\documentclass[10pt]{jsarticle} % use larger type; default would be 10pt
%\usepackage[utf8]{inputenc} % set input encoding (not needed with XeLaTeX)
%\usepackage[round,comma,authoryear]{natbib}
%\usepackage{nruby}
\usepackage{okumacro}
\usepackage{longtable}
%\usepqckage{tablefootnote}
\usepackage[polutonikogreek,english,japanese]{babel}
%\usepackage{amsmath}
\usepackage{latexsym}
\usepackage{color}

\bibliographystyle{jplain}

\title{{\bf PRIMA PARS}\\{\HUGE Summae Theologiae}\\Sancti Thomae
Aquinatis\\{\sffamily QUEAESTIO QUADRAGESIMAOCTAVA}\\DE DISTINCIONE RERUM IN SPECIALI}
\author{Japanese translation\\by Yoshinori {\sc Ueeda}}
\date{Last modified \today}



%%%% コピペ用
%\begin{center}
% {\Large {\bf }}\\
% {\large }\\
% {\footnotesize }\\
% {\Large \\}
%\end{center}
%
%\begin{longtable}{p{21em}p{21em}}
%
%&
%
%\\
%\end{longtable}
%\newpage



\begin{document}
\maketitle

\begin{center}
{\Large 第四十八問\\
諸事物の特殊的な区別について}
\end{center}

\begin{longtable}{p{21em}p{21em}}

Deinde considerandum est de distinctione rerum in speciali. Et primo, de
distinctione boni et mali; deinde de distinctione spiritualis et
corporalis creaturae. Circa primum, quaerendum est de malo; et de causa
mali. Circa malum quaeruntur sex. 

\begin{enumerate}
 \item utrum malum sit natura aliqua. 
 \item utrum malum inveniatur in rebus.
 \item utrum bonum sit subiectum mali.
 \item utrum malum totaliter corrumpat bonum.
 \item de divisione mali per poenam et culpam.
 \item quid habeat plus de ratione mali, utrum poena vel culpa.
\end{enumerate}

&

次に、諸事物の特殊的な区別について考察されるべきである。
第一に、善と悪の区別について、次に、霊的被造物と物体的被造物の区別について。
第一にかんして、悪について、そして、悪の原因について。
悪にかんして、六つのことが問われる。

\begin{enumerate}
 \item 悪は何らかの本性か。
 \item 悪は事物において見いだされるか。
 \item 善は悪の基体か。
 \item 悪は善を完全に消滅させるか。
 \item 罪と罰による悪の区別について。
 \item 罪と悪とで、どちらがより悪の性格を持つか。
\end{enumerate}

\end{longtable}
\newpage

\begin{center}
 {\Large {\bf ARTICULUS PRIMUS}}\\
 {\large UTRUM MALUM SIT NATURA QUAEDAM}\\
 {\footnotesize II {\itshape Sent.}, d.~34, a.~2; III {\itshape SCG.}, c.~7 sqq.; {\itshape de Malo}, q.~1, a.~1; {\itshape Compend.~Theol.}, c.~115; {\itshape De Div.~Nom.}, c.~4, lect.~14.}\\
 {\Large 第一項\\悪は何らかの本性か}
\end{center}

\begin{longtable}{p{21em}p{21em}}

{\huge A}{\scshape d primum sic proceditur}. Videtur quod malum sit
natura quaedam. Quia omne genus est natura quaedam. Sed malum est
quoddam genus, dicitur enim in {\itshape Praedicamentis}, quod {\itshape
bonum et malum non sunt in genere, sed sunt genera aliorum}. Ergo malum
est natura quaedam.

&
第一にかんしては次のように進められる。
悪は何らかの本性であると思われる。
なぜなら、すべての類は何らかの本性である。
しかるに、悪は何らかの類である。なぜなら、『カテゴリー論』で「善と悪は類の中にあるのではなく、むしろ他のものどもの類である」と言われているからである。
ゆえに、悪は何らかの本性である。

\\


2.~{\scshape Praeterea}, omnis differentia constitutiva alicuius speciei
est natura quaedam. Malum autem est differentia constitutiva in
moralibus, differt enim specie malus habitus a bono, ut liberalitas ab
illiberalitate. Ergo malum significat naturam quandam.

&

さらに、ある種を構成する種差はすべて何らかの本性である。しかるに、悪は、
道徳的なことがらにおける構成的種差である。なぜなら、悪い習態は、よい習態
と、種において異なるからである。たとえば、「太っ腹」と「ドケチ」のように。
ゆえに、悪は何らかの本性を表示する。

\\


3.~{\scshape Praeterea}, utrumque contrariorum est natura quaedam. Sed
malum et bonum non opponuntur ut privatio et habitus, sed ut contraria,
ut probat Philosophus, in {\itshape Praedicamentis}, per hoc quod inter
bonum et malum est aliquid medium, et a malo potest fieri reditus ad
bonum. Ergo malum significat naturam quandam.

&

さらに、相反する二つのものは、どちらも何らかの本性である。しかるに、哲学
者が『カテゴリー論』で、善と悪の間には中間のものがあり、悪から善へ戻ると
いうことがありうる、ということから証明しているように、悪と善は、欠如と所
有のように対立するのではなく、相反するものとして対立する。
ゆえに、悪は何らかの本性を表示する。

\\


4.~{\scshape Praeterea}, quod non est, non agit. Sed malum agit, quia
corrumpit bonum. Ergo malum est quoddam ens, et natura quaedam.

&

さらに、存在しないものは作用しない。しかるに、悪は作用する。なぜなら、善
を消滅させるからである。ゆえに、悪は何らかの有であり、何らかの本性である。

\\


5.~{\scshape Praeterea}, ad perfectionem universitatis rerum non
pertinet nisi quod est ens et natura quaedam. Sed malum pertinet ad
perfectionem universitatis rerum, dicit enim Augustinus, in {\itshape Enchirid}.,
quod {\itshape ex omnibus consistit universitatis admirabilis pulchritudo; in qua
etiam illud quod malum dicitur, bene ordinatum, et suo loco positum,
eminentius commendat bona}. Ergo malum est natura quaedam.

&

さらに、諸事物の宇宙の完全性には、有と何らかの本性以外のものは属さない。しかるに、悪は諸事物の宇宙の完全性に属する。なぜなら、アウグスティヌスは『エンキリディオン』で、「宇宙の驚嘆すべき美しさはすべてのものに基づいている。その美しさの中には、悪と言われるものもまた、よく秩序づけられ、自らの場所に置かれたものとして、より優れたかたちで善を推奨している」と述べているからである。ゆえに、悪は何らかの本性である。

\\


{\scshape Sed contra est quod} Dionysius dicit, IV cap. {\itshape de
Div.~Nom}., {\itshape malum non est existens neque bonum}.

&

しかし反対に、ディオニュシウスは『神名論』4巻で「悪は存在せず、善でもない」
と述べている。

\\


{\scshape Respondeo dicendum} quod unum oppositorum cognoscitur per
alterum, sicut per lucem tenebra. Unde et quid sit malum, oportet ex
ratione boni accipere. Diximus autem supra quod bonum est omne id quod
est appetibile, et sic, cum omnis natura appetat suum esse et suam
perfectionem, necesse est dicere quod esse et perfectio cuiuscumque
naturae rationem habeat bonitatis. Unde non potest esse quod malum
significet quoddam esse, aut quandam formam seu naturam. Relinquitur
ergo quod nomine mali significetur quaedam absentia boni. --Et pro tanto
dicitur quod malum {\itshape neque est existens nec bonum}, quia cum
ens, inquantum huiusmodi, sit bonum, eadem est remotio utrorumque.

&
答えて言わなければならない。
対立するものの一方は、他方によって認識される。ちょうど、光によって陰が認識されるように。したがって、悪が何かということも、善の側面から理解しなければならない。
ところで、私たちは上で、善とは欲求されうるものすべてであると言った。また、本性はすべて、自らの存在と完成を欲求するのだから、したがって、どのような本性の存在と完成も善性という性格を持つと言う必要がある。したがって、悪が何らかの存在や形相や本性を表示することはできない。ゆえに、悪という名によって、善の何らかの不在が意味されることになる。
そして以上のような意味を込めて、悪は「存在するものでも善でもない」と言われる。というのも、有は、有である限りにおいて善であるので、悪の隔たりはどちらに対しても同じだからである。



\\

{\scshape Ad primum ergo dicendum} quod Aristoteles ibi loquitur
secundum opinionem Pythagoricorum, qui malum existimabant esse naturam
quandam, et ideo ponebant bonum et malum genera. Consuevit enim
Aristoteles, et praecipue in libris logicalibus, ponere exempla quae
probabilia erant suo tempore, secundum opinionem aliquorum
philosophorum. Vel dicendum, sicut dicit philosophus in X {\itshape Metaphys}.,
quod {\itshape prima contrarietas est habitus et privatio}, quia scilicet in
omnibus contrariis salvatur, cum semper unum contrariorum sit
imperfectum respectu alterius, ut nigrum respectu albi, et amarum
respectu dulcis. Et pro tanto bonum et malum dicuntur genera, non
simpliciter, sed contrariorum, quia sicut omnis forma habet rationem
boni, ita omnis privatio, inquantum huiusmodi, habet rationem mali.

&

第一に対しては、それゆえ、次のように言われるべきで
アリストテレスは、そこで、ピタゴラス派の人々の意見に従って語っている。この人々は、悪が何らかの本性であると考え、したがって、善と悪が類だとした。
じっさい、アリストテレスはよく、とくに論理的著作において、彼の時代にもっともらしく思われていた説を、誰かの哲学者の意見に従って示すことがある。あるいは、次のように言われるべきである。哲学者が『形而上学』10巻で「第一の反対対立は所有と欠如である」と述べるように、というのも、黒が白に対して、苦みが甘さに対してそうであるように、反対対立するものの一方は、つねに、他方との関係において不完全なので、このことは反対対立するすべてのものにおいて成り立つからだが、この同じ意味で、善と悪は類と言われる。ただし、端的にではなく、反対対立するものどもにおける類としてである。というのも、形相がすべて善の性格を持つように、欠如はすべて、欠如である限りにおいて、悪の性格を持つからである。

\\


{\scshape Ad secundum dicendum} quod bonum et malum non sunt
differentiae constitutivae nisi in moralibus, quae recipiunt speciem ex
fine, qui est obiectum voluntatis, a qua moralia dependent. Et quia
bonum habet rationem finis, ideo bonum et malum sunt differentiae
specificae in moralibus; bonum per se, sed malum inquantum est remotio
debiti finis. Nec tamen remotio debiti finis constituit speciem in
moralibus, nisi secundum quod adiungitur fini indebito, sicut neque in
naturalibus invenitur privatio formae substantialis, nisi adiuncta
alteri formae. Sic igitur malum quod est differentia constitutiva in
moralibus, est quoddam bonum adiunctum privationi alterius boni, sicut
finis intemperati est, non quidem carere bono rationis, sed delectabile
sensus absque ordine rationis. Unde malum, inquantum malum, non est
differentia constitutiva; sed ratione boni adiuncti.

&第二に対しては、次のように言われるべきである。善と悪は、道徳的なものにお
いてのみ構成的種差である。道徳的なものは、種を目的から受け取るが、目的は
意志の対象であり、道徳的なことがらは意志に依存するからである。また、善は
目的の性格を持ち、それゆえ善と悪は道徳的なことがらにおいて種を作る種差で
あるが、善はそれ自体によって、しかし悪は、しかるべき目的の除去である限り
で、そのような種差である。しかし、ちょうど自然物においても、他の形相に結
びつけられない限り、実体的形相の欠如が見いだされないように、しかるべき目
的の除去は、不適切な目的に結びつけられることによらないかぎり、道徳的なこ
とがらにおいて種を構成しない。ゆえに、ちょうど不節制な人の目的が、善の性
格を欠いているのではなく、理性による秩序付けを欠いた感覚の喜びである
ように、道徳的な事柄における構成的種差である悪は、他の善の欠如に結びつけ
られた一種の善である。


\\


{\scshape Et per hoc etiam patet responsio ad tertium}. Nam ibi
philosophus loquitur de bono et malo, secundum quod inveniuntur in
moralibus. Sic enim inter bonum et malum invenitur medium, prout bonum
dicitur quod est ordinatum; malum autem, quod non solum est deordinatum,
sed etiam nocivum alteri. Unde dicit philosophus in IV {\itshape Ethic}., quod
prodigus vanus quidem est, sed non malus. Ab hoc etiam malo quod est
secundum morem, contingit fieri reditum ad bonum; non autem ex quocumque
malo. Non enim ex caecitate fit reditus ad visionem, cum tamen caecitas
sit malum quoddam.

&

そしてこのことから、第三異論に対する解答も明らかである。すなわち、そこで
哲学者は、道徳的な事柄において見いだされるかぎりでの善と悪について語って
いる。すなわち、善と悪の間に中間のものが見いだされるのは、善が秩序づけら
れたものについて言われるのに対して、悪は、たんに秩序づけられてないだけで
なく、他に害をなすものについて言われるからである。このことから、哲学者は
『ニコマコス倫理学』4巻で、浪費家は信頼できないが、悪人ではない、と述べて
いる。
さらに、道徳的な悪からも、善へ戻ることが生じうる。しかし、どんな悪からも、これは生じない。それはちょうど、盲目から見ることへ戻ることが生じないようなものである。盲は一種の悪なのだから。


\\


{\scshape Ad quartum dicendum} quod aliquid agere dicitur
 tripliciter. Uno modo, formaliter, eo modo loquendi quo dicitur albedo
 facere album. Et sic malum, etiam ratione ipsius privationis, dicitur
 corrumpere bonum, quia est ipsa corruptio vel privatio boni. Alio modo
 dicitur aliquid agere effective, sicut pictor dicitur facere album
 parietem. Tertio modo, per modum causae finalis, sicut finis dicitur
 efficere, movendo efficientem. His autem duobus modis malum non agit
 aliquid per se, idest secundum quod est privatio quaedam, sed secundum
 quod ei bonum adiungitur, nam omnis actio est ab aliqua forma, et omne
 quod desideratur ut finis, est perfectio aliqua. Et ideo, ut Dionysius
 dicit, {\scshape iv} cap.~{\itshape de Div.~Nom}., malum non agit neque
 desideratur nisi virtute boni adiuncti; per se autem est {\itshape
 infinitum}, et {\itshape praeter voluntatem et intentionem}.


&

第四に対しては、次のように言われるべきである。
何かが作用する、ということは、三通りに言われる。
一つには、形相的にであり、白性が白いものを作る、と言われるような言い方で
 ある。この意味では、悪は、それが欠如だということを根拠にしても、善を消滅させる
 と言われる。なぜなら、悪は、善の消滅ないし欠如それ自体だから。
もう一つには、或るものが作出的に作用とすると言われる。たとえば、画家が壁
 を白くすると言われるように。三つ目には、目的因のありかたによってであり、
 目的が、作出者を動かすことによって作出する、と言われる場合のように。
しかしこの[後者の]二つの言い方では、悪が何かに、自体的に、すなわち或る欠如
 る限りにおいて、作用することはなく、ただ何かに善が結びつけられる限りに
 おいて作用する。というのも、作用はすべて何らかの形相によるのであり、目
 的として望まれるものはすべて何らかの完全性だからである。それゆえ、ディ
 オニュシウスが『神名論』4章で述べるように、悪は、結びつけられた善の知か
 らによらない限り、働くことも望まれることもないと言う。悪はそれ自体にお
 いて、「限られてなく、意志と意図の外にある」。

\\


{\scshape Ad quintum dicendum} quod, sicut supra dictum est, partes
 universi habent ordinem ad invicem, secundum quod una agit in alteram,
 et est finis alterius et exemplar. Haec autem, ut dictum est, non
 possunt convenire malo, nisi ratione boni adiuncti. Unde malum neque ad
 perfectionem universi pertinet, neque sub ordine universi concluditur,
 nisi per accidens, idest ratione boni adiuncti.


&

第五に対しては、次のように言われるべきである。
上述のように、世界の諸部分は、一つが他のものへ作用し、他のものの目的であ
 り範型であるという点で、相互に秩序をもっている。すでに述べられたように、
 これらのことは、結びつけられた善を根拠としてでなければ、悪に適合しえな
 い。したがって、悪は、附帯的に、つまり、結びつけられた善を根拠にしてで
 ないかぎり、世界の完全性に属さず、世界の秩序にも含まれない。


\end{longtable}
\newpage

Articulus 2

[30475] Iª q. 48 a. 2 arg. 1 Ad secundum sic proceditur. Videtur quod malum non inveniatur in rebus. Quidquid enim invenitur in rebus, vel est ens aliquod, vel privatio entis alicuius, quod est non ens. Sed Dionysius dicit, IV cap. de Div. Nom., quod malum distat ab existente, et adhuc plus distat a non existente. Ergo malum nullo modo invenitur in rebus.

[30476] Iª q. 48 a. 2 arg. 2 Praeterea, ens et res convertuntur. Si ergo malum est ens in rebus, sequitur quod malum sit res quaedam. Quod est contra praedicta.

[30477] Iª q. 48 a. 2 arg. 3 Praeterea, albius est quod est nigro impermixtius, ut dicitur in III libro Topic. Aristotelis. Ergo et melius est quod est malo impermixtius. Sed Deus facit semper quod melius est, multo magis quam natura. Ergo in rebus a Deo conditis nihil malum invenitur.

[30478] Iª q. 48 a. 2 s. c. Sed contra est quod secundum hoc removerentur omnes prohibitiones et poenae, quae non sunt nisi malorum.

[30479] Iª q. 48 a. 2 co. Respondeo dicendum quod, sicut supra dictum est, perfectio universi requirit inaequalitatem esse in rebus, ut omnes bonitatis gradus impleantur. Est autem unus gradus bonitatis ut aliquid ita bonum sit, quod nunquam deficere possit. Alius autem gradus bonitatis est, ut sic aliquid bonum sit, quod a bono deficere possit. Qui etiam gradus in ipso esse inveniuntur, quaedam enim sunt, quae suum esse amittere non possunt, ut incorporalia; quaedam vero sunt, quae amittere possunt, ut corporalia. Sicut igitur perfectio universitatis rerum requirit ut non solum sint entia incorruptibilia, sed etiam corruptibilia; ita perfectio universi requirit ut sint quaedam quae a bonitate deficere possint; ad quod sequitur ea interdum deficere. In hoc autem consistit ratio mali, ut scilicet aliquid deficiat a bono. Unde manifestum est quod in rebus malum invenitur, sicut et corruptio, nam et ipsa corruptio malum quoddam est.

[30480] Iª q. 48 a. 2 ad 1 Ad primum ergo dicendum quod malum distat et ab ente simpliciter, et non ente simpliciter, quia neque est sicut habitus, neque sicut pura negatio, sed sicut privatio.

[30481] Iª q. 48 a. 2 ad 2 Ad secundum dicendum quod, sicut dicitur in V Metaphys., ens dupliciter dicitur. Uno modo, secundum quod significat entitatem rei, prout dividitur per decem praedicamenta, et sic convertitur cum re. Et hoc modo, nulla privatio est ens, unde nec malum. Alio modo dicitur ens, quod significat veritatem propositionis, quae in compositione consistit, cuius nota est hoc verbum est, et hoc est ens quo respondetur ad quaestionem an est. Et sic caecitatem dicimus esse in oculo, vel quamcumque aliam privationem. Et hoc modo etiam malum dicitur ens. Propter huius autem distinctionis ignorantiam, aliqui, considerantes quod aliquae res dicuntur malae, vel quod malum dicitur esse in rebus, crediderunt quod malum esset res quaedam.

[30482] Iª q. 48 a. 2 ad 3 Ad tertium dicendum quod Deus et natura, et quodcumque agens, facit quod melius est in toto; sed non quod melius est in unaquaque parte, nisi per ordinem ad totum, ut supra dictum est. Ipsum autem totum quod est universitas creaturarum, melius et perfectius est, si in eo sint quaedam quae a bono deficere possunt, quae interdum deficiunt, Deo hoc non impediente. Tum quia providentiae non est naturam destruere, sed salvare, ut Dionysius dicit, IV cap. de Div. Nom., ipsa autem natura rerum hoc habet, ut quae deficere possunt, quandoque deficiant. Tum quia, ut dicit Augustinus in Enchirid., Deus est adeo potens, quod etiam potest bene facere de malis. Unde multa bona tollerentur, si Deus nullum malum permitteret esse. Non enim generaretur ignis, nisi corrumperetur aer; neque conservaretur vita leonis, nisi occideretur asinus; neque etiam laudaretur iustitia vindicans, et patientia sufferens, si non esset iniquitas.

Articulus 3

[30483] Iª q. 48 a. 3 arg. 1 Ad tertium sic proceditur. Videtur quod malum non sit in bono sicut in subiecto. Omnia enim bona sunt existentia. Sed Dionysius dicit, IV cap. de Div. Nom., quod malum non est existens, neque in existentibus. Ergo malum non est in bono sicut in subiecto.

[30484] Iª q. 48 a. 3 arg. 2 Praeterea, malum non est ens, bonum vero est ens. Sed non ens non requirit ens, in quo sit sicut in subiecto. Ergo nec malum requirit bonum, in quo sit sicut in subiecto.

[30485] Iª q. 48 a. 3 arg. 3 Praeterea, unum contrariorum non est subiectum alterius. Sed bonum et malum sunt contraria. Ergo malum non est in bono sicut in subiecto.

[30486] Iª q. 48 a. 3 arg. 4 Praeterea, id in quo est albedo sicut in subiecto, dicitur esse album. Ergo et id in quo est malum sicut in subiecto, est malum. Si ergo malum sit in bono sicut in subiecto, sequitur quod bonum sit malum, contra id quod dicitur Isai. V, vae, qui dicitis malum bonum, et bonum malum.

[30487] Iª q. 48 a. 3 s. c. Sed contra est quod Augustinus dicit, in Enchirid., quod malum non est nisi in bono.

[30488] Iª q. 48 a. 3 co. Respondeo dicendum quod, sicut dictum est, malum importat remotionem boni. Non autem quaelibet remotio boni malum dicitur. Potest enim accipi remotio boni et privative, et negative. Remotio igitur boni negative accepta, mali rationem non habet, alioquin sequeretur quod ea quae nullo modo sunt, mala essent; et iterum quod quaelibet res esset mala, ex hoc quod non habet bonum alterius rei, utpote quod homo esset malus, quia non habet velocitatem capreae, vel fortitudinem leonis. Sed remotio boni privative accepta, malum dicitur, sicut privatio visus caecitas dicitur. Subiectum autem privationis et formae est unum et idem, scilicet ens in potentia, sive sit ens in potentia simpliciter, sicut materia prima, quae est subiectum formae substantialis et privationis oppositae; sive sit ens in potentia secundum quid et in actu simpliciter, ut corpus diaphanum, quod est subiectum tenebrarum et lucis. Manifestum est autem quod forma per quam aliquid est actu, perfectio quaedam est, et bonum quoddam, et sic omne ens in actu, bonum quoddam est. Et similiter omne ens in potentia, inquantum huiusmodi, bonum quoddam est, secundum quod habet ordinem ad bonum, sicut enim est ens in potentia, ita et bonum in potentia. Relinquitur ergo quod subiectum mali sit bonum.

[30489] Iª q. 48 a. 3 ad 1 Ad primum ergo dicendum quod Dionysius intelligit malum non esse in existentibus sicut partem, aut sicut proprietatem naturalem alicuius existentis.

[30490] Iª q. 48 a. 3 ad 2 Ad secundum dicendum quod non ens negative acceptum non requirit subiectum. Sed privatio est negatio in subiecto, ut dicitur in IV Metaphys., et tale non ens est malum.

[30491] Iª q. 48 a. 3 ad 3 Ad tertium dicendum quod malum non est sicut in subiecto in bono quod ei opponitur, sed in quodam alio bono, subiectum enim caecitatis non est visus, sed animal. Videtur tamen, ut Augustinus dicit, hic fallere dialecticorum regula, quae dicit contraria simul esse non posse. Hoc tamen intelligendum est secundum communem acceptionem boni et mali, non autem secundum quod specialiter accipitur hoc bonum et hoc malum. Album autem et nigrum, dulce et amarum, et huiusmodi contraria, non accipiuntur nisi specialiter, quia sunt in quibusdam generibus determinatis. Sed bonum circuit omnia genera, unde unum bonum potest simul esse cum privatione alterius boni.

[30492] Iª q. 48 a. 3 ad 4 Ad quartum dicendum quod propheta imprecatur vae illis qui dicunt id quod est bonum, secundum quod est bonum, esse malum. Hoc autem non sequitur ex praemissis, ut per praedicta patet.

Articulus 4

[30493] Iª q. 48 a. 4 arg. 1 Ad quartum sic proceditur. Videtur quod malum corrumpat totum bonum. Unum enim contrariorum totaliter corrumpitur per alterum. Sed bonum et malum sunt contraria. Ergo malum potest corrumpere totum bonum.

[30494] Iª q. 48 a. 4 arg. 2 Praeterea, Augustinus dicit, in Enchirid., quod malum nocet inquantum adimit bonum. Sed bonum est sibi simile et uniforme. Ergo totaliter tollitur per malum.

[30495] Iª q. 48 a. 4 arg. 3 Praeterea, malum, quandiu est, nocet et aufert bonum. Sed illud a quo semper aliquid aufertur, quandoque consumitur, nisi sit infinitum; quod non potest dici de aliquo bono creato. Ergo malum consumit totaliter bonum.

[30496] Iª q. 48 a. 4 s. c. Sed contra est quod Augustinus dicit, in Enchirid., quod malum non potest totaliter consumere bonum.

[30497] Iª q. 48 a. 4 co. Respondeo dicendum quod malum non potest totaliter consumere bonum. Ad cuius evidentiam, considerandum est quod est triplex bonum. Quoddam, quod per malum totaliter tollitur, et hoc est bonum oppositum malo; sicut lumen totaliter per tenebras tollitur, et visus per caecitatem. Quoddam vero bonum est, quod nec totaliter tollitur per malum, nec diminuitur, scilicet bonum quod est subiectum mali; non enim per tenebras aliquid de substantia aeris diminuitur. Quoddam vero bonum est, quod diminuitur quidem per malum, sed non totaliter tollitur, et hoc bonum est habilitas subiecti ad actum. Diminutio autem huius boni non est accipienda per subtractionem, sicut est diminutio in quantitatibus, sed per remissionem, sicut est diminutio in qualitatibus et formis. Remissio autem huius habilitatis est accipienda e contrario intensioni ipsius. Intenditur enim huiusmodi habilitas per dispositiones quibus materia praeparatur ad actum; quae quanto magis multiplicantur in subiecto, tanto habilius est ad recipiendum perfectionem et formam. Et e contrario remittitur per dispositiones contrarias; quae quanto magis multiplicatae sunt in materia, et magis intensae, tanto magis remittitur potentia ad actum. Si igitur contrariae dispositiones in infinitum multiplicari et intendi non possunt, sed usque ad certum terminum, neque habilitas praedicta in infinitum diminuitur vel remittitur. Sicut patet in qualitatibus activis et passivis elementorum, frigiditas enim et humiditas, per quae diminuitur sive remittitur habilitas materiae ad formam ignis, non possunt multiplicari in infinitum. Si vero dispositiones contrariae in infinitum multiplicari possunt, et habilitas praedicta in infinitum diminuitur vel remittitur. Non tamen totaliter tollitur, quia semper manet in sua radice, quae est substantia subiecti. Sicut si in infinitum interponantur corpora opaca inter solem et aerem, in infinitum diminuetur habilitas aeris ad lumen, nunquam tamen totaliter tollitur, manente aere, qui secundum naturam suam est diaphanus. Similiter in infinitum potest fieri additio in peccatis, per quae semper magis ac magis minuitur habilitas animae ad gratiam, quae quidem peccata sunt quasi obstacula interposita inter nos et Deum secundum illud Isaiae LIX, peccata nostra diviserunt inter nos et Deum. Neque tamen tollitur totaliter ab anima praedicta habilitas, quia consequitur naturam ipsius.

[30498] Iª q. 48 a. 4 ad 1 Ad primum ergo dicendum quod bonum quod opponitur malo, totaliter tollitur, sed alia bona non totaliter tolluntur, ut dictum est.

[30499] Iª q. 48 a. 4 ad 2 Ad secundum dicendum quod habilitas praedicta est media inter subiectum et actum. Unde ex ea parte qua attingit actum, diminuitur per malum, sed ex ea parte qua tenet se cum subiecto, remanet. Ergo, licet bonum in se sit simile, tamen, propter comparationem eius ad diversa, non totaliter tollitur, sed in parte.

[30500] Iª q. 48 a. 4 ad 3 Ad tertium dicendum quod quidam, imaginantes diminutionem boni praedicti ad similitudinem diminutionis quantitatis, dixerunt quod, sicut continuum dividitur in infinitum, facta divisione secundum eandem proportionem (ut puta quod accipiatur medium medii, vel tertium tertii), sic in proposito accidit. Sed haec ratio hic locum non habet. Quia in divisione in qua semper servatur eadem proportio, semper subtrahitur minus et minus, minus enim est medium medii quam medium totius. Sed secundum peccatum non de necessitate minus diminuit de habilitate praedicta, quam praecedens, sed forte aut aequaliter, aut magis. Dicendum est ergo quod, licet ista habilitas sit quoddam finitum, diminuitur tamen in infinitum, non per se, sed per accidens, secundum quod contrariae dispositiones etiam in infinitum augentur, ut dictum est.

Articulus 5

[30501] Iª q. 48 a. 5 arg. 1 Ad quintum sic proceditur. Videtur quod malum insufficienter dividatur per poenam et culpam. Omnis enim defectus malum quoddam esse videtur. Sed in omnibus creaturis est quidam defectus, quod se in esse conservare non possunt, qui tamen nec poena nec culpa est. Non ergo sufficienter malum dividitur per poenam et culpam.

[30502] Iª q. 48 a. 5 arg. 2 Praeterea, in rebus irrationalibus non invenitur culpa nec poena. Invenitur tamen in eis corruptio et defectus, quae ad rationem mali pertinent. Ergo non omne malum est poena vel culpa.

[30503] Iª q. 48 a. 5 arg. 3 Praeterea, tentatio quoddam malum est. Nec tamen est culpa, quia tentatio cui non consentitur, non est peccatum, sed materia exercendae virtutis, ut dicitur in Glossa II Cor. XII. Nec etiam poena, quia tentatio praecedit culpam, poena autem subsequitur. Insufficienter ergo malum dividitur per poenam et culpam.

[30504] Iª q. 48 a. 5 s. c. Sed contra, videtur quod divisio sit superflua. Ut enim Augustinus dicit, in Enchirid., malum dicitur quia nocet. Quod autem nocet, poenale est. Omne ergo malum sub poena continetur.

[30505] Iª q. 48 a. 5 co. Respondeo dicendum quod malum, sicut supra dictum est, est privatio boni, quod in perfectione et actu consistit principaliter et per se. Actus autem est duplex, primus, et secundus. Actus quidem primus est forma et integritas rei, actus autem secundus est operatio. Contingit ergo malum esse dupliciter. Uno modo, per subtractionem formae, aut alicuius partis, quae requiritur ad integritatem rei; sicut caecitas malum est, et carere membro. Alio modo, per subtractionem debitae operationis; vel quia omnino non est; vel quia debitum modum et ordinem non habet. Quia vero bonum simpliciter est obiectum voluntatis, malum, quod est privatio boni, secundum specialem rationem invenitur in creaturis rationalibus habentibus voluntatem. Malum igitur quod est per subtractionem formae vel integritatis rei, habet rationem poenae; et praecipue supposito quod omnia divinae providentiae et iustitiae subdantur, ut supra ostensum est, de ratione enim poenae est, quod sit contraria voluntati. Malum autem quod consistit in subtractione debitae operationis in rebus voluntariis, habet rationem culpae. Hoc enim imputatur alicui in culpam, cum deficit a perfecta actione, cuius dominus est secundum voluntatem. Sic igitur omne malum in rebus voluntariis consideratum vel est poena vel culpa.

[30506] Iª q. 48 a. 5 ad 1 Ad primum ergo dicendum quod, quia malum privatio est boni, et non negatio pura, ut dictum est supra; non omnis defectus boni est malum, sed defectus boni quod natum est et debet haberi. Defectus enim visionis non est malum in lapide, sed in animali, quia contra rationem lapidis est, quod visum habeat. Similiter etiam contra rationem creaturae est, quod in esse conservetur a seipsa, quia idem dat esse et conservat. Unde iste defectus non est malum creaturae.

[30507] Iª q. 48 a. 5 ad 2 Ad secundum dicendum quod poena et culpa non dividunt malum simpliciter; sed malum in rebus voluntariis.

[30508] Iª q. 48 a. 5 ad 3 Ad tertium dicendum quod tentatio, prout importat provocationem ad malum, semper malum culpae est in tentante. Sed in eo qui tentatur, non est proprie, nisi secundum quod aliqualiter immutatur, sic enim actio agentis est in patiente. Secundum autem quod tentatus immutatur ad malum a tentante, incidit in culpam.

[30509] Iª q. 48 a. 5 ad 4 Ad quartum dicendum quod de ratione poenae est, quod noceat agenti in seipso. Sed de ratione culpae est, quod noceat agenti in sua actione. Et sic utrumque sub malo continetur, secundum quod habet rationem nocumenti.

Articulus 6

[30510] Iª q. 48 a. 6 arg. 1 Ad sextum sic proceditur. Videtur quod habeat plus de ratione mali poena quam culpa. Culpa enim se habet ad poenam, ut meritum ad praemium. Sed praemium habet plus de ratione boni quam meritum, cum sit finis eius. Ergo poena plus habet de ratione mali quam culpa.

[30511] Iª q. 48 a. 6 arg. 2 Praeterea, illud est maius malum, quod opponitur maiori bono. Sed poena, sicut dictum est, opponitur bono agentis, culpa autem bono actionis. Cum ergo melius sit agens quam actio, videtur quod peius sit poena quam culpa.

[30512] Iª q. 48 a. 6 arg. 3 Praeterea, ipsa privatio finis poena quaedam est, quae dicitur carentia visionis divinae. Malum autem culpae est per privationem ordinis ad finem. Ergo poena est maius malum quam culpa.

[30513] Iª q. 48 a. 6 s. c. Sed contra, sapiens artifex inducit minus malum ad vitandum maius; sicut medicus praecidit membrum, ne corrumpatur corpus. Sed Dei sapientia infert poenam ad vitandam culpam. Ergo culpa est maius malum quam poena.

[30514] Iª q. 48 a. 6 co. Respondeo dicendum quod culpa habet plus de ratione mali quam poena, et non solum quam poena sensibilis, quae consistit in privatione corporalium bonorum, cuiusmodi poenas plures intelligunt; sed etiam universaliter accipiendo poenam, secundum quod privatio gratiae vel gloriae poenae quaedam sunt. Cuius est duplex ratio. Prima quidem est, quia ex malo culpae fit aliquis malus, non autem ex malo poenae; secundum illud Dionysii, IV cap. de Div. Nom., puniri non est malum, sed fieri poena dignum. Et hoc ideo est quia, cum bonum simpliciter consistat in actu, et non in potentia, ultimus autem actus est operatio, vel usus quarumcumque rerum habitarum; bonum hominis simpliciter consideratur in bona operatione, vel bono usu rerum habitarum. Utimur autem rebus omnibus per voluntatem. Unde ex bona voluntate, qua homo bene utitur rebus habitis, dicitur homo bonus; et ex mala, malus. Potest enim qui habet malam voluntatem, etiam bono quod habet, male uti; sicut si grammaticus voluntarie incongrue loquatur. Quia ergo culpa consistit in deordinato actu voluntatis, poena vero in privatione alicuius eorum quibus utitur voluntas; perfectius habet rationem mali culpa quam poena. Secunda ratio sumi potest ex hoc, quod Deus est auctor mali poenae, non autem mali culpae. Cuius ratio est, quia malum poenae privat bonum creaturae, sive accipiatur bonum creaturae aliquid creatum, sicut caecitas privat visum; sive sit bonum increatum, sicut per carentiam visionis divinae tollitur creaturae bonum increatum. Malum vero culpae opponitur proprie ipsi bono increato, contrariatur enim impletioni divinae voluntatis, et divino amori quo bonum divinum in seipso amatur; et non solum secundum quod participatur a creatura. Sic igitur patet quod culpa habet plus de ratione mali quam poena.

[30515] Iª q. 48 a. 6 ad 1 Ad primum ergo dicendum quod, licet culpa terminetur ad poenam, sicut meritum ad praemium, tamen culpa non intenditur propter poenam, sicut meritum propter praemium, sed potius e converso poena inducitur ut vitetur culpa. Et sic culpa est peius quam poena.

[30516] Iª q. 48 a. 6 ad 2 Ad secundum dicendum quod ordo actionis, qui tollitur per culpam, est perfectius bonum agentis, cum sit perfectio secunda, quam bonum quod tollitur per poenam, quod est perfectio prima.

[30517] Iª q. 48 a. 6 ad 3 Ad tertium dicendum quod non est comparatio culpae ad poenam sicut finis et ordinis ad finem, quia utrumque potest privari aliquo modo et per culpam, et per poenam. Sed per poenam quidem, secundum quod ipse homo removetur a fine, et ab ordine ad finem, per culpam vero, secundum quod ista privatio pertinet ad actionem, quae non ordinatur ad finem debitum.

\end{document}