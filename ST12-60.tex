\documentclass[10pt]{jsarticle}
\usepackage{okumacro}
\usepackage{longtable}
\usepackage[polutonikogreek,english,japanese]{babel}
\usepackage{latexsym}
\usepackage{color}
\usepackage{schemata}

%----- header -------
\usepackage{fancyhdr}
\pagestyle{fancy}
\lhead{{\it Summa Theologiae} I-II, q.60}
%--------------------

\bibliographystyle{jplain}

\title{{\bf PRIMA SECUNDAE}\\{\HUGE Summae Theologiae}\\Sancti Thomae
Aquinatis\\{\sffamily QUEAESTIO SEXAGESIMA}\\DE DISTINCTIONE VIRTUTUM MORALIUM AD INVICEM}
\author{Japanese translation\\by Yoshinori {\sc Ueeda}}
\date{Last modified \today}

%%%% コピペ用
%\rhead{a.~}
%\begin{center}
% {\Large {\bf }}\\
% {\large }\\
% {\footnotesize }\\
% {\Large \\}
%\end{center}
%
%\begin{longtable}{p{21em}p{21em}}
%
%&
%
%\\
%\end{longtable}
%\newpage

\begin{document}

\maketitle
\thispagestyle{empty}

\begin{center}
{\LARGE 『神学大全』第二部の一}\\
{\Large 第六十問\\道徳的徳相互の区別について}
\end{center}

\begin{longtable}{p{21em}p{21em}}

Deinde considerandum est de distinctione virtutum moralium ad
invicem. Et circa hoc quaeruntur quinque.
  
\begin{enumerate}
 \item utrum sit tantum una virtus moralis.
 \item utrum distinguantur virtutes morales quae sunt circa operationes, ab his quae sunt circa passiones.
 \item utrum circa operationes sit una tantum moralis virtus.
 \item utrum circa diversas passiones sint diversae morales virtutes.
 \item utrum virtutes morales distinguantur secundum diversa obiecta passionum.
\end{enumerate}

&

 次に道徳的徳相互の区別について考察されるべきである。これを巡っては五つのことが問われる。

\begin{enumerate}
 \item ただ一つの道徳的徳があるか。
 \item 働きにかんする道徳的徳は、情念にかんする徳から区別されるか。
 \item 働きにかんしてはただ一つの道徳的徳があるか。
 \item さまざまな情念にかんしてさまざまな道徳的徳があるか。
 \item 道徳的徳は情念のさまざまな対象に応じて区別されるか。
\end{enumerate}
\end{longtable}
\newpage

\rhead{a.~1}
\begin{center}
{\Large {\bf ARTICULUS PRIMUS}}\\
{\large UTRUM UNA TANTUM VIRTUS MORALIS}\\
{\footnotesize III {\itshape Sent.}, d.33, q.1, a.1, qu$^{a}$1.}\\
{\Large 第一項\\ただ一つの道徳的徳があるか}
\end{center}

\begin{longtable}{p{21em}p{21em}}

{\scshape Ad primum sic proceditur}. Videtur quod sit una tantum
moralis virtus. Sicut enim in actibus moralibus directio pertinet ad
rationem, quae est subiectum intellectualium virtutum; ita inclinatio
pertinet ad vim appetitivam, quae est subiectum moralium virtutum. Sed
una est intellectualis virtus dirigens in omnibus moralibus actibus,
scilicet prudentia. Ergo etiam una tantum est moralis virtus inclinans
in omnibus moralibus actibus.

&

 第一項の問題へ、議論は以下のように進められる。ただ一つの道徳的徳があ
 ると思われる。理由は以下の通り。道徳的行為において、方向付けは理性に
 属し、そしてこれは知性的諸徳の基体であるように、傾向性は欲求的力に属
 し、これが道徳的諸徳の基体である。しかるにすべての道徳的行為において
 指導する知性的徳はただ一つ、すなわち思慮である。ゆえにすべての道徳的
 行為において傾く道徳的徳もただ一つである。

\\



2.~{\scshape Praeterea}, habitus non distinguuntur secundum materialia
obiecta, sed secundum formales rationes obiectorum. Formalis autem
ratio boni ad quod ordinatur virtus moralis, est unum, scilicet modus
rationis. Ergo videtur quod sit una tantum moralis virtus.

&

さらに、習慣は質料的対象に即してではなく、対象の形相的性格に即して区別
される。しかるに道徳的徳がそれへと秩序づけられる善の形相的性格は一つ、
すなわち理性の様態(modus rationis)である。ゆえに道徳的徳はただ一つだと
思われる。

\\



3.~{\scshape Praeterea}, moralia recipiunt speciem a fine, ut supra
dictum est. Sed finis omnium virtutum moralium communis est unus,
scilicet felicitas; proprii autem et propinqui sunt infiniti. Non sunt
autem infinitae virtutes morales. Ergo videtur quod sit una tantum.

&

さらに、前に述べられたとおり、道徳的なものは目的から種を受け取る。しか
るにすべての道徳的徳の目的は共通で一つのもの、すなわち幸福である。ただ
し固有で最近接の目的は無数にあるが。しかるに無数の道徳的徳は存在しない。
ゆえにただ一つ存在すると思われる。

\\



{\scshape Sed contra est} quod unus habitus non potest esse in
diversis potentiis, ut supra dictum est. Sed subiectum virtutum
moralium est pars appetitiva animae, quae per diversas potentias
distinguitur, ut in primo dictum est. Ergo non potest esse una tantum
virtus moralis.

&

しかし反対に、先に述べられたとおり、一つの習慣がさまざまな能力の中にあ
ることはできない。しかるに道徳的徳の基体は魂の欲求的部分であり、第一部
で言われたとおり、これはさまざまな部分によって区別されている。ゆえにた
だ一つの道徳的徳があることはできない。

\\



 {\scshape Respondeo dicendum quod}, sicut supra dictum est, virtutes
 morales sunt habitus quidam appetitivae partis. Habitus autem specie
 differunt secundum speciales differentias obiectorum, ut supra dictum
 est. Species autem obiecti appetibilis, sicut et cuiuslibet rei,
 attenditur secundum formam specificam, quae est ab agente.


&

 解答する。以下のように言われるべきである。上述の如く、道徳的徳は欲求
 的部分のある種の習慣である。しかるに習慣は、すでに述べられたとおり、
 対象の種的な差異に即して異なる。ところが欲求されうる対象の種は、他の
 どんな事物の種とも等しく、種に分ける形相に即して見出され、そしてこの
 形相をもたらすのは作用者である。

\\


 Est autem considerandum quod materia patientis se habet ad agens
 dupliciter. Quandoque enim recipit formam agentis secundum eandem
 rationem, prout est in agente, sicut est in omnibus agentibus
 univocis. Et sic necesse est quod, si agens est unum specie, quod
 materia recipiat formam unius speciei, sicut ab igne non generatur
 univoce nisi aliquid existens in specie ignis.

&

 また、受動するものの質料は作用者に対して二通りに関係することが考察さ
 れるべきである。すなわち、あるときには、作用者の形相を同一の性格に即
 して、作用者の中にあるものとして受け取る。これはすべての一義的作用者
 においてそうである。この場合、もし作用者が種において一つであれば、質
 料が一つの種に属する形相を受け取ることが必然である。たとえば、火によっ
 ては火の種に属する何かが一義的に生み出されるしかない。


\\


 Aliquando vero materia recipit formam ab agente non secundum eandem
 rationem, prout est in agente, sicut patet in generantibus non
 univocis, ut animal generatur a sole. Et tunc formae receptae in
 materia ab eodem agente, non sunt unius speciei sed diversificantur
 secundum diversam proportionem materiae ad recipiendum influxum
 agentis, sicut videmus quod ab una actione solis generantur per
 putrefactionem animalia diversarum specierum secundum diversam
 proportionem materiae.

&

他方で、質料が形相を作用者から同一の性格に即して、作用者の中にあるもの
として受け取らないことがあり、これは動物が太陽によって生み出される場合
のように、一義的に生み出すのでないものにおいて明らかである。そしてこの
場合、同一の作用者から質料において受け取られた形相は一つの種に属さず、
作用者の流入を受け取ることに対する質料のさまざまな関係に即して多様化さ
れる。たとえば、太陽の一つの作用によって、質料のさまざまな関係に即して
さまざまな種に属する動物が腐敗によって生じるのを、私たちは見る。

\\



 Manifestum est autem quod in moralibus ratio est sicut imperans et
 movens; vis autem appetitiva est sicut imperata et mota. Non autem
 appetitus recipit impressionem rationis quasi univoce, quia non fit
 rationale per essentiam, sed per participationem, ut dicitur in I
 {\itshape Ethic}. Unde appetibilia secundum motionem rationis
 constituuntur in diversis speciebus, secundum quod diversimode se
 habent ad rationem. Et ita sequitur quod virtutes morales sint
 diversae secundum speciem, et non una tantum.

&

さて、道徳的な事柄において、理性は命令し動かすものであり、これに対して
欲求的力は命じられ動かされるものであることは明らかである。しかるに欲求
は理性の刻印を一義的に受け取らない。なぜなら、『ニコマコス倫理学』第1
巻で言われているように、それは本質によって理性的なものではなく、分有に
よって理性的なものだからである。したがって、理性の運動に即して欲求され
うるものは、それらの理性へのさまざまな関係に即して、さまざまな種におい
て構成される。その結果、道徳的徳は種においてさまざまであり、ただ一つで
ないことが帰結する。

\\



{\scshape Ad primum ergo dicendum} quod obiectum rationis est
verum. Est autem eadem ratio veri in omnibus moralibus, quae sunt
contingentia agibilia. Unde est una sola virtus in eis dirigens,
scilicet prudentia. Obiectum autem appetitivae virtutis est bonum
appetibile. Cuius est diversa ratio, secundum diversam habitudinem ad
rationem dirigentem.

&

 第一異論に対しては、それゆえ、以下のように言われるべきである。理性の
 対象は真である。しかるにすべての道徳的な事柄において、真の性格は同一
 であり、それは、「行為されうる偶然的な事柄」である。このことから、そ
 れらにおいてただ一つの指導する徳、すなわち思慮がある。これに対して欲
 求的なちからの対象は、「欲求されうる善」である。そしてこれには、指導
 する理性へのさまざまな関係に即して、さまざまな性格が属する。

\\



{\scshape Ad secundum dicendum} quod illud formale est unum genere,
propter unitatem agentis. Sed diversificatur specie, propter diversas
habitudines recipientium, ut supra dictum est.

&

 第二異論に対しては以下のように言われるべきである。異論が言うところの
 形相的なものとは、作用者の一性のために類において一つである。しかし、
 すでに述べられたとおり、受け取るもののさまざまな関係のために、種にお
 いて多様化される。
 

\\



{\scshape Ad tertium dicendum} quod moralia non habent speciem a fine
ultimo sed a finibus proximis, qui quidem, etsi infiniti sint numero,
non tamen infiniti sunt specie.

&

第三異論に対しては以下のように言われるべきである。道徳的なものは究極目
的からではなく、最近接の目的から種を受け取る。それらは数において無限で
あるとしても、種において無限であるわけではない。


\end{longtable}
\newpage

\rhead{a.~2}
\begin{center}
{\Large {\bf ARTICULUS SECUNDUS}}\\
{\large UTRUM VIRTUTES MORALES QUAE SUNT CIRCA OPERATIONES,\\DISTINGUANTUR AB HIS QUAE SUNT CIRCA PASSIONES}\\
{\footnotesize II {\itshape Ethic.}, lect.8.}\\
{\Large 第二項\\働きを巡ってある道徳的徳は\\情念を巡ってある道徳的徳から区別されるか}
\end{center}

\begin{longtable}{p{21em}p{21em}}
 {\scshape Ad secundum sic proceditur}. Videtur quod virtutes morales
 non distinguantur ab invicem per hoc quod quaedam sunt circa
 operationes, quaedam circa passiones. Dicit enim philosophus, in II
 {\itshape Ethic}., quod virtus moralis est {\itshape circa
 delectationes et tristitias optimorum operativa}. Sed voluptates et
 tristitiae sunt passiones quaedam, ut supra dictum est. Ergo eadem
 virtus quae est circa passiones, est etiam circa operationes, utpote
 operativa existens.

 &

 第二項の問題へ、議論は以下のように進められる。道徳的徳は、あるものが
 働きを巡って、あるものが情念を巡ってあるということによって相互に区別
 されることはないと思われる。理由は以下の通り。哲学者は『ニコマコス倫
 理学』第2巻で、道徳的徳は「快楽と悲しみを巡って最善のことをなしうるも
 の」であると述べている。しかるに快楽と悲しみは、前に述べられたとおり、
 ある種の情念である。ゆえに情念を巡ってあるのと同じ徳が、働きうるもの
 として働きを巡ってもある。
 
\\



2.~{\scshape Praeterea}, passiones sunt principia exteriorum
operationum. Si ergo aliquae virtutes rectificant passiones, oportet
quod etiam per consequens rectificent operationes. Eaedem ergo
virtutes morales sunt circa passiones et operationes.

 &

さらに、情念は外的な働きの根源である。ゆえにもしある徳が情念を正しくす
るならば、結果的に働きも正しくしなければならない。ゆえに同じ道徳的徳が
情念と働きを巡ってある。
 
\\



 3.~{\scshape Praeterea}, ad omnem operationem exteriorem movetur
 appetitus sensitivus bene vel male. Sed motus appetitus sensitivi
 sunt passiones. Ergo eaedem virtutes quae sunt circa operationes,
 sunt circa passiones.

 &

さらに、すべての外的な働きへ、感覚的欲求は善くあるいは悪く動かされる。
しかるに感覚的欲求の運動とは情念である。ゆえに、働きを巡ってあるのと同
じ徳が情念を巡ってもある。
 
\\



 {\scshape Sed contra est} quod philosophus ponit iustitiam circa
 operationes; temperantiam autem et fortitudinem et mansuetudinem,
 circa passiones quasdam.

 &

しかし反対に、哲学者は働きを巡って正義を、ある種の情念を巡っては、節制
と勇気と温和があるとしている。
 
\\



 {\scshape Respondeo dicendum quod} operatio et passio dupliciter
 potest comparari ad virtutem. Uno modo, sicut effectus. Et hoc modo,
 omnis moralis virtus habet aliquas operationes bonas, quarum est
 productiva; et delectationem aliquam vel tristitiam, quae sunt
 passiones, ut supra dictum est.

&

 解答する。以下のように言われるべきである。働きと情念は二通りの仕方で
 相互に比較されうる。一つには結果としてであり、このしかたでは、すべて
 の道徳的徳は、それを生み出しうるような何らかの善い働きを持ち、また、
 上述の如く、情念であるところの何らかの快楽と悲しみを持つ。
 
\\

 Alio modo potest comparari operatio ad virtutem moralem, sicut
 materia circa quam est. Et secundum hoc, oportet alias esse virtutes
 morales circa operationes, et alias circa passiones. Cuius ratio est,
 quia bonum et malum in quibusdam operationibus attenditur secundum
 seipsas, qualitercumque homo afficiatur ad eas, inquantum scilicet
 bonum in eis et malum accipitur secundum rationem commensurationis ad
 alterum.

&

 もう一つのしかたでは、働きが道徳的徳に、ちょうどそれを巡ってある質料
 として関係づけられる。この意味では、働きを巡ってある道徳的徳と、情念
 を巡ってある道徳的徳とは別のものでなければならない。その理由は以下の
 通りである。ある働きにおける善と悪は、人間がどのようにそれらの働きへ
 情動するかに関係なく、その働き自体において見出される、すなわち、他者
 に対する公平性(commensuratio)の性格に即して受け取られる。

 
 
\\

 Et in talibus oportet quod sit aliqua virtus directiva operationum
 secundum seipsas, sicut sunt emptio et venditio, et omnes huiusmodi
 operationes in quibus attenditur ratio debiti vel indebiti ad
 alterum. Et propter hoc, iustitia et partes eius proprie sunt circa
 operationes sicut circa propriam materiam.


 &

 そしてそのようなものにおいては、働きをそれ自体に即して導きうる何らか
 の徳がなければならない。たとえば売り買いや、そのような、他者に対して
 「あるべき」あるいは「あるべからざる」という性格が見出されるすべ
 ての働きがあるように。このため、正義とその部分は固有に、ちょうど固有
 の質料をめぐってあるようなかたちで、働きを巡ってある。

 
\\

 In quibusdam vero operationibus bonum et malum attenditur solum
 secundum commensurationem ad operantem. Et ideo oportet in his bonum
 et malum considerari, secundum quod homo bene vel male afficitur
 circa huiusmodi. Et propter hoc, oportet quod virtutes in talibus
 sint principaliter circa interiores affectiones, quae dicuntur animae
 passiones, sicut patet de temperantia, fortitudine et aliis
 huiusmodi.

&

 他方で、ある働きにおいては善と悪が働く者への公平性に即してのみ見出さ
 れる。そしてそれゆえに、これらにおいては善と悪が、その人が善くあるい
 は悪く、そのようなものを巡って情動付けられているかぎりで、考察されな
 ければならない。このため、このようなものにおける徳は、魂の情念と呼ば
 れる内的な情動を巡ってなければならない。それはちょうど節制、勇気、そ
 の他そのようなものについて明らかなとおりである。

 
\\

 Contingit autem quod in operationibus quae sunt ad alterum,
 praetermittatur bonum virtutis propter inordinatam animi
 passionem. Et tunc, inquantum corrumpitur commensuratio exterioris
 operationis, est corruptio iustitiae, inquantum autem corrumpitur
 commensuratio interiorum passionum, est corruptio alicuius alterius
 virtutis. Sicut cum propter iram aliquis alium percutit, in ipsa
 percussione indebita corrumpitur iustitia, in immoderantia vero irae
 corrumpitur mansuetudo. Et idem patet in aliis.

 &

 しかし、他者に対してある働きにおいては、秩序付けられていない魂の情念
 のために、徳の善が失われることがある。その場合、外的な働きの公平性が
 消滅する限りにおいて正義が消滅するが、内的情念の公平性が消滅する限り
 において、何らかの他の徳が消滅する。たとえば、怒りのためにある人が他
 人を殴ったとき、そのあるべからざる打撃自体において正義が消滅し、他方、
 怒りの激しさにおいて温和が消滅する。これは他の徳においても同様である。

 
\\



 Et per hoc patet responsio ad obiecta. Nam prima ratio procedit de
 operatione, secundum quod est effectus virtutis. Aliae vero duae
 rationes procedunt ex hoc, quod ad idem concurrunt operatio et
 passio. Sed in quibusdam virtus est principaliter circa operationem,
 in quibusdam circa passionem, ratione praedicta.

 &

そしてこのことによって異論への解答は明らかである。すなわち、第一異論は
徳の結果であるかぎりでの働きについて論じている。他の二つの異論は、働き
と情念が同じものに合流することに基づいて論じている。しかし上述の理由に
よって、ある事柄において、徳は主要に働きを巡ってあり、別の事柄において
は情念を巡ってある。
 

\end{longtable}
\newpage

\rhead{a.~3}
\begin{center}
{\Large {\bf ARTICULUS TERTIUS}}\\
{\large UTRUM CIRCA OPERATIONES SIT TANTUM UNA VIRTUS MORALIS}\\
{\Large 第三項\\働きを巡ってただ一つの道徳的徳があるか}
\end{center}

\begin{longtable}{p{21em}p{21em}}
{\scshape Ad tertium sic proceditur}. Videtur quod sit una tantum
virtus moralis circa operationes. Rectitudo enim omnium operationum
exteriorum videtur ad iustitiam pertinere. Sed iustitia est una
virtus. Ergo una sola virtus est circa operationes.

&

 第三項の問題へ、議論は以下のように進められる。働きを巡ってただ一つの
 道徳的徳があると思われる。理由は以下の通り。すべての外的な働きの正し
 さは正義に属すると思われる。しかるに正義は一つの徳である。ゆえに働き
 を巡ってはただ一つの徳がある。
 
\\

2.~{\scshape Praeterea}, operationes maxime differentes esse videntur
quae ordinantur ad bonum unius, et quae ordinantur ad bonum
multitudinis. Sed ista diversitas non diversificat virtutes morales,
dicit enim philosophus, in V {\itshape Ethic}., quod iustitia legalis,
quae ordinat actus hominum ad commune bonum, non est aliud a virtute
quae ordinat actus hominis ad unum tantum, nisi secundum
rationem. Ergo diversitas operationum non causat diversitatem virtutum
moralium.


&

 さらに、働きの最大の差異は一つの善へ秩序付けられる働きと、多くのもの
 の善へと秩序付けられる働きの違いであるように思われる。しかるにこの違
 いは道徳的徳を異なるものとしない。というのも哲学者が『ニコマコス倫理
 学』第5巻で、人間の行為を共通の善へと秩序付ける法の正義は、人間の行為
 をただ一つのものへ秩序付ける徳と、ただ概念においてのみ異なると述べて
 いるからである。ゆえに働きの違いは道徳的な徳の違いを生み出さない。

 
\\



3.~{\scshape Praeterea}, si sunt diversae virtutes morales circa
diversas operationes, oporteret quod secundum diversitatem
operationum, esset diversitas virtutum moralium. Sed hoc patet esse
falsum, nam ad iustitiam pertinet in diversis generibus commutationum
rectitudinem statuere, et etiam in distributionibus, ut patet in V
{\itshape Ethic}. Non ergo diversae virtutes sunt diversarum
operationum.


&

さらに、もしさまざまな働きを巡ってさまざまな道徳的徳があるならば、さま
ざまな働きに即してさまざまな道徳的徳がなければならなかったであろう。し
かしこれが偽であることは明らかである。なぜなら、『ニコマコス倫理学』第
5巻で明らかなとおり、正義には、さまざまな類における交換の正しさを打ち
立てることが属し、これは分配においてもそうだからである。ゆえに、異なる
働きに属する異なる徳があるわけではない。
 
\\



{\scshape Sed contra est} quod religio est alia virtus a pietate,
quarum tamen utraque est circa operationes quasdam.


&

しかし反対に、敬神は報恩とは別の徳である。しかしこれらのどちらもある種
の働きを巡ってある。
 
\\

{\scshape Respondeo dicendum quod} omnes virtutes morales quae sunt
circa operationes, conveniunt in quadam generali ratione iustitiae,
quae attenditur secundum debitum ad alterum, distinguuntur autem
secundum diversas speciales rationes. Cuius ratio est quia in
operationibus exterioribus ordo rationis instituitur sicut dictum est,
non secundum proportionem ad affectionem hominis, sed secundum ipsam
convenientiam rei in seipsa; secundum quam convenientiam accipitur
ratio debiti, ex quo constituitur ratio iustitiae, ad iustitiam enim
pertinere videtur ut quis debitum reddat.


&

 解答する。以下のように言われるべきである。働きを巡るすべての道徳的徳
 は正義のある種の類的な性格、すなわち他者に対する債務に即して見出され
 る性格において一致するが、さまざまな種的な性格に即しては区別される。
 その理由は以下の通りである。上述の如く、外的な働きにおいて、理性の秩
 序は人間の情動への関係に即してではなく、それ自体における事物の適合性
 それ自体に即して打ち立てられ、その適合性に即して負債の性格が理解され、
 その負債に基づいて正義の性格が打ち立てられる。というのも、正義には、
 人が負債を返済することが属すると思われるからである。
 
\\

Unde omnes huiusmodi virtutes quae sunt circa operationes, habent
aliquo modo rationem iustitiae. Sed debitum non est unius rationis in
omnibus, aliter enim debetur aliquid aequali, aliter superiori, aliter
minori; et aliter ex pacto, vel ex promisso, vel ex beneficio
suscepto. Et secundum has diversas rationes debiti, sumuntur diversae
virtutes, puta religio est per quam redditur debitum Deo; pietas est
per quam redditur debitum parentibus vel patriae; gratia est per quam
redditur debitum benefactoribus; et sic de aliis.

&

 このことから、働きを巡るこのようなすべての徳は、何らかのかたちで正義
 の性格をもつ。しかし、負債はすべてのものにおいて一つの性格に属するわ
 けではない。というのも、同等の者と上位の者と下位の者には、違うかたち
 で負債を負うからである。また、契約に基づくもの、約束に基づくもの、与
 えられた善意に基づくものでは、負債の負い方が異なる。そして負債のこの
 異なる性格に応じて、さまざまな徳が区別される。たとえば、敬神によって
 神に対する負債が返され、報恩によって両親や祖国への負債が返される。感
 謝は、それによって善意を与えた人々へ負債が返され、その他についても同
 様である。
 
\\



{\scshape Ad primum ergo dicendum} quod iustitia proprie dicta est una
specialis virtus, quae attendit perfectam rationem debiti, quod
secundum aequivalentiam potest restitui. Dicitur tamen et ampliato
nomine iustitia, secundum quamcumque debiti redditionem. Et sic non
est una specialis virtus.


&

 第一異論に対しては、それゆえ、以下のように言われるべきである。正義は、
 厳密に言えば一つの種的な徳であり、等価に即して返済されうる負債の完全な性
 格を追究する。しかし、広義では、正義はなんであれ負債の返済に即して語ら
 れる。この意味では一つの種的な徳ではない。

 
\\



{\scshape Ad secundum dicendum} quod iustitia quae intendit bonum
commune, est alia virtus a iustitia quae ordinatur ad bonum privatum
alicuius, unde et ius commune distinguitur a iure privato; et Tullius
ponit unam specialem virtutem, pietatem, quae ordinat ad bonum
patriae. Sed iustitia ordinans hominem ad bonum commune, est generalis
per imperium, quia omnes actus virtutum ordinat ad finem suum,
scilicet ad bonum commune. Virtus autem secundum quod a tali iustitia
imperatur, etiam iustitiae nomen accipit. Et sic virtus a iustitia
legali non differt nisi ratione, sicut sola ratione differt virtus
operans secundum seipsam, et virtus operans ad imperium alterius.


&

 第二異論に対しては以下のように言われるべきである。共通善を意図する正
 義は、ある人の私的な善へと秩序付けられる正義とは異なる徳である。この
 ことから、公法は私法から区別される。そしてキケロは一つの種的な徳であ
 る報恩を置いて、これが祖国の善へと秩序付けるとしている。しかし人間を
 共通善へ秩序付ける正義は、命令によって類的である。なぜなら、徳のすべ
 ての作用は自らの目的へと、すなわち共通善へと秩序付けるからである。こ
 れに対して、徳は、そのような正義によって命じられるかぎりにおいて、や
 はり正義という名を受け取る。このようにして、徳は法的な正義と概念にお
 いてのみ異なる。ちょうど、それ自身において作用する徳と、他のものの命
 令へと作用する徳が概念においてのみ異なるように。
 
\\



{\scshape Ad tertium dicendum} quod in omnibus operationibus ad
iustitiam specialem pertinentibus, est eadem ratio debiti. Et ideo est
eadem virtus iustitiae, praecipue quantum ad commutationes. Forte enim
distributiva est alterius speciei a commutativa, sed de hoc infra
quaeretur.


&

第三異論に対しては以下のように言われるべきである。種的な正義に属するす
べての働きにおいて、負債という同一の性格がある。ゆえに、とくに交換に関
して、正義という同一の徳がある。おそらく、配分的正義は交感的正義とは異
なる種であろうが、これについては後に探究される。
 


\end{longtable}
\newpage


\rhead{a.~4}
\begin{center}
{\Large {\bf ARTICULUS QUARTUS}}\\
{\large UTRUM CIRCA DIVERSAS PASSIONES\\DIVERSAE SINT VIRTUTES MORALES}\\
 {\Large 第四項\\さまざまな情念を巡って道徳的徳はさまざまであるか}
\end{center}

\begin{longtable}{p{21em}p{21em}}
 Ad quartum sic proceditur. Videtur quod circa diversas passiones non
 sint diversae virtutes morales. Eorum enim quae conveniunt in
 principio et fine, unus est habitus, sicut patet maxime in
 scientiis. Sed omnium passionum unum est principium, scilicet amor;
 et omnes ad eundem finem terminantur, scilicet ad delectationem vel
 tristitiam; ut supra habitum est. Ergo circa omnes passiones est una
 tantum moralis virtus.

 
&

 第四項の問題へ、議論は以下のように進められる。さまざまな情念を巡って
 道徳的徳はさまざまであるわけではない、と思われる。理由は以下の通り。
 始まりと終わりが同じものにおいては一つの習慣がある。たとえばそれは学
 知において最大限に明らかである。しかるにすでに述べられたとおり、すべ
 ての情念には一つの根源(始まり)、すなわち愛が属し、そしてすべての情
 念は同じ目的(終わり)、すなわち快楽または悲しみで終わる。ゆえにすべ
 ての情念を巡ってただ一つの道徳的徳がある。
 
\\

2.{\scshape Praeterea}, si circa diversas passiones essent diversae
virtutes morales, sequeretur quod tot essent virtutes morales quot
passiones. Sed hoc patet esse falsum, quia circa oppositas passiones
est una et eadem virtus moralis, sicut fortitudo circa timores et
audacias, temperantia circa delectationes et tristitias. Non ergo
oportet quod circa diversas passiones sint diversae virtutes morales.

 
&

 さらに、もしさまざまな情念を巡って道徳的徳がさまざまであったならば、
 情念の数だけ道徳的徳があっただろう。しかしこれが偽であることは明らか
 である。なぜなら対立する情念を巡っては一つで同一の道徳的徳があるから
 である。たとえば勇気は怖れと豪胆を巡ってあるし、節制は快楽と悲しみを
 巡ってある。ゆえにさまざまな情念を巡って道徳的徳がさまざまである必要
 はない。

 
\\




3.{\scshape Praeterea}, amor, concupiscentia et delectatio sunt
passiones specie differentes, ut supra habitum est. Sed circa omnes
has est una virtus, scilicet temperantia. Ergo virtutes morales non
sunt diversae circa diversas passiones.
 
&

 さらに、前に述べられたとおり\footnote{Q.23, a.4.}、愛、欲情、快楽は種において異なる情念で
 ある。しかるにこのすべてについて一つの徳、すなわち節制がある。ゆえに
 道徳的徳はさまざまな情念を巡ってさまざまであるわけではない。

 
\\




{\scshape Sed contra est} quod fortitudo est circa timores et
audacias; temperantia circa concupiscentias; mansuetudo circa iras; ut
dicitur in III et IV {\itshape Ethic}.
 
&

 しかし反対に、『ニコマコス倫理学』第3巻と第4巻で言われるように、勇気
 は怖れと豪胆にかんしてあり、節制は欲情にかんして、また温和は怒り
 にかんしてある。

 
\\


 {\scshape Respondeo dicendum} quod non potest dici quod circa omnes
 passiones sit una sola virtus moralis, sunt enim quaedam passiones ad
 diversas potentias pertinentes; aliae namque pertinent ad
 irascibilem, aliae ad concupiscibilem, ut supra dictum est. 
&

 解答する。以下のように言われるべきである。すべての情念を巡ってただ一
つの道徳的徳があると言われることはできない。なぜなら、さまざまな能力に
属する情念があるからである。たとえば前に述べられたとおり
\footnote{Q.23, a.1.}、ある情念は怒情的部分に、別の情念は欲情的部分に
属するといったように。
 
 
\\

Nec tamen oportet quod omnis diversitas passionum sufficiat ad
virtutes morales diversificandas.  Primo quidem, quia quaedam
passiones sunt quae sibi opponuntur secundum contrarietatem, sicut
gaudium et tristitia, timor et audacia, et alia huiusmodi. Et circa
huiusmodi passiones sic oppositas, oportet esse unam et eandem
virtutem. 

 &

しかしまた、情念のすべての異なりが、道徳的徳を異なるものとするのに十分
である必要もない。第一に、ある情念は、反対性において自らに対立する。た
とえば喜びと悲しみ、怖れと豪胆、その他そのようなもののように。そしてこ
のように対立するこのような情念を巡っては、一つで同一の徳がなければなら
ない。
 
\\

Cum enim virtus moralis in quadam medietate consistat,
medium in contrariis passionibus secundum eandem rationem instituitur,
sicut et in naturalibus idem est medium inter contraria, ut inter
album et nigrum.

 &

 なぜなら、道徳的徳はある種の中間において成立するので、相反する情念に
 おける中間は同一の性格において成り立つからである。たとえば、自然的事
 物においても、白と黒のあいだのように、相反するものの中間は同一のもの
 である。

 
 \\

 

 Secundo, quia diversae passiones inveniuntur secundum eundem modum
 rationi repugnantes, puta secundum impulsum ad id quod est contra
 rationem; vel secundum retractionem ab eo quod est secundum
 rationem. 

&

 第二に、同一の仕方で理性に反するさまざまな情念が見出されるからである。
 たとえば、理性に反することへの衝動に即して、あるいは、理性に従うこと
 からの撤退に即してのように。

 
\\


 Et ideo diversae passiones concupiscibilis non pertinent ad diversas
 virtutes morales, quia earum motus secundum quendam ordinem se
 invicem consequuntur, utpote ad idem ordinati, scilicet ad
 consequendum bonum, vel ad fugiendum malum; sicut ex amore procedit
 concupiscentia, et ex concupiscentia pervenitur ad delectationem.  Et
 eadem ratio est de oppositis, quia ex odio sequitur fuga vel
 abominatio, quae perducit ad tristitiam.
 
 &

 このため、欲情的部分に属するさまざまな情念はさまざまな道徳的徳に属さ
 ない。なぜなら、それらの運動はある種の秩序に即して、同じものへ、すな
 わち善を獲得することや悪を避けることへと秩序付けられているかのように
 して、相互に伴うからである。たとえば愛から欲情が生じ、欲情から快楽に
 至るように。 そしてこれは対立するものについても同じことが言える。なぜ
 なら、憎しみには逃げることや嫌悪が後続し、これは悲しみへと導くからで
 ある。

 \\

 --- Sed passiones irascibilis non sunt unius ordinis, sed ad diversa
 ordinantur, nam audacia et timor ordinantur ad aliquod magnum
 periculum; spes et desperatio ad aliquod bonum arduum; ira autem ad
 superandum aliquod contrarium quod nocumentum intulit.


 
 
&

 しかし怒情的部分に属する情念は一つの秩序に属さず、様々なものへと秩序
 付けられる。すなわち豪胆と怖れは何らかの大きな危険へ、希望と絶望は何
 らかの困難な善へ、さらに怒りは害をもたらす相反する何かを克服すること
 へと秩序付けられる。

 
\\

Et ideo circa
 has passiones diversae virtutes ordinantur, utpote temperantia circa
 passiones concupiscibilis; fortitudo circa timores et audacias;
 magnanimitas circa spem et desperationem; mansuetudo circa iras.

 &

 ゆえに、これらの情念を巡ってはさまざまな徳が秩序付けられるのであり、
 たとえば、節制は欲情的部分の情念を巡って、勇気は怖れと豪胆を巡って、
 高邁は希望と絶望を巡って、温和は怒りを巡ってある。
 
 \\
 


 {\scshape Ad primum ergo dicendum} quod omnes passiones conveniunt in
 uno principio et fine communi, non autem in uno proprio principio seu
 fine. Unde hoc non sufficit ad unitatem virtutis moralis.
 
&

 第一異論に対しては、それゆえ、以下のように言われるべきである。すべて
 の情念は一つの共通の始まりと終わりにおいて一致するが、一つの固有の始
 まりや終わりにおいて一致するわけではない。したがってそれは道徳的徳の
 一性には十分でない。
 
\\

 {\scshape Ad secundum dicendum} quod, sicut in naturalibus idem est
 principium quo receditur ab uno principio, et acceditur ad aliud; et
 in rationalibus est eadem ratio contrariorum, ita etiam virtus
 moralis, quae in modum naturae rationi consentit, est eadem
 contrariarum passionum.
 
&

 第二異論に対しては以下のように言われるべきである。自然的事物において、
 一つの根源から離れて別の根源へ近づくところの根源は同一であるように
 \footnote{ひとつの端から離れさせる根源と、別の端へと近づかせる根源は、
 それが一つの運動であるときなどでは、同一の根源でありうる。}、また理性
 的な事柄においても相反するものどもに属する同一の性格があるように、道
 徳的徳も、自然本性の仕方で理性に同意するのだから、相反する情念に同一
 のものが属する。

 \\

 Ad tertium dicendum quod illae tres passiones ad idem obiectum
 ordinantur secundum quendam ordinem, ut dictum est. Et ideo ad eandem
 virtutem moralem pertinent.
 
&

 第三異論に対しては以下のように言われるべきである。上述の如く、それら
 三つの情念は、ある種の秩序に即して同一の対象へと秩序付けられている。
 それゆえ、同一の道徳的徳が属する。

 
\\



\end{longtable}
\newpage




\rhead{a.~5}
\begin{center}
{\Large {\bf ARTICULUS QUINTUS}}\\
{\large UTRUM VIRTUTES MORALES DISTINGUANTUR\\SECUNDUM DIVERSA OBIECTA PASSIONUM}\\
{\footnotesize II {\itshape Ethic.}, lect.8, 9.}\\
{\Large 第五項\\道徳的徳は情念のさまざまな対象に即して区別されるか}
\end{center}

\begin{longtable}{p{21em}p{21em}}

 {\scshape Ad quintum sic proceditur}. Videtur quod virtutes morales
 non distinguantur secundum obiecta passionum. Sicut enim sunt obiecta
 passionum, ita sunt obiecta operationum. Sed virtutes morales quae
 sunt circa operationes, non distinguuntur secundum obiecta
 operationum, ad eandem enim virtutem iustitiae pertinet emere vel
 vendere domum, et equum. Ergo etiam nec virtutes morales quae sunt
 circa passiones, diversificantur per obiecta passionum.

 
 &

 第五項の問題へ、議論は以下のように進められる。道徳的徳は情念の対象に
 即して区別されるのではないと思われる。理由は以下の通り。情念の対象の
 ありかたは、働きの対象のあり方と同じである。しかるに働きを巡ってある
 道徳的徳は働きの対象に即して区別されるのではない。たとえば正義という
 同一の徳が、家の売買と馬の売買に属する。ゆえに情念を巡ってある道徳的
 徳もまた、情念の対象によってさまざまになるわけではない。

\\




2.~{\scshape Praeterea}, passiones sunt quidam actus vel motus
appetitus sensitivi. Sed maior diversitas requiritur ad diversitatem
habituum, quam ad diversitatem actuum. Diversa igitur obiecta quae non
diversificant speciem passionis, non diversificabunt speciem virtutis
moralis. Ita scilicet quod de omnibus delectabilibus erit una virtus
moralis, et similiter est de aliis.
 
&

 さらに、情念は感覚的欲求のある種の作用ないし運動である。しかるに習慣
 の異なりには、作用の異なりよりも大きな異なりが必要とされる。ゆえに情
 念の種を異なるものにしない異なる対象は道徳的徳の種を異なるものにする
 ことはない。すなわちこのようにして、すべての快適なものに一つの道徳的
 徳があり、その他についても同様であろう。
 
\\




3.~{\scshape Praeterea}, magis et minus non diversificant speciem. Sed
diversa delectabilia non differunt nisi secundum magis et minus. Ergo
omnia delectabilia pertinent ad unam speciem virtutis. Et eadem
ratione, omnia terribilia, et similiter de aliis. Non ergo virtus
moralis distinguitur secundum obiecta passionum.
 
&

 さらに、より大、より小ということは種を別のものにしない。しかしさまざ
 まな快楽の対象は、より大より小に即してのみ異なる。ゆえにすべての快楽
 は徳の一つの種に属する。同じ理由で、すべての恐怖の対象やその他のもの
 についても同様である。ゆえに道徳的徳は情念の対象に即して区別されるわ
 けではない。
 
\\




4.~{\scshape Praeterea}, sicut virtus est operativa boni, ita est
impeditiva mali. Sed circa concupiscentias bonorum sunt diversae
virtutes, sicut temperantia circa concupiscentias delectationum
tactus, et eutrapelia circa delectationes ludi. Ergo etiam circa
timores malorum debent esse diversae virtutes.
 
&

 さらに、徳は善に作用しうるものであるように、悪を阻害しうるものである。
 しかるに善への欲情を巡ってはさまざまな徳がある。たとえば触覚の喜びへ
 の欲情を巡っては節制が、遊びの快楽を巡っては勤勉があるように。ゆえに
 悪への怖れを巡っても、さまざまな徳がなければならない。

 
\\




{\scshape Sed contra est} quod castitas est circa delectabilia
venereorum; abstinentia vero est circa delectabilia ciborum; et
eutrapelia circa delectabilia ludorum.
 
&

 しかし反対に、貞潔はビーナス的な(性的な)快楽に関わり、禁欲は食事の
 快楽に、勤勉は遊びの快楽にかかわる。

 
\\



 Respondeo dicendum quod perfectio virtutis ex ratione dependet,
 perfectio autem passionis, ex ipso appetitu sensitivo. Unde oportet
 quod virtutes diversificentur secundum ordinem ad rationem, passiones
 autem, secundum ordinem ad appetitum.



&

 解答する。以下のように言われるべきである。徳の完全性は理性に依存する
 が、情念の完全性は感覚的欲求自体に依存する。このことから、徳は理性へ
 の秩序に即して異なるものとなるが、情念は欲求への秩序に即して異なるも
 のとならねばならない。
 
\\

 Obiecta igitur passionum, secundum quod diversimode comparantur ad
 appetitum sensitivum, causant diversas passionum species, secundum
 vero quod comparantur ad rationem, causant diversas species
 virtutum. 

&

 それゆえ、情念の対象は、さまざまな仕方で感覚的欲求に関係する限りにお
 いて、情念のさまざまな種の原因となるが、理性へ関係する限りにおいては、
 徳のさまざまな種の原因となる。

 
\\

 Non est autem idem motus rationis, et appetitus sensitivi. Unde nihil
 prohibet aliquam differentiam obiectorum causare diversitatem
 passionum, quae non causat diversitatem virtutum, sicut quando una
 virtus est circa multas passiones, ut dictum est, et aliquam etiam
 differentiam obiectorum causare diversitatem virtutum, quae non
 causat diversitatem passionum, cum circa unam passionem, puta
 delectationem, diversae virtutes ordinentur.

&


 しかし理性の運動と感覚的欲求の運動は同じではない。このことから、ある
 対象の差異が情念の異なりの原因とはなるが、徳の異なりの原因とはならな
 いということがあっても差し支えない。たとえば、すでに述べられたとおり、
 一つの徳が多くの情念に関してあるというように。また、ある対象の差異が
 徳の異なりの原因となるが、それは情念の異なりの原因とならないというこ
 ともある。それは、一つの情念、たとえば快楽を巡って、さまざまな徳が秩
 序付けられる場合のようにである。

 
\\

 Et quia diversae passiones ad diversas potentias pertinentes, semper
 pertinent ad diversas virtutes, ut dictum est; ideo diversitas
 obiectorum quae respicit diversitatem potentiarum, semper
 diversificat species virtutum; puta quod aliquid sit bonum absolute,
 et aliquid bonum cum aliqua arduitate.

&

 そして、すでに述べられたとおり、さまざまな能力に属するさまざまな情念
 は、常にさまざまな徳に属するので、能力の異なりに関係する対象の異なり
 は、常に徳の異なりを生み出す。たとえば、あるものは無条件に善だが、別
 のものはある険しさを伴って善であるような場合である。

 
\\


 Et quia ordine quodam ratio inferiores hominis partes regit, et etiam
 se ad exteriora extendit; ideo etiam secundum quod unum obiectum
 passionis apprehenditur sensu vel imaginatione, aut etiam ratione; et
 secundum etiam quod pertinet ad animam, corpus, vel exteriores res;
 diversam habitudinem habet ad rationem; et per consequens natum est
 diversificare virtutes.

&

 そして理性はある秩序によって人間の下位の部分を支配し、さらに外的なも
 のへと自らを拡張させるので、一つの情念の対象が感覚や想像力によって、
 さらには理性によってとらえられる限りにおいて、また、魂や身体や外的な
 事物に属する限りにおいて、理性に対してさまざまな関係をもつ。その結果、
 さまざまな徳を本性的に生み出す。
 
\\

 Bonum igitur hominis, quod est obiectum amoris, concupiscentiae et
 delectationis, potest accipi vel ad sensum corporis pertinens; vel ad
 interiorem animae apprehensionem. Et hoc, sive ordinetur ad bonum
 hominis in seipso, vel quantum ad corpus vel quantum ad animam; sive
 ordinetur ad bonum hominis in ordine ad alios. Et omnis talis
 diversitas, propter diversum ordinem ad rationem, diversificat
 virtutem.

&

ゆえに、愛と欲情と快楽の対象である人間の善は、身体の感覚に属するものと
して、あるいは魂の内的な把握に属するものとして捉えられうる。そしてこれ
は、人間の善それ自体、あるいは身体に関する善、あるいは魂に関する善、あ
るいは、他の人への秩序における人間の善へ、秩序付けられる。これらすべて
異なりは、理性への秩序の異なりのために、徳を異なるものとする。
 
\\

 Sic igitur si consideretur aliquod bonum, si quidem sit per sensum
 tactus apprehensum, et ad consistentiam humanae vitae pertinens in
 individuo vel in specie, sicut sunt delectabilia ciborum et
 venereorum; erit pertinens ad virtutem temperantiae.



 &

 ゆえにこのようにして、もしある善が考察され、それが触覚の感覚によって捉えられ、個人または種における人間の生命の存続に属するならば、たとえば食事やビーナス的な事の快楽として捉えられるならば、節制の徳に属するものとなるだろう。

\\



 Delectationes autem aliorum sensuum, cum non sint vehementes, non
 praestant aliquam difficultatem rationi, et ideo circa eas non
 ponitur aliqua virtus, quae {\itshape est circa difficile, sicut et
 ars}, ut dicitur in II {\itshape Ethic}.

 &

 これに対して、他の感覚の快楽は激しくなく、理性に何らかの困難を提示す
 ることはないので、それらについては何らの徳もない。『ニコマコス倫理学』
 第2巻で言われるように、「技術と同様に、徳は困難なものについてある」か
 らである。

 \\
 

 Bonum autem non sensu, sed interiori virtute apprehensum, ad ipsum
 hominem pertinens secundum seipsum, est sicut pecunia et honor;
 quorum pecunia ordinabilis est de se ad bonum corporis; honor autem
 consistit in apprehensione animae. Et haec quidem bona considerari
 possunt vel absolute, secundum quod pertinent ad concupiscibilem; vel
 cum arduitate quadam, secundum quod pertinent ad irascibilem.

&

 他方、感覚的ではなく内的なちからによって捉えられ、それ自身における人
 間に属する善は、お金とか名誉のようなものである。このうちお金はそれ自
 体身体の善へと秩序づけられうるが、名誉は魂の把握において存する。そし
 てこれらの善は、無条件的に、欲情的部分に属する限りにおいて考察されう
 るし、あるいは、ある種の険しさを伴って、怒情的部分に属する限りにおい
 ても考察されうる。
 
\\

 Quae quidem distinctio non habet locum in bonis quae delectant
 tactum, quia huiusmodi sunt quaedam infima, et competunt homini
 secundum quod convenit cum brutis. Circa bonum igitur pecuniae
 absolute sumptum, secundum quod est obiectum concupiscentiae vel
 delectationis aut amoris, est liberalitas. Circa bonum autem
 huiusmodi cum arduitate sumptum, secundum quod est obiectum spei, est
 magnificentia.


&

この区別は触覚を喜ばせる善の中にはない。なぜなら、そのようなものはある
種最低のものであり、非理性的動物と一致するかぎりでの人間に適合するから
である。ゆえに、欲情的部分や快楽や愛の対象である限りにおいて、無条件的
に理解された金銭の善を巡っては、「気前のよさ」がある。他方、希望の対象
であるかぎりにおいて険しさを伴って理解されたそのような善を巡っては「豪
毅」がある。
 
\\



 Circa bonum vero quod est honor, si quidem sit absolute sumptum,
 secundum quod est obiectum amoris, sic est quaedam virtus quae
 vocatur philotimia, idest amor honoris. Si vero cum arduitate
 consideretur, secundum quod est obiectum spei, sic est
 magnanimitas. Unde liberalitas et philotimia videntur esse in
 concupiscibili, magnificentia vero et magnanimitas in irascibili.

&

一方、名誉の善を巡っては、それが愛の対象として無条件的に理解される場合
には、「大志」つまり名誉への愛と呼ばれる徳がある。他方、希望の対象で
ある限りにおいて、険しさを伴って考察される場合には、「高邁」がある。こ
のことから、「気前のよさ」と「大志」は欲情的部分に、そして「豪毅」と
「高邁」は怒情的部分にあるように思われる。
  
\\

 Bonum vero hominis in ordine ad alium, non videtur arduitatem habere,
 sed accipitur ut absolute sumptum, prout est obiectum passionum
 concupiscibilis. Quod quidem bonum potest esse alicui delectabile
 secundum quod praebet se alteri vel in his quae serio fiunt, idest in
 actionibus per rationem ordinatis ad debitum finem; vel in his quae
 fiunt ludo, idest in actionibus ordinatis ad delectationem tantum,
 quae non eodem modo se habent ad rationem sicut prima.

&

また、他人への秩序における人間の善は、険しさを持つようには見えず、欲情
的部分の情念の対象として無条件的に受け取られたものとして理解される。こ
の善はある人にとって他の人に自らを与える限りにおいて快適でありうるが、
それは真面目に行われること、すなわち理性によってしかるべき目的へと秩序
付けられた行為においても、あるいは遊びで行われること、すなわち快楽だけ
に秩序付けられた行為においてもそうである。ただし、後者は前者と同じよう
に理性に関係するわけではない。
 
\\


 In seriis autem se exhibet aliquis alteri dupliciter. Uno modo, ut
 delectabilem decentibus verbis et factis, et hoc pertinet ad quandam
 virtutem quam Aristoteles nominat amicitiam; et potest dici
 affabilitas. Alio modo praebet se aliquis alteri ut manifestum, per
 dicta et facta, et hoc pertinet ad aliam virtutem, quam nominat
 veritatem. Manifestatio enim propinquius accedit ad rationem quam
 delectatio; et seria quam iocosa. Unde et circa delectationes ludorum
 est alia virtus, quam philosophus eutrapeliam nominat.


&

 さて、真面目な事柄において、人は自らを二つの仕方で他者に示す。一つに
 は、上品な言葉と行為によって気持ちのいい人として示すのであり、これは
 アリストテレスが「友愛」と名付けるような徳に属する。またこれは、「親
 しみやすさ」とも言われうる。もう一つには、人は言われたことと為された
 ことによって、自らを明らかなものとして他人に示す。そしてこれは他の徳
 に属し、それを「真実さ」と名付ける。実際、明らかにすることは気持ちの
 よさよりも、そして真面目なことはふざけたことよりも理性に近づいている。
 したがって、遊びの快楽を巡っても別の徳があり、それを哲学者は「勤勉」
 と名付けている。

 
\\



 Sic igitur patet quod, secundum Aristotelem, sunt decem virtutes
 morales circa passiones, scilicet fortitudo, temperantia,
 liberalitas, magnificentia, magnanimitas, philotimia, mansuetudo,
 amicitia, veritas et eutrapelia. Et distinguuntur secundum diversas
 materias vel secundum diversas passiones; vel secundum diversa
 obiecta. Si igitur addatur iustitia, quae est circa operationes,
 erunt omnes undecim.

 
&

 それゆえ、このようにアリストテレスによれば情念を巡る十個の道徳的徳が
 ある。すなわち、勇気、節制、気前の善さ、豪毅、高邁、大志、温和、友愛、
 真実、勤勉である。そしてこれらはさまざまな質料(対象)に即して、ある
 いはさまざまな情念に即して区別される。ゆえに、働きを巡ってある正義が
 加えられるならば、全部で十一の道徳的徳があることになる。

 
 
\\



 {\scshape Ad primum ergo dicendum} quod omnia obiecta eiusdem
 operationis secundum speciem, eandem habitudinem habent ad rationem;
 non autem omnia obiecta eiusdem passionis secundum speciem, quia
 operationes non repugnant rationi, sicut passiones.
 
&

 第一異論に対しては、それゆえ、以下のように言われるべきである。種にお
 いて同一の働きに属するすべての対象は、理性に対して同一の関係を有する
 が、種において同一の情念に属するすべての対象はそうでない。なぜなら、
 働きは理性に反しないが、情念は反することがあるからである。
 
\\

 {\scshape Ad secundum dicendum} quod alia ratione diversificantur
 passiones, et alia virtutes, sicut dictum est.
 
&

 第二異論に対しては以下のように言われるべきである。すでに述べられたと
 おり、情念と徳は、別の根拠によって様々なものになる。
 
\\

 {\scshape Ad tertium dicendum} quod magis et minus non diversificant
 speciem, nisi propter diversam habitudinem ad rationem.
 
&

 第三異論に対しては以下のように言われるべきである。より大、より小、と
 いうことは、理性に対する異なる関係のためにでないかぎり、種を異なるも
 のとしない。
 
\\


 {\scshape Ad quartum dicendum} quod bonum fortius est ad movendum
 quam malum quia malum non agit nisi virtute boni, ut Dionysius dicit,
 {\scshape iv} cap.~{\itshape de Div.~Nom}. Unde malum non facit
 difficultatem rationi quae requirat virtutem, nisi sit excellens,
 quod videtur esse unum in uno genere passionis. Unde circa iras non
 ponitur nisi una virtus, scilicet mansuetudo, et similiter circa
 audacias una sola, scilicet fortitudo. Sed bonum ingerit
 difficultatem, quae requirit virtutem, etiam si non sit excellens in
 genere talis passionis. Et ideo circa concupiscentias ponuntur
 diversae virtutes morales, ut dictum est.
 
&

 第四異論に対しては以下のように言われるべきである。ディオニュシウスが
 『神名論』第4章で言うように、善は悪よりも力強く動かす。なぜなら悪は善
 のちからによらなければ作用しないからである。このことから悪は徳を必要
 とするような困難を理性に与えない。ただし限度を超えたものでなければ。
 そのような悪は情念の一つの類において一つあると思われる。それゆえ怒り
 を巡っては一つの徳、すなわち温和があると考えられ、同様に豪胆を巡って
 もただ一つだけの徳、すなわち勇気があると考えられる。しかし善は徳を必
 要とする困難を、そのような情念の類において限度を超えたものでなくても
 生み出す。ゆえに、欲情的部分を巡って、上述の如く、さまざまな道徳的徳
 があると考えられる。

\end{longtable}


\newpage

\section{division of virtues}
1st try. To be revised.
\subsection{according to passions}
\schema
{\schemabox{passio}}
{
    {\schemabox{concupiscibilis --- TEMPERANTIA*}}
    {
	\schema
	{\schemabox{irascibilis}}
	{\schemabox{audacia -- timor ---FORTITUDO\\
		    spes -- desperatio --- MAGNANIMITAS**\\
		    ira --- MANSUETUDO}}
    }
}

\vspace{2em}

\subsection{according to the objects of passions}
\schema
{\schemabox{bonum hominis}}
{
    \schemabox{sensus corporis -- tactus --- TEMPERANTIA*}
    \schema{\schemabox{interior animae\\apprehensio}}
    {
	\schema{\schemabox{bonum hominis\\in seipso}}
	{
	    \schema{\schemabox{pecunia}}{\schemabox{LIBERALITAS(con.)\\MAGNIFICENTIA(ira.)}}
	    \schema{\schemabox{honor}}{\schemabox{PHILOTIMIA(con.)\\MAGNANIMITAS**(ira.)}}
	}
	\schemabox{quantum ad corpus}
	\schema{\schemabox{in ordine ad alios}}
	{
	    \schema{\schemabox{serio}}{\schemabox{ut delectabiils---AMICITIA\\ut manifestum---VERITAS}}
	    \schemabox{ludo --- EUTRAPELIA}
	}
    }
}

\vspace{2em}
\noindent
(*, ** --- duplication)


\end{document}


1. temperantia* 節制
2. fortitudo 勇気
3. magnanmitas* 高邁(志の高さ)
4. mansuetudo 温和
5. temperantia* 節制
6. liberalitas 気前の善さ
7. magnificentia 豪毅(物惜しみのなさ)
8. philotimia 大志
9. magnanimitas* 高邁(志の高さ)
10. amicitia 友愛
11. veritas 真実(正直さ)
12. eutrupelia 勤勉

*は重複

