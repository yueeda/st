\documentclass[10pt]{jsarticle}
\usepackage{okumacro}
\usepackage{longtable}
\usepackage[polutonikogreek,english,japanese]{babel}
\usepackage{amsmath}
\usepackage{latexsym}
\usepackage{color}

%----- header -------
\usepackage{fancyhdr}
\pagestyle{fancy}
\lhead{{\it Summa Theologiae} I, q.~3}
%--------------------


\bibliographystyle{jplain}
\title{{\bf Prima Pars}\\{\HUGE Summae Theologiae}\\Sancti Thomae
Aquinatis\\Quaestio Tertia\\{\bf De Dei Simplicitate}}
\author{Japanese translation\\by Yoshinori {\sc Ueeda}}
\date{Last modified \today}


%%%% コピペ用
%\rhead{a.~}
%\begin{center}
% {\Large {\bf }}\\
% {\large }\\
% {\footnotesize }\\
% {\Large \\}
%\end{center}
%
%\begin{longtable}{p{21em}p{21em}}
%
%&
%
%\\
%\end{longtable}
%\newpage



\begin{document}

\maketitle

\begin{center}
 {\Large 第三問\\神の単純性について}
\end{center}


\begin{longtable}{p{21em}p{21em}}

Cognito de aliquo an sit, inquirendum restat quomodo sit, ut sciatur de
eo quid sit. Sed quia de Deo scire non possumus quid sit, sed quid non
sit, non possumus considerare de Deo quomodo sit, sed potius quomodo non
sit. Primo ergo considerandum est quomodo non sit; secundo, quomodo a
nobis cognoscatur; tertio, quomodo nominetur. Potest autem ostendi de
Deo quomodo non sit, removendo ab eo ea quae ei non conveniunt, utpote
compositionem, motum, et alia huiusmodi. Primo ergo inquiratur de
simplicitate ipsius, per quam removetur ab eo compositio. Et quia
simplicia in rebus corporalibus sunt imperfecta et partes, secundo
inquiretur de perfectione ipsius; tertio, de infinitate eius; quarto, de
immutabilitate; quinto, de unitate. Circa primum quaeruntur octo. 


&

或るものについてそれが存在するかどうかが認識されたら、それについて何で
あるかが知られるように、それがどのように在るかを探求することが残されて
いる。しかし、私たちは神についてそれが何であるかを知ることができず、何
でないかを知りうるだけなので、私たちは神についてそれがどのようであるか
を考察することはできず、むしろどのようでないかを考察するのである。ゆえ
に第一に、神はどのようでないか、第二に、どのように私たちによって認識さ
れるか、第三に、どのように名づけられるかが考察されるべきである。さて、
神についてそれがどのようでないかは、神に適さないものを除去することによっ
て明らかにできる。たとえば、複合、運動、その他そのようなものを除去する
ことによってである。ゆえに、第一に神の単純性について考察され、それによっ
て神から複合が除去される。そして、物体的事物において、単純なものは不完
全であり部分であるので、第二に神の完全性について、第三に神の無限性につ
いて、第四に神の不変性について、第五に神の一性について探求される。第一
について八つのことが問われる。

\\

\begin{enumerate}
 \item utrum Deus sit corpus.
 \item utrum sit in eo compositio formae et materiae. 
 \item utrum sit in eo compositio quidditatis, sive
essentiae, vel naturae, et subiecti.
 \item utrum sit in eo compositio quae est ex essentia et esse.
 \item utrum sit in eo compositio generis et differentiae. 
 \item utrum sit in eo compositio subiecti et accidentis.
 \item utrum sit quocumque modo compositus, vel totaliter
simplex.
 \item utrum veniat in compositionem cum aliis.
\end{enumerate}

&

\begin{enumerate}
 \item 神は物体か。
 \item 神の中に、質料と形相の複合があるか。
 \item 神の中に、何性あるいは本質あるいは本性と基体の複合があるか。
 \item 神の中に、本質と存在からの複合があるか。
 \item 神の中に、類と種差の複合があるか。
 \item 神の中に、基体と付帯性の複合はあるか。
 \item 神は、何らかのかたちで複合されたものか、それとも全面的に単純か。
 \item 神は、他のものとの複合に入るか。
\end{enumerate}


\end{longtable}

\newpage
\rhead{a.~1}
\begin{center}
 {\Large {\bf ARTICULUS PRIMUS}}\\
 {\large UTRUM DEUS SIT CORPUS}\\
 {\footnotesize I {\it SCG.}, c.~20; II, c.~3; {\it Compend.~Theol.},
 c.~14.}\\
{\Large 第一項\\神は物体か}
\end{center}

\begin{longtable}{p{21em}p{21em}}


{\huge A}{\sc d primum sic proceditur}. Videtur quod Deus sit
corpus. Corpus enim est quod habet trinam dimensionem. Sed sacra
Scriptura attribuit Deo trinam dimensionem, dicitur enim {\it Iob}
{\sc xi} : {\it excelsior caelo est, et quid facies? Profundior
Inferno, et unde cognosces? Longior terra mensura eius, et latior
mari}. Ergo Deus est corpus.

& 

第一項の問題へ、議論は次のように進められる。神は物体であると思われる。
理由は以下の通り。物体とは三次元の広がりがあるものである。ところが、聖
書は神に三次元の広がりがあるとしている。たとえば、『ヨブ記』11章で、
「彼は天より高い。あなたに何ができるのか。彼は陰府より深い。あなたはど
うやって知るのか。彼の尺度は地よりも長く、海原よりも広い」
\footnote{「高い天に対して何ができる。深い陰府について何が分かる。神は
地の果てよりも遠く、海原よりも広いのに。」(11:8-9)}と言われている。

\\


2.~{\sc Praeterea}, omne figuratum est corpus, cum figura sit qualitas
circa quantitatem. Sed Deus videtur esse figuratus, cum scriptum sit
{\it Gen}.~{\sc i} : {\it Faciamus hominem ad imaginem et
similitudinem nostram :} figura enim imago dicitur, secundum illud
{\it Hebr}.~{\sc i} : {\it cum sit splendor gloriae, et figura
substantiae eius}, idest imago. Ergo Deus est corpus.

&

さらに、形があるものはすべて物体である。なぜなら形とは量をめぐる性質な
のだから\footnote{cf.~{\it ST} I, q.78 a.3 ad 2., {\it ST} I-II, q.49,
a.2 c.}。ところで、神に形があるように思われる。なぜなら『創世記』1章で、
「我々の像と類似に向けて、人間を造ろう」\footnote{「我々にかたどり、我々
に似せて、人を造ろう」(1:26)}と書かれているが、『ヘブライ人への手紙』
第1章の「(御子は)栄光の輝きであり、彼の実体の形、すなわち像なのだか
ら」\footnote{「御子は、神の栄光の反映であり、神の本質の完全な現れであっ
て、」(1:3)}によれば、「像」とは「形」のことである。ゆえに、神は物体で
あると思われる。

\\




3.~{\sc Praeterea}, omne quod habet partes corporeas, est corpus. Sed
Scriptura attribuit Deo partes corporeas, dicitur enim {\it Iob} {\sc
xl} : {\it si habes brachium ut Deus}; et in Psalmo : {\it oculi
domini super iustos}; et, {\it dextera domini fecit virtutem}. Ergo
Deus est corpus.

&

さらに、身体の部分を持つものは全て物体である。しかるに聖書は、神に身体
の部分があると言っている。たとえば、『ヨブ記』40章では「もしお前が神の
ような腕をもつならば」\footnote{おまえは神に劣らぬ腕をもち、神のような
声をもって雷鳴をとどろかせるのか。」(40:9)}と言われるし、『詩編』では
「主の眼は義人の上に」「主の右の手は御力を示した」と言われている。ゆえ
に、神は物体である。

\\


4.~{\sc Praeterea}, situs non convenit nisi corpori. Sed ea quae ad
situm pertinent, in Scripturis dicuntur de Deo, dicitur enim Isaiae
{\sc vi} : {\it vidi dominum sedentem;} et Isaiae {\sc iii} : {\it
stat ad iudicandum Dominus}. Ergo Deus est corpus.

&

さらに、「姿勢」というものは、物体にしかあてはまらない。ところが聖書に、
姿勢にかんすることが神について言われている。たとえば、『イザヤ書』6章
「わたしは、主が座っているのを見た」\footnote{「わたしは、高く天にある
御座に主が座しておられるのを見た。」(6:1)}と言われ、『イザヤ書』3章
「主は裁くために立つ」\footnote{「主は争うために構え、民を裁くために立
たれる。」(3:13)}と言われている。ゆえに、神は物体である。

\\


5.~{\sc Praeterea}, nihil potest esse terminus localis a quo vel ad
quem, nisi sit corpus vel aliquod corporeum. Sed Deus in Scriptura
dicitur esse terminus localis ut ad quem, secundum illud Psalmi, {\it
accedite ad eum, et illuminamini}; et ut a quo, secundum illud
Hierem.~{\sc xvii}: {\it recedentes a te in terra scribentur}. Ergo
Deus est corpus.

 &

さらに、どんなものでも物体かなにか物体的なものでなければ、そこからであ
るような場所の端(出発点)や、そこへである場所の端(到達点)にならない。
ところが、聖書の中で、かの『詩篇』「なんじら神に近づいて照らされよ」
\footnote{「主を仰ぎ見る人は光と輝き、辱めに顔を伏せることはない。」
(34:6)}によれば、神は、そこへであるような場所の端だと言われ、また、か
の『エレミヤ書』17章「あなたから離れ去る人々は、地に記されるだろう」
\footnote{「あなたを離れ去る者は、地下に行く者として記される。」
(17:12)}によれば、そこから、として言われている。ゆえに、神は物体である。

\\



{\sc Sed contra est} quod dicitur Ioan.~{\sc iv} : {\it Spiritus est Deus}.

&

しかし反対に、『ヨハネによる福音書』第4章で「神は霊である」
\footnote{「神は霊である。だから、神を礼拝する者は、霊と真理をもって礼
拝しなければならない。」(4:24)}と言われている。

\\

{\sc Respondeo dicendum} absolute Deum non esse corpus.  Primo quidem,
quia nullum corpus movet non motum: ut patet inducendo per
singula.Ostensum est autem supra quod Deus est primum movens immobile.
Unde manifestum est quod Deus non est corpus.

&

解答する。以下のように言われるべきである。神は絶対に物体ではない。第一
に、どんな物体も、動かされないで動かすことはない。これは、個々の事例か
ら帰納して明らかである。ところで、神が、不動の第一動者であることは、上
で示された。したがって、神が物体でないことは明らかである。

\\


Secundo, quia necesse est id quod est primum ens, esse in actu, et
nullo modo in potentia.  Licet enim in uno et eodem quod exit de
potentia in actum, prius sit potentia quam actus tempore, simpliciter
tamen actus prior est potentia: quia quod est in potentia, non
reducitur in actum nisi per ens actu.  Ostensum autem supra quod Deus
est primum ens.  Impossibile est igitur quod in Deo sit aliquid in
potentia.  Omne autem corpus est in potentia: quia continuum,
inquantum huiusmodi, divisibile est in infinitum.  Impossibile est
igitur quod Deum esse corpus.


&

第二に、第一の存在者であるものは、現実態にあり、決して可能態にないこと
が必然である。なぜなら、たしかに、可能態から現実態に出て行く一つの同一
のものにおいて、時間的に、可能態が現実態より先である。しかし、端的には、
現実態の方が可能態よりも先である。なぜなら、可能態にあるものは、現実態
にあるものによってでなければ、現実態に引き出されることはないからである。
ところで、先に、神は第一の存在者であることが示された。ゆえに、神の中に、
何かが可能態にあることは不可能である。ところが、物体は可能態にある。な
ぜなら、連続体は、連続体である限り、無限に分割可能だから。ゆえに、神が
物体であることは不可能である。

\\

Tertio, quia Deus est id quod est nobilissimum in entibus, ut ex
dictis patet.  Impossibile est autem aliquod corpus esse nobilissimum
in entibus.  Quia corpus aut est vivum, aut non vivum.  Corpus autem
vivum, manifestum est quod est nobilius corpore non vivo.  Corpus
autem vivum non vivit inquantum corpus, quia sic omne corpus viveret:
oportet igitur quod vivat per aliquid aliud, sicut corpus nostrum
vivit per animam.  Illud autem per quod vivit corpus, est nobilius
quam corpus.  Impossibile est igitur Deum esse corpus.

&

第三に、すでに述べられたことから明らかなように、神は、存在するものの中
で最も高貴なものである。しかし、何らかの物体が、存在するものの中で最も
高貴であることは不可能である。物体は、生きているか、生きてないかのどち
らかである。ところが、生きている物体は、生きていない物体よりも高貴であ
ることは、明らかである。また、物体は、物体である限りにおいて、生きてい
るのではない。なぜなら、もしそうだとすると、すべての物体が生きていただ
ろう。ゆえに、生きているものは、たとえば、私たちの肉体(=物体)が魂に
よって生きるように、そういう物体以外の何かによって生きるはずである。α
によって物体が生きるなら、αは物体よりも高貴である。ゆえに、神が物体で
あることは不可能である。

\\

{\sc Ad primum ergo dicendum} quod, sicut supra dictum est, sacra
Scriptura tradit nobis spiritualia et divina sub similitudinibus
corporalium. Unde, cum trinam dimensionem Deo attribuit, sub
similitudine quantitatis corporeae, quantitatem virtualem ipsius
designat, utpote per profunditatem, virtutem ad cognoscendum occulta;
per altitudinem, excellentiam virtutis super omnia; per longitudinem,
durationem sui esse; per latitudinem, affectum dilectionis ad
omnia. -- Vel, ut dicit Dionysius, cap.~{\sc ix} {\it de Div.~Nom.},
per profunditatem Dei intelligitur incomprehensibilitas ipsius
essentiae; per longitudinem, processus virtutis eius, omnia
penetrantis; per latitudinem vero, superextensio eius ad omnia,
inquantum scilicet sub eius protectione omnia continentur.

&

第一異論に対しては、それゆえ、次のように言われるべきである。上述のように、
聖書は、わたしたちに、霊のことや神のことを、物体の類似のもとで伝えてい
る。したがって、神に三次元を帰するときは、物体の量の類似のもとに、神の
力の量を示している。たとえば、深さによって、隠されたものを知る力を、ま
た、高さによって、万物に勝る力の卓越を、長さによって、その存在の持続を、
広さによって、万物に対する情愛を示している。あるいは、ディオニュシウス
が『神名論』9章で言うように、神の深さによって、神の本質が把握され得な
いことが、長さによって、万物を貫く神の力の発出が、広さによって、神の保
護のもとに万物が含まれる限りにおいて、万物への神の広がりが、知性認識さ
れる。

\\



{\sc Ad secundum dicendum} quod homo dicitur esse ad imaginem Dei, non
secundum corpus, sed secundum id quo homo excellit alia animalia,
unde, {\it Gen}.~{\sc i}, postquam dictum est: {\it Faciamus hominem
ad imaginem et similitudinem nostram}, subditur: {\it ut praesit
piscibus maris}, et cetera. Excellit autem homo omnia animalia quantum
ad rationem et intellectum. Unde secundum intellectum et rationem,
quae sunt incorporea, homo est ad imaginem Dei.


&

第二異論に対しては、次のように言われるべきである。人間が神の像に向けてある
と言われるのは、身体においてではなく、そこにおいて人間が他の動物に優越
するところのものにおいてである。したがって、『創世記』1章で、「わたし
たちの像と類似に向けて、人間を造ろう」と言われたあとで、「海の魚を治め
るために」云々と言われている。ところで、人間は、理性と知性にかんして、
他の動物に優越する。ゆえに、非物体的なものである知性と理性において、人
間は、神の像に向けてある。

\\




Ad tertium dicendum quod partes corporeae attribuuntur Deo in
Scripturis ratione suorum actuum, secundum quandam
similitudinem. Sicut actus oculi est videre, unde oculus de Deo
dictus, significat virtutem eius ad videndum modo intelligibili, non
sensibili. Et simile est de aliis partibus.


&

第三異論に対しては、次のように言われるべきである。聖書において身体の部
分が神に帰せられるのは、ある種の類似に従って、それらの働きのためにであ
る。たとえば、目の働きは見ることであるから、神について語られた目は、可
感的なしかたではなく、可知的なしかたで、神の見る力を表示する。他の部分
についても同様である。


\\



Ad quartum dicendum quod etiam ea quae ad situm pertinent, non
attribuuntur Deo nisi secundum quandam similitudinem, sicut dicitur
sedens, propter suam immobilitatem et auctoritatem; et stans, propter
suam fortitudinem ad debellandum omne quod adversatur.


&

第四異論に対しては、次のように言われるべきである。姿勢にかんすることも、
ある種の類似にもとづいてでなければ神に帰せられない。たとえば、神の不動
と権威のために、座っていると言われ、神の強さと、逆らうものすべてに対し
て戦うために、立っていると言われる。

\\


Ad quintum dicendum quod ad Deum non acceditur passibus corporalibus,
cum ubique sit, sed affectibus mentis, et eodem modo ab eo
receditur. Et sic accessus et recessus, sub similitudine localis
motus, designant spiritualem affectum.


&

第五異論に対しては、次のように言われるべきである。神は至る所に存在する
ので、物体的な歩みで神に近づくことはない。近づくのは精神の受動によって
であり、同じしかたで神から退く。このように、接近と後退は、場所的運動の
類似のもとで、霊的な受動を示している。




\end{longtable}



\newpage
\rhead{a.~2}

\begin{center}
 {\Large {\bf ARTICULUS SECUNDUS}}\\
 {\large UTRUM IN DEO SIT COMPOSITIO FORMAE ET MATERIAE}\\
 {\footnotesize I \textit{Sent.}, d.~35, a.~1; I \textit{SCG}, c.~17;
 \textit{Compend.~Theol.}, c.~28.}\\
 {\Large 第二項\\神の中に形相と質料の複合があるか}
\end{center}

\begin{longtable}{p{21em}p{21em}}

{\huge A}{\sc d secundum sic proceditur}. Videtur quod in Deo sit
compositio formae et materiae. Omne enim quod habet animam, est
compositum ex materia et forma, quia anima est forma corporis. Sed
Scriptura attribuit animam Deo, introducitur enim \textit{ad
Hebr.}~\textsc{x}, ex persona Dei: \textit{iustus autem meus ex fide
vivit; quod si subtraxerit se, non placebit animae meae}. Ergo Deus
est compositus ex materia et forma.


& 

第二項の問題へ、議論は次のように進められる。神の中に形相と質料の複合が
あると思われる。理由は以下の通り。すべて魂を持つものは質料と形相から複
合されている。なぜなら魂は身体の形相だからである。ところが、聖書は神に
魂があるとしている。『ヘブライ人への手紙』第10章では、神自身の言葉で
「私の正しい者は信仰によって生きる。もしひるむようなことがあれば、その
者は私の魂に気に入らない」\footnote{「わたしの正しいものは信仰によって
生きる。もしひるむようなことがあれば、そのものはわたしの心に適わない。」
(10:38)}と述べられているからである。ゆえに神は質料と形相から複合されて
いる。


\\


2.~\textsc{Praeterea}, ira, gaudium, et huiusmodi, sunt passiones
coniuncti, ut dicitur I \textit{de Anima}. Sed huiusmodi attribuuntur
Deo in Scriptura dicitur enim in Psalmo : \textit{iratus est furore
dominus in populum suum}. Ergo Deus ex materia et forma est
compositus.

& 


さらに、『デ・アニマ』第1巻で述べられているとおり、怒り、喜び、そのよ
うなものは、(質料と形相が)結びついたものの情念である。しかし、聖書で
はそのようなものが神にあるとされている。実際、『詩篇』には「主は主の民
に向かって激情によって怒った」\footnote{「主の怒りは民に向かって燃え上
がり」(106:40)}とある。ゆえに神は質料と形相から複合されている。

\\


3.~\textsc{Praeterea}, materia est principium individuationis. Sed Deus
 videtur esse individuum, non enim de multis praedicatur. Ergo est
 compositus ex materia et forma.

&

さらに、質料は個体化の根源である。しかるに神は個であるように思われる。
なぜなら、多くのものの述語にならないからである。ゆえに、神は質料と形相
から複合されている。

\\


\textsc{Sed contra}, omne compositum ex materia et forma est corpus,
quantitas enim dimensiva est quae primo inhaeret materiae. Sed Deus
non est corpus, ut ostensum est. Ergo Deus non est compositus ex
materia et forma.

&


しかし反対に、質料と形相から複合されているものはすべて物体である。なぜ
なら、次元量\footnote{「次元量」については以下を参照。(\textit{ST} I,
q.~42 a.~1 ad 1.) Ad primum ergo dicendum quod duplex est
quantitas. Una scilicet quae dicitur quantitas molis, vel quantitas
dimensiva, quae in solis rebus corporalibus est, unde in divinis
personis locum non habet. Sed alia est quantitas virtutis, quae
attenditur secundum perfectionem alicuius naturae vel formae, quae
quidem quantitas designatur secundum quod dicitur aliquid magis vel
minus calidum, inquantum est perfectius vel minus perfectum in
caliditate.}は、第一に質料に内属するものだからである。ところが神はすで
に示されたように物体ではない。ゆえに神は質料と形相から複合されていない。

\\



\textsc{Respondeo dicendum} quod impossibile est in Deo esse materiam.
Primo quidem, quia materia est id quod est in potentia.  Ostensum est
autem quod Deus est purus actus, non habens aliquid de potentialitate.
Unde impossibile est quod Deus sit compositus ex materia et forma.

&

解答する。以下のように言われるべきである。神の中に質料があることは不可
能である。第一に、質料は可能態にあるものである。ところが、神は純粋現実
態であり、可能態性をまったく持たないことが示された。したがって、神が質
料と形相から複合されていることは不可能である。


\\

Secundo, quia omne compositum ex materia et forma est perfectum et
bonum per suam formam, unde oportet quod sit bonum per
participationem, secundum quod materia participat formam.  Primum
autem quod est bonum et optimum, quod Deus est, non est bonum per
participationem, quia bonum per essentiam, prius est bono per
participationem.  Unde impossibile est quod Deus sit compositus ex
materia et forma.

&

第二に、質料と形相から複合されているものはすべてそれ自身の形相によって
完全であり良いものである。したがって、それは質料が形相を分有するかたち
の分有によって良いものでなければならない。しかるに、良い第一のもの、つ
まり最善のもの、つまり神は、分有による善ではない。なぜなら、本質による
善は分有による善に先行するからである。したがって、神が質料と形相から複
合されることは不可能である。
\\


Tertio, quia unumquodque agens agit per suam formam, unde secundum
quod aliquid se habet ad suam formam, sic se habet ad hoc quod sit
agens.  Quod igitur primum est et per se agens, oportet quod sit primo
et per se forma.  Deus autem est primum agens, cum sit prima causa
efficiens, ut ostensum est.  Est igitur per essentiam suam forma; et
non compositus ex materia et forma.

&


第三に、各々のものは自らの形相によって働く。したがってあるものの自己の
形相に対する関係は、働くものであることへの関係に等しい。したがって、そ
れ自体で働く第一のものは、第一に、それ自体によって形相でなければならな
い。しかるに神は第一の作用者である。なぜなら、すでに示されたように、第
一作出因なのだから。ゆえに、神は自らの本質によって形相であり、質料と形
相から複合されたものではない。

\\




\textsc{Ad primum ergo dicendum} quod anima attribuitur Deo per
similitudinem actus. Quod enim volumus aliquid nobis, ex anima nostra
est, unde illud dicitur esse placitum animae Dei, quod est placitum
voluntati ipsius.

&

第一異論に対しては、それゆえ、次のように言われるべきである。神に魂があ
るとされるのは働きの類似性によってである。私たちがなにかを自分のために
欲するということは私たちの魂から生じる。したがって、神の意志に適うこと
が、神の魂に適うと言われるのである。


\\

\textsc{Ad secundum dicendum} quod ira et huiusmodi attribuuntur Deo
secundum similitudinem effectus, quia enim proprium est irati punire,
ira eius punitio metaphorice vocatur.

&

第二異論に対しては次のように言わなければならない。怒りやそのようなもの
が神にあるとされるのは、結果の類似による。なぜなら、罰することは怒る者
に固有であるから、罰が比喩的に神の怒りと呼ばれるのである。


\\


\textsc{Ad tertium dicendum} quod formae quae sunt receptibiles in
materia individuantur per materiam, quae non potest esse in alio, cum
sit primum subiectum substans, forma vero, quantum est de se, nisi
aliquid aliud impediat, recipi potest a pluribus. Sed illa forma quae
non est receptibilis in materia, sed est per se subsistens, ex hoc
ipso individuatur, quod non potest recipi in alio, et huiusmodi forma
est Deus. Unde non sequitur quod habeat materiam.

&

第三異論に対しては次のように言わなければならない。質料に受け取られうる
形相は質料によって個体化される。その質料は、もとにある第一基体だから、
他のものにおいて在ることができないが、形相はそれ自体に関する限り、なに
か他のものが妨げない限り、複数のものによって受け取られうる。しかし、質
料に受け取られることができず、それ自体によって自存する形相は、他のもの
において受け取られないということ自体によって個体化される。神はこのよう
な形相である。したがって、神が質料を持つことは帰結しない。



\end{longtable}


\newpage
\rhead{a.~3}

\begin{center}
 {\Large {\bf ARTICULUS TERTIUS}}\\
 {\large UTRUM SIT IDEM DEUS QUOD SUA ESSENTIA VEL NATURA}\\
 {\footnotesize I \textit{Sent.}, d.~34, q.~1, a.~1; I \textit{SCG.},
 c.~21; Qq.~disp., \textit{de Un.~Vert.}, a.~1; \textit{de Anima},
 a.~17, ad 10; \textit{Quodl.}~II, q.~2, a.~2; \textit{Compend.~Theol.},
 c.~10; Opusc.~37, \textit{de Quatuor Oppos.}~c.~4.}\\
{\Large 第三項\\神はその本質ないし本性と同一か}

\end{center}

\begin{longtable}{p{21em}p{21em}}


\textsc{Ad tertium sic proceditur}. Videtur quod non sit idem Deus
quod sua essentia vel natura. Nihil enim est in seipso. Sed essentia
vel natura Dei, quae est deitas, dicitur esse in Deo. Ergo videtur
quod Deus non sit idem quod sua essentia vel natura.


&

第三項の問題へ、議論は次のように進められる。神は神の本質ないし本性と同
一でないと思われる。理由は以下の通り。何ものも自分自身の中にあるものは
ない。しかるに神の本質ないし本性つまり神性は、神のうちに在ると言われる。
ゆえに神は神の本質ないし本性と同一ではないと思われる。

\\

2.~\textsc{Praeterea}, effectus assimilatur suae causae, quia omne
agens agit sibi simile. Sed in rebus creatis non est idem suppositum
quod sua natura, non enim idem est homo quod sua humanitas. Ergo nec
Deus est idem quod sua deitas.

&

さらに、すべて働くものは、自らに似たものを結果として生み出すから、結果
はその原因に類似する。しかるに創造された事物において個体はそれ自身の本
性と同一ではない。たとえば人間は人間性と同一でない。ゆえに神も神の神性
と同一でない。

\\


\textsc{Contra}, de Deo dicitur quod est vita, et non solum quod est
vivens, ut patet Ioan.~\textsc{xiv} : \textit{Ego sum via, veritas et
vita}. Sicut autem se habet vita ad viventem, ita deitas ad Deum. Ergo
Deus est ipsa deitas.

&

反対に、『ヨハネによる福音書』14章「私は道である、真理である、生命であ
る」で明らかなように、神についてそれは生きているものであるのみならず生
命でもあると言われる。しかるに、生命と生きるものとの関係は、神性と神と
の関係に等しい。ゆえに神は神性そのものである。

\\



\textsc{Respondeo dicendum} quod Deus est idem quod sua essentia vel
natura.  Ad cuius intellectum sciendum est, quod in rebus compositis
ex materia et forma, necesse est quod differant natura vel essentia et
suppositum.  Quia essentia vel natura comprehendit in se illa tantum
quae cadunt in definitione speciei, sicut humanitas comprehendit in se
ea quae cadunt in definitione hominis, his enim homo est homo, et hoc
significat humanitas, hoc scilicet quo homo est homo.

&

答えて言わなければならない。神は、神の本質ないし本性と同一である。これ
を理解するためには次のことが知られるべきである。質料と形相から複合され
た事物において、本性ないし本質と個体が異なることは必然である。理由は以
下の通り。本質ないし本性は自らのうちに種の定義に入るものだけを含む。た
とえば、人間性(という人間の本質ないし本性)は、そのうちに、人間の定義に
入るもの(だけ)を含む。なぜなら、人間は、それら(人間の定義に入るものど
も)によって人間だが、人間性は、人間がそれによって人間であるところのも
のを表示するからである。



\\


Sed materia individualis, cum accidentibus omnibus individuantibus
ipsam, non cadit in definitione speciei, non enim cadunt in
definitione hominis hae carnes et haec ossa, aut albedo vel nigredo,
vel aliquid huiusmodi.  Unde hae carnes et haec ossa, et accidentia
designantia hanc materiam, non concluduntur in humanitate.  Et tamen
in eo quod est homo, includuntur, unde id quod est homo, habet in se
aliquid quod non habet humanitas.  Et propter hoc non est totaliter
idem homo et humanitas, sed humanitas significatur ut pars formalis
hominis; quia principia definientia habent se formaliter, respectu
materiae individuantis.

&

しかし、個的質料は、それ自身を個体化するすべての付帯性と同様、種の定義
に入らない。実際、人間の定義に、この肉やこの骨、あるいは白さや黒さ、あ
るいはその他そのようなものは入らない。したがって、この肉やこの骨、この
質料を指定する付帯性は、人間性に含まれない。しかし、人間であるものには、
それらは含まれる。したがって、人間であるものは、自分の中に、人間性が持っ
ていないようなものを持っている。このために、人間と人間性は全面的に同一
ではなく、むしろ人間性は、人間の形相的部分として表示される。なぜなら、
(定義を)限定する原理は、個体化する質料に対して、形相的な関係にあるから
である。


\\


In his igitur quae non sunt composita ex materia et forma, in quibus
individuatio non est per materiam individualem, idest per hanc
materiam, sed ipsae formae per se individuantur, oportet quod ipsae
formae sint supposita subsistentia.  Unde in eis non differt
suppositum et natura.  Et sic, cum Deus non sit compositus ex materia
et forma, ut ostensum est, oportet quod Deus sit sua deitas, sua vita,
et quidquid aliud sic de Deo praedicatur.

&

ゆえに、質料と形相から複合されていないものの場合、個体化は個的質料、す
なわち「この質料」によるのではなく、むしろ形相そのものが、自体的に個体
化されている。したがって、それらの形相自身が、自存する個体でなければな
らない。ゆえに、それらにおいて、個体と本性は異ならない。このようにして、
すでに示されたように神は質料と形相から複合されていないのだから、神が自
らの神性であり自らの生命であり、その他なんであれ神についてそのように述
語されるものであるのでなければならない。

\\

\textsc{Ad primum ergo dicendum} quod de rebus simplicibus loqui non
possumus, nisi per modum compositorum, a quibus cognitionem accipimus.
Et ideo, de Deo loquentes, utimur nominibus concretis, ut significemus
eius subsistentiam, quia apud nos non subsistunt nisi composita, et
utimur nominibus abstractis, ut significemus eius simplicitatem.  Quod
ergo dicitur deitas vel vita, vel aliquid huiusmodi, esse in Deo,
referendum est ad diversitatem quae est in acceptione intellectus
nostri; et non ad aliquam diversitatem rei.

&

第一異論に対してはそれゆえ次のように言われるべきである。私たちが単純な
事物について語ることができるのは、私たちがそこから認識を受け取る複合体
の様式によってでしかない。ゆえに、神について語るとき、私たちは具体的な
名称をその自存性を表示するために用いる。なぜなら、私たちのもとでは複合
体だけが自存するからである。また抽象的な名称を神の単純性を表示するため
に用いる。ゆえに、神性や生命や何かそのようなものが神のうちにあると言わ
れることは、私たちの知性の把握のうちにある多様性に関係付けられるべきで
あって、決して事物の多様性に関係付けられるべきではない。

\\

Ad secundum dicendum quod effectus Dei imitantur ipsum, non perfecte,
sed secundum quod possunt. Et hoc ad defectum imitationis pertinet,
quod id quod est simplex et unum, non potest repraesentari nisi per
multa, et sic accidit in eis compositio, ex qua provenit quod in eis
non est idem suppositum quod natura.

&

第二異論に対しては、次のように言われるべきである。神の結果が神を模倣す
るのは、完全にではなく、できる限りにである。そして単純で一つのものが多
くのものによってでないと表現され得ないということは、模倣の欠点に属する。
そしてそのようなかたちで、神の結果の中に複合が生じ、その複合から、神の
結果においては個体と本性が同一でないということが生じるのである。


\end{longtable}


\newpage
\rhead{a.~4}

\begin{center}
 {\Large {\bf ARTICULUS QUARTUS}}\\
 {\large UTRUM IN DEO SIT IDEM ESSENTIA ET ESSE}\\
 {\footnotesize I \textit{Sent.}, d.~8, q.~4, a.~1, 2; q.~5, a.~2;
 d.~34, q.~1, a.~1; II, d.~1, q.~1, a.~1; I \textit{SCG.}, c.~22, 52;
 Qq.~disp., \textit{de Pot.}, q.~7, a.~2; \textit{de Spirit.~Creat.},
 a.~1; \textit{Compend.~Theol.}, c.~9; Opusc.~37, \textit{de Quatuor
 Oppos.}, c.~4; \textit{de Ent.~et Ess.}, cap.~5.}\\
{\Large 第四項\\神において本質と存在は同一か}
\end{center}

\begin{longtable}{p{21em}p{21em}}

{\huge A}\textsc{d quartum sic proceditur}. Videtur quod in Deo non
sit idem essentia et esse.  Si enim hoc sit, tunc ad esse divinum
nihil additur.  Sed esse cui nulla fit additio, est esse commune quod
de omnibus praedicatur, sequitur ergo quod Deus sit ens commune
praedicabile de omnibus.  Hoc autem est falsum, secundum illud
\textit{Sap.}~\textsc{xiv}: \textit{incommunicabile nomen lignis et
lapidibus imposuerunt}.  Ergo esse Dei non est eius essentia.

&


第四項の問題へ、議論は次のように進められる。神において、本質と存在は同
一でないと思われる。理由は以下の通り。もし同一なら、神の存在には何も加
えられないだろう。しかし、何も加えられないままの存在とは、すべてのもの
の述語となるような、共通存在である。ゆえに、神の本質が存在だとすると、
神は共通存在であり、すべてのものの述語となりうるものであることになる。
しかし、『知恵の書』14章「彼らは、共有されえない名を、石や木に付けた」
によれば、これは間違いである。ゆえに、神の存在は神の本質ではない。

\\


2.~\textsc{Praeterea}, de Deo scire possumus an sit, ut supra dictum
est.  Non autem possumus scire quid sit.  Ergo non est idem esse Dei,
et quod quid est eius, sive quidditas vel natura.

&

さらに、すでに述べたように、神について、私たちは、存在するかどうかを知
ることができる。しかし私たちは、神が何であることを知ることができない。
ゆえに、神の存在と、神の「何であるか」、あるいは何性や本性は同じでない。

\\

\textsc{Sed contra est} quod Hilarius dicit in VII \textit{de Trin.},
\textit{esse non est accidens in Deo, sed subsistens veritas}.  Id
ergo quod subsistit in Deo, est suum esse.

&

しかし反対に、ヒラリウスは『三位一体論』第七巻で、「神において存在は付
帯性でなく、自存する真理である」と述べている。ゆえに、神において自存す
るものは、神の存在である。

\\

\textsc{Respondeo dicendum} quod Deus non solum est sua essentia, ut
ostensum est, sed etiam suum esse.  Quod quidem multipliciter ostendi
potest.

&

解答する。以下のように言われるべきである。神が自らの本質であることはす
でに示されたが、それだけでなく、神は自らの存在でもある。このことは、多
くのしかたで示すことができる。

\\

Primo quidem, quia quidquid est in aliquo quod est praeter essentiam
eius, oportet esse causatum vel a principiis essentiae, sicut
accidentia propria consequentia speciem, ut risibile consequitur
hominem et causatur ex principiis essentialibus speciei; vel ab aliquo
exteriori, sicut calor in aqua causatur ab igne.  


&

第一に、何かの中に、その何かの本質でないものがある場合、なんであれそれ
は、種に伴う固有の附帯性のように、その本質の根源が原因となって生じたも
のか、または、何か[本質の]外のものが原因となって生じたものかのどちら
かである。前者の例としては、「笑いうる」ということが人間に伴っているが、
これは人間の種の本質的な根源が原因になって生じている、という場合である。
後者の例は、たとえば、水の中の熱が、火によって生じている場合がそれであ
る。

\\


Si igitur ipsum esse rei sit aliud ab eius essentia, necesse est quod
esse illius rei vel sit causatum ab aliquo exteriori, vel a principiis
essentialibus eiusdem rei.  Impossibile est autem quod esse sit
causatum tantum ex principiis essentialibus rei, quia nulla res
sufficit quod sit sibi causa essendi, si habeat esse causatum.
Oportet ergo quod illud cuius esse est aliud ab essentia sua, habeat
esse causatum ab alio.  Hoc autem non potest dici de Deo, quia Deum
dicimus esse primam causam efficientem.  Impossibile est ergo quod in
Deo sit aliud esse, et aliud eius essentia.

&

ゆえに、もしも、事物の存在が、その事物の本質と別のものであるならば、そ
の存在は、その事物の外の何かが原因となって生じたか、または、その事物の
本質的な根源が原因となって生じたかのどちらかである。しかし、存在が、事
物の本質的な根源だけが原因となって生じることはありえない。なぜなら、ど
んなものも、もし原因された存在を持つのであれば、自己にとっての存在原因
であるには不十分だからである。ゆえに、自らの存在が自らの本質と別である
ようなものは、他のものが原因となって生じた存在を持たなければならない。
ところが、これは神について言われえない。なぜなら、私たちは神が第一の作
出因であると言う。ゆえに、神において、それの存在と本質が別であることは
不可能である。

\\

Secundo, quia esse est actualitas omnis formae vel naturae, non enim
bonitas vel humanitas significatur in actu, nisi prout significamus
eam esse.  Oportet igitur quod ipsum esse comparetur ad essentiam quae
est aliud ab ipso, sicut actus ad potentiam.  Cum igitur in Deo nihil
sit potentiale, ut ostensum est supra, sequitur quod non sit aliud in
eo essentia quam suum esse.  Sua igitur essentia est suum esse.

&


第二に、存在は、すべての形相や本性の現実性である。なぜなら、善性であれ
人間性であれ、それが現実に示されるのは、私たちがそれが存在すると示す限
りにおいてでしかないからである。ゆえに、存在そのものと、存在と異なる本
質との関係は、現実態と可能態の関係に等しい。ゆえに、すでに示されたよう
に、神において可能的なものはないのだから、神において、本質は神の存在と
異ならない。ゆえに、神の本質は神の存在である。

\\


Tertio, quia sicut illud quod habet ignem et non est ignis, est
ignitum per participationem, ita illud quod habet esse et non est
esse, est ens per participationem.  Deus autem est sua essentia, ut
ostensum est.  Si igitur non sit suum esse, erit ens per
participationem, et non per essentiam.  Non ergo erit primum ens, quod
absurdum est dicere.  Est igitur Deus suum esse, et non solum sua
essentia.


&

第三に、火を所有しながら火でないものは、分有によって火を所有する。同じ
ように、存在を所有しながら存在でないものは、分有によって存在するもので
ある。ところで、すでに示されたように、神は神の本質である。ゆえに、もし
も、神が神の存在でないならば、神は分有によって存在するのであり、本質に
よって存在するのではないことになる。そうなると、神は第一の存在者ではな
いことになるが、このように語ることは馬鹿げている。ゆえに、神は神の本質
であるだけでなく、神の存在でもある。


\\



\textsc{Ad primum ergo dicendum} quod aliquid cui non fit additio
potest intelligi dupliciter.  Uno modo, ut de ratione eius sit quod
non fiat ei additio; sicut de ratione animalis irrationalis est, ut
sit sine ratione.  Alio modo intelligitur aliquid cui non fit additio,
quia non est de ratione eius quod sibi fiat additio, sicut animal
commune est sine ratione, quia non est de ratione animalis communis ut
habeat rationem; sed nec de ratione eius est ut careat ratione.  Primo
igitur modo, esse sine additione, est esse divinum, secundo modo, esse
sine additione, est esse commune.

&

それゆえ第一異論に対しては次のように言わなければならない。「何も加えら
れないもの」は、二通りに考えることができる。一つには、「加えられない」
ということが、それの概念に含まれているものである。たとえば、「非理性的
動物」という概念には、「理性がない」といことが含まれている。「何も加え
られないもの」は、もう一つのしかたで理解できる。それは、「加えられる」
ということが、概念に含まれていないものである。たとえば、動物一般は理性
をもたないが、それは、動物一般の概念に、「理性を持つ」ということが含ま
れていないからである。しかし、「理性を持たない」ということもまた、動物
一般の概念には含まれていない。ゆえに、神の存在は、最初の意味で、何も加
えられない存在である。他方、共通の存在は、二つ目の意味で、何も加えられ
ない存在である。


\\

\textsc{Ad secundum dicendum} quod esse dupliciter dicitur, uno modo,
significat actum essendi; alio modo, significat compositionem
propositionis, quam anima adinvenit coniungens praedicatum subiecto.
Primo igitur modo accipiendo esse, non possumus scire esse Dei, sicut
nec eius essentiam, sed solum secundo modo. Scimus enim quod haec
propositio quam formamus de Deo, cum dicimus Deus est, vera est. Et
hoc scimus ex eius effectibus, ut supra dictum est.


&

第二異論に対しては、次のように言わなければならない。存在は二通りに言わ
れる。一つは、「存在」という現実態を意味する。もう一つは、述語を主語に
結びつけるときに魂が見つけ出す、命題の複合を意味する。「存在」を第一の
意味で理解するとき、私たちは神の存在を知ることができない。それは、私た
ちが神の本質を知ることができないのと同じである。私たちが知ることができ
る神の存在とは、第二の意味の「存在」である。つまり私たちが「神が存在す
る」と言うとき、私たちが神について形成するそのその命題が真であることを、
知る。そして、上で述べたように、このことを、私たちは神の結果から知る。


\end{longtable}

\newpage
\rhead{a.~5}
\begin{center}
 {\Large {\bf ARTICULUS QUINTUS}}\\
 {\large UTRUM DEUS SIT IN GENERE ALIQUO}\\
 {\footnotesize I {\itshape Sent.}, d.~8, q.~4, a.~2; d.~19, q.~4, a.~2;
 I {\itshape SCG}, c.~25; {\itshape De Pot.}, q.~7, a.~3; {\itshape
 Compend.~Theol.}, c.~12; {\itshape De Ent.~et Ess.}, c.~6.}\\
 {\Large 第五項\\神は何かの類の内にあるか}
\end{center}

\begin{longtable}{p{21em}p{21em}}
{\huge A}{\scshape d quintum sic proceditur}. Videtur quod Deus sit in
 genere aliquo. Substantia enim est ens per se subsistens. Hoc autem
 maxime convenit Deo. Ergo Deus est in genere substantiae.

&

第五項の問題へ、議論は次のように進められる。神は何らかの類のうちにある
と思われる。理由は以下の通り。実体とは、自らによって自存する「在るもの」
である。このことは、最大限に神に適合する。ゆえに、神は実体の類のうちに
ある。

\\

2.~{\scshape Praeterea}, unumquodque mensuratur per aliquid sui
generis; sicut longitudines per longitudinem, et numeri per
numerum. Sed Deus est mensura omnium substantiarum, ut patet per
Commentatorem, X {\itshape Metaphys}. Ergo Deus est in genere
substantiae.

&


さらに、各々のものは、自分の類に属する何かによって測られる。たとえば、
長さは長さによって、数は数によって測られる。しかるに神は、『形而上学』
10巻の注釈家によって明らかであるように、すべての実体の尺度である。ゆえ
に、神は実体の類にある。

\\

{\scshape Sed contra}, genus est prius, secundum intellectum, eo quod
in genere continetur. Sed nihil est prius Deo, nec secundum rem, nec
secundum intellectum. Ergo Deus non est in aliquo genere.

&

しかし反対に、類は、理解において、類に含まれているものより先である。し
かし、理解においても事物においても、神より先であるものは何もない。ゆえ
に、神はどんな類の中にもない。

\\


{\scshape Respondeo dicendum} quod aliquid est in genere dupliciter.
Uno modo simpliciter et proprie; sicut species, quae sub genere
continentur.  Alio modo, per reductionem, sicut principia et
privationes, sicut punctus et unitas reducuntur ad genus quantitatis,
sicut principia; caecitas autem, et omnis privatio, reducitur ad genus
sui habitus.  Neutro autem modo Deus est in genere.

&


解答する。以下のように言われるべきである。何かが類の中にあるのには二通
りのかたちがある。一つには、端的に、そして固有の意味でであり、例えば種
がそうである。種は、この意味で類に含まれる。もう一つには、還元によって
であり、例えば根源や欠如がこの意味で類に含まれる。点や一性は、根源とし
て量の類に還元されるのであり、他方、視力の欠如(盲)やすべての欠如は、
それを持っている状態(habitus所有態)に還元される。しかしこのどちらの
かたちによっても、神は類の中にない。

\\


Quod enim non possit esse species alicuius generis, tripliciter
ostendi potest.  Primo quidem, quia species constituitur ex genere et
differentia.  Semper autem id a quo sumitur differentia constituens
speciem, se habet ad illud unde sumitur genus, sicut actus ad
potentiam.  {\itshape Animal} enim sumitur a natura sensitiva per
modum concretionis; hoc enim dicitur animal, quod naturam sensitivam
habet, rationale vero sumitur a natura intellectiva, quia {\itshape
rationale} est quod naturam intellectivam habet, intellectivum autem
comparatur ad sensitivum, sicut actus ad potentiam. Et similiter
manifestum est in aliis.  Unde, cum in Deo non adiungatur potentia
actui, impossibile est quod sit in genere tanquam species.

&
神が何らかの類の種でありえないということは、三つのしかたで明らかにされうる。
第一に、種は、類と種差から構成される。
ところが、種を構成する種差が、そこから取られるところのそれは、そこから類
 が取られるところのそれに対して、現実態が可能態に対するように関係する。
たとえば、「動物」は感覚的な本性から、具体化というしかたで取られる。それ
 で、感覚的な本性をもつものが「動物」と言われる。他方、「理性的」という
 ことは、知性的な本性から取られる。なぜなら、知性的な本性をもつものが、
 理性的なものだから。ところが、知性的なものは、感覚的なものに対して、現
 実態が可能態に対するように関係する。他のものについても同じことが明らか
 である。
したがって、神の中で可能態が現実態に結びつけられることはないので、神が種
 として類の中にあることは不可能である。

\\

Secundo, quia, cum esse Dei sit eius essentia, ut ostensum est, si
Deus esset in aliquo genere, oporteret quod genus eius esset {\itshape
ens}, nam genus significat essentiam rei, cum praedicetur {\itshape in
eo quod quid est}.  Ostendit autem philosophus in III {\itshape
Metaphys}., quod ens non potest esse genus alicuius, omne enim genus
habet differentias quae sunt extra essentiam generis; nulla autem
differentia posset inveniri, quae esset extra ens; quia {\itshape non
ens} non potest esse differentia.  Unde relinquitur quod Deus non sit
in genere.

&

第二に、すでに示されたように、神の存在は神の本質だから、もし神が何らか
の類にあったら、その類は「在るもの」だっただろう。というのも、類は、
「何であるか」において述語されるものだから、本質を表示するからである。
ところが、哲学者は『形而上学』第三巻で、「在るもの」は何らかのものの類
ではありえないことを明らかにしている。類はすべて、類の本質の外にある種
差を持つが、「在るもの」の外にあるような種差は何もないからである。なぜ
なら、「在らぬもの」は、種差ではありえないのだから。ゆえに、神は類の中
にないことが帰結する。

\\

Tertio, quia omnia quae sunt in genere uno, communicant in quidditate
vel essentia generis, quod praedicatur de eis {\itshape in eo quod
quid est}.  Differunt autem secundum esse, non enim idem est esse
hominis et equi, nec huius hominis et illius hominis.  Et sic oportet
quod quaecumque sunt in genere, differant in eis esse et {\itshape
quod quid est}, idest essentia.  In Deo autem non differt, ut ostensum
est.  Unde manifestum est quod Deus non est in genere sicut species.

&

第三に、全て一つの類の中にあるものは、類の何性ないし本質において共通し、
その類は、それらについて、「何であるか」において述語される。他方、存在
において、それらは異なる。たとえば、人間の存在と馬の存在は同じでないし、
この人間の存在とあの人間の存在も同じでないからである。このようにして、
類の中にあるものは何であれ、それらにおいて、存在と、何であるかつまり本
質が異なる。ところが、神においては、すでに示されたように、これらは異な
らない。したがって、神は、種として類の中にあるのではないことが明らかで
ある。

\\

Et ex hoc patet quod non habet genus, neque differentias; neque est
definitio ipsius; neque demonstratio, nisi per effectum, quia
definitio est ex genere et differentia, demonstrationis autem medium
est definitio.

&


このことからまた、神が類も種差も持たず、神の定義もないことが明らかであ
る。また、結果からでなければ論証もない。なぜなら、定義は類と種差からな
り、論証の媒介(中項)は定義だからである。

\\

Quod autem Deus non sit in genere per reductionem ut principium,
manifestum est ex eo quod principium quod reducitur in aliquod genus,
non se extendit ultra genus illud, sicut punctum non est principium
nisi quantitatis continuae, et unitas quantitatis discretae.  Deus
autem est principium totius esse, ut infra ostendetur. Unde non
continetur in aliquo genere sicut principium.

&

また、神が還元によって、根源として類のうちにあるのでもないことは、ある
類に還元される根源が、その類を超えて広がらないということ、たとえば、点
は、連続量以外の根源ではなく、一性は、離散量以外の根源ではない、という
ことから明らかである。これに対して、以下に明らかにされるように、神は全
存在の根源である。したがって、根源として何らかの類に含まれるのでもない。
\footnote{◎しかし、ここに見られる「神は還元によっても実体でない」とい
う主張は、安定していない。
\\
Ad septimum dicendum, quod licet Deus non pertineat ad genus
substantiae quasi in genere contentum,- sicut species vel individuum
sub genere continentur,- potest tamen dici quod sit in genere
substantiae per reductionem, sicut principium, et sicut punctum est in
genere quantitatis continuae, et unitas in genere numeri; et per hunc
modum est mensura substantiarum omnium, sicut unitas numerorum. (De
potentia, q. 7 a. 3 ad 7.)
\\
Ad tertium dicendum quod, licet Deus non sit in genere substantiae
 tamquam species, pertinet tamen ad genus substantiae sicut generis
 principium. (De potentia, q. 9 a. 3 ad 3.)
\\
◎執筆年代
De Potentia 1259-1268
Summa Theologiae 1265-1273
『神学大全』の第一部と『能力論』とは、ほぼ同じ頃(第一イタリア時代)に書かれている。
}


\\

{\scshape Ad primum ergo dicendum} quod substantiae nomen non
significat hoc solum quod est per se esse, quia hoc quod est esse, non
potest per se esse genus, ut ostensum est. Sed significat essentiam
cui competit sic esse, idest per se esse, quod tamen esse non est ipsa
eius essentia. Et sic patet quod Deus non est in genere substantiae.

&

第一異論に対しては、それゆえ、次のように言うべきである。実体という名称
は、それ自身によって存在するということだけを意味するのではない。なぜな
ら、存在というものは、すでに示されたとおり、それ自体では類ではありえな
いからである。そうではなく、そのように存在する、つまりそれ自体によって
存在するということが適合する本質を意味するのだが、この存在は、その本質
それ自体ではない。このようにして、神が実体の類にないことが明らかである。

\\

{\scshape Ad secundum dicendum} quod obiectio illa procedit de mensura
proportionata, hanc enim oportet esse homogeneam mensurato. Deus autem
non est mensura proportionata alicui. Dicitur tamen mensura omnium, ex
eo quod unumquodque tantum habet de esse, quantum ei appropinquat.

&

第二異論に対しては言わなければならない。かの反論は、比例している尺度に
ついて論じている。じっさい、このような尺度は、測られるものと同類でなけ
ればならない。しかし、神は何かに比例している尺度ではない。そうではなく、
各々のものが、それに近づくだけ、それだけ存在をもつということから、すべ
てのものの尺度と言われるのである。


\end{longtable}



\newpage
\rhead{a.~6}


\begin{center}
 {\Large {\bf ARTICULUS SEXTUS}}\\
 {\large UTRUM IN DEO SINT ALIQUA ACCIDENTIA}\\
 {\footnotesize I {\itshape Sent.}, d.~8, q.~4, a.~3; I {\itshape SCG.},
 c.~23; {\itshape De Pot.}, q.~7, a.~4; {\itshape Compend.~Theol.},
 c.~23.}\\
 {\Large 第六項\\神において何かの附帯性があるか}
\end{center}

\begin{longtable}{p{21em}p{21em}}

{\huge A}{\scshape d sextum sic proceditur}. Videtur quod in Deo sint
 aliqua accidentia. Substantia enim {\itshape nulli est accidens}, ut
 dicitur in I {\itshape Physic}. Quod ergo in uno est accidens, non
 potest in alio esse substantia, sicut probatur quod calor non sit
 forma substantialis ignis, quia in aliis est accidens. Sed sapientia,
 virtus, et huiusmodi, quae in nobis sunt accidentia, Deo
 attribuuntur. Ergo et in Deo sunt accidentia.

&

第六項の問題へ、議論は以下のように進められる。神の中に何らかの付帯性が
あると思われる。理由は以下の通り。『自然学』第1巻で言われるように、実
体は、何にも付帯しない(何にとっても付帯性ではない)。ゆえに、ある一つ
のものにおいて付帯性であるものは、別のあるものにおいて、実体であること
はできない。ちょうど、熱が、火の実体形相でないことが、火以外のものにお
いて熱が付帯性であることから証明されるようにである。しかるに、知恵や徳
や、このようなものは、われわれのうちに存在するが、神にも帰せられる。ゆ
えに、これらは神においても付帯性である。


\\



2.~{\scshape Praeterea}, in quolibet genere est unum primum. Multa autem sunt
 genera accidentium. Si igitur prima illorum generum non sunt in Deo,
 erunt multa prima extra Deum, quod est inconveniens.

&

さらに、どんな類においても、第一のひとつのものがある。しかるに、付帯性
の類はたくさんある。ゆえに、もしも、神の中にそれらの類の第一の類がない
ならば、多くの第一の類が神の外にあることになろう。これは不都合である。

\\


{\scshape Sed contra}, omne accidens in subiecto est. Deus autem non
 potest esse subiectum, quia {\itshape forma simplex non potest esse
 subiectum}, ut dicit Boetius in Lib.~{\itshape de Trin}. Ergo in Deo
 non potest esse accidens.

&

しかし反対に、付帯性はすべて基体においてある。しかるに神は基体でありえ
ない。なぜなら、ボエティウスが『三位一体論』で言うように、「単純形相は
基体でありえない」からである。ゆえに、神のなかに付帯性はあり得ない。

\\

Renspondeo dicendum quod, secundum praemissa, manifeste apparet quod
in Deo accidens esse non potest.  Primo quidem, quia subiectum
comparatur ad accidens, sicut potentia ad actum, subiectum enim
secundum accidens est aliquo modo in actu.  Esse autem in potentia,
omnino removetur a Deo, ut ex praedictis patet.

&

解答する。以下のように言われるべきである。すでに述べられたことに従えば、
神の中に付帯性がありえないことは明白に明らかである。実際、第一に、基体
は付帯性に対して、可能態が現実態に対するように関係する。なぜなら、基体
が何らかの意味で現実態にあるのは、付帯性によるからである。ところが、す
でに述べたことから明らかなとおり、可能態における存在は、あらゆる意味に
おいて、神から排除される。


\\

Secundo, quia Deus est suum esse, et, ut Boetius dicit in
Lib.~{\itshape de Hebdomad}., {\itshape licet id quod est, aliquid
aliud possit habere adiunctum, tamen ipsum esse nihil aliud adiunctum
habere potest}, sicut quod est calidum, potest habere aliquid
extraneum quam calidum, ut albedinem; sed ipse calor nihil habet
praeter calorem.

&

第二に、神は自らの存在である。また、ボエティウスが『デ・ヘブドマディブ
ス』で言うように、「存在するものは、何か他のものを、結び付けられたもの
として持つことができるとしても、しかし存在そのものは、他の何ものも、結
び付けられたものとして持つことができない」。ちょうど、熱いものは、熱い
もの以外の何か外的なもの、たとえば白さを持つことができるが、熱さは、熱
さ以外のものを持つことができないように。

\\

Tertio, quia omne quod est per se, prius est eo quod est per accidens.
Unde, cum Deus sit simpliciter primum ens, in eo non potest esse aliquid
per accidens.-- Sed nec {\itshape accidentia per se} in eo esse possunt,
sicut risibile est per se accidens hominis.  Quia huiusmodi accidentia
causantur ex principiis subiecti, in Deo autem nihil potest esse
causatum, cum sit causa prima. Unde relinquitur quod in Deo nullum sit
accidens.  

&

第三に、すべてそれ自身によってであるものは、付帯的にあるものよりも先で
ある。したがって、神は端的に第一の在るものなので、それにおいて、付帯的
にあるものはなにもない。しかし、ちょうど、笑いうることが、それ自身によっ
て、人間の付帯性であるように、付帯性が、それ自身によって、神のうちにあ
ることもできない。なぜなら、そのような付帯性は、基体の根源が原因となっ
て生じるのだが、神は第一原因だから、神においては、何かが原因によって生
じることはできないからである。ゆえに、神において、付帯性はなにもないこ
とが残る。


\\

{\scshape Ad primum ergo dicendum} quod virtus et sapientia non univoce
 dicuntur de Deo et de nobis, ut infra patebit. Unde non sequitur quod
 accidentia sint in Deo, sicut in nobis.  


&

第一異論に対しては、それゆえ、次のように言わなければならない。徳や知恵
は、神と私たちについて、同義的に言われるのではない。これは以下で明らか
になる。したがって、私たちの中に付帯性があるように、神の中にも付帯性が
あることは帰結しない。

\\

{\scshape Ad secundum dicendum} quod, cum substantia sit prior accidentibus,
 principia accidentium reducuntur in principia substantiae sicut in
 priora. Quamvis Deus non sit primum contentum in genere substantiae,
 sed primum extra omne genus, respectu totius esse.

&

第二異論に対しては、次のように言わなければならない。実体は付帯性よりも
先なので、付帯性の根源は、より先なるものとしての実体の根源へ還元される。
神は、実体の類に含まれる第一のものではないけれども、しかし、全存在に関
して、すべての類の外にある第一のものである。

 
\end{longtable}

\newpage
\rhead{a.~7}

\begin{center}
 {\Large {\bf ARTICULUS SEPTIMUS}}\\
 {\large UTRUM DEUS SIT OMNINO SIMPLEX}\\
 {\footnotesize I {\itshape Sent.}, d.~8, q.~4, a.~1; I {\itshape SCG.},
 c.~16, 18; {\itshape De Pot.}, q.~7, a.~1; {\itshape Compend.~Theol.},
 c.~9; Opusc.~37, {\itshape de Quat.~Oppos.}, c.~4; {\itshape De Caus.},
 lect.~21.}\\
 {\Large 第七項\\神はすべての点で単純か}
\end{center}

\begin{longtable}{p{21em}p{21em}}

{\huge A}{\scshape d septimum sic proceditur}. Videtur quod Deus non
 sit omnino simplex. Ea enim quae sunt a Deo, imitantur ipsum, unde a
 primo ente sunt omnia entia, et a primo bono sunt omnia bona. Sed in
 rebus quae sunt a Deo, nihil est omnino simplex. Ergo Deus non est
 omnino simplex.

&

第七項の問題へ、議論は以下のように進められる。神は、あらゆる点で単純だ
というわけではない、と思われる。理由は以下の通り。神によって存在するも
のどもは、神を真似る。このようにして、第一の存在者によってすべての存在
者が、第一の善によってすべての善がある。しかるに、神によってあるものど
もにおいて、あらゆる点で単純なものはなにもない。ゆえに、神はあらゆる点
で単純だというわけではない。

\\

{\scshape Praeterea}, omne quod est melius, Deo attribuendum est. Sed,
 apud nos, composita sunt meliora simplicibus, sicut corpora mixta
 elementis, et elementa suis partibus. Ergo non est dicendum quod Deus
 sit omnino simplex.

&


さらに、すべて、より良いものは、神に帰せられるべきである。しかるに、私
たちのもとで、複合されたものは単純なものよりもよい。たとえば、混合され
た物体は諸元素よりも、諸元素はそれぞれの部分よりもよい。ゆえに、神があ
らゆる点で単純だ、とは言うべきでない。

\\

{\scshape Sed contra est} quod Augustinus dicit, VI {\itshape de
 Trin.}, quod Deus vere et summe simplex est.

&

しかし反対に、アウグスティヌスは『三位一体論』第6巻で、神は真に最高に
単純である、と述べている。

\\

{\scshape Respondeo dicendum} quod Deum omnino esse simplicem,
 multipliciter potest esse manifestum. Primo quidem per
 supradicta. Cum enim in Deo non sit compositio, neque quantitativarum
 partium, quia corpus non est; neque compositio formae et materiae,
 neque in eo sit aliud natura et suppositum; neque aliud essentia et
 esse, neque in eo sit compositio generis et differentiae; neque
 subiecti et accidentis, manifestum est quod Deus nullo modo
 compositus est, sed est omnino simplex.

&

解答する。以下のように言われるべきである。神があらゆる点で単純であるこ
とは、多くのしかたで明らかにできる。第一に、すでに述べられた事柄によっ
て。というのも、神においては、量的部分からの複合がない。なぜなら神は物
体でないからである。また、形相と質料からの複合がない。また神において、
本性と個体は別のものでない。また、本質と存在も別でない。また神において、
類と種差の複合がない。基体と付帯性の複合もない。神がどんなかたちでも複
合されなく、あらゆる点で単純であることは明らかである。

\\


Secundo, quia omne compositum est posterius suis componentibus, et
dependens ex eis. Deus autem est primum ens, ut supra ostensum est.

&

第二に、すべて複合されているものは、自らの複合の要素よりも、後のもので
あり、それらに依存する。しかるに神は、すでに示されたように、第一の存在
者である。

\\

Tertio, quia omne compositum causam habet, quae enim secundum se
diversa sunt, non conveniunt in aliquod unum nisi per aliquam causam
adunantem ipsa. Deus autem non habet causam, ut supra ostensum est,
cum sit prima causa efficiens.

&

第三に、すべて複合されているものは原因を持つ。なぜならば、それ自体にお
いてさまざまであるものどもは、それらをひとつに結びつける何らかの原因に
よらなければ、なにかひとつのものへと一致することはないからである。とこ
ろが、上で示されたように、神は第一作出因なので、原因を持たない。

\\

Quarto, quia in omni composito oportet esse potentiam et actum, quod
in Deo non est, quia vel una partium est actus respectu alterius; vel
saltem omnes partes sunt sicut in potentia respectu totius.

&

第四に、すべて複合されたものの中には可能態と現実態があるのでなければな
らない。このことは、神においてはそうでない。なぜなら、[可能態と現実態
があるものというのは]諸部分のうちの一つの部分が他の部分に対して現実態
にあるか、あるいは少なくとも、すべての部分が全体に対して可能態としてあ
るかのどちらかでである。[しかし神は純粋現実態なので、可能態をもたない。
ゆえに、神の中に可能態と現実態の複合はなく、したがって、複合されていな
い。]

\\

Quinto, quia omne compositum est aliquid quod non convenit alicui
suarum partium. Et quidem in totis dissimilium partium, manifestum
est, nulla enim partium hominis est homo, neque aliqua partium pedis
est pes. In totis vero similium partium, licet aliquid quod dicitur de
toto, dicatur de parte, sicut pars aeris est aer, et aquae aqua;
aliquid tamen dicitur de toto, quod non convenit alicui partium, non
enim si tota aqua est bicubita, et pars eius. Sic igitur in omni
composito est aliquid quod non est ipsum. Hoc autem etsi possit dici
de habente formam, quod scilicet habeat aliquid quod non est ipsum
(puta in albo est aliquid quod non pertinet ad rationem albi), tamen
in ipsa forma nihil est alienum. Unde, cum Deus sit ipsa forma, vel
potius ipsum esse, nullo modo compositus esse potest. Et hanc rationem
tangit Hilarius, VII {\itshape de Trin.}, dicens, {\itshape Deus, qui
virtus est, ex infirmis non continetur, neque qui lux est, ex obscuris
coaptatur}.

&

第五に、すべて複合されたものは、自らの部分のうちの何かに一致しないよう
なものである。実際、全体に類似しない部分からなる全体においては、このこ
とは明らかである。たとえば、人間のどの部分も人間でなく、足のどの部分も
足でない。他方、全体に類似した部分からなる全体においては、全体について
言われることが部分について言われるものがあるけれども、たとえば、空気の
部分は空気であり、水の部分は水であるように、しかし、部分の一部には適合
しないことが全体について言われることがある。たとえば、全体の水が1メー
トルであれば、その部分は1メートルでない[山田訳:全体が二リットルであ
るとすれば、その部分は二リットルではない]。\footnote{◎bicubitaについ
て。肘から中指の先までをcubitumと言い、古代からよく知られた長さの単位。
何故トマスはここで水が2cubit$\approx$1mだと言うのだろうか。トマスの著
作中に、bicubitumは21箇所に出てくるが、いずれも、量にかんする単位だと
いう認識があり、ある箇所でははっきりと「長さ」の単位だと認識している。
cf. {\itshape In De sensu}, tract.~2, l.~2, n.~2 「人間の想像力には、
1mの人間があらわれることがあるが、知性はそれをその大きさに関係なく人間
として理解する」\\ c\u{u}b\u{\i}tum , i, n.~(c\u{u}b\u{\i}tus , i, m.,
Cels.~8, 1; 8, 16; Non.~p.~201, 16) [id.], I.~the elbow (serving for
leaning upon).  A.~The bending, curvature of a shore B.~As a measure
of length, the distance from the elbow to the end of the middle
finger, an ell, a cubit,
\\
○古代エジプトの単位として、cubic cubitsがある。
%http://en.wikipedia.org/wiki/Ancient_Egyptian_units_of_measurement
1 royal cubit = 52.5cm
1 standard cubit = 45cm
\\
○ノアの方舟の大きさを表すのに、cubitが使われている。
『創世記』(6:15)
This is how you are to make it: the length of the ark three hundred
 cubits, its width fifty cubits, and its height thirty cubits. 
長さ300cubits、幅50cubits、高さ30cubits$\approx$150m $\times$ 25m $\times$ 15m
\\
○ソロモンの神殿の大きさについて、cubitが使われている。
『列王記上』(6:1)
``The house that King Solomon built for the Lord was sixty cubits long,
 twenty cubits wide, and thirty cubits high.''
60cubits $\times$ 20cubits $\times$ 30cubits
}それゆえ、このように、すべて複合
 されたものの中には、それ自身でないものがある。このことは、形相を持つも
 のには言えても、つまり、自らでないものを持つということ(たとえば、白い
 ものの中には、白いものという性格に含まれない何かがあるように)、形相そ
 れ自体の中には、外的なものはなにもない。したがって、神は形相それ自体な
 のだから、いやむしろ、神は存在そのものなのだから、どんなかたちでも、複
 合されてはありえない。この理由に触れて、ヒラリウスは、『三位一体論』第7
 巻で「神は力であり、弱いものを含まず、神は光であり、暗いものと交わらな
 い」と言う。

\\


{\scshape Ad primum ergo dicendum} quod ea quae sunt a Deo, imitantur
Deum sicut causata primam causam. Est autem hoc de ratione causati,
quod sit aliquo modo compositum, quia ad minus esse eius est aliud
quam quod quid est, ut infra\footnote{{\itshape ST} I, q.~50, a.~2}
patebit.

&

それゆえ、第一異論に対しては次のように言わなければならない。神によって
あるものどもは、原因から生まれたものどもが、第一原因に関係するように、
神に関係する。しかるに、何らかのしかたで複合されているということは、原
因から生まれたものの性格に含まれる。なぜなら、少なくとも、その存在は、
その何であるかとは異なるからである。このことは後に明らかになる。

\\

{\scshape Ad secundum dicendum} quod apud nos composita sunt meliora
 simplicibus, quia perfectio bonitatis creaturae non invenitur in uno
 simplici, sed in multis. Sed perfectio divinae bonitatis invenitur in
 uno simplici, ut infra\footnote{{\itshape ST} I, q.~4, a.~2.} ostendetur.

&

第二異論に対しては次のように言われるべきである。私たちのもとで、複合さ
れているものが単純なものよりも良いのは、被造物の善性の完成が、一つの単
純なものではなく、多くのものにおいて見出されるからである。しかし、神の
善性の完全性は、一つの単純なものにおいて見出される。これは、後に明らか
にされる。

\end{longtable}

\newpage
\rhead{a.~8}
\begin{center}
 {\Large {\bf ARTICULUS OCTAVUS}}\\
 {\large UTRUM DEUS IN COMPOSITIONEM ALIORUM VENIAT}\\
 {\footnotesize I {\itshape Sent.}, d.8, q.1, a.2; I {\itshape SCG.},
 cap.17, 26, 27; III, cap.51; {\itshape De Pot.}, q.6, a.6; {\itshape De
 Verit.}, q.21, a.4.}\\
 {\Large 第八項\\神は他のものとの複合に入るか}
\end{center}

\begin{longtable}{p{21em}p{21em}}

{\huge A}{\scshape d octavum sic proceditur}. Videtur quod Deus in
 compositionem aliorum veniat. Dicit enim Dionysius, {\scshape iv}
 cap.~{\itshape Cael.~Hier}., {\itshape esse omnium est, quae super
 esse est deitas}. Sed esse omnium intrat compositionem
 uniuscuiusque. Ergo Deus in compositionem aliorum venit.

&

第八項の問題へ、議論は以下のように進められる。神は他のものとの複合に入
るように思われる。理由は以下の通り。ディオニュシウスは『天上階級論』第
4巻で、「すべてのものの存在は、存在の上にあり、神性である」と言う。し
かるに、すべてのものの存在は、どんなものとも複合に入る。ゆえに、神は他
のものとの複合に入る。

\\



{\scshape Praeterea}, Deus est forma, dicit enim Augustinus, in libro
 {\itshape de Verbis Domini}, quod {\itshape verbum Dei} (quod est
 Deus) {\itshape est forma quaedam non formata}. Sed forma est pars
 compositi. Ergo Deus est pars alicuius compositi.

&

さらに、神は形相である。なぜならアウグスティヌスは『主の言葉について』
で、「神の言葉(これは神である)は、形相を受け取らない一種の形相である」
と述べているからである。しかるに、形相は、複合体の部分である。ゆえに、
神は何らかの複合体の部分である。

\\

{\scshape Praeterea}, quaecumque sunt et nullo modo differunt, sunt
 idem. Sed Deus et materia prima sunt, et nullo modo differunt. Ergo
 penitus sunt idem. Sed materia prima intrat compositionem rerum. Ergo
 et Deus. Probatio mediae, quaecumque differunt, aliquibus differentiis
 differunt, et ita oportet ea esse composita; sed Deus et materia prima
 sunt omnino simplicia; ergo nullo modo differunt.

&

さらに、何であれ、存在し、どんな点でも異ならないものは、同一である。し
かるに、神と第一質料は、存在し、どんな点でも異ならない。ゆえに、それら
はまったく同一である。しかるに、第一質料は、事物との複合に入る。ゆえに
神もまたそうである。中間命題の証明。何であれ異なるものは、何らかの差異
において異なる。したがって、それは複合されていなければならない。しかる
に神と第一質料はあらゆる点で単純である。ゆえに、いかなる点でも異ならな
い。

\\

{\scshape Sed contra est} quod dicit Dionysius, {\scshape ii}
 cap.~{\itshape de Div.~Nom}., quod {\itshape neque tactus est eius}
 (scilicet Dei), {\itshape neque alia quaedam ad partes commiscendi
 communio}. --- {\scshape Praeterea}, dicitur in libro {\itshape de
 Causis}, quod {\itshape causa prima regit omnes res, praeterquam
 commisceatur eis}.

&

しかし反対に、ディオニュシウスは『神名論』第2章で、「彼(=神)の接触
はなく、また、諸部分と混合して他の何らかのかたちで合一することもない」、
と述べている。---さらに、『原因論』には「第一原因は、万物を、それらと
混じり合うことなしに、支配する」と言われている。


\\

{\scshape Respondeo dicendum quod} circa hoc fuerunt tres
 errores. Quidam enim posuerunt quod Deus esset anima mundi, ut patet
 per Augustinum in Lib.~VII {\itshape de Civitate Dei}, et ad hoc
 etiam reducitur, quod quidam dixerunt Deum esse animam primi caeli.
 Alii autem dixerunt Deum esse principium formale omnium rerum. Et
 haec dicitur fuisse opinio Almarianorum.  Sed tertius error fuit
 David de Dinando, qui stultissime posuit Deum esse materiam primam.
 Omnia enim haec manifestam continent falsitatem, neque est possibile
 Deum aliquo modo in compositionem alicuius venire, nec sicut
 principium formale, nec sicut principium materiale.

&

解答する。以下のように言われるべきである。これに関しては、三つの誤りが
あった。ある人々は、神は世界霊魂であると言った。これは、アウグスティヌ
スによって、『神の国』第7巻で明らかである。また、ある人々が、神が第一
天の魂であると言ったということも、これに還元される。また、他の人々は、
神は万物の形相的根源だと言った。そしてこれは、アルマリアヌス派の人々の
意見であった。しかし、第三の誤りは、ディナンドゥスのダヴィドであり、彼
は、極めて愚かにも、神が第一質料だとした。これら全ては、明らかな誤謬を
含んでいる。神は、形相的根源としても第一質料としても、どんなかたちでも、
他のものとの複合に入ることは不可能である。

\\

Primo quidem, quia supra diximus Deum esse primam causam
efficientem. Causa autem efficiens cum forma rei factae non incidit in
idem numero, sed solum in idem specie, homo enim generat
hominem. Materia vero cum causa efficiente non incidit in idem numero,
nec in idem specie, quia hoc est in potentia, illud vero in actu.

&

第一に、私たちは上で、神が第一作出因であると述べた。ところで、作出因は、
作り出された事物の形相と、数において同一のものにはならなず、たんに、種
において同一のものになるだけである。たとえば、人間が人間を産み出すよう
に。他方、質料は、作出因と数において同一のものにはならず、また、種にお
いても同一にならない。なぜなら、質料は可能態にあり、作出因は現実態にあ
るからである。

\\


Secundo, quia cum Deus sit prima causa efficiens, eius est primo et
per se agere. Quod autem venit in compositionem alicuius, non est
primo et per se agens, sed magis compositum, non enim manus agit, sed
homo per manum; et ignis calefacit per calorem. Unde Deus non potest
esse pars alicuius compositi.

&

第二に、神は第一作出因だから、第一に、それ自身によって働くことが属する。
しかるに、何かあるものとの複合に入るものは、第一でそれ自身によって働く
ものではなく、むしろ、複合体がそのようなものである。たとえば、手が働く
のではなく、人が手によって働くのだし、火が熱によって熱するのである。し
たがって、神が何らかの複合体の部分であることはできない。

\\

Tertio, quia nulla pars compositi potest esse simpliciter prima in
entibus; neque etiam materia et forma, quae sunt primae partes
compositorum. Nam materia est in potentia, potentia autem est
posterior actu simpliciter, ut ex dictis patet. Forma autem quae est
pars compositi, est forma participata, sicut autem participans est
posterius eo quod est per essentiam, ita et ipsum participatum; sicut
ignis in ignitis est posterior eo quod est per essentiam. Ostensum est
autem quod Deus est primum ens simpliciter.

&

第三に、複合体のどんな部分も、存在するものどもの中で端的に第一のもので
はありえない。複合体の第一の部分である質料と形相もまた、そのような第一
のものではない。なぜなら、質料は可能態にあるが、述べられたことから明ら
かなように、可能態は、端的に言って、現実態よりも後である。他方、複合体
の部分である形相は、分有された形相であり、しかるに、分有するものは、本
質によってあるものよりも後であるように、分有されたもの自体もまた、後の
ものである。ちょうど、燃やされたものにおける火が、本質によって火である
ものよりも後であるように。しかるに、神が、端的に、第一の在るものである
ことはすでに示された。

\\

{\scshape Ad primum ergo dicendum} quod deitas dicitur esse omnium
 effective et exemplariter, non autem per essentiam.

&

第一異論に対しては、それゆえ、次のように言われるべきである。神性が「す
べてのものの存在」と言われるのは、神がすべてのものを作り出すという意味
で、そして、神がすべてのものの範型であるという意味でであり、本質によっ
てそうだというわけではない。

\\

Ad secundum dicendum{\scshape } quod verbum est forma exemplaris, non
autem forma quae est pars compositi.


&

第二異論に対しては次のように言われるべきである。言葉は、範型としての形
相であり、複合体の部分である形相ではない。

\\

{\scshape Ad tertium dicendum} quod simplicia non differunt aliquibus
aliis differentiis, hoc enim compositorum est. Homo enim et equus
differunt rationali et irrationali differentiis, quae quidem
differentiae non differunt amplius ab invicem aliis
differentiis. Unde, si fiat vis in verbo, non proprie dicuntur
{\itshape differre}, sed {\itshape diversa esse}, nam, secundum
philosophum X {\itshape Metaphys}., {\itshape diversum} absolute
dicitur, sed omne {\itshape differens} aliquo differt. Unde, si fiat
vis in verbo, materia prima et Deus non {\itshape differunt}, sed
{\itshape sunt diversa seipsis}. Unde non sequitur quod sint idem.  &

第三異論に対しては、次のように言われるべきである。単純なものどもは、何
らかの他の差異によって異なるのではない。これ(=差異によって異なること)
は、複合体の異なり方である。たとえば、人間と馬は、理性的であることと非
理性的であることという差異(=種差)によって異なるが、これらの差異は、
さらに相互に、他の差異によって異なるわけではない。したがって、厳密に言
えば、それらは、固有には「異なる」(differre)ではなく、「違う」
(diversa esse)と言われる。というのも、『形而上学』10 巻の哲学者によ
れば、「違う」は無条件に言われるが、すべて異なるものは、何らかの点で異
なるからである。したがって、厳密には、第一質料と神は、「異なる」のでは
なく、それぞれ自身において「違う」。したがって、この二つが同一だという
 ことは帰結しない。


\end{longtable}
\end{document}
