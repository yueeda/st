\documentclass[10pt]{jsarticle} % use larger type; default would be 10pt
%\usepackage[utf8]{inputenc} % set input encoding (not needed with XeLaTeX)
%\usepackage[round,comma,authoryear]{natbib}
%\usepackage{nruby}
\usepackage{okumacro}
\usepackage{longtable}
%\usepqckage{tablefootnote}
\usepackage[polutonikogreek,english,japanese]{babel}
%\usepackage{amsmath}
\usepackage{latexsym}
\usepackage{color}

%----- header -------
\usepackage{fancyhdr}
\pagestyle{fancy}
\lhead{{\it Summa Theologiae} I, q.~10}
%--------------------


\bibliographystyle{jplain}
\title{{\bf Prima Pars}\\{\HUGE Summae Theologiae}\\Sancti Thomae
Aquinatis\\Quaestio Decima\\{\bf De Dei Aeternitate}}
\author{Japanese translation\\by Yoshinori {\sc Ueeda}}
\date{Last modified \today}

%%%% コピペ用
%\rhead{a.~}
%\begin{center}
% {\Large {\bf }}\\
% {\large }\\
% {\footnotesize }\\
% {\Large \\}
%\end{center}
%
%\begin{longtable}{p{21em}p{21em}}
%
%&
%
%\\
%\end{longtable}
%\newpage





\begin{document}
\maketitle
\begin{center}
{\Large 第十問\\神の永遠性について}
\end{center}


\begin{longtable}{p{21em}p{21em}}
Deinde quaeritur de aeternitate. Et circa hoc quaeruntur sex. 
\begin{enumerate}
 \item quid sit aeternitas.
 \item utrum Deus sit aeternus.
 \item utrum esse aeternum sit proprium Dei. 
 \item utrum aeternitas differat a tempore.
 \item de differentia aevi et temporis. 
 \item utrum sit unum aevum tantum, sicut est unum tempus et una
       aeternitas.
\end{enumerate}

&

次に永遠について問われる。これについては6つのことが問われる。
\begin{enumerate}
 \item 永遠とはなにか。
 \item 神は永遠か。
 \item 永遠であることは神に固有か。
 \item 永遠は時間と異なるか。
 \item 永劫と時間の違いについて。
 \item 一つの時間、一つの永遠であるように、永劫もただ一つであるか。
\end{enumerate}
\end{longtable}

\newpage
\rhead{a.~1}

\begin{center}
 {\Large {\bf ARTICULUS PRIMUS}}\\
 {\large UTRUM CONVENIENTER DEFINIATUR AETERNITAS, QUOD EST
 INTERMINABILIS VITAE TOTA SIMUL ET PERFECTA POSSESSIO}\\
 {\footnotesize I {\itshape Sent.}, d.~8, q.~2, a.~1; {\itshape De
 Causis}, lect.~2.}\\
 {\Large 第一項\\永遠は、「限りない生命の、全体的で同時の、完全な所有」
 と\\適切に定義されるか}
\end{center}

\begin{longtable}{p{21em}p{21em}}
{\huge A}{\scshape d primum sic proceditur}. Videtur quod non sit conveniens definitio
aeternitatis, quam Boetius ponit V {\it de Consolatione}, dicens quod
{\it aeternitas est interminabilis vitae tota simul et perfecta
possessio}. Interminabile enim negative dicitur. Sed negatio non est de
ratione nisi eorum quae sunt deficientia, quod aeternitati non
competit. Ergo in definitione aeternitatis non debet poni interminabile.

&

第一項の問題へ議論は以下のように進められる。ボエティウスが『哲学の慰め』第5巻で
定めている永遠の定義、すなわち「永遠とは、限りない生命の、全体的で同時の、完全な所有である」は適切でないと思われる。理由は以下の通り。
「限りない」は否定的に述べられるが、否定は欠陥
があるものどもの概念にしか属さない。ところが欠陥があるということは永遠
には適合しない。ゆえに永遠の定義のなかに「限りない」は置かれるべきで
ない。

\\

2.~{\scshape Praeterea}, aeternitas durationem quandam significat. Duratio autem
magis respicit esse quam vitam. Ergo non debuit poni in definitione
aeternitatis vita, sed magis esse.

&

さらに、永遠は一種の持続を意味する。しかるに持続は生命よりも存在に
関係する。ゆえに永遠の定義の中に生命は置かれるべきでなく、むしろ存在
が置かれるべきであった。

\\


3.~{\scshape Praeterea}, totum dicitur quod habet partes. Hoc autem aeternitati
non convenit, cum sit simplex. Ergo inconvenienter dicitur tota.

&

さらに、全体と言われるのは部分を持つものである。しかるにこのこと(部
分を持つということ)は永遠に適合しない。なぜなら永遠は単純なものだか
ら。ゆえに「全体的で」は適切に語られていない。

\\

4.~{\scshape Praeterea}, plures dies non possunt esse simul, nec plura
tempora. Sed in aeternitate pluraliter dicuntur dies et tempora, dicitur
enim {\it Micheae} {\scshape v} : {\itshape egressus eius ab initio, a
diebus aeternitatis}; et {\it ad Rom.}, {\scshape xvi} cap: {\itshape
secundum revelationem mysterii temporibus aeternis taciti. Ergo
aeternitas non est tota simul}.


&

さらに、複数の日や複数の時が同時にはありえない。しかるに永遠の中で、
日や時間が複数的に語られる。たとえば『ミカ書』第5章では「彼の出るのは、
始めから、永遠の日々からである」\footnote{「お前の中から、わたしのために、
イスラエルを治める者が出る。彼の出生は古く、永遠の昔にさかのぼる。」
(5:1)}と言われ、『ローマの信徒への手紙』16章では「永遠の時間(複数形) 黙
されていた奥義の啓示によって」\footnote{「この福音は、世々にわたって隠さ
れていた、秘められた計画を啓示するものです。」(16:25)}と言われている。ゆ
えに永遠は「全体的で同時の」ではない。

\\

5.~{\scshape Praeterea}, totum et perfectum sunt idem. Posito igitur quod sit
tota, superflue additur quod sit perfecta.

&

さらに、全体的なものと完全なものとは同じである。ゆえに「全体的で」と述
べられたなら「完全な」が加えられるのは余分である。

\\

6.~{\scshape Praeterea}, possessio ad durationem non pertinet. Aeternitas autem
quaedam duratio est. Ergo aeternitas non est possessio.

&

さらに、所有は持続に属さない。しかるに永遠は一種の持続である。ゆえに永遠
は所有ではない。


\\

{\scshape Respondeo dicendum} quod, sicut in cognitionem simplicium oportet nos
venire per composita, ita in cognitionem aeternitatis oportet nos venire
per tempus; quod nihil aliud est quam numerus motus secundum prius et
posterius.

&


解答する。以下のように言われるべきである。ちょうど私たちが、複合
されたものどもを通って単純なものの認識へ至らなければならないように、私た
ちは時間を通って永遠の認識へ至らなければならない。そして時間とは、より
先とより後における運動の数に他ならない。

\\


Cum enim in quolibet motu sit successio, et una pars post alteram, ex
hoc quod numeramus prius et posterius in motu, apprehendimus tempus;
quod nihil aliud est quam numerus prioris et posterioris in motu.

&

というのも、どんな運動においても継起があり、ある部分が他の部分の後にくる。
このことにもとづいて、私たちは運動において、より先より後を数え、時間
を把握する。時間とは、運動におけるより先とより後の数に他ならない。

\\

In eo autem quod caret motu, et semper eodem modo se habet, non est
accipere prius et posterius. Sicut igitur ratio temporis consistit in
numeratione prioris et posterioris in motu, ita in apprehensione
uniformitatis eius quod est omnino extra motum, consistit ratio
aeternitatis.

&

他方、運動を欠き、常に同様にあるものにおいて、より先とより後を捉えるこ
とはできない。ゆえにちょうど時間の性格が、運動においてより先とより後を
数えることに存したように、永遠の性格は、あらゆる点で運動の外にあるものの
一様性の把握に存する。

\\

Item, ea dicuntur tempore mensurari, quae principium et finem
habent in tempore, ut dicitur in IV {\it Physic.}, et hoc ideo, quia in
omni eo quod movetur est accipere aliquod principium et aliquem
finem. Quod vero est omnino immutabile, sicut nec successionem, ita nec
principium aut finem habere potest.

&

さらに、『自然学』第4巻で言われるように、時間によって測られるも
のとは時間のうちに始まりと終わりを持つものである。なぜなら、すべて動く
ものにおいて、なんらかの始まりと終わりを捉えることができるからである。こ
れに対して、あらゆる点で不変のものは、継起を持ち得ないのと同様に、どんな
始まりやどんな終わりも持つことができない。


\\

Sic ergo ex duobus notificatur aeternitas. Primo, ex hoc quod
id quod est in aeternitate, est interminabile, idest principio et fine
carens (ut terminus ad utrumque referatur). Secundo, per hoc quod ipsa
aeternitas successione caret, tota simul existens.

&

ゆえに永遠は二つのことにもとづいて知られる。一つには、永遠のうちにあ
るものは限りがない、つまり始まりと終わりを持たない(「限り」は、この
両方(=始まりと終わり)を指す)。第二にこの永遠は、継起を欠き全体が
同時に存在する。

\\


{\scshape Ad primum ergo dicendum} quod simplicia consueverunt per negationem
definiri, sicut punctus est cuius pars non est. Quod non ideo est, quod
negatio sit de essentia eorum, sed quia intellectus noster, qui primo
apprehendit composita, in cognitionem simplicium pervenire non potest,
nisi per remotionem compositionis.

&

第一異論に対しては、それゆえ、次のように言われるべきである。単純なものは否定
によって定義されるのが慣習である。たとえば、点は「部分のないもの」と定義
される。これは、否定がそれらの本質に属するからというのではなく、第一に
複合体を捉える私たちの知性が、複合を除去することによってでなければ単純
なものの認識へ到達することができないからである。

\\

{\scshape Ad secundum dicendum} quod illud quod est vere aeternum, non solum est
ens, sed vivens, et ipsum vivere se extendit quodammodo ad operationem,
non autem esse. Protensio autem durationis videtur attendi secundum
operationem, magis quam secundum esse, unde et tempus est numerus motus.

&

第二異論に対しては、次のように言われるべきである。真に永遠であるものは、存在
するものであるだけでなく、生きるものである。そして生きることは、なんらか
のかたちで働きまで自らを広げるが、存在することはそうでない。ところで、持
続の延長は、存在よりもむしろ働きにしたがって見いだされるように思われる。
このことから、時間も運動の数なのである。

\\

{\scshape Ad tertium dicendum} quod aeternitas dicitur tota, non quia
habet partes, sed inquantum nihil ei deest.

&

第三異論に対しては、次のように言われるべきである。永遠は、部分を持つから全体
と言われるのではなくて、それになにも欠けていないからそう言われる。

\\

{\scshape Ad quartum dicendum} quod, sicut Deus, cum sit incorporeus,
nominibus rerum corporalium metaphorice in Scripturis nominatur, sic
aeternitas, tota simul existens, nominibus temporalibus successivis.

&

第四異論に対しては、次のように言われるべきである。ちょうど神が、非物体的であるた
めに聖書の中で物体的事物の名称によって比喩的に名付けられているように、
永遠もまた、全体が同時に存在するので、時間的で継起的な名称によってその
ように[=比喩的に]名付けられる。


\\

{\scshape Ad quintum dicendum} quod in tempore est duo considerare,
scilicet ipsum tempus, quod est successivum; et nunc temporis, quod est
imperfectum. Dicit ergo tota simul, ad removendum tempus, et perfecta,
ad excludendum nunc temporis.

&

第五異論に対しては、次のように言われるべきである。時間においては二つのことを
考察できる。すなわち、継起的なものである時間それ自体と、不完全なものであ
る時間の今である。ゆえに、時間を排除するために「全体的で同時の」と言われ、
時間の今を排除するために「完全な」と言われる。


\\

{\scshape Ad sextum dicendum} quod illud quod possidetur, firmiter et
quiete habetur. Ad designandam ergo immutabilitatem et indeficientiam
aeternitatis, usus est nomine possessionis.

&


第六異論に対しては、次のように言われるべきである。所有されるものは、堅く静か
に持たれる。ゆえに、永遠の不変性と無欠陥性を示すために、所有という言葉が
使われた。

\end{longtable}


\newpage
\rhead{a.~2}

\begin{center}
 {\Large {\bf ARTICULUS SECUNDUS}}\\
 {\large UTRUM DEUS SIT AETERNUS}\\
 {\footnotesize I {\itshape Sent.}, d.~19, q.~2, a.~1; I {\itshape
 SCG.}, c.~15; {\itshape De Pot.}, q.~3, a.~17, ad 23; {\itshape
 Compend.~Theol.}, c.~5, 8.}\\
 {\Large 第二項\\神は永遠か}
\end{center}


\begin{longtable}{p{21em}p{21em}}

{\huge A}{\scshape d secundum sic proceditur}. Videtur quod Deus non sit aeternus. Nihil
enim factum potest dici de Deo. Sed aeternitas est aliquid factum, dicit
enim Boetius quod {\itshape nunc fluens facit tempus, nunc stans facit
aeternitatem}; et Augustinus dicit, in libro {\it Octoginta trium Quaest.},
quod {\itshape Deus est auctor aeternitatis}. Ergo Deus non est
 aeternus.

&

第二項の問題へ議論は以下のように進められる。神は永遠でないと思われる。
理由は以下の通り。作られたものは、なにものも、神について語られえない。
しかるに、永遠は、作られたなにかである。じっさい、ボエティウスは「流れ
る今が時間を作り、留まる今が永遠を作る」と言うし、アウグスティヌスも、
『八十三問題集』の中で、「神は永遠の作者である」と言っている。ゆえに、
神は永遠でない。

\\

2.~{\scshape Praeterea}, quod est ante aeternitatem et post aeternitatem, non
mensuratur aeternitate. Sed Deus est ante aeternitatem, ut dicitur in
libro {\it de Causis}, et post aeternitatem; dicitur enim {\it Exod. XV}, quod
{\it dominus regnabit in aeternum et ultra}. Ergo esse aeternum non convenit
Deo.
&

さらに、永遠の前と永遠の後のものは、永遠によって測られない。しかるに、神
は、『原因論』という書物の中で言われるように、永遠の前にある。また、『出
エジプト記』15章で「主は永遠に、そして永遠を越えて支配するだろう」
\footnote{「主は代々限りなく統べ治められる。」(15:18)}と言われているので、
永遠の後にもある。ゆえに、永遠であることは神に適さない。

\\

3.~{\scshape Praeterea}, aeternitas mensura quaedam est. Sed Deo non
convenit esse mensuratum. Ergo non competit ei esse aeternum.  

&

さらに、永遠は、一種の尺度である。
しかるに、「測られる」ということは神にふさわしくない。
ゆえに、永遠であることは神に適合しない。

\\

4.~{\scshape Praeterea}, in aeternitate non est praesens, praeteritum
vel futurum cum sit tota simul, ut dictum est. Sed de Deo dicuntur in
Scripturis verba praesentis temporis, praeteriti vel futuri. Ergo Deus
non est aeternus.

&

さらに、すでに述べられたように、(永遠は)全体が同時
にあるのだから、永遠の中に、現在、過去、未来はない。しかるに、神について、
聖書の中で、現在、過去、未来の言葉が語られている。ゆえに、神は永遠でない。

\\

{\scshape Sed contra est} quod dicit Athanasius, aeternus pater, aeternus
filius, aeternus spiritus sanctus.

&

しかし反対に、アタナシウスは、「永遠なる父、永遠なる子、永遠なる聖霊」と
言う。

\\


{\scshape Respondeo dicendum} quod ratio aeternitatis consequitur
immutabilitatem, sicut ratio temporis consequitur motum, ut ex dictis
patet. Unde, cum Deus sit maxime immutabilis, sibi maxime competit esse
aeternum. Nec solum est aeternus, sed est sua aeternitas, cum tamen
nulla alia res sit sua duratio, quia non est suum esse. Deus autem est
suum esse uniforme, unde, sicut est sua essentia, ita est sua
aeternitas.

&

解答する。以下のように言われるべきである。すでに述べられたことから明らか
なとおり、ちょうど「時間」という性格が運動に伴うように、「永遠」という性
格は、不変性に伴う。したがって、神は最大限に不変なのだから、永遠であるこ
とは、最大限に神に適合する。また、神はたんに永遠であるだけでなく、神は自
らの永遠である。これに対して、神以外の他の事物は、どれも、自らの持続では
ない。なぜなら、自らの存在でないからである。しかし神は、自らの一様な存在
であるので、自らの本質であるのと同様に、自らの永遠である。

\\


{\scshape Ad primum ergo dicendum} quod nunc stans dicitur facere
aeternitatem, secundum nostram apprehensionem. Sicut enim causatur in
nobis apprehensio temporis, eo quod apprehendimus fluxum ipsius nunc,
ita causatur in nobis apprehensio aeternitatis, inquantum apprehendimus
nunc stans. -- Quod autem dicit Augustinus, quod {\itshape Deus est auctor
aeternitatis}, intelligitur de aeternitate participata, eo enim modo
communicat Deus suam aeternitatem aliquibus, quo et suam
immutabilitatem.

&

第一異論に対しては、それゆえ、次のように言われるべきである。
留まる今が永遠を作ると言われるのは、私たちの理解に応じてのことである。
ちょうど、私たちが、今の流れを捉えることによって時間を捉えるように、私た
ちは、留まる今を捉える限りで、私たちの中に、永遠の理解が生じる。
また、アウグスティヌスが、神は永遠の作者だと言うということは、分有された永遠につ
いて理解される。なぜなら、神は、自らの不変性を伝えるように、自らの永遠性
を、あるものどもに伝えるからである。

\\


Et per hoc patet solutio ad secundum. Nam Deus dicitur esse ante
aeternitatem, prout participatur a substantiis immaterialibus. Unde et
ibidem dicitur, quod {\it intelligentia parificatur aeternitati}. 

&

そして、これによって第二異論にたいする解答も明らかである。すなわち、神
が永遠の前に存在すると言われるのは、神が、非質料的実体によって分有され
る限りにおいてである。それゆえ、同箇所でも、「知性体が永遠に等しいもの
とされる」と述べられている。

\\

-- Quod autem
dicitur in {\it Exodo}, {\it dominus regnabit in aeternum et ultra
sciendum} quod aeternum accipitur ibi pro saeculo, sicut habet alia
translatio. Sic igitur dicitur quod regnabit ultra aeternum, quia durat
ultra quodcumque saeculum, idest ultra quamcumque durationem datam,
nihil est enim aliud saeculum quam periodus cuiuslibet rei, ut dicitur
in libro {\it I de caelo}. 

&

また、『出エジプト記』で「主が永遠に、永遠を越えて支配するだろう」と言
われているのは、他の翻訳にあるように、ここの「永遠」が「時代」という意
味で理解される限りにおいてである。それゆえ、この意味では、「永遠を越え
て支配するだろう」と言われるのは、どんな時代も超えて、すなわち、与えら
れたどんな持続も越えて持続する、ということである。なぜなら、時代とは、
『天体論』第1巻で言われるように、各々の事物の「周期」に他ならないから
である。

\\

Vel dicitur etiam ultra aeternum regnare,
quia, si etiam aliquid aliud semper esset (ut motus caeli secundum
quosdam philosophos), tamen Deus ultra regnat, inquantum eius regnum est
totum simul.  

&

あるいは、「永遠を越えて支配する」と言われるのは、
(ある哲学者たちによれば天の運動がそうであるように)もしかりに何か他のも
のが常に存在したとしても、(神の支配は全体を同時にである限りで)神はそれ
を越えて支配するからである(とも考えられる)。


\\


{\scshape Ad tertium dicendum} quod aeternitas non est aliud quam ipse
Deus. Unde non dicitur Deus aeternus, quasi sit aliquo modo mensuratus,
sed accipitur ibi ratio mensurae secundum apprehensionem nostram tantum.

&

第三異論に対しては、次のように言われるべきである。永遠は、神自身に他な
らない。したがって、神が永遠だと言われるのは、なんらかのかたちで測られ
たものとしてではなく、ここで尺度の性格が理解されるのは、私たちの理解に
したがうかぎりのことにすぎない。

\\


{\scshape Ad quartum dicendum} quod verba diversorum temporum attribuuntur Deo,
inquantum eius aeternitas omnia tempora includit, non quod ipse varietur
per praesens, praeteritum et futurum.

&

第四異論に対しては、次のように言われるべきである。さまざまな時間の言葉
が神に帰せられるのは、神の永遠にすべての時間が含まれるからであって、神
が、現在、過去、未来によって変化するからではない。

\end{longtable}

\newpage
\rhead{a.~3}


\begin{center}
 {\Large {\bf ARTICULUS TERTIUS}}\\
 {\large UTRUM ESSE AETERNUM SIT PROPRIUM DEI}\\
 {\footnotesize I {\itshape Sent.}, d.~8, q.~2, a.~2; IV d.~44, q.~1,
 a.~2, q.~3; {\itshape Quodl.}~X, q.~2; {\itshape De Div.~Nom.}, c.~10,
 lect.~3; {\itshape De Causis}, lect.~2.}\\
 {\Large 第三項\\永遠であることは神に固有か}
\end{center}


\begin{longtable}{p{21em}p{21em}}
{\huge A}{\scshape d tertium sic proceditur}. Videtur quod esse aeternum
non sit soli Deo proprium. Dicitur enim {\it Danielis XII}, quod
{\itshape qui ad iustitiam erudiunt plurimos, erunt quasi stellae in
perpetuas aeternitates}. Non autem essent plures aeternitates, si solus
Deus esset aeternus. Non igitur solus Deus est aeternus.


&

第三項の問題へ議論は以下のように進められる。永遠であることは神だけに固
有ではないと思われる。理由は以下の通り。『ダニエル書』12章で、「きわめ
て多くの人々を義へと教え導く者は、永続する永遠に、星のようになるであろ
う」\footnote{「目覚めた人々は大空の光のように輝き、多くの者の救いとなっ
た人々は、とこしえに星と輝く。」(12:3)}と言われている。しかるに、もし
神だけが永遠であったならば、複数の永遠はなかったであろう。ゆえに、神だ
けが永遠だというわけではない。

\\


2.~{\scshape Praeterea}, {\it Matth}.~{\scshape xxv} dicitur, {\itshape
Ite, maledicti, in ignem aeternum}. Non igitur solus Deus est aeternus.

&

さらに、『マタイによる福音書』25章で、「呪われた人々よ、永遠の火に入れ」
 \footnote{「『呪われた者ども、わたしから離れ去り、悪魔とその手下のため
 に用意してある永遠の火に入れ。」(25:41)}
と言われている。ゆえに、神だけが永遠なのではない。

\\

3.~{\scshape Praeterea}, omne necessarium est aeternum. Sed multa sunt
necessaria; sicut omnia principia demonstrationis, et omnes
propositiones demonstrativae. Ergo non solus Deus est aeternus.


&


さらに、すべて必然的なものは永遠である。
しかるに、すべての、論証の原理や、すべての論証的命題のように、多くのもの
が必然的である。ゆえに、神だけが永遠だというわけではない。

\\

{\scshape Sed contra est} quod dicit Hieronymus, ad Marcellam, {\itshape
Deus solus est qui exordium non habet}. Quidquid autem exordium habet,
non est aeternum. Solus ergo Deus est aeternus.  

&

しかし反対に、ヒエロニュムスは、マルセラ宛書簡で、神だけが発端を持たない、
と語っている。しかるに、発端を持つものはなんであれ、永遠ではない。ゆえに、
神だけが永遠である。


\\

{\scshape Respondeo dicendum} quod aeternitas vere et proprie in solo Deo est. Quia
aeternitas immutabilitatem consequitur, ut ex dictis patet. Solus autem
Deus est omnino immutabilis, ut est superius ostensum. 
Secundum tamen quod aliqua ab ipso immutabilitatem percipiunt, secundum
hoc aliqua eius aeternitatem participant.

&

解答する。以下のように言われるべきである。永遠は、真に、そして固有に、
ただ神においてのみある。理由は以下の通り。すでに述べられたことから明ら
かなとおり、永遠は、不変性に伴う。しかるに、上で示されたように、ただ神
のみが、あらゆる点において不変である。しかし、あるものどもが、神から不
変性を受け取る限りにおいて、神の永遠性を分有する。

\\

Quaedam ergo quantum ad hoc immutabilitatem sortiuntur a Deo, quod
nunquam esse desinunt, et secundum hoc dicitur {\it Eccle}.~{\scshape
i} de terra, quod {\itshape in aeternum stat}.

&

ゆえに、あるものどもは、けっして存在をやめないというところまで、神から不
変性を受け取る。そしてこの限りで、『コヘレトの言葉』1章で、地について、
「永遠にある」\footnote{「永遠に耐えるのは大地。」(1:5)}と言われている。

\\

Quaedam etiam aeterna in Scripturis dicuntur propter diuturnitatem
durationis, licet corruptibilia sint, sicut in {\it Psalmo} dicuntur
{\itshape montes aeterni}; et {\it Deuter}.~{\scshape xxxiii} etiam dicitur, {\itshape de
pomis collium aeternorum}.

&

またあるものどもは、たとえば『詩編』で「永遠の山々」\footnote{「あなた
が、餌食の山々から」(76:5)}と言われているように、あるいは、『申命記』
第33章には「永遠の丘々の果実から」\footnote{「いにしえの山々のもたらす
最上の物、とこしえの丘の賜物」 (33:15)}と言われるように、消滅しうるも
のであっても、持続の長さのために、聖書の中で「永遠な」と言われる。

\\

Quaedam autem amplius participant de ratione aeternitatis, inquantum
habent intransmutabilitatem vel secundum esse, vel ulterius secundum
operationem, sicut Angeli et beati, qui verbo fruuntur, quia quantum ad
illam visionem verbi, non sunt in sanctis volubiles cogitationes, ut
dicit Augustinus, XV {\it de Trin.} Unde et videntes Deum dicuntur
habere vitam aeternam, secundum illud {\it Ioann}.~{\scshape xvii}, {\itshape haec
est vita aeterna, ut cognoscant} et cetera.

&


さらにあるものどもは、存在にかんして不変性を持つ限りで、
あるいは、御言葉を享受する天使や至福者のように、さらに働きにかんしても不
変性を持つ限りで永遠という性格を分
有する。アウグスティヌスが『三位一体論』で言うように、かの御言葉を見るこ
とにかんして、聖なる者たちの中に転変する思惟はないからである。
このことから、『ヨハネによる福音書』17章「神を知ることは永遠の生命である云々
」によれば、神を見る人々が、永遠の生命を持つと言われる。

\\


{\scshape Ad primum ergo dicendum} quod dicuntur multae aeternitates,
secundum quod sunt multi participantes aeternitatem ex ipsa Dei
contemplatione.

&

第一異論に対しては、それゆえ、次のように言われるべきである。多くの永遠
が語られるのは、神を観照すること自体にもとづいて永遠性を分有する、多く
の者がいるということにしたがってである。

\\

{\scshape Ad secundum dicendum} quod ignis Inferni dicitur aeternus
propter interminabilitatem tantum. Est tamen in poenis eorum
transmutatio, secundum illud {\it Iob} {\scshape xxiv}, {\itshape ad nimium
calorem transibunt ab aquis nivium}. Unde in Inferno non est vera
aeternitas, sed magis tempus; secundum illud {\it Psalmi}, {\itshape erit tempus
eorum in saecula}.

&

第二異論対しては、次のように言われるべきである。地獄の火が永遠と言われるの
は、終わりがないということのためだけにである。しかし、『ヨブ記』24章「彼
らは雪の水から過激な熱へ移りゆくだろう」\footnote{「暑さと乾燥が雪解け水
をも消し去るように、陰府は罪人を消し去るだろう。」(24:19)}とあるのによれ
ば、彼らの罰には変化がある。したがって、地獄において、本当の永遠はなく、
かの『詩編』「彼らの時間は世々に及ぶだろう」\footnote{「主を憎む者が主に
屈服し、この運命が永劫に続くように。」(81:16)}によれば、むしろ地獄にある
のは時間である。

\\

{\scshape Ad tertium dicendum} quod necessarium significat quendam modum
veritatis. Verum autem, secundum philosophum, VI {\it Metaphys.}, est in
intellectu. Secundum hoc igitur vera et necessaria sunt aeterna, quia
sunt in intellectu aeterno, qui est intellectus divinus solus. Unde non
sequitur quod aliquid extra Deum sit aeternum.

&

第三異論に対しては、次のように言われるべきである。必然的なものは、真理のある
様態を表示する。ところで、哲学者の『形而上学』第6巻によれば、真は知性にお
いてある。ゆえに、これによれば、真や必然的なものが永遠であるのは、永遠の
知性のなかにあるからであり、そして、そのような知性は、ただ神の知性だけで
ある。ゆえに、神以外の何かが永遠だということは帰結しない。


\end{longtable}

\newpage
\rhead{a.~4}

\begin{center}
 {\Large {\bf ARTICULUS QUARTUS}}\\
 {\large UTRUM AETERNITAS DIFFERAT A TEMPORE}\\
 {\footnotesize Infra, a.~5; I {\itshape Sent.}, d.~8, q.~2, a.~2;
 d.~19, q.~2, a.~1; {\itshape De Pot.}, q.~3, a.~14, ad 10, 18;
 {\itshape De Div.~Nom.}, c.~10, lect.~3.}\\
 {\Large 第四項\\永遠は時間と異なるか}
\end{center}

\begin{longtable}{p{21em}p{21em}}

{\huge A}{\scshape d quartum sic proceditur}. Videtur quod aeternitas non sit aliud a
tempore. Impossibile est enim duas esse mensuras durationis simul, nisi
una sit pars alterius, non enim sunt simul duo dies vel duae horae; sed
dies et hora sunt simul, quia hora est pars diei. Sed aeternitas et
tempus sunt simul, quorum utrumque mensuram quandam durationis
importat. Cum igitur aeternitas non sit pars temporis, quia aeternitas
excedit tempus et includit ipsum; videtur quod tempus sit pars
aeternitatis, et non aliud ab aeternitate.

&

第四項の問題へ議論は以下のように進められる。永遠は時間と異ならないと思
われる。理由は以下の通り。二つのものが同時に持続の尺度であることはでき
ず、それができるのは一方の尺度が他方の尺度の部分である場合に限る。たと
えば、二つの「日」や二つの「時」が同時にあることはできないが、「日」と
「時」は同時にありうる。なぜなら「時」は「日」の部分だから。しかし永遠
と時間は同時にあり、このどちらも持続のある種の尺度を意味している。ゆえ
に永遠は時間の部分でないのだから、というのも永遠は時間を超え時間を含む
からだが、時間は永遠の部分であり永遠と別のものではないと思われる。

\\

2.~{\scshape Praeterea}, secundum philosophum in IV {\it Physic.},
{\itshape nunc} temporis manet idem in toto tempore. Sed hoc videtur
constituere rationem aeternitatis, quod sit idem indivisibiliter se
habens in toto decursu temporis. Ergo aeternitas est {\itshape nunc}
temporis. Sed nunc temporis non est aliud secundum substantiam a
tempore. Ergo aeternitas non est aliud secundum substantiam a tempore.


&

さらに『自然学』第4巻の哲学者によれば、時間の今は全時間において同じも
のに留まる。しかるに時間の全期間において不可分の状態で同一であるという
ことは、永遠の性格を構成すると思われる。ゆえに永遠は時間の今である。し
かるに、時間の今は、実体において時間と別のものではない。ゆえに永遠は、
実体において時間と別のものではない。


\\


3.~{\scshape Praeterea}, sicut mensura primi motus est mensura omnium
motuum, ut dicitur in IV {\it Physic.}, ita videtur quod mensura primi esse
sit mensura omnis esse. Sed aeternitas est mensura primi esse, quod est
esse divinum. Ergo aeternitas est mensura omnis esse. Sed esse rerum
corruptibilium mensuratur tempore. Ergo tempus vel est aeternitas, vel
aliquid aeternitatis.

&

さらに、『自然学』第4巻でいわれるように、第一の運動の尺度はすべての運
動の尺度である。ちょうどそのように、第一の存在の尺度は、すべての存在の
尺度であるように思われる。しかるに、永遠は第一の存在、つまり神の存在の
尺度である。ゆえに、永遠はすべての存在の尺度である。しかるに、可滅的諸
事物の存在は時間によって測られる。ゆえに時間は永遠であるか、または永遠
の何かである。

\\


{\scshape Sed contra est} quod aeternitas est tota simul, in tempore
autem est prius et posterius. Ergo tempus et aeternitas non sunt idem.

&

しかし反対に、永遠は全体が同時にあるが、時間には前後がある。
ゆえに時間と永遠は同じではない。


\\


{\scshape Respondeo dicendum} quod manifestum est tempus et aeternitatem
non esse idem. Sed huius diversitatis rationem quidam assignaverunt ex
hoc quod aeternitas caret principio et fine, tempus autem habet
principium et finem.


&

解答する。以下のように言われるべきである。時間と永遠が同じでないことは
明らかである。しかし、この違いの理由を、ある人々は、永遠が始まりと終わ
りを欠くのに対し、時間が始まりと終わりを持つということから指定した。

\\



Sed haec est differentia per accidens, et non per se. Quia dato quod
tempus semper fuerit et semper futurum sit, secundum positionem eorum
qui motum caeli ponunt sempiternum, adhuc remanebit differentia inter
aeternitatem et tempus, ut dicit Boetius in libro {\it de Consolat.}, ex
hoc quod aeternitas est tota simul, quod tempori non convenit, quia
aeternitas est mensura esse permanentis, tempus vero est mensura motus.

&

しかしこの違いは附帯的なものであり、自体的なものではない。なぜなら、天
の運動が永続的だとする人々の立場にしたがって、かりに時間が常にあったし
常にあるであろうとしても、ボエティウスが『哲学の慰め』という書物で述べ
るように、永遠は全体が同時にであるが、これは時間に適合しないということ
から、永遠と時間の違いはなお残るだろうからである。永遠は常に留まる存在
の尺度であるのに対し、時間は運動の尺度なのだから。

\\

Si tamen praedicta differentia attendatur quantum ad mensurata, et non
quantum ad mensuras, sic habet aliquam rationem, quia solum illud
mensuratur tempore, quod habet principium et finem in tempore, ut
dicitur in IV {\it Physic}. Unde si motus caeli semper duraret, tempus
non mensuraret ipsum secundum suam totam durationem, cum infinitum non
sit mensurabile; sed mensuraret quamlibet circulationem, quae habet
principium et finem in tempore.

&

しかし、上述の違いが、尺度にかんしてではなく、尺度によって測られるものに
かんして見いだされるのであれば、それには一理あるのであって、なぜなら、
『自然学』第4巻で述べられるように、時間のうちに始まりと終わりを持つものだ
けが、時間によって測られるからである。それゆえ、もし天の運動が常に持続し
たならば、時間は、そのような持続全体に即して、それ[=天の運動]を測るこ
とはできなかったであろう。無限のものは測られないからである。むしろ、時間
は、時間のうちに始まりと終わりを持つ任意の周期を測ったであろう。



\\



Potest tamen et aliam rationem habere ex parte istarum mensurarum, si
accipiatur finis et principium in potentia. Quia etiam dato quod tempus
semper duret, tamen possibile est signare in tempore et principium et
finem, accipiendo aliquas partes ipsius, sicut dicimus principium et
finem diei vel anni, quod non contingit in aeternitate. Sed tamen istae
differentiae consequuntur eam quae est per se et primo, differentiam,
per hoc quod aeternitas est tota simul, non autem tempus.

&

しかし、始まりと終わりが可能態において理解されるのであれば、尺度の側から
も、別の合理的な意味合いを持つことができる。
なぜなら、かりに時間が常に持続するとしても、ちょうど私たちが日や年の始ま
りと終わりを言うように、時間のなんらかの部分を捉えることによって、
時間の中に始まりと終わりを指定することは可能である。このことは、永
遠においては起こらない。
しかし、この違いは、永遠は全体が同時であるのに対して時間はそうではないと
いう、自体的で第一の差異に伴うのである。

\\


{\scshape Ad primum ergo dicendum} quod ratio illa procederet, si tempus
et aeternitas essent mensurae unius generis, quod patet esse falsum, ex
his quorum est tempus et aeternitas mensura.

&

第一異論に対しては、それゆえ、次のように言われるべきである。かの理論は、
もし時間と永遠が一つの類に属する尺度だったら進んだであろう[=正しかっ
たであろう]。しかしそれが偽であることは、時間と永遠がそれぞれ何の
尺度かということから明らかである。

\\




{\scshape Ad secundum dicendum} quod nunc temporis est idem subiecto in toto
tempore, sed differens ratione, eo quod, sicut tempus respondet motui,
ita nunc temporis respondet mobili; mobile autem est idem subiecto in
toto decursu temporis, sed differens ratione, inquantum est hic et
ibi. Et ista alternatio est motus. Similiter fluxus ipsius nunc,
secundum quod alternatur ratione, est tempus. Aeternitas autem manet
eadem et subiecto et ratione. Unde aeternitas non est idem quod nunc
temporis.

&

第二異論に対しては次のように言われるべきである。時間の今は、時間全体の
中で基体において同一だが、観点(ratio)において異なる。理由は以下の通り。
ちょうど時間が運動に対応するように、時間の今は動きうるものに対応する。
しかるに、動きうるものは時間の全経過を通して基体においては同一だが、
\kenten{ここ}と\kenten{そこ}という点で、観点において異なる。そしてこの
変化が運動である。同様に、今そのものの流れは、観点によって変化を受ける
限りで時間である。これに対して永遠は、基体と観点において同一に留
まる。したがって永遠は時間の今と同一ではない。

\\


{\scshape Ad tertium dicendum} quod, sicut aeternitas est propria
mensura ipsius esse, ita tempus est propria mensura motus. Unde
secundum quod aliquod esse recedit a permanentia essendi et subditur
transmutationi, secundum hoc recedit ab aeternitate et subditur
tempori. Esse ergo rerum corruptibilium, quia est transmutabile, non
mensuratur aeternitate, sed tempore. Tempus enim mensurat non solum
quae transmutantur in actu, sed quae sunt transmutabilia. Unde non
solum mensurat motum, sed etiam quietem; quae est eius quod natum est
moveri, et non movetur.


&

第三異論に対しては、次のように言われるべきである。ちょうど永遠が、存在
そのものに固有の尺度であるように、時間は、運動に固有の尺度である。した
がって、ある存在が存在の永続性から退き、変化に服するかぎりにおいて、そ
の限りで、永遠から退き、時間に服する。ゆえに、可滅的事物の存在は、変化
しうるので、永遠ではなく、時間によって測られる。じっさい、時間は、現実
に変化するものだけでなく、変化しうるものをも測る。このため、時間は運動
を測るだけでなく、静止も測るのである。静止は、動かされる本性を持ちなが
ら、動かされていないものに属する。

\end{longtable}

\newpage
\rhead{a.~5}


\begin{center}
 {\Large {\bf ARTICULUS QUINTUS}}\\
 {\large DE DIFFERENTIA AEVI ET TEMPORIS}\\
 {\footnotesize I {\itshape Sent.}, d.~8, q.~2, a.~2; d.~19, q.~2, a.~1;
 II, d.~2, q.~1, a.~1; {\itshape De Pot.}, q.~3, q.~14, ad 18; {\itshape
 Quodl.}~X, q.~2.}\\
 {\Large 第五項\\永劫と時間の違いについて}
\end{center}


\begin{longtable}{p{21em}p{21em}}

{\huge A}{\scshape d quintum sic proceditur}. Videtur quod aevum non sit
 aliud a tempore. Dicit enim Augustinus, VIII {\it super Gen.~ad Litt.},
 quod {\itshape Deus movet creaturam spiritualem per tempus}. Sed aevum dicitur
 esse mensura spiritualium substantiarum. Ergo tempus non differt ab
 aevo.


&

第五項の問題へ議論は以下のように進められる。永劫は時間と異ならないと思
われる。アウグスティヌスは『創世記逐語注解』第8巻で「神は霊的被造物
を時間をとおして動かす」と述べている。しかるに永劫は霊的実体の尺度
だと言われている。ゆえに時間は永劫と異ならない。


\\

2.~{\scshape Praeterea}, de ratione temporis est quod habeat prius et
 posterius, de ratione vero aeternitatis est quod sit tota simul, ut
 dictum est. Sed aevum non est aeternitas, dicitur enim {\it Eccli. I},
 quod sapientia aeterna {\itshape est ante aevum}. Ergo non est totum simul, sed
 habet prius et posterius, et ita est tempus.


&


さらに、すでに述べられたとおり、時間の性格には「より先より後」ということ
が含まれるが、永遠の性格には「全体が同時に」ということが含まれる。しかるに
永劫は永遠ではない。なぜなら『シラ書』第1章に、永遠の知恵が「永劫の前に
ある」と言われているからである。ゆえに、永劫は全体が同時にではなく、よ
り先とより後があり、したがって時間である。


\\



3.~{\scshape Praeterea}, si in aevo non est prius et posterius,
sequitur quod in aeviternis non differat esse vel fuisse vel futurum
esse. Cum igitur sit impossibile aeviterna non fuisse, sequitur quod
impossibile sit ea non futura esse. Quod falsum est, cum Deus possit
ea reducere in nihilum.

&

さらに、もし永劫の中により先とより後がなければ、永劫的なものにおいて、
「ある」「あった」「あるだろう」は異ならないことが帰結する。ゆえに、永
劫的なものが「存在しなかった」ことは不可能なのだから、それらが「存在し
ないであろう」も不可能であることが帰結するが、これは誤りである。なぜな
ら、神はそれらを無へ帰することができるのだから。



\\


4.~{\scshape Praeterea}, cum duratio aeviternorum sit infinita ex parte
 post, si aevum sit totum simul, sequitur quod aliquod creatum sit
 infinitum in actu, quod est impossibile. Non igitur aevum differt a
 tempore.


&

さらに、永劫的なものの持続は、「後」の側に無限だから、もし永劫が、全体
が同時に、であるならば、創造された何かが現実に無限であることになる。こ
れは不可能である。ゆえに、永劫は時間と異ならない。

\\


{\scshape Sed contra est} quod dicit Boetius, {\itshape qui tempus ab
 aevo ire iubes}.


&

しかし反対に、ボエティウスは「(あなたは)時間が永劫から出ていくように命
じる」と述べている。


\\


{\scshape Respondeo dicendum} quod aevum differt a tempore et ab
aeternitate, sicut medium existens inter illa. Sed horum differentiam
aliqui sic assignant, dicentes quod aeternitas principio et fine caret;
aevum habet principium, sed non finem; tempus autem habet principium et
finem. Sed haec differentia est per accidens, sicut supra dictum est,
quia si etiam semper aeviterna fuissent et semper futura essent, ut
aliqui ponunt; vel etiam si quandoque deficerent, quod Deo possibile
esset, adhuc aevum distingueretur ab aeternitate et tempore.

&

解答する。以下のように言われるべきである。永劫は、時間と永遠の中間のも
のとして、時間とも永遠とも異なる。しかし、これらの違いを、ある人たちは、
以下のように言って指摘している。すなわち、永遠は始めと終わりを欠き、永
劫は始めをもつが終わりをもたず、時間は始めと終わりをもつ、と。しかし、
上で述べられたように、この差異は付帯的なものである。なぜなら、ある人々
がそう考えるように、もし、永劫的なものがつねに在ったしつねに在るであろ
うとしても、あるいは、神にはそれが可能であるように、永劫的なものがなく
なることがあるとしても、なお、永劫は、永遠や時間から区別されるだろうか
ら。

\\


Alii vero assignant
differentiam inter haec tria, per hoc quod aeternitas non habet prius et
posterius; tempus autem habet prius et posterius cum innovatione et
veteratione; aevum habet prius et posterius sine innovatione et
veteratione. Sed haec positio implicat contradictoria. Quod quidem
manifeste apparet, si innovatio et veteratio referantur ad ipsam
mensuram. Cum enim prius et posterius durationis non possint esse simul,
si aevum habet prius et posterius, oportet quod, priore parte aevi
recedente, posterior de novo adveniat, et sic erit innovatio in ipso
aevo, sicut in tempore. 


&


また別の人々は、これら三つの違いを、永遠は前後をもたず、時間は新しくなっ
たり古くなったりすることを伴う前後をもち、永劫は、新しくなったり古くなっ
たりせずに前後をもつ、ということによって指摘する。しかし、この考えは矛
盾を含む。このことは、新しくなることや古くなることが、尺度自体に関係す
る場合に、明らかとなる。たとえば、持続の前後は同時にありえないのだから、
もし永劫が前後をもてば、永劫の先の方の部分が退いたときに、後の方が新た
にやってきて、このようにして永劫の中に、ちょうど時間の中に起こるように、
「新しくなる」ということが生じるだろう。


\\

Si vero referantur ad mensurata, adhuc sequitur
inconveniens. Ex hoc enim res temporalis inveteratur tempore, quod habet
esse transmutabile, et ex transmutabilitate mensurati, est prius et
posterius in mensura, ut patet ex IV {\it Physic}. Si igitur ipsum aeviternum
non sit inveterabile nec innovabile, hoc erit quia esse eius est
intransmutabile. Mensura ergo eius non habebit prius et posterius. 


&


他方、(新しくなることや古くなることが)測られる側に関係するとしても、
やはり不都合が生じる。なぜなら、時間的な事物が、時間によって古くなるの
は、変化しうる存在をもつことに基づくのだが、『自然学』第4巻で明らかな
とおり、測られるものの可変性に基づいて、尺度の中に前後があるからである。
ゆえに、もしも、永劫的なもの自体が、古くなることも新しくなることもあり
えないのならば、このことは、それの存在が不変的だからであろう。ゆえに、
それの尺度は前後をもたないであろう。
 
\\


Est
ergo dicendum quod, cum aeternitas sit mensura esse permanentis,
secundum quod aliquid recedit a permanentia essendi, secundum hoc
recedit ab aeternitate. Quaedam autem sic recedunt a permanentia
essendi, quod esse eorum est subiectum transmutationis, vel in
transmutatione consistit, et huiusmodi mensurantur tempore; sicut omnis
motus, et etiam esse omnium corruptibilium. Quaedam vero recedunt minus
a permanentia essendi, quia esse eorum nec in transmutatione consistit,
nec est subiectum transmutationis, tamen habent transmutationem
adiunctam, vel in actu vel in potentia. 

&

ゆえに、次のように言われるべきである。永遠は、永続する存在の尺度なのだ
から、あるものが存在の永続性から離れるほど、永遠性からも離れる。あるも
のどもは、それらの存在が、変化に服し、変化において成立するというかたち
で永遠から離れる。このようなものは、時間によって測られる。たとえば、す
べての運動や、すべて可滅的なものどもの存在もまた、このようなものである。
またあるものどもは、これほど存在の永続性から離れていない。なぜなら、そ
れらの存在が変化において成立するのではなく、変化に服するのでもないが、
現実態においてであれ可能態においてであれ、付加された転変をもつからであ
る。


\\


Sicut patet in corporibus
caelestibus, quorum esse substantiale est intransmutabile; tamen esse
intransmutabile habent cum transmutabilitate secundum locum. Et
similiter patet de Angelis, quod habent esse intransmutabile cum
transmutabilitate secundum electionem, quantum ad eorum naturam
pertinet; et cum transmutabilitate intelligentiarum et affectionum, et
locorum suo modo. Et ideo huiusmodi mensurantur aevo, quod est medium
inter aeternitatem et tempus.

&

これはちょうど、天体の場合に明らかである。すなわち天体の実体的存在は不変
 的だが、天体は不変的な存在を、場所における変化性とともにもっている。
また同様に、天使についても明らかである。すなわち天使は不変的な存在をもつ
 が、それとともに、それらの本性にかんする限りでは、選択における可変性を
 もち、また、(附帯性にかんする限りでは)知性認識と感情、そして、天使の
 あり方なりの場所の変化性をもつ。
それゆえ、そのようなものは、永遠と時間のあいだである永劫によって測られる。

\\

Esse autem quod mensurat aeternitas, nec
est mutabile, nec mutabilitati adiunctum. Sic ergo tempus habet prius et
posterius, aevum autem non habet in se prius et posterius, sed ei
coniungi possunt, aeternitas autem non habet prius neque posterius,
neque ea compatitur.
&

これに対して、永遠を測る存在は、変化しえず、変化に結びつけられているの
でもない。このようにして、それゆえ、時間は前後をもつが、永劫は、自らの
うちに前後を持たない。しかし、永劫に前後が結びつけられることはありうる。
これに対して永遠は前後をもたず、前後が永遠に受け取られることもない(前
後が永遠と両立することもない)。

\\

{\scshape Ad primum ergo dicendum} quod creaturae spirituales, quantum
 ad affectiones et intelligentias, in quibus est successio, mensurantur
 tempore. Unde et Augustinus ibidem dicit quod per tempus moveri, est
 per affectiones moveri. Quantum vero ad eorum esse naturale,
 mensurantur aevo. Sed quantum ad visionem gloriae, participant
 aeternitatem.

&

第一異論に対しては、それゆえ、次のように言われるべきである。霊的被造物
は、感情と知性認識にかんして、それらの中に継起があるので、時間によって
測られる。このことから、アウグスティヌスは、同箇所で、それらが時間によっ
て動くというのは、 感情によって動くということだと述べている。これに対
して、彼らの自然本性的存在にかんしては、永劫によって測られる。しかし、
栄光の直観にかんしては、永遠を分有する。


\\

{\scshape Ad secundum dicendum} quod aevum est totum simul, non tamen
 est aeternitas, quia compatitur secum prius et posterius.


&

第二異論に対しては次のように言われるべきである。永劫は、全体が同時に存
在する。しかし、永劫は永遠でない。なぜなら、永劫は前後と両立するからで
ある。

\\

{\scshape Ad tertium dicendum} quod in ipso esse Angeli in se
considerato, non est differentia praeteriti et futuri, sed solum
secundum adiunctas mutationes. Sed quod dicimus Angelum esse vel fuisse
vel futurum esse, differt secundum acceptionem intellectus nostri, qui
accipit esse Angeli per comparationem ad diversas partes temporis.


&


第三異論に対しては次のように言われるべきである。それ自体で考察された天
使の存在そのものの中に、過去と未来の差異はない。あるとすればただ、結び
つけられた変化においてである。しかし、私たちが、天使がいる、いた、いる
だろう、と言うことは、天使の存在を時間のさまざまな部分との関連によって
理解する私たちの知性の理解において異なる。

\\


Et cum dicit Angelum esse
vel fuisse, supponit aliquid cum quo eius oppositum non subditur divinae
potentiae, cum vero dicit futurum esse, nondum supponit aliquid. Unde,
cum esse et non esse Angeli subsit divinae potentiae, absolute
considerando, potest Deus facere quod esse Angeli non sit futurum, tamen
non potest facere quod non sit dum est, vel quod non fuerit postquam
fuit.

&

また、天使がいる、とか、いた、と言うときには、それの対立物が、それとと
もには神の能力のもとにないような何かあるものを想定しているが、しかし、
(天使が)未来にいる、と言うときには、まだ何も想定していない。したがっ
て、天使の存在と非存在は神の能力に服するのだから、無条件的に考察される
ならば、神は、天使の存在を、将来存在しないようにすることはできる。しか
し、天使が存在しているあいだは、存在しないようにはできないし、存在した
あとに、存在しなかったことにはできない。


\\


{\scshape Ad quartum dicendum} quod duratio aevi est infinita, quia non
 finitur tempore. Sic autem esse aliquod creatum infinitum, quod non
 finiatur quodam alio, non est inconveniens.


&


第四異論に対しては、次のように言われるべきである。永劫の持続は、時間に
よって限られていないので、無限である。しかし、このように、被造のなんら
かの存在が、他の何かに限られていないために無限であることは、不都合では
ない。

\end{longtable}


\newpage
\rhead{a.~6}

\begin{center}
 {\Large {\bf ARTICULUS SEXTUS}}\\
 {\large UTRUM SIT UNUM AEVUM TANTUM}\\
 {\footnotesize II {\itshape Sent.}, d.~2, q.~1, a.~2; {\itshape
 Quodl.}~V, q.~4; Opusc.~XXXVI, {\itshape de Instant.}, c.~3.}\\
 {\Large 第六項\\ただ一つだけの永劫があるか}
\end{center}

\begin{longtable}{p{21em}p{21em}}

{\huge A}{\scshape d sextum sic proceditur}. Videtur quod non sit tantum unum
aevum. Dicitur enim in apocryphis Esdrae, {\itshape maiestas et potestas aevorum
est apud te, Domine}.

&

第六項の問題へ議論は以下のように進められる。永劫はただ一つだけではない
と思われる。なぜなら、外典『エズラ書』に、「主よ、永劫(複数)の威厳と
権能があなたのもとにある」と言われているからである。

\\


2.~{\scshape Praeterea}, diversorum generum diversae sunt mensurae. Sed quaedam
aeviterna sunt in genere corporalium, scilicet corpora caelestia,
quaedam vero sunt spirituales substantiae, scilicet Angeli. Non ergo est
unum aevum tantum.

&

さらに、異なる類には異なる尺度がある。しかるに、天体のように、物体的な
ものの類の内にある永劫的なものもあれば、別の永劫的なもの、すなわち、天
使は、霊的な実体である。ゆえに、ただ一つの永劫があるわけではない。



\\

3.~{\scshape Praeterea}, cum aevum sit nomen durationis, quorum est unum aevum, est
una duratio. Sed non omnium aeviternorum est una duratio, quia quaedam
post alia esse incipiunt, ut maxime patet in animabus humanis. Non est
ergo unum aevum tantum.

&

さらに、永劫は持続の名称なので、それについて一つの永劫があるものどもに
は、一つの持続が属する。しかるに、すべての永劫的なものどもに一つの持続
があるわけではない。なぜなら、(永劫的なもののうちの)あるものは、他の
もののあとに存在し始めるからである。このことは、人間の魂においてもっと
も明らかである。ゆえに、ただ一つの永劫があるわけではない。



\\


4.~{\scshape Praeterea}, ea quae non dependent ab invicem, non videntur habere unam
mensuram durationis, propter hoc enim omnium temporalium videtur esse
unum tempus, quia omnium motuum quodammodo causa est primus motus, qui
prius tempore mensuratur. Sed aeviterna non dependent ab invicem, quia
unus Angelus non est causa alterius. Non ergo est unum aevum tantum.

&

さらに、相互に依存しないものは、一つの持続の尺度を持たないように思われ
る。なぜなら、すべて時間的なものに一つの時間が属すると思われるのは、す
べての運動のある種の原因が第一の運動であり、それが、先ず、時間によって
測られるからである。しかし、永劫的なものどもは、相互に依存しない。なぜ
なら、一つの天使が他の天使の原因ではないからである。ゆえに、永劫は一つ
だけではない。

\\



{\scshape Sed contra}, aevum est simplicius tempore, et propinquius se habens ad
aeternitatem. Sed tempus est unum tantum. Ergo multo magis aevum.
&

しかし反対に、永劫は時間より単純であり、より永遠に近い。しかし、時間は
ただ一つである。ゆえに、ましてや永劫は、ただ一つである。



\\



{\scshape Respondeo dicendum} quod circa hoc est duplex opinio, quidam enim dicunt
quod est unum aevum tantum; quidam quod multa. Quid autem horum verius
sit, oportet considerare ex causa unitatis temporis, in cognitionem enim
spiritualium per corporalia devenimus. 

&

解答する。以下のように言われるべきである。これについては二通りの意見が
ある。ある人々\footnote{Alexander of Hales (c.1185 -- 21 August 1245),
{\it ST} 1, 66(山田)}は、ただ一つの永劫があると言い、別の人々
\footnote{Bonaventura (1221 -- 15 July 1274), {\it In II Sent}., d. 2,
q. 1, a. 2(山田)}は、多くの永劫があると言う。これらのどちらがより真
であるか(を見るためには)、時間の一性の原因から考察しなければならない。
私たちは、物体的なものを通して霊的なものの認識へ至るからである。

\\


Dicunt autem quidam esse unum tempus omnium temporalium, propter hoc
quod est unus numerus omnium numeratorum, cum tempus sit numerus,
secundum philosophum.  Sed hoc non sufficit, quia tempus non est
numerus ut abstractus extra numeratum, sed ut in numerato existens,
alioquin non esset continuus; quia decem ulnae continuitatem habent,
non ex numero, sed ex numerato. Numerus autem in numerato existens non
est idem omnium, sed diversus diversorum.

&

さて、ある人々は\footnote{Averroes (April 14, 1126 -- December 10,
1198). cf. Aquinas's {\it In II Sent}., d. 2, q. 1, a. 2(山田)}、す
べての時間的なものどもに一つの時間があるのは、すべての数えられるものど
もに一つの数があるからだと述べる。なぜなら、哲学者によれば、時間は数な
のだから。しかしこれは十分でない。なぜなら、時間は、数えられるものの外
に抽象されたものとしての数ではなく、数えられるものの中に存在するものと
しての数だからである。そうでなければ、時間は連続的なものでなかったであ
ろう。というのも、10尺の布が連続性を持つのは、数にもとづいてではなく、
数えられるもの(=布)にもとづいてだから。ところが、数えられるもののの
中にある数は、すべてのものに同一ではなく、異なるものに応じて異なる。


\\

Unde alii assignant causam unitatis temporis ex unitate
aeternitatis, quae est principium omnis durationis. Et sic, omnes
durationes sunt unum, si consideretur earum principium, sunt vero
multae, si consideretur diversitas eorum quae recipiunt durationem ex
influxu primi principii.

&

このことから、他の人々\footnote{Alexander of Hales}は、時間の一性の原
因を、すべての持続の根源である永遠の一性から指定する。このようにして、
すべての持続は、もしそれらの根源が考察されるならば一つであり、また、第
一根源の流入から持続を受け取るものどもの多様性が考察されるならば多であ
る。


\\


Alii vero assignant causam unitatis temporis ex
parte materiae primae, quae est primum subiectum motus, cuius mensura
est tempus.

&

また、他の人々\footnote{Bonaventura}は、時間の一性の原因を、第一質料の
側から指定する。第一質料は、運動〔その尺度は時間である〕の第一の基体だ
からである。

\\


Sed neutra assignatio sufficiens videtur, quia ea quae sunt
unum principio vel subiecto, et maxime remoto, non sunt unum simpliciter
sed secundum quid.

&

しかし、どちらの指定も十分であるとは思えない。なぜなら、根源において一で
あるものや基体において一であるものは、しかももっとも離れた根源や基体において
一であるものは、端的な一ではなく、ある意味における一だからである。

\\


Est ergo ratio unitatis temporis, unitas primi motus, secundum quem, cum
sit simplicissimus, omnes alii mensurantur, ut dicitur in X {\it Metaphys}.

&

ゆえに、時間の一性の根拠は、『形而上学』第10巻で言われるように、もっと
も単純であるために、他のすべての運動がそれによって測られる、第一の運動
\footnote{cf. {\it In VIII Physic.}, l. 23 n. 5. Huius igitur rationis
maior manifesta est. Sed ad evidentiam minoris propositionis,
considerandum quod in corporibus caelestibus invenitur duplex motus:
{\bf unus qui est totius firmamenti, quo scilicet totum firmamentum
revolvitur ab oriente in occidentem motu diurno; et iste est primus
motus}: alius motus est quo stellae moventur e converso ab occidente
in orientem. In hoc autem secundo motu, tanto unumquodque caelestium
corporum velocius movetur, quanto propinquius est centro; ut patet
secundum computationem astrologorum, qui motui lunae deputant tempus
unius mensis, soli vero, Mercurio et Veneri unum annum, Marti autem
duos, Iovi duodecim, Saturno triginta, et stellis fixis triginta sex
millia annorum.\\◎月--1月、太陽、水星、金星--1年、火星--2年、木星--12
年、土星--30年、固定星--36,000年。外惑星の公転周期をほぼ正しく観測して
いた。}の一性である。


\\



Sic ergo tempus ad illum motum comparatur non solum ut mensura ad
mensuratum, sed etiam ut accidens ad subiectum; et sic ab eo recipit
unitatem.

&

かくして、それゆえ、時間がその(第一の)運動にたいする関係は、尺度が測られるもの
にたいする関係ではなく、むしろ附帯性が基体にたいする関係である。こ
のようにして、(時間は)それ(第一の運動)から一性を受け取る。


\\


Ad alios autem motus comparatur solum ut
mensura ad mensuratum. Unde secundum eorum multitudinem non
multiplicatur, quia una mensura separata multa mensurari possunt. 

&

これに対して、(時間は)他の(第一の運動以外の)運動に対して、ただ尺度が測られるものにたい
するように関係する。したがって、それら(測られるもの)の多性に応じて多
数化されることはない。なぜなら、一つの離れた尺度によって、多くのものが
測られうるからである。

\\


Hoc igitur habito, sciendum quod de substantiis spiritualibus duplex
fuit opinio.  Quidam enim dixerunt quod omnes processerunt a Deo in
quadam aequalitate, ut Origenes dixit; vel etiam multae earum, ut quidam
posuerunt. 

&

以上のことを理解した上で、霊的実体について二通りの意見があったことが知
られるべきである。ある人々は、オリゲネスが言ったように、\kenten{すべて
の}霊的実体が、あるいは、ある人々\footnote{Bonaventura}が考えたように、
(すべてではないにしても)やはり\kenten{それらの多く}は、神から一種の
等しさにおいて発出したと言った。


\\


Alii vero dixerunt quod omnes substantiae spirituales processerunt a Deo
quodam gradu et ordine et hoc videtur sentire Dionysius, qui dicit,
cap.~{\scshape x} {\it Cael.~Hier.}, quod inter substantias spirituales
sunt primae, mediae et ultimae, etiam in uno ordine Angelorum.

&

他方、別の人々は、すべての霊的実体は、神から一種の段階と秩序によって発出
したと言った。『天上階級論』10章で、霊的実体のあいだには、最初、中間、最
後があり、天使たちの一つの秩序においてもそれがある、と述べるディオニュシ
ウスは、こう考えていると思われる。



\\

Secundum igitur primam opinionem, necesse est dicere quod sunt plura
aeva, secundum quod sunt plura aeviterna prima aequalia.
 &

ゆえに、第一の意見によれば、複数の、等しい第一の永劫的なものが存在する
かぎりにおいて、複数の永劫が存在すると言う必要がある。



\\

Secundum autem secundam opinionem, oportet dicere quod sit
unum aevum tantum, quia, cum unumquodque mensuretur simplicissimo sui
generis, ut dicitur in X {\it Metaphys.}, oportet quod esse omnium
aeviternorum mensuretur esse primi aeviterni, quod tanto est simplicius,
quanto prius. Et quia secunda opinio verior est, ut infra ostendetur,
concedimus ad praesens unum esse aevum tantum.

&

しかし、第二の意見によれば、永劫は一つだけだと言わなければならない。な
ぜなら、『形而上学』第10巻で言われるように、各々のものは、自分の類のもっ
とも単純なものによって測られるのだから、すべての永劫的なものの存在は、
第一の永劫的なもの(より先であるほど、より単純である)によって測られる
からである。そしてあとで示されるように\footnote{STI, q.47, a.2; STI,
 q.50, a.4.}、第二の意見の方が、より真であ
るので、永劫がただ一つであるということを、現時点では、私たちは認める。


\\


{\scshape Ad primum ergo dicendum} quod aevum aliquando accipitur pro {\itshape saeculo}, quod
est periodus durationis alicuius rei, et sic dicuntur multa aeva, sicut
multa saecula.

&

第一異論に対しては、それゆえ、次のように言われるべきである。「永劫」は、
「時代」と理解されることがある。時代とは、なんらかの事物の持続の期間で
ある。そのかぎりで、多くの時代という意味で、多くの永劫と言われる。

\\

{\scshape Ad secundum dicendum} quod, licet corpora caelestia et spiritualia
differant in genere naturae, tamen conveniunt in hoc, quod habent esse
intransmutabile. Et sic mensurantur aevo.

&

第二異論に対しては次のように言われるべきである。天体と霊的諸事物が、本
性の類において異なるとしても、それらは、不変的な存在を持つという点で一
致する。この限りで、それらは永劫によって測られる。

\\

{\scshape Ad tertium dicendum} quod nec omnia temporalia simul incipiunt, et tamen
omnium est unum tempus, propter primum quod mensuratur tempore. Et sic
omnia aeviterna habent unum aevum propter primum, etiam si non omnia
simul incipiant.

&

第三異論に対しては、次のように言われるべきである。すべての時間的なもの
も、同時に始まるわけではないが、時間によって測られる第一のもの(=第一
の運動)のために、すべてのものに一つの時間がある。同様に、すべて永劫的
なものは、すべてが同時に始まるわけではないにせよ、第一の(永劫的な)も
ののために一つの永劫を持つ


\\

{\scshape Ad quartum dicendum} quod ad hoc quod aliqua mensurentur per aliquod
unum, non requiritur quod illud unum sit causa omnium eorum; sed quod
sit simplicius.

&

第四異論に対しては、次のように言われるべきである。あるものどもが、なに
か一つのものによって測られるために、その一つのものは、それらのすべてよ
り単純である必要があるが、それらの原因である必要はない。


\end{longtable}

\end{document}
