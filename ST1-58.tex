\documentclass[10pt]{jsarticle} % use larger type; default would be 10pt
%\usepackage[utf8]{inputenc} % set input encoding (not needed with XeLaTeX)
%\usepackage[round,comma,authoryear]{natbib}
%\usepackage{nruby}
\usepackage{okumacro}
\usepackage{longtable}
%\usepqckage{tablefootnote}
\usepackage[polutonikogreek,english,japanese]{babel}
%\usepackage{amsmath}
\usepackage{latexsym}
\usepackage{color}

%----- header -------
\usepackage{fancyhdr}
\lhead{{\it Summa Theologiae} I, q.~58}
%--------------------

\bibliographystyle{jplain}

\title{{\bf PRIMA PARS}\\{\HUGE Summae Theologiae}\\Sancti Thomae
Aquinatis\\{\sffamily QUEAESTIO QUINQUAGESIMAOCTAVA}\\DE MODO
COGNITIONIS ANGELICAE}
\author{Japanese translation\\by Yoshinori {\sc Ueeda}}
\date{Last modified \today}


%%%% コピペ用
%\rhead{a.~}
%\begin{center}
% {\Large {\bf }}\\
% {\large }\\
% {\footnotesize }\\
% {\Large \\}
%\end{center}
%
%\begin{longtable}{p{21em}p{21em}}
%
%&
%
%
%\\
%\end{longtable}
%\newpage



\begin{document}
\maketitle
\pagestyle{fancy}

\begin{center}
{\Large 第五十八問\\天使的認識の様態について}
\end{center}

\begin{longtable}{p{21em}p{21em}}
Post haec considerandum est de modo angelicae cognitionis. Et circa
hoc quaeruntur septem.

&

これらの後に、天使的認識の様態について考察されるべきである。これをめぐっ
て、七つのことが問われる。


\\


\begin{enumerate}
 \item utrum intellectus Angeli quandoque sit in potentia, quandoque in actu.
 \item utrum Angelus possit simul intelligere multa.
 \item utrum intelligat discurrendo.
 \item utrum intelligat componendo et dividendo.
 \item utrum in intellectu Angeli possit esse falsitas.
 \item utrum utrum cognitio Angeli possit dici matutina et vespertina.
 \item utrum sit eadem cognitio matutina et vespertina, vel diversae.
\end{enumerate}


 &

 \begin{enumerate}
  \item 天使の知性は、あるときには可能態にあり、別のあるときには現実態に
	あるか。
  \item 天使は、同時に多くのことを知性認識することができるか。
  \item 推論することによって知性認識するか。
  \item 複合分割することによって知性認識するか。
  \item 天使の知性の中に、偽がありうるか。
  \item 天使の認識は、朝の認識、夕べの認識と言われうるか。
  \item 朝の認識と夕べの認識は同一の認識か、それとも異なる認識か。
 \end{enumerate}

\end{longtable}

\newpage

\rhead{a.~1}
\begin{center}
 {\Large {\bf ARTICULUS PRIMUS}}\\
 {\large UTRUM INTELLECTUS ANGELI QUANDOQUE SIT IN POTENTIA, QUANDOQUE
 IN ACTU}\\
 {\footnotesize I$^a$II$^{ae}$, q.1, a.6; II {\itshape SCG} cap.97, 98,
 101; {\itshape De Malo}, q.16, a.5, 6.}\\
 {\Large 第一項\\天使の知性は、あるときには可能態にあり、別のあるときには現実態に
	あるか。}
\end{center}

\begin{longtable}{p{21em}p{21em}}

{\Huge A}{\scshape d primum sic proceditur}. Videtur quod
intellectus Angeli quandoque sit in potentia. {\itshape Motus} enim {\itshape est actus
existentis in potentia}, ut dicitur III {\itshape Physic}. Sed mentes angelicae
intelligendo moventur, ut dicit Dionysius, {\scshape iv} cap.~{\itshape de Div.~Nom}. Ergo
mentes angelicae quandoque sunt in potentia.


&

第一項の問題へ、議論は以下のように進められる。天使の知性は、可能態にあ
ることがあると思われる。理由は以下の通り。『自然学』第3巻で言われるよ
うに、「運動は可能態にあるものの現実態である」。ところで、ディオニュシ
ウスが『神名論』第4章で言うように、天使の精神は、知性認識することによっ
て動く。ゆえに、天使の精神は、可能態にあることがある。

\\


{\scshape 2 Praeterea}, cum desiderium sit rei non
habitae, possibilis tamen haberi, quicumque desiderat aliquid
intelligere, est in potentia ad illud. Sed I Petri {\scshape i}, dicitur, {\itshape in quem
desiderant Angeli prospicere}. Ergo intellectus Angeli quandoque est in
potentia.


&

さらに、欲求は、所有されていないが所有されうる事物にかかわるので、何か
を知性認識することを欲求するものはだれでも、それに対して可能態にある。
ところが、『ペトロの手紙一』第1章で「天使たちは、その人を見ることを願っ
ている」\footnote{「天使たちも見て確かめたいと願っているものなのです」
(1:12)}と言われている。ゆえに、天使の知性は、可能態にあることがある。

\\


{\scshape 3 Praeterea}, in libro {\itshape de Causis} dicitur quod
intelligentia intelligit {\itshape secundum modum suae substantiae}. Sed substantia
Angeli habet aliquid de potentia permixtum. Ergo quandoque intelligit in
potentia.


&


さらに、『原因論』という書物の中で、諸々の知性体は、「自らの実体のあり方に
即して」知性認識すると言われている。ところで、天使の実体は、可能態に由
来する何らかの混合をもっている。ゆえに、可能態において知性認識すること
がある。

\\


{\scshape Sed contra est} quod Augustinus dicit, II
{\itshape super Gen. ad Litt}., quod Angeli, {\itshape ex quo creati sunt, ipsa Verbi
aeternitate, sancta et pia contemplatione perfruuntur}. Sed intellectus
contemplans non est in potentia, sed in actu. Ergo intellectus Angeli
non est in potentia.


&

しかし反対に、アウグスティヌスは『創世記逐語注解』第2巻で、天使たちは
「創造さたとき以来、御言葉の永遠性と敬虔な観照を享受している」と言って
いる。ところで、観照する知性は可能態になく、現実態にある。ゆえに、天使
の知性は、可能態にない。


\\


{\scshape Respondeo dicendum} quod, sicut philosophus
dicit, in III {\itshape de Anima} et in VIII {\itshape Physic}., intellectus dupliciter est in
potentia, uno modo, {\itshape sicut ante addiscere vel invenire}, idest antequam
habeat habitum scientiae; alio modo dicitur esse in potentia, sicut {\itshape cum
iam habet habitum scientiae, sed non considerat}. Primo igitur modo,
intellectus Angeli nunquam est in potentia respectu eorum ad quae eius
cognitio naturalis se extendere potest. Sicut enim corpora superiora,
scilicet caelestia, non habent potentiam ad esse, quae non sit completa
per actum; ita caelestes intellectus, scilicet Angeli, non habent
aliquam intelligibilem potentiam, quae non sit totaliter completa per
species intelligibiles connaturales eis. 
Sed quantum ad ea quae eis
divinitus revelantur, nihil prohibet intellectus eorum esse in potentia,
quia sic etiam corpora caelestia sunt in potentia quandoque ut
illuminentur a sole.
&

解答する。以下のように言われるべきである。哲学者が『デ・アニマ』第3巻
と『自然学』第8巻で述べるように、知性は二通りのかたちで可能態にある。
一つには、「学んだり発見したりする前に」というしかたであり、すなわちそ
れは、知という習態を持つ前に、というかたちである。もう一つのかたちで可
能態にあると言われるのは、「すでに知という習態を持っているが、[実際に]
考察していない」というしかたである。それゆえ、第一のしかたでは、天使の
知性は、その本性的認識が及びうる範囲のものどもにかんして、けっして可能
態にはない。それはちょうど、上位の諸物体、なわち天体が、現実態によって
完成されていないような、存在への可能態を持たないように、そのように、天
上的知性、すなわち天使たちも、それらに生得的である可知的形象によって、
全面的に完成されていないような、いかなる可知的な可能態ももたない。しか
し、天使たちに神によって啓示された事柄にかんしては、彼らの知性が可能態
にあることを妨げるものは何もない。なぜなら、このかたちでなら、諸天体も
また、太陽に照らされることに対して、可能態にあることがあるからである。



\\

Secundo vero modo, intellectus Angeli potest esse
in potentia ad ea quae cognoscit naturali cognitione, non enim omnia
quae naturali cognitione cognoscit, semper actu considerat. Sed ad
cognitionem verbi, et eorum quae in verbo videt, nunquam hoc modo est in
potentia, quia semper actu intuetur verbum, et ea quae in verbo
videt. In hac enim visione eorum beatitudo consistit, beatitudo autem
non consistit in habitu, sed in actu, ut dicit philosophus, in I {\itshape Ethic}.


&

他方、第二のかたちでは、天使の知性は、本性的認識によって認識することが
らに対して、可能態にありうる。というのも、本性的認識によって認識するす
べてのことを、必ずしも常に現実に考察しているわけではないからである。し
かし、御言葉の認識や、彼らが御言葉において見る事柄の認識にかんしては、
けっして、このようなかたちで可能態にあることはない。なぜなら、御言葉と、
御言葉において見るものを、常に現実に直観するからである。じっさい、この
直視において、彼らの至福は成立するが、『ニコマコス倫理学』第1巻で哲学
者が述べるように、至福は、習態ではなく、現実態において成立するからであ
る。


\\


{\scshape Ad primum ergo dicendum} quod motus ibi non
sumitur secundum quod est {\itshape actus imperfecti}, idest existentis in
potentia; sed secundum quod est {\itshape actus perfecti}, idest existentis in
actu. Sic enim intelligere et sentire dicuntur motus, ut dicitur in III
{\itshape de Anima}.


&

第一異論に対しては、それゆえ、以下のように言われるべきである。運動は、
そこで、「不完全なものの現実態」すなわち、可能態にあるものの現実態とい
う意味ではなく、「完全なものの現実態」すなわち、現実態において在るもの
の現実態という意味で理解されている。『デ・アニマ』第3巻で言われるよう
に、知性認識することや感覚することは、この意味で、運動と言われるからで
ある。


\\


{\scshape Ad secundum dicendum} quod desiderium illud
Angelorum non excludit rem desideratam, sed eius fastidium. --Vel dicuntur
desiderare Dei visionem, quantum ad novas revelationes, quas pro
opportunitate negotiorum a Deo recipiunt.


&

第二異論に対しては、以下のように言われるべきである。
天使たちのかの願望は、願望された事物を排除せず、むしろ、それを嫌悪するこ
とを廃城する。あるいは、天使たちが、その任務の機会に応じて神から受け取
る、新しい啓示にかんして、神を見ることを願う、と言われている。


\\


{\scshape Ad tertium dicendum} quod in substantia Angeli
non est aliqua potentia denudata ab actu. Et similiter nec intellectus
Angeli sic est in potentia, quod sit absque actu.


&

第三異論に対しては、以下のように言われるべきである。
天使の実体の中には、現実態をまとわないいかなる可能態もない。
同様に、天使の知性もまた、現実態をもたないというかたちの可能態にはない。


\end{longtable}

\newpage


%%%% コピペ用
\rhead{a.~2}
\begin{center}
{\Large {\bf ARTICULUS SECUNDUS}}\\
{\large UTRUM ANGELUS SIMUL POSSIT MULTA INTELLIGERE}\\
{\footnotesize II {\itshape Sent.}, d.3, part.2, q.2, a.4; II {\itshape
SCG} cap.101; {\itshape De Verit.}, q.8, a.14.}\\
{\Large 第二項\\天使は多くのものを同時に知性認識できるか}
\end{center}

\begin{longtable}{p{21em}p{21em}}

{\Huge A}{\scshape d secundum sic proceditur}. Videtur quod
Angelus non possit simul multa intelligere. Dicit enim philosophus, II
{\itshape Topic}., quod {\itshape contingit multa scire, sed unum tantum intelligere}.


&


第二項の問題へ、議論は以下のように進められる。天使は、多くのものを同時
に知性認識することができないと思われる。理由は以下の通り。哲学者は『ト
ピカ』第2巻で、「多くのものを知っている、ということはありうるが、知性
認識するのはただ一つだけである」と述べている。

\\


{\scshape 2 Praeterea}, nihil intelligitur nisi secundum
quod intellectus formatur per speciem intelligibilem, sicut corpus
formatur per figuram. Sed unum corpus non potest formari diversis
figuris. Ergo unus intellectus non potest simul intelligere diversa
intelligibilia.


&

さらに、ちょうど物体が形状によって形づけられるように、知性が可知的形象
によって形づけられるのでないかぎり、なにものも知性認識されない。ところ
で、一つの物体が、さまざまな形状によって形づけられることはありえない。
ゆえに、一つの知性が、同時に、さまざまな可知的[形象]によって形づけら
れることはありえない。


\\


{\scshape 3 Praeterea}, intelligere est motus
quidam. Nullus autem motus terminatur ad diversos terminos. Ergo non
contingit simul multa intelligere.


&


さらに、知性認識することは、ある種の運動である。ところで、どんな運動も、
さまざまな終端へ終着することはない。ゆえに、同時に多くを知性認識するこ
とも起こらない。

\\


{\scshape Sed contra est} quod dicit Augustinus, IV
{\itshape Sup.~Gen.~ad Litt}., {\itshape potentia spiritualis mentis angelicae cuncta quae
voluerit, facillime simul comprehendit}.


&


しかし反対に、アウグスティヌスは『創世記逐語注解』第4巻で、「天使の精
神の霊的能力は、意志するすべてのことを、いとも簡単に同時に把握する」と
述べている。

\\


{\scshape Respondeo dicendum} quod, sicut ad unitatem
motus requiritur unitas termini, ita ad unitatem operationis requiritur
unitas obiecti. Contingit autem aliqua accipi ut plura, et ut unum;
sicut partes alicuius continui. Si enim unaquaeque per se accipiatur,
plures sunt, unde et non una operatione, nec simul accipiuntur per
sensum et intellectum. Alio modo accipiuntur secundum quod sunt unum in
toto, et sic simul et una operatione cognoscuntur tam per sensum quam
per intellectum, dum totum continuum consideratur, ut dicitur in III {\itshape de
Anima}. Et sic etiam intellectus noster simul intelligit subiectum et
praedicatum, prout sunt partes unius propositionis; et duo comparata,
secundum quod conveniunt in una comparatione. 
Ex quo patet quod multa,
secundum quod sunt distincta, non possunt simul intelligi; sed secundum
quod uniuntur in uno intelligibili, sic simul intelliguntur. 


&

解答する。以下のように言われるべきである。ちょうど、運動の一性のために、
終端の一性が必要であるように、働きの一性には、対象の一性が必要である。
ところで、なんらかの連続体の諸部分がそうであるように、あるものが、複数
として受け取られたり、一つとして受け取られたりすることがある。というの
も、もし、各々の事物がそれ自体によって受け取られるならば、それらは複数
であり、したがって、感覚や知性を通して、一つの働きによっても、同時にも、
受け取られないからである。別のしかたでは、それらは全体における一として
受け取られる。この場合には、『デ・アニマ』第3巻で言われるように、連続
体全体が考察されているあいだ、知性と同様に感覚によっても、同時に、一つ
の働きによって認識される。そしてこのしかたでは、私たちの知性ですら、主
語と述語を、一つの命題の部分として、同時に知性認識するし、二つの被関係
項を、一つの関係に属するかぎりで、同時に知性認識する。このことから明ら
かなとおり、多数のものは、区別されている限り、同時に知性認識されないが、
一つの可知的なものにおいて合一される限り、その限りにおいて、同時に知性
認識される。


\\



Unumquodque
autem est intelligibile in actu, secundum quod eius similitudo est in
intellectu. Quaecumque igitur per unam speciem intelligibilem cognosci
possunt, cognoscuntur ut unum intelligibile; et ideo simul
cognoscuntur. Quae vero per diversas species intelligibiles
cognoscuntur, ut diversa intelligibilia capiuntur. 



&

ところで、各々のものが現実に可知的であるのは、それの類似が知性の中にあ
る限りにおいてである。ゆえに、何であれ一つの可知的形象によって認識され
うる事物は、一つの可知的なものとして認識され、それゆえに、同時に認識さ
れる。他方、多様な可知的形象によって認識されるものは、多様な可知的なも
のとして捉えられる。


\\


Angeli igitur ea cognitione qua cognoscunt res per verbum, omnia
cognoscunt una intelligibili specie, quae est essentia divina. Et ideo
quantum ad talem cognitionem, omnia simul cognoscunt, sicut et in
patria non erunt volubiles nostrae cogitationes, ab aliis in alia
euntes atque redeuntes, sed omnem scientiam nostram simul uno
conspectu videbimus, ut Augustinus dicit in XV {\itshape de
Trin}. --Ea vero cognitione qua cognoscunt res per species innatas,
omnia illa simul possunt intelligere, quae una specie cognoscuntur;
non autem illa quae diversis.


&

ゆえに、天使たちは、御言葉によって諸事物を認識するその認識によって、万
物を一つの可知的形象、すなわち神の本質によって認識する。それゆえ、その
ような認識にかんする限り、万物を同時に認識する。ちょうど、アウグスティ
ヌスが『三位一体論』15巻で、天国では「あれからこれへ、またあれへ、と行
きつ戻りつする私たちの転変する思惟はなく、私たちのすべての知を、同時に、
一つの眼差しによって見るだろう」と述べるように。他方、生得的な形象によっ
て諸事物を認識するその認識によっては、一つの形象によって認識されるもの
どもであれば、そのすべてを同時に知性認識することができるが、しかし、多
様な可知的形象によって認識されるものどもを、同時に認識することはできな
い。



\\


{\scshape Ad primum ergo dicendum} quod intelligere multa ut unum, est
quodammodo unum intelligere.


&

第一異論に対しては、それゆえ、以下のように言われるべきである。多くのも
のを一つのものとして知性認識することは、ある意味で、一つのものを知性認
識することである。


\\


{\scshape Ad secundum dicendum} quod intellectus formatur per
intelligibilem speciem quam apud se habet. Et ideo sic potest una
specie intelligibili multa simul intelligibilia intueri, sicut unum
corpus per unam figuram potest simul multis corporibus assimilari.


&

知性は、自らのもとに持っている可知的形象によって形成される。それゆえ、
ちょうど、一つの物体が、一つの形状によって、同時に多くの物体に類似化さ
れうるように、知性は、一つの可知的形象によって、多くの可知的なものを同
時に直観することができる。


\\

{\scshape Ad tertium dicendum} sicut ad primum.
&

第三異論に対しては、第一異論に対して言われたのと同様のことが言われるべき
である。

\end{longtable}
\newpage

\rhead{a.~3}
\begin{center}
{\Large {\bf ARTICULUS TERTIUS}}\\
{\large UTRUM ANGELUS COGNOSCAT DISCURRENDO}\\
{\footnotesize Infra, q.79, a.8; q.85, a.5; {\itshape De Verit.}, q.8,
a.15; q.15, a.1.}\\
{\Large 第三項\\天使は推論することによって認識するか}
\end{center}

\begin{longtable}{p{21em}p{21em}}



{\Huge A}{\scshape d tertium sic proceditur}. Videtur quod Angelus
cognoscat discurrendo. Discursus enim intellectus attenditur secundum
hoc, quod unum per aliud cognoscitur. Sed Angeli cognoscunt unum per
aliud, cognoscunt enim creaturas per Verbum. Ergo intellectus Angeli
cognoscit discurrendo.


&

第三項の問題へ、議論は以下のように進められる。天使は、推論することによっ
て認識すると思われる。理由は以下の通り。知性の推論は、一つのものが他の
ものを通して認識される限りで見出される。ところが、天使たちは、一つのも
のを他のものを通して認識する。なぜなら、被造物を御言葉を通して認識する
からである。ゆえに、天使の知性は推論することによって認識する。


\\



{\scshape 2 Praeterea}, quidquid potest virtus inferior,
potest et virtus superior. Sed intellectus humanus potest syllogizare,
et in effectibus causas cognoscere, secundum quae discursus
attenditur. Ergo intellectus Angeli, qui superior est ordine naturae,
multo magis hoc potest.


&

さらに、上位の力は、下位の力ができることなら何でもできる。ところが、人
間知性は三段論法を行うことや、結果において原因を認識することができるが、
それらにおいて推論が見出される。ゆえに、天使の知性は、自然の秩序におい
て上位にあるから、はるかに優ってこれができる。


\\



{\scshape 3 Praeterea}, Isidorus dicit quod Daemones per experientiam
multa cognoscunt. Sed experimentalis cognitio est discursiva,
{\itshape ex multis} enim {\itshape memoriis fit unum experimentum, et
ex multis experimentis fit unum universale}, ut dicitur in fine
{\itshape Poster}., et in principio {\itshape Metaphys}. Ergo cognitio
Angelorum est discursiva.


&

さらに、イシドルスは、霊たちは経験を通して多くのことを認識すると言って
いる。ところが、経験的認識は推論的である。なぜなら、『分析論後書』の最
後の巻と、『形而上学』の最初の巻で言われているように、「多くの記憶から
一つの経験が生じ、多くの経験から一つの普遍が生じる」からである。ゆえに、
天使たちの認識は推論的である。


\\



{\scshape Sed contra} est quod Dionysius dicit, {\scshape vii}
cap.~{\itshape de Div.~Nom.}, quod Angeli {\itshape non congregant divinam cognitionem a
sermonibus diffusis, neque ab aliquo communi ad ista specialia simul
aguntur}.


&

しかし反対に、ディオニュシウスは『神名論』第7章で、天使たちは「流れ出
た諸々の教えから神の認識を集めたりせず、また、なんらかの共通的なことか
らその特殊的なことへ、同時に導かれたりはしない」、と述べている。


\\



{\scshape Respondeo dicendum} quod, sicut saepius dictum est, Angeli
illum gradum tenent in substantiis spiritualibus, quem corpora
caelestia in substantiis corporeis, nam et {\itshape caelestes mentes}
a Dionysio dicuntur. Est autem haec differentia inter caelestia et
terrena corpora, quod corpora terrena per mutationem et motum
adipiscuntur suam ultimam perfectionem, corpora vero caelestia statim,
ex ipsa sua natura, suam ultimam perfectionem habent. Sic igitur et
inferiores intellectus, scilicet hominum, per quendam motum et
discursum intellectualis operationis perfectionem in cognitione
veritatis adipiscuntur; dum scilicet ex uno cognito in aliud cognitum
procedunt. Si autem statim in ipsa cognitione principii noti,
inspicerent quasi notas omnes conclusiones consequentes, in eis
discursus locum non haberet. Et hoc est in Angelis, quia statim in
illis quae primo naturaliter cognoscunt, inspiciunt omnia quaecumque
in eis cognosci possunt.




&

解答する。以下のように言われるべきである。非常にしばしば述べられたとお
り、天使たちは、霊的諸実体の中で、天体が物体的実体の中で占めている階級
を占めている。その証拠に、彼らはディオニュシウスによって、「天の精神」
と呼ばれている。ところで、天の物体と地上の物体とのあいだには、地上の物
体が、変化と運動を通して自らの最終的な完成を獲得するのに対して、天の物
体は、直ちに、自らの本性それ自体に基づいて、自らの最終的な完成を所有し
ている、という違いがある。それゆえ、下位の知性、すなわち人間たちの知性
も、知性の働きの、ある種の運動や奔走[=推論]を通して、すなわち、一つ
の認識内容から別の認識内容へと進みながら、真理の認識における完成を獲得
する。これに対して、もし、知られた原理の認識自体において、直ちに、帰結
するすべての結論を知られたものとして看取したならば、それらにおいて推論
は存在する場所がなかったであろう。そして、これが天使の場合である。なぜ
なら、直ちに、第一に本性的に認識するものどもにおいて、何であれそれらの
中で認識されうるものすべてを看取するからである。



\\

Et ideo dicuntur {\itshape intellectuales}, quia etiam apud nos, ea
quae statim naturaliter apprehenduntur, {\itshape intelligi} dicuntur;
unde {\itshape intellectus} dicitur habitus primorum
principiorum. Animae vero humanae, quae veritatis notitiam per quendam
discursum acquirunt, {\itshape rationales} vocantur. -- Quod quidem
contingit ex debilitate intellectualis luminis in eis. Si enim
haberent plenitudinem intellectualis luminis, sicut Angeli, statim in
primo aspectu principiorum totam virtutem eorum comprehenderent,
intuendo quidquid ex eis syllogizari posset.


&

それゆえ、彼らは「知性的なものども」と呼ばれる。というのも、私たちの中
でも、直ちに本性的に捉えられるものどもは、「知性認識される」と言われる
からである。このことから、「知性」は、第一諸原理の所有態と言われる。他
方、真理の認識をある種の推論を通して獲得する人間の魂は、「理性的なもの」
と呼ばれる。じっさい、このことは、人間の中の知性的な光の弱さのために起
こる。というのも、もし、人間が、天使のように、知性的な光を十分に持って
いたならば、直ちに、諸原理を最初に見たときに、何であれそれらから三段論
法によって導かれうる事柄を直観することによって、それらの全ての力を把握
したであろうから。


\\



{\scshape Ad primum ergo dicendum} quod {\itshape discursus}
quendam motum nominat. Omnis autem motus est de uno priori in aliud
posterius. Unde discursiva cognitio attenditur secundum quod ex aliquo
prius noto devenitur in cognitionem alterius posterius noti, quod prius
erat ignotum. Si autem in uno inspecto simul aliud inspiciatur, sicut
in speculo inspicitur simul imago rei et res; non est propter hoc
cognitio discursiva. Et hoc modo cognoscunt Angeli res in verbo.


&

第一異論に対しては、それゆえ、以下のように言われるべきである。「走り回
り・推論」は、ある種の運動の名前である。ところで、あらゆる運動は、より
先であるものから、それとは別の、より後のものへとある。したがって、推論
的な認識は、何かより先に知られているものから、以前には知られていなかっ
た、別のより後に知られたものの認識へと至ることに即して見出される。しか
し、ちょうど、鏡の中に、事物の像と、事物とが同時に見られる場合のように、
もし、一つの見ることにおいて、同時に他のものもが見られるならば、このた
めに認識が推論的になることはない。そして、このかたちで、天使たちは、御
言葉において諸事物を認識する。



\\



{\scshape Ad secundum dicendum} quod Angeli syllogizare
possunt, tanquam syllogismum cognoscentes; et in causis effectus
vident, et in effectibus causas, non tamen ita quod cognitionem
veritatis ignotae acquirant syllogizando ex causis in causata, et ex
causatis in causas.


&

第二異論に対しては、以下のように言われるべきである。天使たちは、三段論
法を認識するものとして、三段論法を行うことができる。彼らは、諸原因の中
に諸結果を見、諸結果の中に諸原因を見るが、知らなかった真理の認識を、諸
原因から諸結果へと、あるいは諸結果から諸原因へと三段論法することによっ
て獲得するのではない。

\\



{\scshape Ad tertium dicendum} quod experientia in
Angelis et Daemonibus dicitur secundum quandam similitudinem, prout
scilicet cognoscunt sensibilia praesentia; tamen absque omni discursu.


&

第三異論に対しては、以下のように言われるべきである。天使や霊たちの中で
経験と言われるのは、ある種の類似に即してであり、現前する可感的諸事物を
認識する限りにおいてである。しかし、天使たちはそれらをあらゆる推論なし
に認識する。


\end{longtable}
\end{document}

%%%% コピペ用
%\rhead{a.~}
%\begin{center}
% {\Large {\bf }}\\
% {\large }\\
% {\footnotesize }\\
% {\Large \\}
%\end{center}
%
%\begin{longtable}{p{21em}p{21em}}
%
%&
%
%
%\\
%\end{longtable}
%\newpage


Articulus 4

[30829] Iª q. 58 a. 4 arg. 1 Ad quartum sic proceditur. Videtur quod Angeli intelligant componendo et dividendo. Ubi enim est multitudo intellectuum, ibi est compositio intellectuum, ut dicitur in III de anima. Sed in intellectu Angeli est multitudo intellectuum, cum per diversas species diversa intelligat, et non omnia simul. Ergo in intellectu Angeli est compositio et divisio.

[30830] Iª q. 58 a. 4 arg. 2 Praeterea, plus distat negatio ab affirmatione, quam quaecumque duae naturae oppositae, quia prima distinctio est per affirmationem et negationem. Sed aliquas naturas distantes Angelus non cognoscit per unum, sed per diversas species, ut ex dictis patet. Ergo oportet quod affirmationem et negationem cognoscat per diversa. Et ita videtur quod Angelus intelligat componendo et dividendo.

[30831] Iª q. 58 a. 4 arg. 3 Praeterea, locutio est signum intellectus. Sed Angeli hominibus loquentes, proferunt affirmativas et negativas enuntiationes, quae sunt signa compositionis et divisionis in intellectu; ut ex multis locis sacrae Scripturae apparet. Ergo videtur quod Angelus intelligat componendo et dividendo.

[30832] Iª q. 58 a. 4 s. c. Sed contra est quod Dionysius dicit, VII cap. de Div. Nom., quod virtus intellectualis Angelorum resplendet conspicaci divinorum intellectuum simplicitate. Sed simplex intelligentia est sine compositione et divisione, ut dicitur in III de anima. Ergo Angelus intelligit sine compositione et divisione.

[30833] Iª q. 58 a. 4 co. Respondeo dicendum quod, sicut in intellectu ratiocinante comparatur conclusio ad principium, ita in intellectu componente et dividente comparatur praedicatum ad subiectum. Si enim intellectus statim in ipso principio videret conclusionis veritatem, nunquam intelligeret discurrendo vel ratiocinando. Similiter si intellectus statim in apprehensione quidditatis subiecti, haberet notitiam de omnibus quae possunt attribui subiecto vel removeri ab eo, nunquam intelligeret componendo et dividendo, sed solum intelligendo quod quid est. Sic igitur patet quod ex eodem provenit quod intellectus noster intelligit discurrendo, et componendo et dividendo, ex hoc scilicet, quod non statim in prima apprehensione alicuius primi apprehensi, potest inspicere quidquid in eo virtute continetur. Quod contingit ex debilitate luminis intellectualis in nobis, sicut dictum est. Unde cum in Angelo sit lumen intellectuale perfectum, cum sit speculum purum et clarissimum, ut dicit Dionysius, IV cap. de Div. Nom.; relinquitur quod Angelus, sicut non intelligit ratiocinando, ita non intelligit componendo et dividendo. Nihilominus tamen compositionem et divisionem enuntiationum intelligit, sicut et ratiocinationem syllogismorum, intelligit enim composita simpliciter, et mobilia immobiliter, et materialia immaterialiter.

[30834] Iª q. 58 a. 4 ad 1 Ad primum ergo dicendum quod non qualiscumque multitudo intellectuum compositionem causat, sed multitudo illorum intellectuum quorum unum attribuitur alteri, vel removetur ab altero. Angelus autem, intelligendo quidditatem alicuius rei, simul intelligit quidquid ei attribui potest vel removeri ab ea. Unde intelligendo quod quid est, intelligit quidquid nos intelligere possumus et componendo et dividendo, per unum suum simplicem intellectum.

[30835] Iª q. 58 a. 4 ad 2 Ad secundum dicendum quod diversae quidditates rerum minus differunt, quantum ad rationem existendi, quam affirmatio et negatio. Tamen quantum ad rationem cognoscendi, affirmatio et negatio magis conveniunt, quia statim per hoc quod cognoscitur veritas affirmationis, cognoscitur falsitas negationis oppositae.

[30836] Iª q. 58 a. 4 ad 3 Ad tertium dicendum quod hoc quod Angeli loquuntur enuntiationes affirmativas et negativas, manifestat quod Angeli cognoscunt compositionem et divisionem, non autem quod cognoscant componendo et dividendo, sed simpliciter cognoscendo quod quid est.


%%%% コピペ用
%\rhead{a.~}
%\begin{center}
% {\Large {\bf }}\\
% {\large }\\
% {\footnotesize }\\
% {\Large \\}
%\end{center}
%
%\begin{longtable}{p{21em}p{21em}}
%
%&
%
%
%\\
%\end{longtable}
%\newpage


Articulus 5

[30837] Iª q. 58 a. 5 arg. 1 Ad quintum sic proceditur. Videtur quod in intellectu Angeli possit esse falsitas. Protervitas enim ad falsitatem pertinet. Sed in Daemonibus est phantasia proterva, ut dicit Dionysius, IV cap. de Div. Nom. Ergo videtur quod in Angelorum intellectu possit esse falsitas.

[30838] Iª q. 58 a. 5 arg. 2 Praeterea, nescientia est causa falsae aestimationis. Sed in Angelis potest esse nescientia, ut Dionysius dicit, VI cap. Eccles. Hier. Ergo videtur quod in eis possit esse falsitas.

[30839] Iª q. 58 a. 5 arg. 3 Praeterea, omne quod cadit a veritate sapientiae, et habet rationem depravatam, habet falsitatem vel errorem in suo intellectu. Sed hoc Dionysius dicit de Daemonibus, VII cap. de Div. Nom. Ergo videtur quod in intellectu Angelorum possit esse falsitas.

[30840] Iª q. 58 a. 5 s. c. Sed contra, philosophus dicit, III de anima, quod intellectus semper verus est. Augustinus etiam dicit, in libro octoginta trium quaest., quod nihil intelligitur nisi verum. Sed Angeli non cognoscunt aliquid nisi intelligendo. Ergo in Angeli cognitione non potest esse deceptio et falsitas.

[30841] Iª q. 58 a. 5 co. Respondeo dicendum quod huius quaestionis veritas aliquatenus ex praemissa dependet. Dictum est enim quod Angelus non intelligit componendo et dividendo, sed intelligendo quod quid est. Intellectus autem circa quod quid est semper verus est, sicut et sensus circa proprium obiectum, ut dicitur in III de anima. Sed per accidens in nobis accidit deceptio et falsitas intelligendo quod quid est, scilicet secundum rationem alicuius compositionis, vel cum definitionem unius rei accipimus ut definitionem alterius; vel cum partes definitionis sibi non cohaerent, sicut si accipiatur pro definitione alicuius rei, animal quadrupes volatile (nullum enim animal tale est); et hoc quidem accidit in compositis, quorum definitio ex diversis sumitur, quorum unum est materiale ad aliud. Sed intelligendo quidditates simplices, ut dicitur in IX Metaphys., non est falsitas, quia vel totaliter non attinguntur, et nihil intelligimus de eis; vel cognoscuntur ut sunt. Sic igitur per se non potest esse falsitas aut error aut deceptio in intellectu alicuius Angeli; sed per accidens contingit. Alio tamen modo quam in nobis. Nam nos componendo et dividendo quandoque ad intellectum quidditatis pervenimus, sicut cum dividendo vel demonstrando definitionem investigamus. Quod quidem in Angelis non contingit; sed per quod quid est rei cognoscunt omnes enuntiationes ad illam rem pertinentes. Manifestum est autem quod quidditas rei potest esse principium cognoscendi respectu eorum quae naturaliter conveniunt rei vel ab ea removentur, non autem eorum quae a supernaturali Dei ordinatione dependent. Angeli igitur boni, habentes rectam voluntatem, per cognitionem quidditatis rei non iudicant de his quae naturaliter ad rem pertinent, nisi salva ordinatione divina. Unde in eis non potest esse falsitas aut error. Daemones vero, per voluntatem perversam subducentes intellectum a divina sapientia, absolute interdum de rebus iudicant secundum naturalem conditionem. Et in his quae naturaliter ad rem pertinent, non decipiuntur. Sed decipi possunt quantum ad ea quae supernaturalia sunt, sicut si considerans hominem mortuum, iudicet eum non resurrecturum; et si videns hominem Christum, iudicet eum non esse Deum.

[30842] Iª q. 58 a. 5 ad 1 Et per hoc patet responsio ad ea quae utrinque obiiciuntur. Nam protervitas Daemonum est secundum quod non subduntur divinae sapientiae. Nescientia autem est in Angelis, non respectu naturalium cognoscibilium, sed supernaturalium. Patet etiam quod intellectus eius quod quid est semper est verus, nisi per accidens, secundum quod indebite ordinatur ad aliquam compositionem vel divisionem.


%%%% コピペ用
%\rhead{a.~}
%\begin{center}
% {\Large {\bf }}\\
% {\large }\\
% {\footnotesize }\\
% {\Large \\}
%\end{center}
%
%\begin{longtable}{p{21em}p{21em}}
%
%&
%
%
%\\
%\end{longtable}
%\newpage


Articulus 6

[30843] Iª q. 58 a. 6 arg. 1 Ad sextum sic proceditur. Videtur quod in Angelis non sit vespertina neque matutina cognitio. Vespere enim et mane admixtionem tenebrarum habent. Sed in cognitione Angeli non est aliqua tenebrositas; cum non sit ibi error vel falsitas. Ergo cognitio Angeli non debet dici matutina vel vespertina.

[30844] Iª q. 58 a. 6 arg. 2 Praeterea, inter vespere et mane cadit nox; et inter mane et vespere cadit meridies. Si igitur in Angelis cadit cognitio matutina et vespertina, pari ratione videtur quod in eis debeat esse meridiana et nocturna cognitio.

[30845] Iª q. 58 a. 6 arg. 3 Praeterea, cognitio distinguitur secundum differentiam cognitorum, unde in III de anima dicit philosophus quod scientiae secantur quemadmodum et res. Triplex autem est esse rerum, scilicet in verbo, in propria natura, et in intelligentia angelica, ut Augustinus dicit, II super Gen. ad Litt. Ergo, si ponatur cognitio matutina in Angelis et vespertina, propter esse rerum in verbo et in propria natura; debet etiam in eis poni tertia cognitio, propter esse rerum in intelligentia angelica.

[30846] Iª q. 58 a. 6 s. c. Sed contra est quod Augustinus, IV super Gen. ad Litt., et XI de Civ. Dei, distinguit cognitionem Angelorum per matutinam et vespertinam.

[30847] Iª q. 58 a. 6 co. Respondeo dicendum quod hoc quod dicitur de cognitione matutina et vespertina in Angelis, introductum est ab Augustino, qui sex dies in quibus Deus legitur fecisse cuncta, Gen. I, intelligi vult non hos usitatos dies qui solis circuitu peraguntur, cum sol quarto die factus legatur; sed unum diem, scilicet cognitionem angelicam sex rerum generibus praesentatam. Sicut autem in die consueto mane est principium diei, vespere autem terminus, ita cognitio ipsius primordialis esse rerum, dicitur cognitio matutina, et haec est secundum quod res sunt in verbo. Cognitio autem ipsius esse rei creatae secundum quod in propria natura consistit, dicitur cognitio vespertina, nam esse rerum fluit a verbo sicut a quodam primordiali principio, et hic effluxus terminatur ad esse rerum quod in propria natura habent.

[30848] Iª q. 58 a. 6 ad 1 Ad primum ergo dicendum quod vespere et mane non accipiuntur in cognitione angelica secundum similitudinem ad admixtionem tenebrarum; sed secundum similitudinem principii et termini. Vel dicendum quod nihil prohibet, ut dicit Augustinus IV super Gen. ad Litt., aliquid in comparatione ad unum dici lux, et in comparatione ad aliud dici tenebra. Sicut vita fidelium et iustorum, in comparatione ad impios, dicitur lux, secundum illud Ephes. V, fuistis aliquando tenebrae, nunc autem lux in domino; quae tamen vita fidelium, in comparatione ad vitam gloriae, tenebrosa dicitur, secundum illud II Petri I, habetis propheticum sermonem, cui bene facitis attendentes quasi lucernae lucenti in caliginoso loco. Sic igitur cognitio Angeli qua cognoscit res in propria natura, dies est per comparationem ad ignorantiam vel errorem, sed obscura est per comparationem ad visionem verbi.

[30849] Iª q. 58 a. 6 ad 2 Ad secundum dicendum quod matutina et vespertina cognitio ad diem pertinet, idest ad Angelos illuminatos, qui sunt distincti a tenebris, idest a malis Angelis. Angeli autem boni, cognoscentes creaturam, non in ea figuntur, quod esset tenebrescere et noctem fieri; sed hoc ipsum referunt ad laudem Dei, in quo sicut in principio omnia cognoscunt. Et ideo post vesperam non ponitur nox, sed mane, ita quod mane sit finis praecedentis diei et principium sequentis, inquantum Angeli cognitionem praecedentis operis ad laudem Dei referunt. Meridies autem sub nomine diei comprehenditur, quasi medium inter duo extrema. Vel potest meridies referri ad cognitionem ipsius Dei, qui non habet principium nec finem.

[30850] Iª q. 58 a. 6 ad 3 Ad tertium dicendum quod etiam ipsi Angeli creaturae sunt. Unde esse rerum in intelligentia angelica comprehenditur sub vespertina cognitione, sicut et esse rerum in propria natura.


%%%% コピペ用
%\rhead{a.~}
%\begin{center}
% {\Large {\bf }}\\
% {\large }\\
% {\footnotesize }\\
% {\Large \\}
%\end{center}
%
%\begin{longtable}{p{21em}p{21em}}
%
%&
%
%
%\\
%\end{longtable}
%\newpage


Articulus 7

[30851] Iª q. 58 a. 7 arg. 1 Ad septimum sic proceditur. Videtur quod una sit cognitio vespertina et matutina. Dicitur enim Gen. I, factum est vespere et mane dies unus. Sed per diem intelligitur cognitio angelica, ut Augustinus dicit. Ergo una et eadem est cognitio in Angelis matutina et vespertina.

[30852] Iª q. 58 a. 7 arg. 2 Praeterea, impossibile est unam potentiam simul duas operationes habere. Sed Angeli semper sunt in actu cognitionis matutinae, quia semper vident Deum et res in Deo, secundum illud Matth. XVIII, Angeli eorum semper vident faciem patris mei et cetera. Ergo, si cognitio vespertina esset alia a matutina, nullo modo Angelus posset esse in actu cognitionis vespertinae.

[30853] Iª q. 58 a. 7 arg. 3 Praeterea, apostolus dicit, I Cor. XIII, cum venerit quod perfectum est, evacuabitur quod ex parte est. Sed si cognitio vespertina sit alia a matutina, comparatur ad ipsam sicut imperfectum ad perfectum. Ergo non poterit simul vespertina cognitio esse cum matutina.

[30854] Iª q. 58 a. 7 s. c. In contrarium est quod dicit Augustinus, IV super Gen. ad Litt., quod multum interest inter cognitionem rei cuiuscumque in verbo Dei, et cognitionem eius in natura eius, ut illud merito pertineat ad diem, hoc ad vesperam.

[30855] Iª q. 58 a. 7 co. Respondeo dicendum quod, sicut dictum est, cognitio vespertina dicitur, qua Angeli cognoscunt res in propria natura. Quod non potest ita intelligi quasi ex propria rerum natura cognitionem accipiant, ut haec praepositio in indicet habitudinem principii, quia non accipiunt Angeli cognitionem a rebus, ut supra habitum est. Relinquitur igitur quod hoc quod dicitur in propria natura, accipiatur secundum rationem cogniti, secundum quod subest cognitioni; ut scilicet cognitio vespertina in Angelis dicatur secundum quod cognoscunt esse rerum quod habent res in propria natura. Quod quidem per duplex medium cognoscunt, scilicet per species innatas, et per rationes rerum in verbo existentes. Non enim, videndo verbum, cognoscunt solum illud esse rerum quod habent in verbo; sed illud esse quod habent in propria natura; sicut Deus per hoc quod videt se, cognoscit esse rerum quod habent in propria natura. Si ergo dicatur cognitio vespertina secundum quod cognoscunt esse rerum quod habent in propria natura, videndo verbum; sic una et eadem secundum essentiam est cognitio vespertina et matutina, differens solum secundum cognita. Si vero cognitio vespertina dicatur secundum quod Angeli cognoscunt esse rerum quod habent in propria natura, per formas innatas; sic alia est cognitio vespertina et matutina. Et ita videtur intelligere Augustinus, cum unam ponat imperfectam respectu alterius.

[30856] Iª q. 58 a. 7 ad 1 Ad primum ergo dicendum quod, sicut numerus sex dierum, secundum intellectum Augustini, accipitur secundum sex genera rerum quae cognoscuntur ab Angelis; ita unitas diei accipitur secundum unitatem rei cognitae, quae tamen diversis cognitionibus cognosci potest.

[30857] Iª q. 58 a. 7 ad 2 Ad secundum dicendum quod duae operationes possunt simul esse unius potentiae, quarum una ad aliam refertur; ut patet cum voluntas simul vult et finem et ea quae sunt ad finem, et intellectus simul intelligit principia et conclusiones per principia, quando iam scientiam acquisivit. Cognitio autem vespertina in Angelis refertur ad matutinam, ut Augustinus dicit. Unde nihil prohibet utramque simul esse in Angelis.

[30858] Iª q. 58 a. 7 ad 3 Ad tertium dicendum quod, veniente perfecto, evacuatur imperfectum quod ei opponitur, sicut fides, quae est eorum quae non videntur, evacuatur visione veniente. Sed imperfectio vespertinae cognitionis non opponitur perfectioni matutinae. Quod enim cognoscatur aliquid in seipso, non est oppositum ei quod cognoscatur in sua causa. Nec iterum quod aliquid cognoscatur per duo media, quorum unum est perfectius et aliud imperfectius, aliquid repugnans habet, sicut ad eandem conclusionem habere possumus et medium demonstrativum et dialecticum. Et similiter eadem res potest sciri ab Angelo per verbum increatum, et per speciem innatam.
