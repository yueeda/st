\documentclass[10pt]{jsarticle} % use larger type; default would be 10pt
%\usepackage[utf8]{inputenc} % set input encoding (not needed with XeLaTeX)
%\usepackage[round,comma,authoryear]{natbib}
%\usepackage{nruby}
\usepackage{okumacro}
\usepackage{longtable}
%\usepqckage{tablefootnote}
\usepackage[polutonikogreek,english,japanese]{babel}
%\usepackage{amsmath}
\usepackage{latexsym}
\usepackage{color}
\usepackage{fancyhdr}
\pagestyle{fancy}
\lhead{{\it Summa Theologiae} I, q.~44}

\bibliographystyle{jplain}


\title{{\bf PRIMA PARS}\\{\HUGE Summae Theologiae}\\Sancti Thomae
Aquinatis\\{\sffamily QUEAESTIO QUADRAGESIMAQUARTA}\\DE PROCESSIONE
CREATURARUM A DEO, ET DE OMNIUM ENTIUM PRIMA CAUSA}
\author{Japanese translation\\by Yoshinori {\sc Ueeda}}
\date{Last modified \today}



%%%% コピペ用
%\rhead{a.~}
%\begin{center}
% {\Large {\bf }}\\
% {\large }\\
% {\footnotesize }\\
% {\Large \\}
%\end{center}
%
%\begin{longtable}{p{21em}p{21em}}
%
%&
%
%\\
%\end{longtable}
%\newpage


\begin{document}

\maketitle

\begin{center}
{\Large 第四十四問\\被造物の神からの発出について\\そして、すべての有の第
 一原因について}
\end{center}

\begin{longtable}{p{21em}p{21em}}

Post considerationem divinarum personarum, considerandum restat de
 processione creaturarum a Deo. Erit autem haec consideratio tripartita,
 ut primo consideretur de productione creaturarum; secundo, de earum
 distinctione; tertio, de conservatione et gubernatione. Circa primum
 tria sunt consideranda, primo quidem, quae sit prima causa entium;
 secundo, de modo procedendi creaturarum a prima causa; tertio vero, de
 principio durationis rerum. Circa primum quaeruntur quatuor. 

\begin{enumerate}
 \item utrum Deus sit causa efficiens omnium entium.
 \item utrum materia
 prima sit creata a Deo, vel sit principium ex aequo coordinatum
 ei.
 \item utrum Deus sit causa exemplaris rerum, vel sint alia
 exemplaria praeter ipsum.
 \item utrum ipse sit causa finalis rerum.
\end{enumerate}

&

神のペルソナの考察のあと、神からの被造物の発出についての考察が残されてい
 る。さて、この考察には三つの部分があるだろう。第一に、被造物の発出につ
 いて考察されよう。第二に、それらの区別について、第三に、保存と統治につ
 いて考察される。第一については、三つのことが考察されるべきである。第一
 に、有の第一原因はなにか、第二に、被造物が第一原因から発出するしかたに
 ついて、第三に、諸事物の持続の根源について、である。
第一については、四つのことが探究される。
\begin{enumerate}
 \item 神は万有の作出因か。
 \item 第一質料は神によって創造されたか、それとも、神と等しく並び立つ根源
 か。
 \item 神は諸事物の範型因か、それとも、神以外に他の範型があるか。
 \item 神は諸事物の目的因か。
\end{enumerate}

\end{longtable}

\newpage
\rhead{a.~1}
\begin{center}
{\Large ARTICULUS PRIMUS}\\
{\large UTRUM SIT NECESSARIUM OMNE ENS ESSE CREATUM A DEO}\\
{\footnotesize Infra, q.~45, a.~1; II {\itshape Sent.}, d.~1, q.~1,
 a.~2; d.~37, q.~1, a.~2; II {\itshape SCG.}, c.~15; {\itshape De Pot.},
 q.~3, a.~5; {\itshape Compend.~Theol.}, c.~68; Opusc.~XV, {\itshape de
 Angelis}, c.~9; {\itshape De Div.~Nom.}, c.~5, lect.~1.}\\
{\large 第一項\\全ての有が神によって創造されたということは必然的なことか}
\end{center}
\begin{longtable}{p{21em}p{21em}}

{\sc Ad primum sic proceditur}. Videtur quod non sit necessarium omne ens esse
 creatum a Deo. Nihil enim prohibet inveniri rem sine eo quod non est de
 ratione rei, sicut hominem sine albedine. Sed habitudo causati ad
 causam non videtur esse de ratione entium quia sine hac possunt aliqua
 entia intelligi. Ergo sine hac possunt esse. Ergo nihil prohibet esse
 aliqua entia non creata a Deo.

&

第一項の問題へ議論は次のように進められる。全ての有が神によって創造されたことは、必
 然的なことでないと思われる。理由は以下の通り。ある事物が、その事物の性格に属し
 ていないものを持たずに見いだされることを妨げるものはなにもない
 \footnote{Aとはなにかということの中に、Bが属していないならば、AはBなしに
 在りうる。}。たとえば、人間は、白性なしに見いだされうる。しかるに、原因
 から生じたもの(causatum)が、原因に対してもつ関係は、有の性格に属していな
 いと思われる。なぜなら、この関係がなくても、何らかの有が理解されうるか
 らである。ゆえに、この関係なしに有は在りうる。ゆえに、神によって創造され
 ない有が存在することを妨げるものはない。

\\



2 {\sc Praeterea}, ad hoc aliquid indiget causa efficiente, ut sit. Ergo
 quod non potest non esse, non indiget causa efficiente. Sed nullum
 necessarium potest non esse, quia quod necesse est esse, non potest non
 esse. Cum igitur multa sint necessaria in rebus, videtur quod non omnia
 entia sint a Deo.

&

さらに、或るものが作出因を必要とするのは、存在するためである。
ゆえに、存在しないことが不可能なものは、作出因を必要としない。
しかるに、どんな必然的なものも、存在しないことがで不可能である。
なぜなら、存在することが必然的であるものは、存在しないことが不可能だからで
 ある。ゆえに、諸事物においては多くの必然的なことがあるのだから、すべて
 の有が神によって在るのではないと思われる


\\


3 {\sc Praeterea}, quorumcumque est aliqua causa, in his potest fieri
 demonstratio per causam illam. Sed in mathematicis non fit demonstratio
 per causam agentem, ut per philosophum patet, in III {\it Metaphys}. Non
 igitur omnia entia sunt a Deo sicut a causa agente.

&

さらに、なんらかの原因を持つものどもはなんであれ、それらにおいて、その原
 因による論証が行われうる。しかるに、『形而上学』第3巻において哲学者によっ
 て明らかなとおり、数学的なものにおいて、作用因による論証はなされない。
 ゆえに、万有が作用因としての神によって存在するのではないと思われる。

\\


{\sc Sed contra est} quod dicitur {\it Rom}.~{\sc xi}. {\it Ex ipso, et
 per ipsum, et in ipso sunt omnia}.

&
しかし反対に、『ローマの信徒への手紙』11章に「万物は、それによって、
 それを通して、それにおいて在る」\footnote{「すべてのものは、神から出て、
 神によって保たれ、神に向かっているのです。」(11:36)}
 と言われている。

\\

{\sc Respondeo dicendum} quod necesse est dicere omne quod quocumque modo est,
 a Deo esse. Si enim aliquid invenitur in aliquo per participationem,
 necesse est quod causetur in ipso ab eo cui essentialiter convenit;
 sicut ferrum fit ignitum ab igne. 

&

解答する。以下のように言わなければならない。どんなかたちであれ存在するもの
はすべて神によって存在する、と語ることは必然である。なぜなら、もし、分有に
よって、BがCのなかに見出されるならば、Bは、Bが本質的に適合するAによって、C
のなかに生み出されていることが必然である。たとえば、鉄が火によって燃える場
合のように。

\\

Ostensum est autem supra, cum de
 divina simplicitate ageretur, quod Deus est ipsum esse per se
 subsistens. Et iterum ostensum est quod esse subsistens non potest esse
 nisi unum, sicut si albedo esset subsistens, non posset esse nisi una,
 cum albedines multiplicentur secundum recipientia. 

&

ところで、前に神の単純性について論じらたとき、神は自
存する存在そのものであることが示された。またさらに次のことも示された。自存する存在は一つでしかありえない。それはちょうど、もし白性が自
存したならば、白性が多数化されるのはそれを受け取るものによってだから、その
[自存する]白性は一つでしかありえないようなものである。

\\

Relinquitur ergo
 quod omnia alia a Deo non sint suum esse, sed participant esse. Necesse
 est igitur omnia quae diversificantur secundum diversam participationem
 essendi, ut sint perfectius vel minus perfecte, causari ab uno primo
 ente, quod perfectissime est. 

&

ゆえに、神以外のすべてのものは、自らの存
在ではなく、存在を分有することになる。ゆえに、存在をより完全に分有した
り、より不完全に分有したりして、存在のさまざまな分有にしたがって多様化され
るすべてのものが、もっとも完全に存在する一つの第一の有によって原因されるこ
とは必然である。

\\

Unde et Plato dixit quod necesse est ante
 omnem multitudinem ponere unitatem. Et Aristoteles dicit, in II
{\it Metaphys}., quod id quod est maxime ens et maxime verum, est causa omnis
 entis et omnis veri, sicut id quod maxime calidum est, est causa omnis
 caliditatis.

&

このことから、プラトンもまた、あらゆる多の前に一を置くこと
が必要だと述べた。また、アリストテレスも『形而上学』第2巻で、最大限に有で
あり最大限に真であるものが、すべての有とすべての真の原因である、と述べる。
ちょうど、最大限に熱いものが、すべての熱の原因であるように。

\\

{\sc Ad primum ergo dicendum} quod, licet habitudo ad causam non intret
 definitionem entis quod est causatum, tamen sequitur ad ea qua sunt de
 eius ratione, quia ex hoc quod aliquid per participationem est ens,
 sequitur quod sit causatum ab alio. Unde huiusmodi ens non potest esse,
 quin sit causatum; sicut nec homo, quin sit risibile. Sed quia esse
 causatum non est de ratione entis simpliciter, propter hoc invenitur
 aliquod ens non causatum.

&

第一異論に対しては、それゆえ、次のように言われるべきである。たしかに、原因へ
の関係は、原因されたものである有の定義に入らないが、しかし、それによってそ
 の性格をもつところのものに伴う。なぜなら、なにかが分有
による有であることから、他によって原因されたものであることが帰結するから
である。したがって、そのような有は、原因されたものでないならば、存在する
 ことができない。ちょうど、人間が、笑いうるものでないならば[存在しえな
 いように]。しかし、原因されたものであることは、有の性格に端的に属する
 のではない。このために、原因されないなんらかの有[つまり神]が見いだされるのである。


\\

2 {\sc Ad secundum dicendum} quod ex hac ratione quidam moti fuerunt ad
 ponendum quod id quod est necessarium non habeat causam, ut dicitur in
 VIII {\it Physic}. Sed hoc manifeste falsum apparet in scientiis
 demonstrativis, in quibus principia necessaria sunt causae conclusionum
 necessariarum. Et ideo dicit Aristoteles, in V {\it Metaphys}., quod sunt
 quaedam necessaria quae habent causam suae necessitatis. Non ergo
 propter hoc solum requiritur causa agens, quia effectus potest non
 esse, sed quia effectus non esset, si causa non esset. Haec enim
 conditionalis est vera, sive antecedens et consequens sint possibilia,
 sive impossibilia.

&

第二異論に対しては、次のように言われるべきである。
『自然学』第8巻で言われるように、この理由から、ある人々は、必然的なもの
 は原因を持たないと考えるように動かされた。しかし、これが誤りであること
 は、論証的なものどもにおいて明らかである。そこでは、必然的な原理が必然
 的な結論の原因である。ゆえに、アリストテレスは『形而上学』第5巻で、自ら
 の必然性の原因を持つ必然的なものが存在する、と述べている。ゆえに、「結果
 が存在しないことがありうる」ということだけから、作用因が必要とされるので
 はなく、「原因がなかったならば結果がなかったであろう」ということから、
 作用因は必要とされる。なぜなら、この条件文は、前件と後件が可能的なもの
 であっても不可能的なものであっても、真だからである。



\\

3 {\sc Ad tertium dicendum} quod mathematica accipiuntur ut abstracta
 secundum rationem, cum tamen non sint abstracta secundum
 esse. Unicuique autem competit habere causam agentem, secundum quod
 habet esse. Licet igitur ea quae sunt mathematica habeant causam
 agentem, non tamen secundum habitudinem quam habent ad causam agentem,
 cadunt sub consideratione mathematici. Et ideo in scientiis
 mathematicis non demonstratur aliquid per causam agentem.


&

第三異論に対しては、次のように言われるべきである。
数学的なものは、概念において抽象されているものとして受け取られるが、しか
 し、存在において抽象されているわけではない。
しかるに、各々のものに、作用因を持つということが適合するのは、それが存在
 を持つ限りにおいてである。ゆえに、数学的であるものどもが、作用因を持つ
 としても、しかし、それらが作用因に対して持つ関係にしたがって、それらが
 数学者の考察の中に入ってくるわけではない。ゆえに、数学的諸学において、
 なにかが作用因によって証明されることはない。

\end{longtable}

\newpage
\rhead{a.~2}

\begin{center}
{\Large ARTICULUS SECUNDUS}\\
{\large UTRUM MATERIA PRIMA SIT CREATA A DEO}\\
{\footnotesize II {\itshape SCG.}, c.~16; {\itshape De Pot.}, q.~3,
 a.~3; {\itshape Compend.~Theol.}, c.=69; VIII {\itshape Physic.}, lect.~2.}\\
{\large 第二項\\第一質料は神によって創造されたか}
 
\end{center}


\begin{longtable}{p{21em}p{21em}}

{\sc Ad secundum sic proceditur}. Videtur quod materia prima non sit creata a
Deo. Omne enim quod fit, componitur ex subiecto et ex aliquo alio, ut
dicitur in I {\it Physic}. Sed materiae primae non est aliquod subiectum. Ergo
materia prima non potest esse facta a Deo.

&

第二項の問題へ、議論は以下のように進められる。第一質料は神によって創造されたのでは
ないと思われる。理由は以下の通り。『自然学』第1巻で言われるように、すべて生じるも
のは、基体と、それとは別の何かから複合されている。しかるに、第一質料には、
何らの基体もない。ゆえに、第一質料が、神によって作られることはありえない。


\\

2 {\sc Praeterea}, actio et passio dividuntur contra se invicem. Sed
sicut primum principium activum est Deus, ita primum principium passivum
est materia. Ergo Deus et materia prima sunt duo principia contra se
invicem divisa, quorum neutrum est ab alio.

&

さらに、能動と受動は、相互に対立的に分けられる。しかるに、能動的な第一根
 源が神であるように、受動的な第一根源は質料である。ゆえに、神と第一質料
 は、互いに対立的な二つの根源であり、そのうちのどちらかが他方によってあ
 ることはない。

\\


3 {\sc Praeterea}, omne agens agit sibi simile, et sic, cum omne agens
 agat inquantum est actu, sequitur quod omne factum aliquo modo sit in
 actu. Sed materia prima est tantum in potentia, inquantum
 huiusmodi. Ergo contra rationem materiae primae est, quod sit facta.

&

さらに、すべて働くものは自らに似たものを生みだす。それゆえ、すべて働くも
 のは、現実態においてある限りで働くのだから、すべて作られたものは、何ら
 かのかたちで現実態においてあることが帰結する。しかるに、第一質料は、そ
 のようなものである限り、可能態においてのみ在る。
 ゆえに、作られた、ということは、第一質料の性格に反する。

\\

{\sc Sed contra est} quod dicit Augustinus, XII {\it Confess}., {\it Duo fecisti,
 Domine, unum prope te}, scilicet Angelum, {\it aliud prope nihil}, scilicet
 materiam primam.

&
しかし反対に、アウグスティヌスは『告白』第12巻で、「主よ、あなたは二つのものを
 作った。一つは、あなたの近くにあるもの(すなわち天使)を、もう一つ
 は、無に近いもの(すなわち第一質料)を」と述べている。

\\

Respondeo dicendum quod antiqui philosophi paulatim, et quasi
 pedetentim, intraverunt in cognitionem veritatis. A principio enim,
 quasi grossiores existentes, non existimabant esse entia nisi corpora
 sensibilia. Quorum qui ponebant in eis motum, non considerabant motum
 nisi secundum aliqua accidentia, ut puta secundum raritatem et
 densitatem, congregationem et segregationem. 

&

解答する。以下のように言われるべきである。
古代の哲学者たちは、少しずつ、あたかも一歩ずつ歩を進めるかのように、真理
の認識へと入っていった。
最初、彼らはあたかも粗野な者であるかのように、可感的な物体以外の存在があ
るとは考えなかった。それらの人々の中で、諸事物において運動を措定した人々
 は、運動を、何らかの付帯性においてしか考察しなかった。たとえ ば、希薄と濃
 厚、凝集と分離における運動のように。


\\

Et supponentes ipsam
 substantiam corporum increatam, assignabant aliquas causas huiusmodi
 accidentalium transmutationum, ut puta amicitiam, litem\footnote
{
l\={\i}s , l\={\i}tis (old form stlis, stlitis, like stlocus for locus; cf. Quint. 1, 4, 16), f. root star-, in sterno; cf. Germ. streiten, to contend,
I.a strife, dispute, quarrel. 
}, intellectum,
 aut aliquid huiusmodi. 

&

そして、彼らは、物体の創造されない実体を措定して、このような付帯的な変化
 の何らかの原因を指定した。例えば、友愛、闘争、知性、その他そのようなも
 のをである。


\\


Ulterius vero procedentes, distinxerunt per
 intellectum inter formam substantialem et materiam, quam ponebant
 increatam; et perceperunt transmutationem fieri in corporibus secundum
 formas essentiales.
 Quarum transmutationum quasdam causas
 universaliores ponebant, ut obliquum circulum, secundum Aristotelem,
 vel ideas, secundum Platonem.
&

他方、さらに進んだ人たちは、知性によって、実体形相と質料を区別し(この質
料を、彼らは創造されないものとしたのだが)、諸事物において、本質的な形相
にしたがって変化が生じることをとらえた。彼らは、これらの変化の原因のうち、
あるものは、より普遍的なものであると考えた。たとえばそれは、アリストテレ
スによれば、傾いた軌道であり、プラトンによればイデアである。


\\

 Sed considerandum est quod materia per
 formam contrahitur ad determinatam speciem; sicut substantia alicuius
 speciei per accidens ei adveniens contrahitur ad determinatum modum
 essendi, ut homo contrahitur per album. Utrique igitur consideraverunt
 ens particulari quadam consideratione, vel inquantum est hoc ens, vel
 inquantum est tale ens. 
Et sic rebus causas agentes particulares
 assignaverunt.
&

しかし、次のことが考えられなければならない。
質料は形相によって限定された種へと制約される。たとえばある種の実体
 は、それに到来する付帯性によって、ある限定された在り方へと制約される。
ちょうど人間が白によって制約されるように。
それゆえこのどちらの人々も、有をある特定の考察によって、あるいは、この有、
 あの有である限りにおいて考察した。
このようにして、彼らは諸事物に個別的な作用因を指定した。


\\

 Et ulterius aliqui erexerunt
 se ad considerandum ens
 inquantum est ens, et consideraverunt causam rerum, non solum secundum
 quod sunt haec vel talia, sed secundum quod sunt entia. Hoc igitur quod
 est causa rerum inquantum sunt entia, oportet esse causam rerum, non
 solum secundum quod sunt talia per formas accidentales, nec secundum
 quod sunt haec per formas substantiales, sed etiam secundum omne illud
 quod pertinet ad esse illorum quocumque modo. Et sic oportet ponere
 etiam materiam primam creatam ab universali causa entium.

&

さらに、ある人々が有である限りにおける有を考察するために立ち上がり、諸事
物の原因を、\kenten{この}事物であるとか、\kenten{このような}事物であると
いう限りにおいてではなく、それらが有である限りにおいて考察した。ゆえに、
有である限りにおける事物の原因であるものは、たんに付帯的形相によって
\kenten{このような}事物であるかぎりにおいてや、実体的形相によって
\kenten{この}事物であるかぎりにおいてだけでなく、どのようなかたちであれ、
それらの存在に属するあらゆる点においてもまた諸事物の原因でなければなら
ない。そしてこのような観点から、第一質料もまた有の普遍的原因によって創造されたものとしなければならない。

\\




{\scshape Ad primum ergo dicendum} quod philosophus in I {\it Physic}.~loquitur de fieri
particulari, quod est de forma in formam, sive accidentalem sive
substantialem nunc autem loquimur de rebus secundum emanationem earum ab
universali principio essendi. A qua quidem emanatione nec materia
excluditur, licet a primo modo factionis excludatur.

&

第一異論に対しては、それ故、次のように言われるべきである。哲学者は、『自然学』
第1巻で、付帯的であれ実体的であれ、形相から形相へ、という個別的な生成につ
いて述べている。これに対して、私たちは今、存在の普遍的な根
源からの、諸事物の流出に即して、諸事物について語っている。
前者のかたちでの作成から質料が排除されるとしても、後者の流出からは、質料は排除されない。

\\

{\scshape Ad secundum dicendum} quod passio est effectus actionis. Unde
 et rationabile est quod primum principium passivum sit effectus primi
 principii activi, nam omne imperfectum causatur a perfecto. Oportet
 enim primum principium esse perfectissimum, ut dicit Aristoteles, in
 XII {\itshape Metaphys}.

&

第二異論に対しては、次のように言われるべきである。
受動は能動の結果である。
したがって、第一の受動的根源が、第一の能動的根源の結果であることは、妥当
 である。全て不完全なものは、完全なものが原因となって生じるのだから。
じっさい、アリストテレスが『形而上学』第12巻で述べるように、第一根源はもっ
 とも完全でなければならない。


\\

{\scshape Ad tertium dicendum} quod ratio illa non ostendit quod materia
 non sit creata, sed quod non sit creata sine forma. Licet enim omne
 creatum sit in actu, non tamen est actus purus. Unde oportet quod etiam
 illud quod se habet ex parte potentiae, sit creatum, si totum quod ad
 esse ipsius pertinet, creatum est.

&

第三に対しては、次のように言われるべきである。この異論は、質料が創造され
ないことではなく、形相なしに創造されないことを示している。すなわち、全て
創造されたものは現実態にあるが、純粋現実態ではない。それゆえ、存在それ自
体に属する全体が創造されたのであれば、可能態の側にあるものもまた、創造さ
れたのでなければならない

\end{longtable}
\newpage
\rhead{a.~3}

\begin{center}
 {\Large {\bf ARTICULUS TERTIUS}}\\
 {\large UTRUM CAUSA EXEMPLARIS SIT ALIQUID PRETER DEUM}\\
 {\footnotesize Supra, q.~15; I {\itshape Sent.}, d.~36, q.~2; I
 {\itshape SCG.}, c.~54; {\itshape de Verit.}, q.~3, a.~1, 2; {\itshape
 De Div.~Nom.}, c.~5, lect.~3; I {\itshape Metaphysic.}, lect.~15.}\\
 {\Large 第三項\\範型因は神以外の何かか}
\end{center}

\begin{longtable}{p{21em}p{21em}}
{\huge A}{\scshape d tertium sic proceditur}. Videtur quod
causa exemplaris sit aliquid praeter Deum. Exemplatum enim habet
similitudinem exemplaris. Sed creaturae longe sunt a divina
similitudine. Non ergo Deus est causa exemplaris earum.

&

第三項の問題へ、議論は以下のように進められる。
範型因は、神以外の何かだと思われる。
理由は以下の通り。範型によって生じたものは、範型にたいして類似性をもつ。
ところが、被造物は、神の類似から遠く隔たっている。
ゆえに、神が被造物の範型因なのではない。


\\


2.~{\scshape Praeterea}, omne quod est per participationem, reducitur ad
aliquid per se existens, ut ignitum ad ignem, sicut iam dictum est. Sed
quaecumque sunt in sensibilibus rebus, sunt solum per participationem
alicuius speciei, quod ex hoc patet, quod in nullo sensibilium invenitur
solum id quod ad rationem speciei pertinet, sed adiunguntur principiis
speciei principia individuantia. Oportet ergo ponere ipsas species per
se existentes, ut {\itshape per se hominem}, et {\itshape per se equum},
et huiusmodi. Et haec dicuntur exemplaria. Sunt igitur exemplaria res
quaedam extra Deum.

&

さらに、すでに述べられたとおり、全て分有によって在るものは、それ自体によっ
 て在る何かへ還元される。たとえば、燃えるものが火に還元されるように。とこ
 ろで、可感的諸事物においてあるものはすべて、何らかの種を分有することによっ
 てのみある。このことは、可感的なものどものどれにおいても、種の性格に属
 するものだけが見出されることはなく、個体化の根源が種の根源に結合されてい
 ることから明らかである。ゆえに、「人間それ自体」や「馬それ自体」のよう
 な、それ自体で存在する種を措定しなければならない。そして、これらが範型
 と言われる。ゆえに、神以外に、何らかの範型的事物が存在する。

\\


3.~{\scshape Praeterea}, scientiae et definitiones sunt
de ipsis speciebus, non secundum quod sunt in particularibus, quia
particularium non est scientia nec definitio. Ergo sunt quaedam entia,
quae sunt entia vel species non in singularibus. Et haec dicuntur
exemplaria. Ergo idem quod prius.


&

さらに、学知と定義は種それ自体についてあるのであって、種が個においてある
 限りであるのではない。なぜなら、個的なものには学知も定義もないのだから。
ゆえに、個においてあるのではない有、ないし種であるような、なんらかの有が
 存在し、これが範型と言われる。ゆえに、前の異論に同じ。
\\

4.~{\scshape Praeterea}, hoc idem videtur per Dionysium, qui dicit,
{\scshape v} cap.~{\itshape de Div.~Nom}., quod {\itshape ipsum secundum
se esse, prius est eo quod est per se vitam esse, et eo quod est per se
sapientiam esse}.

&

さらに、同じことが、ディオニュシウスによってもそうだと思われる。彼は『神
 名論』5章で、「それ自体で存在であるものは、それ自体で生命であるものより
 も、それ自体で知恵であるものよりも、先である」と述べている。
\\

{\scshape Sed contra est} quod exemplar est idem quod idea. Sed ideae,
secundum quod Augustinus libro {\itshape Octoginta trium Quaest}.~dicit,
sunt {\itshape formae principales, quae divina intelligentia
continentur}. Ergo exemplaria rerum non sunt extra Deum.

&
しかし反対に、範型はイデアと同じである。しかしイデアは、『八十三問題集』
 でのアウグスティヌスによれば、「神の知性認識に含まれている根源的形相」
 である。ゆえに、諸事物の範型が神の外にあるわけではない。

\\


{\scshape Respondeo dicendum} quod Deus est prima causa exemplaris
omnium rerum. Ad cuius evidentiam, considerandum est quod ad
productionem alicuius rei ideo necessarium est exemplar, ut effectus
determinatam formam consequatur, artifex enim producit determinatam
formam in materia, propter exemplar ad quod inspicit, sive illud sit
exemplar ad quod extra intuetur, sive sit exemplar interius mente
conceptum. 

&

解答する。以下のように言われるべきである。神は全ての事物の第一範型因である。これを明ら
かにするためには、以下のことが考察されるべきである。ある事物を産出するた
めに範型が必要なのは、結果が限定された形相を獲得するためである。たとえば、
職人は、限定された形相[かたち]を質料[材料]の中に生み出すが、それは、
外を見るのであれ、内側に精神によってとらえられたものであれ、そこへ目を向
ける範型のためである。

\\

Manifestum est autem quod ea quae naturaliter fiunt,
determinatas formas consequuntur. Haec autem formarum determinatio
oportet quod reducatur, sicut in primum principium, in divinam
sapientiam, quae ordinem universi excogitavit, qui in rerum distinctione
consistit. 

&

さて、自然的に生じるものどもにおいて、限定された形
相が獲得されていることは明らかである。さらに、この形相の限定は、第一根源
としての神の知恵に還元されなければならない。神の知恵は、諸事物の区別にお
いて成り立つ世界の秩序を考え出したのだから。

\\

Et ideo oportet dicere quod in divina sapientia sunt rationes
omnium rerum, quas supra diximus {\itshape ideas}, id est formas
exemplares in mente divina existentes. Quae quidem licet multiplicentur
secundum respectum ad res, tamen non sunt realiter aliud a divina
essentia, prout eius similitudo a diversis participari potest
diversimode. 

&

ゆえに、神の知恵の中に、全て
の事物の根拠(ratio)があると言わなければならない。この根拠を、私たちは前に
[第15問第1項]「イデア」と呼んだ。また、これは、諸事物への関係において多
数化されるが、実在的には神の本質と別のものではない。つまり、神の類似がさ
まざまなものによってさまざまに分有されるのである。

\\

Sic igitur ipse Deus est primum exemplar omnium. -- Possunt
etiam in rebus creatis quaedam aliorum exemplaria dici, secundum quod
quaedam sunt ad similitudinem aliorum, vel secundum eandem speciem, vel
secundum analogiam alicuius imitationis.


&

このような意味で、それゆえ、神は万物の第一の範型である。--さらに、創造さ
 れた事物の中で、何かが別の何かの範型であると言われることは可能であ
 る。それは、同一の種においてであれ、ある模倣のアナロギアにおいてであれ、
 何かが別の何かの類似に向けてある限りにおいてである。

\\

{\scshape Ad primum ergo dicendum} quod, licet creaturae
non pertingant ad hoc quod sint similes Deo secundum suam naturam,
similitudine speciei, ut homo genitus homini generanti; attingunt tamen
ad eius similitudinem secundum repraesentationem rationis intellectae a
Deo, ut domus quae est in materia, domui quae est in mente artificis.


&

第一異論に対しては、それゆえ、次のように言われるべきである。
たしかに、被造物は、産み出された人間が産む人間に似るようなかたちで、自ら
 の本性において、種の類似性において神に似るということまでは到達していな
 いが、しかし、質料においてある家[=造られた家]が職人の精神においてあ
 る家[=家の設計図]に似るようなかたちで、神によって知性認識された根拠
 (ratio)の表現であるという点では、神の類似にまで到達している。

\\

{\scshape Ad secundum dicendum} quod de ratione hominis
est quod sit in materia, et sic non potest inveniri homo sine
materia. Licet igitur hic homo sit per participationem speciei, non
tamen potest reduci ad aliquid existens per se in eadem specie; sed ad
speciem superexcedentem, sicut sunt substantiae separatae. Et eadem
ratio est de aliis sensibilibus.


&

第二異論に対しては、次のように言われるべきである。
人間の根拠(ratio)には、質料において在るということが属しているので、質料
 なしに人間が見出されるということはありえない。それゆえ、「この人間」が
 種を分有することによって在るということはその通りだが、同じ種においてそ
 れ自体によって存在する何か[=「人間それ自体」のようなもの]に還元され
 ることはありえなず、むしろ、離在諸実体のような、優越的な種に還元される。
 このことは、他の可感的諸事物についても言える。


\\


{\scshape Ad tertium dicendum} quod, licet quaelibet
scientia et definitio sit solum entium, non tamen oportet quod res
eundem modum habeant in essendo, quem intellectus habet in
intelligendo. Nos enim, per virtutem intellectus agentis, abstrahimus
species universales a particularibus conditionibus, non tamen oportet
quod universalia praeter particularia subsistant, ut particularium
exemplaria.

&
第三異論に対しては、次のように言われるべきである。
どんな学知や定義もただ有についてのみあるけれども、事物は、存在することに
 おいてと、知性がそれを知性認識することにおいてもつことにおいてとで、同
 じあり方をしなければならないということはない。
じっさい、私たちは能動知性の力によって、個的状態から普遍的形象を抽象する
 が、しかし、普遍的なものが個的なものの外に、個的なものの範型として自存
 しなければならないということはない。


\\


{\scshape Ad quartum dicendum} quod, sicut dicit Dionysius, {\scshape
xi} cap.~{\itshape de Div.~Nom}., {\itshape per se vitam} et {\itshape
per se sapientiam} quandoque nominat ipsum Deum, quandoque virtutes
ipsis rebus datas, non autem quasdam subsistentes res, sicut antiqui
posuerunt.


&

第四異論に対しては、次のように言われるべきである。
ディオニュシウスが『神名論』9章で述べているように、「それ自体による生命」や
 「それ自体による知恵」 は、神そのもののこと指している場合もあれば、諸事物に与えら
 れた力を指している場合もある。しかし、古代の人々が考えたような、自存
 する事物のことを言っているわけではない。

\end{longtable}
\newpage
\rhead{a.~4}


\begin{center}
 {\Large {\bf ARTICULUS QUARTUS}}\\
 {\large UTRUM DEUS SIT CAUSA FINALIS OMNIUM}\\
 {\footnotesize Infra, q.~65, a.~2; q.~103, a.~2; II {\itshape Sent.},
 d.~1, q.~2, a.~1, 2; III {\itshape SCG.}, c.~17, 18; {\itshape
 Compend.~Theol.}}, c.~100, 101;\\
 {\Large 第四項\\神は万物の目的因か}
\end{center}


\begin{longtable}{p{21em}p{21em}}
{\huge A}{\scshape d quartum sic proceditur}. Videtur quod
Deus non sit causa finalis omnium. Agere enim propter finem videtur esse
alicuius indigentis fine. Sed Deus nullo est indigens. Ergo non competit
sibi agere propter finem.

&

第四項の問題へ、議論は次のように進められる。
神は万物の目的因でないと思われる。
なぜなら、目的のために働くことは、目的を必要としているようなものに属する
 と思われる。ところが、神は、何も必要としない。ゆえに、目的のために働く
 ことは、神にふさわしくない。

\\


2.~{\scshape Praeterea}, finis generationis et forma generati et agens
non incidunt in idem numero, ut dicitur in II {\itshape Physic}., quia
finis generationis est forma generati. Sed Deus est primum agens
omnium. Non ergo est causa finalis omnium.

&

さらに、『自然学』第2巻で言われているとおり、生成の目的、生み出されたも
 のの形相、作用者は、数的に同じものではない。なぜなら、生成の目的は、生
 み出されたものの形相だから。ところが、神は、万物[を生み出す]第一作用
 者である。ゆえに、万物の目的因ではない。

\\


3.~{\scshape Praeterea}, finem omnia appetunt. Sed Deum
non omnia appetunt, quia neque omnia ipsum cognoscunt. Deus ergo non est
omnium finis.

&

さらに、万物は目的を欲求する。ところで、万物が神を欲求するわけではない。
 なぜなら、万物が神を認識するわけではないからである。ゆえに、神は万物の
 目的でない。

\\

4.~{\scshape Praeterea}, finalis causa est prima
causarum. Si igitur Deus sit causa agens et causa finalis, sequitur quod
in eo sit prius et posterius. Quod est impossibile.

&
さらに、目的因は、原因の中で第一のものである。ゆえに、もし神が作用因であ
 り目的因だとすると、神の中に、先のものと後のものがあることになる。これ
 は不可能である。

\\

{\scshape Sed contra est} quod dicitur {\itshape Prov}.~{\scshape xvi},
{\itshape universa propter semetipsum operatus est dominus}.

&
しかし反対に、『箴言』16節で、「主は、宇宙を自分自身のために作った」
 \footnote{「主は御旨によって全てのことをされる」(16:4)}と言
 われている。
\\


{\scshape Respondeo dicendum} quod omne agens agit
propter finem, alioquin ex actione agentis non magis sequeretur hoc quam
illud, nisi a casu. Est autem idem finis agentis et patientis, inquantum
huiusmodi, sed aliter et aliter, unum enim et idem est quod agens
intendit imprimere, et quod patiens intendit recipere. 

&

解答する。以下のように言われるべきである。
全て作用者は目的のために作用する。そうでなければ、偶然によらない限り、作
 用者の作用から、あれよりこれが帰結する[=ある特定の結果が生じる]とい
 うことがない。
ところで、作用者の目的と、受動者の目的は、目的である限りにおいて同一であ
 るが、そのあり方が異なる。つまり、一つで同一のものが、作用者が刻印しよ
 うとするものであり、また、受動者が受け取ろうとするものである。

\\

Sunt autem
quaedam quae simul agunt et patiuntur, quae sunt agentia imperfecta, et
his convenit quod etiam in agendo intendant aliquid acquirere. Sed primo
agenti, qui est agens tantum, non convenit agere propter acquisitionem
alicuius finis; sed intendit solum communicare suam perfectionem, quae
est eius bonitas. Et unaquaeque creatura intendit consequi suam
perfectionem, quae est similitudo perfectionis et bonitatis divinae. Sic
ergo divina bonitas est finis rerum omnium.


&


また、同時に作用しかつ受動するような作用者が存在するが、それは不完全な作
用者である。そしてこれらには、作用することにおいてもまた、何かを獲得すること
 を意図することが適合する。しかし、第一作用者は、[受動せず]作用するだ
 けなので、何らかの目的を獲得するために作用することは適合せず、自らの完
 全性、つまり彼の善性を伝達することだけを意図する。また、各々の被造物は、
 自らの完全性を獲得することを意図するが、それは神の完全性すなわち善性の
 類似である。それゆえこのような意味で、神の善性は全ての事物の目的である。

\\


{\scshape Ad primum ergo dicendum} quod agere propter
indigentiam non est nisi agentis imperfecti, quod natum est agere et
pati. Sed hoc Deo non competit. Et ideo ipse solus est maxime liberalis,
quia non agit propter suam utilitatem, sed solum propter suam bonitatem.

&


第一異論に対しては、それゆえ、次のように言われるべきである。
必要のために作用することは、不完全な作用者にのみ属するのであり、そのよう
 な作用者は、作用しかつ受動することが本性的である。しかし、このことは、
 神には適合しない。ゆえに、神だけが、最高度に自由である。なぜなら、自ら
 の有用性のためではなく、自らの善性のためだけに作用するからである。

\\


{\scshape Ad secundum dicendum} quod forma generati non
est finis generationis nisi inquantum est similitudo formae generantis,
quod suam similitudinem communicare intendit. Alioquin forma generati
esset nobilior generante, cum finis sit nobilior his quae sunt ad finem.

&

第二異論に対しては、次のように言われるべきである。
生み出されたものの形相が、生成の目的であるのは、それが、自らの類似を伝達
 することを意図する、生み出すものの形相の類似であるというまさにその点に
 おいてである。そうでなければ、目的は、目的のためにあるものどもよりも高
 貴だから、生み出されたものの形相が、生み出すものよりも高貴だということ
 になったであろう。

\\

{\scshape Ad tertium dicendum} quod omnia appetunt Deum
ut finem, appetendo quodcumque bonum, sive appetitu intelligibili, sive
sensibili, sive naturali, qui est sine cognitione, quia nihil habet
rationem boni et appetibilis, nisi secundum quod participat Dei
similitudinem.

&

第三異論に対しては、次のように言われるべきである。
知性的欲求であれ、感覚的欲求であれ、また、認識なしにある自然本性的欲求で
 あれ、どんな善を欲求することによっても、万物は神を目的として欲求する。な
 ぜなら、どんなものも、神の類似を分有するという点においてでなければ、善
 や欲求されうるものという性格をもたないからである。

\\


{\scshape Ad quartum dicendum} quod, cum Deus sit causa
efficiens, exemplaris et finalis omnium rerum, et materia prima sit ab
ipso, sequitur quod primum principium omnium rerum sit unum tantum
secundum rem. Nihil tamen prohibet in eo considerari multa secundum
rationem, quorum quaedam prius cadunt in intellectu nostro quam alia.


&

第四異論に対しては、次のように言われるべきである。
神は、全ての事物の、作出因、範型因、目的因であり、また、第一質料も神によっ
 て存在するので、全ての事物の第一根源は、事物という点で唯一であることが
 帰結する。しかし、その一つにおいて、多くの事柄が概念の点で考察され、そ
 のうちのあるものは他のものよりも私たちの知性の中に先に入ってくるという
 ことがあっても、何ら差し支えはない。

\end{longtable}

\end{document}