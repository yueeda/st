\documentclass[10pt]{jsarticle} % use larger type; default would be 10pt
%\usepackage[utf8]{inputenc} % set input encoding (not needed with XeLaTeX)
%\usepackage[round,comma,authoryear]{natbib}
%\usepackage{nruby}
\usepackage{okumacro}
\usepackage{longtable}
%\usepqckage{tablefootnote}
\usepackage[polutonikogreek,english,japanese]{babel}
%\usepackage{amsmath}
\usepackage{latexsym}
\usepackage{color}

%----- header -------
\usepackage{fancyhdr}
\pagestyle{fancy}
\lhead{{\it Summa Theologiae} I$^a$ II$^{ae}$, q.~57}
%--------------------

\bibliographystyle{jplain}

\title{{\bf PRIMA SECUNDAE}\\{\HUGE Summae Theologiae}\\Sancti Thomae
Aquinatis\\{\sffamily QUEAESTIO QUINQUAGESIMASEPTIMA}\\DE DISTINCTIONE
VIRTUTUM INTELLECTUALIUM}
\author{Japanese translation\\by Yoshinori {\sc Ueeda}}
\date{Last modified \today}


%%%% コピペ用
%\rhead{a.~}
%\begin{center}
% {\Large {\bf }}\\
% {\large }\\
% {\footnotesize }\\
% {\Large \\}
%\end{center}
%
%\begin{longtable}{p{21em}p{21em}}
%
%&
%
%\\
%\end{longtable}
%\newpage


\begin{document}
\maketitle

\begin{center}
{\LARGE 『神学大全』第二部の一}\\
{\Large 第五十七問\\知的徳の区別について}
\end{center}
\newpage

\begin{longtable}{p{21em}p{21em}}

{\sc {\Large D}einde} considerandum est de distinctione virtutum. Et
 primo, quantum ad virtutes intellectuales; secundo, quantum ad morales;
 tertio, quantum ad theologicas. Circa primum quaeruntur sex. 

\begin{enumerate}
 \item utrum habitus intellectuales speculativi sint virtutes.
 \item utrum sint tres, scilicet sapientia, scientia et intellectus.
 \item utrum habitus intellectualis qui est ars, sit virtus.
 \item utrum prudentia sit virtus distincta ab arte.
 \item utrum prudentia sit virtus necessaria homini. 
 \item utrum eubulia, synesis et gnome sint virtutes adiunctae prudentiae.
\end{enumerate}

&
次に、徳の区別について考察されるべきである。
第一に、知的徳について、第二に、道徳的徳について、第三に、神学的徳につい
 て。第一について、六つのことが問われる。
\begin{enumerate}
 \item 観照的な知的習態は徳か
 \item それらは三つ、すなわち、知恵、知、知性か
 \item 技術という知的習態は徳か
 \item 思慮は技術から区別された徳か
 \item 思慮は人間に必要な徳か
 \item eubulia, synesis, gnomeは、思慮に結びつけられた徳か
\end{enumerate}
\end{longtable}
\newpage
\rhead{a.~1}
\begin{center}
{\Large {\bfseries ARTICULUS PRIMUS}}\\
{\large UTRUM HABITUS INTELLECTUALES SPECULATIVI SINT VIRTUTES}\\
{\footnotesize III {\itshape Sent.}, d.~23, q.~1, a.~4, qu$^a$ 1;
 {\itshape De Verit.}, q.~1, a.~7.}\\
{\Large 第一項\\観照的な知性的習態は徳か }
\end{center}

\begin{longtable}{p{21em}p{21em}}

{\huge A}{\sc d primum sic proceditur}. 
Videtur quod habitus intellectuales speculativi non sint
 virtutes. Virtus enim est habitus operativus, ut supra dictum est. Sed
 habitus speculativi non sunt operativi, distinguitur enim speculativum
 a practico, idest operativo. Ergo habitus intellectuales speculativi
 non sunt virtutes.

&

第一に対しては次のように進められる。観照的な知的習態は徳でないと思われる。
 なぜなら、徳とは、上で述べられたとおり、働きうる習態である。ところが、
 観照的な習態は、働きうるものではない。なぜなら、それは実践的なもの、す
 なわち、働きうるものから区別されるからである。ゆえに、観照的な知的習態
 は徳でない。

\\

{\scshape 2 Praeterea}, virtus est eorum per quae fit homo felix sive
 beatus, eo quod {\itshape felicitas est virtutis praemium}, ut dicitur
 in I {\itshape Ethic}. Sed habitus intellectuales non considerant actus
 humanos, aut alia bona humana, per quae homo beatitudinem adipiscitur,
 sed magis res naturales et divinas. Ergo huiusmodi habitus virtutes
 dici non possunt.


&
さらに、徳は、人間を幸福ないし至福にするものに属する。なぜなら、『ニコマ
 コス倫理学』1巻で述べられるように、幸福は、徳の褒美だからである。ところ
 が、知的習態は、人間の働きや他の人間的善を考察せず、むしろ自然的事物や
 神的事柄を考察する。しかし、前者によって、人間は、至福を獲得する。ゆえ
 に、これらの習態は、徳と言われえない。


\\




{\scshape 3 Praeterea}, scientia est habitus speculativus. Sed scientia
 et virtus distinguuntur sicut diversa genera non subalternatim posita;
 ut patet per philosophum, in IV {\itshape Topic}. Ergo habitus
 speculativi non sunt virtutes.

&

さらに、知は観照的な習態である。ところが、『トピカ』4巻の哲学者によって
 明らかなとおり、知と徳は、相互に下位とならない異なる類として区別される。
 ゆえに、観照的な習態は徳でない。

\\




{\scshape Sed contra}, soli habitus speculativi considerant necessaria
 quae impossibile est aliter se habere. Sed philosophus ponit, in VI
 {\itshape Ethic}., quasdam virtutes intellectuales in parte animae quae
 considerat necessaria quae non possunt aliter se habere. Ergo habitus
 intellectuales speculativi sunt virtutes.

&

しかし反対に、観照的な習態だけが、他のかたちでありえない必然的な事柄を考
 察する。ところが、哲学者は『ニコマコス倫理学』6巻で、ある種の知的徳を、
 他のかたちでありえない必然的な事柄を考察する魂の側に置いている。ゆえに、
 知的習態は徳である。


\\




{\scshape Respondeo dicendum} quod, cum omnis virtus
 dicatur in ordine ad bonum, sicut supra dictum est, duplici ratione
 aliquis habitus dicitur virtus, ut supra dictum est, uno modo, quia
 facit facultatem bene operandi; alio modo, quia cum facultate, facit
 etiam usum bonum. Et hoc, sicut supra dictum est, pertinet solum ad
 illos habitus qui respiciunt partem appetitivam, eo quod vis appetitiva
 animae est quae facit uti omnibus potentiis et habitibus. 

&

答えて言わなければならない。上で述べられたとおり、\footnote{STI-II,
q.~55, a.~3, c.}すべての徳は、善への秩序という点で語られるから、これも上
で述べられたとおり、\footnote{STI-II, q.~56, a.~3, c.}二つの観点から、あ
る習態が徳と言われる。一つには、よく働く機能を作るからであり、もう一つに
は、機能と共に、善い使用を作るからである。
そして、上で述べられたとおり、この後者のことは、欲求的部分に関係する習態
 だけに属する。なぜなら、魂の欲求能力は、すべての能力と習態を用いるよう
 にする力だからである。


\\



Cum igitur
 habitus intellectuales speculativi non perficiant partem appetitivam,
 nec aliquo modo ipsam respiciant, sed solam intellectivam; possunt
 quidem dici virtutes inquantum faciunt facultatem bonae operationis,
 quae est consideratio veri (hoc enim est bonum opus intellectus), non
 tamen dicuntur virtutes secundo modo, quasi facientes bene uti potentia
 seu habitu. Ex hoc enim quod aliquis habet habitum scientiae
 speculativae, non inclinatur ad utendum, sed fit potens speculari verum
 in his quorum habet scientiam, sed quod utatur scientia habita, hoc est
 movente voluntate. 

&

ゆえに、観照的な知的習態は、欲求的部分を完成せず、どの点でもそれに関係せ
 ず、むしろ、知的部分にのみ関係するので、たしかに、真(これが知性の善い
 業である)を考察するという善い働きの機能を作る限りで、徳と言われうるが、
 この能力や習態を善く使用するようにするものとして、二番目の意味で徳と言
 われることはない。
じっさい、ある人が観照的な知の習態をもつことから、それを用いることへと傾
 くのではなく、それについて知をもつ事柄において、真を観照することができ
 るようになる。しかし、所有する知を用いることは、動かす意志による。

\\


Et ideo virtus quae perficit voluntatem, ut caritas
 vel iustitia, facit etiam bene uti huiusmodi speculativis habitibus. Et
 secundum hoc etiam, in actibus horum habituum potest esse meritum, si
 ex caritate fiant, sicut Gregorius dicit, in VI {\itshape Moral}., quod
 {\itshape contemplativa est maioris meriti quam activa}.


&

ゆえに、愛や正義のような、意志を完成する徳が、このような観照的な習態をよ
 く使用させるようにもする。
そして、これに従えば、グレゴリウスが『道徳論』6巻で「観想的生活は、活動
 的生活よりも、より報いが大きい」と述べるように、これらの習態の活動にお
 いても、もしそれが愛から生じるのであれば、報いがありうる。

\\


{\scshape Ad primum ergo dicendum} quod duplex est
 opus, scilicet exterius, et interius. Practicum ergo, vel operativum,
 quod dividitur contra speculativum, sumitur ab opere exteriori, ad quod
 non habet ordinem habitus speculativus. Sed tamen habet ordinem ad
 interius opus intellectus, quod est speculari verum. Et secundum hoc
 est habitus operativus.

&
第一に体しては、それゆえ、次のように言われるべきである。業績は二通りある。
 すなわち、外的なものと内的なものとである。ゆえに、観照的なものに対して
 分かたれる実践的なもの、ないし働きうるものは、外的な業績から取られ、観
 照的習態は、これに対する秩序をもたない。しかし、知性の内的な業績、つま
 り、真を観照することへの秩序をもつ。そしてこの点で、それは働きうる習態
 である。

\\


{\scshape Ad secundum dicendum} quod virtus est
 aliquorum dupliciter. Uno modo, sicut obiectorum. Et sic huiusmodi
 virtutes speculativae non sunt eorum per quae homo fit beatus; nisi
 forte secundum quod ly per dicit causam efficientem vel obiectum
 completae beatitudinis, quod est Deus, quod est summum
 speculabile. Alio modo dicitur virtus esse aliquorum sicut actuum. Et
 hoc modo virtutes intellectuales sunt eorum per quae homo fit
 beatus. Tum quia actus harum virtutum possunt esse meritorii, sicut
 dictum est. Tum etiam quia sunt quaedam inchoatio perfectae
 beatitudinis, quae in contemplatione veri consistit, sicut supra dictum
 est.

&

第二に対しては、次のように言われるべきである。
徳があるものに属するのに二通りある。
一つには、対象の徳であり、この意味で、このような観照的徳は、それによって
 人間が幸福になるようなものに属さない。ただし、この「によって」が、作出
 因、ないし、完全な至福の対象である神を言うのでないかぎりは。神は最高度
 に観照されうるものだからである。
徳があるものに属すると言われるもう一つの意味は、それが作用の徳という意味
 であり、この場合、知的徳は、人間がそれによって幸福になるようなものに属
 する。それは、このような徳の作用が、すでに述べられたように、報いをもた
 らしうるからであり、また、上で述べられたとおり、真の観想において成立す
 る完全な至福の一種の始まりだからである。

\\




Ad tertium dicendum{\scshape } quod scientia
 dividitur contra virtutem secundo modo dictam, quae pertinet ad vim
 appetitivam.

&

第三に体しては、次のように言われるべきである。
知は、第二の意味で言われた徳に対して分かたれる。これは、欲求力に属するか
 らである。



\end{longtable}

\newpage


\rhead{a.~2}
\begin{center}
 {\Large {\bf ARTICULUS SECUNDUS}}\\
 {\large UTRUM SINT TANTUM TRES HABITUS INTELLECTUALES
 SPECULATIVI,\\SCILICET SAPIENTIA, SCIENTIA ET INTELLECTUS}\\
 {\footnotesize {\itshape De Virtut.}, q.~1, a.~12.}\\
 {\Large 第二項\\知恵、知識、知的直観という、ただ三つの知的習態があるか}
\end{center}

\begin{longtable}{p{21em}p{21em}}


{\huge A}{\scshape d secundum sic proceditur}. Videtur quod
 inconvenienter distinguantur tres virtutes intellectuales speculativae,
 scilicet sapientia, scientia et intellectus. Species enim non debet
 condividi generi. Sed sapientia est quaedam scientia, ut dicitur in VI
 {\itshape Ethic}. Ergo sapientia non debet condividi scientiae, in
 numero virtutum intellectualium.

&
第二にかんしては、次のように進められる。
知恵、知識、知的直観、という三つの観照的な知的徳が区別されるのは不適切だと思
 われる。なぜなら、種が類と同じレベルで区別されるべきではない。ところが、
 『ニコマコス倫理学』6巻で述べられるように、知恵は一種の知識である。ゆえ
 に、知的徳の数において、知恵は、知識と同じレベルで区別されるべきでない。

\\



{\scshape 2 Praeterea}, in distinctione
 potentiarum, habituum et actuum, quae attenditur secundum obiecta,
 attenditur principaliter distinctio quae est secundum rationem formalem
 obiectorum, ut ex supradictis patet. Non ergo diversi habitus debent
 distingui secundum materiale obiectum; sed secundum rationem formalem
 illius obiecti. Sed principium demonstrationis est ratio sciendi
 conclusiones. Non ergo intellectus principiorum debet poni habitus
 alius, aut alia virtus, a scientia conclusionum.

&
さらに、上で述べられたことから明らかなとおり、対象に即して見出される、能
 力、習態、作用の区別においては、対象の形相的根拠に基づいて、主要な区別が見
 出される。ゆえに、質料的対象に即してではなく、その対象の形相的根拠に即
 して、異なる習態が区別されるべきである。ところが、論証の原理は、結論を
 知る根拠である。ゆえに、原理についての知的直観が、結論についての知識と
 異なる習態ないし徳として置かれるべきではない。


\\



{\scshape 3 Praeterea}, virtus intellectualis
 dicitur quae est in ipso rationali per essentiam. Sed ratio, etiam
 speculativa, sicut ratiocinatur syllogizando demonstrative; ita etiam
 ratiocinatur syllogizando dialectice. Ergo sicut scientia, quae
 causatur ex syllogismo demonstrativo, ponitur virtus intellectualis
 speculativa; ita etiam et opinio.

&

さらに、知的徳とは、その本質によって、理性的なものの中にある徳である。
ところで、理性は、観照的理性ですら、論証的三段論法によって推論するように、
 対話的三段論法によっても推論する。ゆえに、論証的三段論法によって原因さ
 れる学知が観照的な知的徳とされるのと同様、意見もまた、そのような徳とさ
 れる。

\\



{\scshape Sed contra est} quod philosophus, VI
 {\itshape Ethic}., ponit has solum tres virtutes intellectuales speculativas,
 scilicet sapientiam, scientiam et intellectum.

&

哲学者が『ニコマコス倫理学』6巻で、知恵、知識、知的直観の三つだけを、観
 照的な知的徳としている。


\\



{\scshape Respondeo dicendum} quod, sicut iam dictum
 est, virtus intellectualis speculativa est per quam intellectus
 speculativus perficitur ad considerandum verum, hoc enim est bonum opus
 eius. Verum autem est dupliciter considerabile, uno modo, sicut per se
 notum; alio modo, sicut per aliud notum. Quod autem est per se notum,
 se habet ut principium; et percipitur statim ab intellectu. Et ideo
 habitus perficiens intellectum ad huiusmodi veri considerationem,
 vocatur intellectus, qui est habitus principiorum. 

&
答えて言わなければならない。
すでに述べられたように、観照的な知的徳は、それによって、観照的知性が、知
 性の善い業である真を考察することへ完成されるところのものである。
ところで、真は、二通りに考察されうる。一つには、それ自体によって知られた
 ものとして、もう一つは、他のものを通して知られたものとして。
さて、それ自体によって知られたものは、原理の位置にあり、知性によって直ち
 に捉えられる。ゆえに、このような真の考察に向けて知性を完成する習態は、
 知的直観と言われ、これは諸原理についての習態である。


\\



Verum autem quod est
 per aliud notum, non statim percipitur ab intellectu, sed per
 inquisitionem rationis, et se habet in ratione termini. Quod quidem
 potest esse dupliciter, uno modo, ut sit ultimum in aliquo genere; alio
 modo, ut sit ultimum respectu totius cognitionis humanae. 


&

他方、他のものを通して知られる真は、知性によって直ちに捉えられるのではな
 く、理性の探究によって捉えられ、終端という性格を持つ。これは二通りのか
 たちでありうる。一つには、何らかの類において最終のものとしてであり、も
 う一つには、人間の全認識にかんして最終のものとしてである。


\\



Et quia {\itshape ea
 quae sunt posterius nota quoad nos, sunt priora et magis nota secundum
 naturam}, ut dicitur in I {\itshape Physic}.; ideo id quod est ultimum respectu
 totius cognitionis humanae, est id quod est primum et maxime
 cognoscibile secundum naturam. Et circa huiusmodi est {\itshape sapientia}, quae
 considerat altissimas causas, ut dicitur in I {\itshape Metaphys}. Unde
 convenienter iudicat et ordinat de omnibus, quia iudicium perfectum et
 universale haberi non potest nisi per resolutionem ad primas causas. 


&

そして、『自然学』1巻で言われるように、「私たちにとってより後に知られるもの
 が、本性においてはより先でありより知られたものである」から、人間の全認
 識にかんして最終であるものは、本性において、第一で、最大限に知られうる
 ものである。そして、このようなものについて、知恵がある。『形而上学』1巻
 で言われるように、知恵は、最高の原因をこうさつするからである。
したがって、知恵は、すべての事柄について適切に判断し秩序づける。なぜなら、
 第一の諸原因へとさかのぼることによらなければ、完全で普遍的な判断は所有
 されえないからである。


\\



---Ad
 id vero quod est ultimum in hoc vel in illo genere cognoscibilium,
 perficit intellectum scientia. Et ideo secundum diversa genera
 scibilium, sunt diversi habitus scientiarum, cum tamen sapientia non
 sit nisi una.

&

他方、知られうるものどものうち、これやあれの類のなかで最終の事柄に向けて
 知性を完成するのは知識である。ゆえに、知られうるさまざまな類に即して、
 さまざまな知識の習態がある。しかし、知恵は、一つしかない。


\\



{\scshape Ad primum ergo dicendum} quod sapientia
 est quaedam scientia, inquantum habet id quod est commune omnibus
 scientiis, ut scilicet ex principiis conclusiones demonstret. Sed quia
 habet aliquid proprium supra alias scientias, inquantum scilicet de
 omnibus iudicat; et non solum quantum ad conclusiones, sed etiam
 quantum ad prima principia, ideo habet rationem perfectioris virtutis
 quam scientia.

&

第一に対しては、それゆえ、次のように言われるべきである。
知恵が一種の知識であるのは、すべての知識に共通であること、つまり、原理か
 ら結論を論証するということをもつ点でである。しかし、知恵は、他の知識に
 まさる固有なことがら、すなわち、すべてについて判断するということをもつ。
 しかもしれは、たんに結論にかんしてだけでなく、第一原理にかんしても判断
 する。ゆえに、知識よりも完全な徳という性格をもつ。


\\



{\scshape Ad secundum dicendum} quod quando ratio
 obiecti sub uno actu refertur ad potentiam vel habitum, tunc non
 distinguuntur habitus vel potentiae penes rationem obiecti et obiectum
 materiale, sicut ad eandem potentiam visivam pertinet videre colorem,
 et lumen, quod est ratio videndi colorem et simul cum ipso
 videtur. 


&
第二に対しては、次のように言われるべきである。
一つの作用のもとにある対象の根拠が、能力や習態へと関係するとき、その習態
 や能力が、対象の根拠や質料的対象に応じて区別されることはない。たとえば、
光は、色を見ることの根拠であり、かつ、色と共に見られるものだが、色を見る
 ことと光を見ることは、同じ視覚という能力に属する。



\\


Principia vero demonstrationis possunt seorsum considerari,
 absque hoc quod considerentur conclusiones. Possunt etiam considerari
 simul cum conclusionibus, prout principia in conclusiones
 deducuntur. Considerare ergo hoc secundo modo principia, pertinet ad
 scientiam, quae considerat etiam conclusiones, sed considerare
 principia secundum seipsa, pertinet ad intellectum. 


&

さて、論証の諸原理は、結論が考察されることなしに、切り離して考察されうる
 し、原理が結論へ導かれるものとして、結論と共に同時にも考察されうる。
ゆえに、この後者のかたちで原理を考察することが知識に属し、というのも、知
 識は結論もまた考察するからだが、他方、それ自体において原理を考察するこ
 とが、知的直観に属する。


\\

Unde, si quis recte
 consideret, istae tres virtutes non ex aequo distinguuntur ab invicem,
 sed ordine quodam; sicut accidit in totis potentialibus, quorum una
 pars est perfectior altera, sicut anima rationalis est perfectior quam
 sensibilis, et sensibilis quam vegetabilis. Hoc enim modo, scientia
 dependet ab intellectu sicut a principaliori. Et utrumque dependet a
 sapientia sicut a principalissimo, quae sub se continet et intellectum
 et scientiam, ut de conclusionibus scientiarum diiudicans, et de
 principiis earundem.

&

したがって、正しく考察するならば、これら三つの徳は、相互に同じレベルで区
 別されているのではなく、一種の秩序によって区別されている。
これはちょうど、理性的魂が感覚的魂よりも完全で、感覚的魂は植物的魂より完
 全であるように、その一部が他の部分よりも完全であるような、全能力におい
 ても生じているようにである。
この意味で、学知は、知的直観に、より根源的なものに依存するようにして依存
 する。そしてこの両者が、もっとも根源的なものとしての知恵に依存する。知
 恵は、そのしたに、知的直観と知識を含む。知識の結論について判断し、それ
 らの結論の根源について判断するものとして。



\\



{\scshape Ad tertium dicendum} quod, sicut supra
 dictum est, habitus virtutis determinate se habet ad bonum, nullo autem
 modo ad malum. Bonum autem intellectus est verum, malum autem eius est
 falsum. Unde soli illi habitus virtutes intellectuales dicuntur, quibus
 semper dicitur verum, et nunquam falsum. Opinio vero et suspicio
 possunt esse veri et falsi. Et ideo non sunt intellectuales virtutes,
 ut dicitur in VI {\itshape Ethic}.

&

第三に対しては、次のように言われるべきである。
上で述べられたとおり、徳の習態は、善に対して限定的に関係し、けっして悪へ
 は関係しない。ところで、知性の善は真であり、悪は偽である。したがって、
 知的習態は、常にそれらによって真が語られけっして偽が語られないようなも
 のだけが、徳と言われる。しかし、意見や疑念は、真でも偽でもありうる。ゆ
 えに、『ニコマコス倫理学』6巻で言われるように、それらは知的徳ではない。




\end{longtable}


\newpage


\rhead{a.~3}
\begin{center}
 {\Large {\bf ARTICULUS TERTIUS}}\\
 {\large UTRUM HABITUS INTELLECTUALIS QUI EST ARS, SIT VIRTUS}\\
 {\footnotesize {\itshape De Verit.}, q.~1, a.~7; VI {\itshape Ethic.}, lect.~3.}\\
 {\Large 第三項\\技術という知的習態は徳か}
\end{center}

\begin{longtable}{p{21em}p{21em}}

{\huge A}{\scshape d tertium sic proceditur}. Videtur quod ars non sit
 virtus intellectualis. Dicit enim Augustinus, in libro {\itshape de
 libero arbitrio}, quod {\itshape virtute nullus male utitur}. Sed arte
 aliquis male utitur, potest enim aliquis artifex, secundum scientiam
 artis suae, male operari. Ergo ars non est virtus.

&
第三に対しては次のように進められる。
技術は知的徳でないと思われる。なぜなら、アウグスティヌスは『自由意志につ
 いて』という書物の中で、「だれも徳を悪く用いない」と述べている。ところ
 が、ある人は、技術を悪く用いる。たとえば、ある技術者が、自分の技術の知識
 にしたがって、悪く働くことがありうる。ゆえに、技術は徳でない。


\\


{\scshape 2 Praeterea}, virtutis non est virtus. {\itshape Artis} autem
 {\itshape est aliqua virtus}, ut dicitur in VI {\itshape Ethic}. Ergo
 ars non est virtus.

&

さらに、徳に属する徳はない。ところが、『ニコマコス倫理学』6巻で「技術に
 は、何らかの徳が属する」と言われている。ゆえに、技術は徳でない。

\\


{\scshape 3 Praeterea}, artes liberales sunt
 excellentiores quam artes mechanicae. Sed sicut artes mechanicae sunt
 practicae, ita artes liberales sunt speculativae. Ergo si ars esset
 virtus intellectualis, deberet virtutibus speculativis annumerari.

&

さらに、自由な技術は機械的な技術よりも優れている。ところが、機械的な技術
 が実践的であるように、自由な技術は観照的である。ゆえに、もし技術が知的
 徳だったとしたら、観照的徳に加えられたであろう。

\\


{\scshape Sed contra est} quod philosophus, in VI {\itshape Ethic}.,
ponit artem esse virtutem; nec tamen connumerat eam virtutibus
speculativis, quarum subiectum ponit scientificam partem animae.

&

しかし反対に、哲学者は『ニコマコス倫理学』6巻で、技術が徳であるとしている。
 しかし、観照的徳にそれを加えていない。彼は観照的徳の基体が、魂の知識を作
 る部分だとしている。

\\


{\scshape Respondeo dicendum} quod ars nihil aliud
 est quam ratio recta aliquorum operum faciendorum. Quorum tamen bonum
 non consistit in eo quod appetitus humanus aliquo modo se habet, sed in
 eo quod ipsum opus quod fit, in se bonum est. Non enim pertinet ad
 laudem artificis, inquantum artifex est, qua voluntate opus faciat; sed
 quale sit opus quod facit. Sic igitur ars, proprie loquendo, habitus
 operativus est. 


&

答えて言わなければならない。技術は、作られるべき何らかの仕事の正しい理拠
に他ならない。しかし、それらの善は、人間の欲求が何らかのかたちで関係する
ものにおいてではなく、それ自体において善であるような、為された仕事そのも
のにおいて成立する。たとえば、技術者の値打ちは、彼が技術者であり、その意志に
よって仕事をする点にあるのではなく、彼がする仕事がどのようなものかという
点にある。このように、技術は、厳密に言うならば、働きうる習態である。

\\


Et tamen in aliquo convenit cum habitibus speculativis,
 quia etiam ad ipsos habitus speculativos pertinet qualiter se habeat
 res quam considerant, non autem qualiter se habeat appetitus humanus ad
 illas. Dummodo enim verum geometra demonstret, non refert qualiter se
 habeat secundum appetitivam partem, utrum sit laetus vel iratus, sicut
 nec in artifice refert, ut dictum est. 


&

しかし、技術は、ある点で、観照的習態と一致する。なぜなら、観照的習態にも、
 人間の欲求が、考察する事物にどのように関係するかではなく、考察する事物
 が、どのようにあるかということが属するからである。たとえば、幾何学者が
 真を論証するとき、彼が欲求的部分に即してどのようにあるか、喜んでいるか
 怒っているか、というようなことは、すでに述べられたとおり関係しないが、
 これは技術者においても同様である。

\\


Et ideo eo modo ars habet
 rationem virtutis, sicut et habitus speculativi, inquantum scilicet nec
 ars, nec habitus speculativus, faciunt bonum opus quantum ad usum, quod
 est proprium virtutis perficientis appetitum; sed solum quantum ad
 facultatem bene agendi.

& 

ゆえに、観照的な習態と同様、技術が、欲求を完成する徳に固有である、使用に
かんして善い仕事を行うということでなく、ただ、善く作用する機能に関しての
み、善い仕事を行うという点で、技術は、観照的な習態と同様に、徳の性格を持
つ。


\\


{\scshape Ad primum ergo dicendum} quod, cum
 aliquis habens artem operatur malum artificium, hoc non est opus artis,
 immo est contra artem, sicut etiam cum aliquis sciens verum mentitur,
 hoc quod dicit non est secundum scientiam, sed contra scientiam. Unde
 sicut scientia se habet semper ad bonum, ut dictum est, ita et ars, et
 secundum hoc dicitur virtus. In hoc tamen deficit a perfecta ratione
 virtutis, quia non facit ipsum bonum usum, sed ad hoc aliquid aliud
 requiritur, quamvis bonus usus sine arte esse non possit.

&

第一に対しては、それゆえ、次のように言われるべきである。
技術を持っている人が、悪い作品を作るとき、それは技術の仕事ではなく、むし
 ろ、技術に反している。ちょうど、知識がある人が、真を偽るとして、それは、
 知識にしたがってではなく、知識に反して語っているのと同様である。
したがって、すでに述べられたとおり、知識が常に善へ関係するのと同様、技術
 もまたそうであり、この点で、それは徳と言われる。しかし、善い使用自体を
 作らず、善い使用のためには何か他のものを必要とする点で、完全な徳の性格
 から欠けている。ただし、善い使用は、技術なしにありえないのだが。


\\


{\scshape Ad secundum dicendum} quod, quia ad hoc
 ut homo bene utatur arte quam habet, requiritur bona voluntas, quae
 perficitur per virtutem moralem; ideo philosophus dicit quod artis est
 virtus, scilicet moralis, inquantum ad bonum usum eius aliqua virtus
 moralis requiritur. Manifestum est enim quod artifex per iustitiam,
 quae facit voluntatem rectam, inclinatur ut opus fidele faciat.

&

第二に対しては、次のように言われるべきである。
人間が所有する技術を善く用いるためには、道徳的徳によって完成される善い意
 志が必要とされる。ゆえに、哲学者は、技術には徳が、すなわち、道徳的徳が
 属すると言うが、それは、技術の善い使用のためには、何らかの道徳的徳が必
 要とされるという理由である。たとえば、技術者は、正しい意志がそれを作る正
 義によって、信頼できる仕事をすることへと傾くことは明らかである。

\\


{\scshape Ad tertium dicendum} quod etiam in ipsis
 speculabilibus est aliquid per modum cuiusdam operis, puta constructio
 syllogismi aut orationis congruae aut opus numerandi vel mensurandi. Et
 ideo quicumque ad huiusmodi opera rationis habitus speculativi
 ordinantur, dicuntur per quandam similitudinem artes, sed liberales; ad
 differentiam illarum artium quae ordinantur ad opera per corpus
 exercita, quae sunt quodammodo serviles, inquantum corpus serviliter
 subditur animae, et homo secundum animam est liber. Illae vero
 scientiae quae ad nullum huiusmodi opus ordinantur, simpliciter
 scientiae dicuntur, non autem artes. Nec oportet, si liberales artes
 sunt nobiliores, quod magis eis conveniat ratio artis.

&
第三に対しては、次のように言われるべきである。
観照されうる事柄においても、ある種の仕事のかたちによるものがある。たとえ
 ば、三段論法や整合的な議論の構築だとか、記憶する仕事、測る仕事などであ
 る。ゆえに、このような理性の仕事に秩序づけられる観照的な習態はどれも、
 一種の類似によって、技術と言われるが、ただし自由な技術と言われる。
それは、動かされる身体によって仕事へ秩序づけられる技術と区別するためであ
 る。身体は、魂に奴隷的に従属する点で、ある種、奴隷的なものである。また
 人間は、魂において自由である。
他方、このような仕事に秩序づけられていない知識は、端的に知識と言われ、技
 術とは言われない。また、もし自由な技術がより高貴であるとしても、知識に
 技術の性格がより適合しなければならぬ、ということはない。





\end{longtable}
\newpage




\rhead{a.~4}
\begin{center}
 {\Large {\bf ARTICULUS QUARTUS}}\\
 {\large UTRUM PRUDENTIA SIT VIRTUS DISTINCTA AB ARTE}\\
 {\footnotesize II$^a$ II$^{ae}$, q.~47, a.~4, ad 2; ad 5; {\itshape De
 Verit.}, q.~1, a.~12; VI {\itshape Ethic.}, lect.~4.}\\
 {\Large 第四項\\思慮は技術から区別された徳か}
\end{center}

\begin{longtable}{p{21em}p{21em}}

{\huge A}{\scshape d quartum sic proceditur}. Videtur
 quod prudentia non sit alia virtus ab arte. Ars enim est ratio recta
 aliquorum operum. Sed diversa genera operum non faciunt ut aliquid
 amittat rationem artis, sunt enim diversae artes circa opera valde
 diversa. Cum igitur etiam prudentia sit quaedam ratio recta operum,
 videtur quod etiam ipsa debeat dici ars.


&
第四にかんしては次のように進められる。
思慮は、技術と別の徳ではないと思われる。なぜなら、
技術は、何らかの仕事の正しい理拠である。
ところで、仕事の類がさまざまであることが、あるものに技術の性格を捨てさせ
 ることはない。じっさい、非常に異なる仕事についてさまざまな技術が存在す
 る。ゆえに、思慮もまた、仕事の正しい理拠なので、それは技術と言われるべ
 きだと思われる。

\\



{\scshape 2 Praeterea}, prudentia magis convenit cum arte quam habitus
 speculativi, utrumque enim eorum est {\itshape circa contingens aliter
 se habere}, ut dicitur in VI {\itshape Ethic}. Sed quidam habitus
 speculativi dicuntur artes. Ergo multo magis prudentia debet dici ars.


&
さらに、思慮は、観照的な習態よりも技術により一致する。なぜなら、この両者
 とも、『ニコマコス倫理学』6巻で言われるように、「他のようにもありうるも
 のにかかわる」からである。しかし、ある観照的な習態は技術と言われる。ゆ
 えに、思慮は、いっそう、技術と言われるべきである。

\\



{\scshape 3 Praeterea}, {\itshape ad prudentiam pertinet bene
 consiliari}, ut dicitur in VI {\itshape Ethic}. Sed etiam in quibusdam
 artibus consiliari contingit, ut dicitur in III {\itshape Ethic}.,
 sicut in arte militari, et gubernativa, et medicinali. Ergo prudentia
 ab arte non distinguitur.

&
さらに、『ニコマコス倫理学』6巻で言われるように、「よく相談することは、
 思慮に属する」。ところが、『ニコマコス倫理学』3巻で言われるように、ある
 技術においても、相談することが生じている。たとえば、軍事的な技術、統治
 の技術、そして、医学の技術において。ゆえに、思慮は技術から区別されない。


\\



{\scshape Sed contra} est quod philosophus
 distinguit prudentiam ab arte, in VI {\itshape Ethic}.

&
しかし反対に、哲学者は『ニコマコス倫理学』6巻で、思慮を技術から区別して
 いる。

\\



{\scshape Respondeo dicendum} quod ubi invenitur
 diversa ratio virtutis, ibi oportet virtutes distingui. Dictum est
 autem supra quod aliquis habitus habet rationem virtutis ex hoc solum
 quod facit facultatem boni operis, aliquis autem ex hoc quod facit non
 solum facultatem boni operis, sed etiam usum. Ars autem facit solum
 facultatem boni operis, quia non respicit appetitum. Prudentia autem
 non solum facit boni operis facultatem, sed etiam usum, respicit enim
 appetitum, tanquam praesupponens rectitudinem appetitus. 

&
答えて言わなければならない。
徳の異なる性格が見出されるところでは、諸徳が区別されなければならない。
ところで、ある習態は、善い仕事の機能を作ることだけから、徳の性格をもつが、
 別のある習態は、善い仕事の機能を作ることだけではなく、その使用も作るこ
 とから、徳の性格をもつ、ということが上で述べられた。
技術は、欲求と関係しないので、ただ、善い仕事の機能だけを作る。
他方、思慮は、善い仕事の機能を作るだけでなく、使用も作る。なぜなら、思慮
 は、欲求の正しさを前提するというかたちで、欲求にかかわるからである。

\\



Cuius differentiae ratio est, quia ars est {\itshape recta ratio
 factibilium}; prudentia vero est {\itshape recta ratio
 agibilium}. Differt autem facere et agere quia, ut dicitur in IX
 {\itshape Metaphys}., factio est actus transiens in exteriorem
 materiam, sicut aedificare, secare, et huiusmodi; agere autem est actus
 permanens in ipso agente, sicut videre, velle, et huiusmodi. Sic igitur
 hoc modo se habet prudentia ad huiusmodi actus humanos, qui sunt usus
 potentiarum et habituum, sicut se habet ars ad exteriores factiones,
 quia utraque est perfecta ratio respectu illorum ad quae comparatur.



&

この違いの根拠は、技術が「作られうる事柄の正しい理拠」であるのに対し、思
 慮は「為されうる事柄の正しい理拠」であることにある。
『形而上学』9巻で述べられているように、作ることと為すことが異なるのは、
 作ることが、たとえば建築する、切る、など、外の質料へ出て行く作用である
 のに対し、為すことは、たとえば見る、意志するなど、作用するものの中に留
 まる作用だからである。ゆえに、思慮は、能力や習態の使用であるこのような
 人間的行為に対して、技術が外的な制作に関係するように関係する。なぜなら、
 両者とも、それへと関係するものどものにかんする完全な理拠だからで
 ある。

\\




Perfectio autem et rectitudo rationis in speculativis, dependet ex
 principiis, ex quibus ratio syllogizat, sicut dictum est quod scientia
 dependet ab intellectu, qui est habitus principiorum, et praesupponit
 ipsum. In humanis autem actibus se habent fines sicut principia in
 speculativis, ut dicitur in VII {\itshape Ethic}. Et ideo ad
 prudentiam, quae est recta ratio agibilium, requiritur quod homo sit
 bene dispositus circa fines, quod quidem est per appetitum rectum. Et
 ideo ad prudentiam requiritur moralis virtus, per quam fit appetitus
 rectus.

& ところで、観照的なものどもにおける理拠の完成ないし正しさは、理性がそれ
に基づいて三段論法を行う原理に依存する。それは、知識が、諸原理の習態であ
る知性に依存し、知性を前提すると言われたとおりである。
しかるに、『ニコマコス倫理学』7巻で言われるように、観照的なものどもにお
 ける諸原理の位置を、人間的行為において占めるのは、目的である。
ゆえに、為されるべき事柄の正しい理虚である思慮には、人間が、目的に関して
 善い状態付けがされていることが要求されるが、このことは、正しい欲求によ
 る。ゆえに、思慮には、正しい欲求がそれによって生じるところの道徳的徳が
 要求される。

\\



Bonum autem artificialium non est bonum appetitus humani, sed
 bonum ipsorum operum artificialium, et ideo ars non praesupponit
 appetitum rectum. Et inde est quod magis laudatur artifex qui volens
 peccat, quam qui peccat nolens; magis autem contra prudentiam est quod
 aliquis peccet volens, quam nolens, quia rectitudo voluntatis est de
 ratione prudentiae, non autem de ratione artis. Sic igitur patet quod
 prudentia est virtus distincta ab arte.


&

これに対して、技術的なものどもの善は、人間的欲求の善ではなく、技術的な仕
 事自体の善であり、したがって、技術は正しい欲求を前提しない。
ゆえに、しくじることを意志する技術者は、それを意志しない技術者よりも賞賛され
 るが、失敗を意志しない人よりも、意志する人の方が、より思慮に反する。な
 ぜなら、意志の正しさは思慮の性格に属するが、技術の性格には属さないから
 である。ゆえに、このようにして、思慮が技術から区別された徳であることが
 明らかである。


\\



{\scshape Ad primum ergo dicendum} quod diversa
 genera artificialium omnia sunt extra hominem, et ideo non
 diversificatur ratio virtutis. Sed prudentia est recta ratio ipsorum
 actuum humanorum, unde diversificatur ratio virtutis, ut dictum est.


&
第一に対しては、それゆえ、次のように言われるべきである。
技術的なもののさまざまな類はすべて、人間の外にあるので、徳の性格は多様化
 されない。これに対して、思慮は、すでに述べられたとおり、人間的行為自体
 の正しい理拠なので、徳の性格を多様化する。

\\



{\scshape Ad secundum dicendum} quod prudentia
 magis convenit cum arte quam habitus speculativi, quantum ad subiectum
 et materiam, utrumque enim est in opinativa parte animae, et circa
 contingens aliter se habere. Sed ars magis convenit cum habitibus
 speculativis in ratione virtutis, quam cum prudentia, ut ex dictis
 patet.


&

第二に対しては次のように言われるべきである。
思慮が観想的習態よりも技術に一致するのは、基体と質料に関してである。なぜ
 なら両者とも、魂の臆見的部分[を基体としてそこ]にあり、他のかたちでも
 ありうるものにかかわるからである。
しかし、述べられたことから明らかなとおり、技術は、、徳の性格において、思
 慮よりも観想的習態と、より一致する。


\\



{\scshape Ad tertium dicendum} quod prudentia est
 bene consiliativa de his quae pertinent ad totam vitam hominis, et ad
 ultimum finem vitae humanae. Sed in artibus aliquibus est consilium de
 his quae pertinent ad fines proprios illarum artium. Unde aliqui,
 inquantum sunt bene consiliativi in rebus bellicis vel nauticis,
 dicuntur prudentes duces vel gubernatores, non autem prudentes
 simpliciter, sed illi solum qui bene consiliantur de his quae conferunt
 ad totam vitam.

&

第三に対しては、次のように言われるべきである。
思慮は、人間の全人生や、人生の究極目的に属する事柄について、よく相談しう
 る。しかし、ある技術においては、その技術に固有の目的に属する事柄につい
 ての相談がある。したがって、ある人々は、戦争や航海の事柄についてよく相
 談しうるという点で、思慮ある指導者とか思慮ある航海士と言われる。しかし、
 端的に思慮ある人と言われているのではなく、そう言われるのは、全人生にか
 かわる事柄についてよく相談されうる人だけである。



\end{longtable}
\newpage

\rhead{a.~5}
\begin{center}
 {\Large {\bf ARTICULUS QUINTUS}}\\
 {\large UTRUM PRUDENTIA SIT VIRTUS NECESSARIA HOMINI}\\
 {\footnotesize II$^a$ II$^{ae}$, q.~11, a.~3, ad 3; {\itshape De
 Verit.}, q.~1, a.~6.}\\
 {\Large 第五項\\思慮は人間に必要な徳か}
\end{center}

\begin{longtable}{p{21em}p{21em}}


{\huge A}{\scshape d quintum sic proceditur}. Videtur quod prudentia
 non sit virtus necessaria ad bene vivendum. Sicut enim se habet ars ad
 factibilia, quorum est ratio recta; ita se habet prudentia ad agibilia,
 secundum quae vita hominis consideratur, est enim eorum recta ratio
 prudentia, ut dicitur in VI {\itshape Ethic}. Sed ars non est
 necessaria in rebus factibilibus nisi ad hoc quod fiant, non autem
 postquam sunt factae. Ergo nec prudentia est necessaria homini ad bene
 vivendum, postquam est virtuosus, sed forte solum quantum ad hoc quod
 virtuosus fiat.

&

第四にかんしては、次のように進められる。
思慮は、善く生きるために必要な徳ではないと思われる。
なぜなら、技術は作られうるものどもに関係し、作られうるものどもに正しい理
 拠が属する。これと同じように、思慮は為されうる事柄、これに即して人間の
 生は考察されるが、に関係し、『ニコマコス倫理学』6巻で言われるように、思
 慮とは、為されうる事柄に属する正しい理拠である。
ところが、技術は、作られうるものどもにおいて、それが作られることのために
 は必要だが、それが作られた後は必要とされない。ゆえに、思慮も、有徳になっ
 た後は人間が善く生きるために必要でなく、おそらく、ただ有徳になるためだ
 けに必要である。



\\



{\scshape 2 Praeterea}, {\itshape prudentia est per quam recte
 consiliamur}, ut dicitur in VI {\itshape Ethic}. Sed homo potest ex
 bono consilio agere non solum proprio, sed etiam alieno. Ergo non est
 necessarium ad bene vivendum quod ipse homo habeat prudentiam; sed
 sufficit quod prudentum consilia sequatur.

&

さらに、『ニコマコス倫理学』6巻で言われるように、「思慮とは、それによっ
 て、私たちがよく相談するところのものである」。
ところが、人間は、自分自身の善い相談に基づいてだけでなく、他人の善い相談
 に基づいても行為しうる。ゆえに、善く生きるためには、その人自身が思慮を
 もつ必要はなく、思慮ある人々の相談に従うことで十分である。

\\



{\scshape 3 Praeterea}, virtus intellectualis est secundum quam
 contingit semper dicere verum, et nunquam falsum. Sed hoc non videtur
 contingere secundum prudentiam, non enim est humanum quod in
 consiliando de agendis nunquam erretur; cum humana agibilia sint
 contingentia aliter se habere. Unde dicitur {\itshape Sap}.~{\scshape
 ix}, {\itshape cogitationes mortalium timidae, et incertae providentiae
 nostrae}. Ergo videtur quod prudentia non debeat poni intellectualis
 virtus.

&

さらに、知的徳には、それに従えば常に真を語り、けっして偽を語らないことが
 属する。ところが、このことは、思慮にしたがっては起こらないと思われる。
 なぜなら、為されるべき事柄について相談するとき、けっして誤らないという
 のは、人間的なことではないからである。したがって、『知恵の書』9章で「道
 徳的な事柄にかんする思惟は恐ろしい。私たちの思慮は不確かだ」と言われて
 いる。ゆえに、思慮は、知的徳とされるべきではない。

\\



{\scshape Sed contra est} quod {\itshape Sap}.~{\scshape viii},
 connumeratur aliis virtutibus necessariis ad vitam humanam, cum dicitur
 de divina sapientia, {\itshape sobrietatem et prudentiam docet, iustitiam et
 virtutem, quibus utilius nihil est in vita hominibus}.


&

しかし反対に、『知恵の書』8章で、神の知恵について、「節度と思慮、正義と徳
 を、人生の中でこれ以上に有用なものはないものとして教える」と言われるとき、
 それは人生に必要な他の諸徳とともに数えられている。


\\



{\scshape Respondeo dicendum} quod prudentia est
 virtus maxime necessaria ad vitam humanam. Bene enim vivere consistit
 in bene operari. Ad hoc autem quod aliquis bene operetur, non solum
 requiritur quid faciat, sed etiam quomodo faciat; ut scilicet secundum
 electionem rectam operetur, non solum ex impetu aut passione. 

&
答えて言わなければならない。
思慮は、人生にもっとも必要な徳である。
善く生きることは、よく働くことにおいて成立する。
また、ある人がよく働くためには、何を作るかだけでなく、どのように作るかが
 必要とされる。
たとえば、衝動や感情だけからではなく、正しい選択にしたがって働くというよ
 うに。


\\


Cum autem
 electio sit eorum quae sunt ad finem, rectitudo electionis duo
 requirit, scilicet debitum finem; et id quod convenienter ordinatur ad
 debitum finem. Ad debitum autem finem homo convenienter disponitur per
 virtutem quae perficit partem animae appetitivam, cuius obiectum est
 bonum et finis. Ad id autem quod convenienter in finem debitum
 ordinatur, oportet quod homo directe disponatur per habitum rationis,
 quia consiliari et eligere, quae sunt eorum quae sunt ad finem, sunt
 actus rationis. 

&

ところで、選択は、目的に対してあるものどもにかかわるので、選択の正しさに
 は二つのことが必要とされる。一つはしかるべき目的であり、もう一つは、し
 かるべき目的へと適切に秩序づけられるものである。
しかるべき目的に向かって、人間は、その対象が善ないし目的である、魂の欲求
 的部分を完成する徳によって適切に整えられる。
他方、しかるべき目的へ適切に秩序づけられるものに対しては、人間は、直接的
 に、理性の習態によって整えられる。なぜなら、目的に対してあるものどもに
 かかわる相談することや選ぶことは、理性の作用だからである。

\\


Et ideo necesse est in ratione esse aliquam virtutem
 intellectualem, per quam perficiatur ratio ad hoc quod convenienter se
 habeat ad ea quae sunt ad finem. Et haec virtus est prudentia. Unde
 prudentia est virtus necessaria ad bene vivendum.


&

ゆえに、理性の中に、目的に対してあるものどもに適切に関係するように理性が
 完成される何らかの知的徳が必要である。
そして、この徳が思慮である。したがって、思慮は、善く生きるために必要な徳
 である。

\\



{\scshape Ad primum ergo dicendum} quod bonum artis
 consideratur non in ipso artifice, sed magis in ipso artificiato, cum
 ars sit ratio recta factibilium, factio enim, in exteriorem materiam
 transiens, non est perfectio facientis, sed facti, sicut motus est
 actus mobilis; ars autem circa factibilia est. 


&

第一に対しては、それゆえ、次のように言われるべきである。
技術の善は、技術者自身の中でではなく、むしろ、作品において考察されるべきで
 ある。技術は、作られうるものどもについての正しい理拠だが、作ることは、
 外の質料へ出て行くものであり、作る人の完成ではなく、作られたものの完成
 だからである。ちょうど、運動が、動かされうるものの現実態であるように。
そして技術は、作られうるものについてある。

\\



Sed prudentiae bonum
 attenditur in ipso agente, cuius perfectio est ipsum agere, est enim
 prudentia recta ratio agibilium, ut dictum est. Et ideo ad artem non
 requiritur quod artifex bene operetur, sed quod bonum opus
 faciat. Requireretur autem magis quod ipsum artificiatum bene
 operaretur, sicut quod cultellus bene incideret, vel serra bene
 secaret; si proprie horum esset agere, et non magis agi, quia non
 habent dominium sui actus. 


&

これに対して、思慮の善は、行為する者、その完成が行為自体であるが、自身の
 中に見出される。思慮は、すでに述べられたとおり、為されうるものどもの正
 しい理拠だからである。ゆえに、技術には、技術者がよく働くことは要求されず、
 ただ善い仕事を作ることが求められる。
しかし、たとえば、ナイフが善く削れることや、鋸が善く切れることが求められ
 るように、技術作品自体がよく働くことが必要とされただろう。もし、これら
 が「作用される」ではなく、「作用する」のであったとしたならば。それらは、
 自らの作用の主導権をもたないので、実際にはそうではないのだが。


\\



Et ideo ars non est necessaria ad bene
 vivendum ipsi artificis; sed solum ad faciendum artificiatum bonum, et
 ad conservandum ipsum. 
Prudentia autem est necessaria homini ad bene
 vivendum, non solum ad hoc quod fiat bonus.


&

ゆえに、技術は、技術者自身が善く生きるために必要でなく、善い技術作品を作
 るために、そして、それを維持するために必要である。
これに対して、思慮は、たんに善い人になるためだけでなく、善く生きるために、
 人間に必要である。

\\


{\scshape Ad secundum dicendum} quod, cum homo
 bonum operatur non secundum propriam rationem, sed motus ex consilio
 alterius; nondum est omnino perfecta operatio ipsius, quantum ad
 rationem dirigentem, et quantum ad appetitum moventem. Unde si bonum
 operetur, non tamen simpliciter bene; quod est bene vivere.


&

第二に対しては次のように言われるべきである。
人間が、自分の理性にしたがってではなく、他人の相談に基づく運動にしたがっ
 て善を働くとき、導く理性に関しても、動かす欲求に関しても、いまだまった
 く完全なその人の働きではない。したがって、もし善を働くとしても、端的に
 善く働くのではない。善く働くことが、善く生きることである。

\\



{\scshape Ad tertium dicendum} quod verum intellectus practici aliter
 accipitur quam verum intellectus speculativi, ut dicitur in VI
 {\itshape Ethic}. Nam verum intellectus speculativi accipitur per
 conformitatem intellectus ad rem. Et quia intellectus non potest
 infallibiliter conformari rebus in contingentibus, sed solum in
 necessariis; ideo nullus habitus speculativus contingentium est
 intellectualis virtus, sed solum est circa necessaria. 

&

第三に対しては次のように言われるべきである。
『ニコマコス倫理学』6巻で言われるように、実践的知性の真は、観照的知性の
 真とは異なって理解される。
つまり、観照的知性の真は、知性と事物の一致によって理解されるが、非必然的
 な事柄においては、知性は間違いのないしかたで事物に一致することができず、
 それができるのは、必然的な事柄においてだけなので、非必然的な事柄につい
 ての観照的な習態は存在せず、ただたんに、必然的なものにかかわる。

\\

Verum autem
 intellectus practici accipitur per conformitatem ad appetitum
 rectum. Quae quidem conformitas in necessariis locum non habet, quae
 voluntate humana non fiunt, sed solum in contingentibus quae possunt a
 nobis fieri, sive sint agibilia interiora, sive factibilia
 exteriora. Et ideo circa sola contingentia ponitur virtus intellectus
 practici, circa factibilia quidem, ars; circa agibilia vero prudentia.

&

これに対して、実践的知性の真は、正しい欲求への一致によって理解される。
この一致は、人間の意志が作るのではない必然的なものどもにおいては起こらず、
内的に為され
 うる事柄であれ、外的に作られうる事柄であれ、私たちによって作られうる非
 必然的な事柄においてのみ起こる。
ゆえに、非必然的な事柄に関してのみ、実践的知性の徳が考えられる。作られう
 るものについては技術が、為されうるものについては思慮が。


\end{longtable}
\newpage

\rhead{a.~6}
\begin{center}
 {\Large {\bf ARTICULUS SEXTUS}}\\
 {\large UTRUM EUBULIA, SYNESIS ET GNOME SINT VIRTUTES ADIUNCTAE PRUDENTIAE}\\
 {\footnotesize II$^a$ II${^ae}$, q.~48, 51; III {\itshape Sent.},
 d.~33, q.~3, a.~1, q$^a$ 3, 4; {\itshape De Virut.}, q.~1, a.~12, ad
 26; q.~5, a.~1.}\\
 {\Large 第六項\\eubulia, synesis, gnomeは、思慮に結びついた徳か}
\end{center}

\begin{longtable}{p{21em}p{21em}}

{\huge A}{\scshape d sextum sic proceditur}. Videtur quod inconvenienter
adiungantur prudentiae eubulia, synesis et gnome. Eubulia enim est
habitus quo bene consiliamur, ut dicitur in VI {\itshape Ethic}. Sed
bene consiliari pertinet ad prudentiam, ut in eodem libro dicitur. Ergo
eubulia non est virtus adiuncta prudentiae, sed magis est ipsa
prudentia.


&
eubulia, synesis, gnomeが、思慮に結びつけられるのは不適切だと思われる。
 なぜなら、
Eubuliaは、『ニコマコス倫理学』6巻で言われているように、それによって、私
 たちがよく相談する習態である。ところが、よく相談することは、同書で言わ
 れているように、思慮に属する。ゆえに、Eubuliaは、思慮に結びついた徳では
 なく、むしろ、思慮そのものである。

\\

{\scshape 2 Praeterea}, ad superiorem pertinet de
inferioribus iudicare. Illa ergo virtus videtur suprema, cuius est actus
iudicium. Sed synesis est bene iudicativa. Ergo synesis non est virtus
adiuncta prudentiae, sed magis ipsa est principalis.

&
さらに、下位の徳について判断することは、上位の徳に属する。
ゆえに、判断する作用が属する徳は、最高の徳だと思われる。
ところが、Synesisは、善く判断しうる徳である。ゆえに、Synesisは、思慮に結
 びついた徳ではなく、むしろSynesisが、根源的な徳である。

\\


{\scshape 3 Praeterea}, sicut diversa sunt ea de
quibus est iudicandum, ita etiam diversa sunt ea de quibus est
consiliandum. Sed circa omnia consiliabilia ponitur una virtus, scilicet
eubulia. Ergo ad bene iudicandum de agendis, non oportet ponere, praeter
synesim, aliam virtutem, scilicet gnomen.


&
さらに、

\\


[35925] Iª-IIae q. 57 a. 6 arg. 4 Praeterea, Tullius ponit, in sua
rhetorica, tres alias partes prudentiae, scilicet memoriam
praeteritorum, intelligentiam praesentium, et providentiam
futurorum. Macrobius etiam ponit, super somnium Scipionis, quasdam alias
partes prudentiae, scilicet cautionem, docilitatem, et alia
huiusmodi. Non videntur igitur solae huiusmodi virtutes prudentiae
adiungi.


&


\\


[35926] Iª-IIae q. 57 a. 6 s. c. Sed contra est auctoritas philosophi,
in VI Ethic., qui has tres virtutes ponit prudentiae adiunctas.

&


\\


[35927] Iª-IIae q. 57 a. 6 co. Respondeo dicendum quod in omnibus
potentiis ordinatis illa est principalior, quae ad principaliorem actum
ordinatur. Circa agibilia autem humana tres actus rationis inveniuntur,
quorum primus est consiliari, secundus iudicare, tertius est
praecipere. Primi autem duo respondent actibus intellectus speculativi
qui sunt inquirere et iudicare, nam consilium inquisitio quaedam
est. Sed tertius actus proprius est practici intellectus, inquantum est
operativus, non enim ratio habet praecipere ea quae per hominem fieri
non possunt. Manifestum est autem quod in his quae per hominem fiunt,
principalis actus est praecipere, ad quem alii ordinantur. Et ideo
virtuti quae est bene praeceptiva, scilicet prudentiae, tanquam
principaliori, adiunguntur tanquam secundariae, eubulia, quae est bene
consiliativa, et synesis et gnome, quae sunt partes iudicativae; de
quarum distinctione dicetur.


&


\\


[35928] Iª-IIae q. 57 a. 6 ad 1 Ad primum ergo dicendum quod prudentia
est bene consiliativa, non quasi bene consiliari sit immediate actus
eius, sed quia hunc actum perficit mediante virtute sibi subiecta, quae
est eubulia.


&


\\


[35929] Iª-IIae q. 57 a. 6 ad 2 Ad secundum dicendum quod iudicium in
agendis ad aliquid ulterius ordinatur, contingit enim aliquem bene
iudicare de aliquo agendo, et tamen non recte exequi. Sed ultimum
complementum est, quando ratio iam bene praecipit de agendis.


&


\\


[35930] Iª-IIae q. 57 a. 6 ad 3 Ad tertium dicendum quod iudicium de
unaquaque re fit per propria principia eius. Inquisitio autem nondum est
per propria principia, quia his habitis, non esset opus inquisitione,
sed iam res esset inventa. Et ideo una sola virtus ordinatur ad bene
consiliandum, duae autem virtutes ad bene iudicandum, quia distinctio
non est in communibus principiis, sed in propriis. Unde et in
speculativis una est dialectica inquisitiva de omnibus, scientiae autem
demonstrativae, quae sunt iudicativae, sunt diversae de
diversis. Distinguuntur autem synesis et gnome secundum diversas regulas
quibus iudicatur, nam synesis est iudicativa de agendis secundum
communem legem; gnome autem secundum ipsam rationem naturalem, in his in
quibus deficit lex communis; sicut plenius infra patebit.


&


\\


[35931] Iª-IIae q. 57 a. 6 ad 4 Ad quartum dicendum quod memoria,
intelligentia et providentia, similiter etiam cautio et docilitas, et
alia huiusmodi, non sunt virtutes diversae a prudentia, sed quodammodo
comparantur ad ipsam sicut partes integrales, inquantum omnia ista
requiruntur ad perfectionem prudentiae. Sunt etiam et quaedam partes
subiectivae, seu species prudentiae, sicut oeconomica, regnativa, et
huiusmodi. Sed praedicta tria sunt quasi partes potentiales prudentiae,
quia ordinantur sicut secundarium ad principale. Et de his infra
dicetur.



&


\end{longtable}
\newpage

\end{document}