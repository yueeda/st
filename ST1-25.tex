\documentclass[10pt]{jsarticle} % use larger type; default would be 10pt
%\usepackage[utf8]{inputenc} % set input encoding (not needed with XeLaTeX)
%\usepackage[round,comma,authoryear]{natbib}
%\usepackage{nruby}
\usepackage{okumacro}
\usepackage{longtable}
%\usepqckage{tablefootnote}
\usepackage[polutonikogreek,english,japanese]{babel}
%\usepackage{amsmath}
\usepackage{latexsym}
\usepackage{color}

%----- header -------
\usepackage{fancyhdr}
\pagestyle{fancy}
\lhead{{\it Summa Theologiae} I, q.~25}
%--------------------


\title{{\bf PRIMA PARS}\\{\HUGE Summae Theologiae}\\Sancti Thomae
Aquinatis\\{\sffamily QUEAESTIO VIGESIMAQUINTA}\\DE DIVINA
POTENTIA}
\author{Japanese translation\\by Yoshinori {\sc Ueeda}}
\date{Last modified \today}

%%%% コピペ用
%\rhead{a.~}
%\begin{center}
% {\Large {\bf }}\\
% {\large }\\
% {\footnotesize }\\
% {\Large \\}
%\end{center}
%
%\begin{longtable}{p{21em}p{21em}}
%
%&
%
%\\
%\end{longtable}
%\newpage



\begin{document}

\maketitle

\begin{center}
{\Large 第二十五問\\神の能力について}
\end{center}


\begin{longtable}{p{21em}p{21em}}

{\huge P}{\scshape ost} considerationem divinae scientiae et voluntatis,
et eorum quae ad hoc pertinent, restat considerandum de divina
potentia. Et circa hoc quaeruntur sex. 

\begin{enumerate}
 \item utrum in Deo sit potentia.
 \item utrum eius potentia sit infinita. 
 \item utrum sit omnipotens. 
 \item utrum possit facere quod ea quae sunt praeterita, non fuerint. 
 \item utrum Deus possit facere quae non facit, vel praetermittere quae
       facit. 
 \item utrum quae facit, possit facere meliora.
\end{enumerate}

&
神の知と意志、そしてそれらに属する事柄の考察のあとに、神の能力
 \footnote{本問題の日本語訳では、とくに、同じpotentiaという語が、文脈に
 応じて「能力」と「可能態」の二つの言葉に訳されることに注意。}について考
 察されるべきことが残っている。これにかんして六つのことが問われる。

\begin{enumerate}
 \item 神の中に能力があるか。
 \item 神の能力は無限か。
 \item 神は全能か。
 \item 過去にあったことをなかったことにできるか。
 \item 神は作らないものを作ることができるか、あるいは、作るものを作らな
       いことができるか
 \item 神は作ったものをより善くすることができるか。
\end{enumerate}

\end{longtable}

\newpage

\rhead{a.~1}
\begin{center}
 {\Large {\bf ARTICULUS PRIMUS}}\\
 {\large UTRUM IN DEO SIT POTENTIA}\\
 {\footnotesize I {\itshape Sent.}, d.42, q.1, a.1; I {\itshape
 SCG.}, cap.16; II, cap.7; {\itshape De Pot.}, q.1, a.1; q.7, a.1.}\\
 {\Large 第一項\\神の中に能力があるか}
\end{center}

\begin{longtable}{p{21em}p{21em}}


{\Huge}{\scshape d primum sic proceditur}. Videtur quod in
Deo non sit potentia. Sicut enim prima materia se habet ad potentiam,
ita Deus, qui est agens primum, se habet ad actum. Sed prima materia,
secundum se considerata, est absque omni actu. Ergo agens primum, quod
est Deus, est absque potentia.

&

第一項の問題へ、議論は以下のように進められる。
神の中に能力はないと思われる。理由は以下の通り。
ちょうど、第一質料が可能態に関係するように、第一作用者である神は、現実態
 に関係する。ところが、第一質料は、それ自体において考察されたとき、あら
 ゆる現実態を欠いている。ゆえに、第一作用者、つまり神は、あらゆる可能態
 を欠いている。


\\


{\scshape 2 Praeterea}, secundum philosophum, in IX
Metaphys., qualibet potentia melior est eius actus, nam forma est melior
quam materia, et actio quam potentia activa; est enim finis eius. Sed
nihil est melius eo quod est in Deo, quia quidquid est in Deo, est Deus,
ut supra ostensum est. Ergo nulla potentia est in Deo.

&

さらに、『形而上学』第9巻の哲学者によれば、どんな可能態よりも、それの現実態
の方がよい。なぜなら、形相は質料よりもよく、作用は能動的能力よりもよいか
 らである。なぜなら、作用は能力の目的だから。ところで、すでに述べられた
 とおり、すべて神の中にあるものは神だから、神の中にあるものよりもよいも
 のはない。ゆえに、神の中に可能態はない。

\\


{\scshape 3 Praeterea}, potentia est principium
operationis. Sed operatio divina est eius essentia, cum in Deo nullum
sit accidens. Essentiae autem divinae non est aliquod principium. Ergo
ratio potentiae Deo non convenit.

&

さらに、能力は、働きの根源である。
ところで、神の働きは、その本質である。
なぜなら、神の中には、どんな附帯性もないのだから。
ところで、神の本質には、どんな根源もない。
ゆえに、能力という性格は、神には適合しない。

\\


{\scshape 4 Praeterea}, supra ostensum est quod scientia
Dei et voluntas eius sunt causa rerum. Causa autem et principium idem
sunt. Ergo non oportet in Deo assignare potentiam, sed solum scientiam
et voluntatem.

&

さらに、神の知とその意志は事物の原因であるということが前に示された。
ところで、原因と根源は同じものである。
ゆえに、神の中に能力を指定する必要はなく、知と意志だけでよい。

\\


{\scshape Sed contra} est quod dicitur in Psalmo
{\scshape lxxxviii}, {\itshape potens es, domine, et veritas tua in circuitu tuo}.

&

しかし反対に、『詩編』88「あなたは力強く、主よ、あなたの真理があなたのまわりに」
と言われている。

\\


{\scshape Respondeo dicendum} quod duplex est potentia,
scilicet passiva, quae nullo modo est in Deo; et activa, quam oportet in
Deo summe ponere. Manifestum est enim quod unumquodque, secundum quod
est actu et perfectum, secundum hoc est principium activum alicuius,
patitur autem unumquodque, secundum quod est deficiens et
imperfectum. 


&

解答する。以下のように言われるべきである。
能力にも二通りあり、受動的能力は神の中にどんな意味においてもないが、能動
 的能力は、神の中に最高度に措定すべきである。
理由は以下の通り。
各々のものは、現実態にあり完全
 であるかぎりで、何かに対する作用の根源であり、また、欠点があり不完全であるかぎりで受動することは明らかである。


\\

Ostensum est autem supra quod Deus est purus actus, et
simpliciter et universaliter perfectus; neque in eo aliqua imperfectio
locum habet. Unde sibi maxime competit esse principium activum, et nullo
modo pati. Ratio autem activi principii convenit potentiae activae. Nam
potentia activa est principium agendi in aliud, potentia vero passiva
est principium patiendi ab alio, ut philosophus dicit, V
Metaphys. Relinquitur ergo quod in Deo maxime sit potentia activa.

&

ところで、前に、神は純粋現実態であり、端的に、普遍的に完全であり、神の中
 のどこにも不完全性がない、ということが示された。
したがって、神には、能動的根源であるということと、決して受
 動しないこととが、最大限に適合する。
ところで、能動的根源という性格は、能動的能力に適合する。
なぜなら、哲学者が『形而上学』第5巻で言うように、能動的能力は、何かへと
 作用することの根源であり、受動的能力は、他から受けることの根源だから。
ゆえに、神の中に、最大限に能動的能力があることが帰結する。


\\


{\scshape Ad primum ergo dicendum} quod potentia activa
non dividitur contra actum, sed fundatur in eo, nam unumquodque agit
secundum quod est actu. Potentia vero passiva dividitur contra actum,
nam unumquodque patitur secundum quod est in potentia. Unde haec
potentia excluditur a Deo, non autem activa.

&

第一異論に対しては、それゆえ、以下のことが言われるべきである。
能動的能力は、現実態に対立して区別されるのではなく、現実態に基礎をもつ。
 なぜなら、各々のものは、現実態にあるかぎりで働くからである。
他方、受動的能力は、現実態に対立して区別される。なぜなら、各々のものは、
 可能態にあるかぎりで受動するからである。
したがって、後者の能力は神から排除されるが、前者は排除されない。

\\


Ad secundum dicendum quod, quandocumque actus
est aliud a potentia, oportet quod actus sit nobilior potentia. Sed
actio Dei non est aliud ab eius potentia, sed utrumque est essentia
divina, quia nec esse eius est aliud ab eius essentia. Unde non oportet
quod aliquid sit nobilius quam potentia Dei.

&

第二異論に対しては、以下のように言われるべきである。
現実態が可能態と別であるときにはいつも、現実態が可能態よりも高貴でなけれ
 ばならない。
しかし、神の作用は、神の能力と別でなく、両者は神の本質である。
なぜなら、神の存在と神の本質も別でないからである。
したがって、なにかが神の能力よりも高貴である必要はない。


\\


{\scshape Ad tertium dicendum} quod potentia in rebus
creatis non solum est principium actionis, sed etiam effectus. Sic
igitur in Deo salvatur ratio potentiae quantum ad hoc, quod est
principium effectus, non autem quantum ad hoc, quod est principium
actionis, quae est divina essentia. Nisi forte secundum modum
intelligendi, prout divina essentia, quae in se simpliciter praehabet
quidquid perfectionis est in rebus creatis, potest intelligi et sub
ratione actionis, et sub ratione potentiae; sicut etiam intelligitur et
sub ratione suppositi habentis naturam, et sub ratione naturae.


&

第三異論に対しては、以下のように言われるべきである。
被造物の中にある能力は、作用の根源であるだけでなく、結果の根源でもある。
したがって、神の中で能力という性格が拾われるのは、結果の根源であることに
 かんしてであり、作用の根源であるということに関してではない。なぜなら、
 神の作用は神の本質だから。
あるいは、理解のしかたによっては以下のようにも言える。
神の本質は、その中に、被造物の中にあるすべての完全性をなんであれ端的に先
 に持っているので、作用という性格のもとに、そしてまた能力という性格のも
 とに理解されうる。
ちょうど、神が、本性を持つ個体という性格のもとでも、本性という性格のもと
 でも理解されるように。


\\


{\scshape Ad quartum dicendum} quod potentia non ponitur
in Deo ut aliquid differens a scientia et voluntate secundum rem, sed
solum secundum rationem; inquantum scilicet potentia importat rationem
principii exequentis id quod voluntas imperat, et ad quod scientia
dirigit; quae tria Deo secundum idem conveniunt. Vel dicendum quod ipsa
scientia vel voluntas divina, secundum quod est principium effectivum,
habet rationem potentiae. Unde consideratio scientiae et voluntatis
praecedit in Deo considerationem potentiae, sicut causa praecedit
operationem et effectum.

&

第四異論に対しては、以下のように言われるべきである。
神において、能力は、知と意志と、事物においてではなく、
ただ概念において異なるものとして措定される。
すなわち、能力が、意志が命じるものを、知が指し示すものへと遂行する根源と
 いう性格を意味する限りで。
この三つは、同一のものとして神に一致する。
あるいは以下のように言われるべきである。
神の知や神の意志は、作出的根源であるかぎりで、能力の性格を持つ。
したがって、神において、知と意志についての考察が、能力の考察に先行する。
ちょうど、原因が、働きと結果に先行するように。

\\


\end{longtable}
\newpage



\rhead{a.~2}
\begin{center}
 {\Large {\bf ARTICULUS SECUNDUS}}\\
 {\large UTRUM POTENTIA DEI SIT INFINITA}\\
 {\footnotesize I {\itshape Sent.}, d.43, q.1, a.1; I {\itshape SCG.}, cap.43; {\itshape De Pot.}, q.1, a.2; {\itshape Compend.~Theol.}, cap.19; VIII{\itshape Physic.}, lect.23; XII{\itshape Metaphys.}, lect.8.}\\
 {\Large 第二項\\神の能力は無限か}
\end{center}

\begin{longtable}{p{21em}p{21em}}

{\Huge A}{\scshape d secundum sic proceditur}. Videtur quod potentia Dei
 non sit infinita. Omne enim infinitum est imperfectum, secundum
 philosophum, in III {\itshape Physic}. Sed potentia Dei non est imperfecta. Ergo
 non est infinita.

&

第二問の問題へ、議論は以下のように進められる。神の能力は無限ではないと思
 われる。理由は以下の通り。『自然学』第3巻における哲学者によれば、すべて
 無限なものは不完全である。ところが、神の能力は不完全ではない。ゆえに、
 無限ではない。


\\


{\scshape 2 Praeterea}, omnis potentia manifestatur per effectum, alias
 frustra esset. Si igitur potentia Dei esset infinita, posset facere
 effectum infinitum. Quod est impossibile.


&

さらに、すべての能力は、結果を通して明らかにされる。もしそうでなかったら、
 能力は無駄なものだっただろう。ゆえに、もし神の能力が無限だったなら、無
 限の結果を作ることができただろう。これは不可能である。


\\


{\scshape 3 Praeterea}, philosophus probat in VIII {\itshape Physic}.,
 quod si potentia alicuius corporis esset infinita, moveret in
 instanti. Deus autem non {\itshape movet in instanti, sed movet creaturam
 spiritualem per tempus, creaturam vero corporalem per locum et tempus},
 secundum Augustinum, VIII {\itshape super Genesim ad litteram}. Non ergo est eius
 potentia infinita.


&

さらに、哲学者は『自然学』第8巻で、もしある物体の能力が無限だったなら、
 瞬間的に動かすことを証明している。ところが、アウグスティヌス『創世記逐
 語注解』第8巻によれば、神は「瞬間的に動かさず、霊
 的被造物を時間を通して動かし、物体的被造物を場所と時間を通して動かす」。
 ゆえに、神の能力は無限でない。


\\


{\scshape Sed contra est} quod dicit Hilarius, VIII {\itshape de Trin}.,
 quod Deus est {\itshape immensae virtutis, vivens, potens}. Omne autem
 immensum est infinitum. Ergo virtus divina est infinita.


&

しかし反対に、ヒラリウス『三位一体論』第8巻によれば、神は「はかりきれない
 力を持つ、生きて能力のあるもの」である。ところで、すべてはかりきれないも
 のは無限である。ゆえに、神の力は無限である。

\\


{\scshape Respondeo dicendum} quod, sicut iam dictum est, secundum hoc
 potentia activa invenitur in Deo, secundum quod ipse actu est. Esse
 autem eius est infinitum, inquantum non est limitatum per aliquid
 recipiens; ut patet per ea quae supra dicta sunt, cum de infinitate
 divinae essentiae ageretur. Unde necesse est quod activa potentia Dei
 sit infinita. 


&


解答する。以下のように言われるべきである。
すでに述べられたとおり、神の中に能動的能力が見出されるのは、神が現実態に
 あるかぎりにおいてである。
ところで、前に、神の本質の無限性について論じられたときに述べられたことから明らかなとおり、神の存在は、それを受け取る何かによって限られていない点で、無限
 である。
したがって、神の能動的能力は無限であることが必然である。



\\

In omnibus enim agentibus hoc invenitur, quod quanto
 aliquod agens perfectius habet formam qua agit, tanto est maior eius
 potentia in agendo. Sicut quanto est aliquid magis calidum, tanto habet
 maiorem potentiam ad calefaciendum, et haberet utique potentiam
 infinitam ad calefaciendum, si eius calor esset infinitus. Unde, cum
 ipsa essentia divina, per quam Deus agit, sit infinita, sicut supra
 ostensum est, sequitur quod eius potentia sit infinita.


&

じっさい、すべて作用するものにおいては、作用者がそれによって作用する形相
 をより完全に持てばもつほど、それだけ一層、作用することにおける作用者の
 能力が大きい。たとえば、あるものが熱ければ熱いほど、より大きな熱する能
 力を持つように。そして、もしその熱が無限であったならば、無限の熱する能
 力を持ったであろう。したがって、神は神の本質によって作用するが、その本
 質は、前に述べられたとおり無限であるから、その能力が無限であることが帰
 結する。


\\


{\scshape Ad primum ergo dicendum} quod philosophus loquitur de infinito
 quod est ex parte materiae non terminatae per formam; cuiusmodi est
 infinitum quod congruit quantitati. Sic autem non est infinita divina
 essentia, ut supra ostensum est; et per consequens nec eius
 potentia. Unde non sequitur quod sit imperfecta.


&

第一異論に対しては、それゆえ、以下のように言われるべきである。
哲学者は、形相によって限定されていない質料の側からある無限について語って
 いる。量に適合するのはこの無限である。ところで、前に示されたとおり、神
 の本質は、この意味で無限なのではない。したがって、神の能力も〔この意味
 で無限なのではない〕。ゆえに、それが不完全であることは帰結しない。


\\


{\scshape Ad secundum dicendum} quod potentia agentis univoci tota
 manifestatur in suo effectu, potentia enim generativa hominis nihil
 potest plus quam generare hominem. Sed potentia agentis non univoci non
 tota manifestatur in sui effectus productione, sicut potentia solis non
 tota manifestatur in productione alicuius animalis ex putrefactione
 generati. 


&

第二異論に対しては、以下のように言われるべきである。
一義的作用者の能力は、全体が、その結果において明示される。たとえば、人間
 を産む能力は、人間を産むこと以上の能力がない。しかし、一義的でない作用
 者の能力は、自らの結果の産出において全体が明示されない。たとえば、太陽
 の能力は、産み出されたものの腐敗から何かの動物を産み出すことにおいて、
 全体が明示されない。

\\


Manifestum est autem quod Deus non est agens univocum, nihil
 enim aliud potest cum eo convenire neque in specie, neque in genere, ut
 supra ostensum est. Unde relinquitur quod effectus eius semper est
 minor quam potentia eius. Non ergo oportet quod manifestetur infinita
 potentia Dei in hoc, quod producat effectum infinitum. 


&

さて、神が一義的作用者でないことは明らかである。なぜなら、前に示されたと
 おり、何かが、種や類において、神に一致することはありえないからである。
したがって、神の結果は、常に、神の能力よりも小さいことが帰結する。ゆえに、
 神の能力が、無限の結果を生み出すことにおいて示されなくてもよい。

\\


Et tamen, etiam
 si nullum effectum produceret, non esset Dei potentia frustra. Quia
 frustra est quod ordinatur ad finem, quem non attingit, potentia autem
 Dei non ordinatur ad effectum sicut ad finem, sed magis ipsa est finis
 sui effectus.


&

しかし、仮に神がまったく結果を生み出さなかったとしても、神の能力が無駄で
 あることにはならない。なぜなら、無駄なものとは、目的へ秩序付けられなが
 ら、それに達しないものであるが、神の能力は、目的としてその結果へ秩序付
 けられているのではなく、むしろ神の能力が、自らの結果の目的だからである。


\\


{\scshape Ad tertium dicendum} quod philosophus in VIII Physic., probat,
 quod si aliquod corpus haberet potentiam infinitam, quod moveret in non
 tempore. Et tamen ostendit, quod potentia motoris caeli est infinita,
 quia movere potest tempore infinito. Relinquitur ergo secundum eius
 intentionem, quod potentia infinita corporis si esset, moveret in non
 tempore, non autem potentia incorporei motoris. 

&

第三異論に対しては、以下のように言われるべきである。
哲学者は『自然学』第8巻で、もしある物体が無限の能力をもっていたならば、
 時間の中で動かすことはなかったであろう、ということを証明している。
しかし、天の動者の能力は、無限の時間動かすことができるので、無限であるこ
 とも示している。
それゆえ、彼の意図としては、物体の能力が無限であったならば、時間の中で動
 かさなかっただろうが、しかし、非物体的な動者の能力の場合は、そうでない、
 とういことであるしかない。


\\

Cuius ratio est, quia
 corpus movens aliud corpus, est agens univocum. Unde oportet quod tota
 potentia agentis manifestetur in motu. Quia igitur quanto moventis
 corporis potentia est maior, tanto velocius movet, necesse est quod si
 fuerit infinita, moveat improportionabiliter citius, quod est movere in
 non tempore. 

&

その理由はなぜかと言えば、他の物体を動かす物体は、一義的作用者である。ゆ
 えに、作用者の能力全体が運動において示されなければならない。ゆえに、動
 かす物体の能力が大きいほど、それだけ速く動かすので、もし無限だったなら
 ば、比較を絶して速く動かすことになり、これはすなわち、時間の中を動かさ
 ないことである。

\\


Sed movens incorporeum est agens non univocum. Unde non
 oportet, quod tota virtus eius manifestetur in motu ita, quod moveat in
 non tempore. Et praesertim, quia movet secundum dispositionem suae
 voluntatis.


&

しかし、非物体的な動者は一義的作用者でない。したがって、時間の中を動かさ
 ないというかたちで、運動において、その力の全体が示されなくてもよい。
そしてとくに、そういう動者は、自らの意志の態勢によって動かすからである。





\end{longtable}
\newpage



\rhead{a.~3}
\begin{center}
 {\Large {\bf ARTICULUS TERTIUS}}\\
 {\large UTRUM DEUS SIT OMNIPOTENS}\\
 {\footnotesize Parte III, q.13, a.1; I {\itshape Sent.}, d.42, q.2,
 a.2; III, d.1, q.2, a.3; II {\itshape SCG}, cap.22, 25; {\itshape De
 Pot.}, q.1, a.7; q.5, a.3; {\itshape Quodl}.~III, q.1, a.1; V, q.2,
 a.1; XII, q.2, a.1; VI {\itshape Ethic.}, Lect. 2.}\\
 {\Large 第三項\\神は全能か}
\end{center}

\begin{longtable}{p{21em}p{21em}}


{\Huge A}{\scshape d tertium sic proceditur}. Videtur quod
Deus non sit omnipotens. Moveri enim et pati aliquid omnium est. Sed hoc
Deus non potest, est enim immobilis, ut supra dictum est. Non igitur est
omnipotens.


&

第三項の問題へ、議論は以下のように進められる。神は全能でないと思われる。
 理由は以下の通り。
「すべてのもの」の中には、動かされることや受動することがある。しかし、神
 はこれをすることができない。なぜなら、前に述べられたとおり、神は不変だ
 からである。ゆえに、全能ではない。

\\


{\scshape 2 Praeterea}, peccare aliquid agere est. Sed
Deus non potest peccare, neque seipsum negare, ut dicitur II
Tim. II. Ergo Deus non est omnipotens.


&

さらに、罪を犯すことは、何かをなすことである。しかし、『テモテの信徒への
 手紙2』で言われているように、神は罪を犯すことが
 できず、自らを否定することができない。ゆえに、神は全能でない。



\\


{\scshape 3 Praeterea}, de Deo dicitur quod
{\itshape omnipotentiam suam parcendo maxime et miserando manifestat}. Ultimum
igitur quod potest divina potentia, est parcere et misereri. Aliquid
autem est multo maius quam parcere et misereri; sicut creare alium
mundum, vel aliquid huiusmodi. Ergo Deus non est omnipotens.


&

さらに、
神について、「許すことと憐れむことによって、自らの全能を最大限に明らかに示
 す」と言われる。それゆえ、神の全能がなしうる究極のことは、許すことと憐
 れむことである。ところが、許すことと憐れむことより遥かに大きいことがあ
 る。たとえば、別の世界を作るとか、何かそのようなことである。ゆえに神は
 全能でない。

\\


{\scshape 4 Praeterea}, super illud I {\itshape Cor}.~{\scshape i}, {\itshape stultam
fecit Deus sapientiam huius mundi}, dicit Glossa, {\itshape sapientiam huius mundi
fecit Deus stultam, ostendendo possibile, quod illa impossibile
iudicabat}. Unde videtur quod non sit aliquid iudicandum possibile vel
impossibile secundum inferiores causas, prout sapientia huius mundi
iudicat; sed secundum potentiam divinam. Si igitur Deus sit omnipotens,
omnia erunt possibilia. Nihil ergo impossibile. Sublato autem
impossibili, tollitur necessarium, nam quod necesse est esse,
impossibile est non esse. Nihil ergo erit necessarium in rebus, si Deus
est omnipotens. Hoc autem est impossibile. Ergo Deus non est omnipotens.


&

さらに、かの『コリントの信徒への手紙1』第1章「神はこの世界の知恵を愚かなもの
 にした」について、『注解』は「神は、この世の知恵が不可能だと判断したこ
 とを可能だと示すことによって、この世の知恵を愚かなものにした」と述べて
 いる。
したがって、何かが可能か不可能かということは、この世の知恵のような下位の
 諸原因にしたがってではなく、神の能力にしたがって判断されるべきだと思わ
 れる。
ゆえに、もし神が全能であれば、すべてのことが可能なことになるであろう。
ゆえに、不可能なものは何もないことになる。
しかし、不可能なことがなくなると、必然的なものもなくなる。なぜなら、必然
 的なものとは、そうでないことが不可能なものだからである。
ゆえに、もし神が全能であれば、事物の中に必然的なものはないことになる。これ
 は不可能である。ゆえに、神は全能でない。


\\


{\scshape Sed contra est} quod dicitur Luc.~{\scshape i}, {\itshape non erit
impossibile apud Deum omne verbum}.


&

しかし反対に、『ルカによる福音書』第1章で「神のもとで、あらゆる言葉は不
 可能でないだろう」と言われている。

\\


{\scshape Respondeo dicendum} quod communiter confitentur
omnes Deum esse omnipotentem. Sed rationem omnipotentiae assignare
videtur difficile. Dubium enim potest esse quid comprehendatur sub ista
distributione, cum dicitur omnia posse Deum. Sed si quis recte
consideret, cum potentia dicatur ad possibilia, cum Deus omnia posse
dicitur, nihil rectius intelligitur quam quod possit omnia possibilia,
et ob hoc omnipotens dicatur. 


&

解答する。以下のように言われるべきである。
すべての人は共通して、神は全能だと告白する。しかし、「全能」を合理的に説
 明するのは困難だと思われる。というのも、「神はすべてのことをなすことが
 できる」と言われるとき、この周延\footnote{もともとはアリストテレス三段
 論法で、全称命題の主語に含まれる範囲のこと。転じて「すべて」のも
 とに含まれる範囲のこと。}のもとに何が含まれるかについて疑問があ
 るからである。
しかし、もし正しく考察するならば、能力は可能なものに対して語られるのだか
 ら、神がすべてのものをなすことができると言われるとき、それはすべての可
 能なものをなすことができ、それゆえ、全
 能と言われる、とこの上なく正しく理解される。

\\



{\itshape Possibile} autem dicitur dupliciter,
secundum philosophum, in V Metaphys. Uno modo, per respectum ad aliquam
potentiam, sicut quod subditur humanae potentiae, dicitur esse {\itshape possibile
homini}. Non autem potest dici quod Deus dicatur omnipotens, quia potest
omnia quae sunt possibilia naturae creatae, quia divina potentia in
plura extenditur. 


&

さて、哲学者の『形而上学』第5巻によれば、「可能なもの」は二通りの意味で
 言われる。
一つには、ある能力への関係によってであり、たとえば、人間の能力のもとにあ
 るものが、人間にとって「可能なも」と言われる。
しかし、神が全能と言われるのは、被造の本性にとって可能なことをすべてなす
 ことができるからである、と言われることはできない。なぜなら、神の能力は、
 複数のものへと広がっているからである。

\\


Si autem dicatur quod Deus sit omnipotens, quia potest
omnia quae sunt possibilia suae potentiae, erit circulatio in
manifestatione omnipotentiae, hoc enim non erit aliud quam dicere quod
Deus est omnipotens, quia potest omnia quae potest. Relinquitur igitur
quod Deus dicatur omnipotens, quia potest {\itshape omnia possibilia absolute},
quod est alter modus dicendi {\itshape possibile}. 


&

しかし、「神は、自分の能力にとって可能なすべてのことをなすことができるから、
 全能だ」と言われるならば、全能の明示において循環があるだろう。なぜなら、
 これは、「神は、神がなしうることすべてをなしうるので、全能である」と言
 うことに他ならないからである\footnote
{\begin{enumerate}
  \item God can do everything.
  \item What are included in ``everything''?
  \item Whatever God can do.
  \item God can do whatever God can do.
 \end{enumerate}
}。ゆえに、神が全能と言われるのは、「可能な
 こと」を語る別の意味である、「無条件に可能なこと」をすべてなしうるからだ、と
 いうことが残される。


\\


Dicitur autem aliquid possibile
vel impossibile absolute, ex habitudine terminorum, possibile quidem,
quia praedicatum non repugnat subiecto, ut Socratem sedere; impossibile
vero absolute, quia praedicatum repugnat subiecto, ut hominem esse
asinum. 



&

ところで、あるものが無条件に可能とか不可能とか言われるのは、主語述語の関係に
 基づく。じっさい、「可能なもの」は、たとえば「ソクラテスが座っている」のように、
 述語が主語に反しないから可能である。他方、「無条件に不可能なもの」は、「人間は
 ロバである」のように、述語が主語に反するからである。

\\


Est autem considerandum quod, cum unumquodque agens agat sibi
simile, unicuique potentiae activae correspondet possibile ut obiectum
proprium, secundum rationem illius actus in quo fundatur potentia
activa, sicut potentia calefactiva refertur, ut ad proprium obiectum, ad
esse calefactibile. 

&

ところで、以下のことが考察されるべきである。各々の作用者は、自らに似た作
 用を行うので、各々の能動的能力には、その能動的能力が基礎付けられる作用
 の性格にしたがって、「可能なもの」が、固有対象として対応する。たとえば、
 熱する能力は、固有対象として、「熱せられることが可能なもの」に関係する。

\\


Esse autem divinum, super quod ratio divinae
potentiae fundatur, est esse infinitum, non limitatum ad aliquod genus
entis, sed praehabens in se totius esse perfectionem. Unde quidquid
potest habere rationem entis, continetur sub possibilibus absolutis,
respectu quorum Deus dicitur omnipotens. 

&

しかし、神の能力の性格が基づく神の存在は、無限であり、存在の何らかの類に
 制限されてなく、その中に、存在のあらゆる完全性を予め所有している。
したがって、何であれ存在の性格をもちうるものは、それとの関係で神が全能だ
 と言われる「無条件に可能なもの」のもとに含まれる。


\\

Nihil autem opponitur rationi
entis, nisi non ens. 
Hoc igitur repugnat rationi possibilis absoluti,
quod subditur divinae omnipotentiae, quod implicat in se esse et non
esse simul. Hoc enim omnipotentiae non subditur, non propter defectum
divinae potentiae; sed quia non potest habere rationem factibilis neque
possibilis. 

&

ところで、存在の性格に反するのは、存在しないもの以外にはない。ゆえに、
神の全能のもとにある「無条件に可能なもの」の性格に反するのは、自らのうち
 に存在と非存在を同時に含むものである。というのも、これが全能のもとにな
 いのは、神の能力の欠陥のためではなく、それが、「なされうるもの」「可能なもの」とい
 う性格をもたないからである。

\\


Quaecumque igitur contradictionem non implicant, sub illis
possibilibus continentur, respectu quorum dicitur Deus omnipotens. Ea
vero quae contradictionem implicant, sub divina omnipotentia non
continentur, quia non possunt habere possibilium rationem. 


&

ゆえに、何であれ矛盾を含意しないものは、かの可能的なもののもとに含まれ、
 それに対して神の全能がある。他方、矛盾を含むものは、「可能なもの」とい
 う性格を持ちえないので、神の全能のもとに含まれない。

\\


Unde
convenientius dicitur quod {\itshape non possunt fieri}, quam quod {\itshape Deus non potest
ea facere}. Neque hoc est contra verbum Angeli dicentis, {\itshape non erit
impossibile apud Deum omne verbum}. Id enim quod contradictionem
implicat, verbum esse non potest, quia nullus intellectus potest illud
concipere.


&

したがって、それらは、「神がそれをなしえない」ではなく「なされえない」と
 言われるのがより適切である。またこれは、「神のもとですべての言葉は不可
 能でないであろう」と語る天使の言葉にも反しない。なぜなら、矛盾を含むも
 のは、言葉でありえないからである。なぜなら、どんな知性も、それを捉えら
 れないからである。

\\


{\scshape Ad primum ergo dicendum} quod Deus dicitur
omnipotens secundum potentiam activam, non secundum potentiam passivam,
ut dictum est. Unde, quod non potest moveri et pati, non repugnat
omnipotentiae.


&

第一異論に対しては、それゆえ、以下のように言われるべきである。
すでに述べられたとおり、神は、受動的能力ではなく能動的能力に即して、全能
 と言われる。したがって、動かされたり受動したりできないことは、全能に反
 しない。

\\


{\scshape Ad secundum dicendum} quod peccare est
deficere a perfecta actione, unde posse peccare est posse deficere in
agendo, quod repugnat omnipotentiae. Et propter hoc, Deus peccare non
potest, qui est omnipotens. 


&

第二異論に対しては、以下のように言われるべきである。
罪を犯すことは、完全な行為から欠落することである。したがって、罪を犯すこ
 とができると言うことは、行為することにおいて欠けることができるというこ
 とであり、これは全能に反する。このため、全能である神は、罪を犯すことが
 できない。



\\

Quamvis philosophus dicat, in IV {\itshape Topic}.,
quod {\itshape potest Deus et studiosus prava agere}. 
Sed hoc intelligitur vel sub
conditione cuius antecedens sit impossibile, ut puta si dicamus quod
potest Deus prava agere si velit, nihil enim prohibet conditionalem esse
veram, cuius antecedens et consequens est impossibile; sicut si dicatur,
si homo est asinus, habet quatuor pedes. 

&

ただし哲学者は、『トピカ』第4巻で「神は妬み深く悪事をすること
 ができる」と述べている。
しかしこれは、一つには、その前件が不可能であるような条件文として理解され
 る。たとえば私たちが、「神がそれを意志するならば、悪をなすことができる」
 と言う場合のように。というのも、前件と後件が偽である条件文が真であるこ
 とを妨げるものはないからである。「もし人間がロバならば、人間は四本の足
 を持つ」のように。



\\



Vel ut intelligatur quod Deus
potest aliqua agere, quae nunc prava videntur; quae tamen si ageret,
bona essent. Vel loquitur secundum communem opinionem gentilium, qui
homines dicebant transferri in deos, ut Iovem\footnote{活用に注意。
 Iupiter, Iovis, Iovi, Iovem, Iove} vel Mercurium.

&

あるいはもう一つには、神は、今は悪いように見える何か
 をなすことができる、と理解できる。しかし、もし神がそれをなすならば、そ
 れは善いものであろう。あるいは、異教徒の共通の意見に従って語っていると
 も考えられる。彼らは、人間が、ジュピターやマーキュリーのような神々に変
 容できると言っていたからである。


\\

{\scshape Ad tertium dicendum} quod Dei omnipotentia
ostenditur maxime in parcendo et miserando, quia per hoc ostenditur Deum
habere summam potestatem, quod libere peccata dimittit, eius enim qui
superioris legi astringitur, non est libere peccata condonare. 

&

第三異論に対しては、以下のように言われるべきである。
神の全能は、許すことと憐れむことにおいて、最大限に示される。
なぜなら、自由に罪を赦すことによって、神が最高の権能を持つことが示される
 からである。じっさい、上位の法に縛られる者は、自由に罪を赦すことができ
 ない。

\\

Vel quia,
parcendo hominibus et miserando, perducit eos ad participationem
infiniti boni, qui est ultimus effectus divinae virtutis. Vel quia, ut
supra dictum est, effectus divinae misericordiae est fundamentum omnium
divinorum operum, nihil enim debetur alicui nisi propter id quod est
datum ei a Deo non debitum. In hoc autem maxime divina omnipotentia
manifestatur, quod ad ipsam pertinet prima institutio omnium bonorum.

&

あるいは、人間を許し憐れむことによって、人間を、神の力の究
 極の結果である無限の善の分有へ導く。
あるいは、前に述べられたとおり、神の憐れみの
 結果は、すべての神の業の基礎である。なぜなら、神から無償で与えられたも
 ののためにでなければ、何かを誰かに負うことはないからである。
しかし、すべての善の第一の設立に属することにおいて、最大限に神の全能は明
 示される。

\\



{\scshape Ad quartum dicendum} quod possibile absolutum
non dicitur neque secundum causas superiores, neque secundum causas
inferiores sed secundum seipsum. Possibile vero quod dicitur secundum
aliquam potentiam, nominatur possibile secundum proximam causam. 


&

第四異論に対しては、以下のように言われるべきである。
「無条件に可能なもの」は、上位の原因に即しても、下位の原因に即しても語ら
 れず、それ自身に即して語られる。しかし、何らかの能力に即して「可
 能なもの」と言われるものは、近接原因に即して「可能なもの」と名付けられ
 る。


\\


Unde ea
quae immediate nata sunt fieri a Deo solo, ut creare, iustificare, et
huiusmodi, dicuntur possibilia secundum causam superiorem, quae autem
nata sunt fieri a causis inferioribus, dicuntur possibilia secundum
causas inferiores. 



&

したがって、創造や義化
などのように、直接的に神によってのみ生じるようなものは、
上位の原因に即して「可能なもの」と言われ、下位の諸原因によって生じるよう
 なものは、下位の諸原因に即して「可能なもの」と言われる。

\\


Nam secundum conditionem causae proximae, effectus
habet contingentiam vel necessitatem, ut supra dictum est. In hoc autem
reputatur stulta mundi sapientia, quod ea quae sunt impossibilia
naturae, etiam Deo impossibilia iudicabat. Et sic patet quod
omnipotentia Dei impossibilitatem et necessitatem a rebus non excludit.


&

そして、前に述べられたとおり、近接原因の状態に応じて、結果は偶然性や必然
 性を持つ。しかし、自然本性にとって不可能なことが、神にとっても不可能だ
 と判断するという点で、世の知恵は愚かだとされる。このように、神の全能が、
 諸事物から不可能や必然を排除しないことが明らかである。


\end{longtable}
\newpage



\rhead{a.~4}
\begin{center}
 {\Large {\bf ARTICULUS QUARTUS}}\\
 {\large UTRUM DEUS POSSIT FACERE QUOD PRAETERITA NON FUERINT}\\
 {\footnotesize II-IIae, q.152, a.3, ad 3; I {\itshape Sent.}, d.42,
 q.2, a.2; II {\itshape SCG}, cap.25; {\itshape De Pot.}, q.1, a.3, ad
 9; {\itshape Quodl}.~V, q.2, a.1; VI {\itshape Ethic.}, lect.2. }\\
 {\Large 第四項\\神は過去をなかったことにすることができるか}
\end{center}

\begin{longtable}{p{21em}p{21em}}


{\Huge A}{\scshape d quartum sic proceditur}. Videtur quod Deus possit
facere quod praeterita non fuerint. Quod enim est impossibile per se,
magis est impossibile quam quod est impossibile per accidens. Sed Deus
potest facere id quod est impossibile per se, ut caecum illuminare, vel
mortuum resuscitare. Ergo multo magis potest Deus facere illud quod est
impossibile per accidens. Sed praeterita non fuisse, est impossibile per
accidens, accidit enim Socratem non currere esse impossibile, ex hoc
quod praeteriit. Ergo Deus potest facere quod praeterita non fuerint.

&

第四項の問題へ、議論は以下のように進められる。神は過去をなかっ
たことにすることができると思われる。理由は以下の通り。自体的に不可能なも
のは、附帯的に不可能なものよりも、不可能である。ところが、神は、目の見え
ない人を見えるようにしたり、死者を蘇らせたりするように、自体的に不可能な
ことをすることができる。ゆえに、附帯的に不可能なことは、より一層、するこ
とができる。ところが、「過去がなかった」ということは、附帯的に不可能なことである。な
ぜなら、「ソクラテスが走らないこと」が不可能であることは、それが過ぎ去ったこ
と(ソクラテスが走ったこと)から附帯的に生じるからである。ゆえに、神は過
去をなかったことにすることができる。




\\


{\scshape 2 Praeterea}, quidquid Deus facere potuit,
potest, cum eius potentia non minuatur. Sed Deus potuit facere, antequam
Socrates curreret, quod non curreret. Ergo, postquam cucurrit, potest
Deus facere quod non cucurrerit.


&

さらに、神がすることができたものはなんであれ、それを、現在、神はすることがで
 きる。なぜなら、神の能力は減らないからである。ところが、神はソクラテス
 が走る前に、ソクラテスが走らないようにすることができた。ゆえに、彼が走っ
 た後も、神は彼が走らなかったようにすることができる。



\\


{\scshape 3 Praeterea}, caritas est maior virtus quam
virginitas. Sed Deus potest reparare caritatem amissam. Ergo et
virginitatem. Ergo potest facere quod illa quae corrupta fuit, non
fuerit corrupta.


&

さらに、愛徳は処女性よりも大きな徳である。しかし神は、失われた愛徳を回復
させることができる。ゆえに、処女性も回復させることができる。ゆえに、堕落
したものを、堕落しなかったようにすることができる。

\\


{\scshape Sed contra est} quod Hieronymus dicit, {\itshape cum
Deus omnia possit, non potest de corrupta facere incorruptam}. Ergo eadem
ratione non potest facere de quocumque alio praeterito quod non fuerit.


&


しかし反対に、ヒエロニュムスは、神はすべてのことを為すことができるが、腐
敗したものを腐敗していないものにすることはできない、と述べている。ゆえに、
同じ理由で、他のなんであれ、過去のことについて、それをなかったことにする
 ことはできない。

\\


{\scshape Respondeo dicendum} quod, sicut supra dictum
est, sub omnipotentia Dei non cadit aliquid quod contradictionem
implicat. Praeterita autem non fuisse, contradictionem implicat. Sicut
enim contradictionem implicat dicere quod Socrates sedet et non sedet,
ita, quod sederit et non sederit. 


&

解答する。以下のように言われるべきである。
前に述べられたとおり、神の全能のもとに、矛盾を含むものは含まれない。
ところで、「過去がなかった」は矛盾を含む。
ちょうど、「ソクラテスが、現在、座っていてかつ座っていない」が矛盾を含むよう
に、「ソクラテスが、過去に、座っていてかつ座っていなかった」も矛盾を含む
ように。



\\


Dicere autem quod sederit, est dicere
quod sit praeteritum, dicere autem quod non sederit, est dicere quod non
fuerit. Unde praeterita non fuisse, non subiacet divinae potentiae. Et
hoc est quod Augustinus dicit, {\itshape contra Faustum}, {\itshape quisquis ita dicit, si
Deus omnipotens est, faciat ut quae facta sunt, facta non fuerint, non
videt hoc se dicere, si Deus omnipotens est, faciat ut ea quae vera
sunt, eo ipso quod vera sunt, falsa sint}. Et philosophus dicit, in VI
{\itshape Ethic}., quod {\itshape hoc solo privatur Deus, ingenita facere quae sunt facta}.


&

ところが、座っていたと言うことは、それが過去であった、と言うことであり、
座っていなかった、と言うことは、それが過去でなかった、と言うことである。
ゆえに、「過去が過去でなかったこと」は、神の能力に含まれない。そしてこれ
が、アウグスティヌスが『ファウストゥス駁論』で以下のように言うことである。
「もし神が全能なら、為されたことを為されなかったことにしてみよ、と言う者
は誰でも、それは、もし神が全能なら、真であるものを、それが真であるがゆえ
に偽であるものにせよ、と自分が言っていることがわかっていない」。また哲学
者も『ニコマコス倫理学』第6巻で「為されたことを為されなかったことにする
ことだけは、神にできない」と述べている。


\\


{\scshape Ad primum ergo dicendum} quod, licet praeterita non fuisse sit
impossibile per accidens, si consideretur id quod est praeteritum, idest
cursus Socratis; tamen, si consideretur praeteritum sub ratione
praeteriti, ipsum non fuisse est impossibile non solum per se, sed
absolute, contradictionem implicans. Et sic est magis impossibile quam
mortuum resurgere, quod non implicat contradictionem, quod dicitur
impossibile secundum aliquam potentiam, scilicet naturalem. Talia enim
impossibilia divinae potentiae subduntur.


&

第一異論に対しては、それゆえ、以下のように言われるべきである。もし、ソク
ラテスが走ったことのような、過去の事実が考察されるならば、過去がなかった
ことは附帯的に不可能である。しかし、過去の性格のもとで過去が考察されるな
らば、それがなかったことは、たんに自体的にだけでなく、無条件的に不可能であ
る。したがって、それは死者を復活させることよりも不可能である。なぜなら、
後者は、何らかの能力、つまり自然本性的能力にしたがって不可能と言われるの
であり、矛盾を含むわけではないから。じっさい、このような不可能性は、神の
能力のもとに服する。

\\


{\scshape Ad secundum dicendum} quod sicut Deus, quantum
est ad perfectionem divinae potentiae, omnia potest, sed quaedam non
subiacent eius potentiae, quia deficiunt a ratione possibilium; ita, si
attendatur immutabilitas divinae potentiae, quidquid Deus potuit,
potest; aliqua tamen olim habuerunt rationem possibilium, dum erant
fienda, quae iam deficiunt a ratione possibilium, dum sunt facta. Et sic
dicitur Deus ea non posse, quia ea non possunt fieri.


&

第二異論に対しては、以下のように言われるべきである。神は、神の能力の完全
性にかんする限り、すべてを為すことができるが、あるものは、それが可能なも
のの性格を欠いているから、その能力に服していない。ちょうどそのように、も
し、神の能力の不変性が注目されるならば、神が過去に為すことができたことは、
現在、神はそれを為すことができる。しかし、あるものは、かつて、つまり為さ
れるべきものであった間は可能なものの性格を持っていたが、しかし、為されて
しまったかぎり、それは可能なものという性格を欠いている。その意味で、神は、
それらが為されえないことないので、それらを為すことができないと言われる。

\\


{\scshape Ad tertium dicendum} quod omnem corruptionem
mentis et corporis Deus auferre potest a muliere corrupta, hoc tamen ab
ea removeri non poterit, quod corrupta non fuerit. Sicut etiam ab aliquo
peccatore auferre non potest quod non peccaverit, et quod caritatem non
amiserit.


&


第三異論に対しては、以下のように言われるべきである。
神は、精神と身体のあらゆる堕落を、堕落した女性から取り去ることができる。
 しかし、「堕落した」ということを、その女性から取り去ることはできない。ちょうど、
 罪人から罪を犯したことや愛徳を失ったことを取り去ることができない
 ように。


\end{longtable}
\newpage




\rhead{a.~5}
\begin{center}
 {\Large {\bf ARTICULUS QUINTUS}}\\
 {\large UTRUM DEUS POSSIT FACERE QUAE NON FACIT}\\
 {\footnotesize I {\itshape Sent.}, d.43, q.2; II {\itshape SCG.},
 cap.23, 24, 27; III, cap.98; {\itshape De Pot.}, q.1, a.5.}\\
 {\Large 第五項\\神は今作っていないものを作ることができるか}
\end{center}

\begin{longtable}{p{21em}p{21em}}


{\Huge A}{\scshape d quintum sic proceditur}. Videtur quod
Deus non possit facere nisi ea quae facit. Deus enim non potest facere
quae non praescivit et praeordinavit se facturum. Sed non praescivit
neque praeordinavit se facturum, nisi ea quae facit. Ergo non potest
facere nisi ea quae facit.

&

第五項の問題へ、議論は以下のように進められる。
神は、今作っているもの以外のものを作ることができないと思われる。理由は以下
の通り。
神は、自分が作るであろうことを予め知り、予め秩序付けたもの以外のものを作ること
ができない。しかし、自分が作るであろうことを予め知り、予め秩序付けるも
のとは、今作っているものに他ならない。ゆえに、今作っている以外のものを作ること
はできない。



\\


{\scshape 2 Praeterea}, Deus non potest facere nisi quod
debet, et quod iustum est fieri. Sed Deus non debet facere quae non
facit, nec iustum est ut faciat quae non facit. Ergo Deus non potest
facere nisi quae facit.


&

さらに、神は、すべきであること、それがなされることが正しいこと以外のこと
 を、することができない。ところが、神は、今しないことをするべきではないし、していないこ
とをするのが正しいのでもない。ゆえに、神は、今しないことをすることがで
きない。

\\


{\scshape 3 Praeterea}, Deus non potest facere nisi quod
bonum est, et conveniens rebus factis. Sed rebus factis a Deo non est
bonum nec conveniens aliter esse quam sint. Ergo Deus non potest facere
nisi quae facit.


&

神は、善であるものと、作られた事物に適合するもの以外のものを作ることができない。
しかし、神によって作られた事物にとって、今あるのとちがうようにあることは、
 善いことでも適合することでもない。ゆえに、神は、今作っていないものを作
 ることができない。

\\


{\scshape Sed contra est} quod dicitur {\itshape Matt}.~{\scshape xxvi}, {\itshape an
non possum rogare patrem meum, et exhibebit mihi modo plus quam duodecim
legiones Angelorum?} Neque autem ipse rogabat, neque pater exhibebat ad
repugnandum Iudaeis. Ergo Deus potest facere quod non facit.


&

しかし反対に、『マタイによる福音書』26章で、「私が父にお願いして、12以上
の天使軍団をすぐに遣わしてもらうことができないだろうか」\footnote{「わ
たしが父にお願いできないとでも思うのか。お願いすれば、父は十二軍団以上
の天使を今すぐ送ってくださるであろう。」(26:53)}と言われている。とこ
ろが、イエスはこれを願わなかったし、父はユダヤ人たちを罰するために軍団
を遣わすことがなかった。ゆえに、神は、今しないことをすることができる。


\\


{\scshape Respondeo dicendum} quod circa hoc quidam
dupliciter erraverunt. Quidam enim posuerunt Deum agere quasi ex
necessitate naturae; ut sicut ex actione rerum naturalium non possunt
alia provenire nisi quae eveniunt, utpote ex semine hominis homo, ex
semine olivae oliva; ita ex operatione divina non possint aliae res, vel
alius ordo rerum effluere, nisi sicut nunc est. 


&

解答する。以下のように言われるべきである。
これについては、人々は二通りのかたちで誤った。ある人々は、神が言わば自然
の必然性によって働くと考えた。そして、人間の種からは人間が、オリーブの
種からはオリーブが生じるように、自然的諸事物の作用からは、今現在生じて
いること以外のものが生じることがありえないように、神の働きからも、今そ
うある以外の他の事物や、諸事物の他の秩序が生じることはありえないと考え
た。


\\


Sed supra ostendimus
Deum non agere quasi ex necessitate naturae, sed voluntatem eius esse
omnium rerum causam; neque etiam ipsam voluntatem naturaliter et ex
necessitate determinari ad has res. Unde nullo modo iste cursus rerum
sic ex necessitate a Deo provenit, quod alia provenire non possent. 



&

しかし、神がいわば自然の必然性に基づいて働くのでなく、むしろ神の意志が、す
 べての諸事物の原因であるということを、私たちは前に示した。
そしてまた、神の意志が、本性的に、必然性によって、これらの事物に限定されるのではないこと
 も示した。したがって、現在のこの諸事物のあり方が、必然に基づいて、他の
 ものは生じ得なかったというようなかたちで、神から
 来たのではない。

\\


Alii
vero dixerunt quod potentia divina determinatur ad hunc cursum rerum,
propter ordinem sapientiae et iustitiae divinae, sine quo Deus nihil
operatur. Cum autem potentia Dei, quae est eius essentia, non sit aliud
quam Dei sapientia, convenienter quidem dici potest quod nihil sit in
Dei potentia, quod non sit in ordine divinae sapientiae, nam divina
sapientia totum posse potentiae comprehendit. 


&

また、別の人々は次のように言った。神の能力は、現在のこの諸事物のあり方へ、
 それなしに神が働くことがないところの、神の知恵と正義の秩序のために定められている。
ところが、神の能力は、神の本質であり、神の知恵と別のものではありえないので、
 神の知恵の秩序の中にないものは、神の能力の中にない、と言うことは適切で
 ある。なぜなら、神の知恵は、能力に属するすべてを把握するから。



\\


Sed tamen ordo a divina
sapientia rebus inditus, in quo ratio iustitiae consistit, ut supra
dictum est, non adaequat divinam sapientiam, sic ut divina sapientia
limitetur ad hunc ordinem. Manifestum est enim quod tota ratio ordinis,
quam sapiens rebus a se factis imponit, a fine sumitur. 
Quando igitur
finis est proportionatus rebus propter finem factis, sapientia facientis
limitatur ad aliquem determinatum ordinem. Sed divina bonitas est finis
improportionabiliter excedens res creatas. Unde divina sapientia non
determinatur ad aliquem certum ordinem rerum, ut non possit alius cursus
rerum ab ipsa effluere. Unde dicendum est simpliciter quod Deus potest
alia facere quam quae facit.



&

しかし、神の知恵から諸事物へと与えられた秩序は、前に述べられたとおり、そこにおいて正義の性格が
 成立するのだが、神の知恵がこの秩序へ限定されるよう
 なかたちで神の知恵と対等するのではない。
じっさい、知者が、自分が作ったものへ与える秩序の全性格は、目的から取られ
 る。それゆえ、目的が、目的のために作られた事物と比例するときには、作る
 者の知恵が何らかの限定された秩序へと制限される。しかし、神の善性は、創
 造された諸事物を、比を絶して超越する目的である。
したがって、神の知恵は、神からそれ以外の諸事物のあり方が出てくることがな
 いようなかたちで、ある特定の事物の秩序へと限定されることがない。したがっ
 て、神は、今している以外のことをすることができると端的に言われるべきで
 ある。

\\




&

\\


{\scshape Ad primum ergo dicendum} quod in nobis, in
quibus est aliud potentia et essentia a voluntate et intellectu, et
iterum intellectus aliud a sapientia, et voluntas aliud a iustitia,
potest esse aliquid in potentia, quod non potest esse in voluntate
iusta, vel in intellectu sapiente. Sed in Deo est idem potentia et
essentia et voluntas et intellectus et sapientia et iustitia. Unde nihil
potest esse in potentia divina, quod non possit esse in voluntate iusta
ipsius, et in intellectu sapiente eius. 


&

第一異論に対しては、それゆえ、以下のように言われるべきである。
私たちにおいては、能力と本質が意志と知性と別であり、さらに、知性は知恵か
ら、意志は正義から異なっているので、正しい意志や知恵ある知性のなかにあ
りえないものが、能力の中にありうる。しかし、神においては、能力と本質、
意志と知性、知恵と正義は同一である。したがって、神の
正しい意志や、神の知恵ある知性の中にありえないものは、神の能力の中にも
ありえない。


\\


Tamen, quia voluntas non
determinatur ex necessitate ad haec vel illa, nisi forte ex
suppositione, ut supra dictum est; neque sapientia Dei et iustitia
determinantur ad hunc ordinem, ut supra dictum est; nihil prohibet esse
aliquid in potentia Dei, quod non vult, et quod non continetur sub
ordine quem statuit rebus. 


&

しかし、前に述べられたとおり、意志は、仮定によるのでないかぎり、これやあ
れに必然的に限定されることはない。また、すでに述べられたとおり、神の知
恵と正義も、この秩序へ限定されることはない。したがって、神の能力の中に
ある何かが、神がそれを意志せず、諸事物の中に設置した秩序に含まれない、
ということがありうる。

\\

Et quia potentia intelligitur ut exequens,
voluntas autem ut imperans, et intellectus et sapientia ut dirigens,
quod attribuitur potentiae secundum se consideratae, dicitur Deus posse
secundum potentiam absolutam. Et huiusmodi est omne illud in quo potest
salvari ratio entis, ut supra dictum est.
 Quod autem attribuitur
potentiae divinae secundum quod exequitur imperium voluntatis iustae,
hoc dicitur Deus posse facere de potentia ordinata.

&


また、能力は遂行するものとして、意志は命じる
 ものとして、そして知性と知恵は方向を与えるものとして理解されるので、そ
 れ自体において考察された能力に帰せられるものを、神は、無条件的な能力に
 おいてなすことができると言われる。
そして、そのようなものとは、前に述べられたとおり、有の性格が汲み取られう
 るすべてのものである。
また、正しい意志の命令が遂行される限りでの、神の能力に帰せられるものは、
 神はそれを、秩序付けられた能力からなすことができる、と言われる。


\\


 Secundum hoc ergo,
dicendum est quod Deus potest alia facere, de potentia absoluta, quam
quae praescivit et praeordinavit se facturum, non tamen potest esse quod
aliqua faciat, quae non praesciverit et praeordinaverit se
facturum. Quia ipsum facere subiacet praescientiae et praeordinationi,
non autem ipsum posse, quod est naturale. Ideo enim Deus aliquid facit,
quia vult, non tamen ideo potest, quia vult, sed quia talis est in sua
natura.


&

それゆえ、以上のことを踏まえると、神は、その無条件的な能力について、自分が作ることを
 予め知り、予め秩序付けたこと以外のものを作ることができると言われるべき
 である。しかし、これは、自分が作るであろうことを予め知らなかったことや、予
 め秩序付けていなかったことを作ることができる、というわけではない。なぜな
 ら、作ること・なすことは、先行する知や先行する秩序付けに属するが、作り
 うること・なしうることは、本性的なことなので、それらに属さないからであ
 る。じっさい、神は、それを意志するから何かを作るが、しかし、意志するからそれを作
 り\kenten{うる}のではない。そうではなく、それがその本性においてそのよう
 なものだから、それを作りうる。



\\


{\scshape Ad secundum dicendum} quod Deus non debet
aliquid alicui nisi sibi. Unde, cum dicitur quod Deus non potest facere
nisi quod debet nihil aliud significatur nisi quod Deus non potest
facere nisi quod ei est conveniens et iustum. Sed hoc quod dico
conveniens et iustum, potest intelligi dupliciter. 



&

第二異論に対しては、以下のように言われるべきである。
神は、自分以外の何かに債務を持たない。したがって、「神はあるべきこと以外
 をなすことができない」と言われるとき、これが意味するのは、神は、それに
 適合し、正しいこと以外をなすことができない、ということに他ならない。
しかし、私が「適合し正しい」と言うことは、二通りに理解できる。

\\

Uno modo, sic quod
hoc quod dico conveniens et iustum, prius intelligatur coniungi cum hoc
verbo est, ita quod restringatur ad standum pro praesentibus; et sic
referatur ad potentiam. 
Et sic falsum est quod dicitur, est enim sensus,
Deus non potest facere nisi quod modo conveniens est et iustum. 


&

一つには、「適合し正しい」と私が言うことが、「ある」という言葉に結び
 つけられて先に理解され、現在の状態に制限され、その意味で、能力へ帰せら
 れる。そしてこの意味で言われることは偽である。つまり「神は、現在、適合
 して正しいこと以外をなすことができない」という意味になるからである。


\\


Si vero
prius coniungatur cum hoc verbo potest, quod habet vim ampliandi, et
postmodum cum hoc verbo est, significabitur quoddam praesens confusum,
et erit locutio vera, sub hoc sensu, Deus non potest facere nisi id
quod, si faceret, esset conveniens et iustum.


&

他方、「できる」という言葉に先に結びつけられ、それによって意味が
 強められ、その後に「ある」という言葉に結びつけられるならば、ある種の漠
 然とした現在を意味し、それは真であるだろう。つまりその意味は「神は、も
 しそれが生じるならば、適合し、正しいであろうこと以外のことを作ることは
 できない」となる。


\\


{\scshape Ad tertium dicendum} quod, licet iste cursus
rerum sit determinatus istis rebus quae nunc sunt, non tamen ad hunc
cursum limitatur divina sapientia et potestas. Unde, licet istis rebus
quae nunc sunt, nullus alius cursus esset bonus et conveniens, tamen
Deus posset alias res facere, et alium eis imponere ordinem.


&

第三異論に対しては、以下のように言われるべきである。
現在の諸事物のあり方は、現在存在するこれらの諸事物へ定められているが、し
 かし、このあり方へ、神の知恵と権能が限定されるわけではない。したがって、
 現在存在するこれらの諸事物にとって、他のあり方は善でも適合的でもないが、
 しかし神は、他の諸事物を作り、他の秩序をそれらに与えることができる。


\end{longtable}


\newpage
\rhead{a.~6}
\begin{center}
 {\Large {\bf ARTICULUS SEXTUS}}\\
 {\large UTRUM DEUS POSSIT MELIORA FACERE EA QUAE FACIT}\\
 {\footnotesize I {\itshape Sent.}, d.~44, a.~1, 2, 3.}\\
 {\Large 第六項\\神は作っているものをより善くすることができるか}
\end{center}

\begin{longtable}{p{21em}p{21em}}


{\huge A}{\scshape d sextum sic proceditur}. Videtur quod Deus
non possit meliora facere ea quae facit. Quidquid enim Deus facit,
potentissime et sapientissime facit. Sed tanto fit aliquid melius,
quanto fit potentius et sapientius. Ergo Deus non potest aliquid facere
melius quam facit.

&

第六に対しては次のように進められる。
神は作ったものをより善くすることができないと思われる。なぜなら、
なんであれ神が作るものは、最高度の能力と知恵において作る。
ところが、能力と知恵が高度であればあるだけ、それだけ善いものが生じる。
ゆえに、神が現に作るものよりも善いものを作ることはできない。


\\


{\scshape 2 Praeterea}, Augustinus, contra Maximinum, sic argumentatur,
{\itshape si Deus potuit, et noluit, gignere filium sibi aequalem,
invidus fuit}. Eadem ratione, si Deus potuit res meliores facere quam
fecerit, et noluit, invidus fuit. Sed invidia est omnino relegata a
Deo. Ergo Deus unumquodque fecit optimum. Non ergo Deus potest aliquid
facere melius quam fecit.


&

さらに、アウグスティヌスは、マクシミヌスに反対して次のように論じている。
 「もし神が、自分に等しい息子を生むことができたのにそれを欲しなかったと
 すれば、神は妬んでいたことになる」。
同じ理由で、もし神が、作ったよりもよく事物を作ることができたが、それを欲
 しなかったとすれば、神は妬んでいた。ところが、妬みは、神からまったく取
 り除かれる。ゆえに、神は、各々のものを最善のものとして作った。ゆえに、
 神はなにかを現に作っているよりも善くすることはできない。

\\

{\scshape 3 Praeterea}, id quod est maxime et valde bonum, non potest
melius fieri, quia maximo nihil est maius. Sed, sicut Augustinus dicit
in {\itshape Enchirid}., {\itshape bona sunt singula quae Deus fecit,
sed simul universa valde bona, quia ex omnibus consistit universitatis
admirabilis pulchritudo}. Ergo bonum universi non potest melius fieri a
Deo.


&

さらに、最高度にまったく善であるものを、より善くすることはできない。なぜ
 なら、最大のものより大きいものはないからである。ところが、アウグスティ
 ヌスが『提要』で述べているように、「神が作った個々のものは善であるが、
 同時に、宇宙もまた真に善である。というのも、万物から、驚嘆すべき宇宙の
 美が成り立っているから」。ゆえに、宇宙の善は、神によって、より善くされ
 ることはできない。

\\


{\scshape 4 Praeterea}, homo Christus est plenus gratia
et veritate, et spiritum habet non ad mensuram, et sic non potest esse
melior. Beatitudo etiam creata dicitur esse summum bonum, et sic non
potest esse melius. Beata etiam virgo Maria est super omnes choros
Angelorum exaltata, et sic non potest esse melior. Non igitur omnia quae
fecit Deus, potest facere meliora.

&

さらに、人間キリストは恩恵と真理に満たされ、測りがたい霊をもつ。したがっ
 て、より善くはありえない。さらに、被造の至福は最高善であると言われ、し
 たがって、より善くはありえない。また、至福である処女マリアは、天使たち
 のすべての舞台の上に上げられているので、より善くはありえない。ゆえに、
 神が作ったすべてのものを、より善くすることはできない。


\\


{\scshape Sed contra est} quod dicitur {\itshape ad Ephes}.~{\scshape
iii}, quod {\itshape Deus potens est omnia facere abundantius quam
petimus aut intelligimus}.


&
しかし反対に、『エフェソの信徒への手紙』三節で「神は私たちが望み理解する
 よりも豊かに万物を作ることができる」\footnote{「私たちたちが求めたり、
 思ったりすることすべてを、遙かに超えてかなえることのおできになる方に」(3:20)}と言われている。
\\


{\scshape Respondeo dicendum} quod bonitas alicuius rei est duplex. Una
quidem, quae est de essentia rei; sicut esse rationale est de essentia
hominis. Et quantum ad hoc bonum, Deus non potest facere aliquam rem
meliorem quam ipsa sit, licet possit facere aliquam aliam ea
meliorem. Sicut etiam non potest facere quaternarium maiorem, quia, si
esset maior, iam non esset quaternarius, sed alius numerus. Sic enim se
habet additio differentiae substantialis in definitionibus, sicut
additio unitatis in numeris, ut dicitur in VIII {\itshape
Metaphys}. 


&

解答する。以下のように言われるべきである。
ある事物の善性には二通りある。一つは、
その事物の本質に属するものであり、たとえば、理性的であることは、人間の本
質に属する。そしてこの善にかんしては、神が何らかの事物をそれ自体よりも善
く作ることができない。ただし、それとは別のもので、それより善いものを作ることはで
きる。それはちょうど、四を四より大きくすることができないようなも
のである。というのも、もし、四が四より大きかったならば、それは四ではなく、
別の数だっただろうから。この意味で、『形而上学』8巻で言われるように、定義
において実体的種差を加えることは、数において一を加えるようなものである。


\\

Alia bonitas est, quae est extra essentiam rei; sicut bonum
hominis est esse virtuosum vel sapientem. Et secundum tale bonum, potest
Deus res a se factas facere meliores. 

&


もう一つの善性は、事物の本質の外にあるものであり、たとえば、人間の善は、
 有徳であることや知者であることである。このような善については、神は自分
 によって作られた事物を、より善くすることができる。

\\

Simpliciter autem loquendo,
qualibet re a se facta potest Deus facere aliam meliorem.

&

しかし端的に言うならば、
神は、自分によって作られたどんな事物よりも、\kenten{別の}より善い事物を作ることが
 できる。

\\


{\scshape Ad primum ergo dicendum} quod, cum dicitur
Deum posse aliquid facere melius quam facit, si ly melius sit nomen,
verum est, qualibet enim re potest facere aliam meliorem. Eandem vero
potest facere meliorem quodammodo, et quodammodo non, sicut dictum
est. Si vero ly melius sit adverbium, et importet modum ex parte
facientis, sic Deus non potest facere melius quam sicut facit, quia non
potest facere ex maiori sapientia et bonitate. Si autem importet modum
ex parte facti, sic potest facere melius, quia potest dare rebus a se
factis meliorem modum essendi quantum ad accidentalia, licet non quantum
ad essentialia.

&

第一異論に対しては、それゆえ、次のように言われるべきである。
「神は、現に作るものより善いものを作ることができる」と言われるとき、こ
 の「より善いもの」が名詞である場合には、それは真である。神は、どんな事
 物よりも、より善い別の事物を作ることができるのだから。そして、すでに述
 べられたとおり、同一の事物を、あるかたちでは、より善くすることができる
 が、あるかたちでは、それができない。
他方、「より善いもの」が副詞であるならば[つまり、「神は、現に作るよりも
 善く作ることができる」という意味ならば]、そして、作るものの側からのあ
 りかたを意味しているならば、その場合、神は、現に作るようなあり方よりも
 善いしかたで作ることはできない。なぜなら、より以上の知恵と善性によって作ること
 はありえないからである。
しかし、作られたものの側からのあり方を意味する場合、その場合には、より善
 く作ることができる。なぜなら、自分によって作られた事物に、より善いあり方
 を与えることができるのだから。ただしそれは、附帯的なものにかんしてであ
 り、本質的なものにかんしてではない。




\\


{\scshape Ad secundum dicendum} quod de ratione filii
est quod aequetur patri, cum ad perfectum venerit, non est autem de
ratione creaturae alicuius, quod sit melior quam a Deo facta est. Unde
non est similis ratio.


&

第二異論に対しては、以下のように言われるべきである。
「子」の概念には、完全な者へ到達するという意味で、「父に等しくなる」とい
 うことが含まれる。しかし、ある「被造物」の概念には、「神によって作られ
 たよりもよくある」ということは含まれない。ゆえに、同じ論拠は当てはまらない。


\\


{\scshape Ad tertium dicendum} quod universum,
suppositis istis rebus, non potest esse melius; propter decentissimum
ordinem his rebus attributum a Deo, in quo bonum universi
consistit. Quorum si unum aliquod esset melius, corrumperetur proportio
ordinis, sicut, si una chorda plus debito intenderetur, corrumperetur
citharae melodia. Posset tamen Deus alias res facere, vel alias addere
istis rebus factis, et sic esset illud universum melius.


&

第三異論に対しては、以下のように言われるべきである。
宇宙は、これらの諸事物を前提とすれば、よりよく在ることができない。それは、
 そこにおいて宇宙の善が成立する、神がこれらの事物に与えたこの上なく素晴
 らしい秩序のためにである。それらのうちのひとつがよりよくなったら、秩序
 のバランスは崩れたであろう。ちょうど、弦楽器の一つの弦が、あるべき強さ
 よりも少し強く張られるならば、その楽器のメロディーが壊れただろうように。
 しかし、神は、別の諸事物を作ったり、今ある事物に別の事物を加えたりする
 ことができて、そのようにして、宇宙はよりよくあったであろう。



\\


{\scshape Ad quartum dicendum} quod humanitas Christi ex
hoc quod est unita Deo, et beatitudo creata ex hoc quod est fruitio Dei,
et beata virgo ex hoc quod est mater Dei, habent quandam dignitatem
infinitam, ex bono infinito quod est Deus. Et ex hac parte non potest
aliquid fieri melius eis, sicut non potest aliquid melius esse Deo.


&
第四異論に対しては、以下のように言われるべきである。
キリストの人間性は、神に合一しているために、被造の至福は、神の享受である
 ために、至福なる処女は、神の母であるために、神である無限の善に基づいて、
 ある種の、無限な価値(dignitas)をもつ。そして、この方面では、何もそれら
 より善くはありえない。ちょうど、何も神より善くはありえないように。



\end{longtable}
\newpage


\end{document}