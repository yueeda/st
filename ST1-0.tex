\documentclass[10pt]{jsarticle}
\usepackage{okumacro}
\usepackage{longtable}
\usepackage[polutonikogreek,english,japanese]{babel}
\usepackage{latexsym}
\usepackage{color}

%----- header -------
\usepackage{fancyhdr}
\lhead{{\it Summa Theologiae} I, PROLOGUS}
%--------------------

\bibliographystyle{jplain}

\title{{\bf PRIMA PARS}\\{\HUGE Summae Theologiae}\\Sancti Thomae
Aquinatis\\{\sffamily PROLOGUS}}
\author{Japanese translation\\by Yoshinori {\sc Ueeda}}
\date{Last modified \today}


%%%% コピペ用
%\rhead{a.~}
%\begin{center}
% {\Large {\bf }}\\
% {\large }\\
% {\footnotesize }\\
% {\Large \\}
%\end{center}
%
%\begin{longtable}{p{21em}p{21em}}
%
%&
%
%
%\\
%\end{longtable}
%\newpage



\begin{document}
\maketitle
\pagestyle{fancy}
\newpage
\begin{center}
{\Large 序文}
\end{center}

\begin{longtable}{p{21em}p{21em}}
Quia Catholicae veritatis doctor non solum provectos debet instruere,
sed ad eum pertinet etiam incipientes erudire, secundum illud apostoli
I {\itshape ad Corinth}.~{\scshape iii}: {\itshape tanquam parvulis in
Christo, lac vobis potum dedi, non escam}; propositum nostrae
intentionis in hoc opere est, ea quae ad Christianam religionem
pertinent, eo modo tradere, secundum quod congruit ad eruditionem
incipientium.

&

『コリントの信徒への手紙一』3章のかの使徒の「キリストにおける子供とし
て、あなたたちには大人の食べ物ではなく、飲み物として乳を与えた」
\footnote{「わたしはあなたがたに乳を飲ませて、固い食物は与えませんでし
た」(3:1)}によれば、西方教会の真理の教師は、勉学が進んだ者たちをさらに
鍛錬しなければならないだけでなく、勉学を始めた者たちを教え導びくことも
またその任務の一部であるから、この著作における私たちの意図は、キリスト
教に属することがらを、学び始めた者たちを導くのにふさわしいかたちで伝え
ることである。

\\

Consideravimus namque huius
doctrinae novitios, in his quae a diversis conscripta sunt, plurimum
impediri, partim quidem propter multiplicationem inutilium quaestionum,
articulorum et argumentorum; partim etiam quia ea quae sunt necessaria
talibus ad sciendum, non traduntur secundum ordinem disciplinae, sed
secundum quod requirebat librorum expositio, vel secundum quod se
praebebat occasio disputandi; partim quidem quia eorundem frequens
repetitio et fastidium et confusionem generabat in animis
auditorum. 

&

じっさい、私たちは、この教えを学び始めた人たちが、さまざまな人々によっ
て書かれたものの中で、ひどく妨げられているのを見てきた。一つにはそれは、
不必要な問題、項、議論の増加のためであり、また一つには、そのような人々
にとって、知識を得るために必要なことがらが、学びの順序に従ってではなく、
書物の解説が必要としたのに応じて、あるいは、討論の機会に提出されたのに
応じて書かれているからであり、あるいはさらに、同じことの頻繁な繰り返し
が、聞く人々の心に、嫌悪感と混乱を生み出したからである。


\\

Haec igitur et alia huiusmodi evitare studentes, tentabimus,
cum confidentia divini auxilii, ea quae ad sacram doctrinam pertinent,
breviter ac dilucide prosequi, secundum quod materia patietur.

&

ゆえに、こういったことやその他これに類したことがらを避けるように努力し
ながら、神の助けに信頼しつつ、聖なる教えに含まれることがらを、題材が許
すかぎり手短に、かつ明快に、考究してみたいと思う。

\end{longtable}
\end{document}



