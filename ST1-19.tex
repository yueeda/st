\documentclass[10pt]{jsarticle} % use larger type; default would be 10pt
%\usepackage[utf8]{inputenc} % set input encoding (not needed with XeLaTeX)
%\usepackage[round,comma,authoryear]{natbib}
%\usepackage{nruby}
\usepackage{okumacro}
\usepackage{longtable}
%\usepqckage{tablefootnote}
\usepackage[polutonikogreek,english,japanese]{babel}
%\usepackage{amsmath}
\usepackage{latexsym}
\usepackage{color}

%----- header -------
\usepackage{fancyhdr}
\lhead{{\it Summa Theologiae} I, q.~19}
%--------------------

\bibliographystyle{jplain}

\title{{\bf PRIMA PARS}\\{\HUGE Summae Theologiae}\\Sancti Thomae
Aquinatis\\{\sffamily QUEAESTIO DECIMANONA}\\DE VOLUNTATE DEI}
\author{Japanese translation\\by Yoshinori {\sc Ueeda}}
\date{Last modified \today}


%%%% コピペ用
%\rhead{a.~}
%\begin{center}
% {\Large {\bf }}\\
% {\large }\\
% {\footnotesize }\\
% {\Large \\}
%\end{center}
%
%\begin{longtable}{p{21em}p{21em}}
%
%&
%
%
%\\
%\end{longtable}
%\newpage



\begin{document}
\maketitle
\pagestyle{fancy}

\begin{center}
{\Large 第十九問\\神の意志について}
\end{center}

\newpage
\begin{longtable}{p{21em}p{21em}}

{\huge P}{\scshape ost} considerationem eorum quae ad divinam scientiam pertinent,
considerandum est de his quae pertinent ad voluntatem divinam, ut sit
prima consideratio de ipsa Dei voluntate; secunda, de his quae ad
voluntatem absolute pertinent; tertia, de his quae ad intellectum in
ordine ad voluntatem pertinent. 

&

神の知に属する事柄が考察された後、神の意志に属する事柄が考察されるべきで
 ある。第一の考察は神の意志そのものについて、第二の考察は無条件的に意志
 に属することについて、第三の考察は意志への秩序において知性に属する事柄
 についてである。

\\


Circa ipsam autem voluntatem quaeruntur
duodecim. 

\begin{enumerate}
 \item utrum in Deo sit voluntas.
 \item utrum Deus velit alia a se.
 \item utrum quidquid Deus vult, ex necessitate velit.
 \item utrum voluntas Dei sit causa rerum.
 \item utrum voluntatis divinae sit assignare aliquam causam.
 \item utrum voluntas divina semper impleatur.
 \item utrum voluntas Dei sit mutabilis.
 \item utrum voluntas Dei necessitatem rebus volitis imponat.
 \item utrum in Deo sit voluntas malorum.
 \item utrum Deus habeat liberum arbitrium.
 \item utrum sit distinguenda in Deo voluntas signi.
 \item utrum convenienter circa divinam voluntatem ponantur quinque signa.
\end{enumerate}
 
&

意志それ自体に関して12のことが問われる。
 
\begin{enumerate}
 \item 神の中に意志があるか。
 \item 神は自分以外のものを意志するか。
 \item 神は、何であれ意志するものを、必然的に意志するか。
 \item 神の意志は事物の原因か。
 \item 神の意志には、なんらかの原因を指定することが属するか。
 \item 神の意志は常に満たされるか。
 \item 神の意志は変化しうるか。
 \item 神の意志は、意志された事物に必然性を課すか。
 \item 神の中に悪の意志があるか。
 \item 神は自由選択を持つか。
 \item 神の中で、しるしの意志が区別されるべきか。
 \item 神の意志をめぐって、五つのしるしが適切に措定されているか。
\end{enumerate}


\end{longtable}



\newpage

\rhead{a.~1}
\begin{center}
 {\Large {\bf ARTICULUS PRIMUS}}\\
 {\large UTRUM IN DEO SIT VOLUNTAS}\\
 {\footnotesize Infra, q.~54, a.~2; I {\itshape Sent.}, d.~45, a.~1;
 {\itshape SCG.}, cap.~72, 73; IV, cap.~19; {\itshape De Verit.}, q.~23,
 a.~1; {\itshape Compend.~Theol.}, cap.~32.}\\
 {\Large 第一項\\神の中に意志があるか}
\end{center}

\begin{longtable}{p{21em}p{21em}}



{\huge A}{\scshape d primum sic proceditur}. Videtur quod in
Deo non sit voluntas. Obiectum enim voluntatis est finis et bonum. Sed
Dei non est assignare aliquem finem. Ergo voluntas non est in Deo.

&

 第一項の問題へ、議論は以下のように進められる。
 神の中に意志はないと思われる。
 理由は以下の通り。
 意志の対象は、目的であり、善である。
 ところが、神になんらかの目的を割り当てることはできない。
 ゆえに、神の中に意志はない。
 
\\


{\scshape 2 Praeterea}, voluntas est appetitus
quidam. Appetitus autem, cum sit rei non habitae, imperfectionem
designat, quae Deo non competit. Ergo voluntas non est in Deo.

&

 さらに、意志は、ある種の欲求である。
 ところで、欲求は、持っていない事柄にかかわるので、不完全性を指し示すが、
 これは神に適合しない。ゆえに、神の中に意志はない。

\\


{\scshape 3 Praeterea}, secundum philosophum, in III {\itshape de
Anima}, voluntas est movens motum. Sed Deus est primum movens
immobile, ut probatur VIII {\itshape Physic}. Ergo in Deo non est voluntas.

&

 さらに、『デ・アニマ』第3巻の哲学者によれば、意志は、動かされて動かすも
 のである\footnote{欲求は、欲求の対象(善)によって動かされ、身体を動か
 す。}。ところで、『自然学』第3巻で証明されているとおり、神は、不動の
 第一動者である。
 ゆえに、神の中に意志はない。

\\


{\scshape  Sed contra est} quod dicit apostolus,
{\itshape Rom}.~{\scshape xii}, {\itshape ut probetis quae sit voluntas Dei}.

&

 しかし反対に、使徒は『ローマの信徒への手紙』12で、「神の意志が何である
 かを、あなた方が示すように」\footnote{「何が神の御心であるか...をわきま
 えるようになりなさい。」(12:2)}と言われている。
 
\\


{\scshape Respondeo dicendum} in Deo voluntatem esse,
sicut et in eo est intellectus, voluntas enim intellectum
consequitur. Sicut enim res naturalis habet esse in actu per suam
formam, ita intellectus intelligens actu per suam formam
intelligibilem. Quaelibet autem res ad suam formam naturalem hanc habet
habitudinem, ut quando non habet ipsam, tendat in eam; et quando habet
ipsam, quiescat in ea. Et idem est de qualibet perfectione naturali,
quod est bonum naturae. Et haec habitudo ad bonum, in rebus carentibus
cognitione, vocatur appetitus naturalis. Unde et natura intellectualis
ad bonum apprehensum per formam intelligibilem, similem habitudinem
habet, ut scilicet, cum habet ipsum, quiescat in illo; cum vero non
habet, quaerat ipsum. Et utrumque pertinet ad voluntatem. Unde in
quolibet habente intellectum, est voluntas; sicut in quolibet habente
sensum, est appetitus animalis. Et sic oportet in Deo esse voluntatem,
cum sit in eo intellectus. Et sicut suum intelligere est suum esse, ita
suum velle.

&

解答する。
意志は知性に伴うから、神の中に知性があるように神の中には意志があると言われるべきである。
理由は以下の通り。ちょうど自然的諸事物が自分の形相を通して現実態において存在を持つよ
 うに、知性は自分がもつ可知的な形相を通して現実に知性認識するもので
 ある。
 ところでどんな事物も、自分がもつ自然本性的な形相に対して以下のよう
 な関係をもつ。すなわち、それをもっていないときにはそれへと傾くが、そ
 れをもっているときはそれの内に休らう。このことはどんな自然本性的な
 完全性すなわち自然の善についても同じである。
 そしてこの善への関係が、認識を欠く事物においては自然本性的欲求と呼
 ばれる。したがって知性的本性も、可知的形相を通して把握された善へと、
 同様の関係をもつ。すなわち、それをもっているときにはその中で休らい、もっ
 ていないときにはそれを求める。そしてこのどちらも意志に属する。
 したがって、ちょうど感覚をもつどんなものの中にも動物的欲求があるよう
 に、知性を持つどんなものの中にも意志がある。このようにして、神の中に
 は知性があるのだから意志もあるのでなければならない。そしてちょうど、
 自らの知性認識が自らの存在であるように、自らの意志もまた自らの存在であ
 る。

\\


{\scshape Ad primum ergo dicendum} quod, licet nihil
aliud a Deo sit finis Dei, tamen ipsemet est finis respectu omnium quae
ab eo fiunt. Et hoc per suam essentiam, cum per suam essentiam sit
bonus, ut supra ostensum est, finis enim habet rationem boni.

&

 第一異論に対しては、それゆえ、以下のように言われるべきである。
たしかに、神以外の何かが神の目的ではないが、神自身が、神から生じるす
 べてのもの\footnote{「神から生じるすべてのもの」と言えば、すべての被造
 物のことであるように思えるが、ここの論旨を考えると、とくに神の意志を指すのだ
 ろう。}に関して目的である。そしてこのことは神の本質による。な
 ぜなら前に示されたとおり\footnote{第6問第3項。}、神は自らの本質によって善だからである。と
 いうのも、目的は善の性格をもつからである。

 
\\


{\scshape Ad secundum dicendum} quod voluntas in nobis
pertinet ad appetitivam partem, quae licet ab appetendo nominetur, non
tamen hunc solum habet actum, ut appetat quae non habet; sed etiam ut
amet quod habet, et delectetur in illo. Et quantum ad hoc voluntas in
Deo ponitur; quae semper habet bonum quod est eius obiectum, cum sit
indifferens ab eo secundum essentiam, ut dictum est.

&

 第二異論に対しては以下のように言われるべきである。
 私たちにおいて意志は欲求的部分に属する。
 この部分は、欲求するということから名付けられているが、しかし、もってい
 ないものを欲求するという働きだけをもつのではなく、もっているものを愛し、
 そこにおいて喜ぶという働きももつ。そしてこの後者にかんして、意志は神
 の中に措定される。すでに述べられたように、神の中の意志はその対象と本
 質において異なることがないので、対象である善を常に所有している。

\\


{\scshape Ad tertium dicendum} quod voluntas cuius
obiectum principale est bonum quod est extra voluntatem, oportet quod
sit mota ab aliquo. Sed obiectum divinae voluntatis est bonitas sua,
quae est eius essentia. Unde, cum voluntas Dei sit eius essentia, non
movetur ab alio a se, sed a se tantum, eo modo loquendi quo intelligere
et velle dicitur motus. Et secundum hoc Plato dixit quod primum movens
movet seipsum.

&

 第三異論に対しては以下のように言われるべきである。
 主要な目的が意志の外に在る善であるような意志は、何かによって動かされ
 たものでなければならない。しかし神の意志の対象は自分の善性であり、
 それは自分の本質である。したがって神の意志は神の本質なので、自分以外
 の他のものによって動かされることはなく、ただ自分によって動かされる。た
 だし知性認識することや意志することが運動と言われる言い方によってでは
 あるが。この意味で、プラトンは第一の動者は自分を動かすと言った
 \footnote{『パイドロス』245C-D参照。}。


\end{longtable}



\newpage





\rhead{a.~2}
\begin{center}
 {\Large {\bf ARTICULUS SECUNDUS}}\\
 {\large UTRUM DEUS VELIT ALIA A SE}\\
 {\footnotesize I {\itshape Sent.}, d.~14, a.~2; I {\itshape SCG.},
 cap.~75, 76, 77; {\itshape De Verit.}, q.~23, a.~4.}\\
 {\Large 第二項\\神は自分以外のものを意志するか}
\end{center}

\begin{longtable}{p{21em}p{21em}}
{\huge A}{\scshape d secundum sic proceditur}. Videtur quod Deus non
 velit alia a se. Velle enim divinum est eius esse. Sed Deus non est
 aliud a se. Ergo non vult aliud a se.


&

第二項の問題へ、議論は以下のように進められる。
神は自分以外のものを意志しないと思われる。理由は以下の通り。
神の「意志すること」は、神の存在(=「あること」)である。ところが、神は、自分以外のもので
 はない。ゆえに、自分以外のものを意志しない。


\\





{\scshape 2 Praeterea}, volitum movet voluntatem, sicut appetibile
 appetitum, ut dicitur in III {\itshape de Anima}. Si igitur Deus velit aliquid
 aliud a se, movebitur eius voluntas ab aliquo alio, quod est
 impossibile.


&

さらに、『デ・アニマ』第三巻で言われるように、ちょうど欲求されうるものが
 欲求を動かすように、意志されたものは意志を動かす。ゆえに、もし神が自分
 以外のものを意志するならば、神は何か他のものによって動かされることにな
 るが、これは不可能である。

\\





{\scshape 3 Praeterea}, cuicumque voluntati sufficit aliquod volitum,
 nihil quaerit extra illud. Sed Deo sufficit sua bonitas, et voluntas
 eius ex ea satiatur. Ergo Deus non vult aliquid aliud a se.


&


さらに、意志された何かが十分であるようなどんな意志にとっても、それ以外に
 何かを求めることはない。ところで、神の善性は神にとって十分であり、神の
 意志は、それによって満たされる。ゆえに、神は、自分以外の何も意志しない。

\\





{\scshape 4 Praeterea}, actus voluntatis multiplicatur secundum
 volita. Si igitur Deus velit se et alia a se, sequitur quod actus
 voluntatis eius sit multiplex, et per consequens eius esse, quod est
 eius velle. Hoc autem est impossibile. Non ergo vult alia a se.


&

さらに、意志の作用は、意志されたものにしたがって多数化される。ゆえに、も
 し神が自分と自分以外のものを意志するならば、神の意志の作用は多様である
 ことになり、したがって、神の意志であるところの神の存在もまた[多様であ
 ることになる]。これは不可能である。ゆえに、神は自分以外のものを意志し
 ない。

\\





{\scshape  Sed contra est} quod apostolus dicit, I {\itshape Thess}.~{\scshape iv}, {\itshape haec est
 voluntas Dei, sanctificatio vestra}.


&

しかし反対に、司徒は『テサロニケの信徒への手紙1』4で、「これが神の意
 志である。あなたたちを聖なる者とすることが。」\footnote{「神の御心は、
 あなたがたが聖なる者となることです。」(4:3)}と述べている。


\\





{\scshape Respondeo dicendum} quod Deus non solum se vult, sed etiam
 alia a se. Quod apparet a simili prius introducto. Res enim naturalis
 non solum habet naturalem inclinationem respectu proprii boni, ut
 acquirat ipsum cum non habet, vel ut quiescat in illo cum habet; sed
 etiam ut proprium bonum in alia diffundat, secundum quod possibile
 est. Unde videmus quod omne agens, inquantum est actu et perfectum,
 facit sibi simile. Unde et hoc pertinet ad rationem voluntatis, ut
 bonum quod quis habet, aliis communicet, secundum quod possibile
 est. Et hoc praecipue pertinet ad voluntatem divinam, a qua, per
 quandam similitudinem, derivatur omnis perfectio. Unde, si res
 naturales, inquantum perfectae sunt, suum bonum aliis communicant,
 multo magis pertinet ad voluntatem divinam, ut bonum suum aliis per
 similitudinem communicet, secundum quod possibile est. Sic igitur vult
 et se esse, et alia. Sed se ut finem, alia vero ut ad finem, inquantum
 condecet divinam bonitatem etiam alia ipsam participare.


&

解答する。以下のように言われるべきである。
神は自分を意志するだけでなく、自分以外のものも意志する。
このことは、前に導入された類似の事柄から明らかである。
つまり、自然的事物は、自分に固有の善にかんして、それをもたないときにはそ
 れを獲得し、もっているときにはそこに休らうという本性的な傾向性をもつだ
 けでなく、できる限り、その固有の善を他へと及ぼそうとする。
それゆえ、すべての作用者は、それが現実態にあり完全であるかぎり、自分
 と似たものを作り出すのを私たちは見る。
したがって、自分がもっている善を、できる限り、他のものに伝達するというこ
 ともまた、意志の性格に属している。そして、このことはとくに神の意志に属
 する。すべての完全性は、神の意志から、なんらかの類似を通して、派生する
 からである。したがって、もし、自然的事物が、それが完全であるかぎりで、
 自分がもっている善を他のものに伝達するならば、なおいっそう、神には、自分がもっている善を、で
 きるかぎり、類似を通して他のものに伝達することが属する。
それゆえ、このようなかたちで、神は自分が存在することも、他のものが存在す
 ることも意志する。しかし、自分を目的として意志し、他のものは、それらも
 また神の善性を分有するのにふさわしい限りで、目的のためにあるものとして、
 意志する。




\\





{\scshape Ad primum ergo dicendum} quod, licet divinum velle sit eius
 esse secundum rem, tamen differt ratione, secundum diversum modum
 intelligendi et significandi, ut ex superioribus patet. In hoc enim
 quod dico Deum esse, non importatur habitudo ad aliquid, sicut in hoc
 quod dico Deum velle. Et ideo, licet non sit aliquid aliud a se, vult
 tamen aliquid aliud a se.


&

第一異論に対して、それゆえ、以下のように言われるべきである。
神の「意志すること」が、事物として、神の「存在すること」であるのはその通
 りだが、しかし、前に論じられた事柄から明らかなとおり、知性認識のしかた
 や表示のしかたが異なるのに応じて、それらは概念において(ratione)異なる。
 たとえば、「神が意志する」と私が言うとき、そこには何かへの関係が含意さ
 れているが、「神が存在する」と私が言うとき、その関係は含意されていない。
 ゆえに、神は自分自身以外の何ものでもないが、しかし、自分以外の何かを意
 志する。



\\





{\scshape Ad secundum dicendum} quod in his quae volumus propter finem,
 tota ratio movendi est finis, et hoc est quod movet voluntatem. Et hoc
 maxime apparet in his quae volumus tantum propter finem. Qui enim vult
 sumere potionem amaram, nihil in ea vult nisi sanitatem, et hoc solum
 est quod movet eius voluntatem. Secus autem est in eo qui sumit
 potionem dulcem, quam non solum propter sanitatem, sed etiam propter se
 aliquis velle potest. Unde, cum Deus alia a se non velit nisi propter
 finem qui est sua bonitas, ut dictum est, non sequitur quod aliquid
 aliud moveat voluntatem eius nisi bonitas sua. Et sic, sicut alia a se
 intelligit intelligendo essentiam suam, ita alia a se vult, volendo
 bonitatem suam.


&

第二異論に対しては、以下のように言われるべきである。
私たちが目的のために意志するものどもにおいては、運動の全根拠(ratio)は目的であり、
意志を動かすのは目的である。このことは、ただ目的のためだけに私たちが欲す
 る事柄において、最大限に明らかである。
たとえば、苦い飲み薬を飲むことを意志する者は、その薬において、健康しか意
 志していないのであり、彼の意志を動かしているのは、このことだけである。
これに対して、甘い飲み薬を飲むことを意志する者においては異なる。人は、
 健康だけでなく、その薬のために、それを意志するということもありうる。したがって、
すでに述べられたように、神は、自分の善性という目的のためにでなければ、自分以外のものを意志しな
 いので、神の善性以外の何かが神の意志を動かすということは帰結しない。こ
 のようにして、神は自分の本質を知性認識することによって自分以外のものを
 知性認識するように、自分の善性を意志することによって、自分以外のものを
 意志する。


\\





{\scshape Ad tertium dicendum} quod ex hoc quod voluntati divinae
 sufficit sua bonitas, non sequitur quod nihil aliud velit, sed quod
 nihil aliud vult nisi ratione suae bonitatis. Sicut etiam intellectus
 divinus, licet sit perfectus ex hoc ipso quod essentiam divinam
 cognoscit, tamen in ea cognoscit alia.


&

第三異論に対しては、以下のように言われるべきである。
神の意志にとって、自分の善性が十分であるということから、他のものを何も意
 志しないということは帰結せず、むしろ、自分の善性という根拠以外によっては
 他の何も意志しないということが帰結する。ちょうど、神の知性が、神の本質
 を認識するということ自体に基づいて、完全であるけれども、しかし、そこに
 おいて他のものを認識するように。


\\





{\scshape Ad quartum dicendum} quod, sicut intelligere divinum est unum,
 quia multa non videt nisi in uno; ita velle divinum est unum et
 simplex, quia multa non vult nisi per unum, quod est bonitas sua.


&



第四異論に対しては、以下のように言われるべきである。
神の「知性認識すること」が、一においてでなければ多を見ないがゆえに、一で
 あるように、神の「意志すること」も、自分の善性である一によってでなけれ
 ば多を意志しないために、一であり単純である。




\end{longtable}
\newpage





\rhead{a.~3}
\begin{center}
 {\Large {\bf ARTICULUS TERTIUS}}\\
 {\large UTRUM QUIDQUID DEUS VULT, EX NECESSITATE VELIT}\\
 {\footnotesize I {\itshape SCG.}, cap.~80 sqq.; III, cap.~98; {\itshape
 De Verit.}, q.~23, a.~4; {\itshape De Pot.}, q.~1, a.~5; q.~10, a.~2,
 ad 6.}\\
 {\Large 第三項\\なんであれ神が意志することは、それを必然に基づいて意志するか}
\end{center}

\begin{longtable}{p{21em}p{21em}}

{\huge A}{\scshape d tertium sic proceditur}. Videtur quod
quidquid Deus vult, ex necessitate velit. Omne enim aeternum est
necessarium. Sed quidquid Deus vult, ab aeterno vult, alias, voluntas
eius esset mutabilis. Ergo quidquid vult, ex necessitate vult.


&

 第三項の問題へ、議論は以下のように進められる。
なんであれ神が意志することを、必然に基づいて意志すると思われる。
 理由は以下の通り。すべて永遠のものは必然的である。ところが、なんであれ
 神が意志するものは、永遠からそれを意志する。さもなければ、神の意志は可
 変的だということになっただろう。ゆえに、なんであれ意志するものを、神は
 必然に基づいて意志する。
 
\\


{\scshape 2 Praeterea}, Deus vult alia a se, inquantum
vult bonitatem suam. Sed Deus bonitatem suam ex necessitate vult. Ergo
alia a se ex necessitate vult.


&

 さらに、神は自分以外のものを、自分の善性を意志する限りで、意志する。
 ところで、神は自分の善性を必然に基づいて意志する。ゆえに、自分以外のも
 のを、必然に基づいて意志する。
 

\\


{\scshape 3 Praeterea}, quidquid est Deo naturale, est
necessarium, quia Deus est per se necesse esse, et principium omnis
necessitatis, ut supra ostensum est. Sed naturale est ei velle quidquid
vult, quia in Deo nihil potest esse praeter naturam, ut dicitur in V
{\itshape Metaphys}. Ergo quidquid vult, ex necessitate vult.


&

さらに、神にとって本性的なものは何であれ、必然的なものである。なぜなら、
 前に示されたように、神は自体的に必然的なものであり、すべての必然性の根
 源だからである。ところで、『形而上学』第5巻で言われるように、神の中には
 本性以外のものはないから、なんであれ神が意志するものは、神にとって、意志
 することが本性的である。ゆえに、なんであれ意志するものを、必然に基づい
 て意志する。
 

\\


{\scshape 4 Praeterea}, non necesse esse, et possibile
non esse, aequipollent. Si igitur non necesse est Deum velle aliquid
eorum quae vult, possibile est eum non velle illud; et possibile est eum
velle illud quod non vult. Ergo voluntas divina est contingens ad
utrumlibet. Et sic imperfecta, quia omne contingens est imperfectum et
mutabile.


&
さらに、「$P$であることが必然的でない\footnote{$\lnot \Box P$}」と、
 「$P$でないことが可能\footnote{$\Diamond \lnot P$}」は
 同値である。
 ゆえに、もし、神が意志するものどものうちの何かを、神が意志することが必
 然でないならば、神がそれを意志しないことが可能である。また、〔同じ論拠に
 よって〕神が意志しないことを、意志することも可能である\footnote{前提す
 る公理は、$\lnot \Box P \equiv \Diamond \lnot P$。この$P$に、「神がaを
 意志する」を代入した場合が前者、「神がaを意志しない」を代入した場合が後
 者。}。ゆえに、神の意志は、どちらへも
 向かいうる非必然的なものである。このようにして、神の意志は不完全である
 ことになる。なぜなら、すべて非必然的なものは不完全で可変的だからである。
 \footnote{背理法。}

 
\\


{\scshape 5 Praeterea}, ab eo quod est ad utrumlibet,
non sequitur aliqua actio, nisi ab aliquo alio inclinetur ad unum, ut
dicit Commentator, in II {\itshape Physic}. Si ergo voluntas Dei in aliquibus se
habet ad utrumlibet, sequitur quod ab aliquo alio determinetur ad
effectum. Et sic habet aliquam causam priorem.


&

 さらに、注釈者が『自然学』第2巻の中で述べているように、どちらへも向かい
 うるということからは、一方へと傾かせる他の何かによらない限り、いかなる
 作用も帰結しない。ゆえに、もし神の意志が、何かにおいて、どちらへも向か
 いうるものであるならば、他の何かによって、結果へと限定されることになる。
 このようにして、神はより先のなんらかの原因をもつ。\footnote{同じく背理
 法の途中。}

\\



{\scshape 6 Praeterea}, quidquid Deus scit, ex
necessitate scit. Sed sicut scientia divina est eius essentia, ita
voluntas divina. Ergo quidquid Deus vult, ex necessitate vult.


&

さらに、なんであれ神が知るものは、必然に基づいて知る。ところで、神の知が
 神の本質であるように、神の意志もまたそうである。ゆえに、神が意志するも
 のは何であれ、必然に基づいて意志する。

 
\\


{\scshape  Sed contra est} quod dicit apostolus,
{\itshape Ephes}. I, {\itshape qui operatur omnia secundum consilium voluntatis suae}. Quod
autem operamur ex consilio voluntatis, non ex necessitate volumus. Non
ergo quidquid Deus vult, ex necessitate vult.


&

しかし反対に、使徒は『エフェソの信徒への手紙』1で、「自分の意志の考慮に
 したがって、すべてのことを行う方」\footnote{「御心のままにすべてのことを
 行われる方の御計画によって前もって定められ、」(1:11)}と述べている。
 ところで、意志の考慮に基づいて私たちが行うことを、私たちは必然に基づい
 て行うことはない。ゆえに、なんであれ神が意志するものを、神は必然に基づいて
 意志するわけではない。
 

\\


{\scshape Respondeo dicendum} quod necessarium dicitur
aliquid dupliciter, scilicet absolute, et ex suppositione. Necessarium
absolute iudicatur aliquid ex habitudine terminorum, utpote quia
praedicatum est in definitione subiecti, sicut necessarium est hominem
esse animal; vel quia subiectum est de ratione praedicati, sicut hoc est
necessarium, numerum esse parem vel imparem. Sic autem non est
necessarium Socratem sedere. Unde non est necessarium absolute, sed
potest dici necessarium ex suppositione, supposito enim quod sedeat,
 necesse est eum sedere dum sedet.


 &

 解答する。以下のように言われるべきである。
 何かが必然的であるというのは二通りの意味で、すなわち、無条件的にと、仮
 定に基づいて、という意味で語られる。あるものが無条件的に必然的であると
 判断されるのは、項の関係によってであり、たとえば、「人間は動物である」
 が必然的であるように、述語が主語の定義の中に在るようなときである。
 あるいは、「数は、偶数か奇数かのどちらかである」のように、主語が述語の
 概念に属する場合である。これに対して、この意味では「ソクラテスは座って
 いる」は必然的でない。したがって、それは無条件的に必然的ではないが、し
 かし、仮定に基づいて必然的であるとは言われうる。つまり、ソクラテスが座っ
 ているという仮定の下では、彼が座っている間、ソクラテスが座っていること
 は必然的である。

 \\
 

 Circa divina igitur volita hoc
considerandum est, quod aliquid Deum velle est necessarium absolute, non
tamen hoc est verum de omnibus quae vult. Voluntas enim divina
necessariam habitudinem habet ad bonitatem suam, quae est proprium eius
obiectum. Unde bonitatem suam esse Deus ex necessitate vult; sicut et
voluntas nostra ex necessitate vult beatitudinem. Sicut et quaelibet
alia potentia necessariam habitudinem habet ad proprium et principale
obiectum, ut visus ad colorem; quia de sui ratione est, ut in illud
 tendat.

 &

 それゆえ、神によって意志された事柄については、以下のことが考察されるべ
 きである。すなわち、神は、あるものを、無条件的に必然的に意志するが、し
 かし、これは神が意志するすべてのことについて当てはまるわけではない。理
 由は以下の通り。
神の意志は、自分の善性に対して必然的な関係をもつ。それはその固有対象だか
 らである。したがって、自分の善性を、神は必然性に基づいて意志する。それ
 はちょうど、私たちの意志もまた、必然性に基づいて至福を意志するのと同様
 であるし、また、他のどんな能力も、固有で主要な対象へと必然的な関係をも
 つようにである。たとえば、視覚は色に対してそのような関係をもつ。それへ
 と向かうことが、その本質規定に含まれているからである。


 \\


 Alia autem a se Deus vult, inquantum ordinantur ad suam
bonitatem ut in finem. Ea autem quae sunt ad finem, non ex necessitate
volumus volentes finem, nisi sint talia, sine quibus finis esse non
potest, sicut volumus cibum, volentes conservationem vitae; et navem,
volentes transfretare. Non sic autem ex necessitate volumus ea sine
quibus finis esse potest, sicut equum ad ambulandum, quia sine hoc
possumus ire; et eadem ratio est in aliis. Unde, cum bonitas Dei sit
perfecta, et esse possit sine aliis, cum nihil ei perfectionis ex aliis
accrescat; sequitur quod alia a se eum velle, non sit necessarium
absolute. Et tamen necessarium est ex suppositione, supposito enim quod
velit, non potest non velle, quia non potest voluntas eius mutari.

&

これに対して、神が自分以外のものを意志するのは、それらが、目的としての神
 の善性へと秩序づけられる限りでである。ところで、目的のためにあるものど
 もを、私たちが必然的に意志するのは、私たちが、命を保つことを意志して食
 物を意志したり、あるいは、海を渡ることを意志して船を意志する場合のよう
 に、それがなくては目的がありえない場合に限られる。他方、それがなくても
 目的があるようなものを、この意味で私たちが必然的に意志することはない。
 たとえば、旅をするのに馬を意志する場合のように、というのも、馬がなくて
 も行くことができるからであり、その他のことにおいても同じことである。
 したがって、神の善性は完全であり、他のものが無くてもありうるのだから、
 また、神にとって完全性に属する何かが、他のものによって増幅されるという
 ことはないので、神が自分以外のものを意志することは、無条件的に必然では
 ないことが帰結する。しかし、仮定に基づいてであれば、必然的である。つま
 り、神の意志は変化しえないので、神がそれを意志することが仮定されるなら
 ば、神がそれを意志しないことはありえない。
 

\\


{\scshape Ad primum ergo dicendum} quod ex hoc quod Deus
ab aeterno vult aliquid, non sequitur quod necesse est eum illud velle,
nisi ex suppositione.

&

 第一異論に対しては、それゆえ、以下のように言われるべきである。
 神が永遠に何かを意志しているということから、神がそれを必然的に意志して
 いるということは帰結しない。ただし、仮定に基づいて、という意味でない限
 り。

\\


{\scshape Ad secundum dicendum} quod, licet Deus ex
necessitate velit bonitatem suam, non tamen ex necessitate vult ea quae
vult propter bonitatem suam, quia bonitas eius potest esse sine aliis.

&

 第二異論に対しては、以下のように言われるべきである。
 神が自分の善性を必然的に意志するのはたしかだが、しかし、自分の善性のた
 めに意志するものどもを、必然的に意志するわけではない。なぜなら、他のも
 のどもがなくても、神の善性はありうるからである。
 

\\


{\scshape Ad tertium dicendum} quod non est naturale Deo
velle aliquid aliorum, quae non ex necessitate vult. Neque tamen
innaturale, aut contra naturam, sed est voluntarium.

&

 第三異論に対しては、以下のように言われるべきである。
 神が必然的に意志しない、神以外のもののうちの何かを意志することは、神にとって本性的な
 ことではない。しかし、それが非本性的であるとか、本性に反するとかという
 ことはなく、むしろ意志的な事柄である。
 

 

\\


{\scshape Ad quartum dicendum} quod aliquando aliqua
causa necessaria habet non necessariam habitudinem ad aliquem effectum,
quod est propter defectum effectus, et non propter defectum
causae. Sicut virtus solis habet non necessariam habitudinem ad aliquid
eorum quae contingenter hic eveniunt, non propter defectum virtutis
solaris, sed propter defectum effectus non necessario ex causa
provenientis. Et similiter, quod Deus non ex necessitate velit aliquid
eorum quae vult, non accidit ex defectu voluntatis divinae, sed ex
defectu qui competit volito secundum suam rationem, quia scilicet est
tale, ut sine eo esse possit perfecta bonitas Dei. Qui quidem defectus
consequitur omne bonum creatum.

&

 第四異論に対しては、以下のように言われるべきである。
 必然的な原因が、結果に対して必然的でない関係をもつとき、それが原因の欠
 陥のせいでなく、結果の欠陥のためであるということがある。たとえば、太陽
 の力が、ここ[=地上]で非必然的に生じる事柄の何かに対して、必然的でない関係をもつ
 のは、太陽の力の欠陥のせいではなく、必然的に原因から出て来ない結果の欠
 陥のためである。同様に、神が、意志する事柄の何かを必然的に意志しないこ
 とは、神の意志の欠陥から生じるのではなく、意志されたものに、それ自らの
 性格にしたがって属する欠陥から生じる。なぜなら、意志されたものは、それがなくても神の
 完全な善性がありうるようなものだからである。実際、この欠陥は、すべての
 被造の善に伴っている。

 

\\


{\scshape Ad quintum ergo dicendum} quod causa quae est
ex se contingens, oportet quod determinetur ab aliquo exteriori ad
effectum. Sed voluntas divina, quae ex se necessitatem habet, determinat
seipsam ad volitum, ad quod habet habitudinem non necessariam.

&

 第五異論に対しては、以下のように言われるべきである。
 もともと非必然的である原因は、外的な何かによって、結果へと限定されなけ
 ればならない。しかし神の意志は、もともと必然性をもつから、自分自身を、意
 志されるものへと限定し、そしてそれへ、必然的でない関係をもつ。
 

\\


{\scshape Ad sextum dicendum} quod, sicut divinum esse
in se est necessarium, ita et divinum velle et divinum scire, sed
divinum scire habet necessariam habitudinem ad scita, non autem divinum
velle ad volita. Quod ideo est, quia scientia habetur de rebus, secundum
quod sunt in sciente, voluntas autem comparatur ad res, secundum quod
sunt in seipsis. Quia igitur omnia alia habent necessarium esse secundum
quod sunt in Deo; non autem secundum quod sunt in seipsis, habent
necessitatem absolutam ita quod sint per seipsa necessaria; propter hoc
Deus quaecumque scit, ex necessitate scit, non autem quaecumque vult, ex
necessitate vult.

&

 第六異論に対しては、以下のように言われるべきである。
 神の存在がそれ自体において必然的であるように、神の「意志すること」や
「『知ること」もまた必然的である。しかし、神の「知ること」が知られたもの
 へ必然的な関係をもつのに対して、神の「意志すること」は意志されたものへ
 必然的な関係をもたない。どうしてそうなるかと言えば、事物について知が持
 たれるのは、事物が知る者のうちにある限りにおいてであるのに対し、意志は
 が事物に関係するのは、事物がその事物そのものであるかぎりにおいてだからである。ゆえに、他のすべての
 ものが、神の中にあるかぎりで、必然的な存在を持つのに対し、事物そのもの
 である限りにおいては、それ自体によって必然的であるというような、無条件
 的な必然性を持たない。このため、神は、なんであれ知るものを必然的に知る
 が、しかし、意志するものをすべて必然的に意志するわけではない。



\end{longtable}
\newpage




\rhead{a.~4}
\begin{center}
 {\Large {\bf ARTICULUS QUARTUS}}\\
 {\large UTRUM VOLUNTAS DEI SIT CAUSA RERUM}\\
 {\footnotesize I {\itshape Sent.}, d.~43, q.~2, a.~1; d.~45, a.~3; II
 {\itshape SCG.}, cap.~23; {\itshape De Pot.}, q.~1, a.~5; q.~3, a.~15.}\\
 {\Large 第四項\\神の意志は事物の原因か}
\end{center}

\begin{longtable}{p{21em}p{21em}}


{\scshape {\huge A}d quartum sic proceditur}. Videtur quod
 voluntas Dei non sit causa rerum. Dicit enim Dionysius, cap.~{\scshape iv} {\itshape de
 Div.~Nom}., {\itshape sicut noster sol, non ratiocinans aut praeeligens, sed per
 ipsum esse illuminat omnia participare lumen ipsius valentia; ita et
 bonum divinum per ipsam essentiam omnibus existentibus immittit
 bonitatis suae radios}. Sed omne quod agit per voluntatem, agit ut
 ratiocinans et praeeligens. Ergo Deus non agit per voluntatem. Ergo
 voluntas Dei non est causa rerum.

 
&

 第四項の問題へ、議論は以下のように進められる。
 神の意志は事物の原因でないと思われる。理由は以下の通り。
 ディオニュシウスは『神名論』第4章で以下のように述べている。
 「ちょうど私たちの太陽が推論や選択をすることなく、存在そのものを通して、
 その光を分有しうるすべてのものを照らすように、神の善もまた、その本質自
 体によって、すべての存在するものどもに、自分の善性の光線を放出する」。
 ところで、すべて意志によって働くものは、推論し、選択するものとして働く。
 ゆえに、神は意志によって働くのではない。ゆえに、神の意志は事物の原因で
 ない。\footnote{Camestres\begin{enumerate}
		  \item すべて意志によって働くものは、推論し、選択する。(Apm)
		  \item (すべての)神は推論し、選択しない。(Esm)
		  \item ゆえに、(すべての)神は意志によって働かない。(Esp)
		 \end{enumerate}}
 


\\



{\scshape 2 Praeterea}, id quod est per essentiam, est
 primum in quolibet ordine sicut in ordine ignitorum est primum, quod
 est ignis per essentiam.\footnote{Busa版のpunctuationをLeo版に倣って変
 更。} Sed Deus est primum agens. Ergo est agens per
 essentiam suam, quae est natura eius. Agit igitur per naturam, et non
 per voluntatem. Voluntas igitur divina non est causa rerum.

 
&

さらに、どの秩序においても、本質によって何かであるものは、第一のものであ
 る。たとえば、火がついたものの秩序の中で第一のものは、本質によって火で
 あるものであるように。ところで、神は第一の作用者である。ゆえに、神は、
 自分の本質による作用者だが、その本質は神の本性である。ゆえに、神は本性
 によって働くのであり、意志によって働くのではない。ゆえに、神の意志は、
 事物の原因でない。
%\footnote{\begin{enumerate}
%			      \item Xの秩序で第一にものはすべて、本質に
%				    よってXである。
%			      \item 神は作用者の秩序で第一のものである。
%			      \item ゆえに、神は本質によって作用者である。
%			      \item 神の本質は神の本性である。
%			      \item ゆえに、神は本性による作用者である。
%			      \item すべて本性による作用者は意志による作
%				    用者でない。
%			      \item ゆえに、神は意志による作用者でない。
%			     \end{enumerate}}
 
\\



{\scshape 3 Praeterea}, quidquid est causa alicuius per
 hoc quod est tale, est causa per naturam, et non per voluntatem, ignis
 enim causa est calefactionis, quia est calidus; sed artifex est causa
 domus, quia vult eam facere. Sed Augustinus dicit, in I de
 {\itshape Doct.~Christ}., quod {\itshape quia Deus bonus est, sumus}. Ergo Deus per suam
 naturam est causa rerum, et non per voluntatem.

 
&

 さらに、「これこれである」ということによって、何かの原因であるものはす
 べて、意志によってではなく本性によって原因である。たとえば、火は「熱い
 ものである」ということによって、熱することの原因だが、建築家は、[建築
 家であるということによってではなく]家を作
 ることを意志するから、家の原因である。ところで、アウグスティヌスは『キ
 リスト教の教え』第1巻で、「神は善であるから、私たちがいる」と述べている。
 ゆえに、神は、自分の本性によって事物の原因であり、意志によってではない。

 

\\



{\scshape 4 Praeterea}, unius rei una est causa. Sed
 rerum creatarum est causa scientia Dei, ut supra dictum est. Ergo
 voluntas Dei non debet poni causa rerum.

 
&

さらに、一つの事物には一つの原因がある。ところが、前に述べられたとおり、
 被造の諸事物の原因は、神の知である。ゆえに、神の意志が、諸事物の原因と
 して措定される必要はない。

 
\\



{\scshape  Sed contra est} quod dicitur {\itshape Sap}.~{\scshape xi}, {\itshape quomodo
 posset aliquid permanere, nisi tu voluisses?}

 
&

しかし反対に、『知恵の書』11「もしあなたが意志しなかったならば、どのよう
 にして何かが存続できたでしょうか」と言われている。
 

\\



{\scshape Respondeo dicendum} quod necesse est dicere
 voluntatem Dei esse causam rerum, et Deum agere per voluntatem, non per
 necessitatem naturae, ut quidam existimaverunt. Quod quidem apparere
 potest tripliciter. Primo quidem, ex ipso ordine causarum agentium. Cum
 enim propter finem agat et intellectus et natura, ut probatur in II
{\itshape  Physic}., necesse est ut agenti per naturam praedeterminetur finis, et
 media necessaria ad finem, ab aliquo superiori intellectu; sicut
 sagittae praedeterminatur finis et certus modus a sagittante. Unde
 necesse est quod agens per intellectum et voluntatem, sit prius agente
 per naturam. Unde, cum primum in ordine agentium sit Deus, necesse est
 quod per intellectum et voluntatem agat.

 &

 解答する。以下のように言われるべきである。
 神の意志は諸事物の原因であると語ること、また、ある人々が考えたように、
 神が本性の必然性によって働くのではなく、神は意志によって働くと語ること
 が必要である。
このことは、三つの仕方で明らかでありうる。第一には、作用因の秩序そのものからである。
 『自然学』第2巻で証明されているように、知性も本性も、目的のために働くので、
 ちょうど、矢が、矢を射る人によって、その的と、特定の限度があらかじめ定
 められているように、本性によって働くものには、なんらかの上位の知
 性によって、目的と、その目的に必要な媒介が、前もって決められている必要
 がある。したがって、知性と意志によって働くものは、自然本性によって働く
 ものに先行する。したがって、働くものどもの秩序の中で、神が第一のものだ
 から、神が知性と意志によって働くことは必然である。


 
 \\

 Secundo, ex ratione naturalis
 agentis, ad quod pertinet ut unum effectum producat, quia natura uno et
 eodem modo operatur, nisi impediatur. Et hoc ideo, quia secundum quod
 est tale, agit, unde, quandiu est tale, non facit nisi tale. Omne enim
 agens per naturam, habet esse determinatum. Cum igitur esse divinum non
 sit determinatum, sed contineat in se totam perfectionem essendi, non
 potest esse quod agat per necessitatem naturae, nisi forte causaret
 aliquid indeterminatum et infinitum in essendo; quod est impossibile,
 ut ex superioribus patet. Non igitur agit per necessitatem naturae sed
 effectus determinati ab infinita ipsius perfectione procedunt secundum
 determinationem voluntatis et intellectus ipsius.

&

第二に、自然本性的に働くものの性格に基づいて、そのこと[=自然本性的な働
 き]には、一つの結果を生み出すことが属する。なぜなら、自然は、妨害され
 ない限り、同じ一つのしかたで働くからである。そしてこれはなぜかというと、
 ものは、かくかくであるということに応じて働くから、かくかくである間、そ
 のようにしか行わないからである。というのも、すべて自然本性によって働く
 ものは、限定された存在をもつからである。それゆえ、神の存在は限定されず、
 その中に、存在の全完全性を含むから、神が本性の必然性によって働くことは不可
 能である。ただし、神が、存在において無規定で無限のものの原因となる、と
 いうのでないならば。しかし、これが不可能であることは、前に述べられたこ
 と\footnote{{\itshape ST} I, q.~8, a.~2.}から明らかである。
 ゆえに、神は、本性の必然性によって働くのではなく、限定された諸結果が、神
 の無限の完全性から、神の意志と知性の限定にしたがって出てくる。

 \\
 
 Tertio, ex habitudine
 effectuum ad causam. Secundum hoc enim effectus procedunt a causa
 agente, secundum quod praeexistunt in ea, quia omne agens agit sibi
 simile. Praeexistunt autem effectus in causa secundum modum
 causae. Unde, cum esse divinum sit ipsum eius intelligere, praeexistunt
 in eo effectus eius secundum modum intelligibilem. Unde et per modum
 intelligibilem procedunt ab eo. Et sic, per consequens, per modum
 voluntatis, nam inclinatio eius ad agendum quod intellectu conceptum
 est, pertinet ad voluntatem. Voluntas igitur Dei est causa rerum.

 
&

第三に、結果の原因にたいする関係に基づく。結果が作用因から出てくるのは、
 その原因の中に先在していることに応じてである。働くものは、働くものに似
 たものを生み出すからである。ところで、結果は、原因のあり方に応じて、原
 因の中に先在する。したがって、神の存在は神の知性認識そのものなので、神
 の結果は、神の中に、可知的なあり方で先在する。したがって、可知的なあり
 方で、神から出ていく。この意味で、帰結として、それらは意志のあり方によっ
 て出ていく。つまり、知性によって捉えられたことを行うことへ向かう神の傾向性
 は、意志に属するからである。ゆえに、神の意志は、諸事物の原因である。


\\



{\scshape Ad primum ergo dicendum} quod Dionysius per
 verba illa non intendit excludere electionem a Deo simpliciter, sed
 secundum quid, inquantum scilicet, non quibusdam solum bonitatem suam
 communicat, sed omnibus, prout scilicet electio discretionem quandam
 importat.

 
&

第一異論に対しては、それゆえ、以下のように言われるべきである。
ディオニュシウスは、かの言葉によって、神から端的に選択を排除することを意
 図しているのではない。そうではなく、ある意味において、つまり「選択」はなんら
 かの「選別」を含意する限りで、選択を排除すること意図している。神は自らの善性を、あるもの
 どもにだけではなく、すべてのものに、伝達するのだから。


\\



{\scshape Ad secundum dicendum} quod, quia essentia Dei
 est eius intelligere et velle, ex hoc ipso quod per essentiam suam
 agit, sequitur quod agat per modum intellectus et voluntatis.

 
&

第二異論に対しては、以下のように言われるべきである。
神の本質は、神の知性認識であり意志であるから、自らの本質によって働くとい
 うこと自体によって、知性と意志のあり方によって働くということが帰結する。


\\



{\scshape Ad tertium dicendum} quod bonum est obiectum
 voluntatis. Pro tanto ergo dicitur, quia Deus bonus est, sumus,
 inquantum sua bonitas est ei ratio volendi omnia alia, ut supra dictum
 est.

 
&

第三異論に対しては、以下のように言われるべきである。
善は意志の対象である。ゆえに、前に述べられたとおり、神の善性が、神が他のすべ
 てのものを意志する神にとっての根拠であるという、その意味で、神が善であ
 るから私たちが存在する、と言われる。\footnote{神は善であるから、神の意
 志は神自身を意志し、その結果、私たちを含むすべての被造物を意志する。}


\\



{\scshape Ad quartum dicendum} quod unius et eiusdem
 effectus, etiam in nobis, est causa scientia ut dirigens, qua
 concipitur forma operis, et voluntas ut imperans, quia forma, ut est in
 intellectu tantum, non determinatur ad hoc quod sit vel non sit in
 effectu, nisi per voluntatem. Unde intellectus speculativus nihil dicit
 de operando. Sed potentia est causa ut exequens, quia nominat
 immediatum principium operationis. Sed haec omnia in Deo unum sunt.
 
&

第四異論に対しては、以下のように言われるべきである。
私たちにおいても、一つの同一の結果について、知が企画者として原因であり、
 それによって作品の形相がとらえられ、そして意志は命令者としての原因であ
 る。というのも、形相は、それが知性の中だけにあるものとしては、意志によ
 らないかぎり、結果の中に存在するかしないかが決定されないからである。
観照的な知性が、働きについては何も言わないことも、このことから出てくる。
これに対して、能力は、働きの直接的な根源の名前なので、遂行者としての原因
 である。しかし神において、これらすべては、一つである。


\end{longtable}
\newpage



\rhead{a.~5}
\begin{center}
 {\Large {\bf ARTIUCLUS QUINTUS}}\\
 {\large UTRUM VOLUNTATIS DIVINAE SIT ASSIGNARE ALIQUAM CAUSAM}\\
 {\footnotesize I {\itshape Sent.}, d.~41, a.~3; I {\itshape SCG},
 cap.~86, 87; III, cap.~97; {\itshape De Verit.}, q.~6, a.~2; q.~23,
 a.~1, ad 3; a.~6, ad 6; {\itshape Ephes.}, cap.~1, lect.~1.}\\
 {\Large 第五項\\なんらかの原因を指定することが、神の意志に属する
 か}\footnote{回りくどい表現だが、「神の意志には原因があるか」ということ。}
\end{center}

\begin{longtable}{p{21em}p{21em}}

{\huge A}{\scshape d quintum sic proceditur}. Videtur quod
voluntatis divinae sit assignare aliquam causam. Dicit enim Augustinus,
libro {\itshape Octoginta trium Quaest}., {\itshape Quis audeat dicere Deum irrationabiliter
omnia condidisse?} Sed agenti voluntario, quod est ratio operandi, est
etiam causa volendi. Ergo voluntas Dei habet aliquam causam.


&

第五の問題へ、議論は以下のように進められる。
神の意志に、なんらかの原因を指定することが属すると思われる。理由は以下
 の通り。
アウグスティヌスは、『八十三問題集』の中で、「神は合理的な理由なく万物を
 作ったと、誰があえて言うだろうか」と述べている。ところで、意志的作用者
 にとって、働く理由は、意志することの原因でもある。ゆえに、神の意志は、
 なんらかの原因をもつ。


\\


{\scshape 2 Praeterea}, in his quae fiunt a volente qui
propter nullam causam aliquid vult, non oportet aliam causam assignare
nisi voluntatem volentis. Sed voluntas Dei est causa omnium rerum, ut
ostensum est. Si igitur voluntatis eius non sit aliqua causa, non
oportebit in omnibus rebus naturalibus aliam causam quaerere, nisi solam
voluntatem divinam. Et sic omnes scientiae essent supervacuae, quae
causas aliquorum effectuum assignare nituntur, quod videtur
inconveniens. Est igitur assignare aliquam causam voluntatis divinae.


&

さらに、何の原因のためにでもなく何かを意志するような、そういう意志者によっ
 て為される事柄においては、意志する者の意志以外の、どんな原因も指定する
 べきでない。ところで、神の意志は、すでに示されたとおり、すべての事物の
 原因である。ゆえに、もし神の意志に何の原因もないならば、すべての自然的諸事物において、
神の意志以外の原因を探求すべきでないことになるだろう。そしてまた、諸結果
 の原因を指定することに努めるすべての学問は空しいことになったであろうが、
 これは不都合である。ゆえに、神の意志に、なんらかの原因を指定することが
 属する。

\\


{\scshape 3 Praeterea}, quod fit a volente non propter
aliquam causam, dependet ex simplici voluntate eius. Si igitur voluntas
Dei non habeat aliquam causam, sequitur quod omnia quae fiunt,
dependeant ex simplici eius voluntate, et non habeant aliquam aliam
causam. Quod est inconveniens.


&

さらに、何の原因のためにでもなく意志するものによって為されることは、その
者の単純な意志に依存する。ゆえに、もし神の意志が、何らの原因ももたないな
 らば、生じるものはすべて、神の単純な意志に依存するのであって、他のどん
 な原因ももたないであろう。これは不都合である。


\\


{\scshape  Sed contra est} quod dicit Augustinus, in
libro {\itshape Octoginta trium Quaest}., {\itshape omnis causa efficiens maior est eo quod
efficitur; nihil tamen maius est voluntate Dei; non ergo causa eius
quaerenda est}.


&

しかし反対に、アウグスティヌスは『八十三問題集』の中で以下のように述べて
 いる。「すべての作出因は、作り出されたものよりも偉大である。しかるに、
 神の意志より偉大なものはない。ゆえに、それの原因が求められるべきでない」。



\\


{\scshape Respondeo dicendum} quod nullo modo voluntas
Dei causam habet. Ad cuius evidentiam, considerandum est quod, cum
voluntas sequatur intellectum, eodem modo contingit esse causam alicuius
volentis ut velit, et alicuius intelligentis ut intelligat. In
intellectu autem sic est quod, si seorsum intelligat principium, et
seorsum conclusionem, intelligentia principii est causa scientiae
conclusionis. Sed si intellectus in ipso principio inspiceret
conclusionem, uno intuitu apprehendens utrumque, in eo scientia
conclusionis non causaretur ab intellectu principiorum, quia idem non
est causa sui ipsius. Sed tamen intelligeret principia esse causas
conclusionis. 


&

解答する。以下のように言われるべきである。
どんなかたちによっても、神の意志は原因をもたない。これを明らかにするため
 に、以下のことが考察されるべきである。すなわち、意志は知性に伴うので、
 意志するかぎりでの意志する者の原因であるということは、知性認識するかぎ
 りでの、知性認識する者の原因であるということと、同じしかたで生じる。
さて、知性においては以下の通りである。原理と結論を別々に知性認識す
 る場合、原理についての直知は、結論についての知の原因である。これに対し
 て、知性が、原理そのものの中に結論を看取する、つまり、一つの直観によっ
 て両者を捉える場合、そこでは、結論についての知が、原理についての直知に
 よって原因されることはない。なぜなら、同一のものが、自分自身の原因では
 ないからである。しかしながら、[その場合でも]諸原理が結論の原因であることは、知性認識
 するであろう。

\\


Similiter est ex parte voluntatis, circa quam sic se habet
finis ad ea quae sunt ad finem, sicut in intellectu principia ad
conclusiones. Unde, si aliquis uno actu velit finem, et alio actu ea
quae sunt ad finem, velle finem erit ei causa volendi ea quae sunt ad
finem. Sed si uno actu velit finem et ea quae sunt ad finem, hoc esse
non poterit, quia idem non est causa sui ipsius. Et tamen erit verum
dicere quod velit ordinare ea quae sunt ad finem, in finem. 

&

意志の側からも同様である。意志については、目的が、目的のためにあるものど
 も[=手段]に対する関係は、知性において、原理が結論に対する関係に等しい。したがっ
 て、ある人が、一つの作用によって、目的を意志し、別の作用によって、目的
 のためにあるものどもを意志する場合、目的を意志することは、目的のために
 あるものどもを意志することの原因であるだろう。これに対して、一つの作用
 によって、目的と、目的のためにあるものどもを意志する場合、このようなこ
 とはありえない。なぜなら、同一のものが、自分自身の原因ではないからであ
 る。しかし、目的のためにあるものどもを、目的へと秩序づけることを意志す
 る、と言うことは正しいであろう。

\\


Deus autem,
sicut uno actu omnia in essentia sua intelligit, ita uno actu vult omnia
in sua bonitate. Unde, sicut in Deo intelligere causam non est causa
intelligendi effectus, sed ipse intelligit effectus in causa; ita velle
finem non est ei causa volendi ea quae sunt ad finem, sed tamen vult ea
quae sunt ad finem, ordinari in finem. Vult ergo hoc esse propter hoc,
sed non propter hoc vult hoc.


&

さて、神は、ちょうど一つの作用によって万物を自らの本質において知性認識す
 るように、一つの作用によって、万物を自らの善性において意志する。したがっ
 て、ちょうど、神において、原因を知性認識することが、結果を知性認識する
 ことの原因でなく、神は諸結果を原因において知性認識するように、神にとっ
 て、目的を意志することは、目的のためにあるものどもを意志することの原因
 ではない。しかし、目的のためにあるものどもを、目的へと秩序づける
 ことを意志する。ゆえに、神は「甲のために乙があること」を意志するが、甲
 のために乙を意志するわけではない。

\\


{\scshape Ad primum ergo dicendum} quod voluntas Dei
rationabilis est, non quod aliquid sit Deo causa volendi, sed inquantum
vult unum esse propter aliud.


&

第一異論に対しては、それゆえ、以下のように言われるべきである。
神の意志が合理的根拠をもつのは、何かが、神が意志することの原因だ
 からではなく、神が、或るものが他のもののためにあることを意志する限りに
 おいてである。


\\


{\scshape Ad secundum dicendum} quod, cum velit Deus
effectus sic esse, ut ex causis certis proveniant, ad hoc quod servetur
ordo in rebus; non est supervacuum, etiam cum voluntate Dei, alias
causas quaerere. Esset tamen supervacuum, si aliae causae quaererentur
ut primae, et non dependentes a divina voluntate. Et sic loquitur
Augustinus in III {\itshape de Trin}., {\itshape placuit vanitati philosophorum etiam aliis
causis effectus contingentes tribuere, cum omnino videre non possent
superiorem ceteris omnibus causam, idest voluntatem Dei}.


&

第二異論に対しては、以下のように言われるべきである。
神は、諸結果が特定の諸原因から到来することを、諸事物における秩序が保たれ
 るために意志するのだから、神の意志があるとしても、その他の諸原因を探究
 することは空虚なことではない。しかしもし、その他の諸原因が、第一の原因
 として、神の意志に依存しないものとして求められたならば、それは空しいこ
 とだったであろう。そして、アウグスティヌスが『三位一体論』第3巻で、以下
 のように語るのはこのような意味でである。「他の諸原因にも偶然的な諸結果
 を帰することが、哲学者たちの虚栄心を喜ばせた。なぜなら、他のすべてのも
 のにまさる原因、すなわち神の意志を、彼らはまったく見ることができなかっ
 たからである」。

\\


{\scshape Ad tertium dicendum} quod, cum Deus velit
effectus esse propter causas, quicumque effectus praesupponunt aliquem
alium effectum, non dependent ex sola Dei voluntate, sed ex aliquo
alio. Sed primi effectus ex sola divina voluntate dependent. Utpote si
dicamus quod Deus voluit hominem habere manus, ut deservirent
intellectui, operando diversa opera, et voluit eum habere intellectum,
ad hoc quod esset homo, et voluit eum esse hominem, ut frueretur ipso,
vel ad complementum universi. Quae quidem non est reducere ad alios
fines creatos ulteriores. Unde huiusmodi dependent ex simplici voluntate
Dei, alia vero ex ordine etiam aliarum causarum.


&

第三異論に対しては、以下のように言われるべきである。
神は、諸結果が諸原因のためにあることを意志するので、どんな結果であれ、他
 のなんらかの結果を前提し、ただ神の意志だけに依存するのではなく、他の何
 かに依存する。しかし、第一の諸結果は、ただ神の意志だけに依存する。
たとえば、私たちが次のように言うとする。「神が、人間が手を持つことを意志
 したのは、それがさまざまな働きをすることで知性を助けるためである」、
 「神が、人間が知性を持つことを意志したのは、人間が人間であるためである」、
 「神が、人間が人間であることを意志したのは、人間が神を享受するために、あるいは、宇宙の完成
 のためにである」。[最後の]これは、さら
 に別の被造の諸目的へ帰することができない。したがって、このようなことは、
 神の単純な意志に依存する。他方、このようではない他の事柄は、他の諸原因
 の秩序にも依存する。


\end{longtable}
\newpage


\rhead{a.~6}
\begin{center}
 {\Large {\bf ARTICULUS SEXTUS}}\\
 {\large UTRUM VOLUNTAS DEI SEMPER IMPLEATUR}\\
 {\footnotesize I {\itshape Sent.}, d.~46, a.~1; d.~47, a.~1, 3;
 {\itshape De Verit.}, q.~23, a.~2; I {\itshape Tim}, cap.~2, lect.~1.}\\
 {\Large 第六項\\神の意志は常に満たされるか}
\end{center}

\begin{longtable}{p{21em}p{21em}}

{\huge A}{\scshape d sextum sic proceditur}. Videtur quod
voluntas Dei non semper impleatur. Dicit enim apostolus, I {\itshape ad
 Tim}.~{\scshape ii}, quod Deus {\itshape vult omnes homines salvos fieri, et ad agnitionem veritatis
venire}. Sed hoc non ita evenit. Ergo voluntas Dei non semper impletur.

&

第六項の問題へ、議論は以下のように進められる。
神の意志は常に満たされるとは限らないと思われる。理由は以下の通り。
使徒は『テモテへの手紙一』2章で、神は「すべての人間が救われた者と
 なること、そして真理の認識へと来ることを欲している」\footnote{「神は、
 すべての人々が救われて真理を知るようになることを望んでおられます。」(2:4)}と述べている。しか
 し、これは、その通りにはなっていない。ゆえに、神の意志が常に満たされる
 とは限らない。

\\


{\scshape 2 Praeterea}, sicut se habet scientia ad
verum, ita voluntas ad bonum. Sed Deus scit omne verum. Ergo vult omne
bonum. Sed non omne bonum fit, multa enim bona possunt fieri, quae non
fiunt. Non ergo voluntas Dei semper impletur.

&

さらに、意志は善に対して、知が真に対するのと同様に関係する。
ところで、神はすべての真を知る。ゆえに、神はすべての善を意志する。しかし、
 すべての善が生じるわけではない。つまり、生じうる多くの善が、実際には生
 じない。ゆえに、神の意志が常に満たされるわけではない。


\\


{\scshape 3 Praeterea}, voluntas Dei, cum sit causa
prima, non excludit causas medias, ut dictum est. Sed effectus causae
primae potest impediri per defectum causae secundae, sicut effectus
virtutis motivae impeditur propter debilitatem tibiae. Ergo et effectus
divinae voluntatis potest impediri propter defectum secundarum
causarum. Non ergo voluntas Dei semper impletur.


&

さらに、神の意志は、第一原因であるから、すでに述べられたとおり、中間の諸
 原因を排除しない。ところで、第一原因の結果は、第二原因の欠陥によって阻
 害されうる。たとえば、動く力の結果は、脛の弱さのために妨げられる。
ゆえに、神の意志の結果も、第二諸原因の欠陥のために妨げられうる。ゆえに、
 神の意志が常に満たされるとは限らない。


\\


{\scshape  Sed contra est quod} dicitur in Psalmo {\scshape cxiii},
{\itshape omnia quaecumque voluit Deus, fecit}.


&

しかし反対に、『詩編』113で「神は、何であれ意志したすべてのことを行った」
 \footnote{「御旨のままにすべてを行われる。」(115:3) Leo版の注では11節となっ
 ている。新共同訳だと、114編の最後が8節なので、115編の3節目は通算11節目
 となる。}
 と言われている。

\\


{\scshape Respondeo dicendum} quod necesse est voluntatem
Dei semper impleri. Ad cuius evidentiam, considerandum est quod, cum
effectus conformetur agenti secundum suam formam, eadem ratio est in
causis agentibus, quae est in causis formalibus. In formis autem sic est
quod, licet aliquid possit deficere ab aliqua forma particulari, tamen a
forma universali nihil deficere potest, potest enim esse aliquid quod
non est homo vel vivum, non autem potest esse aliquid quod non sit
ens. Unde et hoc idem in causis agentibus contingere oportet. 


&

解答する。以下のように言われるべきである。
神の意志が常に満たされるのは必然である。
これを明らかにするためには、以下のことが考察されるべきである。
結果は、その形相に即して、作用者に一致するので、形相的諸原因の中にある同
 一の性格が、作用的諸原因の中にもある。
ところで、形相においては次のようになっている。すなわち、なんらかの個別的
 な形相を欠くことはありえても、普遍的な形相を欠くことはありえない。たと
 えば、何かが、人間でないとか、生きるものでないということはありえても、
 「存在するもの」でないことはありえないように。したがって、この同じこと
 が、作用的諸原因においても生じなければならない。


\\



Potest
enim aliquid fieri extra ordinem alicuius causae particularis agentis,
non autem extra ordinem alicuius causae universalis, sub qua omnes
causae particulares comprehenduntur. Quia, si aliqua causa particularis
deficiat a suo effectu, hoc est propter aliquam aliam causam
particularem impedientem, quae continetur sub ordine causae universalis,
unde effectus ordinem causae universalis nullo modo potest exire. Et hoc
etiam patet in corporalibus. 



&

つまり、何かが、ある個別的な作用者の秩序の外に生じることはありえるが、し
 かし、それのもとにすべての個別的原因が包含されるような、なんらかの普遍
 的作用因の秩序の外に生じることはありえない。
なぜなら、もし、なんらかの個別的原因が、自らの結果を生み出さないとすれば、
 それは、それを阻害する他の個別的原因のためであるが、その阻害する原因も
 普遍的原因の秩序に含まれているから、結果は、決して、普遍的原因の秩序の
 外に出ることができないからである。



\\


Potest enim impediri quod aliqua stella non
inducat suum effectum, sed tamen quicumque effectus ex causa corporea
impediente in rebus corporalibus consequatur, oportet quod reducatur per
aliquas causas medias in universalem virtutem primi caeli. Cum igitur
voluntas Dei sit universalis causa omnium rerum, impossibile est quod
divina voluntas suum effectum non consequatur. 


&

たとえば、なんらかの星が、自らの結果を引き起こさないように妨げられること
 はありうるが、しかし、妨げる物体的原因に基づいて、どのような結果が物体的諸事物
 の中に結果として生じるとしても、それらは、なんらかの媒介となる諸原因を通して、
 第一天の普遍的な力へと還元されなければならないように。
それゆえ、神の意志は、すべての事物の普遍的な原因だから、神の意志が、自ら
 の結果を伴わないことはありえない。


\\


Unde quod recedere
videtur a divina voluntate secundum unum ordinem, relabitur in ipsam
secundum alium, sicut peccator, qui, quantum est in se, recedit a divina
voluntate peccando, incidit in ordinem divinae voluntatis, dum per eius
iustitiam punitur.


&

したがって、ある秩序においては神の意志から退くように見えるものも、他の秩
 序においては、そこへ戻る。たとえば、罪人は、罪を犯すことによって、それ
 自体においては神の意志から退くが、神の正義によって罰せられるかぎり、神
 の意志の秩序に入るように。


\\


{\scshape Ad primum ergo dicendum} quod illud verbum
apostoli, quod Deus {\itshape vult omnes homines salvos fieri} etc., potest
tripliciter intelligi. Uno modo, ut sit accommoda distributio, secundum
hunc sensum, {\itshape Deus vult salvos fieri omnes homines qui salvantur}, non
quia ``nullus homo sit quem salvum fieri non velit, sed quia nullus salvus
fit, quem non velit salvum fieri,'' ut dicit Augustinus. 







&

第一異論に対しては、それゆえ、次のように言われるべきである。
神は「すべての人間が救われた者になることを意志する」云々、という使徒の言
 葉は、三通りに理解されうる。
一つには、制限された周延になるようにであり、この意味では、「神は、救われ
 るすべての人々が救われた者となることを意志する」となる。それは、アウグ
 スティヌスが言うように、「救われた者となることを神が意志しない人がいな
 いからではなく、救われた者となることを意志しない人はだれも救われた者とな
 らない」からである。


\\

Secundo potest
intelligi, ut fiat distributio pro generibus singulorum, et non pro
singulis generum, secundum hunc sensum, {\itshape Deus vult de quolibet statu
hominum salvos fieri, mares et feminas, Iudaeos et gentiles, parvos et
magnos; non tamen omnes de singulis statibus}. 

&


第二の意味では、類に含まれる個々のものとしてではなく、個々のものの類とし
 ての周延となるようにであり、この意味では、「神は、男であれ女であれ、ユ
 ダヤ人であれ外国人であれ、小さいものも大きいものも、人間のどの状態につ
 いても救われたものとなることを意志するが、個々の状態にあるすべての人が
 救われた者となることを意志するわけではない」となる。


\\



Tertio, secundum
Damascenum, intelligitur de voluntate {\itshape antecedente}, non de voluntate
{\itshape consequente}. 
Quae quidem distinctio non accipitur ex parte ipsius
voluntatis divinae, in qua nihil est prius vel posterius; sed ex parte
volitorum. Ad cuius intellectum, considerandum est quod unumquodque,
secundum quod bonum est, sic est volitum a Deo. 

&

第三に、ダマスケヌスによれば、これは先行する意志について理解されるべきで
 あり、後続する意志についてではない。ただし、この区別は、神の意志自体の
 側から受け取られるのではない。神の意志の中に、より先、より後はないから
 である。そうではなく、意志されたものどもの側から受け取られる。
これを理解するためには、以下のことが考察されるべきである。すなわち、各々
 のものは、それが善であるかぎりで、神によって意志されたものである。



\\

Aliquid autem potest
esse in prima sui consideratione, secundum quod absolute consideratur,
bonum vel malum, quod tamen, prout cum aliquo adiuncto consideratur,
quae est consequens consideratio eius, e contrario se habet. Sicut
hominem vivere est bonum, et hominem occidi est malum, secundum
absolutam considerationem, sed si addatur circa aliquem hominem, quod
sit homicida, vel vivens in periculum multitudinis, sic bonum est eum
occidi, et malum est eum vivere. 


&

とこ
 ろで、あるものが、その最初の考察において、無条件的に考察されるかぎりで
 は善ないし悪であるが、しかし、何かが付け加わって考察されるとき、すなわ
 ち、それの後続する考察では、逆になるということがありうる。
たとえば、無条件的な考察において、人間が生きることは善であり、人間が殺さ
 れることは悪である。しかし、ある人間について、彼が人殺しであるとか、あ
 るいは、彼が生きていれば多くの人が危険にさらされる、というようなことが
 付け加えられるならば、その場合には、彼を殺すことが善であり、彼が生きること
 は悪となる。


\\




Unde potest dici quod iudex iustus
antecedenter vult omnem hominem vivere; sed consequenter vult homicidam
suspendi. Similiter Deus antecedenter vult omnem hominem salvari; sed
consequenter vult quosdam damnari, secundum exigentiam suae
iustitiae. 


&

したがって、次のように言われうる。すなわち、正しい裁判官は、先行するかた
 ちで、すべての人間が生きることを意志するが、しかし、後続するかたちで、
 殺人者が絞首刑にされることを意志する。同様に、神は、先行して、すべての
 人間が救われることを意志するが、後続して、自らの正義の要求にしたがって、
 ある人々が断罪されることを意志する。

\\



Neque tamen id quod antecedenter volumus, simpliciter
volumus, sed secundum quid. 
Quia voluntas comparatur ad res, secundum
quod in seipsis sunt, in seipsis autem sunt in particulari, unde
simpliciter volumus aliquid, secundum quod volumus illud consideratis
omnibus circumstantiis particularibus, quod est consequenter velle. 




&


しかし、私たちが先行して意志することは、端的に意志するのではなく、ある意
 味でである。なぜなら、意志は、事物に対して、事物がそれら自身において在るか
 ぎりで関係するが、事物は、それら自身において、個別的に在るからである。したがっ
 て、私たちは、すべての個別的な環境が考察された上で意志するものを、端的
 に意志するのであり、これは、後続的に意志することである。


\\


Unde
potest dici quod iudex iustus simpliciter vult homicidam suspendi, sed
secundum quid vellet eum vivere, scilicet inquantum est homo. Unde magis
potest dici velleitas, quam absoluta voluntas. Et sic patet quod
quidquid Deus simpliciter vult, fit; licet illud quod antecedenter vult,
non fiat.


&

このことから、次のように言われうる。正しい裁判官は、端的に、殺人者が絞首刑になるこ
 とを意志するが、ある意味では、すなわち、彼が人間であるかぎりでならば、彼が生き
 ることを意志したであろう。したがって、それは無条件的な意志というより、「反
 事実的な意志(velle-itas)」\footnote{山田訳では「条件的意志」。裁判官が、
 殺人犯を、人間であるという観点だけで裁くことは現実にはありえないから、
 反事実的とした。文法的にも、接続法未完了過去は、現在の事実に反する事柄
 を表現する。}と言われうる。この意味で、神が先行的に意志す
 ることは生じないことがあるが、神が端的に意志することは、何であれ生じる
 ことが明らかである。


\\


{\scshape Ad secundum dicendum} quod actus cognoscitivae
virtutis est secundum quod cognitum est in cognoscente, actus autem
virtutis appetitivae est ordinatus ad res, secundum quod in seipsis
sunt. Quidquid autem potest habere rationem entis et veri, totum est
virtualiter in Deo; sed non totum existit in rebus creatis. Et ideo Deus
cognoscit omne verum, non tamen vult omne bonum, nisi inquantum vult se,
in quo virtualiter omne bonum existit.


&

第二異論に対しては、次のように言われるべきである。
認識能力の作用は、認識されたものが認識するものの中にあるかぎりで成り立つ
 が、欲求能力の作用は、それ自体であるかぎりにおける事物へと秩序づけられ
 ている。ところで、有や真の性格を持ちうるものはすべて、全体が潜在的に
 (virtualiter)神の中にある。しかし、その全体が、創造された事物において生
 じるわけではない。ゆえに、神はすべての真を認識するが、すべての善を意志
 するわけではない。ただし、神が自らを意志するかぎりで、というのであれば
 別である。神の中には、潜在的にすべての善が存在するからである。


\\


{\scshape Ad tertium dicendum} quod causa prima tunc
potest impediri a suo effectu per defectum causae secundae, quando non
est universaliter prima, sub se omnes causas comprehendens, quia sic
effectus nullo modo posset suum ordinem evadere. Et sic est de voluntate
Dei, ut dictum est.


&

第三異論に対しては、以下のように言われるべきである。
第一原因は、自らのもとにすべての原因を包含する普遍的に第一の原因でないと
 きに、第二原因の欠陥を通して、自らの結果を生み出すことから妨げられうる。
 なぜなら、そうでない場合には、結果は決して自らの秩序から逃れられないからで
 ある。そして、すでに述べられたように、神の意志については、このような状
 態にある。


\end{longtable}
\newpage


\rhead{a.~7}
\begin{center}
 {\Large {\bf ARTICULUS SEPTIMUS}}\\
 {\large UTRUM VOLUNTAS DEI SIT MUTABILIS}\\
 {\footnotesize I {\itshape Sent}., d.~39, q.~1, a.~1; d.~48, q.~2,
 a.~1, ad 2; I {\itshape SCG.}, cap.~82; III, cap.~91, 96, 98; {\itshape
 De Verit.}, q.~12, a.~11, ad 3; {\itshape Hebr}., cap.~6, lect.~4.}\\
 {\Large 第七項\\神の意志は可変的か}
\end{center}

\begin{longtable}{p{21em}p{21em}}

{\huge A}{\scshape d septimum sic proceditur}. Videtur quod
 voluntas Dei sit mutabilis. Dicit enim dominus {\itshape Genes}.~{\scshape vi}, {\itshape poenitet me
 fecisse hominem}. Sed quemcumque poenitet de eo quod fecit, habet
 mutabilem voluntatem. Ergo Deus habet mutabilem voluntatem.

&

第七の問題へ、議論は以下のように進められる。
神の意志は可変的だと思われる。理由は以下の通り。
主は、『創世記』4章で「私は人間を作ったことを後悔する」\footnote{「主は、
 (中略)地上に人を造ったことを後悔し、心を痛められた。」(6:5-6)}と述べて
 いる。
ところが、だれであれ、行ったことについて後悔する者は、可変的な意志を持っ
 ている。ゆえに、神は、可変的な意志を持つ。

\\



{\scshape 2 Praeterea}, Ierem.~{\scshape xviii}, ex persona domini
 dicitur, {\itshape loquar adversus gentem et adversus regnum, ut eradicem et
 destruam et disperdam illud; sed si poenitentiam egerit gens illa a
 malo suo, agam et ego poenitentiam super malo quod cogitavi ut facerem
 ei}. Ergo Deus habet mutabilem voluntatem.


&

さらに、『エレミア書』18章で、主のペルソナから次のように言われている。
 「私は、民族や王国に対して、それを引き抜き、壊し、滅ぼす、と言うであろ
 う。しかし、もしその民族が、自分の悪から後悔を引き出すならば、私も、そ
 の民族に行おうと考えていた悪について、後悔をするであろう」
 \footnote{「あるとき、わたしは一つの民や王国を断罪して、抜き、壊し、滅
 ぼすが、もし、断罪したその民が、悪を悔いるならば、わたしはその民に災い
 をくだそうとしたことを思いとどまる。」(18:7)}。ゆえに、神
 は可変的な意志を持つ。

\\



{\scshape 3 Praeterea}, quidquid Deus facit, voluntarie
 facit. Sed Deus non semper eadem facit, nam quandoque praecepit legalia
 observari, quandoque prohibuit. Ergo habet mutabilem voluntatem.


&

さらに、神が為すことは何であれ、意志的に為す。ところで、神は、常に同じこ
 とを為すわけではない。なぜなら、神は、法にかんすることが守ら
 れることを命じたこともあれば、それを禁じたこともあるからである。ゆえに、
 神は可変的な意志を持つ。


\\



{\scshape 4 Praeterea}, Deus non ex necessitate vult
 quod vult, ut supra dictum est. Ergo potest velle et non velle
 idem. Sed omne quod habet potentiam ad opposita, est mutabile, sicut
 quod potest esse et non esse, est mutabile secundum substantiam; et
 quod potest esse hic et non esse hic, est mutabile secundum locum. Ergo
 Deus est mutabilis secundum voluntatem.


&

さらに、神は、前に述べられたとおり、意志することを必然的に意志するわけで
 はない。ゆえに、神は同一のことを意志することも意志しないこともできる。
ところで、反対のことへの可能性を持つものはすべて、可変的である。
たとえば、存在することもしないことも可能であるものは、実体において可変的
 であり、ここにあることもないことも可能なものは、場所において可変的であ
 る。ゆえに、神は意志において可変的である。


\\



{\scshape  Sed contra est} quod dicitur {\itshape Num}.~{\scshape xxiii}, {\itshape non est Deus, quasi homo, ut mentiatur; neque ut filius hominis, ut
 mutetur}.


&

しかし反対に、『民数記』23章で、「神は人間のようではなく、その結果、嘘を
 つかない。また、人間の子のようではないので、変化しない」\footnote{「神
 は人ではないから、偽ることはない。人の子ではないから、悔いることはない。」(23:19)}と言われている。


\\



{\scshape Respondeo dicendum} quod voluntas Dei est
 omnino immutabilis. Sed circa hoc considerandum est, quod aliud est
 mutare voluntatem; et aliud est velle aliquarum rerum
 mutationem. Potest enim aliquis, eadem voluntate immobiliter
 permanente, velle quod nunc fiat hoc, et postea fiat contrarium. Sed
 tunc voluntas mutaretur, si aliquis inciperet velle quod prius non
 voluit, vel desineret velle quod voluit. Quod quidem accidere non
 potest, nisi praesupposita mutatione vel ex parte cognitionis, vel
 circa dispositionem substantiae ipsius volentis. 


&

解答する。以下のように言われるべきである。
神の意志は、まったく不変的である。
しかし、これについては、意志が変化することと、なんらかの事物の変化を意志
 することとは異なるという点が考察されるべきである。
というのも、ある人が、同一の意志に変わらず留まりながら、今これが生じ、あ
 とで反対のものが生じることを意志することがありうるからである。しかし、
 もし、ある人が、以前には意志していなかったことを意志し始めたり、あるい
 は、意志していたことを意志しなくなったりすると、その場合に意志は変化す
 る。このことは、認識の側から、あるいは、意志するもの自身の実体の状態
 ををめぐって、変化が前提されないかぎり起こりえない。



\\





Cum enim voluntas sit
 boni, aliquis de novo dupliciter potest incipere aliquid velle. Uno
 modo sic, quod de novo incipiat sibi illud esse bonum. Quod non est
 absque mutatione eius, sicut adveniente frigore, incipit esse bonum
 sedere ad ignem, quod prius non erat. Alio modo sic, quod de novo
 cognoscat illud esse sibi bonum, cum prius hoc ignorasset, ad hoc enim
 consiliamur, ut sciamus quid nobis sit bonum. Ostensum est autem supra
 quod tam substantia Dei quam eius scientia est omnino immutabilis. Unde
 oportet voluntatem eius omnino esse immutabilem.


&

その理由は以下の通りである。
意志は善にかかわるので、ある人が、新たに何かを意志し始めるのは、以下の二
 通りのかたちによる。一つには、それがその人にとって新たに善であり始めるこ
 とによる。このことは、その人の変化なしにはない。たとえば、寒さがやって
 来たとき、火の近くに座ることが、以前はそうでなかったのに、善いことであ
 り始める場合のように。
もう一つには、以前には知らなかったのに、新たに、それが自分にとって善であ
 ることを知るというかたちである。じっさい、私たちは、何が私たちにとって
 善であるかを知るために、さまざまなことを探究する\footnote{consiliamur
 山田訳「知るようにとすすめられている」}。ところで、神の実体は、神の知と
 同様に、まったく不変であることが前に\footnote{Q.~9, a.~1; Q.~14, a.~15.}示された。したがって、神の意志はまっ
 たく不変でなければならない。


\\



{\scshape Ad primum ergo dicendum} quod illud verbum
 domini metaphorice intelligendum est, secundum similitudinem nostram,
 cum enim nos poenitet, destruimus quod fecimus. Quamvis hoc esse possit
 absque mutatione voluntatis, cum etiam aliquis homo, absque mutatione
 voluntatis, interdum velit aliquid facere, simul intendens postea illud
 destruere. Sic igitur Deus poenituisse dicitur, secundum similitudinem
 operationis, inquantum hominem quem fecerat, per diluvium a facie
 terrae delevit\footnote{d\={e}l\u{e}o, l\={e}vi, l\={e}tum. to abolish,
 destroy, annihilate.}.


&

第一異論に対しては、それゆえ、以下のように言われるべきである。
主のかの言葉は、私たちの類似に従って、比喩的に理解されるべきである。とい
 うのも、私たちは、悔いるとき、作ったものを破壊するからである。
ただし、こういうことは、意志の変化なしにありうる。なぜなら、時として、ある人が、意志の変化な
 しに、何かを作ろうと意志し、同時に、作ったあとでそれを破壊することを意
 図するということがあるからである。
それゆえ、このようにして、神は、造った人間を、洪水によって地表から取り去っ
 た点で、働きの類似に従って、悔いたと言われる。

\\



{\scshape Ad secundum dicendum} quod voluntas Dei, cum
 sit causa prima et universalis, non excludit causas medias, in quarum
 virtute est ut aliqui effectus producantur. Sed quia omnes causae
 mediae non adaequant virtutem causae primae, multa sunt in virtute et
 scientia et voluntate divina, quae non continentur sub ordine causarum
 inferiorum; sicut resuscitatio Lazari. 


&

第二異論に対しては、以下のように言われるべきである。
神の意志は、第一の普遍的な原因だから、中間の諸原因を排除しない。そしてそ
 の中間の諸原因の力の中に、なんらかの諸結果が生み出されるということがあ
 る。しかし、すべての中間原因は、第一原因の力に匹敵しないので、下位の諸
 原因の秩序のもとに含まれる多くのものが、神の知と意志の力の中にある。
たとえば、ラザロの復活がそれである。


\\



Unde aliquis respiciens ad
 causas inferiores, dicere poterat, {\itshape Lazarus non resurget}, respiciens
 vero ad causam primam divinam, poterat dicere, {\itshape Lazarus resurget}. Et
 utrumque horum Deus vult, scilicet quod aliquid quandoque sit futurum
 secundum causam inferiorem, quod tamen futurum non sit secundum causam
 superiorem; vel e converso. 



&

したがって、ある人は、下位の諸原因を見て、「ラザロは復活しないだろう」と
 言うことができたが、神の第一原因を見る人は、「ラザロは復活するであろう」
 と言うことができた。そして、これらのどちらも神は意志する。すなわち、あ
 ることが、下位の原因に従えば生じるであろうが、上位の原因に従えば生じ
 ないだろうということを、あるいは、その逆のことを、時として意志する。


\\


Sic ergo dicendum est quod Deus aliquando
 pronuntiat aliquid futurum, secundum quod continetur in ordine causarum
 inferiorum, ut puta secundum dispositionem naturae vel meritorum; quod
 tamen non fit, quia aliter est in causa superiori divina. Sicut cum
 praedixit Ezechiae, {\itshape dispone domui tuae\footnote{この与格の文
 法?}, quia morieris et non vives}, ut
 habetur Isaiae {\scshape xxxviii}; neque tamen ita evenit, quia ab aeterno aliter
 fuit in scientia et voluntate divina, quae immutabilis est. 


&

ゆえに、神は時として、ある将来のことを、それが下位の諸原因の秩序に含まれ
 ているかぎりで、たとえば、自然の、あるいは功績の状態において語る。しか
 し、それは、神の上位の原因の中では異なっているので、実際には生じない。
たとえば、『イザヤ書』33章で、エゼキアに「あなたの家を整えよ。あなたは死んで生きないだろうか
 ら」\footnote{そのころ、ヒゼキヤは死の病にかかった。預言者、アモツの子
 イザヤが訪ねて来て、「主はこう言われる。『あなたは死ぬことになっていて、
 命はないのだから、家族に遺言をしなさい』」と言った。(38:1)}と前もって語ったが、それは起こらなかった。不変である神の知と意志の
 中では、永遠から、そうではなかったからである。


\\


Propter
 quod dicit Gregorius, quod {\itshape Deus immutat sententiam, non tamen mutat
 consilium}, scilicet voluntatis suae. Quod ergo dicit, {\itshape poenitentiam agam
 ego}, intelligitur metaphorice dictum, nam homines quando non implent
 quod comminati sunt, poenitere videntur.


&

このため、グレゴリウスは「神は言明を変えるが、思慮を変えない」と言う。思
 慮とはすなわち自分の意志のことである。ゆえに、「私は後悔するであろう」
 と言う、ということは、比喩的に語られたものとして理解される。なぜなら、
 人々は、脅したことを実行しないとき、悔いているように見えるからである。


\\



{\scshape Ad tertium dicendum} quod ex ratione illa non
 potest concludi quod Deus habeat mutabilem voluntatem; sed quod
 mutationem velit.


&

第三異論に対しては、以下のように言われるべきである。
かの根拠からは、神が可変的な意志を持っていることは帰結しえず、変化を意志
 することが帰結しうるのみである。


\\



{\scshape Ad quartum dicendum} quod, licet Deum velle
 aliquid non sit necessarium absolute, tamen necessarium est ex
 suppositione, propter immutabilitatem divinae voluntatis, ut supra
 dictum est.


&

第四異論に対しては、以下のように言われるべきである。
前に述べられたとおり\footnote{a.~3.}、神が何かを意志することは、無条件的に必然的なことでは
 ないが、しかし、神の意志の不変性のために、仮定に基づいて、必然的である。



\end{longtable}
\newpage



\rhead{a.~8}
\begin{center}
 {\Large {\bf ARTICULUS OCTAVUS}}\\
 {\large UTRUM VOLUNTAS DEI NECESSITATEM REBUS VOLITIS IMPONAT}\\
 {\footnotesize I {\itshape SCG}, cap.~85; II cap.~29, 30; {\itshape De
 Verit.}, q.~23, a.~5; {\itshape De Malo}, q.~16, a.~7, ad 15: {\itshape
 Quodl}.~XI, q.~3; XII, q.~3, ad 1; I {\itshape Periherm.}, lect.~14.}\\
 {\Large 第八項\\神の意志は、意志された事物に必然性を与えるか}
\end{center}

\begin{longtable}{p{21em}p{21em}}


{\huge A}{\scshape d octavum sic proceditur}. Videtur quod
voluntas Dei rebus volitis necessitatem imponat. Dicit enim Augustinus,
in {\itshape Enchirid}., {\itshape nullus fit salvus, nisi quem Deus voluerit salvari. Et
ideo rogandus est ut velit, quia necesse est fieri, si voluerit}.

&

第八項の問題へ、議論は以下のように進められる。
神の意志は、意志された事物に必然性を与えると思われる。
理由は以下の通り。
アウグスティヌスは、『エンキリディオン』で次のように述べている。「だれも、神がそ
 の人が救われることを意志した人でなければ、救われた者とならない。ゆえに、
 神がそう意志するように請うべきである。なぜなら、神が意志したら、そうな
 ることが必然だからである」。


\\


{\scshape 2 Praeterea}, omnis causa quae non potest
impediri, ex necessitate suum effectum producit, quia et natura semper
idem operatur, nisi aliquid impediat, ut dicitur in II {\itshape Physic}. Sed
voluntas Dei non potest impediri, dicit enim apostolus, {\itshape ad Rom}.~{\scshape ix},
{\itshape voluntati enim eius quis resistit?} Ergo voluntas Dei imponit rebus
volitis necessitatem.


&

さらに、妨げられることが不可能な原因はすべて、自らの結果を必然的に生み出
 す。なぜなら、『自然学』2巻で言われるように、自然も、何かが妨げないかぎ
 り、常に同じ働きをするからである。ところで、神の意志は妨げられない。な
 ぜなら、使徒が『ローマの信徒への手紙』9章で、「彼の意志にだれが抵抗しよ
 うか」\footnote{「だれが神の御心に逆らうことができようか」(9:19)}と述べているからである。ゆえに、神の意志は、意志された事物に必然
 性を与える。


\\


{\scshape 3 Praeterea}, illud quod habet necessitatem ex
priori, est necessarium absolute, sicut animal mori est necessarium,
quia est ex contrariis compositum. Sed res creatae a Deo, comparantur ad
voluntatem divinam sicut ad aliquid prius, a quo habent necessitatem,
cum haec conditionalis sit vera, {\itshape si aliquid Deus vult, illud est}; omnis
autem conditionalis vera est necessaria. Sequitur ergo quod omne quod
Deus vult, sit necessarium absolute.


&

さらに、先行するものから必然性を持つものは、無条件に必然的なものである。
 たとえば、動物が死ぬことは必然的なことだが、それは、[動物が]反対のものどもから
 構成されているからである。ところで、神によって創造された諸事物が、神の
 意志に対してもつ関係は、先行する何かに対してもつ関係と等しい。ゆえに、
そこから必然性を持つ。なぜなら、「もし神が何かを意志するならば、その通り
 になる」という条件命題は真であるが、すべての真である条件命題は必然的だから
 である\footnote{やや意味が取りにくい。一般に、$p \supset q$と$\Box(p \supset q)$は同値でない。
したがって、「すべての条件命題は必然的である」という主張ではない。ここで
 は「真である条件命題」が問題となっている。この「真である」とは何か。
 「たまたま真である」という意味ではないとすると、「恒真である」とか「定
 理である」という意味か。あるモデルのもとで、ある条件文が真であるならば、
 そのモデルのもとで、あらゆる可能世界でその条件文は真、つまり、$\models p \supset q$であるならば
 $\models \Box(p \supset q)$である、ということか。だとすると、この主張は、
 いわゆる必然化規則(necessitation rule)について述べていて、これは条件命題に限
 らず成立する。\\ あるいは、トマスはここで、この世界の中での統計的必然性につい
 て述べていて、ある条件命題がこの世界で真であるならば、これまでずっと、
 また、これからもずっと真である、と言おうとしているのか。}。ゆえに、神が意志するすべてのことは、無条件的に必然的である。


\\


{\scshape Sed contra}, omnia bona quae fiunt, Deus vult
fieri. Si igitur eius voluntas imponat rebus volitis necessitatem,
sequitur quod omnia bona ex necessitate eveniunt. Et sic perit liberum
arbitrium et consilium, et omnia huiusmodi.


&

しかし反対に、生じるすべての善は、神がそれが生じることを意志する。
ゆえに、もし神の意志が、意志された諸事物に必然性を与えるならば、
すべての善は必然的に生じることになる。
そうすれば、自由選択や思量、そしてすべてそのような事柄は、無くなることに
 なる。


\\


{\scshape Respondeo dicendum} quod divina voluntas
quibusdam volitis necessitatem imponit, non autem omnibus. Cuius quidem
rationem aliqui assignare voluerunt ex causis mediis quia ea quae
producit per causas necessarias, sunt necessaria; ea vero quae producit
per causas contingentes, sunt contingentia. 


&

解答する。以下のように言われるべきである。
神の意志は、意志されたある事柄には必然性を与えるが、すべてのものに与える
 わけではない。ある人々は、この理由を、中間原因に基づいて指定しようとし
 た。
つまり、[神が]必然的な原因によって生み出すものは必然的であり、偶然的な原因に
 よって生み出すものどもは偶然的だからである。


\\



Sed hoc non videtur
sufficienter dictum, propter duo. Primo quidem, quia effectus alicuius
primae causae est contingens propter causam secundam, ex eo quod
impeditur effectus causae primae per defectum causae secundae; sicut
virtus solis per defectum plantae impeditur. Nullus autem defectus
causae secundae impedire potest quin voluntas Dei effectum suum
producat. 

&


しかし、これは、二つのことのために、十分に語られていないと思われる。
第一に、ある第一原因の結果が、第二原因のために偶然的になるのは、第一原因
 の結果が、第二原因の結果によって妨げられるからである。たとえば、太陽の
 力が植物の欠陥によって妨げられるように。ところが、第二原因のどんな欠陥
 も、神の意志が自らの結果を生み出さないように妨げることはできない。

\\



Secundo, quia, si distinctio contingentium a necessariis
referatur solum in causas secundas, sequitur hoc esse praeter
intentionem et voluntatem divinam, quod est inconveniens. 

&


第二に、もし偶然的なものと必然的なものとの区別が、第二原因に関係づけられ
 るだけならば、神の意図と意志の外にあることになるが、これは不都合である。

\\




Et ideo melius
dicendum est, quod hoc contingit propter efficaciam divinae
voluntatis. Cum enim aliqua causa efficax fuerit ad agendum, effectus
consequitur causam non tantum secundum id quod fit, sed etiam secundum
modum fiendi vel essendi, ex debilitate enim virtutis activae in semine,
contingit quod filius nascitur dissimilis patri in accidentibus, quae
pertinent ad modum essendi. 

&

それゆえ、よりよく以下のように言われるべきである。
このことが生じるのは、神の意志の強力さのためである。
ある原因の作用が強力であるとき、結果は、生じるという点で原因に伴うだけで
 なく、どのように生じるか、どのように在るか[=存在様態]、という点でもまた、原因に伴
 う。たとえば、種子の中にある作用的力の弱さのために、産まれた息子が附帯性の点で
 父に似ていないということがあるが、これは、どのように在るか、ということ
 に属する。


\\



Cum igitur voluntas divina sit
efficacissima, non solum sequitur quod fiant ea quae Deus vult fieri;
sed quod eo modo fiant, quo Deus ea fieri vult. Vult autem quaedam fieri
Deus necessario, et quaedam contingenter, ut sit ordo in rebus, ad
complementum universi. 

&

したがって、神の意志は最も強力だから、神が生じることを意志するものが生じ
 るだけでなく、神がそのように生じることを意志する、そのあり方で生じる、
 ということが帰結する。ところで、神は、宇宙の完成のために、諸事物の中に
 秩序があるように、あるものどもが必然的に生じ、別の
 あるものどもが偶然的に生じることを意志する。

\\



Et ideo quibusdam effectibus aptavit causas
necessarias, quae deficere non possunt, ex quibus effectus de
necessitate proveniunt, quibusdam autem aptavit causas contingentes
defectibiles, ex quibus effectus contingenter eveniunt. Non igitur
propterea effectus voliti a Deo, eveniunt contingenter, quia causae
proximae sunt contingentes, sed propterea quia Deus voluit eos
contingenter evenire, contingentes causas ad eos praeparavit.


&


ゆえに、[神は]ある結果には、欠けることがありえず、そこから必然的に結果が生じる
 ような、必然的な原因を適用し、別のある結果には、欠けることがありえ、そ
 こから偶然的に結果が生じるような、偶然的な原因を適用した。
ゆえに、神によって意志された結果が偶然的に生じるのは、近接原因が偶然的だ
 からではなく、神が、それが偶然的に生じることを意志したので、それに偶然
 的な原因を準備したからである。

\\


{\scshape Ad primum ergo dicendum} quod per illud verbum
Augustini intelligenda est necessitas in rebus volitis a Deo, non
absoluta, sed conditionalis, necesse est enim hanc conditionalem veram
esse, si Deus hoc vult, necesse est hoc esse.


&

第一異論に対しては、それゆえ、以下のように言われるべきである。
かのアウグスティヌスの言葉によって、神に意志された諸事物における無条件的
 な必然性ではなく、条件的な必然性が理解されるべきである。
なぜなら、「もし神がこれを意志すれば、これがあることは必然である」という
 条件命題が真であることは、必然的だからである。

\\


{\scshape Ad secundum dicendum} quod, ex hoc ipso quod
nihil voluntati divinae resistit, sequitur quod non solum fiant ea quae
Deus vult fieri; sed quod fiant contingenter vel necessario, quae sic
fieri vult.


&

第二異論に対しては、以下のように言われるべきである。
何も神の意志に逆らわないということからは、神が生じることを意志するものが
 生じる、ということだけでなく、神が意志するとおり、偶然的に、あるいは必
 然的に生じる、ということも帰結する。


\\


{\scshape Ad tertium dicendum} quod posteriora habent
necessitatem a prioribus, secundum modum priorum. Unde et ea quae fiunt
a voluntate divina, talem necessitatem habent, qualem Deus vult ea
habere, scilicet, vel absolutam, vel conditionalem tantum. Et sic, non
omnia sunt necessaria absolute.


&

第三異論に対しては、以下のように言われるべきである。
後続するものは、先行するものから、先行するもののあり方に従って、必然性を
 受け取る。したがって、神の意志によって生じるものどもは、神が、それらが
 持つことを意志するような必然性、すなわち、無条件的な必然性か、または、たんに
 条件的な必然性を持つ。このようにして、すべてのものが無条件的に必然的で
 あるわけではない。




\end{longtable}
\newpage





\rhead{a.~9}
\begin{center}
 {\Large {\bf ARTICULUS NONUS}}\\
 {\large UTRUM VOLUNTAS DEI SIT MALORUM}\\
 {\footnotesize Infra, q.~48, a.~6; I {\itshape Sent.}, d.~46, a.~4; I
 {\itshape SCG}, cap.~95; {\itshape De Pot.}, q.~1, a.~6; {\itshape De
 Malo}, q.~2, a.~1, ad 16.}\\
 {\Large 第九項\\神の意志は悪にかかわるか}
\end{center}

\begin{longtable}{p{21em}p{21em}}

{\huge A}{\scshape d nonum sic proceditur}. Videtur quod
voluntas Dei sit malorum. Omne enim bonum quod fit, Deus vult. Sed mala
fieri bonum est, dicit enim Augustinus, in {\itshape Enchirid}., {\itshape quamvis ea quae
mala sunt, inquantum mala sunt, non sint bona; tamen, ut non solum bona,
sed etiam ut sint mala, bonum est}. Ergo Deus vult mala.

&

第9項の問題へ、議論は以下のように進められる。
神の意志は悪にかかわると思われる。理由は以下の通り。
すべて生じる善は、神がそれを意志する。ところが、悪が生じることは善である。
 なぜなら、アウグスティヌスは『エンキリディオン』で、「悪であるものども
 は、それが悪である限りにおいて善でないにしても、しかし、善だけでなく、
 悪もまた存在することは、善である」と述べているからである。ゆえに、神は悪
 を意志する。

\\


{\scshape 2 Praeterea}, dicit Dionysius, {\scshape iv} cap.~{\itshape de
Div.~Nom.}, {\itshape erit malum ad omnis (idest universi) perfectionem
conferens}. Et Augustinus dicit, in {\itshape Enchirid}., {\itshape ex omnibus consistit
universitatis admirabilis pulchritudo; in qua etiam illud quod malum
dicitur, bene ordinatum, et loco suo positum, eminentius commendat bona;
ut magis placeant, et laudabiliora sint, dum comparantur malis}. Sed Deus
vult omne illud quod pertinet ad perfectionem et decorem universi, quia
hoc est quod Deus maxime vult in creaturis. Ergo Deus vult mala.


&

さらに、ディオニュシウスは『神名論』四章で「悪は、万物の(すなわち宇宙の)
 完全性のために有益なものとしてあるだろう」と述べている。また、アウグスティ
 ヌスは、『エンキリディオン』で「すべてのものから、驚嘆すべき宇宙の美し
 さが成り立っている。その美しさにおいては、悪と言われるものもまた、よく
 秩序づけられ、自らの場所に置かれて、より優れたかたちで善を褒め称えてい
 る。悪と比較されることで、より気に入られ、より賞賛されるものであるよう
 に」と述べる。ところで、神は、宇宙の完全性と魅力に属するすべてのことを意
 志する。なぜなら、これが、神が被造物において最大限に意志することだから
 である。ゆえに、神は悪を意志する。

\\


{\scshape 3 Praeterea}, mala fieri, et non fieri, sunt
contradictorie opposita. Sed Deus non vult mala non fieri, quia, cum
mala quaedam fiant, non semper voluntas Dei impleretur. Ergo Deus vult
mala fieri.


&

さらに、悪が生じることと、悪が生じないこととは、矛盾するかたちで対立する。
 ところが、神は、悪が生じないことを意志しない。なぜなら、なんらかの悪が
 生じているのだから、[もし神が悪が生じないことを意志していたならば]神
 の意志は常に満たされていないことになったであろうから。ゆえに、神は悪を
 意志する。


\\


{\scshape Sed contra est} quod dicit Augustinus, in
libro {\itshape octoginta trium quaest}., {\itshape nullo sapiente homine auctore, fit homo
deterior; est autem Deus omni sapiente homine praestantior; multo igitur
minus, Deo auctore, fit aliquis deterior. Illo autem auctore cum
dicitur, illo volente dicitur}. Non ergo volente Deo, fit homo
deterior. Constat autem quod quolibet malo fit aliquid deterius. Ergo
Deus non vult mala.


&

しかし反対に、アウグスティヌスは、『八十三問題集』という書物の中で、次の
 ように述べている。「知恵ある人間が作者となって、劣った人間が生じることはない。
神は、知恵あるどんな人間よりも優れている。
それゆえ、ましてや、神が作者となって、劣った誰かが生じることはない。
ところで、ある人が作者と言われるとき、それを意志することでそう言われている」。
ゆえに、神が意志して、より劣った人間が生じるのではない。
ところで、どんな悪によっても、なんらかの劣ったものが生じることは明らかで
 ある。ゆえに、神は悪を意志しない。


\\


{\scshape Respondeo dicendum} quod, cum ratio boni sit
ratio appetibilis, ut supra dictum est, malum autem opponatur bono;
impossibile est quod aliquod malum, inquantum huiusmodi, appetatur,
neque appetitu naturali, neque animali, neque intellectuali, qui est
voluntas. Sed aliquod malum appetitur per accidens, inquantum
consequitur ad aliquod bonum. Et hoc apparet in quolibet appetitu. Non
enim agens naturale intendit privationem vel corruptionem; sed formam,
cui coniungitur privatio alterius formae; et generationem unius, quae
est corruptio alterius. Leo etiam, occidens cervum, intendit cibum, cui
coniungitur occisio animalis. Similiter fornicator intendit
delectationem, cui coniungitur deformitas culpae. 


&

解答する。以下のように言われるべきである。
前に述べられたとおり、善の性格は、欲求されうるという性格であり、また、悪
 は善に対立するので、なんらかの悪が、そのようなものである限りで、欲求さ
 れることは不可能である。それは、自然本性的欲求によっても、動物的な欲求によって
 も、知性的な欲求によっても、ない。この知性的欲求が、意志である。
しかし、なんらかの悪が、なんらかの善に伴う限りで、附帯的に欲求されること
 はある。そして、このことは、どんな欲求においても明らかである。たとえば、
 自然的作用者は、欠如や消滅へ向かわず、形相へ向かうが、
 どんな形相にも、他の形相の欠如が結びつけられている。また、あるものの生成へ
 と向かうが、それは他のものの消滅でもある。ライオンも、シカを殺すときエ
 サを求めているが、
それには動物を殺すことが結びつけられている。
 同様に、姦淫する人は快楽を求めるが、それには、罪の醜さが結びつけられて
 いる。


\\



Malum autem quod
coniungitur alicui bono, est privatio alterius boni. Nunquam igitur
appeteretur malum, nec per accidens, nisi bonum cui coniungitur malum,
magis appeteretur quam bonum quod privatur per malum. Nullum autem bonum
Deus magis vult quam suam bonitatem, vult tamen aliquod bonum magis quam
aliud quoddam bonum. Unde malum culpae, quod privat ordinem ad bonum
divinum, Deus nullo modo vult. Sed malum naturalis defectus, vel malum
poenae vult, volendo aliquod bonum, cui coniungitur tale malum, sicut,
volendo iustitiam, vult poenam; et volendo ordinem naturae servari, vult
quaedam naturaliter corrumpi.


&

ところで、なんらかの善に結びつけられている悪は、なんらかの善の欠如である。
ゆえに、悪が結びつけられている善が、その悪によって欠如する善よりも欲せら
 れるのでない限り、附帯的にであっても、悪が欲せられることはけっしてない。
 また、神は、自らの善性以上に、なんらかの善を意志することはないが、ある
 何らかの善を、他のなんらかの善よりも意志するということはある。したがっ
 て、罪の悪は、神という善への秩序を奪うものなので、神はそれをけっして意志しな
 いが、自然的欠陥の悪や、罰の悪は意志する。それは、そのような悪が
 結びつけられているなんらかの善を意志してである。たとえば、正義を意志し
 ながら罰を意志し、自然の秩序が保たれることを意志しつつ、あるものどもが
 自然本性的に消滅することを意志するように。


\\


{\scshape Ad primum ergo dicendum} quod quidam dixerunt
quod, licet Deus non velit mala, vult tamen mala esse vel fieri, quia,
licet mala non sint bona, bonum tamen est mala esse vel fieri. Quod ideo
dicebant, quia ea quae in se mala sunt, ordinantur ad aliquod bonum,
quem quidem ordinem importari credebant in hoc quod dicitur, {\itshape mala esse
vel fieri}. Sed hoc non recte dicitur. Quia malum non ordinatur ad bonum
per se, sed per accidens. Praeter intentionem enim peccantis est, quod
ex hoc sequatur aliquod bonum; sicut praeter intentionem tyrannorum
fuit, quod ex eorum persecutionibus claresceret patientia martyrum. Et
ideo non potest dici quod talis ordo ab bonum importetur per hoc quod
dicitur, quod malum esse vel fieri sit bonum, quia nihil iudicatur
secundum illud quod competit ei per accidens, sed secundum illud quod
competit ei per se.


&

第一異論に対しては、それゆえ、以下のように言われるべきである。
ある人々は、「神は悪を意志しないが、悪が存在することや生じることを意志す
 る。なぜなら、悪は善でないが、悪が存在することや生じることは善であるか
 ら」と述べた。こう言ったのは、それ自体において悪であるものも、なんらか
 の善に秩序づけられていて、その秩序が、「悪が存在することや生じること」
 と言われることの中に含意されると信じたからである。しかし、これは正しく
 語られていない。なぜなら、悪が善に秩序づけられるのは、自体的にではなく、
 附帯的にだからである。たとえば、殉教者たちの忍耐が、彼らの迫害から明ら
 かになることが、暴君の意図の外にあったように、なんらかの善が伴うことは、
 罪人の意図の外にあるからである。それゆえ、善へのそのような秩序が、「悪
 が存在することや生じることは善である」と言われることによって含意され
 る、と言われることはできない。なぜなら、なにものも、それに附帯的に適合
 することにしたがってではなく、自体的に適合することによって、判断される
 からである。



\\


{\scshape Ad secundum dicendum} quod malum non operatur
ad perfectionem et decorem universi nisi per accidens, ut dictum
est. Unde et hoc quod dicit Dionysius, quod malum est ad universi
perfectionem conferens, concludit inducendo quasi ad inconveniens.


&


第二異論に対しては、以下のように言われるべきである。
すでに述べられたように、悪は、附帯的にでなければ、宇宙の完全性や魅力のた
 めに働くことがない。したがって、ディオニュシウスが、「悪は宇宙の完全性
 のために有益なもの」述べることは、いわば不都合なことへ導くかたちで、結
 論している。

\\


{\scshape Ad tertium dicendum} quod, licet mala fieri,
et mala non fieri, contradictorie opponantur; tamen velle mala fieri, et
velle mala non fieri, non opponuntur contradictorie, cum utrumque sit
affirmativum. Deus igitur neque vult mala fieri, neque vult mala non
fieri, sed vult permittere mala fieri. Et hoc est bonum.

&

第三異論に対しては、以下のように言われるべきである。
「悪が生じる」と「悪が生じない」は矛盾対立するが、しかし、「悪が生じるこ
 とを意志する」と「悪が生じないことを意志する」は矛盾対立しない。なぜな
 ら、どちらも肯定命題だからである。ゆえに、神は、悪が生じることを意志す
 るのでも、悪が生じないことを意志するのでもなく、悪が生じるに任せるこ
 とを意志する。そして、このことは善である。


\end{longtable}
\newpage

\rhead{a.~10}
\begin{center}
 {\Large {\bf ARTICULUS DECIMUS}}\\
 {\large UTRUM DEUS HABEAT LIBERUM ARBITRIUM}\\
 {\footnotesize II {\itshape Sent.}, d.~25, q.~1, a.~1; {\itshape SCG},
 cap.~88; {\itshape De Verit.}, q.~24, a.~3; {\itshape De Malo}, q.~16, a.~5.}\\
 {\Large 第十項\\神は自由決定力をもつか}
\end{center}

\begin{longtable}{p{21em}p{21em}}
{\huge A}{\scshape d decimum sic proceditur}. Videtur quod
Deus non habeat liberum arbitrium. Dicit enim Hieronymus, in homilia de
Filio Prodigo, {\itshape solus Deus est, in quem peccatum non cadit, nec cadere
potest; cetera, cum sint liberi arbitrii, in utramque partem flecti
possunt}.

&

第十項の問題へ、議論は以下のように進められる。
神は自由決定力を持たないと思われる。理由は以下の通り。
ヒエロニムスは『放蕩息子についての教話』で、「ただ神だけが、罪がそれへと
 落ちず、また、落ち得ないものである。それ以外のものは、自由決定力を持つ
 ので、どちらの面へも曲げられうる」と述べている。

\\


{\scshape 2 Praeterea}, liberum arbitrium est facultas
rationis et voluntatis, qua bonum et malum eligitur. Sed Deus non vult
malum, ut dictum est. Ergo liberum arbitrium non est in Deo.


&

さらに、自由決定力は理性と意志の機能であり、それによって善と悪が選ばれる。
 ところが、神は、すでに述べられたように、悪を意志しない。ゆえに、自由決
 定力は、神の中にない。

\\


{\scshape Sed contra est} quod dicit Ambrosius, in
libro {\itshape de fide}, {\itshape spiritus sanctus dividit singulis prout vult, idest pro
liberae voluntatis arbitrio, non necessitatis obsequio}.


&


しかし反対に、アンブロシウスは、『信仰について』の中で、「聖霊は、意志す
 るとおりに、つまり、自由な意志の決定として、各々のものに分けるのであって、
 必然性にしたがってではない」と述べている。

\\


{\scshape Respondeo dicendum} quod liberum arbitrium
habemus respectu eorum quae non necessario volumus, vel naturali
instinctu. Non enim ad liberum arbitrium pertinet quod volumus esse
felices, sed ad naturalem instinctum. Unde et alia animalia, quae
naturali instinctu moventur ad aliquid, non dicuntur libero arbitrio
moveri. Cum igitur Deus ex necessitate suam bonitatem velit, alia vero
non ex necessitate, ut supra ostensum est; respectu illorum quae non ex
necessitate vult, liberum arbitrium habet.


&


解答する。以下のように言われるべきである。
私たちは、必然的に意志したり、あるいは自然本性の衝動によって意志したりするのでないことが
 らについて、自由決定力をもつ。
たとえば、私たちが幸せであることを意志するのは、自由決定力ではなく自然本
 性的衝動に属する。したがって、自然本性的衝動によって何かへと動かされる
 他の動物も、自由決定力によって動くとは言われない。
それゆえ、前に示されたとおり、神は、必然的に自らの善性を意志するが、それ以外のものを必然性に
 基づいて意志しないのだから、必然性に基づいて意志しないことがらについて、
 自由決定力をもつ。


\\


{\scshape Ad primum ergo dicendum} quod Hieronymus
videtur excludere a Deo liberum arbitrium, non simpliciter, sed solum
quantum ad hoc quod est deflecti in peccatum.


&


第一異論に対しては、それゆえ、以下のように言われるべきである。
ヒエロニムスは、神から端的に自由決定力を排除しているのではなく、罪へと陥
 る者に属することにかんしてのみ、自由決定力を排除していると思われる。

\\


{\scshape Ad secundum dicendum} quod, cum malum culpae
dicatur per aversionem a bonitate divina, per quam Deus omnia vult, ut
supra ostensum est, manifestum est quod impossibile est eum malum culpae
velle. Et tamen ad opposita se habet, inquantum velle potest hoc esse
vel non esse. Sicut et nos, non peccando, possumus velle sedere, et non
velle sedere.


&


第二異論に対しては、以下のように言われるべきである。
罪の悪は、神の善性からの逸脱によって語られ、また、前に示されたとおり、そ
 の神の善性を通して、神はすべてのことを意志するから、神が罪の悪を意志す
 ることは不可能である。しかし、これこれであること、または、これこれでな
 いことを意志しうる限りで、反対のものに関係する。ちょうど、私たちもまた、
 罪を犯すことなしに、座ることも座らないことも意志することができるように。


\end{longtable}
\newpage



\rhead{a.~11}
\begin{center}
 {\Large {\bf ARTICULUS UNDECIMUS}}\\
 {\large UTRUM SIT DISTINGUENDA IN DEO VOLUNTAS SIGNI}\\
 {\footnotesize I {\itshape Sent.}, d.~45, a.~4; {\itshape De Verit.},
 q.~23, a.~3.}\\
 {\Large 第十一項\\神において「しるしの意志」が区別されるべきか}
\end{center}

\begin{longtable}{p{21em}p{21em}}


{\huge A}{\scshape d undecimum sic proceditur}. Videtur quod
non sit distinguenda in Deo voluntas signi. Sicut enim voluntas Dei est
causa rerum, ita et scientia. Sed non assignantur aliqua signa ex parte
divinae scientiae. Ergo neque debent assignari aliqua signa ex parte
divinae voluntatis.


&

第十一項の問題へ、議論は以下のように進められる。
神において「しるしの意志」が区別されるべきでないと思われる。理由は以下の
 通り。
神の意志が諸事物の原因であるように、神の知もまた諸事物の原因である。
ところが、神の知の側からは、何らのしるしも指定されていない。
ゆえに、神の意志の側からも、なんらかのしるしが指定される必要はない。


\\


{\scshape 2 Praeterea}, omne signum quod non concordat
ei cuius est signum, est falsum. Si igitur signa quae assignantur circa
voluntatem divinam, non concordant divinae voluntati, sunt falsa, si
autem concordant, superflue assignantur. Non igitur sunt aliqua signa
circa voluntatem divinam assignanda.


&


さらに、それのしるしであるところのものに一致しないすべてのしるしは、偽で
 ある。ゆえに、もし、神の意志をめぐって指定されるしるしが、神の意志に一
 致しないならば、それは偽であるし、もしも、一致するならば、その指定は空
 虚である。ゆえに、なんらかのしるしが神の意志をめぐって指定されるべき
 ではない。

\\


{\scshape Sed contra est} quod voluntas Dei est una,
cum ipsa sit Dei essentia. Quandoque autem pluraliter significatur, ut
cum dicitur, {\itshape magna opera domini, exquisita in omnes voluntates
eius}. Ergo oportet quod aliquando signum voluntatis pro voluntate
accipiatur.


&

しかし反対に、神の意志は、神の本質だから、一である。ところが、「主の業は
 偉大であり、神のすべての諸々の意志へと求められた」と言われるときのよう
 に、複数のかたちで表示されることもある。ゆえに、あるときには、意志のし
 るしが、意志として理解されなければならない。


\\


{\scshape Respondeo dicendum} quod in Deo quaedam
dicuntur proprie, et quaedam secundum metaphoram, ut ex supradictis
patet. Cum autem aliquae passiones humanae in divinam praedicationem
metaphorice assumuntur, hoc fit secundum similitudinem effectus, unde
illud quod est signum talis passionis in nobis, in Deo nomine illius
passionis metaphorice significatur. Sicut, apud nos, irati punire
consueverunt, unde ipsa punitio est signum irae, et propter hoc, ipsa
punitio nomine irae significatur, cum Deo attribuitur. 


&

解答する。以下のように言われるべきである。
前に述べられたことから明らかなとおり\footnote{{\itshape ST} I, q.~13,
 a.~3.}、神において、ある事柄は固有の意味で、またある事柄は比喩に即して
 語られる。ところで、人間のある情念が、神の述語へと、比喩的に取り入れら
 れるとき、そうされるのは結果の類似に即してである。したがって、私たちに
 おいてそのような情念のしるしであるものは、神において、その情念の名によっ
 て、比喩的に意味表示される。たとえば、私たちのもとで、怒る人々が罰する
 のが常であるので、罰する働きが、怒りのしるしである。そしてこのために、
 罰すること自体が、神に帰せられるとき、それは怒りという名で表示される。
 
\\


Similiter id quod
solet esse in nobis signum voluntatis, quandoque metaphorice in Deo
voluntas dicitur. Sicut, cum aliquis praecipit aliquid, signum est quod
velit illud fieri, unde praeceptum divinum quandoque metaphorice
{\itshape voluntas Dei} dicitur, secundum illud Matth.~{\scshape vi}, {\itshape fiat voluntas tua, sicut
in caelo et in terra}. 

&

同様に、私たちのもとで、意志のしるしであるのが常であるものが、比喩的に、
 神において意志と言われることがある。たとえば、ある人が何かを命じるとき、
 それは、それが為されることを意志していることのしるしである。したがって、
 神の命令が、比喩的に、「神の意志」と言われることがある。たとえば、かの
 『マタイによる福音書』六章「あな
 たの意志が、天においそうであるように、地においても為されますように」
 \footnote{「御心が行われますように、天におけるように地の上にも」(6:11)}
 のように。


\\

Sed hoc distat inter voluntatem et iram, quia ira
de Deo nunquam proprie dicitur, cum in suo principali intellectu
includat passionem, voluntas autem proprie de Deo dicitur. Et ideo in
Deo distinguitur voluntas proprie, et metaphorice dicta. Voluntas enim
proprie dicta, vocatur {\itshape voluntas beneplaciti}, voluntas autem metaphorice
dicta, est {\itshape voluntas signi}, eo quod ipsum signum voluntatis voluntas
dicitur.


&
しかし、意志と怒りは、次の点で異なっている。すなわち、怒りは、
 その主要な了解内容の中に、情念を含んでいるので、
 神について、けっして固有の意味では語られないが、意志は固有の意味で神に
 ついて語られる。それゆえ、神において、固有の意味で言われる意志と、比喩
 的に言われる意志とが区別される。
じっさい、固有の意味で言われる意志は、「喜びの意志(voluntas beneplaciti)」
 と呼ばれ、比喩的に言われる意志は、「しるしの意志」と呼ばれる。意志のし
 るしが、意志と言われるからである。



\\



{\scshape Ad primum ergo dicendum} quod scientia non
est causa eorum quae fiunt, nisi per voluntatem, non enim quae scimus
facimus, nisi velimus. Et ideo signum non attribuitur scientiae, sicut
attribuitur voluntati.


&

第一異論に対しては、それゆえ、以下のように言われるべきである。
知は、意志を通してでなければ、生じるものの原因ではない。たとえば、私たち
 は、それを意志するのでない限り、知っているものを作らない。ゆえに、「し
 るし」は知に帰せられないが、意志には帰せられる。


\\


{\scshape Ad secundum dicendum} quod signa voluntatis
dicuntur voluntates divinae, non quia sint signa quod Deus velit, sed
quia ea quae in nobis solent esse signa volendi, in Deo divinae
voluntates dicuntur. Sicut punitio non est signum quod in Deo sit ira,
sed punitio, ex eo ipso quod in nobis est signum irae, in Deo dicitur
ira.


&

第二異論に対しては、以下のように言われるべきである。
意志のしるしが神の意志と言われるのは、それが、神が意志していることのしるしだからでは
 なく、私たちのもとで、意志のしるしであるのが常であるようなことが、神に
 おいて、神の意志と言われるからである。たとえば、罰することは、神の中に
 怒りがあることのしるしではないが、むしろ、私たちのもとで、罰することが怒
 りのしるしであることに基づいて、神において怒りと言われる。



\end{longtable}
\newpage





\rhead{a.~12}
\begin{center}
 {\Large {\bf ARTICULUS DUODECIMUS}}\\
 {\large UTRUM CONVENITENTER CIRCA DIVINAM VOLUNTATEM PONANTUR QUINQUE SIGNA}\\
 {\footnotesize I {\itshape Sent.}, d.~45, a.~4; {\itshape De Verit.},
 q.~23, a.~3.}\\
 {\Large 第十二項\\神の意志について、五つのしるしが適切に措定されているか}
\end{center}

\begin{longtable}{p{21em}p{21em}}


{\huge A}{\scshape d duodecimum sic proceditur}. Videtur quod
inconvenienter circa divinam voluntatem ponantur quinque signa,
scilicet, {\itshape prohibitio, praeceptum, consilium, operatio et permissio}. Nam
eadem quae nobis {\itshape praecipit} Deus vel {\itshape consulit}, in nobis quandoque
{\itshape operatur}, et eadem quae {\itshape prohibet}, quandoque {\itshape permittit}. Ergo non debent
ex opposito dividi.


&

第十二項の問題へ、議論は以下のように進められる。
神の意志について、五つのしるし、すなわち「禁止」「命令」「勧告」「働き」
 「許し」が措定されるのは適切でないと思われる。というのも、神が私たちに命
 令し、勧告するその同じことを、神は私たちの中で働くこともあるし、また、禁止する同じ
 ことを、許すこともある。ゆえに、それらが対立的に分けられるべきではない。


\\


{\scshape 2 Praeterea}, nihil Deus operatur, nisi
volens, ut dicitur {\itshape Sap}.~{\scshape xi}. Sed voluntas signi distinguitur a voluntate
beneplaciti. Ergo {\itshape operatio} sub voluntate signi comprehendi non debet.


&


さらに、『知恵の書』11章で言われるように、神は、意志するのでない限り、働
 かない。ところが、しるしの意志は、喜びの意志から区別される。
ゆえに、「働き」が、しるしの意志のもとに含まれるべきではない。

\\


{\scshape 3 Praeterea}, {\itshape operatio} et {\itshape permissio}
communiter ad omnes creaturas pertinent, quia in omnibus Deus operatur,
et in omnibus aliquid fieri permittit. Sed {\itshape praeceptum}, {\itshape consilium} et
{\itshape prohibitio} pertinent ad solam rationalem creaturam. Ergo non veniunt
convenienter in unam divisionem, cum non sint unius ordinis.


&

さらに、「働き」と「許し」は、すべての被造物に共通に属する。なぜなら、神
 は万物の中で働き、万物の中に何かが生じることを許すからである。しかし、
 「命令」「勧告」「禁止」は、ただ理性的被造物だけに属する。ゆえに、それ
 らは一つの秩序に属していないから、一つの分割へと適切にまとまることがない。

\\


{\scshape 4 Praeterea}, malum pluribus modis contingit
quam bonum, quia bonum contingit uno modo, sed malum omnifariam, ut
patet per philosophum in II {\itshape Ethic}., et per Dionysium in {\scshape iv} cap.~{\itshape de
Div.~Nom}. Inconvenienter igitur respectu mali assignatur unum signum
tantum, scilicet {\itshape prohibitio}; respectu vero boni, duo signa, scilicet
{\itshape consilium} et {\itshape praeceptum}.


&


さらに、悪は、善よりも、複数のあり方で生じる。なぜなら、『ニコマコス倫理
 学』第2巻の哲学者や、『神名論』第4章のディオニュシウスによって明らかな
 とおり、善は一つのしかたで生じるが、悪はあらゆるしかたで生じるからであ
 る。
ゆえに、悪にかんして一つのしるしだけ、つまり、「禁止」だけが指定され、他
 方、善にかんして二つのしるし、つまり「勧告」と「命令」が指定されるのは
 不適切である。

\\


{\scshape Respondeo dicendum} quod huiusmodi signa
voluntatis dicuntur ea, quibus consuevimus demonstrare nos aliquid
velle. Potest autem aliquis declarare se velle aliquid, vel per seipsum,
vel per alium. Per seipsum quidem, inquantum facit aliquid, vel directe,
vel indirecte et per accidens. Directe quidem, cum per se aliquid
operatur, et quantum ad hoc, dicitur esse signum operatio. 


&

解答する。以下のように言われるべきである。
意志のしるしと言われるのは、それによって、私たちが何かを意志することを示
 すことが常であるようなものである。
さて、ある人が、自分が何かを意志していることを、自分自身によって示すこと
 もできるし、他のものによって示すこともできる。
自分自身によってというのは、何かを、直接または間接に、あるいは附帯的に、
 行うことによって[それを示す場合である]。
直接に、と言うのは、自分自身によって、何かの働きをなす場合であり、これに
 かんして、「働き」が、しるしと言われる。

\\


Indirecte
autem, inquantum non impedit operationem, nam removens prohibens dicitur
movens per accidens, ut dicitur in VIII {\itshape Physic}. Et quantum ad hoc,
dicitur signum {\itshape permissio}. Per alium autem declarat se aliquid velle,
inquantum ordinat alium ad aliquid faciendum; vel necessaria inductione,
quod fit {\itshape praecipiendo} quod quis vult, et {\itshape prohibendo} contrarium; vel
aliqua persuasoria inductione, quod pertinet ad {\itshape consilium}. 


&

間接に、と言うのは、働きを阻害しないかぎりでであり、たとえば、『自然学』第
 8巻で言われるように、邪魔を取り除くものが、附帯的に、動かすものと言われ
 る。
他方、他のものによって自分が何かを意志していることを示すのは、他のものを、
 何かを行うために秩序づける限りでである。これは、有無を言わさぬこととし
 てか、あるいは、説得して行うかであり、前者の場合、その人が意志している
 ことを「命令する」ことによってか、あるいは、反対のものを「禁止する」こ
 とによって行い、後者の場合、それは、「勧告」に属する。


\\


Quia igitur
his modis declaratur aliquem velle aliquid, propter hoc ista quinque
nominantur interdum nomine {\itshape voluntatis divinae}, tanquam signa
voluntatis. Quod enim praeceptum, consilium et prohibitio dicantur Dei
voluntas, patet per id quod dicitur Matth.~{\scshape vi}: {\itshape fiat voluntas tua, sicut
in caelo et in terra}. Quod autem permissio vel operatio dicantur Dei
voluntas patet per Augustinum, qui dicit in {\itshape Enchirid}., {\itshape nihil fit, nisi
omnipotens fieri velit, vel sinendo ut fiat, vel faciendo}. 


&

ゆえに、このようなしかたである人が何かを意志することが示されるので、この
 ため、この五つが意志のしるしとして、「神の意志」という名で呼ばれることがある。
つまり、命令、勧告、禁止が神の意志と言われることは、『マタイによる福音書』
 6章「あなたの意志が、天におけるように、地においても行われるように」と言
 われていることから明らかである。また、許しや働きが神の意志と言われるこ
 とは、『エンキリディオン』で「生じることを許すことによって
 であれ、それを為すことによってであれ、全能者が生じることを意志する
 のでないかぎり、何ものも生じない」と述
 べるアウグスティヌスによって明らかである。


\\


Vel potest
dici quod {\itshape permissio} et {\itshape operatio} referuntur ad praesens permissio quidem
ad malum, operatio vero ad bonum. Ad futurum vero, {\itshape prohibitio}, respectu
mali; respectu vero boni necessarii, {\itshape praeceptum}; respectu vero
superabundantis boni, {\itshape consilium}.


&

あるいは、次のように言われることもできる。「許し」と「働き」は、現在の事
 柄に関係するが、許しは悪にかんして、働きは善にかんしてある。他方、未来
 に対しては、悪にかんして「禁止」があり、必要な善にかんして「命令」が
 あり、必要以上の善については「勧告」がある。


\\


{\scshape Ad primum ergo dicendum} quod nihil prohibet,
circa eandem rem, aliquem diversimode declarare se aliquid velle, sicut
inveniuntur multa nomina idem significantia. Unde nihil prohibet idem
subiacere praecepto et consilio et operationi, et prohibitioni vel
permissioni.


&

第一異論に対しては、それゆえ、以下のように言われるべきである。
ちょうど、同一のものを意味する多くの名前が見出されるように、同一の事物に
 ついて、ある人が、自分が何かを
 意志することをさまざまなかたちで示すことを妨げるものは何もない。
したがって、同一のものが、命令と勧告と働きに従属すること、あるいは、禁止
 または許しに従属することを妨げるものは何もない。


\\


{\scshape Ad secundum dicendum} quod, sicut Deus potest
significari metaphorice velle id quod non vult voluntate proprie
accepta, ita potest metaphorice significari velle id quod proprie
vult. Unde nihil prohibet de eodem esse voluntatem beneplaciti, et
voluntatem signi. Sed operatio semper est eadem cum voluntate
beneplaciti, non autem praeceptum vel consilium, tum quia haec est de
praesenti, illud de futuro; tum quia haec per se est effectus
voluntatis, illud autem per alium, ut dictum est.


&

第二異論に対しては、以下のように言われるべきである。
ちょうど、厳密な意味で理解された意志によって、意志していないものを、比喩
 的に、神が意志すると表示されうるように、厳密な意味で意志しているものも、
 比喩的に、意志していると表示されうる。
このことから、同一のものについて、喜びの意志としるしの意志があることを、
 何も妨げない。
しかし、働きは、常に、喜びの意志と同一であるが、命令や勧告はそうでない。
 というのも、一つには、前者[=働き]は現在にかかわるが、後者[=命令または勧告]
 は未来にかかわるからであり、また一つには、前者は、それ自体によって意志
 の結果であるが、後者は、他のものを通してそうだからである。


\\


{\scshape Ad tertium dicendum} quod creatura rationalis
est domina sui actus, et ideo circa ipsam specialia quaedam signa
divinae voluntatis assignantur, inquantum rationalem creaturam Deus
ordinat ad agendum voluntarie et per se. Sed aliae creaturae non agunt
nisi motae ex operatione divina, et ideo circa alias non habent locum
nisi operatio et permissio.

&

第三異論に対しては、以下のように言われるべきである。
理性的被造物は自らの行為の主である。ゆえに、神が理性的被造物を、意志的に、
 それ自体によって働くように秩序づけている限りで、理性的被造物をめぐって
 は、ある種特別な、神の意志のしるしが指定されている。しかし、他の被造物
 は、神の働きによって動かされる限りでなければ働かないので、それら他の被
 造物をめぐっては、働きと許し以外のものはない。


\\


{\scshape Ad quartum dicendum} quod omne malum culpae, licet
 multipliciter contingat, tamen in hoc convenit, quod discordat a
 voluntate divina et ideo unum signum respectu malorum assignatur,
 scilicet {\itshape prohibitio}. Sed diversimode bona se habent ad bonitatem
 divinam. Quia quaedam sunt, sine quibus fruitionem divinae bonitatis
 consequi non possumus, et respectu horum est {\itshape praeceptum}. Quaedam vero
 sunt, quibus perfectius consequimur, et respectu horum est
 {\itshape consilium}. --Vel dicendum quod {\itshape consilium} est non solum de melioribus
 bonis assequendis, sed etiam de minoribus malis vitandis.


&

第四異論に対しては、以下のように言われるべきである。
すべて罪の悪は、たしかに多くのしかたで生じるけれども、神の意志と調和しな
 いという点で一致する。ゆえに、諸悪については一つのしるし、つまり「禁止」
 が指定される。しかし、善は、さまざまなしかたで神の善性に関係する。理由
 は以下。あるものどもは、それなしには私たちが神の善性の享受を獲得することが不可
 能なものであり、これらについては「命令」がある。またあるものどもは、そ
 れによって、私たちがより完全にそれを獲得するのであり、これらについては
 「勧告」がある。あるいは、以下のように言われるべきである。「勧告」は、
 獲得されるべきよりよい善についてだけでなく、避けられるべきより小さい悪
 についてもある。


\end{longtable}
\end{document}