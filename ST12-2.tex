\documentclass[10pt]{jsarticle} % use larger type; default would be 10pt
%\usepackage[utf8]{inputenc} % set input encoding (not needed with XeLaTeX)
%\usepackage[round,comma,authoryear]{natbib}
%\usepackage{nruby}
\usepackage{okumacro}
\usepackage{longtable}
%\usepqckage{tablefootnote}
\usepackage[polutonikogreek,english,japanese]{babel}
%\usepackage{amsmath}
\usepackage{latexsym}
\usepackage{color}

%----- header -------
\usepackage{fancyhdr}
\lhead{{\it Summa Theologiae} I-IIae, q.~2}
%--------------------

\bibliographystyle{jplain}

\title{{\bf PRIMA SECUNDAE}\\{\HUGE Summae Theologiae}\\Sancti Thomae
Aquinatis}
\author{Japanese translation\\by Yoshinori {\sc Ueeda}}
\date{Last modified \today}


%%%% コピペ用
%\rhead{a.~}
%\begin{center}
% {\Large {\bf }}\\
% {\large }\\
% {\footnotesize }\\
% {\Large \\}
%\end{center}
%
%\begin{longtable}{p{21em}p{21em}}
%
%&
%
%
%\\
%\end{longtable}
%\newpage







\begin{document}
\maketitle
\pagestyle{fancy}

\rhead{Prologos}

\begin{center}
 {\Large {\bf QUAESTIO SECUNDA}}\\
 {\large DE HIS IN QUIBUS HOMINIS BEATITUDO CONSISTIT}\\

 {\Large 第二問\\人間の至福が成立するものについて}
\end{center}

\begin{longtable}{p{21em}p{21em}}
Deinde considerandum est de beatitudine, nam beatitudo nominat
 adeptionem ultimi finis\footnote{Leo版には、namからfinisまでがない。}. Primo quidem, in quibus sit; secundo, quid
 sit; tertio, qualiter eam consequi possimus. Circa primum quaeruntur
 octo.

 \begin{enumerate}
  \item utrum beatitudo consistat in divitiis.
  \item utrum in honoribus.
  \item utrum in fama, sive in gloria.
  \item utrum in potestate.
  \item utrum in aliquo corporis bono. 
  \item utrum in voluptate.
  \item utrum in aliquo bono animae.
  \item utrum in aliquo bono creato.
 \end{enumerate}
 

 
&

 次いで、至福について考察されるべきである。なぜなら、至福とは究極目的の
 獲得のことを言うからである。第一に、何においてあるか、第二に、それは何
 か、第三に、私たちはどのようにしてそれを獲得できるか。第一をめぐって、
 四つのことが問われる。
 \begin{enumerate}
  \item 至福は富においてあるか。
  \item 名誉においてあるか。
  \item 名声ないし栄光においてあるか。
  \item 権力においてあるか。
  \item 身体の何らかの善においてあるか。
  \item 快楽においてあるか。
  \item 魂の何らかの善においてあるか。
  \item 何らかの被造の善においてあるか。
 \end{enumerate}

\end{longtable}

\newpage
\rhead{a.1}
\begin{center}
 {\Large {\bf ARTICULUS PRIMUS}}\\
 {\large UTRUM BEATITUDO HOMINIS CONSISTAT IN DIVITIIS}\\
 {\footnotesize III {\itshape SCG}, cap.30; I {\itshape Ethic.}, lect.5.}\\
 {\Large 第一項\\人間の至福は富においてあるか}
\end{center}

\begin{longtable}{p{21em}p{21em}}
{\scshape  Ad primum sic proceditur}. Videtur quod
 beatitudo hominis in divitiis consistat. Cum enim beatitudo sit ultimus
 finis hominis, in eo consistit quod maxime in hominis affectu
 dominatur. Huiusmodi autem sunt divitiae, dicitur enim Eccle. {\scshape x},
 {\itshape pecuniae obediunt omnia}. Ergo in divitiis beatitudo hominis consistit.

 
&

 第一項の問題へ、議論は以下のように進められる。
 人間の 至福は富において成立すると思われる。理由は以下の通り。
 至福は、人間の究極目的だから、人間の情念において、最大限に支配されると
 ころにおいて成立する。ところが、富はそのようなものである。なぜなら、
 『コヘレトの言葉』第10章で「万物は金銭に従う」\footnote{「銀はすべてに
 こたえてくれる」(10:19)}と言われているからである。
 ゆえに、富において、人間の至福は成立する。

\\




{\scshape 2 Praeterea}, secundum Boetium, in III {\itshape de
 Consol}., beatitudo est {\itshape status omnium bonorum aggregatione
 perfectus}. Sed in pecuniis omnia possideri videntur, quia, ut
 philosophus dicit in V {\itshape Ethic}., ad hoc nummus est inventus, ut sit quasi
 fideiussor habendi pro eo quodcumque homo voluerit. Ergo in divitiis
 beatitudo consistit.

 
&

 さらに、『哲学の慰め』第3巻にけるボエティウスによると、至福は「すべての
 善を集めて完全な状態」である。ところが、富において、すべてのものが所有
 されると思われる。なぜなら、哲学者が『ニコマコス倫理学』第5巻で述べてい
 るように、何であれ人間が欲することに対して、言わば保証人をもつために、
 金銭は発明されたからである。ゆえに、富において、至福が成立する。
 
 

\\




{\scshape 3 Praeterea}, desiderium summi boni, cum
 nunquam deficiat, videtur esse infinitum. Sed hoc maxime in divitiis
 invenitur, quia avarus non implebitur pecunia, ut dicitur
 Eccle. V. Ergo in divitiis beatitudo consistit.

 
&

さらに、最高善への欲望は、けっして減ることがないので無限であると思われる。
しかるに、このことは最大限に心的な事柄において見出される。なぜなら「神的
 な事柄において至福が成立する」


\\




[33472] Iª-IIae q. 2 a. 1 s. c. Sed contra, bonum hominis in retinendo
 beatitudinem magis consistit quam in emittendo ipsam. Sed sicut Boetius
 in II de Consol. dicit, divitiae effundendo, magis quam coacervando,
 melius nitent, siquidem avaritia semper odiosos, claros largitas
 facit. Ergo in divitiis beatitudo non consistit.

 
&


\\




[33473] Iª-IIae q. 2 a. 1 co. Respondeo dicendum quod impossibile est
 beatitudinem hominis in divitiis consistere. Sunt enim duplices
 divitiae, ut philosophus dicit in I Polit., scilicet naturales, et
 artificiales. Naturales quidem divitiae sunt, quibus homini subvenitur
 ad defectus naturales tollendos, sicut cibus, potus, vestimenta,
 vehicula et habitacula, et alia huiusmodi. Divitiae autem artificiales
 sunt, quibus secundum se natura non iuvatur, ut denarii; sed ars humana
 eos adinvenit propter facilitatem commutationis, ut sint quasi mensura
 quaedam rerum venalium. Manifestum est autem quod in divitiis
 naturalibus beatitudo hominis esse non potest. Quaeruntur enim
 huiusmodi divitiae propter aliud, scilicet ad sustentandam naturam
 hominis, et ideo non possunt esse ultimus finis hominis, sed magis
 ordinantur ad hominem sicut ad finem. Unde in ordine naturae omnia
 huiusmodi sunt infra hominem, et propter hominem facta; secundum illud
 Psalmi VIII, omnia subiecisti sub pedibus eius. Divitiae autem
 artificiales non quaeruntur nisi propter naturales, non enim
 quaererentur, nisi quia per eas emuntur res ad usum vitae
 necessariae. Unde multo minus habent rationem ultimi finis. Impossibile
 est igitur beatitudinem, quae est ultimus finis hominis, in divitiis
 esse.

 
&


\\




[33474] Iª-IIae q. 2 a. 1 ad 1 Ad primum ergo dicendum quod omnia
 corporalia obediunt pecuniae, quantum ad multitudinem stultorum, qui
 sola corporalia bona cognoscunt, quae pecunia acquiri possunt. Iudicium
 autem de bonis humanis non debet sumi a stultis, sed a sapientibus,
 sicut et iudicium de saporibus ab his qui habent gustum bene
 dispositum.

 
&


\\




[33475] Iª-IIae q. 2 a. 1 ad 2 Ad secundum dicendum quod pecunia possunt
 haberi omnia venalia, non autem spiritualia, quae vendi non
 possunt. Unde dicitur Proverb. XVII, quid prodest stulto divitias
 habere, cum sapientiam emere non possit?
 
&


\\




[33476] Iª-IIae q. 2 a. 1 ad 3 Ad tertium dicendum quod appetitus
 naturalium divitiarum non est infinitus, quia secundum certam mensuram
 naturae sufficiunt. Sed appetitus divitiarum artificialium est
 infinitus, quia deservit concupiscentiae inordinatae, quae non
 modificatur, ut patet per philosophum in I Polit. Aliter tamen est
 infinitum desiderium divitiarum, et desiderium summi boni. Nam summum
 bonum quanto perfectius possidetur, tanto ipsummet magis amatur, et
 alia contemnuntur, quia quanto magis habetur, magis cognoscitur. Et
 ideo dicitur Eccli. XXIV, qui edunt me, adhuc esurient. Sed in appetitu
 divitiarum, et quorumcumque temporalium bonorum, est e converso, nam
 quando iam habentur, ipsa contemnuntur, et alia appetuntur; secundum
 quod significatur Ioan. IV, cum dominus dicit, qui bibit ex hac aqua,
 per quam temporalia significantur, sitiet iterum. Et hoc ideo, quia
 eorum insufficientia magis cognoscitur cum habentur. Et ideo hoc ipsum
 ostendit eorum imperfectionem, et quod in eis summum bonum non
 consistit.
 
&



\end{longtable}


\newpage


\begin{longtable}{p{21em}p{21em}}


&

\end{longtable}


\end{document}




Articulus 1


Articulus 2

[33477] Iª-IIae q. 2 a. 2 arg. 1 Ad secundum sic proceditur. Videtur quod beatitudo hominis in honoribus consistat. Beatitudo enim, sive felicitas, est praemium virtutis, ut philosophus dicit in I Ethic. Sed honor maxime videtur esse id quod est virtutis praemium, ut philosophus dicit in IV Ethic. Ergo in honore maxime consistit beatitudo.

[33478] Iª-IIae q. 2 a. 2 arg. 2 Praeterea, illud quod convenit Deo et excellentissimis, maxime videtur esse beatitudo, quae est bonum perfectum. Sed huiusmodi est honor, ut philosophus dicit in IV Ethic. Et etiam I Tim. I, dicit apostolus, soli Deo honor et gloria. Ergo in honore consistit beatitudo.

[33479] Iª-IIae q. 2 a. 2 arg. 3 Praeterea, illud quod est maxime desideratum ab hominibus, est beatitudo. Sed nihil videtur esse magis desiderabile ab hominibus quam honor, quia homines patiuntur iacturam in omnibus aliis rebus ne patiantur aliquod detrimentum sui honoris. Ergo in honore beatitudo consistit.

[33480] Iª-IIae q. 2 a. 2 s. c. Sed contra, beatitudo est in beato. Honor autem non est in eo qui honoratur, sed magis in honorante, qui reverentiam exhibet honorato, ut philosophus dicit in I Ethic. Ergo in honore beatitudo non consistit.

[33481] Iª-IIae q. 2 a. 2 co. Respondeo dicendum quod impossibile est beatitudinem consistere in honore. Honor enim exhibetur alicui propter aliquam eius excellentiam; et ita est signum et testimonium quoddam illius excellentiae quae est in honorato. Excellentia autem hominis maxime attenditur secundum beatitudinem, quae est hominis bonum perfectum; et secundum partes eius, idest secundum illa bona quibus aliquid beatitudinis participatur. Et ideo honor potest quidem consequi beatitudinem, sed principaliter in eo beatitudo consistere non potest.

[33482] Iª-IIae q. 2 a. 2 ad 1 Ad primum ergo dicendum quod, sicut philosophus ibidem dicit, honor non est praemium virtutis propter quod virtuosi operantur, sed accipiunt honorem ab hominibus loco praemii, quasi a non habentibus aliquid maius ad dandum. Verum autem praemium virtutis est ipsa beatitudo, propter quam virtuosi operantur. Si autem propter honorem operarentur, iam non esset virtus, sed magis ambitio.

[33483] Iª-IIae q. 2 a. 2 ad 2 Ad secundum dicendum quod honor debetur Deo et excellentissimis, in signum vel testimonium excellentiae praeexistentis, non quod ipse honor faciat eos excellentes.

[33484] Iª-IIae q. 2 a. 2 ad 3 Ad tertium dicendum quod ex naturali desiderio beatitudinis, quam consequitur honor, ut dictum est, contingit quod homines maxime honorem desiderant. Unde quaerunt homines maxime honorari a sapientibus, quorum iudicio credunt se esse excellentes vel felices.

Articulus 3

[33485] Iª-IIae q. 2 a. 3 arg. 1 Ad tertium sic proceditur. Videtur quod beatitudo hominis consistat in gloria. In eo enim videtur beatitudo consistere, quod redditur sanctis pro tribulationibus quas in mundo patiuntur. Huiusmodi autem est gloria, dicit enim apostolus, Rom. VIII, non sunt condignae passiones huius temporis ad futuram gloriam, quae revelabitur in nobis. Ergo beatitudo consistit in gloria.

[33486] Iª-IIae q. 2 a. 3 arg. 2 Praeterea, bonum est diffusivum sui, ut patet per Dionysium, IV cap. de Div. Nom. Sed per gloriam bonum hominis maxime diffunditur in notitiam aliorum, quia gloria, ut Ambrosius dicit, nihil aliud est quam clara cum laude notitia. Ergo beatitudo hominis consistit in gloria.

[33487] Iª-IIae q. 2 a. 3 arg. 3 Praeterea, beatitudo est stabilissimum bonorum. Hoc autem videtur esse fama vel gloria, quia per hanc quodammodo homines aeternitatem sortiuntur. Unde Boetius dicit, in libro de Consol., vos immortalitatem vobis propagare videmini, cum futuri famam temporis cogitatis. Ergo beatitudo hominis consistit in fama seu gloria.

[33488] Iª-IIae q. 2 a. 3 s. c. Sed contra, beatitudo est verum hominis bonum. Sed famam seu gloriam contingit esse falsam, ut enim dicit Boetius, in libro III de Consol., plures magnum saepe nomen falsis vulgi opinionibus abstulerunt. Quo quid turpius excogitari potest? Nam qui falso praedicantur, suis ipsi necesse est laudibus erubescant. Non ergo beatitudo hominis consistit in fama seu gloria.

[33489] Iª-IIae q. 2 a. 3 co. Respondeo dicendum quod impossibile est beatitudinem hominis in fama seu gloria humana consistere. Nam gloria nihil aliud est quam clara notitia cum laude, ut Ambrosius dicit. Res autem cognita aliter comparatur ad cognitionem humanam, et aliter ad cognitionem divinam, humana enim cognitio a rebus cognitis causatur, sed divina cognitio est causa rerum cognitarum. Unde perfectio humani boni, quae beatitudo dicitur, non potest causari a notitia humana, sed magis notitia humana de beatitudine alicuius procedit et quodammodo causatur ab ipsa humana beatitudine, vel inchoata vel perfecta. Et ideo in fama vel in gloria non potest consistere hominis beatitudo. Sed bonum hominis dependet, sicut ex causa, ex cognitione Dei. Et ideo ex gloria quae est apud Deum, dependet beatitudo hominis sicut ex causa sua, secundum illud Psalmi XC, eripiam eum, et glorificabo eum, longitudine dierum replebo eum, et ostendam illi salutare meum. Est etiam aliud considerandum, quod humana notitia saepe fallitur, et praecipue in singularibus contingentibus, cuiusmodi sunt actus humani. Et ideo frequenter humana gloria fallax est. Sed quia Deus falli non potest, eius gloria semper vera est. Propter quod dicitur, II ad Cor. X, ille probatus est, quem Deus commendat.

[33490] Iª-IIae q. 2 a. 3 ad 1 Ad primum ergo dicendum quod apostolus non loquitur ibi de gloria quae est ab hominibus, sed de gloria quae est a Deo coram Angelis eius. Unde dicitur Marc. VIII, filius hominis confitebitur eum in gloria patris sui, coram Angelis eius.

[33491] Iª-IIae q. 2 a. 3 ad 2 Ad secundum dicendum quod bonum alicuius hominis quod per famam vel gloriam est in cognitione multorum, si cognitio quidem vera sit, oportet quod derivetur a bono existente in ipso homine, et sic praesupponit beatitudinem perfectam vel inchoatam. Si autem cognitio falsa sit, non concordat rei, et sic bonum non invenitur in eo cuius fama celebris habetur. Unde patet quod fama nullo modo potest facere hominem beatum.

[33492] Iª-IIae q. 2 a. 3 ad 3 Ad tertium dicendum quod fama non habet stabilitatem, immo falso rumore de facili perditur. Et si stabilis aliquando perseveret, hoc est per accidens. Sed beatitudo habet per se stabilitatem, et semper.

Articulus 4

[33493] Iª-IIae q. 2 a. 4 arg. 1 Ad quartum sic proceditur. Videtur quod beatitudo consistat in potestate. Omnia enim appetunt assimilari Deo, tanquam ultimo fini et primo principio. Sed homines qui in potestatibus sunt, propter similitudinem potestatis, maxime videntur esse Deo conformes, unde et in Scriptura dii vocantur, ut patet Exod. XXII, diis non detrahes. Ergo in potestate beatitudo consistit.

[33494] Iª-IIae q. 2 a. 4 arg. 2 Praeterea, beatitudo est bonum perfectum. Sed perfectissimum est quod homo etiam alios regere possit, quod convenit his qui in potestatibus sunt constituti. Ergo beatitudo consistit in potestate.

[33495] Iª-IIae q. 2 a. 4 arg. 3 Praeterea, beatitudo, cum sit maxime appetibilis, opponitur ei quod maxime est fugiendum. Sed homines maxime fugiunt servitutem, cui contraponitur potestas. Ergo in potestate beatitudo consistit.

[33496] Iª-IIae q. 2 a. 4 s. c. Sed contra, beatitudo est perfectum bonum. Sed potestas est maxime imperfecta. Ut enim dicit Boetius, III de Consol., potestas humana sollicitudinum morsus expellere, formidinum aculeos vitare nequit. Et postea, potentem censes cui satellites latus ambiunt qui quos terret, ipse plus metuit? Non igitur beatitudo consistit in potestate.

[33497] Iª-IIae q. 2 a. 4 co. Respondeo dicendum quod impossibile est beatitudinem in potestate consistere, propter duo. Primo quidem, quia potestas habet rationem principii, ut patet in V Metaphys. Beatitudo autem habet rationem ultimi finis. Secundo, quia potestas se habet ad bonum et ad malum. Beatitudo autem est proprium et perfectum hominis bonum. Unde magis posset consistere beatitudo aliqua in bono usu potestatis, qui est per virtutem, quam in ipsa potestate. Possunt autem quatuor generales rationes induci ad ostendendum quod in nullo praemissorum exteriorum bonorum beatitudo consistat. Quarum prima est quia, cum beatitudo sit summum hominis bonum, non compatitur secum aliquod malum. Omnia autem praedicta possunt inveniri et in bonis et in malis. Secunda ratio est quia, cum de ratione beatitudinis sit quod sit per se sufficiens, ut patet in I Ethic., necesse est quod, beatitudine adepta, nullum bonum homini necessarium desit. Adeptis autem singulis praemissorum, possunt adhuc multa bona homini necessaria deesse, puta sapientia, sanitas corporis, et huiusmodi. Tertia, quia, cum beatitudo sit bonum perfectum, ex beatitudine non potest aliquod malum alicui provenire. Quod non convenit praemissis, dicitur enim Eccle. V, quod divitiae interdum conservantur in malum domini sui; et simile patet in aliis tribus. Quarta ratio est quia ad beatitudinem homo ordinatur per principia interiora, cum ad ipsam naturaliter ordinetur. Praemissa autem quatuor bona magis sunt a causis exterioribus, et ut plurimum a fortuna, unde et bona fortunae dicuntur. Unde patet quod in praemissis nullo modo beatitudo consistit.

[33498] Iª-IIae q. 2 a. 4 ad 1 Ad primum ergo dicendum quod divina potestas est sua bonitas, unde uti sua potestate non potest nisi bene. Sed hoc in hominibus non invenitur. Unde non sufficit ad beatitudinem hominis quod assimiletur Deo quantum ad potestatem, nisi etiam assimiletur ei quantum ad bonitatem.

[33499] Iª-IIae q. 2 a. 4 ad 2 Ad secundum dicendum quod, sicut optimum est quod aliquis utatur bene potestate in regimine multorum, ita pessimum est si male utatur. Et ita potestas se habet et ad bonum et ad malum.

[33500] Iª-IIae q. 2 a. 4 ad 3 Ad tertium dicendum quod servitus est impedimentum boni usus potestatis, et ideo naturaliter homines eam fugiunt, et non quasi in potestate hominis sit summum bonum.

Articulus 5

[33501] Iª-IIae q. 2 a. 5 arg. 1 Ad quintum sic proceditur. Videtur quod beatitudo hominis consistat in bonis corporis. Dicitur enim Eccli. XXX, non est census supra censum salutis corporis. Sed in eo quod est optimum, consistit beatitudo. Ergo consistit in corporis salute.

[33502] Iª-IIae q. 2 a. 5 arg. 2 Praeterea, Dionysius dicit, V cap. de Div. Nom., quod esse est melius quam vivere, et vivere melius quam alia quae consequuntur. Sed ad esse et vivere hominis requiritur salus corporis. Cum ergo beatitudo sit summum bonum hominis, videtur quod salus corporis maxime pertineat ad beatitudinem.

[33503] Iª-IIae q. 2 a. 5 arg. 3 Praeterea, quanto aliquid est communius, tanto ab altiori principio dependet quia quanto causa est superior, tanto eius virtus ad plura se extendit. Sed sicut causalitas causae efficientis consideratur secundum influentiam, ita causalitas finis attenditur secundum appetitum. Ergo sicut prima causa efficiens est quae in omnia influit, ita ultimus finis est quod ab omnibus desideratur. Sed ipsum esse est quod maxime desideratur ab omnibus. Ergo in his quae pertinent ad esse hominis, sicut est salus corporis, maxime consistit eius beatitudo.

[33504] Iª-IIae q. 2 a. 5 s. c. Sed contra, secundum beatitudinem homo excellit omnia alia animalia. Sed secundum bona corporis, a multis animalibus superatur, sicut ab elephante in diuturnitate vitae, a leone in fortitudine, a cervo in cursu. Ergo beatitudo hominis non consistit in bonis corporis.

[33505] Iª-IIae q. 2 a. 5 co. Respondeo dicendum quod impossibile est beatitudinem hominis in bonis corporis consistere, propter duo. Primo quidem, quia impossibile est quod illius rei quae ordinatur ad aliud sicut ad finem, ultimus finis sit eiusdem conservatio in esse. Unde gubernator non intendit, sicut ultimum finem, conservationem navis sibi commissae; eo quod navis ad aliud ordinatur sicut ad finem, scilicet ad navigandum. Sicut autem navis committitur gubernatori ad dirigendum, ita homo est suae voluntati et rationi commissus; secundum illud quod dicitur Eccli. XV, Deus ab initio constituit hominem, et reliquit eum in manu consilii sui. Manifestum est autem quod homo ordinatur ad aliquid sicut ad finem, non enim homo est summum bonum. Unde impossibile est quod ultimus finis rationis et voluntatis humanae sit conservatio humani esse. Secundo quia, dato quod finis rationis et voluntatis humanae esset conservatio humani esse, non tamen posset dici quod finis hominis esset aliquod corporis bonum. Esse enim hominis consistit in anima et corpore, et quamvis esse corporis dependeat ab anima, esse tamen humanae animae non dependet a corpore, ut supra ostensum est; ipsumque corpus est propter animam, sicut materia propter formam, et instrumenta propter motorem, ut per ea suas actiones exerceat. Unde omnia bona corporis ordinantur ad bona animae, sicut ad finem. Unde impossibile est quod in bonis corporis beatitudo consistat, quae est ultimus hominis finis.

[33506] Iª-IIae q. 2 a. 5 ad 1 Ad primum ergo dicendum quod, sicut corpus ordinatur ad animam sicut ad finem, ita bona exteriora ad ipsum corpus. Et ideo rationabiliter bonum corporis praefertur bonis exterioribus, quae per censum significantur, sicut et bonum animae praefertur omnibus bonis corporis.

[33507] Iª-IIae q. 2 a. 5 ad 2 Ad secundum dicendum quod esse simpliciter acceptum, secundum quod includit in se omnem perfectionem essendi, praeeminet vitae et omnibus subsequentibus, sic enim ipsum esse praehabet in se omnia subsequentia. Et hoc modo Dionysius loquitur. Sed si consideretur ipsum esse prout participatur in hac re vel in illa, quae non capiunt totam perfectionem essendi, sed habent esse imperfectum, sicut est esse cuiuslibet creaturae; sic manifestum est quod ipsum esse cum perfectione superaddita est eminentius. Unde et Dionysius ibidem dicit quod viventia sunt meliora existentibus, et intelligentia viventibus.

[33508] Iª-IIae q. 2 a. 5 ad 3 Ad tertium dicendum quod, quia finis respondet principio, ex illa ratione probatur quod ultimus finis est primum principium essendi, in quo est omnis essendi perfectio, cuius similitudinem appetunt, secundum suam proportionem, quaedam quidem secundum esse tantum, quaedam secundum esse vivens, quaedam secundum esse vivens et intelligens et beatum. Et hoc paucorum est.

Articulus 6

[33509] Iª-IIae q. 2 a. 6 arg. 1 Ad sextum sic proceditur. Videtur quod beatitudo hominis in voluptate consistat. Beatitudo enim, cum sit ultimus finis, non appetitur propter aliud, sed alia propter ipsam. Sed hoc maxime convenit delectationi, ridiculum est enim ab aliquo quaerere propter quid velit delectari, ut dicitur in X Ethic. Ergo beatitudo maxime in voluptate et delectatione consistit.

[33510] Iª-IIae q. 2 a. 6 arg. 2 Praeterea, causa prima vehementius imprimit quam secunda, ut dicitur in libro de causis. Influentia autem finis attenditur secundum eius appetitum. Illud ergo videtur habere rationem finis ultimi, quod maxime movet appetitum. Hoc autem est voluptas, cuius signum est quod delectatio intantum absorbet hominis voluntatem et rationem, quod alia bona contemnere facit. Ergo videtur quod ultimus finis hominis, qui est beatitudo, maxime in voluptate consistat.

[33511] Iª-IIae q. 2 a. 6 arg. 3 Praeterea, cum appetitus sit boni, illud quod omnia appetunt, videtur esse optimum. Sed delectationem omnia appetunt, et sapientes et insipientes, et etiam ratione carentia. Ergo delectatio est optimum. Consistit ergo in voluptate beatitudo, quae est summum bonum.

[33512] Iª-IIae q. 2 a. 6 s. c. Sed contra est quod Boetius dicit, in III de Consol., tristes exitus esse voluptatum, quisquis reminisci libidinum suarum volet, intelliget. Quae si beatos efficere possent, nihil causae est quin pecudes quoque beatae esse dicantur.

[33513] Iª-IIae q. 2 a. 6 co. Respondeo dicendum quod, quia delectationes corporales pluribus notae sunt, assumpserunt sibi nomen voluptatum, ut dicitur VII Ethic., cum tamen sint aliae delectationes potiores. In quibus tamen beatitudo principaliter non consistit. Quia in unaquaque re aliud est quod pertinet ad essentiam eius, aliud est proprium accidens ipsius, sicut in homine aliud est quod est animal rationale mortale, aliud quod est risibile. Est igitur considerandum quod omnis delectatio est quoddam proprium accidens quod consequitur beatitudinem, vel aliquam beatitudinis partem, ex hoc enim aliquis delectatur quod habet bonum aliquod sibi conveniens, vel in re, vel in spe, vel saltem in memoria. Bonum autem conveniens, si quidem sit perfectum, est ipsa hominis beatitudo si autem sit imperfectum est quaedam beatitudinis participatio, vel propinqua, vel remota, vel saltem apparens. Unde manifestum est quod nec ipsa delectatio quae consequitur bonum perfectum, est ipsa essentia beatitudinis; sed quoddam consequens ad ipsam sicut per se accidens. Voluptas autem corporalis non potest etiam modo praedicto sequi bonum perfectum. Nam sequitur bonum quod apprehendit sensus, qui est virtus animae corpore utens. Bonum autem quod pertinet ad corpus, quod apprehenditur secundum sensum, non potest esse perfectum hominis bonum. Cum enim anima rationalis excedat proportionem materiae corporalis, pars animae quae est ab organo corporeo absoluta, quandam habet infinitatem respectu ipsius corporis et partium animae corpori concretarum, sicut immaterialia sunt quodammodo infinita respectu materialium, eo quod forma per materiam quodammodo contrahitur et finitur, unde forma a materia absoluta est quodammodo infinita. Et ideo sensus, qui est vis corporalis, cognoscit singulare, quod est determinatum per materiam, intellectus vero, qui est vis a materia absoluta, cognoscit universale, quod est abstractum a materia, et continet sub se infinita singularia. Unde patet quod bonum conveniens corpori, quod per apprehensionem sensus delectationem corporalem causat, non est perfectum bonum hominis, sed est minimum quiddam in comparatione ad bonum animae. Unde Sap. VII, dicitur quod omne aurum, in comparatione sapientiae, arena est exigua. Sic igitur neque voluptas corporalis est ipsa beatitudo, nec est per se accidens beatitudinis.

[33514] Iª-IIae q. 2 a. 6 ad 1 Ad primum ergo dicendum quod eiusdem rationis est quod appetatur bonum, et quod appetatur delectatio, quae nihil est aliud quam quietatio appetitus in bono, sicut ex eadem virtute naturae est quod grave feratur deorsum, et quod ibi quiescat. Unde sicut bonum propter seipsum appetitur, ita et delectatio propter se, et non propter aliud appetitur, si ly propter dicat causam finalem. Si vero dicat causam formalem, vel potius motivam, sic delectatio est appetibilis propter aliud, idest propter bonum, quod est delectationis obiectum, et per consequens est principium eius, et dat ei formam, ex hoc enim delectatio habet quod appetatur, quia est quies in bono desiderato.

[33515] Iª-IIae q. 2 a. 6 ad 2 Ad secundum dicendum quod vehemens appetitus delectationis sensibilis contingit ex hoc quod operationes sensuum, quia sunt principia nostrae cognitionis, sunt magis perceptibiles. Unde etiam a pluribus delectationes sensibiles appetuntur.

[33516] Iª-IIae q. 2 a. 6 ad 3 Ad tertium dicendum quod eo modo omnes appetunt delectationem, sicut et appetunt bonum, et tamen delectationem appetunt ratione boni, et non e converso, ut dictum est. Unde non sequitur quod delectatio sit maximum et per se bonum, sed quod unaquaeque delectatio consequatur aliquod bonum, et quod aliqua delectatio consequatur id quod est per se et maximum bonum.

Articulus 7

[33517] Iª-IIae q. 2 a. 7 arg. 1 Ad septimum sic proceditur. Videtur quod beatitudo consistat in aliquo bono animae. Beatitudo enim est quoddam hominis bonum. Hoc autem per tria dividitur, quae sunt bona exteriora, bona corporis, et bona animae. Sed beatitudo non consistit in bonis exterioribus, neque in bonis corporis, sicut supra ostensum est. Ergo consistit in bonis animae.

[33518] Iª-IIae q. 2 a. 7 arg. 2 Praeterea, illud cui appetimus aliquod bonum, magis amamus quam bonum quod ei appetimus, sicut magis amamus amicum cui appetimus pecuniam, quam pecuniam. Sed unusquisque quodcumque bonum sibi appetit. Ergo seipsum amat magis quam omnia alia bona. Sed beatitudo est quod maxime amatur, quod patet ex hoc quod propter ipsam omnia alia amantur et desiderantur. Ergo beatitudo consistit in aliquo bono ipsius hominis. Sed non in bonis corporis. Ergo in bonis animae.

[33519] Iª-IIae q. 2 a. 7 arg. 3 Praeterea, perfectio est aliquid eius quod perficitur. Sed beatitudo est quaedam perfectio hominis. Ergo beatitudo est aliquid hominis. Sed non est aliquid corporis, ut ostensum est. Ergo beatitudo est aliquid animae. Et ita consistit in bonis animae.

[33520] Iª-IIae q. 2 a. 7 s. c. Sed contra, sicut Augustinus dicit in libro de Doctr. Christ., id in quo constituitur beata vita, propter se diligendum est. Sed homo non est propter seipsum diligendus, sed quidquid est in homine, est diligendum propter Deum. Ergo in nullo bono animae beatitudo consistit.

[33521] Iª-IIae q. 2 a. 7 co. Respondeo dicendum quod, sicut supra dictum est, finis dupliciter dicitur, scilicet ipsa res quam adipisci desideramus; et usus, seu adeptio aut possessio illius rei. Si ergo loquamur de ultimo fine hominis quantum ad ipsam rem quam appetimus sicut ultimum finem, impossibile est quod ultimus finis hominis sit ipsa anima, vel aliquid eius. Ipsa enim anima, in se considerata, est ut in potentia existens, fit enim de potentia sciente actu sciens, et de potentia virtuosa actu virtuosa. Cum autem potentia sit propter actum, sicut propter complementum, impossibile est quod id quod est secundum se in potentia existens, habeat rationem ultimi finis. Unde impossibile est quod ipsa anima sit ultimus finis sui ipsius. Similiter etiam neque aliquid eius, sive sit potentia, sive habitus, sive actus. Bonum enim quod est ultimus finis, est bonum perfectum complens appetitum. Appetitus autem humanus, qui est voluntas, est boni universalis. Quodlibet bonum autem inhaerens ipsi animae, est bonum participatum, et per consequens particulatum. Unde impossibile est quod aliquod eorum sit ultimus finis hominis. Sed si loquamur de ultimo fine hominis quantum ad ipsam adeptionem vel possessionem, seu quemcumque usum ipsius rei quae appetitur ut finis, sic ad ultimum finem pertinet aliquid hominis ex parte animae, quia homo per animam beatitudinem consequitur. Res ergo ipsa quae appetitur ut finis, est id in quo beatitudo consistit, et quod beatum facit, sed huius rei adeptio vocatur beatitudo. Unde dicendum est quod beatitudo est aliquid animae; sed id in quo consistit beatitudo, est aliquid extra animam.

[33522] Iª-IIae q. 2 a. 7 ad 1 Ad primum ergo dicendum quod, secundum quod sub illa divisione comprehenduntur omnia bona quae homini sunt appetibilia, sic bonum animae dicitur non solum potentia aut habitus aut actus, sed etiam obiectum, quod est extrinsecum. Et hoc modo nihil prohibet dicere id in quo beatitudo consistit, esse quoddam bonum animae.

[33523] Iª-IIae q. 2 a. 7 ad 2 Ad secundum dicendum, quantum ad propositum pertinet, quod beatitudo maxime amatur tanquam bonum concupitum, amicus autem amatur tanquam id cui concupiscitur bonum; et sic etiam homo amat seipsum. Unde non est eadem ratio amoris utrobique. Utrum autem amore amicitiae aliquid homo supra se amet, erit locus considerandi cum de caritate agetur.

[33524] Iª-IIae q. 2 a. 7 ad 3 Ad tertium dicendum quod beatitudo ipsa, cum sit perfectio animae, est quoddam animae bonum inhaerens, sed id in quo beatitudo consistit, quod scilicet beatum facit, est aliquid extra animam, ut dictum est.

Articulus 8

[33525] Iª-IIae q. 2 a. 8 arg. 1 Ad octavum sic proceditur. Videtur quod beatitudo hominis consistat in aliquo bono creato. Dicit enim Dionysius, VII cap. de Div. Nom., quod divina sapientia coniungit fines primorum principiis secundorum, ex quo potest accipi quod summum inferioris naturae sit attingere infimum naturae superioris. Sed summum hominis bonum est beatitudo. Cum ergo Angelus naturae ordine sit supra hominem, ut in primo habitum est; videtur quod beatitudo hominis consistat in hoc quod aliquo modo attingit ad Angelum.

[33526] Iª-IIae q. 2 a. 8 arg. 2 Praeterea, ultimus finis cuiuslibet rei est in suo perfecto, unde pars est propter totum, sicut propter finem. Sed tota universitas creaturarum, quae dicitur maior mundus, comparatur ad hominem, qui in VIII Physic. dicitur minor mundus, sicut perfectum ad imperfectum. Ergo beatitudo hominis consistit in tota universitate creaturarum.

[33527] Iª-IIae q. 2 a. 8 arg. 3 Praeterea, per hoc homo efficitur beatus, quod eius naturale desiderium quietat. Sed naturale desiderium hominis non extenditur ad maius bonum quam quod ipse capere potest. Cum ergo homo non sit capax boni quod excedit limites totius creaturae, videtur quod per aliquod bonum creatum homo beatus fieri possit. Et ita beatitudo hominis in aliquo bono creato consistit.

[33528] Iª-IIae q. 2 a. 8 s. c. Sed contra est quod Augustinus dicit, XIX de Civ. Dei, ut vita carnis anima est, ita beata vita hominis Deus est; de quo dicitur, beatus populus cuius dominus Deus eius.

[33529] Iª-IIae q. 2 a. 8 co. Respondeo dicendum quod impossibile est beatitudinem hominis esse in aliquo bono creato. Beatitudo enim est bonum perfectum, quod totaliter quietat appetitum, alioquin non esset ultimus finis, si adhuc restaret aliquid appetendum. Obiectum autem voluntatis, quae est appetitus humanus, est universale bonum; sicut obiectum intellectus est universale verum. Ex quo patet quod nihil potest quietare voluntatem hominis, nisi bonum universale. Quod non invenitur in aliquo creato, sed solum in Deo, quia omnis creatura habet bonitatem participatam. Unde solus Deus voluntatem hominis implere potest; secundum quod dicitur in Psalmo CII, qui replet in bonis desiderium tuum. In solo igitur Deo beatitudo hominis consistit.

[33530] Iª-IIae q. 2 a. 8 ad 1 Ad primum ergo dicendum quod superius hominis attingit quidem infimum angelicae naturae per quandam similitudinem; non tamen ibi sistit sicut in ultimo fine, sed procedit usque ad ipsum universalem fontem boni, qui est universale obiectum beatitudinis omnium beatorum, tanquam infinitum et perfectum bonum existens.

[33531] Iª-IIae q. 2 a. 8 ad 2 Ad secundum dicendum quod, si totum aliquod non sit ultimus finis, sed ordinetur ad finem ulteriorem, ultimus finis partis non est ipsum totum, sed aliquid aliud. Universitas autem creaturarum, ad quam comparatur homo ut pars ad totum, non est ultimus finis, sed ordinatur in Deum sicut in ultimum finem. Unde bonum universi non est ultimus finis hominis, sed ipse Deus.

[33532] Iª-IIae q. 2 a. 8 ad 3 Ad tertium dicendum quod bonum creatum non est minus quam bonum cuius homo est capax ut rei intrinsecae et inhaerentis, est tamen minus quam bonum cuius est capax ut obiecti, quod est infinitum. Bonum autem quod participatur ab Angelo, et a toto universo, est bonum finitum et contractum.
