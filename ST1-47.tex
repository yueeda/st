\documentclass[10pt]{jsarticle} % use larger type; default would be 10pt
%\usepackage[utf8]{inputenc} % set input encoding (not needed with XeLaTeX)
%\usepackage[round,comma,authoryear]{natbib}
%\usepackage{nruby}
\usepackage{okumacro}
\usepackage{longtable}
%\usepqckage{tablefootnote}
\usepackage[polutonikogreek,english,japanese]{babel}
%\usepackage{amsmath}
\usepackage{latexsym}
\usepackage{color}
%----- header -------
\usepackage{fancyhdr}
\pagestyle{fancy}
\lhead{{\it Summa Theologiae} I, q.~47}
%--------------------

\bibliographystyle{jplain}

\title{{\bf Prima Pars}\\{\HUGE Summa Theologiae}\\Sancti Thomae
Aquinatis\\Quaestio QUDRAGESIMASEPTIMA\\{\bf DE DISTINCTIONE RERUM IN COMMUNI}}
\author{Japanese translation\\by Yoshinori {\sc Ueeda}}
\date{最終変更日 \today}


%%%% コピペ用
%\rhead{a.~}
%\begin{center}
% {\Large {\bf }}\\
% {\large }\\
% {\footnotesize }\\
% {\Large \\}
%\end{center}
%
%\begin{longtable}{p{21em}p{21em}}
%
%&
%
%\\
%\end{longtable}
%\newpage


\begin{document}
\maketitle

\begin{center}
 {\Large 第四十七問\\諸事物の区別一般について}
\end{center}
\begin{longtable}{p{21em}p{21em}}



{\huge P}{\sc ost} productionem creaturarum in esse,
considerandum est de distinctione earum. Erit autem haec consideratio
tripartita. Nam primo considerabimus de distinctione rerum in communi;
secundo, de distinctione boni et mali; tertio, de distinctione
spiritualis et corporalis creaturae. Circa primum quaeruntur
tria. 
\begin{enumerate}
 \item de ipsa rerum multitudine seu distinctione. 
 \item de earum inaequalitate. 
 \item de unitate mundi.
\end{enumerate}

&
被造物を存在へと産出することのあとには、それらの区別について考察されるべ
 きである。ところで、この考察は、三つの部分をもつだろう。
すなわち、第一に、わたしたちは諸事物の区別を共通的に考察し、第二に、善と
 悪の区別について、第三に、霊的被造物と物体的被造物の区別について考察す
 る。第一については、三つのことが問われる。
\begin{enumerate}
 \item 諸事物の多性、あるいは、区別そのものについて。
 \item それらの不等性について。
 \item 世界の一性について。
\end{enumerate}

\end{longtable}
\newpage
\rhead{a.~1}

\begin{center}
 {\Large {\bf ARTICULUS PRIMUS}}\\
 {\large UTRUM RERUM MULTITUDO ET DISTINCTIO SIT A DEO}\\
 {\footnotesize II {\it SCG.}, c.~34 usque 45 inlusu.; lib.~III, c.~97;
 {\it De Pot.}, q.~3, a.~1, ad 9; a.~16; {\it Compend.~Theol.}, c.~72, 102;
 XII {\it Metaphysic.}, l.~2; {\it De Causis,} l.~24.}\\
{\Large 第1項\\諸事物の多性と区別は神によってあるか}
\end{center}

\begin{longtable}{p{21em}p{21em}}


{\huge A}{\sc d primum sic proceditur}. Videtur quod
 rerum multitudo et distinctio non sit a Deo. Unum enim semper natum est
 unum facere. Sed Deus est maxime unus, ut ex praemissis patet. Ergo non
 producit nisi unum effectum.

&
第一に対しては、次のように進められる。
諸事物の多性と区別は、神によってあるのではないと思われる。
なぜなら、一つのものは常に一つのものを作るように生まれついている。
しかるに、すでに述べられたことから明らかなとおり、神は最大限に一である。
ゆえに、神は一つの結果しか造らない。

\\





2.~{\sc Praeterea}, exemplatum assimilatur suo
 exemplari. Sed Deus est causa exemplaris sui effectus, ut supra dictum
 est. Ergo, cum Deus sit unus, effectus eius est unus tantum, et non
 distinctus.


&
さらに、範型から生じたものは、自らの範型に類似化される。
しかるに、上で述べられたとおり、神は自らの結果の範型因である。
ゆえに、神は一つなので、神の結果はただ一つであり、区別されていない。

\\





3.~{\sc Praeterea}, ea quae sunt ad finem,
 proportionantur fini. Sed finis creaturae est unus, scilicet divina
 bonitas, ut supra ostensum est. Ergo effectus Dei non est nisi unus.


&
さらに、目的に対してあるものは、目的に比例する。
しかるに、上で示されたとおり、被造物の目的は一つ、すなわち、神の善性であ
 る。
ゆえに、神の結果は一つしかない。

\\





{\sc Sed contra est} quod dicitur {\it Gen.}~{\sc i}, quod {\it Deus
 distinxit lucem a tenebris}, et {\it divisit aquas ab aquis}. Ergo
 distinctio et multitudo rerum est a Deo.


&

しかし反対に、『創世記』1章で「神は闇から光を区別」し、「水から水を分けた」
 \footnote{「神は光と闇を分け」(1:4)、「神は大空を造り、大空の下と大空の
 上に水を分けさせられた。」(1:7)} と述べられている。ゆえに、諸事物の区別
 と多性は神による。



\\





{\sc Respondeo dicendum} quod causam distinctionis
 rerum multipliciter aliqui assignaverunt. Quidam enim attribuerunt eam
 materiae, vel soli, vel simul cum agente. Soli quidem materiae, sicut
 Democritus, et omnes antiqui naturales, ponentes solam causam
 materialem, secundum quos distinctio rerum provenit a casu, secundum
 motum materiae. -- Materiae vero et agenti simul distinctionem et
 multitudinem rerum attribuit Anaxagoras, qui posuit intellectum
 distinguentem res, extrahendo quod erat permixtum in materia. 



&

答えて言わなければならない。
人々は、多くのかたちで、諸事物の区別の原因を指定した。
ある人々は、それを質料単独に、あるいは作用者と同時に質料に帰した。
質料単独に帰したのは、たとえば、デモクリトスや、古代のすべての自然哲学者
 たちであり、彼らは質料因だけを考えたので、彼らによれば、諸事物の区別は、
 質料の運動にしたがって、偶然に生じる。
他方、アナクサゴラスは、諸事物の区別と多性を質料と作用者の両方に帰する。
 彼は、質料において混合されていたものを抽出することによって、諸事物を区
 別する知性を措定した。

\\



Sed hoc
 non potest stare propter duo. Primo quidem, quia supra ostensum est
 quod etiam ipsa materia a Deo creata est. Unde oportet et
 distinctionem, si qua est ex parte materiae, in altiorem causam
 reducere. -- Secundo, quia materia est propter formam, et non e
 converso. Distinctio autem rerum est per formas proprias. Non ergo
 distinctio est in rebus propter materiam, sed potius e converso in
 materia creata est difformitas, ut esset diversis formis
 accommodata. 
&

しかし、この考えは二つの理由のために成り立たない。第一に、質料自身もまた、
 神によって創造されたことが上で明らかにされたからである。したがって、か
 りにそれが質料の側からであるとしても、より高い原因へと遡らなければなら
 ない。第二に、質料は形相のためにあるのであり、その逆ではない。しかるに、
 諸事物の区別は固有の形相によってある。
ゆえに、質料のために、諸事物の中に区別があるのではなく、むしろ逆に、さま
 ざまな形相に適合するように、創造された質料においては無形相性がある。

\\

Quidam vero attribuerunt distinctionem rerum secundis
 agentibus. Sicut Avicenna, qui dixit quod Deus, intelligendo se,
 produxit intelligentiam primam, in qua, quia non est suum esse, ex
 necessitate incidit compositio potentiae et actus, ut infra
 patebit. Sic igitur prima intelligentia, inquantum intelligit causam
 primam, produxit secundam intelligentiam; inquantum autem intelligit se
 secundum quod est in potentia, produxit corpus caeli, quod movet;
 inquantum vero intelligit se secundum illud quod habet de actu,
 produxit animam caeli. 

&

またある人々は、諸事物の区別を、第二作用者に帰した。たとえば、アヴィセン
 ナは、神は、自らを知性認識することによって、第一の知性体を産出し、そこ
 において、それは自らの存在でないので、必然的に、可能態と現実態の複合
 が始まったと言った。これは下で明らかとなるであろう。
かくして、それゆえ、第一の知性体が、第一原因を知性認識する限りにおいて、
 第二の知性体を産出し、さらに、[第二の知性体は]自らを、可能態において
 在る限りにおいて知性認識揮する限りにおいて、天の物体を産出してそれを動
 かし、また、自らが現実態を分け持つものである点で知性認識する限りで、天
 の魂を産出した。


\\


Sed hoc non potest stare propter duo. Primo
 quidem, quia supra ostensum est quod creare solius Dei est. Unde ea
 quae non possunt causari nisi per creationem, a solo Deo producuntur,
 et haec sunt omnia quae non subiacent generationi et
 corruptioni. Secundo, quia secundum hanc positionem, non proveniret ex
 intentione primi agentis universitas rerum, sed ex concursu multarum
 causarum agentium. Tale autem dicimus provenire a casu. Sic igitur
 complementum universi, quod in diversitate rerum consistit, esset a
 casu, quod est impossibile. 

&

しかし、これは二つの理由のために成立し得ない。第一に、創造することは、た
だ神にのみ属することが上で明らかにされたからである。したがって、創造によ
らなければ原因されないものどもは、ただ神によってのみ生み出されるが、その
ようなものとは、生成消滅のもとにないすべてのものどもである。第二に、この
考えによれば、第一作用者の意図に基づいてではなく、多くの作用因の合同によっ
て諸事物の世界が出てきたことになる。しかし、そのようなことを、わたしたち
は、偶然生じると言う。ゆえに、諸事物の多様性において成立する世界の完成は、
偶然であることになる。これは不可能である。


\\


Unde dicendum est quod distinctio rerum et multitudo est ex intentione
 primi agentis, quod est Deus. Produxit enim res in esse propter suam
 bonitatem communicandam creaturis, et per eas repraesentandam. Et quia
 per unam creaturam sufficienter repraesentari non potest, produxit
 multas creaturas et diversas, ut quod deest uni ad repraesentandam
 divinam bonitatem, suppleatur ex alia, nam bonitas quae in Deo est
 simpliciter et uniformiter, in creaturis est multipliciter et
 divisim. Unde perfectius participat divinam bonitatem, et repraesentat
 eam, totum universum, quam alia quaecumque creatura. -- Et quia ex divina
 sapientia est causa distinctionis rerum, ideo Moyses dicit res esse
 distinctas verbo Dei, quod est conceptio sapientiae. Et hoc est quod
 dicitur {\it Gen}.~{\sc i} : {\sc Dixit Deus, fiat lux. Et divisit
 lucem a tenebris}.

&
したがって、諸事物の区別と多性は、第一作用者、すなわち神の意図に基づくと
 言われるべきである。神は事物を、自らの善性を被造物に伝達し、それらによっ
 て自らの善性が表現されるために、存在へと産出した。
そして、被造物の内の一つでは、十分に表現され得ないので、神の善性を表現す
 るために、一つの被造物に欠けているところは別の被造物によって補われるよ
 うに、多くのさまざまな被造物を生み出した。神における善性は単純で一つの
 かたちに在るが、被造物においては多様でさまざまな形で在るからである。
このことから、全世界は、それ以外の各々の被造物よりも、より完全に神の善性
 を分有し、それを表現する。
そして、諸事物の区別の原因は、神の知恵に基づくので、モーセは、神の言葉に
 よって、すなわち知恵の概念(懐胎)によって、区別されたという。これが、『創世記』
 1章「神は言った。光あれ。そして光を闇から分けた」と言われていることなの
 である。


\\





{\sc Ad primum ergo dicendum} quod agens per
 naturam agit per formam per quam est, quae unius tantum est una, et
 ideo non agit nisi unum. Agens autem voluntarium, quale est Deus, ut
 supra ostensum est, agit per formam intellectam. Cum igitur Deum multa
 intelligere non repugnet unitati et simplicitati ipsius, ut supra
 ostensum est, relinquitur quod, licet sit unus, possit multa facere.


&

第一に対しては、それゆえ、次のように言われるべきである。
本性による作用者は、それによって在るところの形相を通して働くが、その形相
 は、一つの作用者に一つである。ゆえに、一つの働きしか為さない。
これに対して、意志的な作用者、神はこのようなものだが、は、上で明らかにさ
 れたとおり、知性認識された形相によって働く。
ゆえに、上で示されたように、神が多くのものを知性認識することは、神の一性
 と単純性に矛盾しないので、神は一つであるが、多くのことを為すことができ
 ることが帰結する。

\\





{\sc Ad secundum dicendum} quod ratio illa teneret
 de exemplato quod perfecte repraesentat exemplar, quod non
 multiplicatur nisi materialiter. Unde imago increata, quae est
 perfecta, est una tantum. Sed nulla creatura repraesentat perfecte
 exemplar primum, quod est divina essentia. Et ideo potest per multa
 repraesentari. Et tamen secundum quod ideae dicuntur exemplaria,
 pluralitati rerum correspondet in mente divina pluralitas idearum.

&

第二に対しては、次のように言われるべきである。
かの論は、範型を完全に表現するような、範型によって生じたものについてなら
 成り立ったであろう。そのようなものは、質料的にしか多数化されないからで
 ある。したがって、完全なものである創造されざる像は、ただ一つである。
しかし、どんな被造物も、神の本質である第一の範型を完全に表現しない。
ゆえに、多くのものによって表現されうる。しかし、イデアが範型と言われる限
 りにおいては、諸事物の複数性には、神の精神の中の、イデアの複数性が対応
 する。


\\





{\sc Ad tertium dicendum} quod in speculativis
 medium demonstrationis, quod perfecte demonstrat conclusionem, est unum
 tantum, sed media probabilia sunt multa. Et similiter in operativis,
 quando id quod est ad finem adaequat, ut ita dixerim, finem, non
 requiritur quod sit nisi unum tantum. Sed creatura non sic se habet ad
 finem qui est Deus. Unde oportuit creaturas multiplicari.


&

第三に対しては、次のように言われるべきである。
理論的な事柄においては、結論を完全に証明する論証の媒介は、一つだけである。
 しかし、蓋然的な媒介は、多数ある。
同様に、働きのもとにある事柄において、目的のためにあるものが、目的にいわ
 ば等しい場合には、一つしか必要とされない。しかし、被造物は、神である目
 的へ、そのようには関係しない。ゆえに、被造物は多様化される必要があった。




\end{longtable}
\newpage
\rhead{a.~2}

\begin{center}
 {\Large {\bf ARTICULUS SECUNDUS}}\\
 {\large UTRUM INAEQUALITAS RERUM SIT A DEO}\\
 {\footnotesize Infra q.~65, a.~2; II {\it SCG.}, c.~44, 45; III, c.~97;
 {\itshape De Pot.}, q.~3, a.~16; Qu.~{\itshape de Anima} a.~7;
 {\itshape Compend.~Theol.}, c.~73, 102; {\itshape De Div.~Nom.}, c.~4,
 l.=16}\\
{\Large 第2項\\諸事物の不等性は神によってあるか}
\end{center}

\begin{longtable}{p{21em}p{21em}}

{\huge A}{\sc d secundum} sic proceditur. Videtur quod inaequalitas
rerum non sit a Deo. Optimi enim est optima adducere. Sed inter optima
unum non est maius altero. Ergo Dei, qui est optimus, est omnia aequalia
facere.

&

第二に対しては、次のように進められる。諸事物の不等性は神によってあるわけ
 ではないと思われる。なぜなら、最善の事柄をもたらすことは、最善のものに
 属する。しかるに、最善のものどもの中で、或るものが他のものよりも偉大で
 あることはない。ゆえに、最善である神には、すべて等しいものを造ることが
 属する。

\\

2.~{\scshape Praeterea}, aequalitas est effectus unitatis, ut dicitur in
V {\itshape Metaphys}. Sed Deus est unus. Ergo fecit omnia aequalia.

&
さらに、『形而上学』第5巻で言われているように、等しさは一性の結果である。
しかるに、神は一である。ゆえに、神はすべて等しいものを造った。


\\


3.~{\scshape Praeterea}, iustitiae est inaequalia
inaequalibus dare. Sed Deus est iustus in omnibus operibus suis. Cum
ergo operationi eius, qua esse rebus communicat, non praesupponatur
aliqua inaequalitas rerum, videtur quod fecerit omnia aequalia.

&
さらに、等しくないものを、等しくないものに与えることは、正義に属する。し
 かるに、神は、自らのすべての技において正しい。ゆえに、それによって神が
 事物に存在を伝達するところの神の働きには、どんな諸事物の不等性も前提さ
 れていないので、すべて等しいものを造ったと思われる。


\\

{\scshape Sed contra est} quod dicitur {\itshape Eccli.}~{\scshape
xxxiii}: {\itshape Quare dies diem superat, et iterum lux lucem, et
annus annum, sol solem?  A domini scientia separata sunt}.


&
しかし反対に、『シラ書』33節に「なぜ、日が日の上に、さらに光が光の上に、
 年が年の上に、太陽が太陽の上にあるか。主の知恵によって、それらは分かた
 れた」と言われている。


\\

{\scshape Respondeo dicendum} quod Origenes, volens
excludere positionem ponentium distinctionem in rebus ex contrarietate
principiorum boni et mali, posuit a Deo a principio omnia creata esse
aequalia. Dicit enim quod Deus primo creavit creaturas rationales
tantum, et omnes aequales, in quibus primo exorta est inaequalitas ex
libero arbitrio, quibusdam conversis in Deum secundum magis et minus,
quibusdam etiam secundum magis et minus a Deo aversis. 

&

答えて言わなければならない。
オリゲネス\footnote{オリゲネス(Origenes Adamantius, 182年? - 251年)は、
 古代キリスト教最大の神学者。いわゆるギリシア教父とよばれる神学者群の一
 人で、アレクサンドリア学派といわれるグループの代表的存在。『諸原理につ
 いて』(De Principiis)など膨大な著作を著したが、死後異端の疑惑をかけら
 れたため、多くの著作が処分された。キリスト教の教義学を初めて確立し、そ
 の後の西欧思想史に大きな影響を与えたと評される。(Wikipedia)}は、善と悪
 という対立する根源から諸事物における区別があると考える立場を排除しよう
 として、根源である神から、万物は等しいものとして創造されたとした。
じっさい、彼は、神が第一に理性的被造物だけを創造し、それらは等しく、それ
 らの中に、自由選択から第一に不等性が生じ、ある者たちは、より多くまたは
 より少なく神へと向かい、ある者たちは、より多くまたはより少なく、神から
 離反した、と言う。

\\

Illae igitur
rationales creaturae quae ad Deum per liberum arbitrium conversae sunt,
promotae sunt ad diversos ordines Angelorum, pro diversitate
meritorum. Illae autem quae aversae sunt a Deo, sunt corporibus
alligatae diversis, secundum diversitatem peccati, et hanc causam dicit
esse creationis et diversitatis corporum. 

&
ゆえに、自由選択によって神へ向き直った理性的被造物は、功績の多様性に応じて、天使のさまざまな階級へと動かされた。他方、神から離反した理性的被造物は、罪の多様さにしたがって、さまざまな物体に結びつけられた。そして、これが、物体の創造と多様性の原因だという。

\\

Sed secundum hoc, universitas corporalium creaturarum non esset propter
bonitatem Dei communicandam creaturis, sed ad puniendum peccatum. Quod
est contra illud quod dicitur {\itshape Gen}.~{\scshape i} : {\itshape
Vidit Deus cuncta quae fecerat, et erant valde bona}. Et ut Augustinus
dicit, XI {\itshape de Civ.~Dei}: {\itshape Quid stultius dici potest,
quam istum solem, ut in uno mundo unus esset, non decori pulchritudinis,
vel saluti rerum corporalium consuluisse artificem Deum; sed hoc potius
evenisse, quia una anima sic peccaverat? Ac per hoc, si centum animae
peccassent, centum soles haberet hic mundus}.

&
しかし、これによれば、物体的被造物の宇宙は、被造物に神の善性を伝えるためにあるのではなく、罪を罰するためにあることになる。これは、『創世記』1章「神は作ったものすべてを見た。そして、それらは非常に善かった」と言われていることに反する。また、アウグスティヌスが『神の国』11巻で言うように、「制作者である神が、美の飾りのために、あるいは、物体的事物の健やかさのために、一つの世界に一つあるように、この太陽を作ったのではなく、一つの魂が、そのように罪を犯したから、そうなったのだと語るほど、愚かしいことがあろうか。これによれば、もし100の魂が罪を犯したならば、100の太陽が、この世に在ったことになる」。

\\

Et ideo dicendum est quod, sicut
sapientia Dei est causa distinctionis rerum, ita et inaequalitatis. Quod
sic patet. Duplex enim distinctio invenitur in rebus, una formalis, in
his quae differunt specie; alia vero materialis, in his quae differunt
numero tantum. Cum autem materia sit propter formam, distinctio
materialis est propter formalem. Unde videmus quod in rebus
incorruptibilibus non est nisi unum individuum unius speciei, quia
species sufficienter conservatur in uno, in generabilibus autem et
corruptibilibus, sunt multa individua unius speciei, ad conservationem
speciei. Ex quo patet quod principalior est distinctio formalis quam
materialis. 

& ゆえに、次のように言われるべきである。神の知恵が諸事物の区別の原因であ
るように、不当性の原因でもある。これは以下のようにして明らかである。諸事
物において二通りの区別が見いだされる。一つは、種において異なるものどもに
おける形相的な区別であり、もう一つは、数においてのみ異なるものどもにおけ
る質料的な区別である。しかるに、質料は形相のためにあるから、質料的な区別
は形相的な区別のためにある。したがって、不可滅的なものどもにおいて、一つ
の種に一つの個しかないのを私たちは見るが、それは、種が、一つのものにおい
て十分に保存されるからである。これに対し、生成消滅するものどもにおいては、
種を保存するために、一つの種に属する多くの個が存在する。このことから、質
料的区別よりも、形相的区別の方が、より根源的であることが明らかである。


\\


Distinctio autem formalis semper requirit inaequalitatem, quia, ut
dicitur in VIII {\itshape Metaphys.}, formae rerum sunt sicut numeri, in
quibus species variantur per additionem vel subtractionem unitatis. Unde
in rebus naturalibus gradatim species ordinatae esse videntur, sicut
mixta perfectiora sunt elementis, et plantae corporibus mineralibus, et
animalia plantis, et homines aliis animalibus; et in singulis horum una
species perfectior aliis invenitur. Sicut ergo divina sapientia causa
est distinctionis rerum propter perfectionem universi, ita et
inaequalitatis. Non enim esset perfectum universum, si tantum unus
gradus bonitatis inveniretur in rebus.


&

しかるに、形相的区別は、常に、不等性を必要とする。なぜなら、『形而上学』
 8巻で言われるように、諸事物の形相は数のようなものであり、そこにおいて、
 1を足したり引いたりすることによって種が変化する。
したがって、自然的諸事物において、種が段階的に秩序づけられているのが見ら
 れる。たとえば、混合物は元素よりも、植物は鉱物よりも、動物は植物よりも、
 人間は他の動物よりも、それぞれ完全であるように。
これらの各々において、一つの種が他の種よりも完全であることが見出される。
ゆえに、神の知恵は、宇宙の完全性のために、諸事物の区別の原因であるように、
 不等性の原因でもある。もし、諸事物において、ただ一つの善性の段階しか見
 出されなかったら、宇宙は完全ではなかったであろう。

\\


{\scshape Ad primum ergo dicendum} quod optimi agentis est producere
totum effectum suum optimum, non tamen quod quamlibet partem totius
faciat optimam simpliciter, sed optimam secundum proportionem ad totum,
tolleretur enim bonitas animalis, si quaelibet pars eius oculi haberet
dignitatem. Sic igitur et Deus totum universum constituit optimum,
secundum modum creaturae, non autem singulas creaturas, sed unam alia
meliorem. Et ideo de singulis creaturis dicitur {\itshape
Gen.}~{\scshape i}: {\itshape Vidit Deus lucem quod esset bona}, et
similiter de singulis, sed de omnibus simul dicitur: {\itshape Vidit
Deus cuncta quae fecerat, et erant valde bona}.

&
第一に対しては、それゆえ、次のように言われるべきである。
最善の作用者には、自らの結果全体を最善に作り出すことが属する。しかし、全
 体のどの部分も端的に完全にするのではなく、全体とのバランスにおいて部分
 を最善にする。たとえば、動物の善性は、もしそのどの部分も目に相当する価
 値を持っていたら、なくなっていたであろう。
それゆえ、このように、神も、全宇宙を被造物のあり方に即して最善に作ったの
 であり、個々の被造物を最善に作ったのではない。そうではなく、一つの被造
 物を他の被造物よりも善いようにした。このことから、個々の被造物について
 は、『創世記』1章「神は、善いものである光を見た」\footnote{「神は光を見
 て、良しとされた」(1:4)}のであり、同様に、個々
 のものについてだが、すべてのものについて同時に、「神は造ったすべてのも
 のを見た。それらは実に善かった」\footnote{「神はお造りになったすべてのものを御覧になった。見よ、それは極めて良かった」(1:31)}と述べられている。


\\


{\scshape Ad secundum dicendum} quod primum quod
procedit ab unitate, est aequalitas; et deinde procedit
multiplicitas. Et ideo a patre, cui, secundum Augustinum, appropriatur
unitas, processit filius, cui appropriatur aequalitas; et deinde
creatura, cui competit inaequalitas. Sed tamen etiam a creaturis
participatur quaedam aequalitas, scilicet proportionis.

&
第二に対しては、次のように言われるべきである。
一性から出てくる第一のものは等しさである。そして、続いて多性が出てくる。
ゆえに、アウグスティヌスによれば一性が固有化されるところの父から、等しさ
 が固有化される子が発出した。続いて、不等性が適合する被造物が発出した。
しかし、被造物によっても、ある種の等しさ、すなわち、比の等しさが分有され
 ている。


\\


{\scshape Ad tertium dicendum} quod ratio illa est quae
movit Origenem, sed non habet locum nisi in retributione praemiorum,
quorum inaequalitas debetur inaequalibus meritis. Sed in constitutione
rerum non est inaequalitas partium per quamcumque inaequalitatem
praecedentem vel meritorum vel etiam dispositionis materiae; sed propter
perfectionem totius. Ut patet etiam in operibus artis, non enim propter
hoc differt tectum a fundamento, quia habet diversam materiam; sed ut
sit domus perfecta ex diversis partibus, quaerit artifex diversam
materiam, et faceret eam si posset.


&

第三に対しては、次のように言われるべきである。
この理屈は、オリゲネスを動かしたものだが、これは報償の分配においてのみ有
 効なものであって、それに、功績の不等性に応じた不等性が必要とされる。
しかし、諸事物の構築において部分の不等性が存在するのは、功績であれ、質料
 の配置であれ、先行するなんらかの不等性によるのではなく、全体の完全性の
 ためである。
これはちょうど、技術による作品においても明らかである。たとえば、屋根が基
 礎と違うのは、異なる質料をもつからではない。さまざまな部分からなる家が
 完全であるために、技術者はさまざまな質料(材料)を求め、もし可能であればそれを
 造る。



\end{longtable}
\newpage
\rhead{a.~3}

\begin{center}
 {\Large {\bf ARTICULUS TERTIUS}}\\
 {\large UTRUM SIT UNUS MUNDUS TANTUM}\\
 {\footnotesize {\itshape De Pot.}, q.~3, a.~16, ad 1; XII {\itshape
 Metaphys.}, l.~10; I {\itshape de Cael.~et Mund.}, l.~16 sqq.}\\
 {\Large 第三項\\ただ一つの世界だけが在るか}
\end{center}

\begin{longtable}{p{21em}p{21em}}


{\scshape Ad tertium sic proceditur}. Videtur quod non sit unus mundus
tantum, sed plures. Quia, ut Augustinus dicit, in libro {\itshape
Octoginta trium Quaest}., inconveniens est dicere quod Deus sine ratione
res creavit. Sed ea ratione qua creavit unum, potuit creare multos, cum
eius potentia non sit limitata ad unius mundi creationem, sed est
infinita, ut supra ostensum est. Ergo Deus plures mundos produxit.

&
第三に対しては次のように進められる。
世界はただ一つではなく、複数あると思われる。なぜなら、
アウグスティヌスが『八十三問題集』で述べているように、神が理由なく事物を
 創造したと言うのは不適切である。しかし、上で示されたように、神の能力は
 一つの世界を創造することに制限されているわけではなく、無限であるから、
 一つの世界を創造したその理由によって、多くの世界を創造することができた。
ゆえに、神は複数の世界を生み出した。

\\


2.~{\scshape Praeterea}, natura facit quod melius est, et
multo magis Deus. Sed melius esset esse plures mundos quam unum, quia
plura bona paucioribus meliora sunt. Ergo plures mundi facti sunt a Deo.

&

さらに、自然はより善いものを造る。ましてや神はそうである。しかるに、より
 多くの善は、より少ない善よりも良いので、複数の世界が存在することは、一
 つの世界が存在することよりも良い。ゆえに、複数の世界が神によって造られ
 た。

\\


3.~{\scshape Praeterea}, omne quod habet formam in materia, potest
multiplicari secundum numerum, manente eadem specie, quia multiplicatio
secundum numerum est ex materia. Sed mundus habet formam in materia,
sicut enim cum dico {\itshape homo}, significo formam, cum autem dico
{\itshape hic homo}, significo formam in materia; ita, cum dicitur
{\itshape mundus}, significatur forma, cum autem dicitur {\itshape hic
mundus}, significatur forma in materia. Ergo nihil prohibet esse plures
mundos.  &

さらに、すべて質料において形相をもつものは、同一の種に留まりながら、数に
 おいて多数化されうる。なぜなら、数における多数化は質料によるからである。
 しかるに、世界は質料において形相をもつ。ちょうど、わたしが「人間」と言
 うとき、形相を意味表示するが、「この人間」と言うとき、質料における形相
 を意味表示するように、「世界」と言われるとき、形相が表示されるが、「こ
 の世界」と言われるとき、質料における形相が表示される。ゆえに、複数の世
 界が存在することを何も妨げない。 

\\
{\scshape Sed contra est} quod dicitur Ioan.~{\scshape i}: {\itshape
Mundus per ipsum factus est}; ubi singulariter {\itshape mundum}
nominavit, quasi uno solo mundo existente. 
 &
しかし反対に、『ヨハネによる福音書』1章「世界が、彼によって造られた」と
 言われていて、そこで「世界」と単数形で呼ばれている。これはただ一つの世
 界が存在するかのようにである。

\\

{\scshape Respondeo dicendum} quod ipse ordo in rebus sic
a Deo creatis existens, unitatem mundi manifestat. Mundus enim iste unus
dicitur unitate ordinis, secundum quod quaedam ad alia
ordinantur. Quaecumque autem sunt a Deo, ordinem habent ad invicem et ad
ipsum Deum, ut supra ostensum est. Unde necesse est quod omnia ad unum
mundum pertineant. Et ideo illi potuerunt ponere plures mundos, qui
causam mundi non posuerunt aliquam sapientiam ordinantem, sed casum; ut
Democritus, qui dixit ex concursu atomorum factum esse hunc mundum, et
alios infinitos.

&

答えて言わなければならない。
このように神によって創造された諸事物における秩序自体が、世界の一性を明示
 している。
じっさい、この世界は、或るものが他のものに秩序づけられる秩序の一性によっ
 て、一つと言われる。
しかるに、上で示されたように、神によってあるものはなんでも、相互に秩序を
 もち、また、神自身へ秩序をもつ。したがって、万物が、一つの世界に属する
 ことが必要である。
ゆえに、かの人々が複数の世界を措定することができたのは、世界の原因が、秩序づける
 なんらかの知恵ではなく、偶然だと考えたからである。たとえば、デモクリト
 スは、原子の衝突からこの世界が造られ、また、他の無数の世界が造られたと
 言った。

\\


{\scshape Ad primum ergo dicendum} quod haec ratio est quare mundus est
unus, quia debent omnia esse ordinata uno ordine, et ad unum. Propter
quod Aristoteles, in XII Metaphys.{\itshape }, ex unitate ordinis in
rebus existentis concludit unitatem Dei gubernantis. Et Plato ex unitate
exemplaris probat unitatem mundi, quasi exemplati.  &

第一に対しては、それゆえ、次のように言われるべきである。
なぜ世界が一つであるかの理由は、万物が一つの秩序によって、そして一つのも
 のへと秩序づけられなければならないからである。
このために、アリストテレスは『形而上学』12巻で、諸事物の中にある秩序の一
 性から、統治する神の一性を結論している。またプラトンも、範型の一性から、
 範型によって生じたものとしての世界の一性を証明している


\\


{\scshape Ad secundum dicendum} quod nullum agens intendit pluralitatem
materialem ut finem, quia materialis multitudo non habet certum
terminum, sed de se tendit in infinitum; infinitum autem repugnat
rationi finis. Cum autem dicitur plures mundos esse meliores quam unum,
hoc dicitur secundum multitudinem materialem. Tale autem melius non est
de intentione Dei agentis, quia eadem ratione dici posset quod, si
fecisset duos, melius esset quod essent tres; et sic in infinitum.  & 第
二対しては、次のように言われるべきである。どんな作用者も、質料的な複数性
を目的として意図したりはしない。なぜなら、質料的多数性は、確かな終端をも
たず、それ自体から、無限に広がるが、無限は、目的の性格に反するからである。
しかるに、複数の世界が一つの世界よりも良いと言われるとき、これは、質料的
な多数性にしたがって言われている。しかし、このような良さは、作用者である
神の意図に属さない。なぜなら、[もしその理屈が成り立つならば]同じ理由に
よって、二つの世界を造ったのであれば、三つの世界を造った方がより善かった、
と言われ得たであろうし、同様に無限に進んだであろうから。

\\


{\scshape Ad tertium dicendum} quod mundus constat ex
sua tota materia. Non enim est possibile esse aliam terram quam istam,
quia omnis terra ferretur naturaliter ad hoc medium, ubicumque esset. Et
eadem ratio est de aliis corporibus quae sunt partes mundi.
&
第三に対しては、次のように言われるべきである。
世界は、それ自身の全質料に基づいて成立している。
なぜなら、この大地意外に他の大地はありえないからである。
というのも、すべての地は、それがどこであろうと、この中心へ、本性的に運ば
 れるからである。同様に、世界の部分である他の物体についても同様である。



\end{longtable}
\end{document}