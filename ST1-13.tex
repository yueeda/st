\documentclass[paper=a4paper,fontsize=10pt,jafontsize=9pt,titlepage]{jlreq}
\usepackage{pxrubrica} %ルビ、傍点
\usepackage{longtable}
\usepackage[polutonikogreek,english,japanese]{babel}
\usepackage{latexsym}
\usepackage{color}
\usepackage{url}
%----- header -------
\usepackage{fancyhdr}
\pagestyle{fancy}
\fancyhead[C]{}
\fancyhead[L]{{\itshape Summa Theologiae} I, q.13}
%--------------------


\title{{\bfseries PRIMA PARS}\\Summae Theologiae\\Sancti Thomae
Aquinatis\\{\sffamily QUEAESTIO DECIMATERTIA}\\DE NOMINIBUS DEI}

\author{Japanese translation\\by Yoshinori {\scshape Ueeda}}
\date{Last modified \today}

%%%% コピペ用
%\rhead{a.~}
%\begin{center}
% {\Large \{bfseries }}\\
% {\large }\\
% {\footnotesize }\\
% {\Large \\}
%\end{center}
%
%\begin{longtable}{p{21em}p{21em}}
%
%
%
%\\
%\end{longtable}
%\newpage

\begin{document}

\maketitle

\begin{center}
{\Large 第十三問\\神の名について}
\end{center}

\begin{longtable}{p{21em}p{21em}}
{\LARGE C}onsideratis his quae ad divinam cognitionem pertinent,
procedendum est ad considerationem divinorum nominum, unumquodque enim
nominatur a nobis, secundum quod ipsum cognoscimus. Circa hoc ergo
quaeruntur duodecim.

\begin{enumerate}
 \item utrum Deus sit nominabilis a nobis.
 \item utrum aliqua nomina dicta de Deo, praedicentur de ipso
substantialiter.
 \item utrum aliqua nomina dicta de Deo, proprie dicantur de ipso; an
omnia attribuantur ei metaphorice.
 \item utrum multa nomina dicta de Deo, sint synonyma.
 \item utrum nomina aliqua dicantur de Deo et creaturis univoce, vel
aequivoce.
 \item supposito quod dicantur analogice, utrum dicantur de Deo per
prius, vel de creaturis.
 \item utrum aliqua nomina dicantur de Deo ex tempore.
 \item utrum hoc nomen {\itshape Deus} sit nomen naturae, vel operationis.
 \item utrum hoc nomen {\itshape Deus} sit nomen communicabile.
 \item utrum accipiatur univoce vel aequivoce, secundum quod
significat Deum per naturam, et per participationem, et secundum
opinionem.
 \item utrum hoc nomen {\itshape Qui est} sit maxime proprium nomen Dei.
 \item utrum
propositiones affirmativae possint formari de Deo.
\end{enumerate}

&

神の認識に属することが考察されたので、次に神の名の考察へと進むべきであ
る。なぜなら各々のものは、私たちがそれを認識するのに従って私たちによっ
て名付けられるからである。これについては12のことが問われる。

\vspace{1.3em}
\begin{enumerate}
 \item 神は私たちによって名付けられうるか。
 \item 神について言われるなんらかの名は、それについて実体的に述語付け
 られるか。
 \item 神について言われるなんらかの名は、それについて固有に言われるか、
       それとも、すべての名は比喩的に神に帰せられるか。
 \item 神について言われる多くの名は、同義的か。
 \item ある名は、神と被造物について一義的に言われるか、それとも異義的
 にか。
 \item アナロギア的に言われるとして、神についてより先に言われるのか、
 それとも被造物についてか。
 \item ある名は神について時間に基づいて言われるか。
 \item 「神」というこの名は本性の名か働きの名か。
 \item 「神」というこの名は共通でありうるか。
 \item 「神」というこの名は、本性、分有、意見によって、一義的にあるい
 は異義的に理解されるか。
 \item 「在る者」というこの名は、最大限に神に固有の名か。
 \item 神について肯定命題が形成されうるか。
\end{enumerate}
\end{longtable}

\newpage
\fancyhead[R]{a.1}
\begin{center}
{\Large {\bfseries ARTICULUS PRIMUS}\\UTRUM DEUS SIT NOMINABILIS A NOBIS}\\
{\large 第一項\\神は私たちによって名付けられうるか}
\end{center}

\begin{longtable}{p{21em}p{21em}}

{\scshape Ad primum sic proceditur}. Videtur quod nullum nomen Deo
conveniat. Dicit enim Dionysius, I cap.\ {\itshape de Div.\ Nom}., quod
{\itshape neque nomen eius est, neque opinio}. Et {\itshape Prov}.\ {\scshape xxx}
dicitur, {\itshape Quod nomen eius, et quod nomen filii eius, si nosti?}

&

第一項の問題へ、議論は次のように進められる。どんな名も神に適合しないと
思われる。理由は以下の通り。ディオニュシウスは『神名論』第1章で「彼に
は名も意見もない」と述べているし、『箴言』30で「もしあなたが知っている
ならば、どんな彼の名か、どんな彼の息子の名か」\footnote{「その名は何と
いうのか。その子の名は何というのか。あなたは知っているのか。」(30:4)}
と言われている。

\\

2.~{\scshape Praeterea}, omne nomen aut dicitur in abstracto, aut in
concreto. Sed nomina significantia in concreto, non competunt Deo, cum
simplex sit, neque nomina significantia in abstracto, quia non
significant aliquid perfectum subsistens. Ergo nullum nomen potest
dici de Deo.

&

さらに、名はすべて抽象的に言われるか具体的に言われるかである。ところが、
神は単純だから、具体的に表示する名は神に相応しくない。また、抽象的に表
示する名もふさわしくない。なぜなら、そういう名は、自存する完全な何かを
表示しないからである。ゆえに、どんな名も、神について語られ得ない。

\\

3.~{\scshape Praeterea}, nomina significant substantiam cum qualitate;
verba autem et participia significant cum tempore; pronomina autem cum
demonstratione vel relatione. Quorum nihil competit Deo, quia sine
qualitate est et sine omni accidente, et sine tempore; et sentiri non
potest, ut demonstrari possit; nec relative significari, cum relativa
sint aliquorum antedictorum recordativa, vel nominum, vel
participiorum, vel pronominum demonstrativorum. Ergo Deus nullo modo
potest nominari a nobis.

&

さらに、名詞\footnote{nomenは名詞と形容詞を含むので「名詞と形容詞」と
訳すべきだが、省略して「名詞」と訳す。厳密には、名詞はnomen
substantivum、形容詞はnomen adjectivumと言われる。cf.~{\itshape ST}I,
q.36, a.4, ad7.}は性質とともに実体を表示する。他方、動詞や分詞は時間と
ともに〔実体を〕表示する。また、代名詞は明示(=指示)あるいは関係とと
もに〔実体を〕表示する。これらのどれも神に適合しない。理由は以下の通り
である。神は性質をもたず、あらゆる付帯性をもたず、また時間ももたない。
また明示されうるように感覚されることも関係的に表示されることも不可能で
ある。なぜなら、関係的なものとは、名詞であれ分詞であれ明示的な代名詞で
あれ、先に述べられた何かを思い出させうるものだからである。ゆえに神は、
いかなる意味でも私たちによって名付けられない。

\\

{\scshape Sed contra est} quod dicitur {\itshape Exod}.\ {\scshape xv}: {\itshape Dominus
quasi vir pugnator, Omnipotens nomen eius}.

&

しかし反対に、『出エジプト記』15章で「主人はいわば戦士であり全能は彼の
名である」\footnote{「主こそいくさびと、その名は主」(15:3)}と言われて
いる。

\\

{\scshape Respondeo dicendum} quod, secundum Philosophum, voces sunt signa
intellectuum, et intellectus sunt rerum similitudines. Et sic patet
quod voces referuntur ad res significandas, mediante conceptione
intellectus. Secundum igitur quod aliquid a nobis intellectu cognosci
potest, sic a nobis potest nominari.

&

解答する。以下のように言われるべきである。哲学者によれば、音声は知性概
念\footnote{ここで「知性概念」と訳したのはintellectus。この語は通常
「知性」と訳し、認識能力を指すが、ここではそのような能力によって獲得さ
れた理解内容を言う。音声が精神のうちにある概念のしるしであるという思想
は、アリストテレスの『命題論』冒頭の箇所で述べられている。
\foreignlanguage{greek}{>'Esti m`en o>~un t`a 'en t~h| fwn~h| t~wn 'en
t~h| yuq~h| pajhm'atwn s'umbola,} ($16^{a}3$--4)(この箇所のラテン語訳
は``Sunt ergo ea quae sunt in voce, earum quae sunt in anima passionum
notae:'') アリストテレスの言葉は「魂の中にある受動」だが、トマスは注
解でconceptio intellectus(知性の懐念)と読み替えている(l.~2, n.~15)。}の
しるしであり、知性概念は事物の類似である。このように、音声は知性の懐念
\footnote{「概念」と訳されるratioと区別したいとき、conceptioに「懐念」
いう訳語があてられることがある。これはconceptioの別義が「懐胎」「妊娠」
であることによる。ratioはギリシア語「ロゴス」の訳語であり世界に満ちて
いる「理」のようなニュアンスを持つ。それが知性に受け取られて育まれたも
のがconceptioである。いずれにしてもconceptio ( $< concipio = cum +
capio$)にはratioにはない「抱かれ、育まれた」というニュアンスがある。}
を媒介にして、表示されるべき事物へ関係することが明らかである。ゆえに、
あるものは、知性によって、私たちに認識されうるかぎりにおいて、私たちに
よって名付けられうる。

\\

Ostensum est autem supra quod Deus in hac vita non potest a nobis
videri per suam essentiam; sed cognoscitur a nobis ex creaturis,
secundum habitudinem principii, et per modum excellentiae et
remotionis. Sic igitur potest nominari a nobis ex creaturis, non tamen
ita quod nomen significans ipsum, exprimat divinam essentiam secundum
quod est, sicut hoc nomen {\itshape homo} exprimit sua significatione
essentiam hominis secundum quod est, significat enim eius
definitionem, declarantem eius essentiam; ratio enim quam significat
nomen, est definitio.

&

ところで、神は現生においてその本質によって私たちに知られることは不可能
だが、被造物をもとにして根源というあり方において卓越と排除の方法によっ
て認識されることが、前に\footnote{第12問第2項}示された。それゆえ、神は
私たちによって被造物をもとにして名付けられるが、神を表示する名が、神の
本質をあるがままに表現するのではない。これは、「人間」という名が、その
表示によって人間である限りにおける人間の本質を表現するのとは異なる。じっ
さい「人間」は人間の本質を明らかにする人間の定義を意味するが、名が表示
する概念は定義だからである。

\\

{\scshape Ad primum ergo dicendum} quod ea ratione dicitur Deus non habere
nomen, vel esse supra nominationem, quia essentia eius est supra id
quod de Deo intelligimus et voce significamus.

&

第一異論に対してはそれゆえ次のように言われるべきである。神の本質は、私
たちが神について知性認識し音声によって意味表示することを超えているから、
そのために、神は名を持たないとか命名を超えてあるとかと言われる。

\\

{\scshape Ad secundum dicendum} quod, quia ex creaturis in Dei cognitionem
venimus, et ex ipsis eum nominamus, nomina quae Deo attribuimus, hoc
modo significant, secundum quod competit creaturis materialibus, quarum
cognitio est nobis connaturalis, ut supra dictum est. 

&

第二異論に対しては次のように言われるべきである。私たちは被造物に基づい
て神の認識へと至り、被造物に基づいて神を名付けるので、私たちが神に帰す
る名は質料的被造物に適するようなしかたで意味表示する。すでに述べられた
ように、私たちにとって本性にかなっているのは、質料的被造物の認識だから
である。

\\

Et quia in huiusmodi creaturis, ea quae sunt perfecta et subsistentia
sunt composita; forma autem in eis non est aliquid completum
subsistens, sed magis quo aliquid est, inde est quod omnia nomina a
nobis imposita ad significandum aliquid completum subsistens,
significant in concretione, prout competit compositis; quae autem
imponuntur ad significandas formas simplices, significant aliquid non
ut subsistens, sed ut quo aliquid est, sicut albedo significat ut quo
aliquid est album.

&

そしてそのような被造物において、完全で自存するのは複合体である。他方、
それらにおける形相は自存する完全なものでなく、むしろ、\kenten{それ}に
よって何かがあるところの\kenten{それ}である。それゆえ、自存する完全な
ものを表示するために私たちによってつけられたすべての名は、複合体に適合
するものとして、具体的に表示する。他方、単純形相を表示するためにつけら
れた名は、あるものを自存するものとしてではなく、\kenten{それ}によって
それがあるところの\kenten{それ}として表示する。例えば、白性は
\kenten{それ}によってなにかが白いところの\kenten{それ}を表示する。

\\

Quia igitur et Deus simplex est, et subsistens est, attribuimus ei et
nomina abstracta, ad significandam simplicitatem eius; et nomina
concreta, ad significandum subsistentiam et perfectionem ipsius,
quamvis utraque nomina deficiant a modo ipsius, sicut intellectus
noster non cognoscit eum ut est, secundum hanc vitam.

&

ゆえに、神は単純でありかつ自存するので、私たちは神の単純性を表示するた
めに抽象名詞を神に帰し、神の自存や完全性を表示するために具体名詞を帰す
る。ただしどちらの名も、私たちの知性が現生においては神をあるがままに認
識しないのと同じく、神自身のあり方と比べれば不完全なのだが。

\\

{\scshape Ad tertium dicendum} quod significare substantiam cum qualitate,
est significare suppositum cum natura vel forma determinata in qua
subsistit. Unde, sicut de Deo dicuntur aliqua in concretione, ad
significandum subsistentiam et perfectionem ipsius, sicut iam dictum
est, ita dicuntur de Deo nomina significantia substantiam cum
qualitate.

&

第三異論に対しては次のように言われるべきである。実体を性質とともに表示
することは、個体がそこにおいて自存するところの限定された本性や形相とと
もに個体を表示することである。したがって、すでに述べられたとおり、神の
自存性や完全性を表示するために神について何かが具体的に言われるのと同じ
ようにして、神について性質とともに実体を表示する名が語られる。

\\

Verba vero et participia consignificantia tempus dicuntur de ipso, ex
eo quod aeternitas includit omne tempus, sicut enim simplicia
subsistentia non possumus apprehendere et significare nisi per modum
compositorum, ita simplicem aeternitatem non possumus intelligere vel
voce exprimere, nisi per modum temporalium rerum; et hoc propter
connaturalitatem intellectus nostri ad res compositas et temporales.

&

他方、時間をともに表示する動詞と分詞は、永遠性がすべての時間を包含する
ことから神について語られる。ちょうど私たちが、複合体のあり方によってで
なければ自存する単純なものをとらえたり表示したりできないように、私たち
は単純な永遠性を時間的事物のあり方によってでなければ、知性認識したり音
声によって表現したりすることができない。それは私たちの知性が、複合され
て時間的な事物に本性上の適合性をもつためである。

\\

Pronomina vero demonstrativa dicuntur de Deo, secundum quod faciunt
demonstrationem ad id quod intelligitur, non ad id quod sentitur,
secundum enim quod a nobis intelligitur, secundum hoc sub
demonstratione cadit. Et sic, secundum illum modum quo nomina et
participia et pronomina demonstrativa de Deo dicuntur, secundum hoc et
pronominibus relativis significari potest.

&

また指示代名詞が神について語られるのは、感覚されたものではなく知性認識
されたものを指示するためである。すなわち、神は私たちに知性認識されるこ
とに応じて、そのかぎりで指示のもとに入ってくる。このように、名詞、分詞、
指示代名詞が神について語られるその仕方によって、関係代名詞によっても神
は表示されうる。

\end{longtable}

\newpage
\fancyhead[R]{a.~2}

\begin{center}
{\Large {\bfseries ARTICULUS SECUNDUS}}\\
{\large UTRUM ALIQUOD NOMEN DICATUR DE DEO SUBSTANTIALITER\\
第二項\\
神について何らかの名が実体的に語られるか}
\end{center}

\begin{longtable}{p{21em}p{21em}}

{\scshape Ad secundum sic proceditur}. Videtur quod nullum nomen dicatur de
Deo substantialiter. Dicit enim Damascenus: {\itshape Oportet singulum
eorum quae de Deo dicuntur, non quid est secundum substantiam
significare, sed quid non est ostendere, aut habitudinem quandam, aut
aliquid eorum quae assequuntur naturam vel operationem}.

&

第二項の問題へ議論は以下のように進められる。どんな名も、神について実体
的に語られないと思われる。理由は以下の通り。ダマスケヌスは「神について
語られる個々のことは、実体に即して「何であるか」を表示せず、「何でない
か」を明示するのでなければならない。それが一種の関係であれ、本性や働き
に伴うものの何かであれ」と述べている。

\\

2.~{\scshape Praeterea}, dicit Dionysius, {\scshape i} cap.\ {\itshape de Div.\
Nom.}: {\itshape Omnem sanctorum theologorum hymnum invenies, ad bonos
thearchiae processus, manifestative et laudative Dei nominationes
dividentem}, et est sensus, quod nomina quae in divinam laudem sancti
doctores assumunt, secundum processus ipsius Dei distinguuntur. Sed
quod significat processum alicuius rei, nihil significat ad eius
essentiam pertinens. Ergo nomina dicta de Deo, non dicuntur de ipso
substantialiter.

&

さらに、ディオニュシウスは『神名論』第1章で「あなたは、聖なる神学者た
ちのすべての賛美歌を見いだすだろう。神的根源のよき発出に沿って、明らか
に称えつつ神の命名を分けながら」と述べている。この意味は、聖なる博士た
ちが神を褒め称えるために集める名は、神自身の発出にしたがって区別される
ということである。しかしある事物の発出を表示するものは、その事物の本質
に属する何ものも表示しない。ゆえに神について語られる名は、神について実
体的に語られるのではない。

\\

3.~{\scshape Praeterea}, secundum hoc nominatur aliquid a nobis, secundum
quod intelligitur. Sed non intelligitur Deus a nobis in hac vita
secundum suam substantiam. Ergo nec aliquod nomen impositum a nobis,
dicitur de Deo secundum suam substantiam.

&

さらに、あるものはそれが知性認識されるのに応じて私たちによって名付けら
れる。しかるに、現生において、神は私たちによってその実体において知性認
識されない。ゆえに、私たちによってつけられた何らかの名が、神についてそ
の実体に即して語られることはない。

\\

{\scshape Sed contra est} quod dicit Augustinus, VI {\itshape de Trin.}: {\itshape
Deo hoc est esse, quod fortem esse vel sapientem esse, et si quid de
illa simplicitate dixeris, quo eius substantia significatur}. Ergo
omnia nomina huiusmodi significant divinam substantiam.

&

しかし反対に、アウグスティヌスは『三位一体論』6巻で「神にとって、存在
することは、強くあることであり、知恵あることであり、その他何であれ、あ
なたが神の単純性について語るようなものであり、それによって神の実体が意
味される」と述べている。ゆえに、このようなすべての名は、神の実体を表示
する。

\\

{\scshape Respondeo dicendum} quod de nominibus quae de Deo dicuntur
negative, vel quae relationem ipsius ad creaturam significant,
manifestum est quod substantiam eius nullo modo significant; sed
remotionem alicuius ab ipso, vel relationem eius ad alium, vel potius
alicuius ad ipsum. Sed de nominibus quae absolute et affirmative de
Deo dicuntur, sicut {\itshape bonus, sapiens,} et huiusmodi, multipliciter
aliqui sunt opinati.

&

解答する。以下のように言われるべきである。神について否定的に語られる名
や、神の被造物への関係を意味する名については、それが神の実体を決して意
味せず、むしろ神から何かを排除することや、神の他のものへの関係、否むし
ろ何かから神への関係を意味することが明らかである。しかし「善いもの」や
「知恵あるもの」など、神について肯定的に非関係的\footnote{absolutusは
「絶対的な」と訳したくなるが、$< ab + solvere$からわかるように「解き放
つ」の意味をもつ。トマスでは、relative(関係的)と対比的に用いられるこ
とが多い。その場合には「非関係的」と訳した方が意味がはっきりする。もち
ろん「絶対的」という言葉の原義も同じだが神について語られる「絶対的」は
余計なニュアンスを持ち込む危険がある。}に語られる名については、多くの
様々な意見があった。

\\

Quidam enim dixerunt quod haec omnia nomina, licet affirmative
de Deo dicantur, tamen magis inventa sunt ad aliquid removendum a Deo,
quam ad aliquid ponendum in ipso. Unde dicunt quod, cum dicimus Deum
esse {\itshape viventem}, significamus quod Deus non hoc modo est, sicut res
inanimatae, et similiter accipiendum est in aliis. Et hoc posuit Rabbi
Moyses. 

&

たとえば、ある人々は、これらの名はすべて、たしかに肯定的に神について語
られるけれども、神において何かを措定する\footnote{「措定する」は
p\={o}n\={o}の定訳だが以下の点に注意。「措定」は日本語として「根拠はな
いが仮にそう定める」というニュアンスがあるが、トマスにおいては必ずしも
そのような仮定的な意味を含むとは限らない(p\={o}n\={o}をそのような意味
で用いることは不可能ではない)。むしろこの箇所に現れているように「排除」
や「除去」の対義語として用いられる。「排除せずそこに置く」「そこにある
ことを認める」という意味。「認める」「是認する」という訳語の方がいいか
もしれない。}ためではなく、神から何かを排除するために見いだされたのだ
と言った。したがってこの人々は、私たちが神が「生きている」と言うとき、
神が無生物のようではないということを意味しているのであり、その他につい
ても同様であると言う。マイモニデスはこのような意見であった。

\\

Alii vero dicunt quod haec nomina imposita sunt ad significandum
habitudinem eius ad creata, ut, cum dicimus {\itshape Deus est bonus}, sit
sensus, {\itshape Deus est causa bonitatis in rebus}. Et eadem ratio est in
aliis.

&

他方別の人々は、これらの名は神の被造物への関係を表示するためにつけられ
たと言う。その結果、私たちが「神は善い」と言うとき、その意味は「神は諸
事物における善性の原因である」ということになる。その他の名においても同
様である。

\\

Sed utrumque istorum videtur esse inconveniens, propter tria. Primo
quidem, quia secundum neutram harum positionum posset assignari ratio
quare quaedam nomina magis de Deo dicerentur quam alia. Sic enim est
causa corporum, sicut est causa bonorum, unde, si nihil aliud
significatur, cum dicitur {\itshape Deus est bonus}, nisi {\itshape Deus est
causa bonorum}, poterit similiter dici quod {\itshape Deus est corpus},
quia est causa corporum.

&

しかし、このどちらも、三つの理由のために不適切である。第一に、このどち
らの立場も、なぜ、ある名が、他の名よりも神について語られるのかを説明し
ない。たとえば、神は善の原因であるのと同様に、石の原因でもある。したがっ
て、もし「神は善である」と言われるとき、たんに「神は善の原因である」と
いうことだけが意味されるのであれば、神は物体の原因なのだから、同様に
「神は物体である」とも言われ得たであろう。

\\

Item, per hoc quod dicitur quod est corpus, removetur quod non sit ens
in potentia tantum, sicut materia prima.

&

さらに、物体であると言われることによって、第一質料のように、たんに可能
態だけにあるのではないということが除去によって示される(のだから、「神
は物体である」と言ってもよかったはずである)。\footnote{「神はPである」
と語られるとき、それは「神はPの原因である」を意味するという主張と「神
はPと矛盾するQでない」を意味するという主張を批判している。この箇所は後
者に関係する。「物体である」ことは第一質料のような「たんに可能態にある」
ことに矛盾するので、神はたんに可能態にあるのではないことを意味するため
に「神は物体である」と言ってもいいはずだが、それは許されないではないか、
という反論。原文にはnonがあるので「可能態だけにあるのではない」が排除
されることになるが、このremoveturは「排除のしかたによってquod以下が意
味される」ほどの意か。Blackfriars訳はremoveturを「含意する」impliesと
訳しているが蓋し荒技。}

\\

Secundo, quia sequeretur quod omnia nomina dicta de Deo, per posterius
dicerentur de ipso, sicut {\itshape sanum} per posterius dicitur de
medicina, eo quod significat hoc tantum quod sit causa sanitatis in
animali, quod per prius dicitur sanum.

&

第二に、すべて神について語られる名が、より後なる意味で語られることになっ
たであろう。例えば「健康な」は、より先なる意味において、動物における健
康を意味するが、その動物における健康の原因であるということだけを意味す
るために、薬はより後なる意味で健康と言われる。\footnote{「健康薬品」と
「健康な動物」の「健康」の意味の違いに注目している。この例は、アリスト
テレスがアナロギアを論じる際に用いた有名な例。『形而上学』第4巻参照。
ここでの要点は、神がXの原因であるためにXと言われるとすれば、それは薬が
健康と言われるのと同じ理由なので、より先なる意味で(=本来的に)神がX
であると言っているのではないことになる、ということ。}

\\

Tertio, quia hoc est contra intentionem loquentium de Deo. Aliud enim
intendunt dicere, cum dicunt Deum viventem, quam quod sit causa vitae
nostrae, vel quod differat a corporibus inanimatis.

&

第三に、これは、神について語る人々の意図に反する。人々が「神が生きてい
る」と言うとき、彼らは「神が私たちの生命の原因である」や「神は魂のない
物体とは違う」とは別の何かを語ること意図している。

\\

Et ideo aliter dicendum est, quod huiusmodi quidem nomina significant
substantiam divinam, et praedicantur de Deo substantialiter, sed
deficiunt a repraesentatione ipsius. Quod sic patet. Significant enim
sic nomina Deum, secundum quod intellectus noster cognoscit
ipsum. Intellectus autem noster, cum cognoscat Deum ex creaturis, sic
cognoscit ipsum, secundum quod creaturae ipsum repraesentant.

&

それゆえ、以下のように別のかたちで言われるべきである。これらの名は神の
実体を表示し神について実体的に述語されるが、神を完全に表現するわけでは
ない。このことは以下のようにして明らかである。これらの名が神を表示する
のは、私たちの知性が神を認識するかぎりにおいてである。しかるに、私たち
の知性は、神を被造物に基づいて認識するので、被造物が神を表現するかぎり
において神を認識する。

\\

Ostensum est autem supra quod Deus in se praehabet omnes perfectiones
creaturarum, quasi simpliciter et universaliter perfectus. Unde
quaelibet creatura intantum eum repraesentat, et est ei similis,
inquantum perfectionem aliquam habet, non tamen ita quod repraesentet
eum sicut aliquid eiusdem speciei vel generis, sed sicut excellens
principium, a cuius forma effectus deficiunt, cuius tamen aliqualem
similitudinem effectus consequuntur; sicut formae corporum inferiorum
repraesentant virtutem solarem. Et hoc supra expositum est, cum de
perfectione divina agebatur. Sic igitur praedicta nomina divinam
substantiam significant, imperfecte tamen, sicut et creaturae
imperfecte eam repraesentant.

&

しかしすでに、神が、端的に普遍的に完全なものとして、自らのうちに被造物
のすべての完全性をあらかじめ有していることが示された。したがって、どん
な被造物も、何らかの完全性を持っているかぎり、そのかぎりで、神を表現し、
神に似ている。しかし、自分と同じ種や類に属するもののように神を表現する
のではなく、結果がその形相を十分に持たず、しかしその何らかの類似を結果
が獲得するといった、卓越した原因として表現する。ちょうど、下位の物体の
形相が、太陽の力を表現する場合のように。このことは、神の完全性について
論じられたときに説明されたことである。ゆえに、前述の名は、神の実体を表
示するが、被造物が神を不完全に表現するのと同様に、神を不完全に表示する。

\\

Cum igitur dicitur {\itshape Deus est bonus}, non est sensus, {\itshape Deus est
causa bonitatis}, vel {\itshape Deus non est malus}, sed est sensus, {\itshape
id quod bonitatem dicimus in creaturis, praeexistit in Deo}, et hoc
quidem secundum modum altiorem. Unde ex hoc non sequitur quod Deo
competat esse bonum inquantum causat bonitatem, sed potius e converso,
quia est bonus, bonitatem rebus diffundit, secundum illud Augustini,
{\itshape de Doct.\ Christ.}: {\itshape inquantum bonus est, sumus}.

&

ゆえに、「神は善である」と言われるとき、その意味は、「神は善性の原因で
ある」や「神は悪でない」ではなく、「私たちが被造物において善性と言うも
のが、神の中に先在している」という意味である。しかも、より高いあり方で
先在する。したがって、このことから、善であることが神に適合するのは、神
が善性の原因だからだ、ということは帰結せず、むしろ逆に、かのアウグスティ
ヌス『キリスト教の教え』「神が善である限りにおいて、私たちは存在する」
にしたがって、神が善であるから、神は諸事物に善を及ぼすのである。

\\

{\scshape Ad primum ergo dicendum} quod Damascenus ideo dicit quod haec
nomina non significant quid est Deus, quia a nullo istorum nominum
exprimitur quid est Deus perfecte, sed unumquodque imperfecte eum
significat, sicut et creaturae imperfecte eum repraesentant.

&

第一異論に対してはそれゆえ以下のように言われるべきである。ダマスケヌス
がこれらの名称が神の何であるかを表示しないと言うのは、これらの名のどれ
によっても完全に神の何であるかが表現されないからである。ちょうど被造物
も不完全に神を表現するように、各々の名称は不完全に神を表示する。

\\

{\scshape Ad secundum dicendum} quod in significatione nominum, aliud est
quandoque a quo imponitur nomen ad significandum, et id ad quod
significandum nomen imponitur, sicut hoc nomen {\itshape lapis} imponitur
ab eo quod {\itshape laedit pedem}, non tamen imponitur ad hoc
significandum quod significet laedens pedem, sed ad significandam
quandam speciem corporum; alioquin omne laedens pedem esset lapis.

&

第二異論に対しては次のように言われるべきである。名の表示においては、時
として、表示のために、ある名が\kenten{そこから}付けられたところのもの
と、\kenten{それを表示するために}その名が付けられたところのものが異な
ることがある。例えば、lapis(石)というこの名は、laedit pedem(足を傷
つける)から付けられているが、しかし、「足を傷つけるもの」を表示するた
めにではなく、むしろ、ある種の物体を表示するために名付けられている。そ
うでなければ、すべて、足を傷つけるものは石であっただろう。

\\

Sic igitur dicendum est quod huiusmodi divina nomina imponuntur quidem
a processibus deitatis, sicut enim secundum diversos processus
perfectionum, creaturae Deum repraesentant, licet imperfecte; ita
intellectus noster, secundum unumquemque processum, Deum cognoscit et
nominat.

&

このようなわけで、それゆえ、以下のように言われるべきである。〔異論が言
うような〕そのような名は、たしかに神性の発出によって付けられてはいる。
すなわち、ちょうどその様々な完全性の発出に即して被造物が不完全にではあ
れ神を表現するように、私たちの知性も、各々の発出に即して神を認識し名付
ける。

\\

Sed tamen haec nomina non imponit ad significandum ipsos processus,
ut, cum dicitur {\itshape Deus est vivens}, sit sensus, {\itshape ab eo procedit
vita}, sed ad significandum ipsum rerum principium, prout in eo
praeexistit vita, licet eminentiori modo quam intelligatur vel
significetur.

&

しかしこれらの名は、「神は生きている」の意味が「神から生命が発出する」
であるというように、この発出自体を表示するために付いているのではなく、
神のうちに生命が先在するというように、諸事物の根源自体を表示するために
付いている。ただし生命は知性認識され表示される以上の優れたあり方で神の
うちに先在するのだが。

\\

{\scshape Ad tertium dicendum} quod essentiam Dei in hac vita cognoscere
non possumus secundum quod in se est, sed cognoscimus eam secundum
quod repraesentatur in perfectionibus creaturarum. Et sic nomina a
nobis imposita eam significant.

&

第三異論に対しては次のように言われるべきである。私たちは神の本質を、現
生において、それ自体においてある限りで認識することができず、それが被造
物の完全性において表現される限りにおいて認識する。このようにして、私た
ちによって付けられたそれらの名は、それ(神の本質)を表示する。

\end{longtable}

\newpage
\fancyhead[R]{a.~3}

\begin{center}
{\Large {\bfseries ARTICULUS TERTIUS}}\\
{\large UTRUM ALIQUOD NOMEN DICATUR DE DEO PROPRIE}\\
{\large 第三項\\何らかの名が神について固有に語られるか} 
\end{center}

\begin{longtable}{p{21em}p{21em}}

{\scshape Ad tertium sic proceditur}. Videtur quod nullum nomen dicatur de
Deo proprie. Omnia enim nomina quae de Deo dicimus, sunt a creaturis
accepta, ut dictum est. Sed nomina creaturarum metaphorice dicuntur de
Deo, sicut cum dicitur {\itshape Deus est lapis}, vel {\itshape leo}, vel
aliquid huiusmodi. Ergo omnia nomina dicta de Deo, dicuntur
metaphorice.

&

第三項の問題へ議論は以下のように進められる。どんな名も神について固有に
語られないと思われる。理由は以下の通り。すべて神について私たちが語る名
は、すでに述べられたように、被造物から取られている。しかし被造物の名は
神について比喩的に語られる。例えば「神は石である」とか「神は獅子である」
などのように。ゆえに神について語られるすべての名は、比喩的に語られる。

\\

2.~{\scshape Praeterea}, nullum nomen proprie dicitur de aliquo, a quo
verius removetur quam de eo praedicetur. Sed omnia huiusmodi nomina,
{\itshape bonus, sapiens}, et similia, verius removentur a Deo quam de eo
praedicentur, ut patet per Dionysium, {\scshape ii} cap.\ {\itshape Cael.\
Hier}. Ergo nullum istorum nominum proprie dicitur de Deo.

&

さらに、どんな名も、それに述語されるよりも取り除かれる方がより真である
ようなものについて、固有に語れることはない。しかし「善」「知者」など、
すべてこのような名は、神について述語されるよりも神から取り除かれる方が
より真である。これはディオニュシウス『天上階級論』2章によって明らかで
ある。ゆえにこれらのうちのどんな名も神について固有に語られない。

\\

3.~{\scshape Praeterea}, nomina corporum non dicuntur de Deo nisi
metaphorice, cum sit incorporeus. Sed omnia huiusmodi nomina implicant
quasdam corporales conditiones, significant enim cum tempore, et cum
compositione, et cum aliis huiusmodi, quae sunt conditiones
corporum. Ergo omnia huiusmodi nomina dicuntur de Deo metaphorice.

&

さらに、神は非物体的なものなので、物体の名は、比喩的にでなければ神につ
いて語られない。しかしこれらすべての名は、何らかの物体的な条件をその意
味に含んでいる。たとえば、時間や複合やその他そのようなものを伴って表示
するが、それらは物体の条件である。ゆえにこのようなすべての名は、神につ
いて比喩的に語られる。

\\

{\scshape Sed contra est} quod dicit Ambrosius, in Lib.\ II {\itshape de Fide}:
{\itshape Sunt quaedam nomina, quae evidenter proprietatem divinitatis
ostendunt; et quaedam quae perspicuam divinae maiestatis exprimunt
veritatem; alia vero sunt, quae translative per similitudinem de Deo
dicuntur}. Non igitur omnia nomina dicuntur de Deo metaphorice, sed
aliqua dicuntur proprie.

&

しかし反対に、アンブロシウスは『信仰について』第2巻で次のように述べて
いる。「神の固有性をはっきりと明示するような名もあれば、神の偉大さの明
らかな真理を表現する名もある。他方、類似を通して神について比喩的に語ら
れる名もある」。ゆえに、神についてすべての名が比喩的に語られるのではな
く固有に語られる名もある。

\\

{\scshape Respondeo dicendum} quod, sicut dictum est, Deum cognoscimus ex
perfectionibus procedentibus in creaturas ab ipso; quae quidem
perfectiones in Deo sunt secundum eminentiorem modum quam in
 creaturis.

 &

解答する。以下のように言われるべきである。すでに述べられたように、私た
ちは神を、神から被造物へ発出する完全性に基づいて認識するが、それらの完
全性は神のうちに被造物のうちよりも優れたかたちで存在する。
 
 \\

 Intellectus autem noster eo modo apprehendit eas, secundum
quod sunt in creaturis, et secundum quod apprehendit, ita significat
per nomina. In nominibus igitur quae Deo attribuimus, est duo
considerare, scilicet, perfectiones ipsas significatas, ut bonitatem,
 vitam, et huiusmodi; et modum significandi.

 &

 ところが、私たちの知性はそれらの完全性を被造物のうちにある限りにおい
てとらえ、そのようにとらえる限りにおいて名によって表示する。それゆえ私
たちが神に帰する名において、二つのことを考察できるのであり、一つは、善
性、生命のような、表示されている\kenten{完全性}と、その表示の
\kenten{仕方}である。


 \\


 Quantum igitur ad id quod significant huiusmodi nomina, proprie
competunt Deo, et magis proprie quam ipsis creaturis, et per prius
dicuntur de eo. Quantum vero ad modum significandi, non proprie
dicuntur de Deo, habent enim modum significandi qui creaturis
competit.

&

ゆえに、このような名が表示しているものにかんして言えば、これらの名は神
に固有に適合するのであり、被造物よりも固有に、そして被造物よりも先に、
神について語られる。他方、表示の仕方にかんしては、神について固有に語ら
れず、被造物に適する表示の仕方を持っている。

\\

{\scshape Ad primum ergo dicendum} quod quaedam nomina significant
huiusmodi perfectiones a Deo procedentes in res creatas, hoc modo quod
ipse modus imperfectus quo a creatura participatur divina perfectio,
in ipso nominis significato includitur, sicut {\itshape lapis} significat
aliquid materialiter ens, et huiusmodi nomina non possunt attribui Deo
nisi metaphorice. Quaedam vero nomina significant ipsas perfectiones
absolute, absque hoc quod aliquis modus participandi claudatur in
eorum significatione, ut {\itshape ens, bonum vivens}, et huiusmodi, et
talia proprie dicuntur de Deo.

&

第一異論に対しては、それゆえ次のように言われるべきである。神から被造の
諸事物へ発出するこれらの完全性を、ある名は、神の完全性が被造物によって
分有される不完全なあり方が、名の意味自体の中に含まれるというかたちで表
示する。たとえば、「石」は、ある質料的に存在するものを表示する。そして、
このような名は神に比喩的にでなければ帰されない。他方で、ある名は、その
完全性を、無条件的に、何らかの分有のあり方をその意味の中に含まずに表示
する。たとえば、「存在するもの」「善」「生きるもの」などである。そして、
このような名は、神に固有に語られる。

\\

{\scshape Ad secundum dicendum} quod ideo huiusmodi nomina dicit Dionysius
negari a Deo, quia id quod significatur per nomen, non convenit eo
modo ei, quo nomen significat, sed excellentiori modo. Unde ibidem
dicit Dionysius quod Deus est {\itshape super omnem substantiam et vitam}.

&

第二異論に対しては、次のように言われるべきである。ディオニュシウスが、
これらの名は神から除かれると言うのは、名によって表示されるものが、名が
表示するあり方でではなく、より優れたあり方で、神に適合するからである。
このため、ディオニュシウスは、同じ箇所で、神は「あらゆる実体と生命を超
えている」と述べている。

\\

{\scshape Ad tertium dicendum} quod ista nomina quae proprie dicuntur de
Deo important conditiones corporales, non in ipso significato nominis,
sed quantum ad modum significandi. Ea vero quae metaphorice de Deo
dicuntur, important conditionem corporalem in ipso suo significato.

&

第三異論に対しては、次のように言われるべきである。神について固有に語ら
れる名も物体的条件を含んでいるが、しかしそれは、名の意味自体の中にでは
なく、表示の仕方にかんしてである。他方、神について比喩的に語られる名は、
物体的条件を、その意味自体の中に含んでいる。

\\

\end{longtable}

%【考察】\\
%「石」と「善」の違いは、前者が、その意味の中に不完全性を含んでいるのに対して、後者は含んでいないという点にある。ただし「善」の方も、名の意味と区別される「表示の仕方」について言えば、物体的条件を伴っている。たとえば時間や複合などを含意する限りで。
%
%イデア論の文脈で言えば、神において、前者はイデア、後者は本質的属性である。
%イデアとは、分有の不完全さをその意味のうちに含むもの、という理解でいいか
%どうか。
%
%「意味のうちに含む」という点は、cum/sine praecisioneという理解でいいか。

\newpage
\fancyhead[R]{a.~4}

\begin{center}
{\Large {\bfseries ARTICULUS QUARTUS}}\\
{\large UTRUM NOMINA DICTA DE DEO SINT NOMINA SYNONYMA\\
第四項\\
神について語られる名は同義語か}
\end{center}

\begin{longtable}{p{21em}p{21em}}

{\scshape Ad quartum sic proceditur}. Videtur quod
ista nomina dicta de Deo, sint nomina synonyma. Synonyma enim nomina
dicuntur, quae omnino idem significant. Sed ista nomina dicta de Deo,
omnino idem significant in Deo, quia bonitas Dei est eius essentia, et
similiter sapientia. Ergo ista nomina sunt omnino synonyma.

&

第四項の問題へ議論は以下のように進められる。神について語られるそれらの
名は同義語だと思われる。理由は以下の通り。まったく同じものを表示する名
が同義語と言われる。\footnote{異なる言葉が同じ意味を持つとき、それらは
同義語(nomina synonyma)と言われる。これに対して、「XはAである」「Yは
Aである」という二つの文において(Aは音声レベルで一致)、Aの意味が同じ
であるとき、「Aは一義的(univocum)な述語である」「Aは一義的に述語され
る」と言われ、異なるとき、「Aは異義的(aequivocum)な述語である」「Aは
異義的に述語される」と言われる。次項で論じられるように、この中間である
とき、それはアナロギア的(analogum)だと言われる。「同義語」が異なる複
数の言葉の意味の関係であるのに対して、「一義的」「異義的」「アナロギア
的」は、音声レベルで一致する複数の述語における意味の相互関係を言う。}
しかるに、神について語られるそれらの名は、神においてまったく同じものを
表示する。すなわち神の善性と神の本質、そして同様に(神の)知恵である。
ゆえにそれらの名はまったく同義語である。

\\

2.~Si dicatur quod ista nomina significant idem secundum rem, sed
secundum rationes diversas, contra: Ratio cui non respondet aliquid in
re, est vana; si ergo istae rationes sunt multae, et res est una,
videtur quod rationes istae sint vanae.

&

もし、それらの名が、事物においては同一のものを表示するが、異なる概念に
おいて表示するのだ、と言われるならば、次のように反対する。事物において、
何ものにも対応しない概念は空虚である。ゆえに、もしそれらの概念が多数で
あって、事物は一つであるならば、それらの概念は空虚だということになる。

\\

3.~{\scshape Praeterea}, magis est unum quod est unum re et ratione, quam
quod est unum re et multiplex ratione. Sed Deus est maxime unus. Ergo
videtur quod non sit unus re et multiplex ratione. Et sic nomina dicta
de Deo non significant rationes diversas, et ita sunt synonyma.

&

さらに、事物においても概念においても一であるものの方が、事物において一
だが概念において多であるものよりもより一である。しかるに、神は最大限に
一である。ゆえに神は事物において一で概念において多であるのではない。ゆ
えに神について語られる名は、様々な概念を表示するのではなく、したがって
それらは同義語である。

\\

{\scshape Sed contra}, omnia synonyma, sibi invicem adiuncta, nugationem
adducunt, sicut si dicatur {\itshape vestis indumentum}. Si igitur omnia
nomina dicta de Deo sunt synonyma, non posset convenienter dici {\itshape
Deus bonus}, vel aliquid huiusmodi; cum tamen scriptum sit Ierem.\
{\scshape xxxii}: {\itshape Fortissime, magne, potens, dominus exercituum nomen
tibi}.

&

しかし反対に、たとえば「衣服は服である」と言われる場合のように、同義語
はすべて相互に結びつけられると無意味な反復となる。ゆえに神について語ら
れるすべての名が同義語だとすると、「神は善である」などのようなことが適
切に言われ得ない。しかし聖書には、『エレミア書』32節「あなたの名は、最
も強く、大きく、力ある、軍隊の主」\footnote{「大いなる神、力ある神、そ
の御名は万軍の主」(32: 18)}とある。

\\

{\scshape Respondeo dicendum} quod huiusmodi nomina dicta de Deo, non sunt
synonyma. Quod quidem facile esset videre, si diceremus quod huiusmodi
nomina sunt inducta ad removendum, vel ad designandum habitudinem
causae respectu creaturarum, sic enim essent diversae rationes horum
nominum secundum diversa negata, vel secundum diversos effectus
connotatos.

&

答えて言わなければならない。神について語られるこれらの名は同義語ではな
い。このことは、もし私たちが、このような名が排除のために、あるいは原因
の結果への関係を示すために付けられたのだと言っていたとしたら、理解する
のが簡単だっただろう。なぜなら、否定されるものがさまざまであり、示され
る結果がさまざまであることに応じて、これらの名のさまざまな概念があった
だろうからである。

\\

Sed secundum quod dictum est huiusmodi nomina substantiam divinam
significare, licet imperfecte, etiam plane apparet, secundum
praemissa, quod habent rationes diversas. Ratio enim quam significat
nomen, est conceptio intellectus de re significata per
nomen. Intellectus autem noster, cum cognoscat Deum ex creaturis,
format ad intelligendum Deum conceptiones proportionatas
perfectionibus procedentibus a Deo in creaturas.

&

しかし、これらの名が、不完全にではあれ神の実体を表示すると言われた限り
においても、いっそう明らかに、これまで述べられたことにしたがって、それ
らがさまざまな概念を持つことが明らかである。というのも、名が表示する概
念(ratio)は、その名によって表示される事物について知性認識された懐念
(conceptio)である。しかるに、私たちの知性は、神を被造物に基づいて認識
するので、神を認識するために、神から被造物へ発出する完全性に比例した懐
念を形成する。

\\

Quae quidem perfectiones in Deo praeexistunt unite et simpliciter, in
creaturis vero recipiuntur divise et multipliciter. Sicut igitur
diversis perfectionibus creaturarum respondet unum simplex principium,
repraesentatum per diversas perfectiones creaturarum varie et
multipliciter; ita variis et multiplicibus conceptibus intellectus
nostri respondet unum omnino simplex, secundum huiusmodi conceptiones
imperfecte intellectum. Et ideo nomina Deo attributa, licet
significent unam rem, tamen, quia significant eam sub rationibus
multis et diversis, non sunt synonyma.

&

それらの完全性は、神において、ひとつで単純なものとして先在するが、被造
物においては、分割されて多様なかたちで受け取られる。ゆえに、ちょうど、
被造物のさまざまな完全性に、被造物のさまざまな完全性によって、さまざま
に多様に表現されるひとつの単純な根源が対応するように、私たちの知性のさ
まざまで多様な概念に、そのような諸懐念によって不完全に認識されたひとつ
のまったく単純なものが、対応する。ゆえに、神に帰される名は、たしかにひ
とつの事物を表示するが、しかし、その事物を多数のさまざまな概念のもとに
表示するので、同義語ではない。

\\

{\scshape Et sic patet solutio ad primum}: quia nomina synonyma dicuntur,
quae significant unum secundum unam rationem. Quae enim significant
rationes diversas unius rei, non primo et per se unum significant,
quia nomen non significat rem, nisi mediante conceptione intellectus,
ut dictum est.

&

以上のようにして、第一異論への解答は明らかである。同義語とは、ひとつの
概念にしたがって、ひとつのものを表示する名のことなのだから。というのも、
ひとつの事物のさまざまな概念を表示する名は、第一に、自体的に、ひとつの
ものを表示するのではない。なぜなら、すでに述べられたように、名は、知性
の懐念を媒介にしなければ、事物を表示しないからである。

\\

{\scshape Ad secundum dicendum} quod rationes plures horum nominum non sunt
cassae et vanae, quia omnibus eis respondet unum quid simplex, per
omnia huiusmodi multipliciter et imperfecte repraesentatum.

&

第二異論に対しては次のように言われるべきである。これらの名の複数の概念
は、空虚でも無意味でもない。なぜなら、これらすべてには、これらすべての
名をとおして多様に、かつ不完全に表現された、ひとつの単純な何かが対応す
るからである。

\\

{\scshape Ad tertium dicendum} quod hoc ipsum ad perfectam Dei unitatem
pertinet, quod ea quae sunt multipliciter et divisim in aliis, in ipso
sunt simpliciter et unite. Et ex hoc contingit quod est unus re et
plures secundum rationem, quia intellectus noster ita multipliciter
apprehendit eum, sicut res multipliciter ipsum repraesentant.

&

第三異論に対しては、次のように言われるべきである。他のものにおいて多様
で分割されてあるものが、神において単純にひとつのものとしてある、という
こと自体が、神の完全な一性に属する。そしてこのことから、事物においてひ
とつであり、かつ概念において複数だということが生じる。なぜなら、私たち
の知性は、事物が神を多様に表現するのと同じように、神を多様にとらえるか
らである。

\end{longtable}

\newpage
\fancyhead[R]{a.~5}
\begin{center}
{\Large {\bfseries ARTICULUS QUINTUS}}\\
{UTRUM EA QUAE DE DEO DICUNTUR ET CREATURIS, UNIVOCE DICANTUR DE IPSIS}
\\{\large 第五項\\神と被造物について語られる名は、それらに一義的に語られるか}
\end{center}

\begin{longtable}{p{21em}p{21em}}

{\scshape Ad quintum sic proceditur}. Videtur quod ea quae dicuntur de Deo
et creaturis, univoce de ipsis dicantur. Omne enim aequivocum
reducitur ad univocum, sicut multa ad unum. Nam si hoc nomen {\itshape
canis} aequivoce dicitur de latrabili,
%\footnote{l\={a}tr\={a}bilis, e, adj: barking} 
et marino
%\footnote{m\u{a}r\={\i}nus, a, um, adj.: of or belonging to the sea}
oportet quod de aliquibus univoce dicatur, scilicet de omnibus
latrabilibus, aliter enim esset procedere in infinitum.

&

第五項の問題へ議論は以下のように進められる。神と被造物について語られる
ものは、一義的にそれらについて語られると思われる。理由は以下の通り。ちょ
うど、多が一に還元されるように、すべての異義語は一義語に還元される。た
とえば、「canis」というこの名が、吠えるもの(犬)と、海にいるもの(ア
ザラシ)について異義的に語られるとしても、あるものどもについて、すなわ
ち、全ての吠えるものどもについては一義的に語られなければならない。そう
でなければ、無限に進むことができただろう。\footnote{canisという語が犬
とアザラシについて異義的に語られるとする。しかし犬について語られるかぎ
りでのcanisは一義的である。もしそれが異義的だとすると、canisと呼ばれる
「犬」をさらに限定して、一義的に語られる犬の部分集合「犬A」を取り出す
ことができるはずである。異論は、それが限りなく進むことはありえないとす
るが、じっさいには個体に到達して止まる。この場合、ある一頭の犬が典型的
な「犬」であり、その他の犬は、その典型的な「犬」に似ているなどの関係を
もつかぎりで「犬のようなもの」という意味の省略として「犬」と呼ばれるで
あろう。これはイデア論か?}。

\\

Inveniuntur autem quaedam agentia univoca, quae conveniunt cum suis
effectibus in nomine et definitione, ut homo generat hominem; quaedam
vero agentia aequivoca, sicut sol causat calidum, cum tamen ipse non
sit calidus nisi aequivoce. Videtur igitur quod primum agens, ad quod
omnia agentia reducuntur, sit agens univocum. Et ita, quae de Deo et
creaturis dicuntur, univoce praedicantur.

&

ところで、ある作用者は一義的であり、それらは自らの結果と、名と定義にお
いて一致する。たとえば人間が人間を生む場合のように。他方、ある作用者は
異義的であり、たとえば太陽が結果として熱いものを生み出す場合がそれであ
る。なぜなら、太陽は異義的にでなければ「熱いもの」ではないからである
\footnote{「太陽が熱い」と「たき火が熱い」の「熱い」は意味が異なる。天
体は地上の物体と成り立ちがまったく異なるというアリストテレスの自然学に
基づく主張。}。ゆえに、すべての作用者がそこへ還元されるところの第一作
用者は、一義的作用者であると思われる。したがって、神と被造物について語
られるものは、一義的に語られる。

\\

2.~{\scshape Praeterea}, secundum aequivoca non attenditur aliqua
similitudo. Cum igitur creaturae ad Deum sit aliqua similitudo,
secundum illud {\itshape Genes}.\ {\scshape i}: {\itshape Faciamus hominem ad
imaginem et similitudinem nostram}, videtur quod aliquid univoce de
Deo et creaturis dicatur.

&

さらに、異義的なものどもにおいては、なんらの類似性も見出されない。ゆえ
に、『創世記』第1章「人間を、私達の像と類似に向けて作ろう」
\footnote{「我々にかたどり、我々に似せて、人を造ろう」(1:26)}によれば、
被造物の神へのなんらかの類似があるのだから、何かが神と被造物について一
義的に語られると思われる。

\\

3.~{\scshape Praeterea}, mensura est homogenea mensurato, ut dicitur in X
{\itshape Metaphys}. Sed Deus est prima mensura omnium entium, ut ibidem
dicitur. Ergo Deus est homogeneus creaturis. Et ita aliquid univoce de
Deo et creaturis dici potest.

&

さらに、『形而上学』10巻\footnote{第1章。「つねに尺度は〔この尺度で測
られる事物と〕同種的である。」\textgreek{>ae`i d`e suggen`es t`o
m'etron.}($1053^{a}24-5$)}で言われるように、尺度は尺度によって測られる
ものと類を同じくする。しかるに、同じ箇所で言われるように、神は全て存在
するものの第一の尺度である。ゆえに神は被造物と類を同じくする。したがっ
て何かが神と被造物について一義的に語られうる。

\\

{\scshape Sed contra}, quidquid praedicatur de aliquibus secundum idem
nomen et non secundum eandem rationem, praedicatur de eis
aequivoce. Sed nullum nomen convenit Deo secundum illam rationem,
secundum quam dicitur de creatura, nam sapientia in creaturis est
qualitas, non autem in Deo; genus autem variatum mutat rationem, cum
sit pars definitionis. Et eadem ratio est in aliis. Quidquid ergo de
Deo et creaturis dicitur, aequivoce dicitur.

&

しかし反対に、あるものどもについて、同じ名において、かつ、同じではない
概念において述語されるものはすべて、それらについて異義的に述語される。
しかるに、どんな名も、それが被造物について語られるような概念に即して、
神に適合することはない。たとえば知恵は、被造物においては性質だが、神に
おいてはそうでない。しかし類は定義の部分だから類が変われば概念も変わる。
% \footnote{被造物において、知恵は性質という類に属するが、神においてはそ
% うでない。したがって、被造物の知恵と神の知恵は類を異にする。類が違うと
% 概念が異なる。ゆえに、被造物の知恵と神の知恵は概念が異なる。すなわち異
% 義語である。}
他の名称についても同様である。ゆえに神と被造物について言われるものはな
んであれ異義的に語られる。

\\

2.~{\scshape Praeterea}, Deus plus distat a creaturis, quam quaecumque
creaturae ab invicem. Sed propter distantiam quarundam creaturarum,
contingit quod nihil univoce de eis praedicari potest; sicut de his
quae non conveniunt in aliquo genere. Ergo multo minus de Deo et
creaturis aliquid univoce praedicatur, sed omnia praedicantur
aequivoce.

&

さらに、神はどんな被造物相互の隔たりよりも被造物から隔たっている。しか
るに、ある被造物の距離のために、それらについて何も一義的に述語されえな
いということが生じている。たとえば、なんらの類においても一致しないもの
どもがそうである。ゆえに神と被造物については、よりいっそう一義的に何か
が語られることがないのであって、すべては異義的に述語される。

\\

{\scshape Respondeo dicendum} quod impossibile est aliquid praedicari de
Deo et creaturis univoce. Quia omnis effectus non adaequans virtutem
causae agentis, recipit similitudinem agentis non secundum eandem
rationem, sed deficienter, ita ut quod divisim et multipliciter est in
effectibus, in causa est simpliciter et eodem modo; sicut sol secundum
unam virtutem, multiformes et varias formas in istis inferioribus
producit. Eodem modo, ut supra dictum est, omnes rerum perfectiones,
quae sunt in rebus creatis divisim et multipliciter, in Deo
praeexistunt unite.

&

答えて言わなければならない。何かが神と被造物について一義的に語られるこ
とは不可能である。なぜなら、作用因の力に及ばない結果はすべて、同じ性格
においてではなく、異なるかたちで、作用者の類似を受け取る。したがって、
結果において分割され多様であるものが、原因においては単純に同一のあり方
で存在する。たとえば、太陽が、ひとつの力にしたがって、その下位のものど
ものうちに多様でさまざまな形相を生み出すように。同様に、前に述べられた
ように、被造の諸事物において分割され多様なかたちで存在する諸事物のすべ
ての完全性は、神において一つのものとして先在する。

\\

Sic igitur, cum aliquod nomen ad perfectionem pertinens de creatura
dicitur, significat illam perfectionem ut distinctam secundum rationem
definitionis ab aliis, puta cum hoc nomen {\itshape sapiens} de homine
dicitur, significamus aliquam perfectionem distinctam ab essentia
hominis, et a potentia et ab esse ipsius, et ab omnibus huiusmodi. Sed
cum hoc nomen de Deo dicimus, non intendimus significare aliquid
distinctum ab essentia vel potentia vel esse ipsius.

&

したがってそれゆえ、完全性に属するある名が被造物について語られるとき、
その完全性を、定義の概念において他から区別されたものとして表示する。た
とえば「知恵あるもの」というこの名が人間について語られるとき、私たちは、
人間の本質や能力や人間の存在や、そのようなすべてのものとは区別されたあ
る完全性を表示する。しかし私たちがこの名を神について語るとき、なにか神
の本質や能力や存在から区別されたものを表示しようと意図するわけではない。

\\

Et sic, cum hoc nomen {\itshape sapiens} de homine dicitur, quodammodo
circumscribit et comprehendit rem significatam, non autem cum dicitur
de Deo, sed relinquit rem significatam ut incomprehensam, et
excedentem nominis significationem. Unde patet quod non secundum
eandem rationem hoc nomen {\itshape sapiens} de Deo et de homine
dicitur. Et eadem ratio est de aliis. Unde nullum nomen univoce de Deo
et creaturis praedicatur.

&

したがって、「知恵あるもの」というこの名が人間について語られるとき、な
んらかのかたちで、[その名は]表示された事物を取り囲み包含するが、神に
ついて語られるときはそのようなことはなく、むしろ表示された事物を包みき
れないもの、名の意味を超えるものとして残す。したがってこの「知恵あるも
の」という名が同一の概念において神と人間について語られるのではないこと
が明らかである。これは他の名についても同様である。したがってどんな名も、
神と被造物について一義的に述語されることはない。

\\

Sed nec etiam pure aequivoce, ut aliqui dixerunt. Quia secundum hoc,
ex creaturis nihil posset cognosci de Deo, nec demonstrari; sed semper
incideret fallacia aequivocationis. Et hoc est tam contra philosophos,
qui multa demonstrative de Deo probant, quam etiam contra apostolum
dicentem, {\itshape Rom}. {\scshape i}: {\itshape invisibilia Dei per ea quae facta
sunt, intellecta, conspiciuntur}.

&

しかし、ある人々が言ったように、純粋に異義的に語られるというわけでもな
い。なぜなら、この説に従えば、被造物に基づいては、神について何も認識さ
れえず、また、論証もされえない。むしろ常に、異義性の誤謬に陥ってしまう。
これは、多くのことを神について証明している哲学者に反するのと同様、『ロー
マの信徒への手紙』1章「神の見られえない事柄は作られてものどもを通して
認識され、見られる」\footnote{「目に見えない神の性質、つまり神の永遠の
力と神性は被造物に現れており、これを通して神を知ることができます」
(1:20) }と述べる使徒にも反する。

\\

Dicendum est igitur quod huiusmodi nomina dicuntur de Deo et creaturis
secundum analogiam, idest proportionem. Quod quidem dupliciter
contingit in nominibus, vel quia multa habent proportionem ad unum,
sicut sanum dicitur de medicina et urina, inquantum utrumque habet
ordinem et proportionem ad sanitatem animalis, cuius hoc quidem signum
est, illud vero causa; vel ex eo quod unum habet proportionem ad
alterum, sicut sanum dicitur de medicina et animali, inquantum
medicina est causa sanitatis quae est in animali.

&

ゆえに、次のように言わなければならない。このような名は、神と被造物につ
いて、アナロギア、すなわち比例にしたがって語られる。名において、これは
二通りのかたちで生じる。ひとつは、多くのものが一つのものに比を有する場
合である。たとえば、「健康な」が薬と尿について語られるが、それは、どち
らも動物の健康に秩序と比を有するかぎりにおいてである。つまり、後者はそ
の徴候であり、前者はその原因である。もうひとつは、一つのものが別のもの
に比を有する場合である。たとえば、「健康な」が、薬が動物においてある健
康の原因であるかぎりにおいて、薬と動物について語られるように。

\\

Et hoc modo aliqua dicuntur de Deo et creaturis analogice, et non
aequivoce pure, neque univoce. Non enim possumus nominare Deum nisi ex
creaturis, ut supra dictum est. Et sic, quidquid dicitur de Deo et
creaturis, dicitur secundum quod est aliquis ordo creaturae ad Deum,
ut ad principium et causam, in qua praeexistunt excellenter omnes
rerum perfectiones.

&

そして、この後者のかたちで、何かが神と被造物についてアナロギア的に語ら
れる。それは純粋に異義的にでも、一義的にでもない。じっさい、上で示され
たように、私たちは、被造物に基づいてでなければ、神を名付けることができ
ない。したがって、何であれ神と被造物について語られるものは、諸事物のす
べての完全性がより優れたかたちで先在する根源や原因としての神にたいする、
被造物の何らかの秩序に基づいて、語られる。

\\

Et iste modus communitatis medius est inter puram aequivocationem et
simplicem univocationem. Neque enim in his quae analogice dicuntur,
est una ratio, sicut est in univocis; nec totaliter diversa, sicut in
aequivocis; sed nomen quod sic multipliciter dicitur, significat
diversas proportiones ad aliquid unum; sicut {\itshape sanum}, de urina
dictum, significat signum sanitatis animalis, de medicina vero dictum,
significat causam eiusdem sanitatis.

&

そして、この共通性のあり方は、純粋な異義性と単純な一義性の中間にある。
なぜなら、アナロギア的に語られるものどもにおいては、一義的な語における
ように、一つの概念が存在するわけではないが、しかし、異義的な語における
ように、まったく異なる概念があるわけでもないからである。むしろ、このよ
うに多様に語られる名は、ある一つのものに対するさまざまな比を表示する。
たとえば、尿について語られる「健康な」は、動物の健康の徴候を表示し、薬
について語られた「健康な」は、動物の健康の原因を表示する。

\\

{\scshape Ad primum ergo dicendum} quod, licet in praedicationibus oporteat
aequivoca ad univoca reduci, tamen in actionibus agens non univocum ex
necessitate praecedit agens univocum. Agens enim non univocum est
causa universalis totius speciei, ut sol est causa generationis omnium
hominum. Agens vero univocum non est causa agens universalis totius
speciei (alioquin esset causa sui ipsius, cum sub specie contineatur),
sed est causa particularis respectu huius individui, quod in
participatione speciei constituit.  Causa igitur universalis totius
speciei non est agens univocum.

&

第一異論に対してはそれゆえ次のように言われるべきである。たしかに、述語
付けにおいて、異義語は一義語に還元されるが、しかし、作用者の働きにおい
ては、一義的でない作用者が、必然的に、一義的な作用者に先行する。なぜな
ら、太陽が、すべての人間の生成の原因であるように、一義的でない作用者は、
種全体の普遍的な原因だからである。これに対して、一義的作用者は、種全体
の普遍的な作用因ではなく(もしそうだったら、それ自体も種に含まれるのだ
から、自己自身の原因となったであろう)、種を分有することにおいて成立す
る、この個物にかんする、個別的な原因である。ゆえに、種全体の普遍的な原
因は、一義的作用者ではない。

\\

Causa autem universalis est prior particulari. Hoc autem agens
universale, licet non sit univocum, non tamen est omnino aequivocum,
quia sic non faceret sibi simile; sed potest dici agens analogicum,
sicut in praedicationibus omnia univoca reducuntur ad unum primum, non
univocum, sed analogicum, quod est ens.

&

ところで普遍的な原因は個別的な原因に先行する。しかもこの普遍的作用者は
一義的でないにしてもまったく異義的でもない。なぜなら、もしまったく異義
的であったならば、自らに似たものを生み出すということがなかったであろう
から。むしろそれはアナロギア的作用者と言われうる。ちょうど、カテゴリー
において、あらゆる一義的な名が一つの第一のものに還元されるが、その第一
のものは一義的ではなくアナロギア的なもの、すなわち「在るもの・有」であ
るように。

\\

{\scshape Ad secundum dicendum} quod similitudo creaturae ad Deum est
imperfecta, quia etiam nec idem secundum genus repraesentat, ut supra
dictum est.

&

第二異論に対しては次のように言われるべきである。被造物の神に対する類似
は不完全である。なぜなら、上で述べられたとおり\footnote{第1部第4問第3
項「何らかの被造物は神に似たものでありうるか」。}、類において同一のも
のを表現しないからである。

\\

{\scshape Ad tertium dicendum} quod Deus non est mensura proportionata
mensuratis. Unde non oportet quod Deus et creaturae sub uno genere
contineantur.

&

第三異論に対しては次のように言われるべきである。神は測られるものに比例
した尺度ではない。したがって、神と被造物が一つの類のもとに含まれなくて
はならないということはない。

\\

Ea vero quae sunt in contrarium, concludunt quod non univoce huiusmodi
nomina de Deo et creaturis praedicentur, non autem quod aequivoce.

&

反対異論は、これらの名が、神と被造物について一義的に述語されるのではな
いと結論しているが、しかし異義的に述語されるのでもない。

\end{longtable}

\newpage
\fancyhead[R]{a.~6}

\begin{center}
{\Large {\bfseries ARTICULUS SEXTUS}}\\
{\large UTRUM NOMINA PER PRIUS DICANTUR DE CREATURIS QUAM DE DEO}\\
{\footnotesize Supra, a.~3; I {\itshape Sent.}, d.~22, a.~2; I
{\itshape SCG.}, c.~34; Compend.~Theol.}, c.~27; {\itshape Ephes.},
c.~3, lect.~4.\\
{\Large 第六項\\神よりも先に被造物について名が語られるか}
\end{center}

\begin{longtable}{p{21em}p{21em}}
{\huge A}{\scshape d sextum sic proceditur}. Videtur quod nomina per
prius dicantur de creaturis quam de Deo. Secundum enim quod
cognoscimus aliquid, secundum hoc illud nominamus; cum nomina,
secundum philosophum, sint signa intellectuum. Sed per prius
cognoscimus creaturam quam Deum. Ergo nomina a nobis imposita, per
prius conveniunt creaturis quam Deo.

&

第六項の問題へ議論は以下のように進められる。神より先に被造物について名
が語られると思われる。理由は以下の通り。私たちが何かを認識する限りにお
いて、私たちはそれを名付ける。というのも、哲学者によれば、名とは知性認
識内容のしるし(記号)だからである。しかるに、私たちは、神よりも先に被
造物を認識する。ゆえに、私たちに付けられた名は、神より先に被造物に適合
する。

\\

2.~{\scshape Praeterea}, secundum Dionysium, in libro de {\itshape
Div.~Nom.}, Deum ex creaturis nominamus. Sed nomina a creaturis
translata in Deum, per prius dicuntur de creaturis quam de Deo; sicut
{\itshape leo}, {\itshape lapis}, et huiusmodi. Ergo omnia nomina quae
de Deo et de creaturis dicuntur, per prius de creaturis quam de Deo
dicuntur.

&

さらに、『神名論』のディオニュシウスによれば、私たちは、被造物に基づい
て神に名を付ける。しかるに、「獅子」や「石」のように、被造物から神へと
移された名は、神より先に被造物について語られている。ゆえに、神と被造物
について語られるすべての名は、神より先に被造物について語られる。

\\

3.~{\scshape Praeterea}, omnia nomina quae communiter de Deo et
creaturis dicuntur, dicuntur de Deo sicut de causa omnium, ut dicit
Dionysius. Sed quod dicitur de aliquo per causam, per posterius de
illo dicitur, per prius enim dicitur animal sanum quam medicina, quae
est causa sanitatis. Ergo huiusmodi nomina per prius dicuntur de
creaturis quam de Deo.

&

さらに、ディオニュシウスが言うように、神と被造物について共通に語られ
る名は、神について、万物の原因として語られる。しかるに、何かについて、
原因として語られるものは、それについて、より後に語られる。たとえば、薬
が[動物の]健康の原因であるとしても、薬よりも先に動物が「健康」と言わ
れる。ゆえに、このような名は、神より先に被造物について語られる。

\\

{\scshape Sed contra est} quod dicitur {\itshape Ephes}.~{\scshape
iii}, {\itshape Flecto genua mea ad Patrem Domini nostri Iesu, ex quo
omnis paternitas in caelo et in terra nominatur}. Et eadem ratio
videtur de nominibus aliis quae de Deo et creaturis dicuntur. Ergo
huiusmodi nomina per prius de Deo quam de creaturis dicuntur.

&

しかし反対に、『エフェソの信徒への手紙』3章で「わたしは、私たちの主人
であるイエスの父に向いてひざまずくが、そこからすべての父性が、天と地に
おいて名付けられる」\footnote{「こういうわけで、わたしは御父の前にひざ
まずいて祈ります。御父から、天と地にあるすべての家族がその名を与えられ
ています。」(3:14, 15) }と言われている。神と被造物について語られるすべ
ての名について、同じ理屈があると思われる。ゆえに、このような名は、被造
物より先に神について語られる。

\\

{\scshape Respondeo dicendum} quod in omnibus nominibus quae de
pluribus analogice dicuntur, necesse est quod omnia dicantur per
respectum ad unum, et ideo illud unum oportet quod ponatur in
definitione omnium.

&

答えて言わなければならない。複数のものについてアナロギア的に語られるす
べての名において、すべてのものが、一つのものとの関係によって語られること
が必要である。ゆえに、その一つのものが、すべてのものの定義の中に置かれな
ければならない。

\\

Et quia ratio quam significat nomen, est definitio, ut dicitur in IV
{\itshape Metaphys}., necesse est quod illud nomen per prius dicatur
de eo quod ponitur in definitione aliorum, et per posterius de aliis,
secundum ordinem quo appropinquant ad illud primum vel magis vel
minus, sicut {\itshape sanum} quod dicitur de animali, cadit in
definitione {\itshape sani} quod dicitur de medicina, quae dicitur
{\itshape sana} inquantum causat sanitatem in animali; et in
definitione {\itshape sani} quod dicitur de urina, quae dicitur
{\itshape sana} inquantum est signum sanitatis animalis.

&

また、『形而上学』第4巻で言われるように、名が表示する概念は定義である
から、その名は、他のものどもの定義の中に置かれるものについてより先に語
られ、その他のものどもについては、かの第一のものに対してより多くあるい
はより少なく接近することに応じて、より後に語られるのが必然である。たと
えば、動物について語られる「健康」は、動物における健康を結果として生み
出すかぎりにおいて薬について言われる「健康」の定義の中に入るし、また、
動物の健康の徴候である限りにおいて「健康」と言われる尿についての「健康」
の定義にも入る。

\\

Sic ergo omnia nomina quae metaphorice de Deo dicuntur, per prius de
creaturis dicuntur quam de Deo, quia dicta de Deo, nihil aliud
significant quam similitudines ad tales creaturas.

&

それゆえ、この同じ理由で、神について比喩的に語られるすべての名は、神よ
りも被造物について先に語られる。というのも、神について語られた[比喩的
な]名は、そのような被造物への類似以外を表示しないからである。

\\

Sicut enim {\itshape ridere}, dictum de prato, nihil aliud significat
quam quod pratum similiter se habet in decore cum floret, sicut homo
cum ridet, secundum similitudinem proportionis; sic nomen {\itshape
leonis}, dictum de Deo, nihil aliud significat quam quod Deus
similiter se habet ut fortiter operetur in suis operibus, sicut leo in
suis. Et sic patet quod, secundum quod dicuntur de Deo, eorum
significatio definiri non potest, nisi per illud quod de creaturis
dicitur.

&

たとえば、草原について言われた「笑う」は、草原が花咲いたときに、比の類
似にしたがって、人間が笑うときに似た美しさになるということをまさに表示
する。同様に、神について言われる「獅子」という名は、神がその業において、
獅子と同様に力強く働くことをまさに表示する。このように、そられの名の意
味は、それが神について語られるかぎりにおいては定義され得ず、被造物につ
いて語られることによってのみ定義されることが明らかである。

\\

De aliis autem nominibus, quae non metaphorice dicuntur de Deo, esset
etiam eadem ratio, si dicerentur de Deo causaliter tantum, ut quidam
posuerunt. Sic enim, cum dicitur {\itshape Deus est bonus}, nihil
aliud esset quam {\itshape Deus est causa bonitatis creaturae}, et sic
hoc nomen {\itshape bonum}, dictum de Deo, clauderet in suo intellectu
bonitatem creaturae. Unde {\itshape bonum} per prius diceretur de
creatura quam de Deo.

&

他方神について、比喩的でない意味で語られる名についても、もしある人々が
考えたように、それらが原因という観点からのみ神について語られるのだった
ならば、同じ議論ができただろう。つまりその場合には「神は善である」と言
われるとき「神は被造物の善性の原因である」以外のことを表示しないことに
なり、神について言われるこの「善」という名が、その意味内容の中に被造物
の善性を含むことになっていただろう。このことから「善」は神より先に被造
物について語られたであろう。

\\

Sed supra ostensum est quod huiusmodi nomina non solum dicuntur de Deo
causaliter, sed etiam essentialiter. Cum enim dicitur {\itshape Deus
est bonus}, vel {\itshape sapiens}, non solum significatur quod ipse
sit causa sapientiae vel bonitatis, sed quod haec in eo eminentius
praeexistunt. Unde, secundum hoc, dicendum est quod, quantum ad rem
significatam per nomen, per prius dicuntur de Deo quam de creaturis,
quia a Deo huiusmodi perfectiones in creaturas manant. Sed quantum ad
impositionem nominis, per prius a nobis imponuntur creaturis, quas
prius cognoscimus. Unde et modum significandi habent qui competit
creaturis, ut supra dictum est.

&

しかし、このような名は、神について、原因という観点からのみ語られるので
はなく、本質的にも語られることが前に示された。つまり、「神は善である」
とか「神は知者である」とかと言われるとき、神が知恵や善性の原因だという
ことが表示されるだけでなく、これらが神の中に、より優れたかたちで先在し
ていることもまた表示される。したがって、この点にかんする限り、名によっ
て表示される事物にかんしては、被造物より先に神について[これらの名が]
語られると言われるべきである。なぜなら、神から、このような完全性が被造
物へと流れ出るからである。しかし、命名にかんする限りでは、私たちがより
先に認識する被造物に、[それらの名が]私たちによって付けられる。したがっ
て、上述のように、被造物に適合する表示様態も[それらの名は]もつのであ
る。

\\

{\scshape Ad primum ergo dicendum} quod obiectio illa procedit quantum
ad impositionem nominis.

&

第一異論に対してはそれゆえ次のように言われるべきである。この反論は、命
名にかんして論じている。

\\

{\scshape Ad secundum dicendum} quod non est eadem ratio de nominibus
quae metaphorice de Deo dicuntur, et de aliis, ut dictum est.

&

第二異論に対しては、次のように言われるべきである。すでに述べられたよう
に、神について比喩的に語られる名については、同じことは言えない。

\\

{\scshape Ad tertium dicendum} quod obiectio illa procederet, si
huiusmodi nomina solum de Deo causaliter dicerentur et non
essentialiter, sicut sanum de medicina.

&

第三異論に対しては、次のように言われるべきである。その反論は、もしその
ような名が、ちょうど薬について言われる「健康」のように、神について原因
の観点からのみ語られ、本質的に語られなかったとすれば成立したであろう。

\end{longtable}
\newpage
\fancyhead[R]{a.~7}

\begin{center}
{\Large {\bfseries ARTICULUS SEPTIMUS}}\\
{\large UTRUM NOMINA QUAE IMPORTANT RELATIONEM AD CREATURAS, DICANTUR
DE DEO EX TEMPORE}\\
{\footnotesize Infra, q.~34, a.~3, ad 2; I {\itshape Sent.}, d.~30,
a.~1; d.~37, q.~2, a.~3.}\\
{\Large 第七項\\被造物への関係を含む名は神について時間的に語られるか}
\end{center}

\begin{longtable}{p{21em}p{21em}}

{\huge A}{\scshape d septimum sic proceditur}. Videtur quod nomina
quae important relationem ad creaturas, non dicantur de Deo ex
tempore. Omnia enim huiusmodi nomina significant divinam substantiam,
ut communiter dicitur. Unde et Ambrosius dicit quod hoc nomen
{\itshape Dominus} est nomen potestatis, quae est divina substantia,
et {\itshape Creator} significat Dei actionem, quae est eius
essentia. Sed divina substantia non est temporalis, sed aeterna. Ergo
huiusmodi nomina non dicuntur de Deo ex tempore, sed ab aeterno.

&

第七項の問題へ議論は以下のように進められる。被造物への関係を含む名は神
について時間的に語られるのではないと思われる。理由は以下の通り。このよ
うなすべての名は、[諸権威によって]共通に言われているように、神の実体
を表示する。このことからアンブロシウスも、この「主人」という名は神の実
体である権能の名であり、「創造者」は神の実体である神の働きを表示すると
言う。しかるに神の実体は時間的ではなく永遠である。ゆえにこのような名は
神について時間的に語られるのではなく永遠から語られる。

\\

2.~{\scshape Praeterea}, cuicumque convenit aliquid ex tempore, potest
dici factum, quod enim ex tempore est album, fit album. Sed Deo non
convenit esse factum. Ergo de Deo nihil praedicatur ex tempore.

&

さらに、時間的に何かが適合するものは何であれ、「なった」と言われうる。
たとえば、時間的に白いものは、「白くなる」[と言われうる]。しかし神に
は「なった」ということは適合しない。ゆえに神については何も時間的に述語
付けされない。

\\

3.~{\scshape Praeterea}, si aliqua nomina dicuntur de Deo ex tempore
propter hoc quod important relationem ad creaturas, eadem ratio
videtur de omnibus quae relationem ad creaturas important. Sed quaedam
nomina importantia relationem ad creaturas, dicuntur de Deo ab
aeterno, ab aeterno enim scivit creaturam et dilexit, secundum illud
{\itshape Ierem}.~{\scshape xxxi}, {\itshape in caritate perpetua
dilexi te}. Ergo et alia nomina quae important relationem ad
creaturas, ut {\itshape Dominus} et {\itshape Creator}, dicuntur de
Deo ab aeterno.

&

さらにもし、被造物への関係を含むからという理由でなんらかの名が神につい
て時間的に言われるとすると、被造物への関係を含むすべての名について同じ
ことが言えると思われる。しかし被造物への関係を含むある名は、神について
永遠から言われる。たとえばかの『エレミア書』31章「わたしは永遠の愛にお
いてあなたを愛した」\footnote{「遠くから、主はわたしに現れた。わたしは、
とこしえの愛をもってあなたを愛し、変わることなく慈しみを注ぐ。」
(31:3)}によれば、神は永遠から被造物を知りそして愛した。ゆえに「主人」
や「創造者」のような被造物への関係を含む他の名も、神について永遠から言
われる。

\\

4.~{\scshape Praeterea}, huiusmodi nomina relationem
significant. Oportet igitur quod relatio illa vel sit aliquid in Deo,
vel in creatura tantum. Sed non potest esse quod sit in creatura
tantum, quia sic Deus denominaretur {\itshape Dominus} a relatione
opposita, quae est in creaturis; nihil autem denominatur a suo
opposito. Relinquitur ergo quod relatio est etiam aliquid in Deo. Sed
in Deo nihil potest esse ex tempore, cum ipse sit supra tempus. Ergo
videtur quod huiusmodi nomina non dicantur de Deo ex tempore.

&

さらに、このような名は関係を表示する。ゆえに、その関係は、神において在
る何かであるか、あるいは、被造物のみにおいて在る何かかのどちらかである。
しかるに、被造物においてのみ在る何かではありえない。なぜなら、その場合、
被造物において在る反対の関係[つまり「僕」という関係]によって、神が主
人と名付けられたことになるが、何ものも、自らに反対のものによっては名付
けられないからである。ゆえに、その関係は、神においても何かであることが
帰結する。しかし、神は時間を越えているから、神においては、何も時間的に
ありえない。ゆえに、このような名は、時間的に神について言われるのでない
と思われる。

\\

5.~{\scshape Praeterea}, secundum relationem dicitur aliquid relative,
puta secundum dominium dominus, sicut secundum albedinem albus. Si
igitur relatio dominii non est in Deo secundum rem, sed solum secundum
rationem, sequitur quod Deus non sit realiter dominus, quod patet esse
falsum.

&

さらに、ちょうど、あるものが白性によって「白い」と言われるように、関係
にしたがって、関係的に語られる。たとえば、支配にしたがって「主人」と言
われる。ゆえに、もし、支配という関係が、神の中に事物に即して存在してお
らず、たんに概念に即してのみ存在しているとすれば、神は、事実として主人
であるわけではないことになるが、これは、明らかに偽である。

\\

6.~{\scshape Praeterea}, in relativis quae non sunt simul natura, unum
potest esse, altero non existente, sicut scibile existit, non
existente scientia, ut dicitur in {\itshape Praedicamentis}. Sed
relativa quae dicuntur de Deo et creaturis, non sunt simul
natura. Ergo potest aliquid dici relative de Deo ad creaturam, etiam
creatura non existente. Et sic huiusmodi nomina, {\itshape Dominus} et
{\itshape Creator}, dicuntur de Deo ab aeterno, et non ex tempore.

&

さらに、『カテゴリー論』で言われるように、本性において同時でない関係的
なものどもにおいては、一方が、他方が存在しなくても、存在しうる。たとえ
ば、知識が存在しなくても、知られうるものは存在する。しかるに、神と被造
物について語られる関係的なものは、本性において同時ではない。ゆえに、あ
る名は、神について、被造物が存在しなくても被造物との関係において語られ
うる。このようにして、「主人」や「創造者」のような名は、神について永遠
に語られるのであり、時間的にではない。

\\

{\scshape Sed contra est} quod dicit Augustinus, V {\itshape de
Trin}., quod haec relativa appellatio {\itshape Dominus} Deo convenit
ex tempore.

&

しかし反対に、アウグスティヌスは『三位一体論』5巻で、この「主人」とい
う関係的な名付けは、時間から神に適合すると述べている。

\\

{\scshape Respondeo dicendum} quod quaedam nomina importantia
relationem ad creaturam, ex tempore de Deo dicuntur, et non ab
aeterno.

&

解答する。以下のように言われるべきである。被造物への関係を含む名の中に
は、永遠からではなく、時間的に、神について語られるものがある。

\\

Ad cuius evidentiam, sciendum est quod quidam posuerunt relationem non
esse rem naturae, sed rationis tantum. Quod quidem apparet esse
falsum, ex hoc quod ipsae res naturalem ordinem et habitudinem habent
ad invicem.

&

これを明らかにするためには次のことが知られるべきである。ある人々は、関
係が、本性の事物[=実在的なもの]でなく、ただ概念の事物[=概念的なも
の]に過ぎないと考えた。このことは、事物自体が相互に自然的な秩序と関係
を持っていることから偽であることが明らかである。

\\

Veruntamen sciendum est quod, cum relatio requirat duo extrema,
tripliciter se habere potest ad hoc quod sit res naturae et
rationis. Quandoque enim ex utraque parte est res rationis tantum,
quando scilicet ordo vel habitudo non potest esse inter aliqua, nisi
secundum apprehensionem rationis tantum, utpote cum dicimus {\itshape
idem eidem idem}.

&

しかしまた次のことも知られるべきである。関係は、二つの端を必要とするの
で、関係が実在的か概念的かということについて、三通りの場合がありうる。
すなわち、ある場合には、[関係は]どちらの側からもたんに概念的である。
たとえば、「同一のものは同一のものにとって同一である」と私たちが言うと
きのように、理性の把握においてのみ以外には、秩序や関係が、何らかのもの
のあいだに存在しえないような場合である。

\\

Nam secundum quod ratio apprehendit bis aliquod unum, statuit illud ut
duo; et sic apprehendit quandam habitudinem ipsius ad seipsum. Et
similiter est de omnibus relationibus quae sunt inter ens et non ens;
quas format ratio, inquantum apprehendit non ens ut quoddam
extremum. Et idem est de omnibus relationibus quae consequuntur actum
rationis, ut genus et species, et huiusmodi.

&

つまり、理性は、なにか一つのものを二度捉え、それを二つのものとして立て、
そうして、そのものの自分自身への何らかの関係を捉える。存在するものと存
在しないものとのあいだにあるすべての関係についても同様である。理性は、
存在しないものを、[関係の]何らかの端として捉え、そういった関係を形成
する。類、種など、理性の働きに伴うすべての関係についてもまた、同様であ
る。

\\

Quaedam vero relationes sunt, quantum ad utrumque extremum, res
naturae, quando scilicet est habitudo inter aliqua duo secundum
aliquid realiter conveniens utrique. Sicut patet de omnibus
relationibus quae consequuntur quantitatem, ut magnum et parvum,
duplum et dimidium, et huiusmodi, nam quantitas est in utroque
extremorum. Et simile est de relationibus quae consequuntur actionem
et passionem, ut motivum et mobile, pater et filius, et similia.

&

他方、ある関係は、そのどちらの端にかんしても実在的であり、それはすなわ
ち、どちらにも実在的に適合する何かに即して、関係が何らかのもののあいだ
にある場合である。たとえばこれは、大と小、二倍と半分のような量に伴うす
べての関係について明らかである。つまりどちらの関係項にも量がある。また
動かしうるものと動かされうるもの、父と息子等々、能動と受動に伴う関係に
ついても同様である。

\\

Quandoque vero relatio in uno extremorum est res naturae, et in altero
est res rationis tantum. Et hoc contingit quandocumque duo extrema non
sunt unius ordinis. Sicut sensus et scientia referuntur ad sensibile
et scibile, quae quidem, inquantum sunt res quaedam in esse naturali
existentes, sunt extra ordinem esse sensibilis et intelligibilis, et
ideo in scientia quidem et sensu est relatio realis, secundum quod
ordinantur ad sciendum vel sentiendum res; sed res ipsae in se
consideratae, sunt extra ordinem huiusmodi.

&

他方また、関係が、両端の一方においては実在的だが、他方においてはたんに
概念的に過ぎないこともある。これは、両端が一つの秩序に属していないとき
にはいつでも生じる。たとえば、感覚や知が、可感的なものや可知的なものに
関係する場合がそれである。可感的なものや可知的なものは、自然的存在にお
いて存在する事物である限りにおいて、可感的なものや可知的なものの存在の
秩序の外にある\footnote{''extra ordinem esse sensibilis et
intelligibilis''のesseを属格の名詞として読む。}。ゆえに、知と感覚にお
いては、それらが事物を知ることや感覚することへと秩序づけられている限り
において、その関係は実在的だが、それ自体において考察された事物そのもの
は、そのような秩序の外にある。

\\

Unde in eis non est aliqua relatio realiter ad scientiam et sensum;
sed secundum rationem tantum, inquantum intellectus apprehendit ea ut
terminos relationum scientiae et sensus. Unde philosophus dicit, in V
{\itshape Metaphys}., quod non dicuntur relative eo quod ipsa
referantur ad alia, sed quia alia referuntur ad ipsa. Et similiter
{\itshape dextrum} non dicitur de columna, nisi inquantum ponitur
animali ad dextram, unde huiusmodi relatio non est realiter in
columna, sed in animali.

&

したがって、それらの事物の中には、知や感覚に対する関係が、実在的にある
わけではなく、たんに概念的に、知性がそれらを知や感覚の関係の端として捉
える限りにおいて、ある。このことから、哲学者は『形而上学』5巻で、それ
が他ものへ関係づけられるからではなく、他のものがそれへ関係づけられるか
ら、関係的に言われるのだと言う。同様に、「右」ということが柱について言
われるのは、動物にとって右側にあるからに他ならない。したがって、このよ
うな関係は、柱において実在的に存在しないが、動物においては実在的である。
\footnote{「左右」などの関係が決まるためには、基準となる方向(どちらを
向いているか)が定まることが必要。動物は、動きなどによってどちらを向い
ているかが分かるので、左右が定まる。しかし、柱などの無生物は、それ自体
ではどちらに向いているかが定まらないので、「柱から見て右」というような
言い方は無意味である。それゆえ、柱にとって、左右は実在的な関係ではない。}

\\

Cum igitur Deus sit extra totum ordinem creaturae, et omnes creaturae
ordinentur ad ipsum, et non e converso, manifestum est quod creaturae
realiter referuntur ad ipsum Deum; sed in Deo non est aliqua realis
relatio eius ad creaturas, sed secundum rationem tantum, inquantum
creaturae referuntur ad ipsum. Et sic nihil prohibet huiusmodi nomina
importantia relationem ad creaturam, praedicari de Deo ex tempore, non
propter aliquam mutationem ipsius, sed propter creaturae mutationem;
sicut columna fit dextera animali, nulla mutatione circa ipsam
existente, sed animali translato.

&

ゆえに、神は被造物の全秩序の外にあり、すべての被造物は神に秩序づけられ
るが、逆はそうでないのだから、被造物が実在的に神自身へ関係づけられるこ
とは明らかである。しかし、神の中には、神の被造物へのどんな実在的関係も
存在せず、被造物が神へ関係する限りにおいて、たんに概念的な関係が存在す
るに過ぎない。この意味で、被造物への関係を含むこのような名が、時間的に
神に述語づけられることを妨げるものはなにもない。それ[が時間的であるの]
は、神自身の何らかの変化のためではなく、被造物の変化のためである。ちょ
うど、柱が動物の右になるのは、柱自身をめぐって存在する何らかの変化によっ
てではなく、移動した動物のためであるように。

\\

{\scshape Ad primum ergo dicendum} quod relativa quaedam sunt imposita
ad significandum ipsas habitudines relativas, ut {\itshape dominus},
{\itshape servus}, {\itshape pater} et {\itshape filius}, et
huiusmodi, et haec dicuntur relativa {\itshape secundum esse}. Quaedam
vero sunt imposita ad significandas res quas consequuntur quaedam
habitudines, sicut {\itshape movens} et {\itshape motum}, {\itshape
caput} et {\itshape capitatum}, et alia huiusmodi, quae dicuntur
relativa {\itshape secundum dici}.

&

第一異論に対してはそれゆえ次のように言われるべきである。関係的な名のあ
るものは、「主人」「奴隷」「父」「子」のように、関係そのものを表示する
ために付けられている。これらは、「存在において」関係的な名と言われる。
他方、ある名は、事物を表示するために付けられるが、その事物に何らかの関
係が伴う。たとえば「動かすもの」「動かされるもの」「頭」「頭を持つもの」
などであり、これらは、「語りにおいて」関係的な名と言われる。

\\

Sic igitur et circa nomina divina haec differentia est
consideranda. Nam quaedam significant ipsam habitudinem ad creaturam,
ut {\itshape Dominus}. Et huiusmodi non significant substantiam
divinam directe, sed indirecte, inquantum praesupponunt ipsam, sicut
dominium praesupponit potestatem, quae est divina substantia.

&

このようだから、それゆえ、神の名についても、この違いが考慮されるべきで
ある。すなわち、「主人」のように、ある名は、被造物への関係自体を表示す
る。そして、このような名は、神の実体を直接には表示せず、間接的に、神の
実体を前提とする限りにおいて、[神の実体を]表示する。たとえば、主人で
あることには、権能が必要だが、この権能は、神の実体である。

\\

Quaedam vero significant directe essentiam divinam, et ex consequenti
important habitudinem; sicut {\itshape Salvator}, {\itshape Creator},
et huiusmodi, significant actionem Dei, quae est eius
essentia. Utraque tamen nomina ex tempore de Deo dicuntur quantum ad
habitudinem quam important, vel principaliter vel consequenter, non
autem quantum ad hoc quod significant essentiam, vel directe vel
indirecte.

&

他方、ある名は神の本質を直接的に表示し、その結果として関係を意味する。
たとえば、「救済者」「創造者」などは、神の働きを表示するが、これらの働
きは神の本質である。しかし、このどちらの名も、それが主要にせよ随伴的に
せよ意味する関係の点で、時間的に神について語られるのであり、直接にせよ
間接にせよ、本質を表示する点で、時間的に神について語られることはない。

\\

{\scshape Ad secundum dicendum} quod, sicut relationes quae de Deo
dicuntur ex tempore, non sunt in Deo nisi secundum rationem, ita nec
fieri nec factum esse dicitur de Deo, nisi secundum rationem, nulla
mutatione circa ipsum existente, sicut est id, {\itshape domine
refugium factus es nobis}.

&

第二異論に対しては、次のように言われるべきである。神について時間的に語
られる関係が神においてあるのは、概念的にである。したがって、神について
はどんな変化もないのだから、「成る」も「作られた」も、概念的にでなけれ
ば、神について語られない。かの「主よ、あなたは私たちにとっての避難所と
なった」\footnote{「主よ、あなたは代々にわたしたちの宿るところ」『詩編』
(90:1)}は、この意味で言われている。

\\

{\scshape Ad tertium dicendum} quod operatio intellectus et voluntatis
est in operante, et ideo nomina quae significant relationes
consequentes actionem intellectus vel voluntatis, dicuntur de Deo ab
aeterno. Quae vero consequuntur actiones procedentes, secundum modum
intelligendi, ad exteriores effectus, dicuntur de Deo ex tempore, ut
{\itshape salvator}, {\itshape creator}, et huiusmodi.

&

第三異論に対しては、次のように言われるべきである。知性と意志の働きは、
働くもののうちにある。ゆえに、知性や意志の作用に伴う関係を表示する名は、
永遠から神について言われる。他方、理解のしかたにおいて、外の結果へと出
て行く作用に伴うものは、時間的に神について言われる。「救済者」「創造者」
などがこのような名である。

\\

{\scshape Ad quartum dicendum} quod relationes significatae per
huiusmodi nomina quae dicuntur de Deo ex tempore, sunt in Deo secundum
rationem tantum, oppositae autem relationes in creaturis sunt secundum
rem. Nec est inconveniens quod a relationibus realiter existentibus in
re, Deus denominetur, tamen secundum quod cointelliguntur per
intellectum nostrum oppositae relationes in Deo. Ut sic Deus dicatur
relative ad creaturam, quia creatura refertur ad ipsum, sicut
philosophus dicit, in V {\itshape Metaphys}., quod scibile dicitur
relative, quia scientia refertur ad ipsum.

&

第四異論に対しては、次のように言われるべきである。神について時間的に語
られるこのような名によって表示される関係は、神において、たんに概念的に
存在するが、対応する関係は、被造物において、事物に即して[=実在的に]
存在する。また、事物において実在的に存在する関係によって神が名付けられ
るのも不都合ではない。ただしそれは、わたしたちの知性によって、対応する
関係が神において共に理解されることによる。ちょうど、哲学者が『形而上学』
5巻で、「知られうるもの」が関係的に言われるのは、知がそれへ関係づけら
れるからだ、と言うように、神が被造物に対して関係的に言われるのは、被造
物が神へ関係づけられるからである。

\\

{\scshape Ad quintum dicendum} quod, cum ea ratione referatur Deus ad
creaturam, qua creatura refertur ad ipsum; cum relatio subiectionis
realiter sit in creatura, sequitur quod Deus non secundum rationem
tantum, sed realiter sit dominus. Eo enim modo dicitur dominus, quo
creatura ei subiecta est.

&

第五異論に対しては、次のように言われるべきである。被造物が神へ関係づけ
られることが根拠となって、神は被造物へ関係づけられるのだから、また、服
従の関係は、実在的に被造物において在るのだから、神はたんに概念的だけで
なく、実在的に主であることが帰結する。被造物が神に服従するそのかたちに
よって、[神は]主と言われる。

\\

{\scshape Ad sextum dicendum} quod, ad cognoscendum utrum relativa
sint simul natura vel non, non oportet considerare ordinem rerum de
quibus relativa dicuntur, sed significationes ipsorum relativorum. Si
enim unum in sui intellectu claudat aliud et e converso, tunc sunt
simul natura, sicut duplum et dimidium, pater et filius, et
similia. Si autem unum in sui intellectu claudat aliud, et non e
converso, tunc non sunt simul natura.

&

第六異論に対しては、次のように言われるべきである。関係的な名が本性にお
いて同時か否かを知るためには、関係的な名がそれについて言われているとこ
ろの事物の秩序を考察する必要はなく、それら関係的な名の意味を考察すれば
よい。たとえば、一方が、他方をその内容に含み、逆もまたそうであるならば、
その場合それらは本性において同時である。たとえば「二倍」と「半分」、
「父」と「子」などである。他方、一方が他方をその内容に含むが、逆はそう
でない場合、それらは本性において同時ではない。(次ページに続く)

\\

Et hoc modo se habent scientia et scibile. Nam scibile dicitur
secundum potentiam, scientia autem secundum habitum, vel secundum
actum. Unde scibile, secundum modum suae significationis, praeexistit
scientiae. Sed si accipiatur scibile secundum actum, tunc est simul
cum scientia secundum actum, nam scitum non est aliquid nisi sit eius
scientia. Licet igitur Deus sit prior creaturis, quia tamen in
significatione domini clauditur quod habeat servum, et e converso,
ista duo relativa, dominus et servus, sunt simul natura. Unde Deus non
fuit dominus, antequam haberet creaturam sibi subiectam.

&

「知」と「知られうるもの」は、このような関係にある。つまり、「知られう
るもの」は、可能態において言われるが、「知」は、習態、あるいは現実態に
おいて言われる。したがって、「知られうるもの」は、その意味のありかたに
おいて、知に先立って存在する。しかし、もし「知られうるもの」が、現実態
に即して理解されるならば、その場合には、現実態に即してそれは知と同時で
ある。なぜなら、「知られたもの」は、それの知に他ならないからである。ゆ
えに、神は被造物より先だが、「主」の意味の中に、奴隷を持つことが含まれ、
逆もまたそうなので、「主」と「奴隷」というこの二つの関係的な名は、本性
において同時である。したがって、神は、自らに服従する被造物をもつ前は、
主でなかった。

\end{longtable}
\newpage
\fancyhead[R]{a.~8}

\begin{center}
{\Large {\bfseries ARTICULUS OCTAVUS}}\\
{\large UTRUM HOC NOMEN {\itshape DEUS} SIT NOMEN NATURAE}\\
{\footnotesize I {\itshape Sent.}, d.~2, expos.~lit.}\\
{\Large 第八項\\「神」というこの名は本性の名か}
\end{center}

\begin{longtable}{p{21em}p{21em}}

{\huge A}{\scshape d octavum sic proceditur}. Videtur quod hoc nomen
{\itshape Deus} non sit nomen naturae. Dicit enim Damascenus, in I
libro, quod {\itshape Deus dicitur a theein}, quod est currere, et
{\itshape fovere universa; vel ab aethein, idest ardere (Deus enim
noster ignis consumens est omnem malitiam}); {\itshape vel a
theasthai}, quod est considerare, {\itshape omnia}. Haec autem omnia
ad operationem pertinent. Ergo hoc nomen {\itshape Deus} operationem
significat, et non naturam.

&

第八項の問題へ議論は以下のように進められる。「神」というこの名は本性の
名でないと思われる。理由は以下の通り。ダマスケヌスは[『正統信仰論』の]
1巻で「神はtheein、つまり「走る」\footnote{この``currere''は多くの写本
で ``curare''となっているが、Leonina版は``currere''を採っている
(\textgreek{j'ew} = to run)。}から言われ、宇宙を慈しむ。あるいは、
athein すなわち「火を付ける」から言われる(なぜなら私たちの神は、すべ
ての悪を焼き尽くす火だから)。あるいは、
theasthai\footnote{\textgreek{je'aomai} = to gaze at, behold,
contemplate}、つまり全てのものを「考察する」ことから言われる」と述べて
いるが、これらは全て働きに属する。ゆえに、「神」というこの名は、働きを
表示するのであって、本性を表示するのではない。

\\

2.~{\scshape Praeterea}, secundum hoc aliquid nominatur a nobis,
secundum quod cognoscitur. Sed divina natura est nobis ignota. Ergo
hoc nomen {\itshape Deus} non significat naturam divinam.

&

さらに、何かが私たちによって名付けられるのは、それが認識される限りにお
いてである。しかるに、神の本性は私たちに知られない。ゆえに、この「神」
という名は、神の本性を表示しない。

\\

{\scshape Sed contra est} quod dicit Ambrosius, in libro I {\itshape
de Fide}, quod {\itshape Deus} est nomen naturae.

&

しかし反対に、アンブロシウスは『信仰論』1巻で、「神」は本性の名だと言っ
ている。

\\

{\scshape Respondeo dicendum} quod non est semper idem id a quo
imponitur nomen ad significandum, et id ad quod significandum nomen
imponitur.

&

解答する。以下のように言われるべきである。表示するために、名がどこから
付けられるかと、何を表示するために名が付けられるかとは、必ずしも常に同
じではない。

\\

Sicut enim substantiam rei ex proprietatibus vel operationibus eius
cognoscimus, ita substantiam rei denominamus quandoque ab aliqua eius
operatione vel proprietate, sicut substantiam lapidis denominamus ab
aliqua actione eius, quia laedit pedem; non tamen hoc nomen impositum
est ad significandum hanc actionem, sed substantiam lapidis. Si qua
vero sunt quae secundum se sunt nota nobis, ut calor, frigus, albedo,
et huiusmodi, non ab aliis denominantur. Unde in talibus idem est quod
nomen significat, et id a quo imponitur nomen ad significandum.

&

ちょうど私たちが事物の実体を、その属性や働きに基づいて認識するように、
私たちは事物の実体をその働きや属性から名付けることがある。たとえば私た
ちは、石の実体を、それの何らかの働きから名付ける。つまり足を傷つけると
いうことから。しかしこの[「石」という]名は、この働きを表示するためで
はなく、石の実体を表示するために付けられた。しかし、熱さ、冷たさ、白な
ど、なんであれそれ自体において私たちに知られているものであれば、それら
が他のものによって名付けられることはない。したがって、そのようなものに
おいては、名が表示するものと、表示するために名がそこから付けられたとこ
ろのものとは同じである。

\\

Quia igitur Deus non est notus nobis in sui natura, sed innotescit
nobis ex operationibus vel effectibus eius, ex his possumus eum
nominare, ut supra dictum est. Unde hoc nomen {\itshape Deus} est
nomen operationis, quantum ad id a quo imponitur ad
significandum. Imponitur enim hoc nomen ab universali rerum
providentia, omnes enim loquentes de Deo, hoc intendunt nominare Deum,
quod habet providentiam universalem de rebus. Unde dicit Dionysius,
{\scshape xii} cap.~{\itshape de Div.~Nom.}, quod {\itshape deitas est
quae omnia videt providentia et bonitate perfecta}. Ex hac autem
operatione hoc nomen {\itshape Deus} assumptum, impositum est ad
significandum divinam naturam.

& ゆえに、神は私たちにその本性において知られず、神の働きや結果に基づい
て、私たちに[自らを]知らせるのだから、既述のように、これらに基づいて、
私たちは神を名付けることができる。したがって、この「神」という名は、表
示するためにそこから付けられたところのものにかんする限り、働きの名であ
る。なぜなら、この名は、諸事物の普遍的な摂理から付けられていて、神につ
いて語る全ての人は、諸事物について普遍的な摂理をもつことを、神と名付け
ることを意図しているからである。このことから、ディオニュシウスは『神名
論』12章で「神性とは、完全な摂理と善性によって万物を見るものである」と
述べる。しかし、この働きから「神」というこの名は取られたが、神の本性を
表示するために付けられた。

\\

{\scshape Ad primum ergo dicendum} quod omnia quae posuit Damascenus,
pertinent ad providentiam, a qua imponitur hoc nomen Deus ad
significandum.

&

第一異論に対してはそれゆえ以下のように言われるべきである。ダマスケヌス
がそこで書いた全てのことは、摂理に関係している。その摂理から、この「神」
という名は、表示するために付けられた。

\\

{\scshape Ad secundum dicendum} quod, secundum quod naturam alicuius
rei ex eius proprietatibus et effectibus cognoscere possumus, sic eam
nomine possumus significare. Unde, quia substantiam lapidis ex eius
proprietate possumus cognoscere secundum seipsam, sciendo quid est
lapis, hoc nomen lapis ipsam lapidis naturam, secundum quod in se est,
significat, significat enim definitionem lapidis, per quam scimus quid
est lapis.  Ratio enim quam significat nomen, est definitio, ut
dicitur in IV {\itshape Metaphys}.

&

第二異論に対しては次のように言われるべきである。私たちがある事物の本性
をそれの固有性や結果に基づいて認識することができるかぎりにおいて、私た
ちはその本性を名によって表示することができる。したがって、私たちは石の
実体を、それの固有性に基づいてそれ自体に即して石が何であるかを知ること
によって、認識することができるので、「石」というこの名は石の本性そのも
のを、それ自体において表示する。なぜなら、それは私たちがそれを通して石
が何かを知るところの石の定義を表示するからである。なぜなら『形而上学』
4巻で言われるように、名が表示する概念は定義なのだから。

\\

Sed ex effectibus divinis divinam naturam non possumus cognoscere
secundum quod in se est, ut sciamus de ea quid est; sed per modum
eminentiae et causalitatis et negationis, ut supra dictum est. Et sic
hoc nomen Deus significat naturam divinam. Impositum est enim nomen
hoc ad aliquid significandum supra omnia existens, quod est principium
omnium, et remotum ab omnibus. Hoc enim intendunt significare
nominantes Deum.

&

しかし私たちは、神の結果に基づいて、神の本性をそれ自体において、それに
ついて「何であるか」を知るようには認識することができず、むしろ上述のよ
うに、卓越性と原因性と否定というかたちによって認識する。だから「神」と
いうこの名は神の本性を表示する。なぜなら、この名は万物の根源であり万物
から隔たっている万物の上に存在する何かを表示するために付けられているか
らである。神を名付ける人々はこれを表示することを意図している。

\end{longtable}
\newpage
\fancyhead[R]{a.~9}

\begin{center}
{\Large {\bfseries ARTICULUS NONUS}}\\
{\large UTRUM HOC NOMEN {\itshape DEUS} SIT COMMUNICABILE}\\
{\Large 第九項\\「神」というこの名は共有されうるか}
\end{center}

\begin{longtable}{p{21em}p{21em}}

{\huge A}{\scshape d nonum sic proceditur}. Videtur quod hoc nomen
Deus sit communicabile. Cuicumque enim communicatur res significata
per nomen, communicatur et nomen ipsum. Sed hoc nomen {\itshape Deus},
ut dictum est, significat divinam naturam, quae est communicabilis
aliis, secundum illud II Pet.~{\scshape i}, {\itshape magna et
pretiosa promissa nobis donavit, ut per hoc efficiamur divinae
consortes naturae}. Ergo hoc nomen {\itshape Deus} est communicabile.

&

第九項の問題へ議論は以下のように進められる。「神」というこの名は共有さ
れうると思われる。理由は以下の通り。あるものに、名によって表示された事
物が共有されるとき、その名自体も共有されうる。ところで、この「神」とい
う名は、すでに述べられたとおり、神の本性を表示するが、神の本性は『ペト
ロの手紙 二』第1章「彼は私たちに偉大で貴重な約束を与えたが、それは、
このことによって、私たちが神と同種の本性になるためである」
\footnote{「わたしたちは尊くすばらしい約束を与えられています。それは
(中略)神の本性にあずからせていただくようになるためです。」(1:4)}によ
れば、他のものと共有されうる。ゆえに「神」というこの名は共有されうる。

\\

2.~{\scshape Praeterea}, sola nomina propria non sunt
communicabilia. Sed hoc nomen Deus non est nomen proprium, sed
appellativum, quod patet ex hoc quod habet plurale, secundum illud
Psalmi {\scshape lxxxi}, {\itshape Ego dixi, dii estis}. Ergo hoc
nomen {\itshape Deus} est communicabile.

&

さらに、共有されえないのは固有名詞だけである。しかしこの「神」という名
は固有名詞ではなく普通名詞である。このことは、『詩編』第81章「私は言っ
た。あなた方は神々である」\footnote{「わたしは言った「あなたたちは神々
なのか、皆、いと高き方の子らなのか」と。」(82:6)}によれば、これが複数
形をもつことから明らかである。ゆえにこの「神」という名は共有されうる。

\\

3.~{\scshape Praeterea}, hoc nomen {\itshape Deus} imponitur ab
operatione, ut dictum est. Sed alia nomina quae imponuntur Deo ab
operationibus, sive ab effectibus, sunt communicabilia, ut {\itshape
bonus, sapiens} et huiusmodi. Ergo et hoc nomen {\itshape Deus} est
communicabile.

&

さらに、「神」というこの名は、すでに述べられたとおり、働きから付けられ
ている。ところで「善なる者」「知者」といった働きやその結果から神に付け
られている他の名は共有されうる。ゆえに「神」というこの名も共有されうる。

\\

{\scshape Sed contra est} quod dicitur {\itshape Sap}.~{\scshape xiv},
{\itshape incommunicabile nomen lignis et lapidibus imposuerunt}; et
loquitur de nomine deitatis. Ergo hoc nomen Deus est nomen
incommunicabile.

&

しかし反対に、『知恵の書』14章に「彼らは、共有されえない名を材木や石に
付けた」と言われているが、これは神性の名について語られている。ゆえに
「神」というこの名は共有されえない。

\\

{\scshape Respondeo dicendum} quod aliquod nomen potest esse
communicabile dupliciter, uno modo, proprie; alio modo, per
similitudinem. Proprie quidem communicabile est, quod secundum totam
significationem nominis, est communicabile multis. Per similitudinem
autem communicabile est, quod est communicabile secundum aliquid eorum
quae includuntur in nominis significatione. Hoc enim nomen {\itshape
leo} proprie communicatur omnibus illis in quibus invenitur natura
quam significat hoc nomen {\itshape leo}, per similitudinem vero
communicabile est illis qui participant aliquid leoninum, ut puta
audaciam vel fortitudinem, qui metaphorice {\itshape leones} dicuntur.

&

解答する。以下のように言われるべきである。ある名は、二通りのしかたで共
有されうる。一つには固有にであり、もう一つには類似によってである。固有
に共有されうるのは、名の意味全体に即して多くのものに共有されうるもので
ある。これに対して、類似によって共有されうるのは、名の意味の中に含まれ
ているもののうち、何かあるものに即して共有されうるものである。たとえば、
「獅子」というこの名は、「獅子」という名が表示する本性が見出されるすべ
てのものども[つまりすべてのライオンの個体]に、固有に共有されている。
他方、大胆さや強さといった獅子の何らかの点を分有するものに、類似によっ
て共有され、それらは比喩的に「獅子」と言われる。

\\

Ad sciendum autem quae nomina proprie sunt communicabilia,
considerandum est quod omnis forma in supposito singulari existens,
per quod individuatur, communis est multis, vel secundum rem vel
secundum rationem saltem, sicut natura humana communis est multis
secundum rem et rationem, natura autem solis non est communis multis
secundum rem, sed secundum rationem tantum; potest enim natura solis
intelligi ut in pluribus suppositis existens.  &

ところで、どのような名が固有に共有されうるかを知るためには、以下のこと
が考察されるべきである。個的な個体において存在する形相はすべて、その個
体を通して個体化されているが、多くのものに、実在的に共通する場合と、た
んに概念的に共通する場合とがある。たとえば、人間本性は、多くのものに、
実在的にも概念的にも共通するが、太陽の本性は、多くのものに、実在的には
共通せず、たんに概念的に共通するにすぎない。というのは、太陽の本性は、
複数の個体において存在すると考えられうる\footnote{「考えられうる」
はpotest intelligi。ここでintelligiとはその可能性があることを整合的に
理解できるというほどの意。}からである。

\\

Et hoc ideo, quia intellectus intelligit naturam cuiuslibet speciei
per abstractionem a singulari, unde esse in uno supposito singulari
vel in pluribus, est praeter intellectum naturae speciei, unde,
servato intellectu naturae speciei, potest intelligi ut in pluribus
existens.

&

これは、知性が、どんな種の本性も、個物から抽象して認識するからであり、
それゆえ、一つの個的な個体において存在するか、複数の個体において存在す
るかということは、種の本性の理解の外にある。したがって、種の本性の理解
を保ったままで、それが複数のものにおいて存在するものとして理解されうる。

\\

Sed singulare, ex hoc ipso quod est singulare, est divisum ab omnibus
aliis. Unde omne nomen impositum ad significandum aliquod singulare,
est incommunicabile et re et ratione, non enim potest nec in
apprehensione cadere pluralitas huius individui.

&

しかし個物は、それが個物であるということから、他の全てのものから分割さ
れている。そのため、何らかの個物を表示するために付けられたすべての名は、
実在的にも概念的にも、共有されえない。この個体の把握のうちに、複数性が
入ることができないからである。

\\

Unde nullum nomen significans aliquod individuum, est communicabile
multis proprie, sed solum secundum similitudinem; sicut aliquis
metaphorice potest dici {\itshape Achilles}, inquantum habet aliquid
de proprietatibus Achillis, scilicet fortitudinem.

&

したがって、何らかの個体を表示する名は、多くのものに、固有に共有されえ
ず、ただ類似によって共有されうる。たとえば、ある人が、アキレスの固有性、
つまり強さのいくらかをもっているという点で、比喩的に「アキレス」と言わ
れうるように。

\\

Formae vero quae non individuantur per aliquod suppositum, sed per
seipsas (quia scilicet sunt formae subsistentes), si intelligerentur
secundum quod sunt in seipsis, non possent communicari nec re neque
ratione; sed forte per similitudinem, sicut dictum est de individuis.

&

他方、何らかの個体によってではなく、自己自身によって(つまり、それが自
存する形相であるため)個体化されている形相は、それ自体において理解され
るならば、実在的にも概念的にも共有されえず、(共有されるとしてもそれは)
個物について言われたのと同様、せいぜい、類似によって共有される。

\\

Sed quia formas simplices per se subsistentes non possumus intelligere
secundum quod sunt, sed intelligimus eas ad modum rerum compositarum
habentium formas in materia; ideo, ut dictum est, imponimus eis nomina
concreta significantia naturam in aliquo supposito.

&

しかし、私たちは自存する単純形相をあるがままに知性認識することができず、
質料において形相をもつ複合的事物のあり方に即してそれらを認識するので、
すでに述べられたとおり、それらには、何らかの個体において本性を表示する
具体名の名を付ける。

\\

Unde, quantum pertinet ad rationem nominum, eadem ratio est de
nominibus quae a nobis imponuntur ad significandum naturas rerum
compositarum, et de nominibus quae a nobis imponuntur ad significandum
naturas simplices subsistentes.

&

したがって、名称の性格にかんする限り、複合的事物の本性を表示するために
私たちによって付けられる名称と、自存する単純本性を表示するために私たち
によって付けられる名称とは、同じ性格のものとなる。

\\

Unde, cum hoc nomen {\itshape Deus} impositum sit ad significandum
naturam divinam, ut dictum est; natura autem divina multiplicabilis
non est, ut supra ostensum est, sequitur quod hoc nomen Deus
incommunicabile quidem sit secundum rem, sed communicabile sit
secundum opinionem, quemadmodum hoc nomen sol esset communicabile
secundum opinionem ponentium multos soles.

&

それゆえ、すでに述べられたとおり、「神」というこの名は神の本性を表示す
るために付けられているが、上で示されたとおり、神の本性は多数化されえな
いので、「神」というこの名は、実在的には共有されえないが、意見という観
点から見れば、共有されうる。ちょうど、「太陽」というこの名が、多数の太
陽を想定する人々の意見に従うならば、共有されていたであろうように。

\\

Et secundum hoc dicitur {\itshape Gal}.~{\scshape iv}, {\itshape his
qui natura non sunt dii, serviebatis}; Glossa, {\itshape non sunt dii
natura, sed opinione hominum}.

&

そしてこの意味で、『ガラテヤの信徒への手紙』第4章「本性において神々で
ないものたちにあなたたちは仕えていた」\footnote{「ところで、あなたがた
はかつて、神を知らずに、もともと神でない神々に奴隷として仕えていました」
(4:8)}と言われているが、これは注解では「本性において神々ではなく、人間
の意見において」と言われる。

\\

--- Est nihilominus communicabile hoc nomen {\itshape Deus}, non
secundum suam totam significationem, sed secundum aliquid eius, per
quandam similitudinem, ut {\itshape dii} dicantur, qui participant
aliquid divinum per similitudinem, secundum illud, {\itshape ego dixi,
dii estis}.

&

しかしそれでも、かの「私は言った。あなた方は神々である」によれば、この
「神」という名は、その意味全体に即してではなく、その一部に即して、何ら
かの類似によって共有されうるのであり、その結果、何らかの神的なものを類
似によって分有するものどもが、「神々」と言われる。

\\

Si vero esset aliquod nomen impositum ad significandum Deum non ex
parte naturae, sed ex parte suppositi, secundum quod consideratur ut
{\itshape hoc aliquid}, illud nomen esset omnibus modis
incommunicabile, sicut forte est nomen tetragrammaton apud
Hebraeos. Et est simile si quis imponeret nomen soli designans hoc
individuum.

&

他方で、もし、本性の側からではなく、個体の側から、すなわち「このあるも
の」として考えられるという点で神を表示するために付けられた名があったと
したら、その名は、どんなかたちによっても、共有されえなかったであろう。
たとえば、ヘブライ人たちのいう「四文字からなる名」\footnote{神を示す
YHVHという四つの子音からなる言葉。ヤハウエを表す。}のように。また、も
しだれかが、太陽に、その個体を指示する名を付けたならば、同じようになる。

\\

{\scshape Ad primum ergo dicendum} quod natura divina non est
communicabilis nisi secundum similitudinis participationem.

&

第一異論に対してはそれゆえ次のように言われるべきである。神の本性は、類
似の分有に即してでなければ共有されえない。
\\

{\scshape Ad secundum dicendum} quod hoc nomen Deus est nomen
appellativum, et non proprium, quia significat naturam divinam ut in
habente; licet ipse Deus, secundum rem, non sit nec universalis nec
particularis. Nomina enim non sequuntur modum essendi qui est in
rebus, sed modum essendi secundum quod in cognitione nostra est. Et
tamen, secundum rei veritatem, est incommunicabile, secundum quod
dictum est de hoc nomine {\itshape sol}.

&

第二異論に対しては、次のように言われるべきである。この「神」という名は、
普通名詞であって固有名詞ではない。なぜなら、神の本性を、神の本性を持つ
ものの中にあるものとして表示するからである。ただし、神自身は、実在とし
て見れば、普遍的でも個別的でもない。名は、実在におけるあり方に伴うので
はなく、私たちの認識の中にある限りでのあり方に伴うからである。しかし、
事物の真理に即して言えば、「太陽」というこの名について言われたような意
味で、共有されえない。

\\

{\scshape Ad tertium dicendum} quod haec nomina {\itshape bonus},
{\itshape sapiens}, et similia, imposita quidem sunt a perfectionibus
procedentibus a Deo in creaturas, non tamen sunt imposita ad
significandum divinam naturam, sed ad significandum ipsas perfectiones
absolute. Et ideo etiam secundum rei veritatem sunt communicabilia
multis. Sed hoc nomen {\itshape Deus} impositum est ab operatione
propria Deo, quam experimur continue, ad significandum divinam
naturam.

&

第三異論に対しては、次のように言われるべきである。「善いもの」「知恵あ
るもの」などの名は、神から被造物へ発出する完全性によって名付けられてい
るが、神の本性を表示するために名付けられているのではなく、それらの完全
性を無条件的に表示するために名付けられている。ゆえに、事物の真理に即し
ても、多数のものに共有されうる。しかし「神」というこの名は、神の本性を
表示するために、私たちが絶えず経験している神に固有の働きから付けられて
いる。

\end{longtable}
\newpage
\fancyhead[R]{a.~10}

\begin{center}
{\Large {\bfseries ARTICULUS DECIMUS}}\\
{\large UTRUM HOC NOMEN {\itshape DEUS} UNIVOCE DICATUR DE DEO PER
PARTICIPATIONEM, SECUNDUM NATURAM, ET SECUNDUM OPINIONEM}\\
{\Large 第十項\\「神」というこの名は\\分有、本性、意見において一義的に語られるか}
\end{center}

\begin{longtable}{p{21em}p{21em}}

{\huge A}{\scshape d decimum sic proceditur}. Videtur quod hoc nomen
{\itshape Deus} univoce dicatur de Deo per naturam, et per
participationem, et secundum opinionem. Ubi enim est diversa
significatio, non est contradictio affirmantis et negantis,
aequivocatio enim impedit contradictionem sed Catholicus dicens
{\itshape idolum non est Deus}, contradicit Pagano dicenti {\itshape
idolum est Deus}. Ergo {\itshape Deus} utrobique sumptum univoce
dicitur.

&

第十項の問題へ議論は以下のように進められる。この「神」という名は、神に
ついて、本性、分有、意見によって、一義的に語られると思われる。理由は以
下の通り。[音声が同じでも]意味が異なるとき、それを肯定する者と否定す
る者との矛盾対立は生じない。同名異義は矛盾対立を阻止するからである。と
ころが、「偶像は神でない」と言う正統信者と、「偶像は神である」と言う異
教徒とは矛盾対立する。ゆえに、どちらに取られた「神」[という名]も、一
義的に語られている。

\\

{\scshape 2 Praeterea}, sicut idolum est Deus secundum opinionem et
non secundum veritatem, ita fruitio carnalium delectationum dicitur
felicitas secundum opinionem, et non secundum veritatem. Sed hoc nomen
{\itshape beatitudo} univoce dicitur de hac beatitudine opinata, et de
hac beatitudine vera. Ergo et hoc nomen Deus univoce dicitur de Deo
secundum veritatem, et de Deo secundum opinionem.

&

さらに、ちょうど、神が偶像であるのは意見においてであり、真理においてで
はないように、身体的快楽の享受が幸福であるのは、意見においてであり、真
理においてではない。ところが、「至福」というこの名は、意見による至福と、
真の至福とについて、一義的に語られる。ゆえに、この「神」という名も、真
理における神と、意見における神について、一義的に語られる。

\\

{\scshape 3 Praeterea}, univoca dicuntur quorum est ratio una. Sed
Catholicus, cum dicit unum esse Deum, intelligit nomine Dei rem
omnipotentem, et super omnia venerandam, et hoc idem intelligit
gentilis, cum dicit idolum esse Deum. Ergo hoc nomen Deus univoce
dicitur utrobique.

&

さらに、一義的なものとは、それらに一つの概念が属するものどものことであ
る。ところが、正統信者が「神は一つである」と言うとき、彼は「神」という
名で、全能であり、すべてにまさって畏敬されるべきものを理解するが、この
同じものを、異教徒は、「偶像は神である」と言うときに理解している。ゆえ
に、「神」というこの名は、どちらについても一義的に語られる。

\\

{\scshape Sed contra}, illud quod est in intellectu, est similitudo
eius quod est in re, ut dicitur in I {\itshape Periherm}. Sed
{\itshape animal}, dictum de animali vero et de animali picto,
aequivoce dicitur. Ergo hoc nomen Deus, dictum de Deo vero et de Deo
secundum opinionem, aequivoce dicitur.

&

しかし反対に、『命題論』1巻で言われているように、知性のうちにあるもの
は実在するものの類似である。ところが「動物」は動物と、描かれた動物
\footnote{ギリシャ語の「動物」\textgreek{z\~w|on}は、「動物」と「絵」
という二つの意味がある。必ずしも「絵に描いた動物」でなくてもよい。ラテ
ン語のanimalには「絵」の意味がないので、やや内容がずれている。}につい
て異義的に語られる。ゆえにこの「神」という名は、真の神について言われる
ときと、意見に即して言われるときとでは、異義的に語られる。

\\

{\scshape 2 Praeterea}, nullus potest significare id quod non
cognoscit, sed gentilis non cognoscit naturam divinam, ergo, cum dicit
idolum est Deus, non significat veram deitatem. Hanc autem significat
Catholicus dicens unum esse Deum. Ergo hoc nomen Deus non dicitur
univoce, sed aequivoce, de Deo vero, et de Deo secundum opinionem.

&

さらに、だれも認識しないものを意味表示することはできない。しかし異教徒
は神の本性を認識しない。ゆえに「偶像が神である」と語るとき、真の神性を
意味表示していない。しかし正統信者が「神は一つである」と言うとき、彼は
それを意味表示している。ゆえにこの「神」という名は真の神についてと意見
に即しての神について一義的にではなく異義的に語られている。

\\

{\scshape Respondeo dicendum} quod hoc nomen Deus, in praemissis
tribus significationibus, non accipitur neque univoce neque aequivoce,
sed analogice. Quod ex hoc patet. Quia univocorum est omnino eadem
ratio, aequivocorum est omnino ratio diversa, in analogicis vero,
oportet quod nomen secundum unam significationem acceptum, ponatur in
definitione eiusdem nominis secundum alias significationes accepti.

&

解答する。以下のように言われるべきである。先述の三つの表示において、
「神」というこの名は、一義的にでも異義的にでもなく、アナロギア的に理解
される。これは以下のことから明らかである。一義的なものどもにはまったく
同じ概念が、異義的なものどもにはまったく異なる概念が属するが、アナロギ
ア的なものどもにおいては、ある一つの意味に即して理解された名が、他の意
味に即して理解された同じ名の定義の中に置かれなければならない。

\\

Sicut ens de substantia dictum, ponitur in definitione entis secundum
quod de accidente dicitur; et sanum dictum de animali, ponitur in
definitione sani secundum quod dicitur de urina et de medicina; huius
enim sani quod est in animali, urina est significativa, et medicina
factiva.

&

たとえば、実体について語られた有は、附帯性について語られる場合の有の定
義の中に置かれるし、動物について語られた健康が、尿や薬について語られる
場合の健康の定義の中に置かれる。じっさい、尿が表示し、薬が作るのは、動
物の中にある健康である。

\\

Sic accidit in proposito. Nam hoc nomen Deus, secundum quod pro Deo
vero sumitur, in ratione Dei sumitur secundum quod dicitur Deus
secundum opinionem vel participationem. Cum enim aliquem nominamus
Deum secundum participationem, intelligimus nomine Dei aliquid habens
similitudinem veri Dei.

&

今の問題についても同様である。つまり、真の神のために用いられるこの「神」
という名が、意見や分有という面で「神」と言われる場合に、その「神」の概
念の中に用いられる。たとえば、私たちがある人を分有によって神と名付ける
とき、「神」という名によって、私たちは、真の神の類似をもつ何かを理解す
る。

\\

Similiter cum idolum nominamus Deum, hoc nomine Deus intelligimus
significari aliquid, de quo homines opinantur quod sit Deus. Et sic
manifestum est quod alia et alia est significatio nominis, sed una
illarum significationum clauditur in significationibus aliis. Unde
manifestum est quod analogice dicitur.

&

同様に、私たちが偶像を神と名付けるとき、「神」というこの名によって、人々
がそれが神だと意見するものについて、何かが意味されていると理解する。こ
のように、それぞれの場合の名の意味は異なるが、それらの意味の一つが、他
の意味の中に含まれていることが明らかである。したがって、[「神」という
名が]アナロギア的に語られていることが明らかである。

\\

{\scshape Ad primum ergo dicendum} quod nominum multiplicitas non
attenditur secundum nominis praedicationem, sed secundum
significationem, hoc enim nomen homo, de quocumque praedicetur, sive
vere sive false, dicitur uno modo. Sed tunc multipliciter diceretur,
si per hoc nomen {\itshape homo} intenderemus significare diversa,
puta, si unus intenderet significare per hoc nomen {\itshape homo} id
quod vere est homo, et alius intenderet significare eodem nomine
lapidem, vel aliquid aliud.

&

第一異論に対してはそれゆえ次のように言われるべきである。名の多さは、名
の述語付けという点ではなく、名の意味の点で見出される。たとえば「人間」
というこの名は、どんなものについて述語されようとも、それが真にであれ偽
にであれ一様に語られる。しかしもし、この「人間」という名によって、私た
ちが異なるものを意味しようとしたならば、たとえば、ある人は「人間」とい
うこの名で真に人間であるものを意味しようとし、他の人は同じ名で石や何か
他のものを意味しようとしたとすると、その場合には多様に語られたであろう。

\\

Unde patet quod Catholicus dicens idolum non esse Deum, contradicit
Pagano hoc asserenti, quia uterque utitur hoc nomine Deus ad
significandum verum Deum.

&

したがって、偶像は神でないと言う正統信者は、それが神であると言う異教徒
と矛盾対立する。なぜなら、どちらもこの「神」という名を真の神を意味する
ために用いているからである。

\\

Cum enim Paganus dicit idolum esse Deum, non utitur hoc nomine
secundum quod significat Deum opinabilem, sic enim verum diceret, cum
etiam Catholici interdum in tali significatione hoc nomine utantur, ut
cum dicitur, {\itshape omnes dii gentium Daemonia}.

&

じっさい、異教徒が偶像が神であると言うとき、この[「神」という]名を意
見として語られる神を意味するものとして用いているのではない。もしそうだ
としたら真を語っていることになっただろう。「すべての神々は部族の霊」
\footnote{「諸国の民の神々はすべて空しい」『詩編』(96:5)}と言われる場
合のように、正統信者でも時としてこの名をそのような意味で使うからである。

\\

Et similiter dicendum ad secundum et tertium. Nam illae rationes
procedunt secundum diversitatem praedicationis nominis, et non
secundum diversam significationem.

&

第二異論と第三異論に対しても同様に言われるべきである。これらの異論は、
名の述語付けの多様性に即して論じているのであり、意味の多様性に即して論
じているのではない。

\\

{\scshape Ad quartum dicendum} quod animal dictum de animali vero et
de picto, non dicitur pure aequivoce; sed philosophus largo modo
accipit aequivoca, secundum quod includunt in se analoga. Quia et ens,
quod analogice dicitur, aliquando dicitur aequivoce praedicari de
diversis praedicamentis.

&

第四異論に対しては次のように言われるべきである。真の動物と描かれた動物
について語られた「動物」は、純粋に異義的ではない。しかし哲学者は、アナ
ロギアをその中に含むような、広い意味で、異義語を理解している。なぜなら、
有も、アナロギア的に言われるが、時として、さまざまなカテゴリーについて
異義的に述語されると言われるからである。

\\

{\scshape Ad quintum dicendum} quod ipsam naturam Dei prout in se est,
neque Catholicus neque Paganus cognoscit, sed uterque cognoscit eam
secundum aliquam rationem causalitatis vel excellentiae vel
remotionis, ut supra dictum est. Et secundum hoc, in eadem
significatione accipere potest gentilis hoc nomen Deus, cum dicit
idolum est Deus, in qua accipit ipsum Catholicus dicens idolum non est
Deus. Si vero aliquis esset qui secundum nullam rationem Deum
cognosceret, nec ipsum nominaret, nisi forte sicut proferimus nomina
quorum significationem ignoramus.

&

第五異論に対しては、次のように言われるべきである。上で述べられたとおり、
正統信者も異教徒も、神の本性をあるがままに認識することはなく、どちらも、
なんらかの原因性や卓越性や排除という性格に基づいて、それを認識する。そ
の意味で、異教徒が偶像は神であると言うとき、正統信者が偶像は神でないと
言うときと同じ意味で、この「神」という名を理解することができる。他方、
もしも、どんな概念に即しても神を認識しないような人がいたとしたら、彼は、
神を名付けることもなかったであろう。私たちが、その意味を知らないものの
名を口にする場合のようにでなかったならば。

\end{longtable}
\newpage
\fancyhead[R]{a.~11}

\begin{center}
{\Large {\bfseries ARTICULUS UNDECIMUS}}\\
{\large UTRUM HOC NOMEN {\itshape QUI EST} MAXIME NOMEN DEI PROPRIUM}\\
{\footnotesize I {\itshape Sent.}, d.~8, q.~1, a.~1, 3; {\itshape De
Pot.}, q.~2, a.~1; q.~7, a.~3; q.~10, a.~1, ad 9; {\itshape De
Div.~Nom.}, c.~5, lect.~1.}\\
{\Large 第十一項\\「在る者」というこの名は、最大限に神に固有の名か}
\end{center}

\begin{longtable}{p{21em}p{21em}}

{\huge A}{\scshape d undecimum sic proceditur}. Videtur quod hoc nomen
{\itshape Qui est} non sit maxime proprium nomen Dei. Hoc enim nomen
{\itshape Deus} est nomen incommunicabile, ut dictum est. Sed hoc
nomen {\itshape Qui est} non est nomen incommunicabile. Ergo hoc nomen
{\itshape Qui est} non est maxime proprium nomen Dei.

&

第十一項の問題へ議論は以下のように進められる。「在る者」というこの名は、
最大限に固有な神の名ではないと思われる。理由は以下の通り。「神」という
この名は、すでに述べられたように、共有されえない。しかし、「在る者」と
いう名は、共有されえない名ではない。ゆえに、「在る者」という名は、最大
限に固有な神の名ではない。

\\

{\scshape 2 Praeterea}, Dionysius dicit, {\scshape iii} cap.~{\itshape
de Div.~Nom}., quod {\itshape boni nominatio est manifestativa omnium
Dei processionum}. Sed hoc maxime Deo convenit, quod sit universale
rerum principium. Ergo hoc nomen {\itshape bonum} est maxime proprium
Dei, et non hoc nomen {\itshape Qui est}.

&

さらに、ディオニュシウスは『神名論』3章で「善という名付けは、神のすべ
ての発出を明らかにしうる」と述べている。ところで、諸事物の普遍的根源で
あるということは、最大限に神に適合する。ゆえに「善」というこの名が、最
大限に神に固有であって、「在る者」という名ではない。

\\

{\scshape 3 Praeterea}, omne nomen divinum videtur importare
relationem ad creaturas, cum Deus non cognoscatur a nobis nisi per
creaturas. Sed hoc nomen {\itshape Qui est} nullam importat
habitudinem ad creaturas. Ergo hoc nomen {\itshape Qui est} non est
maxime proprium nomen Dei.

&

神は私たちによって被造物を通してしか認識されないので、神のすべての名は
被造物への関係を意味の内に含んでいると思われる。ところで、「在る者」と
いう名は、被造物への関係をまったく含んでいない。ゆえに「在る者」という
名は、最大限に固有な神の名ではない。

\\

{\scshape Sed contra est} quod dicitur {\itshape Exod}.~{\scshape
iii}, quod Moysi quaerenti, {\itshape si dixerint mihi, quod est nomen
eius? Quid dicam eis?}  Et respondit ei dominus, {\itshape sic dices
eis, {\scshape qui est} misit me ad vos}. Ergo hoc nomen {\itshape Qui
est} est maxime proprium nomen Dei.

&

しかし反対に、『出エジプト記』3章で、「彼の名は何かと彼らがわたしに言っ
たなら、わたしは彼らにどう言いましょうか」と尋ねるモーセに対し、主が彼
に、「「在る者」がわたしをあなたたちに遣わした、と彼らに答えよ」と答え
ている\footnote{「「わたしは、今、イスラエルの人々のところへ参ります。
彼らに、『あなたたちの先祖の神が、わたしをここに遣わされたのです』と言
えば、彼らは、『その名は一体何か』と問うにちがいありません。彼らになん
と答えるべきでしょうか。」 神はモーセに、「わたしはある。わたしはある
という者だ」と言われ、また、「イスラエルの人々にこう言うがよい。『わた
しはある』という方がわたしをあなたたちに遣わされたのだと。」(3:13-14)}。
ゆえに、「在る者」は、最大限に固有な神の名である。

\\

{\scshape Respondeo dicendum} quod hoc nomen {\itshape Qui est}
triplici ratione est maxime proprium nomen Dei. Primo quidem, propter
sui significationem. Non enim significat formam aliquam, sed ipsum
esse. Unde, cum esse Dei sit ipsa eius essentia, et hoc nulli alii
conveniat, ut supra ostensum est, manifestum est quod inter alia
nomina hoc maxime proprie nominat Deum, unumquodque enim denominatur a
sua forma.

&

解答する。以下のように言われるべきである。「在る者」というこの名は、三
つの理由で、最大限に固有な神の名である。第一に、その意味表示のためにで
ある。この名は、何らの形相も表示せず、存在そのものを表示する。したがっ
て、神の存在は神の本質であり、上で示されたように、このことは他のどんな
ものにも適合しないので、他の名の中で、これが最大限に神を名付けることは
明らかである。各々のものは、自らの形相によって名付けられるのだから。

\\

Secundo, propter eius universalitatem. Omnia enim alia nomina vel sunt
minus communia; vel, si convertantur cum ipso, tamen addunt aliqua
supra ipsum secundum rationem; unde quodammodo informant et
determinant ipsum. Intellectus autem noster non potest ipsam Dei
essentiam cognoscere in statu viae, secundum quod in se est, sed
quemcumque modum determinet circa id quod de Deo intelligit, deficit a
modo quo Deus in se est. Et ideo, quanto aliqua nomina sunt minus
determinata, et magis communia et absoluta, tanto magis proprie
dicuntur de Deo a nobis.

&

第二に、その普遍性のためにである。他のすべての名は、「在る者」よりも共
通性が少ないか、あるいは、「在る者」と置換されたとしても、何らかの概念
を「在る者」に加える。それゆえ、なんらかのかたちで、「在る者」を形相付
け[に情報を加え]、限定する。さて、私たちの知性は、現生において、神の
本質をあるがままに認識することができず、神について知性認識するものにつ
いてどのようなかたちでそれを限定しようと、神それ自身があるあり方から欠
ける。ゆえに、名は、より限定されていないほど、そしてより共通的で無条件
的であるほど、私たちによって、より固有に神について語られる。

\\

Unde et Damascenus dicit quod {\itshape principalius omnibus quae de
Deo dicuntur nominibus, est {\scshape qui est}, totum enim in seipso
comprehendens, habet ipsum esse velut quoddam pelagus substantiae
infinitum et indeterminatum}. Quolibet enim alio nomine determinatur
aliquis modus substantiae rei, sed hoc nomen {\itshape Qui est} nullum
modum essendi determinat, sed se habet indeterminate ad omnes; et ideo
nominat ipsum {\itshape pelagus substantiae infinitum}.

&

このことから、ダマスケヌスも「神について語られるすべての名よりも、「在
る者」がより根源的である。なぜなら、その中にすべてを包含するものとして、
存在を、無限で無限定な、一種の実体の海としてもつから」と述べる。じっさ
い、どんな他の名によっても、事物の実体の何らかのあり方が限定されるが、
「在る者」というこの名は、どんな存在のあり方も限定せず、あらゆるあり方
に対して非限定なものとしてある。ゆえに、神を「実体の無限な海」として名
付ける。

\\

Tertio vero, ex eius consignificatione. Significat enim esse in
praesenti, et hoc maxime proprie de Deo dicitur, cuius esse non novit
praeteritum vel futurum, ut dicit Augustinus in V {\itshape de Trin}.

&

第三には、それが共に表示するものに基づいてである。つまりそれは、存在を、
現在形で表示するが、これは、最大限に神について語られる。アウグスティヌ
スが『三位一体論』5巻で言うように、神の存在は、過去も未来も知らないか
らである。

\\

{\scshape Ad primum ergo dicendum} quod hoc nomen {\itshape Qui est}
est magis proprium nomen Dei quam hoc nomen {\itshape Deus}, quantum
ad id a quo imponitur, scilicet ab esse, et quantum ad modum
significandi et consignificandi, ut dictum est. Sed quantum ad id ad
quod imponitur nomen ad significandum, est magis proprium hoc nomen
Deus, quod imponitur ad significandum naturam divinam. Et adhuc magis
proprium nomen est tetragrammaton, quod est impositum ad significandam
ipsam Dei substantiam incommunicabilem, et, ut sic liceat loqui,
singularem.

&

第一異論に対してはそれゆえ次のように言われるべきである。「在る者」とい
うこの名は、上述のように、何から名付けられているか、すなわち存在から名
付けられているという点、意味表示のしかた、共に表示するものの点で、「神」
という名よりも固有な神の名である。しかし、その名が何を表示するために付
けられているかという点では、「神」というこの名の方が固有である。なぜな
らこれは、神の本性を表示するために付けられているから。また、なおさらに
固有なのは、四文字から成る名(tetragrammaton)であり、これは共有されえず、
また、もしこう言うことが許されるならば、個的な神の実体そのものを表示す
るために付けられている。

\\

{\scshape Ad secundum dicendum} quod hoc nomen {\itshape bonum} est
principale nomen Dei inquantum est causa, non tamen simpliciter, nam
esse absolute praeintelligitur causae.

&

第二異論に対しては、次のように言われるべきである。「善」という名は、原
因としての神の主要な名であるが、端的に主要な神の名ではない。なぜなら、
原因より先に、存在が、無条件的に理解されるからである。

\\

{\scshape Ad tertium dicendum} quod non est necessarium quod omnia
nomina divina importent habitudinem ad creaturas; sed sufficit quod
imponantur ab aliquibus perfectionibus procedentibus a Deo in
creaturas. Inter quas prima est ipsum esse, a qua sumitur hoc nomen
{\itshape Qui est}.

&

第三異論に対しては、次のように言われるべきである。神のあらゆる名が被造
物への関係を含意しなければならないということはない。むしろ、神から被造
物へ発出する何らかの完全性によって名付けられていることで足りる。それら
の完全性の中で、第一のものが存在であり、そこから「在る者」というこの名
が取られている。

\end{longtable}

\newpage
\fancyhead[R]{a.~12}

\begin{center}
{\Large {\bfseries ARTICULUS DUODECIMUS}}\\
{\large UTRUM PROPOSITIONES AFFIRMATIVAE POSSINT FORMARI DE DEO}\\
{\footnotesize I {\itshape Sent.}, d.~4, q.~2, a.~1; d.~22, a.~2, ad 1;
I {\itshape SCG.}, c.~34; {\itshape De Pot.}, q.~7, a.~5, ad 2.}\\
{\Large 第十二項\\神について肯定命題が形成されうるか}
\end{center}

\begin{longtable}{p{21em}p{21em}}

{\huge A}{\scshape d duodecimum sic proceditur}. Videtur quod
propositiones affirmativae non possunt formari de Deo. Dicit enim
Dionysius, {\scshape ii} cap.~{\itshape Cael.~Hier}., quod {\itshape
negationes de Deo sunt verae, affirmationes autem incompactae}.

&

第十二項の問題へ議論は以下のように進められる。神について肯定命題は作成
されえないと思われる。理由は以下の通り。ディオニュシウスは『天上階級論』
第2章で「神について否定は真だが肯定は不確かである」と述べている。

\\

{\scshape 2 Praeterea}, Boetius dicit, in libro {\itshape de Trin}.,
quod {\itshape forma simplex subiectum esse non potest}. Sed Deus
maxime est forma simplex, ut supra ostensum est. Ergo non potest esse
subiectum. Sed omne illud de quo propositio affirmativa formatur,
accipitur ut subiectum. Ergo de Deo propositio affirmativa formari non
potest.

&

さらに、ボエティウスは『三位一体論』で「単純形相は基体でありえない」と
言う。ところが、神は、上で示されたように、最大限に単純な形相である。ゆ
えに、神は基体でありえない。しかし、それについて肯定命題が形成されるよ
うなものはすべて、基体として理解される。ゆえに、神について肯定命題は形
成されえない。

\\

{\scshape 3 Praeterea}, omnis intellectus intelligens rem aliter quam
sit, est falsus. Sed Deus habet esse absque omni compositione, ut
supra probatum est. Cum igitur omnis intellectus affirmativus
intelligat aliquid cum compositione, videtur quod propositio
affirmativa vere de Deo formari non possit.

&

さらに、事物を、それがあるのとはちがうように認識する知性はすべて誤って
いる。ところが、上で証明されたとおり、神は、あらゆる複合から離れた存在
をもつ。ゆえに、肯定するすべての知性は何かを複合と共に認識するから、肯
定命題が真に神について形成されることはありえないと思われる。

\\

{\scshape Sed contra est} quod fidei non subest falsum. Sed
propositiones quaedam affirmativae subduntur fidei, utpote quod Deus
est trinus et unus, et quod est omnipotens. Ergo propositiones
affirmativae possunt vere formari de Deo.

&

しかし反対に、信仰のもとに偽が入ることはない。ところが、ある種の肯定命
題が、信仰に与えられている。たとえば、神は三にして一であるとか、神は全
能である、とかである。ゆえに、肯定命題が、真に、神について形成されうる。

\\

{\scshape Respondeo dicendum} quod propositiones affirmativae possunt
vere formari de Deo. Ad cuius evidentiam, sciendum est quod in
qualibet propositione affirmativa vera, oportet quod praedicatum et
subiectum significent idem secundum rem aliquo modo, et diversum
secundum rationem. Et hoc patet tam in propositionibus quae sunt de
praedicato accidentali, quam in illis quae sunt de praedicato
substantiali.

&

解答する。以下のように言われるべきである。神について、真に、肯定命題が
作成されうる。これを明らかにするために、次のことが知られるべきである。
どんな真の肯定命題においても、述語と主語が、何らかのかたちで、実在的に
同一でありかつ概念的に異なるものを意味表示していなければならない。これ
は、附帯的述語をとる命題においてと同様、実体的述語をとる命題においても
明らかである。

\\

Manifestum est enim quod homo et albus sunt idem subiecto, et
differunt ratione, alia enim est ratio hominis, et alia ratio albi. Et
similiter cum dico homo est animal, illud enim ipsum quod est homo,
vere animal est; in eodem enim supposito est et natura sensibilis, a
qua dicitur animal, et rationalis, a qua dicitur homo.  Unde hic etiam
praedicatum et subiectum sunt idem supposito, sed diversa ratione.

&

たとえば、人間と白い者は、基体において同一だが、概念において異なる。人
間の概念と白い者の概念は異なるからである。同様に、「人間は動物である」
と私が言うとき、人間であるそのものが、真に、動物である。つまり、同一の
個体において、感覚的本性(これによって動物と言われる)と理性的本性(こ
れによって人間と言われる)が存在している。したがって、そこでも、述語と
主語は、個体において同一だが、概念において異なる。

\\

Sed et in propositionibus in quibus idem praedicatur de seipso, hoc
aliquo modo invenitur; inquantum intellectus id quod ponit ex parte
subiecti, trahit ad partem suppositi, quod vero ponit ex parte
praedicati, trahit ad naturam formae in supposito existentis, secundum
quod dicitur quod praedicata tenentur formaliter, et subiecta
materialiter. Huic vero diversitati quae est secundum rationem,
respondet pluralitas praedicati et subiecti, identitatem vero rei
significat intellectus per ipsam compositionem.

&

しかしこのことは、あるものが自分自身に述語付けられるような命題において
もまた、何らかのかたちで見出される。それは、知性が、主語の側に置くもの
を、個体の側に関連させ、他方で、述語の側に置くものを、その個体において
存在する形相の本性に関係させることによるが、そのかぎりで、述語は形相的
に、主語は質料的に了解される。また、概念的な多様性には、述語と主語とい
う複数性が対応し、他方で、実在的な同一性を、知性は、複合それ自体によっ
て表示する。

\\

Deus autem, in se consideratus, est omnino unus et simplex, sed tamen
intellectus noster secundum diversas conceptiones ipsum cognoscit, eo
quod non potest ipsum ut in seipso est, videre.

&

さて、神は、それ自体で考察されるならば、あらゆる点で一であり単純である
が、私たちの知性は、神をあるがままに見ることができないので、それを多様
な懐念に即して認識する。

\\

Sed tamen, quamvis intelligat ipsum sub diversis conceptionibus,
cognoscit tamen quod omnibus suis conceptionibus respondet una et
eadem res simpliciter. Hanc ergo pluralitatem quae est secundum
rationem, repraesentat per pluralitatem praedicati et subiecti,
unitatem vero repraesentat intellectus per compositionem.

&

しかし、私たちの知性は神を多様な懐念のもとで知性認識するけれども、それ
ら[神にかんして知性がもつ]すべての懐念には一つの同一の実在が端的に対
応することもまた認識する。ゆえに[私たちの]知性は、概念的なこの複数性
を、述語と主語という複数性によって表現し、一性を複合によって表現する。

\\

{\scshape Ad primum ergo dicendum} quod Dionysius dicit affirmationes
de Deo esse incompactas, vel {\itshape inconvenientes} secundum aliam
translationem, inquantum nullum nomen Deo competit secundum modum
significandi, ut supra dictum est.

&

第一異論に対しては、それゆえ、次のように言われるべきである。ディオニュ
シウスは、上で述べられたとおり、どんな名も、表示のしかたの点で神に適合
しないという意味で、神についての肯定が不確かである、あるいは、別の翻訳
によれば「不適切である」と言っている。

\\

{\scshape Ad secundum dicendum} quod intellectus noster non potest
formas simplices subsistentes secundum quod in seipsis sunt,
apprehendere, sed apprehendit eas secundum modum compositorum, in
quibus est aliquid quod subiicitur, et est aliquid quod inest. Et ideo
apprehendit formam simplicem in ratione subiecti, et attribuit ei
aliquid.

&

第二異論に対しては、次のように言われるべきである。私たちの知性は、自存
する単純形相を、あるがままに捉えることができず、複合体のあり方に即して
それらを捉える。この複合体においては、何かが基体となり、何かがそれに内
在する。ゆえに、単純形相を、基体という観点で捉え、そしてそれに何かを帰
属させる。

\\

{\scshape Ad tertium dicendum} quod haec propositio, {\itshape
intellectus intelligens rem aliter quam sit, est falsus}, est duplex,
ex eo quod hoc adverbium {\itshape aliter} potest determinare hoc
verbum {\itshape intelligit} ex parte intellecti, vel ex parte
intelligentis. Si ex parte intellecti, sic propositio vera est, et est
sensus, quicumque intellectus intelligit rem esse aliter quam sit,
falsus est. Sed hoc non habet locum in proposito, quia intellectus
noster, formans propositionem de Deo, non dicit eum esse compositum,
sed simplicem.

&

第三異論に対しては、次のように言われるべきである。この、「事物を、それ
があるのとはちがうように認識する知性は誤っている」という命題は、「ちが
うように」というこの副詞が、「認識する」という動詞を、認識対象の側から
限定するか、認識主体の側から限定するかによって、二通りにある。認識対象
の側からだとすると、その場合、この命題は真である。その意味は、どんな知
性であれ、事物をそれがあるのとはちがうように「ある」と認識する知性は誤っ
ている、ということになる。しかし、これは今の問題とは関係しない。なぜな
ら、私たちの知性は、神について命題を形成するとき、神が複合体で「ある」
と言うのではなく、単純で「ある」と言うからである。

\\

Si vero ex parte intelligentis, sic propositio falsa est. Alius est
enim modus intellectus in intelligendo, quam rei in
essendo. Manifestum est enim quod intellectus noster res materiales
infra se existentes intelligit immaterialiter; non quod intelligat eas
esse immateriales, sed habet modum immaterialem in intelligendo. Et
similiter, cum intelligit simplicia quae sunt supra se, intelligit ea
secundum modum suum, scilicet composite, non tamen ita quod intelligat
ea esse composita. Et sic intellectus noster non est falsus, formans
compositionem de Deo.

&

他方、認識主体の側からだとすると、その場合この命題は偽である。なぜなら
知性認識するときの知性のあり方と、存在するときの事物のあり方とは異なる
からである。たとえば、私たちの知性は、自分の下位に存在する質料的事物を
非質料的に認識するが、それらが非質料的なものであると認識するのではなく、
知性認識において非質料的なあり方をしていると認識することは明らかである。
同様に、[私たちの知性が]自分の上位にある単純なものどもを認識するとき
も、それらを自分のあり方に即して、つまり複合的に認識するが、しかしそれ
らが複合されて在ると認識するわけではない。この意味で、神について複合
[=肯定命題]を形成する私たちの知性は、誤っているわけではない。

\end{longtable}

\end{document}