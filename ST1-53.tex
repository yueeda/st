\documentclass[10pt]{jsarticle} % use larger type; default would be 10pt
%\usepackage[utf8]{inputenc} % set input encoding (not needed with XeLaTeX)
%\usepackage[round,comma,authoryear]{natbib}
%\usepackage{nruby}
\usepackage{okumacro}
\usepackage{longtable}
%\usepqckage{tablefootnote}
\usepackage[polutonikogreek,english,japanese]{babel}
%\usepackage{amsmath}
\usepackage{latexsym}
\usepackage{color}

%----- header -------
\usepackage{fancyhdr}
\lhead{{\it Summa Theologiae} I, q.~53}
%--------------------

\bibliographystyle{jplain}

\title{{\bf PRIMA PARS}\\{\HUGE Summae Theologiae}\\Sancti Thomae
Aquinatis\\{\sffamily QUEAESTIO QUINQUAGESIMATERTIA}\\DE MOTU LOCALI ANGELORUM}
\author{Japanese translation\\by Yoshinori {\sc Ueeda}}
\date{Last modified \today}


%%%% コピペ用
%\rhead{a.~}
%\begin{center}
% {\Large {\bf }}\\
% {\large }\\
% {\footnotesize }\\
% {\Large \\}
%\end{center}
%
%\begin{longtable}{p{21em}p{21em}}
%
%&
%
%
%\\
%\end{longtable}
%\newpage



\begin{document}
\maketitle
\pagestyle{fancy}

\begin{center}
{\Large 第五十三問\\天使の場所的運動について}
\end{center}

\begin{longtable}{p{21em}p{21em}}
Consequenter considerandum est de motu locali Angelorum. Et circa hoc
 quaeruntur tria.

\begin{enumerate}
 \item utrum Angelus possit moveri localiter.
 \item utrum moveatur de loco ad locum, pertranseundo medium.
 \item utrum motus Angeli sit in tempore vel in instanti.
\end{enumerate}

&

引き続き、天使の場所的運動について考察されるべきである。これにかんして、
 三つのことが問われる。

\begin{enumerate}
 \item 天使は場所的に動くか。
 \item 天使は、中間地点を経過することで、場所から場所へと動くか。
 \item 天使の動きは時間の中にあるか、それとも、瞬間的か。
\end{enumerate}


\end{longtable}

\newpage

\rhead{a.~1}
\begin{center}
 {\Large {\bf ARTICULUS PRIMUS}}\\
 {\large UTRUM ANGELUS POSSIT MOVERI LOCALITER}\\
 {\footnotesize I {\itshape Sent.}, d.~37, q.~4, a.~1; Opusc.~XV,
 {\itshape de Angelis}, c.~18.}\\
 {\Large 第一項\\天使は場所的に動きうるか}
\end{center}

\begin{longtable}{p{21em}p{21em}}

{\huge A}{\scshape d primum sic proceditur}. Videtur quod Angelus non
possit moveri localiter. Ut enim probat philosophus in VI {\itshape
Physic}., {\itshape nullum impartibile movetur}: quia dum aliquid est in
termino a quo, non movetur; nec etiam dum est in termino ad quem, sed
tunc mutatum est, unde relinquitur quod omne quod movetur, dum movetur,
partim est in termino a quo, et partim in termino ad quem. Sed Angelus
est impartibilis. Ergo Angelus non potest moveri localiter.

&

第一項の問題へ、議論は次のように進められる。天使は場所的に動きえないと思
 われる。理由は以下の通り。哲学者が『自然学』6巻で証明しているように、
 「どんな不可分なものも動かない」。なぜなら、あるものが、出発点に在るとき
 には動かず、また、終着点に在るときにも動かない。そのときには、すでに動き
 が完了しているからである。したがって、すべて動くものは、動いているあいだ、
 一部は出発点に在り、一部は終着点に在ることになる。ところが、天使は不可分
 である。ゆえに、天使は、場所的に動きえない。

\\


2 {\scshape Praeterea}, motus est actus imperfecti, ut dicitur in III
{\itshape Physic}. Sed Angelus beatus non est imperfectus. Ergo Angelus
beatus non movetur localiter.

&
さらに、『自然学』3巻で言われるとおり、運動は不完全なものの現実態である。
 ところが、至福である天使は不完全なものではない。ゆえに、至福な天使は場
 所的に動かない。

\\


3 {\scshape Praeterea}, motus non est nisi propter
indigentiam. Sed sanctorum Angelorum nulla est indigentia. Ergo sancti
Angeli localiter non moventur.

&
さらに、運動は、足りないものを求めることにのみよって生じる。ところが、聖
 なる天使たちに、足りないものはない。ゆえに、聖なる天使たちは場所的に動
 かない。

\\


{\scshape Sed contra}, eiusdem rationis est Angelum
beatum moveri, et animam beatam moveri. Sed necesse est dicere animam
beatam localiter moveri, cum sit articulus fidei quod Christus secundum
animam, descendit ad Inferos. Ergo Angelus beatus movetur localiter.


&

しかし反対に、至福な天使が動くことと、至福な魂が動くこととには、同じ理論
 的根拠がが当てはまる。ところで、至福な魂は、場所的に動くと言われる必要が
 ある。なぜなら、キリストが、魂において地獄の人々のもとへ降りたということ
 は、信仰箇条だからである\footnote{「そして、霊においてキリストは、捕らわ
 れていた霊たちのところへ行って宣教されました。」『ペトロの手紙一』
 (3:19) ``in quo et his, qui in carcere erant, spiritibus adveniens
 praedicavit,'' }。ゆえに、至福な天使は、場所的に動く。


\\


{\scshape Respondeo dicendum} quod Angelus beatus potest
moveri localiter. Sed sicut esse in loco aequivoce convenit corpori et
Angelo, ita etiam et moveri secundum locum. Corpus enim est in loco,
inquantum continetur sub loco, et commensuratur loco. Unde oportet quod
etiam motus corporis secundum locum, commensuretur loco, et sit secundum
exigentiam eius. Et inde est quod secundum continuitatem magnitudinis
est continuitas motus; et secundum prius et posterius in magnitudine,
est prius et posterius in motu locali corporis, ut dicitur in IV
{\itshape Physic}. 



&

解答する。以下のように言われるべきである。
至福な天使は、場所的に動きうる。しかし、「場所に存在する」ということが、
 物体と天使には異義的に語られるように、「場所に即して動く」ということも
 また[異義的に語られる]。
物体は、場所に含まれ、場所とともに測られるかぎりで、場所に存在する。した
 がって、場所に即した物体の運動もまた、場所とともに測られ、場所の必要性
 に応じてある。したがって、大きさの連続にしたがって、運動の連続があり、
 大きさにおける前後関係に応じて、物体の場所的運動のなかに前後関係がある。
 これは、『自然学』4巻で言われるとおりである。

\\


Sed Angelus non est in loco ut commensuratus et contentus, sed
magis ut continens. Unde motus Angeli in loco, non oportet quod
commensuretur loco, nec quod sit secundum exigentiam eius, ut habeat
continuitatem ex loco; sed est motus non continuus. Quia enim Angelus
non est in loco nisi secundum contactum virtutis, ut dictum est, necesse
est quod motus Angeli in loco nihil aliud sit quam diversi contactus
diversorum locorum successive et non simul, quia Angelus non potest
simul esse in pluribus locis, ut supra dictum est. Huiusmodi autem
contactus non est necessarium esse continuos. 


&

しかし、天使は、ともに測られ、含まれるものとしての場所には存在せず、むし
 ろ、天使の方が場所を含む。したがって、場所における天使の運動は、場所と
 ともに測られる必要はないし、場所に基づいて連続性をもつように、場所の必
 要性に応じてある必要もない。むしろ、天使の運動は連続していない。
すでに述べられたとおり、天使が場所に存在するのは、力の接触においてに他な
 らないから、天使の運動が場所にあるのは、さまざまな場所のさまざまな接触
 が、継起的に、同時にではなく生じることに他ならない。これは、先に述べら
 れたとおり、天使が、同時に複数の場所に存在できないことによる。
しかし、このような接触が、連続している必要はない。

\\


Potest tamen in huiusmodi
contactibus continuitas quaedam inveniri. Quia, ut dictum est, nihil
prohibet Angelo assignare locum divisibilem, per contactum suae
virtutis; sicut corpori assignatur locus divisibilis, per contactum suae
magnitudinis. Unde sicut corpus successive, et non simul, dimittit locum
in quo prius erat, et ex hoc causatur continuitas in motu locali eius;
ita etiam Angelus potest dimittere successive locum divisibilem in quo
prius erat, et sic motus eius erit continuus. Et potest etiam totum
locum simul dimittere, et toti alteri loco simul se applicare, et sic
motus eius non erit continuus.

&

しかしながら、このような接触において、一種の連続性が見出されることは可能
 である。なぜなら、先に述べられたとおり、天使に、天使の力の接触をとおして、
 分割可能な場所が割り当てられることを妨げるものは何もないからである。した
 がって、ちょうど物体が、同時にではなく継起的に、先にいた場所を離れ、そし
 てこのことが原因となって連続性が結果として生じるように、天使もまた、先に
 いた場所を継起的に離れることが可能であり、もしそういうことがあれば、そ
 の天使の運動は連続的なものになるであろう。また、すべての場所を同時に離れ、そして別の場所全体に、
 みずからを接触させることも可能であり、このような場合には、天使の運
 動は連続的でないだろう。


\\


{\scshape Ad primum ergo dicendum} quod illa ratio
dupliciter deficit in proposito. Primo quidem, quia demonstratio
Aristotelis procedit de indivisibili secundum quantitatem, cui respondet
locus de necessitate indivisibilis. Quod non potest dici de
Angelo. 


&

第一異論に対しては、それゆえ、以下のように言われるべきである。
この論は、現在の問題について、二通りに間違っている。
一つには、アリストテレスの論証は、量の点で不可分であるものについてなされ
 ており、そういうものには、必然的に、不可分な場所が対応する。しかし、こ
 の意味での不可分なものが、天使について言われることはできない。


\\



Secundo, quia demonstratio Aristotelis procedit de motu
continuo. Si enim motus non esset continuus, posset dici quod aliquid
movetur dum est in termino a quo, et dum est in termino ad quem, quia
ipsa successio diversorum ubi circa eandem rem, motus diceretur; unde in
quolibet illorum ubi res esset, illa posset dici moveri. Sed continuitas
motus hoc impedit, quia nullum continuum est in termino suo, ut patet,
quia linea non est in puncto. Et ideo oportet quod illud quod movetur,
non sit totaliter in altero terminorum, dum movetur; sed partim in uno,
et partim in altero. Secundum ergo quod motus Angeli non est continuus,
demonstratio Aristotelis non procedit in proposito. 


&

二つめには、アリストテレスの論証は、連続的な運動について行われている。
じっさい、もし運動が連続的なものでなかったならば、あるものが、出発点に在
 るあいだに動くとか、終着点に在るときに動くと言われることができただろう。
 なぜなら、同一の事物にかんするさまざまな「どこ」の継起自体が、運動と言
 われたであろうから。したがって、それらのどの「どこ」に事物が在っても、
 その事物が動くと言われることできただろう。しかし、運動の連続性がこれを
 妨げる。というのも、明らかなとおり、点の中に線は存在しないから、どんな連続体
 も端には存在しないからである。ゆえに、動くものは、動いているあいだ、全
 体として、もう一方の端にあることはなく、部分的に一方の端に、部分的に他
 方の端にある。ゆえに、天使の運動が連続的でないという点で、アリストテレ
 スの論証は、現在の問題には当てはまらない。

\\

Sed secundum quod
motus Angeli ponitur continuus, sic concedi potest quod Angelus, dum
movetur, partim est in termino a quo, et partim in termino ad quem (ut
tamen partialitas non referatur ad substantiam Angeli, sed ad locum),
quia in principio sui motus continui, Angelus est in toto loco
divisibili a quo incipit moveri; sed dum est in ipso moveri, est in
parte primi loci quem deserit, et in parte secundi loci quem occupat. 

&

しかし、天使の運動が連続的だと仮定されるならば、その場合には、天使が動く
 あいだ、部分的に出発点に在り、部分的に終着点に在るということを認めるこ
 とはできる(ただし、この部分性は、天使の実体に関係づけられているので
 なく、場所に関係づけられている)。
なぜなら、みずからの運動の始まりにおいて、天使は、そこから運動を始める分
 割可能な場所全体に存在するが、運動している最中は、部分的に、そこを離れ
 る第一の場所にあり、そして、そこを占める第二の場所にあるからである。



\\



Et
hoc quidem quod possit occupare partes duorum locorum, competit Angelo
ex hoc quod potest occupare locum divisibilem per applicationem suae
virtutis sicut corpus per applicationem magnitudinis. Unde sequitur de
corpore mobili secundum locum, quod sit divisibile secundum magnitudinem
de Angelo autem, quod virtus eius possit applicari alicui divisibili.

&

そして、二つの場所の諸部分を占めることができるというこのことは、天使が、
 あたかも物体が量を宛てがうことによるようにして、自らの力を宛てがうこと
 によって、分割可能な場所を占めることができることに基づく。したがって、
 場所的に動きうる物体には量において分割可能であることが伴い、他方、天使
 には、天使の力が、何らかの分割可能なものに当てはめられうるということが
 伴う。


\\


2 {\scshape Ad secundum dicendum} quod motus existentis in
potentia, est actus imperfecti. Sed motus qui est secundum applicationem
virtutis, est existentis in actu, quia virtus rei est secundum quod actu
est.

&

第二異論に対しては、次のように言われるべきである。
可能態にあるものの運動は、不完全なものの現実態である。
しかし、力を当てはめることによってある運動は、現実態にあるものの運動であ
 る。なぜなら、事物の力は、事物が現実に存在することによってあるからであ
 る。

\\


3 {\scshape Ad tertium dicendum} quod motus existentis in potentia, est
propter indigentiam suam, sed motus existentis in actu, non est propter
indigentiam suam, sed propter indigentiam alterius. Et hoc modo Angelus,
propter indigentiam nostram, localiter movetur, secundum illud {\itshape
Heb}.~{\scshape i}, {\itshape Omnes sunt administratorii spiritus, in
ministerium missi propter eos qui haereditatem capiunt salutis}.

&

第三異論に対しては、次のように言われるべきである。
可能態にあるものの運動は、自らの必要のためにあるが、しかし、現実態にある
 ものの運動は、自らの必要のためではなく、他のものの必要のためにある。こ
 のようにして、かの『ヘブライ人への手紙』1章「全ては仕える霊であり、救い
 の遺産をつかむ人々のために、奉仕のために使わされた」\footnote{「天使た
 ちは皆、奉仕する霊であって、救いを受け継ぐことになっている人々に仕える
 ために、遣わされたのではなかったですか」(1:14)}によれば、天使は、私達の
 必要のために、場所的に動く。



\end{longtable}


\newpage

\rhead{a.~2}
\begin{center}
 {\Large {\bf ARTICULUS SECUNDUS}}\\
 {\large UTRUM ANGELUS TRANSEAT PER MEDIUM}\\
 {\footnotesize I {\itshape Sent.}, d.~38, q.~4, a.~2; {\itshape
 Quodl.}~I, q.~3, a.~2.}\\
 {\Large 第二項\\天使は中間を通って移動するか}
\end{center}

\begin{longtable}{p{21em}p{21em}}

{\huge A}{\scshape d secundum sic proceditur}. Videtur quod
Angelus non transeat per medium. Omne enim quod pertransit medium, prius
pertransit locum sibi aequalem, quam maiorem. Locus autem aequalis
Angeli, qui est indivisibilis, est locus punctalis. Si ergo Angelus in
suo motu pertransit medium, oportet quod numeret puncta infinita suo
motu, quod est impossibile.

&

第二項の問題へ向けて、議論は以下のように進められる。
天使は中間を通って移動するのではないと思われる。理由は以下の通り。
中間を通って移動するものはすべて、自らに等しい場所を、それより大きい場所
 の前に、通過しなければならない。ところが、天使は不可分だから、天使に等
 しい場所とは点である。ゆえに、もし天使が自らの運動において中間を通るの
 であれば、自らの運動によって、無限の点を数え上げなければならないが、そ
 れは不可能である。

\\


{\scshape 2 Praeterea}, Angelus est simplicioris
substantiae quam anima nostra. Sed anima nostra sua cogitatione potest
transire de uno extremo in aliud, non pertranseundo medium, possum enim
cogitare Galliam et postea Syriam, nihil cogitando de Italia, quae est
in medio. Ergo multo magis Angelus potest de uno extremo transire ad
aliud, non per medium.

&

さらに、天使は、私たちの魂よりも単純な実体である。
ところで、私たちの魂は、自らの認識によって、中間を通過せずに、ひとつの端
 から他の端へ移ることができる。たとえば、ガリアを考えて、その次に、中間
 にあるイタリアについて考えることなしに、シリアを考えることができる。
ゆえに、ましてや天使は、ひとつの端から中間を通らずに別の端へ移動すること
 できる。


\\


{\scshape Sed contra}, si Angelus movetur de uno loco
ad alium, quando est in termino ad quem, non movetur, sed mutatus
est. Sed ante omne mutatum esse praecedit mutari, ergo alicubi existens
movebatur. Sed non movebatur dum erat in termino a quo. Movebatur ergo
dum erat in medio. Et ita oportet quod pertranseat medium.

&

しかし反対に、もし天使がひとつの場所から別の場所へと動くならば、終着点に
 いるとき、その天使は動かず、すでに動いたものである。ところが、すべて、動いたも
 のであるということの前には、動くということがある。ゆえに、天使は、どこ
 かに存在しつつ、動いていた。ところが、天使が出発点にいたときには動いて
 いたかった。ゆえに、中間にいたときに、天使は動いていたのである。このよ
 うにして、天使は中間を通過するのでなければならない。
\\


{\scshape Respondeo dicendum} quod, sicut supra dictum
est, motus localis Angeli potest esse continuus, et non continuus. Si
ergo sit continuus, non potest Angelus moveri de uno extremo in alterum,
quin transeat per medium, quia, ut dicitur in {\itshape V Physic}., {\itshape medium est in
quod prius venit quod continue mutatur, quam in quod mutatur ultimum};
ordo enim prioris et posterioris in motu continuo, est secundum ordinem
prioris et posterioris in magnitudine, ut dicitur in {\itshape IV
 Physic}. 

&

解答する。以下のように言われるべきである。
前に述べられたように、天使の場所的運動は、連続的でも非連続的でもありうる。
 ゆえに、連続的である場合には、天使がひとつの端からもう一方の端へと、中
 間を通らずに動くことはできない。なぜなら、『自然学』5巻で言われるように、
 「中間とは、最後にそこへと動くところよりも先に、連続的に動くものが、そ
 こへと到達するところ」だからである。じっさい、『自然学』4巻で言われると
 おり、連続的運動における前後の順序は、大きさにおける前後関係による。

\\


Si autem
motus Angeli non sit continuus, possibile est quod pertranseat de aliquo
extremo in aliud, non pertransito medio. Quod sic patet. Inter quaelibet
enim duo extrema loca sunt infinita loca media; sive accipiantur loca
divisibilia, sive indivisibilia. Et de indivisibilibus quidem manifestum
est, quia inter quaelibet duo puncta sunt infinita puncta media, cum
nulla duo puncta consequantur se invicem sine medio, ut in {\itshape VI
Physic}. probatur. 


&

他方、天使の運動が非連続的である場合、ある端から別の端へ、中間を通過せず
に到達することができる。これは以下のようにして明らかである。どんな二つの
端の場所の間にも、中間の場所が無限にある。このことは、場所が分割可能なものと
 理解されても、分割不可能なものと理解されても同じである。
分割不可能な場所について、このことは明らかである。なぜなら、『自然学』6
 巻で証明されているとおり、どんな二つの点も、中間のものなしに、相互に伴
 うことがないので、どんな二つの点のあいだにも、中間の点が無限に
 存在するからである。


\\


-- De locis autem divisibilibus necesse est etiam hoc
dicere. Et hoc demonstratur ex motu continuo alicuius corporis. Corpus
enim non movetur de loco ad locum nisi in tempore. In toto autem tempore
mensurante motum corporis, non est accipere duo nunc, in quibus corpus
quod movetur non sit in alio et alio loco, quia si in uno et eodem loco
esset in duobus nunc, sequeretur quod ibi quiesceret; cum nihil aliud
sit quiescere quam in loco eodem esse nunc et prius. Cum igitur inter
primum nunc et ultimum temporis mensurantis motum, sint infinita nunc,
oportet quod inter primum locum, a quo incipit moveri, et ultimum locum,
ad quem terminatur motus, sint infinita loca.


&

また、分割可能な場所についてもまた、この同じことを言うことが必然である。
 これは、ある物体の連続的な運動から示される。つまり、物体は、時間の中で
 なければ、ある場所から別の場所へと動くことがない。ところが、物体のある
 運動を測る全時間の中で、動いている物体が別々の場所にないような二つの
 「今」を理解することはできない。なぜなら、もし、一つの同じ場所で、二つ
 の「今」があったとしたら、[その物体は]そこで休んでいたということが帰
 結するだろうからである。というのも、「休む」とは、同一の場所に、以前も
 あることに他ならないからである。ゆえに、最初の「今」と、その運動を測る
 時間の最後とのあいだには、無限の「今」が存在するから、そこから運動が始
 まる最初の場所と、運動がそこへと終わる最後の場所とのあいだには、無限の
 場所がなければならない。

\\


-- Et hoc sic etiam
sensibiliter apparet. Sit enim unum corpus unius palmi, et sit via per
quam transit, duorum palmorum, manifestum est quod locus primus, a quo
incipit motus, est unius palmi; et locus, ad quem terminatur motus, est
alterius palmi. Manifestum est autem quod, quando incipit moveri,
paulatim deserit primum palmum, et subintrat secundum. Secundum ergo
quod dividitur magnitudo palmi, secundum hoc multiplicantur loca media,
quia quodlibet punctum signatum in magnitudine primi palmi, est
principium unius loci; et punctum signatum in magnitudine alterius
palmi, est terminus eiusdem. Unde cum magnitudo sit divisibilis in
infinitum, et puncta sint etiam infinita in potentia in qualibet
magnitudine; sequitur quod inter quaelibet duo loca sint infinita loca
media. 


&

そしてこのことは、感覚的にも明らかである。
1パルム\footnote{palmus, 長さの単位。12 digits≒25cm。}の長さの一つの物
 体と、2パルムの長さの経路があり、その物体がそこを通過するとすると、そこ
 から運動が始まる最初の場所は、一つのパルムに属し、運動がそこへと終局す
 る場所は、もう一つのパルムに属することは明らかである。
さらに、それが動き始めるとき、少しずつ第一のパルムを離れ、第二のパルムに
 入っていくことも明らかである。それゆえ、パルムの大きさが分割されるのに
 応じて、中間の場所は多数化される。なぜなら、第一のパルムの大きさの中で
 指定されたどの点も、一つの場所の始まりであり、また、もう一つのパルムの
 大きさの中で指定された点は、その同じ場所の終端だからである。したがって、
 大きさは無限に分割可能であり、また、点もまた、どんな大きさの中にも可能
 態において無数にあるから、どんな二つの場所のあいだにも無数の中間の場所
 があることが帰結する。

\\


Mobile autem infinitatem mediorum locorum non consumit nisi per
continuitatem motus, quia sicut loca media sunt infinita in potentia,
ita et in motu continuo est accipere infinita quaedam in potentia. Si
ergo motus non sit continuus, omnes partes motus erunt numeratae in
actu. Si ergo mobile quodcumque moveatur motu non continuo, sequitur
quod vel non transeat omnia media, vel quod actu numeret media infinita,
quod est impossibile. Sic igitur secundum quod motus Angeli non est
continuus, non pertransit omnia media. Hoc autem, scilicet moveri de
extremo in extremum et non per medium, potest convenire Angelo sed non
corpori. Quia corpus mensuratur et continetur sub loco, unde oportet
quod sequatur leges loci in suo motu. Sed substantia Angeli non est
subdita loco ut contenta, sed est superior eo ut continens, unde in
potestate eius est applicare se loco prout vult, vel per medium vel sine
medio.

&
ところで、動きうるものが、中間の場所の無限性を取り尽くすのは、運動の連続
性によってでしかない。なぜなら、中間の場所が可能態において無限であるよう
 に、連続的な運動においても、可能態において、ある種の無限を理解する
 ことができるからである。
ゆえに、もし運動が連続的でないならば、運動のすべての部分が、現実に数えら
 れることになる。
ゆえに、もし、なんであれ動きうるものが、連続的でない運動によって動くなら
 ば、すべて中間を通過するわけではないということか、あるいは、無限の中間
 を現実に数え上げることになるが、それは不可能である。このようにして、そ
 れゆえ、天使の運動が連続的でない場合には、すべての中間を通過するわけで
 はない。しかし、このこと、つまり、端から端へ中間を通らずに動くというこ
 とは、天使に適合しうるが、物体には適合しえない。なぜなら、舞台は場所に
 測られ、場所に含まれるので、その運動において、場所の法則に従うからであ
 る。しかし、天使の実体は、含まれるものとして場所に従属するのではなく、
 むしろ、場所を含むものとして場所より上位にある。したがって、天使の権能
 の中には、中間を通って、あるいは中間を通らずに、自らを意志するままに場
 所に当てはめるということが属している。


\\


{\scshape Ad primum ergo dicendum} quod locus Angeli non
accipitur ei aequalis secundum magnitudinem, sed secundum contactum
virtutis, et sic locus Angeli potest esse divisibilis, et non semper
punctalis. Sed tamen loca media etiam divisibilia, sunt infinita, ut
dictum est, sed consumuntur per continuitatem motus, ut patet ex
praedictis.

&

第一異論に対しては、それゆえ、以下のように言われるべきである。
天使の場所は、大きさにしたがって、天使に等しいと理解されるのではなくて、
 力の接触にしたがってである。この意味で、天使の場所は分割可能でありうる
 のであり、常に点としての場所にあるのではない。しかし、すでに述べられた
 とおり、中間の場所は、分割可能である場合でも無限である。そしててそれは、
 既述のことから明らかなとおり、運動の連続性によって取り尽くされる。

\\


{\scshape Ad secundum dicendum} quod Angelus dum movetur
localiter, applicatur eius essentia diversis locis, animae autem
essentia non applicatur rebus quas cogitat, sed potius res cogitatae
sunt in ipsa. Et ideo non est simile.

&
第二異論に対しては、以下のように言われるべきである。
天使が場所的に動いているとき、天使の本質はさまざまな場所に適用される。し
 かし魂の本質は、認識する事物に適用されることはなく、むしろ、認識された
 事物が魂の中にある。ゆえに、同じようにはならない。

\\


{\scshape Ad tertium dicendum} quod in motu continuo
mutatum esse non est pars moveri, sed terminus unde oportet quod moveri
sit ante mutatum esse. Et ideo oportet quod talis motus sit per
medium. Sed in motu non continuo mutatum esse est pars, sicut unitas est
pars numeri, unde successio diversorum locorum, etiam sine medio,
constituit talem motum.

&

第三\footnote{反対異論にたいする解答。}に対しては、以下のように言われるべきである。
連続的な運動においては、「動いた」ということは、動くことの部分ではなく終
 極である。したがって、「動いた」となる前に、動くことがなければならない。
ゆえに、そのような運動は中間を通るのでなければならない。
しかし、連続的でない運動において、「動いた」ということは、ちょうど、一が数の部
 分であるように、部分である。したがって、さまざまな場所の継起が、中間は
 ないけれども、そのような運動を構成する。


\end{longtable}
\newpage







\rhead{a.~3}
\begin{center}
 {\Large {\bf ARTICULUS TERTIUS}}\\
 {\large UTRUM MOTUS ANGELI SIT IN INSTANTI}\\
 {\footnotesize I {\itshape Sent.}, d.~37, q.~4, a.~3; {\itshape
 Quodl.}, IX, q.~4, a.~4; XI, q.~9.}\\
 {\Large 第三項\\天使の運動は瞬間的か}
\end{center}

\begin{longtable}{p{21em}p{21em}}

{\huge A}{\scshape d tertium sic proceditur}. Videtur quod
motus Angeli sit in instanti. Quanto enim virtus motoris fuerit fortior,
et mobile minus resistens motori, tanto motus est velocior. Sed virtus
Angeli moventis seipsum, improportionabiliter excedit virtutem moventem
aliquod corpus. Proportio autem velocitatum est secundum minorationem
temporis. Omne autem tempus omni tempori proportionabile est. Si igitur
aliquod corpus movetur in tempore, Angelus movetur in instanti.

&


第三項の問題へ、議論は以下のように進められる。
天使の運動は瞬間的だと思われる。理由は以下の通り。
動くものの力が強いほど、そして、動かされるものが、動かすものに抵抗するこ
 とが少ないほど、運動はより速くなる。また、自らを動かす天使の運動は、
 なんらかの物体を動かす力を、比を絶して凌駕する。ところが、複数の速さの比は、
 時間の短縮に応じてある。また、あらゆる時間は、あらゆる時間にたいして比
 例しうる。ゆえに、もし、なんらかの物体が時間の中で動くのであれば、天
 使は瞬間的に動く。

\\


{\scshape 2 Praeterea}, motus Angeli simplicior est quam
aliqua mutatio corporalis. Sed aliqua mutatio corporalis est in
instanti, ut illuminatio, tum quia non illuminatur aliquid successive,
sicut calefit successive; tum quia radius non prius pertingit ad
propinquum quam ad remotum. Ergo multo magis motus Angeli est in
instanti.

&
さらに、天使の運動は、どんな物体的変化よりも単純である。
ところが、ある物体的変化は瞬間的であり、たとえば照明がそうである。[照明
 が瞬間的であることの]理由は、ひとつには、何かを継起的に熱するようなか
 たちでは、継起的に照らされることがないからであり、もうひとつには、光線
 は、遠くのところより近くのところへ、先に到達するというようなことがない
 からである。ゆえに、さらにいっそう、天使の運動は瞬間的である。

\\


{\scshape 3 Praeterea}, si Angelus movetur in tempore de
loco ad locum, manifestum est quod in ultimo instanti illius temporis
est in termino ad quem, in toto autem tempore praecedenti, aut est in
loco immediate praecedenti, qui accipitur ut terminus a quo; aut partim
in uno et partim in alio. Si autem partim in uno et partim in alio,
sequitur quod sit partibilis, quod est impossibile. Ergo in toto tempore
praecedenti est in termino a quo. Ergo quiescit ibi, cum quiescere sit
in eodem esse nunc et prius, ut dictum est. Et sic sequitur quod non
moveatur nisi in ultimo instanti temporis.

&

さらに、もし天使が場所から場所へ時間的に動くならば、その時間の最後の瞬間
 に、目的地点(terminus ad quem)に存在し、そして、その直前の全時間には、
 直接的に先行する場所、つまり出発地点(terminus a quo)に存在するか、ある
 いは、その部分のひとつが目的地点にあり、別の部分が出発地点にあるかのど
 ちらかだということは、明らかである。
ところが、もし、部分的に別々の場所にあるならば、部分に分けられうることに
 なり、それは不可能である。ゆえに、先行する全時間において、出発地点にあ
 る。ゆえに、そこで静止している。というのも、すでに述べられたとおり、静
 止するとは、今と先の時間において、同一の場所に存在することだからである。
 ゆえに、もし、時間の最後の瞬間においてでなければ動かないということが帰
 結する。

\\


{\scshape  Sed contra}, in omni mutatione est prius et
posterius. Sed prius et posterius motus numeratur secundum tempus. Ergo
omnis motus est in tempore, etiam motus Angeli; cum in eo sit prius et
posterius.

&
しかし反対に、あらゆる変化において、前後がある。ところが、運動の前後は、
 時間によって測られる。ゆえに、すべての運動は、それが天使の運動であった
 としても、時間の中にある。なぜなら、そこに前後があるからである。

\\


{\scshape Respondeo dicendum} quod quidam dixerunt motum
localem Angeli esse in instanti. Dicebant enim quod, cum Angelus movetur
de uno loco ad alium, in toto tempore praecedenti Angelus est in termino
a quo, in ultimo autem instanti illius temporis est in termino ad
quem. Nec oportet esse aliquod medium inter duos terminos; sicut non est
aliquod medium inter tempus et terminum temporis. Inter duo autem nunc
temporis, est tempus medium, unde dicunt quod non est dare ultimum nunc
in quo fuit in termino a quo. Sicut in illuminatione, et in generatione
substantiali ignis, non est dare ultimum instans in quo aer fuit
tenebrosus, vel in quo materia fuit sub privatione formae ignis, sed est
dare ultimum tempus, ita quod in ultimo illius temporis est vel lumen in
aere, vel forma ignis in materia. Et sic illuminatio et generatio
substantialis dicuntur motus instantanei. 

&
解答する。以下のように言われるべきである。
ある人々は、天使の場所的運動は瞬間的だと言った。
たとえば、彼らは次のように論じていた。天使がある場所から別の場所へ動くと
 き、先行する全時間において、天使は始端に在り、そしてその時間の最後の瞬
 間に、終端に在る。そして、二つの端の間に、なんらかの中間のものがある必
 要はない。それはちょうど、時間と、時間の端の間に、なんら中間のものがな
 いようなものである。
ところで、二つの、「時間の今」の間には、中間の時間がある。したがって、彼
 らが言うには、始端において、そのときに存在するところの最終の「今」を与
 えることはできない。それはちょうど、照明や、火の実体の生成において、空
 気が暗かった最後の瞬間や、材料が火の形相をもたなかった最後の瞬間を与え
 ることができず、最後の時間を与えることができるように、そのように、その
 時間の最後において、光が空気の中にあり、あるいは、火の形相が材料の中に
 ある。このような意味で、照明や、実体の生成は、瞬間的な運動と言われる。


\\


Sed hoc non habet locum in
proposito. Quod sic ostenditur. De ratione enim quietis est quod
quiescens non aliter se habeat nunc et prius, et ideo in quolibet nunc
temporis mensurantis quietem, quiescens est in eodem et in primo, et in
medio, et in ultimo. Sed de ratione motus est quod id quod movetur,
aliter se habeat nunc et prius, et ideo in quolibet nunc temporis
mensurantis motum, mobile se habet in alia et alia dispositione, unde
oportet quod in ultimo nunc habeat formam quam prius non habebat. Et sic
patet quod quiescere in toto tempore in aliquo, puta in albedine, est
esse in illo in quolibet instanti illius temporis, unde non est
possibile ut aliquid in toto tempore praecedenti quiescat in uno termino
et postea in ultimo instanti illius temporis sit in alio termino. Sed
hoc est possibile in motu, quia moveri in toto aliquo tempore, non est
esse in eadem dispositione in quolibet instanti illius temporis. Igitur
omnes huiusmodi mutationes instantaneae sunt termini motus continui,
sicut generatio est terminus alterationis materiae, et illuminatio
terminus motus localis corporis illuminantis. 

&

しかし、これは今の問題には当てはまらない。このことは、以下のように明らか
 にされる。「静止」の概念には、静止するものが、以前と今とで同じ状態にある
 ことが含まれる。ゆえに、静止するものを測る時間のどの「今」においても、
 静止するものは、以前と、中間と、最後において、同じ状態になければならな
 い。しかし、「運動」の概念には、動かされるものが、今と以前とで別の状
 態にあることが含まれる。したがって、運動を測る時間のどの「今」において
 も、動くものは、別々の状態にある。したがって、最後の「今」において、以
 前にはもっていなかった形相をもつのでなければならない。このようにして、
 全時間において、何かに、たとえば「白」に静止することは、その時間のどの
 瞬間においてもそこに在ることだということが明らかである。したがって、何
 かあるものが、先行する全時間において、ひとつの端に存在し、その後、その
 時間の最後の瞬間において、もう一つの端に存在する、ということは不可能で
 ある。しかし、このことは、運動においては可能である。なぜなら、なんらか
 の全時間において動かされることは、その時間のどの瞬間においても同一の状
 態にあることではないからである。ゆえに、すべてのこのような瞬間的な変化
 が、連続的な運動の端にはある。たとえば、生成は、質料の変化の端であるし、
 照明は、照明する物体の場所的運動の端であるように。

\\
Motus autem localis Angeli
non est terminus alicuius alterius motus continui, sed est per seipsum,
a nullo alio motu dependens. Unde impossibile est dicere quod in toto
tempore sit in aliquo loco, et in ultimo nunc sit in alio loco. Sed
oportet assignare nunc in quo ultimo fuit in loco praecedenti. Ubi autem
sunt multa nunc sibi succedentia, ibi de necessitate est tempus, cum
tempus nihil aliud sit quam numeratio prioris et posterioris in
motu. Unde relinquitur quod motus Angeli sit in tempore. In continuo
quidem tempore, si sit motus eius continuus; in non continuo autem, si
motus sit non continuus (utroque enim modo contingit esse motum Angeli,
ut dictum est), continuitas enim temporis est ex continuitate motus, ut
dicitur in IV {\itshape Physic}. 

&

ところで、天使の場所的運動は、なんらかの別の連続的な運動の端ではなく、そ
 れ自身によって在り、他の何らの運動にも依存しない。したがって、全時間に
 おいて在る場所に存在し、最後の「今」に別の場所に在る、と言うことはでき
 ない。むしろ、先行する場所に存在した最後の「今」を指定しなければならな
 い。ところで、それ自身に継起する多くの「今」があるところには、必然的に、
 時間がある。時間とは、運動における前後の数に他ならないのだから。したがっ
 て、天使の運動は時間の中にあるということが帰結する。実際、もし天使の運
 動が連続的であれば、それは連続的な時間のなかにあり、もし、[天使の運動
 が]連続的でないならば(前に述べられたとおり、天使の運動はどちらでもあ
 りうる)それは連続的でない時間の中にある。『自然学』4巻で言われるとおり、
 時間の連続性は、運動の連続性に基づくからである。

\\




Sed istud tempus, sive sit tempus continuum sive
non, non est idem cum tempore quod mensurat motum caeli, et quo
mensurantur omnia corporalia, quae habent mutabilitatem ex motu
caeli. Motus enim Angeli non dependet ex motu caeli.

&

しかし、この時間は、連続的な時間であれ、そうでない時間であれ、天の運動を
 測る時間と同じではないし、また、すべての物体的なもの---これらは天の運動に
 基づいて可変性を有するのだが---が測られる時間とも同じでない。なぜなら、天
 使の運動は、天の運動に依存しないからである。

\\


{\scshape Ad primum ergo dicendum} quod, si tempus motus
Angeli non sit continuum, sed successio quaedam ipsorum nunc, non
habebit proportionem ad tempus quod mensurat motum corporalium, quod est
continuum, cum non sit eiusdem rationis. Si vero sit continuum, est
quidem proportionabile, non quidem propter proportionem moventis et
mobilis sed propter proportionem magnitudinum in quibus est motus. Et
praeterea, velocitas motus Angeli non est secundum quantitatem suae
virtutis; sed secundum determinationem suae voluntatis.

&

第一異論に対しては、それゆえ、以下のように言われるべきである。
もし天使の運動の時間が連続的でなく、諸々の「今」の一種の継起である場合、
 物体的運動---これは連続的である---を測る時間に対する比をもたない。なぜ
 なら、同一の性格に属さないからである。他方、[天使の運動の時間が]連続的であ
 る場合、たしかに、比率関係にありうるが、動かすものの動かされるものに対
 する比のためにではなく、運動がそこにあるところの大きさの比のためである。
さらに、天使の運動の速さは、その力の量によるのではなく、その意志の決定に
 よる。

\\


{\scshape Ad secundum dicendum} quod illuminatio est
terminus motus; et est alteratio, non motus localis ut intelligatur
lumen moveri prius ad propinquum, quam ad remotum. Motus autem Angeli
est localis, et non est terminus motus. Unde non est simile.

&

第二異論に対しては、以下のように言われるべきである。照明は運動の端であり、
また、変化であって、ちょうど、光が、遠くよりも先に近くに動かされると理解
されるようなかたちで、場所的運動ではない。
これに対して、天使の運動は場所的であり、運動の端ではない。ゆえに類似して
 いない。

\\


{\sc Ad tertium dicendum} quod obiectio illa
procedit de tempore continuo. Tempus autem motus Angeli potest esse non
continuum. Et sic Angelus in uno instanti potest esse in uno loco, et in
alio instanti in alio loco, nullo tempore intermedio existente. Si autem
tempus motus Angeli sit continuum, Angelus in toto tempore praecedenti
ultimum nunc, variatur per infinita loca, ut prius expositum est. Est
tamen partim in uno locorum continuorum et partim in alio, non quod
substantia illius sit partibilis; sed quia virtus sua applicatur ad
partem primi loci et ad partem secundi, ut etiam supra dictum est.

&

第三異論に対しては、以下のように言われるべきである。この異論は連続的な時
間について論じられている。しかし、天使の運動の時間は、連続的でないことが
ありうる。だから、天使は、ある瞬間にある場所にいて、どんな時間も介在せず、
次の瞬間に、別の場所に存在することができる。他方、天使の運動の時間が連続
的である場合には、天使は、最後の「今」に先行する全時間において、先に示さ
れたとおり、無限の場所にわたって変化しうる。しかし、部分的に連続的な場所
 のひとつのところにあり、別の部分が別の場所にあるのは、天使の実体が部分
 に分けられうるからではなく、天使の力が、部分的に最初の場所に適用され、
 部分的に第二の場所に適用されるからである。このことも、前に述べられた。


\end{longtable}
\end{document}

