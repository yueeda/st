\documentclass[10pt]{jsarticle} % use larger type; default would be 10pt
%\usepackage[utf8]{inputenc} % set input encoding (not needed with XeLaTeX)
%\usepackage[round,comma,authoryear]{natbib}
%\usepackage{nruby}
\usepackage{okumacro}
\usepackage{longtable}
%\usepqckage{tablefootnote}
\usepackage[polutonikogreek,english,japanese]{babel}
%\usepackage{amsmath}
\usepackage{latexsym}
\usepackage{color}

%----- header -------
\usepackage{fancyhdr}
\lhead{{\it Summa Theologiae} III, Prologus}
%--------------------

\bibliographystyle{jplain}

\title{{\bf TERTIA PARS}\\{\HUGE Summae Theologiae}\\Sancti Thomae
Aquinatis\\{\sffamily PROLOGUS}}
\author{Japanese translation\\by Yoshinori {\sc Ueeda}}
\date{Last modified \today}

%%%% コピペ用
%\rhead{a.~}
%\begin{center}
% {\Large {\bf }}\\
% {\large }\\
% {\footnotesize }\\
% {\Large \\}
%\end{center}
%
%\begin{longtable}{p{21em}p{21em}}
%
%&
%
%
%\\
%\end{longtable}
%\newpage



\begin{document}
\maketitle
\pagestyle{fancy}


\begin{longtable}{p{21em}p{21em}}
Quia salvator noster dominus Iesus Christus, teste Angelo, {\it populum
suum salvum faciens a peccatis eorum}, viam veritatis nobis in seipso
demonstravit, per quam ad beatitudinem immortalis vitae resurgendo
pervenire possimus, necesse est ut, ad consummationem totius theologici
negotii, post considerationem ultimi finis humanae vitae et virtutum ac
vitiorum, de ipso omnium salvatore ac beneficiis eius humano generi
praestitis nostra consideratio subsequatur. 


&


私たちの救済者、主イエスキリストは、天使の証言によって、
「自らの民を、彼らの罪から救われたものとし」\footnote{「マリアは男の子を産む。その子をイエスと名付けなさい。この子は自分の民を罪から救うからである。」『マタイによる福音
 書』(1:21)}、真理の道を私たちに自らにおいて示し、私たちはその道を通って、
 不死の生命という至福へ、再び上昇するこ
 とによって到達することができるのだから、神学的議論全体を尽くすためには、人間の生の究極目的と、
 徳と悪徳の考察のあとに、万物の救済者自身について、そして、人類に与えら
 れた彼の好意の考察が続くことが必要である。


\\

Circa quam, primo
considerandum occurrit de ipso salvatore; secundo, de sacramentis eius,
quibus salutem consequimur; tertio, de fine immortalis vitae, ad quem
per ipsum resurgendo pervenimus. 


&


この考察をめぐって、第一に救済者自身について、第二に、それに
 よって私たちが救済を獲得するところの、彼の秘蹟について\footnote{q.60.}、第三に、私たち
 が、彼を通して、そこへと再上昇することによって到達する不死なる生命の目
 的について\footnote{Supplem. q.69.}、考察が行われる。



\\


Circa primum duplex consideratio
occurrit, prima est de ipso incarnationis mysterio, secundum quod Deus
pro nostra salute factus est homo; secunda de his quae per ipsum
salvatorem nostrum, idest Deum incarnatum, sunt acta et passa.

&


第一については二つの考察があるが、第一は、それによって神が私たちの救済の
 ために人間となったところの、受肉の秘蹟について、第二に、私たちの救済者、
 すなわち受肉した神によって、行われたり被られた事柄についてである\footnote{q.27.}。

\end{longtable}


\end{document}