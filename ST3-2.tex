\documentclass[10pt]{jsarticle} % use larger type; default would be 10pt
%\usepackage[utf8]{inputenc} % set input encoding (not needed with XeLaTeX)
%\usepackage[round,comma,authoryear]{natbib}
%\usepackage{nruby}
\usepackage{okumacro}
\usepackage{longtable}
%\usepqckage{tablefootnote}
\usepackage[polutonikogreek,english,japanese]{babel}
%\usepackage{amsmath}
\usepackage{latexsym}
\usepackage{color}

%----- header -------
\usepackage{fancyhdr}
\lhead{{\it Summa Theologiae} III, q.~2}
%--------------------

\bibliographystyle{jplain}

\title{{\bf TERTIA PARS}\\{\HUGE Summae Theologiae}\\Sancti Thomae
Aquinatis\\{\sffamily QUEAESTIO SECUNDA}\\DE MODO UNIONIS VERBI
INCARNATI\\QUANTUM AD IPSAM UNIONEM}
\author{Japanese translation\\by Yoshinori {\sc Ueeda}}
\date{Last modified \today}


%%%% コピペ用
%\rhead{a.~}
%\begin{center}
% {\Large {\bf }}\\
% {\large }\\
% {\footnotesize }\\
% {\Large \\}
%\end{center}
%
%\begin{longtable}{p{21em}p{21em}}
%
%&
%
%
%\\
%\end{longtable}
%\newpage



\begin{document}
\maketitle
\pagestyle{fancy}

\begin{center}
{\Large 第二問\\受肉した言葉の合一のしかたについて:合一それ自体にかんし
 て}
\end{center}
\newpage

\begin{longtable}{p{21em}p{21em}}

Deinde considerandum est de modo unionis Verbi incarnati. Et primo
 quantum ad ipsam unionem; secundo, quantum ad personam assumentem;
 tertio, quantum ad naturam assumptam. Circa primum quaeruntur
 duodecim. 


&


次に、受肉した言葉の合一のしかたについて考察されるべきである。第一に、合
 一それ自体について、第二に、受容するペルソナについて、第三に、受容の本性に
 ついて。第一をめぐって、12のことが問われる。

\\

\begin{enumerate}
 \item utrum unio Verbi incarnati sit facta in natura.
 \item utrum sit facta in persona.
 \item utrum sit facta in supposito vel hypostasi.
 \item utrum persona vel hypostasis Christi post incarnationem sit
       composita.
 \item utrum sit facta aliqua unio animae et corporis in Christo.
 \item utrum natura humana fuerit unita Verbo accidentaliter.
 \item utrum ipsa unio sit aliquid creatum.
 \item utrum sit idem quod assumptio.
 \item utrum sit maxima unionum.
 \item utrum unio duarum naturarum in Christo fuerit facta per gratiam.
 \item utrum eam aliqua merita praecesserint.
 \item utrum aliqua gratia fuerit homini Christo naturalis.
\end{enumerate}


&



\begin{enumerate}
 \item 受肉した言葉の合一は、本性においてなされたか。
 \item ペルソナにおいてなされたか。
 \item 個体、ないしヒュポスタシスにおいてなされたか。
 \item キリストのペルソナないしヒュポスタシスは、受肉の後、複合されたも
       のか。
 \item キリストにおいて、魂と身体のなんらかの合一がなされたか。
 \item 人間本性は、附帯的に、言葉に合一されたか。
 \item 合一そのものは創造されたものか。
 \item 合一は受容と同じか。
 \item それは合一の中で最大のものか。
 \item キリストにおける二つの本性の合一は恩恵によって生じたか。
 \item その恩恵には何らかの功績が先行したか。
 \item 何らかの恩恵が人間キリストにとって自然本性的だったか。
\end{enumerate}
\end{longtable}
\newpage


\rhead{a.~1}
\begin{center}
 {\Large {\bf ARTICULUS PRIMUS}}\\
 {\large UTRUM UNIO VERBI INCARNATI SIT FACTA IN UNA NATURA}\\
 {\footnotesize III {\itshape Sent.}, d.5, q.1, a.2; IV {\itshape SCG},
 cap.35, 41; {\itshape De Unione Verbi}, a.1; {\itshape De Verit.},
 q.20, a.1; {\itshape Compend.~Theol.}, cap.206; {\itshape In Ioan.},
 cap.1, lect.7; {\itshape Ad Rom.}, cap.1, lect.2; {\itshape Ad
 Philipp.}, cap.2, lect.2.}\\
 {\Large 第一項\\受肉した言葉の合一は一つの本性においてなされたか}
\end{center}


\begin{longtable}{p{21em}p{21em}}

{\Huge A}{\scshape d primum sic proceditur}. Videtur quod unio Verbi incarnati sit facta in
una natura. Dicit enim Cyrillus, et inducitur in gestis Concilii
Chalcedonensis, {\itshape non oportet intelligere duas naturas, sed unam naturam
Dei Verbi incarnatam}. Quod quidem non fieret nisi unio esset in
natura. Ergo unio Verbi incarnati facta est in natura.


&

第一項の問題へ、議論は以下のように進められる。
受肉した言葉の合一は、一つの本性において生じたと思われる。理由は以下の通
 り。
キュリッルス(キリロ)が語り、カルケドン公会議の議事録から引用されること
 だが、「二つの本性を知解すべきでなく、神の言葉の受肉した一つの本性を知解すべきであ
 る」。このことは、合一が本性においてでなかったならば生じなかったであろう。
 ゆえに、受肉した言葉の合一は、本性においてなされた。



\\



2. {\scshape Praeterea}, Athanasius dicit, sicut anima rationalis et caro conveniunt
in constitutione humanae naturae, sic Deus et homo conveniunt in
constitutione alicuius unius naturae. Ergo facta est unio in natura.


&


さらに、アタナシウスは、理性的魂と肉とが人間本性の構成において一致するよ
 うに、神と人間は、ある一つの本性の構成において一致する。ゆえに、合一は
 本性においてなされた。

\\



3. {\scshape Praeterea}, duarum naturarum una non denominatur ex altera nisi aliquo
modo in invicem transmutentur. Sed divina natura et humana in Christo ab
invicem denominantur, dicit enim Cyrillus divinam naturam esse
{\itshape incarnatam}; et Gregorius Nazianzenus dicit naturam humanam esse
{\itshape deificatam}; ut patet per Damascenum. Ergo ex duabus naturis videtur esse
facta una natura.


&

さらに、二つの本性のうち、一方が他方によって名付けられるのは、何らかの相
 互的な変容による以外にない。しかし、神と人間の本性は、キリストにおいて
 相互に名付けられる。たとえば、キュリッルスは、神の本性が受肉したと言い、
 ナジアンズスのグレゴリウスは、ダマスケヌスによって明らかなとおり、人間
 本性が神化されたと言う。ゆえに、二つの本性から一つの本性が生じたと思わ
 れる。


\\



{\scshape Sed contra est} quod dicitur in determinatione Concilii Chalcedonensis,
{\itshape confitemur in novissimis diebus filium Dei unigenitum inconfuse,
immutabiliter, indivise, inseparabiliter agnoscendum, nusquam sublata
differentia naturarum propter unionem}. Ergo unio non est facta in
natura.


&

しかし反対に、カルケドン公会議の決定において、次のように言われている。
 「神の一人息子が、混乱することなく、不変に、分割されず、分離され
 ず、合一のために本性の差異がどこにおいてもなくなることなく、知られるべ
 きであることを、この直近の日々において、私たちは告白する」。ゆえに、合
 一が本性においてなされたのではない。



\\

{\scshape Respondeo dicendum} quod ad huius quaestionis evidentiam, oportet
considerare quid sit natura. Sciendum est igitur quod nomen naturae a
nascendo est dictum vel sumptum. Unde primo est impositum hoc nomen ad
significandum generationem viventium, quae nativitas vel pullulatio
dicitur, ut dicatur natura quasi nascitura. 

&

解答する。以下のように言われるべきである。
この問題を明らかにするためには、本性とは何かを考察しなければならない。
さて、本性というこの名は、「生む」ことに由来して言われる。このことから、
 この名は、第一に、生物の生成を表示するために付けられた。「本性」(natura)は
「生まれるであろうもの」(nascitura)のように語られるが、そのように、それ
 は「出産」や「発生」と呼ばれる。


\\



Deinde translatum est nomen
naturae ad significandum principium huius generationis. Et quia
principium generationis in rebus viventibus est intrinsecum, ulterius
derivatum est nomen naturae ad significandum quodlibet principium
intrinsecum motus, secundum quod philosophus dicit, in II {\itshape Physic}., quod
{\itshape natura est principium motus in eo in quo est per se et non secundum
accidens}. 


&

次に、「本性」という名は、この生成の根源を表示するように転化した。そして、
 生物における生成の根源は内在的なものなので、さらに、「本性」という名は、
 何であれ運動の内在的根源を表示するように派生した。これは、哲学者が『自然
 学』第2巻で「本性は、附帯的にではなくそれ自体においてあるものにおける運
 動の根源である」による。


\\


Hoc autem principium vel forma est, vel materia. Unde
quandoque natura dicitur forma, quandoque vero materia. Et quia finis
generationis naturalis est, in eo quod generatur, essentia speciei, quam
significat definitio, inde est quod huiusmodi essentia speciei vocatur
etiam natura. 


&

ところで、この根源は、形相であるか、あるいは質料であるかである。このこと
 から、あるときには形相が本性と言われ、あるときには質料が本性と言われる。
 そして、自然本性的な生成の目的は、生成するものにおいて、定義が表示する
 種の本質であるから、そのような種の本質もまた、「本性」と呼ばれる。


\\


Et hoc modo Boetius naturam definit, in libro de duabus
naturis, dicens, {\itshape natura est unamquamque rem informans specifica
differentia}, quae scilicet complet definitionem speciei. Sic ergo nunc
loquimur de natura, secundum quod natura significat essentiam, vel quod
quid est, sive quidditatem speciei. 



&

そしてこの意味で、ボエティウスは、『二つの本性について』という書物で「本
 性」を定義して、「本性とは、各々の事物に種を構成する種差を与えるもので
 ある」と述べる。それはすなわち、種の定義を完成するものである。それゆえ、
 このようにして、私たちは今、「本性」が、本質、あるいは「何である
 か」あるいは、種の何性を表示するかぎりにおいて、本性について語っている。


\\


Hoc autem modo accipiendo naturam,
impossibile est unionem Verbi incarnati esse factam in
natura. Tripliciter enim aliquid unum ex duobus vel pluribus
constituitur. Uno modo, ex duobus perfectis integris remanentibus. 



&

ところで、このように本性を理解するとき、受肉した言葉の合一が、本性におい
 てなされることは不可能である。理由は以下の通り。
ある一つのものが、二つないし複数のものから構成されるしかたには三つある。
一つは、二つの完全なものがそのまま完全なものに留まりつつ、その二つから
 生じるしかたである。


\\

Quod
quidem fieri non potest nisi in his quorum forma est compositio, vel
ordo, vel figura, sicut ex multis lapidibus absque aliquo ordine
adunatis per solam compositionem fit acervus; ex lapidibus autem et
lignis secundum aliquem ordinem dispositis, et etiam ad aliquam figuram
redactis, fit domus. 
Et secundum hoc, posuerunt aliqui unionem esse per
modum confusionis, quae scilicet est sine ordine; vel commensurationis,
quae est cum ordine. 

&

これは、その形相が複合や秩序、あるいは形態であるものにおいてしか生じえない。
 たとえば、多くの石から、何らの秩序も加えられることなく、ただの複合によっ
 て、堆積が生じる場合のように。あるいは、石と木材とから、ある秩序にした
 がって配置され、さらには何らかの形態へと作られて、家が生じる場合のよ
 うに。
そして、ある人々はこうしたことに即して、秩序がない場合には、合一が混沌と
 いうかたちである、あるいは、秩序がある場合には、尺度によるかたちであると考えた。



\\


Sed hoc non potest esse. Primo quidem, quia
compositio, ordo vel figura non est forma substantialis, sed
accidentalis. Et sic sequeretur quod unio incarnationis non esset per
se, sed per accidens, quod infra improbabitur. Secundo, quia ex
huiusmodi non fit unum simpliciter, sed secundum quid, remanent enim
plura actu. Tertio, quia forma talium non est natura, sed magis ars,
sicut forma domus. Et sic non constitueretur una natura in Christo, ut
ipsi volunt. 

&

しかしこれはありえない。第一に、複合、秩序、ないし形態は、実体的形相では
 なく、附帯的形相である。したがって、受肉の合一が、自体的にでなく附帯的
 にあることになるが、これは、後で論駁される。第二に、このようなことから
 は端的な一は生じず、ある意味における一が生じるだけであり、現実態におい
 て、複数のものが留まる。第三に、そのようなものの形相は本性ではなく、む
 しろ、家の形相のように、技術である。そうして、彼らが意志するように、キ
 リストにおいて一つの本性は構成されないであろう。


\\


Alio modo fit aliquid ex perfectis, sed transmutatis, sicut
ex elementis fit mixtum. Et sic aliqui dixerunt unionem incarnationis
esse factam per modum complexionis. 


&

もう一つのしかたでは、あるものが、完全なものから、しかし変容したものから
 生じる。たとえば、元素から混合物が生じる場合のように。この意味で、ある
 人々は、受肉の合一は、混合のしかたで生じたと言った。


\\


Sed hoc non potest esse. Primo
quidem, quia natura divina est omnino immutabilis, ut in prima parte
dictum est. Unde nec ipsa potest converti in aliud, cum sit
incorruptibilis, nec aliud in ipsam, cum ipsa sit
ingenerabilis. 


&


しかしこれはありえない。第一に、第一部で述べられたとおり、神の本性は、あ
 らゆる点で不変である。したがって、それは不滅なので、他のものへと転化し
 えないし、それは不生なので、他のものも、それへ転化しえない。


\\


Secundo, quia id quod est commixtum, nulli miscibilium
est idem specie, differt enim caro a quolibet elementorum specie. Et sic
Christus nec esset eiusdem naturae cum patre, nec cum matre. Tertio,
quia ex his quae plurimum distant non potest fieri commixtio, solvitur
enim species unius eorum, puta si quis guttam aquae amphorae vini
apponat. Et secundum hoc, cum natura divina in infinitum excedat
humanam, non potest esse mixtio, sed remanebit sola natura
divina. 

&

第二に、混合したものは、種において、どの混合の要素とも同じでない。たとえ
 ば、肉は、どの元素とも種において同一でない。だから、キリストは、父とも
 母とも同一の本性でなかったことになる。第三に、遠く隔たっているものから
 混合は生じえず、それらの一つの種が解かれる。たとえば、水の一滴をワイン
 の壺に注ぐ場合のように。したがって、神の本性は無限に人間本性を越えるの
 で、混合されることはできず、ただ神の本性に留まるであろう。

\\


Tertio modo fit aliquid ex aliquibus non permutatis, sed
imperfectis, sicut ex anima et corpore fit homo; et similiter ex
diversis membris. 


&

第三のしかたで、あるものは、変化したものではなく不完全なののから生じる。
 たとえば、魂と身体とから人間が生じ、また同様に、さまざまな四
 肢から生じるように。


\\


Sed hoc dici non potest de incarnationis
mysterio. Primo quidem, quia utraque natura est secundum suam rationem
perfecta, divina scilicet et humana. Secundo, quia divina et humana
natura non possunt constituere aliquid per modum partium
quantitativarum, sicut membra constituunt corpus, quia natura divina est
incorporea. 


&


しかし、これは、受肉の秘蹟について語られえない。第一に、どちらの本性も、
つまり神の本性も人間の本性も、それ自身の性格において完全である。
第二に、神の本性と人間の本性は、四肢が身体を構成するようなかたちで、部分
 のしかたによって何かを構成することができない。それは、神の本性が非物体
 的だからである。


\\


Neque per modum formae et materiae, quia divina natura non
potest esse forma alicuius, praesertim corporei. Sequeretur etiam quod
species resultans esset communicabilis pluribus, et ita essent plures
Christi. Tertio, quia Christus neque esset humanae naturae, neque
divinae, differentia enim addita variat speciem, sicut unitas in
numeris, sicut dicitur in VIII Metaphys.


&

また、形相と質料のかたちでもない。なぜなら、神の本性は、どんなものの、と
 くに物体的なものの形相ではありえないからである。また、その結果として生
 じる種が、複数のものに共通するものとなり、複数のキリストがいることになっ
 たであろう。第三に、キリストには、人間本性も神の本性もないことになった
 であろう。『形而上学』第8巻で言われるように、数における一性のように、付
 加された種差は種を変えるからである。


\\



{\scshape Ad primum ergo dicendum} quod illa auctoritas Cyrilli exponitur in quinta
synodo sic, {\itshape si quis, unam naturam Dei Verbi incarnatam dicens, non sic
accipit sicut patres docuerunt, quia ex divina natura et humana unione
secundum subsistentiam facta, sed ex talibus vocibus naturam unam sive
substantiam divinitatis et carnis Christi introducere conatur, talis
anathema sit}. Non ergo sensus est quod in incarnatione ex duabus naturis
sit una natura constituta, sed quia una natura Dei Verbi carnem univit
in persona.


&

第一異論に対しては、それゆえ、以下のように言われるべきである。
かのキュリッルスの権威は、第五信教において、以下のように説明されている。
 「もし誰かが、神の言葉の受肉した一つの本性を語り、父祖たちが教えたよう
 に、つまり神と人間の本性から、自存における合一が生じたというように理解
 せず、そのような音声によって、神性とキリストの肉の一つの本性ないし実体を導
 入しようとするならば、それは異端である」。ゆえに、受肉において、二つの
 本性から一つの本性が構成されるという意味ではなく、むしろ、神の言葉の一
 つの本性が、ペルソナにおいて、肉と合一したという意味である。

\\



{\scshape Ad secundum dicendum} quod ex anima et corpore constituitur in unoquoque
nostrum duplex unitas, naturae, et personae. Naturae quidem, secundum
quod anima unitur corpori, formaliter perficiens ipsum, ut ex duabus
fiat una natura, sicut ex actu et potentia, vel materia et forma. Et
quantum ad hoc non attenditur similitudo, quia natura divina non potest
esse corporis forma, ut in primo probatum est. Unitas vero personae
constituitur ex eis inquantum est unus aliquis subsistens in carne et
anima. Et quantum ad hoc attenditur similitudo, unus enim Christus
subsistit in divina natura et humana.


&

第二異論に対しては、以下のように言われるべきである。
魂と身体とから、私たちの各々において、二つの一性、すなわち、本性とペルソ
 ナの一性が構成される。本性の一性は、魂が、形相的に身体を完成するものと
 して身体に合一されることによるもので、現実態と可能態から、あるいは質料
 と形相から、二つのものから一つの本性が生じる。
そしてこれにかんして、類似は見出されない。なぜなら、第一部で証明されたと
 おり、神の本性は身体の形相でありえないからである。
他方、ペルソナの一性は、ある一人の、肉と魂において自存する人がいる
限りにおいて、それらから構成される。そして、この点で、類似が見出される。
 すなわち、キリストは、神と人間の本性において自存する。




\\



{\scshape Ad tertium dicendum} quod, sicut Damascenus dicit, natura divina dicitur
incarnata, quia est unita carni personaliter: non quod sit in naturam
carnis conversa. Similiter etiam caro dicitur deificata, ut ipse dicit,
non per conversionem, sed per unionem ad Verbum, salvis suis
proprietatibus, ut intelligatur caro deificata quia facta est Dei Verbi
caro, non quia facta sit Deus.


&


第三異論に対しては、以下のように言われるべきである。
ダマスケヌスが言うように、神の本性が受肉したと言われるのは、それがペルソ
 ナ的に肉に合一するからであり、それが肉の本性に転化したからではない。同
 様に、彼が言うとおり、肉が神化されると言われるのも、転化によってではな
 く、それ自身の固有性を保持しつつ、言葉への合一によってである。神化され
 た肉とは、肉が神の言葉に作られたと理解されるのであり、肉が神になったわけで
 はない。



\end{longtable}
\newpage






\rhead{a.~2}
\begin{center}
 {\Large {\bf ARTICULUS SECUNDUS}}\\
 {\large UTRUM UNIO VERBI INCARNATI SIT FACTA IN PERSONA}\\
 {\footnotesize III {\itshape Sent.}, d.5, q.1, a.3; IV {\itshape SCG},
 cap.41; {\itshape De Unione Verbi}, a.1; {\itshape Ad Philipp.},
 cap.2, lect.2.}\\
 {\Large 第二項\\受肉した言葉の合一はペルソナにおいてなされたか}
\end{center}

\begin{longtable}{p{21em}p{21em}}




{\Huge A}{\scshape d secundum sic proceditur}. Videtur quod unio Verbi incarnati non sit
facta in persona. Persona enim Dei non est aliud a natura ipsius, ut
habitum est in primo. Si ergo unio non est in natura, sequitur quod non
sit facta in persona.


&


第二項の問題へ、議論は以下のように進められる。
受肉した言葉の合一は、ペルソナにおいて為されたのではないと思われる。理由
 は以下の通り。
第一部で論じられたとおり、神のペルソナは、神の本性と別のものでない。
ゆえに、もし、合一が本性においてでないならば、ペルソナにおいてなされたの
 でもないことが帰結する。

\\



2. {\scshape Praeterea}, natura humana non est minoris dignitatis in Christo quam in
nobis. Personalitas autem ad dignitatem pertinet, ut in primo habitum
est. Cum ergo natura humana in nobis propriam personalitatem habeat,
multo magis habuit propriam personalitatem in Christo.


&


さらに、人間本性は、キリストにおいて、私たちにおいて持つより小さい偉大さ
 を持つわけでない。しかし、第一部で論じられたとおり、ペルソナ性は、偉大
 さに属する。ゆえに、人間本性は私たちにおいて固有のペルソナ性をもつので、
 ましてや、キリストにおいて、固有のペルソナ性をもった。

\\



3. {\scshape Praeterea}, sicut Boetius dicit, in libro {\itshape de Duabus Naturis}, {\itshape persona est
rationalis naturae individua substantia}. Sed Verbum Dei assumpsit
naturam humanam individuam, {\itshape natura} enim {\itshape universalis non sistit secundum
se, sed in nuda contemplatione consideratur}, ut Damascenus dicit. Ergo
humana natura habet suam personalitatem. Non ergo videtur quod sit facta
unio in persona.


&

さらに、ボエティウスが『二つの本性について』という書物で言うように、ペル
 ソナとは、理性的本性をもつ個体的実体である。しかし、神の言葉は、人間の
 個体的本性を受容した。なぜなら、ダマスケヌスが、「普遍はそれ自体において
 存立せず、裸の観照によって考察される」と言うからである。ゆえに、人間本
 性は、自らのペルソナ性をもつ。ゆえに、ペルソナにおいて合一がなされると
 は思われない。


\\



{\scshape Sed contra est} quod in Chalcedonensi synodo legitur, {\itshape non in duas
personas partitum aut divisum, sed unum et eundem filium unigenitum
dominum nostrum Iesum Christum confitemur}. Ergo facta est unio Verbi in
persona.


&

しかし、カルケドン信教には次のようにある。「二つのペルソナに分かれず分離さ
 れず、一人で同一の一人息子が、私たちの主であるイエス・キリストであると
 私たちは告白する」。ゆえに、言葉の合一はペルソナにおいてなされた。


\\



{\scshape Respondeo dicendum} quod persona aliud significat quam natura. Natura
enim significat essentiam speciei, quam significat definitio. Et si
quidem his quae ad rationem speciei pertinent nihil aliud adiunctum
inveniri posset, nulla necessitas esset distinguendi naturam a supposito
naturae, quod est individuum subsistens in natura illa, quia unumquodque
individuum subsistens in natura aliqua esset omnino idem cum sua
natura. 


&

解答する。以下のように言われるべきである。
「ペルソナ」は、「本性」と違うものを意味表示する。
「本性」は、種の本質を表示し、それを定義が表示する。
もし、種の性格に属するものに結びつけられた、他の何かが見出されえなかった
 ならば、本性を本性の個体(suppositum)、つまりその本性において自存する個体(individuum)から区別する必要性はなかったであろう。なぜなら、その場合、ある本性
 において自存する各々の個体は、自らの本性とあらゆる点で同一だっただろう
 から。

\\

Contingit autem in quibusdam rebus subsistentibus inveniri
aliquid quod non pertinet ad rationem speciei, scilicet accidentia et
principia individuantia, sicut maxime apparet in his quae sunt ex
materia et forma composita. 

&

しかし、ある自存する諸事物において、種の性格に属さない何か、すなわち、附
 帯性や個体化の根源が見出されることがある。たとえば、質料と形相とから
 複合されているものにおいて、これは最大限に明らかである。

\\

Et ideo in talibus etiam secundum rem
differt natura et suppositum, non quasi omnino aliqua separata, sed quia
in supposito includitur ipsa natura speciei, et superadduntur quaedam
alia quae sunt praeter rationem speciei. 


&

それゆえ、そのようなものにおいては、事物に即してでさえ(実在的にでさえ)、
 本性と個体は異なるが、それらがまったく分離しているというわけではなく、個体のなかに本
 性自身が含まれていて、種の性格の外にある他のものどもが付加されている。


\\

Unde suppositum significatur ut
totum, habens naturam sicut partem formalem et perfectivam sui. Et
propter hoc in compositis ex materia et forma natura non praedicatur de
supposito, non enim dicimus quod hic homo sit sua humanitas. 

&

したがって、個体は、本性を形相的で自らを完成する部分としてもつ全体として表示される。このため、質料と形相から複合されたものにおいて、本
 性は個体に述語付けられない。たとえば、私たちは、「この人間は自らの人間性であ
 る」とは言わない。


\\

Si qua vero
res est in qua omnino nihil est aliud praeter rationem speciei vel
naturae suae, sicut est in Deo, ibi non est aliud secundum rem
suppositum et natura, sed solum secundum rationem intelligendi, quia
natura dicitur secundum quod est essentia quaedam; eadem vero dicitur
suppositum secundum quod est subsistens. 


&

他方、もし、自らの種や本性の性格以外のものをまったく何ももたないものがあ
 れば、たとえば神のように、そこでは、個体と本性は、事物に即して(実在的
 に)別のものでなく、むしろ、知解の性格に即してのみ異なる。なぜなら、そ
 れは、ある種の本質であるかぎりで本性と言われ、同じものが、自存するもの
 であるかぎりで個体と言われるからである。


\\

Et quod est dictum de
supposito, intelligendum est de persona in creatura rationali vel
intellectuali, quia nihil aliud est persona quam rationalis naturae
individua substantia, secundum Boetium. 
Omne igitur quod inest alicui
personae, sive pertineat ad naturam eius sive non, unitur ei in
persona. 

&

そして、個体について語られたことは、理性的、ないし知性的被造物において、
 ペルソナについて語られる。なぜなら、ペルソナとは、ボエティウスによれば、
 理性的本性を持つ個体的実体だからである。
ゆえに、あるペルソナに内在する全てのものは、それがその本性に属していても
 いなくても、ペルソナにおいてそれに合一されている。

\\




Si ergo humana natura Verbo Dei non unitur in persona, nullo
modo ei unitur. 
Et sic totaliter tollitur incarnationis fides, quod est
subruere totam fidem Christianam. Quia igitur Verbum habet naturam
humanam sibi unitam, non autem ad suam naturam divinam pertinentem
consequens est quod unio sit facta in persona Verbi, non autem in
natura.


&

それゆえ、もし、人間本性が神
 の言葉にペルソナにおいて合一していないならば、どんなかたちでもそれに合
 一しないであろう。
かくして、受肉の信仰は完全に倒され、それは、キリスト教全体を破壊すること
 である。ゆえに、言葉は、人間本性を、自らの神の本
 性に属するものとしてではなく、自らに合一されたものとして持つのだから、
 その合一が、本性においてではなく、言葉のペルソナにおいてなされたことが
 帰結する。

\\



{\scshape Ad primum ergo dicendum} quod, licet in Deo non sit aliud secundum rem
natura et persona, differt tamen secundum modum significandi, sicut
dictum est, quia persona significat per modum subsistentis. Et quia
natura humana sic unitur Verbo ut Verbum in ea subsistat, non autem ut
aliquid addatur ei ad rationem suae naturae, vel ut eius natura in
aliquid transmutetur, ideo unio facta est in persona, non in natura.


&

第一異論に対しては、それゆえ、以下のように言われるべきである。
神において、実在的に、本性とペルソナが異ならないとしても、すでに述べられ
 たとおり、表示のしかたが異なる。「ペルソナ」は自存するもののありかたに
 よって表示するからである。そして、人間本性は、言葉がその本性において自
 存するようなかたちで言葉に合一されるのであり、自らの本性の性格に何かが付加さ
 れるようにでも、その本性が何かへ転化するようにでもない。ゆえに、合一は、ペルソ
 ナにおいてなされたのであり、本性においてではない。

\\



{\scshape Ad secundum dicendum} quod personalitas necessario intantum pertinet ad
dignitatem alicuius rei et perfectionem, inquantum ad dignitatem et
perfectionem eius pertinet quod per se existat, quod in nomine personae
intelligitur. Dignius autem est alicui quod existat in aliquo se
digniori, quam quod existat per se. 

&

第二異論に対しては、以下のように言われるべきである
ペルソナ性がある事物の偉大さや完全性に必然的に属するのは、自存することが、
 それの偉大さや完全性に属するのに応じてであり、この「自存すること」が「ペルソナ」と
 いう名において知解される。ところで、自らによって存在するものよりも、自
 らよりも偉大なものにおいて存在することの方が、そのものにとって偉大であ
 る。


\\


Et ideo ex hoc ipso humana natura
dignior est in Christo quam in nobis, quia in nobis, quasi per se
existens, propriam personalitatem habet in Christo autem existit in
persona Verbi. Sicut etiam esse completivum speciei pertinet ad
dignitatem formae, tamen sensitivum nobilius est in homine, propter
coniunctionem ad nobiliorem formam completivam, quam sit in bruto
animali, in quo est forma completiva.


&

ゆえに、このことから、人間本性は、私たちにおいてよりもキリストにおいてよ
 り偉大である。なぜなら、それ自身によって存在するものとしての私たちにお
 いて、人間本性は固有のペルソナ性をもつが、キリストにおいては、言葉のペ
 ルソナにおいて存在するからである。
さらに、種を完成するということは、形相の偉大さに属するが、感覚的であるこ
 とは、非理性的動物においてよりも人間において、より高貴である。非理性的動物において、
 それは種を完成させる形相だが、人間においては、より高貴な完成する形相と
 結びついているからである。


\\



{\scshape Ad tertium dicendum} quod Dei Verbum non assumpsit naturam humanam in
universali, sed in atomo, idest in individuo, sicut Damascenus dicit,
alioquin oporteret quod cuilibet homini conveniret esse Dei Verbum,
sicut convenit Christo. 



&

第三異論に対しては、以下のように言われるべきである。
神の言葉は、ダマスケヌスが言うとおり、普遍において人間本性を受容したのでは
 なく、アトムにおいて、つまり個体において取った。もしそうでなければ、
 「神の言葉である」ということが、キリストに適合したように、どんな人間にも適
 合しなければならなかったであろう。


\\


Sciendum est tamen quod non quodlibet individuum
in genere substantiae, etiam in rationali natura, habet rationem
personae, sed solum illud quod per se existit, non autem illud quod
existit in alio perfectiori. Unde manus Socratis, quamvis sit quoddam
individuum, non est tamen persona, quia non per se existit, sed in
quodam perfectiori, scilicet in suo toto. 



&

しかし、実体の類のうち、どんな個体も、それが理性的本性においてあるものですら、ペルソナの
 性格を持つわけではなく、ただ、自らによって存在するものだけがその性格を
 持ち、より完全な別の者の中に存在するものは、その性格を持たないことが知
 られるべきである。したがって、ソクラテスの手は、何らかの個体だが、ペル
 ソナではない。なぜなら、自らによって存在せず、なんらかのより完全なもの、
 すなわち、自らの全体において存在するからである。


\\


Et hoc etiam potest
significari in hoc quod persona dicitur substantia individua, non enim
manus est substantia completa, sed pars substantiae. Licet igitur humana
natura sit individuum quoddam in genere substantiae, quia tamen non per
se separatim existit, sed in quodam perfectiori, scilicet in persona Dei
Verbi, consequens est quod non habeat personalitatem propriam. Et ideo
facta est unio in persona.


&

このことは、ペルソナが個体的実体と言われることにおいても表示されている。
つまり、手は完全な実体ではなく、実体の部分である。ゆえに、人間本性が、実
 体の類における何らかの個体であるにしても、それ自体によって、多から離れ
 て存在せず、より完全な何か、すなわち神の言葉のペルソナにおいて存在する
 ので、固有のペルソナ性をもたないことが帰結する。ゆえに、合一はペルソナ
 においてなされた。


\end{longtable}
\newpage





\rhead{a.~3}
\begin{center}
 {\Large {\bf ARTICULUS TERTIUS}}\\
 {\large UTRUM UNIO VERBI INCARNATI SIT FACTA \\IN SUPPOSITO, SIVE IN HYPOSTASI}\\
 {\footnotesize III {\itshape Sent.}, d.6, q.1, a.1, qu$^a$ 1, 2; d.7,
 q.1, a.1; IV {\itshape SCG}, cap.38, 39; {\itshape De Union.~Verbi},
 a.2; {\itshape Cont.~Error.~Graec.}, cap.20; {\itshape
 Cont.~Graec.~Armen.~etc.}, cap.6; {\itshape Compend.~Theol.}, cap.210,
 221; {\itshape Quodl}.~IX, q.2, a.3; {\itshape In Ioan.}, cap.1, lect.7.}\\
 {\Large 第三項\\受肉した言葉の合一は個体、ないしヒュポスタシスにおいて
 なされたか}
\end{center}

\begin{longtable}{p{21em}p{21em}}



{\Huge A}{\scshape d tertium sic proceditur}. Videtur quod unio Verbi incarnati non sit
facta in supposito, sive in hypostasi. Dicit enim Augustinus, in
Enchirid., {\itshape divina substantia et humana utrumque est unus Dei filius, sed
aliud propter Verbum, et aliud propter hominem}. Leo Papa etiam dicit, in
epistola ad Flavianum, {\itshape unum horum coruscat miraculis, et aliud succumbit
iniuriis. Sed omne quod est aliud et aliud, differt supposito}. Ergo unio
Verbi incarnati non est facta in supposito.

&

第三項の問題へ、議論は以下のように進められる。
受肉した言葉の合一は、個体ないしヒュポスタシスにおいてなされたのではない
 と思われる。理由は以下の通り。
アウグスティヌスは『エンキリディオン』で次のように言う。「神の実体と人間の実体は、両方
 とも、一人の神の息子である。しかし、一方は言葉のため、他方は人間のため
 である」。また、教皇レオも『フラウィアヌス宛書簡』で言う。「これらのう
 ちのひとつは栄光によって輝き、他方は不正によって沈む。しかしすべて別々
 のものは、個体において異なる」。ゆえに、受肉した言葉の合一は個体におい
 て為されたのではないと思われる。


\\



2. {\scshape Praeterea}, hypostasis nihil est aliud quam {\itshape substantia particularis}, ut
Boetius dicit, in libro {\itshape de Duabus Naturis}. Manifestum est autem quod in
Christo est quaedam alia substantia particularis praeter hypostasim
Verbi, scilicet corpus et anima et compositum ex eis. Ergo in Christo
est alia hypostasis praeter hypostasim Verbi Dei.

&

さらに、ボエティウスが『二つの本性について』で述べるように、ヒュポスタシ
 スとは、「個的実体」に他ならない。ところで、キリストにおいて、言葉のヒュ
 ポスタシス以外に、肉体と魂と、それらからの複合体という、別の個的実体が
 あることは明らかである。ゆえに、キリストにおいて、神の言葉のヒュポスタ
 シス以外に、他のヒュポスタシスがある。


\\



3. {\scshape Praeterea}, hypostasis Verbi non continetur in aliquo genere, neque sub
specie, ut patet ex his quae in prima parte dicta sunt. Sed Christus,
secundum quod est factus homo, continetur sub specie humana, dicit enim
Dionysius, {\scshape i} cap.~{\itshape de Div.~Nom}., {\itshape intra nostram factus est naturam qui
omnem ordinem secundum omnem naturam supersubstantialiter excedit}. Non
autem continetur sub specie humana nisi sit hypostasis quaedam humanae
speciei. Ergo in Christo est alia hypostasis praeter hypostasim Verbi
Dei. Et sic idem quod prius.

&

さらに、第一部で述べられたことから明らかなとおり、言葉のヒュポスタシスは、
 どんな類にも種にも含まれない。しかしキリストは、人となった限りで、人間
 の種に含まれる。じっさい、ディオニュシウスは『神名論』第1章で「全ての秩
 序を全ての本性において超実体的に超えるものが、私たちの本性へと作られた」
 と言う。しかし、人間の種に含まれるものは、人間の種を持つ何らかのヒュポ
 スタシス以外にはない。ゆえに、キリストの中に、神の言葉のヒュポスタシス
 以外に、他のヒュポスタシスがある。ゆえに、前の異論と同じ。

\\



{\scshape Sed contra est} quod Damascenus dicit, in III libro, {\itshape in domino nostro
Iesu Christo duas naturas cognoscimus, unam autem hypostasim}.

&

しかし反対に、ダマスケヌスは、第三巻で「私たちの主であるイエス・キリ
 ストにおいて、私たちは二つの本性と一つのヒュポスタシスを知る」と言う。

\\



{\scshape Respondeo dicendum} quod quidam, ignorantes habitudinem hypostasis ad
personam, licet concederent in Christo unam solam personam, posuerunt
tamen aliam hypostasim Dei et aliam hominis, ac si unio sit facta in
persona, non in hypostasi. Quod quidem apparet erroneum
tripliciter. 


&

解答する。以下のように言われるべきである。
ある人々は、ヒュポスタシスとペルソナの関係を知らず、キリストの中にただ一
 つのペルソナがあることを認めながら、神のヒュポスタシスと人間のヒュポス
 タシスが別であり、合一はペルソナにおいてなされたのであり、ヒュポスタシ
 スにおいてではないと主張した。これが誤りであることは、三通りのしかたで
 明らかである。


\\


Primo, ex hoc quod persona supra hypostasim non addit nisi
determinatam naturam, scilicet rationalem; secundum quod Boetius dicit,
in libro {\itshape de Duabus Naturis}, quod {\itshape persona est rationalis naturae
individua substantia}. 
Et ideo idem est attribuere propriam hypostasim
humanae naturae in Christo, et propriam personam.

&

第一に、ボエティウスが『二つの本性について』という書物で「ペルソナとは理
 性的本性を持つ個的実体である」と言うのによれば、ペルソナはヒュポスタシスに、限定された本性、すなわち理性的である
 ことを加えるだけである。ゆえに、キリストにおいて、人間本性に固有のヒュ
 ポスタシスを帰するのは、固有のペルソナを帰するのと同じことである。


\\


 Quod intelligentes
sancti patres, utrumque in Concilio quinto, apud Constantinopolim
celebrato, damnaverunt, dicentes, {\itshape si quis introducere conetur in
mysterio Christi duas subsistentias seu duas personas, talis anathema
sit, nec enim adiectionem personae vel subsistentiae suscepit sancta
Trinitas, incarnato uno de sancta Trinitate, Deo Verbo}. 


&

聖なる教父たちは、このことを理解して、祝福されたコンスタンティノープルで
 の第五公会議において、次のように述べて両方を断罪した。「もしだれかが、
 キリストの秘蹟において、二つの実体ないし二つのペルソナを持ち込もうとす
 るならば、そのような人は異端である。なぜなら、聖なる三位の一つ、言葉で
 ある神が受肉するとき、聖なる三位は、どんなペルソナないし自存性の追加も
 受けなかったのだから」。

\\


``Subsistentia''
autem idem est quod res subsistens, quod est proprium hypostasis, ut
patet per Boetium, in libro {\itshape de Duabus Naturis}. 



&

ところで、この「自存性」とは、自存する事物と同じであり、それは、『二つの本性につ
 いて』という書物におけるボエティウスによって明らかなとおり、ヒュポスタ
 シスに固有である。


\\

Secundo quia, si detur
quod persona aliquid addat supra hypostasim in quo possit fieri unio,
hoc nihil est aliud quam proprietas ad dignitatem pertinens, secundum
quod a quibusdam dicitur quod {\itshape persona est hypostasis proprietate
distincta ad dignitatem pertinente}. 



&

第二に、仮にペルソナが、そこにおいて合一が生じうるヒュポスタシスに何かを
 加えるとしても、それは、偉大さに属する何かに他ならない。その限りで、あ
 る人々によって、ペルソナは、偉大さに属する固有性によって際立つヒュポス
 タシスと言われている。


\\

Si ergo facta sit unio in persona et
non in hypostasi, consequens erit quod non sit facta unio nisi secundum
dignitatem quandam. 


&

ゆえに、もし、合一がペルソナにおいてなされ、ヒュポスタシスにおいてでない
 ならば、何らかの偉大さに即して合一がなされたに違いないことになろう。


\\


Et hoc est, approbante synodo Ephesina, damnatum a
Cyrillo sub his Verbis, {\itshape si quis in uno Christo dividit subsistentias
post adunationem, sola copulans eas coniunctione quae secundum
dignitatem vel auctoritatem est vel secundum potentiam, et non magis
concursu secundum adunationem naturalem, anathema sit}. 


&

そしてこれは、エフェソス公会議で認められ、キュリッルスによって以下の言葉
 で異端とされた。「もしだれかが、一人のキリストにおいて、合一のあとに、
 自存性を分割し、偉大さないし権威ないし権能に即してある結合によってそれ
 らを結びつけ、本性の合一に即した一致によってそうしないならば、異端であ
 る」。


\\


Tertio, quia
tantum hypostasis est cui attribuuntur operationes et proprietates
naturae, et ea etiam quae ad naturae rationem pertinent in concreto,
dicimus enim quod {\itshape hic homo} ratiocinatur, et est risibilis, et est animal
rationale. 


&

第三に、働きや本性の固有性、そして本性の性格に具体的に属するものが帰せら
 れるのは、ヒュポスタシスだけである。たとえば、私たちは、「この人間」が、
 推論し、笑いうるし、理性的動物である、と言う。

\\


Et hac ratione {\itshape hic homo} dicitur esse suppositum, quia
scilicet supponitur his quae ad hominem pertinent, eorum praedicationem
recipiens. Si ergo sit alia hypostasis in Christo praeter hypostasim
Verbi, sequetur quod de aliquo alio quam de Verbo verificentur ea quae
sunt hominis, puta esse natum de virgine, passum, crucifixum et
sepultum. 


&

このため、「この人間」が、個体(suppositum)と言われる。なぜなら、人間の属するものども
 のもとにあり(supponitur)、それらの述語付けを受け取るものだからである。
 ゆえに、もし、キリストの中に、言葉のヒュポスタシスとは別のヒュポスタシ
 スがあれば、言葉以外の何か別のものについて、人間に属すること、たとえば、
 処女から産まれたこと、受難を受け、十字架にかけられ、埋葬されたというよ
 うなことが、真として語られることになる。

\\


Et hoc etiam damnatum est, approbante Concilio Ephesino, sub
his Verbis, {\itshape si quis personis duabus vel subsistentiis eas quae sunt in
evangelicis et apostolicis Scripturis impartitur voces, aut de Christo a
sanctis dictas, aut ab ipso de se; et quasdam quidem velut homini
praeter illud ex Deo Verbum specialiter intellecto applicat, quasdam
vero, velut Deo decibiles, soli ex Deo patre Verbo, anathema sit. }


&

そしてこれもまた、エフェソス公会議の承認のもと、以下の言葉によって断罪さ
 れた。
「もしだれかが、二つのペルソナないし自存性に、福音や使徒の書物の中で述べ
 る言葉を、キリストについて聖人たちによって語られた言葉であれ、キリスト
 自身によって自身について語られた言葉であれ、ある言葉は、神からの彼の言
 葉を除いて人間にとくに理解されたものとして適用し、別の言葉は、神によっ
 て語られうるものとして、父なる神からの言葉だけに適用するならば、それ
 は異端である」。

\\


Sic
igitur patet esse haeresim ab olim damnatam dicere quod in Christo sunt
duae hypostases vel duo supposita, sive quod unio non sit facta in
hypostasi vel supposito. Unde in eadem synodo legitur, {\itshape si quis non
confitetur carni secundum subsistentiam unitum ex Deo patre Verbum,
unumque esse Christum cum sua carne, eundem scilicet Deum et hominem,
anathema sit}.

&

ゆえに、以上のようにして、キリストにおいて二つのヒュポスタシスないし二つ
 の個体があるとか、合一がヒュポスタシスないし個体においてなされなかった
 と述べることは、かつて断罪された異端である。このことから、同じ信教の中
 で、以下のように書かれている。「もしだれかが、肉に、自存性に即して、父
 なる神からの言葉が合一し、自らの肉を伴うキリストが、一つの同一のもの、
 すなわち神でありかつ人間であると告白しないならば、異端である」。


\\





{\scshape Ad primum ergo dicendum} quod, sicut accidentalis differentia facit
alterum, ita differentia essentialis facit aliud. Manifestum est autem
quod alteritas, quae provenit ex differentia accidentali, potest ad
eandem hypostasim vel suppositum in rebus creatis pertinere, eo quod
idem numero potest diversis accidentibus subesse, non autem contingit in
rebus creatis quod idem numero subsistere possit diversis essentiis vel
naturis. 


&

第一異論に対しては、それゆえ、以下のように言われるべきである。
附帯的な差異が「別様のもの」を作るように、本質的差異は「別のもの」を作る。
 ところで、附帯的差異に基づいて到来する「別様さ」が、被造物において、同
 一のヒュポスタシスないし個体に属することは可能である。なぜなら、数的に
 同一なものが、さまざまな附帯性のもとにあることが可能だから。しかし、被
 造物において、数的に同一のものが、さまざまな本質ないし本性のもとにあるこ
 とは不可能である。

\\





Unde sicut quod dicitur alterum et alterum in creaturis, non
significat diversitatem suppositi, sed solum diversitatem formarum
accidentalium; ita quod Christus dicitur aliud et aliud, non importat
diversitatem suppositi sive hypostasis, sed diversitatem naturarum. 

&

したがって、被造物において、別様であることが、個体の多様さを意味せず、た
 だ、附帯形相の多様さを意味するだけであるように、キリストが別々のものであると
言われることは、個体ないしヒュポスタシスの多様性ではなく、諸本性の多様さを
 意味する。


\\


Unde
Gregorius Nazianzenus dicit, in epistola {\itshape ad Chelidonium}, {\itshape aliud et aliud
sunt ea ex quibus salvator est, non alius autem et alius. Dico vero
aliud et aliud e contrario quam in Trinitate habet. Ibi enim alius et
alius dicimus, ut non subsistentias confundamus, non aliud autem et
aliud}.

&


このことから、ナジアンゾスのグレゴリウスは『ケリドニウス宛書簡』で次のよ
 うに述べている。「別々の物とは、救済者がそれからあるところのものであっ
 て、別々の者ではない。私が別々の物というのは、逆に、三位一体において
 もつものである。つまりそこでは、私は別々の者と言うが、自存体
 と混同しないように、別々の物とは言わない」。

\\



{\scshape Ad secundum dicendum} quod hypostasis significat substantiam particularem
non quocumque modo, sed prout est in suo complemento. Secundum vero quod
venit in unionem alicuius magis completi, non dicitur hypostasis, sicut
manus aut pes. 

&

第二異論に対しては、以下のように言われるべきである。
ヒュポスタシスは、個的実体を意味するが、どんなしかたによっても、というわ
 けではなく、自らの完成においてあるものとして意味する。逆に、手や足のよ
 うに、より完全な何かとの合一に入るかぎりでは、ヒュポスタシスとは言われない。


\\

Et similiter humana natura in Christo, quamvis sit
substantia particularis, quia tamen venit in unionem cuiusdam completi,
scilicet totius Christi prout est Deus et homo, non potest dici
hypostasis vel suppositum, sed illud completum ad quod concurrit,
dicitur esse hypostasis vel suppositum.

&

同様に、キリストにおける人間本性は、個的実体ではあるが、ある完全なもの、す
 なわち、神であり人間であるキリスト全体との合一に入るので、ヒュポスタシ
 スないし個体とは言われえない。むしろ、それへと集まるところのかの完全な
 ものが、ヒュポスタシスないし個体と言われる。

\\



{\scshape Ad tertium dicendum} quod etiam in rebus creatis res aliqua singularis
non ponitur in genere vel specie ratione eius quod pertinet ad eius
individuationem, sed ratione naturae, quae secundum formam determinatur,
cum individuatio magis sit secundum materiam in rebus compositis. Sic
igitur dicendum est quod Christus est in specie humana ratione naturae
assumptae, non ratione ipsius hypostasis.

&


第三異論に対しては、以下のように言われるべきである。
被造物においても、ある個的な事物が類や種におかれるのは、その個体性に属す
 るものによってではなく、本性によってである。本性は形相によって限定され、
 他方、個体性は、むしろ、複合的事物においては質料に即してある。したがっ
 てこの意味で、キリストは、そのヒュポスタシスのためにではなく、受容され
 た本
 性のために、人間の種の中にあると言われるべきである。


\end{longtable}
\newpage







\rhead{a.~4}
\begin{center}
 {\Large {\bf ARTICULUS QUARTUS}}\\
 {\large UTRUM PERSONA CHRISTI SIT COMPOSITA}\\
 {\footnotesize III {\itshape Sent.}, d.6, q.2, a.3.}\\
 {\Large 第四項\\キリストのペルソナは複合されているか}
\end{center}

\begin{longtable}{p{21em}p{21em}}

{\Huge A}{\scshape d quartum sic proceditur}. Videtur quod persona Christi non sit
composita. Persona enim Christi non est aliud quam persona vel
hypostasis Verbi, ut ex dictis patet. Sed in Verbo non est aliud persona
et natura, ut patet ex his quae dicta sunt in prima parte. Cum ergo
natura Verbi sit simplex, ut in primo ostensum est, impossibile est quod
persona Christi sit composita.


&

第四項の問題へ、議論は以下のように進められる。
キリストのペルソナは複合されてないと思われる。理由は以下の通り。
キリストのペルソナは、上述のことから明らかなとおり\footnote{a.2.}、言葉のペルソナないし
 ヒュポスタシスに他ならない。しかし、第一部で言われたことから明らかなと
 おり\footnote{STI, q.39, a.1.}、言葉において、ペルソナと本性は別で
 ない。ゆえに、第一部で明らかにされたとおり\footnote{STI, q.3, a.7.}、言葉の本性は単純なので、キリストのペルソナが複合されていることは不可能である。

\\



2. {\scshape Praeterea}, omnis compositio videtur esse ex partibus. Sed divina natura
non potest habere rationem partis, quia omnis pars habet rationem
imperfecti. Ergo impossibile est quod persona Christi sit composita ex
duabus naturis.


&

さらに、全ての複合は部分からであるように思われる。しかし、神の本性は部分
 という性格を持つことがありえない。なぜなら、全て部分は不完全なものとい
 う性格を持つからである。ゆえに、キリストのペルソナが二つの本性から複合
 されていることは不可能である。

\\



3. {\scshape Praeterea}, quod componitur ex aliquibus, videtur esse homogeneum eis,
sicut ex corporibus non componitur nisi corpus. Si igitur ex duabus
naturis aliquid sit in Christo compositum, consequens erit quod illud
non erit persona, sed natura. Et sic in Christo erit facta unio in
natura. Quod est contra praedicta.


&


さらに、たとえば、物体から複合されたものは物体であるように、あるものども
 から複合されたものは、それらと類を同じくすると思われる。ゆえに、もし、
 二つの本性からキリストにおいて何かが複合されるならば、それはペルソナで
 はなく本性であるだろう。そして、キリストにおいて、本性における合一がな
 されたであろう。これは上述のこと\footnote{a.1.}に反する。

\\



{\scshape Sed contra est} quod Damascenus dicit, III libro, {\itshape in Domino Iesu Christo
duas naturas cognoscimus, unam autem hypostasim, ex utrisque compositam}.


&

しかし反対に、ダマスケヌスは第三巻で「主イエス・キリストにおいて、私たち
 は、二つの本性と、それら両方から複合された一つのヒュポスタシスを知る」
 と述べている。

\\



{\scshape Respondeo dicendum} quod persona sive hypostasis Christi dupliciter
considerari potest. Uno modo, secundum id quod est in se. Et sic est
omnino simplex, sicut et natura Verbi. Alio modo, secundum rationem
personae vel hypostasis, ad quam pertinet subsistere in aliqua
natura. Et secundum hoc, persona Christi subsistit in duabus
naturis. Unde, licet sit ibi unum subsistens, est tamen ibi alia et alia
ratio subsistendi. Et sic dicitur persona composita, inquantum unum
duobus subsistit.


&

解答する。以下のように言われるべきである。
キリストのペルソナないしヒュポスタシスは、二通りに考えられうる。
一つは、それ自体においてであり、その場合、それは言葉の本性と同様、あらゆ
 る点で単純である。もう一つには、ペルソナないしヒュポスタシスの性格に即
 してであり、この性格に、ある本性において自存することが属する。
そしてこの限りで、キリストのペルソナは、二つの本性において自存する。した
 がって、そこには一つの自存するものがあるが、しかし、別々の自存の性格が
 ある。そして、この意味で、一つのものが二つのものによって自存する限りで、
 ペルソナが複合さていると言われる。

\\



Et per hoc patet responsio ad primum.


&

これによって、第一異論への解答は明らかである。

\\



{\scshape Ad secundum dicendum} quod illa compositio personae ex naturis non
dicitur esse ratione partium, sed potius ratione numeri, sicut omne
illud in quo duo conveniunt, potest dici ex eis compositum.


&

第二異論に対しては、以下のように言われるべきである。
本性からのかのペルソナの複合は、部分の性格によってあるとは言われず、むし
 ろ、数の性格によってある。ちょうど、二つのものが一致するものは、それら
 から複合されていると言われうるように。


\\



{\scshape Ad tertium dicendum} quod non in omni compositione hoc verificatur quod
illud quod componitur sit homogeneum componentibus, sed solum in
partibus continui; nam continuum non componitur nisi ex
continuis. Animal vero componitur ex anima et corpore, quorum neutrum
est animal.


&


第三異論に対しては、以下のように言われるべきである。
複合されたものは複合の要素と類を同じくするということは、全ての複合におい
 て真であるわけではなく、ただ、連続体の部分にだけあてはまる。
連続体は、連続体によってしか複合されないからである。
これに対して、動物は、魂と肉体から複合されるが、このどちらも動物ではない。

\\

\end{longtable}
\newpage


\rhead{a.~5}
\begin{center}
 {\Large {\bf ARTICULUS QUINTUS}}\\
 {\large UTRUM IN CHRISTO FUERIT UNIO ANIMAE ET CORPUS}\\
 {\footnotesize III {\itshape Sent.}, d.2, q.1, a.3, qu$^a$ 3; d.6, q.3,
 a.1; IV {\itshape SCG}, cap.37; {\itshape Compend.~Theol.}, cap.209.}\\
 {\Large 第五項\\キリストにおいて魂と肉体の合一があったか}
\end{center}

\begin{longtable}{p{21em}p{21em}}


{\Huge A}{\scshape d quintum sic proceditur}. Videtur quod in Christo non fuerit unio
animae et corporis. Ex unione enim animae et corporis in nobis causatur
persona vel hypostasis hominis. Si ergo anima et corpus fuerunt in
Christo unita, sequitur quod fuerit ex unione eorum aliqua hypostasis
constituta. Non autem hypostasis Verbi Dei, quae est aeterna. Ergo in
Christo erit aliqua persona vel hypostasis praeter hypostasim
Verbi. Quod est contra praedicta.


&

第五項の問題へ、議論は以下のように進められる。
キリストにおいて、魂と肉体の合一はなかったと思われる。
理由は以下の通り。私たちにおいて、魂と肉体の合一から、人間のペルソナないしヒュ
 ポスタシスが結果として生じる。ゆえに、もし、キリストにおいて魂と肉体が
 合一されていたならば、それらの合一から何らかのヒュポスタシスが構成され
 たであろう。しかし、神の言葉のヒュポスタシスは永遠なので、そのように構
 成されることがない。ゆうえに、キリストの中に、言葉のヒュポスタシス意外
 に、別のペルソナないしヒュポスタシスがあることになる。これは、先述のこ
 と\footnote{a.2.}に反する。


\\



2. {\scshape Praeterea}, ex unione animae et corporis constituitur natura humanae
speciei. Damascenus autem dicit, in III libro, quod {\itshape in Domino nostro
Iesu Christo non est communem speciem accipere}. Ergo in eo non est facta
compositio animae et corporis.


&


さらに、魂と肉体の合一から、人間の種の本性が構成される。
しかし、ダマスケヌスは第三巻で「私たちの主であるイエス・キリストにおいて、
 共通の種を理解することはできない」と言う。ゆえに、その中に、魂と肉体の
 複合がなされることはなかった。

\\



3. {\scshape Praeterea}, anima non coniungitur corpori nisi ut vivificet ipsum. Sed
corpus Christi poterat vivificari ab ipso Verbo Dei, quod est fons et
principium vitae. Ergo in Christo non fuit unio animae et corporis.


&


さらに、魂は身体に、肉体を生かすことによってのみ結び付けられる。しかし、
 キリストの肉体は、生命の泉であり根源である神の言葉自身によって生かされ
 ることが可能だった。ゆえに、キリストにおいて魂と肉体の合一はなかった。


\\



{\scshape Sed contra est} quod corpus non dicitur animatum nisi ex unione
animae. Sed corpus Christi dicitur animatum, secundum illud quod
Ecclesia cantat, {\itshape animatum corpus assumens, de virgine nasci dignatus
est}. Ergo in Christo fuit unio animae et corporis.


&

しかし反対に、肉体は、魂の複合によらなければ、生かされたものと言われない。
 しかし、かの「生かされた肉体を受容し、処女から生まれるに値する」と教会が
 歌うのによれば、キリストの肉体は生かされたものと言われている。ゆえに、
 キリストにおいて魂と肉体の合一があった。

\\



{\scshape Respondeo dicendum} quod Christus dicitur homo univoce cum hominibus
aliis, utpote eiusdem speciei existens, secundum illud apostoli,
Philipp. II, in similitudinem hominum factus. Pertinet autem ad rationem
speciei humanae quod anima corpori uniatur, non enim forma constituit
speciem nisi per hoc quod sit actus materiae; et hoc est ad quod
generatio terminatur, per quam natura speciem intendit. Unde necesse est
dicere quod in Christo fuerit anima unita corpori, et contrarium est
haereticum, utpote derogans veritati humanitatis Christi.


&

解答する。以下のように言われるべきである。
使徒の、かの『フィリピの信徒への手紙』「人間の類似へと作られた」
 \footnote{「かえって自分を無にして、僕の身分になり、人間と同じ者になら
 れました。人間の姿で現れ、」(2:7)}によれば、キリストは、同じ種を持って
 存在するものとして、他の人間たちと一義的に「人間」と言われる。
しかし、人間の種の性格には、魂が肉体に合一されることが属する。なぜなら、
 形相が種を構成するのは、それが質料の現実態であることによる以外にないか
 らである。生成はそこに終局し、自然は生成を通して種を意図する。
したがって、キリストにおいて魂と肉体の合一があることは必然であり、その反
 対は異端である。つまり、それはキリストの人間性の真理を低めるものである。


\\



{\scshape Ad primum ergo dicendum} quod ex hac ratione moti fuerunt illi qui
negaverunt unionem animae et corporis in Christo, ne per hoc scilicet
cogerentur personam novam aut hypostasim in Christo inducere; quia
videbant quod in puris hominibus ex unione animae ad corpus constituitur
persona. 


&


第一異論に対しては、それゆえ、以下のように言われるべきである。
キリストにおいて魂と肉体の合一を否定した人々は、純粋な人間において、魂の
 肉体への合一からペルソナが構成されるのを見ていたので、それに動かされ、
 その合一から、キリストにおいて新たなペルソナないしヒュポスタシスを導出
 しなければならないのを避けるために、そうしたと思われる。



\\

Sed hoc ideo in puris hominibus accidit quia anima et corpus
sic in eis coniunguntur ut per se existant. Sed in Christo uniuntur ad
invicem ut adiuncta alteri principaliori quod subsistit in natura ex eis
composita. 


&

しかし、これが純粋な人間において生じるのは、魂と肉体が、それら自体によっ
 て存在するようなかたちで結び付けられるからである。しかし、キリストにお
 いては、それらの複合から、本性において自存する、より根源的な他のものに
 沿うかたちで、相互に結び付けられる。

\\

Et propter hoc ex unione animae et corporis in Christo non
constituitur nova hypostasis seu persona, sed advenit ipsum coniunctum
personae seu hypostasi praeexistenti. 


&

このため、キリストにおける魂と肉体の合一から、新しいペルソナないしヒュポ
 スタシスは構成されず、むしろ、結びついたもの自体が、その前に存在するペ
 ルソナないしヒュポスタシスに到来する。

\\

Nec propter hoc sequitur quod sit
minoris efficaciae unio animae et corporis in Christo quam in
nobis. Quia ipsa coniunctio ad nobilius non adimit virtutem aut
dignitatem, sed auget, sicut anima sensitiva in animalibus constituit
speciem, quia consideratur ut ultima forma; non autem in hominibus,
quamvis in eis sit nobilior et virtuosior; et hoc per adiunctionem
ulterioris et nobilioris perfectionis animae rationalis, ut etiam supra
dictum est.


&

また、このことから、私たちにおいてよりもキリストにおいて、魂と肉体の合一
 の有効性がより小さいということが帰結するわけでもない。
なぜなら、より高貴なものへの合一は、力や偉大さを取り去らず、むしろ増やす
 からである。たとえば、動物における感覚的魂は、最終の形相と考えられるの
 で、種を構成するが、人間においてはそうでない。しかし、より高貴で力がある。
それは、理性的魂という、より最終的で高貴な完全性に結
 びつけられることによる。これもまた、前に\footnote{a.2, ad 2.}述べられたことである。

\\



{\scshape Ad secundum dicendum} quod verbum Damasceni potest intelligi
dupliciter. Uno modo, ut referatur ad humanam naturam. Quae quidem non
habet rationem communis speciei secundum quod est in uno solo individuo,
sed secundum quod est abstracta ab omni individuo, prout in nuda
contemplatione consideratur; vel secundum quod est in omnibus
individuis. 


&

第二異論に対しては、以下のように言われるべきである。
ダマスケヌスの言葉は二通りに理解できる。一つには、人間の本性に言及するも
 のとしてである。実際、その本性は、ただ一人の個人の中にあるかぎりでは、
 共通の種という性格を持たず、むしろ、まったくの観想において考察されるも
 のとして、全ての個人から抽象される限りで、あるいは、全ての個人の中にあ
 る限りで、そのような性格を持つ。


\\


Filius autem Dei non assumpsit humanam naturam prout est in
sola consideratione intellectus, quia sic non assumpsisset ipsam rem
humanae naturae. Nisi forte diceretur quod humana natura esset quaedam
idea separata, sicut Platonici posuerunt hominem sine materia. Sed tunc
filius Dei non assumpsisset carnem, contra id quod dicitur Luc. ult.,
spiritus carnem et ossa non habet, sicut me videtis habere. 



&


しかし神の息子は、たんなる知性の考察においてあるものとしての人間本性を受
 容したのではない。なぜなら、もしそうだとすると、人間本性の実物それ自体を
 (ipsam rem)受容しなかったであろうから。ただし、プラトン派の人々が、人間を
 質料無しに考えたように、人間本性が、ある種の離在するイデアであると言う
 のなら別である。しかし、その場合には、神の息子は肉を受容しなかったであろ
 うが、これに反して、『ルカによる福音書』最終章で「あな
 た方がわたしが持っているのを見るようには、霊は肉と骨をもたない」
 \footnote{「わたしの手や足を見なさい。まさしくわたしだ。触ってよく見な
 さい。亡霊には肉も骨もないが、あなたがたに見えるとおり、わたしにはそれ
 がある。」 (24:39)}と言われている。

\\


Similiter
etiam non potest dici quod filius Dei assumpsit humanam naturam prout
est in omnibus individuis eiusdem speciei, quia sic omnes homines
assumpsisset. Relinquitur ergo, ut Damascenus postea dicit in eodem
libro, quod assumpserit naturam humanam {\itshape in atomo}, idest in individuo,
{\itshape non quidem in alio individuo, quod sit suppositum vel hypostasis illius
naturae, quam in persona Filii Dei}. 



&

同様に、神の息子が、同一の種に属する全ての個人の中にあるものとしての人間
 本性を受容したと言われることもできない。なぜなら、その場合、全ての人間を
 乗したことになるから。ゆえに、ダマスケヌスが同書で、その後に言うように、
 「アトムにおいて」、すなわち、個人において、ただし、「その本性を持つ個
 体ないしヒュポスタシスである他の個人においてではなく、神の息子のペルソ
 ナにおいて」人間本性を受容した。

\\


Alio modo potest intelligi dictum
Damasceni ut non referatur ad naturam humanam, quasi ex unione animae et
corporis non resultet una communis natura, quae est humana, sed est
referendum ad unionem duarum naturarum, divinae scilicet et humanae, ex
quibus non componitur aliquid tertium, quod sit quaedam natura communis;
quia sic illud esset natum praedicari de pluribus. 


&

また、ダマスケヌスが言ったことは別の意味で理解される。つまり、魂と肉体
 の合一から、一つの共通の本性、すなわち人間本性が結果として生じるのでは
 なく、二つの本性、すなわち神と人間の本性の合一に言及されるべきものとし
 てである。この二つの本性から、何か第三の、何らかの共通の本性であるもの
 が複合されるのではない。なぜなら、もうしそうなら、その第三のものが、複
 数のものについて述語されたであろうから。


\\


Et hoc ibi
intendit. Unde subdit, {\itshape neque enim generatus est, neque unquam
generabitur alius Christus, ex deitate et humanitate, in deitate et
humanitate, Deus perfectus, idem et homo perfectus}.


&

そして、ここで、彼はこれを意図している。したがって彼はその下で、
以下のように述べている。「神性と人性から、神性と人性において、生成したの
 でなく、他のキリストが将来生まれることもなく、完全な神であり、同じもの
 が完全な人間でもある」。





\\



{\scshape Ad tertium dicendum} quod duplex est principium vitae corporalis. Unum
quidem effectivum. Et hoc modo Verbum Dei est principium omnis
vitae. Alio modo est aliquid principium vitae formaliter. Cum enim
{\itshape vivere viventibus sit esse}, ut dicit philosophus, in II {\itshape De Anima}; sicut
unumquodque formaliter est per suam formam, ita corpus vivit per
animam. Et hoc modo non potuit corpus vivere per Verbum, quod non potest
esse corporis forma.


&

第三異論に対しては、以下のように言われるべきである。
物体的なものの生命の根源には二通りある。一つは作出的根源であり、このしか
 たで、神の言葉は
全ての生命の根源である。もう一つは、生命の形相的な根源である。哲学者が
 『デ・アニマ』第2巻で言うように「生きることは、生きるものにとって、存在
 することである」ので、つまり各々のものは、自らの形相によって存在するの
 で、肉体は魂によって存在する。この意味では、肉体が言葉によって生きるこ
 とは不可能であった。言葉が肉体の形相であることは不可能だからである。


\end{longtable}
\newpage


\rhead{a.~6}
\begin{center}
 {\Large {\bf ARTICULUS SEXTUS}}\\
 {\large UTRUM HUMANA NATURA FUERIT UNITA VERBO DEI ACCIDENTALITER}\\
 {\footnotesize III {\itshape Sent.}, d.6, q.3, a.2; IV {\itshape SCG},
 cap.34, 37, 41, 49; {\itshape De Unione Verbi}, a.1; {\itshape
 Compend.~Theol.}, cap.203, 209, 210; {\itshape Ad Rom.}, cap.1, lect.2;
 {\itshape Ad Philipp.}, cap.2, lect.2.}\\
 {\Large 第六項\\人間本性は神の言葉に附帯的に合一したか}
\end{center}

\begin{longtable}{p{21em}p{21em}}
{\scshape Ad sextum sic proceditur}. Videtur quod humana natura fuerit unita Verbo
Dei accidentaliter. Dicit enim apostolus, {\itshape Philipp}.~{\scshape ii}, de filio Dei,
quod {\itshape habitu inventus est ut homo}. Sed habitus accidentaliter advenit ei
cuius est, sive accipiatur habitus prout est unum de decem generibus;
sive prout est species qualitatis. Ergo humana natura accidentaliter
unita est filio Dei.

&

第六項の問題へ、議論は以下のように進められる。人間本性は、神の言葉に附帯
 的に合一したのだと思われる。理由は以下の通り。
使徒は『フィリピの信徒への手紙』第2章で、神の息子について「外見において
 人間として見出された」\footnote{「かえって自分を無にして、僕の身分になり、人間と同じ者になら
 れました。人間の姿で現れ、」(2:7)}と言われている。しかし、「外見・習態」は、
 それが十の類の一つとして理解されても、性質の種として理解されても、基体
 に附帯的に到来する。ゆえに、人間本性は、神の息子に附帯的に合一した。

\\



2. {\scshape Praeterea}, omne quod advenit alicui post esse completum, advenit ei
accidentaliter, hoc enim dicimus accidens quod potest alicui et adesse
et abesse praeter subiecti corruptionem. Sed natura humana advenit ex
tempore filio Dei habenti esse perfectum ab aeterno. Ergo advenit ei
accidentaliter.

&


さらに、全て何かに、完全な存在の後に到来するものは、附帯的に到来する。な
 ぜなら私たちは、基体が消滅することなく、なにかに付着したり離れたりしうるものを
 附帯的なものと言うからである。しかし、人間本性は、永遠から完全な存在を
 持つ神の息子に、時間的に到来する。ゆえに、附帯的に到来する。

\\



3. {\scshape Praeterea}, quidquid non pertinet ad naturam seu essentiam alicuius rei,
est accidens eius, quia omne quod est vel est substantia, vel est
accidens. Sed humana natura non pertinet ad essentiam vel naturam filii
Dei divinam, quia non est facta unio in natura, ut supra dictum
est. Ergo oportet quod natura humana accidentaliter filio Dei advenerit.

&

さらに、何であれ或るものの本性や本質に属さないものは、それの附帯性である。
 なぜなら、全てのものは、実体か附帯性かのどちらかであるから。しかし、人
 間本性は、神の息子の神的本質ないし本性に属さない。なぜなら、前に述べら
 れたとおり、本性において合一したのではないから。ゆえに、人間本性は、附
 帯的に、神の息子に到来したのでなければならない。

\\



4. {\scshape Praeterea}, instrumentum accidentaliter advenit. Sed natura humana in
Christo fuit divinitatis instrumentum, dicit enim Damascenus, in III
libro, quod {\itshape caro Christi instrumentum divinitatis existit}. Ergo videtur
quod humana natura fuerit filio Dei unita accidentaliter.

&

さらに、道具は附帯的に到来する。しかし、人間本性は、キリストにおいて、神
 性の道具である。なぜなら、ダマスケヌスが第三巻で「キリストの肉は神性の
 道具である」と述べているから。ゆえに、人間本性は、附帯的に、神の息子に
 合一されたと思われる。


\\



{\scshape Sed contra est} quod illud quod accidentaliter praedicatur, non praedicat
aliquid, sed {\itshape quantum vel quale vel aliquo modo se habens}. Si igitur
humana natura accidentaliter adveniret, cum dicitur Christus esse homo,
non praedicaretur aliquid, sed quale aut quantum aut aliquo modo se
habens. Quod est contra decretalem Alexandri Papae dicentis, {\itshape cum
Christus sit perfectus Deus et perfectus homo, qua temeritate audent
quidam dicere quod Christus, secundum quod est homo, non est aliquid?}

&


しかし反対に、附帯的に述語されるものは、「或るもの」ではなく、「どれだけ、
 どのように、どのような関係で」という内容を表す。ゆえに、もし人間本性が
 附帯的に到来したならば、「キリストは人間である」と言われるとき、「或るもの」ではなく、「どれだけ、
 どのように、どのような関係で」という内容を表したであろう。これは、以下
 のように言う教皇アレクサンドロスの教皇教令に反する。「キリストは完全な
 神でありかつ完全な人であるので、キリストが、人間であるかぎりで、何か或
 るものでないと、誰が不注意にもあえて言うだろうか」。

\\



{\scshape Respondeo dicendum} quod, ad huius quaestionis evidentiam, sciendum est
quod circa mysterium unionis duarum naturarum in Christo, duplex
haeresis insurrexit. 

&

解答する。以下のように言われるべきである。
この問題を明らかにするためには、キリストにおける二つの本性の合一の秘蹟を
 巡って、二つの異端が生じたことが知られるべきである。

\\

Una quidem confundentium naturas, sicut Eutychetis
et Dioscori, qui posuerunt quod ex duabus naturis est constituta una
natura; ita quod confitentur Christum esse {\itshape ex duabus naturis}, quasi ante
unionem distinctis; non autem {\itshape in duabus naturis}, quasi post unionem
naturarum distinctione cessante. 

&

一つは、二つの本性を混合する人たちの異端であり、たとえば、エウテュケスや
 ディオスコルスのように、二つの本性から一つの本性が構成されると考えて、
 「二つの本性から」すなわち、合一の前には区別された本性から、
 「二つの本性において」すなわち、本性の合一の後はその区別が消滅するとい
 うかたちで、キリストがあると告白する。

\\

Alia vero fuit haeresis Nestorii et
Theodori Mopsuesteni separantium personas. Posuerunt enim aliam esse
personam filii Dei, et filii hominis. Quas dicebant sibi invicem esse
unitas, primo quidem, {\itshape secundum inhabitationem}, inquantum scilicet Verbum
Dei habitavit in illo homine sicut in templo. Secundo, {\itshape secundum unitatem
affectus}, inquantum scilicet voluntas illius hominis est semper
conformis voluntati Dei. 


&

他方、別の異端は、ネストリウスとモプスエステヌスのテオドルスの、ペルソナ
 を切り離す人々の異端である。
彼らは、神の息子のペルソナと、人間の息子のペルソナは別々だと主張した。
彼らが言うには、この二つのペルソナは、相互に一性であり、それは第一に、
 「住まうことにおいて」つまり、神の言葉が、彼の人間の中に、ちょうど神殿
 の中に住むように、住む限りでであり、第二に、「愛情の一性において」すな
 わち、彼の人間の意志が常に、神の意志に一致する限りにおいてである。

\\

Tertio modo, {\itshape secundum operationem}, prout
scilicet dicebant hominem illum esse Dei Verbi instrumentum. 



&


第三に、「働きにおいて」すなわち、彼の人間は、神の言葉の道具であると彼ら
 は言っていた。

\\

Quarto,
{\itshape secundum dignitatem honoris}, prout omnis honor qui exhibetur filio Dei,
exhibetur filio hominis, propter coniunctionem ad filium Dei. 

&

第四に、「名誉の偉大さにおいて」すなわち、神の息子によって示される全ての
 名誉が、神の息子への結合のために、彼の人間の息子によって示される限りに
 おいて。


\\

Quinto,
{\itshape secundum aequivocationem}, idest secundum communicationem nominum, prout
scilicet dicimus illum hominem esse Deum et filium Dei. Manifestum est
autem omnes istos modos accidentalem unionem importare. 

&

第五に、「両義性において」すなわち、名称の共有に即して、つまり、私たちは、
 彼の人間が神であり、神の息子であると言うかぎりで。しかし、これら全ての
 しかたが、附帯的な合一を意味することは明らかである。


\\

Quidam autem
posteriores magistri, putantes se has haereses declinare, in eas per
ignorantiam inciderunt. Quidam enim eorum concesserunt unam Christi
personam, sed posuerunt duas hypostases, sive duo supposita; dicentes
hominem quendam, compositum ex anima et corpore, a principio suae
conceptionis esse assumptum a Dei Verbo. 
Et haec est prima opinio quam
Magister ponit in sexta distinctione tertii libri sententiarum.

&

しかし後世のある教師たちは、これらの異端を避けようとして、無知によって、
 それらへと落ち込んでいった。たとえば彼らのある人々は、キリストの一つの
 ペルソナを認めたが、魂と肉体から複合されたある人間が、その懐胎の最初か
 ら、神の言葉によって受容されたと述べて、二つのヒュポスタシスないし二つの個体を主張した。
そして、これが、『命題集』第三巻の第六区分で、教師(ロンバルドゥス)が論じ
 ている最初の意見である。


\\

 Alii
vero, volentes servare unitatem personae, posuerunt Christi animam non
esse corpori unitam, sed haec duo, separata ab invicem, esse unita Verbo
accidentaliter, ut sic non cresceret numerus personarum. Et haec est
tertia opinio quam Magister ibidem ponit. 

&

他方、別の人たちは、ペルソナの一性を守ろうとして、キリストの魂が肉体に合
 一されず、この二つが、ペルソナの数が増えないように、相互に分離したまま、
 附帯的に、言葉において合一していると論じた。そしてこれは、教師が第三の
 意見として同箇所で論じているものである。

\\

Utraque autem harum opinionum
incidit in haeresim Nestorii. Prima quidem, quia idem est ponere duas
hypostases vel duo supposita in Christo, quod ponere duas personas, ut
supra dictum est. 

&

しかしこのどちらの意見も、ネストリウスの異端に陥る。第一の意見は、二つの
 ヒュポスタシスないし二つの個体をキリストにおいて措定することは、すでに
 述べられたとおり\footnote{a.3.}、二つの
 ペルソナを措定することと同じだからである。

\\

Et si fiat vis in nomine personae, considerandum est
quod etiam Nestorius utebatur unitate personae, propter unitatem
dignitatis et honoris. Unde et quinta synodus definit anathema eum qui
dicit {\itshape unam personam secundum dignitatem, honorem et adorationem, sicut
Theodorus et Nestorius insanientes conscripserunt}. 


&

もし、「ペルソナ」という名称に重点を置くならば、ネストリウスもまた、ペル
 ソナの一性を、偉大さや名誉の一性のために用いていたことが考察されるべき
 である。このことから、第五公会議で、「テオドルスとネストリウスが愚かに
 も書いたように、偉大さ、名誉、崇拝において一つのペルソナ」と語る彼を、異端と限定している。


\\

Alia vero opinio
incidit in errorem Nestorii quantum ad hoc, quod posuit unionem
accidentalem. Non enim differt dicere quod Verbum Dei unitum est homini
Christo secundum inhabitationem sicut in templo suo, sicut dicebat
Nestorius; et dicere quod unitum fuit Verbum homini secundum induitionem
sicut vestimento, sicut dicit tertia opinio. 
Quae etiam dicit peius
aliquid quam Nestorius, quod anima et corpus non sunt unita. 

&

他方、別の意見は、附帯的合一を論じた点で、ネストリウスの誤謬に陥った。つ
 まり、ネストリウスが言ったように、神の言葉が人間キリストに、ちょうど自
 らの神殿の中にあるように住まうことで合一したとか、そして、第三の意見が言うように、言葉は人
 間に、服を着るようにして合一した、と言うのと変わらない。この後者の意見
 は、ネストリウスよりも悪い。魂と身体が合一していないからである。


\\

Fides autem
Catholica, medium tenens inter praedictas positiones, neque dicit esse
unionem factam Dei et hominis secundum essentiam vel naturam; neque
etiam secundum accidens; sed medio modo, secundum subsistentiam seu
hypostasim. 

&

しかし、カトリックの信仰は、上述の諸々の立場の中間を保持し、神と人との合
 一は、本質ないし本性においてなされたと言わず、また、附帯的になされたと
 も言わない。それは、中間のあり方によって、自存体、あるいはヒュポスタシ
 スにおいてなされたと言う。

\\

Unde in quinta synodo legitur, {\itshape cum multis modis unitas
intelligatur, qui iniquitatem Apollinarii et Eutychetis sequuntur,
interemptionem eorum quae convenerunt colentes}, (idest, interimentes
utramque naturam), {\itshape unionem secundum confusionem dicunt; Theodori autem
et Nestorii sequaces, divisione gaudentes, affectualem unitatem
introducunt, sancta vero Dei Ecclesia, utriusque perfidiae impietatem
reiiciens unionem Dei Verbi ad carnem secundum compositionem confitetur,
quod est secundum subsistentiam}.

&

このことから、第五信教には次のように書かれている。
「一性は多くのしかたで理解されるので、アポリナリウスとエウテュケスの不合
 理に従う人々は、合致した者どもを破壊しようとして(すなわち、両方の本性を
 破壊しようとして)、混同における合一を語る。しかし、テオドルスとネストリ
 ウスに従う人々は、分割を喜び、愛情による合一を導入する。しかし、神聖な神の教
 会は、どちらの不信仰の不敬も排除して、神の言葉の肉への、自存体において
 ある複合を告白する」。


\\


Sic igitur patet quod secunda trium
opinionum quas Magister ponit, quae asserit unam hypostasim Dei et
hominis, non est dicenda opinio, sed sententia Catholicae
fidei. 
Similiter etiam prima opinio, quae ponit duas hypostases; et
tertia, quae ponit unionem accidentalem; non sunt dicendae opiniones,
sed haereses in Conciliis ab Ecclesia damnatae.


&

ゆえに、このようにして、教師が述べる三つの意見のうち、神と人間の一つのヒュ
 ポスタシスを主張する第二の意見が、意見と言われるべきでなく、むしろ、カ
 トリック信仰の教義であると言われれるべきである。同様に、二つのヒュポス
 タシスを述べる第一の意見と、附帯的合一を述べる第三の意見は、意見と言わ
 れるべきでなく、公会議において教会によって断罪された異端と言われるべき
 である。


\\



{\scshape Ad primum ergo dicendum} quod, sicut Damascenus dicit, in III libro, {\itshape non
necesse autem omnifariam et indefective assimilari exempla, quod enim in
omnibus simile, idem utique erit, et non exemplum. Et maxime in divinis,
impossibile enim simile exemplum invenire et in theologia}, idest in
deitate personarum, et {\itshape in dispensatione}, idest in mysterio
incarnationis. 



&

第一異論に対しては、それゆえ、以下のように言われるべきである。
ダマスケヌスは第3巻で以下のように述べる。「あらゆる点で、欠点なく、範型が真似
 られる必要はない。なぜなら、全て似ているものどもにおいて、もしそうであっ
 たら、それらは同一のものであり、範型でもないからである。そして、最大限
 に、神において、神学(すなわちペルソナの神性)と免除(すなわち受肉の秘蹟)において、類似と範型を見出すことは不可能である」。

\\



Humana igitur natura in Christo assimilatur habitui,
idest vestimento, non quidem quantum ad accidentalem unionem, sed
quantum ad hoc, quod Verbum videtur per humanam naturam, sicut homo per
vestimentum. Et etiam quantum ad hoc, quod vestimentum mutatur, quia
scilicet formatur secundum figuram eius qui induit ipsum, qui a sua
forma non mutatur propter vestimentum, et similiter humana natura
assumpta a Verbo Dei est meliorata, ipsum autem Verbum Dei non est
mutatum; ut exponit Augustinus, in libro octogintatrium quaestionum.

&

ゆえに、キリストにおける人間本性が外見に、すなわち衣服に似ているのは、附
 帯的合一に関してではなく、ちょうど人間が衣服を通して見られるように、言
 葉が、人間本性を通して見られるからである。さらにまた、衣服は、その服を
 着る人の姿に応じて形が決まるので、変化するが、服を着る人は、服のため
 に、その姿を離れて変化したりしない。ちょうどそのように、人間本性は、神
 の言葉に受容されることでより良くなるが、神の言葉自体は、変化しない。これはアウグ
 スティヌスが『八十三問題集』で説明しているとおりである。



\\



{\scshape Ad secundum dicendum} quod illud quod advenit post esse completum,
accidentaliter advenit, nisi trahatur in communionem illius esse
completi. Sicut in resurrectione corpus adveniet animae praeexistenti,
non tamen accidentaliter, quia ad idem esse assumetur, ut scilicet
corpus habeat esse vitale per animam. 


&

第二異論に対しては、以下のように言われるべきである。
完全な存在の後に到来するものは、その完全な存在との共有へと引き入れられる
 のでない場合に、附帯的に到来する。たとえば、復活において、肉体が、先在
 する魂に到来するが、それは附帯的に到来するわけではない。なぜなら、身体が、魂
 によって、生きた存在を持つように、同一の存在へと受容されるからである。



\\


Non est autem sic de albedine,
quia aliud est esse albi, et aliud esse hominis cui advenit
albedo. Verbum autem Dei ab aeterno esse completum habuit secundum
hypostasim sive personam, ex tempore autem advenit ei natura humana, non
quasi assumpta ad unum esse prout est naturae, sicut corpus assumitur ad
esse animae; sed ad unum esse prout est hypostasis vel personae. Et ideo
humana natura non unitur accidentaliter filio Dei.

&

しかし、白性についてはこのようでない。なぜなら、白いものの存在と、それに
 白性が到来する人間の存在は別々だからである。しかし神の言葉は、ヒュポス
 タシスないしペルソナにおいて、永遠から完全な存在をもっていたが、時間的
 に、それに人間本性が到来した。そしてそれは、肉体が魂の存在へ採られるよ
 うに、本性に属するものとしての一つの存在へと受容されるのではなく、ヒュポ
 スタシスないしペルソナに属する一つの存在へと採られた。ゆえに、人間本性
 は、神の息子に、附帯的に合一されるのではない。


\\



{\scshape Ad tertium dicendum} quod accidens dividitur contra
substantiam. Substantia autem, ut patet V {\itshape Metaphys}., dupliciter dicitur,
uno modo, essentia sive natura; alio modo, pro supposito sive
hypostasi. Unde sufficit ad hoc quod non sit unio accidentalis, quod sit
facta unio secundum hypostasim, licet non sit facta unio secundum
naturam.

&

第三異論に対しては、以下のように言われるべきである。
附帯性は実体に対して分かたれる。しかし実体は、『形而上学』第5巻で明らか
 なとおり、二つの意味で言われる。一つには、本質ないし本性としてであり、
 もう一つには、個体ないしヒュポスタシスとしてである。したがって、附帯的
 合一でないためには、本性における合一でなかったとしても、ヒュポスタシス
 における合一がなされたことで十分である。


\\



{\scshape Ad quartum dicendum} quod non omne quod assumitur ut instrumentum,
pertinet ad hypostasim assumentis, sicut patet de securi et gladio: nihil
tamen prohibet illud quod assumitur ad unitatem hypostasis, se habere ut
instrumentum, sicut corpus hominis vel membra eius. Nestorius igitur
posuit quod natura humana est assumpta a Verbo solum per modum
instrumenti, non autem ad unitatem hypostasis. 
Et ideo non concedebat
quod homo ille vere esset filius Dei, sed instrumentum eius.

&

第四異論に対しては、以下のように言われるべきである。
道具として受容されるものがすべて、受容するヒュポスタシスに属するわけではない。
 それは、ノコギリや剣の場合を見れば明らかである。しかし、ヒュポスタシス
 との合一へと受容されるものが、道具として関係することは、なんら差し支えな
 い。たとえば、人間の身体や、その四肢がそうであるように。ゆえに、ネスト
 リウスは、人間本性が言葉によって受容されたのは、ただ道具としてであって、
 ヒュポスタシスの合一へでないと主張した。それゆえ、彼は、かの人が真に神の
 息子であることを認めず、神の道具だと主張したのである。

\\



 Unde
Cyrillus dicit, in {\itshape Epistola ad Monachos Aegypti}, {\itshape hunc Emanuelem}, idest
Christum, {\itshape non tanquam instrumenti officio sumptum dicit Scriptura, sed
tanquam Deum vere humanatum}, idest hominem factum. Damascenus autem
posuit naturam humanam in Christo esse sicut instrumentum ad unitatem
hypostasis pertinens.

&

ゆえに、キュリッルスは『エジプトの君主への手紙』で以下のように述べる。
 「この無垢な者(つまりキリスト)は、道具の役割として採られたと聖書は述べ
 ず、むしろ、真に人間化した(つまり人となった)神であると述べる」。しかし
 ダマスケヌスは、キリストにおける人間本性は、あたかもヒュポスタシスの一性に道具
 として属するものであると論じた。





\end{longtable}
\newpage





\rhead{a.~7}
\begin{center}
 {\Large {\bf ARTICULUS SEPTIMUS}}\\
 {\large UTRUM UNIO DIVINAE ET HUMANAE NATURAE SIT ALIQUID CREATUM}\\
 {\footnotesize III {\itshape Sent.}, d.2, q.2, a.2, qu$^a$ 3, ad 2;
 d.5, q.1, a.1, qu$^a$ 1, d.7, q.2, a.1.}\\
 {\Large 第七項\\神と人間の本性の合一は創造されたものか}
\end{center}

\begin{longtable}{p{21em}p{21em}}

{\Huge A}{\scshape d septimum sic proceditur}. Videtur quod unio divinae et humanae naturae
non sit aliquid creatum. Nihil enim in Deo creatum potest esse, quia
quidquid est in Deo, Deus est. Sed unio est in Deo, quia ipse Deus est
humanae naturae unitus. Ergo videtur quod unio non sit aliquid creatum.


&

第七項の問題へ、議論は以下のように進められる。
神と人間本性の合一は、創造されたものではないと思われる。理由は以下の通り。
神の中に、何も創造されたものはない。なぜなら、神の中にあるものはなんであ
 れ神だからである。
しかし、合一は神の中にある。なぜなら、神自身が、人間本性に合一されている
 からである。ゆえに、合一は、創造されたものではないと思われる。


\\



2. {\scshape Praeterea}, finis est potissimum in unoquoque. Sed finis unionis est
divina hypostasis sive persona, ad quam terminata est unio. Ergo videtur
quod huiusmodi unio maxime debeat iudicari secundum conditionem divinae
hypostasis. Quae non est aliquid creatum. Ergo nec ipsa unio est aliquid
creatum.


&

さらに、目的は各々のものにおいて、もっとも力がある。しかし、合一の目的は、
 神のヒュポスタシスないしペルソナであり、合一はそれへと極まる。ゆえに、
 そのような合一は、最大限に、神のヒュポスタシスの条件において判断される
 べきである。そうすると、それは、創造されたものではない。ゆえに、合一もまた創造され
 た何かではない。

\\



3. {\scshape Praeterea}, propter quod unumquodque, et illud magis. Sed homo dicitur
esse creator propter unionem. Ergo multo magis ipsa unio non est aliquid
creatum, sed creator.


&

さらに、各々のものが、そのために何かであるところのそれは、よりその何かで
 ある\footnote{$a$が$b$のために$P$である場合、$b$は$a$よりも$P$である。
 e.g. 石が火のために熱い場合、火は石よりも熱い。}。しかし、人間は、合一のために、創造するものであると言われる。ゆえに、ま
 してや、合一自体は、創造されたものではなく、創造するものである。

\\



{\scshape Sed contra est}, quod incipit esse ex tempore, est creatum. Sed unio illa
non fuit ab aeterno, sed incoepit esse ex tempore. Ergo unio est aliquid
creatum.


&


しかし反対に、時間的な始まりを持つものは創造されたものである。しかし、か
 の合一は、永遠からではなく、時間的な始まりをもつ。ゆえに、合一は、創造
 された何かである。

\\



{\scshape Respondeo dicendum} quod unio de qua loquimur est relatio quaedam quae
consideratur inter divinam naturam et humanam, secundum quod conveniunt
in una persona filii Dei. 


&

解答する。以下のように言われるべきである。
私たちが語っている合一は、神の息子の一つのペルソナにおいて一致する限りで、
 神の本性と人間の本性の間に考えられる、一種の関係である。

\\

Sicut autem in prima parte dictum est, omnis
relatio quae consideratur inter Deum et creaturam, realiter quidem est
in creatura, per cuius mutationem talis relatio innascitur, non autem
est realiter in Deo, sed secundum rationem tantum, quia non nascitur
secundum mutationem Dei. 


&

しかし、第一部で言われたとおり、神と被造物の間に考えられる全ての関係は、
 被造物においては実在的で、それの変化によってそのような関係が生まれるが、
 神においては実在的でなく、たんに概念的なものである。なぜなら、神の変化
 によって生み出されるのではないからである。


\\


Sic igitur dicendum est quod haec unio de qua
loquimur, non est in Deo realiter, sed secundum rationem tantum in
humana autem natura, quae creatura quaedam est, est realiter. Et ideo
oportet dicere quod sit quoddam creatum.


&

それゆえ、この意味で、私たちが語っているこの合一は、神においては実在的で
 なく、たんに概念的であり、他方人間本性においては、何らかの被造物であり、
 実在的に存在すると言われるべきである。ゆえに、それは創造された何かだと
 言うべきである。


\\



{\scshape Ad primum ergo dicendum} quod haec unio non est in Deo realiter sed solum
secundum rationem tantum, dicitur enim Deus unitus creaturae ex hoc quod
creatura unita est ei, absque Dei mutatione.


&


第一異論に対しては、以下のように言われるべきである。
この合一は、神においては実在的でなく、たんに概念的であり、神が被造物に合
 一したと言われるのは、神がなんら変化することなく、
被造物が神に合一されたからであり、

\\



{\scshape Ad secundum dicendum} quod ratio relationis, sicut et motus dependet ex
fine vel termino, sed esse eius dependet ex subiecto. Et quia unio talis
non habet esse reale nisi in natura creata, ut dictum est, consequens
est quod habeat esse creatum.


&

第二異論に対しては、以下のように言われるべきである。
関係の性格は、運動と同様、目的ないし終極に依存するが、その存在は、基体に
 依存する。
そして、そのような合一は、上述の通り、被造の本性においてのみ実在的な存在
 をもつので、被造の存在をもつことが帰結する。

\\



{\scshape Ad tertium dicendum} quod homo dicitur et est Deus propter unionem
inquantum terminatur ad hypostasim divinam. Non tamen sequitur quod ipsa
unio sit creator vel Deus, quia quod aliquid dicatur creatum, hoc magis
respicit esse ipsius quam relationem.


&

第三異論に対しては、以下のように言われるべきである。
人間が、合一のために、神であると言われるのは、神のヒュポスタシスへと終
 極するからである。しかし、そのことから、合一自体が創造するものや神であ
 ることは帰結しない。なぜなら、あるものが被造物であるということは、関係
 よりもむしろ、それ自体の存在に関係するからである。

\\




\end{longtable}
\newpage






\rhead{a.~8}
\begin{center}
 {\Large {\bf ARTICULUS OCTAVUS}}\\
 {\large UTRUM IDEM SIT UNIO QUOD ASSUMPTIO}\\
 {\footnotesize III {\itshape Sent.}, d.5, q.1, a.1, qu$^a$3.}\\
 {\Large 第八項\\合一は受容と同一か}
\end{center}

\begin{longtable}{p{21em}p{21em}}




{\Huge A}{\scshape d octavum sic proceditur}. Videtur quod idem sit unio quod
assumptio. Relationes enim, sicut et motus, specificantur secundum
terminum. Sed idem est terminus assumptionis et unionis, scilicet divina
hypostasis. Ergo videtur quod non differant unio et assumptio.


&

第八項の問題へ、議論は以下のように進められる。
合一と受容は同じだと思われる。理由は以下の通り。
関係は、運動と同様、終極に応じて種に分けられる。しかし、受容と合一の終極
 は同じもの、すなわち、神のヒュポスタシスである。ゆえに、合一と受容は異
 ならないと思われる。


\\



2. {\scshape Praeterea}, in mysterio incarnationis idem videtur esse uniens et
assumens, unitum et assumptum. Sed unio et assumptio videntur sequi
actionem et passionem unientis et uniti, vel assumentis et
assumpti. Ergo videtur idem esse unio quod assumptio.


&


さらに、受肉の秘蹟において、合一するものと、受容するものは同じだと思われ
 る。しかし、合一と受容は、合一するものと合一されるもの、受容するものと
 受容されるものの能動と受動に伴う。ゆえに、合一は受容と同じである。

\\



3. {\scshape Praeterea}, Damascenus dicit, in III libro, {\scshape aliud est unio, aliud
incarnatio. Nam unio solam demonstrat copulationem, ad quid autem facta
est, non adhuc. Incarnatio autem et humanatio determinant ad quem sit
facta copulatio}. Sed similiter assumptio non determinat ad quem facta
sit copulatio. Ergo videtur idem esse unio et assumptio.


&


さらに、ダマスケヌスは第3巻で以下のように述べている。「合一と受肉は別で
 ある。すなわち、合一は、結びつきだけを示して、何へとそれがなされた
 かはいまだ示さない。しかし、受肉と人間化は、結びつきが何へとなされ
 かを限定する」。しかし、受容も同様に、何へ結びつきがなされたかを限定し
 ない。ゆえに、合一と受容は同一だと思われる。

\\



{\scshape Sed contra est} quod divina natura dicitur unita, non autem assumpta.


&

しかし反対に、神の本性は、合一されたと言われるが、受容されたとは言われな
 い。

\\



{\scshape Respondeo dicendum} quod, sicut dictum est, unio importat relationem
divinae naturae et humanae secundum quod conveniunt in una
persona. Omnis autem relatio quae incipit esse ex tempore, ex aliqua
mutatione causatur. Mutatio autem consistit in actione et passione. 


&

解答する。以下のように言われるべきである。
すでに述べられたとおり、合一は、神の本性と人間本性の関係を、一つのペルソ
 ナにおいて一致する限りにおいて、意味する。しかし、時間において存在を開
 始する全ての関係は、何らかの変化が原因となって生じる。ところで、変化は、
 能動と受動において成立する。


\\

Sic
igitur dicendum est quod prima et principalis differentia inter unionem
et assumptionem est quod unio importat ipsam relationem, assumptio autem
actionem secundum quam dicitur aliquis assumens, vel passionem secundum
quam dicitur aliquid assumptum. 


&

したがって、合一と受容の第一の主要な差異は、合一が関係自体を意味し、受容
 が、あるものが受容すると言われるかぎりで、その能動を意味し、あるものが
 受容されると言われる限りで、その受動を意味することの間にあると言われる
 べきである。


\\

Ex hac autem differentia accipitur
secundo alia differentia. Nam assumptio dicitur sicut in fieri, unio
autem sicut in facto esse. Et ideo uniens dicitur esse unitum, assumens
autem non dicitur esse assumptum. 


&

しかしこの違いから、第二に、もう一つの違いが理解される。すなわち、受容は、
 「なること」において言われ、合一は、「なされたこと」において言われる。
ゆえに、「合一するものは合一されたものである」とは言うが、「受容するものは
 受容されるものである」とは言わない。


\\

Natura enim humana significatur ut in
termino assumptionis ad hypostasim divinam per hoc quod dicitur {\itshape homo},
unde vere dicimus quod filius Dei, qui est uniens sibi humanam naturam,
est homo. Sed humana natura in se considerata, idest in abstracto,
significatur ut assumpta, non autem dicimus quod filius Dei sit humana
natura. 


&

たとえば、人間本性は、「人間」と言われることによって、神のヒュポスタシス
への受容の終極にあるものとして表示される。したがって、私たちは、真
 に、自らに人間本性を合一させる者である神の息子が、人間であると言う。し
 かし、それ自体で、すなわち抽象的に考えられた、人間本性は、受容された者
 として表示されるので、私たちは、神の息子が人間本性であるとは言わない。


\\

Ex eodem etiam sequitur tertia differentia, quod relatio,
praecipue aequiparantiae, non magis se habet ad unum extremum quam ad
aliud; actio autem et passio diversimode se habent ad agens et patiens,
et ad diversos terminos. 


&

同じことから、第三の差異も帰結する。それは、関係、とくに比較の関係は、一
 方より他方へ関係すると言うことがない。しかし、能動と受動は、能動者と受
 動者に、そしてさまざまな終極へ、異なる形で関係する。


\\

Et ideo assumptio determinat terminum et a quo
et ad quem, dicitur enim assumptio quasi ab alio ad se sumptio, unio
autem nihil horum determinat. Unde indifferenter dicitur quod humana
natura est unita divinae, et e converso. 


&

それゆえ、受容が終極を限定し、どこからどこへを限定するので、受容は、他の
 ものから自らへの受容として言われるが、合一は、これらのどれも限定しない。
 したがって、人間本性が神の本性に合一されるということと、その逆とが、区
 別なく言われる。

\\

Non autem dicitur divina natura
assumpta ab humana, sed e converso, quia humana natura adiuncta est ad
personalitatem divinam, ut scilicet persona divina in humana natura
subsistat.


&

しかし、神の本性が、民現本性によって受容されるは言われず、言われるのはそ
 の逆である。なぜなら、人間本性は、神のペルソナ性へ結び付けられる、すな
 わち、神のペルソナが、人間本性において自存するからである。

\\



{\scshape Ad primum ergo dicendum} quod unio et assumptio non eodem modo se habent
ad terminum, sed diversimode, sicut dictum est.


&

第一異論に対しては、合一と受容は、すでに述べられたとおり、同じしかたで終
 極に関係せず、異なるしかたで関係する、と言われるべきである。


\\



{\scshape Ad secundum dicendum} quod uniens et assumens non omnino sunt idem. Nam
omnis persona assumens est uniens, non autem e converso. Nam persona
patris univit naturam humanam filio, non autem sibi et ideo dicitur
uniens, non assumens. Et similiter non est idem unitum et assumptum. Nam
divina natura dicitur unita, non assumpta.


&

第二異論に対しては、以下のように言われるべきである。
合一するものと受容するものは、あらゆる点で同じであるわけではない。なぜな
 ら、全て受容するペルソナは合一するものだが、その逆ではないからである。
 つまり、父のペルソナは、子に、人間本性を合一させたが、自らに合一させた
 わけではない。ゆえに、合一させるものとは言われても、受容するものとは言
 われない。同様に、合一されたものと受容されたものとも同じではない。
なぜなら、神の本性は合一されたものとは言われるが、受容されたものとは言わ
 れないからである。


\\



{\scshape Ad tertium dicendum} quod assumptio determinat cui facta est copulatio ex
parte assumentis, inquantum assumptio dicitur quasi ad se sumptio. Sed
incarnatio et humanatio ex parte assumpti, quod est caro, vel natura
humana. Et ideo assumptio differt ratione et ab unione, et ab
incarnatione seu humanatione.


&


第三異論に対しては、以下のように言われるべきである。
受容は、「自らにたいして採る」というような意味で言われる限りで、受容するも
 のの側から、それへと結合する対象を限定する。しかし、受肉と人間化は、受
 容されるもの、すなわち肉あるいは人間本性の側からそれを限定する。ゆえに、
 受容は、合一、受肉、人間化から、概念において異なる。

\\



\end{longtable}
\newpage



\rhead{a.~9}
\begin{center}
 {\Large {\bf ARTICULUS NONUS}}\\
 {\large UTRUM UNIO DUARUM NATURARUM SIT MAXIMA UNIONUM}\\
 {\footnotesize III {\itshape Sent.}, d.5, q.1, a.1, qu$^a$2.}\\
 {\Large 第九項\\二つの本性の合一は、合一の中で最大か}
\end{center}

\begin{longtable}{p{21em}p{21em}}



{\Huge A}{\scshape d nonum sic proceditur}. Videtur quod unio duarum naturarum non sit
maxima unionum. Unitum enim deficit in ratione unitatis ab eo quod est
unum, eo quod unitum dicitur per participationem, unum autem per
essentiam. Sed in rebus creatis aliquid dicitur esse simpliciter unum,
sicut praecipue patet de ipsa unitate quae est principium numeri. Ergo
huiusmodi unio de qua loquimur, non importat maximam unitatem.


&

第九項の問題へ、議論は以下のように進められる。
二つの本性の合一は、合一の中で最大ではないと思われる。
理由は以下の通り。
合一したものは、一であるものから、一性の性格において劣っている。なぜなら、
 合一したものは分有によって言われるが、一は本質によって言われるからであ
 る。しかし、被造物の中には、端的に一と言われるものがある。とくに、数の
 始まりである一性それ自体について、それは明らかである。
ゆえに、私たちが語っているような合一は、最大限の一性を意味しない。

\\



2. {\scshape Praeterea}, quanto ea quae uniuntur magis distant, tanto minor est
unio. Sed ea quae secundum hanc unionem uniuntur, maxime distant,
scilicet natura divina et humana, distant enim in infinitum. Ergo
huiusmodi est minima unio.


&

さらに、合一されたものが隔たっているほど、合一は小さい。しかし、この合一
 によって合一されるものは、最大限に隔たっ
 ている。なぜなら、神の本性と人間の本性は、無限に隔たるからである。ゆえ
 に、そのようなものは、最少の合一である。


\\



3. {\scshape Praeterea}, per unionem aliquid fit unum. Sed ex unione animae et
corporis in nobis fit aliquid unum in persona et natura, ex unione autem
divinae et humanae naturae fit aliquid unum solum in persona. Ergo maior
est unio animae ad corpus quam divinae naturae ad humanam. Et sic unio
de qua loquimur, non importat maximam unitatem.


&

さらに、合一によって、何かが一つになる。しかし、魂と身体から、私たちにお
 いて、ペルソナと本性においてなにか一つのものが生じるが、神と人間の本性
 の合一からは、ペルソナにおいてのみ、なにか一つのものが生じる。ゆえに、
 魂と身体の合一の方が、神の本性と2限本性の合一よりも、大きな合一である。
 したがって、私たちが語っている合一は、最大の一性を意味しない。


\\



{\scshape Sed contra est} quod Augustinus dicit, in I {\itshape de Trin}., quod {\itshape homo potius
est in filio quam filius in patre}. Filius autem est in patre per
unitatem essentiae, homo autem est in filio per unionem
incarnationis. Ergo maior est unio incarnationis quam unitas divinae
essentiae. Quae tamen est maxima unitatum. Et sic, per consequens, unio
incarnationis importat maximam unitatem.


&

しかし反対に、アウグスティヌスは『三位一体論』第1巻で「人間は、息子が神
 においてあるよりも、息子においてある」と言う。しかし、息子は父において、
 本質の一性においてあるが、人間は息子において、受肉の一性によってある。
 ゆえに、受肉の一性は、神の本質の一性よりも大きい。しかし、神の本質の一
 性は、一性の中で最大である。このようにして、受肉の合一は、最大の一性を意味す
 ることが帰結する。

\\



{\scshape Respondeo dicendum} quod unio importat coniunctionem aliquorum in aliquo
uno. Potest ergo unio incarnationis dupliciter accipi, uno modo, ex
parte eorum quae coniunguntur; et alio modo, ex parte eius in quo
coniunguntur. 


&

解答する。以下のように言われるべきである。
合一は、あるものどもの、ある一つのものにおける結合を意味する。ゆえに、受
 肉の一性は、二通りに理解でき、一つには結合されるものどもの側から、もう
 一つには、結合が生じるものの側からである。


\\

Et ex hac parte huiusmodi unio habet praeeminentiam inter
alias uniones, nam unitas personae divinae, in qua uniuntur duae
naturae, est maxima. Non autem habet praeeminentiam ex parte eorum quae
coniunguntur.


&

この後者の側から、このような合一は、他の合一の中で傑出している。すなわち、
 二つの本性がそこにおいて合一される神のペルソナの一性は、最大である。し
 かし、結び付けられるものどもの側からは、傑出していない。


\\



{\scshape Ad primum ergo dicendum} quod unitas personae divinae est maior quam
unitas numeralis, quae scilicet est principium numeri. Nam unitas
divinae personae est unitas per se subsistens, non recepta in aliquo per
participationem, est etiam in se completa, habens in se quidquid
pertinet ad rationem unitatis. 


&

第一異論に対しては、それゆえ、以下のように言われるべきである。
神のペルソナの一性は、数的一性、すなわち数の根源である一性よりも大きい。
 なぜなら、神のペルソナの一性は、それ自体で自存する一性であり、分有によっ
 て何かに受け取られず、さらにそれ自体で完全であり、自らの中に、一性の性
 格に属するものはなんでも持っているからである。


\\


Et ideo non competit sibi ratio partis,
sicut unitati numerali, quae est pars numeri, et quae participatur in
rebus numeratis. Et ideo quantum ad hoc unio incarnationis praeeminet
unitati numerali, ratione scilicet unitatis personae. Non autem ratione
naturae humanae, quae non est ipsa unitas personae divinae, sed est ei
unita.


&

ゆえに、数的一性のように、部分の性格が自らに属さない。数的一性は、数の部
 分であり、数えられた事物に分有される。ゆえに、この点で、受肉の一性は、
 ペルソナの一性のゆえに、数的一性よりも優れている。しかし、人間本性のゆ
 えにではない。人間本性は、神のペルソナの一性自体ではなく、むしろそれに
 合一されているものなのである。


\\



{\scshape Ad secundum dicendum} quod ratio illa procedit ex parte coniunctorum, non
autem ex parte personae in qua est facta unio.


&

第二異論に対しては、この論は、結び付けられたものどもの側からなされていて、
 そこに合一がなされるペルソナの側からではない、と言われるべきである。

\\



{\scshape Ad tertium dicendum} quod unitas divinae personae est maior unitas quam
unitas et personae et naturae in nobis. Et ideo unio incarnationis est
maior quam unio animae et corporis in nobis.


&


第三異論に対しては、以下のように言われるべきである。
神のペルソナの一性は、私たちにおける、ペルソナと本性の一性よりも大きな一
 性である。
ゆえに、受肉の合一は、私たちにおける魂と身体の合一よりも大きい。

\\



Quia vero id quod in contrarium obiicitur falsum supponit, scilicet quod
maior sit unio incarnationis quam unitas personarum divinarum in
essentia, dicendum est ad auctoritatem Augustini quod humana natura non
est magis in filio Dei quam filius Dei in patre sed multo minus, sed
ipse homo, quantum ad aliquid, est magis in filio quam filius in patre;
inquantum scilicet idem supponitur in hoc quod dico homo, prout sumitur
pro Christo, et in hoc quod dico, filius Dei; non autem idem est
suppositum patris et filii.


&

他方、反対異論において、受肉の合一が本質における神のペルソナの一性よりも
 大きいという偽を措定しているので、アウグスティヌスの権威によって、以下
 のように言われるべきである。人間
 本性が神の息子において、神の息子が父においてあるよりも大きくはなく、む
 しろ、遥かに小さい。しかし、人間自体は、ある点に関して、父の中の
 息子よりも息子の中にある。すなわち、私が「人間」と言うものにおいて、同
 じものが、キリストと理解されるときと、神の息子と私が言うことにおいてあ
 るという点で。これに対して、父と息子の個体は同じではない。


\\




\end{longtable}
\newpage




\rhead{a.~10}
\begin{center}
 {\Large {\bf ARTICULUS DECIMUS}}\\
 {\large UTRUM UNIO INCARNATIONIS SIT PER GRATIAM}\\
 {\footnotesize Infra, q.6; III {\itshape Sent.}, d.13, q.3, a.1;
 {\itshape De Verit.}, q.29, a.2; {\itshape Ad Coloss.}, cap.2, lect.2.}\\
 {\Large 第十項\\受肉の合一は恩恵によるか}
\end{center}

\begin{longtable}{p{21em}p{21em}}

{\Huge A}{\scshape d decimum sic proceditur}. Videtur quod unio incarnationis non sit per
gratiam. Gratia enim est accidens quoddam, ut in secunda parte habitum
est. Sed unio humanae naturae ad divinam non est facta per accidens, ut
supra ostensum est. Ergo videtur quod unio incarnationis non sit facta
per gratiam.

&

第十項の問題へ、議論は以下のように進められる。
受肉の合一は恩恵によってあるのではないと思われる。
理由は以下の通り。
恩恵は、第二部で\footnote{STI-II, q.110, a.2, ad 2.}論じられたとおり、ある種の附帯性である。しかし、人間本性
の神の本性への合一は、前に示されたとおり\footnote{a.6.}、附帯性によるのではない。ゆえに、
受肉の一性は、恩恵によってなされたのではない。

\\



2. {\scshape Praeterea}, gratiae subiectum est anima. Sed sicut dicitur {\itshape Coloss}.~{\scshape ii}, {\itshape in
Christo habitavit plenitudo divinitatis corporaliter}. Ergo videtur quod
illa unio non sit facta per gratiam.

&

さらに、恩恵の期待は魂である。しかし、『コロサイの信徒への手紙』第2章で言わ
れるように、「キリストにおいて、神性の充満が肉体的に住んでいた」
\footnote{「キリストの内には、満ちあふれる神性が、余すところなく、見える
形をとって宿っており、」(2:9)}。ゆえに、かの合一は、恩恵によってなされた
のではないと思われる。

\\



3. {\scshape Praeterea}, quilibet sanctus Deo unitur per gratiam. Si igitur unio
incarnationis fuit per gratiam, videtur quod non aliter dicatur Christus
esse Deus quam alii sancti homines.

&

さらに、聖人は誰でも恩恵によって神に合一されている。ゆえに、もし受肉の合
 一が恩恵によってあったならば、他の聖人たちとは違って、キリストが神であ
 ると言われることはなかったと思われる。

\\



{\scshape Sed contra est} quod Augustinus dicit, in libro {\itshape de Praedest.~Sanctorum},
{\itshape ea gratia fit ab initio fidei suae homo quicumque Christianus, qua
gratia homo ille ab initio suo factus est Christus}. Sed homo ille factus
est Christus per unionem ad divinam naturam. Ergo unio illa fuit per
gratiam.

&

しかし反対に、アウグスティヌスは、『聖人たちの予定について』という書物で、
 「かの人が、自らの初めからキリストとされたその同じ恩恵によって、どんな
 人も、自らの信仰の初めから、キリスト教徒となる」と述べている。しかし、
 かの人は、神の本性への合一によって、キリストとされた。ゆえに、かの合一
 は、恩恵によってあった。

\\



{\scshape Respondeo dicendum} quod, sicut in secunda parte dictum est, gratia
dupliciter dicitur, uno modo, ipsa voluntas Dei gratis aliquid dantis;
alio modo, ipsum gratuitum donum Dei. 

&

解答する。第二部で言われたとおり\footnote{STI-II, q.110, a.1.}、恩恵は二通りの意味で言われる。一つは、
 無償で何かを与える神の意志自体であり、もう一つは、神の無償の贈り物自体
 である。


\\

Indiget autem humana natura
gratuita Dei voluntate ad hoc quod elevetur in Deum, cum hoc sit supra
facultatem naturae suae. 

&

ところで、人間本性は、神へと引き上げられるために、神の無償の意志を必要と
 する。なぜなら、このことは、自らの本性の機能を越えているからである。

\\

Elevatur autem humana natura in Deum
dupliciter. Uno modo, per operationem, qua scilicet sancti cognoscunt et
amant Deum. 

&

しかし、人間本性が神へ引き上げられるのに二通りある。一つは、働きによって
 であり、すなわち聖人たちは、働きによって、神を認識し、愛する。

\\

Alio modo, per esse personale, qui quidem modus est
singularis Christo, in quo humana natura assumpta est ad hoc quod sit
personae filii Dei. Manifestum est autem quod ad perfectionem
operationis requiritur quod potentia sit perfecta per habitum, sed quod
natura habeat esse in supposito suo, non fit mediante aliquo habitu. 

&


もう一つには、ペルソナ的存在によってであり、そのありかたは、キリスト独自
 のものである。そこにおいて、人間本性は、神の息子のペルソナに属すること
 に向けて受容される。さて、働きの完成のためには、能力が習態を通して完成
 されることが必要であるが、本性が自らの個体において存在をもつということ
 は、どんな習態を媒介することもなく生じることが明らかである。

\\


Sic
igitur dicendum est quod, si gratia accipiatur ipsa Dei voluntas gratis
aliquid faciens, vel gratum seu acceptum aliquem habens, unio
incarnationis facta est per gratiam, sicut et unio sanctorum ad Deum per
cognitionem et amorem. 

&

それゆえ、以下のように言われるべきである。もし恩恵が、無償で何かを作る、
 あるいは、与えられたものや受け取ったものを持たせる神の意志そのものと理解されるならば、受肉の合一は、ちょうど認識と
 愛を通した聖人たちの神への合一が恩恵によってなされるように、恩恵によっ
 てなされた。


\\

Si vero gratia dicatur ipsum gratuitum Dei donum,
sic ipsum quod est humanam naturam esse unitam personae divinae, potest
dici quaedam gratia, inquantum nullis praecedentibus meritis hoc est
factum, non autem ita quod sit aliqua gratia habitualis qua mediante
talis unio fiat.

&

他方、もし恩恵が、神の無償の贈り物として語られるならば、人間本性が神のペ
 ルソナに合一したこと自体は、どんな先行する功績もなくこれがなされた限り
 で、ある種の恩恵と言われうるが、それは、何らかの習態的恩恵が存在し、その媒介で
 このような合一がなされたというわけではない。

\\



{\scshape Ad primum ergo dicendum} quod gratia quae est accidens, est quaedam
similitudo divinitatis participata in homine. Per incarnationem autem
humana natura non dicitur participasse similitudinem aliquam divinae
naturae, sed dicitur esse coniuncta ipsi naturae divinae in persona
filii. Maius autem est ipsa res quam similitudo eius participata.

&

第一異論に対しては、それゆえ、以下のように言われるべきである。
附帯性である恩恵は、人間に分有された神性の一種の類似である。しかし受肉に
 よって、人間本性は、神の本性の何らかの類似を分有したとは言われず、むし
 ろ、子のペルソナにおいて、神の本性自体に結合されたと言われる。
しかし、ものそれ自体の方が、そのものの類似を分有したものよりも、大きい。

\\



Ad secundum dicendum quod gratia habitualis est solum in anima, sed
gratia, idest gratuitum Dei donum quod est uniri divinae personae,
pertinet ad totam naturam humanam, quae componitur ex anima et
corpore. 

&

第二異論に対しては、以下のように言われるべきである。
習態的恩恵は魂の中だけにあるが、神のペルソナに合一されうる神の無償の贈り
 物は、魂と身体から複合される人間本性全体に属する。


\\

Et per hunc modum dicitur plenitudo divinitatis in Christo
corporaliter habitasse, quia est unita divina natura non solum animae,
sed etiam corpori. 

&

そして、この意味で、神性の充満がキリストにおいて物体的に住んでいた、と言われ
 る。なぜなら、神の本性は、魂だけでなく、身体にも合一したからである。


\\

Quamvis etiam possit dici quod dicitur habitasse in
Christo corporaliter, idest {\itshape non umbraliter}, sicut habitavit in
sacramentis veteris legis, de quibus ibidem subditur quod sunt {\itshape umbra
futurorum, corpus autem est Christus}, prout scilicet corpus contra
umbram dividitur. 

&

ただし、キリストにおいて物体的に住んでいたと言われることが、「影において
 でなく」という意味に読むことは可能である。ちょうど、旧約の秘蹟において
 住んでいたように。その秘蹟については、同じ箇所で、身体が影に対して分け
 られるもの理解されて、「未来の影。身体はキリスト」と付け加えられている。


\\

Dicunt etiam quidam quod divinitas dicitur in Christo
habitasse corporaliter, scilicet tribus modis, sicut corpus habet tres
dimensiones, uno modo, per essentiam, praesentiam et potentiam, sicut in
ceteris creaturis; alio modo, per gratiam gratum facientem, sicut in
sanctis tertio modo, per unionem personalem, quod est proprium sibi.

&

さらに、ある人々は、神性がキリストにおいて物体的に住んでいたと言われるこ
 とは、ちょうど物体が三つの次元をもつように三つの意味があり、一つには、
 ある被造物においてあるように、本質、現在、能力によってであり、もう一つ
 には、聖人たちにおいてあるように、無償の恩恵によって、そして第三には、
 キリスト自らに固有のこととして、ペルソナ的合一によってであると言う。

\\



Unde patet responsio ad tertium, quia scilicet unio incarnationis non
est facta solum per gratiam habitualem, sicut alii sancti uniuntur Deo;
sed secundum subsistentiam, sive personam.

&

このことから、第三異論への答えは明らかである。すなわち、受肉の合一は、他
 の聖人たちが神に合一する場合のように、習態的恩恵によってのみあるのでは
 なく、自存体、ないしペルソナにおいてある。

\end{longtable}
\newpage





\rhead{a.~11}
\begin{center}
 {\Large {\bf ARTICULUS UNDECIMUS}}\\
 {\large UTRUM UNIO INCARNATIONIS FUERIT ALIQUA MERITA SUBSECUTA}\\
 {\footnotesize I$^a$ II$^{ae}$, q.98, a.4; III {\itshape Sent.}, d.4,
 q.3, a.1; {\itshape Ad Heb.}, cap.1, lect.4.}\\
 {\Large 第十一項\\受肉の合一は何らかの功績の後に続いたか}
\end{center}

\begin{longtable}{p{21em}p{21em}}


{\Huge A}{\scshape d undecimum sic proceditur}. Videtur quod unio incarnationis fuerit
aliqua merita subsecuta. Quia super illud Psalmi, {\itshape fiat misericordia tua,
domine, super nos, quemadmodum speravimus in te}, dicit Glossa,
 {\itshape hic
insinuatur desiderium prophetae de incarnatione, et meritum
impletionis}. Ergo incarnatio cadit sub merito.


&


第十一項の問題へ、議論は以下のように進められる。
受肉の合一は、何らかの功績の後に続いたと思われる。理由は以下の通り。
かの『詩編』の「主よ、私たちがあなたにおいて望んだように、私たちの上に憐れみを
 かけてください」\footnote{「主よ、あなたの慈しみが/我らの上にあるよう
 に/主を待ち望む我らの上に。」(31:22)}について、『注解』は「ここで、受肉についての預言者の願
 いと、成就の功績が示唆されている」と言っている。ゆえに、受肉は、功績の
 もとにある。


\\



2. {\scshape Praeterea}, quicumque meretur aliquid, meretur illud sine quo illud
haberi non potest. Sed antiqui patres merebantur vitam aeternam, ad quam
pervenire non poterant nisi per incarnationem, dicit enim Gregorius, in
libro Moral., 



&

さらに、だれであれ、ある功績によって何かを獲得するならば、それがないと獲
 得できないようなものもまた、獲得している。しかし、古代の父祖たちは、功
 績によって永遠の生命を獲得したが、受肉を通してでなければ、それに到達す
 ることはできなかった。なぜなら、グレゴリウスが『道徳論』という書物で次
 のように述べているからである。


\\

{\itshape hi qui ante Christi adventum in hunc mundum venerunt,
quantamlibet iustitiae virtutem haberent, ex corporibus educti in sinum
caelestis patriae statim recipi nullo modo poterant, quia nondum ille
venerat qui iustorum animas in perpetua sede collocaret}. Ergo videtur
quod meruerint incarnationem.


&

「キリストの到来以前にこの世に生まれた人々は、どれだけ多くの正義の徳をもっ
 ていたとしても、物体的なものどもから引き抜かれ、天の祖国の胸へ、直ちに
 受け取られることは決してできなかった。なぜなら、義人たちの魂を永続する
 座席に座らせる彼の人がまだ来ていなかったからである」。ゆえに、受肉は報
 償として与えられた。


\\



3. {\scshape Praeterea}, de beata virgine cantatur quod {\itshape dominum omnium meruit portare},
quod quidem factum est per incarnationem. Ergo incarnatio cadit sub
merito.


&

さらに、至福な処女について「万人の主をもたらす褒美を与えられ」と歌われて
 いる。しかしこれは、受肉によって行われた。ゆえに、受肉は功績のもとにあ
 る。


\\



{\scshape Sed contra est} quod Augustinus dicit, in libro {\itshape de Praedest.~Sanctorum},
{\itshape quisquis in capite nostro praecedentia merita singularis illius
generationis invenerit, ipse in nobis, membris eius, praecedentia merita
multiplicatae regenerationis inquirat}. 


&

しかし反対に、アウグスティヌスは『聖人たちの予定について』という書物で、
 以下のように述べている。「私たちの頭において、先行する彼の個々の生成の
 功績を見出した人はだれでも、私たちにおいて、彼の四肢において、数多くに
 された再生の先行する功績を求めるだろう」。


\\


Sed nulla merita praecesserunt
regenerationem nostram secundum illud {\itshape Tit}.~{\scshape iii}, {\itshape non ex operibus
iustitiae quae fecimus nos, sed secundum suam misericordiam salvos nos
fecit per lavacrum regenerationis}. Ergo nec illam Christi generationem
aliqua merita praecesserunt.


&

しかし、かの『テトスへの手紙』第3章「私たちが行った義の業に基づいてでな
 く、自分の憐れみに従って、再生の水盤によって私たちを救われたものとした」
 \footnote{「神は、わたしたちが行った義の業によってではなく、御自分の憐
 れみによって、わたしたちを救ってくださいました。この救いは、聖霊によっ
 て新しく生まれさせ、新たに造りかえる洗いを通して実現したのです。」
 (3:5)}によれば、私たちの再生に先行する功績は何もない。ゆえに、キリスト
 のかの出生に先行する功績もなかった。

\\



{\scshape Respondeo dicendum} quod, quantum ad ipsum Christum, manifestum est ex
praemissis quod nulla eius merita potuerunt praecedere unionem. Non enim
ponimus quod ante fuerit purus homo, et postea per meritum bonae vitae
obtinuerit esse filius Dei, sicut posuit Photinus, sed ponimus quod a
principio suae conceptionis ille homo vere fuerit filius Dei, utpote non
habens aliam hypostasim quam filium Dei, secundum illud Luc.~{\scshape i}, {\itshape quod ex
te nascetur sanctum, vocabitur filius Dei}. 


&

解答する。以下のように言われるべきである。
キリスト自身にかんして言えば、彼の功績が合一に先行することがあえ理恵なかっ
 たことが、先述のことから明らかである。なぜなら、ポティヌスが述べたよう
 に、彼が、以前には純粋な人間であったが、それ以後、善い生の功績によって、神の息子であ
 ることを獲得した、と私たちは考えないからである。そうではなく、私たち
 は、自らの懐胎の最初から、かの人は、真に神の息子であると考える。それは、
 『ルカによる福音書』第1章「あなたから生まれるものは、聖なるもの、神の息子と呼
 ばれるでしょう」\footnote{「天使は答えた。「聖霊があなたに降り、いと高
 き方の力があなたを包む。だから、生まれる子は聖なる者、神の子と呼ばれる。」
 (1:35) }によれば、神の息子以外のヒュポスタシスを持たないものとしてある。

\\


Et ideo omnis operatio illius
hominis subsecuta est unionem. Unde nulla eius operatio potuit esse
meritum unionis. 

&

 ゆえに、かの人の全ての働きは、合一の後にある。したがって、彼のどの働き
 も、合一のための功績ではない。


\\


Sed neque etiam opera cuiuscumque alterius hominis
potuerunt esse meritoria huius unionis ex condigno.
Primo quidem, quia
opera meritoria hominis proprie ordinantur ad beatitudinem, quae est
virtutis praemium, et consistit in plena Dei fruitione. Unio autem
incarnationis, cum sit in esse personali, transcendit unionem mentis
beatae ad Deum, quae est per actum fruentis. Et ita non potest cadere
sub merito.


&

また、他のどんな人の業も、この合一に値する功績ではありえなかった。一つに
 は、人間の功績の業は、厳密には、至福へと秩序付けられているが、至福は、
 徳の報償である。そしてそれは、十分な神の享受において成立する。しかし、
 受肉の合一は、ペルソナ的存在においてあるので、享受する者の働きを通して
 ある至福な精神の神への合一を越える。したがって、それは功績のもとに入り
 えない。


\\

 Secundo, quia gratia non potest cadere sub merito, quia est
merendi principium. Unde multo minus incarnatio cadit sub merito, quae
est principium gratiae, secundum illud Ioan.~{\scshape i}, {\itshape gratia et veritas per
Iesum Christum facta est}. 


&


第二に、恩恵は、何かの功績があって与えられるのではない。なぜなら、恩恵は、
 報償を与えることの根源だからである。したがって、『ヨハネによる福音書』第1
 章「恩恵と真理はイエス・キリストによってなされた」\footnote{「律法はモー
 セを通して与えられたが、恵みと真理はイエス・キリストを通して現れたから
 である。」(1:17)}によれば、受肉は恩恵の根源だから、ましてや、それが功績
 のために与えられることはない。

\\



Tertio, quia incarnatio Christi est
reformativa totius humanae naturae. Et ideo non cadit sub merito
alicuius hominis singularis, quia bonum alicuius puri hominis non potest
esse causa boni totius naturae. Ex congruo tamen meruerunt sancti patres
incarnationem, desiderando et petendo. Congruum enim erat ut Deus
exaudiret eos qui ei obediebant.


&


第三に、キリストの受肉は人間本性全体を変えるものである。それゆえ、どんな
 個人の功績によってもない。なぜなら、ある純粋な人間の善は、全本性の善の
 原因でありえないからである。しかし、聖なる父祖たちは、望むことと乞うこ
 とによって、集団的に、受肉の功績であった。なぜなら、神が、神に従ってい
 た人々の声を聞くことは適切であったから。

\\



Et per hoc patet responsio ad primum.

&


これによって、第一異論への解答は明らかである。

\\


{\scshape Ad secundum dicendum} hoc esse falsum, quod sub merito cadat omne illud
sine quo praemium esse non potest. Quaedam enim sunt quae non solum
requiruntur ad praemium, sed etiam praeexiguntur ad meritum, sicut
divina bonitas et eius gratia, et ipsa hominis natura. Et similiter
incarnationis mysterium est principium merendi, quia {\itshape de plenitudine
Christi omnes accepimus}, ut dicitur Ioan.~{\scshape i}.


&

第二異論に対しては、以下のように言われるべきである。
それがなければ報償がありえないものはすべて、功績の褒美として与えられる、
 というのは偽である。なぜなら、神の善性、恩恵、そして人間の本性のように、
 報償のために必要であるだけでなく、功績を積むために必要なものもあ
 るからである。同様に、受肉の秘蹟は功績を積むことの根源である。なぜなら、『ヨ
 ハネによる福音書』第1章で言われるように、「キリストの充満から、私たちは
 皆、受け取った」\footnote{「わたしたちは皆、この方の満ちあふれる豊かさ
 の中から、恵みの上に、更に恵みを受けた。」(1:16)}からである。


\\



{\scshape Ad tertium dicendum} quod beata virgo dicitur meruisse portare dominum
Iesum Christum, non quia meruit Deum incarnari, sed quia meruit, ex
gratia sibi data, illum puritatis et sanctitatis gradum ut congrue
posset esse mater Dei.


&


第三異論に対しては、以下のように言われるべきである。
至福な処女が、主イエス・キリストをもたらす報償を受けたと言われるのは、神
 が受肉することを報償として受けたからではなく、彼女に与えられた恩恵によっ
 て、神の母であることが適切でありうるような段階の純粋さと聖性とを、報償と
 して得たからである。

\\


\end{longtable}
\newpage




\rhead{a.~12}
\begin{center}
 {\Large {\bf ARTICULUS DUODECIMUS}}\\
 {\large UTRUM GRATIA UNIONIS FUERIT CHRISTO HOMINI NATURALIS}\\
 {\footnotesize Infra q.34, a.3, ad 2; III {\itshape Sent.}, d.4, q.3,
 a.2, qu$^a$ 1.}\\
 {\Large 第十二項\\合一の恩恵は人間キリストにとって自然本性的であったか}
\end{center}

\begin{longtable}{p{21em}p{21em}}



{\Huge A}{\scshape d duodecimum sic proceditur}. Videtur quod gratia unionis non fuerit
Christo homini naturalis. Unio enim incarnationis non est facta in
natura, sed in persona, ut supra dictum est. Sed unumquodque denominatur
a termino. Ergo gratia illa magis debet dici personalis quam naturalis.


&

第十二項の問題へ、議論は以下のように進められる。
合一の恩恵は、人間キリストにとって自然本性的ではなかったと思われる。
理由は以下の通り。
前に言われたとおり\footnote{a.1, 2.}、受肉の合一は本性ではなくペルソナに
 おいてなされた。しかし、各々のものは、終極によって名付けられる。ゆえに
 かの恩恵は、自然本性的ではなく、ペルソナ的と言われるべきである。


\\



2. {\scshape Praeterea}, gratia dividitur contra naturam, sicut gratuita, quae sunt a
Deo, distinguuntur contra naturalia, quae sunt a principio
intrinseco. Sed eorum quae ex opposito dividuntur, unum non denominatur
ab alio. Ergo gratia Christi non est ei naturalis.


&


さらに、ちょうど、神からである恩恵的なものが、内在的根源からである自然本性的なも
 のから分けられるように、恩恵は自然に対立して分けられる。しかし、対立し
 て分けられるものどもは、一方が他方によって名付けられることがない。ゆえ
 に、キリストの恩恵は自然本性的でない。

\\



3. {\scshape Praeterea}, naturale dicitur quod est secundum naturam. Sed gratia
unionis non est naturalis Christo secundum naturam divinam, quia sic
conveniret etiam aliis personis. Neque etiam naturalis est ei secundum
naturam humanam, quia sic conveniret omnibus hominibus qui sunt eiusdem
naturae cum ipso. Ergo videtur quod nullo modo gratia unionis sit
Christo naturalis.


&

さらに、自然本性にしたがってあるものが、自然本性的なものと言われる。しか
 し、合一の恩恵は、神の本性にしたがって、自然本性的なキリストにあるので
 はない。なぜなら、もしそうなら、他のペルソナにもそれが適合したであろう
 から。さらに、人間本性にしたがって、キリストにとって自然本性的なのでも
 ない。なぜなら、もしそうなら、彼と同じ本性をもつ全ての人間に適合したで
 あろうから。ゆえに、どんなかたちでも、合一の恩恵は、自然本性的なキリス
 トにないと思われる。


\\



{\scshape Sed contra est} quod Augustinus dicit, in Enchirid., {\itshape in naturae humanae
susceptione fit quodammodo ipsa gratia illi homini naturalis, qua nullum
possit admittere peccatum}.



&

しかし反対に、アウグスティヌスは、『エンキリディオン』で、「人間本性を受
 けることにおいて、恩恵自体があるかたちで彼の人に自然本性的となる。それ
 によって、どんな罪も受け入れることが不可能となるように」と述べている。


\\


{\scshape Respondeo dicendum} quod, secundum philosophum, in V {\itshape Metaphys}., natura
uno modo dicitur ipsa nativitas, alio modo essentia rei. Unde naturale
potest aliquid dici dupliciter. Uno modo, quod est tantum ex principiis
essentialibus rei, sicut igni naturale est sursum ferri. 


&


解答する。以下のように言われるべきである。
『形而上学』第5巻の哲学者によれば、本性は、一つには生み出す力のことを言
 い、もう一つには事物の本質を言う。ゆえに、何かが自然本性的であるという
 のも、二通りに言われうる。一つには、たとえば、火にとって、上へ上ること
 が自然本性的であるように、たんに事物の本質的根源に基づくものがそう言わ
 れる。

\\

Alio modo
dicitur esse homini naturale quod ab ipsa nativitate habet, secundum
illud {\itshape Ephes}.~{\scshape ii}, {\itshape eramus natura filii irae}; et {\itshape Sap}.~{\scshape xii}, {\itshape nequam est natio
eorum, et naturalis malitia ipsorum}. 


&

もう一つには、『エフェソの信徒への手紙』第2章「私たちは、自然本性によって、
 怒りの息子だった」\footnote{「わたしたちも皆、こういう者たちの中にいて、
 以前は肉の欲望の赴くままに生活し、肉や心の欲するままに行動していたので
 あり、ほかの人々と同じように、生まれながら神の怒りを受けるべき者でした。」
 (12:3)}や、『知恵の書』第12章「彼らの生まれが悪く、彼らの悪意は自然本性的で」\footnote{「しかし、あなたは徐々に罰を加えながら、/悔い改めの機会を与えておられた。彼らの血統が悪く、その悪意が生来のもので、/その思いが永久に変わらないことを、/ご存じであったにもかかわらず。」(12:10)}
によれば、生み出す力からそれを持つために、人間にとって自然本性的と言
 われる。


\\

Gratia igitur Christi, sive unionis
sive habitualis, non potest dici naturalis quasi causata ex principiis
naturae humanae in ipso, quamvis possit dici naturalis quasi proveniens
in naturam humanam Christi causante divina natura ipsius. 



&


ゆえに、キリストの恩恵は、合一の恩恵であれ習態的な恩恵であれ、彼の中の人
 間本性の諸根源が原因となって生じたという意味で、自然本性的と言われるこ
 とはできない。ただし、それらが、彼の神の
 本性が原因となってキリストの人間本性へとやって来るものとして、自然本性
 的と言われることは可能である。

\\

Dicitur autem
naturalis utraque gratia in Christo inquantum eam a nativitate habuit,
quia ab initio conceptionis fuit natura humana divinae personae unita,
et anima eius fuit munere gratiae repleta.


&

これに対して、どちらの恩恵も、キリストにおいて、生み出す力によってそれを
 持った限りにおいて、自然本性的と言われる。なぜなら、懐胎の初めから、人
 間本性は神のペルソナに合一され、彼の魂は、恩恵の賜物で満たされていたか
 らである。

\\



{\scshape Ad primum ergo dicendum} quod, licet unio non sit facta in natura, est
tamen causata ex virtute divinae naturae, quae est vere natura
Christi. Et etiam convenit Christo a principio nativitatis.


&

第一異論に対しては、それゆえ、以下のように言われるべきである。
合一は本性においてなされたのではないが、しかし、神の本性の力が原因となっ
 て生じた。そして、神の本性は、真に、キリストの本性である。また、それは、
 生み出す力の根源から、キリストに適合している。


\\



{\scshape Ad secundum dicendum} quod non secundum idem dicitur gratia, et
naturalis. Sed gratia quidem dicitur inquantum non est ex merito,
naturalis autem dicitur inquantum est ex virtute divinae naturae in
humanitate Christi ab eius nativitate.


&

第二異論に対しては、以下のように言われるべきである。
同一のものに即して、恩恵でありかつ自然本性的だと言われるのではない。
恩恵と言われるのは、功績に基づいてでない限りでであり、自然本性的と言われ
 るのは、神の本性の力に基づいて、彼の生み出す力によって、キリストの人
 間性においてある限りにおいてである。

\\



{\scshape Ad tertium dicendum} quod gratia unionis non est naturalis Christo
secundum humanam naturam, quasi ex principiis humanae naturae
causata. Et ideo non oportet quod conveniat omnibus hominibus.


&

第三異論に対しては、以下のように言われるべきである。
合一の恩恵は、キリストにとって、人間本性の根源が原因になって生じるという
 意味で、人間本性に即して自然本性的なのではない。
ゆえに、全ての人間に適合しなくてもよい。

\\

 Est tamen
naturalis ei secundum humanam naturam, propter proprietatem nativitatis
ipsius, prout sic conceptus est ex spiritu sancto ut esset idem
 naturalis filius Dei et hominis. 



&

しかし、彼の生み出す力という固有性のために、神の息子と人間が自然本性的に
 同一であるように、聖霊によって懐胎されたという意味で、人間本性にしたがっ
 て自然本性的である。


\\

Secundum vero divinam naturam
est ei naturalis, inquantum divina natura est principium activum huius
gratiae. Et hoc convenit toti Trinitati, scilicet huius gratiae esse
activum principium.


&

他方、神の本性に即して、彼にとって自然本性的であるのは、神の本性が、この
 恩恵の作用的根源である限りにおいてである。そして、これ、すなわち、この
 恩恵の作用的根源であることは、三位一体全体に適合する。


\end{longtable}

\end{document}