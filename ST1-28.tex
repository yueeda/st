\documentclass[10pt]{jsarticle} % use larger type; default would be 10pt
%\usepackage[utf8]{inputenc} % set input encoding (not needed with XeLaTeX)
%\usepackage[round,comma,authoryear]{natbib}
%\usepackage{nruby}
\usepackage{okumacro}
\usepackage{longtable}
%\usepqckage{tablefootnote}
\usepackage[polutonikogreek,english,japanese]{babel}
%\usepackage{amsmath}
\usepackage{latexsym}
\usepackage{color}

%----- header -------
\usepackage{fancyhdr}
\pagestyle{fancy}
\lhead{{\it Summa Theologiae} I, q.~28}
%--------------------


\title{{\bf PRIMA PARS}\\{\HUGE Summae Theologiae}\\Sancti Thomae
Aquinatis\\{\sffamily QUEAESTIO VIGESIMAOCTAVA}\\DE RELATIONIBUS DIVINIS}
\author{Japanese translation\\by Yoshinori {\sc Ueeda}}
\date{Last modified \today}

%%%% コピペ用
%\rhead{a.~}
%\begin{center}
% {\Large {\bf }}\\
% {\large }\\
% {\footnotesize }\\
% {\Large \\}
%\end{center}
%
%\begin{longtable}{p{21em}p{21em}}
%
%&
%
%\\
%\end{longtable}
%\newpage



\begin{document}

\maketitle

\begin{center}
{\Large 第二十八問\\神の関係について}
\end{center}


\begin{longtable}{p{21em}p{21em}}

Deinde considerandum est de relationibus divinis. Et circa hoc
 quaeruntur quatuor. 


\begin{enumerate}
 \item utrum in Deo sint aliquae relationes reales.
 \item utrum illae relationes sint ipsa essentia divina, vel sint extrinsecus affixae.
 \item utrum possint esse in Deo plures relationes realiter distinctae ab invicem.
 \item de numero harum relationum.
\end{enumerate}


&

次に、神の関係について考察されるべきである。これを巡って、四つのことが問
 われる。

\begin{enumerate}
 \item 神の中に何らかの関係があるか。
 \item その関係は、神の本質それ自体か、それとも、外から加えられたものか。
 \item 神の中に、相互に実在的に区別される複数の関係がありうるか。
 \item それらの関係の数について。
\end{enumerate}


\end{longtable}



\newpage



\rhead{a.~1}
\begin{center}
 {\Large {\bf ARTICULUS PRIMUS}}\\
 {\large UTRUM IN DEO SINT ALIQUAE RELATIONES REALES}\\
 {\footnotesize I {\itshape Sent.}, d.26, q.2, a.1; IV {\itshape SCG},
 cap.14; {\itshape De Pot.}, q.8, a.1; {\itshape Compend.~Theol.},
 cap.53; In {\itshape Ioan.}, cap.16, lect.4.}\\
 {\Large 第一項\\神の中に何らかの実在的関係があるか}
\end{center}

\begin{longtable}{p{21em}p{21em}}




{\Huge A}{\scshape d primum sic proceditur}. Videtur quod in Deo non sint aliquae
relationes reales. Dicit enim Boetius, in libro {\itshape de Trin}., quod {\itshape cum quis
praedicamenta in divinam vertit praedicationem, cuncta mutantur in
substantiam quae praedicari possunt; ad aliquid vero omnino non potest
praedicari}. Sed quidquid est realiter in Deo, de ipso praedicari
potest. Ergo relatio non est realiter in Deo.


&

第一項の問題へ、議論は以下のように進められる。
神の中に、実在的関係はないと思われる。理由は以下の通り。
ボエティウスは、『三位一体論』という書物の中で、以下のように述べている。
「だれかが、カテゴリーを神の述語にするとき、述語付けられうるすべてのものは、実体へ変えられる。しかし、関係は、どんな意味で
 も神について述語付けられない」。しかし、神において実在的にあるものは、
 神について、述語付けられうる。ゆえに、関係は、神の中に実在的にあるわけ
 ではない。




\\



2. {\scshape Praeterea}, dicit Boetius in eodem libro, quod {\itshape similis est relatio in
Trinitate patris ad filium, et utriusque ad spiritum sanctum, ut eius
quod est idem, ad id quod est idem}. Sed huiusmodi relatio est rationis
tantum, quia omnis relatio realis exigit duo extrema realiter. Ergo
relationes quae ponuntur in divinis, non sunt reales relationes, sed
rationis tantum.


&

さらに、ボエティウスは、同書で、「三位一体において、父の子への関係、そし
 て、両者の聖霊への関係は、同じものの同じものへの関係として類似している」
 と述べている。しかし、このような関係はたんに概念的なものにすぎない。な
 ぜなら、すべての実在的関係は、実在的に、二つの項を必要とするからである。
 ゆえに、神の中に措定される関係は、実在的ではなく、たんに概念的な関係で
 ある。

\\



3. {\scshape Praeterea}, relatio paternitatis est relatio principii. Sed cum dicitur,
{\itshape Deus est principium creaturarum}, non importatur aliqua relatio realis,
sed rationis tantum. Ergo nec paternitas in divinis est relatio
realis. Et eadem ratione nec aliae relationes quae ponuntur ibi.


&

さらに、「父」という関係は、根源の関係である。しかし、「神は被造物の根源
 である」と言われるとき、それは実在的関係を意味せず、たんに概念的な関係
 を意味するだけである。ゆえに、神における「父」も、実在的でない。同じ理
 由で、そこに措定される他の関係も実在的でない。



\\



4. {\scshape Praeterea}, generatio in divinis est secundum intelligibilis verbi
processionem. Sed relationes quae consequuntur operationem intellectus,
sunt relationes rationis. Ergo paternitas et filiatio, quae dicuntur in
divinis secundum generationem, sunt relationes rationis tantum.


&

さらに、神における生成は、可知的な言葉の発出に即してある。しかし、知性の
 働きに伴う関係は、概念的関係である。ゆえに、生成に即して神において言わ
 れる「父」「息子」は、たんに概念的な関係にすぎない。


\\



{\scshape Sed contra est} quod pater non dicitur nisi a paternitate, et filius a
filiatione. Si igitur paternitas et filiatio non sunt in Deo realiter,
sequitur quod Deus non sit realiter pater aut filius, sed secundum
rationem intelligentiae tantum, quod est haeresis Sabelliana.


&

しかし反対に、父は父性によって、息子は息子性によってのみ語られる。ゆえに、
 もし、父性と息子性が、神の中で実在的にあるのでなければ、神は、実在的に
ではなく、たんに理解の特質に即してのみ父であり息子であることになる。しか
 しこれは、サベリウス派の異端である。

\\



{\scshape Respondeo dicendum} quod relationes quaedam sunt in divinis realiter. Ad
cuius evidentiam, considerandum est quod solum in his quae dicuntur ad
aliquid, inveniuntur aliqua secundum rationem tantum, et non secundum
rem. 


&

解答する。以下のように言われるべきである。
神の中に、実在的な関係がある。
これを明らかにするために、関係的に語られるものどもの中にだけ、たんに概念
 的なものと、事物に即してあるものとが見出されることが考察されるべきであ
 る。

\\

Quod non est in aliis generibus, quia alia genera, ut quantitas et
qualitas, secundum propriam rationem significant aliquid alicui
inhaerens. Ea vero quae dicuntur ad aliquid, significant secundum
propriam rationem solum respectum ad aliud. Qui quidem respectus
aliquando est in ipsa natura rerum; utpote quando aliquae res secundum
suam naturam ad invicem ordinatae sunt, et invicem inclinationem
habent. 
Et huiusmodi relationes oportet esse reales. Sicut in corpore
gravi est inclinatio et ordo ad locum medium, unde respectus quidam est
in ipso gravi respectu loci medii. Et similiter est de aliis
huiusmodi. 


&

このことは、他の類では起こらない。なぜなら、量や性質などの他の類は、固有
 の性格に即して、何かが何かに内属することを意味表示するからである。しか
 し、関係的に言われるものどもは、固有の性格に即して、ただ他のものへの関
 係を表示する。この関係は、あるときには、事物の本性それ自体の中にある。
 たとえば、ある事物が、それ自身の本性に即して、相互に秩序付けられていて、
 相互に傾向性を持つ場合がそれである。
そして、そのような関係は、実在的である。たとえば、重い物体の中には、中心の
 場所への傾向性と秩序があるので、その関係は、中心の場所に対して、重いも
 の自体の中にある。他のものについても同様である。

\\

Aliquando vero respectus significatus per ea quae dicuntur ad
aliquid, est tantum in ipsa apprehensione rationis conferentis unum
alteri, et tunc est relatio rationis tantum; sicut cum comparat ratio
hominem animali, ut speciem ad genus. 

&


他方、別の場合には、関係的に語られるものによって表示される関係が、一方を他
 方へ関係づける理性の把握の中だけにある場合がある。その場合、関係は、た
 んに理性にのみ属する。たとえば、種を類へ関係づけるようにして、人間を動
 物に関係づけるときのように。

\\


Cum autem aliquid procedit a
principio eiusdem naturae, necesse est quod ambo, scilicet procedens et
id a quo procedit, in eodem ordine conveniant, et sic oportet quod
habeant reales respectus ad invicem. Cum igitur processiones in divinis
sint in identitate naturae, ut ostensum est, necesse est quod relationes
quae secundum processiones divinas accipiuntur, sint relationes reales.

&

しかし、同じ本性の根源から何かが発出するときには、両者、つまり、発出するも
 のと、それがそこから発出したところのもの(発出させたもの)とが、同じ秩序において一致する必要があり、
 したがって、相互に実在的な関係をもたなければならない。ゆえに、神におけ
 る発出は、すでに示されたとおり、本性の同一性に即したものなので、神の発
 出に即して理解される諸関係は実在的な関係であることが必然である。

\\




{\scshape Ad primum ergo dicendum} quod {\itshape ad aliquid} dicitur omnino non praedicari in
Deo, secundum propriam rationem eius quod dicitur ad aliquid; inquantum
scilicet propria ratio eius quod ad aliquid dicitur, non accipitur per
comparationem ad illud cui inest relatio, sed per respectum ad
alterum. Non ergo per hoc excludere voluit quod relatio non esset in
Deo, sed quod non praedicaretur per modum inhaerentis secundum propriam
relationis rationem, sed magis per modum ad aliud se habentis.


&


第一異論に対しては、それゆえ、以下のように言われるべきである。
「関係」は、他のものへの
 関係によって理解されるのであり、その関係が内在するものへの関係によって
 理解されるのではないという意味で、「あるものに対して」と言
 われるその固有の性格に即しては、決して、神において述語されないと言われ
 る。ゆえに、ボエティウスは、これによって、関係が神の中にないと主張した
 かったのではなく、関係の固有の性格に即して、内在するもののしかたで述語
 されるのではなく、むしろ、他に対して関わるというしかたで述語されると主
 張したかったのである。


\\



{\scshape Ad secundum dicendum} quod relatio quae importatur per hoc nomen {\itshape idem},
est relatio rationis tantum, si accipiatur simpliciter idem, quia
huiusmodi relatio non potest consistere nisi in quodam ordine quem ratio
adinvenit alicuius ad seipsum, secundum aliquas eius duas
considerationes. Secus autem est, cum dicuntur aliqua eadem esse, non in
numero, sed in natura generis sive speciei. Boetius igitur relationes
quae sunt in divinis, assimilat relationi identitatis, non quantum ad
omnia, sed quantum ad hoc solum, quod per huiusmodi relationes non
diversificatur substantia, sicut nec per relationem identitatis.


&

第二異論に対しては、以下のように言われるべきである。「同じ」という言葉で
 意味される関係は、「同じ」が単純に理解される場合、たんに概念的な関係で
 ある。
なぜなら、そのような関係は、理性が、何らかの二つの考察に即して見
 出した、あるものの自分自身に対するある種の秩序においてのみ成立するから
 である。
しかし、あるものが、数においてではなく、類や種の本性において同一と言われ
 るときは事情が異なる。それゆえ、ボエティウスは、神における諸関係を、あ
 らゆる点にかんして同一性になぞらえたのではなく、そのような関係によっては、同一
 性関係と同じように、実体が多様化されないことだけに即して、同一性になぞ
 らえたのである。




\\



{\scshape Ad tertium dicendum} quod, cum creatura procedat a Deo in diversitate
naturae, Deus est extra ordinem totius creaturae, nec ex eius natura est
eius habitudo ad creaturas. Non enim producit creaturas ex necessitate
suae naturae, sed per intellectum et per voluntatem, ut supra dictum
est. Et ideo in Deo non est realis relatio ad creaturas. Sed in
creaturis est realis relatio ad Deum, quia creaturae continentur sub
ordine divino, et in earum natura est quod dependeant a Deo. Sed
processiones divinae sunt in eadem natura. Unde non est similis ratio.


&

第三異論に対しては、次のように言われるべきである。
被造物は、本性の多様性において、神から発出するので、神は全被造物の秩序の
 外にあり、被造物への関係がその本性に含まれることもない。
前に述べられたとおり\footnote{ST I, q.14, a.8; q.19, a.4.}、自らの本性の
 必然性によってではなく、知性と意志によって、被造物を生みだしたからであ
 る。ゆえに、神において、被造物への実在的な関係はない。しかし、被造物に
 は、神への実在的な関係がある。なぜなら、被造物は神の秩序のもとに含まれ、
 その本性の中には、神に依存することがあるからである。しかし、神の諸々の発出は、
 同一の本性にある。ゆえに、両者は事情が異なる。


\\



{\scshape Ad quartum dicendum} quod relationes quae consequuntur solam operationem
intellectus in ipsis rebus intellectis, sunt relationes rationis tantum,
quia scilicet eas ratio adinvenit inter duas res intellectas. Sed
relationes quae consequuntur operationem intellectus, quae sunt inter
verbum intellectualiter procedens et illud a quo procedit, non sunt
relationes rationis tantum, sed rei, quia et ipse intellectus et ratio
est quaedam res, et comparatur realiter ad id quod procedit
intelligibiliter, sicut res corporalis ad id quod procedit
corporaliter. Et sic paternitas et filiatio sunt relationes reales in
divinis.


&

第四異論に対しては、以下のように言われるべきである。
知性認識された事物において、知性の働きのみに伴う関係は、たんに概念的なも
 のである。なぜなら、その関係を、理性が、二つの知性認識された事物の間に
 見出すからである。しかし、知性の働きに伴う関係、すなわち、知性的に発出する言葉と、そこから発出するもと
 のものとの関係は、たんに概念の関係ではなく、事物の関係である。なぜなら、
 知性も理性もある種の事物であり、可知的に発出するものに対して、実在的に
 関係するからである。ちょうど、物体的事物が、物体的に発出するものに関係
 するように。この意味で、父性と息子性は、神において実在的な関係である。

\end{longtable}
\newpage





\rhead{a.~2}
\begin{center}
 {\Large {\bf ARTICULUS SECUNDUS}}\\
 {\large UTRUM RELATIO IN DEO SIT IDEM QUOD SUA ESSENTIA}\\
 {\footnotesize I {\itshape Sent.}, d.33, a.1; IV {\itshape SCG},
 cap.14; {\itshape De Pot.}, q.8, a.2; {\itshape Quodl}.~VI, q.1;
 {\itshape Compend.~Theol.}, cap.54, 66, 67.}\\
 {\Large 第二項\\神における関係はその本質と同じか}
\end{center}

\begin{longtable}{p{21em}p{21em}}





{\Huge A}{\scshape d secundum sic proceditur}. Videtur quod relatio in Deo non sit idem
quod sua essentia. Dicit enim Augustinus, in V {\itshape de Trin}., quod {\itshape non omne
quod dicitur in Deo, dicitur secundum substantiam. Dicitur enim ad
aliquid, sicut pater ad filium, sed haec non secundum substantiam
dicuntur}. Ergo relatio non est divina essentia.


&

第二項の問題へ、議論は以下のように進められる。
神における関係は、その本質ではないと思われる。理由は以下の通り。
アウグスティヌスは、『三位一体論』第5巻で次のように言っている。「全て神
 において語られることが、実体に即して語られるわけではない。父や息子のよ
 うに、関係が語られることもある。しかし、これらは、実体に即して語られて
 いるのではない」。ゆえに、関係は、神の実体ではない。


\\




2. {\scshape Praeterea}, Augustinus dicit, VII {\itshape de Trin}., {\itshape omnis res quae relative
dicitur, est etiam aliquid excepto relativo; sicut homo dominus, et homo
servus}. Si igitur relationes aliquae sunt in Deo, oportet esse in Deo
aliquid aliud praeter relationes. Sed hoc aliud non potest esse nisi
essentia. Ergo essentia est aliud a relationibus.

&

さらに、アウグスティヌスは、『三位一体論』第7巻で、以下のように述べてい
 る。「関係的に語られる全てのものは、関係するものが取り除かれても、やは
 り何かである。たとえば、主人は人間であり、奴隷も人間である」。ゆえに、
 もし神の中に何らかの関係があるならば、関係以外にも、神の中に何かがなけ
 ればならない。しかし、これは、本質以外のものではありえない。ゆえに、本
 質は、関係と別である。



\\



3. {\scshape Praeterea}, esse relativi est ad aliud se habere, ut dicitur in
{\itshape Praedicamentis}. Si igitur relatio sit ipsa divina essentia, sequitur
quod esse divinae essentiae sit ad aliud se habere, quod repugnat
perfectioni divini esse, quod est maxime absolutum et per se subsistens,
ut supra ostensum est. Non igitur relatio est ipsa essentia divina.


&


さらに、『カテゴリー論』で言われるように、関係的なものの存在は、他のもの
 に関係することである。ゆえに、もし関係が神の本質それ自体だったならば、
 神の本質は、他のものの本質に関係することだっただろう。しかしこれは、前
 に示されたとおり\footnote{STI, q.3, a.4.}、最大限に独立で自体的に自存する神の存在の完全性に反す
 る。ゆえに、関係は神の本質そのものではない。


\\



{\scshape Sed contra}, omnis res quae non est divina essentia, est creatura. Sed
relatio realiter competit Deo. Si ergo non est divina essentia, erit
creatura, et ita ei non erit adoratio latriae exhibenda, contra quod in
Praefatione cantatur, {\itshape ut in personis proprietas, et in maiestate
adoretur aequalitas}.


&

しかし反対に、神の本質でない全てのものは被造物である。しかし、関係は、実
 在的に神に適合する。ゆえに、もしそれが神の本質でないならば、被造物であ
 ることになり、それに対してラトレイアの崇拝が示されるべきでないことにな
 ろう。しかしこれは、「序」で「ペルソナにおける固有性と、偉大さにおける
 等しさが崇められるように」と歌われるのに反する。


\\



{\scshape Respondeo dicendum} quod circa hoc dicitur Gilbertus Porretanus errasse,
sed errorem suum postmodum in Remensi Concilio revocasse. Dixit enim
quod relationes in divinis sunt assistentes, sive extrinsecus
affixae. 



&

解答する。以下のように言われるべきである。
これについては、ギルベルトゥス・ポッレターヌスが誤ったが、自らの誤りを、
後にランス公会議で取り消したと言われている。彼は、神における諸関係は、補助
 的、ないし外的に付け加えられたものと言った。


\\


Ad cuius evidentiam, considerandum est quod in quolibet novem
generum accidentis est duo considerare. Quorum unum est esse quod
competit unicuique ipsorum secundum quod est accidens. Et hoc communiter
in omnibus est inesse subiecto, accidentis enim esse est inesse. 



&

これを明らかにするために、以下のことが考察されるべきである。すなわち、附
 帯性の9つの類のそれぞれにおいて、二つのことを考えられる。
一つは、それぞれに、それが附帯性であることに即して適合する、存在である。
 そしてこれは、共通に、すべてのものにおいて「基体に内在する」ということ
 である。附帯性の存在は、内在なのだから。


\\


Aliud
quod potest considerari in unoquoque, est propria ratio uniuscuiusque
illorum generum. Et in aliis quidem generibus a relatione, utpote
quantitate et qualitate, etiam propria ratio generis accipitur secundum
comparationem ad subiectum, nam quantitas dicitur mensura substantiae,
qualitas vero dispositio substantiae. 


&

各々の類においてもう一つ考えられるのは、それら各々の類の独自の性格であ
 る。そして、量や性質など、関係以外の類においては、類の独自の性格もまた、
 基体への関係に即して理解される。量は、実体の尺度であり、性質は実体の状
 態である、というように。


\\



Sed ratio propria relationis non
accipitur secundum comparationem ad illud in quo est, sed secundum
comparationem ad aliquid extra. 



&


しかし、関係独自の性格は、そこにおいてあるところのもの(基体)への関係ではなく、
 何か外のものへの関係に即して理解される。

\\


Si igitur consideremus, etiam in rebus
creatis, relationes secundum id quod relationes sunt, sic inveniuntur
esse assistentes, non intrinsecus affixae; quasi significantes respectum
quodammodo contingentem ipsam rem relatam, prout ab ea tendit in
alterum. 



&

ゆえに、被造物においても、関係を、関係であるかぎりにおいて考察するならば、
それは補助的なものであり、内在的に付着するものではないことが見出される。
むしろそれは、関係づけられる事物に付随する、ある種の関係を、その事物から別の
 ものへ向かうものとして表示するものである。


\\



Si vero consideretur relatio secundum quod est accidens, sic
est inhaerens subiecto, et habens esse accidentale in ipso. Sed
Gilbertus Porretanus consideravit relationem primo modo tantum. 


&

他方、もし、関係を、それが附帯性であるかぎりで考察するならば、それは基体
 に内在するものであり、そこにおいて、附帯的存在をもつものである。
しかし、ギルベルトゥス・ポッレターヌスは、第一のしかたにおいてのみ、附帯
 性を考えた。


\\



Quidquid
autem in rebus creatis habet esse accidentale, secundum quod transfertur
in Deum, habet esse substantiale, nihil enim est in Deo ut accidens in
subiecto, sed quidquid est in Deo, est eius essentia. Sic igitur ex ea
parte qua relatio in rebus creatis habet esse accidentale in subiecto,
relatio realiter existens in Deo habet esse essentiae divinae, idem
omnino ei existens. 



&

さて、なんであれ被造物の中で附帯的存在をもつものが神へ転用される場合に
 は、実体的存在をもつので、神の中に、基体に附帯性が存在するようなしか
 たで存在するものは何もない。むしろ、神の中にあるものは、神の本質である。
ゆえに、関係が被造物において、基体の中に附帯的存在を持つという側面から言
 えば、実在的に神の中に存在する関係は、神の本質の存在をもち、あらゆる点
 で神の本質と同一である。

\\


In hoc vero quod ad aliquid dicitur, non
significatur aliqua habitudo ad essentiam, sed magis ad suum
oppositum. Et sic manifestum est quod relatio realiter existens in Deo,
est idem essentiae secundum rem; et non differt nisi secundum
intelligentiae rationem, prout in relatione importatur respectus ad suum
oppositum, qui non importatur in nomine essentiae. Patet ergo quod in
Deo non est aliud esse relationis et esse essentiae, sed unum et idem.


&

他方、関係と言われる点では、それは本質への何らかの関わりを表示するのでは
 なく、むしろ、対置されるものへの関わりを表示する。この意味で、神の中に
 実在的に存在する関係は、事物に即しては本質と同一であり、関係において、
 本質という名では意味されない、自らに対置されるものへの関わりが意味され
 る点で、知性の理解に即してのみ本質と異なる。ゆえに、神の中で、関係の存在と本
 質の存在は別でなく、一つであり同じであることが明らかである。

\\



{\scshape Ad primum ergo dicendum} quod verba illa Augustini non pertinent ad hoc,
quod paternitas, vel alia relatio quae est in Deo, secundum esse suum
non sit idem quod divina essentia; sed quod non praedicatur secundum
modum substantiae, ut existens in eo de quo dicitur, sed ut ad alterum
se habens. 


&

第一異論に対しては、それゆえ、以下のように言われるべきである。
アウグスティヌスのかの言葉は、父性その他の神の中の関係が、それ自身の存在
 に即して、神の本質と同一でないと言っているのではなく、実体のあり方に即
 して述語されるのではないと言っている。つまり、関係は、Aについて語られるそのAに
 内在するものとしてではなく、他のものへ関わるものとして述語される。


\\

Et propter hoc dicuntur duo tantum esse praedicamenta in
divinis. Quia alia praedicamenta important habitudinem ad id de quo
dicuntur, tam secundum suum esse, quam secundum proprii generis
rationem, nihil autem quod est in Deo, potest habere habitudinem ad id
in quo est, vel de quo dicitur, nisi habitudinem identitatis, propter
summam Dei simplicitatem.


&

このため、神の中にはただ二つのカテゴリーがあると言われる。なぜなら、他の
 カテゴリーは、自らの存在に即してだけでなく、類独自の性格に即しても、Aに
 ついて語られる、そのAへの関わり(=基体への関わり)を意味するが、神にお
 いては、その最高の単純性のために、基体への関わりは同一性以外にないから
 である。

\\



{\scshape Ad secundum dicendum} quod, sicut in rebus creatis, in illo quod dicitur
relative, non solum est invenire respectum ad alterum, sed etiam aliquid
absolutum, ita et in Deo, sed tamen aliter et aliter. 


&

第二異論に対しては、以下のように言われるべきである。
被造物で、関係的に言われるるものにおいて、他のものへの関わりだけでなく、
 何か独自のものが見出されるが、神においてもそれは同じである。しかし、そ
 れぞれ異なるしかたによる。

\\

Nam id quod
invenitur in creatura praeter id quod continetur sub significatione
nominis relativi, est alia res, in Deo autem non est alia res, sed una
et eadem, quae non perfecte exprimitur relationis nomine, quasi sub
significatione talis nominis comprehensa. 


&

すなわち、被造物において、関係的名称の意味のもとに含まれるもの以外のもの
 が見出されるのは、別の事物だが、神においては、別の事物ではなく、同一の
 事物であって、関係的名称によって、いわばそのような名称の意味に含まれた
 ものとして完全に表示されていないものである。

\\

Dictum est enim supra, cum de
divinis nominibus agebatur, quod plus continetur in perfectione divinae
essentiae, quam aliquo nomine significari possit. Unde non sequitur quod
in Deo, praeter relationem, sit aliquid aliud secundum rem; sed solum
considerata nominum ratione.


&

じっさい、前に、神の名について論じられたとき、神の本質の完全性には、その
 名によって表示されうるものよりも多くのことが含まれていると語られた。し
 たがって、神の中に、関係以外に、事物に即して他の何かであるようなものが
 存在することは帰結せず、むしろ、名称の概念においてのみそう考えられる。

\\



{\scshape Ad tertium dicendum} quod, si in perfectione divina nihil plus
contineretur quam quod significat nomen relativum, sequeretur quod esse
eius esset imperfectum, utpote ad aliquid aliud se habens, sicut si non
contineretur ibi plus quam quod nomine sapientiae significatur, non
esset aliquid subsistens. 


&

第三異論に対しては、以下のように言われるべきである。
もし、神の完全性に、関係的名称が表示する以上のことが含まれていないならば、
 神の存在は、他の何かに関わるものとして、不完全だっただろう。ちょうど、
 もし、知恵という名称で表示されるもの以上のものが含まれていなかったなら
 ば、自存する何かではなかっただろうように。


\\

Sed quia divinae essentiae perfectio est maior
quam quod significatione alicuius nominis comprehendi possit, non
sequitur, si nomen relativum, vel quodcumque aliud nomen dictum de Deo,
non significat aliquid perfectum, quod divina essentia habeat esse
imperfectum, quia divina essentia comprehendit in se omnium generum
perfectionem, ut supra dictum est.


&

むしろ、神の本質の完全性は、どんな名称の意味によっても包含されないほど
 大きいので、関係的名称、あるいは、なんであれ神について語られる他の名称
 が、何か完全なものを表示しないとしても、そのことから、神の本質が不
 完全な存在を持つことは帰結しない。神の本質は、前に述べられたとおり、あ
 らゆる類の完全性をそのうちに包含するからである。


\end{longtable}
\newpage




\rhead{a.~3}
\begin{center}
 {\Large {\bf ARTICULUS TERTIUS}}\\
 {\large UTRUM RELATIONES QUAE SUNT IN DEO, REALITER AB INVICEM DISTINGUANTUR}\\
 {\footnotesize I {\itshape Sent.}, d.26, q.2, a.2; {\itshape De Pot.},
 q.2, a.5, 6.}\\
 {\Large 第三項\\神の中にある諸関係は、相互に実在的に区別されるか}
\end{center}

\begin{longtable}{p{21em}p{21em}}





{\Huge A}{\scshape d tertium sic proceditur}. Videtur quod relationes quae sunt in Deo,
realiter ab invicem non distinguantur. Quaecumque enim uni et eidem sunt
eadem, sibi invicem sunt eadem. Sed omnis relatio in Deo existens est
idem secundum rem cum divina essentia. Ergo relationes secundum rem ab
invicem non distinguuntur.


&


第三の問題へ、議論は以下のように進められる。
神の中にある諸関係は、相互に実在的に区別されないと思われる。理由は以下の
 通り。
なんであれ、ある一つのものと同一である者どもは、相互に同一である
 \footnote{if $a=b$ and $a=c$, then $b=c$.}。しかし、神の中に存在するあ
 らゆる関係は、神の本質と、事物に即して同一である。ゆえに、諸関係は、事
 物に即して、相互に区別されない。

\\



2 {\scshape Praeterea}, sicut paternitas et filiatio secundum nominis rationem
distinguuntur ab essentia divina, ita et bonitas et potentia. Sed
propter huiusmodi rationis distinctionem non est aliqua realis
distinctio bonitatis et potentiae divinae. Ergo neque paternitatis et
filiationis.


&

さらに、ちょうど父性と息子性が、名称の概念に即して神の本質から区別される
 ように、善性と能力もまた、そのように区別される。しかし、このような概念
 の区別のために、神の善性と能力の実在的な区別のようなものは生じない。ゆ
 えに、父性と息子性もまたそうである。

\\



3 {\scshape Praeterea}, in divinis non est distinctio realis nisi secundum
originem. Sed una relatio non videtur oriri ex alia. Ergo relationes non
distinguuntur realiter ab invicem.


&

さらに、神において、実在的な区別は、起源に即して以外にはない。しかし、あ
 る関係が、他の関係を起源とするとは思われない。ゆえに、諸関係は、相互に
 実在的に区別されない。

\\



{\scshape Sed contra est} quod dicit Boetius, in libro {\itshape de Trin}., quod {\itshape substantia in
divinis continet unitatem, relatio multiplicat Trinitatem}. Si ergo
relationes non distinguuntur ab invicem realiter, non erit in divinis
Trinitas realis, sed rationis tantum, quod est Sabelliani erroris.


&

しかし反対に、ボエティウスは、『三位一体論』という書物の中で、「神におい
 て、実体は一性を保ち、関係は三性を生む」と述べている。ゆえに、もし、諸
 関係が実在的に相互に区別されないならば、神における三性は実在的でなく、
 たんに概念的であることになるが、これは、サベリウスの誤謬に属する。


\\



{\scshape Respondeo dicendum} quod ex eo quod aliquid alicui attribuitur, oportet
quod attribuantur ei omnia quae sunt de ratione illius, sicut cuicumque
attribuitur {\itshape homo}, oportet quod attribuatur ei esse rationale. De ratione
autem relationis est respectus unius ad alterum, secundum quem aliquid
alteri opponitur relative. 

&

解答する。以下のように言われるべきである。AにBが帰せられるとき、Bの性格に属す
 るすべてのものが、Aに帰せられる。たとえば、「人間」が帰せられるすべてのものに、
 「理性的である」ことが帰せられるように。ところで、関係の性格には、ある
 ものが別のものに関係的に対置されることに即して、ある
 ものの他のものへの関わりということが属する。



\\

Cum igitur in Deo realiter sit relatio, ut
dictum est, oportet quod realiter sit ibi oppositio. Relativa autem
oppositio in sui ratione includit distinctionem. Unde oportet quod in
Deo sit realis distinctio, non quidem secundum rem absolutam, quae est
essentia, in qua est summa unitas et simplicitas; sed secundum rem
relativam.


&

ゆえに、すでに述べられたとおり、神の中に実在的に関係があるので、そこには
 実在的に対置があるはずである。しかし、「関係的な対置」は、その性
 格の中に、「区別」を含む。したがって、神の中に実在的な区別がなければならな
 い。ただしそれは、無関係的な事物に即してではない。そのようなものは神の
 本質であり、そこには最高の一性と単純性がある。そうではなく、関係的な事
 物に即してである。


\\



{\scshape Ad primum ergo dicendum} quod, secundum philosophum in III {\itshape Physic}.,
argumentum illud tenet, quod quaecumque uni et eidem sunt eadem, sibi
invicem sunt eadem, in his quae sunt idem re et ratione, sicut tunica et
indumentum, non autem in his quae differunt ratione. 

&

第一異論に対しては、それゆえ、以下のように言われるべきである。
『自然学』第3巻における哲学者によれば、「同一のものに同一であるものどもは
 相互に同一である」が当てはまるのは、「服」と「衣」のように事物と概念に
 おいて同一のものどもであって、概念的に異なるものどもではない。


\\

Unde ibidem dicit
quod, licet actio sit idem motui, similiter et passio, non tamen
sequitur quod actio et passio sint idem, quia in actione importatur
respectus ut a quo est motus in mobili, in passione vero ut qui est ab
alio. 


&

このことから、同箇所で、彼は以下のように言っている。能動と運動は同一であり、受動と運動も同一だが、能動
 と受動が同一であることは帰結しない。なぜなら、能動においては、可動的な
 ものにおける運動の起源への関係が意味されるのに対して、受動においては、
 他によってあるものとしての運動が意味されるからである。


\\

Et similiter, licet paternitas sit idem secundum rem cum essentia
divina, et similiter filiatio, tamen haec duo in suis propriis
rationibus important oppositos respectus. Unde distinguuntur ab invicem.


&

同様に、父性は、事物に即して神の本質と同一であり、息子性も同様であるが、
 この二つはそれぞれに独自の概念において、対立的な関わりを意味する。した
 がって、相互に区別される。


\\



{\scshape Ad secundum dicendum} quod potentia et bonitas non important in suis
rationibus aliquam oppositionem, unde non est similis ratio.


&

第二異論に対しては、以下のように言われるべきである。
能力と善性は、その概念の中に、何らかの対置を含まない。ゆえに、同じ議論は
 成り立たない。

\\



{\scshape Ad tertium dicendum} quod, quamvis relationes, proprie loquendo, non
oriantur vel procedant ab invicem, tamen accipiuntur per oppositum
secundum processionem alicuius ab alio.


&

第三異論に対しては、以下のように言われるべきである。
諸関係は、厳密に言えば、相互に起源となったり発出したりしないが、あるもの
 が他のものから発出することに即して、対置によって理解される。

\end{longtable}
\newpage



\rhead{a.~4}
\begin{center}
 {\Large {\bf ARTICULUS QUARTUS}}\\
 {\large UTRUM IN DEO SINT TANTUM QUATUOR RELATIONES REALIS,\\SCILICET
 PATERNITAS, FILIATIO, SPIRATIO ET PROCESSIO}\\
 {\Large 第四項\\神の中には、父性、息子性、霊発、発出の四つだけ実在的関係があるか}
\end{center}

\begin{longtable}{p{21em}p{21em}}



{\Huge A}{\scshape d quartum sic proceditur}. Videtur quod in Deo non sint tantum quatuor
relationes reales, scilicet paternitas, filiatio, spiratio et
processio. Est enim considerare in Deo relationes intelligentis ad
intellectum, et volentis ad volitum, quae videntur esse relationes
reales, neque sub praedictis continentur. Non ergo sunt solum quatuor
relationes reales in Deo.


&


第四項の問題へ、議論は以下のように進められる。
神の中に、父性、息子性、霊発、発出の四つだけ実在的関係があるわけではない
 と思われる。理由は以下の通り。
神の中に、知性認識するものと知性認識されるもの、意志するものと意志される
 ものの関係を見て取ることができ、これらは実在的関係であり、また、上述の
 ものの中に含まれないと思われる。ゆえに、神の中に、実在的関係は四つだけ
 でないと思われる。

\\



2. {\scshape Praeterea}, relationes reales accipiuntur in Deo secundum processionem
intelligibilem verbi. Sed relationes intelligibiles multiplicantur in
infinitum, ut Avicenna dicit. Ergo in Deo sunt infinitae relationes
reales.


&


さらに、実在的関係は、神の中で、可知的な言葉の発出に即して理解される。し
 かし、アヴィセンナが言うように、可知的関係は無限に多様化される。ゆえに、
 神の中には、無数の実在的関係がある。

\\



3. {\scshape Praeterea}, ideae sunt in Deo ab aeterno, ut supra dictum est. Non autem
distinguuntur ab invicem nisi secundum respectum ad res, ut supra dictum
est. Ergo in Deo sunt multo plures relationes aeternae.


&


さらに、前に述べられたとおり、神の中には、永遠からイデアが存在する。し
 かし、前に述べられたとおり、それらは、事物に関係する限りでのみ、相互に
 区別される。ゆえに、神の中には、さらに多くの永遠の関係がある。

\\



4. {\scshape Praeterea}, aequalitas et similitudo et identitas sunt relationes
quaedam; et sunt in Deo ab aeterno. Ergo plures relationes sunt ab
aeterno in Deo, quam quae dictae sunt.


&

さらに、等しさと類似性と同一性は一種の関係であり、それらは神の中に永遠か
 ら存在する。ゆえに、述べられているのよりも多くの関係が、神の中に永遠か
 らある。


\\



{\scshape Sed contra}, videtur quod sint pauciores. Quia secundum philosophum, in
III {\itshape Physic}., eadem via est de Athenis ad Thebas, et de Thebis ad
Athenas. Ergo videtur quod pari ratione eadem sit relatio de patre ad
filium, quae dicitur paternitas, et de filio ad patrem, quae dicitur
filiatio. Et sic non sunt quatuor relationes in Deo.


&

しかし反対に、それより少ないと思われる。というのも、『自然学』第3巻の哲
 学者によれば、アテナイからテーバイへの道は、テーバイからアテナイへの道と同
 一である。ゆえに、同じ理由で、「父性」と言われる父から息子への関係と、
 「息子性」と言われる息子から父への関係は同一である。こうして、神の中に
 四つの関係があるわけではない。


\\



{\scshape Respondeo dicendum} quod, secundum philosophum, 
in V {\itshape Metaphys}., relatio
omnis fundatur vel supra quantitatem, ut duplum et dimidium; vel supra
actionem et passionem, ut faciens et factum, pater et filius, dominus et
servus, et huiusmodi. 


&

解答する。以下のように言われるべきである。
『形而上学』第5巻の哲学者によれば、全ての関係は、「二倍」や「半分」のよ
 うに量に基づくか、「するもの」と「されたもの」、「父」と「息子」、「主
 人」と「奴隷」などのように、能動受動に基づくかのどちらかである。



\\

Cum autem quantitas non sit in Deo (est enim {\itshape sine
quantitate magnus}, ut dicit Augustinus), relinquitur ergo quod realis
relatio in Deo esse non possit, nisi super actionem fundata. Non autem
super actiones secundum quas procedit aliquid extrinsecum a Deo, quia
relationes Dei ad creaturas non sunt realiter in ipso, ut supra dictum
est. 


&

ところで、神の中に量はないので(なぜなら、アウグスティヌスが言うように、「量なしに
 大きい」から)、神の中の実在的関係は、能動(作用)に基づくもの以外にはない。し
 かし、神から何か外のものへ出ていく作用に基づくことはない。なぜなら、前
 に述べられたとおり\footnote{STI, q.13, a.7.}、神から被造物への関係は、神において、実在的にあるわ
 けではないからである。



\\

Unde relinquitur quod relationes reales in Deo non possunt accipi,
nisi secundum actiones secundum quas est processio in Deo, non extra,
sed intra. Huiusmodi autem processiones sunt duae tantum, ut supra
dictum est, quarum una accipitur secundum actionem intellectus, quae est
processio verbi; alia secundum actionem voluntatis, quae est processio
amoris. 


&

したがって、神における実在的関係は、神の中の発出、それも外へではなく中へ
 の発出に即した作用において見出され、それ以外にはありえない。
そのような発出は、前に述べられたとおり、二つだけであり、その一つは知性の
 作用に即して理解される、言葉の発出であり、もう一つは、意志の作用に即し
 て理解される、愛の発出である。


\\


Secundum quamlibet autem processionem oportet duas accipere
relationes oppositas, quarum una sit procedentis a principio, et alia
ipsius principii. Processio autem verbi dicitur generatio, secundum
propriam rationem qua competit rebus viventibus. Relatio autem principii
generationis in viventibus perfectis dicitur {\itshape paternitas}, relatio vero
procedentis a principio dicitur {\itshape filiatio}. 


&


さらに、どんな発出に即しても、対置される二つの関係を理解しなければならな
 い。一つは根源から発出するものがもつ関係であり、もう一つは、その根源自
 体がもつ関係である。言葉の発出は、生成と言われるが、それは、生きている
 事物に適合する独自の性格に即したものである。また、完全な生物において、
 生成の根源がもつ関係は「父性」と言われ、根源から発出するものがもつ関係
 は「息子性」と言われる。


\\


Processio vero amoris non
habet nomen proprium, ut supra dictum est, unde neque relationes quae
secundum ipsam accipiuntur. Sed vocatur relatio principii huius
processionis {\itshape spiratio}; relatio autem procedentis, {\itshape processio}; quamvis
haec duo nomina ad ipsas processiones vel origines pertineant, et non ad
relationes.


&

他方、愛の発出は、前に述べられたとおり、独自の名称をもたない。それで、こ
 の発出に即して理解される関係も、独自の名称をもたない。しかし、この発出
 の根源がもつ関係は、「霊発」と呼ばれ、発出するものの関係は「発出」と呼
 ばれる。ただ、この二つの名称は、発出自体、あるいは起源に属し、関係に属するの
 ではないのだけれども。



\\



{\scshape Ad primum ergo dicendum} quod in his in quibus differt intellectus et
intellectum, volens et volitum, potest esse realis relatio et scientiae
ad rem scitam, et volentis ad rem volitam. Sed in Deo est idem omnino
intellectus et intellectum, quia intelligendo se intelligit omnia alia,
et eadem ratione voluntas et volitum. 

&

第一異論に対しては、それゆえ、以下のように言われるべきである。
知性と知性認識されたもの、意志するものと意志されたものが異なるものにおい
 ては、知の、知られた事物への関係や、意志するものの、意志されたものへの関
 係は、実在的なものでありうる。しかし、神においては、知性と知性認識された
 ものはまったく同一である。なぜなら、自分を知性認識することによって、他
 のすべてを知性認識するからである。意志と意志されたものについても同じこ
 とが言える。


\\


Unde in Deo huiusmodi relationes
non sunt reales, sicut neque relatio eiusdem ad idem. Sed tamen relatio
ad verbum est realis, quia verbum intelligitur ut procedens per actionem
intelligibilem, non autem ut res intellecta. Cum enim intelligimus
lapidem, id quod ex re intellecta concipit intellectus, vocatur verbum.


&

したがって、神において、このような関係は実在的でない。これは、自分自身へ
 の関係が実在的でないのと同じである。しかし、言葉への関係は実在的である。
 なぜなら、言葉は、知性認識された事物としてではなく、可知的な作用によっ
 て発出するものとして理解されるからである。じっさい、私たちが石ころを
 知性認識するとき、知性が、知性認識された事物に基づいて懐念するものが、言葉
 と呼ばれるのである。


\\



{\scshape Ad secundum dicendum} quod in nobis relationes intelligibiles in
infinitum multiplicantur, quia alio actu intelligit homo lapidem, et
alio actu intelligit se intelligere lapidem, et alio etiam intelligit
hoc intelligere, et sic in infinitum multiplicantur actus intelligendi,
et per consequens relationes intellectae. Sed hoc in Deo non habet
locum, quia uno actu tantum omnia intelligit.


&

第二異論に対しては、以下のように言われるべきである。
私たちにおいて、可知的な関係は無限に増える。なぜなら、ある作用によって、ある人が石を知
 性認識し、別の作用によって、自分が石を知性認識することを知性認識し、ま
 た別の作用によって、それを知性認識することを知性認識し、というふうに知
 性認識の作用が無限に増えていき、結果的に、知性認識された関係も無限に増
 えるからである。しかし、神はただ一つの作用で万物を知性認識するので、神
 においてこのようなことが起こる余地はない。

\\



{\scshape Ad tertium dicendum} quod respectus ideales sunt ut intellecti a
Deo. Unde ex eorum pluralitate non sequitur quod sint plures relationes
in Deo, sed quod Deus cognoscat plures relationes.


&

第三異論に対しては、以下のように言われるべきである。
イデアの関係は、神によって知性認識されたものに属するものとしてある。した
 がって、それらの複数性から、複数の関係が神の中にあることは帰結しない。
 むしろ、神が複数の関係を認識することが帰結するだけである。


\\



{\scshape Ad quartum dicendum} quod aequalitas et similitudo in Deo non sunt
relationes reales, sed rationis tantum, ut infra patebit.


&

第四異論に対しては、以下のように言われるべきである。
後に明らかになるとおり\footnote{STI, q.42, a.1, ad 4.}、等しさと類似は、神の中で実在的な関係ではなく、たんに概念的な関係である。



\\




{\scshape Ad quintum dicendum} quod via est eadem ab uno termino ad alterum, et e
converso; sed tamen respectus sunt diversi. Unde ex hoc non potest
concludi quod eadem sit relatio patris ad filium, et e converso, sed
posset hoc concludi de aliquo absoluto, si esset medium inter ea.


&

第四異論に対しては、以下のように言われるべきである。
道は、一つの端から他の端へと、その逆とで、同じ道である。しかし、関係は異
 なる。したがって、このことから、父から息子への関係と、その逆の関係が同
 一であると結論されることはできない。しかし、仮に、父と息子の間に何か中
 間のものがあったならば、そのなんらかの非関係的なものについて、この
 こと(父と息子の間に同一の何かがあること)が結論されえたであろう。


\end{longtable}
\newpage



\end{document}
