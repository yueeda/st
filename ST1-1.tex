\documentclass[10pt]{jsarticle} % use larger type; default would be 10pt
%\usepackage[utf8]{inputenc} % set input encoding (not needed with XeLaTeX)
%\usepackage[round,comma,authoryear]{natbib}
%\usepackage{nruby}
\usepackage{okumacro}
\usepackage{longtable}
%\usepqckage{tablefootnote}
\usepackage[polutonikogreek,english,japanese]{babel}
%\usepackage{amsmath}
\usepackage{latexsym}
\usepackage{color}

%----- header -------
\usepackage{fancyhdr}
\lhead{{\it Summa Theologiae} I, q.~1}
%--------------------

\bibliographystyle{jplain}

\title{{\bf PRIMA PARS}\\{\HUGE Summae Theologiae}\\Sancti Thomae
Aquinatis\\{\sffamily QUEAESTIO PRIMA}\\DE SACRA DOCTRINA, QUALIS SIT,
ET AD QUAE SE EXTENDAT}
\author{Japanese translation\\by Yoshinori {\sc Ueeda}}
\date{Last modified \today}


%%%% コピペ用
%\rhead{a.~}
%\begin{center}
% {\Large {\bf }}\\
% {\large }\\
% {\footnotesize }\\
% {\Large \\}
%\end{center}
%
%\begin{longtable}{p{21em}p{21em}}
%
%&
%
%
%\\
%\end{longtable}
%\newpage



\begin{document}
\maketitle
\pagestyle{fancy}

\begin{center}
{\Large 第一問\\神聖な教えについて、それはどのようなもので、どの範囲に及
 ぶか}
\end{center}

\begin{longtable}{p{21em}p{21em}}

Et ut intentio nostra sub aliquibus certis limitibus comprehendatur,
 necessarium est primo investigare de ipsa sacra doctrina, qualis sit,
 et ad quae se extendat. 

Circa quae quaerenda sunt decem. 

\begin{enumerate}
 \item de necessitate huius doctrinae.
 \item utrum sit scientia.
 \item utrum sit una vel plures.
 \item utrum sit speculativa vel practica.
 \item de comparatione eius ad alias scientias.
 \item utrum sit sapientia.
 \item quid sit subiectum eius. 
 \item utrum sit argumentativa.
 \item utrum uti debeat metaphoricis vel symbolicis locutionibus.
 \item utrum Scriptura sacra huius doctrinae sit secundum plures sensus exponenda.

\end{enumerate}

&

私たちが考究しようとすることがらが、ある一定の範囲内に収まるように、まず、
 神聖な教えそのものについて、それがどのようなものであり、どの範囲に及ぶかを探
 究する必要がある。

これを巡って、十のことが問われるべきである。

\begin{enumerate}
 \item この教えの必要性について。
 \item それは学か。
 \item それは一つか、あるいは複数か。
 \item それは観照的か実践的か。
 \item 他の学との関係について。
 \item それは知恵か。
 \item その主題は何か。
 \item それは推論的か。
 \item それは比喩的あるいは象徴的な語り方を用いるべきか。
 \item この教えの聖書は複数の意味で説明されるべきか。
\end{enumerate}

\\

\end{longtable}

\newpage

\rhead{a.~1}
\begin{center}
 {\Large {\bf ARTICULUS PRIMUS}}\\
 {\large UTRUM SIT NECESSARIUM, PRAETER PHILOSOPHICAS
 DISCIPLINAS,\\ALIAM DOCTRINAM HABERI}\\
 {\footnotesize II$^a$ II$^{ae}$, q.~2, a.~3, 4; I {\itshape Sent.},
 Prol., a.~1; I {\itshape SCG.}, c.~4, 5; {\itshape De Verit.}, q.~14, a.~10.}\\
 {\Large 第一項\\哲学的諸学とは別に、他の教えをもつことは必要か}
\end{center}

\begin{longtable}{p{21em}p{21em}}
{\huge A}{\scshape d primum sic proceditur}. Videtur quod non sit
necessarium, praeter philosophicas disciplinas, aliam doctrinam
haberi. Ad ea enim quae supra rationem sunt, homo non debet conari,
secundum illud {\itshape Eccli}.~{\scshape iii}, {\itshape altiora te ne
quaesieris}. Sed ea quae rationi subduntur, sufficienter traduntur in
philosophicis disciplinis. Superfluum igitur videtur, praeter
philosophicas disciplinas, aliam doctrinam haberi.

&

第一項の問題へ向けて、議論は次のように進められる。
哲学的諸学と別に、他の教えをもつ必要はないと考えられる。
なぜなら、『集会の書』3章「あなたより高いものを問うてはならない」 によれ
 ば、人間は、理性を越えることがらを知ろうとすべきでない。しかし、理性の
 もとにあることがらは、哲学的諸学の中で十分に論じられている。ゆえに、哲
 学的諸学意外に、他の教えをもつことは無駄であるように思える。

\\

{\scshape 2 Praeterea}, doctrina non potest esse nisi de ente, nihil
enim scitur nisi verum, quod cum ente convertitur. Sed de omnibus
entibus tractatur in philosophicis disciplinis, et etiam de Deo, unde
quaedam pars philosophiae dicitur theologia, sive scientia divina, ut
patet per philosophum in VI {\itshape Metaphys}. Non fuit igitur
necessarium, praeter philosophicas disciplinas, aliam doctrinam haberi.


&
さらに、教えは、有についての教え以外にありえない。なぜなら、真でなければ
 何も知られないが、真は有と置換されるからである。
しかし、すべての有について、哲学的諸学の中で論じられるし、神についてすら、
 論じられている。このことから、『形而上学』6巻の哲学者によって明らかなと
 おり、哲学のある部分は「神学」や「神の学」と言われる。ゆえに、哲学的諸
 学以外に、他の教えをもつ必要はなかった。


\\



Sed contra est quod dicitur II {\itshape ad Tim}.~{\scshape iii},
{\itshape omnis Scriptura divinitus inspirata utilis est ad docendum, ad
arguendum, ad corripiendum, ad erudiendum ad iustitiam}. Scriptura autem
divinitus inspirata non pertinet ad philosophicas disciplinas, quae sunt
secundum rationem humanam inventae. Utile igitur est, praeter
philosophicas disciplinas, esse aliam scientiam divinitus inspiratam.

&

しかし反対に、『テモテへの手紙二』3章で次のように言われている。「神の霊
 感によって書かれたすべての聖書は、教え、戒め、矯正し、正義へ導くため
 に有用である」。\footnote{「聖書はすべて神の霊の導きの下に書かれ、人を
 教え、戒め、誤りを正し、義に導く訓練をするうえに有益です。」(3:16)}ところで、神の霊感によって書かれた聖書は、哲学的諸学に
 属さない。哲学的諸学は、人間理性にしたがって見出されるものだからである。
 ゆえに、哲学的諸学以外に、神の霊感による他の学があることが有用である。

\\



{\scshape Respondeo dicendum} quod necessarium fuit ad
humanam salutem, esse doctrinam quandam secundum revelationem divinam,
praeter philosophicas disciplinas, quae ratione humana
investigantur. Primo quidem, quia homo ordinatur ad Deum sicut ad
quendam finem qui comprehensionem rationis excedit, secundum illud
{\itshape Isaiae} {\scshape lxiv}, {\itshape oculus non vidit Deus absque te, quae praeparasti
diligentibus te}. Finem autem oportet esse praecognitum hominibus, qui
suas intentiones et actiones debent ordinare in finem. Unde necessarium
fuit homini ad salutem, quod ei nota fierent quaedam per revelationem
divinam, quae rationem humanam excedunt. 

&

解答する。以下のように言われるべきである。
人間の救済のためには、人間理性によって探究される哲学的諸学以外に、神の啓
 示による何らかの教えが存在することが必要であった。
その理由の第一には、『イザヤ書』64章「神よ、あなたを愛する者たちに準備したものどもを、
 あなたなしに、目が見ることがなかった」\footnote{「あなたを待つ者に計らっ
 てくださる方は、神よ、あなたのほかにはありません。昔から、他に聞いた者
 も耳にした者も、目に見た者もありません」(64:3)}によれば、人間は、神を目的と
 して秩序づけられているが、その神は、理性の把握を越えている。しかし、自
 分の意図と行為を目的へ秩序づけなければならない人間には、目的があらかじ
 め知られていなければならない。したがって、人間には、救いのために、人間
 理性を越えることがらが、神の啓示を通して知られたものとなる必要があっ
 た。


\\





Ad ea etiam quae de Deo ratione
humana investigari possunt, necessarium fuit hominem instrui revelatione
divina. Quia veritas de Deo, per rationem investigata, a paucis, et per
longum tempus, et cum admixtione multorum errorum, homini proveniret, a
cuius tamen veritatis cognitione dependet tota hominis salus, quae in
Deo est. Ut igitur salus hominibus et convenientius et certius
proveniat, necessarium fuit quod de divinis per divinam revelationem
instruantur. 


&

神について、人間理性によって探究されうることがらについてもまた、神の啓示
 によって、人間が教えられることが必要であった。なぜなら、神についての真
 理は、少数の優れた人々によって理性を通して探究されたならば、長い時間を
 かけ、また多くの誤りを含みながら、人間のもとにやって来ただろうが、しか
 し、その神にかんする真理の認識に、人間の全救済がかかっているからである。
 その救済は神の中にあるのだから。ゆえに、救済が、人々に、より適切に確実
 にやってくるために、神のことがらについて、神の啓示を通して教えられる
 ことが必要であった。


\\




Necessarium igitur fuit, praeter philosophicas disciplinas,
quae per rationem investigantur, sacram doctrinam per revelationem
haberi.

&

ゆえに、理性によって探究される哲学的諸学以外に、啓示によって神聖な教えをもつこ
 とが必要であった。


\\



{\scshape Ad primum ergo dicendum} quod, licet ea quae sunt altiora
hominis cognitione, non sint ab homine per rationem inquirenda, sunt
tamen, a Deo revelata, suscipienda per fidem. Unde et ibidem subditur,
{\itshape plurima supra sensum hominum ostensa sunt tibi}. Et in
huiusmodi sacra doctrina consistit.

&

第一異論に対しては、それゆえ、次のように言われるべきである。
人間の認識より高いものは、人間によって理性を通して探究されるべきでないが、
 しかし、神によって啓示されるそのようなものは、信仰によって保持されるべ
 きである。したがって、同じ箇所で、「人間の感覚を越えた多くのものがあな
 たに示された」と、続いて述べられている。神聖な教えは、そのようなものにおいて、
 成立している。

\\



Ad secundum dicendum quod diversa ratio
cognoscibilis diversitatem scientiarum inducit. Eandem enim conclusionem
demonstrat astrologus et naturalis, puta quod terra est rotunda, sed
astrologus per medium mathematicum, idest a materia abstractum;
naturalis autem per medium circa materiam consideratum. Unde nihil
prohibet de eisdem rebus, de quibus philosophicae disciplinae tractant
secundum quod sunt cognoscibilia lumine naturalis rationis, et aliam
scientiam tractare secundum quod cognoscuntur lumine divinae
revelationis. Unde theologia quae ad sacram doctrinam pertinet, differt
secundum genus ab illa theologia quae pars philosophiae ponitur.


&

第二異論に対しては、次のように言われるべきである。
学の違いをもたらすのは、認識の性格[根拠]の違いである。
たとえば、天文学者と自然学者は、同じ結論、たとえば「地球は丸い」という結
 論を証明するが、天文学者は数学的な媒介[中項]、つまり、質料から抽象さ
 れたことがらによってそれを証明するが、他方で自然学者は、質料をめぐって
 考察された媒介[中項]によって証明する。
このことから、同じ事物について、哲学的諸学が、自然理性の光によって考察され
 うるかぎりで論じ、他方で別の学が、神の啓示の光によって認識されうるかぎ
 りで論じるとしても、それは差し支えない。
このことから、神聖な教えに含まれる神学は、哲学の部分とされる神学とは、類において
 異なる。


\end{longtable}

\newpage


\rhead{a.~2}
\begin{center}
 {\Large {\bf ARTICULUS SECUNDUS}}\\
 {\large UTRUM SACRA DOCTRINA SIT SCIENTIA}\\
 {\footnotesize II$^a$ II$^{ae}$, q.~1, a.~5, ad 2; I {\itshape Sent.},
 Prol., a.~3, q$^a$ 2; {\itshape De Verit.}, q.~14, a.~9, ad 3; in
 {\itshape Boet.~de Trin.}, q.~2, a.~2.}\\
 {\Large 第二項\\神聖な教えは学か}
\end{center}

\begin{longtable}{p{21em}p{21em}}

{\huge A}{\scshape d secundum sic proceditur}. Videtur quod sacra
doctrina non sit scientia. Omnis enim scientia procedit ex principiis
per se notis. Sed sacra doctrina procedit ex articulis fidei, qui non
sunt per se noti, cum non ab omnibus concedantur, {\itshape non enim
omnium est fides}, ut dicitur II {\itshape Thessalon}.~{\scshape
iii}. Non igitur sacra doctrina est scientia.


&

第二項の問題へ、議論は次のように進められる。
神聖な教えは学でないと思われる。なぜなら、すべて学は自明な原理から出発する。し
 かし、神聖な教えは、信仰箇条から出発するが、その信仰箇条はすべての人に認めら
 れているわけではない。『テサロニケの信徒への手紙二』3章で、「信仰はすべ
 ての人に属するわけではない」\footnote{「すべての人に、信仰があるわけで
 はないのです。」(3:2)}と言われるように。ゆえに、神聖な教えは学でない。

\\


{\scshape 2 Praeterea}, scientia non est singularium. Sed
sacra doctrina tractat de singularibus, puta de gestis Abrahae, Isaac et
Iacob, et similibus. Ergo sacra doctrina non est scientia.


&

さらに、学は個的なことがらにかかわらない。ところが、神聖な教えは、アブラハム、
 イサク、ヤコブ、その他、似た人々の行為についてなど、個的なことがらを扱
 う。ゆえに、神聖な教えは学でない。

\\


Sed contra est quod Augustinus dicit, XIV {\itshape de
Trinitate}: {\itshape Huic scientiae attribuitur illud tantummodo quo fides
saluberrima gignitur, nutritur, defenditur, roboratur}. Hoc autem ad
nullam scientiam pertinet nisi ad sacram doctrinam. Ergo sacra doctrina
est scientia.


&

しかし反対に、アウグスティヌスは『三位一体論』14巻で「この学には、もっと
 も健康的な信仰が、それによって生み出され、養われ、守られ、強化されるも
 のだけが帰せられる」と述べる。しかし、このことは、神聖な教え以外のどの学にも属さ
 ない。ゆえに、神聖な教えは学である。


\\


{\scshape Respondeo dicendum} sacram doctrinam esse
scientiam. Sed sciendum est quod duplex est scientiarum genus. Quaedam
enim sunt, quae procedunt ex principiis notis lumine naturali
intellectus, sicut arithmetica, geometria, et huiusmodi. Quaedam vero
sunt, quae procedunt ex principiis notis lumine superioris scientiae,
sicut perspectiva procedit ex principiis notificatis per geometriam, et
musica ex principiis per arithmeticam notis. Et hoc modo sacra doctrina
est scientia, quia procedit ex principiis notis lumine superioris
scientiae, quae scilicet est scientia Dei et beatorum. Unde sicut musica
credit principia tradita sibi ab arithmetico, ita doctrina sacra credit
principia revelata sibi a Deo.


&

解答する。次のように言われるべきである。
神聖な教えは学である。しかし、学の類に二つあることが知られるべきである。ある学
 は、自然的知性の光によって知られる原理から出発する。たとえば、算術、幾
 何学などがそれである。他方、ある学は、上位の学の光によって知られた原理
 から出発する。たとえば、透視図法は、幾何学によって知られた原理から始め
 るし、音楽は、算術学によって知られた原理から始める。そして、この後者の
 しかたで、神聖な教えは学である。なぜなら、上位の学、すなわち、神と至福者の知
 の光によって知られた原理から出発するからである。このことから、ちょうど
 音楽が、算術から自らに伝えられた原理を信じるように、神聖な教えは、神から自ら
 に啓示された原理を信じる。



\\


{\scshape Ad primum ergo dicendum} quod principia
cuiuslibet scientiae vel sunt nota per se, vel reducuntur ad notitiam
superioris scientiae. Et talia sunt principia sacrae doctrinae, ut
dictum est.


&

それゆえ、第一異論に対しては次のように言われるべきである。
どんな学の原理も、それ自体によって知られているか、あるいは、上位の学の知
 へ還元されるかのいずれかである。すでに述べられたとおり、神聖な教えの原理はそ
 のようなものである。

\\


{\scshape Ad secundum dicendum} quod singularia traduntur
in sacra doctrina, non quia de eis principaliter tractetur, sed
introducuntur tum in exemplum vitae, sicut in scientiis moralibus; tum
etiam ad declarandum auctoritatem virorum per quos ad nos revelatio
divina processit, super quam fundatur sacra Scriptura seu doctrina.


&

第二異論に対しては、次のように言われるべきである。
個的なものが神聖な教えの中で扱われるのは、それらについて主要に論じられているか
 らではない。個的なものが導入されるのは、倫理的な知においてそうであるよ
 うに、生の事例としてであり、さらにまた、神の啓示は、ある人々を通して私
 たちまでやって来たのだが、その人々の権威を明示するためである。聖書、あ
 るいは神聖な教えは、この神の啓示に基礎を持つ。




\end{longtable}
\newpage

\rhead{a.~3}
\begin{center}
 {\Large {\bf ARTICULUS TERTIUS}}\\
 {\large UTRUM SACRA DOCTRINA SIT UNA SCIENTIA}\\
 {\footnotesize Infra art.~3; I {\itshape Sent.}, Prol., a.~2, 4.}\\
 {\Large 第三項\\神聖な教えは一つの学か}
\end{center}

\begin{longtable}{p{21em}p{21em}}

{\huge A}{\scshape d tertium sic proceditur}. Videtur quod sacra
doctrina non sit una scientia. Quia secundum philosophum in I {\itshape
Poster}., {\itshape una scientia est quae est unius generis
subiecti}. Creator autem et creatura, de quibus in sacra doctrina
tractatur, non continentur sub uno genere subiecti. Ergo sacra doctrina
non est una scientia.

&

第三の問題へ向けて、以下のように議論が進められる。
神聖な教えは一つの学でないと思われる。なぜなら、『分析論後書』1巻の哲学者によ
 れば「一つの学とは、一つの類の主題を扱うものである」。しかし、神聖な教えの中
 で、創造者と被造物が論じられるが、この二つは主題の一つの類に含まれない。
 ゆえに、神聖な教えは一つの学でない。

\\


{\scshape 2 Praeterea}, in sacra doctrina tractatur de
Angelis, de creaturis corporalibus, de moribus hominum. Huiusmodi autem
ad diversas scientias philosophicas pertinent. Igitur sacra doctrina non
est una scientia.


&

さらに、神聖な教えの中では、天使について、物体的被造物について、人間の行いにつ
 いて論じられる。しかし、これらは異なる哲学的諸学に属する。ゆえに、神聖な教え
 は一つの学でない。

\\


{\scshape Sed contra} est quod sacra Scriptura de ea loquitur sicut de
una scientia, dicitur enim {\itshape Sap}.~{\scshape x}, {\itshape dedit
illi scientiam sanctorum}.


&
しかし反対に、聖書は、それ[=神聖な教え]について、一つの学であるかのように語っ
 ている。たとえば、『知恵の書』10章「彼に、聖なるものどもの知[単数形]を与えた」
 と言われるように。

\\


{\scshape Respondeo dicendum} sacram doctrinam unam
scientiam esse. Est enim unitas potentiae et habitus consideranda
secundum obiectum, non quidem materialiter, sed secundum rationem
formalem obiecti, puta homo, asinus et lapis conveniunt in una formali
ratione colorati, quod est obiectum visus. Quia igitur sacra Scriptura
considerat aliqua secundum quod sunt divinitus revelata, secundum quod
dictum est, omnia quaecumque sunt divinitus revelabilia, communicant in
una ratione formali obiecti huius scientiae. Et ideo comprehenduntur sub
sacra doctrina sicut sub scientia una.


&


解答する。以下のように言われるべきである。神聖な教えは一つの学である。なぜなら、
能力や習態の一性は、対象の観点から考察されるべきだが、その際、対象は、質
料的にではなく、対象の形相的性格に即して考察される。たとえば、人間、ロバ、
石は、「色があるもの」という形相的性格の点で一致するが、これは、視覚の対象で
ある。ゆえに、すでに述べられたことによれば、聖書は、神によって啓示された
ものであるという点で、何事についても考察するのだから、なんであれ神によっ
て啓示されうるすべてのものは、この学の対象の形相的性格において共通する。
ゆえに、一つの学としての神聖な教えの中に、それらは包含される。

\\


{\scshape Ad primum ergo dicendum} quod sacra doctrina
non determinat de Deo et de creaturis ex aequo, sed de Deo
principaliter, et de creaturis secundum quod referuntur ad Deum, ut ad
principium vel finem. Unde unitas scientiae non impeditur.


&

第一異論に対しては、次のように言われるべきである。
神聖な教えは、神と被造物とについて、等しく論じるのではなく、神について主要に論
 じ、被造物については、それらが根源や目的としての神に関係するかぎりで
 論じる。したがって、学の一性は損なわれない。
\\


{\scshape Ad secundum dicendum} quod nihil prohibet
inferiores potentias vel habitus diversificari circa illas materias,
quae communiter cadunt sub una potentia vel habitu superiori, quia
superior potentia vel habitus respicit obiectum sub universaliori
ratione formali. 


&

第二異論に対しては、次のように言われるべきである。
一つの上位の能力や習態のもとに共通にある題材について、下位の能力や習態が多
 様化することがあってもなんら問題ない。なぜなら、上位の能力や習態は、よ
 り普遍的な形相的性格のもとに、対象に関係するからである。

\\

Sicut obiectum sensus communis est sensibile, quod
comprehendit sub se visibile et audibile, unde sensus communis, cum sit
una potentia, extendit se ad omnia obiecta quinque sensuum. Et similiter
ea quae in diversis scientiis philosophicis tractantur, potest sacra
doctrina, una existens, considerare sub una ratione, inquantum scilicet
sunt divinitus revelabilia, ut sic sacra doctrina sit velut quaedam
impressio divinae scientiae, quae est una et simplex omnium.


&

たとえば、共通感覚の対象は「可感的なもの」だが、これは、「見られうるもの」
 と「聞かれうるもの」をその中に含む。したがって、共通感覚は、一つの能力
 だが、五つの感覚のすべての対象に及ぶ。同様に、さまざまな哲学的諸学の中
 で論じられるものを、神聖な教えが、一つのものでありながら、一つの性格のもとに、
 すなわち、それが神によって啓示されうるものであるかぎりで、考察すること
 が可能である。この意味で、神聖な教えは、万物について一つで単純である神の知の、
 いわばある種の刻印である。



\end{longtable}
\newpage



\rhead{a.~4}
\begin{center}
 {\Large {\bf ARTICULUS QUARTUS}}\\
 {\large UTRUM SACRA DOCTRINA SIT SCIENTIA PRACTICA}\\
 {\footnotesize I {\itshape Sent.}, Prol., a.3, qu$^a$1.}\\
 {\Large 第四項\\神聖な教えは実践的学か}
\end{center}

\begin{longtable}{p{21em}p{21em}}

{\Huge A}{\scshape d quartum sic proceditur}. Videtur quod sacra
 doctrina sit scientia practica. Finis enim practicae est operatio,
 secundum philosophum in II Metaphys. Sacra autem doctrina ad
 operationem ordinatur, secundum illud Iac. I, estote factores verbi, et
 non auditores tantum. Ergo sacra doctrina est practica scientia.



&

第四項の問題へ、議論は以下のように進められる。
神聖な教えは実践的学だと思われる。理由は以下の通り。
『形而上学』第2巻の哲学者によれば、実践的学の目的は働きである。しかし、
『ヤコブへの手紙』1章「言葉をただ聞いているだ
 けの人ではなく、行う人でありなさい」によれば、
 神聖な教えは、働きへ秩序付けられている。ゆえに、神聖な教えは実践的学である。


\\


{\scshape 2 Praeterea}, sacra doctrina dividitur per
legem veterem et novam. Lex autem pertinet ad scientiam moralem, quae
est scientia practica. Ergo sacra doctrina est scientia practica.


&

さらに、神聖な教えは、旧法と新法によって分かたれる。ところが、法は道徳学に属し、
 それは実践的学である。ゆえに神聖な教えは実践的学である。



\\


{\scshape Sed contra}, omnis scientia practica est de
rebus operabilibus ab homine; ut moralis de actibus hominum, et
aedificativa de aedificiis. Sacra autem doctrina est principaliter de
Deo, cuius magis homines sunt opera. Non ergo est scientia practica, sed
magis speculativa.


&

しかし反対に、すべての実践的額は、人間によってなされうる事柄にかかわる。
 たとえば、道徳学は人々の行為について、建築学は建築物についてである。
しかし、神聖な教えは、主として神についてであり、むしろ人間はその神の作品である。
 ゆえに、それは実践的学でなく、むしろ観照的学である。


\\


{\scshape Respondeo dicendum} quod sacra doctrina, ut
dictum est, una existens, se extendit ad ea quae pertinent ad diversas
scientias philosophicas, propter rationem formalem quam in diversis
attendit, scilicet prout sunt divino lumine cognoscibilia. Unde licet in
scientiis philosophicis alia sit speculativa et alia practica, sacra
tamen doctrina comprehendit sub se utramque; sicut et Deus eadem
scientia se cognoscit, et ea quae facit. Magis tamen est speculativa
quam practica, quia principalius agit de rebus divinis quam de actibus
humanis; de quibus agit secundum quod per eos ordinatur homo ad
perfectam Dei cognitionem, in qua aeterna beatitudo consistit.


&

解答する。以下のように言われるべきである。
神聖な教えは、すでに述べられたとおり、神の光によって認識可能なこと、という、さ
 まざまな事柄において見出される形相的性格のために、さまざまな哲学的諸学
 に属する事柄へ、一つのものとして及ぶ。したがって、哲学的諸学においては、
 感傷的学と実践的学は異なる学だが、神聖な教えはそのどちらも含む。ちょうどそれ
 は、神もまた、同一の知によって、自分自身と、自分が作るものを認識するよ
 うに。しかし、どちらかと言えば、それは実践的と言うよりは観照的である。
 なぜなら、より主要には、人間の行為よりは神の事柄について論じるからであ
 る。人間の行為については、それが、そこにおいて永遠の至福が成立するとこ
 ろの神の完全な認識へと人間が秩序付けられるかぎりにおいて論じる。


\\


Et per hoc patet responsio ad obiecta.


&

以上によって、異論への解答は明らかである。


\end{longtable}
\newpage


\rhead{a.~5}
\begin{center}
 {\Large {\bf ARTICULUS QUINTUS}}\\
 {\large UTRUM SACRA DOCTRINA SIT DIGNIOR ALIIS SCIENTIIS}\\
 {\footnotesize I-IIae, q.66, a.5, ad 3; I {\itshape Sent.}, Prol.~a.1;
 II {\itshape SCG}, cap.4.}\\
 {\Large 第五項\\神聖な教えは他の諸学よりも価値があるか}
\end{center}

\begin{longtable}{p{21em}p{21em}}

{\Huge A}{\scshape d quintum sic proceditur}. Videtur quod
sacra doctrina non sit dignior aliis scientiis. Certitudo enim pertinet
ad dignitatem scientiae. Sed aliae scientiae, de quarum principiis
dubitari non potest, videntur esse certiores sacra doctrina, cuius
principia, scilicet articuli fidei, dubitationem recipiunt. Aliae igitur
scientiae videntur ista digniores.


&

第五項の問題へ、議論は以下のように進められる。
神聖な教えは他の諸学より価値が高いわけではないと思われる。
理由は以下の通り。
学問の価値には確実性が関係する。しかし、その原理が疑いえない他の諸学は、
 信仰箇条という、疑いを容れる原理をもつ神聖な教えよりも、確実だと思われる。
ゆえに、他の諸学の方が、これよりも価値があると思われる。


\\


{\scshape 2 Praeterea}, inferioris scientiae est a
superiori accipere, sicut musicus ab arithmetico. Sed sacra doctrina
accipit aliquid a philosophicis disciplinis, dicit enim Hieronymus in
epistola {\itshape Ad Magnum Oratorem Urbis Romae}, quod doctores antiqui {\itshape intantum
philosophorum doctrinis atque sententiis suos resperserunt libros, ut
nescias quid in illis prius admirari debeas, eruditionem saeculi, an
scientiam Scripturarum}. Ergo sacra doctrina est inferior aliis
scientiis.


&

さらに、音楽が算術から〔原理を〕受け取るように、上位の学から
 〔原理を〕受け取ることは、下位の学に属する。しかし、神聖な教えが、哲学的諸学
 から受け取るものがある。たとえば、ヒエロニュムスは、『ローマの偉大な弁論
 家宛書簡』で、古代の学者たちは「哲学者たちの教説や文章を、自分たちの本
 の中にまき散らしたので、世俗の学問と聖書の学知と、どちらが先に賞賛され
 るべきか、あなたが途方に暮れるほどです」と書いている。ゆえに、神聖な教えは、他の諸学
 よりも下位である。



\\


{\scshape Sed contra est} quod aliae scientiae dicuntur
ancillae huius, {\itshape Prov}.~{\scshape ix}, {\itshape misit ancillas suas vocare ad arcem}.


&


しかし反対に、他の諸学は、この学のメイドと言われる。『箴言』第9章「自分
 のメイドたちを城に招くと、彼は伝えた」。

\\


{\scshape Respondeo dicendum} quod, cum ista scientia
quantum ad aliquid sit speculativa, et quantum ad aliquid sit practica,
omnes alias transcendit tam speculativas quam practicas. 

&

解答する。以下のように言われるべきである。
この学は、ある点にかんして観照的であり、別の点にかんして実践的であるから、
鑑賞的諸学も実践的諸学も、他のすべての学を超越する。


\\


Speculativarum
enim scientiarum una altera dignior dicitur, tum propter certitudinem,
tum propter dignitatem materiae. 
Et quantum ad utrumque, haec scientia
alias speculativas scientias excedit. Secundum certitudinem quidem, quia
aliae scientiae certitudinem habent ex naturali lumine rationis humanae,
quae potest errare, haec autem certitudinem habet ex lumine divinae
scientiae, quae decipi non potest. Secundum dignitatem vero materiae,
quia ista scientia est principaliter de his quae sua altitudine rationem
transcendunt, aliae vero scientiae considerant ea tantum quae rationi
subduntur. 



&


観照的諸学の中で、ある学問が他の学問より価値があると言われるのは、一つには
 確実性のためであり、もう一つには題材の価値のためにである。
そしてこのどちらについても、この学は、他の観照的諸学にまさる。
確実性に即しては、他の諸学は人間理性の自然的な光に基づいて確実性を持つが、
その光は誤りうる。それに対してこの学は、神の知の光に基づいて確実性を持ち、
 その光は欺かれ得ない。また、題材に即して言えば、この学は、主に、その高
 さによって理性を越えるものにかんしてあるが、他の諸学は、理性に服するも
 のだけを考察する。

\\


Practicarum vero scientiarum illa dignior est, quae ad
ulteriorem finem ordinatur, sicut civilis militari, nam bonum exercitus
ad bonum civitatis ordinatur. Finis autem huius doctrinae inquantum est
practica, est beatitudo aeterna, ad quam sicut ad ultimum finem
ordinantur omnes alii fines scientiarum practicarum. Unde manifestum
est, secundum omnem modum, eam digniorem esse aliis.


&

他方、実践的諸学の中で、より価値が高いのは、より究極的な目的へ秩序付けら
 れているものである。たとえば、政治学が兵学より価値があるのは、軍隊の善
 は国家の善に秩序付けられているからである。ところで、この教えの目的は、
 それが実践的学である限りにおいて、永遠の至福であり、他のすべての実践的
 学問の目的は、究極
 目的として、それへと秩序付けられている。したがって、あらゆる意味におい
 て、これがその他のものより価値があることが明らかである。


\\


{\scshape Ad primum ergo dicendum} quod nihil prohibet id
quod est certius secundum naturam, esse quoad nos minus certum, propter
debilitatem intellectus nostri, qui {\itshape se habet ad manifestissima naturae,
sicut oculus noctuae ad lumen solis}, sicut dicitur in II {\itshape Metaphys}. Unde
dubitatio quae accidit in aliquibus circa articulos fidei, non est
propter incertitudinem rei, sed propter debilitatem intellectus
humani. Et tamen minimum quod potest haberi de cognitione rerum
altissimarum, desiderabilius est quam certissima cognitio quae habetur
de minimis rebus, ut dicitur in XI {\itshape De Animalibus}.


&


第一異論に対しては、それゆえ、以下のように言われるべきである。
『形而上学』第2巻で言われるように、私たちの知性は、「自然のもっとも明ら
 かな事柄へ、あたかもフクロウの目が太陽の光に関係するように関係する」の
 で、本性に即してより確実なものが、私たちにとってより確実でないというこ
 とが、私たちの知性の弱さのために起こりうる。したがって、信仰箇条をめぐっ
 て、ある人々の中に生じている疑いは、実際の不確実性のせいではなく、人
 間知性の弱さのせいである。しかし、『動物論』第11巻で言われるように、最
 高の諸事物の認識について持ちうる最少のことは、最少の諸事物について持ち
 うる最高に確実な認識よりも、望ましいものである。


\\


{\scshape Ad secundum dicendum} quod haec scientia
accipere potest aliquid a philosophicis disciplinis, non quod ex
necessitate eis indigeat, sed ad maiorem manifestationem eorum quae in
hac scientia traduntur. Non enim accipit sua principia ab aliis
scientiis, sed immediate a Deo per revelationem. Et ideo non accipit ab
aliis scientiis tanquam a superioribus, sed utitur eis tanquam
inferioribus et ancillis; sicut architectonicae utuntur
subministrantibus, ut civilis militari. Et hoc ipsum quod sic utitur
eis, non est propter defectum vel insufficientiam eius, sed propter
defectum intellectus nostri; qui ex his quae per naturalem rationem (ex
qua procedunt aliae scientiae) cognoscuntur, facilius manuducitur in ea
quae sunt supra rationem, quae in hac scientia traduntur.


&


第二異論に対しては、以下のように言われるべきである。
この学が、哲学的諸学から受け取りうるものがあるのは、必要性に基づいてそれ
 を必要とするからではなく、この学で扱われる事柄をより明らかに示すために
 である。じっさい、その諸原理を他の諸学からではなく、直接に神から啓示を
 通して受け取るので、上位のものとしての他の諸学から受け取るのではなく、
 むしろそれらを下位のもの、メイドとして使う。ちょうど、建築的諸学がそれ
 に仕える諸学を用いるように。政治学が兵学を用いる場合がそれである。そし
 て、それらをそのように用いることは、それの欠点や不十分さのためではなく、
 私たちの知性の欠陥のためである。人間知性は、自然的理性(他の諸学はここ
 から出発する)をとおして認識するものから、理性を越えるものへと導かれる
 方がわかりやすいのだが、この後者を、この学は扱う。

\end{longtable}
\newpage



\rhead{a.~6}
\begin{center}
 {\Large {\bf ARTICULUS SEXTUS}}\\
 {\large UTRUM HAEC DOCTRINA SIT SAPIENTIA}\\
 {\footnotesize I {\itshape Sent.}, Prol., a.3, qu$^a$1, 3; II, Prol.,
 princ.; II {\itshape SCG}, cap.4.}\\
 {\Large 第六項\\この教えは知恵か}
\end{center}

\begin{longtable}{p{21em}p{21em}}



{\Huge A}{\scshape d sextum sic proceditur}. Videtur quod haec
doctrina non sit sapientia. Nulla enim doctrina quae supponit sua
principia aliunde, digna est nomine sapientiae, quia {\itshape sapientis est
ordinare, et non ordinari (I Metaphys.}). Sed haec doctrina supponit
principia sua aliunde, ut ex dictis patet. Ergo haec doctrina non est
sapientia.


&

第六項の問題へ、議論は以下のように進められる。
この教えは知恵でないと思われる。理由は以下の通り。
「知恵には、秩序付けられることではなく、秩序
 付けることが属する」(『形而上学』第1巻)ので、どこか他のところから原理を
 受け取るどんな教えも、知恵の名に値しない。しかし、この教えは、すでに述
 べられたことから明らかなとおり、どこか他のところからその原理を受け取る。
ゆえに、この教えは知恵でない。

\\


{\scshape 2 Praeterea}, ad sapientiam pertinet probare
principia aliarum scientiarum, unde {\itshape ut caput} dicitur scientiarum, ut VI
{\itshape Ethic}.~patet. Sed haec doctrina non probat principia aliarum
scientiarum. Ergo non est sapientia.


&

さらに、知恵には、他の諸学の原理を証明することが属する。それゆえ、『ニコ
 マコス倫理学』第6巻で明らかなとおり、「頭のようなもの」と言われる。しか
 し、この教えは、他の諸学の原理を証明しない。ゆえに、知恵ではない。


\\


{\scshape 3 Praeterea}, haec doctrina per studium
acquiritur. Sapientia autem per infusionem habetur, unde inter septem
dona spiritus sancti connumeratur, ut patet {\itshape Isaiae} {\scshape xi}. Ergo haec
doctrina non est sapientia.


&

さらに、この教えは、勉強を通して得られる。しかし、『イザヤ書』第11章で明
 らかなとおり、知恵は注入によって得られ、それゆえに、聖霊の七つの贈り物
 のなかに数えられる。ゆえに、この教えは知恵でない。


\\


{\scshape Sed contra est} quod dicitur {\itshape Deut}.~{\scshape iv}, in
principio legis, {\itshape haec est nostra sapientia et intellectus coram populis}.

&

しかし反対に、『申命記』第4章の最初の法で「これは、人々を前にした私たち
 の知恵、知性である」と言われている。

\\


{\scshape Respondeo dicendum} quod haec doctrina maxime
sapientia est inter omnes sapientias humanas, non quidem in aliquo
genere tantum, sed simpliciter. Cum enim sapientis sit ordinare et
iudicare, iudicium autem per altiorem causam de inferioribus habeatur;
ille sapiens dicitur in unoquoque genere, qui considerat causam
altissimam illius generis. Ut in genere aedificii, artifex qui disponit
formam domus, dicitur sapiens et architector, respectu inferiorum
artificum, qui dolant ligna vel parant lapides, unde dicitur I {\itshape Cor}.~{\scshape iii},
ut sapiens architector fundamentum posui. 


&


解答する。以下のように言われるべきである。
この教えは、人間のすべての知恵の中で、なんらかの類においてだけでなく、端
 的に、最高度に知恵である。理由は以下の通り。
知者には、秩序付けることと判断することが属するが、判断は、より高い原因を
 通して、より下位のものについて持たれるので、どの類においても、その類の
 最高の原因を考察するものが、知者と言われる。たとえば、建築の類では、木
 材を削り、石を準備する下位の職人たちとの関係において、家の形を整える建
 築家が、知者であり棟梁と言われる。このことから、『コリントの信徒への手紙
 一』第3章で、「知者であり棟梁である者として、私は基礎を置いた」と言われ
 る。


\\


Et rursus, in genere totius
humanae vitae, prudens sapiens dicitur, inquantum ordinat humanos actus
ad debitum finem, unde dicitur {\itshape Prov}.~{\scshape x}, {\itshape sapientia est viro
prudentia}. Ille igitur qui considerat simpliciter altissimam causam
totius universi, quae Deus est, maxime sapiens dicitur, unde et
sapientia dicitur esse {\itshape divinorum cognitio}, ut patet per Augustinum, XII
{\itshape de Trinitate}. 


&


さらに、人間の生全体の類において、思慮が知恵と言われるが、それは、人間の
 行為をしかるべき目的へ秩序付ける限りにおいてである。それゆえ、『箴言』
 第10章で「人にとって思慮は知恵である」と言われる。
ゆえに、端的に、世界全体の最高原因を考察する者、すなわち神が、最高度に知
 者と言われる。それゆえ、アウグスティヌス『三位一体論』第12巻で明らかな
 とおり、知恵は、「神的な事柄についての認識」とも言われる。

\\


Sacra autem doctrina propriissime determinat de Deo
secundum quod est altissima causa, quia non solum quantum ad illud quod
est per creaturas cognoscibile (quod philosophi cognoverunt, ut dicitur
{\itshape Rom}.~{\scshape i}, {\itshape quod notum est Dei, manifestum est illis}); sed etiam quantum ad
id quod notum est sibi soli de seipso, et aliis per revelationem
communicatum. Unde sacra doctrina maxime dicitur sapientia.


&

しかし、もっとも固有の意味では、神聖な教えは、神について、それが最高原因である
 かぎりにおいて、論じる。なぜなら、それは、被造物をとおして認識されうる
 ものだけでなく(『ローマの信徒への手紙』第1章「神について知られることは、
 彼らに明らかである」によれば、哲学者たちがこれらを認識する)、それ自体
 についてそれ自体で知られることや、啓示によって他の人々に伝えられるもの
 にかんしても論じるからである。したがって、神聖な教えは、最大限に知恵と言われ
 る。


\\


{\scshape Ad primum ergo dicendum} quod sacra doctrina
non supponit sua principia ab aliqua scientia humana, sed a scientia
divina, a qua, sicut a summa sapientia, omnis nostra cognitio ordinatur.


&

第一異論に対しては、それゆえ、以下のように言われるべきである。
神聖な教えは、人間的な学の何かから自らの原理を受け取るのではなく、神的な学から
 である。私たちのすべての認識は、最高の知恵によって秩序付けられるように
 して、それによって秩序付けられている。


\\


{\scshape Ad secundum dicendum} quod aliarum scientiarum
principia vel sunt per se nota, et probari non possunt, vel per aliquam
rationem naturalem probantur in aliqua alia scientia. Propria autem
huius scientiae cognitio est, quae est per revelationem, non autem quae
est per naturalem rationem. 


&

第二異論に対しては、以下のように言われるべきである。
他の諸学の原理は、自明であり証明されえないか、あるいは、他の何かの学の中
 で、自然的理性によって証明されるかである。しかし、この学の認識に固有な
 のは、それが啓示によってあり、自然理性によってあるのではないということ
 である。
%% 学の原理が啓示によって得られるとはどういうこと?

\\

Et ideo non pertinet ad eam probare
principia aliarum scientiarum, sed solum iudicare de eis, quidquid enim
in aliis scientiis invenitur veritati huius scientiae repugnans, totum
condemnatur ut falsum, unde dicitur II {\itshape Cor}.~{\scshape x}, {\itshape consilia destruentes, et
omnem altitudinem extollentem se adversus scientiam Dei}.


&

ゆえに、他の諸学の原理を証明することは、この学に属さず、ただ、それらにつ
 いて判断することが属する。つまり、なんであれ、この学の真理に反するもの
 が他の諸学に見出されるならば、その全体が偽と断罪される。このことから、
 『コリントの信徒への手紙二』第10章で、「思慮と、神の知に反対して自らを高
 めるすべての高みをを破壊し」と言われる。


\\


{\scshape Ad tertium dicendum} quod, cum iudicium ad
sapientem pertineat, secundum duplicem modum iudicandi, dupliciter
sapientia accipitur. 

&

第三異論に対しては、以下のように言われるべきである。
判断は、知者に属するので、二つの判断のしかたに即して、二通りに知恵が理解
 される。


\\

Contingit enim aliquem iudicare, uno modo per modum
inclinationis, sicut qui habet habitum virtutis, recte iudicat de his
quae sunt secundum virtutem agenda, inquantum ad illa inclinatur, unde
et in X {\itshape Ethic}.~dicitur quod virtuosus est mensura et regula actuum
humanorum. 

&

すなわち、ある人が判断するとき、一つには、傾向性によって判断することがあ
 る。たとえば、徳の習慣を持つ人は、為されるべき徳に即してある事柄につい
 て、それへと傾く限りにおいて正しく判断する。このことから、『ニコマコス
 倫理学』第10巻で、「有徳な人は、人々の行為の尺度であり基準である」と言
 われている。


\\

Alio modo, per modum cognitionis, sicut aliquis instructus in
scientia morali, posset iudicare de actibus virtutis, etiam si virtutem
non haberet. 

&

もう一つには、認識のしかたによる。たとえば、道徳学において訓練された人が、
徳を持っていなくても、有徳な行為について判断するように。

\\

Primus igitur modus iudicandi de rebus divinis, pertinet ad
sapientiam quae ponitur donum spiritus sancti secundum illud I Cor. II,
spiritualis homo iudicat omnia, etc., et Dionysius dicit, II cap. de
divinis nominibus, Hierotheus doctus est non solum discens, sed et
patiens divina. 

&

それゆえ、神的な事柄について判断する第一のしかたは、かの『コリントの信徒
 への手紙一』第2章「霊的な人はすべてを判断する云々」や、ディオニュシウス
 が『神名論』第2章で「ヒエロテウスは神的な事柄を学ぶだけでなく受け取って
 学識がある」 と述べていることにしたがえば、聖霊の贈り物とされる。


\\

Secundus autem modus iudicandi pertinet ad hanc
doctrinam, secundum quod per studium habetur; licet eius principia ex
revelatione habeantur.


&

これに対して、判断する第二のしかたは、勉強によって所有される限りで、この
 学に属する。ただし、その原理は啓示によって所有されるのだが。


\end{longtable}
\newpage

\rhead{a.~7}
\begin{center}
 {\Large {\bf ARTICULUS SEPTIMUS}}\\
 {\large UTRUM DEUS SIT SUBIECTUM HUIUS SCIENTIAE}\\
 {\footnotesize Supra, a.3, ad 1; I {\itshape Sent.}, Prol., a.4; in
 Boet.~{\itshape De Trin.}, q.5, a.4.}\\
 {\Large 第七項\\この学の主題は神か}
\end{center}

\begin{longtable}{p{21em}p{21em}}


{\Huge A}{\scshape d septimum sic proceditur}. Videtur quod
Deus non sit subiectum huius scientiae. In qualibet enim scientia
oportet supponere de subiecto {\itshape quid est}, secundum philosophum in I
{\itshape Poster}. Sed haec scientia non supponit de Deo quid est, dicit enim
Damascenus, {\itshape in Deo quid est, dicere impossibile est}. Ergo Deus non est
subiectum huius scientiae.


&

第七項の問題へ、議論は以下のように進められる。
神はこの学の主題でないと思われる。理由は以下の通り。
『分析論後書』第1巻の哲学者によれば、どの学でも、その主題について、それ
 が何かを提示しなければならない。しかし、この学は、神について、それが何
 かを提示しない。なぜなら、ダマスケヌスは、「神において、何であるかを語
 ることは不可能である云々」と述べているからである。ゆえに神はこの学の主
 題ではない。


\\


{\scshape 2 Praeterea}, omnia quae determinantur in
aliqua scientia, comprehenduntur sub subiecto illius scientiae. Sed in
sacra Scriptura determinatur de multis aliis quam de Deo, puta de
creaturis, et de moribus hominum. Ergo Deus non est subiectum huius
scientiae.


&

さらに、すべてなんらかの学の中で論じられることは、その学の主題のもとに含
 まれる。しかし、聖書の中で、神以外の多くのこと、たとえば被造物や人間の
 行いなどについて論じられる。ゆえに神はこの学の主題ではない。


\\


{\scshape Sed contra}, illud est subiectum scientiae, de
quo est sermo in scientia. Sed in hac scientia fit sermo de Deo, dicitur
enim theologia, quasi sermo de Deo. Ergo Deus est subiectum huius
scientiae.


&


しかし反対に、それについて学の中で教説があるものが、学の主題である。しか
 し、この学においては、神についての教説がなされ、神についての教説、とい
 う意味で、神学と言われる。ゆえに、神はこの学の主題である。


%% このSed Contraが正しいならば、Sacra doctrinaとは神学のこと。つまり神
%% 学大全のこと。

\\


{\scshape Respondeo dicendum} quod Deus est subiectum
huius scientiae. Sic enim se habet subiectum ad scientiam, sicut
obiectum ad potentiam vel habitum. Proprie autem illud assignatur
obiectum alicuius potentiae vel habitus, sub cuius ratione omnia
referuntur ad potentiam vel habitum, sicut homo et lapis referuntur ad
visum inquantum sunt colorata, unde coloratum est proprium obiectum
visus. 



&

解答する。以下のように言われるべきである。
神はこの学の主題である。理由は以下の通り。
主題と学の関係は、対象と能力ないし習態との関係と同じである。
ところで、ある能力ないし習態の対象に指定されるのは、それの性格のもとに、
 すべてがその能力や習態に関係づけられるものである。たとえば、
人間と石は「色があるもの」である限りで視覚に関係づ
 けられるので、視覚の対象は「色があるもの」であるように。


\\


Omnia autem pertractantur in sacra doctrina sub ratione Dei, vel
quia sunt ipse Deus; vel quia habent ordinem ad Deum, ut ad principium
et finem. Unde sequitur quod Deus vere sit subiectum huius
scientiae. 




&

ところで、すべてのものは、神の性格のもとに、あるいは神であるために、ある
 いは、根源や目的としての神に秩序付けられるかぎりで、神聖な教え
 の中で論じられる。したがって、神は、真に、この学の主題であることが帰結
 する。



\\

Quod etiam manifestum fit ex principiis huius scientiae, quae
sunt articuli fidei, quae est de Deo, idem autem est subiectum
principiorum et totius scientiae, cum tota scientia virtute contineatur
in principiis. 


&

このことは、この学の原理からも明らかとされる。すなわち、この学の原理は信
 仰箇条であるが、これは神にかんする。ところが、原理と学全体の主題は同一
 である。なぜなら、学全体は、原理の中に潜在的に含まれているからである。


\\

Quidam vero, attendentes ad ea quae in ista scientia
tractantur, et non ad rationem secundum quam considerantur,
assignaverunt aliter subiectum huius scientiae, vel res et signa; vel
opera reparationis; vel totum Christum, idest caput et membra. De
omnibus enim istis tractatur in ista scientia, sed secundum ordinem ad
Deum.


&

しかし、ある人々は、それに即して考察される性格ではなく、この学で論じられ
 ることに注目して、この学の主題を違うかたちで指定した。事柄としるしであ
 るとか、償いの業であるとか、キリストの全体、すなわち、頭と四肢であると
 か。じっさい、これらすべてについて、この学の中で論じられるが、それは神
 への秩序に即してである。


\\


{\scshape Ad primum ergo dicendum} quod, licet de Deo non
possimus scire quid est, utimur tamen eius effectu, in hac doctrina, vel
naturae vel gratiae, loco definitionis, ad ea quae de Deo in hac
doctrina considerantur, sicut et in aliquibus scientiis philosophicis
demonstratur aliquid de causa per effectum, accipiendo effectum loco
definitionis causae.


&

第一異論に対しては、それゆえ、以下のように言われるべきである。
神について、私たちはその何であるかを知ることができないが、しかし、私たち
 はこの教えの中で、その結果を、自然の結果であれ恩恵の結果であれ、この教
 えの中で神について考察される事柄のために、定義の代わりに用いる。それは
 ちょうど、ある哲学的諸学の中で、原因の定義の代わりに結果を当てはめるこ
 とによって、ある事柄が結果を通して原因について論証されるようにである。


\\


{\scshape Ad secundum dicendum} quod omnia alia quae
determinantur in sacra doctrina, comprehenduntur sub Deo, non ut partes
vel species vel accidentia, sed ut ordinata aliqualiter ad ipsum.


&

第二異論に対しては、以下のように言われるべきである。
神聖な教えの中で論じられる他のすべてのことは、部分や種や附帯性として神のもとに
 含まれるのではなく、何らかのかたちで、神へと秩序付けられるものとしてで
 ある。


\end{longtable}
\newpage




\rhead{a.~8}
\begin{center}
 {\Large {\bf ARTICULUS OCTAVUS}}\\
 {\large UTRUM HAEC DOCTRINA SIT ARGUMENTATIVA}\\
 {\footnotesize II-IIae, q.1, a.5, ad 1; I {\itshape Sent.}, Prol., a.5;
 I {\itshape SCG}, cap.9; in Boet.~{\itshape De Trin.}, q.2, a.3;
 {\itshape Quodlib}.~IV, q.9, a.3.}\\
 {\Large 第八項\\この学は推論的か}
\end{center}

\begin{longtable}{p{21em}p{21em}}

{\Huge A}{\scshape d octavum sic proceditur}. Videtur quod haec
doctrina non sit argumentativa. Dicit enim Ambrosius in libro I {\itshape de fide
Catholica}, {\itshape tolle argumenta, ubi fides quaeritur}. Sed in hac doctrina
praecipue fides quaeritur, unde dicitur {\itshape Ioan}.~{\scshape xx}, {\itshape haec scripta sunt ut
credatis}. Ergo sacra doctrina non est argumentativa.


&

第八項の問題へ、議論は以下のように進められる。この教えは推論的でないと思
 われる。理由は以下の通り。アンブロシウスは『カトリックの信仰について』
 第1巻で「信仰が求められるところでは推論を捨てよ」と言っている。しかし、
 この教えにおいて、とくに信仰が求められる。ゆえに、『ヨハネによる福音書』
 第20章「あなたたちが信じるためにこの書がある」と言われる。
ゆえに、神聖な教えは推論的でない。

%% この異論だと、sacra doctrinaはキリスト教信仰そのもののこと。

\\


{\scshape 2 Praeterea}, si sit argumentativa, aut
argumentatur ex auctoritate, aut ex ratione. Si ex auctoritate, non
videtur hoc congruere eius dignitati, nam locus ab auctoritate est
infirmissimus, secundum Boetium. Si etiam ex ratione, hoc non congruit
eius fini, quia secundum Gregorium in homilia, {\itshape fides non habet meritum,
ubi humana ratio praebet experimentum}. Ergo sacra doctrina non est
argumentativa.


&

さらに、もし推論的であるならば、それは、権威に基づくか、あるいは理性に基
 づくかのどちらかである。もし、権威に基づくならば、それは、この教えの威
 厳に合致しない。なぜなら、ボエティウスによれば、権威によるものは、もっ
 とも下位にあるからである。また、もし理性に基づくならば、それは、その目
 的に合致しない。なぜなら、グレゴリウス『講解』によれば、「人間の理性が
 証明を与えるところでの信仰は功績を持たないい」からである。ゆえに、神聖な教え
 は推論的でない。


\\


{\scshape Sed contra} est quod dicitur {\itshape ad Titum} {\scshape i}, de
episcopo, {\itshape amplectentem eum qui secundum doctrinam est, fidelem sermonem,
ut potens sit exhortari in doctrina sana, et eos qui contradicunt
arguere}.


&

しかし反対に、『テトスヘの手紙』第1章で、司教について「教えに即している
 信仰深い言葉を抱き、健全な教えにおいて励まされ、反論する者たちを説得す
 るように」と言われている。


\\


{\scshape Respondeo dicendum} quod, sicut aliae scientiae
non argumentantur ad sua principia probanda, sed ex principiis
argumentantur ad ostendendum alia in ipsis scientiis; ita haec doctrina
non argumentatur ad sua principia probanda, quae sunt articuli fidei;
sed ex eis procedit ad aliquid aliud ostendendum; sicut apostolus, I {\itshape ad
Cor}.~{\scshape xv}, ex resurrectione Christi argumentatur ad resurrectionem
communem probandam. 



&

解答する。以下のように言われるべきである。
ちょうど、他の諸学が、自らの諸原理へと推論することがなく、むしろ諸原理か
 ら、その学の中の他の事柄を示すために推論するように、この教えも、自らの
 諸原理、つまり信仰箇条を証明するために推論せず、それらから何か他のもの
 を示すために論を進める。ちょうど、使徒が『コリントの信徒への手紙一』第
 15章で、キリストの復活から、共通の復活を証明することへ推論しているよう
 に。


\\


Sed tamen considerandum est in scientiis
philosophicis, quod inferiores scientiae nec probant sua principia, nec
contra negantem principia disputant, sed hoc relinquunt superiori
scientiae, suprema vero inter eas, scilicet metaphysica, disputat contra
negantem sua principia, si adversarius aliquid concedit, si autem nihil
concedit, non potest cum eo disputare, potest tamen solvere rationes
ipsius. 



&

しかし、以下のことが考察されるべきである。
哲学的諸学においては、下位の学が、自らの諸原理を証明することも、その原
 理を否定する人に反対して議論することもせず、それを上位の学に委ねるが、
 そのうちの最高の学、すなわち形而上学は、反対者が何かを譲歩するならば、
 その原理を否定する者に反論するが、もし何も譲歩しないなら、そのものと議
 論することができない。ただし、彼の議論に解答することはできるが。



\\


Unde sacra Scriptura, cum non habeat superiorem, disputat cum
negante sua principia, argumentando quidem, si adversarius aliquid
concedat eorum quae per divinam revelationem habentur; sicut per
auctoritates sacrae doctrinae disputamus contra haereticos, et per unum
articulum contra negantes alium. 



&

したがって、神聖な教えは、それより上位の学を持たないので、もし反対者が、神の啓示によって持たれる事柄の何かを認め
 るならば、その原理を否定する者と、推論によって、議論する。ちょうど、私
 たちが、異端者たちと、神聖な教えの権威によって議論し、一つの信仰箇条によって、
 他の信仰箇条を否定する者に反対して議論するように。


\\


Si vero adversarius nihil credat eorum
quae divinitus revelantur, non remanet amplius via ad probandum
articulos fidei per rationes, sed ad solvendum rationes, si quas
inducit, contra fidem. Cum enim fides infallibili veritati innitatur,
impossibile autem sit de vero demonstrari contrarium, manifestum est
probationes quae contra fidem inducuntur, non esse demonstrationes, sed
solubilia argumenta.


&

他方、反対者が、神によって啓示された事柄を何も認めない場合には、理性
 によって信仰箇条を証明する道にそれ以上とどまることをせず、むしろ、彼が
 何か信仰に反すること論じるならば、それに解答する道を行く。
じっさい、信仰は、不可謬の真理に発し、真に、それに反することが証明
 されることは不可能であるから、信仰に反することを導く議論は、証明ではな
 く、間違いが示されうる議論であることが明らかである。


\\


{\scshape Ad primum ergo dicendum} quod, licet argumenta
rationis humanae non habeant locum ad probandum quae fidei sunt, tamen
ex articulis fidei haec doctrina ad alia argumentatur, ut dictum est.


&

第一異論に対しては、それゆえ、以下のように言われるべきである。
人間理性の議論は、信仰に属するものを証明する立場にないが、しかし、すでに
 述べられたとおり、信仰箇条に基づいて、この教えを他の学に対して論じるこ
 とができる。




\\


{\scshape Ad secundum dicendum} quod argumentari ex
auctoritate est maxime proprium huius doctrinae, eo quod principia huius
doctrinae per revelationem habentur, et sic oportet quod credatur
auctoritati eorum quibus revelatio facta est. 

&

第二異論に対しては、以下のように言われるべきである。
権威に基づいて論じることは、この教えに最大限に属する。なぜなら、この教え
 の原理は、啓示によって持たれ、それゆえ、それによって啓示が行われたもの
 どもの権威が信じられるべきだからである。

\\

Nec hoc derogat dignitati
huius doctrinae, nam licet locus ab auctoritate quae fundatur super
ratione humana, sit infirmissimus; locus tamen ab auctoritate quae
fundatur super revelatione divina, est efficacissimus. Utitur tamen
sacra doctrina etiam ratione humana, non quidem ad probandum fidem, quia
per hoc tolleretur meritum fidei; sed ad manifestandum aliqua alia quae
traduntur in hac doctrina. 


&


またこのことが、この学の価値を損なうことはない。たしかに、人間理性に基づ
 く権威からの議論は、 もっとも弱いものだが、神の啓示に基づく権威からの議
 論は、もっとも強力である。しかし、神聖な教えは、人間理性を用いるが、それは信
 仰を証明するためにではない。もしそうだとすると、信仰の功徳をなくしてし
 まうだろうから。そうではなく、この教えの中で伝えられる他の何かを明示す
 るためにである。



\\

Cum enim gratia non tollat naturam, sed
perficiat, oportet quod naturalis ratio subserviat fidei; sicut et
naturalis inclinatio voluntatis obsequitur caritati. Unde et apostolus
dicit, II {\itshape ad Cor}.~{\scshape x}, {\itshape in captivitatem redigentes omnem intellectum in
obsequium Christi}. 


&

じっさい、恩恵は自然を破壊せず、むしろ完成させるのだから、自然本性的理性
 は、信仰に奉仕するのでなければならない。たとえば、意志の自然本性的傾向
 性が、愛徳に奉仕するように。このことから、使徒も、『コリントの信徒への
 手紙二』第10章で「捕らえて、すべての知性をキリストに従わせる人々は」\footnote{「神の知識に逆らうあらゆる高慢を打ち倒し、あらゆる思惑
 をとりこにしてキリストに従わせ」(10:5)}と言っている。


\\

Et inde est quod etiam auctoritatibus philosophorum
sacra doctrina utitur, ubi per rationem naturalem veritatem cognoscere
potuerunt; sicut Paulus, {\itshape Actuum} {\scshape xvii}, inducit verbum Arati, dicens,
{\itshape sicut et quidam poetarum vestrorum dixerunt, genus Dei sumus}. Sed tamen
sacra doctrina huiusmodi auctoritatibus utitur quasi extraneis
argumentis, et probabilibus. 


&

そしてそれゆえ、神聖な教えは、自然本性的理性によって真理を認識することができる
 ところでは、哲学者たちの権威もまた用いる。たとえば、パウロは『使徒言行
 録』第17章で、アラトスの言葉を引いて次のように語る。「あなたたちの詩人
 の誰かが言ったように、私たちは神の一族である」\footnote{「皆さんのうち
 のある詩人たちも、/『我らは神の中に生き、動き、存在する』/『我らもそ
 の子孫である』と、/言っているとおりです。 」(17:28)}
しかし、聖書はこのような権威を言わば外的で証明可能な議論として用いる。


\\

Auctoritatibus autem canonicae Scripturae
utitur proprie, ex necessitate argumentando. Auctoritatibus autem
aliorum doctorum Ecclesiae, quasi arguendo ex propriis, sed
probabiliter. Innititur enim fides nostra revelationi apostolis et
prophetis factae, qui canonicos libros scripserunt, non autem
revelationi, si qua fuit aliis doctoribus facta. 


&

聖書正典の権威は、これを議論の必要性に基づいて、固有の意味で用いる。
また、教会の他の学者たちの権威は、これを、固有のものに基づいて、議論を明
 らかにするものとして用いるが、絶対に正しいものとしてではない。
じっさい、私たちの信仰は、正典の書物を書いた使徒や預言者になされた啓示に
 依拠するのであって、もし他の学者たちに啓示がなされたとすれば、そのよう
 な啓示に依拠するのではない。

\\

Unde dicit Augustinus,
in epistola ad Hieronymum, {\itshape solis eis Scripturarum libris qui canonici
appellantur, didici hunc honorem deferre, ut nullum auctorem eorum in
scribendo errasse aliquid firmissime credam. Alios autem ita lego, ut,
quantalibet sanctitate doctrinaque praepolleant, non ideo verum putem,
quod ipsi ita senserunt vel scripserunt}.


&

このことから、アウグスティヌスは『ヒエロニュムス宛書簡』で以下のように述
 べている。
「私は、聖書の中で正典と呼ばれる書物だけに、彼らのうちのどの著者も、決して書く
 ときに誤らなかったと私が信じるという名誉を与えることを学びました。しか
 し、他の書物は、私は、どれほど神々しさや学識が際立っ
 ていても、彼らがそう考えたり書いたりしたから、それが真だと考えないよう
 に読みます」。

\end{longtable}
\newpage


\rhead{a.~9}
\begin{center}
 {\Large {\bf ARTICULUS NONUS}}\\
 {\large UTRUM SACRA SCRIPTURA DEBEAT UTI METAPHORIS}\\
 {\footnotesize I {\itshape Sent.}, Prol., a.5; d.34, q.3, a.1, 2; III
 {\itshape SCG}, cap.119; in Boet.~{\itshape de Trin.}, q.2, a.4.}\\
 {\Large 第九項\\聖書は比喩を用いるべきか}
\end{center}

\begin{longtable}{p{21em}p{21em}}


{\Huge A}{\scshape d nonum sic proceditur}. Videtur quod sacra
Scriptura non debeat uti metaphoris. Illud enim quod est proprium
infimae doctrinae, non videtur competere huic scientiae, quae inter
alias tenet locum supremum, ut iam dictum est. Procedere autem per
similitudines varias et repraesentationes, est proprium poeticae, quae
est infima inter omnes doctrinas. Ergo huiusmodi similitudinibus uti,
non est conveniens huic scientiae.


&

第九項の問題へ、議論は以下のように進められる。
聖書は比喩を用いるべきでないと思われる。理由は以下の通り。
最低の教えに固有の事柄が、すでに言われたとおり、他の諸学の中で最高の位置を占めるこの学にふさわ
 しいとは思えない。しかし、さまざまな類似や表象をとおして進むことは、す
 べての教えの中で最も低い詩学に固有である。ゆえに、そのような類似を用い
 ることは、この学にふさわしくない。



\\


{\scshape 2 Praeterea}, haec doctrina videtur esse
ordinata ad veritatis manifestationem, unde et manifestatoribus eius
praemium promittitur, {\itshape Eccli}.~{\scshape xxiv}, {\itshape qui elucidant me, vitam aeternam
habebunt}. Sed per huiusmodi similitudines veritas occultatur. Non ergo
competit huic doctrinae divina tradere sub similitudine corporalium
rerum.


&

さらに、この教えは、真理の明示へと秩序付けられていると思われる。
それゆえ、それを明示する人々にも、報償が与えられる。『集会の書』24章「私
 を明らかにする者たちは、永遠の生命を持つだろう」\footnote{新共同訳聖書に該
 当箇所なし。}。しかし、このような類
 似によって、真理は隠される。ゆえに、この教えにとって、神の事柄を物体的事物の
 類似のもとに伝えることはふさわしくない。


\\


{\scshape 3 Praeterea}, quanto aliquae creaturae sunt
sublimiores, tanto magis ad divinam similitudinem accedunt. Si igitur
aliquae ex creaturis transumerentur ad Deum, tunc oporteret talem
transumptionem maxime fieri ex sublimioribus creaturis, et non ex
infimis. Quod tamen in Scripturis frequenter invenitur.


&

さらに、被造物は、より高位にあるほど、より神の類似に近づく。ゆえに、もし、
 被造物の何かが神になぞらえられるならば、その場合には、最下位のものではな
 く、より高位の被造物
 に基づいて、そのような喩えがなされるべきであろう。そして、聖書では、そ
 のようなことが頻繁に見出される。


\\


{\scshape Sed contra est} quod dicitur {\itshape Osee} {\scshape xii}, ego
visionem multiplicavi eis, et in manibus prophetarum assimilatus
sum. Tradere autem aliquid sub similitudine, est metaphoricum. Ergo ad
sacram doctrinam pertinet uti metaphoris.


&

しかし反対に、『ホセア書』第12章「私は、見たことを、多くの人に多数化した。
 そして、預言者たちの手において、私に似たものとした」\footnote{「わたし
 は預言者たちに言葉を伝え/多くの幻を示し/預言者たちによってたとえを示
 した。」(12:11)}と言われている。
しかし、何かを類似のもとに伝えることは、比喩である。ゆえに、神聖な教えには、比
 喩を用いることが属する。



\\


{\scshape Respondeo dicendum} quod conveniens est sacrae
Scripturae divina et spiritualia sub similitudine corporalium
tradere. Deus enim omnibus providet secundum quod competit eorum
naturae. Est autem naturale homini ut per sensibilia ad intelligibilia
veniat, quia omnis nostra cognitio a sensu initium habet. Unde
convenienter in sacra Scriptura traduntur nobis spiritualia sub
metaphoris corporalium. 


&

解答する。以下のように言われるべきである。
聖書にとって、神や霊のことを、物体の類似のもとに伝えることが便利である。
 理由は以下の通り。
神は、万物を、その本性に適したものに即して摂理する。ところが、人間には、
 可感的なものや可知的なものを通して進むことが適合している。なぜなら、私
 たちの認識はすべて、感覚に端緒を発するからである。ゆえに、聖書の中で、
 霊的な事柄が、物体的な比喩のもとで私たちに伝えられることが便利である。



\\

Et hoc est quod dicit Dionysius, I
cap. {\itshape Caelestis Hierarchiae}, {\itshape impossibile est nobis
 aliter lucere divinum radium, nisi varietate sacrorum velaminum 
circumvelatum}.


&

そしてこれが、ディオニュシウスが『天上階級論』第1章で「さまざまな聖なるヴェールで
覆われてれていなければ、それ以外のしかたで、神の光線が私たちに照らすこと
 は不可能である」と言っている意味である。

\\

 Convenit etiam
sacrae Scripturae, quae communiter omnibus proponitur (secundum illud ad
{\itshape Rom}.~{\scshape i}, {\itshape sapientibus et insipientibus debitor sum}), ut spiritualia sub
similitudinibus corporalium proponantur; ut saltem vel sic rudes eam
capiant, qui ad intelligibilia secundum se capienda non sunt idonei.


&


さらに、聖書は、共通に、すべての人に示されているので(『ローマの信徒への
 手紙』第1章「知恵ある人にもそうでない人にも、私はすべきことがある」
 \footnote{「わたしは、ギリシア人にも未開の人にも、知恵のある人にもない
 人にも、果たすべき責任があります。」(1:14)})、霊的なことを、物体的な類
 似のもとに提示することや、可知的なことをそれ自体において捕らえることに
 適していない粗野な人々が、聖書を理解しやすいようにすることが、適してい
 る。


\\


{\scshape Ad primum ergo dicendum} quod poeta utitur
metaphoris propter repraesentationem, repraesentatio enim naturaliter
homini delectabilis est. Sed sacra doctrina utitur metaphoris propter
necessitatem et utilitatem, ut dictum est.


&

第一異論に対しては、それゆえ、以下のように言われるべきである。
詩人は、表象のために、比喩を用いる。表象は、本性的に、人間にとって快適だ
 からである。
しかし、すでに述べられたとおり、神聖な教えは、比喩を必要性と有益さのために用い
 る。


\\


{\scshape Ad secundum dicendum} quod radius divinae
revelationis non destruitur propter figuras sensibiles quibus
circumvelatur, ut dicit Dionysius, sed remanet in sua veritate; ut
mentes quibus fit revelatio, non permittat in similitudinibus permanere,
sed elevet eas ad cognitionem intelligibilium; et per eos quibus
revelatio facta est, alii etiam circa haec instruantur. Unde ea quae in
uno loco Scripturae traduntur sub metaphoris, in aliis locis expressius
exponuntur. Et ipsa etiam occultatio figurarum utilis est, ad exercitium
studiosorum, et contra irrisiones infidelium, de quibus dicitur,
{\itshape Matth}.~{\scshape vii}, {\itshape nolite sanctum dare canibus}.


&

第二異論に対しては、以下のように言われるべきである。
ディオニュシウスが言うように、神の啓示の光線は、それを隠す可感的な形態の
 ために破壊されたりはせず、むしろ、その真理にとどまるので、啓示が下され
 る精神が、その類似にとどまることを許さず、可知的なものの認識へ、精神を
 高める。そして啓示が下される人々を通して、他の人々が、それについて、教
 えられる。したがって、聖書のある箇所で、比喩のもとに伝えられることが、
 他の箇所でははっきりと説明される。また、さまざまな形態のもとに隠すこと
 は、勉学に励む者の訓練や、冷笑的な不信心者への対抗のために有益である。こ
 の後者について、『マタイによる福音書』第7章「聖なるものを犬に与えるな」\footnote{「神聖なものを犬に与えてはならず、また、真珠を豚に投げてはならない。それを足で踏みにじり、向き直ってあなたがたにかみついてくるだろう。」(7:6)}
 は語られている。

\\


{\scshape Ad tertium dicendum} quod, sicut docet
Dionysius, cap. II {\itshape Cael.~Hier}., magis est conveniens quod divina in
Scripturis tradantur sub figuris vilium corporum, quam corporum
nobilium. Et hoc propter tria. 


&


第三異論に対しては、以下のように言われるべきである。
ディオニュシウスが『天上階級論』第2章で教えているとおり、聖書の中で、神
 の事柄は、高貴な物体よりも卑俗な物体の形態のもとに伝えられる方が、より
 適している。これは、三つの理由による。


\\

Primo, quia per hoc magis liberatur
humanus animus ab errore. Manifestum enim apparet quod haec secundum
proprietatem non dicuntur de divinis, quod posset esse dubium, si sub
figuris nobilium corporum describerentur divina; maxime apud illos qui
nihil aliud a corporibus nobilius excogitare noverunt. 

&

第一に、このことによって、人間の心はより誤謬から解放される。
というのも、それが、神のことについて、文字通りの厳密な意味で語られているのではない
 ことが一目瞭然だからである。とくに、高貴な物体以外のものを考え
 られない人々にとっては、もしそれが高貴な物体の形態によって記述されてい
 たならば、疑わしかっただろう。


\\

Secundo, quia hic
modus convenientior est cognitioni quam de Deo habemus in hac
vita. Magis enim manifestatur nobis de ipso quid non est, quam quid est,
et ideo similitudines illarum rerum quae magis elongantur a Deo,
veriorem nobis faciunt aestimationem quod sit supra illud quod de Deo
dicimus vel cogitamus. 


&

第二に、この語り方が、この生において私たちが神についてもつ認識には適して
 いる。なぜなら、私たちには、それが何でないかの方が、何であるかよりも明
 らかに示されるので、神から遠く離れている事物の類似が、私たちが神につい
 て語り考えることを越えて、より正しい判断を私たちの中に作るからである。


\\

Tertio, quia per huiusmodi, divina magis
occultantur indignis.


&


第三に、このようなものによって、神の事柄が、よりそれにふさわしくない者か
 ら隠される。

\end{longtable}
\newpage






\rhead{a.~10}
\begin{center}
 {\Large {\bf ARTRICULUS DECIMUS}}\\
 {\large UTRUM SACRA SCRIPTURA SUB UNA LITTERA HABEAT PLURES SENSUS}\\
 {\footnotesize I {\itshape Sent.}, Prol.~a.5; IV, d.21, q.1, a.2, qu$^a$ 1, ad 3; {\itshape de Pot.}, q.4, a.1; {\itshape Quodl}.~III, q.14, a.1; VII, q.6, per tot.; {\itshape ad Gal.}, cap.4, lect.7.}\\
 {\Large 第十項\\聖書は一つの言葉のもとに複数の意味を持つか}
\end{center}

\begin{longtable}{p{21em}p{21em}}

{\Huge A}{\scshape d decimum sic proceditur}. Videtur quod
 sacra Scriptura sub una littera non habeat plures sensus, qui sunt
 historicus vel litteralis, allegoricus, tropologicus sive moralis, et
 anagogicus. Multiplicitas enim sensuum in una Scriptura parit
 confusionem et deceptionem, et tollit arguendi firmitatem, unde ex
 multiplicibus propositionibus non procedit argumentatio, sed secundum
 hoc aliquae fallaciae assignantur. Sacra autem Scriptura debet esse
 efficax ad ostendendam veritatem absque omni fallacia. Ergo non debent
 in ea sub una littera plures sensus tradi.



&

第10項の問題へ、議論は以下のように進められる。
聖書は、一つの言葉のもとに複数の意味、すなわち、歴史的、字義的、寓意的、
 トポス的ないし道徳的、そして神秘的意味、をもたないと思われる。理由は以下の通り。
一つの書物の中に複数の意味があれば、混乱と欺きをもたらし、論証の堅固さを
 だめにする。したがって、多重命題から推論は生まれず、これに従えば、何
 らかの虚偽が指定される。しかし、聖書は、全くの誤りなく真理を明示する力
 があるべきである。ゆえに、その中に、複数の意味を伝える一つの言葉がある
 べきでない。


\\


{\scshape 2 Praeterea}, Augustinus dicit in libro {\itshape De Utilitate Credendi},
 quod {\itshape Scriptura quae testamentum vetus vocatur, quadrifariam traditur,
 scilicet, secundum historiam, secundum aetiologiam, secundum analogiam,
 secundum allegoriam}. Quae quidem quatuor a quatuor praedictis videntur
 esse aliena omnino. Non igitur conveniens videtur quod eadem littera
 sacrae Scripturae secundum quatuor sensus praedictos exponatur.



&

さらに、アウグスティヌスは、『信仰の有用さ』で、以下のように述べている。
 「旧約と呼ばれる聖書は、四通りのしかたで伝えられている。すなわち、歴史
 に即して、原因に即して、類比に即して、そして寓意に即して」。これら四つ
 は、上で述べられている四つとまったく異なるように思われる。ゆえに、聖書
 の同じ言葉が、前述の四つの意味で説明されるようには思われない。


\\


{\scshape 3 Praeterea}, praeter praedictos sensus,
invenitur sensus parabolicus, qui inter illos sensus quatuor non
continetur.


&


さらに、上述の意味の他に、寓話的意味があり、これは上の四つの意味に含まれ
 ていない。

\\


{\scshape Sed contra est} quod dicit Gregorius, XX
{\itshape Moralium}, {\itshape sacra Scriptura omnes scientias ipso locutionis suae more
transcendit, quia uno eodemque sermone, dum narrat gestum, prodit
mysterium}.


&

しかし反対に、グレゴリウスは『道徳論』第20巻で次のように述べている。「聖書は、自らの語り方に
 よってあらゆる知を越える。なぜなら、一つの同じ教説によって、為されたこ
 とを語りながら、神秘を示している」。


\\


{\scshape Respondeo dicendum} quod auctor sacrae
Scripturae est Deus, in cuius potestate est ut non solum voces ad
significandum accommodet (quod etiam homo facere potest), sed etiam res
ipsas. Et ideo, cum in omnibus scientiis voces significent, hoc habet
proprium ista scientia, quod ipsae res significatae per voces, etiam
significant aliquid. Illa ergo prima significatio, qua voces significant
res, pertinet ad primum sensum, qui est sensus historicus vel
litteralis. Illa vero significatio qua res significatae per voces,
iterum res alias significant, dicitur sensus spiritualis; qui super
litteralem fundatur, et eum supponit. 



&

解答する。以下のように言われるべきである。
聖書の著者は神であり、その権能には、意味するために音声を使うだけでなく
 (これは人間にもできる)、事物そのものも用いることが含まれる。ゆえに、
すべての学において、音声が表示するので、音声によって表示された事物が、さ
 らに何かを意味することが、その学に固有である。ゆえに、この第一の意味表
 示、つまり、音声が事物を表示する作用は、第一の意味に属し、それは、歴史
 的、ないし字義的意味である。他方、音声によって表示された事物が、さらに
 別の事物を表示する作用は、霊的意味と言われる。それは、字義的意味に基づ
 き、またそれを前提とする。




\\


Hic autem sensus spiritualis
trifariam dividitur. Sicut enim dicit apostolus, {\itshape ad Hebr}.~{\scshape vii}, lex vetus
figura est novae legis, et ipsa nova lex, ut dicit Dionysius in
{\itshape Ecclesiastica Hierarchia}, est figura futurae gloriae, in nova etiam
lege, ea quae in capite sunt gesta, sunt signa eorum quae nos agere
debemus. Secundum ergo quod ea quae sunt veteris legis, significant ea
quae sunt novae legis, est sensus allegoricus, secundum vero quod ea
quae in Christo sunt facta, vel in his quae Christum significant, sunt
signa eorum quae nos agere debemus, est sensus moralis, prout vero
significant ea quae sunt in aeterna gloria, est sensus anagogicus. 



&

この霊的意味が、三通りに分かたれる。ちょうど使徒が『ヘブライ人への手紙』
 第7章で述べるように、旧法は新法のたとえであり、また、ディオニュシウスが
 『天上階級論』で述べるように、新法は、未来の栄光のたとえであるように。
 さらに、新法の中で、その頭で為されたことは、私たちがしなければならない
 ことのしるしである。
それゆえ、旧法に属することが新法に属することを表示するかぎりで、それは寓
 意的意味であり、キリストにおいて為されたことや、キリストを表示する事柄
 において為されたことが、私たちがしなければならないことのしるしであるか
 ぎりで、道徳的意味がある。また、永遠の栄光においてあるものを意味する限
 りで、神秘的意味がある。



\\


Quia
vero sensus litteralis est, quem auctor intendit, auctor autem sacrae
Scripturae Deus est, qui omnia simul suo intellectu comprehendit, non
est inconveniens, ut dicit Augustinus XII {\itshape Confessionum}, si etiam
secundum litteralem sensum in una littera Scripturae plures sint sensus.


&

ところで、字義的意味は、著者が意図するものだが、聖書の著者は、万物を同時に自らの知
 性によって把握する神なので、アウグスティヌスが『告白』第12巻で言うよう
 に、もし字義的意味において、聖書の一つの言葉に複数の意味があっても、そ
 れは不適当なことではない。


\\


{\scshape Ad primum ergo dicendum} quod multiplicitas
horum sensuum non facit aequivocationem, aut aliam speciem
multiplicitatis, quia, sicut iam dictum est, sensus isti non
multiplicantur propter hoc quod una vox multa significet; sed quia ipsae
res significatae per voces, aliarum rerum possunt esse signa. Et ita
etiam nulla confusio sequitur in sacra Scriptura, cum omnes sensus
fundentur super unum, scilicet litteralem; ex quo solo potest trahi
argumentum, non autem ex his quae secundum allegoriam dicuntur, ut dicit
Augustinus in epistola contra Vincentium Donatistam. Non tamen ex hoc
aliquid deperit sacrae Scripturae, quia nihil sub spirituali sensu
continetur fidei necessarium, quod Scriptura per litteralem sensum
alicubi manifeste non tradat.


&


第一異論に対しては、それゆえ、以下のように言われるべきである。
これらの意味の多さは、同音異義やその他の多様さの種類を生まない。なぜなら、
 すでに語られたとおり、これらの意味は、一つの音声が多くのことを意味する
 ためにではなく、むしろ、音声によって表示された事物が、
他の事物のしるしでありうるために、多数化されるからである。したがって、す
 べての意味は、一つの意味、すなわち字義的意味に基づくので、聖書において、
 いかなる混乱も帰結しない。アウグスティヌスが、ドナティストのウィケンティ
 ウスに反論して書簡で述べるように、字義的意味からのみ、推論は引き出されるのであっ
 て、寓意によって語られたものからではない。
しかし、このことによって、聖書の何かが失われるわけではない。なぜなら、霊
 的意味のもとに含まれる信仰に必要なことで、聖書が他の箇所で、字義的意味
 によって明白に伝えてないものはないからである。


\\


{\scshape Ad secundum dicendum} quod illa tria,
historia, aetiologia, analogia, ad unum litteralem sensum pertinent. Nam
historia est, ut ipse Augustinus exponit, cum simpliciter aliquid
proponitur, aetiologia vero, cum causa dicti assignatur, sicut cum
dominus assignavit causam quare Moyses permisit licentiam repudiandi
uxores, scilicet propter duritiam cordis ipsorum, {\itshape Matt}.~{\scshape xix}, analogia
vero est, cum veritas unius Scripturae ostenditur veritati alterius non
repugnare. Sola autem allegoria, inter illa quatuor, pro tribus
spiritualibus sensibus ponitur. Sicut et Hugo de sancto Victore sub
sensu allegorico etiam anagogicum comprehendit, ponens in tertio suarum
{\itshape Sententiarum} solum tres sensus, scilicet historicum, allegoricum et
tropologicum.


&

第二異論に対しては、以下のように言われるべきである。
歴史的、原因的、類比的というこの三つは、一つの字義的意味に属する。すなわ
 ち、歴史的意味は、アウグスティヌス自身が説明しているように、端的に何か
 が提示されるときに使われ、原因的意味は、語られたことの原因が指定される
 ときである。たとえば、『マタイによる福音書』第19章で\footnote{「イエスは言われた。「あなたたちの心が頑固なので、モーセは妻を離縁することを許したのであって、初めからそうだったわけではない。」(19:8) }
主が、モーゼが妻たちを離縁する許可を出した原因、つまり彼らの心の頑なさを
 指定したときがそれである。
他方、類比的意味は、聖書のある箇所の真理が、別の箇所の真理と矛盾しないこ
 とが示されるときである。
この四つの中で、寓意的意味だけが、三つの霊的意味の中に数えられる。たとえ
 ば、サンヴィクトルのヒューゴも、彼の『命題集』第三巻で、寓意的意味のも
 とに寓話的意味を含めて、ただ三つの意味、歴史的、寓意的、トポス的意味を
 示している。


\\


{\scshape Ad tertium dicendum} quod sensus parabolicus
sub litterali continetur, nam per voces significatur aliquid proprie, et
aliquid figurative; nec est litteralis sensus ipsa figura, sed id quod
est figuratum. Non enim cum Scriptura nominat Dei brachium, est
litteralis sensus quod in Deo sit membrum huiusmodi corporale, sed id
quod per hoc membrum significatur, scilicet virtus operativa. In quo
patet quod sensui litterali sacrae Scripturae nunquam potest subesse
falsum.


&

第三異論に対しては、以下のように言われるべきである。
神秘的意味は、字義的意味のもとに含まれる。
なぜなら、音声によって、ある事柄が固有に表示される場合と、たとえによって表示される
 ことがあるが、たとえ自体は字義的意味ではなく、たとえられたものが、字義
 的意味を持つからである。じっさい、聖書が「神の腕」と言うとき、神の中に、そのような身体的部
 分があるという字義的な意味ではなく、この部分によって表示されるもの、つ
 まり働く力があるという意味である。この例で、聖書の字義的意味に、
 決して偽が入り込まないことが明らかである。


\end{longtable}


\end{document}


