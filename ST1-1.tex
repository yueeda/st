\documentclass[10pt]{jsarticle}
\usepackage{okumacro}
\usepackage{longtable}
\usepackage[polutonikogreek,english,japanese]{babel}
\usepackage{latexsym}
\usepackage{color}

%----- header -------
\usepackage{fancyhdr}
\lhead{{\it Summa Theologiae} I, q.~1}
%--------------------

\bibliographystyle{jplain}

\title{{\bf PRIMA PARS}\\{\HUGE Summae Theologiae}\\Sancti Thomae
Aquinatis\\{\sffamily QUEAESTIO PRIMA}\\DE SACRA DOCTRINA, QUALIS SIT,
ET AD QUAE SE EXTENDAT}
\author{Japanese translation\\by Yoshinori {\sc Ueeda}}
\date{Last modified \today}


%%%% コピペ用
%\rhead{a.~}
%\begin{center}
% {\Large {\bf }}\\
% {\large }\\
% {\footnotesize }\\
% {\Large \\}
%\end{center}
%
%\begin{longtable}{p{21em}p{21em}}
%
%&
%
%
%\\
%\end{longtable}
%\newpage



\begin{document}
\maketitle
\pagestyle{fancy}

\begin{center}
{\Large 第一問\\神聖な教えについて、それはどのようなもので、どの範囲に及
 ぶか}
\end{center}

\begin{longtable}{p{21em}p{21em}}

Et ut intentio nostra sub aliquibus certis limitibus comprehendatur,
 necessarium est primo investigare de ipsa sacra doctrina, qualis sit,
 et ad quae se extendat. 

Circa quae quaerenda sunt decem. 

\begin{enumerate}
 \item de necessitate huius doctrinae.
 \item utrum sit scientia.
 \item utrum sit una vel plures.
 \item utrum sit speculativa vel practica.
 \item de comparatione eius ad alias scientias.
 \item utrum sit sapientia.
 \item quid sit subiectum eius. 
 \item utrum sit argumentativa.
 \item utrum uti debeat metaphoricis vel symbolicis locutionibus.
 \item utrum Scriptura sacra huius doctrinae sit secundum plures sensus exponenda.

\end{enumerate}

&

私たちが考究しようとすることがらが、ある一定の範囲内に収まるように、まず、
 神聖な教えそのものについて、それがどのようなものであり、どの範囲に及ぶかを探
 究する必要がある。

これを巡って、十のことが問われるべきである。

\begin{enumerate}
 \item この教えの必要性について。
 \item それは学か。
 \item それは一つか、あるいは複数か。
 \item それは観照的か実践的か。
 \item 他の学との関係について。
 \item それは知恵か。
 \item その主題は何か。
 \item それは推論的か。
 \item それは比喩的あるいは象徴的な語り方を用いるべきか。
 \item この教えの聖書は複数の意味で説明されるべきか。
\end{enumerate}

\\

\end{longtable}

\newpage

\rhead{a.~1}
\begin{center}
 {\Large {\bf ARTICULUS PRIMUS}}\\
 {\large UTRUM SIT NECESSARIUM, PRAETER PHILOSOPHICAS
 DISCIPLINAS,\\ALIAM DOCTRINAM HABERI}\\
 {\footnotesize II$^a$ II$^{ae}$, q.~2, a.~3, 4; I {\itshape Sent.},
 Prol., a.~1; I {\itshape SCG.}, c.~4, 5; {\itshape De Verit.}, q.~14, a.~10.}\\
 {\Large 第一項\\哲学的諸学とは別に、他の教えをもつことは必要か}
\end{center}

\begin{longtable}{p{21em}p{21em}}
{\huge A}{\scshape d primum sic proceditur}. Videtur quod non sit
necessarium, praeter philosophicas disciplinas, aliam doctrinam
haberi. Ad ea enim quae supra rationem sunt, homo non debet conari,
secundum illud {\itshape Eccli}.~{\scshape iii}, {\itshape altiora te ne
quaesieris}. Sed ea quae rationi subduntur, sufficienter traduntur in
philosophicis disciplinis. Superfluum igitur videtur, praeter
philosophicas disciplinas, aliam doctrinam haberi.

&

第一項の問題へ向けて、議論は次のように進められる。
哲学的諸学と別に、他の教えをもつ必要はないと考えられる。
なぜなら、『集会の書』3章「あなたより高いものを問うてはならない」 によれ
 ば、人間は、理性を越えることがらを知ろうとすべきでない。しかし、理性の
 もとにあることがらは、哲学的諸学の中で十分に論じられている。ゆえに、哲
 学的諸学意外に、他の教えをもつことは無駄であるように思える。

\\

{\scshape 2 Praeterea}, doctrina non potest esse nisi de ente, nihil
enim scitur nisi verum, quod cum ente convertitur. Sed de omnibus
entibus tractatur in philosophicis disciplinis, et etiam de Deo, unde
quaedam pars philosophiae dicitur theologia, sive scientia divina, ut
patet per philosophum in VI {\itshape Metaphys}. Non fuit igitur
necessarium, praeter philosophicas disciplinas, aliam doctrinam haberi.


&


さらに、教えは、有についての教え以外にありえない。なぜなら、真でなけれ
ば何も知られないが、真は有と置換されるからである。しかし、すべての有に
ついて、哲学的諸学の中で論じられるし、神についてすら、論じられている。
このことから、『形而上学』6巻の哲学者によって明らかなとおり、哲学のあ
る部分は「神学」や「神の学」と言われる。ゆえに、哲学的諸学以外に、他の
教えをもつ必要はなかった。


\\



Sed contra est quod dicitur II {\itshape ad Tim}.~{\scshape iii},
{\itshape omnis Scriptura divinitus inspirata utilis est ad docendum, ad
arguendum, ad corripiendum, ad erudiendum ad iustitiam}. Scriptura autem
divinitus inspirata non pertinet ad philosophicas disciplinas, quae sunt
secundum rationem humanam inventae. Utile igitur est, praeter
philosophicas disciplinas, esse aliam scientiam divinitus inspiratam.

&

しかし反対に、『テモテへの手紙二』3章で次のように言われている。「神の霊
 感によって書かれたすべての聖書は、教え、戒め、矯正し、正義へ導くため
 に有用である」。\footnote{「聖書はすべて神の霊の導きの下に書かれ、人を
 教え、戒め、誤りを正し、義に導く訓練をするうえに有益です。」(3:16)}ところで、神の霊感によって書かれた聖書は、哲学的諸学に
 属さない。哲学的諸学は、人間理性にしたがって見出されるものだからである。
 ゆえに、哲学的諸学以外に、神の霊感による他の学があることが有用である。

\\



{\scshape Respondeo dicendum} quod necessarium fuit ad humanam
salutem, esse doctrinam quandam secundum revelationem divinam, praeter
philosophicas disciplinas, quae ratione humana investigantur. Primo
quidem, quia homo ordinatur ad Deum sicut ad quendam finem qui
comprehensionem rationis excedit, secundum illud {\itshape Isaiae}
{\scshape lxiv}, {\itshape oculus non vidit Deus absque te, quae
praeparasti diligentibus te}. Finem autem oportet esse praecognitum
hominibus, qui suas intentiones et actiones debent ordinare in
finem. Unde necessarium fuit homini ad salutem, quod ei nota fierent
quaedam per revelationem divinam, quae rationem humanam excedunt.

&

解答する。以下のように言われるべきである。人間の救済のためには、人間理
性によって探究される哲学的諸学以外に、神の啓示による何らかの教えが存在
することが必要であった。その理由の第一には、『イザヤ書』64章「神よ、あ
なたを愛する者たちに準備したものどもを、あなたなしに、目が見ることがな
かった」\footnote{「あなたを待つ者に計らってくださる方は、神よ、あなた
のほかにはありません。昔から、他に聞いた者も耳にした者も、目に見た者も
ありません」(64:3)}によれば、人間は、神を目的として秩序づけられている
が、その神は、理性の把握を越えている。しかし、自分の意図と行為を目的へ
秩序づけなければならない人間には、目的があらかじめ知られていなければな
らない。したがって、人間には、救いのために、人間理性を越えることがらが、
神の啓示を通して知られたものとなる必要があった。


\\

Ad ea etiam quae de Deo ratione humana investigari possunt,
necessarium fuit hominem instrui revelatione divina. Quia veritas de
Deo, per rationem investigata, a paucis, et per longum tempus, et cum
admixtione multorum errorum, homini proveniret, a cuius tamen
veritatis cognitione dependet tota hominis salus, quae in Deo est. Ut
igitur salus hominibus et convenientius et certius proveniat,
necessarium fuit quod de divinis per divinam revelationem instruantur.


&

神について、人間理性によって探究されうることがらについてもまた、神の啓
示によって、人間が教えられることが必要であった。なぜなら、神についての
真理は、少数の優れた人々によって理性を通して探究されたならば、長い時間
をかけ、また多くの誤りを含みながら、人間のもとにやって来ただろうが、し
かし、その神にかんする真理の認識に、人間の全救済がかかっているからであ
る。その救済は神の中にあるのだから。ゆえに、救済が、人々に、より適切に
確実にやってくるために、神のことがらについて、神の啓示を通して教えられ
ることが必要であった。


\\




Necessarium igitur fuit, praeter philosophicas disciplinas,
quae per rationem investigantur, sacram doctrinam per revelationem
haberi.

&

ゆえに、理性によって探究される哲学的諸学以外に、啓示によって神聖な教え
をもつことが必要であった。


\\



{\scshape Ad primum ergo dicendum} quod, licet ea quae sunt altiora
hominis cognitione, non sint ab homine per rationem inquirenda, sunt
tamen, a Deo revelata, suscipienda per fidem. Unde et ibidem subditur,
{\itshape plurima supra sensum hominum ostensa sunt tibi}. Et in
huiusmodi sacra doctrina consistit.

&

第一異論に対しては、それゆえ、次のように言われるべきである。人間の認識
より高いものは、人間によって理性を通して探究されるべきでないが、しかし、
神によって啓示されるそのようなものは、信仰によって保持されるべきである。
したがって、同じ箇所で、「人間の感覚を越えた多くのものがあなたに示され
た」と、続いて述べられている。神聖な教えは、そのようなものにおいて、成
立している。

\\



{\scshape Ad secundum dicendum} quod diversa ratio cognoscibilis diversitatem
scientiarum inducit. Eandem enim conclusionem demonstrat astrologus et
naturalis, puta quod terra est rotunda, sed astrologus per medium
mathematicum, idest a materia abstractum; naturalis autem per medium
circa materiam consideratum. Unde nihil prohibet de eisdem rebus, de
quibus philosophicae disciplinae tractant secundum quod sunt
cognoscibilia lumine naturalis rationis, et aliam scientiam tractare
secundum quod cognoscuntur lumine divinae revelationis. Unde theologia
quae ad sacram doctrinam pertinet, differt secundum genus ab illa
theologia quae pars philosophiae ponitur.


&

第二異論に対しては、次のように言われるべきである。学の違いをもたらすの
は、認識の性格[根拠]の違いである。たとえば、天文学者と自然学者は、同
じ結論、たとえば「地球は丸い」という結論を証明するが、天文学者は数学的
な媒介[中項]、つまり、質料から抽象されたことがらによってそれを証明す
るが、他方で自然学者は、質料をめぐって考察された媒介[中項]によって証
明する。このことから、同じ事物について、哲学的諸学が、自然理性の光によっ
て考察されうるかぎりで論じ、他方で別の学が、神の啓示の光によって認識さ
れうるかぎりで論じるとしても、それは差し支えない。このことから、神聖な
教えに含まれる神学は、哲学の部分とされる神学とは、類において異なる。


\end{longtable}

\newpage


\rhead{a.~2}
\begin{center}
 {\Large {\bf ARTICULUS SECUNDUS}}\\
 {\large UTRUM SACRA DOCTRINA SIT SCIENTIA}\\
 {\footnotesize II$^a$ II$^{ae}$, q.~1, a.~5, ad 2; I {\itshape Sent.},
 Prol., a.~3, q$^a$ 2; {\itshape De Verit.}, q.~14, a.~9, ad 3; in
 {\itshape Boet.~de Trin.}, q.~2, a.~2.}\\
 {\Large 第二項\\神聖な教えは学か}
\end{center}

\begin{longtable}{p{21em}p{21em}}

{\huge A}{\scshape d secundum sic proceditur}. Videtur quod sacra
doctrina non sit scientia. Omnis enim scientia procedit ex principiis
per se notis. Sed sacra doctrina procedit ex articulis fidei, qui non
sunt per se noti, cum non ab omnibus concedantur, {\itshape non enim
omnium est fides}, ut dicitur II {\itshape Thessalon}.~{\scshape
iii}. Non igitur sacra doctrina est scientia.


&

第二項の問題へ、議論は次のように進められる。神聖な教えは学でないと思わ
れる。なぜなら、すべて学は自明な原理から出発する。しかし、神聖な教えは、
信仰箇条から出発するが、その信仰箇条はすべての人に認められているわけで
はない。『テサロニケの信徒への手紙二』3章で、「信仰はすべての人に属す
るわけではない」\footnote{「すべての人に、信仰があるわけではないので
す。」(3:2)}と言われるように。ゆえに、神聖な教えは学でない。

\\


{\scshape 2 Praeterea}, scientia non est singularium. Sed
sacra doctrina tractat de singularibus, puta de gestis Abrahae, Isaac et
Iacob, et similibus. Ergo sacra doctrina non est scientia.


&

さらに、学は個的なことがらにかかわらない。ところが、神聖な教えは、アブ
ラハム、イサク、ヤコブ、その他、似た人々の行為についてなど、個的なこと
がらを扱う。ゆえに、神聖な教えは学でない。

\\


Sed contra est quod Augustinus dicit, XIV {\itshape de Trinitate}:
{\itshape Huic scientiae attribuitur illud tantummodo quo fides
saluberrima gignitur, nutritur, defenditur, roboratur}. Hoc autem ad
nullam scientiam pertinet nisi ad sacram doctrinam. Ergo sacra
doctrina est scientia.


&

しかし反対に、アウグスティヌスは『三位一体論』14巻で「この学には、もっ
とも健康的な信仰が、それによって生み出され、養われ、守られ、強化される
ものだけが帰せられる」と述べる。しかし、このことは、神聖な教え以外のど
の学にも属さない。ゆえに、神聖な教えは学である。


\\


{\scshape Respondeo dicendum} sacram doctrinam esse scientiam. Sed
sciendum est quod duplex est scientiarum genus. Quaedam enim sunt,
quae procedunt ex principiis notis lumine naturali intellectus, sicut
arithmetica, geometria, et huiusmodi. Quaedam vero sunt, quae
procedunt ex principiis notis lumine superioris scientiae, sicut
perspectiva procedit ex principiis notificatis per geometriam, et
musica ex principiis per arithmeticam notis. Et hoc modo sacra
doctrina est scientia, quia procedit ex principiis notis lumine
superioris scientiae, quae scilicet est scientia Dei et beatorum. Unde
sicut musica credit principia tradita sibi ab arithmetico, ita
doctrina sacra credit principia revelata sibi a Deo.


&

解答する。次のように言われるべきである。神聖な教えは学である。しかし、
学の類に二つあることが知られるべきである。ある学は、自然的知性の光によっ
て知られる原理から出発する。たとえば、算術、幾何学などがそれである。他
方、ある学は、上位の学の光によって知られた原理から出発する。たとえば、
透視図法は、幾何学によって知られた原理から始めるし、音楽は、算術学によっ
て知られた原理から始める。そして、この後者のしかたで、神聖な教えは学で
ある。なぜなら、上位の学、すなわち、神と至福者の知の光によって知られた
原理から出発するからである。このことから、ちょうど音楽が、算術から自ら
に伝えられた原理を信じるように、神聖な教えは、神から自らに啓示された原
理を信じる。



\\


{\scshape Ad primum ergo dicendum} quod principia cuiuslibet scientiae
vel sunt nota per se, vel reducuntur ad notitiam superioris
scientiae. Et talia sunt principia sacrae doctrinae, ut dictum est.


&

それゆえ、第一異論に対しては次のように言われるべきである。どんな学の原
理も、それ自体によって知られているか、あるいは、上位の学の知へ還元され
るかのいずれかである。すでに述べられたとおり、神聖な教えの原理はそのよ
うなものである。

\\


{\scshape Ad secundum dicendum} quod singularia traduntur in sacra
doctrina, non quia de eis principaliter tractetur, sed introducuntur
tum in exemplum vitae, sicut in scientiis moralibus; tum etiam ad
declarandum auctoritatem virorum per quos ad nos revelatio divina
processit, super quam fundatur sacra Scriptura seu doctrina.


&

第二異論に対しては、次のように言われるべきである。個的なものが神聖な教
えの中で扱われるのは、それらについて主要に論じられているからではない。
個的なものが導入されるのは、倫理的な知においてそうであるように、生の事
例としてであり、さらにまた、神の啓示は、ある人々を通して私たちまでやっ
て来たのだが、その人々の権威を明示するためである。聖書、あるいは神聖な
教えは、この神の啓示に基礎を持つ。




\end{longtable}
\newpage

\rhead{a.~3}
\begin{center}
 {\Large {\bf ARTICULUS TERTIUS}}\\
 {\large UTRUM SACRA DOCTRINA SIT UNA SCIENTIA}\\
 {\footnotesize Infra art.~3; I {\itshape Sent.}, Prol., a.~2, 4.}\\
 {\Large 第三項\\神聖な教えは一つの学か}
\end{center}

\begin{longtable}{p{21em}p{21em}}

{\huge A}{\scshape d tertium sic proceditur}. Videtur quod sacra
doctrina non sit una scientia. Quia secundum philosophum in I {\itshape
Poster}., {\itshape una scientia est quae est unius generis
subiecti}. Creator autem et creatura, de quibus in sacra doctrina
tractatur, non continentur sub uno genere subiecti. Ergo sacra doctrina
non est una scientia.

&

第三の問題へ向けて、以下のように議論が進められる。神聖な教えは一つの学
でないと思われる。なぜなら、『分析論後書』1巻の哲学者によれば「一つの
学とは、一つの類の主題を扱うものである」。しかし、神聖な教えの中で、創
造者と被造物が論じられるが、この二つは主題の一つの類に含まれない。ゆえ
に、神聖な教えは一つの学でない。

\\


{\scshape 2 Praeterea}, in sacra doctrina tractatur de Angelis, de
creaturis corporalibus, de moribus hominum. Huiusmodi autem ad
diversas scientias philosophicas pertinent. Igitur sacra doctrina non
est una scientia.


&

さらに、神聖な教えの中では、天使について、物体的被造物について、人間の
行いについて論じられる。しかし、これらは異なる哲学的諸学に属する。ゆえ
に、神聖な教えは一つの学でない。

\\


{\scshape Sed contra} est quod sacra Scriptura de ea loquitur sicut de
una scientia, dicitur enim {\itshape Sap}.~{\scshape x}, {\itshape dedit
illi scientiam sanctorum}.


&

しかし反対に、聖書は、それ[=神聖な教え]について、一つの学であるかの
ように語っている。たとえば、『知恵の書』10章「彼に、聖なるものどもの知
[単数形]を与えた」と言われるように。

\\


{\scshape Respondeo dicendum} sacram doctrinam unam scientiam
esse. Est enim unitas potentiae et habitus consideranda secundum
obiectum, non quidem materialiter, sed secundum rationem formalem
obiecti, puta homo, asinus et lapis conveniunt in una formali ratione
colorati, quod est obiectum visus. Quia igitur sacra Scriptura
considerat aliqua secundum quod sunt divinitus revelata, secundum quod
dictum est, omnia quaecumque sunt divinitus revelabilia, communicant
in una ratione formali obiecti huius scientiae. Et ideo
comprehenduntur sub sacra doctrina sicut sub scientia una.


&


解答する。以下のように言われるべきである。神聖な教えは一つの学である。
なぜなら、能力や習態の一性は、対象の観点から考察されるべきだが、その際、
対象は、質料的にではなく、対象の形相的性格に即して考察される。たとえば、
人間、ロバ、石は、「色があるもの」という形相的性格の点で一致するが、こ
れは、視覚の対象である。ゆえに、すでに述べられたことによれば、聖書は、
神によって啓示されたものであるという点で、何事についても考察するのだか
ら、なんであれ神によって啓示されうるすべてのものは、この学の対象の形相
的性格において共通する。ゆえに、一つの学としての神聖な教えの中に、それ
らは包含される。

\\


{\scshape Ad primum ergo dicendum} quod sacra doctrina non determinat
de Deo et de creaturis ex aequo, sed de Deo principaliter, et de
creaturis secundum quod referuntur ad Deum, ut ad principium vel
finem. Unde unitas scientiae non impeditur.


&

第一異論に対しては、次のように言われるべきである。神聖な教えは、神と被
造物とについて、等しく論じるのではなく、神について主要に論じ、被造物に
ついては、それらが根源や目的としての神に関係するかぎりで論じる。したがっ
て、学の一性は損なわれない。
\\


{\scshape Ad secundum dicendum} quod nihil prohibet inferiores
potentias vel habitus diversificari circa illas materias, quae
communiter cadunt sub una potentia vel habitu superiori, quia superior
potentia vel habitus respicit obiectum sub universaliori ratione
formali.


&

第二異論に対しては、次のように言われるべきである。一つの上位の能力や習
態のもとに共通にある題材について、下位の能力や習態が多様化することがあっ
てもなんら問題ない。なぜなら、上位の能力や習態は、より普遍的な形相的性
格のもとに、対象に関係するからである。

\\

Sicut obiectum sensus communis est sensibile, quod comprehendit sub se
visibile et audibile, unde sensus communis, cum sit una potentia,
extendit se ad omnia obiecta quinque sensuum. Et similiter ea quae in
diversis scientiis philosophicis tractantur, potest sacra doctrina,
una existens, considerare sub una ratione, inquantum scilicet sunt
divinitus revelabilia, ut sic sacra doctrina sit velut quaedam
impressio divinae scientiae, quae est una et simplex omnium.


&

たとえば、共通感覚の対象は「可感的なもの」だが、これは、「見られうるも
の」と「聞かれうるもの」をその中に含む。したがって、共通感覚は、一つの
能力だが、五つの感覚のすべての対象に及ぶ。同様に、さまざまな哲学的諸学
の中で論じられるものを、神聖な教えが、一つのものでありながら、一つの性
格のもとに、すなわち、それが神によって啓示されうるものであるかぎりで、
考察することが可能である。この意味で、神聖な教えは、万物について一つで
単純である神の知の、いわばある種の刻印である。



\end{longtable}
\newpage



\rhead{a.~4}
\begin{center}
 {\Large {\bf ARTICULUS QUARTUS}}\\
 {\large UTRUM SACRA DOCTRINA SIT SCIENTIA PRACTICA}\\
 {\footnotesize I {\itshape Sent.}, Prol., a.3, qu$^a$1.}\\
 {\Large 第四項\\神聖な教えは実践的学か}
\end{center}

\begin{longtable}{p{21em}p{21em}}

{\Huge A}{\scshape d quartum sic proceditur}. Videtur quod sacra
doctrina sit scientia practica. Finis enim practicae est operatio,
secundum philosophum in II Metaphys. Sacra autem doctrina ad
operationem ordinatur, secundum illud Iac. I, estote factores verbi,
et non auditores tantum. Ergo sacra doctrina est practica scientia.



&

第四項の問題へ、議論は以下のように進められる。神聖な教えは実践的学だと
思われる。理由は以下の通り。『形而上学』第2巻の哲学者によれば、実践的
学の目的は働きである。しかし、『ヤコブへの手紙』1章「言葉をただ聞いて
いるだけの人ではなく、行う人でありなさい」によれば、神聖な教えは、働き
へ秩序付けられている。ゆえに、神聖な教えは実践的学である。


\\


{\scshape 2 Praeterea}, sacra doctrina dividitur per legem veterem et
novam. Lex autem pertinet ad scientiam moralem, quae est scientia
practica. Ergo sacra doctrina est scientia practica.


&

さらに、神聖な教えは、旧法と新法によって分かたれる。ところが、法は道徳
学に属し、それは実践的学である。ゆえに神聖な教えは実践的学である。



\\


{\scshape Sed contra}, omnis scientia practica est de
rebus operabilibus ab homine; ut moralis de actibus hominum, et
aedificativa de aedificiis. Sacra autem doctrina est principaliter de
Deo, cuius magis homines sunt opera. Non ergo est scientia practica, sed
magis speculativa.


&

しかし反対に、すべての実践的額は、人間によってなされうる事柄にかかわる。
たとえば、道徳学は人々の行為について、建築学は建築物についてである。し
かし、神聖な教えは、主として神についてであり、むしろ人間はその神の作品
である。ゆえに、それは実践的学でなく、むしろ観照的学である。


\\


{\scshape Respondeo dicendum} quod sacra doctrina, ut dictum est, una
existens, se extendit ad ea quae pertinent ad diversas scientias
philosophicas, propter rationem formalem quam in diversis attendit,
scilicet prout sunt divino lumine cognoscibilia. Unde licet in
scientiis philosophicis alia sit speculativa et alia practica, sacra
tamen doctrina comprehendit sub se utramque; sicut et Deus eadem
scientia se cognoscit, et ea quae facit. Magis tamen est speculativa
quam practica, quia principalius agit de rebus divinis quam de actibus
humanis; de quibus agit secundum quod per eos ordinatur homo ad
perfectam Dei cognitionem, in qua aeterna beatitudo consistit.


&

解答する。以下のように言われるべきである。神聖な教えは、すでに述べられ
たとおり、神の光によって認識可能なこと、という、さまざまな事柄において
見出される形相的性格のために、さまざまな哲学的諸学に属する事柄へ、一つ
のものとして及ぶ。したがって、哲学的諸学においては、感傷的学と実践的学
は異なる学だが、神聖な教えはそのどちらも含む。ちょうどそれは、神もまた、
同一の知によって、自分自身と、自分が作るものを認識するように。しかし、
どちらかと言えば、それは実践的と言うよりは観照的である。なぜなら、より
主要には、人間の行為よりは神の事柄について論じるからである。人間の行為
については、それが、そこにおいて永遠の至福が成立するところの神の完全な
認識へと人間が秩序付けられるかぎりにおいて論じる。


\\


Et per hoc patet responsio ad obiecta.


&

以上によって、異論への解答は明らかである。


\end{longtable}
\newpage


\rhead{a.~5}
\begin{center}
 {\Large {\bf ARTICULUS QUINTUS}}\\
 {\large UTRUM SACRA DOCTRINA SIT DIGNIOR ALIIS SCIENTIIS}\\
 {\footnotesize I-IIae, q.66, a.5, ad 3; I {\itshape Sent.}, Prol.~a.1;
 II {\itshape SCG}, cap.4.}\\
 {\Large 第五項\\神聖な教えは他の諸学よりも価値があるか}
\end{center}

\begin{longtable}{p{21em}p{21em}}

{\Huge A}{\scshape d quintum sic proceditur}. Videtur quod sacra
doctrina non sit dignior aliis scientiis. Certitudo enim pertinet ad
dignitatem scientiae. Sed aliae scientiae, de quarum principiis
dubitari non potest, videntur esse certiores sacra doctrina, cuius
principia, scilicet articuli fidei, dubitationem recipiunt. Aliae
igitur scientiae videntur ista digniores.


&

第五項の問題へ、議論は以下のように進められる。神聖な教えは他の諸学より
価値が高いわけではないと思われる。理由は以下の通り。学問の価値には確実
性が関係する。しかし、その原理が疑いえない他の諸学は、信仰箇条という、
疑いを容れる原理をもつ神聖な教えよりも、確実だと思われる。ゆえに、他の
諸学の方が、これよりも価値があると思われる。


\\


{\scshape 2 Praeterea}, inferioris scientiae est a superiori accipere,
sicut musicus ab arithmetico. Sed sacra doctrina accipit aliquid a
philosophicis disciplinis, dicit enim Hieronymus in epistola {\itshape
Ad Magnum Oratorem Urbis Romae}, quod doctores antiqui {\itshape
intantum philosophorum doctrinis atque sententiis suos resperserunt
libros, ut nescias quid in illis prius admirari debeas, eruditionem
saeculi, an scientiam Scripturarum}. Ergo sacra doctrina est inferior
aliis scientiis.


&

さらに、音楽が算術から〔原理を〕受け取るように、上位の学から〔原理を〕
受け取ることは、下位の学に属する。しかし、神聖な教えが、哲学的諸学から
受け取るものがある。たとえば、ヒエロニュムスは、『ローマの偉大な弁論家
宛書簡』で、古代の学者たちは「哲学者たちの教説や文章を、自分たちの本の
中にまき散らしたので、世俗の学問と聖書の学知と、どちらが先に賞賛される
べきか、あなたが途方に暮れるほどです」と書いている。ゆえに、神聖な教え
は、他の諸学よりも下位である。



\\


{\scshape Sed contra est} quod aliae scientiae dicuntur ancillae
huius, {\itshape Prov}.~{\scshape ix}, {\itshape misit ancillas suas
vocare ad arcem}.


&


しかし反対に、他の諸学は、この学のメイドと言われる。『箴言』第9章「自
分のメイドたちを城に招くと、彼は伝えた」。

\\


{\scshape Respondeo dicendum} quod, cum ista scientia
quantum ad aliquid sit speculativa, et quantum ad aliquid sit practica,
omnes alias transcendit tam speculativas quam practicas. 

&

解答する。以下のように言われるべきである。この学は、ある点にかんして観
照的であり、別の点にかんして実践的であるから、鑑賞的諸学も実践的諸学も、
他のすべての学を超越する。


\\


Speculativarum enim scientiarum una altera dignior dicitur, tum
propter certitudinem, tum propter dignitatem materiae.  Et quantum ad
utrumque, haec scientia alias speculativas scientias excedit. Secundum
certitudinem quidem, quia aliae scientiae certitudinem habent ex
naturali lumine rationis humanae, quae potest errare, haec autem
certitudinem habet ex lumine divinae scientiae, quae decipi non
potest. Secundum dignitatem vero materiae, quia ista scientia est
principaliter de his quae sua altitudine rationem transcendunt, aliae
vero scientiae considerant ea tantum quae rationi subduntur.



&


観照的諸学の中で、ある学問が他の学問より価値があると言われるのは、一つ
には確実性のためであり、もう一つには題材の価値のためにである。そしてこ
のどちらについても、この学は、他の観照的諸学にまさる。確実性に即しては、
他の諸学は人間理性の自然的な光に基づいて確実性を持つが、その光は誤りう
る。それに対してこの学は、神の知の光に基づいて確実性を持ち、その光は欺
かれ得ない。また、題材に即して言えば、この学は、主に、その高さによって
理性を越えるものにかんしてあるが、他の諸学は、理性に服するものだけを考
察する。

\\


Practicarum vero scientiarum illa dignior est, quae ad ulteriorem
finem ordinatur, sicut civilis militari, nam bonum exercitus ad bonum
civitatis ordinatur. Finis autem huius doctrinae inquantum est
practica, est beatitudo aeterna, ad quam sicut ad ultimum finem
ordinantur omnes alii fines scientiarum practicarum. Unde manifestum
est, secundum omnem modum, eam digniorem esse aliis.


&

他方、実践的諸学の中で、より価値が高いのは、より究極的な目的へ秩序付け
られているものである。たとえば、政治学が兵学より価値があるのは、軍隊の
善は国家の善に秩序付けられているからである。ところで、この教えの目的は、
それが実践的学である限りにおいて、永遠の至福であり、他のすべての実践的
学問の目的は、究極目的として、それへと秩序付けられている。したがって、
あらゆる意味において、これがその他のものより価値があることが明らかであ
る。


\\


{\scshape Ad primum ergo dicendum} quod nihil prohibet id quod est
certius secundum naturam, esse quoad nos minus certum, propter
debilitatem intellectus nostri, qui {\itshape se habet ad
manifestissima naturae, sicut oculus noctuae ad lumen solis}, sicut
dicitur in II {\itshape Metaphys}. Unde dubitatio quae accidit in
aliquibus circa articulos fidei, non est propter incertitudinem rei,
sed propter debilitatem intellectus humani. Et tamen minimum quod
potest haberi de cognitione rerum altissimarum, desiderabilius est
quam certissima cognitio quae habetur de minimis rebus, ut dicitur in
XI {\itshape De Animalibus}.


&


第一異論に対しては、それゆえ、以下のように言われるべきである。『形而上
学』第2巻で言われるように、私たちの知性は、「自然のもっとも明らかな事
柄へ、あたかもフクロウの目が太陽の光に関係するように関係する」ので、本
性に即してより確実なものが、私たちにとってより確実でないということが、
私たちの知性の弱さのために起こりうる。したがって、信仰箇条をめぐって、
ある人々の中に生じている疑いは、実際の不確実性のせいではなく、人間知性
の弱さのせいである。しかし、『動物論』第11巻で言われるように、最高の諸
事物の認識について持ちうる最少のことは、最少の諸事物について持ちうる最
高に確実な認識よりも、望ましいものである。


\\


{\scshape Ad secundum dicendum} quod haec scientia accipere potest
aliquid a philosophicis disciplinis, non quod ex necessitate eis
indigeat, sed ad maiorem manifestationem eorum quae in hac scientia
traduntur. Non enim accipit sua principia ab aliis scientiis, sed
immediate a Deo per revelationem. Et ideo non accipit ab aliis
scientiis tanquam a superioribus, sed utitur eis tanquam inferioribus
et ancillis; sicut architectonicae utuntur subministrantibus, ut
civilis militari. Et hoc ipsum quod sic utitur eis, non est propter
defectum vel insufficientiam eius, sed propter defectum intellectus
nostri; qui ex his quae per naturalem rationem (ex qua procedunt aliae
scientiae) cognoscuntur, facilius manuducitur in ea quae sunt supra
rationem, quae in hac scientia traduntur.


&


第二異論に対しては、以下のように言われるべきである。この学が、哲学的諸
学から受け取りうるものがあるのは、必要性に基づいてそれを必要とするから
ではなく、この学で扱われる事柄をより明らかに示すためにである。じっさい、
その諸原理を他の諸学からではなく、直接に神から啓示を通して受け取るので、
上位のものとしての他の諸学から受け取るのではなく、むしろそれらを下位の
もの、メイドとして使う。ちょうど、建築的諸学がそれに仕える諸学を用いる
ように。政治学が兵学を用いる場合がそれである。そして、それらをそのよう
に用いることは、それの欠点や不十分さのためではなく、私たちの知性の欠陥
のためである。人間知性は、自然的理性(他の諸学はここから出発する)をと
おして認識するものから、理性を越えるものへと導かれる方がわかりやすいの
だが、この後者を、この学は扱う。

\end{longtable}
\newpage



\rhead{a.~6}
\begin{center}
 {\Large {\bf ARTICULUS SEXTUS}}\\
 {\large UTRUM HAEC DOCTRINA SIT SAPIENTIA}\\
 {\footnotesize I {\itshape Sent.}, Prol., a.3, qu$^a$1, 3; II, Prol.,
 princ.; II {\itshape SCG}, cap.4.}\\
 {\Large 第六項\\この教えは知恵か}
\end{center}

\begin{longtable}{p{21em}p{21em}}



{\Huge A}{\scshape d sextum sic proceditur}. Videtur quod haec
doctrina non sit sapientia. Nulla enim doctrina quae supponit sua
principia aliunde, digna est nomine sapientiae, quia {\itshape
sapientis est ordinare, et non ordinari (I Metaphys.}). Sed haec
doctrina supponit principia sua aliunde, ut ex dictis patet. Ergo haec
doctrina non est sapientia.


&

第六項の問題へ、議論は以下のように進められる。この教えは知恵でないと思
われる。理由は以下の通り。「知恵には、秩序付けられることではなく、秩序
付けることが属する」(『形而上学』第1巻)ので、どこか他のところから原理
を受け取るどんな教えも、知恵の名に値しない。しかし、この教えは、すでに
述べられたことから明らかなとおり、どこか他のところからその原理を受け取
る。ゆえに、この教えは知恵でない。

\\


{\scshape 2 Praeterea}, ad sapientiam pertinet probare principia
aliarum scientiarum, unde {\itshape ut caput} dicitur scientiarum, ut
VI {\itshape Ethic}.~patet. Sed haec doctrina non probat principia
aliarum scientiarum. Ergo non est sapientia.


&

さらに、知恵には、他の諸学の原理を証明することが属する。それゆえ、『ニ
コマコス倫理学』第6巻で明らかなとおり、「頭のようなもの」と言われる。
しかし、この教えは、他の諸学の原理を証明しない。ゆえに、知恵ではない。


\\


{\scshape 3 Praeterea}, haec doctrina per studium
acquiritur. Sapientia autem per infusionem habetur, unde inter septem
dona spiritus sancti connumeratur, ut patet {\itshape Isaiae}
{\scshape xi}. Ergo haec doctrina non est sapientia.


&

さらに、この教えは、勉強を通して得られる。しかし、『イザヤ書』第11章で
明らかなとおり、知恵は注入によって得られ、それゆえに、聖霊の七つの贈り
物のなかに数えられる。ゆえに、この教えは知恵でない。


\\


{\scshape Sed contra est} quod dicitur {\itshape Deut}.~{\scshape iv},
in principio legis, {\itshape haec est nostra sapientia et intellectus
coram populis}.

&

しかし反対に、『申命記』第4章の最初の法で「これは、人々を前にした私た
ちの知恵、知性である」と言われている。

\\


{\scshape Respondeo dicendum} quod haec doctrina maxime sapientia est
inter omnes sapientias humanas, non quidem in aliquo genere tantum,
sed simpliciter. Cum enim sapientis sit ordinare et iudicare, iudicium
autem per altiorem causam de inferioribus habeatur; ille sapiens
dicitur in unoquoque genere, qui considerat causam altissimam illius
generis. Ut in genere aedificii, artifex qui disponit formam domus,
dicitur sapiens et architector, respectu inferiorum artificum, qui
dolant ligna vel parant lapides, unde dicitur I {\itshape
Cor}.~{\scshape iii}, ut sapiens architector fundamentum posui.


&


解答する。以下のように言われるべきである。この教えは、人間のすべての知
恵の中で、なんらかの類においてだけでなく、端的に、最高度に知恵である。
理由は以下の通り。知者には、秩序付けることと判断することが属するが、判
断は、より高い原因を通して、より下位のものについて持たれるので、どの類
においても、その類の最高の原因を考察するものが、知者と言われる。たとえ
ば、建築の類では、木材を削り、石を準備する下位の職人たちとの関係におい
て、家の形を整える建築家が、知者であり棟梁と言われる。このことから、
『コリントの信徒への手紙一』第3章で、「知者であり棟梁である者として、
私は基礎を置いた」と言われる。


\\


Et rursus, in genere totius humanae vitae, prudens sapiens dicitur,
inquantum ordinat humanos actus ad debitum finem, unde dicitur
{\itshape Prov}.~{\scshape x}, {\itshape sapientia est viro
prudentia}. Ille igitur qui considerat simpliciter altissimam causam
totius universi, quae Deus est, maxime sapiens dicitur, unde et
sapientia dicitur esse {\itshape divinorum cognitio}, ut patet per
Augustinum, XII {\itshape de Trinitate}.


&


さらに、人間の生全体の類において、思慮が知恵と言われるが、それは、人間
の行為をしかるべき目的へ秩序付ける限りにおいてである。それゆえ、『箴言』
第10章で「人にとって思慮は知恵である」と言われる。ゆえに、端的に、世界
全体の最高原因を考察する者、すなわち神が、最高度に知者と言われる。それ
ゆえ、アウグスティヌス『三位一体論』第12巻で明らかなとおり、知恵は、
「神的な事柄についての認識」とも言われる。

\\


Sacra autem doctrina propriissime determinat de Deo secundum quod est
altissima causa, quia non solum quantum ad illud quod est per
creaturas cognoscibile (quod philosophi cognoverunt, ut dicitur
{\itshape Rom}.~{\scshape i}, {\itshape quod notum est Dei, manifestum
est illis}); sed etiam quantum ad id quod notum est sibi soli de
seipso, et aliis per revelationem communicatum. Unde sacra doctrina
maxime dicitur sapientia.


&

しかし、もっとも固有の意味では、神聖な教えは、神について、それが最高原
因であるかぎりにおいて、論じる。なぜなら、それは、被造物をとおして認識
されうるものだけでなく(『ローマの信徒への手紙』第1章「神について知ら
れることは、彼らに明らかである」によれば、哲学者たちがこれらを認識す
る)、それ自体についてそれ自体で知られることや、啓示によって他の人々に
伝えられるものにかんしても論じるからである。したがって、神聖な教えは、
最大限に知恵と言われる。


\\


{\scshape Ad primum ergo dicendum} quod sacra doctrina non supponit
sua principia ab aliqua scientia humana, sed a scientia divina, a qua,
sicut a summa sapientia, omnis nostra cognitio ordinatur.


&

第一異論に対しては、それゆえ、以下のように言われるべきである。神聖な教
えは、人間的な学の何かから自らの原理を受け取るのではなく、神的な学から
である。私たちのすべての認識は、最高の知恵によって秩序付けられるように
して、それによって秩序付けられている。


\\


{\scshape Ad secundum dicendum} quod aliarum scientiarum principia vel
sunt per se nota, et probari non possunt, vel per aliquam rationem
naturalem probantur in aliqua alia scientia. Propria autem huius
scientiae cognitio est, quae est per revelationem, non autem quae est
per naturalem rationem.


&

第二異論に対しては、以下のように言われるべきである。他の諸学の原理は、
自明であり証明されえないか、あるいは、他の何かの学の中で、自然的理性に
よって証明されるかである。しかし、この学の認識に固有なのは、それが啓示
によってあり、自然理性によってあるのではないということである。
%% 学の原理が啓示によって得られるとはどういうこと?

\\

Et ideo non pertinet ad eam probare principia aliarum scientiarum, sed
solum iudicare de eis, quidquid enim in aliis scientiis invenitur
veritati huius scientiae repugnans, totum condemnatur ut falsum, unde
dicitur II {\itshape Cor}.~{\scshape x}, {\itshape consilia
destruentes, et omnem altitudinem extollentem se adversus scientiam
Dei}.


&

ゆえに、他の諸学の原理を証明することは、この学に属さず、ただ、それらに
ついて判断することが属する。つまり、なんであれ、この学の真理に反するも
のが他の諸学に見出されるならば、その全体が偽と断罪される。このことから、
『コリントの信徒への手紙二』第10章で、「思慮と、神の知に反対して自らを
高めるすべての高みをを破壊し」と言われる。


\\


{\scshape Ad tertium dicendum} quod, cum iudicium ad sapientem
pertineat, secundum duplicem modum iudicandi, dupliciter sapientia
accipitur.

&

第三異論に対しては、以下のように言われるべきである。判断は、知者に属す
るので、二つの判断のしかたに即して、二通りに知恵が理解される。


\\

Contingit enim aliquem iudicare, uno modo per modum inclinationis,
sicut qui habet habitum virtutis, recte iudicat de his quae sunt
secundum virtutem agenda, inquantum ad illa inclinatur, unde et in X
{\itshape Ethic}.~dicitur quod virtuosus est mensura et regula actuum
humanorum.

&

すなわち、ある人が判断するとき、一つには、傾向性によって判断することが
ある。たとえば、徳の習慣を持つ人は、為されるべき徳に即してある事柄につ
いて、それへと傾く限りにおいて正しく判断する。このことから、『ニコマコ
ス倫理学』第10巻で、「有徳な人は、人々の行為の尺度であり基準である」と
言われている。


\\

Alio modo, per modum cognitionis, sicut aliquis instructus in scientia
morali, posset iudicare de actibus virtutis, etiam si virtutem non
haberet.

&

もう一つには、認識のしかたによる。たとえば、道徳学において訓練された人
が、徳を持っていなくても、有徳な行為について判断するように。

\\

Primus igitur modus iudicandi de rebus divinis, pertinet ad sapientiam
quae ponitur donum spiritus sancti secundum illud I Cor. II,
spiritualis homo iudicat omnia, etc., et Dionysius dicit, II cap. de
divinis nominibus, Hierotheus doctus est non solum discens, sed et
patiens divina.

&

それゆえ、神的な事柄について判断する第一のしかたは、かの『コリントの信
徒への手紙一』第2章「霊的な人はすべてを判断する云々」や、ディオニュシ
ウスが『神名論』第2章で「ヒエロテウスは神的な事柄を学ぶだけでなく受け
取って学識がある」 と述べていることにしたがえば、聖霊の贈り物とされる。


\\

Secundus autem modus iudicandi pertinet ad hanc doctrinam, secundum
quod per studium habetur; licet eius principia ex revelatione
habeantur.


&

これに対して、判断する第二のしかたは、勉強によって所有される限りで、こ
の学に属する。ただし、その原理は啓示によって所有されるのだが。


\end{longtable}
\newpage

\rhead{a.~7}
\begin{center}
 {\Large {\bf ARTICULUS SEPTIMUS}}\\
 {\large UTRUM DEUS SIT SUBIECTUM HUIUS SCIENTIAE}\\
 {\footnotesize Supra, a.3, ad 1; I {\itshape Sent.}, Prol., a.4; in
 Boet.~{\itshape De Trin.}, q.5, a.4.}\\
 {\Large 第七項\\この学の主題は神か}
\end{center}

\begin{longtable}{p{21em}p{21em}}


{\Huge A}{\scshape d septimum sic proceditur}. Videtur quod Deus non
sit subiectum huius scientiae. In qualibet enim scientia oportet
supponere de subiecto {\itshape quid est}, secundum philosophum in I
{\itshape Poster}. Sed haec scientia non supponit de Deo quid est,
dicit enim Damascenus, {\itshape in Deo quid est, dicere impossibile
est}. Ergo Deus non est subiectum huius scientiae.


&

第七項の問題へ、議論は以下のように進められる。神はこの学の主題でないと
思われる。理由は以下の通り。『分析論後書』第1巻の哲学者によれば、どの
学でも、その主題について、それが何かを提示しなければならない。しかし、
この学は、神について、それが何かを提示しない。なぜなら、ダマスケヌスは、
「神において、何であるかを語ることは不可能である云々」と述べているから
である。ゆえに神はこの学の主題ではない。


\\


{\scshape 2 Praeterea}, omnia quae determinantur in aliqua scientia,
comprehenduntur sub subiecto illius scientiae. Sed in sacra Scriptura
determinatur de multis aliis quam de Deo, puta de creaturis, et de
moribus hominum. Ergo Deus non est subiectum huius scientiae.


&

さらに、すべてなんらかの学の中で論じられることは、その学の主題のもとに
含まれる。しかし、聖書の中で、神以外の多くのこと、たとえば被造物や人間
の行いなどについて論じられる。ゆえに神はこの学の主題ではない。


\\


{\scshape Sed contra}, illud est subiectum scientiae, de quo est sermo
in scientia. Sed in hac scientia fit sermo de Deo, dicitur enim
theologia, quasi sermo de Deo. Ergo Deus est subiectum huius
scientiae.


&


しかし反対に、それについて学の中で教説があるものが、学の主題である。し
かし、この学においては、神についての教説がなされ、神についての教説、と
いう意味で、神学と言われる。ゆえに、神はこの学の主題である。


%% このSed Contraが正しいならば、Sacra doctrinaとは神学のこと。つまり神
%% 学大全のこと。

\\


{\scshape Respondeo dicendum} quod Deus est subiectum huius
scientiae. Sic enim se habet subiectum ad scientiam, sicut obiectum ad
potentiam vel habitum. Proprie autem illud assignatur obiectum
alicuius potentiae vel habitus, sub cuius ratione omnia referuntur ad
potentiam vel habitum, sicut homo et lapis referuntur ad visum
inquantum sunt colorata, unde coloratum est proprium obiectum visus.



&

解答する。以下のように言われるべきである。神はこの学の主題である。理由
は以下の通り。主題と学の関係は、対象と能力ないし習態との関係と同じであ
る。ところで、ある能力ないし習態の対象に指定されるのは、それの性格のも
とに、すべてがその能力や習態に関係づけられるものである。たとえば、人間
と石は「色があるもの」である限りで視覚に関係づけられるので、視覚の対象
は「色があるもの」であるように。


\\


Omnia autem pertractantur in sacra doctrina sub ratione Dei, vel quia
sunt ipse Deus; vel quia habent ordinem ad Deum, ut ad principium et
finem. Unde sequitur quod Deus vere sit subiectum huius scientiae.




&

ところで、すべてのものは、神の性格のもとに、あるいは神であるために、あ
るいは、根源や目的としての神に秩序付けられるかぎりで、神聖な教えの中で
論じられる。したがって、神は、真に、この学の主題であることが帰結する。



\\

Quod etiam manifestum fit ex principiis huius scientiae, quae sunt
articuli fidei, quae est de Deo, idem autem est subiectum principiorum
et totius scientiae, cum tota scientia virtute contineatur in
principiis.


&

このことは、この学の原理からも明らかとされる。すなわち、この学の原理は
信仰箇条であるが、これは神にかんする。ところが、原理と学全体の主題は同
一である。なぜなら、学全体は、原理の中に潜在的に含まれているからである。


\\

Quidam vero, attendentes ad ea quae in ista scientia tractantur, et
non ad rationem secundum quam considerantur, assignaverunt aliter
subiectum huius scientiae, vel res et signa; vel opera reparationis;
vel totum Christum, idest caput et membra. De omnibus enim istis
tractatur in ista scientia, sed secundum ordinem ad Deum.


&

しかし、ある人々は、それに即して考察される性格ではなく、この学で論じら
れることに注目して、この学の主題を違うかたちで指定した。事柄としるしで
あるとか、償いの業であるとか、キリストの全体、すなわち、頭と四肢である
とか。じっさい、これらすべてについて、この学の中で論じられるが、それは
神への秩序に即してである。


\\


{\scshape Ad primum ergo dicendum} quod, licet de Deo non possimus
scire quid est, utimur tamen eius effectu, in hac doctrina, vel
naturae vel gratiae, loco definitionis, ad ea quae de Deo in hac
doctrina considerantur, sicut et in aliquibus scientiis philosophicis
demonstratur aliquid de causa per effectum, accipiendo effectum loco
definitionis causae.


&

第一異論に対しては、それゆえ、以下のように言われるべきである。神につい
て、私たちはその何であるかを知ることができないが、しかし、私たちはこの
教えの中で、その結果を、自然の結果であれ恩恵の結果であれ、この教えの中
で神について考察される事柄のために、定義の代わりに用いる。それはちょう
ど、ある哲学的諸学の中で、原因の定義の代わりに結果を当てはめることによっ
て、ある事柄が結果を通して原因について論証されるようにである。


\\


{\scshape Ad secundum dicendum} quod omnia alia quae determinantur in
sacra doctrina, comprehenduntur sub Deo, non ut partes vel species vel
accidentia, sed ut ordinata aliqualiter ad ipsum.


&

第二異論に対しては、以下のように言われるべきである。神聖な教えの中で論
じられる他のすべてのことは、部分や種や附帯性として神のもとに含まれるの
ではなく、何らかのかたちで、神へと秩序付けられるものとしてである。


\end{longtable}
\newpage


\rhead{a.~8}
\begin{center}
 {\Large {\bf ARTICULUS OCTAVUS}}\\
 {\large UTRUM HAEC DOCTRINA SIT ARGUMENTATIVA}\\
 {\footnotesize II-IIae, q.1, a.5, ad 1; I {\itshape Sent.}, Prol., a.5;
 I {\itshape SCG}, cap.9; in Boet.~{\itshape De Trin.}, q.2, a.3;
 {\itshape Quodlib}.~IV, q.9, a.3.}\\
 {\Large 第八項\\この学は推論的か}
\end{center}

\begin{longtable}{p{21em}p{21em}}

{\Huge A}{\scshape d octavum sic proceditur}. Videtur quod haec
doctrina non sit argumentativa. Dicit enim Ambrosius in libro I
{\itshape de fide Catholica}, {\itshape tolle argumenta, ubi fides
quaeritur}. Sed in hac doctrina praecipue fides quaeritur, unde
dicitur {\itshape Ioan}.~{\scshape xx}, {\itshape haec scripta sunt ut
credatis}. Ergo sacra doctrina non est argumentativa.


&

第八項の問題へ、議論は以下のように進められる。この教えは推論的でないと
思われる。理由は以下の通り。アンブロシウスは『カトリックの信仰について』
第1巻で「信仰が求められるところでは推論を捨てよ」と言っている。しかし、
この教えにおいて、とくに信仰が求められる。ゆえに、『ヨハネによる福音書』
第20章「あなたたちが信じるためにこの書がある」と言われる。ゆえに、神聖
な教えは推論的でない。

%% この異論だと、sacra doctrinaはキリスト教信仰そのもののこと。

\\


{\scshape 2 Praeterea}, si sit argumentativa, aut argumentatur ex
auctoritate, aut ex ratione. Si ex auctoritate, non videtur hoc
congruere eius dignitati, nam locus ab auctoritate est infirmissimus,
secundum Boetium. Si etiam ex ratione, hoc non congruit eius fini,
quia secundum Gregorium in homilia, {\itshape fides non habet meritum,
ubi humana ratio praebet experimentum}. Ergo sacra doctrina non est
argumentativa.


&

さらに、もし推論的であるならば、それは、権威に基づくか、あるいは理性に
基づくかのどちらかである。もし、権威に基づくならば、それは、この教えの
威厳に合致しない。なぜなら、ボエティウスによれば、権威によるものは、もっ
とも下位にあるからである。また、もし理性に基づくならば、それは、その目
的に合致しない。なぜなら、グレゴリウス『講解』によれば、「人間の理性が
証明を与えるところでの信仰は功績を持たないい」からである。ゆえに、神聖
な教えは推論的でない。


\\


{\scshape Sed contra} est quod dicitur {\itshape ad Titum} {\scshape
i}, de episcopo, {\itshape amplectentem eum qui secundum doctrinam
est, fidelem sermonem, ut potens sit exhortari in doctrina sana, et
eos qui contradicunt arguere}.


&

しかし反対に、『テトスヘの手紙』第1章で、司教について「教えに即してい
る信仰深い言葉を抱き、健全な教えにおいて励まされ、反論する者たちを説得
するように」と言われている。


\\


{\scshape Respondeo dicendum} quod, sicut aliae scientiae non
argumentantur ad sua principia probanda, sed ex principiis
argumentantur ad ostendendum alia in ipsis scientiis; ita haec
doctrina non argumentatur ad sua principia probanda, quae sunt
articuli fidei; sed ex eis procedit ad aliquid aliud ostendendum;
sicut apostolus, I {\itshape ad Cor}.~{\scshape xv}, ex resurrectione
Christi argumentatur ad resurrectionem communem probandam.



&

解答する。以下のように言われるべきである。ちょうど、他の諸学が、自らの
諸原理へと推論することがなく、むしろ諸原理から、その学の中の他の事柄を
示すために推論するように、この教えも、自らの諸原理、つまり信仰箇条を証
明するために推論せず、それらから何か他のものを示すために論を進める。ちょ
うど、使徒が『コリントの信徒への手紙一』第15章で、キリストの復活から、
共通の復活を証明することへ推論しているように。


\\


Sed tamen considerandum est in scientiis philosophicis, quod
inferiores scientiae nec probant sua principia, nec contra negantem
principia disputant, sed hoc relinquunt superiori scientiae, suprema
vero inter eas, scilicet metaphysica, disputat contra negantem sua
principia, si adversarius aliquid concedit, si autem nihil concedit,
non potest cum eo disputare, potest tamen solvere rationes ipsius.



&

しかし、以下のことが考察されるべきである。哲学的諸学においては、下位の
学が、自らの諸原理を証明することも、その原理を否定する人に反対して議論
することもせず、それを上位の学に委ねるが、そのうちの最高の学、すなわち
形而上学は、反対者が何かを譲歩するならば、その原理を否定する者に反論す
るが、もし何も譲歩しないなら、そのものと議論することができない。ただし、
彼の議論に解答することはできるが。



\\


Unde sacra Scriptura, cum non habeat superiorem, disputat cum negante
sua principia, argumentando quidem, si adversarius aliquid concedat
eorum quae per divinam revelationem habentur; sicut per auctoritates
sacrae doctrinae disputamus contra haereticos, et per unum articulum
contra negantes alium.



&

したがって、神聖な教えは、それより上位の学を持たないので、もし反対者が、
神の啓示によって持たれる事柄の何かを認めるならば、その原理を否定する者
と、推論によって、議論する。ちょうど、私たちが、異端者たちと、神聖な教
えの権威によって議論し、一つの信仰箇条によって、他の信仰箇条を否定する
者に反対して議論するように。


\\


Si vero adversarius nihil credat eorum quae divinitus revelantur, non
remanet amplius via ad probandum articulos fidei per rationes, sed ad
solvendum rationes, si quas inducit, contra fidem. Cum enim fides
infallibili veritati innitatur, impossibile autem sit de vero
demonstrari contrarium, manifestum est probationes quae contra fidem
inducuntur, non esse demonstrationes, sed solubilia argumenta.


&

他方、反対者が、神によって啓示された事柄を何も認めない場合には、理性に
よって信仰箇条を証明する道にそれ以上とどまることをせず、むしろ、彼が何
か信仰に反すること論じるならば、それに解答する道を行く。じっさい、信仰
は、不可謬の真理に発し、真に、それに反することが証明されることは不可能
であるから、信仰に反することを導く議論は、証明ではなく、間違いが示され
うる議論であることが明らかである。


\\


{\scshape Ad primum ergo dicendum} quod, licet argumenta rationis
humanae non habeant locum ad probandum quae fidei sunt, tamen ex
articulis fidei haec doctrina ad alia argumentatur, ut dictum est.


&

第一異論に対しては、それゆえ、以下のように言われるべきである。人間理性
の議論は、信仰に属するものを証明する立場にないが、しかし、すでに述べら
れたとおり、信仰箇条に基づいて、この教えを他の学に対して論じることがで
きる。




\\


{\scshape Ad secundum dicendum} quod argumentari ex auctoritate est
maxime proprium huius doctrinae, eo quod principia huius doctrinae per
revelationem habentur, et sic oportet quod credatur auctoritati eorum
quibus revelatio facta est.

&

第二異論に対しては、以下のように言われるべきである。権威に基づいて論じ
ることは、この教えに最大限に属する。なぜなら、この教えの原理は、啓示に
よって持たれ、それゆえ、それによって啓示が行われたものどもの権威が信じ
られるべきだからである。

\\

Nec hoc derogat dignitati huius doctrinae, nam licet locus ab
auctoritate quae fundatur super ratione humana, sit infirmissimus;
locus tamen ab auctoritate quae fundatur super revelatione divina, est
efficacissimus. Utitur tamen sacra doctrina etiam ratione humana, non
quidem ad probandum fidem, quia per hoc tolleretur meritum fidei; sed
ad manifestandum aliqua alia quae traduntur in hac doctrina.


&


またこのことが、この学の価値を損なうことはない。たしかに、人間理性に基
づく権威からの議論は、 もっとも弱いものだが、神の啓示に基づく権威から
の議論は、もっとも強力である。しかし、神聖な教えは、人間理性を用いるが、
それは信仰を証明するためにではない。もしそうだとすると、信仰の功徳をな
くしてしまうだろうから。そうではなく、この教えの中で伝えられる他の何か
を明示するためにである。



\\

Cum enim gratia non tollat naturam, sed perficiat, oportet quod
naturalis ratio subserviat fidei; sicut et naturalis inclinatio
voluntatis obsequitur caritati. Unde et apostolus dicit, II {\itshape
ad Cor}.~{\scshape x}, {\itshape in captivitatem redigentes omnem
intellectum in obsequium Christi}.


&

じっさい、恩恵は自然を破壊せず、むしろ完成させるのだから、自然本性的理
性は、信仰に奉仕するのでなければならない。たとえば、意志の自然本性的傾
向性が、愛徳に奉仕するように。このことから、使徒も、『コリントの信徒へ
の手紙二』第10章で「捕らえて、すべての知性をキリストに従わせる人々は」
\footnote{「神の知識に逆らうあらゆる高慢を打ち倒し、あらゆる思惑をとり
こにしてキリストに従わせ」(10:5)}と言っている。


\\

Et inde est quod etiam auctoritatibus philosophorum sacra doctrina
utitur, ubi per rationem naturalem veritatem cognoscere potuerunt;
sicut Paulus, {\itshape Actuum} {\scshape xvii}, inducit verbum Arati,
dicens, {\itshape sicut et quidam poetarum vestrorum dixerunt, genus
Dei sumus}. Sed tamen sacra doctrina huiusmodi auctoritatibus utitur
quasi extraneis argumentis, et probabilibus.


&

そしてそれゆえ、神聖な教えは、自然本性的理性によって真理を認識すること
ができるところでは、哲学者たちの権威もまた用いる。たとえば、パウロは
『使徒言行録』第17章で、アラトスの言葉を引いて次のように語る。「あなた
たちの詩人の誰かが言ったように、私たちは神の一族である」\footnote{「皆
さんのうちのある詩人たちも、/『我らは神の中に生き、動き、存在する』/
『我らもその子孫である』と、/言っているとおりです。 」(17:28)}しかし、
聖書はこのような権威を言わば外的で証明可能な議論として用いる。


\\

Auctoritatibus autem canonicae Scripturae utitur proprie, ex
necessitate argumentando. Auctoritatibus autem aliorum doctorum
Ecclesiae, quasi arguendo ex propriis, sed probabiliter. Innititur
enim fides nostra revelationi apostolis et prophetis factae, qui
canonicos libros scripserunt, non autem revelationi, si qua fuit aliis
doctoribus facta.


&

聖書正典の権威は、これを議論の必要性に基づいて、固有の意味で用いる。ま
た、教会の他の学者たちの権威は、これを、固有のものに基づいて、議論を明
らかにするものとして用いるが、絶対に正しいものとしてではない。じっさい、
私たちの信仰は、正典の書物を書いた使徒や預言者になされた啓示に依拠する
のであって、もし他の学者たちに啓示がなされたとすれば、そのような啓示に
依拠するのではない。

\\

Unde dicit Augustinus, in epistola ad Hieronymum, {\itshape solis eis
Scripturarum libris qui canonici appellantur, didici hunc honorem
deferre, ut nullum auctorem eorum in scribendo errasse aliquid
firmissime credam. Alios autem ita lego, ut, quantalibet sanctitate
doctrinaque praepolleant, non ideo verum putem, quod ipsi ita
senserunt vel scripserunt}.


&

このことから、アウグスティヌスは『ヒエロニュムス宛書簡』で以下のように
述べている。「私は、聖書の中で正典と呼ばれる書物だけに、彼らのうちのど
の著者も、決して書くときに誤らなかったと私が信じるという名誉を与えるこ
とを学びました。しかし、他の書物は、私は、どれほど神々しさや学識が際立っ
ていても、彼らがそう考えたり書いたりしたから、それが真だと考えないよう
に読みます」。

\end{longtable}
\newpage


\rhead{a.~9}
\begin{center}
 {\Large {\bf ARTICULUS NONUS}}\\
 {\large UTRUM SACRA SCRIPTURA DEBEAT UTI METAPHORIS}\\
 {\footnotesize I {\itshape Sent.}, Prol., a.5; d.34, q.3, a.1, 2; III
 {\itshape SCG}, cap.119; in Boet.~{\itshape de Trin.}, q.2, a.4.}\\
 {\Large 第九項\\聖書は比喩を用いるべきか}
\end{center}

\begin{longtable}{p{21em}p{21em}}


{\Huge A}{\scshape d nonum sic proceditur}. Videtur quod sacra
Scriptura non debeat uti metaphoris. Illud enim quod est proprium
infimae doctrinae, non videtur competere huic scientiae, quae inter
alias tenet locum supremum, ut iam dictum est. Procedere autem per
similitudines varias et repraesentationes, est proprium poeticae, quae
est infima inter omnes doctrinas. Ergo huiusmodi similitudinibus uti,
non est conveniens huic scientiae.


&

第九項の問題へ、議論は以下のように進められる。聖書は比喩を用いるべきで
ないと思われる。理由は以下の通り。最低の教えに固有の事柄が、すでに言わ
れたとおり、他の諸学の中で最高の位置を占めるこの学にふさわしいとは思え
ない。しかし、さまざまな類似や表象をとおして進むことは、すべての教えの
中で最も低い詩学に固有である。ゆえに、そのような類似を用いることは、こ
の学にふさわしくない。



\\


{\scshape 2 Praeterea}, haec doctrina videtur esse ordinata ad
veritatis manifestationem, unde et manifestatoribus eius praemium
promittitur, {\itshape Eccli}.~{\scshape xxiv}, {\itshape qui
elucidant me, vitam aeternam habebunt}. Sed per huiusmodi
similitudines veritas occultatur. Non ergo competit huic doctrinae
divina tradere sub similitudine corporalium rerum.


&

さらに、この教えは、真理の明示へと秩序付けられていると思われる。それゆ
え、それを明示する人々にも、報償が与えられる。『集会の書』24章「私を明
らかにする者たちは、永遠の生命を持つだろう」\footnote{新共同訳聖書に該
当箇所なし。}。しかし、このような類似によって、真理は隠される。ゆえに、
この教えにとって、神の事柄を物体的事物の類似のもとに伝えることはふさわ
しくない。


\\


{\scshape 3 Praeterea}, quanto aliquae creaturae sunt sublimiores,
tanto magis ad divinam similitudinem accedunt. Si igitur aliquae ex
creaturis transumerentur ad Deum, tunc oporteret talem transumptionem
maxime fieri ex sublimioribus creaturis, et non ex infimis. Quod tamen
in Scripturis frequenter invenitur.


&

さらに、被造物は、より高位にあるほど、より神の類似に近づく。ゆえに、も
し、被造物の何かが神になぞらえられるならば、その場合には、最下位のもの
ではなく、より高位の被造物に基づいて、そのような喩えがなされるべきであ
ろう。そして、聖書では、そのようなことが頻繁に見出される。


\\


{\scshape Sed contra est} quod dicitur {\itshape Osee} {\scshape xii},
ego visionem multiplicavi eis, et in manibus prophetarum assimilatus
sum. Tradere autem aliquid sub similitudine, est metaphoricum. Ergo ad
sacram doctrinam pertinet uti metaphoris.


&

しかし反対に、『ホセア書』第12章「私は、見たことを、多くの人に多数化し
た。そして、預言者たちの手において、私に似たものとした」\footnote{「わ
たしは預言者たちに言葉を伝え/多くの幻を示し/預言者たちによってたとえ
を示した。」(12:11)}と言われている。しかし、何かを類似のもとに伝えるこ
とは、比喩である。ゆえに、神聖な教えには、比喩を用いることが属する。



\\


{\scshape Respondeo dicendum} quod conveniens est sacrae Scripturae
divina et spiritualia sub similitudine corporalium tradere. Deus enim
omnibus providet secundum quod competit eorum naturae. Est autem
naturale homini ut per sensibilia ad intelligibilia veniat, quia omnis
nostra cognitio a sensu initium habet. Unde convenienter in sacra
Scriptura traduntur nobis spiritualia sub metaphoris corporalium.


&

解答する。以下のように言われるべきである。聖書にとって、神や霊のことを、
物体の類似のもとに伝えることが便利である。理由は以下の通り。神は、万物
を、その本性に適したものに即して摂理する。ところが、人間には、可感的な
ものや可知的なものを通して進むことが適合している。なぜなら、私たちの認
識はすべて、感覚に端緒を発するからである。ゆえに、聖書の中で、霊的な事
柄が、物体的な比喩のもとで私たちに伝えられることが便利である。



\\

Et hoc est quod dicit Dionysius, I cap. {\itshape Caelestis
Hierarchiae}, {\itshape impossibile est nobis aliter lucere divinum
radium, nisi varietate sacrorum velaminum circumvelatum}.


&

そしてこれが、ディオニュシウスが『天上階級論』第1章で「さまざまな聖な
るヴェールで覆われてれていなければ、それ以外のしかたで、神の光線が私た
ちに照らすことは不可能である」と言っている意味である。

\\

Convenit etiam sacrae Scripturae, quae communiter omnibus proponitur
(secundum illud ad {\itshape Rom}.~{\scshape i}, {\itshape sapientibus
et insipientibus debitor sum}), ut spiritualia sub similitudinibus
corporalium proponantur; ut saltem vel sic rudes eam capiant, qui ad
intelligibilia secundum se capienda non sunt idonei.


&


さらに、聖書は、共通に、すべての人に示されているので(『ローマの信徒へ
の手紙』第1章「知恵ある人にもそうでない人にも、私はすべきことがある」
\footnote{「わたしは、ギリシア人にも未開の人にも、知恵のある人にもない
人にも、果たすべき責任があります。」(1:14)})、霊的なことを、物体的な
類似のもとに提示することや、可知的なことをそれ自体において捕らえること
に適していない粗野な人々が、聖書を理解しやすいようにすることが、適して
いる。


\\


{\scshape Ad primum ergo dicendum} quod poeta utitur metaphoris
propter repraesentationem, repraesentatio enim naturaliter homini
delectabilis est. Sed sacra doctrina utitur metaphoris propter
necessitatem et utilitatem, ut dictum est.


&

第一異論に対しては、それゆえ、以下のように言われるべきである。詩人は、
表象のために、比喩を用いる。表象は、本性的に、人間にとって快適だからで
ある。しかし、すでに述べられたとおり、神聖な教えは、比喩を必要性と有益
さのために用いる。


\\


{\scshape Ad secundum dicendum} quod radius divinae revelationis non
destruitur propter figuras sensibiles quibus circumvelatur, ut dicit
Dionysius, sed remanet in sua veritate; ut mentes quibus fit
revelatio, non permittat in similitudinibus permanere, sed elevet eas
ad cognitionem intelligibilium; et per eos quibus revelatio facta est,
alii etiam circa haec instruantur. Unde ea quae in uno loco Scripturae
traduntur sub metaphoris, in aliis locis expressius exponuntur. Et
ipsa etiam occultatio figurarum utilis est, ad exercitium studiosorum,
et contra irrisiones infidelium, de quibus dicitur, {\itshape
Matth}.~{\scshape vii}, {\itshape nolite sanctum dare canibus}.


&

第二異論に対しては、以下のように言われるべきである。ディオニュシウスが
言うように、神の啓示の光線は、それを隠す可感的な形態のために破壊された
りはせず、むしろ、その真理にとどまるので、啓示が下される精神が、その類
似にとどまることを許さず、可知的なものの認識へ、精神を高める。そして啓
示が下される人々を通して、他の人々が、それについて、教えられる。したがっ
て、聖書のある箇所で、比喩のもとに伝えられることが、他の箇所でははっき
りと説明される。また、さまざまな形態のもとに隠すことは、勉学に励む者の
訓練や、冷笑的な不信心者への対抗のために有益である。この後者について、
『マタイによる福音書』第7章「聖なるものを犬に与えるな」\footnote{「神
聖なものを犬に与えてはならず、また、真珠を豚に投げてはならない。それを
足で踏みにじり、向き直ってあなたがたにかみついてくるだろう。」(7:6)}は
語られている。

\\


{\scshape Ad tertium dicendum} quod, sicut docet Dionysius, cap. II
{\itshape Cael.~Hier}., magis est conveniens quod divina in Scripturis
tradantur sub figuris vilium corporum, quam corporum nobilium. Et hoc
propter tria.


&


第三異論に対しては、以下のように言われるべきである。ディオニュシウスが
『天上階級論』第2章で教えているとおり、聖書の中で、神の事柄は、高貴な
物体よりも卑俗な物体の形態のもとに伝えられる方が、より適している。これ
は、三つの理由による。


\\

Primo, quia per hoc magis liberatur humanus animus ab
errore. Manifestum enim apparet quod haec secundum proprietatem non
dicuntur de divinis, quod posset esse dubium, si sub figuris nobilium
corporum describerentur divina; maxime apud illos qui nihil aliud a
corporibus nobilius excogitare noverunt.

&

第一に、このことによって、人間の心はより誤謬から解放される。というのも、
それが、神のことについて、文字通りの厳密な意味で語られているのではない
ことが一目瞭然だからである。とくに、高貴な物体以外のものを考えられない
人々にとっては、もしそれが高貴な物体の形態によって記述されていたならば、
疑わしかっただろう。


\\

Secundo, quia hic modus convenientior est cognitioni quam de Deo
habemus in hac vita. Magis enim manifestatur nobis de ipso quid non
est, quam quid est, et ideo similitudines illarum rerum quae magis
elongantur a Deo, veriorem nobis faciunt aestimationem quod sit supra
illud quod de Deo dicimus vel cogitamus.


&

第二に、この語り方が、この生において私たちが神についてもつ認識には適し
ている。なぜなら、私たちには、それが何でないかの方が、何であるかよりも
明らかに示されるので、神から遠く離れている事物の類似が、私たちが神につ
いて語り考えることを越えて、より正しい判断を私たちの中に作るからである。


\\

Tertio, quia per huiusmodi, divina magis occultantur indignis.

&


第三に、このようなものによって、神の事柄が、よりそれにふさわしくない者
から隠される。

\end{longtable}
\newpage






\rhead{a.~10}
\begin{center}
 {\Large {\bf ARTRICULUS DECIMUS}}\\
 {\large UTRUM SACRA SCRIPTURA SUB UNA LITTERA HABEAT PLURES SENSUS}\\
 {\footnotesize I {\itshape Sent.}, Prol.~a.5; IV, d.21, q.1, a.2, qu$^a$ 1, ad 3; {\itshape de Pot.}, q.4, a.1; {\itshape Quodl}.~III, q.14, a.1; VII, q.6, per tot.; {\itshape ad Gal.}, cap.4, lect.7.}\\
 {\Large 第十項\\聖書は一つの言葉のもとに複数の意味を持つか}
\end{center}

\begin{longtable}{p{21em}p{21em}}

{\Huge A}{\scshape d decimum sic proceditur}. Videtur quod sacra
Scriptura sub una littera non habeat plures sensus, qui sunt
historicus vel litteralis, allegoricus, tropologicus sive moralis, et
anagogicus. Multiplicitas enim sensuum in una Scriptura parit
confusionem et deceptionem, et tollit arguendi firmitatem, unde ex
multiplicibus propositionibus non procedit argumentatio, sed secundum
hoc aliquae fallaciae assignantur. Sacra autem Scriptura debet esse
efficax ad ostendendam veritatem absque omni fallacia. Ergo non debent
in ea sub una littera plures sensus tradi.



&

第十項の問題へ、議論は以下のように進められる。聖書は、一つの言葉のもと
に複数の意味、すなわち、歴史的、字義的、寓意的、トポス的ないし道徳的、
そして神秘的意味、をもたないと思われる。理由は以下の通り。一つの書物の
中に複数の意味があれば、混乱と欺きをもたらし、論証の堅固さをだめにする。
したがって、多重命題から推論は生まれず、これに従えば、何らかの虚偽が指
定される。しかし、聖書は、全くの誤りなく真理を明示する力があるべきであ
る。ゆえに、その中に、複数の意味を伝える一つの言葉があるべきでない。


\\


{\scshape 2 Praeterea}, Augustinus dicit in libro {\itshape De
Utilitate Credendi}, quod {\itshape Scriptura quae testamentum vetus
vocatur, quadrifariam traditur, scilicet, secundum historiam, secundum
aetiologiam, secundum analogiam, secundum allegoriam}. Quae quidem
quatuor a quatuor praedictis videntur esse aliena omnino. Non igitur
conveniens videtur quod eadem littera sacrae Scripturae secundum
quatuor sensus praedictos exponatur.



&

さらに、アウグスティヌスは、『信仰の有用さ』で、以下のように述べている。
「旧約と呼ばれる聖書は、四通りのしかたで伝えられている。すなわち、歴史
に即して、原因に即して、類比に即して、そして寓意に即して」。これら四つ
は、上で述べられている四つとまったく異なるように思われる。ゆえに、聖書
の同じ言葉が、前述の四つの意味で説明されるようには思われない。


\\


{\scshape 3 Praeterea}, praeter praedictos sensus, invenitur sensus
parabolicus, qui inter illos sensus quatuor non continetur.


&


さらに、上述の意味の他に、寓話的意味があり、これは上の四つの意味に含ま
れていない。

\\


{\scshape Sed contra est} quod dicit Gregorius, XX {\itshape
Moralium}, {\itshape sacra Scriptura omnes scientias ipso locutionis
suae more transcendit, quia uno eodemque sermone, dum narrat gestum,
prodit mysterium}.


&

しかし反対に、グレゴリウスは『道徳論』第20巻で次のように述べている。
「聖書は、自らの語り方によってあらゆる知を越える。なぜなら、一つの同じ
教説によって、為されたことを語りながら、神秘を示している」。


\\


{\scshape Respondeo dicendum} quod auctor sacrae Scripturae est Deus,
in cuius potestate est ut non solum voces ad significandum accommodet
(quod etiam homo facere potest), sed etiam res ipsas. Et ideo, cum in
omnibus scientiis voces significent, hoc habet proprium ista scientia,
quod ipsae res significatae per voces, etiam significant aliquid. Illa
ergo prima significatio, qua voces significant res, pertinet ad primum
sensum, qui est sensus historicus vel litteralis. Illa vero
significatio qua res significatae per voces, iterum res alias
significant, dicitur sensus spiritualis; qui super litteralem
fundatur, et eum supponit.



&

解答する。以下のように言われるべきである。聖書の著者は神であり、その権
能には、意味するために音声を使うだけでなく(これは人間にもできる)、事
物そのものも用いることが含まれる。ゆえに、すべての学において、音声が表
示するので、音声によって表示された事物が、さらに何かを意味することが、
その学に固有である。ゆえに、この第一の意味表示、つまり、音声が事物を表
示する作用は、第一の意味に属し、それは、歴史的、ないし字義的意味である。
他方、音声によって表示された事物が、さらに別の事物を表示する作用は、霊
的意味と言われる。それは、字義的意味に基づき、またそれを前提とする。




\\


Hic autem sensus spiritualis trifariam dividitur. Sicut enim dicit
apostolus, {\itshape ad Hebr}.~{\scshape vii}, lex vetus figura est
novae legis, et ipsa nova lex, ut dicit Dionysius in {\itshape
Ecclesiastica Hierarchia}, est figura futurae gloriae, in nova etiam
lege, ea quae in capite sunt gesta, sunt signa eorum quae nos agere
debemus. Secundum ergo quod ea quae sunt veteris legis, significant ea
quae sunt novae legis, est sensus allegoricus, secundum vero quod ea
quae in Christo sunt facta, vel in his quae Christum significant, sunt
signa eorum quae nos agere debemus, est sensus moralis, prout vero
significant ea quae sunt in aeterna gloria, est sensus anagogicus.



&

この霊的意味が、三通りに分かたれる。ちょうど使徒が『ヘブライ人への手紙』
第7章で述べるように、旧法は新法のたとえであり、また、ディオニュシウス
が『天上階級論』で述べるように、新法は、未来の栄光のたとえであるように。
さらに、新法の中で、その頭で為されたことは、私たちがしなければならない
ことのしるしである。それゆえ、旧法に属することが新法に属することを表示
するかぎりで、それは寓意的意味であり、キリストにおいて為されたことや、
キリストを表示する事柄において為されたことが、私たちがしなければならな
いことのしるしであるかぎりで、道徳的意味がある。また、永遠の栄光におい
てあるものを意味する限りで、神秘的意味がある。



\\


Quia vero sensus litteralis est, quem auctor intendit, auctor autem
sacrae Scripturae Deus est, qui omnia simul suo intellectu
comprehendit, non est inconveniens, ut dicit Augustinus XII {\itshape
Confessionum}, si etiam secundum litteralem sensum in una littera
Scripturae plures sint sensus.


&

ところで、字義的意味は、著者が意図するものだが、聖書の著者は、万物を同
時に自らの知性によって把握する神なので、アウグスティヌスが『告白』第12
巻で言うように、もし字義的意味において、聖書の一つの言葉に複数の意味が
あっても、それは不適当なことではない。


\\


{\scshape Ad primum ergo dicendum} quod multiplicitas horum sensuum
non facit aequivocationem, aut aliam speciem multiplicitatis, quia,
sicut iam dictum est, sensus isti non multiplicantur propter hoc quod
una vox multa significet; sed quia ipsae res significatae per voces,
aliarum rerum possunt esse signa. Et ita etiam nulla confusio sequitur
in sacra Scriptura, cum omnes sensus fundentur super unum, scilicet
litteralem; ex quo solo potest trahi argumentum, non autem ex his quae
secundum allegoriam dicuntur, ut dicit Augustinus in epistola contra
Vincentium Donatistam. Non tamen ex hoc aliquid deperit sacrae
Scripturae, quia nihil sub spirituali sensu continetur fidei
necessarium, quod Scriptura per litteralem sensum alicubi manifeste
non tradat.


&


第一異論に対しては、それゆえ、以下のように言われるべきである。これらの
意味の多さは、同音異義やその他の多様さの種類を生まない。なぜなら、すで
に語られたとおり、これらの意味は、一つの音声が多くのことを意味するため
にではなく、むしろ、音声によって表示された事物が、他の事物のしるしであ
りうるために、多数化されるからである。したがって、すべての意味は、一つ
の意味、すなわち字義的意味に基づくので、聖書において、いかなる混乱も帰
結しない。アウグスティヌスが、ドナティストのウィケンティウスに反論して
書簡で述べるように、字義的意味からのみ、推論は引き出されるのであって、
寓意によって語られたものからではない。しかし、このことによって、聖書の
何かが失われるわけではない。なぜなら、霊的意味のもとに含まれる信仰に必
要なことで、聖書が他の箇所で、字義的意味によって明白に伝えてないものは
ないからである。


\\


{\scshape Ad secundum dicendum} quod illa tria, historia, aetiologia,
analogia, ad unum litteralem sensum pertinent. Nam historia est, ut
ipse Augustinus exponit, cum simpliciter aliquid proponitur,
aetiologia vero, cum causa dicti assignatur, sicut cum dominus
assignavit causam quare Moyses permisit licentiam repudiandi uxores,
scilicet propter duritiam cordis ipsorum, {\itshape Matt}.~{\scshape
xix}, analogia vero est, cum veritas unius Scripturae ostenditur
veritati alterius non repugnare. Sola autem allegoria, inter illa
quatuor, pro tribus spiritualibus sensibus ponitur. Sicut et Hugo de
sancto Victore sub sensu allegorico etiam anagogicum comprehendit,
ponens in tertio suarum {\itshape Sententiarum} solum tres sensus,
scilicet historicum, allegoricum et tropologicum.


&

第二異論に対しては、以下のように言われるべきである。歴史的、原因的、類
比的というこの三つは、一つの字義的意味に属する。すなわち、歴史的意味は、
アウグスティヌス自身が説明しているように、端的に何かが提示されるときに
使われ、原因的意味は、語られたことの原因が指定されるときである。たとえ
ば、『マタイによる福音書』第19章で\footnote{「イエスは言われた。「あな
たたちの心が頑固なので、モーセは妻を離縁することを許したのであって、初
めからそうだったわけではない。」(19:8) }主が、モーゼが妻たちを離縁する
許可を出した原因、つまり彼らの心の頑なさを指定したときがそれである。他
方、類比的意味は、聖書のある箇所の真理が、別の箇所の真理と矛盾しないこ
とが示されるときである。この四つの中で、寓意的意味だけが、三つの霊的意
味の中に数えられる。たとえば、サンヴィクトルのヒューゴも、彼の『命題集』
第三巻で、寓意的意味のもとに寓話的意味を含めて、ただ三つの意味、歴史的、
寓意的、トポス的意味を示している。


\\


{\scshape Ad tertium dicendum} quod sensus parabolicus sub litterali
continetur, nam per voces significatur aliquid proprie, et aliquid
figurative; nec est litteralis sensus ipsa figura, sed id quod est
figuratum. Non enim cum Scriptura nominat Dei brachium, est litteralis
sensus quod in Deo sit membrum huiusmodi corporale, sed id quod per
hoc membrum significatur, scilicet virtus operativa. In quo patet quod
sensui litterali sacrae Scripturae nunquam potest subesse falsum.


&

第三異論に対しては、以下のように言われるべきである。神秘的意味は、字義
的意味のもとに含まれる。なぜなら、音声によって、ある事柄が固有に表示さ
れる場合と、たとえによって表示されることがあるが、たとえ自体は字義的意
味ではなく、たとえられたものが、字義的意味を持つからである。じっさい、
聖書が「神の腕」と言うとき、神の中に、そのような身体的部分があるという
字義的な意味ではなく、この部分によって表示されるもの、つまり働く力があ
るという意味である。この例で、聖書の字義的意味に、決して偽が入り込まな
いことが明らかである。

\end{longtable}
\end{document}


