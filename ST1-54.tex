\documentclass[10pt]{jsarticle} % use larger type; default would be 10pt
%\usepackage[utf8]{inputenc} % set input encoding (not needed with XeLaTeX)
%\usepackage[round,comma,authoryear]{natbib}
%\usepackage{nruby}
\usepackage{okumacro}
\usepackage{longtable}
%\usepqckage{tablefootnote}
\usepackage[polutonikogreek,english,japanese]{babel}
%\usepackage{amsmath}
\usepackage{latexsym}
\usepackage{color}

%----- header -------
\usepackage{fancyhdr}
\lhead{{\it Summa Theologiae} I, q.~54}
%--------------------

\bibliographystyle{jplain}

\title{{\bf PRIMA PARS}\\{\HUGE Summae Theologiae}\\Sancti Thomae
Aquinatis\\{\sffamily QUEAESTIO QUINQUAGESIMAQUARTA}\\DE COGNITIONE ANGELORUM}
\author{Japanese translation\\by Yoshinori {\sc Ueeda}}
\date{Last modified \today}


%%%% コピペ用
%\rhead{a.~}
%\begin{center}
% {\Large {\bf }}\\
% {\large }\\
% {\footnotesize }\\
% {\Large \\}
%\end{center}
%
%\begin{longtable}{p{21em}p{21em}}
%
%&
%
%
%\\
%\end{longtable}
%\newpage



\begin{document}
\maketitle
\pagestyle{fancy}

\begin{center}
{\Large 第五十四問\\天使たちの認識について}
\end{center}

\begin{longtable}{p{21em}p{21em}}

Consideratis his quae ad substantiam Angeli pertinent, procedendum est
ad cognitionem ipsius. Haec autem consideratio erit quadripartita, nam
primo considerandum est de his quae pertinent ad virtutem cognoscitivam
Angeli; secundo, de his quae pertinent ad medium cognoscendi ipsius;
tertio, de his quae ab eo cognoscuntur; quarto, de modo cognitionis
ipsorum.

Circa primum quaeruntur quinque. 

\begin{enumerate}
 \item utrum intelligere Angeli sit sua substantia.
 \item utrum eius esse sit suum intelligere. 
 \item utrum eius substantia sit sua virtus intellectiva.
 \item utrum in Angelis sit intellectus agens et possibilis. 
 \item utrum in eis sit aliqua alia potentia cognoscitiva quam intellectus.
\end{enumerate}

&

天使の実体に属する事柄が考察されたので、その認識[について]の考察へと進
まれるべきである。この考察は、四つの部分をもつだろう。すなわち、第一に、
天使の認識能力に属する事柄について考察されるべきであり、第二に、それの認
識の媒介に属する事柄について、第三に、天使によって認識される事柄について、
第四に、天使たちの認識方法について、考察されるべきである。
第一のことを巡って、五つのことが問われる。

\begin{enumerate}
 \item 天使の知性認識の働きはその実体か。
 \item 天使の存在は、その知性認識の働きか。
 \item 天使の実体は、その知性認識能力か。
 \item 天使の中に、能動知性と可能知性があるか。
 \item 天使の中に、知性以外の他の認識能力があるか。
\end{enumerate}

\end{longtable}
\newpage


\rhead{a.~1}
\begin{center}
 {\Large {\bf ARTICULUS PRIMUS}}\\
 {\large UTRUM INTELLIGERE ANGELI SIT EIUS SUBSTANTIA}\\
 {\footnotesize Opusc.~XV, {\itshape de Angelis}, cap.~{\scshape xiii}}\\
 {\Large 第一項\\天使の知性認識の働きはその実体か}
\end{center}

\begin{longtable}{p{21em}p{21em}}

{\huge A}{\scshape d primum sic proceditur}. Videtur quod intelligere
Angeli sit eius substantia. Angelus enim est sublimior et simplicior
quam intellectus agens animae. Sed substantia intellectus agentis est
sua actio; ut patet in III {\itshape de Anima} per Aristotelem, et eius
Commentatorem. Ergo, multo fortius, substantia Angeli est sua actio,
quae est intelligere.


&

第一項の問題へ、議論は以下のように進められる。天使の知性認識の働きは、天
使の実体であると思われる。理由は以下の通り。
天使は魂の能動知性よりも崇高で単純である。ところが、『デ・アニマ』3巻
 の中で、アリストテレスと彼の注釈者によって明らかなとおり、能動知性の実
 体は、その作用である。ゆえに、さらにもまして、天使の実体は、その作用、
 すなわち、知性認識の働きである。
 

\\

{\scshape  2 Praeterea}, philosophus dicit, in XII Metaphys., quod {\itshape actio
 intellectus est vita}. Sed cum {\itshape vivere sit esse viventibus}, ut dicitur in
 II {\itshape de Anima}, videtur quod jvita sit essentia. Ergo actio intellectus est
 essentia intelligentis Angeli.

&

さらに、哲学者は『形而上学』12巻で「知性の作用は生命である」と述べている。
 ところが、『デ・アニマ』2巻で言われるように、「生きることは、生きるものに
 とって、存在することである」のだから、生命は、本質であると思われる。
ゆえに、知性の働きは、知性認識する天使の本質である。



\\


{\scshape  3 Praeterea}, si extrema sunt unum, medium non differt ab
 eis, quia extremum magis distat ab extremo, quam medium. Sed in Angelo
 idem est intellectus et intellectum, ad minus inquantum intelligit
 essentiam suam. Ergo intelligere, quod cadit medium inter intellectum
 et rem intellectam, est idem cum substantia Angeli intelligentis.


&
さらに、もし、両端が一つのものであるならば、中間は、それら両端と異ならな
 い。なぜなら、一方の端は、中間よりも、他方の端から離れているからである。
 ところが、少なくとも、自らの本質を認識する限り、天使の中で、知性と知性
 認識されるものとは同一である。ゆえに、知性と知性認識された事物の中間で
 ある知性認識の働きは、知性認識する天使の実体と同一である。


\\


{\scshape  Sed contra,} plus differt actio rei a substantia eius, quam
 ipsum esse eius. Sed nullius creati suum esse est sua substantia, hoc
 enim solius Dei proprium est, ut ex superioribus patet. Ergo neque
 Angeli, neque alterius creaturae, sua actio est eius substantia.

&

しかし反対に、事物の作用は、それの存在それ自体よりも、その実体から離れて
 いる。ところが、自らの存在がその実体であることは、どんな被造物にも当て
 はまらず、前述のことから明らかなとおり、ただ神にのみに固有のことである。
ゆえに、天使も、そして他の被造物も、その作用は自らの実体でない。
\\


{\scshape Respondeo dicendum} quod impossibile est quod actio Angeli,
 vel cuiuscumque alterius creaturae, sit eius substantia. Actio enim est
 proprie actualitas virtutis; sicut esse est actualitas substantiae vel
 essentiae. Impossibile est autem quod aliquid quod non est purus actus,
 sed aliquid habet de potentia admixtum, sit sua actualitas, quia
 actualitas potentialitati repugnat. Solus autem Deus est actus
 purus. Unde in solo Deo sua substantia est suum esse et suum
 agere. 

&

解答する。以下のように言われるべきである。天使の、あるいはなんであれ他の
 被造物の作用は、その実体ではない。作用とは、固有の意味で、能力の現実性
 である。たとえば、存在は、実体ないし本質の現実性である。ところが、純粋
 現実態でなく、可能態の混合をもつものが、自らの現実性であることは不可能
 である。なぜなら、現実性は可能態性に背馳するからである。しかし、ただ神
 だけが、純粋現実態である。したがって、ただ神の中でのみ、自らの実体が、自
 らの存在であり、自らの作用である。



\\

Praeterea, si intelligere Angeli esset sua substantia, oporteret
 quod intelligere Angeli esset subsistens. Intelligere autem subsistens
 non potest esse nisi unum; sicut nec aliquod abstractum
 subsistens. Unde unius Angeli substantia non distingueretur neque a
 substantia Dei, quae est ipsum intelligere subsistens; neque a
 substantia alterius Angeli. 

&

さらに、もし天使の知性認識の働きが自らの実体であったとしたら、天使の知性
 認識の働きは自存したであろう。ところが、自存する知性認識の働きは一つで
 しかありえない。ちょうど、何らかの抽象されたものが自存するとしたら、そ
 れは一つでしかありえないように。したがって、一人の天使の実体が、自存す
 る知性認識の働きそのものである神の実体からも、他の天使の実体からも、区
 別されなかったであろう。


\\

Si etiam Angelus ipse esset suum
 intelligere, non possent esse gradus in intelligendo perfectius et
 minus perfecte, cum hoc contingat propter diversam participationem
 ipsius intelligere.

&


またさらに、もし、天使自身が自らの知性認識の働きであったならば、より完
 全に認識する、より不完全に認識する、ということにおける段階がありえなかっ
 ただろう。なぜなら、このことは、知性認識の働きそれ自体を、さまざまに分
 有するために生じるからである。

\\


{\scshape  Ad primum ergo dicendum} quod, cum dicitur quod intellectus
 agens est sua actio, est praedicatio non per essentiam, sed per
 concomitantiam, quia cum sit in actu eius substantia, statim quantum
 est in se, concomitatur ipsam actio. Quod non est de intellectu
 possibili, qui non habet actiones nisi postquam fuerit factus in actu.

&

第一異論に対しては、それゆえ、以下のように言われるべきである。
能動知性が自らの働きであると言われるとき、それは本質による述語付けではな
 く、随伴による述語付けである。なぜなら、能動知性の実体は現実態にあるの
 で、直ちに、自らのうちにあるかぎりの働きが実体自体に随伴するからである。
このことは、現実態にされたあとでなければ作用をもたない可能知性には起こら
 ない。

\\


{\scshape  Ad secundum dicendum} quod vita non hoc modo se habet ad
 vivere, sicut essentia ad esse; sed sicut cursus ad currere, quorum
 unum significat actum in abstracto, aliud in concreto. Unde non
 sequitur si vivere sit esse, quod vita sit essentia. Quamvis etiam
 quandoque vita pro essentia ponatur; secundum quod Augustinus dicit, in
 libro {\itshape de Trin}., quod {\itshape memoria et intelligentia et voluntas sunt una
 essentia, una vita}. Sed sic non accipitur a philosopho cum dicit quod
 {\itshape actio intellectus est vita}.

&

第二異論に対しては、以下のように言われるべきである。
vitaはvivereにたいして、essentiaがesseにたいするのと同じように関係するのでは
 なく、むしろ、cursusがcurrereにたいするように関係する。つまり、その一方
 が、作用を抽象的に表示し、他方が具体的に表示する。したがって、vivereが
 esseであるとしても、vitaがessentiaであることは帰結しない。
さらに、アウグスティヌスが『三位一体論』という書物で「記憶と知性と意志は、
 ひとつの本質でありひとつの生命である」と述べるのによれば、生命が本質と
 して考えられることもあるけれども、しかし、哲学者が「知性の作用は生命で
 ある」と言うときに、彼によって理解されているのはそのような意味ではない。



\\


{\scshape Ad tertium dicendum} quod actio quae transit in aliquid
 extrinsecum, est realiter media inter agens et subiectum recipiens
 actionem. Sed actio quae manet in agente, non est realiter medium inter
 agens et obiectum, sed secundum modum significandi tantum, realiter
 vero consequitur unionem obiecti cum agente. Ex hoc enim quod
 intellectum fit unum cum intelligente, consequitur intelligere, quasi
 quidam effectus differens ab utroque.


&

第三異論に対しては、以下のように言われるべきである。
外の何かへ越え出ていく作用は、実在的に、作用者と、その作用を受ける基体の
 間にある媒介である。しかし、作用者のうちに留まる作用は、実在的に、作用
 者と対象の間にある媒介ではなく、たんに、表示の仕方においてであり、実在
 的には、対象の作用者への合一に伴う。つまり、知性認識されるものが知性認
 識するものとひとつになることに基づいて、いわばその両者から流れ出るもの
 として、知性認識の作用が伴う。


\end{longtable}

\newpage




\rhead{a.~2}
\begin{center}
 {\Large {\bf ARTICULUS SECUNDUS}}\\
 {\large UTRUM INTELLIGERE ANGELI SIT EIUS ESSE}\\
 {\Large 第二項\\天使の知性認識はその存在か}
\end{center}

\begin{longtable}{p{21em}p{21em}}

{\huge A}{\scshape d secundum sic proceditur}. Videtur quod intelligere
Angeli sit eius esse. {\itshape Vivere} enim {\itshape viventibus est
esse}, ut dicitur in II {\itshape de Anima}. Sed intelligere est quoddam
vivere, ut in eodem dicitur. Ergo intelligere Angeli est eius esse.


&

第二項の問題へ、議論は以下のように進められる。
天使の知性認識はその存在だと思われる。理由は以下の通り。
『デ・アニマ』二巻で言われるように、「生きることは、生きるものにとって、
 存在することである」。ところが、同じ箇所で言われているように、知性認識
 することは、一種の生きることである。ゆえに、天使の知性認識はその存在で
 ある。

\\


{\scshape 2 Praeterea}, sicut se habet causa ad causam,
ita effectus ad effectum. Sed forma per quam Angelus est, est eadem cum
forma per quam intelligit ad minus seipsum. Ergo eius intelligere est
idem cum suo esse.


&

さらに、結果相互の関係は、それぞれの原因相互の関係に等しい。
ところが、天使がそれによって存在するところの形相は、天使がそれによって知
 性認識するところの形相と、少なくとも自己認識にかんする限り、同一である。
 ゆえに、天使の知性認識は、その存在と同一である。

\\


{\scshape Sed contra}, intelligere Angeli est motus eius; ut patet per
Dionysium, {\scshape iv} cap.~{\itshape de Div.~Nom}. Sed esse non est
motus. Ergo esse Angeli non est intelligere eius.


&

しかし反対に、『神名論』四章のディオニュシウスによって明らかなとおり、天
 使の知性認識は、その運動である。ところが、存在は運動でない。ゆえに、天
 使の存在はその知性認識でない。

\\


{\scshape Respondeo dicendum} quod actio Angeli non est
eius esse, neque actio alicuius creaturae. Duplex enim est actionis
genus, ut dicitur IX {\itshape Metaphys}. Una scilicet actio est quae transit in
aliquid exterius, inferens ei passionem, sicut \={u}r\u{e}re et secare. Alia
vero actio est quae non transit in rem exteriorem, sed manet in ipso
agente, sicut sentire, intelligere et velle, per huiusmodi enim actionem
non immutatur aliquid extrinsecum, sed totum in ipso agente agitur. 


&

解答する。以下のように言われるべきである。
天使の作用はその存在でない。またどんな被造物の作用も、その存在ではない。
『形而上学』九巻で言われるとおり、作用には二通りの類がある。たとえば、
 「燃やす」や「切る」のように、ある作用は、何か外のものへ、それに受動を
 もたらしつつ、越え出ていく。他方、別のある作用は、たとえば「感覚する」
 や「知性認識する」や「意志する」のように、外の事物へ出ていかず、
 作用者自身の中に留まる。
そして、このような作用によって、外的な何かが変化を被ることはなく、全体が、
 作用者自身の中で行われる。


\\

De
prima ergo actione manifestum est quod non potest esse ipsum esse
agentis, nam esse agentis significatur intra ipsum, actio autem talis
est effluxus in actum ab agente. 
Secunda autem actio de sui ratione
habet infinitatem, vel simpliciter, vel secundum quid. Simpliciter
quidem, sicut intelligere, cuius obiectum est verum, et velle, cuius
obiectum est bonum, quorum utrumque convertitur cum ente; et ita
intelligere et velle, quantum est de se, habent se ad omnia; et utrumque
recipit speciem ab obiecto. 

&

それゆえ、前者の作用について、それが作用者の存在そのものでありえないこと
 が明らかである。なぜなら、作用者の存在は、作用者自身の内部にあるものと
 して表示されるが、そのような作用は、作用者から作用を受けるものへ流れ出
 るものだからである。
他方、後者の作用は、その性格のなかに、端的に、あるいはなんらかの点で、無
 限性がある。端的にと言うのは、たとえば、知性はその対象が真であり、意志
 はその対象が善であるが、この両者とも、有と置換される。したがって、知性
 認識の作用と意志の作用は、それ自身にかんする限り、すべてのものに関係し、
 両者とも、対象から形象を受け取る。


\\


Secundum quid autem infinitum est sentire,
quod se habet ad omnia sensibilia, sicut visus ad omnia visibilia. Esse
autem cuiuslibet creaturae est determinatum ad unum secundum genus et
speciem, esse autem solius Dei est simpliciter infinitum, in se omnia
comprehendens, ut dicit Dionysius, {\scshape v} cap.~{\itshape de Div.~Nom.} Unde solum esse
divinum est suum intelligere et suum velle.


&

他方、ある点で無限なのは感覚の作用であり、それは、すべての可感的なものに
 関係する。たとえば、視覚は、すべての可視的なものに関係する。ところが、
 どんな被造物の存在も、類と種によって一つのものに限定される。これに対し
 て、『神名論』五章のディオニュシウスが言うように、ただ神の存在だけが、
 自らの中に万物を包含しつつ、端的に無限である。
したがって、ただ神の存在だけが、その知性認識であり意志である。


\\



{\scshape Ad primum ergo dicendum} quod vivere quandoque
sumitur pro ipso esse viventis, quandoque vero pro operatione vitae,
idest per quam demonstratur aliquid esse vivens. Et hoc modo philosophus
dicit quod intelligere est vivere quoddam, ibi enim distinguit diversos
gradus viventium secundum diversa opera vitae.


&

第一異論に対しては、それゆえ、以下のように言われるべきである。
生きることは、生きるものの存在自体とされるときもあれば、生命の働き、すな
 わち、それによって何かが生き物であることが明示されるところのもの、とされ
 ることもある。知性認識することは、一種の生きることだと哲学者が言うのは、
 この後者の意味でである。なぜなら、彼はその箇所で、生命のさまざまな働き
 にしたがって、生物のさまざまな段階を区別しているからである。

\\


{\scshape Ad secundum dicendum} quod ipsa essentia
Angeli est ratio totius sui esse, non autem est ratio totius sui
intelligere, quia non omnia intelligere potest per suam essentiam. Et
ideo secundum propriam rationem, inquantum est talis essentia,
comparatur ad ipsum esse Angeli. Sed ad eius intelligere comparatur
secundum rationem universalioris obiecti, scilicet veri vel entis. Et
sic patet quod, licet sit eadem forma, non tamen secundum eandem
rationem est principium essendi et intelligendi. Et propter hoc non
sequitur quod in Angelo sit idem esse et intelligere.


&

第二異論に対しては、以下のように言われるべきである。
天使の本質自体は、自らの存在全体の根拠だが、自らの知性認識全体の根拠では
 ない。なぜなら、天使は自らの本質によって万物を知性認識することができな
 いからである。ゆえに、天使は、天使の存在自体にたいして、固有の性格によっ
 て、つまり、そのような本質であるかぎりで関係するが、その知性認識に対し
 ては、対象の、つまり、真あるいは有の、より普遍的な性格によって関係する。
 以上のことから以下のことが明らかである。[天使において、存在の根源と知
 性認識の根源とは]たしかに同一の形相であるが、しかし、同一の性
 格において、存在の根源と知性認識の根源であるわけではない。
そしてこのために、[存在と知性認識の根源が同一であることから]天使におい
 て、存在と知性認識が同一あることは帰結しない。




\end{longtable}
\newpage


\rhead{a.~3}
\begin{center}
 {\Large {\bf ARTICULUS TERTIUS}}\\
 {\large UTRUM POTENTIA INTELLECTIVA ANGELI SIT EIUS ESSENTIA}\\
 {\footnotesize Infra qu.~77, a.~1; q.~79, a.~1.}\\
 {\Large 第三項\\天使の知性的能力はその本質か}
\end{center}

\begin{longtable}{p{21em}p{21em}}

{\huge A}{\scshape d tertium sic proceditur}. Videtur quod
virtus vel potentia intellectiva in Angelo non sit aliud quam sua
essentia. {\itshape Mens} enim et {\itshape intellectus} nominant potentiam intellectivam. Sed
Dionysius in pluribus locis suorum librorum, nominat ipsos Angelos
{\itshape intellectus} et {\itshape mentes}. Ergo Angelus est sua potentia intellectiva.

&
第三項の問題へ、議論は以下のように進められる。
天使の中にある知性的な力ないし能力は、その本質に他ならないと思われる。理
 由は以下の通り。
「精神」や「知性」は、知性的能力の名である。ところが、ディオニュシウスは、
 彼の書物の複数の箇所で、天使自身を「知性」や「精神」と名付けている。ゆ
 えに、天使は自らの知性的能力である。


\\


{\scshape 2 Praeterea}, si potentia intellectiva in
Angelo est aliquid praeter eius essentiam, oportet quod sit accidens,
hoc enim dicimus esse accidens alicuius, quod est praeter eius
essentiam. Sed {\itshape forma simplex subiectum esse non potest}, ut Boetius
dicit, in libro {\itshape de Trin}. Ergo Angelus non esset forma simplex, quod est
contra praemissa.

&
さらに、もし天使の中の知性的能力がその本質以外の何かであるならば、それは、
 付帯性でなければならない。なぜなら、私たちは、あるものの本質の外に在る
 ものは付帯性であると言うからである。ところが、ボエティウスが『三位一体
 論』で述べるように、「単純形相は基体でありえない」。ゆえに、[知性的能
 力が本質でないならば]天使は単純形相でなかったことになる。これは、既述
 のことに反する。

\\


{\scshape 3 Praeterea}, Augustinus dicit, XII {\itshape Confess}.,
quod Deus fecit angelicam naturam {\itshape prope se}, materiam autem primam {\itshape prope
nihil}, ex quo videtur quod Angelus sit simplicior quam materia prima,
utpote Deo propinquior. Sed materia prima est sua potentia. Ergo multo
magis Angelus est sua potentia intellectiva.

&
さらに、アウグスティヌスは『告白』12巻で、神は天使的本性を「自分の近く
 に」、第一質料を「無の近くに」作った、と述べている。このことから、天使
 は、神により近いものとして、第一質料よりも単純であると思われる。ところ
 が、第一質料は、自らの能力(potentia)である。ゆえに、さらにいっそう、天使は自ら
 の知性的能力である。

\\


{\scshape  Sed contra est quod} Dionysius dicit, {\scshape xi}
cap.~{\itshape Angel.~Hier}., quod Angeli {\itshape dividuntur in substantiam, virtutem et
operationem}. Ergo aliud est in eis substantia, et aliud virtus, et aliud
operatio.

&
しかし反対に、ディオニュシウスは、『天使階級論』9章で、天使たちは「実体、
 力、働きへ」分けられる、と述べている。ゆえに、それらにおいて、実体、力、
 働きは別のものである。

\\


{\scshape Respondeo dicendum} quod nec in Angelo nec in
aliqua creatura, virtus vel potentia operativa est idem quod sua
essentia. Quod sic patet. Cum enim potentia dicatur ad actum, oportet
quod secundum diversitatem actuum sit diversitas potentiarum, propter
quod dicitur quod proprius actus respondet propriae potentiae. In omni
autem creato essentia differt a suo esse, et comparatur ad ipsum sicut
potentia ad actum, ut ex supra dictis patet. Actus autem ad quem
comparatur potentia operativa, est operatio. In Angelo autem non est
idem intelligere et esse, nec aliqua alia operatio aut in ipso aut in
quocumque alio creato, est idem quod eius esse. Unde essentia Angeli non
est eius potentia intellectiva, nec alicuius creati essentia est eius
operativa potentia.

&

解答する。以下のように言われるべきである。天使の中でも、他のどんな被造物
の中でも、力ないし働きの能力が、その本質と同一ではない。このことは以下の
ようにして明らかである。すなわち、能力(potentia)は作用(actus)に対応して言
われるので、作用の異なりに対応して能力も異なるのでなければならない。この
ために、固有の作用は固有の能力に対応すると言われる。さて、前に述べられた
ことから明らかなとおり、すべての被造物の中で、本質は、その存在と異なって
いて、存在に対して、可能態(potentia)が現実態(actus)に対するように関係する。
また、働きの能力が関係する作用とは、働き(operatio)である。天使の中では、知性認識の
働き(intelligere)と存在の働き(esse)は同一でなく、天使においても他のどんな被造物においても、
他の何らかの働きがその存在と同一であることはない。したがって、天使の本質
は、その知性的能力ではなく、また、どんな被造物の本質も、その働きの能力で
ない。

\\


{\scshape Ad primum ergo dicendum} quod Angelus dicitur
intellectus et mens, quia tota eius cognitio est
intellectualis. Cognitio autem animae partim est intellectualis, et
partim sensitiva.

&

第一異論に対しては、それゆえ、以下のように言われるべきである。
天使が「知性」や「精神」と言われるのは、その認識全体が知性的だからである。
 これに対して魂の認識は、ある部分は知性的で、別の部分は感覚的である。

\\


{\scshape Ad secundum dicendum} quod forma simplex quae
est actus purus, nullius accidentis potest esse subiectum, quia
subiectum comparatur ad accidens ut potentia ad actum. Et huiusmodi est
solus Deus. Et de tali forma loquitur ibi Boetius. Forma autem simplex
quae non est suum esse, sed comparatur ad ipsum ut potentia ad actum,
potest esse subiectum accidentis, et praecipue eius quod consequitur
speciem, huiusmodi enim accidens pertinet ad formam (accidens vero quod
est individui, non consequens totam speciem, consequitur materiam, quae
est individuationis principium). Et talis forma simplex est Angelus.

&

第二異論に対しては、以下のように言われるべきである。
純粋現実態である単純形相は、どんな付帯性の基体でもありえない。なぜなら、
 基体は付帯性に対して、可能態が現実態に対するように関係するからである。
このようなものは神だけである。
そして、ボエティウスはそこで、このような形相について語っている。
---これに対して、自らの存在でなく、自らの存在に対して可能態が現実態に対
 するように関係する単純形相は、付帯性の基体でありえ、とくに、その種(species)に伴
 うものの基体でありうる。なぜなら、そのような付帯性は、形相に属するから
 である。(---他方、個体に属する付帯性は、種全体に伴わず、質料に伴う。質
 料は個体化の根源だからである。)そして、このような単純形相が天使である。

\\


{\scshape Ad tertium dicendum} quod potentia materiae
est ad ipsum esse substantiale, et non potentia operativa, sed ad esse
accidentale. Unde non est simile.

&

第三異論に対しては、以下のように言われるべきである。
質料の可能態は、実体的存在そのものへ関係する。また、それは働く能力
 (potentia)ではなく、[働く能力は]むしろ付帯的存在へ関係する。
したがって、それは似ていない。


\end{longtable}
\newpage




\rhead{a.~4}
\begin{center}
 {\Large {\bf ARTICULUS QUARTUS}}\\
 {\large UTRUM IN ANGELO SIT INTELLECTUS AGENS ET POSSIBILIS}\\
 {\footnotesize II {\itshape Cont.~Gent.}, cap.~96.}\\
 {\Large 第四項\\天使の中に能動知性と可能知性があるか}
\end{center}

\begin{longtable}{p{21em}p{21em}}

{\huge A}{\scshape d quartum sic proceditur}. Videtur quod in Angelo sit
intellectus agens et possibilis. Dicit enim philosophus, in III
{\itshape de Anima}, quod {\itshape sicut in omni natura est aliquid quo
est omnia fieri, et aliquid quo est omnia facere, ita etiam in
anima}\footnote{Postquam philosophus determinavit de intellectu
 possibili, nunc determinat de intellectu agente. Et circa hoc duo
 facit. Primo ostendit esse intellectum agentem, praeter possibilem, et
 ratione, et exemplo. Secundo ostendit huius intellectus naturam, ibi,
 et hic intellectus. Ponit ergo circa primum talem rationem. In omni
 natura quae est quandoque in potentia et quandoque in actu, oportet
 ponere aliquid, quod est sicut materia in unoquoque genere, quod
 scilicet est in potentia ad omnia quae sunt illius generis. Et aliud,
 quod est sicut causa agens, et factivum; quod ita se habet in faciendo
 omnia, sicut ars ad materiam. Sed anima secundum partem intellectivam
 quandoque est in potentia, et quandoque in actu. Necesse est igitur in
 anima intellectiva esse has differentias: ut scilicet unus sit
 intellectus, in quo possint omnia intelligibilia fieri, et hic est
 intellectus possibilis, de quo supra dictum est: et alius intellectus
 sit ad hoc quod possit omnia intelligibilia facere in actu; qui vocatur
 intellectus agens, et est sicut habitus quidam. ({\itshape In III De
 Anima}, lect.~10, n.~728.)}. Sed Angelus est natura quaedam. Ergo in eo est intellectus agens
et possibilis.

&

第四項の問題へ、議論は以下のように進められる。
天使の中に、能動知性と可能知性があると思われる。理由は以下の通り。
哲学者は『デ・アニマ』3巻で「すべての自然本性の中で、あるものは、それによっ
 てすべてのものになるところのものであり、またあるものは、それによって
 すべてのものを作るところのものであるから、魂においてもまたそうである」
 と述べている。ところが、天使は、ある自然本性である。ゆえに、天使の中に、
 能動知性と可能知性がある。

\\


2 {\scshape  Praeterea}, recipere est proprium
intellectus possibilis, illuminare autem est proprium intellectus
agentis, ut patet in III {\itshape de Anima}. Sed Angelus recipit illuminationem a
superiori, et illuminat inferiorem. Ergo in eo est intellectus agens et
possibilis.

&

さらに、『デ・アニマ』3巻で明らかなとおり、受け取ることは可能知性に固有
 であり、照らすことは能動知性に固有である。ところが、天使は、上位の天使
 から照明を受け取り、下位の天使を照らす。ゆえに、その中に、能動知性と可能
 知性がある。

\\


{\scshape  Sed contra est} quod in nobis intellectus
agens et possibilis est per comparationem ad phantasmata; quae quidem
comparantur ad intellectum possibilem ut colores ad visum, ad
intellectum autem agentem ut colores ad lumen, ut patet ex III {\itshape de
anima}. Sed hoc non est in Angelo. Ergo in Angelo non est intellectus
agens et possibilis.

&

しかし反対に、私たちの中に能動知性と可能知性があるのは、表象像との関係に
 よってである。じっさい、『デ・アニマ』3巻で明らかなとおり、表象像は、可
 能知性に対しては、ちょうど色が視覚 に対するように関係し、能動知性に対し
 ては、色が光に対するように関係する。ところが、天使の中にこういうことは
 ない。ゆえに、天使の中に、能動知性と可 能知性はない。


\\


{\scshape Respondeo dicendum} quod necessitas ponendi intellectum
possibilem in nobis, fuit propter hoc, quod nos invenimur quandoque
intelligentes in potentia et non in actu, unde oportet esse quandam
virtutem, quae sit in potentia ad intelligibilia ante ipsum intelligere,
sed reducitur in actum eorum cum fit sciens, et ulterius cum fit
considerans. Et haec virtus vocatur intellectus possibilis. 

&

解答する。以下のように言われるべきである。
私たちの中に可能知性を措定する必要性は次のことのためにであった。すなわち、
 私たちは、あるときには可能態において知性認識するものであり、現実態にお
 いてではない、ということが見いだされるので、したがって、知性認識の働き
 それ自体の以前に、可知的なものに対して可能態にあるが、知る者になると
 き、そしてさらに考察する者になるときには、それら〔可知的なものども〕の
 現実態に引き出されるような、何らかの能力が存在しなければならないからで
 ある。そして、この能力が、可能知性と呼ばれる。



\\



Necessitas
autem ponendi intellectum agentem fuit, quia naturae rerum materialium,
quas nos intelligimus, non subsistunt extra animam immateriales et
intelligibiles in actu, sed sunt solum intelligibiles in potentia, extra
animam existentes, et ideo oportuit esse aliquam virtutem, quae faceret
illas naturas intelligibiles actu. Et haec virtus dicitur intellectus
agens in nobis. 


&

他方、能動知性を措定すべき必要性は次のことであった。すなわち、私たちが認
 識する質料的事物の本性は、非質料的で可知的な魂の外に自存してはおらず、
 魂の外に存在しているときにはただ可能態において可知的であるに過ぎない。
 それゆえ、何らかの能力が存在して、それがその本性を現実態において可知的
 なものにするのでなければならなかった。この能力が、私たちの中にある能動
 知性と言われる。

\\




Utraque autem necessitas deest in Angelis. Quia neque
sunt quandoque intelligentes in potentia tantum, respectu eorum quae
naturaliter intelligunt, neque intelligibilia eorum sunt intelligibilia
in potentia, sed in actu; intelligunt enim primo et principaliter res
immateriales, ut infra patebit. Et ideo non potest in eis esse
intellectus agens et possibilis, nisi aequivoce.

&

ところが、このどちらの必要性も、天使にはない。
なぜなら、天使たちが本性的に認識する事柄に関して、彼らが時として単に可能態におい
 てのみ認識するということはないし、彼らがもつ可知的なものどもは、可能態
 においてではなく、現実態において可知的である。なぜなら、あとで明らかに
 なるとおり、彼らは、第一に、そして主要に、非質料的事物を認識するからで
 ある。ゆえに、彼らの中に能動知性と受動知性が存在することは不可能である。
 ただし、異義的にでない限り。


\\


{\scshape Ad primum ergo dicendum} quod philosophus
intelligit ista duo esse in omni natura in qua contingit esse generari
vel fieri, ut ipsa verba demonstrant. In Angelo autem non generatur
scientia, sed naturaliter adest. Unde non oportet ponere in eis agens et
possibile.

&


第一異論に対しては、それゆえ、以下のように言われるべきである。
哲学者は、言葉自体が明示しているとおり、生成や消滅が生じるようなすべて
 の自然本性の中に、その二つがあると考えている。ところが、天使の中に、知
 が生じることはなく、自然本性的に、そこにある。したがって、彼らの中に、
 能動と受動(の知性)を措定する必要はない。

\\


{\scshape Ad secundum dicendum} quod intellectus
agentis est illuminare non quidem alium intelligentem, sed
intelligibilia in potentia, inquantum per abstractionem facit ea
intelligibilia actu. Ad intellectum autem possibilem pertinet esse in
potentia respectu naturalium cognoscibilium, et quandoque fieri
actu. Unde quod Angelus illuminat Angelum, non pertinet ad rationem
intellectus agentis. Neque ad rationem intellectus possibilis pertinet,
quod illuminatur de supernaturalibus mysteriis, ad quae cognoscenda
quandoque est in potentia. Si quis autem velit haec vocare intellectum
agentem et possibilem, aequivoce dicet, nec de nominibus est curandum.


&

第二異論に対しては、以下のように言われるべきである。
能動知性に属するのは、知性認識する他の者を照らすことではなく、抽象によっ
 て現実に可知的なものにするという意味で、可能態において可知的なものを照ら
 すことである。また、可能知性には、自然本性的に認識されうるものどもに対
 して可能態にあること、そして、時として現実態になることが属する。したがっ
 て、天使が天使を照らすことは、能動知性の性格に属さない。また、それを認
 識することに対して時として可能態にあるような、超自然的な神秘について照
 らされることが、可能知性の性格に属するのでもない。しかし、もし誰かが、
 これらを能動知性や可能知性と呼びたいのであれば、それは異義的に語ってい
 る。しかし、そういう名称について心を煩わせるべきではない。

\end{longtable}
\newpage



\rhead{a.~5}
\begin{center}
 {\Large {\bf ARTICULUS QUINTUS}}\\
 {\large UTRUM IN ANGELIS EST SOLA INTELLECTIVA COGNITIO}\\
 {\footnotesize III {\itshape Cont.~Gent.}, cap.~108; {\itshape De
 Malo}, q.~16, a.~1, ad 14.}\\
 {\Large 第五項\\天使の中には、ただ知性的認識だけがあるか}
\end{center}

\begin{longtable}{p{21em}p{21em}}


{\huge A}{\scshape d quintum sic proceditur}. Videtur quod in
Angelis non sit sola intellectiva cognitio. Dicit enim Augustinus, VIII
{\itshape de Civ.~Dei}, quod in Angelis est {\itshape vita quae intelligit et sentit}. Ergo in
eis est potentia sensitiva.

&

第五項の問題へ、議論は以下のように進められる。天使たちの中に、知性的認識
 だけがあるのではないと思われる。理由は以下の通り。
アウグスティヌスは『神の国』第8巻で、天使たちの中に「知性認識し、感覚す
 る生命」があると述べている。ゆえに、彼らの中に、感覚的能力がある。


\\


{\scshape  2 Praeterea}, Isidorus dicit quod Angeli multa
noverunt per experientiam. Experientia autem fit ex multis memoriis, ut
dicitur in I {\itshape Metaphys}. Ergo in eis est etiam memorativa potentia.

&
さらに、イシドロスは、天使たちが経験を通して多くのことを知っていると述べ
 ている。ところが、『形而上学』第1巻で言われるように、経験は、多くの記憶
 から生じる。ゆえに、彼らの中には、記憶能力もある。

\\


{\scshape  3 Praeterea}, Dionysius dicit, {\scshape iv} cap.~{\itshape de
Div.~Nom}., quod in Daemonibus est {\itshape phantasia proterva}. Phantasia autem ad
vim imaginativam pertinet. Ergo in Daemonibus est vis imaginativa. Et
eadem ratione in Angelis, quia sunt eiusdem naturae.

&

さらに、ディオニュシウスは、『神名論』第4章で、悪霊たちの中には「傲慢な
 表象力」があると述べている。ところが、表象力は想像力に属する。ゆえに、
 悪霊たちの中には想像力がある。ゆえに、天使たちは同じ本性をもっているの
 だから、同じ理由で、天使たちの中には想像力がある。

\\


{\scshape Sed contra est} quod Gregorius dicit, in
homilia de ascensione, quod {\itshape homo sentit cum pecoribus, et intelligit cum
Angelis}.



&
しかし反対に、グレゴリウスは『昇天講話』の中で「人間は獣たちとともに感覚
 し、天使たちとともに知性認識する」と述べている。

\\



{\scshape Respondeo dicendum} quod in anima nostra sunt
quaedam vires, quarum operationes per organa corporea exercentur, et
huiusmodi vires sunt actus quarundam partium corporis, sicut est visus
in oculo, et auditus in aure. Quaedam vero vires animae nostrae sunt,
quarum operationes per organa corporea non exercentur, ut intellectus et
voluntas, et huiusmodi non sunt actus aliquarum partium corporis. Angeli
autem non habent corpora sibi naturaliter unita, ut ex supra dictis
patet. Unde de viribus animae non possunt eis competere nisi intellectus
et voluntas. 



&

解答する。以下のように言われるべきである。
私たちの魂の中には、その働きが、身体器官によって遂行されるような力がある
 が、そのような力は、身体のある部分の現実態である。たとえば、目における
 視覚や、耳における聴覚がそれである。他方、私たちの魂のある力は、その働
 きが身体器官によっては遂行されない。たとえば、知性や意志がそれである。
そして、そのようなものは、身体のある部分の現実態ではない。
さて、前に述べられたことから明らかなとおり、天使たちは、本性的に自らに合
 一された身体をもたない。したがって、魂の諸能力のなかでは、知性と意志以
 外に彼らに適合しえるものはない。


\\


Et hoc etiam Commentator dicit, XII {\itshape Metaphys}., quod
substantiae separatae dividuntur in intellectum et voluntatem. --- Et hoc
convenit ordini universi, ut suprema creatura intellectualis sit
totaliter intellectiva; et non secundum partem, ut anima nostra. Et
propter hoc etiam Angeli vocantur intellectus et mentes, ut supra dictum
est.

&

そして、このことを、注釈家も、『形而上学』12巻で、離在実体は知性と意志に分
 かたれると述べている。--そして、最高の知性的被造物が全体的に知性認識的
 であり、私たちの魂のように、部分的にでないということは、宇宙の秩序にも
 適合する。--また、このために、前に述べられたとおり\footnote{第3項第1異
 論解答}、天使たちは「知性」や「精神」と呼ばれる。

\\

{\scshape Ad ea vero quae in contrarium obiiciuntur},
potest dupliciter responderi. Uno modo, quod auctoritates illae
loquuntur secundum opinionem illorum qui posuerunt Angelos et Daemones
habere corpora naturaliter sibi unita. Qua opinione frequenter
Augustinus in libris suis utitur, licet eam asserere non intendat, unde
dicit, XXI {\itshape de Civ.~Dei}, quod {\itshape super hac inquisitione
 non est multum laborandum}. 

&

反対に論じられた諸々の議論に対しては、二通りに解答されうる。
一つには、それらの権威は、天使や悪魔たちが、本性的に自らに合一された身体
 をもっていると主張した人々の意見に従って語っている、と。
アウグスティヌスは、そのような意見を、彼の本の中で頻繁に用いている。しか
 し、彼はその意見を主張しようとはしておらず、それゆえ、『神の国』21巻で
 は「この探究にはあまり多くの労力を使うべきでない」と言っている。


\\



Alio modo potest dici, quod auctoritates illae, et
consimiles, sunt intelligendae per quandam similitudinem. Quia cum
sensus certam apprehensionem habeat de proprio sensibili, est in usu
loquentium ut etiam secundum certam apprehensionem intellectus aliquid
{\itshape sentire} dicamur. Unde etiam {\itshape sententia} nominatur. 

&

別の仕方では、かの権威たちや、似た人たちは、なんらかの類似によって理解さ
 れるべきだと言うことができる。なぜなら、感覚は、固有感覚対象について確
 かな把握をもつので、知性が何かを確かな把握にしたがって「感じる」と私た
 ちが言うことも、語る人々の言葉の使用範囲に含まれる。したがって、知性は
 またsententiaとも名付けられる。

\\
Experientia vero
Angelis attribui potest per similitudinem cognitorum, etsi non per
similitudinem virtutis cognoscitivae. 
Est enim in nobis experientia, dum
singularia per sensum cognoscimus, Angeli autem singularia cognoscunt,
ut infra patebit, sed non per sensum. 

&
他方、「経験」が天使に帰せられうるのは、認識されるものの類似によってであ
 り、認識能力の類似によってではない。なぜなら、経験が私たちの中にあるの
 は、個物を感覚によって認識するときだが、後で明らかになるように、天使は
 感覚によって個物を認識するのではないからである。


\\

Sed tamen {\itshape memoria} in Angelis
potest poni, secundum quod ab Augustino ponitur in mente; licet non
possit eis competere secundum quod ponitur pars animae
sensitivae. --Similiter dicendum quod {\itshape phantasia proterva} attribuitur
Daemonibus, ex eo quod habent falsam practicam existimationem de vero
bono, deceptio autem in nobis proprie fit secundum phantasiam, per quam
interdum similitudinibus rerum inhaeremus sicut rebus ipsis, ut patet in
dormientibus et amentibus.

&

しかし、記憶が天使の中に措定されうるのは、それがアウグスティヌスによって
 精神の中に措定されているかぎりにおいてである。ただし、それが感覚的魂の
 部分だとされるかぎりにおいては、記憶は天使たちに適合しえない。同様に、悪魔たち
 に「邪悪な表象力」が帰せられるのも、彼らが真の善について偽の実践的判断
 をもつことに基づくと言われるべきである。私たちにおいて、誤りは固有に表
 象力において生じる。眠っている人や狂気の人において明らかなように、表象
 力によって、時として、私たちは、事物の類似に、事物そのものであるかのよ
 うに、固着する。



\end{longtable}




\end{document}