\documentclass[10pt]{jsarticle}
\usepackage[utf8]{inputenc}
\usepackage[T1]{fontenc}
\usepackage{okumacro}
\usepackage{longtable}
\usepackage[polutonikogreek,english,japanese]{babel}
\usepackage{amsmath}
\usepackage{latexsym}
\usepackage{color}
\usepackage{otf}
\usepackage{schemata}
%----- header -------
\usepackage{fancyhdr}
\pagestyle{fancy}
\lhead{{\it Summa Theologiae} I-II, q.56}
%--------------------

\bibliographystyle{jplain}

\title{{\bf Prima Secundae}\\{\HUGE Summae Theologiae}\\Sancti Thomae
Aquinatis\\{\sffamily QUAESTIO QUINQUAGESIMASEXTA}\\{\bf DE SUBIECTO VIRTUTIS}}
\author{Japanese translation\\by Yoshinori {\sc Ueeda}}
\date{Last modified \today}

%%%% コピペ用
%\rhead{a.~}
%\begin{center}
% {\Large {\bf }}\\
% {\large }\\
% {\footnotesize }\\
% {\Large \\}
%\end{center}
%
%\begin{longtable}{p{21em}p{21em}}
%
%&
%
%\\
%\end{longtable}
%\newpage

\begin{document}

\maketitle
\thispagestyle{empty}
\begin{center}
{\Large 『神学大全』第二部の一\\第五十六問\\徳の基体について}
\end{center}


\begin{longtable}{p{21em}p{21em}}
Deinde considerandum est de subiecto virtutis. Et circa hoc quaeruntur sex. 

\begin{enumerate}
 \item utrum virtus sit in potentia animae sicut in subiecto.
 \item utrum una virtus possit esse in pluribus potentiis.
 \item utrum intellectus possit esse subiectum virtutis.
 \item utrum irascibilis et concupiscibilis.
 \item utrum vires apprehensivae sensitivae.
 \item utrum voluntas.
\end{enumerate}

&

次に徳の基体について考察されるべきである。これを巡って六つのことが問われる。

\begin{enumerate}
 \item 徳は、基体としての魂の能力の中にあるか。
 \item 一つの徳が複数の能力の中にありうるか。
 \item 知性は徳の基体でありうるか。
 \item 怒情的部分と欲情的部分は基体でありうるか。
 \item 感覚的把握の力は基体でありうるか。
 \item 意志は基体でありうるか。
\end{enumerate}
\end{longtable}


\newpage
\rhead{a.~1}
\begin{center}
{\Large {\bf ARTICULUS PRIMUS}}\\
{\large UTRUM VIRTUS SIT IN POTENTIA ANIMAE SICUT IN SUBIECTO}\\
{\footnotesize III {\itshape Sent.}, III, d.33, q.2, a.4, qu$^{a}$1; {\itshape De Virtut.}, q.1, a.3.}\\
{\Large 第一項\\徳は基体としての魂の能力の中にあるか。}
\end{center}

\begin{longtable}{p{21em}p{21em}}
{\scshape Ad primum sic proceditur}. Videtur quod virtus non sit in
potentia animae sicut in subiecto. Dicit enim Augustinus, in II
{\itshape de Lib.~Arbit}., quod {\itshape virtus est qua recte
vivitur}. Vivere autem non est per potentiam animae, sed per eius
essentiam. Ergo virtus non est in potentia animae, sed in eius
essentia.

&

第一項の問題へ、議論は以下のように進められる。徳は基体としての魂の能力
の中にあるのではないと思われる。理由は以下の通り。アウグスティヌスは
『自由意志』第2巻で「徳とは、それによって正しくいきられるところのもの
である」と述べている。しかるに生きることは魂の能力によってあるのではな
く、魂の本質によってある。ゆえに徳は魂の能力ではなく、魂の本質の中にあ
る。

\\



2. {\scshape Praeterea}, philosophus dicit, in II {\itshape Ethic}.,
{\itshape virtus est quae bonum facit habentem, et opus eius bonum
reddit}. Sed sicut opus constituitur per potentiam, ita habens
virtutem constituitur per essentiam animae. Ergo virtus non magis
pertinet ad potentiam animae, quam ad eius essentiam.

&

さらに、哲学者は『ニコマコス倫理学』第2巻で「徳とは、それを持つ人を善
い人にし、彼の業を善いものにするものである」と述べている。しかるに、業
が能力によって構成されるのと同じように、徳をもつ人は魂の本質によって構
成される。ゆえに徳は魂の能力と同様に、魂の本質にも属する。

\\



3. {\scshape Praeterea}, potentia est in secunda specie
qualitatis. Virtus autem est quaedam qualitas, ut supra dictum
est. Qualitatis autem non est qualitas. Ergo virtus non est in
potentia animae sicut in subiecto.

&

さらに、能力は性質の第二の種である。しかるに徳は、先に述べられたとおり、
ある種の性質である。しかるに性質が性質に属することはない。ゆえに徳は基
体としての魂の能力の中にあるのではない。

\\

{\scshape Sed contra}, {\itshape virtus est ultimum potentiae}, ut
dicitur in I {\itshape de Caelo}. Sed ultimum est in eo cuius est
ultimum. Ergo virtus est in potentia animae.

&

しかし反対に、『天体論』第1巻で言われるように「徳は能力の究極である」。しかるに、究極は、それの究極であるところのものの中にある。ゆえに徳は魂の能力の中にある。

\\

{\scshape Respondeo dicendum} quod virtutem pertinere ad potentiam
animae, ex tribus potest esse manifestum. Primo quidem, ex ipsa
ratione virtutis, quae importat perfectionem potentiae, perfectio
autem est in eo cuius est perfectio.

&

解答する。以下のように言われるべきである。徳が魂の能力に属すると言うこ
とは、三つのことから明らかにされうる。第一に、能力の完成を意味するとい
う徳の性格自体に基づいてである。完全性は、それの完全性であるところのも
のの中にある。

\\

Secundo, ex hoc quod est habitus operativus, ut supra dictum est,
omnis autem operatio est ab anima per aliquam potentiam.

&

第二に、先に述べられたとおり、それが活動的な習慣であることに基づく。す
べての働きは、何らかの能力を通して魂によってある。

\\

Tertio, ex hoc quod disponit ad optimum, optimum autem est finis, qui
vel est operatio rei, vel aliquid consecutum per operationem a
potentia egredientem. Unde virtus humana est in potentia animae sicut
in subiecto.

&

第三に、それが最善のものへ向かって態勢付けることに基づく。最善のものと
は目的であり、それは事物の働きであるか、能力からでてくる働きによって何
か伴うものかのいずれかである。したがって、人間の徳は、基体としての魂の
能力の中にある。

\\



{\scshape Ad primum ergo dicendum} quod vivere dupliciter
sumitur. Quandoque enim dicitur vivere ipsum esse viventis, et sic
pertinet ad essentiam animae, quae est viventi essendi
principium. Alio modo vivere dicitur operatio viventis, et sic virtute
recte vivitur, inquantum per eam aliquis recte operatur.

&

第一異論に対しては、それゆえ、以下のように言われるべきである。生きるこ
とは二通りに理解される。すなわち、時として、生きるものの「存在すること」
それ自体が「生きること」と言われる。この意味では、(生きることは)生き
るものにとっての存在の根源である魂の本質に属する。もう一つの意味では、
生きるものの働きが「生きること」といわれる。この意味で、ある人が徳によっ
て正しく働く限りにおいて、徳によって正しく生きられる。

\\



{\scshape Ad secundum dicendum} quod bonum vel est finis, vel in
ordine ad finem dicitur. Et ideo, cum bonum operantis consistat in
operatione, hoc etiam ipsum quod virtus facit operantem bonum,
refertur ad operationem, et per consequens ad potentiam.

&

第二異論に対しては以下のように言われるべきである。善は目的であるか、あ
るいは目的への秩序において言われる。ゆえに働くものの善は働きにおいて成
立するので、徳が働きを善いものにすること自体が、働きへ関係づけられ、そ
の結果として能力へ関係づけられる。

\\

{\scshape Ad tertium dicendum} quod unum accidens dicitur esse in alio
sicut in subiecto, non quia accidens per seipsum possit sustentare
aliud accidens, sed quia unum accidens inhaeret substantiae mediante
alio accidente, ut color corpori mediante superficie; unde superficies
dicitur esse subiectum coloris. Et eo modo potentia animae dicitur
esse subiectum virtutis.

&

第三異論に対しては以下のように言われるべきである。一つの附帯性が、基体
としての他のものの中にあると言われるのは、附帯性がそれ自体によって他の
附帯性を支持することができるからではなく、一つの附帯性が、他の附帯性が
媒介となって、実体に内在するからである。たとえば色が物体に内在するのは
表面が媒介となってである。このことから、表面は色の基体であると言われる。
同じしかたで、魂の能力が徳の基体と言われる。


\end{longtable}
\newpage

\rhead{a.~2}
\begin{center}
{\Large {\bf ARTICULUS SECUNDUS}}\\
{\large UTRUM UNA VIRTUS POSSIT ESSE IN PLURIBUS POTENTIIS}\\
{\footnotesize Infra, q.60, a.5; IV {\itshape Sent.}, d.14, q.1, a.3, qu$^{a}$ 1; {\itshape De Verit.}, q.14, a.4, ad 7.}\\
{\Large 第二項\\一つの徳が複数の能力の中にありうるか}
\end{center}

\begin{longtable}{p{21em}p{21em}}



\newpage
{\scshape Ad secundum sic proceditur}. Videtur quod una virtus possit
esse in duabus potentiis. Habitus enim cognoscuntur per actus. Sed
unus actus progreditur diversimode a diversis potentiis, sicut
ambulatio procedit a ratione ut a dirigente, a voluntate sicut a
movente, et a potentia motiva sicut ab exequente. Ergo etiam unus
habitus virtutis potest esse in pluribus potentiis.

&

第二項の問題へ、議論は以下のように進められる。一つの徳が二つの能力の中
にありうると思われる。理由は以下の通り。習慣は作用をとおして認識される。
しかるに一つの作用が、さまざまな仕方でさまざまな能力からでてくる。たと
えば歩く作用は、指示するものとしての理性と、動かすものとしての意志と、
実行するものとしての運動能力からでてくる。ゆえに一つの徳の習慣も、複数
の能力の中にありうると思われる。

\\



2. {\scshape Praeterea}, philosophus dicit, in II {\itshape Ethic}.,
quod ad virtutem tria requiruntur, scilicet {\itshape scire, velle et
immobiliter operari}. Sed scire pertinet ad intellectum, velle ad
voluntatem. Ergo virtus potest esse in pluribus potentiis.

&

さらに、哲学者は『ニコマコス倫理学』第2巻で、徳には三つのもの、すなわ
ち「知ること、意志すること、変わらない仕方で働くこと」が必要とされると
述べている。しかるに知ることは知性に、意志することは意志に属する。ゆえ
に得が複数の能力に属することは可能である。

\\


3. {\scshape Praeterea}, prudentia est in ratione, cum sit {\itshape
recta ratio} agibilium, ut dicitur in VI {\itshape Ethic}. Est etiam
in voluntate, quia non potest esse cum voluntate perversa, ut in eodem
libro dicitur. Ergo una virtus potest esse in duabus potentiis.

&

さらに、『ニコマコス倫理学』第6巻で言われるように、思慮は「為されうる
事柄についての正しい理」であるので、理性の中にある。また、意志の中にも
ある。なぜなら、同じ本の中で言われているように、ひねくれた意志と共にあ
ることができないからである。ゆえに一つの徳が二つの能力の中にありうる。

\\

{\scshape Sed contra}, virtus est in potentia animae sicut in
subiecto. Sed idem accidens non potest esse in pluribus
subiectis. Ergo una virtus non potest esse in pluribus potentiis
animae.

&

しかし反対に、徳は、基体としての魂の能力の中にある。しかるに同一の附帯
性が複数の基体の中にはありえない。ゆえに一つの徳が複数の魂の能力の中に
はありえない。

\\

{\scshape Respondeo dicendum} quod aliquid esse in duobus, contingit
dupliciter. Uno modo, sic quod ex aequo sit in utroque. Et sic
impossibile est unam virtutem esse in duabus potentiis, quia
diversitas potentiarum attenditur secundum generales conditiones
obiectorum, diversitas autem habituum secundum speciales; unde
ubicumque est diversitas potentiarum, est diversitas habituum, sed non
convertitur.

&

解答する。以下のように言われるべきである。あるものが二つのものの中にあ
る、ということは二通りのしかたで起こる。一つには、等しく両方の中にとい
うしかたであり、この意味では、一つの徳が二つの能力の中にあることは不可
能である。なぜなら、能力の違いは対象の一般的な条件に従って見出されるが、
習慣の違いは特殊的な条件に従って見出されるからである。したがって、能力
の違いがあるところにはいつも習慣の違いがあるが、その逆は成り立たない。

\\

Alio modo potest esse aliquid in duobus vel pluribus, non ex aequo,
sed ordine quodam. Et sic una virtus pertinere potest ad plures
potentias; ita quod in una sit principaliter, et se extendat ad alias
per modum diffusionis, vel per modum dispositionis; secundum quod una
potentia movetur ab alia, et secundum quod una potentia accipit ab
alia.

&

もう一つの仕方で、あるものが二つのものや複数のものの中にあるのは、等し
くではなく、何らかの秩序に基づいてである場合である。この意味では、一つ
の徳が複数の能力に属することがありうる。つまり一つの能力の中に主要にあ
り、それが他の能力へ流出する仕方で、あるいは配置の仕方で及ぶ場合である。
それは、ある能力が他の能力によって動かされる限りにおいて、そしてある能
力が他の能力から何かを受け取る限りにおいて生じる。

\\

{\scshape Ad primum ergo dicendum} quod idem actus non potest
aequaliter, et eodem ordine, pertinere ad diversas potentias, sed
secundum diversas rationes, et diverso ordine.

&

第一異論に対しては、それゆえ以下のように言われるべきである。同一の作用
が等しく同一の秩序によって複数の能力に属することはできないが、異なる性
格において、異なる秩序によってであれば、それが可能である。

\\

{\scshape Ad secundum dicendum} quod scire praeexigitur ad virtutem
moralem, inquantum virtus moralis operatur secundum rationem
rectam. Sed essentialiter in appetendo virtus moralis consistit.

&

第二異論に対しては以下のように言われるべきである。知ることは、道徳的な
徳が正しい理にしたがって働く限りにおいて、道徳的な徳のために前もって必
要とされる。しかし本質的には、道徳的な徳は欲求することにおいて成立する。

\\

{\scshape Ad tertium dicendum} quod prudentia realiter est in ratione
sicut in subiecto, sed praesupponit rectitudinem voluntatis sicut
principium, ut infra dicetur.

&

第三異論に対しては以下のように言われるべきである。思慮は、実在的には基
体としての理性の中にあるが、後で述べられるように、根源として、意志の正
しさを前提とする。


\end{longtable}
\end{document}

%\rhead{a.~}
%\begin{center}
% {\Large {\bf }}\\
% {\large }\\
% {\footnotesize }\\
% {\Large \\}
%\end{center}
%
%\begin{longtable}{p{21em}p{21em}}
%
%&
%
%\\
%\end{longtable}
%\newpage


ARTICULUS 3
[35847] Iª-IIae q. 56 a. 3 arg. 1
Ad tertium sic proceditur. Videtur quod intellectus non sit subiectum virtutis. Dicit enim Augustinus, in libro de moribus Eccles., quod omnis virtus est amor. Subiectum autem amoris non est intellectus, sed solum vis appetitiva. Ergo nulla virtus est in intellectu.

[35848] Iª-IIae q. 56 a. 3 arg. 2
Praeterea, virtus ordinatur ad bonum, sicut ex supradictis patet. Bonum autem non est obiectum intellectus, sed appetitivae virtutis. Ergo subiectum virtutis non est intellectus, sed appetitiva virtus.

[35849] Iª-IIae q. 56 a. 3 arg. 3
Praeterea, virtus est quae bonum facit habentem, ut philosophus dicit. Sed habitus perficiens intellectum non facit bonum habentem, non enim propter scientiam vel artem dicitur homo bonus. Ergo intellectus non est subiectum virtutis.

[35850] Iª-IIae q. 56 a. 3 s. c.
Sed contra est quod mens maxime dicitur intellectus. Subiectum autem virtutis est mens; ut patet ex definitione virtutis supra inducta. Ergo intellectus est subiectum virtutis.

[35851] Iª-IIae q. 56 a. 3 co.
Respondeo dicendum quod, sicut supra dictum est, virtus est habitus quo quis bene operatur. Dupliciter autem habitus aliquis ordinatur ad bonum actum. Uno modo, inquantum per huiusmodi habitum acquiritur homini facultas ad bonum actum, sicut per habitum grammaticae habet homo facultatem recte loquendi. Non tamen grammatica facit ut homo semper recte loquatur, potest enim grammaticus barbarizare aut soloecismum facere. Et eadem ratio est in aliis scientiis et artibus. Alio modo, aliquis habitus non solum facit facultatem agendi, sed etiam facit quod aliquis recte facultate utatur, sicut iustitia non solum facit quod homo sit promptae voluntatis ad iusta operandum, sed etiam facit ut iuste operetur. Et quia bonum, sicut et ens, non dicitur simpliciter aliquid secundum id quod est in potentia, sed secundum id quod est in actu; ideo ab huiusmodi habitibus simpliciter dicitur homo bonum operari, et esse bonus, puta quia est iustus vel temperatus; et eadem ratio est de similibus. Et quia virtus est quae bonum facit habentem, et opus eius bonum reddit, huiusmodi habitus simpliciter dicuntur virtutes, quia reddunt bonum opus in actu, et simpliciter faciunt bonum habentem. Primi vero habitus non simpliciter dicuntur virtutes, quia non reddunt bonum opus nisi in quadam facultate, nec simpliciter faciunt bonum habentem. Non enim dicitur simpliciter aliquis homo bonus, ex hoc quod est sciens vel artifex, sed dicitur bonus solum secundum quid, puta bonus grammaticus, aut bonus faber. Et propter hoc, plerumque scientia et ars contra virtutem dividitur, quandoque autem virtutes dicuntur, ut patet in VI Ethic. Subiectum igitur habitus qui secundum quid dicitur virtus, potest esse intellectus, non solum practicus, sed etiam intellectus speculativus, absque omni ordine ad voluntatem, sic enim philosophus, in VI Ethic., scientiam, sapientiam et intellectum, et etiam artem, ponit esse intellectuales virtutes. Subiectum vero habitus qui simpliciter dicitur virtus, non potest esse nisi voluntas; vel aliqua potentia secundum quod est mota a voluntate. Cuius ratio est, quia voluntas movet omnes alias potentias quae aliqualiter sunt rationales, ad suos actus, ut supra habitum est, et ideo quod homo actu bene agat, contingit ex hoc quod homo habet bonam voluntatem. Unde virtus quae facit bene agere in actu, non solum in facultate, oportet quod vel sit in ipsa voluntate; vel in aliqua potentia secundum quod est a voluntate mota. Contingit autem intellectum a voluntate moveri, sicut et alias potentias, considerat enim aliquis aliquid actu, eo quod vult. Et ideo intellectus, secundum quod habet ordinem ad voluntatem, potest esse subiectum virtutis simpliciter dictae. Et hoc modo intellectus speculativus, vel ratio, est subiectum fidei, movetur enim intellectus ad assentiendum his quae sunt fidei, ex imperio voluntatis; nullus enim credit nisi volens. Intellectus vero practicus est subiectum prudentiae. Cum enim prudentia sit recta ratio agibilium, requiritur ad prudentiam quod homo se bene habeat ad principia huius rationis agendorum, quae sunt fines; ad quos bene se habet homo per rectitudinem voluntatis, sicut ad principia speculabilium per naturale lumen intellectus agentis. Et ideo sicut subiectum scientiae, quae est ratio recta speculabilium, est intellectus speculativus in ordine ad intellectum agentem; ita subiectum prudentiae est intellectus practicus in ordine ad voluntatem rectam.

[35852] Iª-IIae q. 56 a. 3 ad 1
Ad primum ergo dicendum quod verbum Augustini intelligendum est de virtute simpliciter dicta non quod omnis talis virtus sit simpliciter amor; sed quia dependet aliqualiter ab amore, inquantum dependet a voluntate, cuius prima affectio est amor, ut supra dictum est.

[35853] Iª-IIae q. 56 a. 3 ad 2
Ad secundum dicendum quod bonum uniuscuiusque est finis eius, et ideo, cum verum sit finis intellectus, cognoscere verum est bonus actus intellectus. Unde habitus perficiens intellectum ad verum cognoscendum, vel in speculativis vel in practicis, dicitur virtus.

[35854] Iª-IIae q. 56 a. 3 ad 3
Ad tertium dicendum quod ratio illa procedit de virtute simpliciter dicta.



%\rhead{a.~}
%\begin{center}
% {\Large {\bf }}\\
% {\large }\\
% {\footnotesize }\\
% {\Large \\}
%\end{center}
%
%\begin{longtable}{p{21em}p{21em}}
%
%&
%
%\\
%\end{longtable}
%\newpage


ARTICULUS 4
[35855] Iª-IIae q. 56 a. 4 arg. 1
Ad quartum sic proceditur. Videtur quod irascibilis et concupiscibilis non possint esse subiectum virtutis. Huiusmodi enim vires sunt communes nobis et brutis. Sed nunc loquimur de virtute secundum quod est propria homini, sic enim dicitur virtus humana. Non igitur humanae virtutis potest esse subiectum irascibilis et concupiscibilis, quae sunt partes appetitus sensitivi, ut in primo dictum est.

[35856] Iª-IIae q. 56 a. 4 arg. 2
Praeterea, appetitus sensitivus est vis utens organo corporali. Sed bonum virtutis non potest esse in corpore hominis, dicit enim apostolus, Rom. VII, scio quod non habitat in carne mea bonum. Ergo appetitus sensitivus non potest esse subiectum virtutis.

[35857] Iª-IIae q. 56 a. 4 arg. 3
Praeterea, Augustinus probat, in libro de moribus Eccles., quod virtus non est in corpore, sed in anima, eo quod per animam corpus regitur, unde quod aliquis corpore bene utatur, totum refertur ad animam; sicut si mihi auriga obtemperans, equos quibus praeest, recte regit, hoc totum mihi debetur. Sed sicut anima regit corpus, ita etiam ratio regit appetitum sensitivum. Ergo totum rationali parti debetur, quod irascibilis et concupiscibilis recte regantur. Sed virtus est qua recte vivitur, ut supra dictum est. Virtus igitur non est in irascibili et concupiscibili, sed solum in parte rationali.

[35858] Iª-IIae q. 56 a. 4 arg. 4
Praeterea, principalis actus virtutis moralis est electio, ut dicitur in VIII Ethic. Sed electio non est actus irascibilis et concupiscibilis, sed rationis, ut supra dictum est. Ergo virtus moralis non est in irascibili et concupiscibili, sed in ratione.

[35859] Iª-IIae q. 56 a. 4 s. c.
Sed contra est quod fortitudo ponitur esse in irascibili, temperantia autem in concupiscibili. Unde philosophus dicit, in III Ethic., quod hae virtutes sunt irrationabilium partium.

[35860] Iª-IIae q. 56 a. 4 co.
Respondeo dicendum quod irascibilis et concupiscibilis dupliciter considerari possunt. Uno modo secundum se, inquantum sunt partes appetitus sensitivi. Et hoc modo, non competit eis quod sint subiectum virtutis. Alio modo possunt considerari inquantum participant rationem, per hoc quod natae sunt rationi obedire. Et sic irascibilis vel concupiscibilis potest esse subiectum virtutis humanae, sic enim est principium humani actus, inquantum participat rationem. Et in his potentiis necesse est ponere virtutes. Quod enim in irascibili et concupiscibili sint aliquae virtutes, patet. Actus enim qui progreditur ab una potentia secundum quod est ab alia mota, non potest esse perfectus, nisi utraque potentia sit bene disposita ad actum, sicut actus artificis non potest esse congruus, nisi et artifex sit bene dispositus ad agendum, et etiam ipsum instrumentum. In his igitur circa quae operatur irascibilis et concupiscibilis secundum quod sunt a ratione motae, necesse est ut aliquis habitus perficiens ad bene agendum sit non solum in ratione, sed etiam in irascibili et concupiscibili. Et quia bona dispositio potentiae moventis motae, attenditur secundum conformitatem ad potentiam moventem; ideo virtus quae est in irascibili et concupiscibili, nihil aliud est quam quaedam habitualis conformitas istarum potentiarum ad rationem.

[35861] Iª-IIae q. 56 a. 4 ad 1
Ad primum ergo dicendum quod irascibilis et concupiscibilis secundum se consideratae, prout sunt partes appetitus sensitivi, communes sunt nobis et brutis. Sed secundum quod sunt rationales per participationem, ut obedientes rationi, sic sunt propriae hominis. Et hoc modo possunt esse subiectum virtutis humanae.

[35862] Iª-IIae q. 56 a. 4 ad 2
Ad secundum dicendum quod, sicut caro hominis ex se quidem non habet bonum virtutis, fit tamen instrumentum virtuosi actus, inquantum, movente ratione, membra nostra exhibemus ad serviendum iustitiae, ita etiam irascibilis et concupiscibilis ex se quidem non habent bonum virtutis sed magis infectionem fomitis; inquantum vero conformantur rationi, sic in eis adgeneratur bonum virtutis moralis.

[35863] Iª-IIae q. 56 a. 4 ad 3
Ad tertium dicendum quod alia ratione regitur corpus ab anima, et irascibilis et concupiscibilis a ratione. Corpus enim ad nutum obedit animae absque contradictione, in his in quibus natum est ab anima moveri, unde philosophus dicit, in I Polit., quod anima regit corpus despotico principatu, idest sicut dominus servum. Et ideo totus motus corporis refertur ad animam. Et propter hoc in corpore non est virtus, sed solum in anima. Sed irascibilis et concupiscibilis non ad nutum obediunt rationi, sed habent proprios motus suos, quibus interdum rationi repugnant, unde in eodem libro philosophus dicit quod ratio regit irascibilem et concupiscibilem principatu politico, quo scilicet reguntur liberi, qui habent in aliquibus propriam voluntatem. Et propter hoc etiam oportet in irascibili et concupiscibili esse aliquas virtutes, quibus bene disponantur ad actum.

[35864] Iª-IIae q. 56 a. 4 ad 4
Ad quartum dicendum quod in electione duo sunt, scilicet intentio finis, quae pertinet ad virtutem moralem; et praeacceptio eius quod est ad finem, quod pertinet ad prudentiam; ut dicitur in VI Ethic. Quod autem habeat rectam intentionem finis circa passiones animae, hoc contingit ex bona dispositione irascibilis et concupiscibilis. Et ideo virtutes morales circa passiones, sunt in irascibili et concupiscibili, sed prudentia est in ratione.



%\rhead{a.~}
%\begin{center}
% {\Large {\bf }}\\
% {\large }\\
% {\footnotesize }\\
% {\Large \\}
%\end{center}
%
%\begin{longtable}{p{21em}p{21em}}
%
%&
%
%\\
%\end{longtable}
%\newpage


ARTICULUS 5
[35865] Iª-IIae q. 56 a. 5 arg. 1
Ad quintum sic proceditur. Videtur quod in viribus sensitivis apprehensivis interius, possit esse aliqua virtus. Appetitus enim sensitivus potest esse subiectum virtutis, inquantum obedit rationi. Sed vires sensitivae apprehensivae interius, rationi obediunt, ad imperium enim rationis operatur et imaginativa et cogitativa et memorativa. Ergo in his viribus potest esse virtus.

[35866] Iª-IIae q. 56 a. 5 arg. 2
Praeterea, sicut appetitus rationalis, qui est voluntas, in suo actu potest impediri, vel etiam adiuvari, per appetitum sensitivum; ita etiam intellectus vel ratio potest impediri, vel etiam iuvari, per vires praedictas. Sicut ergo in viribus sensitivis appetitivis potest esse virtus, ita etiam in apprehensivis.

[35867] Iª-IIae q. 56 a. 5 arg. 3
Praeterea, prudentia est quaedam virtus, cuius partem ponit Tullius memoriam, in sua rhetorica. Ergo etiam in vi memorativa potest esse aliqua virtus. Et eadem ratione, in aliis interioribus apprehensivis viribus.

[35868] Iª-IIae q. 56 a. 5 s. c.
Sed contra est quod omnes virtutes vel sunt intellectuales, vel morales, ut dicitur in II Ethic. Morales autem virtutes omnes sunt in parte appetitiva, intellectuales autem in intellectu vel ratione, sicut patet in VI Ethic. Nulla ergo virtus est in viribus sensitivis apprehensivis interius.

[35869] Iª-IIae q. 56 a. 5 co.
Respondeo dicendum quod in viribus sensitivis apprehensivis interius, ponuntur aliqui habitus. Quod patet ex hoc praecipue quod philosophus dicit, in libro de memoria, quod in memorando unum post aliud, operatur consuetudo, quae est quasi quaedam natura, nihil autem est aliud habitus consuetudinalis quam habitudo acquisita per consuetudinem, quae est in modum naturae. Unde de virtute dicit Tullius, in sua rhetorica, quod est habitus in modum naturae, rationi consentaneus. In homine tamen id quod ex consuetudine acquiritur in memoria, et in aliis viribus sensitivis apprehensivis, non est habitus per se; sed aliquid annexum habitibus intellectivae partis, ut supra dictum est. Sed tamen si qui sunt habitus in talibus viribus, virtutes dici non possunt. Virtus enim est habitus perfectus, quo non contingit nisi bonum operari, unde oportet quod virtus sit in illa potentia quae est consummativa boni operis. Cognitio autem veri non consummatur in viribus sensitivis apprehensivis; sed huiusmodi vires sunt quasi praeparatoriae ad cognitionem intellectivam. Et ideo in huiusmodi viribus non sunt virtutes, quibus cognoscitur verum; sed magis in intellectu vel ratione.

[35870] Iª-IIae q. 56 a. 5 ad 1
Ad primum ergo dicendum quod appetitus sensitivus se habet ad voluntatem, quae est appetitus rationis, sicut motus ab eo. Et ideo opus appetitivae virtutis consummatur in appetitu sensitivo. Et propter hoc, appetitus sensitivus est subiectum virtutis. Virtutes autem sensitivae apprehensivae magis se habent ut moventes respectu intellectus, eo quod phantasmata se habent ad animam intellectivam, sicut colores ad visum, ut dicitur in III de anima. Et ideo opus cognitionis in intellectu terminatur. Et propter hoc, virtutes cognoscitivae sunt in ipso intellectu vel ratione.

[35871] Iª-IIae q. 56 a. 5 ad 2
Et per hoc patet solutio ad secundum.

[35872] Iª-IIae q. 56 a. 5 ad 3
Ad tertium dicendum quod memoria non ponitur pars prudentiae, sicut species est pars generis, quasi ipsa memoria sit quaedam virtus per se, sed quia unum eorum quae requiruntur ad prudentiam, est bonitas memoriae; ut sic quodammodo se habeat per modum partis integralis.



%\rhead{a.~}
%\begin{center}
% {\Large {\bf }}\\
% {\large }\\
% {\footnotesize }\\
% {\Large \\}
%\end{center}
%
%\begin{longtable}{p{21em}p{21em}}
%
%&
%
%\\
%\end{longtable}
%\newpage


ARTICULUS 6
[35873] Iª-IIae q. 56 a. 6 arg. 1
Ad sextum sic proceditur. Videtur quod voluntas non sit subiectum alicuius virtutis. Ad id enim quod convenit potentiae ex ipsa ratione potentiae, non requiritur aliquis habitus. Sed de ipsa ratione voluntatis, cum sit in ratione, secundum philosophum in III de anima, est quod tendat in id quod est bonum secundum rationem, ad quod ordinatur omnis virtus, quia unumquodque naturaliter appetit proprium bonum, virtus enim est habitus per modum naturae, consentaneus rationi, ut Tullius dicit in sua rhetorica. Ergo voluntas non est subiectum virtutis.

[35874] Iª-IIae q. 56 a. 6 arg. 2
Praeterea, omnis virtus aut est intellectualis, aut moralis, ut dicitur in I et II Ethic. Sed virtus intellectualis est, sicut in subiecto, in intellectu et ratione, non autem in voluntate, virtus autem moralis est, sicut in subiecto, in irascibili et concupiscibili, quae sunt rationales per participationem. Ergo nulla virtus est in voluntate sicut in subiecto.

[35875] Iª-IIae q. 56 a. 6 arg. 3
Praeterea, omnes actus humani, ad quos virtutes ordinantur, sunt voluntarii. Si igitur respectu aliquorum humanorum actuum sit aliqua virtus in voluntate, pari ratione respectu omnium actuum humanorum erit virtus in voluntate. Aut ergo in nulla alia potentia erit aliqua virtus, aut ad eundem actum ordinabuntur duae virtutes, quod videtur inconveniens. Voluntas ergo non potest esse subiectum virtutis.

[35876] Iª-IIae q. 56 a. 6 s. c.
Sed contra est quod maior perfectio requiritur in movente quam in moto. Sed voluntas movet irascibilem et concupiscibilem. Multo ergo magis debet esse virtus in voluntate, quam in irascibili et concupiscibili.

[35877] Iª-IIae q. 56 a. 6 co.
Respondeo dicendum quod, cum per habitum perficiatur potentia ad agendum, ibi indiget potentia habitu perficiente ad bene agendum, qui quidem habitus est virtus, ubi ad hoc non sufficit propria ratio potentiae. Omnis autem potentiae propria ratio attenditur in ordine ad obiectum. Unde cum, sicut dictum est, obiectum voluntati sit bonum rationis voluntati proportionatum, quantum ad hoc non indiget voluntas virtute perficiente. Sed si quod bonum immineat homini volendum, quod excedat proportionem volentis; sive quantum ad totam speciem humanam, sicut bonum divinum, quod transcendit limites humanae naturae, sive quantum ad individuum, sicut bonum proximi; ibi voluntas indiget virtute. Et ideo huiusmodi virtutes quae ordinant affectum hominis in Deum vel in proximum, sunt in voluntate sicut in subiecto; ut caritas, iustitia et huiusmodi.

[35878] Iª-IIae q. 56 a. 6 ad 1
Ad primum ergo dicendum quod ratio illa habet locum de virtute quae ordinat ad bonum proprium ipsius volentis, sicut temperantia et fortitudo, quae sunt circa passiones humanas et alia huiusmodi, ut ex dictis patet.

[35879] Iª-IIae q. 56 a. 6 ad 2
Ad secundum dicendum quod rationale per participationem non solum est irascibilis et concupiscibilis; sed omnino, idest universaliter, appetitivum, ut dicitur in I Ethic. Sub appetitivo autem comprehenditur voluntas. Et ideo, si qua virtus est in voluntate, erit moralis, nisi sit theologica, ut infra patebit.

[35880] Iª-IIae q. 56 a. 6 ad 3
Ad tertium dicendum quod quaedam virtutes ordinantur ad bonum passionis moderatae, quod est proprium huius vel illius hominis, et in talibus non est necessarium quod sit aliqua virtus in voluntate, cum ad hoc sufficiat natura potentiae, ut dictum est. Sed hoc solum necessarium est in illis virtutibus quae ordinantur ad aliquod bonum extrinsecum.


