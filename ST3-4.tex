\documentclass[10pt]{jsarticle} % use larger type; default would be 10pt
%\usepackage[utf8]{inputenc} % set input encoding (not needed with XeLaTeX)
%\usepackage[round,comma,authoryear]{natbib}
%\usepackage{nruby}
\usepackage{okumacro}
\usepackage{longtable}
%\usepqckage{tablefootnote}
\usepackage[polutonikogreek,english,japanese]{babel}
%\usepackage{amsmath}
\usepackage{latexsym}
\usepackage{color}

%----- header -------
\usepackage{fancyhdr}
\lhead{{\it Summa Theologiae} III, q.~4}
%--------------------

\bibliographystyle{jplain}

\title{{\bf TERTIA PARS}\\{\HUGE Summae Theologiae}\\Sancti Thomae
Aquinatis\\{\sffamily QUEAESTIO QUARTA}\\DE UNIONE EX PARTE ASSUMPTI}
\author{Japanese translation\\by Yoshinori {\sc Ueeda}}
\date{Last modified \today}


%%%% コピペ用
%\rhead{a.~}
%\begin{center}
% {\Large {\bf }}\\
% {\large }\\
% {\footnotesize }\\
% {\Large \\}
%\end{center}
%
%\begin{longtable}{p{21em}p{21em}}
%
%&
%
%
%\\
%\end{longtable}
%\newpage



\begin{document}
\maketitle
\pagestyle{fancy}


\begin{center}
{\Large 第四問\\合一について:受容されるものの側から}
\end{center}


\begin{longtable}{p{21em}p{21em}}




{\Huge D}{\scshape einde} considerandum est de unione ex parte
 assumpti. Circa quod primo considerandum occurrit de his quae sunt a
 Verbo Dei assumpta; secundo, de coassumptis, quae sunt perfectiones et
 defectus. 

&

次に、受容されたものの側から合一について考察されるべきである。これを巡っ
 て、第一に、神の言葉によって受容されたものについて、第二に、共に受容さ
 れたもの、すなわち、完全性と欠点について、考察されるべきである。

\\



Assumpsit autem Filius Dei humanam naturam, et partes eius. Unde circa
 primum triplex consideratio occurrit, prima est, quantum ad ipsam
 naturam humanam; secunda est, quantum ad partes ipsius; tertia, quantum
 ad ordinem assumptionis. 



&

ところで、神の息子は人間本性とそれの諸部分を受容した。したがって、第一の
 ことを巡って、三通りの考察が生じる。第一は、人間本性自体にかんして、第
 二に、その諸部分に関して、第三に、受容への秩序に関して。


\\



Circa primum quaeruntur sex. 

\begin{enumerate}
 \item utrum humana natura fuerit magis assumptibilis a filio Dei quam
       aliqua alia natura.
 \item utrum assumpserit personam.
 \item utrum assumpserit hominem.
 \item utrum fuisset conveniens quod assumpsisset humanam naturam a
       singularibus separatam.
 \item utrum fuerit conveniens quod assumpsisset humanam naturam in
       omnibus singularibus.
 \item utrum fuerit conveniens quod assumeret humanam naturam in aliquo homine ex stirpe Adae progenito.
\end{enumerate}


&

第一のことを巡って、六つのことが問われる。
\begin{enumerate}
 \item 人間本性は、他の本性よりも、神の息子によって受容されうるものだっ
       たか。
 \item ペルソナを受容したか。
 \item 人間を受容したか。
 \item 個々の人から分離された人間本性を受容したとしたら、それは適切だっ
       たか。
 \item 全ての個人の中にある人間本性を受容したとしたら、それは適切だった
       か。
 \item アダムの家系から生まれたある人間において、人間本性を受容したこと
       は適切だったか。
\end{enumerate}

\end{longtable}

\newpage



\rhead{a.~1}
\begin{center}
 {\Large {\bf ARTICULUS PRIMUS}}\\
 {\large UTRUM HUMANA NATURA FUERIT MAGIS ASSUMPTIBILIS\\A FILIO DEI
 QUAM QUAELIBET ALIA NATURA}\\
 {\footnotesize III {\itshape Sent.}, d.2, a.1, a.1; IV {\itshape SCG},
 cap.53, 55; {\itshape Ad Hebr.}, cap.2, letc.4.}\\
 {\Large 第一項\\人間本性は、他のどの本性よりも、\\神の息子によって受容さ
 れうるものだったか}
\end{center}

\begin{longtable}{p{21em}p{21em}}

{\Huge A}{\scshape d primum sic proceditur}. Videtur quod humana natura non fuerit magis
assumptibilis a filio Dei quam quaelibet alia natura. Dicit enim
Augustinus, in epistola {\itshape ad Volusianum}, {\itshape in rebus mirabiliter factis tota
ratio facti est potentia facientis}. Sed potentia Dei facientis
incarnationem, quae est opus maxime mirabile, non limitatur ad unam
naturam, cum potentia Dei sit infinita. Ergo natura humana non est magis
assumptibilis a Deo quam aliqua alia creatura.


&

第一項の問題へ、議論は以下のように進められる。
人間本性は、他のどんな本性よりも、神の息子によって受容されうるものだった
 わけではないと思われる。理由は以下の通り。
アウグスティヌスは、『ウォルシアヌス宛』書簡で、「驚くべきかたちで作られ
 た事物において、作られたものの根拠は作るものの能力である」と述べている。
しかし、受肉を行う神の能力、これは最大限に驚くべきものだが、は、神能力が
 無限なので、一つの本性に限られていない。ゆえに、人間本性は、他のどんな
被造物よりも、神によって受容されうるものであるわけではない。



\\



2. {\scshape Praeterea}, similitudo est ratio faciens ad congruitatem incarnationis
divinae personae, ut supra dictum est. Sed sicut in natura rationali
invenitur similitudo imaginis, ita in natura irrationali invenitur
similitudo vestigii. Ergo creatura irrationalis assumptibilis fuit,
sicut humana natura.


&


さらに、前に述べられたとおり、類似は、神のペルソナの受肉の適切さを根拠付
 ける。しかし、理性的本性において、似姿への類似が見出されるように、非理
 性的本性においても、痕跡の類似が見出される。ゆえに、非理性的被造物は、
 人間本性と同様に、受容可能であった。

\\



3. {\scshape Praeterea}, in natura angelica invenitur expressior Dei similitudo quam
in natura humana, sicut Gregorius dicit, in Homilia {\itshape de Centum Ovibus},
introducens illud Ezech.~{\scshape xxviii}, {\itshape tu signaculum similitudinis}. Invenitur
etiam in Angelo peccatum, sicut in homine, secundum illud {\itshape Iob}
 {\scshape iv}, {\itshape in
Angelis suis reperit pravitatem}. Ergo natura angelica fuit ita
assumptibilis sicut natura hominis.


&

さらに、グレゴリウスが『百頭の羊について』の講解で、かの『エゼキエル書』
 第28章「あなたは、類似のしるし」\footnote{「人の子よ、ティルスの王に対して嘆きの歌
 をうたい、彼に言いなさい。主なる神はこう言われる。お前はあるべき姿を印
 章としたものであり/知恵に満ち、美しさの極みである。」(28:12)」}を引い
 て述べるように、天使の本性において、人間本性におけるよりも、より明らか
 な神の類似が見出される。さらに、『ヨブ記』第4章「自分の天使たちの中で、
 醜さを見つけ出す」\footnote{「神はその僕たちをも信頼せず/御使いたちを
 さえ賞賛されない。」(4:18)}によれば、天使の中には、人間と同様に罪も見出
 される。ゆえに、天使の本性は、人間本性と同様に、受容可能であった。





\\



4. {\scshape Praeterea}, cum Deo competat summa perfectio, tanto magis est Deo aliquid
simile, quanto est magis perfectum. Sed totum universum est magis
perfectum quam partes eius, inter quas est humana natura. Ergo totum
universum est magis assumptibile quam humana natura.


&


さらに、神には最高の完全性が適合するので、神に似ているほど、それだけ完全
 である。しかし、全宇宙は、その部分よりもより完全であり、人間本性は、そ
 の部分である。ゆえに、全宇宙は、人間本性よりも、受容可能である。

\\



{\scshape Sed contra est} quod dicitur {\itshape Prov}.~{\scshape viii}, ex ore sapientiae genitae,
{\itshape deliciae meae esse cum filiis hominum}. Et ita videtur esse quaedam
congruentia unionis filii Dei ad humanam naturam.


&


しかし反対に、『箴言』第8章で、生まれた知恵の口から「人間の息子たちとと
 もに、私の喜びがある」「\footnote{「主の造られたこの地上の人々と共に楽
 を奏し/人の子らと共に楽しむ。」(8:31)}と言われている。したがって、神の
 息子の人間本性への合一の、ある種の適合性があるように思われる。

\\



{\scshape Respondeo dicendum} quod aliquid assumptibile dicitur quasi aptum assumi
a divina persona. Quae quidem aptitudo non potest intelligi secundum
potentiam passivam naturalem, quae non se extendit ad id quod
transcendit ordinem naturalem, quem transcendit unio personalis
creaturae ad Deum. 


&

解答する。以下のように言われるべきである。
あるものが、受容可能であるということは、神のペルソナによって受容されるこ
 とが適しているという意味で語られる。
この適性は自然的な受動能力に即しては理解されえない。なぜなら、そのような
 能力は、自然の秩序を超えないが、被造物の神へのペルソナ的合一は、その秩
 序を越えているからである。


\\

Unde relinquitur quod assumptibile aliquid dicatur
secundum congruentiam ad unionem praedictam. Quae quidem congruentia
attenditur secundum duo in humana natura, scilicet secundum eius
dignitatem; et necessitatem. 



&

したがって、残っているのは、既述の合一への適合性に即して、なにかが受容可
 能と言われることである。この適合性は、人間本性において、二つの点で見出
 される。すなわち、その偉大さの点、そして必要性の点である。


\\

Secundum dignitatem quidem, quia humana
natura, inquantum est rationalis et intellectualis, nata est contingere
aliqualiter ipsum Verbum per suam operationem, cognoscendo scilicet et
amando ipsum. Secundum necessitatem autem, quia indigebat reparatione,
cum subiaceret originali peccato.


&

偉大さの点では、人間本性は、理性的であり知性的であるかぎりにおいて、何ら
 かのかたちで、すなわち、それを認識し愛するというかたちで、自分の働きを
 通して言葉自体に触れるような本性を持つ。また、必要性に即して言えば、そ
 れが原罪のもとにあったために、償いを必要していたからである。



\\

 Haec autem duo soli humanae naturae
conveniunt, nam creaturae irrationali deest congruitas dignitatis;
naturae autem angelicae deest congruitas praedictae necessitatis. Unde
relinquitur quod sola natura humana sit assumptibilis.


&

しかし、これら二つは、人間本性だけに適合する。なぜなら、非理性的な被造物
 は偉大さという適性を持たず、また、天使的本性は、上述の必要性の面での適
 性を持たないからである。したがって、人間本性だけが、受容可能であるとい
 うことが残る。\footnote{このあたりの議論は循環している。あるいは論点先
 取。}


\\



{\scshape Ad primum ergo dicendum} quod creaturae denominantur aliquales ex eo quod
competit eis secundum proprias causas, non autem ex eo quod convenit eis
secundum primas causas et universales, sicut dicimus aliquem morbum esse
incurabilem, non quia non potest curari a Deo, sed quia per propria
principia subiecti curari non potest. Sic ergo dicitur aliqua creatura
non esse assumptibilis, non ad subtrahendum aliquid divinae potentiae,
sed ad ostendendum conditionem creaturae quae ad hoc aptitudinem non
habet.


&

第一異論に対しては、それゆえ、以下のように言われるべきである。
被造物がある性質を持つものとして語られるのは、それに固有の原因に即してで
 あり、第一原因や普遍的原因に即してそれに適合するものによってではない。
 たとえば、私たちが、ある病気が不治であると言うのは、神によっても治療で
 きないからではなく、その基体に固有の諸根源によって治療できないからであ
 る。ゆえに、ある被造物が受容されえないと言われるのは、神の能力のなにか
 を取り去るためではなく、受容に対して適性を持たない被造物の条件を明らか
 にするためにである。


\\



{\scshape Ad secundum dicendum} quod similitudo imaginis attenditur in natura
humana secundum quod est capax Dei, scilicet ipsum attingendo propria
operatione cognitionis et amoris. Similitudo autem vestigii attenditur
solum secundum repraesentationem aliquam ex impressione divina in
creatura existentem, non autem ex eo quod creatura irrationalis, in qua
est sola talis similitudo possit ad Deum attingere per solam suam
operationem. 


&

第二異論に対しては、以下のように言われるべきである。
似姿の類似が人間本性において見出されるのは、それが神を受け入れうる点で、
 すなわち、自分に固有の認識と愛の働きによって、神に触れることで、そうし
 うる限りにおいてである。これに対して、痕跡の類似は、存在する被造物にお
 ける神の何らかの刻印によって、何らかの表現に即してのみ見出されるのであ
 り、そのような類似だけが見出される非理性的被造物が、自らの働きだけによっ
 て、神に触れることができるからではない。


\\

Quod autem deficit a minori, non habet congruitatem ad id
quod est maius, sicut corpus quod non est aptum perfici anima sensitiva,
multo minus est aptum perfici anima intellectiva. Multo autem est maior
et perfectior unio ad Deum secundum esse personale quam quae est
secundum operationem. Et ideo creatura irrationalis, quae deficit ab
unione ad Deum per operationem, non habet congruitatem ut uniatur ei
secundum esse personale.


&

ところで、より小さいものにも足りないものは、より大きいものへの適合性を持
 たない。たとえば、感覚的魂によって完成される適性を持たない物体は、知性
 的魂によって完成される適性を、なおさら持たない。しかし、ペルソナ的存在
 において神へ合一することは、働きに即してそうすることよりも、はるかに大
 きく完全な合一である。ゆえに、非理性的被造物は、働きによる神への合一に
 も足りないので、ペルソナ的存在における神への合一への適性を持たない。

\\



{\scshape Ad tertium dicendum} quod quidam dicunt Angelum non esse assumptibilem,
quia a principio suae creationis est in sua personalitate perfectus, cum
non subiaceat generationi et corruptioni. Unde non potuisset in unitatem
divinae personae assumi nisi eius personalitas destrueretur, quod neque
convenit incorruptibilitati naturae eius; neque bonitati assumentis, ad
quam non pertinet quod aliquid perfectionis in creatura assumpta
corrumpat. 



&

第三異論に対しては、以下のように言われるべきである。
ある人々は次のように言う。天使は生成消滅しないため、それの創造の最初から、それのペルソ
 ナ性において完全なので、受容可能でない。したがって、天使のペル
 ソナ性が破壊されないかぎり、神のペルソナとの合一へ受容されえなかったが、
 この破壊は、天使の本性の不滅性と適合せず、また、受容された被造物におい
 て何らかの完全性を破壊するのは、受容する者の善性にも適合しない、と。

\\

Sed hoc non videtur totaliter excludere congruitatem
assumptionis angelicae naturae. Potest enim Deus producendo novam
angelicam naturam, copulare eam sibi in unitate personae, et sic nihil
praeexistens ibi corrumperetur. Sed, sicut dictum est, deest congruitas
ex parte necessitatis, quia, etsi natura angelica in aliquibus peccato
subiaceat, est tamen eius peccatum irremediabile ut in prima parte
habitum est.


&


しかしこのことは、天使的本性の受容の適性を、全面的に排除するとは思われな
 い。というのも、神は新しい天使的本性を生み出すことによって、それを自ら
 にペルソナの一性へと結びつけ、そうして、そこで前に存在していた何ものも
 消滅しないようにできるからである。そうではなく、すでに述べられたとおり、
 天使には、必要性の側からの適性がないのである。仮に天使的本性がある事柄
 において罪のもとにあるとしても、第一部で述べられたとおり\footnote{Q.64,
 a.2.} 、その罪は償う
 ことが不可能である。

\\



{\scshape Ad quartum dicendum} quod perfectio universi non est perfectio unius
personae vel suppositi, sed eius quod est unum sub positione vel
ordine. Cuius plurimae partes non sunt assumptibiles, ut dictum
est. Unde relinquitur quod solum natura humana sit assumptibilis.


&

第四異論に対しては、以下のように言われるべきである。
宇宙の完全性は、一つのペルソナや個体の完全性ではなく、位置ないし秩序のも
 とで一つであるものの完全性である。すでに述べられたとおり、それらの極め
 て多くの部分は受容不可能である。したがって、人間本性だけが受容可能であ
 ることが残される。

\end{longtable}
\newpage



\rhead{a.~2}
\begin{center}
 {\Large {\bf ARTICULUS SECUNDUS}}\\
 {\large UTRUM FILIUS DEI ASSUMPSERIT PERSONAM}\\
 {\footnotesize III {\itshape Sent.}, d.5, q.3, a.3.}\\
 {\Large 第二項\\神の息子はペルソナを受容したか}
\end{center}

\begin{longtable}{p{21em}p{21em}}

{\Huge A}{\scshape d secundum sic proceditur}. Videtur quod Filius Dei assumpserit
personam. Dicit enim Damascenus, in III libro, quod Filius Dei {\itshape assumpsit
humanam naturam in atomo}, idest, in individuo. Sed {\itshape individuum rationalis
naturae} est persona, ut patet per Boetium, in libro {\itshape de Dduabus
Naturis}. Ergo Filius Dei personam assumpsit.


&

第二項の問題へ、議論は以下のように進められる。
神の息子はペルソナを受容したと思われる。理由は以下の通り。
ダマスケヌスは第三巻で、神の息子は「人間本性をアトムにおいて」つまり個体
 において「受容した」と述べている。しかし、『二つの本性に
 ついて』のボエティウスによって明らかなとおり、「理性的本性を持つ個体」
 はペルソナである。ゆえに、神の息子はペルソナを受容した。


\\




2. {\scshape Praeterea}, Damascenus dicit quod Filius Dei {\itshape assumpsit ea quae in natura
nostra plantavit}. Plantavit autem ibi personalitatem. Ergo Filius Dei
assumpsit personam.


&

さらに、ダマスケヌスは、神の息子が「私たちの本性に植えたものを受容した」
 と言う。しかし、ここで、植えたのはペルソナ性である。ゆえに、神の息子は、
 ペルソナを受容した。


\\




3. {\scshape Praeterea}, nihil consumitur nisi quod est. Sed Innocentius III dicit, in
quadam Decretali, quod {\itshape persona Dei consumpsit personam hominis}. Ergo
videtur quod persona hominis fuit prius assumpta.


&

さらに、存在するものでなければ、何も消費されない。しかし、イノケンティウ
 ス3世は、ある教皇庁令で、「神のペルソナが人間のペルソナを消費した」と言
 う。ゆえに、人間のペルソナが、最初に受容されたと思われる。

\\




{\scshape Sed contra est} quod Augustinus dicit, in libro {\itshape de Fide ad Petrum}, quod
{\itshape Deus naturam hominis assumpsit, non personam}.


&

しかし反対に、アウグスティヌスは、『ペトルス宛書簡:信仰について』で「神
 は人間のペルソナではなく本性を受容した」と述べている。

\\




{\scshape Respondeo dicendum} quod aliquid dicitur assumi ex eo quod ad aliquid
sumitur. Unde illud quod assumitur oportet praeintelligi assumptioni,
sicut id quod movetur localiter praeintelligitur ipsi motui. Persona
autem non praeintelligitur in humana natura assumptioni, sed magis se
habet ut terminus assumptionis, ut supra dictum est. 

&

解答する。以下のように言われるべきである。
あるものが受容されるということは、あるものへと採られることから言われる。
したがって、受容に先立って、受容されるものが理解されるべきである。ちょう
 ど、場所的に動かされるものが、運動自体に先立って理解されるように。
さて、ペルソナは、人間本性において、受容に先立って理解されず、むしろ、前
 に述べられたとおり、受容の終極として関係する。


\\



Si enim
praeintelligeretur, vel oporteret quod corrumperetur, et sic frustra
esset assumpta. Vel quod remaneret post unionem, et sic essent duae
personae, una assumens et alia assumpta; quod est erroneum, ut supra
ostensum est. Unde relinquitur quod nullo modo Filius Dei assumpsit
humanam personam.


&

なぜなら、もしも先行して理解されるならば、それは消滅したか、合一の後も残っ
 たかのどちらかだが、消滅したならば、それを受容することは無駄だっただろ
 うし、残ったならば、受容するものと受容されるものとして二つのペルソナが
 あったことになるが、これは、前に示されたとおり、誤りである。したがって、
 神の息子は決して人間のペルソナを受容しなかったことが帰結する。


\\




{\scshape Ad primum ergo dicendum} quod naturam humanam assumpsit Filius Dei {\itshape in
atomo}, idest, {\itshape in individuo quod non est aliud a supposito increato quod
est persona Filii Dei}. Unde non sequitur quod persona sit assumpta.


&

第一異論に対しては、それゆえ、以下のように言われるべきである。
人間本性を、神の息子が受容したのが「アトムにおいて」と言われるのは、「神の息子
 のペルソナである、創造されざる個体(suppositum)と別でない個体
 (individuum)において」という意味である。ゆえに、ペルソナが受容されたこ
 とにはならない。

\\




{\scshape Ad secundum dicendum} quod naturae assumptae non deest propria
personalitas propter defectum alicuius quod ad perfectionem humanae
naturae pertineat, sed propter additionem alicuius quod est supra
humanam naturam, quod est unio ad divinam personam.


&

第二異論に対しては、以下のように言われるべきである。
受容された本性に固有のペルソナ性がないのは、人間本性に属する何らかの欠陥
 のためではなく、人間本性の上にある何らかの付加、つまり、神のペルソナへ
 の合一のためにである。


\\




{\scshape Ad tertium dicendum} quod {\itshape consumptio} ibi non importat destructionem
alicuius quod prius fuerat, sed impeditionem eius quod aliter esse
posset. Si enim humana natura non esset assumpta a divina persona,
natura humana propriam personalitatem haberet. Et pro tanto dicitur
persona consumpsisse personam, licet improprie, quia persona divina sua
unione impedivit ne humana natura propriam personalitatem haberet.


&

第三異論に対しては、以下のように言われるべきである。
ここでの「消費」というのは、前に存在していたものの破壊を意味せず、他のし
 かたでありえたものの阻止を意味する。つまり、もし人間本性が神のペルソナ
 によって受容されなかったならば、人間本性は、固有のペルソナ性をもってい
 たであろう。そして、神のペルソナが、自らの合一によって、人間本性が固有
 のペルソナ性を持つことを阻止するため、厳密な言葉の使用ではないが、ペル
 ソナがペルソナを消費したと言われる。



\end{longtable}
\newpage








\rhead{a.~3}
\begin{center}
 {\Large {\bf ARTICULUS TERTIUS}}\\
 {\large UTRUM PERSONA DIVINA ASSUMPSERIT HOMINEM}\\
 {\footnotesize Supra, q.2, a.6; III {\itshape Sent.}, d.6, q.1, a.2;
 {\itshape Cont.~Error.~Graec.}, cap.20; {\itshape Compend.~Theol.},
 cap.210; {\itshape Ad Rom.}, cap.1, lect.3; {\itshape Ad Philipp.},
 cap.2, lect.2.}\\
 {\Large 第三項\\神のペルソナは人間を受容したか}
\end{center}

\begin{longtable}{p{21em}p{21em}}


{\Huge A}{\scshape d tertium sic proceditur}. Videtur quod persona divina assumpserit
hominem. Dicitur enim in Psalmo, {\itshape beatus quem elegisti et assumpsisti},
quod Glossa exponit de Christo. Et Augustinus dicit, in libro {\itshape de agone
Christiano}, {\itshape Filius Dei hominem assumpsit, et in illo humana perpessus
est}.


&

第三項の問題へ、議論は以下のように進められる。
神のペルソナは人間を受容したと思われる。理由は以下の通り。
『詩編』で「あなたが選び受容した人は幸いだ」\footnote{「いかに幸いなことでしょう/あなたに選ばれ、近づけられ/あなたの庭に宿る人は。恵みの溢れるあなたの家、聖なる神殿によって/わたしたちが満ち足りますように。」(65:5)}と言われていて、『注解』は、
 それをキリストについて説明している。また、アウグスティヌスは、『キリス
 ト教徒の戦い』で「神の息子は人間を受容し、そこにおいて人間本性が保たれ
 ている」と言う。

\\



2. {\scshape Praeterea}, hoc nomen {\itshape homo} significat naturam humanam. Sed Filius Dei
assumpsit humanam naturam. Ergo assumpsit hominem.


&


さらに、「人間」というこの名は、人間本性を意味する。しかし、神の息子は、
 人間本性を受容した。ゆえに、人間を受容した。

\\



3. {\scshape Praeterea}, Filius Dei est homo. Sed non est homo quem non assumpsit quia
sic esset pari ratione Petrus, vel quilibet alius homo. Ergo est homo
quem assumpsit.


&

さらに、神の息子は人間である。しかし、彼が受容しなかった人間ではない。な
 ぜなら、もし受容しなかった人間であるならば、同じ理由で、神の息子は、ペトロや、他のどの人間でもあっただろ
 うから。ゆえに、神の息子は、彼が受容した人間である。

\\



{\scshape Sed contra est} auctoritas Felicis Papae et martyris, quae introducitur
in Ephesina synodo, {\itshape credimus in dominum nostrum Iesum Christum, de
virgine Maria natum quia ipse est Dei sempiternus Filius et Verbum, et
non homo a Deo assumptus, ut alter sit praeter illum. Neque enim hominem
assumpsit Dei Filius ut alter sit praeter ipsum}.


&

しかし反対に、教皇であり殉教者であるフェリックスの権威は、エフェソ公会議
 に導入されたものだが、以下のように述べている。「私たちは、処女マリアか
 ら生まれた私たちの主イエス・キリストを信じる。なぜなら、彼は神の永遠の
 息子であり、言葉であり、彼以外でありえるような神によって受容された人間
 ではない。なぜなら、神の息子が人間を受容したのは、彼以外でありえ
 るものとしてではないからである」。


\\



{\scshape Respondeo dicendum} quod, sicut supra dictum est, id quod assumitur non
est terminus assumptionis, sed assumptioni praeintelligitur. Dictum est
autem quod individuum in quo assumitur natura humana, non est aliud quam
divina persona, quae est terminus assumptionis. 

&

解答する。以下のように言われるべきである。
前に述べられたとおり、受容されるものは、受容の終極でなく、受容に先立って
 理解される。ところで、すでに言われたことだが、そこにおいて人間本性が受容される個体は、受容の終
 極である、神のペルソナに他ならない。

\\


Hoc autem nomen {\itshape homo}
significat humanam naturam prout est nata in supposito esse, quia, ut
dicit Damascenus, sicut hoc nomen {\itshape Deus} significat {\itshape eum qui habet divinam
naturam}, ita hoc nomen {\itshape homo} significat {\itshape eum qui habet humanam naturam}. Et
ideo non est proprie dictum quod Filius Dei assumpsit hominem,
supponendo, sicut rei veritas se habet, quod in Christo sit unum
suppositum et una hypostasis. 



&

また、「人間」というこの名は、個体において存在するように生まれついている
 ものとしての人間本性を表示する。それは、ダマスケヌスが言うとおり、「神」というこの
 名が、「神の本性を持つもの」を意味するように、「人間」というこの名は、
 「人間本性を持つもの」を意味する。
それゆえ、事実として、キリストにおいて、一つの個体と一つのヒュポスタシス
 があることが想定されるのだから、厳密には「神の息子が人間を受容した」と
 は言われない。


\\

Sed secundum illos qui ponunt in Christo
duas hypostases vel duo supposita, convenienter et proprie dici posset
quod Filius Dei hominem assumpsisset. Unde et prima opinio quae ponitur
sexta distinctione tertii libri sententiarum concedit hominem esse
assumptum. Sed illa opinio erronea est, ut supra ostensum est.


&


しかし、キリストの中に二つのヒュポスタシスや二つの個体を措定する人々によ
 れば、神の息子が人間を受容したということが、適切に、厳密な意味で言われ
 えたであろう。このことから、『命題集』第三巻第6区分の第一の意見は、人間
 が受容されることを認めている。しかし、この意見は、前に示されたとおり、
 誤りである。


\\



{\scshape Ad primum ergo dicendum} quod huiusmodi locutiones non sunt extendendae,
tanquam propriae, sed pie sunt exponendae, ubicumque a sacris doctoribus
ponuntur; ut dicamus {\itshape hominem assumptum}, quia eius natura est assumpta;
et quia assumptio terminata est ad hoc quod Filius Dei sit homo.


&

第一異論に対しては、それゆえ、以下のように言われるべきである。
このような語り方は、それが聖なる学者たちによって言われる場合にはどこでも、
 厳密なものとして拡張されるべきでなく、むしろ、敬虔に説明されるべきであ
 る。「受容された人間」と私たちが言うのは、その本性が受容されたからであ
 り、受容が、神の息子が人間であるという事へ終極するからである、というよ
 うに。

\\



{\scshape Ad secundum dicendum} quod hoc nomen {\itshape homo} significat naturam humanam in
concreto, prout scilicet est in aliquo supposito. Et ideo, sicut non
possumus dicere quod suppositum sit assumptum, ita non possumus dicere
quod homo sit assumptus.


&

第二異論に対しては、以下のように言われるべきである。
「人間」というこの名は、ある個体においてあるものとして、人間本性を具体的
 に表示する。ゆえに、私たちは、個体が受容されたと言うことができないように、人間が
 受容されたと言うこともまたできない。


\\



{\scshape Ad tertium dicendum} quod Filius Dei non est homo quem assumpsit; sed
cuius naturam assumpsit.


&

第三異論に対しては、以下のように言われるべきである。
神の息子は、受容した人間ではない。そうではなく、その本性を受容した。



\end{longtable}
\newpage





\rhead{a.~4}
\begin{center}
 {\Large {\bf ARTICULUS QUARTUS}}\\
 {\large UTRUM FILIUS DEI DEBUERIT ASSUMERE NATURAM HUMANAM\\ABSTRACTAM
 AB OMNIBUS INDIVIDUIS}\\
 {\footnotesize Supra, q.2, a.2, ad 3; a.5, ad 2; III {\itshape Sent.},
 d.2, Expos.~litt.}\\
 {\Large 第四項\\神の息子は、全ての個人から抽象された\\人間本性を受容しなけ
 ればならなかったか}
\end{center}

\begin{longtable}{p{21em}p{21em}}

{\Huge A}{\scshape d quartum sic proceditur}. Videtur quod Filius Dei debuit assumere
naturam humanam abstractam ab omnibus individuis. Assumptio enim naturae
humanae facta est ad communem omnium hominum salutem, unde dicitur I
{\itshape Tim}.~{\scshape iv}, de Christo, quod {\itshape est salvator omnium hominum, maxime
fidelium}. Sed natura prout est in individuis, recedit a sua
communitate. Ergo Filius Dei debuit humanam naturam assumere prout est
ab omnibus individuis abstracta.


&


第一項の問題へ、議論は以下のように進められる。
神の息子は、全ての個人から抽象された人間本性を受容しなければならなかった
 と思われる。理由は以下の通り。
人間本性の受容は、全ての人間の共通な救済のために行われた。このことから、
 『テモテへの手紙一』第4章で、キリストについて、「全ての人間の、と
 くに信心深い人々の救済者」\footnote{「わたしたちが労苦し、奮闘するのは、
 すべての人、特に信じる人々の救い主である生ける神に希望を置いているから
 です。」(4:10)}と言われている。しかし、個人の中にあるものとしての本性は、
 その共通性において後退する。ゆえに、神の息子は、全ての個人から抽象され
 た人間本性を受容しなければならなかった。

\\



2. {\scshape Praeterea}, in omnibus quod nobilissimum est Deo est attribuendum. Sed in
unoquoque genere id quod est per se potissimum est. Ergo Filius Dei
debuit assumere {\itshape per se hominem}. Quod quidem, secundum Platonicos, est
humana natura ab individuis separata. Hanc ergo debuit Filius Dei
assumere.


&


さらに、あらゆるものにおいて、もっとも高貴なものは、神へ帰せられるべきで
 ある。しかし、各々の類において、自体的に存在するものは、もっとも力があ
 る。ゆえに、神の息子は、「自体的な人間」を受容すべきだった。これは、プ
 ラトン派の人々によれば、個人から分離した人間本性である。ゆえに、これを、
 神の息子は受容すべきだった。

\\



3. {\scshape Praeterea}, natura humana non est assumpta a Filio Dei prout significatur
in concreto per hoc nomen {\itshape homo}, ut dictum est. Sic autem significatur
prout est in singularibus, ut ex dictis patet. Ergo Filius Dei assumpsit
humanam naturam prout est ab individuis separata.


&

さらに、すでに述べられたとおり、人間本性は、「人間」というこの名によって具体的に表示されるものと
 して、神の息子に受容されたのではない。この意味では、すでに述べられたこ
 とから明らかなとおり、それは個人において
 存在するものとして表示される。ゆえに、神の息子は、個人から分離したもの
 としての本性を受容した。

\\



{\scshape Sed contra est} quod dicit Damascenus, in III libro, {\itshape Dei Verbum
incarnatum neque eam quae nuda contemplatione consideratur naturam
assumpsit. Non enim incarnatio hoc, sed deceptio, et fictio
incarnationis}. Sed natura humana prout est a singularibus separata vel
abstracta, {\itshape in nuda contemplatione cogitatur, quia secundum seipsam non
subsistit}, ut idem Damascenus dicit. Ergo Filius Dei non assumpsit
humanam naturam secundum quod est a singularibus separata.


&

しかし反対に、ダマスケヌスは第三巻で次のように述べている。「受肉した神の
 言葉は、裸の観照によって考察される本性を受容したのではない。なぜなら、
 それは受肉ではなく、受肉の欺きないし創作だから」。しかし、個人から分離
 され抽象されたものとしての人間本性は、ダマスケヌスが同じ箇所で言うよう
 に、「裸の観照において思惟されたも
 のである。なぜなら、それ自体において自存しないから」。ゆえに、神の息子
 は、個人から分離したかぎりでの人間本性を受容したのではない。


\\



{\scshape Respondeo dicendum} quod natura hominis, vel cuiuscumque alterius rei
sensibilis, praeter esse quod in singularibus habet, dupliciter potest
intelligi, uno modo, quasi per seipsam esse habeat praeter materiam,
sicut Platonici posuerunt; alio modo, sicut in intellectu existens, vel
humano vel divino. 

&

解答する。以下のように言われるべきである。
人間の、あるいは何であれ他の可感的事物の本性は、個物において持つ存在を離
 れて、二通りに理解されうる。一つには、質料を離れてそれ自身によって
 存在を持つものとしてであり、これはプラトン派の人々が考えたようにである。
 もう一つには、人間であれ神であれ、知性において存在するものとしてである。

\\

Per se quidem subsistere non potest, ut philosophus
probat, in VII {\itshape Metaphys}., quia ad naturam speciei rerum sensibilium
pertinet materia sensibilis, quae ponitur in eius definitione; sicut
carnes et ossa in definitione hominis. 
Unde non potest esse quod natura
humana sit praeter materiam sensibilem.
&

しかし、哲学者が『形而上学』第7巻で証明するように、そのような可感的事物の本性は、それ自身によって自存することができな
 い。なぜなら、可感的事物の種の本性には、可感的質料が属し、それの
 定義の中に置かれるからである。たとえば、人間の定義の中に、肉と骨が置か
 れるように。それゆえ、人間本性が、可感的質料を離れて存在することはでき
 ない。

\\

 Si tamen esset hoc modo
subsistens natura humana, non fuisset conveniens ut a Verbo Dei
assumeretur. 


&

しかし、仮に、人間本性がこのように自存するものだったとしても、神の言葉に
 よって受容されることは、適切でなかった。

\\

Primo quidem, quia assumptio ista terminatur ad
personam. Hoc autem est contra rationem formae communis, ut sic in
persona individuetur. 


&

第一に、この受容はペルソナに終極する。しかし、このようにペルソナにおいて
 個体化されることは、共通な形相の性
 格に反する。


\\

Secundo, quia naturae communi non possunt attribui
nisi operationes communes et universales, secundum quas homo nec meretur
nec demeretur, cum tamen illa assumptio ad hoc facta sit ut Filius Dei
in natura assumpta nobis mereretur. 

&

第二に、共通な本性には、共通で普遍的な働きしか帰せられないが、それらの働
 きに応じて、人間が功績を積んだり積まなかったりはしない。しかし、この
 受容は、神の息子が、受容された本性において、私たちのために功績を積むた
 めになされた。


\\

Tertio quia natura sic existens non
est sensibilis, sed intelligibilis. Filius autem Dei assumpsit humanam
naturam ut hominibus in ea visibilis appareret, secundum illud Baruch
III, post haec in terris visus est, et cum hominibus conversatus
est. 

&

第三に、このように存在する本性は、可感的でなく可知的なものである。しかし、
『バルク書』第3章「この後、地上に見られ、人々と共に住んだ」
 \footnote{「その後、知恵は地上に現れ、人々の中に住んだ。」(3:38)}によれ
 ば、神の息子は人間本性を受容したが、その結果、その本性において、人々に
 見られうるものとして現れた。

\\

Similiter etiam non potuit assumi natura humana a Filio Dei
secundum quod est in intellectu divino. Quia sic nihil aliud esset quam
natura divina, et per hunc modum, ab aeterno esset in Filio Dei humana
natura. 


&

同様に、人間本性が、神の知性の中にあるかぎりで、神の息子によって受容され
 ることもできなかった。なぜなら、もしそうならば、それは神の本性以外で
 なかっただろうから。そしてこのしかたによれば、永遠から神の息子において
 人間本性があったことになる。


\\

Similiter non convenit dicere quod Filius Dei assumpserit
humanam naturam prout est in intellectu humano. Quia hoc nihil aliud
esset quam si intelligeretur assumere naturam humanam. Et sic, si non
assumeret eam in rerum natura, esset intellectus falsus. Nec aliud esset
quam {\itshape fictio quaedam incarnationis}, ut Damascenus dicit.


&

同様に、神の息子が人間本性を、人間の知性の中にあるものとして受容したとい
 うのも適切でない。なぜなら、これは、人間本性を受容することが知性認
 識されることと異ならなかったであろう。そしてもし、現実には受容しな
 かったとしたら、その知性認識は偽だったであろうし、ダマスケヌスが言うよ
 うに、「受肉の一種の創作」に他ならなかったであろう。


\\



{\scshape Ad primum ergo dicendum} quod Filius Dei incarnatus est communis omnium
salvator, non communitate generis vel speciei, quae attribuitur naturae
ab individuis separatae, sed communitate causae, prout Filius Dei
incarnatus est universalis causa salutis humanae.


&

第一異論に対しては、それゆえ、以下のように言われるべきである。
受肉した神の息子は、万人の共通の救済者だが、個人から分離された本性に帰せ
 られる類や種の共通性においてではなく、受肉した神の息子が人間の救済の普
 遍的原因だという意味で、原因の共通性においてである。


\\



{\scshape Ad secundum dicendum} quod {\itshape per se homo} non invenitur in rerum natura ita
quod sit praeter singularia, sicut Platonici posuerunt. Quamvis quidam
dicant quod Plato non intellexit hominem separatum esse nisi in
intellectu divino. Et sic non oportuit quod assumeretur a Verbo, cum ab
aeterno sibi affuerit.


&

第二異論に対しては、以下のように言われるべきである。
「自体的な人間」は、現実には見出されず、したがって、プラトン派の人々が考
 えたように、個を離れて存在することはない。ただし、ある人々は、プラトン
 は、ただ神の知性の中にあるものとしてのみ、分離した人間を考えた、と言う。
 そしてこの意味では、それは言葉によって受容される必要はない。なぜなら、
 永遠からそこに在ったのだから。



\\



{\scshape Ad tertium dicendum} quod natura humana, quamvis non sit assumpta in
concreto ut suppositum praeintelligatur assumptioni, sic tamen assumpta
est in individuo, quia assumpta est ut sit in individuo.


&

第三異論に対しては、以下のように言われるべきである。
人間本性は、たしかに、受容に先立って個体が理解されているようなかたちで、具体的なものにおいて
 受容されたわけではない。しかし、受容された結果、個人において存在するようなかたちで、個人において受容された。

\\



\end{longtable}
\newpage




\rhead{a.~5}
\begin{center}
 {\Large {\bf ARTICULUS QUINTUS}}\\
 {\large UTRUM FILIUS DEI HUMANAM NATURAM ASSUMERE DEBUIT\\ IN OMNIBUS IDIVIDUIS}\\
 {\footnotesize III {\itshape Sent.}, d.2, q.1, a.2, qu$^a$ 1.}\\
 {\Large 第五項\\神の息子は、人間本性を、\\全ての個人において受容しなけれ
 ばならなかったか}
\end{center}

\begin{longtable}{p{21em}p{21em}}

{\Huge A}{\scshape d quintum sic proceditur}. Videtur quod Filius Dei humanam naturam
assumere debuit in omnibus individuis. Id enim quod primo et per se
assumptum est, est natura humana. Sed quod convenit per se alicui
naturae, convenit omnibus in eadem natura existentibus. Ergo conveniens
fuit ut natura humana assumeretur a Dei Verbo in omnibus suppositis.


&

第五項の問題へ、議論は以下のように進められる。
神の息子は、人間本性を、全ての個人において受容しなければならなかった。
理由は以下の通り。
第一に、自体的に受容されたのは、人間本性である。しかし、何らかの本性に自
 体的に適合するものは、その同一の本性において存在する全てのものに適合す
 る。ゆえに、人間本性が、すべての個体において神の言葉によって受容される
 ことが適切であった。

\\



2. {\scshape Praeterea}, incarnatio divina processit ex divina caritate, ideo dicitur
Ioan.~{\scshape iii}, {\itshape sic Deus dilexit mundum ut Filium suum unigenitum daret}. Sed
caritas facit ut aliquis se communicet amicis quantum possibile
est. Possibile autem fuit Filio Dei ut plures naturas hominum assumeret,
ut supra dictum est, et, eadem ratione, omnes. Ergo conveniens fuit ut
Filius Dei assumeret naturam in omnibus suis suppositis.


&


さらに、神の受肉は神の愛から出てきた。それゆえ、『ヨハネによる福音書』第
 3章で、「神は、自分の一人息子を与えるほどに世界を愛した」\footnote{「神
 は、その独り子をお与えになったほどに、世を愛された。独り子を信じる者が
 一人も滅びないで、永遠の命を得るためである。」(3:16)}と言われている。
しかし、愛は、できる限り、愛するものに自分を伝達するようにさせる。しかし、
 前に述べられたとおり、神の息子が、人間の複数の本性を受容することが可能
 だった。同様に、全ての人間の本性を受容することも可能だった。ゆえに、神
 の息子が、そのすべての個体において、本性を受容することが適切だった。

\\



3. {\scshape Praeterea}, sapiens operator perficit opus suum breviori via qua
potest. Sed brevior via fuisset si omnes homines assumpti fuissent ad
naturalem filiationem, quam quod per unum Filium naturalem {\itshape multi in
adoptionem filiorum adducantur}, ut dicitur {\itshape Galat.}~{\scshape iv}. Ergo natura humana
debuit a Filio Dei assumi in omnibus suppositis.


&


さらに、知恵は、自分の仕事を、最短の道で完成させる。しかし、『ガラテヤの
 信徒への手紙』第4章で「多くの人が、息子たちを養子にすることへと導かれる」
 \footnote{「それは、律法の支配下にある者を贖い出して、わたしたちを神の
 子となさるためでした。」(4:5)}と言われるように、もし全ての人間が、自然
 の息子性へと受容されたならば、それは、一人の自然の息子によるよりも、短
 い道程だっただろう。ゆえに、人間本性は、すべての個体において、神の息子
 によって受容されるべきであった。

\\



{\scshape Sed contra est} quod Damascenus dicit, in III libro, quod {\itshape Filius Dei non
assumpsit humanam naturam quae in specie consideratur, neque enim omnes
hypostases eius assumpsit}.


&

しかし反対に、ダマスケヌスは第3巻で以下のように述べている。「神の息子は、
 種において考察される人間本性を受容したのでも、その全てのヒュポスタシス
 を受容したのでもない」。

\\



{\scshape Respondeo dicendum} quod non fuit conveniens quod humana natura in
omnibus suis suppositis a Verbo assumeretur. 


&

解答する。以下のように言われるべきである。
人間本性が、その全ての個体において、言葉によって受容されることは適切でな
 かった。

\\

Primo quidem, quia
tolleretur multitudo suppositorum humanae naturae, quae est ei
connaturalis. 
Cum enim in natura assumpta non sit considerare aliud
suppositum praeter personam assumentem, ut supra dictum est; si non
esset natura humana nisi assumpta, sequeretur quod non esset nisi unum
suppositum humanae naturae, quod est persona assumens. 

&

第一に、人間本性にとって共本性的である、個体の多数性ということがなくなる
 からである。
というのも、前に述べられたとおり、受容された本性において、受容するペルソナ以外に、個体を考える
ことはできないので、もし、人間本性が、受容されたもの以外になかったならば、
 人間本性の個体は一つ、つまり受容するペルソナしかなかっただろうからで
 ある。


\\

Secundo, quia hoc
derogaret dignitati Filii Dei incarnati, prout est primogenitus in
multis fratribus secundum humanam naturam, sicut est primogenitus omnis
creaturae secundum divinam. Essent enim tunc omnes homines aequalis
dignitatis. 

&

第二に、このことは、神の息子の偉大さを損なうであろう。つまりその場合には、
 神の息子は、人間本性において多くの兄弟たちの中で、最初に生まれたもの、
 あるいは、神の本性において全ての被造物のなかの最初に生まれたもの、とい
 うことになるからである。そして、その場合、全ての人間は、等しい偉大さを
 もったであろう。


\\

Tertio, quia conveniens fuit quod, sicut unum suppositum
divinum est incarnatum, ita unam solam naturam humanam assumeret, ut ex
utraque parte unitas inveniatur.


&

第三に、一つの神の個体が受肉するように、ただ一つの人間本性を受容すること
 が、両方の側から一性が見出されることとして、適切であった。

\\



{\scshape Ad primum ergo dicendum} quod assumi convenit secundum se humanae
naturae, quia scilicet non convenit ei ratione personae, sicut naturae
divinae convenit assumere ratione personae. Non autem quia convenit ei
secundum se sicut pertinens ad principia essentialia eius, vel sicut
naturalis eius proprietas, per quem modum conveniret omnibus eius
suppositis.


&

第一異論に対しては、それゆえ、以下のように言われるべきである。
受容されることは、それ自体において、人間本性に適合する。なぜなら、ペルソ
 ナの性格において適合しないからである。対照的に、
神の本性には、ペルソナの性格において受容することが適合するのだが。
しかし、それ自体においてと言っても、その本質的諸根源に属するものとしてで
 はないし、また、それの本性的な固有性としてでもない。そのようなしかたで
 適合するのであれば、そのすべての個体に適合したであろう。


\\



{\scshape Ad secundum dicendum} quod dilectio Dei ad homines manifestatur non solum
in ipsa assumptione humanae naturae, sed praecipue per ea quae passus
est in natura humana pro aliis hominibus, secundum illud {\itshape Rom}.~{\scshape v},
{\itshape commendat autem Deus suam caritatem in nobis, quia, cum inimici essemus,
Christus pro nobis mortuus est}. Quod locum non haberet si in omnibus
hominibus naturam humanam assumpsisset.


&

第二異論に対しては、以下のように言われるべきである。
『ローマの信徒への手紙』第5章「神は自らの愛を私たちのなかで勧める
 が、それは、私たちが不和だったので、キリストが私たちのために死んだから
 である」\footnote{「しかし、わたしたちがまだ罪人であったとき、キリスト
 がわたしたちのために死んでくださったことにより、神はわたしたちに対する
 愛を示されました。」(5:8)}によれば、神の人間への愛は、人間本性の受容自体において明示されるだけでなく、とくに、
 人間本性において、他のすべての人間のために受難した事柄によって、明示さ
 れる。このことは、もし、全ての人間において、人間本性を受容したならば、
 ありえなかったであろう。



\\



{\scshape Ad tertium dicendum} quod ad brevitatem viae quam sapiens operator
observat, pertinet quod non faciat per multa quod sufficienter potest
fieri per unum. Et ideo convenientissimum fuit quod per unum hominem
alii omnes salvarentur.


&

第三異論に対しては、以下のように言われるべきである。
知恵ある働き手が守る、道程の短さは、一つのものによって十分になされうるこ
 とを、多くのものによって行わないことに属する。ゆえに、一人の人間によって
 他のすべての人間が救われることは、この上なく適したことであった。

\end{longtable}
\newpage




\rhead{a.~6}
\begin{center}
 {\Large {\bf ARTICULUS SEXTUS}}\\
 {\large UTRUM FUERIT CONVENIENS UT FILIUS DEI \\HUMANAM NATURAM ASSUMERET
 EX STIRPE ADAE}\\
 {\footnotesize Infra, q.31, a.1; III {\itshape Sent.}, d.2, q.1, a.2,
 qu$^a$ 2, 3; {\itshape Compend.~Theol.}, cap.217.}\\
 {\Large 第六項\\神の息子が人間本性を\\アダムの家系から受容したのは適切だっ
 たか}
\end{center}

\begin{longtable}{p{21em}p{21em}}

{\Huge A}{\scshape d sextum sic proceditur}. Videtur quod non fuerit conveniens ut Filius
Dei humanam naturam assumeret ex stirpe Adae. Dicit enim apostolus, {\itshape ad
Heb}.~{\scshape vii}, {\itshape decebat ut esset nobis pontifex segregatus a peccatoribus}. Sed
magis esset a peccatoribus segregatus si non assumpsisset humanam
naturam ex stirpe Adae peccatoris. Ergo videtur quod non debuit de
stirpe Adae naturam humanam assumere.


&

第六項の問題へ、議論は以下のように進められる。
神の息子が人間本性をアダムの家系から受容したのは適切でなかったと思われる。
 理由は以下の通り。
使徒は『ヘブライ人への手紙』第7章で、「大祭司が罪人たちから引き離されたの
 は私たちにとってよかった」\footnote{「このように聖であり、罪なく、汚れ
 なく、罪人から離され、もろもろの天よりも高くされている大祭司こそ、わた
 したちにとって必要な方なのです。」(7:26)}と述べている。しかし、もし、罪
 人アダムの家系から人間本性を受容しなかったならば、より罪人たちから離さ
 れていたであろう。ゆえに、アダムの家系から人間本性を受容すべきでなかっ
 たと思われる。

\\



2. {\scshape Praeterea}, in quolibet genere nobilius est principium eo quod est ex
principio. Si igitur assumere voluit humanam naturam, magis debuit eam
assumere in ipso Adam.


&

さらに、どの類においても、根源は、根源から出てくるものよりも高貴である。
 ゆえに、人間本性を受容することを意志したならば、それをむしろアダムにおいて受
 容すべきであった。

\\



3. {\scshape Praeterea}, gentiles fuerunt magis peccatores quam Iudaei, ut dicit
Glossa, {\itshape Galat}.~{\scshape ii}, super illud, {\itshape nos natura Iudaei, non ex gentibus
peccatores}. Si ergo ex peccatoribus naturam humanam assumere voluit,
debuit eam magis assumere ex gentilibus quam ex stirpe Abrahae, qui fuit
iustus.


&

さらに、『注釈』が、かの『ガラテヤの信徒への手紙』第2章「私たちは生まれ
 つきユダヤ人であり、異邦からの罪人ではない」\footnote{「わたしたちは生ま
 れながらのユダヤ人であって、異邦人のような罪人ではありません。」
 (2:15)}について述べるように、異教徒たちはユダヤ人たちよりもより罪人であっ
 た。ゆえに、もし罪人たちから人間本性を受容することを意志したならば、義
 人であったアブラハムの家系からではなく、異教徒から受容すべきであった。


\\



{\scshape Sed contra est} quod Luc.~{\scshape iii} generatio domini reducitur usque ad Adam.


&


 しかし反対に、『ルカによる福音書』第3章\footnote{「イエスが宣教を始められたときはおよそ三十歳であった。イエスはヨセフの子と思われていた。ヨセフはエリの子、それからさかのぼると、マタト、レビ、メルキ、ヤナイ、ヨセフ、マタティア、アモス、ナウム、エスリ、ナガイ、マハト、マタティア、セメイン、ヨセク、ヨダ、ヨハナン、レサ、ゼルバベル、シャルティエル、ネリ、メルキ、アディ、コサム、エルマダム、エル、ヨシュア、エリエゼル、ヨリム、マタト、レビ、シメオン、ユダ、ヨセフ、ヨナム、エリアキム、メレア、メンナ、マタタ、ナタン、ダビデ、エッサイ、オベド、ボアズ、サラ、ナフション、アミナダブ、アドミン、アルニ、ヘツロン、ペレツ、ユダ、ヤコブ、イサク、アブラハム、テラ、ナホル、セルグ、レウ、ペレグ、エベル、シェラ、カイナム、アルパクシャド、セム、ノア、レメク、メトシェラ、エノク、イエレド、マハラルエル、ケナン、エノシュ、セト、アダム。そして神に至る。」(3:23-38)} で、主の出生は、アダムにまでさか
 のぼると言われている。


\\



{\scshape Respondeo dicendum} quod, sicut Augustinus dicit, in XIII {\itshape de Trin}.,
{\itshape poterat Deus hominem aliunde suscipere, non de genere illius Adae qui
suo peccato obligavit genus humanum. Sed melius iudicavit et de ipso
quod victum fuerat genere assumere hominem Deus, per quem generis humani
vinceret inimicum}. 

&

解答する。以下のように言われるべきである。
アウグスティヌスが『三位一体論』第13巻で次のように述べている。「神は人間
 を、自分の罪によって人類に罪を着せたアダムの一族からではなく、どこか他
 から受容することができた。しかし、神が、打ち負かされた一族から人間を受容
 し、それによって、人類の敵を打ち負かすように、よりよく判断した」。


\\

Et hoc propter tria. Primo quidem, quia hoc videtur
ad iustitiam pertinere, ut ille satisfaciat qui peccavit. Et ideo de
natura per ipsum corrupta debuit assumi id per quod satisfactio erat
implenda pro tota natura. 

&

これは三つのことのためにである。第一に、罪を犯した人に償わせることは正義
 に属すると思われるからである。それゆえ、彼によって破壊された本性から、
 それによって全本性のために償いがなされるべきところのものが受容されるべ
 きであった。


\\

Secundo, hoc etiam pertinet ad maiorem hominis
dignitatem, dum ex illo genere victor Diaboli nascitur quod per Diabolum
fuerat victum. Tertio, quia per hoc etiam Dei potentia magis ostenditur,
dum de natura corrupta et infirma assumpsit id quod in tantam virtutem
et dignitatem est promotum.


&

第二に、悪魔に打ち負かされた一族から、悪魔を打ち負かす者が生まれるとき、
 それは、人間のより大きな偉大さに属する。第三に、崩壊し弱体化した本性か
 ら、あれほど大きな力と偉大さが出てきたものを受容したとき、それによって、
 神の能力が、より大きく示される。

\\



{\scshape Ad primum ergo dicendum} quod Christus debuit esse a peccatoribus
segregatus quantum ad culpam, quam venerat destruere, non quantum ad
naturam, quam venerat salvare; secundum quam {\itshape debuit per omnia fratribus
assimilari}, ut idem apostolus dicit, {\itshape Heb}.~{\scshape ii}. Et in hoc etiam mirabilior
est eius innocentia, quod de massa peccato subiecta natura assumpta
tantam habuit puritatem.


&

第一異論に対しては、それゆえ、以下のように言われるべきである。
キリストは罪に関して、罪人たちから切り離されなければならなかったのであり、
 本性にかんしてではない。彼は罪を破壊するために、本性を救うためにやって
 来たのだから。そして、同じ使徒が『ヘブライ人への手紙』第2章で言うように、
 この本性において、「全ての同胞たちによって類似化しなければならなかった」
 \footnote{「それで、イエスは、神の御前において憐れみ深い、忠実な大祭司
 となって、民の罪を償うために、すべての点で兄弟たちと同じようにならねば
 ならなかったのです。」(2:17)}。
そして、罪にまみれた塊から受け取られた本性が、あれほどの純粋さを持ってい
 たことにおいて、彼の無垢は驚くべきものである。

\\



{\scshape Ad secundum dicendum} quod, sicut dictum est, oportuit eum qui peccata
venerat tollere, esse a peccatoribus segregatum quantum ad culpam, cui
Adam subiacuit, et quem Christus {\itshape a suo delicto eduxit}, ut dicitur
{\itshape Sap}.~{\scshape x}. Oportebat autem eum qui mundare omnes venerat, non esse
mundandum, sicut et in quolibet genere motus primum movens est immobile
secundum illum motum, sicut primum alterans est inalterabile. Et ideo
non fuit conveniens ut assumeret humanam naturam in ipso Adam.


&

第二異論に対しては、以下のように言われるべきである。
すでに述べられたとおり、罪を破壊するために来た者は、罪にかんして、罪人た
 ちから離されていなければならないが、アダムは罪のもとにあった。そして、
 キリストは、『知恵の書』第10章で言われるように、「自らの失敗から救い上
 げた」\footnote{「世の父として最初に造られたただ一人の人を、/知恵は守
 り、犯した過ちから救い、万物を治める力を彼に与えた。」(10:1-2)}。
さらに、全ての人を清めるために来た者は、清められるべきではなかった。ちょ
 うど、どの類においても、第一の動かす運動は、その運動にかんして、不動で
 あるように。たとえば、最初の変化を起こすものは変化しえないものである。
 ゆえに、人間本性をアダム自身において受容するのは適切でなかった。

\\



{\scshape Ad tertium dicendum} quod, quia Christus debebat esse maxime a
peccatoribus segregatus quantum ad culpam, quasi summam innocentiae
obtinens, conveniens fuit ut a primo peccatore usque ad Christum
perveniretur mediantibus quibusdam iustis, in quibus perfulgerent
quaedam insignia futurae sanctitatis. Propter hoc etiam in populo ex quo
Christus erat nasciturus instituit Deus quaedam sanctitatis signa, quae
incoeperunt in Abraham, qui primus promissionem accepit de Christo, et
circumcisionem in signum foederis consummandi, ut dicitur {\itshape Gen}.~{\scshape xvii}.


&


第三異論に対しては、以下のように言われるべきである。
キリストは、最高の無垢をもつ者として、最大限に、罪にかんして罪人たちから
 切り離されなければならなかったので、最初の罪人からキリストまで、義人た
 ちを介して至ることが適切であった。その義人たちにおいて、将来の聖性のし
 るしが輝いていた。
このことのために、キリストがそこから生まれた民において、神はある種の聖性
 のしるしを置いたのであり、それは、『創世記』第17章で言われるように\footnote{「わたしは、あなたとの間に、また後に続く子孫との間に契約を立て、それを永遠の契約とする。そして、あなたとあなたの子孫の神となる。わたしは、あなたが滞在しているこのカナンのすべての土地を、あなたとその子孫に、永久の所有地として与える。わたしは彼らの神となる。神はまた、アブラハムに言われた。「だからあなたも、わたしの契約を守りなさい、あなたも後に続く子孫も。あなたたち、およびあなたの後に続く子孫と、わたしとの間で守るべき契約はこれである。すなわち、あなたたちの男子はすべて、割礼を受ける。
包皮の部分を切り取りなさい。
 これが、わたしとあなたたちとの間の契約のしるしとなる」」(17:7-11)}最初にキリストについての約束と、
 契約をしたしるしとしての割礼を受け取ったアブラハムに始まる。

\\






\end{longtable}


\end{document}


