\documentclass[10pt]{jsarticle} % use larger type; default would be 10pt
%\usepackage[utf8]{inputenc} % set input encoding (not needed with XeLaTeX)
%\usepackage[round,comma,authoryear]{natbib}
%\usepackage{nruby}
\usepackage{okumacro}
\usepackage{longtable}
%\usepqckage{tablefootnote}
\usepackage[polutonikogreek,english,japanese]{babel}
%\usepackage{amsmath}
\usepackage{latexsym}
\usepackage{color}
%\usepackage{tikz}

%----- header -------
\usepackage{fancyhdr}
\pagestyle{fancy}
\lhead{{\it Summa Theologiae} I, q.~36}
%--------------------


\title{{\bf PRIMA PARS}\\{\HUGE Summae Theologiae}\\Sancti Thomae
Aquinatis\\{\sffamily QUEAESTIO TRIGESTIMASEXTA}\\DE PERSONA SPIRITUS SANCTI}
\author{Japanese translation\\by Yoshinori {\scshape Ueeda}}
\date{Last modified \today}

%%%% コピペ用
%\rhead{a.~}
%\begin{center}
% {\Large {\bf }}\\
% {\large }\\
% {\footnotesize }\\
% {\Large \\}
%\end{center}
%
%\begin{longtable}{p{21em}p{21em}}
%
%&
%
%\\
%\end{longtable}
%\newpage



\begin{document}

\maketitle

\begin{center}
{\Large 聖トマス・アクィナスの神学大全の第一部\\第三十六問\\聖霊のペル
 ソナについて}
\end{center}

\thispagestyle{empty}
\begin{longtable}{p{21em}p{21em}}

{\Huge P}ost haec considerandum est de his quae pertinent ad personam
 spiritus sancti. Qui quidem non solum dicitur spiritus sanctus, sed
 etiam amor et donum Dei. Circa nomen ergo spiritus sancti quaeruntur
 quatuor. 

\begin{enumerate}
 \item utrum hoc nomen spiritus sanctus sit proprium alicuius divinae personae.
 \item utrum illa persona divina quae spiritus sanctus dicitur, procedat a patre et filio.
 \item utrum procedat a patre per filium.
 \item utrum pater et filius sint unum principium spiritus sancti.
\end{enumerate}


&

これらの後に、聖霊のペルソナに属する事柄が考察されるべきである。彼は聖
 霊と言われるのみならず愛や神の賜物とも言われる。ゆえに聖霊という名称
 について四つのことが問われる。


\begin{enumerate}
 \item 聖霊というこの名称は何らかの神のペルソナに固有か。
 \item 聖霊と言われるこの神のペルソナは父と息子から発出するか。
 \item 父から息子を通して発出するか。
 \item 父と息子は聖霊の一つの根源か。
\end{enumerate}


\end{longtable}



\newpage





\rhead{a.~1}
\begin{center}
{\Large {\bf ARTICULUS PRIMUS}}\\
{\large UTRUM HOC NOMEN {\itshape SPIRITUS SANCTUS} SIT \\PROPRIUM
 NOMEN ALICUIUS DIVINAE PERSONAE}\\
{\footnotesize I {\itshape Sent.}, d.10, a.4; IV {\itshape
 Cont.~Gent.}, cap.19; {\itshape Compend.~Theol.}, cap.46, 47. }\\
{\Large 第一項\\「聖霊」というこの名称は何らかの神のペルソナに固有か}
\end{center}

\begin{longtable}{p{21em}p{21em}}

{\scshape Ad primum sic proceditur}. Videtur quod hoc nomen {\itshape Spiritus Sanctus} non
sit proprium nomen alicuius divinae personae. Nullum enim nomen
commune tribus personis, est proprium alicuius personae. Sed hoc nomen
{\itshape Spiritus Sanctus} est commune tribus personis. Ostendit enim Hilarius,
VIII {\itshape de Trin}., in {\itshape Spiritu Dei} aliquando significari patrem, ut cum
dicitur, {\itshape Spiritus domini super me}; aliquando significari filium, ut
cum dicit filius, {\itshape in Spiritu Dei eiicio Daemonia}, naturae suae
potestate eiicere se Daemonia demonstrans; aliquando spiritum sanctum,
ut ibi, {\itshape Effundam de spiritu meo super omnem carnem}. Ergo hoc nomen
{\itshape Spiritus Sanctus} non est proprium alicuius divinae personae.

&

第一項の問題へ議論は以下のように進められる。
「聖霊」というこの名称は何らかの神のペルソナの固有の名称ではないと思わ
 れる。理由は以下の通り。
三つのペルソナに共通するどんな名称もあるペルソナに属する固有なものでは
 ない。しかるに「聖霊」というこの名称は三つのペルソナに共通する。たと
 えばヒラリウスは『三位一体論』第8巻で、「神の霊」において、「主の霊が私
 の上に」と言われるときのようにあるときには父が表示され、息子が「神の霊において私は
 悪霊を追い払うだろう」と言うとき「自らの本性の権能によって、自ら悪霊を追
 い払うことを示している」のように、あるときには息子が表示され、「私の
 霊からすべての肉へ注ぐ」のように、あるときには聖霊が表示される。ゆえ
 にこの「聖霊」という名称は、何らかの神のペルソナに固有ではない。


\\



2. {\scshape Praeterea}, nomina divinarum personarum ad aliquid dicuntur, ut Boetius
dicit, in libro {\itshape de Trin}. Sed hoc nomen {\itshape Spiritus Sanctus} non dicitur ad
aliquid. Ergo hoc nomen non est proprium divinae personae.

&

さらに、ボエティウスが『三位一体論』で言うように神のペルソナの名称は関
 係的に語られる。しかるに「聖霊」というこの名称は関係的に語られていな
 い。ゆえにこの名称は神のペルソナに固有ではない。

\\



3. {\scshape Praeterea}, quia filius est nomen alicuius divinae personae, non potest
dici filius huius vel illius. Dicitur autem spiritus huius vel illius
hominis. Ut enim habetur {\itshape Num}.~{\scshape xi} : {\itshape Dixit dominus ad Moysen, auferam de
spiritu tuo, tradamque eis}; et IV {\itshape Reg}.~{\scshape ii} : {\itshape Requievit spiritus Eliae
super Elisaeum}. Ergo {\itshape Spiritus Sanctus} non videtur esse proprium nomen
alicuius divinae personae.

&

さらに、息子はある神のペルソナの名称なのでこの人の息子やあの人の息子と
 言われえない。しかしこの人の霊やあの人の霊と言われている。たとえば
 『民数記』第11章「神はモーセに言った。「あなたの霊から取り去りそして
 彼らに伝えるだろう」\footnote{「私はそこに降って、あなたと語り、あな
 たの上にある霊の一部を取って、彼らの上に置こう。そうすれば、彼らはあ
 なたと共に民の重荷を負うことができるようになり、あなた一人で負うこと
 はなくなる。」(11:17)}や『第四列王記』第2章「エリサの上にエリアの
 霊が留まった」\footnote{「エリコの預言者の仲間は、離れた所からエリシャ
 を見ていて、「エリヤの霊がエリシャの上にとどまっている」と言った。彼
 らはエリシャを迎えに来て、その前で地にひれ伏し、」(2:15) }などである。
 ゆえに「聖霊」は何らかの神のペルソナに固有でないと思われる。

\\



{\scshape Sed contra est} quod dicitur I Ioan. ult. : {\itshape Tres sunt qui testimonium
dant in caelo, Pater, Verbum et Spiritus Sanctus}. Ut autem Augustinus
dicit, VII {\itshape de Trin}., cum quaeritur : {\itshape Quid tres?} Dicimus, {\itshape Tres
personae}. Ergo {\itshape Spiritus Sanctus} est nomen divinae personae.

&

しかし反対に、『ヨハネの手紙一』の最後で「天で証言するのは三つ、父、
 言葉、聖霊である」\footnote{「証しするのは三者で、霊と水と血です。こ
 の三者の証しは一致しています。」(5:7-8)}と言われている。さらにアウグ
 スティヌスは『三位一体論』第7巻で「三とは何か」と問われて私たちは「三
 つのペルソナ」と言う、と述べている。ゆえに「聖霊」は神のペルソナの名
 称である。

\\



{\scshape Respondeo dicendum} quod, cum sint duae processiones in divinis, altera
earum, quae est per modum amoris, non habet proprium nomen, ut supra
dictum est. Unde et relationes quae secundum huiusmodi processionem
accipiuntur, innominatae sunt, ut etiam supra dictum est. 
Propter quod
et nomen personae hoc modo procedentis, eadem ratione, non habet
proprium nomen.


&

解答する。以下のように言われるべきである。
神の中には二つの発出があるが、すでに述べられたように愛のしかたによるそ
 れらの一方は固有の名称をもっていない。したがって、このような発出に即
 して理解される関係も、このこともすでに述べられたが、まだ名付けられて
 いない。このため、このしかたで発出するペルソナの名称も同じ理由で固有
 の名称をもっていない。

\\



Sed sicut sunt accommodata aliqua nomina, ex usu
loquentium, ad significandum praedictas relationes, cum nominamus eas
nomine {\itshape processionis} et {\itshape spirationis}, quae, secundum proprietatem
significationis, magis videntur significare actus notionales quam
relationes; ita ad significandum divinam personam quae procedit per
modum amoris, accommodatum est, ex usu Scripturae, hoc nomen {\itshape Spiritus
Sanctus}. 


&

しかし、私たちがそれらを「発出」や「霊発」という名称で名付けるとき、こ
 れらは意味の固有性に応じて、関係よりは識標的な作用を意味するように見え
 るのだが、ある名称が話者たちの使用に基づいて採用されているように、
ちょうどそのように、聖書の使用に基づいて、愛のしかたで発出す
 る神のペルソナを表示するために、この「聖霊」という名称が採用された。


\\



Et huius quidem convenientiae ratio sumi potest ex
duobus. Primo quidem, ex ipsa communitate eius quod dicitur spiritus
sanctus. Ut enim Augustinus dicit, XV {\itshape de Trin.} : {\itshape quia}
Spiritus Sanctus {\itshape communis est ambobus, id vocatur ipse proprie quod ambo communiter,
nam et pater est spiritus, et filius est spiritus; et pater est
sanctus, et filius est sanctus}. 



&

そしてこれが適切であることの根拠は二つのことから取られうる。一つには、
 聖霊と言われるものの共通性自体からである。すなわち、アウグスティヌス
 が『三位一体論』第15巻で述べるように、聖霊は「両者に共通であり、両者
 に共通にとそれ自身が固有に呼ばれる。すなわち父も例であり息子も例であ
 り、父も聖であり息子も聖だからである」。

\\



Secundo vero, ex propria
significatione. Nam nomen spiritus, in rebus corporeis, impulsionem
quandam et motionem significare videtur, nam flatum et ventum spiritum
nominamus. Est autem proprium amoris, quod moveat et impellat
voluntatem amantis in amatum. Sanctitas vero illis rebus attribuitur,
quae in Deum ordinantur. Quia igitur persona divina procedit per modum
amoris quo Deus amatur, convenienter {\itshape Spiritus Sanctus} nominatur.

&


第二に固有の意味に基づいてである。すなわち霊という名称は、物体的事物に
 おいてある種の衝撃や運動を意味するように思われる。たとえば私たちは息
 や風をspiritusと名付ける。しかるに愛する人の意志を愛されるものへ動か
 したり強いたりすることは愛に固有だと思われる。他方で聖性は神へ秩序づ
 けられている事物に帰せられる。ゆえに、それによって神が愛される愛のし
 かたによって神のペルソナが発出するのだから、それは適切に「聖霊」と名
 付けられる。

\\



{\scshape Ad primum ergo dicendum} quod hoc quod dico {\itshape Spiritus Sanctus}, prout
sumitur in virtute duarum dictionum, commune est toti Trinitati. Quia
nomine spiritus significatur immaterialitas divinae substantiae,
spiritus enim corporeus invisibilis est, et parum habet de materia;
unde omnibus substantiis immaterialibus et invisibilibus hoc nomen
attribuimus. Per hoc vero quod dicitur sanctus, significatur puritas
divinae bonitatis. 

&


第一異論に対してはそれゆえ以下のように言われるべきである。
「聖霊」と私が言うものは、それが二語からなるものとして理解されるかぎり
 三性全体に共通する。すなわち、「霊」という名称によって神の実
 体の卑湿両性が表示される。物体的なspiritusは不可視のものであり、もっ
 とも質料をもたない。したがって、すべての非質料的で不可視の実体にこの
 名称が帰せられる。他方「聖」と言われるものによって、神の善性の純粋さ
 が意味される。

\\


Si autem accipiatur hoc quod dico {\itshape Spiritus Sanctus},
in vi unius dictionis, sic ex usu Ecclesiae est accommodatum ad
significandam unam trium personarum, scilicet quae procedit per modum
amoris, ratione iam dicta.


&

他方、私が「聖霊」という言うものが一語として理解されるならば、その場合
 には教会の使用に基づいて三つのペルソナの一つ、すなわちすでに述べられ
 た理由により、愛のしかたで発出するペルソナを表示すべく採用される。



\\




{\scshape Ad secundum dicendum} quod, licet hoc quod dico {\itshape Spiritus Sanctus},
relative non dicatur, tamen pro relativo ponitur, inquantum est
accommodatum ad significandam personam sola relatione ab aliis
distinctam. Potest tamen intelligi etiam in nomine aliqua relatio, si
spiritus intelligatur quasi spiratus.

&

第二異論に対しては以下のように言われるべきである。
「聖霊」と私が言うものが関係的に言われないとしても、ただ関係によっての
 み他から区別されるペルソナを表示すべく採用されている限りにおいて、関
 係的なものとして措定されている。それにしてもしかし、「霊」が「呼吸」の意味で理解
 されるならば、この名称において何らかの関係が理解されることも可能であ
 る。



\\



{\scshape Ad tertium dicendum} quod in nomine filii intelligitur sola relatio
eius qui est a principio, ad principium, sed in nomine patris
intelligitur relatio principii; et similiter in nomine spiritus, prout
importat quandam vim motivam. Nulli autem creaturae competit esse
principium respectu alicuius divinae personae, sed e converso. Et ideo
potest dici pater noster, et spiritus noster, non tamen potest dici
filius noster.

&

第三異論に対しては以下のように言われるべきである。
「息子」という名称においては、根源からであるものの根源への関係だけが理
 解されるが、「父」という名称においては根源の関係が理解される。
 「霊」という名称においても、それが何らかの動かす力を意味するかぎりで
 は同様である。しかし、ある神のペルソナにかんして根源であるとい
 うことは、そんな被造物にも適合しない。ゆえに私たちの父や私たちの霊とは言えるが、
 私たちの息子とは言われえない。

\end{longtable}
\newpage


\rhead{a.~2}
\begin{center}
{\Large {\bf ARTICULUS SECUNDUS}}\\
{\large UTRUM SPIRITUS SANCTUS PROCEDAT A FILIO}\\
{\footnotesize I {\itshape Sent.}, d.11, a.1; IV {\itshape SCG},
 cap.24, 25; {\itshape De Pot.}, q.10, a.4, 5; {\itshape Contra
 errores Graec.}, parte II, cap.27 usque ad 32; {\itshape
 Compend.~Theol.}, cap.49; {\itshape Contra Graecos., Armenos, etc.},
 cap.4; in {\itshape Ioan.}, cap.15, lect.4; cap.14, lect.4.}\\
{\Large 第二項\\聖霊は息子から発出するか}
\end{center}

\begin{longtable}{p{21em}p{21em}}
{\scshape Ad secundum sic proceditur}. Videtur quod Spiritus Sanctus
non procedat a filio. Quia secundum Dionysium, {\itshape non est audendum dicere
aliquid de substantiali divinitate, praeter ea quae divinitus nobis ex
sacris eloquiis sunt expressa}. Sed in Scriptura sacra non exprimitur
quod {\itshape Spiritus Sanctus} a filio procedat, sed solum quod
procedat a patre; ut patet Ioann.~{\scshape xv}, {\itshape spiritum veritatis, qui a patre
procedit}. Ergo Spiritus Sanctus non procedit a filio.

&

第二項の問題へ議論は以下のように進められる。聖霊は息子から発出しないと
 思われる。理由は以下の通り。ディオニュシウスによれば「神について
 私たちに聖なる言葉に基づいて表明されていること以外の何かを敢えて語ろ
 うとすべきでない」。しかるに聖書には聖霊が息子から発出するとは表現さ
 れてなく、たとえば『ヨハネによる福音書』第15章「真理の霊、彼は父から
 発出する」\footnote{「私が父のもとからあなたがたに遣わそうとしている弁護者、すなわち、父のもとから出る真理の霊が来るとき、その方が私について証しをなさるであろう。」(15:26)}のように、ただ父から発出すると表現されている。ゆえに聖霊は
 息子から発出しない。


\\



2. {\scshape Praeterea}, in symbolo Constantinopolitanae synodi sic
 legitur :
{\itshape Credimus in spiritum sanctum, dominum et vivificantem, ex patre
procedentem, cum patre et filio adorandum et glorificandum}. Nullo
igitur modo debuit addi in symbolo nostro quod Spiritus
Sanctus procedat a filio, sed videntur esse anathematis rei, qui hoc
addiderunt.

&

さらに、コンスタンティノープル教会会議の要綱において以下のように書かれ
 ている。「私たちは聖霊、主、生かす者、神から発出する者、父と息子と共
 に崇められるべき者、栄光とされる者を信じる」。ゆえに私たちの要綱のな
 かに聖霊が息子から発出するということを断じて加えるべきではなく、それ
 を加えた人々は破門の事例であると思われる。

\\



3. {\scshape Praeterea}, Damascenus dicit, {\itshape spiritum sanctum ex patre dicimus, et
spiritum patris nominamus, ex filio autem spiritum sanctum non
dicimus, spiritum vero filii nominamus}. Ergo Spiritus
Sanctus non procedit a filio.

&

さらにダマスケヌスは次のように述べている。「私たちは聖霊が父からである
 と言い、父の霊と名付ける。しかし私たちは聖霊が息子からであるとは言わ
 ないが、息子の霊と名付ける」。ゆえに聖霊は息子から発出しない。


\\



4. {\scshape Praeterea}, nihil procedit ab eo in quo quiescit. Sed {\itshape
Spiritus Sanctus} quiescit in filio. Dicitur enim in legenda beati
Andreae, {\itshape pax vobis, et universis qui credunt in unum Deum patrem, et
in unum filium eius, unicum dominum nostrum Iesum Christum, et in unum
spiritum sanctum, procedentem ex patre, et in filio permanentem}. Ergo
Spiritus Sanctus non procedit a filio.

&

さらに、なにものも、そこにおいて休らっているところから発出しない。しか
るに聖霊は息子において休らっている。なぜなら聖アンドレアの聖人伝で次の
 ように言われているからである。「あなたたちに平和を。そして父である一人
 の神、その神の一人の息子、私たちの一人の主人であるイエス・キリスト、
 父から発出し息子の中に留まっている一人の聖霊を信じるすべての人々に平
 和を」。ゆえに聖霊は息子から発出しない。

\\



5. {\scshape Praeterea}, filius procedit ut verbum. Sed spiritus noster in nobis non
videtur procedere a verbo nostro. Ergo nec Spiritus Sanctus
procedit a filio.

&

さらに、息子は言葉として発出する。しかし私たちの中の霊は私たちの言葉か
 ら発出するようには思われない。ゆえに聖霊も息子から発出しない。

\\



6. {\scshape Praeterea},  Spiritus Sanctus perfecte procedit a patre. Ergo
superfluum est dicere quod procedit a filio.

&

さらに、聖霊は完全に父から発出する。ゆえに息子から発出すると語ることは
 無駄なことである。

\\


7. {\scshape Praeterea}, {\itshape in perpetuis non differt esse et posse}, ut dicitur in III
{\itshape Physic}.; et multo minus in divinis. Sed Spiritus Sanctus
potest distingui a filio, etiam si ab eo non procedat. Dicit enim
Anselmus, in libro {\itshape de Processione Spiritus Sancti} : {\itshape Habent utique a
patre esse filius et {\itshape Spiritus Sanctus}, sed diverso modo,
quia alter nascendo, et alter procedendo, ut alii sint per hoc ab
invicem}. Et postea subdit : {\itshape Nam si per aliud non essent plures filius
et Spiritus Sanctus, per hoc solum essent diversi}. Ergo
{\itshape Spiritus Sanctus} distinguitur a filio, ab eo non existens.

&

さらに、『自然学』第3巻で言われるように「永遠なものにおいて\kenten{あ
 る}ことと\kenten{できる}ことは異ならない」。そしてましてや神において
 は異ならない。しかるに聖霊は、かりに息子から発出しなかったとしても息
 子から区別されうる。なぜならアンセルムスが『聖霊の発出について』とい
 う書物の中で以下のように述べているからである。「息子と聖霊は神から存
 在を持つが、そのしかたが異なる。なぜなら一方は生まれることによって、
 他方は発出することによって持つからである。そして相互にこのことによっ
 て他のものである」。そしてその後でこう述べている。「すなわち、もし他
 のことによって息子と聖霊が複数でなかったとしても、このことのみによっ
 て異なるものであっただろう」。ゆえに聖霊は息子から存在するのでないと
 しても息子から区別される。

\\



{\scshape Sed contra est} quod dicit Athanasius : {\itshape Spiritus Sanctus a
patre et filio, non factus, nec creatus, nec genitus, sed procedens}.

&

しかし反対に、アタナシウスは「聖霊は父と息子から、作られず、創造されず、
 生み出されず、発出する」と述べている。

\\

{\scshape Respondeo dicendum quod} necesse est dicere Spiritum Sanctum a Filio
esse. Si enim non esset ab eo, nullo modo posset ab eo personaliter
distingui. Quod ex supra dictis patet. Non enim est possibile dicere
quod secundum aliquid absolutum divinae personae ab invicem
distinguantur, quia sequeretur quod non esset trium una essentia;
quidquid enim in divinis absolute dicitur, ad unitatem essentiae
pertinet. Relinquitur ergo quod solum relationibus divinae personae ab
invicem distinguantur. 

&

解答する。以下のように言われるべきである。
聖霊は息子からあると言うことが必然である。理由は以下の通り。
もし息子からでないならば、聖霊は息子からペルソナ的にどのようなしかたに
 よっても区別さ
 れえなかったであろう。これは前に述べられたことから明らかである。すな
 わち、何らかの非関係的なものに即して、神のペルソナが相互に区別される
 と言うことはできない。なぜなら、(もしそう言えたならば)三つのものに
 属する一つの本質ではないことが帰結しただろうから。というのも、なんで
 あれ神において非関係的に言われるものは本質の一性に属するからである。
 それゆえ、関係によってのみ神のペルソナは相互に区別されることが
帰結する。



\\

Relationes autem personas distinguere non
possunt, nisi secundum quod sunt oppositae. Quod ex hoc patet, quia
Pater habet duas relationes, quarum una refertur ad Filium, et alia ad
Spiritum Sanctum; quae tamen, quia non sunt oppositae, non constituunt
duas personas, sed ad unam personam Patris tantum pertinent. Si ergo
in Filio et in Spiritu Sancto non esset invenire nisi duas relationes
quibus uterque refertur ad Patrem, illae relationes non essent ad
invicem oppositae; sicut neque duae relationes quibus Pater refertur
ad illos. 

&

ところで関係は、それらが対立する限りにおいてでなければペルソナを区別す
 ることができない。これは以下のことから明らかである。すなわち、父は二
 つの関係をもち、その一つは息子へもう一つは聖霊へ関係する。しかしこれ
 らの関係は対立していないので二つのペルソナを構成せずむしろ父というた
 だ一つのペルソナに属する。ゆえにもし息子と聖霊の中に両者が父へ関係す
 る二つの関係しか見出されなかったとしたら、ちょうど父がそれらへ関係す
 る関係が対立していないように、それらの関係は相互に対立し
 ていない。


\\


Unde, sicut persona Patris est una, ita sequeretur quod
persona Filii et  Spiritus Sancti esset una, habens duas relationes
oppositas duabus relationibus Patris. Hoc autem est haereticum, cum
tollat fidem Trinitatis. Oportet ergo quod Filius et  Spiritus Sanctus ad invicem referantur oppositis relationibus. Non
autem possunt esse in divinis aliae relationes oppositae nisi
relationes originis, ut supra probatum est. 




&

それゆえ、ちょうど父のペルソナが一つであるように、息子と聖霊のペルソナ
 も、父の二つの関係に対立する二つの関係をもつものとして一つであっただ
 ろう。これは三位一体の信仰を否定するので異端である。ゆえに、息子と聖
 霊は相互に対立する関係によって関係するのでなければならない。しかるに
 前に証明されたとおり、神において起源の関係以外に対立する関係はありえ
 ない。


\\


Oppositae autem relationes
originis accipiuntur secundum principium, et secundum quod est a
principio. Relinquitur ergo quod necesse est dicere vel Filium esse a
 Spiritu Sancto, quod nullus dicit, vel  Spiritum Sanctum esse a Filio,
quod nos confitemur. 

&

 そして、対立する起源の関係は、根源に即してと根源からであるものに即して理解さ
 れる。ゆえに息子が聖霊からであるかまたは聖霊が息子からであるかが残さ
 れるが、前者を言う人は誰もいない。私たちは後者を認める。


\\



Et huic quidem consonat ratio processionis
utriusque. Dictum enim est supra quod Filius procedit per modum
intellectus, ut verbum; Spiritus Sanctus autem per modum
voluntatis, ut amor. Necesse est autem quod amor a verbo procedat, non
enim aliquid amamus, nisi secundum quod conceptione mentis
apprehendimus. Unde et secundum hoc manifestum est quod
 Spiritus Sanctus procedit a Filio. 



&

そしてこれには両方の発出の性格が調和する。すなわち、息子は言葉として知
 性のあり方によって発出し、また聖霊は愛として
意志のあり方によって発出するということが前に語られた。しかるに愛が言葉
 から発出することは必然である。私たちは精神の懐念によって捉える限りに
 おいてでなければ何も愛さないのだから。したがってこのことに即しても、
 聖霊が息子から発出することは明らかである。

\\



Ipse etiam ordo rerum hoc
docet. Nusquam enim hoc invenimus, quod ab uno procedant plura absque
ordine, nisi in illis solum quae materialiter differunt; sicut unus
faber producit multos cultellos materialiter ab invicem distinctos,
nullum ordinem habentes ad invicem. Sed in rebus in quibus non est
sola materialis distinctio, semper invenitur in multitudine
productorum aliquis ordo. 


&

さらに諸事物の秩序もまたこのことを教えている。すなわち、一つのものから
 複数のものが無秩序に発出するということは、質料的に異なっているものだ
 けにおいて見出されることである。たとえば一人の工作人が質料的に区別さ
 れるが相互に何の秩序もない多くの小刀を作り出す場合のように。しかし、
 質料的な区別だけがあるのではない諸事物においては、常に生み出された多
 の中に何らかの秩序がある。


\\




Unde etiam in ordine creaturarum
productarum, decor divinae sapientiae manifestatur. Si ergo ab una
persona Patris procedunt duae personae, scilicet Filius et 
 Spiritus Sanctus, oportet esse aliquem ordinem eorum ad invicem. Nec
potest aliquis ordo alius assignari, nisi ordo naturae, quo alius est
ex alio. 
Non est igitur possibile dicere quod Filius et 
 Spiritus Sanctus sic procedant a Patre, quod neuter eorum procedat ab
alio, nisi quis poneret in eis materialem distinctionem, quod est
impossibile. 


&

したがって、生み出された被造物の秩序においても神の知恵の装飾が示されて
 いる。ゆえにもし一つの父のペルソナから二つのペルソナすなわち息子と聖
 霊が発出するならば、それら相互に何らかの秩序がなければならない。そし
 て、一方が他方からという本性の秩序以外にはどんな他の秩序も指定される
 ことはできない。
それゆえ、そのどちらからも他方が発出しないというしかたで息子と聖霊が父
 から発出すると語ることは、その人がそれらの中に質料的な区別しか措定し
 ないのでなければ不可能であるが、そのような区別は不可能である。

\\


Unde etiam ipsi Graeci processionem  Spiritus Sancti
aliquem ordinem habere ad Filium intelligunt. 
Concedunt enim  Spiritum
sanctum esse  {\itshape Spiritum Filii}, et esse a Patre {\itshape per Filium}. Et quidam
eorum dicuntur concedere quod sit {\itshape a Filio}, vel {\itshape profluat ab eo}, non
tamen quod {\itshape procedat}. 


&

したがっ
 てギリシア人たちも聖霊の発出がある種の秩序を息子に対して持っていると
 考えている。
つまり彼らは聖霊が「息子の霊」であり、父から「息子を通して」あることを認めてい
 る。そして彼らのうちのある人々は、それが息子からであること、あるいは、
 息子から「発出する」のではないにしても「流れ出す」と認めていると言われている。

\\

Quod videtur vel ex ignorantia, vel ex protervia
esse. Quia si quis recte consideret, inveniet {\itshape processionis} verbum
inter omnia quae ad originem qualemcumque pertinent, communissimum
esse. 


&

 このことは、無知あるいは傲慢からであると思われる。なぜなら、もしだれ
 かが正しく考察していたならば「発出」という言葉はどのようなかたちであれ起源に
 属するすべての事柄の中でもっとも共通であることを見出していたであろう
 から。

\\


Utimur enim eo ad designandum qualemcumque originem; sicut quod
linea procedit a puncto, radius a sole, rivus a fonte; et similiter in
quibuscumque aliis. Unde ex quocumque alio ad originem pertinente,
potest concludi quod {\itshape  Spiritus Sanctus} procedit a Filio.

&

じっさい私たちはそれ(発出という言葉)をどんなものであれ起源を示すために用いる。
 線は点から、太陽から光が、泉から川が発出する、というように。これはどん
 な他のものにおいても同様である。したがって、起源に属する他のどんなも
 のからも、聖霊が息子から発出することが結論されうる。

\\



{\scshape Ad primum ergo dicendum} quod de Deo dicere non debemus quod in sacra
Scriptura non invenitur vel per verba, vel per sensum. Licet autem per
verba non inveniatur in sacra Scriptura quod Spiritus
Sanctus procedit a Filio, invenitur tamen quantum ad sensum; et
praecipue ubi dicit Filius, Ioan. XVI, de  Spiritu Sancto loquens, {\itshape ille
me clarificabit, quia de meo accipiet}. 


&

第一異論に対してはそれゆえ以下のように言われるべきである。
言葉によってであれ意味によってであれ、聖書の中に見出されないことを私た
 ちは神について語るべきではない。しかし聖書の中に、言葉によって聖霊が
 息子から発出することは見出されないにしても、意味によっては見出される。
 とくに息子が『ヨハネによる福音書』第16章で聖霊について語って「彼が私
 を明らかにするだろう。なぜなら私から受け取るだろうから」
 \footnote{「その方は私に栄光を与える。私のものを受けて、あなたがたに
 告げるからである。」(16:14)}と述べるところで。

\\

Regulariter etiam in sacra
Scriptura tenendum est, quod id quod de Patre dicitur, oportet de
filio intelligi, etiam si dictio exclusiva addatur, nisi solum in
illis in quibus Pater et Filius secundum oppositas relationes
distinguuntur. Cum enim dominus, Matth.~{\scshape xi}, dicit, {\itshape nemo novit Filium
nisi Pater}, non excluditur quin Filius seipsum cognoscat. 


&

さらに一般的な規則として聖書において父について言われることは、排他的な
語が吹かされている場合でも、息子についても解されるべきだと考えられるべ
 きである。たあし、父と息子が対立する関係において区別されている場合は
 除いて。たとえば主人が『マタイによる福音書』第11章で「父以外だれも息子を知
 らない」と言うとき、息子が自分自身を認識することは排除されていない。

\\

Sic igitur
cum dicitur quod Spiritus Sanctus a Patre procedit, etiam
si adderetur quod a solo Patre procedit, non excluderetur inde Filius,
quia quantum ad hoc quod est esse principium  Spiritus Sancti, non
opponuntur Pater et Filius; sed solum quantum ad hoc, quod hic est
pater et ille Filius.

&

ゆえにこのようにして聖霊が父から発出すると言われるとき、かりに「ただ」父から
 発出すると付言されたとしても、このことから息子が排除されはしなかった
 だろう。なぜなら聖霊の根源であるということにかんして、父と息子は対立
 せず、これは父であり彼は息子であるということにかんしてのみ対立するか
 らである。


\\



{\scshape Ad secundum dicendum} quod in quolibet Concilio institutum fuit
symbolum aliquod, propter errorem aliquem qui in Concilio
damnabatur. Unde sequens Concilium non faciebat aliud symbolum quam
primum, sed id quod implicite continebatur in primo symbolo, per
aliqua addita explanabatur contra haereses insurgentes. 


&

第二異論に対しては以下のように言われるべきである。
どの教会会議においても、その会議で断罪された何らかの誤りのために要綱が
 決定された。したがって、後続する会議は最初の会議と異なる要綱を設定せ
 ず、第一の要綱に潜在的に含まれているものが、何かを付け加えることによって、
 勃興する異端に反対して説明されていた。


\\

Unde in
determinatione Chalcedonensis synodi dicitur, quod illi qui fuerunt
congregati in Concilio Constantinopolitano, doctrinam de  Spiritu
sancto tradiderunt, {\itshape non quod minus esset in praecedentibus} (qui apud
Nicaeam congregati sunt), {\itshape inferentes; sed intellectum eorum adversus
haereticos declarantes}. 


&

このことから、カルケドン公会議の決定において以下のように言われている。
これはコンスタンティノープル公会議に集まった人々に聖霊について教説を
伝えたが、それは「(ニケア公会議に集まった)先行する人々の中に足りない
ことがあったからではなく」推論して「異端者たちに反対する彼らの理解を明示するた
めにである」。

\\

Quia igitur in tempore antiquorum Conciliorum
nondum exortus fuerat error dicentium  Spiritum Sanctum non procedere a
filio; non fuit necessarium quod hoc explicite poneretur. Sed postea,
insurgente errore quorundam, in quodam Concilio in Occidentalibus
partibus congregato, expressum fuit auctoritate Romani pontificis;
cuius auctoritate etiam antiqua Concilia congregabantur et
confirmabantur. Continebatur tamen implicite in hoc ipso quod
dicebatur Spiritus Sanctus a Patre procedere.

&

ゆえに、古代の公会議の時代には聖霊が息子から発出しないと語る人々の誤り
 はまだ表面に現れていなかったので、このことが明示的に述べられる必要は
 なかった。しかしその後、ある人々の誤りが発生したので、西方のある地方
 に集まった会議において、ローマ教皇の権威によって表明された。この権威
 によって古代の公会議も開かれ確定されたものである。しかしそれは潜在的に、聖
 霊が父から発出すると言われたことの中に含まれていた。

\\



{\scshape Ad tertium dicendum} quod  Spiritum Sanctum non procedere a Filio, primo
fuit a Nestorianis introductum; ut Patet in quodam symbolo
Nestorianorum damnato in Ephesina synodo. Et hunc errorem secutus fuit
Theodoretus Nestorianus, et plures post ipsum; inter quos fuit etiam
Damascenus. Unde in hoc eius sententiae non est standum. Quamvis a
quibusdam dicatur quod Damascenus, sicut non confitetur  Spiritum
sanctum esse a Filio, ita etiam non negat, ex vi illorum verborum.

&

第三異論に対しては以下のように言われるべきである。
聖霊が息子から発出しないということは、最初にネストリウス派の人々によっ
 て導入された。これはエフェソスの公会議で断罪されたネストリウス派の人々
 の教義において明らかである。そしてこの誤りにテオドレトゥス・ネストリ
 アヌスが従い、彼の後に多くの人々が従った。その中にダマスケヌスもいた。
 このことから、この点で彼の主張は保持されるべきでない。
ただしある人々によって、ダマスケヌスは聖霊が息子からであることを告白していな
いけれども、彼の言葉の上ではそれを否定もしていないと言われているのだが。


\\



{\scshape Ad quartum dicendum} quod per hoc quod {\itshape  Spiritus Sanctus}
dicitur quiescere vel manere in Filio, non excluditur quin ab eo
procedat, quia et Filius in Patre manere dicitur, cum tamen a Patre
procedat. Dicitur etiam {\itshape  Spiritus Sanctus} in Filio
quiescere, vel sicut amor amantis quiescit in amato; vel quantum ad
humanam naturam Christi, propter id quod scriptum est, Ioan.~{\scshape
 i} : {\itshape Super
quem videris  Spiritum descendentem, et manentem super eum, hic est qui
baptizat}.

&

第四異論に対しては以下のように言われるべきである。
聖霊が息子の中に休らうとか留まると言われることによって、彼から発出する
 ことが排除されるわけではない。なぜなら息子も父から発出するが、父のの中に留まると言われる
 からである。また、聖霊が息子の内で休らうと言われるのは、愛する人の
 愛が愛される人において休らうという意味であるか、あるいは、キリストの
 人間本性にかんして、『ヨハネによる福音書』第1章で「あなたは彼の上に霊
 が降りて、彼の上に留まるのを見るだろう。彼は洗礼を施す者である」
 \footnote{「私はこの方を知らなかった。しかし、水で洗礼(バプテスマ)を
 授けるようにと、私をお遣わしになった方が私に言われた。『霊が降って、
 ある人にとどまるのを見たら、その人が、聖霊によって洗礼(バプテスマ)を
 授ける人である。』」(1:33)}と言われていることのためにである。



\\



{\scshape Ad quintum dicendum} quod verbum in divinis non accipitur secundum
similitudinem verbi vocalis, a quo non procedit  Spiritus, quia sic
tantum metaphorice diceretur, sed secundum similitudinem verbi
mentalis, a quo amor procedit.

&

第五異論に対しては以下のように言われるべきである。
神において言葉は音声の言葉への類似において理解されるのではない。その言
 葉はたんに比喩的に語られているので、そこから霊は発出しない。そうでは
 なく精神の言葉への類似に即して理解されているのであり、その言葉からは
 愛が発出する。

\\



{\scshape Ad sextum dicendum} quod per hoc quod {\itshape  Spiritus Sanctus}
perfecte procedit a Patre, non solum non superfluum est dicere quod
{\itshape  Spiritus Sanctus} procedat a Filio; sed omnino
necessarium. Quia una virtus est Patris et Filii; et quidquid est a
patre, necesse est esse a Filio, nisi proprietati Filiationis
repugnet. Non enim Filius est a seipso, licet sit a Patre.

&

第六異論に対しては以下のように言われるべきである。聖霊が完全に父から発
 出するということによって、聖霊が息子から発出することが無駄にならない
 のみならず、むしろまったくそれが必然である。なぜなら父と息子に一つの
 力が属するので、父からであるものは、息子の固有性に反しないかぎり、子
 からでもあることが必然だからである。たとえば息子は父からではあるが自
 分自身からではない、というように。

\\



{\scshape Ad septimum dicendum} quod {\itshape  Spiritus Sanctus} distinguitur
personaliter a Filio in hoc, quod origo unius distinguitur ab origine
alterius. Sed ipsa differentia originis est per hoc, quod Filius est
solum a Patre, {\itshape  Spiritus Sanctus} vero a Patre et Filio. Non
enim aliter processiones distinguerentur, sicut supra ostensum est.

&

第七異論に対しては以下のように言われるべきである。
聖霊は息子から一方の起源が他方の起源から区別される点でペルソナ的に区別
 される。しかしこの起源の違いは息子がただ父からであり聖霊が父と子から
 であるという点によってある。なぜなら、もしそうでないならそれらの発出
 が区別されなかったであろうから。これは前に示されたようにである。



\end{longtable}
\newpage




\rhead{a.~3}
\begin{center}
{\Large {\bf ARTICULUS TERTIUS}}\\
{\large UTRUM SPIRITUS SANCTUS PROCEDAT A PATRE PER FILIUM}\\
{\footnotesize I {\itshape Sent.}, d.12, a.3; {\itshape SCG}, II, cap.9.}\\
{\Large 第三項\\聖霊は息子を通して父から発出するか}
\end{center}

\begin{longtable}{p{21em}p{21em}}
{\scshape Ad tertium sic proceditur}. Videtur quod {\itshape  Spiritus Sanctus}
non procedat a Patre per Filium. Quod enim procedit ab aliquo per
aliquem, non procedit ab eo immediate. Si igitur {\itshape  Spiritus
Sanctus} procedit a Patre per Filium, non procedit a Patre
immediate. Quod videtur inconveniens.

&

第三項の問題へ議論は以下のように進められる。
聖霊は息子を通して父から発出するのではないと思われる。
理由は以下の通り。
何かを通して何かから発出するものは、それから直接的に発出するのではない。
ゆえにもし聖霊が父から息子を通して発出するならば父から直接的に発出しな
 い。これは不適当だと思われる。


\\



2. {\scshape Praeterea}, si {\itshape  Spiritus Sanctus} procedit a Patre per
Filium, non procedit a Filio nisi propter Patrem. Sed propter quod
unumquodque, et illud magis. Ergo magis procedit a Patre quam a Filio.

&

さらにもし聖霊が父から息子を通して発出するならば、父のゆえにでなければ
 息子からは発出しない。しかるに「おのおののものが\kenten{それ}のために
 然々であるような\kenten{それ}は、より然々である」\footnote{『分析論後
 書』第1巻第2章。}。ゆえに息子よりも父
 からより発出する。

\\



3. {\scshape Praeterea}, Filius habet esse per generationem. Si igitur {\itshape
Spiritus Sanctus} est a Patre per Filium, sequitur quod prius
generetur Filius, et postea procedat {\itshape  Spiritus Sanctus}. Et
sic processio  Spiritus Sancti non est aeterna. Quod est haereticum.

&

さらに、息子は生成によって存在を持つ。ゆえにもし聖霊が父から息子を通し
 てであるならば、より先に息子が生成し、その後に聖霊が発出する。そうす
 ると聖霊の発出は永遠でないことになる。これは異端である。

\\



4. {\scshape Praeterea}, cum aliquis dicitur per aliquem operari, potest e converso
dici, sicut enim dicimus quod rex operatur per ballivum, ita potest
dici quod ballivus operatur per regem. Sed nullo modo dicimus quod
Filius spiret  Spiritum Sanctum per Patrem. Ergo nullo modo potest
dici quod Pater spiret  Spiritum Sanctum per Filium.

&

さらに、ある人が他の人を通して働くとき、その逆を言うことも可能である。
 たとえば、私たちが王が役人によって働くと言うように、役人が王によって
 働くと言うこともできる。しかし、息子が父を通して聖霊を霊発するとはど
 んな意味においても言わない。ゆえに父が聖霊を息子を通して霊発すると言
 われることはできない。

\\



{\scshape Sed contra est} quod Hilarius dicit, in libro {\itshape de
 Trin}. : {\itshape Conserva hanc,
oro, fidei meae religionem, ut semper obtineam Patrem, scilicet te; et
Filium tuum una tecum adorem; et  Spiritum Sanctum tuum, qui est per
unigenitum tuum, promerear}.

&

しかし反対に、ヒラリウスは『三位一体論』という書物の中で以下のように述
 べている。「私の信仰のこの教義を保つように私は祈る。私が常に父すな
 わちあなたを保持するように、そしてあなたの息子をあなたとともに崇拝す
 るように、そしてあなたの独り子を通してあるあなたの聖霊を獲得するように」。

\\



{\scshape Respondeo dicendum} quod in omnibus locutionibus in quibus dicitur
aliquis per aliquem operari, haec praepositio {\itshape per} designat in causali
aliquam causam seu principium illius actus. Sed cum actio sit media
inter faciens et factum, quandoque illud causale cui adiungitur haec
praepositio per, est causa actionis secundum quod exit ab agente. 

&

解答する。以下のように言われるべきである。
ある人がある人を通して・ある人によって働くと言われるすべての語りにおいて、この「通して・
 よって」という前置詞は、原因的に、その作用の何らかの原因や根源を指示
 する。しかし作用は行う人と行われたものとの中間にあるので、時として「通して・よっ
 て」というこの前置詞が結び付けられる原因的なものが、作用が作用者から
 でていくかぎりでの作用の原因となる。

\\


Et
tunc est causa agenti quod agat; sive sit causa finalis, sive
formalis, sive effectiva vel motiva, finalis quidem, ut si dicamus
quod  artifex operatur {\itshape per cupiditatem lucri}; formalis vero, ut si
dicamus quod operatur {\itshape per artem suam}; motiva vero, si dicamus quod
operatur {\itshape per imperium alterius}. 

&

そして、作用する作用者にとって何かが原因であるとき、それはそれが目的因
 であるか、形相因であるか、作出因ないし動因であるかのいずれかである。
目的因であるのは、たとえば私たちが「技術者が報酬への欲求によって働く」と
 述べる場合であり、形相因なのはたとえば「彼の技術によって働く」と述べ
 る場合であり、また動因であるのは「彼が他の人の命令によって働く」と述
 べる場合である。

\\

Quandoque vero dictio causalis cui adiungitur haec praepositio per,
est causa actionis secundum quod terminatur ad factum; ut cum dicimus,
{\itshape artifex operatur per martellum}. Non enim significatur quod martellus
sit causa artifici quod agat, sed quod sit causa artificiato ut ab
artifice procedat; et quod hoc ipsum habeat ab artifice. 

&

他方で、「通して・よって」というこの前置詞が結び付けられる原因的な語り
 が、為されたことへ終局するかぎりにおける作用の原因である。たとえば私
 たちが「技術者が小槌によって働く」と言う場合のように。つまり小槌が作
 用する技術者の原因なのではなく、技術者から発出するものとしての技術作
 品にとっての原因であるということが表示されていて、(小槌は)そのこと
 自体を技術者によって所有する。


\\

Et hoc est
quod quidam dicunt, quod haec praepositio per quandoque notat
auctoritatem in recto, ut cum dicitur, {\itshape rex operatur per ballivum},
quandoque autem in obliquo, ut cum dicitur, {\itshape ballivus operatur per
regem}. 


&

そしてこのことが、ある人々が、この「通して・よって」という前置詞がたとえば「王が役人によって働く」と
 言われる場合のように表の意味で権威を示すが、「役人が王によって働く」
 と言われる場合のように、裏の意味でそれを示す場合がある、と言うことで
 ある。

\\

Quia igitur Filius habet a Patre quod ab eo procedat 
Spiritus Sanctus, potest dici quod Pater per Filium spirat  Spiritum
Sanctum; vel quod Spiritus Sanctus procedat a Patre per
Filium, quod idem est.

&

ゆえに、息子は父から、彼から聖霊が発出するということをもつのだから、父
 が息子を通して聖霊を霊発するとも言えるし、聖霊は父から息子を通して発
 出するとも言え、これは同じことである。

\\



{\scshape Ad primum ergo dicendum} quod in qualibet actione est duo considerare,
scilicet suppositum agens, et virtutem qua agit; sicut ignis calefacit
calore. Si igitur in Patre et Filio consideretur virtus qua spirant
Spiritum Sanctum, non cadit ibi aliquod medium, quia haec virtus est
una et eadem. 

&

第一異論に対してはそれゆえ以下のように言われるべきである。
どんな作用においても二つのことを考察することができるのであり、すなわち
 それは作用する基体と\kenten{それ}によって作用するところの\kenten{その
 力}とである。たとえば火は熱によって熱する。ゆえに父と息子において
 \kenten{それ}によって聖霊を霊発するところの力が考察されるならば、そこ
 に何らの媒介も入らない。なぜならこの力は一つであり同一だからである。

\\

Si autem considerentur ipsae personae spirantes, sic,
cum  Spiritus Sanctus communiter procedat a Patre et Filio,
invenitur Spiritus Sanctus immediate a Patre procedere,
inquantum est ab eo; et mediate, inquantum est a Filio. 

&

他方、霊発するペルソナ自身が考察されるならば、その場合には、聖霊は父と
 息子から共通に発出するので、聖霊は父からである限りにおいて直接的に父
 から発出し、息子からである限りにおいて間接的に父から発出する。

\\

Et sic dicitur
procedere a Patre per Filium. Sicut etiam Abel processit immediate ab
Adam, inquantum Adam fuit Pater eius; et mediate, inquantum Eva fuit
mater eius, quae processit ab Adam; licet hoc exemplum materialis
processionis ineptum videatur ad significandam immaterialem
processionem divinarum personarum.

&

そしてこの意味で、父から息子を通して発出すると言われる。ちょうど、アベ
ルもまた、アダムが彼の父であるかぎりで直接的にアダムから発出し、アダム
から発出したエワが彼の母であるかぎりで、間接的にアダムから発出したよう
にである。ただしこの質料的な発出の例は、神のペルソナの非質料的な発出を
表示するには不十分であると思われる。

\\



{\scshape Ad secundum dicendum} quod, si Filius acciperet a Patre aliam virtutem
numero ad spirandum  Spiritum Sanctum, sequeretur quod esset sicut
causa secunda et instrumentalis, et sic magis procederet a Patre quam
a Filio. Sed una et eadem numero virtus spirativa est in Patre et
Filio, et ideo aequaliter procedit ab utroque. Licet aliquando dicatur
principaliter vel proprie procedere de Patre, propter hoc quod Filius
habet hanc virtutem a Patre.

&

第二異論に対しては以下のように言われるべきである。
もし息子が父から、聖霊を霊発するための数的に別の力を父から受け取ってい
たならば、息子は第二原因や道具的な原因ということになり、聖霊は息子よ
りも父から発出したことになっただろう。しかし父と息子には数的に一つで同
 一の霊発力があるのであって、それゆえ聖霊は両者から等しく発出する。
ただし息子がこの力を父から持つために、主要にあるいは固有に父から発出す
 ると言われることはある。


\\



{\scshape Ad tertium dicendum} quod, sicut generatio Filii est coaeterna
generanti, unde non prius fuit Pater quam gigneret Filium; ita
processio  Spiritus Sancti est coaeterna suo principio. Unde non fuit
prius Filius genitus, quam Spiritus Sanctus procederet,
sed utrumque aeternum est.

&

第三異論に対しては以下のように言われるべきである。
息子の生成が生成する者と同じ永遠であり、したがって父が息子を生むより先
 にいたのではないように、聖霊の発出も、自らの根源と同じ永遠である。し
 たがって、聖霊が発出したより先に息子が生まれたのではなく両者とも永遠
 である。

\\



{\scshape Ad quartum dicendum} quod, cum aliquis dicitur per aliquid operari, non
semper recipitur conversio, non enim dicimus quod martellus operetur
per fabrum. Dicimus autem quod ballivus operatur per regem, quia
ballivi est agere, cum sit dominus sui actus. Martelli autem non est
agere, sed solum agi, unde non designatur nisi ut
instrumentum. 

&

第四異論に対しては以下のように言われるべきである。
ある人がある人を通して働くと言われるとき、常に逆が成立するわけではない。
 たとえば私たちは小槌が工作人を通して働くとは言わない。しかし役人が王
 を通して働くとは言う。それは作用することが役人に属しているからである。
 つまり役人は自らの作用の主人である。これに対して小槌に作用することは
 属さず、ただ作用されることだけが属する。従って、それは道具としてしか
 指示されない。

\\

Dicitur autem ballivus operari per regem, quamvis haec
praepositio per denotet medium, quia, quanto suppositum est prius in
agendo, tanto virtus eius est immediatior effectui, quia virtus causae
primae coniungit causam secundam suo effectui, unde et prima principia
dicuntur immediata in demonstrativis scientiis. 

&

対照的に、たしかにこの「によって」という前置詞は媒介を指示するけれども、
 役人は王によって働くと言われる。なぜなら、作用することにおいて固体が
 より先のものであればあるほど、それだけいっそう彼の力は結果に直接的で
 ある。なぜなら、第一原因の力が自らの結果にとって第二原因に結合するか
 らである。このことから論証的な諸学において第一諸原理は直接的と言われ
 る。

\\

Sic igitur, inquantum
ballivus est medius secundum ordinem suppositorum agentium, dicitur
rex operari per ballivum, secundum ordinem vero virtutum, dicitur
ballivus operari per regem, quia virtus regis facit quod actio ballivi
consequatur effectum. Ordo autem non attenditur inter Patrem et Filium
quantum ad virtutem; sed solum quantum ad supposita. Et ideo dicitur
quod Pater spirat per Filium, et non e converso.

&

ゆえに、以上のようにして役人が作用する個体の秩序に従って媒介であるかぎ
 りで、王が役人を通して働くと言われるが、力の秩序に従っては、役人が王
 によって働くと言われる。なぜなら王の力が、役人の作用が結果をもたらす
 ようにさせるからである。しかし父と息子のあいだに秩序は力に関して見出
 されるのではなくたんに個体に関して見出される。ゆえに父は息子を通して
 霊発すると言われるが、逆は言われない。

\\


\end{longtable}
\newpage

\rhead{a.~4}
\begin{center}
{\Large {\bf ARTICULUS QUARTUS}}\\
{\large UTRUM PATER ET FILIUS SINT UNUM PRINCIPIUM SPIRITUS SANCTI}\\
{\footnotesize I {\itshape Sent.}, d.11, a.2, 4; d.29, a.4; IV
 {\itshape SCG}, cap.25.}\\
{\Large 第四項\\父と息子は聖霊の一つの根源か}
\end{center}

\begin{longtable}{p{21em}p{21em}}

{\scshape Ad quartum sic proceditur}. Videtur quod Pater et Filius non sint unum
principium  Spiritus Sancti. Quia {\itshape  Spiritus Sanctus} non
videtur a Patre et Filio procedere inquantum sunt unum, neque in
natura, quia Spiritus Sanctus sic etiam procederet a
seipso, qui est unum cum eis in natura; neque etiam inquantum sunt
unum in aliqua proprietate, quia una proprietas non potest esse duorum
suppositorum, ut videtur. Ergo Spiritus Sanctus procedit a
Patre et Filio ut sunt plures. Non ergo Pater et Filius sunt unum
principium  Spiritus Sancti.

&

第四項の問題へ議論は以下のように進められる。
父と息子は聖霊の一つの根源ではないと思われる。理由は以下の通り。
聖霊は父と息子が本性において一であるかぎりにおいて父と息子から発出するのではないと
 思われる。なぜなら、そのようにしてであれば、聖霊は父と息子と本性にお
 いて一である自ら自身からも発出したであろうから。また父と息子が何らかの固有性に
 おいて一である限りにおいて両者から発出したのでもない。なぜなら一つの
 固有性が二つの個体に属することはありえないと思われるからである。ゆえ
 に聖霊は父と息子が複数である限りにおいて両者から発出する。ゆえに父と
 息子は聖霊の一つの根源ではない。

\\



2. {\scshape Praeterea}, cum dicitur, {\itshape Pater et Filius sunt unum principium  Spiritus
Sancti}, non potest ibi designari unitas personalis, quia sic Pater et
Filius essent una persona. Neque etiam unitas proprietatis, quia si
propter unam proprietatem Pater et Filius sunt unum principium
Spiritus Sancti, pari ratione, propter duas proprietates Pater videtur
esse duo principia Filii et  Spiritus Sancti; quod est
inconveniens. Non ergo Pater et Filius sunt unum principium  Spiritus
Sancti.

&

さらに「父と息子は聖霊の一つの根源である」と言われるとき、そこでペル
 ソナ的一性は示されえない。なぜならもし示されているとすると父と息子は
 一つのペルソナだっただろう。さらに固有性の一性も示されえない。なぜな
 らもし一つの固有性のために父と息子が聖霊の一つの根源であるならば、同
 じ理由によって二つの固有性のために父は息子と聖霊の二つの根源であるよ
 うに思われる。これは不都合である。ゆえに父と息子は聖霊の一つの根源で
 はない。

\\



3. {\scshape Praeterea}, Filius non magis convenit cum Patre quam {\itshape
Spiritus Sanctus}. Sed {\itshape  Spiritus Sanctus} et Pater non sunt
unum principium respectu alicuius divinae personae. Ergo neque Pater
et Filius.

&

さらに、息子は聖霊以上に父に適合するわけではない。しかし聖霊と父は何ら
 かの神のペルソナにかんして一つの根源ではない。ゆえに父と息子もそうで
 はない。

\\



4. {\scshape Praeterea}, si Pater et Filius sunt unum principium  Spiritus Sancti
aut unum quod est Pater; aut unum quod non est Pater. Sed neutrum est
dare, quia si unum quod est Pater, sequitur quod Filius sit Pater; si
unum quod non est Pater, sequitur quod Pater non est Pater. Non ergo
dicendum est quod Pater et Filius sint unum principium  Spiritus
Sancti.

&

さらに、もし父と息子が聖霊の一つの根源であるならば、その一つのものは父
であるか父でないかのいずれかである。しかしどちらも認められない。なぜな
らもし父なら息子が父であることになるし、父でないなら父は父でないことに
なるからである。ゆえに父と息子が聖霊の一つの根源だと言われるべきではな
い。

\\



5. {\scshape Praeterea}, si Pater et Filius sunt unum principium  Spiritus Sancti
videtur e converso dicendum quod unum principium  Spiritus Sancti sit
Pater et Filius. Sed haec videtur esse falsa, quia hoc quod dico
{\itshape principium}, oportet quod supponat vel pro persona Patris, vel pro
persona Filii; et utroque modo est falsa. Ergo etiam haec est falsa,
{\itshape Pater et Filius sunt unum principium  Spiritus Sancti}.

&

さらに、父と息子が聖霊の一つの根源であるならば、聖霊の一つの根源は父と
息子であると逆にも言われるべきである。しかしこれは偽であると思われる。
なぜなら「根源」と私が言うものは、父のペルソナとして想定するか、息子の
ペルソナとして想定さするかでなければならないが、どちらの想定においても
その言い換えは偽だからである。ゆえに「父と息子が聖霊の一つの根源である」
は偽である。

\\



6. {\scshape Praeterea}, unum in substantia facit idem. Si igitur Pater et Filius
sunt unum principium  Spiritus Sancti, sequitur quod sint {\itshape idem
principium}. Sed hoc a multis negatur. Ergo non est concedendum quod
Pater et Filius sint unum principium  Spiritus Sancti.

&

さらに、実体における一は同一を生む。ゆえにもし父と息子が聖霊の一つの根
 源であるならば、それらは「同一の根源」であることになるだろう。しかし
 これは多くの人たちによって否定されている。ゆえに父と息子が聖霊の一つ
 の根源であることが認められるべきではない。

\\



7. {\scshape Praeterea}, Pater et Filius et {\itshape  Spiritus Sanctus}, quia sunt
unum principium creaturae, dicuntur esse unus creator. Sed Pater et
Filius non sunt unus spirator, sed {\itshape duo spiratores}, ut a multis
dicitur. Quod etiam consonat dictis Hilarii, qui dicit, in II de
Trin., quod {\itshape  Spiritus Sanctus} a Patre et Filio auctoribus
confitendus est. Ergo Pater et Filius non sunt unum principium
Spiritus Sancti.

&

さらに、父と息子と聖霊は、被造物の一つの根源なので一人の創造者と言われ
 る。しかるに父と息子は一人の霊発者ではなく、多くの人たちによって言わ
 れるように「二人の霊発者」である。このことは『三位一体論』第2巻で、聖
 霊は「諸作者である父と息子とからであることを告白されるべき」と述べる
 ヒラリウスにも一致する。ゆえに父と息子は聖霊の一つの根源ではない。

\\



{\scshape Sed contra est} quod Augustinus dicit, in V de Trin., quod Pater et
Filius non sunt duo principia, sed unum principium  Spiritus Sancti.

&

しかし反対にアウグスティヌスは『三位一体論』第5巻で父と息子は聖霊の二つの根
 源ではなく一つの根源であると述べている。

\\



{\scshape Respondeo dicendum} quod Pater et Filius in omnibus unum sunt, in
quibus non distinguit inter eos relationis oppositio. Unde, cum in hoc
quod est esse principium Spiritus Sancti, non opponantur relative,
sequitur quod Pater et Filius sunt unum principium  Spiritus
Sancti. 

&

解答する。以下のように言われるべきである。
父と息子は、対立関係がそれらのあいだを区別しないようなあらゆるところで一
である。したがって、聖霊の根源であるという点ではそれらは関係的に対立しな
 いのだから、父と息子が聖霊の一つの根源であることが帰結する。

\\



Quidam tamen dicunt hanc esse impropriam, {\itshape Pater et Filius sunt
unum principium  Spiritus Sancti}. Quia cum hoc nomen principium,
singulariter acceptum, non significet personam, sed proprietatem,
dicunt quod sumitur adiective, et quia adiectivum non determinatur per
adiectivum, non potest convenienter dici quod Pater et Filius sint
unum principium Spiritus Sancti, nisi unum intelligatur quasi
adverbialiter positum, ut sit sensus, {\itshape sunt unum principium}, idest {\itshape uno
modo}. Sed simili ratione posset dici Pater duo principia Filii et
Spiritus Sancti, idest duobus modis. 


&

しかしある人々は、「父と息子が聖霊の一つの根源である」という命題が不適
当であると言っている。その理由は、「根源」というこの名称は単数で理解さ
れていて、それはペルソナではなく固有性を表示して形容詞的に解釈されるが、
形容詞は形容詞によって限定されないので、父と息子が聖霊の一つの根源であ
るということは、この「一つの」が副詞的に置かれたものと理解され、その意
味が「一つの根源である」すなわち「一つのしかたによって」となるのでない
限り、適切に言われえないと彼らはいう。しかし(もしこれが正しいなら)同
じ理由で父は息子と聖霊の二つの根源、すなわち「二つのしかたによって」根
源だと言えることになってしまう。

\\

Dicendum est ergo quod, licet hoc
nomen principium significet proprietatem, tamen significat eam per
modum substantivi, sicut hoc nomen Pater vel Filius etiam in rebus
creatis. 
Unde numerum accipit a forma significata, sicut et alia
substantiva. Sicut igitur Pater et Filius sunt unus Deus, propter
unitatem formae significatae per hoc nomen Deus; ita sunt unum
principium Spiritus Sancti, propter unitatem proprietatis
significatae in hoc nomine principium.

&

それゆえ以下のように言われるべきである。この「根源」という名称は固有性
を表示するが、それを名詞のしかたで表示する。それはちょうど、「父」
や「息子」もまた被造物においてそうであるのと同様である。したがって、他
の名詞と同じように、表示された形相から数を受け取る。ゆえに、ちょうど
「神」というこの名称によって表示された形相の一性のために、父と息子は一人
の神であるように、「根源」というこの名称において表示された固有性の
一性のために、(父と息子は)聖霊の一つの根源である。


\\



{\scshape Ad primum ergo dicendum} quod, si attendatur virtus spirativa,
Spiritus Sanctus procedit a Patre et Filio inquantum sunt
unum in virtute spirativa, quae quodammodo significat naturam cum
proprietate, ut infra dicetur. Neque est inconveniens unam
proprietatem esse in duobus suppositis, quorum est una natura. Si vero
considerentur supposita spirationis, sic {\itshape  Spiritus Sanctus}
procedit a Patre et Filio ut sunt plures, procedit enim ab eis ut amor
unitivus duorum.

&

第一異論に対してはそれゆえ以下のように言われるべきである。もし霊発の力
が注目されるならば聖霊は父と息子からそれらが霊発の力において一であるか
ぎりにおいて発出する。この力はある意味で、後で述べられるとおり、固有性
とともに本性を表示する。また、一つの固有性が二つの個体においてあること
は、それらに一つの本性が属するのであれば不都合ではない。他方、霊発する
個体が考察されるならば、その場合に聖霊が複数としての父と息子から発出す
る。なぜならそれらから、二つのものを合一させうる愛として発出するからで
ある。

\\



{\scshape Ad secundum dicendum} quod, cum dicitur, Pater et Filius sunt unum
principium  Spiritus Sancti, designatur una proprietas, quae est forma
significata per nomen. Non tamen sequitur quod propter plures
proprietates possit dici Pater plura principia, quia implicaretur
pluralitas suppositorum.

&

第二異論に対しては以下のように言われるべきである。
父と息子が聖霊の一つの根源であると言われるとき一つの固有性が示される
が、それは名称によって表示された形相である。しかしこのことから、複数の
 固有性のために父が複数の根源であるとは言われえない。なぜならもしそう
 なら個体の複数性が意味されるだろうからである。

\\



{\scshape Ad tertium dicendum} quod secundum relativas proprietates non
attenditur in divinis similitudo vel dissimilitudo, sed secundum
essentiam. Unde, sicut Pater non est similior sibi quam Filio, ita nec
Filius similior Patri quam Spiritus Sanctus.

&

第三異論に対しては以下のように言われるべきである。
関係的な固有性において神の中に類似や不類似は見出されず、それが見出され
 るのは本質においてである。したがって、父が息子よりも自分に似ているの
 ではないように、息子も聖霊より父に似ているわけではない。

\\



{\scshape Ad quartum dicendum} quod haec duo, scilicet, Pater et Filius sunt unum
principium quod est Pater, aut, unum principium quod non est Pater,
non sunt contradictorie opposita. Unde non est necesse alterum eorum
dare. Cum enim dicimus, Pater et Filius sunt unum principium, hoc quod
dico principium, non habet determinatam suppositionem, imo confusam
pro duabus personis simul. Unde in processu est fallacia figurae
dictionis, a confusa suppositione ad determinatam.

&

第四異論に対しては以下のように言われるべきである。このふたつ、すなわち
「父と息子は父である一つの根源である」と「父と息子は父でない一つの根源
である」は矛盾対立しないので、これらのどちらかが成立することは必然では
ない。じっさい、私たちが「父と息子は一つの根源である」と言うとき、私が
「根源」と言うものは限定された個体をもたずむしろ同時に二つのペルソナで
あるものとして渾然とした個体をもつ。それゆえ異論の推論には、渾然とした
個体から限定された個体へといった「語形に由来する誤謬」がある。


\\



{\scshape Ad quintum dicendum} quod haec etiam est vera, unum principium
Spiritus Sancti est Pater et Filius. Quia hoc quod dico principium non
supponit pro una persona tantum, sed indistincte pro duabus, ut dictum
est.

&

第五異論に対しては以下のように言われるべきである。この「聖霊の一つの根
源は父と息子である」も真である。なぜならすでに言われたように私が「根源」
と言うものは単に一人のペルソナだけを示すのではなく、二人のペルソナを区
別せずに示しているからである。



\\



{\scshape Ad sextum dicendum} quod convenienter potest dici quod Pater et Filius
sunt idem principium, secundum quod ly principium supponit confuse et
indistincte pro duabus personis simul.

&

第六異論に対しては以下のように言われるべきである。「父と息子は同一の根
 源である」ということは適切に言われうる。ただしこの「根源」が二人のペ
 ルソナを同時に渾然と区別せずに示す限りにおいて。


\\



{\scshape Ad septimum dicendum} quod quidam dicunt quod Pater et Filius, licet
sint unum principium  Spiritus Sancti, sunt tamen duo spiratores,
propter distinctionem suppositorum, sicut etiam duo spirantes, quia
actus referuntur ad supposita. Nec est eadem ratio de hoc nomine
creator. 

&

第七異論に対しては以下のように言われるべきである。ある人々は父と息子が
聖霊の一つの根源であるとしても、個体の区別のために二人の霊発者であると
言う。それはちょうど、二人の霊発する者という意味であり、作用は個体に関
係づけられるからそう言われる。しかし「創造者」というこの名称について同
じことは言えない。

\\

Quia Spiritus Sanctus procedit a Patre et Filio
ut sunt duae personae distinctae, ut dictum est, non autem creatura
procedit a tribus personis ut sunt personae distinctae, sed ut sunt
unum in essentia. 


&

なぜならすでに述べられたとおり聖霊は父と息子から区別された二人のペルソ
ナとして発出するが、被造物は区別されたペルソナとしての三人のペルソナか
らではなく、本質において一で或るものとしての三人のペルソナから発出する
からである。

\\


Sed videtur melius dicendum quod, quia spirans
adiectivum est, spirator vero substantivum, possumus dicere quod Pater
et Filius sunt duo spirantes, propter pluralitatem suppositorum; non
autem duo spiratores, propter unam spirationem. 


&

しかしこれよりもよいしかたで以下のように言われるべきである。すなわち、
「霊発する」は形容詞だが「霊発者」は名詞なので\footnote{ここでspirans
は分詞の名詞的用法で「霊発するもの」を意味する。分詞は名詞を修飾するか
ぎりで形容詞(nomen adjectivum)と見なせる。これに対してspiratorは同じ
spiroに由来するが、文法的には名詞(nomen substantivum)として理解されて
いる。}、個体の複数性のために私たちは父と息子が二人の霊発する者だとは
言えるが、そこには一つの霊発があるから二人の霊発者とは言えない。

\\


Nam adiectiva nomina
habent numerum secundum supposita, substantiva vero a seipsis,
secundum formam significatam. Quod vero Hilarius dicit, quod 
Spiritus Sanctus est a Patre et Filio auctoribus, exponendum est quod
ponitur substantivum pro adiectivo.

&

なぜなら形容詞は個体に即して数を持つのに対して名詞は表示された形相に即
して自ら自身によって数を持つからである。また、ヒラリウスが「聖霊は諸作
者である父と息子から」と述べていることは、形容詞の代用として名詞が置か
 れていると説明されるべきである。


\end{longtable}
\end{document}