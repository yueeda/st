\documentclass[10pt]{jsarticle} % use larger type; default would be 10pt
%\usepackage[utf8]{inputenc} % set input encoding (not needed with XeLaTeX)
%\usepackage[round,comma,authoryear]{natbib}
%\usepackage{nruby}
\usepackage{okumacro}
\usepackage{longtable}
%\usepqckage{tablefootnote}
\usepackage[polutonikogreek,english,japanese]{babel}
%\usepackage{amsmath}
\usepackage{latexsym}
\usepackage{color}

%----- header -------
\usepackage{fancyhdr}
\lhead{{\it Summa Theologiae} I-IIae, q.~23}
%--------------------

\bibliographystyle{jplain}

\title{{\bf PRIMA SECUNDAE}\\{\HUGE Summae Theologiae}\\Sancti Thomae
Aquinatis}
\author{Japanese translation\\by Yoshinori {\sc Ueeda}}
\date{Last modified \today}


%%%% コピペ用
%\rhead{a.~}
%\begin{center}
% {\Large {\bf }}\\
% {\large }\\
% {\footnotesize }\\
% {\Large \\}
%\end{center}
%
%\begin{longtable}{p{21em}p{21em}}
%
%&
%
%
%\\
%\end{longtable}
%\newpage







\begin{document}
\maketitle
\pagestyle{fancy}

\rhead{Prologos}

\begin{center}
 {\Large {\bf QUAESTIO VESTIMATERTIA}}\\
 {\large DE DIFFERENTIA PASSIONUM AB INVICEM}\\

 {\Large 第二十三問\\情念相互の差異について}
\end{center}

\begin{longtable}{p{21em}p{21em}}

Deinde considerandum est de passionum differentia ad invicem. Et circa
 hoc quaeruntur quatuor.

\begin{enumerate}
 \item utrum passiones quae sunt in concupiscibili, sint diversae ab his quae sunt in irascibili.
 \item utrum contrarietas passionum irascibilis\footnote{Leo版に従う。
       Busa版では`irascibli'。} sit secundum contrarietatem boni et mali.
 \item utrum sit aliqua passio non habens contrarium.
 \item utrum sint aliquae passiones differentes specie, in eadem potentia, non contrariae ad invicem.
\end{enumerate}
 
&

続いて情念相互の差異について考察されるべきである。これをめぐっては四つの
 ことが問われる。

 \begin{enumerate}
  \item 欲情の情念は怒情の情念と異なるか。
	\footnote{それぞれ「欲情」「怒情」と訳したconscibilisと
	irascibilisは、{\itshape ST} I, q.59, a.1, ad 2; a.4; q.81;
	q.82などで既出。感覚的欲求下位区分で二つの種を形成する。}
  \item 怒情の諸情念の反対性は善と悪の反対性に応じてあるか。
  \item 反対のものを持たない情念があるか。
  \item 同一の能力の中に種において異なりながら相互に反対しない情念がある
	か。
 \end{enumerate}

\end{longtable}


\newpage
\rhead{a.~1}
\begin{center}
{\Large {\bf ARTICULUS PRIMUS}}\\
{\large UTRUM PASSIONES QUAE SUNT IN CONCUPISCIBILI, SINT DIVERSAE AB
 HIS QUAE SUNT IN IRASCIBILI}\\
{\footnotesize {\itshape De Verit.}, q.26, a.4.}\\
{\Large 第一項\\怒情の情念は欲情の情念と異なるか}
\end{center}

\begin{longtable}{p{21em}p{21em}}
{\scshape Ad primum sic proceditur}. Videtur quod passiones eaedem sint in
irascibili et in concupiscibili. Dicit enim philosophus, in II {\itshape Ethic}.,
quod passiones animae sunt quas {\itshape sequitur gaudium et tristitia}. Sed
gaudium et tristitia sunt in concupiscibili. Ergo omnes passiones sunt
in concupiscibili. Non ergo sunt aliae in irascibili, et aliae in
concupiscibili.


&

第一項の問題へ、議論は以下のように進められる。
怒情と欲情においてある情念は同一であると思われる。理由は
 以下の通り。『ニコマコス倫理学』第2巻で哲学者は、魂の情念とは「喜びと悲
 しみに伴う」ものだと述べている。しかるに喜びと悲しみは欲情にお
 いてある。ゆえに、ある情念が怒情において、別の情念が欲情
においてあるわけではない。


\\



2. {\scshape Praeterea}, Matth.~{\scshape xiii}, super illud, {\itshape simile est regnum caelorum ferment}o
etc., dicit Glossa Hieronymi, {\itshape in ratione possideamus prudentiam, in
irascibili odium vitiorum, in concupiscibili desiderium virtutum}. Sed
odium est in concupiscibili, sicut et amor, cui contrariatur, ut dicitur
in II {\itshape Topic}. Ergo eadem passio est in concupiscibili et irascibili.


&

次に、かの「天の王国はパン種に似ている」\footnote{「また、別のたとえをお
 話しになった。「天の国は、パン種に似ている。女がこれを取って三サトンの
 小麦粉に混ぜると、やがて全体が膨らむ」」(13:33)}『マタイによる福音書』
 第13章を注解してヒエロニュムスの『注解』は「私たちは理性の中に思慮を、
 怒情の中に悪徳への憎しみを、欲情の中に美徳への願望を持
 つ」と述べている。
しかるに、『トピカ』第2巻で言われるように、憎しみは愛と同様に欲情
の中にあり、愛と反対のものとされる。ゆえに同じ情念が欲情と怒
 情の中にある。


\\


3. {\scshape Praeterea}, passiones et actus differunt specie secundum obiecta. Sed
passionum irascibilis et concupiscibilis eadem obiecta sunt, scilicet
bonum et malum. Ergo eaedem sunt passiones irascibilis et
concupiscibilis.


&

さらに、情念(受動)と作用(能動)は対象に即して種において異なる。しかる
 に諸情念の中で、怒情と欲情には同一の対象、すなわち善と
 悪が属する。ゆえに欲情と怒情に同一の諸情念が属する。


\\



{\scshape Sed contra}, diversarum potentiarum actus sunt specie diversi, sicut
videre et audire. Sed irascibilis et concupiscibilis sunt duae potentiae
dividentes appetitum sensitivum, ut in primo dictum est. Ergo, cum
passiones sint motus appetitus sensitivi, ut supra dictum est, passiones
quae sunt in irascibili, erunt aliae secundum speciem a passionibus quae
sunt in concupiscibili.


&

しかし反対に、異なる能力の作用は種において異なる。たとえば見ることと聞く
 ことのように。しかるに、第一部で\footnote{{\itshape ST} I, q.81, a.2.}述べられたように怒情と欲情
は感覚的欲求を分ける二つの能力である。ゆえに、すでに述べられたとお
 り\footnote{{\itshape ST} I-II, q.22, a.3.}情念は感覚的欲求の運動なのだから、怒情的なものの中にある情念は、欲求
 的なものの中にある情念と種において異なることになる。

\\


{\scshape Respondeo dicendum} quod passiones quae sunt in irascibili et in
concupiscibili, differunt specie. Cum enim diversae potentiae habeant
diversa obiecta, ut in primo dictum est, necesse est quod passiones
diversarum potentiarum ad diversa obiecta referantur. 



&

解答する。以下のように言われるべきである。
怒情と欲情にある情念は種において異なる。理由は以下の通り。
第一部で述べられたとおり\footnote{{\itshape ST}I, q.77, a.3.}、異なる能
 力は異なる対象を持つので、異なる能力の情念(受動)が異なる対象へと関係づけられ
 ることは必然である。


\\

Unde multo magis
passiones diversarum potentiarum specie differunt, maior enim
differentia obiecti requiritur ad diversificandam speciem potentiarum,
quam ad diversificandam speciem passionum vel actuum. Sicut enim in
naturalibus diversitas generis consequitur diversitatem potentiae
materiae, diversitas autem speciei diversitatem formae in eadem materia;
ita in actibus animae, actus ad diversas potentias pertinentes, sunt non
solum specie, sed etiam genere diversi; actus autem vel passiones
respicientes diversa obiecta specialia comprehensa sub uno communi
obiecto unius potentiae, differunt sicut species illius generis. 


&

したがってそれ以上に、異なる能力の情念は種において異なる。なぜなら、受動
 や作用の種を分けるには、能力の種を分ける以上の対象の異なりが必要だから
 である。ちょうど自然物において、類の異なりは質料の可能態(能力)の異な
 りに伴うのに対して、種の異なりはその同じ質料における勁草の異なりに伴う
 ように、魂の作用において、異なる能力に属する作用は、種においてだけでな
 く類においても異なるが、一つの能力の共通な一つの対象の元に含まれる異な
 る特殊な対象に関係する作用や受動は、その類に属する諸々の種のようにして異なる。


\\

Ad
cognoscendum ergo quae passiones sint in irascibili, et quae in
concupiscibili, oportet assumere obiectum utriusque potentiae. Dictum
est autem in primo quod obiectum potentiae concupiscibilis est bonum vel
malum sensibile simpliciter acceptum, quod est delectabile vel
dolorosum. Sed quia necesse est quod interdum anima difficultatem vel
pugnam patiatur in adipiscendo aliquod huiusmodi bonum, vel fugiendo
aliquod huiusmodi malum, inquantum hoc est quodammodo elevatum supra
facilem potestatem animalis; ideo ipsum bonum vel malum, secundum quod
habet rationem ardui vel difficilis, est obiectum
irascibilis. 


&

ゆえに、どの情念が怒情の中にあり、どの情念が欲情の中にあるかを
 知るためには、両方の能力の対象を想定しなければならない。
ところで、第一部で\footnote{{\itshape ST}I, q.81, a.2.}、欲情的能力の対
 象は端的に受け取られた可感的な善と悪、つまり快と苦だと述べられた。
しかるに、時として魂はこのような何らかの善を獲得したり悪を避けたりするときに困難や争いを受
 動することが必ずあるので、というのもそれがある点で動物的な容易な力を越
 えているかぎりでそうなるのだが、それゆえ、このような善や悪は、つらさや
 困難さの性格を持つかぎりで、怒情的なものの対象である。


\\

Quaecumque ergo passiones respiciunt absolute bonum vel
malum, pertinent ad concupiscibilem; ut gaudium, tristitia, amor, odium,
et similia. Quaecumque vero passiones respiciunt bonum vel malum sub
ratione ardui, prout est aliquid adipiscibile vel fugibile cum aliqua
difficultate, pertinent ad irascibilem; ut audacia, timor, spes, et
huiusmodi.


&


それゆえ、無条件的に善や悪に関係するどんな情念も欲情的なものに属する。た
 とえば、喜、悲、愛、憎などである。他方、何らかの困難さを伴って獲得され
 たり避けられたりするものとして、つらさの性格のもとに善や悪へ関係する情
 念はなんであれ怒情的なものに属する。たとえば勇気、怖れ、希望などである。

\\



{\scshape Ad primum ergo dicendum} quod, sicut in primo dictum est, ad hoc vis
irascibilis data est animalibus, ut tollantur impedimenta quibus
concupiscibilis in suum obiectum tendere prohibetur, vel propter
difficultatem boni adipiscendi, vel propter difficultatem mali
superandi. Et ideo passiones irascibilis omnes terminantur ad passiones
concupiscibilis. Et secundum hoc, etiam passiones quae sunt in
irascibili, consequitur gaudium et tristitia, quae sunt in
concupiscibili.


&

第一異論に対しては、それゆえ以下のように言われるべきである。
第一部で\footnote{Ibid.}述べられたとおり、怒情の力は、善を獲得する困難や
悪に打ち克つ困難のために、欲情が自らの対象へ向かうのがそれによっ
 て妨げられるところの障害を取り除くために、動物たちに与えられている。
ゆえに、怒情の情念はすべて欲情の情念において終息する。
 したがって、怒情の中にある情念もまた、欲情的の中にある喜
 びや悲しみに伴う。


\\



{\scshape Ad secundum dicendum} quod odium vitiorum attribuit Hieronymus
irascibili, non propter rationem odii, quae proprie competit
concupiscibili; sed propter impugnationem, quae pertinet ad irascibilem.


&

第二異論に対しては、次のように言われるべきである。ヒエロ
 ニュムスが悪徳への憎しみを怒情に帰したのは憎の性格のためにでは
 ない。憎は欲情に適合するからである。そうではなく、怒情に属する攻撃のためにである。


\\



{\scshape Ad tertium dicendum} quod bonum inquantum est delectabile, movet
concupiscibilem. Sed si bonum habeat quandam difficultatem ad
adipiscendum, ex hoc ipso habet aliquid repugnans concupiscibili. Et
ideo necessarium fuit esse aliam potentiam quae in id tenderet. Et eadem
ratio est de malis. Et haec potentia est irascibilis. Unde ex
consequenti passiones concupiscibilis et irascibilis specie differunt.


&

第三異論に対しては、次のように言われるべきである。
善は快であるかぎり欲情を動かす。しかしもし善が獲得のための何らか
 の困難さをもつなら、そのこと自体から、欲情に抵抗するものを持つ。そしてそ
 れゆえ、それへと向かう他の能力が必要であった。悪についても同様である。
 そしてこの能力が怒情である。したがって、欲情の情念と怒情の情念は
 種において異なる。




\end{longtable}
\newpage

\rhead{a.~2}
\begin{center}
{\Large {\bf ARTICULUS SECUNDUS}}\\
{\large UTRUM CONTRARIETAS PASSIONUM IRASCIBILIS SIT SECUNDUM
 CONTRARIETATEM BONI ET MALI}\\
{\footnotesize III {\itshape Sent.}, d.26, q.1, a.3; {\itshape De
 Verit.}, q.26, a.4.}\\
{\Large 第二項\\怒情的部分に属する諸情念の反対性は善悪の反対性に応じてあるか}
\end{center}

\begin{longtable}{p{21em}p{21em}}
{\scshape Ad secundum sic proceditur}. Videtur quod contrarietas passionum
irascibilis non sit nisi secundum contrarietatem boni et mali. Passiones
enim irascibilis ordinantur ad passiones concupiscibilis, ut dictum
est. Sed passiones concupiscibilis non contrariantur nisi secundum
contrarietatem boni et mali; sicut amor et odium, gaudium et
tristitia. Ergo nec passiones irascibilis.

&

第二項の問題へ議論は以下のように進められる。
怒情的部分の諸情念の反対性は善悪の反対性に応じてしかないと思われる。理由
 は以下の通り。
すでに述べられたとおり怒情の諸情念は欲情の諸情念へと秩序付けられてい
 る。しかるに欲情の諸情念は、愛と憎、喜と悲のように善悪に応じての
 み反対となる。ゆえに怒情の諸情念もそれ以外でない。

\\



2. {\scshape Praeterea}, passiones differunt secundum obiecta; sicut et motus secundum
terminos. Sed contrarietas non est in motibus nisi secundum
contrarietatem terminorum, ut patet in V {\scshape Physic}. Ergo neque in
passionibus est contrarietas nisi secundum contrarietatem
obiectorum. Obiectum autem appetitus est bonum vel malum. Ergo in nulla
potentia appetitiva potest esse contrarietas passionum nisi secundum
contrarietatem boni et mali.


&

さらに、運動が終極に応じて異なるように、情念は対象に応じて異なる。しかる
 に『自然学』第5巻で明らかなとおり、運動における反対性は終極の反対性に応
 じてしかない。ゆえに、情念においても対象の反対性に応じてでなければ反対
 性はない。しかるに欲求の対象は善と悪である。ゆえにどんな欲求能力におい
 ても善悪の反対性に応じてでなければ情念の反対性はない。


\\



3. {\scshape Praeterea}, {\itshape omnis passio animae attenditur secundum accessum et recessum},
ut Avicenna dicit, in {\itshape Sexto de Naturalibus}. Sed accessus causatur ex
ratione boni, recessus autem ex ratione mali, quia sicut {\itshape bonum est quod
omnia appetun}t, ut dicitur in I {\itshape Ethic}., ita malum est quod omnia
fugiunt. Ergo contrarietas in passionibus animae non potest esse nisi
secundum bonum et malum.


&

さらに、アヴィセンナが『自然学第六巻』で述べるように、「魂の情念はすべて
 接近と後退に即して見出される」。しかるに接近は善の性格に基づいて原因さ
 れ、後退は悪の性格に基づいて原因される。なぜなら『ニコマコス倫理学』第1
 巻で言われるように「善とは万物が欲求するところのもの」であり、悪とは万
 物が避けるところのものだからである。ゆえに魂の諸情念における反対性は善
 と悪に即してしかありえない。


\\



{\scshape Sed contra}, timor et audacia sunt contraria, ut patet in III {\itshape Ethic}. sed
timor et audacia non differunt secundum bonum et malum, quia utrumque
est respectu aliquorum malorum. Ergo non omnis contrarietas passionum
irascibilis est secundum contrarietatem boni et mali.


&

しかし反対に、『ニコマコス倫理学』第3巻で明らかなとおり、怖れと勇気は反
 対である。しかるに怖れと勇気が善と悪に即して異なるわけではない。なぜな
 らどちらも何らかの悪にかかわるからである。ゆえに怒情のすべての情念の反
 対性が善悪の反対性によるのではない。

\\



{\scshape Respondeo dicendum} quod passio quidam motus est, ut dicitur in III
{\itshape Physic}. Unde oportet contrarietatem passionum accipere secundum
contrarietatem motuum vel mutationum. Est autem duplex contrarietas in
mutationibus vel motibus, ut dicitur in V {\itshape Physic}. Una quidem secundum
accessum et recessum ab eodem termino, quae quidem contrarietas est
proprie mutationum, idest generationis, quae est mutatio ad esse, et
corruptionis, quae est mutatio ab esse. Alia autem secundum
contrarietatem terminorum, quae proprie est contrarietas motuum, sicut
dealbatio, quae est motus a nigro in album, opponitur denigrationi, quae
est motus ab albo in nigrum. 

&

解答する。以下のように言われるべきである。
『自然学』第3巻で言われるように、情念は一種の運動である。したがって情念
 の反対性は運動ないし変容の反対性に即して理解されなければならない。
ところで、『自然学』第5巻で言われるように、変容や運動には二通りの反対性
 がある。一つは、同一の終端に対する接近と後退によるもので、この反対性は
 固有の意味で変容すなわち存在への変容である生成と存在からの変容である消
 滅に属する。これに対してもう一つの反対性は、終極の反対性に応じてあり、
 これは固有の意味で運動の反対性である。たとえば黒から白への運動である白
 化は白から黒への運動である黒化に対置される。


\\


Sic igitur in passionibus animae duplex
contrarietas invenitur, una quidem secundum contrarietatem obiectorum,
scilicet boni et mali; alia vero secundum accessum et recessum ab eodem
termino. In passionibus quidem concupiscibilis invenitur prima
contrarietas tantum, quae scilicet est secundum obiecta, in passionibus
autem irascibilis invenitur utraque. 

&

ゆえにこのようにして魂の諸情念において二つの反対性が見出される。一つは対
 象の反対性すなわち善と悪に即してであり、もう一つは同一の終端に対する接
 近と後退に即してである。欲情の諸情念においては前者の反対性、すなわち対
 象に即したそれだけが見出されるが、怒情の諸情念においては両方が見出され
 る。


\\

Cuius ratio est quia obiectum
concupiscibilis, ut supra dictum est, est bonum vel malum sensibile
absolute. 
Bonum autem, inquantum bonum, non potest esse terminus ut a
quo, sed solum ut ad quem, quia nihil refugit bonum inquantum bonum, sed
omnia appetunt ipsum. Similiter nihil appetit malum inquantum huiusmodi,
sed omnia fugiunt ipsum, et propter hoc, malum non habet rationem
termini ad quem, sed solum termini a quo. Sic igitur omnis passio
concupiscibilis respectu boni, est ut in ipsum, sicut amor, desiderium
et gaudium, omnis vero passio eius respectu mali, est ut ab ipso, sicut
odium, fuga seu abominatio, et tristitia. Unde in passionibus
concupiscibilis non potest esse contrarietas secundum accessum et
recessum ab eodem obiecto. 

&

その理由は以下の通りである。前に述べられたとおり、欲情の対象は、無条件的に可感的な善や
悪である。しかるに善であるかぎりにおける善は、それから離れていく終端では
 ありえず、それへ向かっていく終端である。なぜならどんなものでも、善であ
 るかぎりでの善を避けることはなく、万物はそれを欲求するからである。同様
 に、どんなものでも、悪であるかぎりでの悪を欲求することはなく、万物はそ
 れを避ける。このために悪はそれへ向かっていく終端の性格を持たず、ただそ
 れから離れていく終端の性格を持つのみである。それゆえこのようにして、愛、
 欲望、喜のような善にかかわる欲情のすべての情念は、善へという在り方を
 し、また、憎、忌避または嫌悪、悲といった、悪にかかわる欲情の情念はすべ
 て、善から離れるという在り方をする。したがって、欲情の諸情念において、
 同一の対象に対する接近と後退に即した反対性はありえない。


\\


Sed obiectum irascibilis est sensibile bonum
vel malum, non quidem absolute, sed sub ratione difficultatis vel
arduitatis, ut supra dictum est. Bonum autem arduum sive difficile habet
rationem ut in ipsum tendatur, inquantum est bonum, quod pertinet ad
passionem spei; et ut ab ipso recedatur, inquantum est arduum vel
difficile, quod pertinet ad passionem desperationis. 


&

しかし怒情の対象は可感的な善や悪だが、前に述べられたとおり、無条件的にではなく困難やつらさとい
 う性格のもとでのそれである。つらい、あるいは困難な善は、それが善である
 限りにおいて、それへと向かうという性格を持ち、これは希望という情念に属
 する。また、それがつらく困難であるかぎりで、それから後退するという性格
 を持つが、これは絶望という情念に属する。

\\

Similiter malum
arduum habet rationem ut vitetur, inquantum est malum, et hoc pertinet
ad passionem timoris, habet etiam rationem ut in ipsum tendatur, sicut
in quoddam arduum, per quod scilicet aliquid evadit subiectionem mali,
et sic tendit in ipsum audacia. Invenitur ergo in passionibus
irascibilis contrarietas secundum contrarietatem boni et mali, sicut
inter spem et timorem, et iterum secundum accessum et recessum ab eodem
termino, sicut inter audaciam et timorem.


&

同様に、つらい悪は、それが悪である限りにおいて避けられるという性格を持つ
 が、それは怖れという情念に属する。また、何かつらいものとしてのその悪へ
 向かい、そのことによって何かが悪に従属することを避けるという性格をもつ
 が、この意味で勇気がその悪へ向かう。ゆえに怒情の諸情念の中には、希望と
 怖れのような、善と悪の反対性に即した反対性と、さらに、勇気と怖れのよう
 な同一の終端にかんする接近と後退に即した反対性とが見出される。

\\


Et per hoc patet responsio ad obiecta.

&

以上のことによって異論への解答は明らかである。

\end{longtable}
\newpage


\rhead{a.~3}
\begin{center}
{\Large {\bf ARTICULUS TERTIUS}}\\
{\large UTRUM SIT ALIQUA PASSIO ANIMAE NON HABENS CONTRARIUM}\\
{\footnotesize Infra, q.46, a.1, ad 2; III{\itshape Sent.}, d.26, q.1,
 a.3; {\itshape De Verit.}, q.26, a.4.}\\
{\Large 第三項\\魂の何らかの情念は反対のものをもたないか}
\end{center}

\begin{longtable}{p{21em}p{21em}}



{\scshape Ad tertium sic proceditur}. Videtur quod omnis passio animae habeat
 aliquid contrarium. Omnis enim passio animae vel est in irascibili vel
 in concupiscibili, sicut supra dictum est. Sed utraeque passiones
 habent contrarietatem suo modo. Ergo omnis passio animae habet
 contrarium.

&

第三の問題へ議論は以下のように進められる。すべての魂の情念は反対のものを
 もつと思われる。理由は以下の通り。前に述べられたとおり、魂のすべての情念は怒情か欲情において
 ある。しかるにちらの情念もそのあり方において反対性をもつ。ゆえに魂のすべて
 の情念は反対のものをもつ。

\\



{\scshape Praeterea}, omnis passio animae habet vel bonum vel malum pro obiecto,
 quae sunt obiecta universaliter appetitivae partis. Sed passioni cuius
 obiectum est bonum, opponitur passio cuius obiectum est malum. Ergo
 omnis passio habet contrarium.

&

さらに、魂のすべての情念は対象として善か悪をもつが、これらは普遍的に欲求
 的部分の対象である。しかるにその対象が善である情念には、その対象が悪で
 ある情念が対立する。ゆえにすべての情念は反対のものをもつ。

\\


{\scshape Praeterea}, omnis passio animae est secundum accessum vel secundum
 recessum, ut dictum est. Sed cuilibet accessui contrariatur recessus,
 et e converso. Ergo omnis passio animae habet contrarium.

&

さらに、魂のすべての情念は、前に述べられたとおり、接近または後退に即して
 ある。しかるにどんな接近にも後退が反し、その逆もまた然りである。ゆえに、
 魂のすべての情念は反対のものをもつ。



\\



{\scshape Sed contra}, ira est quaedam passio animae. Sed nulla passio ponitur
 contraria irae, ut patet in IV {\itshape Ethic}. Ergo non omnis passio habet
 contrarium.

&

しかし反対に、怒りは魂の情念の一種である。しかるに『ニコマコス倫理学』第
 4巻で明らかなとおり、どんな情念も怒りに対置されない。ゆえに全ての情念が
 反対のものをもつわけではない。
 い

\\



{\scshape Respondeo dicendum} quod singulare est in passione irae, quod non potest
 habere contrarium, neque secundum accessum et recessum, neque secundum
 contrarietatem boni et mali. Causatur enim ira ex malo difficili iam
 iniacente. Ad cuius praesentiam, necesse est quod aut appetitus
 succumbat, et sic non exit terminos tristitiae, quae est passio
 concupiscibilis, aut habet motum ad invadendum malum laesivum, quod
 pertinet ad iram. Motum autem ad fugiendum habere non potest, quia iam
 malum ponitur praesens vel praeteritum. Et sic motui irae non
 contrariatur aliqua passio secundum contrarietatem accessus et
 recessus. 

&

解答する。以下のように言われるべきである。怒りという情念において、接近に
 おいても後退においても、また善と悪の反対性においても、反対のものをもた
 ない唯一のものがある。なぜなら怒りはすでに眼前にある困難な
 悪によって引き起こされるからである。それが眼前にあるとき、欲求は降伏し、
 欲情の情念である悲しみの端を出ないか、あるいは、悲痛な悪に立ち向かう運
 動をもつかであるが、この運動が怒りに属する。しかし避ける運動は持つこと
 ができない。なぜなら、すでに悪が現在ないし過去に措定されているからであ
 る。このようにして、怒りの運動に何らかの情念が接近と後退に即して対立
 することはない。


\\

Similiter etiam nec secundum contrarietatem boni et
 mali. Quia malo iam iniacenti opponitur bonum iam adeptum, quod iam non
 potest habere rationem ardui vel difficilis. Nec post adeptionem boni
 remanet alius motus, nisi quietatio appetitus in bono adepto, quae
 pertinet ad gaudium, quod est passio concupiscibilis. 


&

同様に、善と悪の反対性に即してでもありえない。なぜなら、眼前にある悪に対
 立するのは既に獲得された善だが、そのような悪はもはや険しさや困難さとい
 う性格をもちえないからである。また、善の獲得の後には、獲得された善にお
 ける欲求の休息以外の運動は残らないが、この休息は喜びに属し、そして喜び
 は欲情の情念である。

\\

Unde motus irae
 non potest habere aliquem motum animae contrarium, sed solummodo
 opponitur ei cessatio a motu, sicut philosophus dicit, in sua
 rhetorica, quod mitescere opponitur ei quod est irasci, quod non est
 oppositum contrarie, sed negative vel privative.

&

したがって、怒りの運動は何ら魂の反対の運動を持ちえ
 ず、むしろただ運動の停止が対置されるのみである。ちょうど哲学者が自らの
 『修辞学』で「柔和であることは怒っている人に対立する」と言うように。つ
 まりこれは反対によっ て対立するのではなく、否定的や肯定的によって対立す
 る。




\\



Et per hoc patet responsio ad obiecta.

&

以上によって、異論に対する解答は明らかである。


\end{longtable}
\newpage

\rhead{a.~4}
\begin{center}
{\Large {\bf ARTICULUSE QUARTUS}\\
{\large UTRUM SINT ALIQUAE PASSIONES DIFFRENTES SPECIE, IN EADEM
 POTENTIA, NON CONTRARIAE AD INVICEM}}\\
{\footnotesize III {\itshape Sent.}, d.26, q.1, a.3; {\itshape De
 Verit.}, q.26, a.4; II {\itshape Ethic.}, lect.~5.}\\
{\Large 第四項\\同一の能力の中にあるが種において異なる、相互に反対しない
 情念があるか}
\end{center}

\begin{longtable}{p{21em}p{21em}}

{\scshape Ad quartum sic proceditur}. Videtur quod non possint in aliqua potentia
esse passiones specie differentes, et non contrariae ad
invicem. Passiones enim animae differunt secundum obiecta. Obiecta autem
passionum animae sunt bonum et malum, secundum quorum differentiam
passiones habent contrarietatem. Ergo nullae passiones eiusdem
potentiae, non habentes contrarietatem ad invicem, differunt specie.

&

第四項の問題へ議論は以下のように進められる。ある能力の中に種において異な
 りながら相互に反対でないような情念はありえないと思われる。理由は以下の
 通り。魂の情念は対象に即して異なる。ところで魂の情念の対象は善と悪であ
 り、これらの差異に応じて情念は反対性をもつ。ゆえに同一の能力に属し相互
 に反対性を持たないどんな情念も、種において異なる。

\\



2. {\scshape Praeterea}, differentia speciei est differentia secundum formam. Sed
omnis differentia secundum formam, est secundum aliquam contrarietatem,
ut dicitur in X Metaphys. Ergo passiones eiusdem potentiae quae non sunt
contrariae, non differunt specie.

&

さらに、種の差異は形相における差異である。しかるに『形而上学』第10巻で言
 われるように、すべて形相における差異は何らかの反対性においてある。ゆえ
 に、同一の能力に属する反対でない情念が種において異なることはない。

\\



3. {\scshape Praeterea}, cum omnis passio animae consistat in accessu vel recessu ad
bonum vel ad malum, necesse videtur quod omnis differentia passionum
animae sit vel secundum differentiam boni et mali; vel secundum
differentiam accessus et recessus; vel secundum maiorem vel minorem
accessum aut recessum. Sed primae duae differentiae inducunt
contrarietatem in passionibus animae, ut dictum est. Tertia autem
differentia non diversificat speciem, quia sic essent infinitae species
passionum animae. Ergo non potest esse quod passiones eiusdem potentiae
animae differant specie, et non sint contrariae.

&

さらに、魂のすべての情念は、善または悪への接近ないし後退において成立する
 のだから、魂の情念のすべての差異は、善と悪の差異に即してか、接近と後退
 の差異に即してか、あるいは接近ないし後退の度合いに即してかであるのが必
 然であるように思われる。しかるに最初の二つの差異は、既に述べられたとお
 り、魂の情念において反対性を生む。また第三の差異は種の異りをもたらさな
 い。なぜかというと、もしもたらすとしたら魂の情念の種が無数になったであ
 ろうから。ゆえに、魂の同一の能力に属する情念が種において異なり、かつ反
 対しないことは不可能である。

\\



{\scshape Sed contra}, amor et gaudium differunt specie, et sunt in
concupiscibili. Nec tamen contrariantur ad invicem, quin potius unum est
causa alterius. Ergo sunt aliquae passiones eiusdem potentiae quae
differunt specie, nec sunt contrariae.

&

しかし反対に、愛と喜びは種において異なり、かつ欲情の中にある。しかし相互
に反対するわけではない。むしろ一方が他方の原因である。ゆえに、種におい
て異なるが反対しないような、同一の能力に属する情念が存在する。


\\



{\scshape Respondeo dicendum} quod passiones differunt secundum activa, quae sunt
obiecta passionum animae. Differentia autem activorum potest attendi
dupliciter, uno modo, secundum speciem vel naturam ipsorum activorum,
sicut ignis differt ab aqua; alio modo, secundum diversam virtutem
activam. Diversitas autem activi vel motivi quantum ad virtutem movendi,
potest accipi in passionibus animae secundum similitudinem agentium
naturalium. Omne enim movens trahit quodammodo ad se patiens, vel a se
repellit. 


&

解答する。以下のように言われるべきである。情念は、魂の情念の対象である能
 動的なものに即して異なる。ところで、能動的なものの差異は二通りに見出さ
 れうる。一つには種において、ないし能動的なもの自体の本性に即してであり、
 たとえば、火が水と異なる場合がそれである。もう一つには異なる能動の力に
 即してである。ところで、動かす力に関する能動的なものや動かしうるものの
 差異は、自然的能動者の類似に即して魂の緒情念において理解されうる。なぜ
 なら全て動かすものは、何らかのしかたで、受動する自らへ引き付ける、ある
 いは自らから退けるからである。


\\


Trahendo quidem ad se, tria facit in ipso. Nam primo quidem,
dat ei inclinationem vel aptitudinem ut in ipsum tendat, sicut cum
corpus leve, quod est sursum, dat levitatem corpori generato, per quam
habet inclinationem vel aptitudinem ad hoc quod sit sursum. Secundo, si
corpus generatum est extra locum proprium, dat ei moveri ad
locum. Tertio, dat ei quiescere, in locum cum pervenerit, quia ex eadem
causa aliquid quiescit in loco, per quam movebatur ad locum. Et
similiter intelligendum est de causa repulsionis. 

&

実際、自らへ引き付けるとき、自らにおいて三つのことを為す。第一に、

\\

In motibus autem
appetitivae partis, bonum habet quasi virtutem attractivam, malum autem
virtutem repulsivam. Bonum ergo primo quidem in potentia appetitiva
causat quandam inclinationem, seu aptitudinem, seu connaturalitatem ad
bonum, quod pertinet ad passionem amoris. Cui per contrarium respondet
odium, ex parte mali. Secundo, si bonum sit nondum habitum, dat ei motum
ad assequendum bonum amatum, et hoc pertinet ad passionem desiderii vel
concupiscentiae. Et ex opposito, ex parte mali, est fuga vel
abominatio. Tertio, cum adeptum fuerit bonum, dat appetitus quietationem
quandam in ipso bono adepto, et hoc pertinet ad delectationem vel
gaudium. Cui opponitur ex parte mali dolor vel tristitia. 

&


\\

In passionibus
autem irascibilis, praesupponitur quidem aptitudo vel inclinatio ad
prosequendum bonum vel fugiendum malum, ex concupiscibili, quae absolute
respicit bonum vel malum. Et respectu boni nondum adepti, est spes et
desperatio. Respectu autem mali nondum iniacentis, est timor et
audacia. Respectu autem boni adepti, non est aliqua passio in
irascibili, quia iam non habet rationem ardui, ut supra dictum est. Sed
ex malo iam iniacenti, sequitur passio irae. 


&


\\


Sic igitur patet quod in
concupiscibili sunt tres coniugationes passionum, scilicet amor et
odium, desiderium et fuga gaudium et tristitia. Similiter in irascibili
sunt tres, scilicet spes et desperatio, timor et audacia, et ira, cui
nulla passio opponitur. Sunt ergo omnes passiones specie differentes
undecim, sex quidem in concupiscibili, et quinque in irascibili; sub
quibus omnes animae passiones continentur.

&


\\


[34566] Iª-IIae q. 23 a. 4 ad arg.
Et per hoc patet responsio ad obiecta.

&


\end{longtable}
\end{document}