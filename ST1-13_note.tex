\documentclass[paper=a4paper,fontsize=10pt,jafontsize=9pt,titlepage]{jlreq}
% \usepackage{okumacro}
\usepackage{longtable}
\usepackage[polutonikogreek,english,japanese]{babel}
\usepackage{latexsym}
\usepackage{color}
\usepackage{url}
%----- header -------
\usepackage{fancyhdr}
\pagestyle{fancy}
\fancyhead[C,R]{}
\fancyhead[L]{Duns Scotus, {\itshape Ordinatio} I, d.3, Q.2.}
%--------------------
%%%% コピペ用
%\rhead{a.~}
%\begin{center}
% {\Large {\bf }}\\
% {\large }\\
% {\footnotesize }\\
% {\Large \\}
%\end{center}
%
%\begin{longtable}{p{21em}p{21em}}
%
%
%
%\\
%\end{longtable}
%\newpage

\title{スコトゥスの存在の一義牲について}
\author{Japanese translation\\by Yoshinori {Ueeda}}
\date{Last modified \today}

\begin{document}

\maketitle

\newpage

\noindent
テキストは\url{http://www.logicmuseum.com/wiki/Authors/Duns_Scotus/Ordinatio}にあるものを使用。

\section*{存在の一義性について Duns Scotus}
\begin{longtable}{p{21em}p{21em}}
 [27] Et univocationem sic intellectam probo quintupliciter. Primo
 sic: omnis intellectus, certus de uno conceptu et dubius de diversis,
 habet conceptum de quo est certus alium a conceptibus de quibus est
 dubius; subiectum includit praedicatum. Sed intellectus viatoris
 potest esse certus de Deo quod sit ens, dubitando de ente finito vel
 infinito, creato vel increato; ergo conceptus entis de Deo est alius
 a conceptu isto et illo, et ita neuter ex se et in utroque illorum
 includitur; igitur univocus.

 &

 そしてこのように理解された一義性を、私は五通りに証明する。第一に、そ
 れは以下の通りである。ある懐念について確信しているが、それと異なる多
 様な懐念については疑念を持っているような知性はすべて、それについて確
 信している懐念を、それについて疑っている諸懐念とは異なるものとしてもっ
 ている。この主語は述語を含んでいる。しかるに、旅人(現生の人間)の知
 性は、神について、無限の有か有限の有か、被造の有か非被造の有かという
 ことについては疑念を持ちながら、それが有であることは確信している。ゆ
 えに、神についての有の懐念は、この懐念やあの懐念(=有限か無限か、被造
 かそうでないか)とは異なる懐念であり、したがって、自らに基づいてこの
 どちらでもなく、そしてそれらの両方に含まれている。ゆえに、それは一義
 的である。

 \\

 [28] Probatio maioris, quia nullus idem conceptus est certus et
 dubius; ergo vel alius, quod est propositum, vel nullus - et tunc non
 erit certitudo de aliquo conceptu.

 &

 大前提の証明。どんな同一の懐念も、確実でありかつ疑わしいことはない。
 ゆえに、それらは別の懐念であるか、そしてこれが上で提示されていること
 だが、または、それは懐念でないかである。この場合には、何らかの懐念につ
 いて確実性が生じることはない。
 
 \\

[29] Probatio minoris: quilibet philosophus fuit certus, illud quod
posuit primum principium, esse ens, - puta unus de igne et alius de
aqua, certus erat quod erat ensa; non autem fuit certus quod esset ens
creatum vel increatum, primum vel non primum.

 &

 小前提の証明。どんな哲学者も、彼が第一根源とするものが有であることを
 確信している。たとえば、ある人は火について、別の人は水について、そ
 れが有であることを確信している。しかし、それが被造のものかそうでない
 か、あるいは、第一のものかそうでないかについては、確信していなかった。

 \\

 
 Non enim erat certus quod erat primum, quia tunc fuisset certus de
falso, et falsum non est scibile; nec quod erat ens non primum, quia
tunc non posuissent oppositum.

&

 その理由は以下の通りである。それが第一のものであることが確かでなかっ
 たのは、もし確かだったならば、その場合には偽について(=「火は第一のも
 のでない」は偽である、ということについて)確かだっただろうが、それが
 偽であることは知られえないからである。また、それが第一でない有である
 ことも確かではなかった。なぜなら、その場合には、反対のものを置かなかっ
 たであろうから(?)。

 \\

 -- Confirmatur etiam, nam aliquis videns philosophos
discordare potuit esse certus de quocumque quod aliquis posuit primum
principium, esse ens, et tamen propter contrarietatem opinionum eorum
potuit dubitare utrum sit hoc ens vel illud. Et tali dubitanti si
fieret demonstratio concludens vel destruens aliquem conceptum
inferiorem, puta quod ignis non erit ens primum sed aliquod ens
posterius primo ente, non destrueretur ille conceptus primus sibi
certus, quem habuit de ente, sed salvaretur in illo conceptu
particulari probato de igne, - et per hoc probatur propositio,
supposita in ultima consequentia rationis, quae fuit quod ille
conceptus certus, qui est ex se neuter dubiorum, in utroque istorum
salvatur.

 &

 さらに、以下のように確証される。すなわち、哲学者たちが意見を異にする
 のを見る人は、ある人が第一根源だとするものが有であることを確信するが、
 彼らの意見が反対しているために、これこれの有であるか、あるいはそうで
 ない別の有であるかということについて疑念を持つということがありうる。
 そして、そのように疑う人に、もし、下位の何らかの懐念、たとえば、火は
 第一の有ではなく、第一の有より後のなんらかの有であることが証明された
 り、あるいはそのことの否定が証明されたりすれば、彼にとって確実である、
 有について彼がもった第一の懐念は否定されず、火について証明された個別
 的な懐念の中に、保たれるであろう。そしてこのことによって、推論の最終
 結論において措定された命題が証明される。すなわちそれは「それ自身に基
 づいては疑われているどちらでもないような確実な懐念\footnote{たとえば
 ens。}は、それらのどちらの中にも\footnote{ens creatumやens increatum}
 保たれている」というものであった。
 
 
 
 \\

[30] Quod si non cures de auctoritate illa accepta de diversitate
opinionum philosophantium, sed dicas quod quilibet habet duos
conceptus in intellectu suo, propinquos, qui propter propinquitatem
analogiae videntur esse unus conceptus, - contra hoc videtur esse quod
tunc ex ista evasione videretur destructa omnis via probandi unitatem
alicuius conceptus univocam: si enim dicis hominem habere unum
conceptum ad Socratem et Platonem, negabitur tibi, et dicetur quod
sunt duo, sed 'videntur unus' propter magnam similitudinem.

 &

もし、あなたが、哲学する人々の諸々の意見の異なりについて理解されたかの
権威を気にすることなく、自らの知性の中に二つの懐念をもつ人はだれでも、
複数の近い懐念が、そのアナロギアの近さのために一つの懐念に見えるのだと
言うならば、それに反対して、以下のようであるように思われる。すなわち、
その場合には、その逸脱に基づいて、ある懐念の一義的な一性を証明するすべ
ての道が破壊されると思われる。つまり、「人間」がソクラテスとプラトンに
対して一つの懐念をもつとあなたが言うならば、あなたにそれは否定され、そ
れらが二つの懐念であり、大きな類似性のために「一つに見える」と言われる
だろう。
 
 \\

 [31]Praeterea, illi duo conceptus sunt simpliciter simplices; ergo
 non intelligibiles nisi distincte et totaliter. Ergo si nunc non
 'videntur' duo, nec post.

 &

 さらに、その二つの懐念は、端的に単純である。ゆえに、判明にかつ全体的
 にでなければ可知的でない。ゆえに、もし、今二つに「見え」ないのであれ
 ば、今後もそう見えない。
 
 \\

 [32] Item, aut concipiuntur ut omnino disparati, et mirum quomodo
 videntur unus, - aut ut comparati secundum analogiam, aut secundum
 similitudinem, vel distinctionem, et tunc simul vel prius
 concipiuntur ut distincti. Ergo non videntur unus.

 &

 さらに、もしそれらの懐念がまったく分離したものとして捉えられるならば、
 どのようにしてそれらが一つに見えるのかがわからないし、あるいは、もし
 それらの懐念が、アナロギアや類似、あるいは区別に即して関係したものと
 して捉えられるならば、その場合には、同時にせよより先にせよ、区別され
 たものとして捉えられている。ゆえに、一つには見えない。
 
 \\

 [33] Item, ponendo duos conceptus, ponis duo obiecta formalia
 cognita. Quomodo sunt duo cognita formalia et non ut distincta?

 &

 さらに、あなたは、二つの懐念を考えるときに、二つの認識された形相的な
 対象を考える。どのようにして、二つの形相的なものが認識されながら、区
 別さたものでないかたちで認識されるのか。
 
 \\

 [34] Praeterea, si intellectus intelligeret singularia sub propriis
 rationibus, quamvis conceptus duorum eiusdem speciei essent simillimi
 (non est dubium tamen quin multo similiores quam isti duo in
 proposito, quia isti duo differunt specie) adhuc intellectus bene
 distingueret inter tales conceptus singularium. - Haec etiam
 responsio improbatur distinctione 8 quaestione 3, et una alia
 responsio, quae negat maiorem.

 &

 さらに、かりに知性が、固有の性格のもとに個別的なものを知性認識すると
 (事実に反して)仮定したならば、同一の種に属する二つのものの懐念がど
 んなに似ていたとしても(それらが、今問題になっているその二つのものよ
 りも、はるかに似ていることは疑いない。というのも、それらは種において
 異なるのだから)、やはり知性は個物のそのような懐念をよく区別したであ
 ろう。

 またさらに、この回答は第8区分第3問によって誤りが証明される。また、大
 前提を否定する他の一つ回答もまた。

 \\

 [35] Secundo principaliter arguo sic: nullus conceptus realis causatur
 in intellectu viatoris naturaliter nisi ab his quae sunt naturaliter
 motiva intellectus nostri; sed illa sunt phantasma, vel obiectum
 relucens in phantasmate, et intellectus agens; ergo nullus conceptus
 simplex naturaliter fit in intellectu nostro modo nisi qui potest
 fieri virtute istorum.

 &

 主要な証明の第二に\footnote{[27]で「五通りに」と言ったうちの二番目。}、
 私は次のように論じる。どんな実在的な懐念も、旅人の知性の中に、私たち
 の知性を本性的に動かしうるものによってでなければ、原因されない。しか
 るに、それらのものは、表象像または表象像において輝いている対象と能動
 知性である。ゆえに、どんな単純な懐念も、これらのものの力によって生じ
 うるのでなければ、現在、私たちの知性の中に本性的に生じない。

 \\


 Sed conceptus qui non esset univocus obiecto relucenti in
 phantasmate, sed omnino alius, prior, ad quem ille habeat analogiam,
 non potest fieri virtute intellectus agentis et phantasmatis; ergo
 talis conceptus alius, analogus qui ponitur, naturaliter in
 intellectu viatoris numquam erit, - et ita non poterit haberi
 naturaliter aliquis conceptus de Deo, quod est falsum.

 &

 しかるに、表象像の中に輝く対象に一義的でなく、まったく別であり、より
 先であり、それに彼のものがアナロギアをもつ懐念は、能動知性や表象の力
 によって生じえない。ゆえに、そのような別の懐念、アナロギア的なものと
 される懐念は、決して旅人の知性の中に本性的に生じないことになる。そし
 てかくして、本性的に、どんな神についての懐念も持たれえないことになる
 が、これは偽である。

 \\


 Probatio assumpti: obiectum quodcumque, sive relucens in
 phantasmate sive in specie intelligibili, cum intellectu agente vel
 possibili cooperante, secundum ultimum suae virtutis facit sicut
 effectum sibi adaequatum, conceptum suum proprium et conceptum omnium
 essentialiter vel virtualiter inclusorum in eo; sed ille alius
 conceptus qui ponitur analogus, non est essentialiter nec virtualiter
 inclusus in isto, nec etiam est iste; ergo iste non fiet ab aliquo
 tali movente.

 &

 仮定の証明。表象像や可知的形象において輝いているどんな対象も、能動知
 性や可能知性の協力とともに、自らの力の究極にしたがって、それを自らに
 対等する結果、自らに固有の懐念、本質的または能力的にそれに含まれるす
 べてのものの懐念を生み出す。しかし、かの別の、アナロギア的なものとさ
 れている懐念は、本質的にも能力的にも、その懐念の中には含まれず、また
 その懐念でもない。ゆえに、そのような懐念は、なんらそのような動者によっ
 ては生じない。

 \\

 Et confirmatur ratio, quia 'obiectum': praeter conceptum suum
 proprium adaequatum, et inclusum in ipso altero duorum modorum
 praedictorum, nihil potest cognosci ex isto obiecto nisi per
 discursum; sed discursus praesupponit cognitionem istius simplicis ad
 quod discurritur.


 &

 そして、以下の議論が確定される。自らの固有で対等な、そしてこの述語の
 二つのしかたの一方自身の中に含まれている懐念以外の「対象」は、推論に
 よってしかその対象から認識されえない。しかるに、推論は、それへと推論
 されるその単純なものの認識を前提とする。

 
 \\
 

 Formetur igitur ratio sic, quia nullum obiectum
 facit conceptum simplicem proprium, in isto intellectu, conceptum
 simplicem proprium alterius obiecti, nisi contineat illud aliud
 obiectum essentialiter vel virtualiter; obiectum autem creatum non
 continet increatum essentialiter vel virtualiter, et hoc sub ea
 ratione sub qua sibi attribuitur, ut 'posterius essentialiter'
 attribuitur 'priori essentialiter', - quia contra rationem
 'posterioris essentialiter' est includere virtualiter suum prius, et
 patet quod obiectum creatum non essentialiter continet increatum
 secundum aliquid omnino sibi proprium et non commune; ergo non facit
 conceptum simplicem et proprium enti increato.

 &

 それゆえ、議論は以下のように形成されるべきである。すなわち、どんな対
 象も、別の対象を本質的にないし能力的に含んでいなければ、この知性にお
 いて、単純で固有の懐念を、その別の対象の単純で固有の懐念を作らない。
 しかるに被造の対象は、非被造のものを本質的ないし能力的に含まない。
 



 
\end{longtable}

\end{document}





36 Tertio arguitur sic: conceptus proprius alicuius subiecti est sufficiens ratio concludendi de illo subiecto omnia conceptibilia quae sibi necessario insunt; nullum autem conceptum habemus de Deo per quem sufficienter possimus cognoscere omnia concepta a nobis quae necessario sibi insunt - patet de Trinitate et aliis creditis necessariis; ergo etc.	
37 Maior probatur, quia immediatam quamlibet cognoscimus in quantum terminos cognoscimus; igitur patet maior de omni illo conceptibili quod immediate inest conceptui subiecti. Quod si insit mediate, fiet idem argumentum de medio comparato ad idem subiectum, et ubicumque stabitur habetur propositum de immediatis, et ultra per illas scientur mediatae.	
38 Item, quarto, potest sic argui: aut aliqua 'perfectio simpliciter' habet rationem communem Deo et creaturae, et habetur propositum, aut non sed tantum propriam creaturae, et tunc ratio eius non conveniet formaliter Deo, quod est inconveniens; aut habet rationem omnino propriam Deo, et tunc sequitur quod nihil attribuendum est Deo, quia est 'perfectio simpliciter', nam hoc nihil est aliud dicere nisi quod quia ratio eius ut convenit Deo dicit 'perfectionem simpliciter', ideo ipsum ponitur in Deo; et ita peribit doctrina Anselmi Monologion, ubi vult quod 'praetermissis relationibus, in omnibus aliis quidquid est simpliciter melius ipsum quam non ipsum attribuendum est Deo, sicut quodcumque non tale est amovendum ab ipso'. Primo ergo, secundum ipsum, aliquid cognoscitur esse tale, et secundo attribuitur Deo; ergo non est tale praecise ut in Deo. - Hoc etiam confirmatur, quia tunc nulla 'perfectio simpliciter' esset in creatura; consequentia patet, quia nullius talis perfectionis etiam conceptus aliquis convenit creaturae nisi conceptus analogicus (ex hypothesi) - 'talis' secundum se, quia analogicus est imperfectus - et in nullo est eius ratio melior non ipso, quia alias secundum illam rationem analogicam poneretur in Deo.	
39 Confirmatur etiam haec quarta ratio sic:  omnis inquisitio metaphysica de Deo sic procedit, considerando formalem rationem alicuius et auferendo ab illa ratione formali imperfectionem quam habet in creaturis, et reservando illam rationem formalem et attribuendo sibi omnino summam perfectionem, et sic attribuendo illud Deo. Exemplum de formali ratione sapientiae (vel intellectus) vel voluntatis: consideratur enim in se et secundum se; et ex hoc quod ista ratio non concludit formaliter imperfectionem aliquam nec limitationem, removentur ab ipsa imperfectiones quae concomitantur eam in creaturis, et reservata eadem ratione sapientiae et voluntatis attribuuntur ista Deo perfectissime. Ergo omnis inquisitio de Deo supponit intellectum habere conceptum eundem, univocum, quem accepit ex creaturis.	
40 Quod si dicas, alia est formalis ratio eorum quae conveniunt Deo, - ex hoc sequitur inconveniens, quod ex nulla ratione propria eorum prout sunt in creaturis, possunt concludi de Deo, quia omnino alia et alia ratio illorum est et istorum; immo non magis concludetur quod Deus est sapiens formaliter, ex ratione sapientiae quam apprehendimus ex creaturis, quam quod Deus est formaliter lapis: potest enim conceptus aliquis, alius a conceptu lapidis creati, formari, ad quem conceptum lapidis ut est idea in Deo habet iste lapis attributionem, et ita formaliter diceretur 'Deus est lapis', secundum istum conceptum analogum, sicut 'sapiens', secundum illum conceptum analogum.	
41 Ad hoc etiam arguitur, quinto, sic: perfectior creatura potest movere ad perfectiorem conceptum de Deo. Ergo cum aliqua visio Dei, puta infima, non tantum differat ab aliqua intellectione abstractiva data ipsius quantum suprema creatura distat ab infima, videtur sequi quod si infima potest movere ad aliquam abstractivam, quod suprema, vel aliqua citra eam, poterit movere ad intuitivam, quod est impossibile.	
42 Quod si dicas, tot gradus esse in intellectione abstractiva de Deo quot sunt species creatae, licet extremae intellectiones non tantum distent quantum extremae species, quod bene est possibile, quia quilibet gradus in intellectionibus minus distat a sibi proximo quam species creata movens ad illum distat a specie movente ad alium, - contra: differentia intellectionum abstractivarum non est tantum numeralis, quia illae causantur a causis alterius speciei, et per proprias rationes illarum, non in quantum includunt aliquid commune sicut dicit via de univocatione. Ergo sequitur quod inter infimam intellectionem abstractivam et infimam intuitivam sint plures mediae quam inter infimam speciem entium et supremam, vel tot. Quod si est inconveniens consequens, et per consequens - antecedens. Ergo pauciores sunt species intellectionis abstractivae quam entium. Ergo incipiendo ab infima, hinc inde, restat aliqua entis, superior illa quae causat supremam abstractivam. Ergo illa superior causabit intuitivam de Deo.	
43 Item, quare ponentur tot species intellectionum de eodem obiecto, si movet ad proprium? 	
44 Item, ad conclusionem principalem videtur esse quod omnis multitudo reducitur ad unum. Ergo ita est in conceptibus.
45 Qualis autem sit univocatio entis, ad quanta et ad quae, dicetur magis in quaestione de primo obiecto intellectus.	
46 Contra istas rationes instatur. - Contra primam instatur, de toto disiuncto, et illa responsio ponitur distinctione 8, et debilius improbatur quam aliae.	
47 Ad secundam, ut formatur breviter, negatur maior, quia, propter conexionem, effectus potest aliquem conceptum facere de causa, licet non ita perfectum sicut causa de se: conceditur enim quod conclusio facit notitiam de principio demonstratione 'quia', sed illa non est perfectissima notitia principii sed illa qua cognoscitur ex terminis perfecte cognitis. Quare enim, non similiter in conceptibus, effectus simpliciter apprehensus causabit notitiam aliquam habitualem simplicem de causa? 	
48 Ad probationem maioris dico quod licet non possit effectus aequivocus in causam aequivocam exsistentem, nec in aliquid eiusdem rationis cum causa, potest tamen in notitiam aliquam eius, quae est imperfectior non solum ipsa causa in se sed etiam in ipso effectu aequivoco causae, scilicet perfecto conceptu eius. - Sed accipiatur maior sic: 'nullum obiectum potest in conceptum alicuius nisi contineat illum conceptum virtualiter vel essentialiter'. Haec videtur manifesta per rationem causae et effectus aequivoci, et licet attribuatur secundum aliquos actio intellectui (non curo), quomodocumque obiectum requiritur, sic non potest in conceptum perfectiorem conceptu sibi adaequato; talis est proprius, quiditativus; ergo etc. Probatur minor, quia effectuum aequivocorum eiusdem causae ille est perfectissimus qui est simillimus causae: talis est proles intellectualis sive verbum perfectum huius obiecti. Maior probatur, quia tunc perfectio intelligentiae excederet totam virtutem memoriae.	
49 Absolute videtur concedendum quod nullus de Deo conceptus potest fieri in nobis per actionem obiecti creati, qui sit perfectior conceptu perfecto proprio illius obiecti, nec per consequens, ad quem attribuatur iste conceptus proprius obiecti moventis, immo magis, ille conceptus de Deo est imperfectior verbo huius, quia effectus aequivocus, dissimilior causae. Oportet ergo recedere ab opinione Henrici, si ponat conceptum lapidis attribui ad conceptum quem causat lapis de Deo. Praecise salvari potest quod obiectum conceptum attribuitur ad obiectum, non tamen conceptus ad conceptum; et hoc est bene possibile, quia de conceptu perfectiore habetur conceptus imperfectior quam de conceptu imperfectiore. Et quomodo est rationabile in eodem intellectu conceptum proprium de Deo esse simpliciter imperfectiorem conceptu lapidis vel albi, et quomodo erit beatitudo naturalis in cognitione Dei (ex X Ethicorum)?	
50 Sed videtur contra univocationem esse eadem difficultas, quia omnis conceptus de Deo erit minus perfectus conceptu proprio albi perfecto, quia omnis talis continetur in albedine ut conceptus communis in speciali, et communis est simpliciter minus perfectus, quia potentialis et pars respectu conceptus specialis. Quomodo ergo secundum illam erit beatitudo in cognitione naturali Dei?	
51 Responsio. Quilibet conceptus simpliciter simplex, scilicet univocationis, est imperfectior positive quam verbum albi, hoc est non tantam perfectionem ponit; tamen est perfectior permissive, quia abstrahit a limitatione, et ita est conceptibilis sub infinitate: et tunc ille conceptus - simplex quidem, non tamen simpliciter simplex - scilicet 'ens infinitum', erit perfectior verbo albi, et ille proprius Deo, non autem ille prior, communis, abstractus ab albedine. Unde via univocationis tenet quod omnis conceptus proprius Deo est perfectior verbo cuiuscumque creati, sed alia non sic.	
52 Sed instatur contra hanc responsionem dupliciter. Primo, arguendo quod difficultas remanet contra viam univocationis, quia ex duobus conceptibus, quorum uterque est imperfectior verbo albi, non videtur fieri conceptus perfectior illo verbo. Sed conceptus entis, ut conceditur, est imperfectior quam albi, vel lineae, et conceptus infiniti similiter. Probatio, quia 'infinitum' concipitur a nobis per finitum, 'finitum' per lineam, vel aliquod tale obiectum, movens ad conceptum passionis. Igitur conceptus infiniti est imperfectior conceptu lineae. - Confirmatur ratio, quia conceptus includens affirmationem et negationem non est perfectior propter negationem, aut saltem non est perfectior quam concipiendo affirmationem illius negationis; hic 'ens infinitum' non est aliquis conceptus positivi nisi entis; ergo infinitas non facit conceptum perfectum, aut saltem non erit perfectior conceptus entis infiniti quam entis finiti.	
53 Secundo instatur similiter, pro Henrico, quia licet conceptus simpliciter simplex sit imperfectior verbo sive conceptu creaturae, ut arguitur, tamen multi tales possunt coniungi, et unus determinabit alium, - et ille conceptus totus erit perfectior. Nec est hic maior difficultas quam ibi, nisi in duobus: primum, quod hic quilibet conceptus, sive determinans sive determinabilis, ponitur proprius Deo, ibi unus communis et alius proprius; secundum, quod hic aliquis proprius Deo conceditur imperfectior verbo creaturae, ibi nullus. Istorum autem primum non est inconveniens, quia passio bene determinat subiectum: homo risibilis, et tamen utrumque est aeque commune. Secundum oportet omnino concedere, propter istam secundam rationem, - loquendo de conceptu, id est de actu concipiendi, non autem de obiecto concepto.	
54 Quoad istas instantias videtur satis congrue responderi, quod utraque opinio conceptum non simpliciter simplicem ponit perfectiorem verbo illius quod movet ad partem. Sed instantia arguendo facta, videtur contra utramque opinionem, quia quotcumque coniungantur, quilibet illorum conceptuum imprimitur a creatura movente. Ergo est imperfectior verbo illius creaturae. Aggregatio imperfectiorum quomodo faciet conceptum perfectiorem intensive? - Confirmatio etiam bene instat contra illud de infinito. - Non ergo propter istam rationem dimittatur opinio, quia est communis difficultas utrique, et aeque, si analogia conceptuum exponatur de conceptis.	
55 Forte instantiae bene probant quod actus circa Deum non sit perfectissimus intensive; nec hoc requiritur, ut sit ibi beatitudo naturalis, sed quod coniungat obiecto perfectissimo, -- II De animalibus: ((Parum nosse)) etc. Et forte intensius amatur creatum aliquod quam Deus, nec tamen illud amatum nunc beatificat sicut Deus (de hoc in libro IV, 'quomodo beatificamur in obiecto'). Istud de ente infinito verum esset, si 'infinitum' esset praecise modus sub quo obiectum conciperetur, et non pars conceptus, vel modus sic quod conceptus in se (sicut distinguitur in quaestione de unitate Dei, de singularitate ut concepta et ut modo praecise sub quo), quemadmodum etiam certus gradus intensionis est praecise modus sub quo videtur haec albedo. Sic autem non intelligimus ens infinitum, sed ut includens duos conceptus, licet 'alter determinet alterum'. Et forte ille conceptus privativus finiti nihil ponit, quamvis det intelligere positivum, ita quod si habemus conceptum positivum de necessario, magis perfecte positive intelligitur Deus hic, ens simpliciter necessarium. Sed nec forte necessarium nec aeternum concipimus nisi negationem imperfectionis, puta potentiae aliter se habendi vel fluxibilis, seu principii seu finis. Aeternum dicit 'quoddam' infinitum, quia in duratione minime perfectior est infinitas quam in quantitate perfectionis, sicut infinita magnitudo esset perfectior infinito tempore.
56 Tertio dico quod Deus non cognoscitur naturaliter a viatore in particulari et proprie, hoc est sub ratione huius essentiae ut haec et in se. Sed ratio illa posita ad hoc in praecedenti opinione, non concludit. Cum enim arguitur quod non 'cognoscitur aliquid nisi per simile', aut intelligit 'per simile' de similitudine univocationis, aut imitationis. Si primo modo, igitur nihil cognoscitur de Deo secundum illam opinionem, quia in nullo habet similitudinem univocationis secundum illum modum. Si secundo modo, et creaturae non tantum imitantur illam essentiam sub ratione generalis attributi sed etiam essentiam hanc ut est haec essentia (sive ut 'nuda' in se est exsistens, secundum eum) - sic enim magis est idea vel exemplar quam sub ratione generalis attributi - ergo propter talem similitudinem posset creatura esse principium cognoscendi essentiam divinam in se et in particulari.	
57 Est ergo alia ratio huius conclusionis, videlicet quod Deus ut haec essentia in se, non cognoscitur naturaliter a nobis, quia sub ratione talis cognoscibilis est obiectum voluntarium, non naturale, nisi respectu sui intellectus tantum. Et ideo a nullo intellectu creato potest sub ratione huius essentiae ut haec est naturaliter cognosci, nec aliqua essentia naturaliter cognoscibilis a nobis sufficienter ostendit hanc essentiam ut haec, nec per similitudinem univocationis nec imitationis. Univocatio enim non est nisi in generalibus rationibus, imitatio etiam deficit, quia imperfecta, quia creatura imperfecte eum imitatur. Utrum autem sit alia ratio huius impossibilitatis, videlicet propter rationem primi obiecti, sicut alii ponunt, de hoc in quaestione de primo obiecto. 	
58 Quarto dico quod ad multos conceptus proprios Deo possumus pervenire, qui non conveniunt creaturis, - cuiusmodi sunt conceptus omnium perfectionum simpliciter, in summo. Et perfectissimus conceptus, in quo quasi in quadam descriptione perfectissime cognoscimus Deum, est concipiendo omnes perfectiones simpliciter et in summo. Tamen conceptus perfectior simul et simplicior, nobis possibilis, est conceptus entis infiniti. Iste enim est simplicior quam conceptus entis boni, entis veri, vel aliorum similium, quia 'infinitum' non est quasi attributum vel passio entis, sive eius de quo dicitur, sed dicit modum intrinsecum illius entitatis, ita quod cum dico 'infinitum ens', non habeo conceptum quasi per accidens, ex subiecto et passione, sed conceptum per se subiecti in certo gradu perfectionis, scilicet infinitatis, - sicut albedo intensa non dicit conceptum per accidens sicut albedo visibilis, immo intensio dicit gradum intrinsecum albedinis in se. Et ita patet simpli citas huius conceptus 'ens infinitum'.	
59 Probatur perfectio istius conceptus, tum quia iste conceptus, inter omnes nobis conceptibiles conceptus, virtualiter plura includit - sicut enim ens includit virtualiter verum et bonum in se, ita ens infinitum includit verum infinitum et bonum infinitum, et omnem 'perfectionem simpliciter' sub ratione infiniti, - tum quia demonstratione 'quia' ultimo concluditur 'esse' de ente infinito, sicut apparet ex quaestione prima secundae distinctionis; illa autem sunt perfectiora quae ultimo cognoscuntur demonstratione 'quia' ex creaturis, quia propter eorum remotionem a creaturis difficillimum est ea ex creaturis concludere.