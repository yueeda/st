\documentclass[10pt]{jsarticle}
\usepackage{okumacro}
\usepackage{longtable}
\usepackage[polutonikogreek,english,japanese]{babel}
\usepackage{latexsym}
\usepackage{color}
\usepackage{schemata}
\usepackage[T1]{fontenc}
\usepackage{lmodern}

%----- header -------
\usepackage{fancyhdr}
\pagestyle{fancy}
\lhead{{\it Summa Theologiae} I-II, q.62}
%--------------------

\bibliographystyle{jplain}

\title{{\bf PRIMA SECUNDAE}\\{\HUGE Summae Theologiae}\\Sancti Thomae
Aquinatis\\{\sffamily QUEAESTIO SEXAGESIMASECUNDA}\\DE VIRTUTIBUS THEOLOGICIS}
\author{Japanese translation\\by Yoshinori {\sc Ueeda}}
\date{Last modified \today}

%%%% コピペ用
%\rhead{a.~}
%\begin{center}
% {\Large {\bf }}\\
% {\large }\\
% {\footnotesize }\\
% {\Large \\}
%\end{center}
%
%\begin{longtable}{p{21em}p{21em}}
%
%&
%
%\\
%\end{longtable}
%\newpage

\begin{document}

\maketitle
\thispagestyle{empty}

\begin{center}
{\LARGE 『神学大全』第二部の一}\\
{\Large 第六十二問\\神学的徳について}
\end{center}

\begin{longtable}{p{21em}p{21em}}

Deinde considerandum est de virtutibus theologicis. Et circa hoc quaeruntur quatuor. 

\begin{enumerate}
 \item utrum sint aliquae virtutes theologicae.
 \item utrum virtutes theologicae distinguantur ab intellectualibus et moralibus.
 \item quot, et quae sint.
 \item de ordine earum.
\end{enumerate}

&

 次に神学的徳\footnote{virtus theologicaは「神学的徳」と訳す。定訳は「対神徳」だが、それだと「神に対して人間が持つ徳」という意味が含まれることになり、たとえば第二異論が読めない。}について考察されるべきである。これを巡っては四つのことが問われる。  

\begin{enumerate}
 \item 神学的徳が存在するか。
 \item 神学的徳は知性的徳と道徳的徳から区別されるか。
 \item それらはいくつあり、また、何であるか。
 \item それらの秩序について。
\end{enumerate}
\end{longtable}

\newpage

\rhead{a.~1}
\begin{center}
{\Large {\bf ARTICULUS PRIMUS}}\\
{\large UTRUM SINT ALIQUAE VIRTUTES THEOLOGICAE}\\
{\footnotesize III {\itshape Sent.}, d.23, q.1, a.4; qu$^{a}$3; {\itshape De Virtut.}, q.1, a.10, 12.}\\
{\Large 第一項\\神学的徳が存在するか}
\end{center}

\begin{longtable}{p{21em}p{21em}}
{\scshape Ad primum sic proceditur}. Videtur quod non sint aliquae
virtutes theologicae. Ut enim dicitur in VII {\itshape Physic}.,
{\itshape virtus est dispositio perfecti ad optimum, dico autem
perfectum, quod est dispositum secundum naturam}. Sed id quod est
divinum, est supra naturam hominis. Ergo virtutes theologicae non sunt
virtutes hominis.

&

 第一項の問題へ、議論は以下のように進められる。神学的徳は存在しないと
 思われる。理由は以下の通り。『自然学』第7巻で言われているように、「徳
 は最善のものへと完成されたものの態勢であるが、私が完成されたものと言
 うのは、本性にしたがって態勢付けられたものである」。しかるに神的であ
 るものは人間の本性の上位にある。ゆえに神学的徳は人間の徳ではない。

 
\\



2.~{\scshape Praeterea}, virtutes theologicae dicuntur quasi virtutes
divinae. Sed virtutes divinae sunt exemplares, ut dictum est, quae
quidem non sunt in nobis, sed in Deo. Ergo virtutes theologicae non
sunt virtutes hominis.

&

 さらに、神学的徳は神的な徳という意味で語られる。しかるに神的徳は、す
 でに述べられたとおり範型的であり、それは私たちではなく神の中にある。
 ゆえに神学的徳は人間の徳ではない。
 
\\


3.~{\scshape Praeterea}, virtutes theologicae dicuntur quibus
ordinamur in Deum, qui est primum principium et ultimus finis
rerum. Sed homo ex ipsa natura rationis et voluntatis, habet ordinem
ad primum principium et ultimum finem. Non ergo requiruntur aliqui
habitus virtutum theologicarum, quibus ratio et voluntas ordinetur in
Deum.

&

 さらに、神学的徳と言われるのは、それによって、諸事物の第一根源であり
 かつ究極目的である神へと私たちが秩序付けられるところのものである。し
 かし人間は理性と意志の本性自身に基づいて第一根源と究極目的への秩序を
 もつ。ゆえに、それによって理性と意志が神へと秩序付けられるところの神
 学的徳に属する何らかの習慣が必要とされることはない。
 
\\



{\scshape Sed contra est} quod praecepta legis sunt de actibus
virtutum. Sed de actibus fidei, spei et caritatis dantur praecepta in
lege divina, dicitur enim {\itshape Eccli}.~{\scshape ii}: {\itshape
Qui timetis Deum, credite illi; item, sperate in illum; item, diligite
illum}. Ergo fides, spes et caritas sunt virtutes in Deum
ordinantes. Sunt ergo theologicae.

 &

 しかし反対に、法の原則は徳の活動にかかわる。しかるに信仰、希望、愛徳
 について、神の法の中で原則が与えられている。すなわち、『シラ書』第2章
 で「神を恐れるあなたたちは、神を信じなさい、さらに神へと希望しなさい、
 さらに神を愛しなさい」\footnote{「主を畏れる人々よ、主を信頼せよ。/
 そうすれば、お前たちの報いが失われることはない。主を畏れる人々よ、善
 い賜物に/永遠の喜びと憐れみに希望を置け。/〔主の報いは/喜びを伴う
 永遠の賜物だからである。〕」(2:8-9) }と言われている。ゆえに、信仰、希
 望、愛徳は神へと秩序付けられた徳である。ゆえにそれらは神学的である。

\\



{\scshape Respondeo dicendum} quod per virtutem perficitur homo ad
actus quibus in beatitudinem ordinatur, ut ex supradictis patet. Est
autem duplex hominis beatitudo sive felicitas, ut supra dictum
 est.


&

 解答する。以下のように言われるべきである。上述のことから明らかなとお
 り、人間は徳を通して行為へと完成させられ、その行為において至福へと秩
 序付けられる。ところで、上述の如く、人間の至福ないし幸福には二通りあ
 る。
 
\\


 Una quidem proportionata humanae naturae, ad quam scilicet homo
pervenire potest per principia suae naturae. Alia autem est beatitudo
naturam hominis excedens, ad quam homo sola divina virtute pervenire
potest, secundum quandam divinitatis participationem; secundum quod
dicitur II Petr.~{\scshape i}, quod  per Christum facti sumus
{\itshape consortes divinae naturae}.

&

 一つは人間本性に比例した幸福であり、それへと人間は自らの本性に属する
 諸根源を通して到達できる。もう一つは人間本性を超える至福であり、それ
 へ人間は、何らかの神性の分有において神のちからによってのみ到達できる。
 これは『ペトロの手紙一』「私たちはキリストを通して神の本性と同類のも
 のになった」による。

 
\\

 Et quia huiusmodi beatitudo proportionem humanae naturae excedit,
principia naturalia hominis, ex quibus procedit ad bene agendum
secundum suam proportionem, non sufficiunt ad ordinandum hominem in
beatitudinem praedictam.


&

 そしてこのような至福は人間本性の比を越えるので、それに基づいて自らの
 比にしたがって善く行為する事へと進んでいくところの人間の自然本性的根
 源は、上述の至福へ人間を秩序付けるのに十分でない。
 
\\


 Unde oportet quod superaddantur homini divinitus aliqua principia,
per quae ita ordinetur ad beatitudinem supernaturalem, sicut per
principia naturalia ordinatur ad finem connaturalem, non tamen absque
adiutorio divino.

 &

 したがって、人間にはそれを通してそのように本性を越えた至福へと秩序付
 けられるところの何らかの神的な根源が与えられなければならない。それは
 ちょうど、自然本性的根源を通して、しかし神の助けなしに共本性的な目的
 へ秩序付けられるようにである。

\\

 Et huiusmodi principia virtutes dicuntur theologicae, tum quia habent
Deum pro obiecto, inquantum per eas recte ordinamur in Deum; tum quia
a solo Deo nobis infunduntur; tum quia sola divina revelatione, in
sacra Scriptura, huiusmodi virtutes traduntur.

 &

 そしてこのような根源が神学的徳と呼ばれる。一つには、そのような徳を通
 して神へと正しく秩序付けられる限りにおいて、それが神を対象としてもっ
 ているからであり、またもう一つには、ただ神のみからそれらが私たちに注
 入されるからであり、また聖書における神の啓示によってのみそのような徳
 が伝えられているからである。
 
\\



{\scshape Ad primum} ergo dicendum quod aliqua natura potest attribui
alicui rei dupliciter. Uno modo, essentialiter, et sic huiusmodi
virtutes theologicae excedunt hominis naturam. Alio modo,
participative, sicut lignum ignitum participat naturam ignis, et sic
quodammodo fit homo particeps divinae naturae, ut dictum est. Et sic
istae virtutes conveniunt homini secundum naturam participatam.

&

 第一異論に対しては、それゆえ、以下のように言われるべきである。ある本
 性は、ある事物に二通りの仕方で帰せられうる。一つには、本質的にであり、
 この仕方では、このような神学的徳は人間の本性を越える。もう一つには分
 有的にであり、たとえば燃える木材が火を分有する場合がそれである。この
 仕方では、上述のようにある意味で人間は神の本性を分有する。そしてこの
 意味で、これらの徳は分有された本性に即して人間に適合する。
 
\\



{\scshape Ad secundum dicendum} quod istae virtutes non dicuntur
divinae, sicut quibus Deus sit virtuosus, sed sicut quibus nos
efficimur virtuosi a Deo, et in ordine ad Deum. Unde non sunt
exemplares, sed exemplatae.

&

 第二異論に対しては、それゆえ、以下のように言われるべきである。これら
 の徳は、それによって神が有徳であるという意味で神的なと言われているわ
 けではなく、それによって私たちが神によって、神への秩序において有徳な
 ものに作り出されるという意味でである。したがってそれらは範型的ではな
 く、範型によって生み出されたものである。
 
\\



{\scshape Ad tertium dicendum} quod ad Deum naturaliter ratio et
voluntas ordinatur prout est naturae principium et finis, secundum
tamen proportionem naturae. Sed ad ipsum secundum quod est obiectum
beatitudinis supernaturalis, ratio et voluntas secundum suam naturam
non ordinantur sufficienter.

&

第三異論に対しては以下のように言われるべきである。神へと自然本性的に理
性と意志は秩序付けられるが、それは神が本性の比に即してではあるが、本性
の根源であり目的であるかぎりでである。しかし神が本性を越えた至福の対象
である限りにおいては、理性と意志は自らの本性に即しては十分には秩序付け
られない。


\end{longtable}
\newpage

\rhead{a.~2}
\begin{center}
{\Large {\bf ARTICULUS SECUNDUS}}\\
{\large UTRUM VIRTUTES THEOLOGICAE DISTINGUANTUR\\ AB INTELLECTUALIBUS ET MORALIBUS}\\
{\footnotesize III {\itshape Sent.}, d.23, q.1, a.4, qu$^{a}$3, ad 4; {\itshape De Verit.}, q.14, a.3, ad 9; {\itshape De Virtut.}, q.1, a.12.}\\
{\Large 神学的徳は知性的徳と倫理的徳から区別されるか\\}
\end{center}

\begin{longtable}{p{21em}p{21em}}
{\scshape Ad secundum sic proceditur}. Videtur quod virtutes
theologicae non distinguantur a moralibus et
intellectualibus. Virtutes enim theologicae, si sunt in anima humana,
oportet quod perficiant ipsam vel secundum partem intellectivam vel
secundum partem appetitivam. Sed virtutes quae perficiunt partem
intellectivam, dicuntur intellectuales, virtutes autem quae perficiunt
partem appetitivam, sunt morales. Ergo virtutes theologicae non
distinguuntur a virtutibus moralibus et intellectualibus.

&

 第二項の問題へ、議論は以下のように進められる。神学的徳は道徳的徳と知
 性的徳から区別されないと思われる。理由は以下の通り。神学的徳は、もし
 人間の魂の中にあるのならば、魂を知的部分に即してか、または欲求的部分
 に即して完成させるのでなければならない。しかし知性的部分を完成させる
 徳は知性的と言われ、欲求的部分を完成させる徳は道徳的である。ゆえに神
 学的徳は道徳的徳と知性的徳から区別されない。
 
\\



2.~{\scshape Praeterea}, virtutes theologicae dicuntur quae ordinant
nos ad Deum. Sed inter intellectuales virtutes est aliqua quae ordinat
nos ad Deum, scilicet sapientia, quae est de divinis, utpote causam
altissimam considerans. Ergo virtutes theologicae ab intellectualibus
virtutibus non distinguuntur.


&

 さらに、神学的徳と言われるのは私たちを神へ秩序付けるものである。しか
 し知性的徳の中には私たちを神へと秩序付ける徳、すなわち知恵があり、そ
 れは最高の原因を考察するものとして神的な事柄にかかわる。ゆえに神学的
 徳は知性的徳から区別されない。
 
\\



3.~{\scshape Praeterea}, Augustinus, in libro {\itshape de Moribus
Eccles}., manifestat in quatuor virtutibus cardinalibus quod sunt
{\itshape ordo amoris}. Sed amor est caritas, quae ponitur virtus
theologica. Ergo virtutes morales non distinguuntur a theologicis.

&

 さらにアウグスティヌスは『教会の道徳について』という書物の中で、四つ
 の枢要徳においてそれらが「愛の秩序」であることを明らかにしている。し
 かるに愛は愛徳であり、これは神学的徳とされている。ゆえに道徳的徳は神
 学的徳から区別されない。
 
\\

{\scshape Sed contra}, id quod est supra naturam hominis, distinguitur
ab eo quod est secundum naturam hominis. Sed virtutes theologicae sunt
super naturam hominis, cui secundum naturam conveniunt virtutes
intellectuales et morales, ut ex supradictis patet. Ergo distinguuntur
ab invicem.

&

しかし反対に、人間本性を越えているものは人間本性に即してあるものから区
別される。しかるに神学的徳は人間の本性を越えている。そして上述の如く、
それにはその本性に即して知性的徳と道徳的徳が適合する。ゆえにそれらは相
互に区別される。
 
\\


{\scshape Respondeo dicendum} quod, sicut supra dictum est, habitus
specie distinguuntur secundum formalem differentiam
obiectorum. Obiectum autem theologicarum virtutum est ipse Deus, qui
est ultimus rerum finis, prout nostrae rationis cognitionem
excedit. Obiectum autem virtutum intellectualium et moralium est
aliquid quod humana ratione comprehendi potest. Unde virtutes
theologicae specie distinguuntur a moralibus et intellectualibus.


&

 解答する。以下のように言われるべきである。上述の如く習慣は対象の形相
 的な差異に即して種において区別される。さて、神学的徳の対象は神自身で
 あり、それは私たちの理性の認識を越えるものとして諸事物の目的である。
 これに対して知性的徳と道徳的徳の対象は人間の理性によって把握されうる
 何かである。したがって神学的徳は種において道徳的徳と知性的徳から区別
 される。
 
 
\\



{\scshape Ad primum ergo dicendum} quod virtutes intellectuales et
morales perficiunt intellectum et appetitum hominis secundum
proportionem naturae humanae, sed theologicae supernaturaliter.


&

 第一異論に対しては、それゆえ、以下のように言われるべきである。知性的
 徳と道徳的徳は人間の知性と欲求を人間の本性の比に即して完成させるが、
 神学的徳は自然本性を越えたしかたでそうする。
 
\\



{\scshape Ad secundum dicendum} quod sapientia quae a philosopho
ponitur intellectualis virtus, considerat divina secundum quod sunt
investigabilia ratione humana. Sed theologica virtus est circa ea
secundum quod rationem humanam excedunt.


&

 第二異論に対しては以下のように言われるべきである。哲学者によって知性
 的徳とされている知恵は、人間の理性によって探求されうるものである限り
 において、神的なものを考察する。しかし神学的徳は、それが人間の理性を
 越える限りにおいて,それらにかかわる。

 
\\



{\scshape Ad tertium dicendum} quod, licet caritas sit amor, non tamen
omnis amor est caritas. Cum ergo dicitur quod omnis virtus est ordo
amoris, potest intelligi vel de amore communiter dicto; vel de amore
caritatis. Si de amore communiter dicto, sic dicitur quaelibet virtus
esse ordo amoris, inquantum ad quamlibet cardinalium virtutum
requiritur ordinata affectio, omnis autem affectionis radix et
principium est amor, ut supra dictum est. Si autem intelligatur de
amore caritatis, non datur per hoc intelligi quod quaelibet alia
virtus essentialiter sit caritas, sed quod omnes aliae virtutes
aliqualiter a caritate dependeant, ut infra patebit.

&

 第三異論に対しては以下のように言われるべきである。たしかに愛徳は愛で
 あるが、すべての愛が愛徳なのではない。ゆえにすべての徳は愛の秩序で在
 ると言われるとき、それは共通的に言われる愛について、または、愛徳であ
 る愛について理解されうる。もし共通的に言われる愛についてであれば、そ
 の場合には、枢要徳のどれにも秩序付けられた情動が必要とされるが、上述
 の如くすべての情動の根や根源が愛である限りにおいて、どんな徳も愛の秩
 序であると言われる。他方で、愛徳である愛について理解されるのならば、
 このことによってどんな他の徳も本質的に愛徳であるということではなく、
 すべての他の徳は何らかの仕方で愛徳に依存するということが理解される。
 これは後に明らかにされる。

\end{longtable}
\newpage




\rhead{a.~3}
\begin{center}
{\Large {\bf ARTICULUS TERTIUS}}\\
{\large UTRUM CONVENIENTER FIDES, SPES ET CARITAS PONANTUR VIRTUTES THEOLOGICAE}\\
{\footnotesize II${^a}$II${^ae}$, q.17, a.6; III {\itshape Sent.}, d.23, q.1, a.5; d.26, q.2, a.3, qu$^{a}$1; {\itshape De Virtut.}, q.1, a.10, 12; I {\itshape Cor.}, cap.13, lect.2, 4.}\\
{\Large 第三項\\信仰、希望、愛徳が神学的徳に指定されるのは適切か}
\end{center}

\begin{longtable}{p{21em}p{21em}}

{\scshape Ad tertium sic proceditur}. Videtur quod inconvenienter
ponantur tres virtutes theologicae, fides, spes et caritas. Virtutes
enim theologicae se habent in ordine ad beatitudinem divinam, sicut
inclinatio naturae ad finem connaturalem. Sed inter virtutes ordinatas
ad finem connaturalem, ponitur una sola virtus naturalis, scilicet
intellectus principiorum. Ergo debet poni una sola virtus theologica.


&

 第三項の問題へ、議論は以下のように進められる。信仰、希望、愛徳という
 三つの神学的徳は適切に指定されていないと思われる。理由は以下の通り。
 神学的徳は神の至福への秩序においてある。それはちょうど、本性の傾向性
 が共本性的な目的への秩序においてあるのと同様である。しかるに共本性的
 な目的へ秩序付けられる諸徳の中にはただ一つの自然本性的徳が指定される
 のであり、それは諸原理についての直知である。ゆえにただ一つの神学的徳
 が指定されるべきである。
 
\\



2.~{\scshape Praeterea}, theologicae virtutes sunt perfectiores
virtutibus intellectualibus et moralibus. Sed inter intellectuales
virtutes fides non ponitur, sed est aliquid minus virtute, cum sit
cognitio imperfecta. Similiter etiam inter virtutes morales non
ponitur spes, sed est aliquid minus virtute, cum sit passio. Ergo
multo minus debent poni virtutes theologicae.

&

 さらに、神学的徳は知性的徳と道徳的徳よりも完全である。しかるに知性的
 徳の中に信仰は指定されず、むしろそれは不完全な認識なので徳よりも小さ
 い何かである。同様にまた、道徳的徳の中に希望は指定されず、それはむし
 ろ、情念なので徳よりも小さい何かである。ゆえに、ましてやそれらは神学
 的徳とされるべきでない。
 
\\



3.~{\scshape Praeterea}, virtutes theologicae ordinant animam hominis
ad Deum. Sed ad Deum non potest anima hominis ordinari nisi per
intellectivam partem, in qua est intellectus et voluntas. Ergo non
debent esse nisi duae virtutes theologicae, una quae perficiat
intellectum, alia quae perficiat voluntatem.


&

 さらに、神学的徳は人間の愛を神へ秩序付ける。しかし神へ人間の魂が秩序
 付けられるのは、その中に知性と意志がある知性的部分によるしかない。ゆ
 えに二つの神学的徳しかあるべきでない。すなわち、一つは知性を完成する
 もの、もう一つは意志を完成させるものである。
 
\\



{\scshape Sed contra est} quod apostolus dicit, {\itshape I ad
Cor}.~{\scshape xiii}, {\itshape nunc autem manent fides, spes,
caritas, tria haec}.

&

 しかし反対に、使徒は『コリントの信徒への手紙一』第8章で「しかし今、信
 仰と希望と愛徳というこの三つが残っている\footnote{「それゆえ、信仰と、
 希望と、愛、この三つは、いつまでも残ります。その中で最も大いなるもの
 は、愛です。」(13:13)}」と言っている。\label{fides_spes_caritas}
 
\\



{\scshape Respondeo dicendum} quod, sicut supra dictum est, virtutes
theologicae hoc modo ordinant hominem ad beatitudinem supernaturalem,
sicut per naturalem inclinationem ordinatur homo in finem sibi
 connaturalem.


&

 解答する。以下のように言われるべきである。上述の如く、神学的徳は、ちょ
 うど人間が自らに共本性的である目的へ自然本性的傾向性によって秩序付け
 られるような仕方で、人間を本性を越えた至福へと秩序付ける。
 
\\

 Hoc autem contingit secundum duo. Primo quidem, secundum rationem vel
intellectum, inquantum continet prima principia universalia cognita
nobis per naturale lumen intellectus, ex quibus procedit ratio tam in
speculandis quam in agendis. Secundo, per rectitudinem voluntatis
naturaliter tendentis in bonum rationis.

&

 しかしこのことは二通りの仕方で生じる。一つには、理性あるいは知性に即
 して、普遍的な第一諸原理が知性の自然の光を通して私たちに知られる限り
 において、それらに基づいて理性が観照的なことだけでなくなされるべき事
 柄においても進んでいく、という仕方である。第二に、自然本性的に理性の
 善へと向かう意志の正しさを通してである。

 
\\


Sed haec duo deficiunt ab ordine beatitudinis supernaturalis; secundum
illud {\itshape I ad Cor}.~{\scshape ii}, oculus non vidit, et auris
non audivit, et in cor hominis non ascendit, quae praeparavit Deus
diligentibus se.


&

 しかしこの二つは、かの『コリントの信徒への手紙一』第2章「目は見ず耳は
 聞こえず、人間の心へ上ってこないことを、神は神を愛する人々に準備した」
 \footnote{「こう書いてあるとおりです。/「目が見もせず、耳が聞きもせ
 ず/人の心に思い浮かびもしなかったことを/神はご自分を愛する者たちに
 準備された。」(2:9)}によれば、本性を越えた至福への秩序に欠ける。
 
\\


 Unde oportuit quod quantum ad utrumque, aliquid homini
supernaturaliter adderetur, ad ordinandum ipsum in finem
supernaturalem. Et primo quidem, quantum ad intellectum, adduntur
homini quaedam principia supernaturalia, quae divino lumine capiuntur,
et haec sunt credibilia, de quibus est fides.

&

 したがって、このどちらについても、それを本性を越えた目的へ秩序付ける
 ために、何かが人間に本性を越えた仕方で加えられることが必要であった。
 そして第一に、知性については人間に本性を越えたある種の原理が付加され、
 それらは神の光によって受け取られる。それらは信じられうることであり、
 それについて信仰がある。
 
\\


 Secundo vero, voluntas ordinatur in illum finem et quantum ad motum
intentionis, in ipsum tendentem sicut in id quod est possibile
consequi, quod pertinet ad spem, et quantum ad unionem quandam
spiritualem, per quam quodammodo transformatur in illum finem, quod
fit per caritatem. Appetitus enim uniuscuiusque rei naturaliter
movetur et tendit in finem sibi connaturalem, et iste motus provenit
ex quadam conformitate rei ad suum finem.

&

 第二に、意志がかの目的へ、獲得できるものへ向かう意図の運動に関して秩
 序付けられるが、これは希望に属する。また、霊的なある種の合一に関して、
 それによってあるしかたでその目的へと変容されるが、これは愛徳を通して
 行われる。というのも、各々の事物の欲求には、本性的に、自分に共本性的
 な目的へと動き向かうことが属するが、この運動は、事物の自らの目的への
 一致から生じるからである。
 
\\



{\scshape Ad primum ergo} dicendum quod intellectus indiget speciebus
intelligibilibus, per quas intelligat, et ideo oportet quod in eo
ponatur aliquis habitus naturalis superadditus potentiae. Sed ipsa
natura voluntatis sufficit ad naturalem ordinem in finem, sive quantum
ad intentionem finis, sive quantum ad conformitatem ad ipsum. Sed in
ordine ad ea quae supra naturam sunt, ad nihil horum sufficit natura
potentiae. Et ideo oportet fieri superadditionem habitus
supernaturalis quantum ad utrumque.

&

 第一異論に対しては、それゆえ、以下のように言われるべきである。知性は
 可知的形象を必要とし、それを通して知性認識する。ゆえに知性の中に能力
 の上に加えられた本性的な何らかの習慣がなければならない。しかし意志の
 本性は、目的への意図に関しても、目的への一致に関しても、目的への本性
 的な秩序のために十分である。しかし、本性を越えてあるものへの秩序にお
 いて、これらのどれに対しても、能力の本性は十分でない。ゆえに、両者に
 ついて、自然本性を越えた習慣の不可がなされる必要がある。
 
\\

{\scshape Ad secundum dicendum} quod fides et spes imperfectionem
quandam important, quia fides est de his quae non videntur, et spes de
his quae non habentur. Unde habere fidem et spem de his quae subduntur
humanae potestati, deficit a ratione virtutis. Sed habere fidem et
spem de his quae sunt supra facultatem naturae humanae, excedit omnem
virtutem homini proportionatam; secundum illud {\itshape I ad
Cor}.~{\scshape i}, quod {\itshape infirmum est Dei, fortius est
hominibus}.

&

 第二異論に対しては以下のように言われるべきである。信仰と希望はある種
 の不完全性を含んでいる。なぜなら信仰は見えていないものに関わり、希望
 は獲得されていないものに関わるからである。したがって、人間の権能のも
 とにある事柄について信仰や希望を持つことは徳の性格に届かない。しかし、
 人間の本性の機能を越える事柄について信仰や希望を持つことは、人間に比
 例するすべての徳を越える。これは『コリントの信徒への手紙一』第1章「神
 の弱さは人間よりも強い」\footnote{「なぜなら、神の愚かさは人よりも賢
 く、神の弱さは人よりも強いからです。」(1:25)}による。
 
\\



{\scshape Ad tertium dicendum} quod ad appetitum duo pertinent,
scilicet motus in finem; et conformatio ad finem per amorem. Et sic
oportet quod in appetitu humano duae virtutes theologicae ponantur,
scilicet spes et caritas.

&

 第三異論に対しては以下のように言われるべきである。欲求には二つのこと
 が属する。すなわち目的への運動と、愛による目的への一致である。この意
 味で、人間の欲求の中には二つの神学的徳が指定される。すなわち希望と愛
 徳とである。
 
\\


\end{longtable}
\newpage

\rhead{a.~4}
\begin{center}
{\Large {\bf ARTICULUS QUARTUS}}\\
{\large UTRUM FIDES SIT PRIOR SPE, ET SPES CARITATE}\\
{\footnotesize II$^{a}$II$^{ae}$, q.4, a.7; q.17, a.7, 8; III {\itshape Sent.}, d.23, q.2, a.5; d.26, q.2, a.3, qu$^{a}$2; {\itshape De Virtut.}, q.4, a.3.}\\
{\Large 第四項\\信仰は希望に先立ち、希望は愛徳に先立つか}
\end{center}

\begin{longtable}{p{21em}p{21em}}

{\scshape Ad quartum sic proceditur}. Videtur quod non sit hic ordo
theologicarum virtutum, quod fides sit prior spe, et spes prior
caritate. Radix enim est prior eo quod est ex radice. Sed caritas est
radix omnium virtutum; secundum illud {\itshape ad Ephes}.~{\scshape
iii}, {\itshape in caritate radicati et fundati}. Ergo caritas est
prior aliis.


&

 第四項の問題へ、議論は以下のように進められる。神学的徳の順序は項では
 ない、すなわち、信仰が希望より先で、希望が愛徳よりも先ではないと思わ
 れる。理由は以下の通り。根は根からでるものよりも先である。しかるに愛
 徳はすべての徳の根である。これはかの『エフェソの信徒への手紙』第3章
 「愛徳において根付き基づく」\footnote{「あなたがたの信仰によって、キ
 リストがあなたがたの心の内に住んでくださいますように。あなたがたが愛
 に根ざし、愛に基づく者となることによって、」(3:17)}による。ゆえに愛徳
 は他のものよりも先である。

 \\



2.~{\scshape Praeterea}, Augustinus dicit, in I {\itshape de
Doct.~Christ}., {\itshape non potest aliquis diligere quod esse non
crediderit. Porro si credit et diligit, bene agendo efficit ut etiam
speret}. Ergo videtur quod fides praecedat caritatem, et caritas spem.


&

さらに、アウグスティヌスは『キリスト教の教え』第1巻の中で次のように書
いている。「存在していることを信じていないものをだれも愛することはでき
ない。さらにもし信じて愛しているならば、善く行うことによって、希望して
もいることを示す」。ゆえに信仰が愛徳に先行し、愛徳が希望に先行すると思
われる。

 \\



3.~{\scshape Praeterea}, amor est principium omnis affectionis, ut
supra dictum est. Sed spes nominat quandam affectionem; est enim
quaedam passio, ut supra dictum est. Ergo caritas, quae est amor, est
prior spe.


&

 さらに、上述の如く、愛はすべての情動の根源である。しかるに希望とはあ
 る種の情動の名前である。なぜなら、上述の如く\footnote{Q.23, a.4.}、そ
 れはある種の情念だからである。ゆえに愛である愛徳は、希望よりも先であ
 る。


 \\



{\scshape Sed contra est} ordo quo apostolus ista enumerat, dicens,
{\itshape nunc autem manent fides, spes, caritas}.

&

 しかし反対に、使徒がこれを数え上げる順番がある。いわく、「今、信仰、
 希望、愛徳が留まっている」。\footnote{\pageref{fides_spes_caritas}ページの反対異論を参照。}


 \\


 {\scshape Respondeo dicendum} quod duplex est ordo, scilicet
 generationis, et perfectionis. Ordine quidem generationis, quo
 materia est prior forma, et imperfectum perfecto, in uno et eodem;
 fides praecedit spem, et spes caritatem, secundum actus (nam habitus
 simul infunduntur).

&

 解答する。以下のように言われるべきである。順序には二通りあり、すなわ
 ち生成の順序と完全性の順序である。生成の順序では、一つで同一のものに
 おいて、質料が形相よりも、不完全なものが完全なものよりも先であるが、
 作用に即して信仰が希望に先行し、希望は愛徳に先行する(というのも、習
 慣は同時に注がれるので)。


 \\



 Non enim potest in aliquid motus appetitivus tendere vel sperando vel
 amando, nisi quod est apprehensum sensu aut intellectu. Per fidem
 autem apprehendit intellectus ea quae sperat et amat. Unde oportet
 quod, ordine generationis, fides praecedat spem et caritatem.

&

 理由は以下の通り。欲求の運動が希望したり愛したりすることで何かへ向か
 うには、それが感覚や知性によって捉えられていなければならない。したがっ
 て、生成の順序では、信仰が希望と愛徳に先行する。


 \\


 Similiter autem ex hoc homo aliquid amat, quod apprehendit illud ut
 bonum suum. Per hoc autem quod homo ab aliquo sperat se bonum
 consequi posse, reputat ipsum in quo spem habet, quoddam bonum
 suum. Unde ex hoc ipso quod homo sperat de aliquo, procedit ad
 amandum ipsum. Et sic, ordine generationis, secundum actus, spes
 praecedit caritatem.

&

 同様に、人間はそれを自らの善と捉えることに基づいて、それを愛する。し
 かるに、人間があるものによって、自分が善を獲得することができることを
 希望することを通して、そこにおいて希望を持つそのものを自らの善だと評
 価する。したがって、人間が何かについて希望を持つことに基づいて、その
 ものを愛するということへと進む。そしてこのようにして、生成の順序では、
 作用に即して言えば、希望が愛徳に先行する。


 \\

  Ordine vero perfectionis, caritas praecedit fidem et spem, eo quod
 tam fides quam spes per caritatem formatur, et perfectionem virtutis
 acquirit. Sic enim caritas est mater omnium virtutum et radix,
 inquantum est omnium virtutum forma, ut infra dicetur.

&

 他方、完全性の順序では、愛徳が信仰と希望に先行する。なぜなら、希望と
 同様に信仰も、愛徳によって形成され、徳の完全性を獲得するからである。
 実際このようにして、後で述べられるように、愛徳はすべての徳の形相であ
 る限りにおいて、すべての徳の母であり根である。


 \\



Et per hoc patet responsio ad primum.


&

そしてこれによって第一異論への解答は明らかである。

 \\



{\scshape Ad secundum dicendum} quod Augustinus loquitur de spe qua
quis sperat ex meritis iam habitis se ad beatitudinem perventurum,
quod est spei formatae, quae sequitur caritatem. Potest autem aliquis
sperare antequam habeat caritatem, non ex meritis quae iam habet, sed
quae sperat se habiturum.


&

 第二異論に対しては以下のように言われるべきである。アウグスティヌスは、
 すでに獲得した功徳から至福へ到達することを希望するその希望について語っ
 ている。これは形成された希望に属するのであり、この希望は愛徳に後続す
 る。しかし、人は愛をもつ前に、すでに持っている功徳に基づいてではなく、
 持つことを希望する功徳に基づいて希望するということが可能である。


 \\



{\scshape Ad tertium dicendum} quod, sicut supra dictum est, cum de
passionibus ageretur, spes respicit duo. Unum quidem sicut principale
obiectum, scilicet bonum quod speratur. Et respectu huius, semper amor
praecedit spem, nunquam enim speratur aliquod bonum nisi desideratum
et amatum. 


&

 第三異論に対しては以下のように言われるべきである。前に情念について論
 じられたときに述べられたように、希望は二つのことに関係する。一つは主
 要な対象としての希望されている善である。そしてこれに関して、常に愛が
 希望に先行する。なぜなら欲求され愛されないならばどんな善も希望されな
 いからである。

\\

Respicit etiam spes illum a quo se sperat posse consequi
bonum. Et respectu huius, primo quidem spes praecedit amorem; quamvis
postea ex ipso amore spes augeatur. Per hoc enim quod aliquis reputat
per aliquem se posse consequi aliquod bonum, incipit amare ipsum, et
ex hoc ipso quod ipsum amat, postea fortius de eo sperat.

 &

 さらに希望は、それによって善を獲得することができるのを希望するものに
 関係する。そしてこれに関しては、第一に希望は愛に先行する。ただし、後
 でその愛に基づいて希望が増大されることはあるが。たとえば、ある人が、
 だれかを通して、自分が何らかの善を獲得できると評価することによって、
 その人を愛し始める。そしてその人を愛すること自体によって、後でより強
 くその人を希望するように。

\end{longtable}
\end{document}