\documentclass[10pt]{jsarticle} % use larger type; default would be 10pt
%\usepackage[utf8]{inputenc} % set input encoding (not needed with XeLaTeX)
%\usepackage[round,comma,authoryear]{natbib}
%\usepackage{nruby}
\usepackage{okumacro}
\usepackage{longtable}
%\usepqckage{tablefootnote}
\usepackage[polutonikogreek,english,japanese]{babel}
%\usepackage{amsmath}
\usepackage{latexsym}
\usepackage{color}

%----- header -------
\usepackage{fancyhdr}
\lhead{{\it Summa Theologiae} III, q.~75}
%--------------------

\bibliographystyle{jplain}

\title{{\bf TERTIA PARS}\\{\HUGE Summae Theologiae}\\Sancti Thomae
Aquinatis\\{\sffamily QUEAESTIO SEPTUAGESIMAQUINTA}\\DE CONVERSIONE
PANIS ET VINI IN CORPUS ET SANGUINEM CHRISTI}
\author{Japanese translation\\by Yoshinori {\sc Ueeda}}
\date{Last modified \today}


%%%% コピペ用
%\rhead{a.~}
%\begin{center}
% {\Large {\bf }}\\
% {\large }\\
% {\footnotesize }\\
% {\Large \\}
%\end{center}
%
%\begin{longtable}{p{21em}p{21em}}
%
%&
%
%
%\\
%\end{longtable}
%\newpage



\begin{document}
\maketitle
\pagestyle{fancy}

\begin{center}
{\Large 第75問\\パンとワインのキリストの身体と血への変容について}
\end{center}

\begin{longtable}{p{21em}p{21em}}
Deinde considerandum est de conversione panis et vini in corpus et
 sanguinem Christi. Et circa hoc quaeruntur octo. 

\begin{enumerate}
 \item utrum substantia panis et vini remaneat in hoc sacramento post
 consecrationem.
 \item utrum annihiletur.
 \item utrum convertatur in corpus et sanguinem Christi.
 \item utrum remaneant ibi accidentia post conversionem.
 \item utrum remaneat ibi forma substantialis.
 \item utrum conversio ista fiat subito.
 \item utrum sit miraculosior omni alia mutatione.
 \item quibus verbis convenienter exprimi possit.
\end{enumerate}

&

次にパンとワインのキリストの身体と血への変容について考察されるべきであ
 る。これについて八つのことが問われる。

\begin{enumerate}
 \item パンとワインの実体はこの秘蹟において聖別の後に残存するか。
 \item 無に帰するか。
 \item キリストの身体と血へ変容するか。
 \item この変容の後、そこに附帯性が残存するか。
 \item そこに実体的形相は残存するか。
 \item その変容は瞬間的に起こるか。
 \item 他のすべての変化よりも奇跡的か。
 \item どの言葉によって適切に表現されうるか。
\end{enumerate}


\\



\end{longtable}
\newpage
\rhead{a.~1}
\begin{center}
{\Large {\bf ARTICULUS PRIMUS}}\\
{\large UTRUM IN HOC SACRAMENTO SIT CORPUS CHRISTI SECUNDUM\\
 VERITATEM, VEL SOLUM SECUNDUM FIGURAM VEL SICUT IN SIGNO}\\
{\footnotesize IV {\itshape Sent.}, dist.10, art.1; IV {\itshape
 SCG}, cap.61 sqq; {\itshape De Ven.~Sacram.~Altar.~serm.} 11;
 {\itshape In Matth.}, cap.26; {\itshape In Ioan.}, cap.6,
 lect.6; I {\itshape ad Cor.}, cap.11, lect.5.}\\
{\Large 第一項\\この秘蹟においてキリストの体は真理においてあるか、\\それ
 ともたんに形においてのみあるいはしるしにおけるものとしてあるか}
\end{center}

\begin{longtable}{p{21em}p{21em}}

{\scshape Ad primum sic proceditur}. Videtur quod in hoc sacramento non sit
 corpus Christi secundum veritatem, sed solum secundum figuram, vel
 sicut in signo. Dicitur enim Ioan.~{\scshape VI} quod, cum dominus dixisset,
 {\itshape Nisi manducaveritis carnem filii hominis et biberitis eius sanguinem},
 etc., {\itshape multi ex discipulis eius audientes dixerunt, Durus est hic
 sermo}:  quibus ipse : {\itshape Spiritus est qui vivificat, caro non prodest
 quidquam}. Quasi dicat, secundum expositionem Augustini, {\itshape super quartum
 Psalmum} : {\itshape Spiritualiter intellige quae locutus sum. Non hoc corpus
 quod videtis manducaturi estis, et bibituri illum sanguinem quem
 fusuri sunt qui me crucifigent. Sacramentum aliquod vobis
 commendavi. Spiritualiter intellectum vivificabit vos, caro autem non
 prodest quidquam}.



&

第一項の問題へ議論は以下のように進められる。この秘蹟においてキリストの
 身体は真理においてあるのではなく、たんに形においてあるいはしるしにお
 けるものとしてあると思われる。理由は以下の通り。
『ヨハネによる福音書』第6章で主が「もしあなたたちが人間の子の肉を食べ
 彼の血を飲まなかったならば云々」\footnote{「イエスは言われた。「よくよく言っておく。人の子の肉を食べ、その血を飲まなければ、あなたがたの内に命はない。私の肉を食べ、私の血を飲む者は、永遠の命を得、私はその人を終わりの日に復活させる。」」(6:53-54)}と言ったとき「弟子たちの多くがそれを
 聞いて「これは理解しがたい話だ」\footnote{「弟子たちの多くの者はこれを聞いて言った。「これはひどい話だ。誰が、こんなことを聞いていられようか。」」(6:60)}と言ったが、かれらに主は「活かすのは
 霊であり肉は何の役にも立たない」\footnote{「命を与えるのは霊である。
 肉は何の役にも立たない。私があなたがたに話した言葉は霊であり、命であ
 る。 」(6:63)}と言った。『第四詩編講解』のアウグスティヌスによればこ
 れは次のような意味である。「私が語ったことを霊的に理解しなさい。あなたたちが見ているこの身
 体を食べるだろうとか、私を十字架につける者たちが流させるかの血を飲む
 だろうということではない。私はある秘蹟をあなたたちに与えた。霊的に理
 解されたものがあなたたちを活かすだろうが、肉は何の役にも立たない」。



\\



2. {\scshape Praeterea}, Dominus dicit, Matth.~ult.: {\itshape Ecce, ego
 vobiscum sum omnibus diebus usque ad consummationem saeculi}, quod
 exponens Augustinus dicit :  {\itshape Donec saeculum finiatur,
 sursum est dominus, sed tamen et hic nobiscum est veritas
 dominus. Corpus enim in quo resurrexit, uno in loco esse oportet,
 veritas autem eius ubique diffusa est}. Non ergo secundum veritatem
 est corpus Christi in hoc sacramento, sed solum sicut in signo.


&

さらに、主は『マタイによる福音書』の最後で次のように言っている。「見よ、
 私はあなたたちと共に世々の尽きるまでずっといる」\footnote{「あなたが
 たに命じたことをすべて守るように教えなさい。私は世の終わりまで、いつ
 もあなたがたと共にいる。」 」(28:20)}。アウグスティヌスはこれを解説し
 て次のように言う。「世が尽きるまで主は上にいるが、しかし真理である主
 は私たちと共にここにもいる。なぜなら、それにおいて復活した身体は一つ
 の場所に存在しなければならないが、彼の真理は至る所に広がっているから
 である」。
 ゆえに、キリストの身体が真理においてこの秘蹟の中にあるのではなく、た
 だしるしにおいてあるものとしてある。


\\




3. {\scshape Praeterea}, nullum corpus potest esse simul in pluribus locis, cum nec
 Angelo hoc conveniat, eadem enim ratione posset esse ubique. Sed
 corpus Christi est verum corpus, et est in caelo. Ergo videtur quod
 non sit secundum veritatem in sacramento altaris, sed solum sicut in
 signo.


&


さらに、どんな物体も複数の場所に同時に在ることはできない。天使にもそれ
 はできないのだから。というのも同じ理由で至る所に在ることができただろ
 うから。しかしキリストの身体は真の物体(身体)でありそれは天にある。ゆえに真理にお
 いて祭壇の秘蹟においてあるのではなく、ただしるしにおけるものとしてあ
 る。

\\




4. {\scshape Praeterea}, sacramenta Ecclesiae ad utilitatem fidelium ordinantur. Sed
secundum Gregorium, in quadam homilia, regulus reprehenditur quia
{\itshape quaerebat corporalem Christi praesentiam}. Apostoli etiam impediebantur
recipere spiritum sanctum propter hoc quod affecti erant ad eius
praesentiam corporalem, ut Augustinus dicit, super illud Ioan. XVI, {\itshape si
non abiero, Paraclitus non veniet ad vos}. Non ergo Christus secundum
praesentiam corporalem est in sacramento altaris.

&

さらに、教会の秘蹟は信仰に属する事柄の有用性へと秩序づけられている。し
 かしグレゴリウスのある講話で、支配者は「キリストの身体的な現前を求め
 る」せいで非難される。さらに使徒たちは、彼の身体的な現前を熱望したた
 めに聖霊を受け取ることを妨げられた。これはアウグスティヌスが『ヨハネ
 福音書注解』14章「もし私が離れていかないならば、擁護者があなたたちの
 もとに来ないだろう」について言っているようにである。ゆえにキリストは
 身体的な現前において祭壇の秘蹟においてあるのではない。


\\




{\scshape Sed contra est} quod Hilarius dicit, in VIII {\itshape de Trin}. : {\itshape De veritate
carnis et sanguinis Christi non est relictus ambigendi locus. Nunc et
ipsius domini professione, et fide nostra, caro eius vere est cibus et
sanguis eius vere est potus}. Et Ambrosius dicit, VI {\itshape de
 sacramentis} : {\itshape Sicut verus est Dei filius dominus Iesus Christus, ita vera Christi
caro est quam accipimus, et verus sanguis eius est potus}.

&

しかし反対に、ヒラリウスは『三位一体論』第8巻で「キリストの肉と血につ
 いて疑うべきところは残っていない。今や主自身の宣言と私たちの信仰によっ
 て、彼の肉が真に食物であり彼の血は真に飲み物である」と述べている。ま
 たアンブロシウスは『秘蹟について』第6巻で「主イエス・キリストが真の神
 の子であるように、私たちが受け取るのは真のキリストの肉であり、彼の知
 は真の飲み物である」と述べている。


\\




{\scshape Respondeo dicendum} quod verum corpus Christi et sanguinem esse in hoc
sacramento, non sensu deprehendi potest, sed sola fide, quae
auctoritati divinae innititur. Unde super illud Luc.~{\scshape xxii}, {\itshape hoc est
corpus meum quod pro vobis tradetur}, dicit Cyrillus : {\itshape Non dubites an
hoc verum sit, sed potius suscipe verba salvatoris in fide, cum enim
sit veritas, non mentitur}. 


&

解答する。以下のように言われるべきである。キリストの真の身体と血がこの
 秘蹟においてあることは感覚によってではなく、神の権威に依拠する信仰に
 よってのみ把握されうる。このことから、かの『ルカによる福音書』22章の
 「これはあなたたちに伝えられる私の肉である」\footnote{それから、イエ
 スはパンを取り、感謝の祈りを献げてそれを裂き、使徒たちに与えて言われ
 た。「これは、あなたがたのために与えられる私の体である。私の記念とし
 てこのように行いなさい。」(22:19)}についてキュリッルスは「これは本当
 かと疑ってはならない。むしろ救い主の言葉を信仰において保持しなさい。
 彼は真理であり嘘をつかないからである」と述べている。


\\


Hoc autem conveniens est, primo quidem,
perfectioni novae legis. Sacrificia enim veteris legis illud verum
sacrificium passionis Christi continebant solum in figura, secundum
illud {\itshape Heb}.~{\scshape x} : {\itshape Umbram habens lex futurorum bonorum, non ipsam rerum
imaginem}. Et ideo oportuit ut aliquid plus haberet sacrificium novae
legis a Christo institutum, ut scilicet contineret ipsum passum, non
solum in significatione vel figura, sed etiam in rei veritate. Et ideo
hoc sacramentum, quod ipsum Christum realiter continet, ut Dionysius
dicit, {\scshape iii} cap. {\itshape Eccles.~Hierar}., est {\itshape perfectivum omnium sacramentorum
aliorum, in quibus virtus Christi participatur}. 


&

ところで、このことは第一に新法の完全性にとって適切である。理由は以下の
 通り。かの『ヘブライ人への手紙』第10章「諸事物の像自体ではなく将来の
 善の影を持つ法」\footnote{「律法には、やがて来る良いことの影があるば
 かりで、そのものの実体はありません。ですから、年ごとに絶えず献げられ
 る同じいけにえによって、神に近づく人たちを完全な者にすることはできな
 いのです。」(10:1)}によれば、旧法の犠牲は、キリストの受難の真の犠牲を
 形においてのみ含んでいた。ゆえに、キリストによって据えられた新法の犠
 牲はそれに加えて何かを持たなければならなかった。すなわち、受難自体を、
 意味ないし形においてのみならず事物の真理においてもまた含まなければな
 らなかった。ゆえに、キリスト自身を現実に含むこの秘蹟は、ちょうどディオニュシウスが『天上階級論』第3章で言うように、
 キリストの力が参加する「他のすべての秘蹟を完成しうる」ものである。


\\


Secundo, hoc competit
caritati Christi, ex qua pro salute nostra corpus verum nostrae
naturae assumpsit. Et quia maxime proprium amicitiae est, {\itshape convivere
amicis}, ut philosophus dicit, IX {\itshape Ethic}., sui praesentiam corporalem
nobis repromittit in praemium, Matth.~{\scshape xxiv} : {\itshape Ubi fuerit corpus, illuc
congregabuntur et aquilae}. Interim tamen nec sua praesentia corporali
in hac peregrinatione destituit, sed per veritatem corporis et
sanguinis sui nos sibi coniungit in hoc sacramento. Unde ipse dicit,
Ioan.~{\scshape vi} : {\itshape Qui manducat meam carnem et bibit meum sanguinem, in me
manet et ego in eo}. Unde hoc sacramentum est maximae caritatis signum,
et nostrae spei sublevamentum, ex tam familiari coniunctione Christi
ad nos. 


&

第二に、これはキリストの愛徳に適合する。キリストはその愛徳に基づいて私
 たちを救済するために私たちの本性を持つ真の身体を取った。そして哲学者
 が『ニコマコス倫理学』第9巻で言うように、友愛にとって最も固有なことは
 「友人たちと共に生きること」なので、自らを身体的に現前させることを褒
 美として私たちに新たに約束した。『マタイによる福音書』24節「死体があ
 るところには鷹が集まるだろう」\footnote{「屍のある所には、禿鷲が集ま
 るものだ。」(24:28)}。
しかしこの旅路において自らの身体的現前を取り去ってなく、むしろ自らの身
 体と血の真理を通してこの秘蹟において私たちを自らに結び付けている。こ
 のことからキリスト自身が『ヨハネによる福音書』第4章で「私の肉を食べ私
 の血を飲む者は私の中に留まり私も彼の中に留まる」\footnote{「私の肉を
 食べ、私の血を飲む者は、私の内にとどまり、私もまたその人の内にとどま
 る。 」(6:56)}と述べている。したがって、この秘蹟は最大の愛徳のしるし
 であり、キリストの私たちへのこれほど親しい結びつきに基づく私たちの希
 望の称揚である。



\\


Tertio, hoc competit perfectioni fidei, quae, sicut est de
divinitate Christi, ita est de eius humanitate, secundum illud
Ioan.~{\scshape xiv} : {\itshape Creditis in Deum, et in me credite}. Et quia fides est
invisibilium, sicut divinitatem suam nobis exhibet Christus
invisibiliter, ita et in hoc sacramento carnem suam nobis exhibet
invisibili modo. 


&

第三に、これは信仰の完全性に適合する。それはかの『ヨハネによる福音書』
 第14章「あなたたちは神を信じているが私も信じなさい」\footnote{「心を
 騒がせてはならない。神を信じ、また私を信じなさい。」(14:1)}によれば、キリストの神性についてであ
 るのと同様に、彼の人間性についてでもある。そして信仰は見えないものに
 ついてあるから、キリストが自らの神性を私たちに見えない形で示すように、
 この秘蹟において自らの肉を見えない形で示している。

\\



Quae quidam non attendentes, posuerunt corpus et
sanguinem Christi non esse in hoc sacramento nisi sicut in signo. Quod
est tanquam haereticum abiiciendum, utpote verbis Christi
contrarium. Unde et Berengarius, qui primus inventor huius erroris
fuerat, postea coactus est suum errorem revocare, et veritatem fidei
confiteri.

&

これらのことに注意しない人々は、キリストの身体と血がこの秘蹟においてし
 るしにおけるものとしてしかないと考えた。これはキリストの言葉に反対の
 ものとして、異端として遺棄されるべきである。したがって、この誤りに最
 初に陥ったベレンガリウスは、その後強いられて自らの誤りを取り下げ、信仰の真理
 を告白した。


\\




{\scshape Ad primum ergo dicendum} quod ex hac auctoritate praedicti haeretici
occasionem errandi sumpserunt, male verba Augustini intelligentes. Cum
enim Augustinus dicit, {\itshape Non hoc corpus quod videtis manducaturi estis},
non intendit excludere veritatem corporis Christi, sed quod non erat
manducandum in hac specie in qua ab eis videbatur. Per hoc autem quod
subdit, {\itshape Sacramentum vobis aliquod commendavi, spiritualiter
intellectum vivificabit vos}, non intendit quod corpus Christi sit in
hoc sacramento solum secundum mysticam significationem, sed
spiritualiter dici, idest, invisibiliter et per virtutem
spiritus. Unde, {\itshape super Ioan}., exponens illud quod dicitur, {\itshape Caro non
prodest quidquam}, dicit : {\itshape Sed, quo modo illi intellexerunt. Carnem
quippe sic intellexerunt manducandam, quo modo in cadavere dilaniatur
aut in macello venditur, non quo modo spiritu vegetatur. Accedat
spiritus ad carnem, et prodest plurimum, nam, si caro nihil prodesset,
verbum caro non fieret, ut habitaret in nobis}.

&


第一異論に対しては、それゆえ、以下のように言われるべきである。
この権威に基づきアウグスティヌスの言葉を悪く理解して、前述の異端者
 たちは誤謬の機会を得た。つまり、アウグスティヌスが「あなたたちが見て
 いるこの身体を将来食べるのではなく」と言うとき、キリストの身体の真理
 を排除することを意図しているのではなく、彼らによって見られたその種に
 おいて食べられるべきでないということを意図していた。しかし、「あなた
 たちに私が与えた何らかの秘蹟であり、知性的に理解されたものがあなたた
 ちを活かすだろう」とその後に述べられていることは、キリストの身体が、
 ただ神秘的な意味表示においてのみこの秘蹟の中にあるということではなく、
 霊的に、すなわち見えないしかたで、霊の力によって語られているというこ
 とが意図されている。このことから、『ヨハネ福音書注解』で「肉は何の役
 にも立たない云々」を説明して以下のようにいっている。「しかし、彼らは
 どのように理解しただろうか。たしかに彼らは死肉において切り刻み食肉市
 場で売られるように肉が食べられると理解したのであり、霊によって活性化
 されるようなかたちで食べられると理解
 したのではない。霊は肉に近づき大いに役立つ。すなわち、もし肉が何の役
 にも立たなかったならば、肉という言葉が生じて私たちの中で使われたりは
 しなかっただろう」。

\\




{\scshape Ad secundum dicendum} quod verbum illud Augustini, et omnia similia,
sunt intelligenda de corpore Christi secundum quod videtur in propria
specie, secundum quod etiam ipse dominus dixit, Matth.~{\scshape xxvi}
 : {\itshape Me autem
non semper habebitis}. Invisibiliter tamen sub speciebus huius
sacramenti est ubicumque hoc sacramentum perficitur.

&

第二異論に対しては以下のように言われるべきである。
アウグスティヌスのかの言葉やその他の類似した言葉は、キリストの身体が固
 有の種において見られるかぎりで理解されるべきであり、このかぎりでは主
 自身も『マタイによる福音書』26章で「あなたたちは私を常に持つことはな
 いだろう」\footnote{「貧しい人々はいつもあなたがたと一緒にいるが、私
 はいつも一緒にいるわけではない。」(26:11)}と言っている。しかし、この
 秘蹟が完成されるところはどこでも、この秘蹟の種のもとで見えないしかた
 である。

\\




{\scshape Ad tertium dicendum} quod corpus Christi non est eo modo in sacramento
sicut corpus in loco, quod suis dimensionibus loco commensuratur, sed
quodam speciali modo, qui est proprius huic sacramento. Unde dicimus
quod corpus Christi est in diversis altaribus, non sicut in diversis
locis, sed {\itshape sicut in sacramento}. Per quod non intelligimus quod
Christus sit ibi solum sicut in signo, licet sacramentum sit in genere
signi, sed intelligimus corpus Christi esse ibi, sicut dictum est,
secundum modum proprium huic sacramento.

&

第三異論に対しては次のように言われるべきである。
キリストの身体は物体が場所にあるのとおなじしかたで秘蹟においてあるので
 はない。物体が場所にあるのはその物体の次元で場所によって測られるが、
 この秘蹟の場合にはこの秘蹟に固有の特別なしかたによってある。このこと
 から、私たちはキリストの身体がさまざまな祭壇においてあると言うが、これは
 さまざまな場所にという意味ではなく秘蹟においてという意味である。
このことによって私たちは、秘蹟はしるしの類の中にあるとはいえ、キリストが単にしるしにおいてあるようにそこにあ
 ると理解するのではなく、キリストの身体が、今述べたようにこの秘蹟に固
 有のしかたでそこにあると理解する。


\\




{\scshape Ad quartum dicendum} quod ratio illa procedit de praesentia corporis
Christi prout est praesens per modum corporis, idest prout est in sua
specie visibili, non autem prout spiritualiter, idest invisibiliter,
modo et virtute spiritus. Unde Augustinus dicit, {\itshape super
 Ioan}.~: {\itshape Si
intellexisti spiritualiter} verba Christi de carne sua, {\itshape spiritus et
vita tibi sunt, si intellexisti carnaliter, etiam spiritus et vita
sunt, sed tibi non sunt}.

&

第四異論に対しては以下のように言われるべきである。
この論は、キリストの身体の現前について、物体的なしかたで、つまり自らの
 見られうる種においてあるものとして進められていて、霊的に、すなわち霊
 のあり方と力によって見られないしかたでの現前について進められていない。
 このことからアウグスティヌスは『ヨハネ伝注解』で以下のように述べてい
 る。「もしあなたが」自らの肉についてのキリストの言葉を「霊的に理解し
 たならば、霊と命はあなたのものだが、もし肉的に理解したならば、そ
 れでも霊と生命はあるが、しかしあなたのものではない」。


\end{longtable}
\newpage

\rhead{a.~2}
\begin{center}
{\Large {\bf ARTICULUS SECUNDUS}}\\
{\large UTRUM IN HOC SACRAMENTO REMANEAT SUBSTANTIA PANIS ET VINI POST
 CONSECRATIONEM}\\
{\footnotesize IV {\itshape Sent.}, d.11, q.1, a.1, qu$^{a}$1; IV
 {\itshape SCG} cap.63; {\itshape In Matth.}, cap.26; I
 {\itshape ad Cor.}, cap.11, lect.4}\\
{\Large 第二項\\この秘蹟において聖別後にパンとワインの実体は残存するか}
\end{center}

\begin{longtable}{p{21em}p{21em}}

{\scshape Ad secundum sic proceditur}. Videtur quod in hoc sacramento remaneat
substantia panis et vini post consecrationem. Dicit enim Damascenus,
in libro IV : {\itshape Quia consuetudo est hominibus comedere panem et vinum,
coniugavit eis deitatem, et fecit ea corpus et sanguinem suum}. Et
infra : {\itshape Panis communicationis non panis simplex est, sed unitus
deitati}. Sed coniugatio est rerum actu existentium. Ergo panis et
vinum simul sunt in hoc sacramento cum corpore et sanguine Christi.

&

第二項の問題へ議論は以下のように進められる。
この秘蹟において聖別後にパンとワインの実体は残存すると思われる。理由は
 以下の通り。
ダマスケヌスは第五巻で「人々にパンとワインを与えることが慣わしなので、
 それらに神性を結び付け、それらを自らの身体と血にした」と述べている。
 そしてその後で「伝達のパンは単純なパンではなく神性に合一されたパンで
 ある」と述べている。しかるに、結び付けるとは現実に存在している事物を
 結び付けることである。ゆえにパンとワインは、この秘蹟においてキリスト
 の身体と血とともに存在している。


\\



2. {\scshape Praeterea}, inter Ecclesiae sacramenta debet esse conformitas. Sed in
aliis sacramentis substantia materiae manet, sicut in Baptismo
substantia aquae, et in confirmatione substantia chrismatis. Ergo et
in hoc sacramento substantia panis et vini manet.

&

さらに、教会の秘蹟の間には一致があるべきである。しかし、他の秘蹟におい
 ては質料の実体が残存している。たとえば洗礼においては水の実体が、堅信
 においては聖油が。ゆえにこの秘蹟においてもパンとワインの実体が残存し
 ている。


\\



3. {\scshape Praeterea}, panis et vinum assumitur in hoc sacramento inquantum
significat ecclesiasticam unitatem, prout {\itshape unus panis fit ex multis
granis, et unum vinum ex multis racemis}, ut Augustinus dicit, in libro
{\itshape de Symbolo}. Sed hoc pertinet ad ipsam substantiam panis et vini. Ergo
substantia panis et vini remanet in hoc sacramento.

&


さらに、パンとワインは、この秘蹟において教会の一性を表示するために採用
 されている。それはアウグスティヌスが『しるしについて』で「一つのパンが多くの穀物から、一杯のワインが多くのブドウか
 ら生じている」と言うようにである。しかしこのことはパンとワインの実体
 それ自体に属する。ゆえにパンとワインの実体はこの秘蹟において残存して
 いる。

\\



{\scshape Sed contra est} quod Ambrosius dicit, in libro {\itshape de
 Sacramentis} : {\itshape Licet
figura panis et vini videatur, nihil tamen aliud quam caro Christi et
sanguis post consecrationem credenda sunt}.

&

しかし反対に、アンブロシウスは『秘蹟について』という書物で次のように述
 べている。「パンとワインの形が見られるとしても、聖別の後にキリストの
 肉と血以外のものが信じられるべきではない」。

\\



{\scshape Respondeo dicendum} quod quidam posuerunt post consecrationem
substantiam panis et vini in hoc sacramento remanere. --- Sed haec positio
stare non potest. Primo quidem, quia per hanc positionem tollitur
veritas huius sacramenti, ad quam pertinet ut verum corpus Christi in
hoc sacramento existat. Quod quidem ibi non est ante
consecrationem. Non autem aliquid potest esse alicubi ubi prius non
erat, nisi per loci mutationem, vel per alterius conversionem in
ipsum, sicut in domo aliqua de novo incipit esse ignis aut quod illuc
defertur, aut quod ibi generatur. 

&

解答する。以下のように言われるべきである。
ある人々はこの秘蹟において聖別後にパンとワインの実体が残存すると主張し
 た。しかしこの立場は成立しえない。第一にこの立場によって、キリストの
 真の身体がこの秘蹟において存在しているという、この秘蹟の真理が否定さ
 れるからである。
聖別前、キリストの身体はそこに存在しない。しかし、あるものが、以前には
 存在しなかったところに存在しうるのは、場所の変化によるかあるいは他の
 もののそれへの転化による以外にはありえない。たとえばある家の中で新た
 に火が存在し始めるのはその火がどこかから持ち込まれたか、そこで起こさ
 れたかのいずれかである。


\\

Manifestum est autem quod corpus
Christi non incipit esse in hoc sacramento per motum localem. Primo
quidem, quia sequeretur quod desineret esse in caelo, non enim quod
localiter movetur, pervenit de novo ad aliquem locum, nisi deserat
priorem. Secundo, quia omne corpus localiter motum pertransit omnia
media, quod hic dici non potest. Tertio, quia impossibile est quod
unus motus eiusdem corporis localiter moti terminetur simul ad diversa
loca, cum tamen in pluribus locis corpus Christi sub hoc sacramento
simul esse incipiat. Et propter hoc relinquitur quod non possit aliter
corpus Christi incipere esse de novo in hoc sacramento nisi per
conversionem substantiae panis in ipsum. Quod autem convertitur in
aliquid, facta conversione, non manet. Unde relinquitur quod, salva
veritate huius sacramenti, substantia panis post consecrationem
remanere non possit.


&

しかるに、キリストの身体がこの秘蹟において場所的運動によって存在し始め
 ることがないのは明らかである。第一にそれは、もしそうだとすると天にい
 なかったであろうから。場所的に動くものは先の場所を離れることなしに新
 たにある場所に到来することはないからである。第二に、場所的に動かされ
 るすべての物体はすべての中間を渡るが、このことはここで言われ得ないか
 らである。第三に、場所的に動かされる同一の物体の一つの運動が、同時に
 さまざまな場所を終着地点とすることは不可能だが、この秘蹟において、キ
 リストの身体が同時に複数の場所に新たに存在し始めるからである。それゆ
 え、キリストの身体が新たにこの秘蹟において存在し始めるのは
 パンの実体がそれへ変容することによる以外にないことが帰結する。しかし、
 変容が起こった後、何かへ変容したものは残存しない。したがって、この秘
 蹟の真理を救うために、パンの実体は聖別後に残存しえないといういうこと
 が(消去法により)帰結する。


\\



Secundo, quia haec positio contrariatur formae
huius sacramenti, in qua dicitur, {\itshape Hoc est corpus meum}. Quod non esset
verum si substantia panis ibi remaneret, nunquam enim substantia panis
est corpus Christi. Sed potius esset dicendum, {\itshape Hic est corpus
meum}. 


&

第二に、この立場はこの秘蹟の形式に反する。そこでは「これは私の身体であ
 る」と言われているから。もしパンの実体がそこに残存していたらこのこと
 は真ではなかっただろう。なぜなら、パンの実体は決してキリストの身体で
 はないから。むしろ「ここに私の身体がある」と言われるべきだっただろう。


\\



Tertio, quia contrariaretur venerationi huius sacramenti, si
aliqua substantia esset ibi quae non posset adorari adoratione
latriae. 


&


第三に、もし神にのみ捧げられる礼拝によって拝まれえない実体がそこにあっ
 たならば、それはこの秘蹟の真理に反したであろうから。

\\



Quarto, quia contrariaretur ritui Ecclesiae, secundum quem
post corporalem cibum non licet sumere corpus Christi, cum tamen post
unam hostiam consecratam liceat sumere aliam. 

&

第四に、この立場は教会の典礼に反したであろう。これによれば、物体的な食
 物の後にキリストの身体をとることは許されないが、一つの聖体の後に別の
 聖体をとることは許されている。

\\



Unde haec positio
vitanda est tanquam haeretica.

&

したがって、この立場は異端として避けられるべきである。

\\



{\scshape Ad primum ergo dicendum} quod Deus coniugavit divinitatem suam, idest
divinam virtutem, pani et vino, non ut remaneant in hoc sacramento,
sed ut faciat inde corpus et sanguinem suum.

&

第一異論に対しては、それゆえ、以下のように言われるべきである。
神はパンとワインに自らの神性すなわち神の力を結び付けたのは、それらがこ
 の秘蹟において残存するようにではなく、そこから自らの身体と血が生じる
 ようにである。


\\



{\scshape Ad secundum dicendum} quod in aliis sacramentis non est ipse Christus
realiter, sicut in hoc sacramento. Et ideo in sacramentis aliis manet
substantia materiae, non autem in isto.

&

第二異論に対しては、以下のように言われるべきである。
他の秘蹟においては、この秘蹟においてのように現実にキリスト自身が存在す
 ることはない。ゆえに他の秘蹟において質料の実体は残るが、この秘蹟にお
 いては残らない。

\\



{\scshape Ad tertium dicendum} quod species quae remanent in hoc sacramento, ut
infra dicetur, sufficiunt ad significationem huius sacramenti, nam per
accidentia cognoscitur ratio substantiae.

&

第三異論に対しては以下のように言われるべきである。
この秘蹟において残存する種は、以下に述べられるように、この秘蹟の意味表
 示、すなわち附帯性によって実体の性格が認識されることにとっては十分である。

\\


\end{longtable}
\newpage



\rhead{a.~3}
\begin{center}
{\Large {\bf ARTICULUS TERTIUS}}\\
{\large UTRUM SUBSTANTIA PANIS, POST CONSECRATIONEM HUIUS SACRAMENTI,
 ANNIHILETUR, AUT IN PRISTINAM MATERIAM RESOLVATUR}\\
{\footnotesize IV {\itshape Sent.}, d.11, q.1, a.2; IV {\itshape SCG},
 cap.63; {\itshape Quodl}.~V, q.6, a.1; {\itshape In Matth.}, cap.26; I
 {\itshape ad Cor.}, cap.11, lect.5.}\\
{\Large 第三項\\パンの実体はこの秘蹟の聖別後に無化されるか\\あるいは元の
質料へ解体されるか}
\end{center}

\begin{longtable}{p{21em}p{21em}}


{\scshape Ad tertium sic proceditur}. Videtur quod substantia panis, post
consecrationem huius sacramenti, annihiletur, aut in pristinam
materiam resolvatur. Quod enim est aliquid corporale, oportet alicubi
esse. Sed substantia panis, quae est quiddam corporale, non manet in
hoc sacramento, ut dictum est, nec etiam est dare aliquem locum ubi
sit. Ergo non est aliquid post consecrationem. Igitur aut est
annihilata, aut in praeiacentem materiam resoluta.

&

第三項の問題へ議論は以下のように進められる。
パンの実体は、この秘蹟の聖別の後、無化される、あるいは元の質料へ解体さ
 れると思われる。理由は以下の通り。
物体的なものはどこかに存在しなければならない。しかし一種の物体的なもの
 であるパンの実体は、すでに述べられたとおり、この秘蹟において残存しな
 い。またそこに存在する何らかの場所を与えることもできない。
ゆえに、聖別の後それは何かではない。ゆえに無化されたか、あるいは元にあっ
 た質料へ解体されたかである。

\\



2. {\scshape Praeterea}, illud quod est terminus a quo in qualibet mutatione, non
remanet, nisi forte in potentia materiae, sicut, quando ex aere fit
ignis, forma aeris non manet nisi in potentia materiae; et similiter
quando ex albo fit nigrum. Sed in hoc sacramento substantia panis et
vini se habet sicut terminus a quo corpus autem vel sanguis Christi
sicut terminus ad quem, dicit enim Ambrosius, in libro {\itshape de
 Officiis} : {\itshape Ante benedictionem alia species nominatur, post benedictionem corpus
significatur}. Ergo, facta consecratione, substantia panis vel vini non
manet, nisi forte resoluta in suam materiam.

&


さらに、どんな変化においても、変化の出発点は質料の可能態においてでなけ
 れば留まらない。たとえば空気から火が生じるとき、空気の形相は質料の可
 能態においてしか留まらない。白から黒が生じるときも同様である。しかし
 この秘蹟においてパンとワインの実体は変化の出発点として関係し、キリス
 トの身体と血は変化の目的として関係する。実際、アンブロシウスは『義務
 論』で「聖別式の前に他の種で呼ばれるが、聖別式の後は身体が表示される」
 と述べている。ゆえに、聖別がなされると、パンやワインの実体は残存せず、
 むしろ自らの質料へと解体される。

\\



3. {\scshape Praeterea}, oportet alterum contradictoriorum esse verum. Sed haec est
falsa, {\itshape Facta consecratione, substantia panis vel vini est
aliquid}. Ergo haec est vera : {\itshape Substantia panis vel vini est nihil}.

&

さらに、矛盾するものの一方は真でなければならない。しかし「聖別が行われ
た後にパンや血の実体が何かである」は矛盾である。ゆえに、「パンやワインの
 実体は無である」が真である。

\\



{\scshape Sed contra est} quod Augustinus dicit, in libro {\itshape Octogintatrium
Quaestionum} : {\itshape Deus non est causa tendendi in non esse}. Sed hoc
sacramentum divina virtute perficitur. Ergo in hoc sacramento non
annihilatur substantia panis aut vini.

&

しかし反対に、アウグスティヌスは『八十三問題集』という書物の中で「神は
 非存在へ向かう原因ではない」と言っている。しかしこの秘蹟は神の力によっ
 て完成される。ゆえにこの秘蹟においてパンやワインの実体が無化されるこ
 とはない。


\\



{\scshape Respondeo dicendum} quod, quia substantia panis vel vini non manet in
hoc sacramento, quidam, impossibile reputantes quod substantia panis
vel vini in corpus vel sanguinem Christi convertatur, posuerunt quod
per consecrationem substantia panis vel vini vel resolvitur in
praeiacentem materiam, vel quod annihiletur. 


&

解答する。以下のように言われるべきである。
パンや血の実体がこの秘蹟において残存しないので、パンやワインの実体がキ
 リストの身体や血に変容されることが不可能だと考えたある人々は、聖別に
 よってパンや血の実体が元の質料へと解体されるとか、あるいは無化さ
 れると主張した。

\\



Praeiacens autem materia
in quam corpora mixta resolvi possunt, sunt quatuor elementa, non enim
potest fieri resolutio in materiam primam, ita quod sine forma
existat, quia materia sine forma esse non potest. Cum autem post
consecrationem nihil sub speciebus sacramenti remaneat nisi corpus et
sanguis, oportebit dicere quod elementa in quae resoluta est
substantia panis et vini, inde discedant per motum localem. Quod sensu
perciperetur. 



&

ところで、混合物体がそれへと解体されうる元の質料とは四つの元素である。
 なぜなら、質料は形相なしに存在できないので、形相なしに存在する形で第
 一質料へと解体されることは不可能だからである。しかし、聖別後秘蹟の種
 の元に身体と血以外のものは残存しないので、それへと解体された元素がパ
 ンとワインの実体であり、そこから場所的運動によって無くなっていく、と
 言わなければならなかっただろう。(もしこれが正しければ)この様子は感覚によって知覚されたであ
 ろう。


\\


Similiter etiam substantia panis vel vini manet usque ad
ultimum instans consecrationis. In ultimo autem instanti
consecrationis iam est ibi substantia vel corporis vel sanguinis
Christi, sicut in ultimo instanti generationis iam inest forma. Unde
non erit dare aliquod instans in quo sit ibi praeiacens materia. Non
enim potest dici quod paulatim substantia panis vel vini resolvatur in
praeiacentem materiam, vel successive egrediatur de loco
specierum. 


&

また同様に、パンやワインの実体は聖別の最後の瞬間まで残存する。しかし聖
 別の最後の瞬間には、そこにすでにキリストの身体や血の実体が存在する。
 ちょうど生成の最後の瞬間にはすでに形相が内在するように。したがって、
 そこに元の質料が存在する何らかの瞬間を与えることはできないであろう。
 なぜなら、パンやワインの実体が少しずつ元の質料に解体されるとか、継起
 的に種の場所からなくなっていく、とは言われえないからである。



\\

Quia, si hoc inciperet fieri in ultimo instanti suae
consecrationis, simul sub aliqua parte hostiae esset corpus Christi
cum substantia panis, quod est contra praedicta. Si vero incipiat
fieri ante consecrationem, erit dare aliquod tempus in quo sub aliqua
parte hostiae neque erit substantia panis, neque erit corpus Christi,
quod est inconveniens. 



&

なぜなら、もし聖別の最後の瞬間にこのことが起こり始めたならば、同時に、
 聖体のある部分においてキリストの身体がパンの実体と共にあったことにな
 るが、これは前に述べたことに反するからである。他方で、聖別の前にこれ
 が起こり始めるならば、聖体のある部分のもとで、パンの実体でもキリスト
 の身体でもないものがある時間を指定できることになるがこれは不都合であ
 る。

\\




Et hoc ipsimet perpendisse videntur. Unde
posuerunt aliud sub disiunctione, scilicet quod annihiletur. Sed nec
hoc potest esse. Quia non erit dare aliquem modum quo corpus Christi
verum incipiat esse in hoc sacramento, nisi per conversionem
substantiae panis in ipsum, quae quidem conversio tollitur, posita vel
annihilatione panis, vel resolutione in praeiacentem
materiam. 


&

そして彼らはこのことをよく考えたのだと思われる。それゆえ彼らはもう一
 つの選択肢を取り、それが無化されると主張した。しかしこれもまた不可能
 である。なぜなら、この秘蹟においてキリストの真の身体が存在し始めるし
 かたは、パンの実体がその身体へと変容することによる以外にはないが、
 無化や元の質料への解体だとされるならば、この変容が否定されるからである。


\\

Similiter etiam non est dare unde talis resolutio vel
annihilatio in hoc sacramento causetur, cum effectus sacramenti
significetur per formam; neutrum autem horum significatur per haec
verba formae, {\itshape Hoc est corpus meum}. 


&

また同様に、どこからこの解体や無化がこの秘蹟において原因されるかを指定
 することができない。なぜなら、秘蹟の結果は形相によって表示されるが、
 このどちらも形相というこの言葉によっては表示されないからである。「こ
 れは私の身体である」。

\\



Unde patet praedictam positionem esse falsam.

&

したがって、上述の立場が偽であることは明らかである。

\\

{\scshape Ad primum ergo dicendum} quod substantia panis vel vini, facta
consecratione, neque sub speciebus sacramenti manet, neque alibi. Non
tamen sequitur quod annihiletur, convertitur enim in corpus
Christi. Sicut non sequitur, si aer ex quo generatus est ignis, non
sit ibi vel alibi, quod sit annihilatus.

&

第一異論に対しては、それゆえ以下のように言われるべきである。
パンやワインの実体は、聖別が行われると、秘蹟の種の元にも、他のどこかにも
 残らない。しかし無化されるということは帰結せず、むしろキリストの身体
 へと変容される。ちょうど、火がそこから生成する空気が、そこにも他のと
 ころにもなく、また無化されるのでもないのと同様である。


\\



{\scshape Ad secundum dicendum} quod forma quae est terminus a quo, non
convertitur in aliam formam, sed una forma succedit alteri in
subiecto, et ideo prima forma non remanet nisi in potentia
materiae. Sed hic substantia panis convertitur in corpus Christi, ut
supra dictum est. Unde ratio non sequitur.

&

第二異論に対しては以下のように言われるべきである。
出発点である形相が他の形相へ変容されるのではなく、基体において一つの形
 相が他の形相に継起する。それゆえ、第一の形相は、質料の可能態において
 しか残存しない。しかし、前に述べたとおり、ここでパンの実体はキリスト
 の身体へ変容される。それゆえこの異論の推論は間違っている。


\\



{\scshape Ad tertium dicendum} quod, licet post consecrationem haec sit falsa,
{\itshape Substantia panis est aliquid;} id tamen in quod substantia panis
conversa est, est aliquid. Et ideo substantia panis non est
annihilata.

&

第三異論に対しては以下のように言われるべきである。
聖別の後、「パンの実体が何かである」は偽だが、しかしパンの実体がそれへ
 と変容されたものは何かである。ゆえにパンの実体は無化されたのではない。


\end{longtable}
\newpage


\rhead{a.~4}
\begin{center}
{\Large {\bf ARTICULUS QUARTUS}}\\
{\large UTRUM PANIS POSSIT CONVERTI IN CORPUS CHRISTI}\\
{\footnotesize IV {\itshape Sent.}, d.11, q.1, a.3, qu$^a$1; IV
 {\itshape SCG}, cap.63; {\itshape Quodl}.~V, q.6, a.1;\\ {\itshape
 Cont.~Graec., Armen.,} etc., cap.8; I {\itshape ad Cor.}, cap.11,
 lect.4, 5.}\\
{\Large 第四項\\パンがキリストの身体へ変容されうるか}
\end{center}

\begin{longtable}{p{21em}p{21em}}

{\scshape Ad quartum sic proceditur}. Videtur quod panis non possit converti in
corpus Christi. Conversio enim quaedam mutatio est. Sed in omni
mutatione oportet esse aliquod subiectum, quod prius est in potentia
et postea est in actu, ut enim dicitur in III {\itshape Physic}., {\itshape motus est actus
existentis in potentia}. Non est autem dare aliquod subiectum
substantiae panis et corporis Christi, quia de ratione substantiae est
quod {\itshape non sit in subiecto}, ut dicitur in {\itshape Praedicamentis}. Non ergo
potest esse quod tota substantia panis convertatur in corpus Christi.

&

第四項の問題へ議論は以下のように進められる。
パンはキリストの身体へ変容されえないと思われる。理由は以下の通り。
変容は一種の変化である。しかるにどんな変化においても何らかの基体がなけ
 ればならず、それは先には可能態においてあり後に現実態においてある。
なぜなら『自然学』第3巻で「運動は可能態においてあるものの現実態である」
 と言われるようにだから。
しかし、パンの実体とキリストの身体の基体を与えることはできない。なぜな
 ら『カテゴリー論』で言われるように実体の性格には「基体においてない」
 ことが含まれるからである。
ゆえに、パンの全実体がキリストの身体へ変容されることはありえない。

\\

2. {\scshape Praeterea}, forma illius in quod aliquid convertitur, de novo incipit
esse in materia eius quod in ipsum convertitur, sicut, cum aer
convertitur in ignem prius non existentem, forma ignis incipit de novo
esse in materia aeris; et similiter, cum cibus convertitur in hominem
prius non existentem, forma hominis incipit esse de novo in materia
cibi. Si ergo panis convertitur in corpus Christi, necesse est quod
forma corporis Christi de novo incipiat esse in materia panis, quod
est falsum. Non ergo panis convertitur in substantiam corporis
Christi.

&

さらに、あるものがそれへと変容されるものの形相は、それへと変容されるも
 のの質料において新たに存在し始める。たとえば、空気が先に存在していな
 い火へと変容されるとき、火の形相は空気の質料において新たに存在し始め
 る。同様に、食物が以前には存在しなかった人間へ変容されるとき、人間の
 形相が食物の質料において新たに存在し始める。ゆえにもしパンがキリスト
 の身体へ変容されるならば、キリストの身体の形相がパンの質料において新
 たに存在し始めることが必要である。しかしこれは偽である。ゆえにパンが
 キリストの身体の実体へ変容されるのではない。

\\



3. {\scshape Praeterea}, quae sunt secundum se divisa, nunquam unum eorum fit
alterum, sicut albedo nunquam fit nigredo, sed subiectum albedinis fit
subiectum nigredinis, ut dicitur in I {\itshape Physic}. Sed, sicut duae formae
contrariae sunt secundum se divisae, utpote principia formalis
differentiae existentes; ita duae materiae signatae sunt secundum se
divisae, utpote existentes principium materialis divisionis. Ergo non
potest esse quod haec materia panis fiat haec materia qua individuatur
corpus Christi. Et ita non potest esse quod substantia huius panis
convertatur in substantiam corporis Christi.

&

さらに、それ自体において分割されているものは決してその一方が他方になる
 ことはない。たとえば白が黒になることはなく、むしろ『自然学』第1巻で言
 われているように、白の基体が黒の基体となる。しかるに、ちょうど相反す
 る二つの形相が、差異の形相的根源が存在するかぎりにおいてそれ自体にお
 いて分割されているように、二つの指定質料もまた分割の質料的根源が存在
 するものとしてそれ自体において分割されている。ゆえに、パンのこの質料
 が、キリストの身体がそれによって個体化されているこの質料になることは
 ない。したがって、このパンの実体がキリストの身体の実体へ変容されるこ
 とはありえない。



\\



{\scshape Sed contra est} quod Eusebius Emesenus dicit: {\itshape Novum tibi et impossibile
esse non debet quod in Christi substantiam terrena et mortalia
convertuntur}.

&

しかし反対に、エウセビウス・エメセヌスは「地上の死すべきものがキリスト
 の実体へ変容されるということは、あなたにとって新奇なことでも不可能な
 ことでもある必要はない」と述べている。

\\



{\scshape Respondeo dicendum} quod, sicut supra dictum est, cum in hoc sacramento
sit verum corpus Christi, nec incipiat ibi esse de novo per motum
localem; cum etiam nec corpus Christi sit ibi sicut in loco, ut ex
dictis patet, necesse est dicere quod ibi incipiat esse per
conversionem substantiae panis in ipsum. 


&

解答する。以下のように言われるべきである。
上述のごとく、この秘蹟においてキリストの真の身体が、場所的運動によって
 新たにそこにあるのではない。なぜなら既述のことから明らかなとおり、キ
 リストの身体が場所的にそこにあるのではないからである。それゆえ、パン
 の実体のそれへの変容によってそこに存在し始めると言う必要がある。


\\



Haec tamen conversio non est
similis conversionibus naturalibus, sed est omnino supernaturalis,
sola Dei virtute effecta. Unde Ambrosius dicit, in libro {\itshape de
Sacramentis} : {\itshape Liquet quod praeter naturae ordinem virgo generavit. Et
hoc quod conficimus, corpus ex virgine est. Quid igitur quaeris
naturae ordinem in Christi corpore, cum praeter naturam sit ipse
dominus Iesus partus ex virgine?} Et super illud {\itshape Ioan}, {\scshape vi}, {\itshape verba quae
ego locutus sum vobis}, scilicet de hoc sacramento, {\itshape spiritus et vita
sunt}, dicit Chrysostomus : {\itshape Idest, spiritualia sunt, nihil habentia
carnale neque consequentiam naturalem, sed eruta sunt ab omni tali
necessitate quae in terra, et a legibus quae hic positae
sunt}. 

&

しかしこの変容は、いかなる自然的な変容にも似ておらず、まったく超自然的
 な、ただ神の力によってのみ結果されるものである。このことからアンブロ
 シウスは『秘蹟について』という書物の中で次のように述べる。「自然の秩
 序の外でマリアが産んだことは明らかである。そして身体はマリアからある
 ことを私たちは確証する。ゆえにキリストの身体においてあなたはどんな自
 然の秩序を求めるのか、マリアから生まれた主イエス自身が自然の外にある
 のに」。そしてかの『ヨハネによる福音書』第4章「私があなたたちに語る言
 葉」すなわちこの秘蹟について「は霊であり生命である」について、クリュソストムスは次のように言
 う。「すなわち、それらは霊的なものであり、肉的なものや自然的な帰結を
 もたず、地にあるそれらすべての必然性と、ここで指定されている法則から
 引き抜かれている」。


\\



Manifestum est enim quod omne agens agit inquantum est
actu. Quodlibet autem agens creatum est determinatum in suo actu, cum
sit determinati generis et speciei. Et ideo cuiuslibet agentis creati
actio fertur super aliquem determinatum actum. Determinatio autem
cuiuslibet rei in esse actuali est per eius formam. Unde nullum agens
naturale vel creatum potest agere nisi ad immutationem formae. 
Et propter hoc omnis conversio quae fit secundum leges naturae, est
formalis.
&

実際、すべて作用するものは現実態にある限りにおいて作用する。しかるに被
 造のどんな作用者も、限定された類と種に属するので、自らの現実態において
 限定されている。ゆえに被造のどんな作用者の作用も何らかの限定された
 現実態の上へもたらされる。しかるに現実的な存在におけるどんな事物の限
 定も、その事物の形相による。したがってどんな自然的なあるいは被造の作
 用者も形相の変化へでなければ作用できない。このため、自然の法則に従っ
 て生じるどんな変容も形相的である。


\\


Sed Deus est infinitus actus, ut in prima parte habitum
est. Unde eius actio se extendit ad totam naturam entis. Non igitur
solum potest perficere conversionem formalem, ut scilicet diversae
formae sibi in eodem subiecto succedant, sed conversionem totius
entis, ut scilicet tota substantia huius convertatur in totam
substantiam illius. 

&

しかし第一部で論じられたとおり神は無限の現実態である。したがって神の作
 用は存在者の全本性へと広がる。ゆえに、同一の基体においてさまざまな形
 相を自らに継起させるようにして形相的な変容を完成させることができるの
 みならず、それの全実体がかのものの全実体へと変容されるようにして全存
 在者の変容を完成させることができる。


\\



Et hoc agitur divina virtute in hoc
sacramento. Nam tota substantia panis convertitur in totam substantiam
corporis Christi, et tota substantia vini in totam substantiam
sanguinis Christi. Unde haec conversio non est formalis, sed
substantialis. Nec continetur inter species motus naturalis, sed
proprio nomine potest dici {\itshape transubstantiatio}.

&

そしてこれが、神の力によってこの日席において行われている。
すなわち、パンの全実体がキリストの身体の全実体へ、そしてワイ
 ンの全実体がキリストの血の全実体へと変容されている。したがってこの変
 容は形相的でなく実体的である。またこれは自然的な運動の種の中には含ま
 れず「実体間変容」という固有の名によって語られうる。


\\



{\scshape Ad primum ergo dicendum} quod obiectio illa procedit de mutatione
formali, quia formae proprium est in materia vel subiecto esse. Non
autem habet locum in conversione totius substantiae. Unde, cum haec
conversio substantialis importet quendam ordinem substantiarum quarum
una convertitur in alteram, est sicut in subiecto in utraque
substantia, sicut ordo et numerus.

&

第一異論に対しては、それゆえ以下のように言われるべきである。かの反論は
 形相的変化について進められている。なぜなら質料や基体においてあること
 は形相に固有のことだからである。しかるにこのことは全実体の変容におい
 ては当てはまらない。したがって、この実体的変容は、一つの実体型の実体
 へ変容するという諸実体のある種の秩序を意味するので、ちょうど秩序や数
 のように、両方の実体においてあたかも基体においてあるようなかたちであ
 る。


\\



{\scshape Ad secundum dicendum} quod etiam illa obiectio procedit de conversione
formali, seu mutatione, quia oportet, sicut dictum est, formam esse in
materia vel subiecto. Non autem habet locum in conversione totius
substantiae, cuius non est accipere aliquod subiectum.

&

第二異論に対しては以下のように言われるべきである。
この反論もまた形相的な変容ないし変化について進んでいる。なぜなら、すで
 に述べたように、形相は質料や基体において存在しなければならないからで
 ある。しかしこれは何らかの基体を受け取ることがない全実体の変容におい
 ては当てはまらない。

\\



{\scshape Ad tertium dicendum} quod virtute agentis finiti non potest forma in
formam mutari, nec materia in materiam. Sed virtute agentis infiniti,
quod habet actionem in totum ens, potest talis conversio fieri, quia
utrique formae et utrique materiae est communis natura entis; et id
quod entitatis est in una, potest auctor entis convertere ad id quod
est entitatis in altera, sublato eo per quod ab illa distinguebatur.

&

第三異論に対しては以下のように言われるべきである。
有限な作用者の力によっては形相が形相へ、あるいは質料が質料へ変化させら
 れることはない。しかし全存在者への作用を持つ無限の作用者の力によって、
 そのような変容がなされうる。なぜならどちらの形相にもどちらの質料にも、
 存在者という共通の本性が属するからである。そして存在者の制作者は、そ
 れによって一方が他方から区別されていたものを除去しながら、一
 方の存在性に属するものを他方の存在性に属するものへと変容させることが
 できる。


\end{longtable}
\newpage


\rhead{a.~5}
\begin{center}
{\Large {\bf ARTICULUS QUINTUS}}\\
{\large UTRUM IN HOC SACRAMENTO REMANEANT ACCIDENTIA PANIS ET VINI}\\
{\footnotesize IV {\itshape Sent.}, d.11, q.1, a.1, qu$^a$2; d.12,
 q.1, a.1, qu$^a$2; IV {\itshape SCG}, cap.62, 63, 65; \\{\itshape
 Cont.~Graec., Armen.}~etc., cap.8; I {\itshape ad Cor.}, cap11, lect.5.}\\
{\Large 第五項\\この秘蹟においてパンとワインの附帯性は残存するか}
\end{center}

\begin{longtable}{p{21em}p{21em}}
{\scshape Ad quintum sic proceditur}. Videtur quod in hoc sacramento non
remaneant accidentia panis et vini. Remoto enim priori, removetur
posterius. Sed substantia est naturaliter prior accidente, ut probatur
VII {\itshape Metaphys}. Cum ergo, facta consecratione, non remaneat substantia
panis in hoc sacramento, videtur quod non possint remanere accidentia
eius.

&

第四項の問題へ議論は以下のように進められる。
この秘蹟においてパンとワインの附帯性は残存しないと思われる。理由は以下
 の通り。より先のものが取り除かれると、より後のものも取り除かれる。し
 かるに『形而上学』第7巻で証明されるように、実体は自然本性的に附帯性よ
 りも先である。ゆえに、この秘蹟において聖別が行われるとパンの実体は残
 存しないので、それの附帯性が残存することは不可能だと思われる。


\\



2. {\scshape Praeterea}, in sacramento veritatis non debet esse aliqua deceptio. Sed
per accidentia iudicamus de substantia. Videtur ergo quod decipiatur
humanum iudicium, si, remanentibus accidentibus, substantia panis non
remaneat. Non ergo hoc est conveniens huic sacramento.

&

さらに、真理の秘蹟においていかなる欺きもあるべきでない。しかるに私たち
 は附帯性を通して実体について判断する。ゆえにもし附帯性が留まってパン
 の実体が留まらないとすると、人間の判断は欺かれるであろう。ゆえにこの
 ことはこの秘蹟に適合しない。

\\



3. {\scshape Praeterea}, quamvis fides non sit subiecta rationi, non tamen est
contra rationem, sed supra ipsam, ut in principio huius operis dictum
est. Sed ratio nostra habet ortum a sensu. Ergo fides nostra non debet
esse contra sensum, dum sensus noster iudicat esse panem, et fides
nostra credit esse substantiam corporis Christi. Non ergo hoc est
conveniens huic sacramento, quod accidentia panis subiecta sensibus
maneant, et substantia panis non maneat.

&

さらに、この著作の最初に述べられたとおり、信仰は理性に服従しないが理性
 に反せずむしろ優越する。しかし私たちの理性は感覚に端緒をもつ。ゆえに、
 私たちの感覚がパンであると判断し、私たちの信仰がキリストの身体の実体
 だと判断するとき、私たちの信仰は感覚に反しているべきではない。ゆえに、
 感覚に従属するパンの附帯性が留まり、パンの実体が留まらないということ
 は、この秘蹟にに相応しくない。


\\



4. {\scshape Praeterea}, illud quod manet, conversione facta, videtur esse subiectum
mutationis. Si ergo accidentia panis manent conversione facta, videtur
quod ipsa accidentia sint conversionis subiectum. Quod est
impossibile, nam {\itshape accidentis non est accidens}. Non ergo in hoc
sacramento debent remanere accidentia panis et vini.

&

さらに、変容が行われた後に留まっているものは変化の基体だと思われる。ゆ
 えにもしパンの附帯性が、変容が行われた後に留まるならば、その附帯性が
 変容の基体だと思われる。これは不可能である。なぜなら、「附帯性の附帯
 性はない」からである。ゆえにこの秘蹟においてパンとワインの附帯性が留
 まるべきではない。



\\



{\scshape Sed contra est} quod Augustinus dicit, in libro {\itshape
 Sententiarum Prosperi} :
{\itshape Nos in specie panis et vini, quam videmus, res invisibiles, idest
carnem et sanguinem, honoramus}.

&

しかし反対に、アウグスティヌスは『喜ばしい文章』(?)という書物で「私た
 ちは、私たちが見るパンとワインの種において、見えない事物すなわち肉と
 血をたたえる。

\\



{\scshape Respondeo dicendum} quod sensu apparet, facta consecratione, omnia
accidentia panis et vini remanere. Quod quidem rationabiliter per
divinam providentiam fit. Primo quidem, quia non est consuetum
hominibus, sed horribile, carnem hominis comedere et sanguinem bibere,
proponitur nobis caro et sanguis Christi sumenda sub speciebus illorum
quae frequentius in usum hominis veniunt, scilicet panis et
vini. 


&

解答する。以下のように言われるべきである。感覚に明らかであるように、聖
 別が行われた後に、パンとワインのすべての附帯性は残存する。このことは
 実に神の摂理によって理にかなったかたちで生じている。第一に、人間たち
 には人間の肉を食べ血を飲む習慣はなくむしろ身の毛がよだつことなので、
 キリストの肉と血が、より頻繁に人間の使用にやって来るものども、つまり
 パンとワインのかたちのもとに摂取されるように、人間に提示されている。

\\


Secundo, ne hoc sacramentum ab infidelibus irrideretur, si sub
specie propria dominum nostrum manducemus. 


&

第二に、この秘蹟が不信仰の者たちによって笑われないようにである。もし固有の形
 のもとに私たちの主を食べていたとしたら、(笑われたであろう)。

\\


Tertio ut, dum
invisibiliter corpus et sanguinem domini nostri sumimus, hoc proficiat
ad meritum fidei.

&

第三に、見えないしかたで私たちの主の身体と血を摂取することは、信仰の報
 償を増すであろう。

\\



{\scshape Ad primum ergo dicendum} quod, sicut dicitur in libro de causis,
effectus plus dependet a causa prima quam a causa secunda. Et ideo
virtute Dei, qui est causa prima omnium, fieri potest ut remaneant
posteriora, sublatis prioribus.

&

第一異論に対しては、それゆえ、以下のように言われるべきである。
『原因論』という書物で言われるように、結果は第二原因よりも第一原因に依
 存する。ゆえに万物の第一原因である神の力によって、より先のものが取り
 去られても後のものが留まることが可能である。


\\



{\scshape Ad secundum dicendum} quod in hoc sacramento nulla est deceptio, sunt
enim secundum rei veritatem accidentia, quae sensibus
diiudicantur. Intellectus autem, cuius est proprium obiectum
substantia, ut dicitur in III {\itshape de Anima}, per fidem a deceptione
praeservatur.

&

第二異論に対しては以下のように言われるべきである。この秘蹟においてなに
 も欺きではない。すなわち、それらは事物の真理において附帯性であり、そ
 れらは感覚によって判断される。他方、『デ・アニマ』第3巻で言われるよう
 に、その固有の対象が実体である知性は信仰によって欺きからあらかじめ守
 られている。

\\



Et sic patet responsio ad tertium. Nam fides non est contra sensum,
sed est de eo ad quod sensus non attingit.

&

このようにして第三異論への解答は明らかである。すなわち、信仰は感覚に反
 しないが、感覚が到達しないものについてある。

\\



{\scshape Ad quartum dicendum} quod haec conversio non proprie habet subiectum,
ut dictum est. Sed tamen accidentia, quae remanent, habent aliquam
similitudinem subiecti.

&

第四異論に対しては以下のように言われるべきである。
すでに述べられたとおり、この変容は厳密には基体をもたない。しかし残存す
 る附帯性は、基体の何らかの類似をもつ。


\end{longtable}
\newpage






\end{document}


ARTICULUS 6
[50578] IIIª q. 75 a. 6 arg. 1
Ad sextum sic proceditur. Videtur quod, facta consecratione, remaneat in hoc sacramento forma substantialis panis. Dictum est enim quod, facta consecratione, remaneant accidentia. Sed, cum panis sit quiddam artificiale, etiam forma eius est accidens. Ergo remanet, facta consecratione.

[50579] IIIª q. 75 a. 6 arg. 2
Praeterea, forma corporis Christi est anima, dicitur enim in II de anima, quod anima est actus corporis physici potentia vitam habentis. Sed non potest dici quod forma substantialis panis convertatur in animam. Ergo videtur quod remaneat, facta consecratione.

[50580] IIIª q. 75 a. 6 arg. 3
Praeterea, propria operatio rei sequitur formam substantialem eius. Sed illud quod remanet in hoc sacramento, nutrit, et omnem operationem facit quam faceret panis existens. Ergo forma substantialis panis remanet in hoc sacramento, facta consecratione.

[50581] IIIª q. 75 a. 6 s. c.
Sed contra, forma substantialis panis est de substantia panis. Sed substantia panis convertitur in corpus Christi, sicut dictum est. Ergo forma substantialis panis non manet.

[50582] IIIª q. 75 a. 6 co.
Respondeo dicendum quod quidam posuerunt quod, facta consecratione, non solum remanent accidentia panis, sed etiam forma substantialis eius. Sed hoc esse non potest. Primo quidem quia, si forma substantialis remaneret, nihil de pane converteretur in corpus Christi nisi sola materia. Et ita sequeretur quod non converteretur in corpus Christi totum, sed in eius materiam. Quod repugnat formae sacramenti, qua dicitur, hoc est corpus meum. Secundo quia, si forma substantialis panis remaneret, aut remaneret in materia, aut a materia separata. Primum autem esse non potest. Quia, si remaneret in materia panis, tunc tota substantia panis remaneret, quod est contra praedicta. In alia autem materia remanere non posset, quia propria forma non est nisi in propria materia. Si autem remaneret a materia separata, iam esset forma intelligibilis actu, et etiam intellectus, nam omnes formae a materia separatae sunt tales. Tertio, esset inconveniens huic sacramento. Nam accidentia panis in hoc sacramento remanent ut sub eis videatur corpus Christi, non autem sub propria specie, sicut supra dictum est. Et ideo dicendum est quod forma substantialis panis non manet.

[50583] IIIª q. 75 a. 6 ad 1
Ad primum ergo dicendum quod nihil prohibet arte fieri aliquid cuius forma non est accidens, sed forma substantialis, sicut arte possunt produci ranae et serpentes. Talem enim formam non producit ars virtute propria, sed virtute naturalium principiorum. Et hoc modo producit formam substantialem panis, virtute ignis decoquentis materiam ex farina et aqua confectam.

[50584] IIIª q. 75 a. 6 ad 2
Ad secundum dicendum quod anima est forma corporis dans ei totum ordinem esse perfecti, scilicet esse, et esse corporeum, et esse animatum, et sic de aliis. Convertitur igitur forma panis in formam corporis Christi secundum quod dat esse corporeum, non autem secundum quod dat esse animatum tali anima.

[50585] IIIª q. 75 a. 6 ad 3
Ad tertium dicendum quod operationum panis quaedam consequuntur ipsum ratione accidentium, sicut immutare sensum. Et tales operationes inveniuntur in speciebus panis post consecrationem, propter ipsa accidentia, quae remanent. Quaedam autem operationes consequuntur panem vel ratione materiae, sicut quod convertitur in aliquid; vel ratione formae substantialis, sicut est operatio consequens speciem eius, puta quod confirmat cor hominis. Et tales operationes inveniuntur in hoc sacramento, non propter formam vel materiam quae remaneat, sed quia miraculose conferuntur ipsis accidentibus, ut infra dicetur.


ARTICULUS 7
[50586] IIIª q. 75 a. 7 arg. 1
Ad septimum sic proceditur. Videtur quod ista conversio non fiat in instanti, sed fiat successive. In hac enim conversione prius est substantia panis, et postea substantia corporis Christi. Non ergo utrumque est in eodem instanti, sed in duobus instantibus. Sed inter quaelibet duo instantia est tempus medium. Ergo oportet quod haec conversio fiat secundum successionem temporis quod est inter ultimum instans quo est ibi panis, et primum instans quo est ibi corpus Christi.

[50587] IIIª q. 75 a. 7 arg. 2
Praeterea, in omni conversione est fieri et factum esse. Sed haec duo non sunt simul, quia quod fit, non est; quod autem factum est, iam est. Ergo in hac conversione est prius et posterius. Et ita oportet quod non sit instantanea, sed successiva.

[50588] IIIª q. 75 a. 7 arg. 3
Praeterea, Ambrosius dicit, in libro de Sacram., quod istud sacramentum Christi sermone conficitur. Sed sermo Christi successive profertur. Ergo haec conversio fit successive.

[50589] IIIª q. 75 a. 7 s. c.
Sed contra est quod haec conversio perficitur virtute infinita, cuius est subito operari.

[50590] IIIª q. 75 a. 7 co.
Respondeo dicendum quod aliqua mutatio est instantanea triplici ratione. Uno quidem modo, ex parte formae, quae est terminus mutationis. Si enim sit aliqua forma quae recipiat magis aut minus, successive acquiritur subiecto, sicut sanitas. Et ideo, quia forma substantialis non recipit magis et minus, inde est quod subito fit eius introductio in materia. Alio modo, ex parte subiecti, quod quandoque successive praeparatur ad susceptionem formae, et ideo aqua successive calefit. Quando vero ipsum subiectum est in ultima dispositione ad formam, subito recipit ipsam, sicut diaphanum subito illuminatur. Tertio, ex parte agentis, quod est infinitae virtutis, unde statim potest materiam ad formam disponere. Sicut dicitur Marc. VII, quod, cum Christus dixisset, ephphetha, quod est adaperire, statim apertae sunt aures hominis, et solutum est vinculum linguae eius. Et his tribus rationibus haec conversio est instantanea. Primo quidem, quia substantia corporis Christi, ad quam terminatur ista conversio, non suscipit magis neque minus. Secundo, quia in hac conversione non est aliquod subiectum, quod successive praeparetur. Tertio, quia agitur Dei virtute infinita.

[50591] IIIª q. 75 a. 7 ad 1
Ad primum ergo dicendum quod quidam non simpliciter concedunt quod inter quaelibet duo instantia sit tempus medium. Dicunt enim quod hoc habet locum in duobus instantibus quae referuntur ad eundem motum, non autem in duobus instantibus quae referuntur ad diversa. Unde inter instans quod mensurat finem quietis, et aliud instans quod mensurat principium motus, non est tempus medium. Sed in hoc decipiuntur. Quia unitas temporis et instantis, vel etiam pluralitas eorum, non accipitur secundum quoscumque motus, sed secundum primum motum caeli, qui est mensura omnis motus et quietis. Et ideo alii hoc concedunt in tempore quod mensurat motum dependentem ex motu caeli. Sunt autem quidam motus ex motu caeli non dependentes, nec ab eo mensurati, sicut in prima parte dictum est de motibus Angelorum. Unde inter duo instantia illis motibus respondentia, non est tempus medium. Sed hoc non habet locum in proposito. Quia, quamvis ista conversio secundum se non habeat ordinem ad motum caeli, consequitur tamen prolationem verborum, quam necesse est motu caeli mensurari. Et ideo necesse est inter quaelibet duo instantia circa istam conversionem signata esse tempus medium. Quidam ergo dicunt quod instans in quo ultimo est panis, et instans in quo primo est corpus Christi, sunt quidem duo per comparationem ad mensurata, sed sunt unum per comparationem ad tempus mensurans, sicut, cum duae lineae se contingunt, sunt duo puncta ex parte duarum linearum, unum autem punctum ex parte loci continentis. Sed hoc non est simile. Quia instans et tempus particularibus motibus non est mensura intrinseca, sicut linea et punctus corporibus, sed solum extrinseca, sicut corporibus locus. Unde alii dicunt quod est idem instans re, sed aliud ratione. Sed secundum hoc sequeretur quod realiter opposita essent simul. Nam diversitas rationis non variat aliquid ex parte rei. Et ideo dicendum est quod haec conversio, sicut dictum est, perficitur per verba Christi, quae a sacerdote proferuntur, ita quod ultimum instans prolationis verborum est primum instans in quo est in sacramento corpus Christi, in toto autem tempore praecedente est ibi substantia panis. Cuius temporis non est accipere aliquod instans proximo praecedens ultimum, quia tempus non componitur ex instantibus consequenter se habentibus, ut probatur in VI Physic. Et ideo est quidem dare instans in quo est corpus Christi, non est autem dare ultimum instans in quo sit substantia panis, sed est dare ultimum tempus. Et idem est in mutationibus naturalibus, ut patet per philosophum, in VIII physicorum.

[50592] IIIª q. 75 a. 7 ad 2
Ad secundum dicendum quod in mutationibus instantaneis simul est fieri et factum esse, sicut simul est illuminari et illuminatum esse. Dicitur enim in talibus factum esse secundum quod iam est, fieri autem, secundum quod ante non fuit.

[50593] IIIª q. 75 a. 7 ad 3
Ad tertium dicendum quod ista conversio, sicut dictum est, fit in ultimo instanti prolationis verborum, tunc enim completur verborum significatio, quae est efficax in sacramentorum formis. Et ideo non sequitur quod ista conversio sit successiva.


ARTICULUS 8
[50594] IIIª q. 75 a. 8 arg. 1
Ad octavum sic proceditur. Videtur quod haec sit falsa, ex pane fit corpus Christi. Omne enim id ex quo fit aliquid, est id quod fit illud, sed non convertitur, dicimus enim quod ex albo fit nigrum, et quod album fit nigrum; et licet dicamus quod homo fiat niger, non tamen dicimus quod ex homine fiat nigrum ut patet in I Physic. Si ergo verum est quod ex pane fiat corpus Christi, verum erit dicere quod panis fiat corpus Christi. Quod videtur esse falsum, quia panis non est subiectum factionis, sed magis est terminus. Ergo non vere dicitur quod ex pane fiat corpus Christi.

[50595] IIIª q. 75 a. 8 arg. 2
Praeterea, fieri terminatur ad esse, vel ad factum esse. Sed haec nunquam est vera, panis est corpus Christi, vel, panis est factus corpus Christi, vel etiam, panis erit corpus Christi. Ergo videtur quod nec haec sit vera, ex pane fit corpus Christi.

[50596] IIIª q. 75 a. 8 arg. 3
Praeterea, omne id ex quo fit aliquid, convertitur in id quod fit ex eo. Sed haec videtur esse falsa, panis convertitur in corpus Christi, quia haec conversio videtur esse miraculosior quam creatio; in qua tamen non dicitur quod non ens convertatur in ens. Ergo videtur quod etiam haec sit falsa, ex pane fit corpus Christi.

[50597] IIIª q. 75 a. 8 arg. 4
Praeterea, illud ex quo fit aliquid, potest esse illud. Sed haec est falsa, panis potest esse corpus Christi. Ergo etiam haec est falsa, ex pane fit corpus Christi.

[50598] IIIª q. 75 a. 8 s. c.
Sed contra est quod Ambrosius dicit, in libro de sacramentis, ubi accedit consecratio, de pane fit corpus Christi.

[50599] IIIª q. 75 a. 8 co.
Respondeo dicendum quod haec conversio panis in corpus Christi, quantum ad aliquid convenit cum creatione et cum transmutatione naturali, et quantum ad aliquid differt ab utroque. Est enim commune his tribus ordo terminorum, scilicet ut post hoc sit hoc, in creatione enim est esse post non esse, in hoc sacramento corpus Christi post substantiam panis, in transmutatione naturali album post nigrum vel ignis post aerem; et quod praedicti termini non sint simul. Convenit autem conversio de qua nunc loquimur cum creatione, quia in neutra earum est aliquod commune subiectum utrique extremorum. Cuius contrarium apparet in omni transmutatione naturali. Convenit vero haec conversio cum transmutatione naturali in duobus, licet non similiter. Primo quidem, quia in utraque unum extremorum transit in aliud, sicut panis in corpus Christi, et aer in ignem, non autem non ens convertitur in ens. Aliter tamen hoc accidit utrobique. Nam in hoc sacramento tota substantia panis transit in totum corpus Christi, sed in transmutatione naturali materia unius suscipit formam alterius, priori forma deposita. Secundo conveniunt in hoc, quod utrobique remanet aliquid idem, quod non accidit in creatione. Differenter tamen, nam in transmutatione naturali remanet eadem materia vel subiectum; in hoc autem sacramento remanent eadem accidentia. Et ex his accipi potest qualiter differenter in talibus loqui debeamus. Quia enim in nullo praedictorum trium extrema sunt simul ideo in nullo eorum potest unum extremum de alio praedicari per verbum substantivum praesentis temporis, non enim dicimus, non ens est ens, vel, panis est corpus Christi, vel, aer est ignis aut album nigrum. Propter ordinem vero extremorum, possumus uti in omnibus hac praepositione ex, quae ordinem designat. Possumus enim vere et proprie dicere quod ex non ente fit ens, et ex pane corpus Christi, et ex aere ignis vel ex albo nigrum. Quia vero in creatione unum extremorum non transit in alterum, non possumus in creatione uti verbo conversionis, ut dicamus quod non ens convertitur in ens. Quo tamen verbo uti possumus in hoc sacramento, sicut et in transmutatione naturali. Sed quia in hoc sacramento tota substantia in totam mutatur, propter hoc haec conversio proprie transubstantiatio vocatur. Rursus, quia huius conversionis non est accipere aliquod subiectum, ea quae verificantur in conversione naturali ratione subiecti, non sunt concedenda in hac conversione. Et primo quidem, manifestum est quod potentia ad oppositum consequitur subiectum, ratione cuius dicimus quod album potest esse nigrum, vel aer potest esse ignis. Licet haec non sit ita propria sicut prima, nam subiectum albi, in quo est potentia ad nigredinem, est tota substantia albi, non enim albedo est pars eius; subiectum autem formae aeris est pars eius; unde, cum dicitur, aer potest esse ignis, verificatur ratione partis per synecdochen. Sed in hac conversione et similiter in creatione, quia nullum est subiectum, non dicitur quod unum extremum possit esse aliud, sicut quod non ens possit esse ens, vel quod panis possit esse corpus Christi. Et eadem ratione non potest proprie dici quod de non ente fiat ens, vel quod de pane fiat corpus Christi, quia haec praepositio de designat causam consubstantialem; quae quidem consubstantialitas extremorum in transmutationibus naturalibus attenditur penes convenientiam in subiecto. Et simili ratione non conceditur quod panis erit corpus Christi, vel quod fiat corpus Christi, sicut neque conceditur in creatione quod non ens erit ens, vel quod non ens fiat ens, quia hic modus loquendi verificatur in transmutationibus naturalibus ratione subiecti, puta cum dicimus quod album fit nigrum, vel album erit nigrum. Quia tamen in hoc sacramento, facta conversione, aliquid idem manet, scilicet accidentia panis, ut supra dictum est, secundum quandam similitudinem aliquae harum locutionum possunt concedi, scilicet quod panis sit corpus Christi, vel, panis erit corpus Christi, vel, de pane fit corpus Christi; ut nomine panis non intelligatur substantia panis, sed in universali hoc quod sub speciebus panis continetur, sub quibus prius continetur substantia panis, et postea corpus Christi.

[50600] IIIª q. 75 a. 8 ad 1
Ad primum ergo dicendum quod illud ex quo aliquid fit, quandoque quidem importat simul subiectum cum uno extremorum transmutationis, sicut cum dicitur, ex albo fit nigrum. Quandoque vero importat solum oppositum, vel extremum, sicut cum dicitur, ex mane fit dies. Et sic non conceditur quod hoc fiat illud, idest quod mane fiat dies. Et ita etiam in proposito, licet proprie dicatur quod ex pane fiat corpus Christi, non tamen proprie dicitur quod panis fiat corpus Christi, nisi secundum quandam similitudinem, ut dictum est.

[50601] IIIª q. 75 a. 8 ad 2
Ad secundum dicendum quod illud ex quo fit aliquid, quandoque erit illud, propter subiectum quod importatur. Et ideo, cum huius conversionis non sit aliquod subiectum, non est similis ratio.

[50602] IIIª q. 75 a. 8 ad 3
Ad tertium dicendum quod in hac conversione sunt plura difficilia quam in creatione, in qua hoc solum difficile est, quod aliquid fit ex nihilo, quod tamen pertinet ad proprium modum productionis primae causae, quae nihil aliud praesupponit. Sed in hac conversione non solum est difficile quod hoc totum convertitur in illud totum, ita quod nihil prioris remaneat, quod non pertinet ad communem modum productionis alicuius causae, sed etiam habet hoc difficile, quod accidentia remanent corrupta substantia, et multa alia, de quibus in sequentibus agetur. Tamen verbum conversionis recipitur in hoc sacramento, non autem in creatione, sicut dictum est.

[50603] IIIª q. 75 a. 8 ad 4
Ad quartum dicendum quod, sicut dictum est, potentia pertinet ad subiectum, quod non est accipere in hac conversione. Et ideo non conceditur quod panis possit esse corpus Christi, non enim haec conversio fit per potentiam passivam creaturae, sed per solam potentiam activam creatoris.

