\documentclass[10pt]{jsarticle} % use larger type; default would be 10pt
%\usepackage[utf8]{inputenc} % set input encoding (not needed with XeLaTeX)
%\usepackage[round,comma,authoryear]{natbib}
%\usepackage{nruby}
\usepackage{okumacro}
\usepackage{longtable}
%\usepqckage{tablefootnote}
\usepackage[polutonikogreek,english,japanese]{babel}
%\usepackage{amsmath}
\usepackage{latexsym}
\usepackage{color}
%\usepackage{tikz}

%----- header -------
\usepackage{fancyhdr}
\pagestyle{fancy}
\lhead{{\it Summa Theologiae} I, q.~34}
%--------------------


\title{{\bf PRIMA PARS}\\{\HUGE Summae Theologiae}\\Sancti Thomae
Aquinatis\\{\sffamily QUEAESTIO TRIGESTIMAQUARTA}\\DE PERSONA FILII}
\author{Japanese translation\\by Yoshinori {\sc Ueeda}}
\date{Last modified \today}

%%%% コピペ用
%\rhead{a.~}
%\begin{center}
% {\Large {\bf }}\\
% {\large }\\
% {\footnotesize }\\
% {\Large \\}
%\end{center}
%
%\begin{longtable}{p{21em}p{21em}}
%
%&
%
%\\
%\end{longtable}
%\newpage



\begin{document}

\maketitle

\thispagestyle{empty}
\begin{center}
{\Large 聖トマス・アクィナスの神学大全の第一部\\第三十四問\\息子のペル
 ソナについて}
\end{center}



\begin{longtable}{p{21em}p{21em}}
{\Huge D}einde considerandum est de persona filii. Attribuuntur autem tria
 nomina filio, scilicet filius, verbum et imago. Sed ratio filii ex
 ratione patris consideratur. Unde restat considerandum de verbo et
 imagine. Circa verbum quaeruntur tria. 


\begin{enumerate}
 \item utrum verbum dicatur essentialiter in divinis, vel
       personaliter.
 \item utrum sit proprium nomen filii. 
 \item utrum in nomine verbi importetur respectus ad creaturas.
\end{enumerate}

&

次に息子のペルソナについて考察されるべきである。ところで息子には息子、
 言葉、似像という三つの名称が帰せられる。しかし息子という性格は父とい
 う性格から考察される。したがって、言葉と似像について考察されるべきこ
 とが残されている。言葉をめぐっては三つのことが問われる。

\begin{enumerate}
 \item 言葉は神において本質的に語られるか、それともペルソナ的にか。
 \item それは息子の固有の名称か。
 \item 言葉という名称において被造物への関係が意味されているか。
\end{enumerate}

\end{longtable}

\newpage

\rhead{a.~1}
\begin{center}
{\Large {\bf ARTICULUS PRIMUS}}\\
{\large UTRUM VERBUM IN DIVINIS SIT NOMEN PERSONALE}\\
{\footnotesize I$^{a}$II$^{ae}$, q.93, a.1, ad 2; I {\itshape Sent.},
 d.27, q.2, a.2, qu$^{a}$1; {\itshape De Pot.}, q.2, a.9, ad 7;
 {\itshape De Verit.}, q.4, a.2; a.4, ad 4.}\\
{\Large 第一項\\言葉は神においてペルソナ的名称か}
\end{center}

\begin{longtable}{p{21em}p{21em}}

{\scshape Ad primum sic proceditur}. Videtur quod verbum in divinis non sit nomen
 personale. Nomina enim personalia proprie dicuntur in divinis, ut
 pater et filius. Sed verbum metaphorice dicitur in divinis, ut
 Origenes dicit, {\itshape super Ioannem}. Ergo verbum non est personale in
 divinis.

&

第一項の問題へ議論は以下のように進められる。
言葉はかみにおいてペルソナ的名称でないと思われる。理由は以下の通り。
ペルソナ的名称は、父や息子のように、神において固有に語られる。
しかるにオリゲネスが『ヨハネ伝注解』で述べているように、「言葉」は神におい
 て比喩的に語られる。ゆえに「言葉」は神においてペルソナ的ではない。



\\




2. {\scshape Praeterea}, secundum Augustinum, in libro {\itshape de Trin}., {\itshape verbum est notitia
 cum amore}. Et secundum Anselmum, in {\itshape Monol}., {\itshape dicere summo spiritui
 nihil aliud est quam cogitando intueri}. Sed notitia et cogitatio et
 intuitus in divinis essentialiter dicuntur. Ergo verbum non dicitur
 personaliter in divinis.

&

さらに、『三位一体論』のアウグスティヌスによれば「言葉は愛と共にある知であ
 る」。また『モノロギオン』のアンセルムスによれば「最高の霊にとって語ること
 は思惟することによって直視することである」。しかるに知と思惟と直視は
 神において本質的に語られる。ゆえに言葉は神においてペルソナ的に語られ
 るのではない。

\\




3. {\scshape Praeterea}, de ratione verbi est quod dicatur. Sed, secundum Anselmum,
 sicut pater est intelligens, et filius est intelligens, et spiritus
 sanctus est intelligens; ita pater est dicens, filius est dicens, et
 spiritus sanctus est dicens. Et similiter quilibet eorum
 dicitur. Ergo nomen verbi essentialiter dicitur in divinis, et non
 personaliter.

&

さらに言葉の性格には語られることが属する。しかしアンセルムスによればちょ
 うど父が知性認識する者であり息子もまた知性認識する者であり、聖
 霊もまたそうであるように、父は語る者であり息子は語る者であり聖霊は語
 る者である。同様に彼らのだれもがそう言われる。ゆえに言葉という名称は
 神において本質的に語られるのでありペルソナ的にではない。

\\




4. {\scshape Praeterea}, nulla persona divina est facta. Sed verbum Dei est aliquid
 factum, dicitur enim in Psalmo {\scshape cxlviii}, {\itshape ignis, grando, nix, glacies,
 spiritus procellarum, quae faciunt verbum eius}. Ergo verbum non est
 nomen personale in divinis.

&

さらに、神のどのペルソナも作られたものではない。しかるに神の言葉は作ら
 れた何かである。というのも『詩編』148で「火と雹と雪と霧と嵐の霊は、彼
 の言葉を作る」\footnote{「火よ、雹よ、雪よ、霧よ/御言葉を成し遂げる
 激しい風よ」(148:8)}と述べられているからである。ゆえに言葉は神におい
 てペルソナ的名称ではない。


\\




{\scshape Sed contra est} quod dicit Augustinus, in VII {\itshape de Trin}., {\itshape sicut filius
 refertur ad patrem, ita et verbum ad id cuius est verbum}. Sed filius
 est nomen personale, quia relative dicitur. Ergo et verbum.

&

しかし反対に、アウグスティヌスは『三位一体論』第7巻で「ちょうど息子が
 父に関係づけられるように、言葉は、それの言葉であるところのそれへ関係
 づけられる」と述べている。しかるに息子は関係的に語られるがゆえにペル
 ソナ的名称である。ゆえに言葉もまたそうである。

\\




{\scshape Respondeo dicendum} quod nomen verbi in divinis, si proprie sumatur,
 est nomen personale, et nullo modo essentiale. Ad cuius evidentiam,
 sciendum est quod {\itshape verbum} tripliciter quidem in nobis proprie dicitur,
 quarto autem modo, dicitur improprie sive figurative. 

&

解答する。以下のように言われるべきである。
言葉という名称は、厳密に解釈される場合、神においてペルソナ的名称であり
 いかなる意味においても本質的でない。これを明らかにするために以下のこ
 とが知られるべきである。「言葉」は三つの意味で私たちにおいて固有に
 語られるが、第四の意味では固有でないしかたあるいは比喩的に語られる。


\\


Manifestius
 autem et communius in nobis dicitur verbum quod voce profertur. 
Quod
 quidem ab interiori procedit quantum ad duo quae in verbo exteriori
 inveniuntur, scilicet vox ipsa, et significatio vocis. Vox enim
 significat intellectus conceptum, secundum philosophum, in libro I
 {\itshape Periherm}., et iterum vox ex imaginatione procedit, ut in libro {\itshape de
 Anima} dicitur. 

&

さて私たちにおいて、音声によって発せられるものが、より明らかにそして共通に
 言葉と言われる。それは外的な言葉において見出される二つのもの、すなわ
 ち音声それ自体と音声の表示に関して内から出てくる。なぜなら『命題論』
 第1巻の哲学者によれば音声は知性の懐念を表示するからであり、さらに『デ
 アニマ』で言われるように、音声はイメージ\footnote{このimaginatioは日
 本語に訳しにくい。「想像力」とすると一種の能力のように受け取られるが、
 能力ではなくむしろ想像力の中で作られたり蓄えられりしている像のことで
 あると考えられるので「イメージ」と訳してみる。}から出てくるからである。


\\



Vox autem quae non est significativa, verbum dici non
 potest. Ex hoc ergo dicitur verbum vox exterior, quia significat
 interiorem mentis conceptum. Sic igitur primo et principaliter
 interior mentis conceptus verbum dicitur, secundario vero, ipsa vox
 interioris conceptus significativa, tertio vero, ipsa imaginatio
 vocis verbum dicitur. 

&

これに対して、表示しない音声は言葉と言われえない。ゆえにこのことから、外的な音声
が言葉と言われるのは内的な精神の懐念を表示するからである。ゆえに、第一
 にそして主要に、精神の内的な懐念が言葉と言われ、第二に内的な懐念を表
 示しうる音声それ自体が、そして第三に音声がもつイメージ自体が言葉と言われる。


\\


Et hos tres modos verbi ponit Damascenus, in I
 libro, cap.~{\scshape xiii}, dicens quod {\itshape verbum} dicitur
 {\itshape naturalis intellectus motus, secundum quem movetur et intelligit et cogitat, velut lux et
 splendor}, quantum ad primum, {\itshape rursus verbum est quod} non verbo
 profertur, sed {\itshape in corde pronuntiatur}, quantum ad tertium, {\itshape rursus
 etiam verbum est Angelus}, idest nuntius, {\itshape intelligentiae}, quantum ad
 secundum. 

&

そして言葉のこれら三つの意味をダマスケヌスは第1巻第13章\footnote{『正
 統信仰論』。}で次のように述べて確認している。「言葉は知性の本性的な運
 動であり、それにしたがって光や輝きのように動かされ、知性認識し、思惟
 する」と言われる。これは第一の意味にかんしてである。「さらに言葉は」
 言葉によってもたらされるのではなく「心の中で発せられる」。これは第三の意味
 である。「さらにまた言葉は知性の天使」つまり使者である。これは第二の
 意味にかんしてである。


\\


Dicitur autem figurative quarto modo verbum, id quod verbo
 significatur vel efficitur, sicut consuevimus dicere, {\itshape hoc est verbum
 quod dixi tibi, vel quod mandavit rex}, demonstrato aliquo facto quod
 verbo significatum est vel simpliciter enuntiantis, vel etiam
 imperantis. 
&

他方で第四の意味で比喩的に、言葉によって表示されるものや作り出されるも
 のが言葉と言われる。たとえば私たちが「これは私があなたに言った言葉だ」
 とか「これが王が命じた言葉だ」と言う慣わしであるように、たんに表明す
 る人の、あるいはさらに命じる人の言葉で表示された何らかの事実が示され
 たときにそう言われる。


\\

Dicitur autem proprie verbum in Deo, secundum quod verbum
 significat conceptum intellectus. 
Unde Augustinus dicit, in XV {\itshape de
 Trin}., {\itshape quisquis potest intelligere verbum, non solum antequam sonet,
 verum etiam antequam sonorum eius imagines cogitatione involvantur,
 iam potest videre aliquam verbi illius similitudinem, de quo dictum
 est: In principio erat verbum}. 

&

さらには、言葉が知性の概
 念を表示する限りにおいて、固有の意味で神において言葉が言われる。この
 ことからアウグスティヌスは『三位
 一体論』第15巻で「言葉を理解しうる者は誰でも、それが響く前だけでなく、
 彼の音のイメージが思惟によって展開される前に、すでにその言葉の何らか
 の類似を見ることができるが、「はじめに言葉があった」というのは、このこと
 について語られている」。

\\

Ipse autem conceptus cordis de ratione
 sua habet quod ab alio procedat, scilicet a notitia
 concipientis.
 Unde verbum, secundum quod proprie dicitur in divinis,
 significat aliquid ab alio procedens, quod pertinet ad rationem
 nominum personalium in divinis, eo quod personae divinae
 distinguuntur secundum originem, ut dictum est. Unde oportet quod
 nomen verbi, secundum quod proprie in divinis accipitur, non sumatur
 essentialiter, sed personaliter tantum.

&

ところで心の懐念は、それ自身の性格から、他からすなわち懐念する人の知か
 ら発出することが属している。したがって、言葉は、固有に神において語られる限
 りにおいて、他の何かから発出するものを表示し、それは神においてペルソ
 ナ的名称の性格に属する。なぜならすでに述べられたとおり、神のペルソナ
 は起源において区別されるからである。したがって、言葉という名称は
 神において固有に理解されるかぎりにおいて、本質的にではなくたんにペル
 ソナ的に理解される。

\\




{\scshape Ad primum ergo dicendum} quod Ariani, quorum fons Origenes invenitur,
 posuerunt filium alium a patre esse in diversitate substantiae. Unde
 conati sunt, cum filius Dei verbum dicitur, astruere non esse proprie
 dictum; ne, sub ratione verbi procedentis, cogerentur fateri filium
 Dei non esse extra substantiam patris; nam verbum interius sic a
 dicente procedit, quod in ipso manet. 

&

第一異論に対してはそれゆえ以下のように言われるべきである。
アリウス派の人々、彼らの源泉はオリゲネスであることが見出されるが、は
 息子が実体の差異において父と異なると主張した。したがって、神の息子は
 言葉と言われているので、それが固有の意味で言われているのではないと付
 け加えることを強いられた。それは、発出する言葉の性格のもと、神の息子
 が父の実体の外にないと表明することを強いられないようにである。なぜな
 ら内なる言葉は語る者の中に留まる形で語る者から発出するからである。

\\



Sed necesse est, si ponitur
 verbum Dei metaphorice dictum, quod ponatur verbum Dei proprie
 dictum. Non enim potest aliquid metaphorice verbum dici, nisi ratione
 manifestationis, quia vel manifestat sicut verbum, vel est verbo
 manifestatum. Si autem est manifestatum verbo, oportet ponere verbum
 quo manifestetur. 

&

しかし、神の言葉が比喩的に言われると考えるとしても、神の言葉は固有に語られる
 と考えることが必然である。理由は以下の通り。あるものが比喩的に言葉と
 言われうるのは、言葉のように明示する、あるいは言葉によって明示さ
 れるという理由で、明示という性格によって以外にはない。しかるにもし言
 葉によって明示されるのであれば、それによって明示されるところの言葉を
 考えるべきである。


\\


Si autem dicitur verbum quia exterius manifestat,
 ea quae exterius manifestant, non dicuntur verba nisi inquantum
 significant interiorem mentis conceptum, quem aliquis etiam per
 exteriora signa manifestat. Etsi ergo verbum aliquando dicatur
 metaphorice in divinis, tamen oportet ponere verbum proprie dictum,
 quod personaliter dicatur.

&

さらに、もし外的に明示するという理由で言葉と言われるならば、外的に明示
 する事柄は内的な精神の懐念を表示する限りにおいてでなければ言葉と言わ
 れない。ゆえに、言葉は神において比喩的に言われることもあるが、しかし
 固有に語られた言葉という者を考える必要があり、これがペルソナ的に語ら
 れている。


\\




{\scshape Ad secundum dicendum} quod nihil eorum quae ad intellectum pertinent,
 personaliter dicitur in divinis, nisi solum verbum, solum enim verbum
 significat aliquid ab alio emanans. Id enim quod intellectus in
 concipiendo format, est verbum. 

&

第二異論に対しては以下のように言われるべきである。
知性に属するもので神においてペルソナ的に語られるのはただ言葉だけである。
 なぜなら言葉だけが他のものから流出する何かを表示するからである。
じっさい、知性が懐念することにおいて形成するものが言葉である。

\\


Intellectus autem ipse, secundum quod
 est per speciem intelligibilem in actu, consideratur absolute. Et
 similiter intelligere, quod ita se habet ad intellectum in actu,
 sicut esse ad ens in actu, non enim intelligere significat actionem
 ab intelligente exeuntem, sed in intelligente manentem. Cum ergo
 dicitur quod verbum est notitia, non accipitur notitia pro actu
 intellectus cognoscentis, vel pro aliquo eius habitu, sed pro eo quod
 intellectus concipit cognoscendo. 

&

他方、知性自身は可知的形象によって現実にある限りにおいて非関係的に考察
 される。また同様に知性認識の働きも、存在(esse)が現実に存在するもの
 (ens in actu)に関係するようにして現実態における知性に関係する(が、そ
 れも非関係的に考察される)。なぜなら、知性認識の働きは知性認識するも
 のから外に出てくる作用ではなく、知性認識するものの中に留まる作用を表
 示するからである。ゆえに言葉が知であると言われるとき、この知は認識す
 る知性の作用としてでもその何らかの習態としてでもなく、知性が認識する
 ことによって捉えるものとして理解される。

\\



Unde et Augustinus dicit quod
 verbum est {\itshape sapientia genita}, quod nihil aliud est quam ipsa conceptio
 sapientis, quae etiam pari modo notitia genita dici potest. Et per
 eundem modum potest intelligi quod {\itshape dicere} Deo sit {\itshape cogitando intueri},
 inquantum scilicet intuitu cogitationis divinae concipitur verbum
 Dei. 

&

このことからアウグスティヌスも、言葉は「生み出された知恵」であると述べ
 ているが、これは知恵の懐念自体に他ならない。これはまた同じしかたで
 「生み出された知」とも言われうる。そして同じしかたによって、神にとっ
 て「語ることは思惟することによって直視することである」ということも理
 解され、それは、神の認識の直観によって神の言葉が懐念されるかぎりにお
 いてである。

\\


{\itshape Cogitationis} tamen nomen Dei verbo proprie non convenit, dicit
 enim Augustinus, XV {\itshape de Trin}. : {\itshape Ita dicitur illud verbum Dei, ut
 cogitatio non dicatur; ne aliquid esse quasi volubile credatur in
 Deo, quod nunc accipiat formam ut verbum sit, eamque dimittere
 possit, atque informiter quodammodo volutari}. 

&

しかし、「思惟」という名称は神の言葉に固有には適合しない。なぜならア
 ウグスティヌスが『三位一体論』第15巻で次のように述べているからである。
「かの神の言葉と言われているの
 は、思惟と言われないようにであり、それは、今、言葉として形相を受け取っ
 ているが、それを捨てることができ、そしてなんらか形相をもたないかたち
 で変化するというように、あるものが神の中でいわば変化しうると信じられな
 いようにである」。



\\



Cogitatio enim proprie
 in inquisitione veritatis consistit, quae in Deo locum non habet. Cum
 vero intellectus iam ad formam veritatis pertingit, non cogitat, sed
 perfecte veritatem contemplatur. Unde Anselmus improprie accipit
 cogitationem pro contemplatione.

&

つまり思惟は固有には真理の探究において成立するが、このことは神において
 占める場所がない。むしろ知性がすでに真理の形相に到達しているとき、知
 性は思惟せず完全に真理を観照する。したがってアンセルムスは思惟を固有でない
 意味で「観照」の意味で理解している。

\\




{\scshape Ad tertium dicendum} quod, sicut, proprie loquendo, verbum dicitur
 personaliter in divinis et non essentialiter, ita et {\itshape dicere}. Unde,
 sicut verbum non est commune patri et filio et spiritui sancto, ita
 non est verum quod pater et filius et spiritus sanctus sint unus
 dicens. 

&

第三異論に対しては以下のように言われるべきである。
ちょうど固有の意味で言うならば言葉は神において本質的にでなくペルソナ的に語られ
るように、「語る」もまた同様である。したがって、言葉が父と息
 子と聖霊に共通でないように、父と息子と聖霊が一人の語る者であるという
 のも真ではない。

\\



Unde Augustinus dicit, VII {\itshape de Trin}. : {\itshape Dicens illo coaeterno
 verbo non singulus intelligitur in divinis}. Sed dici convenit
 cuilibet personae, dicitur enim non solum verbum sed res quae verbo
 intelligitur vel significatur. Sic ergo uni soli personae in divinis
 convenit dici eo modo quo dicitur verbum, eo vero modo quo dicitur
 res in verbo intellecta, cuilibet personae convenit dici. 

&

このことからアウグスティヌスは『三位一体論』第7巻で「かの共に永遠であ
 る言葉によって語る者は神において各々に一人ずつだとは理解されない」と
 述べている。しかし、「語られること」はどのペルソナにも適合する。言葉
 だけでなく言葉によって知性認識されたり表示されたりする事物もまた語ら
 れるからである。このようにして、それゆえ、神において、言葉が語られる
 しかたで「語られること」はただ一つのペルソナにだけ適合するが、
 事物が言葉において知性認識されるというしかたであれば、「語られること」
 はどのペルソナにも適合する。

\\



Pater enim,
 intelligendo se et filium et spiritum sanctum, et omnia alia quae
 eius scientia continentur, concipit verbum, ut sic tota Trinitas
 verbo dicatur, et etiam omnis creatura; sicut intellectus hominis
 verbo quod concipit intelligendo lapidem, lapidem dicit. 
&

じっさい父は自分自身と息子と聖霊を、そしてその
 知に含まれる他のすべての事柄を知性認識することによって言葉を懐念する。
 その結果そのようにして三性全体が言葉によって語られ、さらにすべての被
 造物も言葉によって語られる。ちょうど人間の知性が石を知性認識すること
 によって懐念する言葉によって石を語るように。

\\

Anselmus
 vero improprie accepit {\itshape dicere} pro {\itshape intelligere}. 
Quae tamen
 differunt. Nam intelligere importat solam habitudinem intelligentis
 ad rem intellectam; in qua nulla ratio originis importatur, sed solum
 informatio quaedam in intellectu nostro, prout intellectus noster fit
 in actu per formam rei intellectae. In Deo autem importat omnimodam
 identitatem, quia in Deo est omnino idem intellectus et intellectum,
 ut supra ostensum est.

&

他方でアンセルムスは「語ること」を非固有的に「知性認識すること」の意味
 で理解している。しかしこれらには違いがある。すなわち、「知性認識する
 こと」は知性認識する者の知性認識された事物への関係だけを意味し、その
 関係においてどんな起源の性格も意味されず、私たちの知性が知性認識され
 た事物の形相によって現実態になるという、私たちの知性におけるある種の
 形成だけが意味される。しかし神においてはあらゆる点における同一性がある。なぜなら前に示された
 とおり\footnote{Q.14, a.2.}神において知
 性と知性認識されたものはあらゆる点で同一だからである。


\\



 Sed dicere importat principaliter habitudinem
 ad verbum conceptum nihil enim est aliud dicere quam proferre
 verbum. 
Sed mediante verbo importat habitudinem ad rem intellectam,
 quae in verbo prolato manifestatur intelligenti. Et sic sola persona
 quae profert verbum, est dicens in divinis, cum tamen singula
 personarum sit intelligens et intellecta, et per consequens verbo
 dicta.

&

しかし「語ること」は主に懐念された言葉に対する関係を意味する。語ること
 は言葉を発することに他ならないからである。しかし、言葉を媒介にして知
 性認識された事物への関係を意味する。この関係は発せられた言葉において
 知性認識するものに明示される。そしてこのようにして、言葉を発するペル
 ソナだけが神において「語る者」である。ペルソナの各々は知性認識する者
 であり知性認識されたものであり、結果的に言葉によって語られたものであ
 るとしても。


\\




{\scshape Ad quartum dicendum} quod verbum sumitur ibi figurative, prout
 significatum vel effectus verbi dicitur verbum. Sic enim creaturae
 dicuntur facere verbum Dei, inquantum exequuntur effectum aliquem, ad
 quem ordinantur ex verbo concepto divinae sapientiae, sicut aliquis
 dicitur facere verbum regis, dum facit opus ad quod ex verbo regis
 instigatur.

&

第四異論に対しては以下のように言われるべきである。
そこで「言葉」は比喩的に理解されていて、意味されたものや言葉の結果が言葉と言われてい
 る。この意味で被造物は、神の知恵がもつ懐念された言葉に基づいてそれへ
 と秩序づけられた何らかの結果を実行する限りにおいて神の言葉を行うと言
 われる。それはちょうど、ある人が王の言葉によって促された仕事を行うと
 きに王の言葉を行うと言われるようにである。


\end{longtable}

\newpage


\rhead{a.~2}
\begin{center}
{\Large {\bf ARTICULUS SECUNDUS}}\\
{\large UTRUM VERBUM SIT PROPRIUM NOMEN FILII}\\
{\footnotesize I {\itshape Sent.}, d.27, q.2, a.2, qu$^{a}$2;
 {\itshape De Verit.}, q.4, a.3; {\itshape Contra errores Graec.},
 cap.12; {\itshape ad Hebr.}, cap.1, lect.2.}\\
{\Large 第二項\\言葉は息子に固有の名称か}
\end{center}

\begin{longtable}{p{21em}p{21em}}
{\scshape Ad secundum sic proceditur}. Videtur quod verbum non sit proprium nomen
filii. Filius enim est persona subsistens in divinis. Sed verbum non
significat rem subsistentem, ut in nobis patet. Ergo verbum non potest
esse proprium nomen personae filii.

&

第二項の問題へ議論は以下のように進められる。
言葉は息子に固有の名称でないと思われる。理由は以下の通り。
息子は神において自存するペルソナである。しかるに言葉は、私たちにおいて
 明らかなとおり自存する事物を表示しない。ゆえに言葉は息子のペルソナに
 固有の名称ではありえない。


\\



2. {\scshape Praeterea}, verbum prolatione quadam procedit a dicente. Si ergo filius
est proprie verbum, non procedit a patre nisi per modum
prolationis. Quod est haeresis Valentini, ut patet per Augustinum, in
libro de haeresibus.

&

さらに、言葉はある種の発声によって語る者から発出する。ゆえに息子が固有
 に言葉であるならば、発声というしかたによらなければ父から発出しないで
 あろう。これはアウグスティヌスによって『異端について』という書物で明
 らかなとおり、ワレンティヌスの異端である。

\\



3. {\scshape Praeterea}, omne nomen proprium alicuius personae significat
proprietatem aliquam eius. Si igitur verbum sit proprium nomen filii,
significabit aliquam proprietatem eius. Et sic erunt plures
proprietates in divinis quam supra enumeratae sunt.

&

さらに、あるペルソナの固有の名称は、それの何らかの固有性を表示する。ゆ
 えにもし言葉が息子に固有の名称であるならば、彼の何らかの固有性を表示
 するはずである。そうすると、前に枚挙されたよりも多くの固有性が神の中
 にあることになるだろう。

\\



4. {\scshape Praeterea}, quicumque intelligit, intelligendo concipit verbum. Sed
filius intelligit. Ergo filii est aliquod verbum. Et sic non est
proprium filii esse verbum.

&

さらに、知性認識する者はだれでも、知性認識することによって言葉を懐念す
 る。しかるに息子は知性認識する。ゆえに息子は何らかの言葉を所有してい
 る。かくして言葉は息子に固有のものではない。

\\



5. {\scshape Praeterea}, {\itshape Hebr}.~{\scshape i} dicitur de filio, {\itshape portans omnia verbo virtutis
suae}, ex quo Basilius accipit quod spiritus sanctus sit verbum
filii. Non est ergo proprium filii esse verbum.

&

さらに『へブル人への手紙』第1章で息子について「自分の力の言葉で万物を
 運ぶ人」\footnote{「御子は神の栄光の輝きであり、神の本質の現れであって、万物をその力ある言葉によって支えておられます。そして、罪の清めを成し遂げて、天の高い所におられる大いなる方の右の座に着かれました。」(1:3)}と言われているが、バシリウスはこのことから聖霊が息子の言葉で
 あると理解している。ゆえに言葉であることは息子に固有のことではない。


\\



{\scshape Sed contra est} quod Augustinus dicit, VI de Trin., verbum solus filius
accipitur.

&

しかし反対にアウグスティヌスは『三位一体論』第6巻で「ただ一人息子だけ
 が言葉と理解される」と述べている。

\\



{\scshape Respondeo dicendum} quod verbum proprie dictum in divinis personaliter
accipitur, et est proprium nomen personae filii. Significat enim
quandam emanationem intellectus, persona autem quae procedit in
divinis secundum emanationem intellectus, dicitur filius, et huiusmodi
processio dicitur generatio, ut supra ostensum est. Unde relinquitur
quod solus filius proprie dicatur verbum in divinis.

&

解答する。以下のように言われるべきである。
言葉は固有に神においてペルソナ的に理解され、息子のペルソナに固有の名称
 である。なぜなら、前に示されたとおり\footnote{Q.27, a.2.}、それは何らかの知性の流出を意味するが、神の中で知性
 の流出において発出するものは息子と言われ、そしてそのような発出は生成
 と言われるからである。したがって、ただ息子だけが固有の意味で神におい
 て言葉と言われることが帰結する。

\\



{\scshape Ad primum ergo dicendum} quod in nobis non est idem esse et
intelligere, unde illud quod habet in nobis esse intelligibile, non
pertinet ad naturam nostram. Sed esse Dei est ipsum eius intelligere,
unde verbum Dei non est aliquod accidens in ipso, vel aliquis effectus
eius; sed pertinet ad ipsam naturam eius. Et ideo oportet quod sit
aliquid subsistens, quia quidquid est in natura Dei, subsistit. Et
ideo Damascenus dicit quod verbum Dei {\itshape est substantiale, et in
hypostasi ens, reliqua vero verba}, scilicet nostra, {\itshape virtutes sunt
animae}.

&

第一異論に対してはそれゆえ以下のように言われるべきである。
私たちにおいて存在することと知性認識することは同じでない。それゆえ私た
 ちの中で可知的な存在を持っているものは私たちの本性に属さない。
しかし神の存在は神の知性認識それ自体である。したがって神の言葉は神にお
 いて何らかの附帯性や神の何らかの結果ではなく、神の本性それ自体に属す
 る。ゆえにそれは自存する何かでなければならない。なぜなら神の本性にお
 いてあるものはなんであれ自存するからである。ゆえにダマスケヌスは、神
 の言葉は「実体的でありヒュポスタシスにおける存在者である。それ以外の
 言葉」つまり私たちの言葉は「魂の力である」と述べている。



\\



{\scshape Ad secundum dicendum} quod non propter hoc error Valentini est
damnatus, quia filium dixit prolatione natum, ut Ariani
calumniabantur, sicut Hilarius refert, VI {\itshape de Trin}., sed propter varium
modum prolationis quem posuit, sicut patet per Augustinum in libro {\itshape de
Haeresibus}.

&

第二異論に対しては以下のように言われるべきである。
ワレンティヌスの誤謬が断罪されたのは、アリウス派の人々が誤って批判した
 ように、息子が発声によって生まれたと言ったからではなく、ヒラリウスが
 『三位一体論』第6巻で言及しているように、彼が考えた異なるしかたの発声
 のためである。これは『異端について』のアウグスティヌスによって明らか
 である。

\\



{\scshape Ad tertium dicendum} quod in nomine verbi eadem proprietas importatur
quae in nomine filii, unde dicit Augustinus, {\itshape eo dicitur verbum, quo
filius}. Ipsa enim nativitas filii, quae est proprietas personalis
eius, diversis nominibus significatur, quae filio attribuuntur ad
exprimendum diversimode perfectionem eius. Nam ut ostendatur
connaturalis patri, dicitur filius; ut ostendatur coaeternus, dicitur
splendor; ut ostendatur omnino similis, dicitur imago; ut ostendatur
immaterialiter genitus, dicitur verbum. Non autem potuit unum nomen
inveniri, per quod omnia ista designarentur.

&

第三異論に対しては以下のように言われるべきである。
言葉という名称においては息子という名称におけるのと同じ固有性が意味され
 る。このことからアウグスティヌスは、それによって息子と言われるそれで
 もって言葉と言われる、と述べている。
息子のペルソナ的固有性である出生それ自体は、さまざまな名称によって表示
 されるが、それらは息子の完全性をさまざまなしかたで表現するために息子
 に帰せられる。すなわち、父と同じ本性であることを示すために「息子」と
 言われ、共に永遠であることを示すために「輝き」と言われ、あらゆる点で
 類似していることを示すために「似像」と言われ、非質料的に生まれたこと
 を示すために「言葉」と言われる。しかしこれらすべてが示される一つの名
 称を見出すことはできなかった。


\\



{\scshape Ad quartum dicendum} quod eo modo convenit filio esse intelligentem,
quo convenit ei esse Deum, cum intelligere essentialiter dicatur in
divinis, ut dictum est. Est autem filius Deus genitus, non autem
generans Deus. Unde est quidem intelligens, non ut producens verbum,
sed ut verbum procedens; prout scilicet in Deo verbum procedens
secundum rem non differt ab intellectu divino, sed relatione sola
distinguitur a principio verbi.

&

第四異論に対しては以下のように言われるべきである。
知性認識する者であることが息子に適合するのは、神であることが彼に適合す
 るのと同じしかたによる。なぜなら、すでに述べられたとおり、知性認識す
 ることは神において本質的に語られるからである。
しかるに息子は生まれた神であり、生み出す神ではない。したがって息子
 は、言葉を生み出す者としてではなく言葉が発出する者としての認識する者
 である。すなわち、神において発出する言葉は事物において神の知性と異な
 らず、たんに関係によってのみ言葉の根源から区別される。

\\



{\scshape Ad quintum dicendum} quod, cum de filio dicitur, {\itshape portans omnia verbo
virtutis suae}, verbum figurate accipitur pro effectu verbi. Unde
Glossa ibi dicit quod verbum sumitur pro imperio; inquantum scilicet
ex effectu virtutis verbi est quod res conserventur in esse, sicut ex
effectu virtutis verbi est quod res producantur in esse. Quod vero
Basilius interpretatur verbum pro spiritu sancto, improprie et
figurate locutus est, prout verbum alicuius dici potest omne illud
quod est manifestativum eius, ut sic ea ratione dicatur spiritus
sanctus verbum filii, quia manifestat filium.

&

第五異論に対しては以下のように言われるべきである。息子について「自らの
 力の言葉によって万物を運ぶ」と言われるときに、「言葉」は比喩的に言葉
 の結果として理解されている。このことからその箇所の『注解』で「言葉」
 は「権力」の意味に取られていると述べている。すなわち、言葉の力の結果
 から事物が存在へと生み出されるように、言葉の力の結果から事物は存在に
 おいて保存される、という意味である。他方で、バシリウスが「言葉」を聖
 霊として解釈していることは、なんであれその人を明示しうるものはその人
 の言葉と言われうるかぎりで、固有の意味ではなく比喩的に語られた。つま
 り、聖霊が息子の言葉と言われるのはそれが息子を明示するからである。






\end{longtable}
\newpage





\rhead{a.~3}
\begin{center}
{\Large {\bf ARTICULUS TERTIUSS}}\\
{\large UTRUM IN NOMINE VERBI IMPORTETUR RESPECTUS AD CREATURAM}\\
{\footnotesize Infra, q.37, a.2, ad 3; I {\itshape Sent.}, d.27, q.2,
 a.3; {\itshape De Verit.}, q.4, a.5; {\itshape Quodl.}~IV, q.4, a.1,
 ad 1.}\\
{\Large 第三項\\言葉という名称の中に被造物への関係が含意されているか}
\end{center}

\begin{longtable}{p{21em}p{21em}}

{\scshape Ad tertium sic proceditur}. Videtur quod in nomine verbi non importetur
respectus ad creaturam. Omne enim nomen connotans effectum in
creatura, essentialiter in divinis dicitur. Sed verbum non dicitur
essentialiter, sed personaliter, ut dictum est. Ergo verbum non
importat respectum ad creaturam.

&

第三項への問題へ議論は以下のように進められる。
言葉という名称の中に被造物への関係は含意されていないと思われる。理由は
 以下の通り。
被造物における結果を意味に含む名称はすべて、神において本質的に語られる。
 しかるにすでに述べられたとおり、言葉は本質的にではなくペルソナ的に語
 られる。ゆえに言葉は被造物への関係を含意しない。

\\



2. {\scshape Praeterea}, quae important respectum ad creaturas, dicuntur de Deo ex
tempore, ut dominus et creator. Sed verbum dicitur de Deo ab
aeterno. Ergo non importat respectum ad creaturam.

&

さらに、被造物への関係を含意するものは神について時間的に語られる。たと
 えば「主人」や「創造者」のように。しかるに言葉は神について永遠から語
 られる。ゆえに被造物への関係を含意しない。

\\



3. {\scshape Praeterea}, verbum importat respectum ad id a quo procedit. Si ergo
importat respectum ad creaturam, sequitur quod procedat a creatura.

&

さらに言葉はそこから発出したところのそれへの関係を含意する。ゆえにもし
 被造物への関係を含意するならば、被造物から発出したことになるだろう。

\\



4. {\scshape Praeterea}, ideae sunt plures secundum diversos respectus ad
creaturas. Si igitur verbum importat respectum ad creaturas, sequitur
quod in Deo non sit unum verbum tantum, sed plura.

&

さらに、イデアは被造物へのさまざまな関係に応じて複数である。ゆえにもし
 言葉が被造物への関係を含意するならば、神においてただ一つの言葉ではな
 く複数の言葉があることになる。

\\



5. {\scshape Praeterea}, si verbum importat respectum ad creaturam, hoc non est nisi
inquantum creaturae cognoscuntur a Deo. Sed Deus non solum cognoscit
entia, sed etiam non entia. Ergo in verbo importabitur respectus ad
non entia, quod videtur falsum.

&

さらにもし言葉が被造物への関係を含意するならば、それは被造物が神によっ
 て認識される限りにおいてに他ならない。しかるに神は存在するものだけで
 なく存在しないものも認識する。ゆえに言葉において存在しないものへの関
 係が含意されることになるがこれは偽であると思われる。

\\



{\scshape Sed contra est} quod dicit Augustinus, in libro {\itshape Octoginta Trium
Quaest}., quod in nomine Verbi {\itshape significatur non solum respectus ad
patrem, sed etiam ad illa quae per verbum facta sunt operativa
potentia}.

&

しかし反対に、アウグスティヌスは『八十三問題集』の中で、言葉という名称
 において「父への関係だけでなく、言葉を通して作用的能力によって作られた
 ものへの関係も意味されている」と述べている。

\\



{\scshape Respondeo dicendum} quod in verbo importatur respectus ad
creaturam. Deus enim, cognoscendo se, cognoscit omnem
creaturam. Verbum autem in mente conceptum, est repraesentativum omnis
eius quod actu intelligitur. Unde in nobis sunt diversa verba,
secundum diversa quae intelligimus. Sed quia Deus uno actu et se et
omnia intelligit, unicum verbum eius est expressivum non solum patris,
sed etiam creaturarum. 


&

解答する。以下のように言われるべきである。
言葉において被造物への関係が含意される。理由は以下の通り。
神は自らを認識することによってすべての被造物を認識する。
しかるに精神において懐念された言葉は現実に知性認識されるすべてのものを
 表現しうる。したがって私たちにおいては、私たちが知性認識するさまざま
 なものに応じてさまざまな言葉がある。しかし神は一つの作用によって自分
 自身と万物を知性認識するのだから、彼の一つの言葉は父だけでなく被造物
 も表現しうる。
\\

Et sicut Dei scientia Dei quidem est
cognoscitiva tantum, creaturarum autem cognoscitiva et factiva; ita
verbum Dei eius quod in Deo patre est, est expressivum tantum,
creaturarum vero est expressivum et operativum. Et propter hoc dicitur
in Psalmo {\scshape xxxii}, {\itshape dixit, et facta sunt}; quia in verbo importatur ratio
factiva eorum quae Deus facit.

&

そして、神の知が、神については認識しうるだけだが、被造物については認識
 しうると共に作りうるものであるように、神の言葉は、神の側にあるものに
 ついてはそれを表現しうるだけだが、被造物についてはそれを表現しうると
 共に働きうるものである。このため、『詩編』32章で「彼は語った、そして
 それらは作られた」\footnote{「主が語ると、そのように成り/主が命じると、そのように立った。」(33:9)}と言われている。なぜなら言葉において神が作るものど
 もの制作的理拠が含意されているからである。

\\



{\scshape Ad primum ergo dicendum} quod in nomine personae includitur etiam
natura oblique, nam persona est rationalis naturae individua
substantia. In nomine igitur personae divinae, quantum ad relationem
personalem, non importatur respectus ad creaturam, sed importatur in
eo quod pertinet ad naturam. Nihil tamen prohibet, inquantum
includitur in significatione eius essentia, quod importetur respectus
ad creaturam, sicut enim proprium est filio quod sit filius, ita
proprium est ei quod sit genitus Deus, vel genitus creator. Et per
hunc modum importatur relatio ad creaturam in nomine verbi.

&

第一異論に対してはそれゆえ以下のように言われるべきである。
ペルソナという名称にの中には裏の意味で本性もまた含まれる。すなわち、ペルソ
 ナとは理性的本性の個的実体である。ゆえに神のペルソナの名称において、
 ペルソナ的関係にかんする限り、被造物への関係は含意されないが、本性に
 属するものにおいてはそれが含意される。しかし、その本質の意味に含まれ
 るかぎりで、被造物への関係が含意されるのは何ら問題ない。たとえば、ちょ
 うど息子には息子であることが固有であるように、生まれた神であることや
 生まれた創造者であることが固有である。このしかたで言葉という名称には
 被造物への関係が含意される。

\\



{\scshape Ad secundum dicendum} quod, cum relationes consequantur actiones,
quaedam nomina important relationem Dei ad creaturam, quae consequitur
actionem Dei in exteriorem effectum transeuntem, sicut creare et
gubernare, et talia dicuntur de Deo ex tempore. Quaedam vero
relationem quae consequitur actionem non transeuntem in exteriorem
effectum, sed manentem in agente, ut scire et velle, et talia non
dicuntur de Deo ex tempore. Et huiusmodi relatio ad creaturam
importatur in nomine verbi. Nec est verum quod nomina importantia
relationem Dei ad creaturas, omnia dicantur ex tempore, sed sola illa
nomina quae important relationem consequentem actionem Dei in
exteriorem effectum transeuntem, ex tempore dicuntur.

&

第二異論に対しては以下のように言われるべきである。
関係は作用に伴うので、ある名称は、創造することや統治することなど外への結果へ越えて
 いく神の作用に伴う神の被造物への関係を含意する。そしてそのような名称
 は時間的に神について語られる。他方で、知ることや意志することなど外への結果へ越えていかず
作用者の内に留まる作用に伴う関係を名称もあり、これらは神について時間的
 に語られない。そして被造物へのこのような関係が言葉という名称において
 含意される。また、神の被造物への関係を含意する名称がすべて時間的に語
 られるというのも真ではない。そうではなく、外の結果へ越えていく神の作
 用に伴う関係を含意する名称だけが時間的に語られる。

\\



{\scshape Ad tertium dicendum} quod creaturae non cognoscuntur a Deo per
scientiam a creaturis acceptam, sed per essentiam suam. Unde non
oportet quod a creaturis procedat verbum, licet verbum sit expressivum
creaturarum.

&

第三異論に対しては以下のように言われるべきである。
被造物は、被造物から受け取られた知を通して神によって認識されるのではな
 く、神の本質をとおして認識される。したがって、言葉が被造物を表現しう
 るとしても被造物から言葉が発出する必要はない。

\\



{\scshape Ad quartum dicendum} quod nomen ideae principaliter est impositum ad
significandum respectum ad creaturam, et ideo pluraliter dicitur in
divinis, neque est personale. Sed nomen verbi principaliter impositum
est ad significandam relationem ad dicentem, et ex consequenti ad
creaturas, inquantum Deus, intelligendo se, intelligit omnem
creaturam. Et propter hoc in divinis est unicum tantum verbum, et
personaliter dictum.

&

第四異論に対しては以下のように言われるべきである。
イデアという名称は主要に被造物への関係を表示するために付けられている。
 それゆえ、神において複数で語られ、そしてペルソナ的には語られない。し
 かし言葉という名称は主要に、語る者への関係を表示し、
 結果的に、神が自分を知性認識することによってすべての被造物を知性認識
 する限りにおいて、被造物への関係を表示するために付けられている。この
 ために、神においてはただ一つの言葉があり、それはペルソナ的に語られる。

\\



{\scshape Ad quintum dicendum} quod eo modo quo scientia Dei est non entium, et
verbum Dei est non entium, quia non est aliquid minus in verbo Dei
quam in scientia Dei, ut Augustinus dicit. Sed tamen verbum est entium
ut expressivum et factivum, non entium autem, ut expressivum et
manifestativum.

&

第五異論に対しては以下のように言われるべきである。
神の知が存在しないものにかかわるのと同じしかたで神の言葉は存在しないも
 のにかかわる。なぜなら、アウグスティヌスが言うように、神の知の中にあるも
 ので神の言葉の中にないものはないからである。しかし言葉は、それを表現
 し、かつ作りうるかぎりで、存在するものにかかわるが、存在しないもの
 へは、それを表現し明示しうるかぎりでかかわる。


\end{longtable}
\newpage


\end{document}