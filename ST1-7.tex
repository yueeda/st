\documentclass[10pt]{jsarticle} % use larger type; default would be 10pt
%\usepackage[utf8]{inputenc} % set input encoding (not needed with XeLaTeX)
%\usepackage[round,comma,authoryear]{natbib}
%\usepackage{nruby}
\usepackage{okumacro}
\usepackage{longtable}
%\usepqckage{tablefootnote}
\usepackage[polutonikogreek,english,japanese]{babel}
%\usepackage{amsmath}
\usepackage{latexsym}
\usepackage{color}

%----- header -------
\usepackage{fancyhdr}
\pagestyle{fancy}
\lhead{{\it Summa Theologiae} I, q.~7}
%--------------------

\bibliographystyle{jplain}
\title{{\bf Prima Pars}\\{\HUGE Summae Theologiae}\\Sancti Thomae
Aquinatis\\Quaestio Septima\\{\bf De Infinitate Dei}}
\author{Japanese translation\\by Yoshinori {\sc Ueeda}}
\date{Last modified \today}


%%%% コピペ用
%\rhead{a.~}
%\begin{center}
% {\Large {\bf }}\\
% {\large }\\
% {\footnotesize }\\
% {\Large \\}
%\end{center}
%
%\begin{longtable}{p{21em}p{21em}}
%
%&
%
%\\
%\end{longtable}
%\newpage


\begin{document}
\maketitle

\begin{center}
 {\Large 第七問\\神の無限性について}
\end{center}

\begin{longtable}{p{21em}p{21em}}

Post considerationem divinae perfectionis, considerandum est de eius
 infinitate, et de existentia eius in rebus, attribuitur enim Deo quod
 sit ubique et in omnibus rebus, inquantum est incircumscriptibilis et
 infinitus. Circa primum quaeruntur quatuor. 


\begin{enumerate}
 \item utrum Deus sit
 infinitus.
 \item utrum aliquid praeter ipsum sit infinitum secundum
 essentiam.
 \item utrum aliquid possit esse infinitum secundum
 magnitudinem.
 \item utrum possit esse infinitum in rebus secundum
 multitudinem.
\end{enumerate}

&

神の完全性の考察の後に、神の無限性について、そして神の諸事物における存在
 について考察されるべきである。なぜなら、神が制限されず無限であるかぎり
 において、至る所に万物のなかに存在することが神に帰せられるからであ
 る。第一について四つのことが問われる。

\begin{enumerate}
 \item 神は無限か。
 \item 神以外の何かが、本質において無限か。
 \item 何かが大きさにおいて無限でありうるか。
 \item 諸事物の中に、多さにおいて無限なものがありうるか。
\end{enumerate}

\end{longtable}

\newpage
\rhead{a.~1}
\begin{center}
 {\Large {\bf ARTICULUS PRIMUS}}\\
 {\large UTRUM DEUS SIT INFINITUS}\\
 {\footnotesize Parte III, q.~10, a.~3, ad 1; I {\itshape Sent.}, d.~43,
 q.~1, a.~1; I {\itshape SCG.}, c.~43; {\itshape De Verit.}, q.~2, a.~2,
 ad 5; q.~29, a.~3; {\itshape De Pot.}, q.~1, a.~2; {\itshape Quodlib.}
 III, a.~3; {\itshape Compend.~Theol.}, c.~18, 20.}\\
 {\Large 第一項\\神は無限か}
\end{center}

\begin{longtable}{p{21em}p{21em}}


{\huge A}{\scshape d primum sic proceditur}. Videtur quod Deus non sit
infinitus. Omne enim infinitum est imperfectum, quia habet rationem
partis et materiae, ut dicitur in III {\itshape Physic}. Sed Deus est
perfectissimus. Ergo non est infinitus.

&

第一の問題へ、議論は以下のように進められる。神は無限でないと思われる。
 理由は以下の通り。すべて無
 限であるものは不完全である。なぜなら『自然学』第三巻で言われるよう
 に、それは部分と質料の性格を持つからである。しかるに神はもっとも完全である。
 ゆえに無限でない。


\\


2.~{\scshape Praeterea}, secundum Philosophum in I {\itshape Physic}., finitum et
infinitum conveniunt quantitati. Sed in Deo non est quantitas, cum non
sit corpus, ut supra ostensum est. Ergo non competit sibi esse
infinitum.

&

さらに、『自然学』第一巻の哲学者によれば、有限と無限は量に適合する。し
 かるに、前に示されたように神は物体でないので神の中に量はない。ゆえに、
 無限であることは神に適合しない。


\\


3.~{\scshape Praeterea}, quod ita est hic quod non alibi, est finitum secundum
locum, ergo quod ita est hoc quod non est aliud, est finitum secundum
substantiam. Sed Deus est hoc, et non est aliud, non enim est lapis nec
lignum. Ergo Deus non est infinitus secundum substantiam.

&

さらに、他の所ではなくここにあるものは場所において限られているし、他のも
 のではなくこれであるものは実体において限られている。しかし神は他のもの
 ではなくこれである。なぜなら神は石や木ではないのだから。ゆえに神は実
 体において無限でない。


\\


{\scshape Sed contra est} quod dicit Damascenus, quod Deus est {\itshape infinitus
et aeternus et incircumscriptibilis}.

&

しかし反対に、ダマスケヌスは「神は無限で永遠で限りがない」と述べている。


\\


{\scshape Respondeo dicendum} quod omnes antiqui philosophi attribuunt infinitum
primo principio, ut dicitur in III {\itshape Physic}., et hoc rationabiliter,
considerantes res effluere a primo principio in infinitum. Sed quia
quidam erraverunt circa naturam primi principii, consequens fuit ut
errarent  circa infinitatem ipsius. Quia enim ponebant primum principium
materiam, consequenter attribuerunt primo principio infinitatem
materialem; dicentes aliquod corpus infinitum esse primum principium
rerum.

&

解答する。以下のように言われるべきである。『自然学』第三巻で言われるように、すべての古
 代の哲学者たちは無限を第一根源に帰している。彼らは第一根源から無限
 に事物が流出すると考えているのだから、これには一理ある。しかしある人々
 は第一根源の本性について誤ったので、結果としてそれの無限性について
 も誤ることになった。なぜなら彼らは第一根源が質料だとしたので、なん
 らかの無限の物体が諸事物の第一根源だと言って、結果的に、第一根源に質料
 的無限を帰したからである。



\\


Considerandum est igitur quod infinitum dicitur aliquid ex eo quod non
est finitum. Finitur autem quodammodo et materia per formam, et forma
per materiam. Materia quidem per formam, inquantum materia, antequam
recipiat formam, est in potentia ad multas formas, sed cum recipit unam,
terminatur per illam. Forma vero finitur per materiam, inquantum forma,
in se considerata, communis est ad multa, sed per hoc quod recipitur in
materia, fit forma determinate huius rei. 

&

それゆえ、何かが無限と言われるのは限られていないことに基づく、というこ
 とが考察されるべきである。ところ
 で、ある意味で質料は形相によって限られ、また形相は質料によって限ら
 れる 。質料が形相によって限られるのは、形相を受け取る以前の質料が多く
 の形相に対して可能態にあるが、ひとつの形相を受け取るときその形相によっ
 て限られるからである。他方、形相が質料によって限られるのは、形相がそ
 れ自体で考えられるならば、多くのものに共通のものだが、しかし質料に受
 け取られることによって、限定的にこの事物の形相となるからである。




\\


-- Materia autem perficitur per formam per quam finitur, et ideo infinitum
secundum quod attribuitur materiae, habet rationem imperfecti; est enim
quasi materia non habens formam. Forma autem non perficitur per
materiam, sed magis per eam eius amplitudo contrahitur, unde infinitum
secundum quod se tenet ex parte formae non determinatae per materiam,
habet rationem perfecti.

&

ところで質料は、それを限定する形相によって完成される。だから質料に帰
せられる無限は不完全という性格をもつ。なぜならそれは、いわば形相を持たない質料(形のな
 い材料)だからである。これに対して、形相は質料によって完成されず、むし
 ろ質料によって形相の豊かさが制約される。したがって質料に限定されない
 形相という側面から保持するかぎりでの無限は、完全という性格をもつ 。



\\


Illud autem quod est maxime formale omnium, est ipsum esse, ut ex
superioribus patet. Cum igitur esse divinum non sit esse receptum in
aliquo, sed ipse sit suum esse subsistens, ut supra ostensum est;
manifestum est quod ipse Deus sit infinitus et perfectus.

&

さて、上述のことから明らかなとおり、万物の中でもっとも形相的なものは存在
それ自体である 。ゆえに上で示されたとおり、神の存在は何かに受け取ら
れた存在ではなく、神自身が自存する存在なので、神自身が無限であり完
全であることは明らかである。

\\



Et per hoc patet responsio ad primum.
&

このことによって、第一異論に対する解答は明らかである。


\\



{\scshape Ad secundum dicendum} quod terminus quantitatis est sicut forma ipsius,
cuius signum est, quod figura, quae consistit in terminatione
quantitatis, est quaedam forma circa quantitatem. Unde infinitum quod
competit quantitati, est infinitum quod se tenet ex parte materiae, et
tale infinitum non attribuitur Deo, ut dictum est.

&

第二異論に対しては、次のように言われるべきである。量の限界は、量の形相のよう
なものである。そのしるしに、形は、量の制限において成立するのだが、量に
まつわる一種の形相である。したがって、量に適合する無限は、質料の側から
ある無限であり、そのような無限は、すでに述べられたように、神に帰せられ
ない。


\\



{\scshape Ad tertium dicendum} quod, ex hoc ipso quod esse Dei est per se
subsistens non receptum in aliquo, prout dicitur infinitum, distinguitur
ab omnibus aliis, et alia removentur ab eo, sicut, si esset albedo
subsistens, ex hoc ipso quod non esset in alio, differret ab omni
albedine existente in subiecto.

&

第三異論に対しては、次のように言われるべきである。神の存在がそれ自体によっ
 て自存し、何にも受け取られていないということ自体から、そしてこのことか
 ら無限と言われるのだが、それは他の全てのものから区別され、また他のも
 のが神から取り除かれる。ちょうど、もしも自存する白があったとしたら、そ
 れが他のものにおいてあるのではないということ自体から、基体において存
 在するすべての白と異なるようなものである。


\end{longtable}
\newpage
\rhead{a.~2}


\begin{center}
 {\Large {\bf ARTICULUS SECUNDUS}}\\
 {\large UTRUM ALIQUID ALIUD QUAM DEUS POSSIT ESSE INFINITUM PER ESSENTIAM}\\
 {\footnotesize Infra, q.~50, a.~2, ad 4; III, q.~10, a.~3, ad 2, 3; I
 {\itshape Sent.}, d.~18, q.~1, a.~2; {\itshape De Verit.}, q.~29, a.~3;
 {\itshape Quodl.}~IX, a.~1; X, q.~2, a.~1, ad 2; XII, q.~2, ad 2; XI
 {\itshape Metaph.}, lect.~10.}\\
 {\Large 第二項\\神以外の何かが本質によって無限でありうるか}
\end{center}

\begin{longtable}{p{21em}p{21em}}

{\huge A}{\scshape d secundum sic proceditur}. Videtur quod aliquid
aliud quam Deus possit esse infinitum per essentiam. Virtus enim rei
proportionatur essentiae eius. Si igitur essentia Dei est infinita,
oportet quod eius virtus sit infinita. Ergo potest producere effectum
infinitum, cum quantitas virtutis per effectum cognoscatur.

&

第二項の問題へ、議論は以下のように進められる。神以外の何かが本質によって無限であ
 りうると思われる。理由は以下の通り。事物の力は事物の本質に比例する。ゆえにもし神の本
 質が無限なら、神の力も無限でなければならない。ゆえに力の量は結果によっ
 て知られるのだから、神は無限の結果を生み出すことができる。


\\

2.~{\scshape Praeterea}, quidquid habet virtutem infinitam, habet essentiam
infinitam. Sed intellectus creatus habet virtutem infinitam, apprehendit
enim universale, quod se potest extendere ad infinita singularia. Ergo
omnis substantia intellectualis creata est infinita.

&

さらに、なんであれ無限の力をもつものは、無限の本質をもつ。しかるに創造
 された知性は無限の力をもつ。なぜなら無限の個物へ広がる普遍を捉えるか
 らである。ゆえにすべての創造された知性的実体は無限である。



\\

3.~{\scshape Praeterea}, materia prima aliud est a Deo, ut supra
ostensum est. Sed materia prima est infinita. Ergo aliquid aliud praeter
Deum potest esse infinitum.

&

さらに、上で示されたとおり、第一質料は神と異なる。しかるに第一質料は無
 限である。ゆえに神以外の何かが無限でありうる。


\\

{\scshape Sed contra est} quod infinitum non potest esse ex principio aliquo, ut
dicitur in III {\itshape Physic}. Omne autem quod est praeter Deum, est ex Deo
sicut ex primo principio. Ergo nihil quod est praeter Deum, potest esse
infinitum.

&

しかし反対に、『自然学』第三巻で言われるように、無限はなんらかの根源から出てくるのではありえない。しかるに神以外の全てのものは、第一根源としての神から出てくる。ゆえに、神以外の何ものも無限ではありえない。


\\


{\scshape Respondeo dicendum} quod aliquid praeter Deum potest esse infinitum
secundum quid, sed non simpliciter. Si enim loquamur de infinito
secundum quod competit materiae, manifestum est quod omne existens in
actu, habet aliquam formam, et sic materia eius est terminata per
formam. Sed quia materia, secundum quod est sub una forma substantiali,
remanet in potentia ad multas formas accidentales; quod est finitum
simpliciter, potest esse infinitum secundum quid, utpote lignum est
finitum secundum suam formam, sed tamen est infinitum secundum quid,
inquantum est in potentia ad figuras infinitas.

&

解答する。以下のように言われるべきである。神以外のものは、ある意味において無限ではあり
 うるが、端的に無限ではありえない。理由は以下のとおり。もし私たちが
 質料に適合する限りでの無限について語るならば、現実に存在する全てのもの
 がなんらかの形相をもつことは明らかであり、またその意味でそれの質料が
 形相によって限られるのは明らかである。しかし質料は、ひとつの実体形相
 のもとにあるかぎりにおいて、多くの付帯形相に対して可能態にとどまる。つ
 まり、端的に言えば有限であるものが、ある意味で無限でありうる。たとえ
 ば木は、自らの形相において有限だが、無限の形に対して可能態にあるかぎ
 りでは、ある意味で無限である。



\\

Si autem loquamur de infinito secundum quod convenit formae, sic
manifestum est quod illa quorum formae sunt in materia, sunt simpliciter
finita, et nullo modo infinita. Si autem sint aliquae formae creatae non
receptae in materia, sed per se subsistentes, ut quidam de Angelis
opinantur, erunt quidem infinitae secundum quid, inquantum huiusmodi
formae non terminantur neque contrahuntur per aliquam materiam, sed quia
forma creata sic subsistens habet esse, et non est suum esse, necesse
est quod ipsum eius esse sit receptum et contractum ad determinatam
naturam. Unde non potest esse infinitum simpliciter.

&

これに対して、形相に属する限りでの無限について私たちが語るならば、その形
 相が質料においてあるものどもは、端的に有限であり、どんな意味においても
 無限ではありえない。しかし、ある人々が天使についてそう考えるように、な
 んらかの創造された形相が、質料に受け取られずそれ自体で自存するのであ
 れば、ある意味においてそれらは無限だろう。それは、そのような形相がな
 んらかの質料によって限定されたり制限されたりしないかぎりにおいてである。
 しかしそのように自存する被造の形相は存在をもち、自らの存在でないので、
それの存在自体は受け取られ限定された本性へと制限されている。
 したがってそれが端的に無限ではありえない。


\\


{\scshape Ad primum ergo dicendum} quod hoc est contra rationem facti, quod
essentia rei sit ipsum esse eius, quia esse subsistens non est esse
creatum, unde contra rationem facti est, quod sit simpliciter
infinitum. Sicut ergo Deus, licet habeat potentiam infinitam, non tamen
potest facere aliquid non factum (hoc enim esset contradictoria esse
simul); ita non potest facere aliquid infinitum simpliciter.

&

第一異論に対しては、それゆえ、次のように言わなければならない。「事物の
 本質がその事物の存在そのものである」ということは「作られたもの」の
性格に反する。なぜなら自存する存在は創造された存在ではないから。
 したがって端的に無限であるということは、「作られたもの」の性格に反す
 る。ゆえにちょうど神が、無限の能力を持っているとしても、作られないも
 のを作ることはできないように(もしそんなことができたとしたら、矛盾する
 ものどもが同時にあるということになったであろうから)、神は端的に無
 限なものを作ることはできない。

\\

{\scshape Ad secundum dicendum} quod hoc ipsum quod virtus intellectus extendit se
quodammodo ad infinita, procedit ex hoc quod intellectus est forma non
in materia; sed vel totaliter separata, sicut sunt substantiae
Angelorum; vel ad minus potentia intellectiva, quae non est actus
alicuius organi, in anima intellectiva corpori coniuncta.

&

第二異論に対しては、次のように言われるべきである。知性の力がある意味で自己
 を無限へ拡張させるということ自体は、知性が質料の中にない形相であること
 から出てくる。それは天使の実体がそうであるように(質料から)完全に
 分離しているか、あるいは身体に結合された知性的魂の知性的能力そうであ
 るように、少なくともどんな器官の現実態でもないかのいずれかである。


\\

{\scshape Ad tertium dicendum} quod materia prima non existit in rerum natura per
seipsam, cum non sit ens in actu, sed potentia tantum, unde magis est
aliquid concreatum, quam creatum. Nihilominus tamen materia prima, etiam
secundum potentiam, non est infinita simpliciter, sed secundum quid,
quia eius potentia non se extendit nisi ad formas naturales.

&

第三異論に対しては、次のように言われるべきである。第一質料は、それ自体によっ
 て諸事物の本性において存在するのではない。なぜならそれは現実態におい
 て在るものではなく、たんに可能態において在るものだからである。したがっ
 て創造されたものと言うよりは、むしろ「共に創造されたもの」である。
 しかしそれでも、第一質料は可能態においてであっても端的に無限ではな
 く、ある意味で無限であるに過ぎない。なぜならそれの可能態は、自然的形
 相にまでしか自らを拡張しないからである。




\end{longtable}
\newpage
\rhead{a.~3}

\begin{center}
 {\Large {\bf ARTICULUS TERTIUS}}\\
 {\large UTRUM POSSIT ESSE ALIQUID INFINITUM ACTU SECUNDUM MAGNITUDINEM}\\
 {\footnotesize {\itshape De Verit.}, q.~2, ad 5; {\itshape Quodl.} IX,
 a.~1; XII, q.~2, ad 2; I {\itshape Physic.}, lect.~9; III, lect.~7
 sqq.; I {\itshape de Caelo}, lect.~9 sqq.}\\
 {\Large 第三項\\何かが大きさにおいて現実に無限でありうるか}
\end{center}

\begin{longtable}{p{21em}p{21em}}
 

{\huge A}{\scshape d tertium sic proceditur}. Videtur quod possit esse
 aliquid infinitum actu secundum magnitudinem. In scientiis enim
 mathematicis non invenitur falsum, quia {\itshape abstrahentium non est
 mendacium}, ut dicitur in II {\itshape Physic}. Sed scientiae
 mathematicae utuntur infinito secundum magnitudinem, dicit enim
 geometra in suis demonstrationibus, {\itshape sit linea talis
 infinita}. Ergo non est impossibile aliquid esse infinitum secundum
 magnitudinem.

&

第三項の問題へ、議論は以下のように進められる。大きさにおいて現実に無限であるもの
 がありうると思われる。理由は以下のとおり。数学的諸学において偽は見
 出されない。なぜなら『自然学』第二巻で言われるように「抽象するものた
 ちに嘘はない」からである。しかるに数学的諸学は大きさにおいて無限で
 あるものを用いる。たとえば幾何学者はその証明の中で、「そのような線
 分が無限だとしてみよう」と言う。ゆえに何かが大きさにおいて無限である
 ことは不可能ではない。


\\


2.~{\scshape Praeterea}, id quod non est contra rationem alicuius, non est
 impossibile convenire sibi. Sed esse infinitum non est contra rationem
 magnitudinis, sed magis finitum et infinitum videntur esse passiones
 quantitatis. Ergo non est impossibile aliquam magnitudinem esse
 infinitam.

&

さらに、あるものの性格に反しないものはそれに属することが不可能ではない。
 しかるに無限であることは大きさの性格に反せず、むしろ有限と無限は
 量の属性であるように見える。ゆえになんらかの大きさが無限であることは不可
 能でない。

\\

3.~{\scshape Praeterea}, magnitudo divisibilis est in infinitum, sic enim definitur
 continuum, {\itshape quod est in infinitum divisibile}, ut patet in III
 {\itshape Physic}. Sed contraria nata sunt fieri circa idem. Cum ergo divisioni
 opponatur additio, et diminutioni augmentum, videtur quod magnitudo
 possit crescere in infinitum. Ergo possibile est esse magnitudinem
 infinitam.

&

さらに、大きさは無限に分割可能である。この意味で『自然学』第三巻で明ら
 かなように、連続体が「無限に分割されうるもの」と定義される。しかるに
 相反するものは同一のものをめぐって生じる本性をもつ。ゆえに、分割には
 付加が対立し減少には増大が対立するので、大きさは無限に増大しうるよ
 うに思われる。ゆえに大きさは無限でありうる。



\\

4.~{\scshape Praeterea}, motus et tempus habent quantitatem et continuitatem a
 magnitudine super quam transit motus, ut dicitur in IV {\itshape Physic}. Sed non
 est contra rationem temporis et motus quod sint infinita, cum
 unumquodque indivisibile signatum in tempore et motu circulari, sit
 principium et finis. Ergo nec contra rationem magnitudinis erit quod
 sit infinita.

&

さらに『自然学』第四巻で言われているように、運動と時間はその上を運動
 が移動していく大きさから量と持続性をもつ。しかるに時間と運動の性格
 に無限であることは反しない。なぜなら時間と円環的運動において、各々の
 不可分で指定された点は、始めであり終わりであるのだから。ゆえに大きさ
 の性格に無限であることは反しないであろう。


\\


{\scshape Sed contra}, omne corpus superficiem habet. Sed omne corpus superficiem
 habens est finitum, quia superficies est terminus corporis finiti. Ergo
 omne corpus est finitum. Et similiter potest dici de superficie et
 linea. Nihil est ergo infinitum secundum magnitudinem.

&

しかし反対に、物体はすべて表面をもつ。しかるにすべて表面をもつ物体は有
 限である。なぜなら表面は有限な物体の限界だから。ゆえにすべて物体
 は有限である。同様に表面と線分についても言われうる。ゆえに大きさに
 おいて無限なものは存在しない。

\\


{\scshape Respondeo dicendum} quod aliud est esse infinitum secundum suam
 essentiam, et secundum magnitudinem. Dato enim quod esset aliquod
 corpus infinitum secundum magnitudinem, utpote ignis vel aer, non tamen
 esset infinitum secundum essentiam, quia essentia sua esset terminata
 ad aliquam speciem per formam, et ad aliquod individuum per
 materiam. Et ideo, habito ex praemissis quod nulla creatura est
 infinita secundum essentiam, adhuc restat inquirere utrum aliquid
 creatum sit infinitum secundum magnitudinem. 

&

解答する。以下のように言われるべきである。自らの本質において無限であることと、大きさに
 おいて無限であることとは異なる。なぜなら、かりに大きさにおいて無限ななん
 らかの物体が存在したとしても、たとえば火や空気のような物体が大きさにお
 いて無限だったとしても、本質において無限な物体が存在するわけではないからであ
 る。なぜならそれの本質は、形相によってなんらかの種へ限定され、また
 質料によってなんらかの個体へと制限されただろうから。それゆえ上述の
 ことから、どんな被造物も本質によって無限ではないことが知られたが、創造
 された何かが大きさにおいて無限であるかどうかを問うことがなお残されて
 いる。


\\

Sciendum est igitur quod corpus, quod est magnitudo completa,
 dupliciter sumitur, scilicet mathematice, secundum quod consideratur in
 eo sola quantitas; et naturaliter, secundum quod consideratur in eo
 materia et forma. Et de corpore quidem naturali, quod non possit esse
 infinitum in actu, manifestum est. Nam omne corpus naturale aliquam
 formam substantialem habet determinatam, cum igitur ad formam
 substantialem consequantur accidentia, necesse est quod ad determinatam
 formam consequantur determinata accidentia; inter quae est
 quantitas. Unde omne corpus naturale habet determinatam quantitatem et
 in maius et in minus. Unde impossibile est aliquod corpus naturale
 infinitum esse. 

&

ところで、完全な大きさである物体(=立方体)は、二通りのかたちでとらえられる
 ことが知られるべきである。ひとつには数学的にであり、この場合、物体にお
 いてただ量だけが考察される。もうひとつは自然学的にであり、この場合、
 物体において質料と形相が考察される。そして自然的物体(=自然学的に理解され
 た物体)にかんして、それが現実態において無限でありえないことは明ら
 かである。なぜなら、すべて自然的な物体はなんらかの限定された実体形相
 をもつが、実体形相には付帯性が伴うので、限定された形相には限定された付
 帯性(この中に量がある)が伴う必要があるからである。したがって、すべて
 の自然的物体は、大きいにしろ小さいにしろ、限定された量をもっている。し
 たがって、なんらかの自然的物体が無限であることは不可能である。


\\


Hoc etiam ex motu patet. Quia omne corpus naturale habet aliquem motum
 naturalem. Corpus autem infinitum non posset habere aliquem motum
 naturalem, nec rectum, quia nihil movetur naturaliter motu recto, nisi
 cum est extra suum locum, quod corpori infinito accidere non posset;
 occuparet enim omnia loca, et sic indifferenter quilibet locus esset
 locus eius. 

&

このことは運動からも明らかである。なぜなら、すべて自然的物体はなんらか
 の自然的運動をもつ。しかるに(かりに無限大の物体があったとして)無限
 な物体はどんな自然的運動ももつことができなかったであろう。まず、直線
 運動をもたない。なぜなら、なにものもそれが自らの場所の外にあるとき
 でなければ、直線運動によって動かされることはないが、そのこと(自らの場
 所の外にあるということ)は無限な物体には起こりえないからである。とい
 うのもそのような物体はすべての場所を占めたであろうし、そのようにして、
 どんな場所もかのものの場所だっただろうから。



\\


Et similiter etiam neque secundum motum circularem. Quia in motu
 circulari oportet quod una pars corporis transferatur ad locum in quo
 fuit alia pars; quod in corpore circulari, si ponatur infinitum, esse
 non posset, quia duae lineae protractae a centro, quanto longius
 protrahuntur a centro, tanto longius distant ab invicem; si ergo corpus
 esset infinitum, in infinitum lineae distarent ab invicem, et sic una
 nunquam posset pervenire ad locum alterius. 

&

同様に、円環運動によっても動かされない。円環運動においては、物体のある部
 分が、(その物体の)他の部分があった場所へ移動することが必要だが、もし
 円状の物体が無限だとされるならば、このことは起こりえなかったであろう。
 なぜなら、中心から引き離された二つの線は、中心からより長く引き離される
 ほど、お互いにより遠く隔たる。ゆえに、もしその物体が無限であったならば、
 それらの線は、相互に無限に離れていただろう。かくして、一方が他方の場所
 へ到達することは、けっしてできなかったであろう。



\\


De corpore etiam mathematico eadem ratio est. Quia si imaginemur corpus
 mathematicum existens actu, oportet quod imaginemur ipsum sub aliqua
 forma, quia nihil est actu nisi per suam formam. Unde, cum forma
 quanti, inquantum huiusmodi, sit figura, oportebit quod habeat aliquam
 figuram. Et sic erit finitum, est enim figura, quae termino vel
 terminis comprehenditur.

&

数学的な物体(数学的に理解された物体)についても同様の推論がある。もし
 私たちが、現実的に存在する数学的な物体を想像するならば、それを何らか
 の形相のもとに想像しなければならない。なぜならなにものも、自らの形相
 によってでなければ現実に存在しないからである。したがって量の形相は、
 量の形相であるかぎりにおいて形であるから、それは何らかの形をもたなけ
 ればならない。このようにして、それは有限なものであろう。なぜならそれ
 は、ひとつまたは複数の限界によって包含されている形だからである。


\\


{\scshape Ad primum ergo dicendum} quod geometer  non indiget sumere aliquam lineam
 esse infinitam actu, sed indiget accipere aliquam lineam finitam actu ,
 a qua possit subtrahi quantum necesse est, et hanc nominat lineam
 infinitam.

&

第一異論に対しては、それゆえ、次のように言われるべきである。幾何学者は、ある
 線が現実に無限であるとみなすことを必要としているのではなく、必要なだけ
 そこから取り去ることができる[現実的に有限な]線を理解することを必要と
 している。そしてその線を無限な線と名付けているのである。

\\


{\scshape Ad secundum dicendum} quod, licet infinitum non sit contra rationem
 magnitudinis in communi, est tamen contra rationem cuiuslibet speciei
 eius, scilicet contra rationem magnitudinis bicubitae vel tricubitae,
 sive circularis vel triangularis, et similium. Non autem est possibile
 in genere esse quod in nulla specie est. Unde non est possibile esse
 aliquam magnitudinem infinitam, cum nulla species magnitudinis sit
 infinita.

&

第二異論に対しては次のように言われるべきである。無限であることは、共通的な大
 きさの性格に反しないけれども、大きさのどの種の性格にも反する。すなわち
 それは、2キュビトや3キュビトの大きさという性格に反するし、円の大きさ、
 三角の大きさなど、これに類似の大きさにも反する。ところで、どんな種の中
 にもないものが類のなかにあることは不可能である。したがって、大きさの
 どの種も無限でないのだから、なんらかの無限な大きさが存在することは不可
 能である。


\\


{\scshape Ad tertium dicendum} quod infinitum quod convenit quantitati, ut dictum
 est, se tenet ex parte materiae. Per divisionem autem totius acceditur
 ad materiam, nam partes se habent in ratione materiae, per additionem
 autem acceditur ad totum, quod se habet in ratione formae. Et ideo non
 invenitur infinitum in additione magnitudinis, sed in divisione tantum.

&

第三に対しては次のように言われるべきである。すでに述べられたように、量に
 適合する無限は質料の側からある。しかるに全体を分割することによって
 質料へ近づく。なぜなら部分は質料の性格をもつからである。これに対し
 て付加することによって全体へと近づくが、全体は形相の性格をもつ。ゆえ
 に、大きさを付加することにおいて無限は見出されず、ただ分割することにおい
 て見出されるのみである。

\\


{\scshape Ad quartum dicendum} quod motus et tempus non sunt secundum totum in
 actu, sed successive, unde habent potentiam permixtam actui. Sed
 magnitudo est tota in actu. Et ideo infinitum quod convenit quantitati,
 et se tenet ex parte materiae, repugnat totalitati magnitudinis, non
 autem totalitati temporis vel motus, esse enim in potentia convenit
 materiae.

&

運動と時間は、全体においてではなく継起的に現実態にある。したがって現実
 態に混じって可能態を持つ。しかし大きさは全体が現実態にある。ゆえに
 量に適合し質料の側からある無限は、大きさの全体性には矛盾するが、時間
 と運動の全体性には矛盾しない。なぜなら可能態において在るということは、
 質料に適合するからである。


\end{longtable}
\newpage
\rhead{a.~4}
 

\begin{center}
 {\Large {\bf ARTICULUS QUARTUS}}\\
 {\large UTRUM POSSIT ESSE INFINITUM IN REBUS SECUNDUM MULTITUDINEM}\\
 {\footnotesize II {\itshape Sent.}, d.~1, q.~1, a.~5, ad 17 sqq.;
 {\itshape De Verit.}, q.~2, a.~10; {\itshape Quodl.}~IX, a.~1; XII,
 q.~2, ad 2; III {\itshape Physic.}, lect.~12.}\\
 {\Large 第四項\\諸事物の中に、多さにおいて無限なものがありうるか}
\end{center}

\begin{longtable}{p{21em}p{21em}}
{\huge A}{\scshape d quartum sic proceditur}. Videtur quod possibile sit esse
 multitudinem infinitam secundum actum. Non enim est impossibile id quod
 est in potentia reduci ad actum. Sed numerus est in infinitum
 multiplicabilis. Ergo non est impossibile esse multitudinem infinitam
 in actu.

&

第四項の問題へ、議論は以下のように進められる。現実態において無限の多がありうる
 と思われる。理由は以下のとおり。可能態にあるものが現実態ヘもたらされる
 ことは不可能ではない。しかるに数は無限に多数化される。ゆえに現実
 態において無限の多があることは不可能ではない。


\\


2.~{\scshape Praeterea}, cuiuslibet speciei possibile est esse aliquod
 individuum in actu. Sed species figurae sunt infinitae. Ergo possibile
 est esse infinitas figuras in actu.

&

さらに、どんな種にとっても、その種に属するある個体が現実に存在することは
 不可能ではない。しかるに形の種は無限である。ゆえに無限の形が現実に
 存在することは不可能ではない。




\\

3.~{\scshape Praeterea}, ea quae non opponuntur ad invicem, non impediunt se
 invicem. Sed, posita aliqua multitudine rerum, adhuc possunt fieri alia
 multa quae eis non opponuntur, ergo non est impossibile aliqua iterum
 simul esse cum eis, et sic in infinitum. Ergo possibile est esse
 infinita in actu.

&

さらに、相互に対立しないものどもは相互に妨げ合わない。しかるに諸事物
 のなんらかの多数性が措定されても、それらに対立しない他の多くのものが生
 じうる。ゆえにこれら(後者のものども)とともに、さらにあるものが同時
 に存在することは不可能ではない。これは無限に続く。ゆえに無限のものが
 現実に存在することが可能である。


\\

{\scshape Sed contra est} quod dicitur {\itshape Sap}.~{\scshape xi}, {\itshape omnia in pondere, numero et mensura
 disposuisti}.

&

しかし反対に、『知恵の書』11章で「あなたは万物を重さ、数、尺度において
 配置した」\footnote{「しかしあなたは、物差しと数と秤とによって/すべ
 てを按配された」(11:20)}と言われている。


\\


Respondeo dicendum quod circa hoc fuit duplex opinio. Quidam enim, sicut
 Avicenna et Algazel, dixerunt quod impossibile est esse multitudinem
 actu infinitam per se, sed infinitam per accidens multitudinem esse,
 non est impossibile. 

&

解答する。以下のように言われるべきである。これについては二通りの意見があった。ある人々
 は、アヴィセンナ やアルガゼル のように、自体的に現実態における無限の
 多はありえないが、付帯的に(現実態における)無限の多があることは不可
 能でないと言った。

\\


Dicitur enim multitudo esse infinita per se, quando requiritur ad
 aliquid ut multitudo infinita sit. Et hoc est impossibile esse, quia
 sic oporteret quod aliquid dependeret ex infinitis; unde eius generatio
 nunquam compleretur, cum non sit infinita pertransire.

&

多が自体的に無限と言われるのは、なにかのために、無限の多が存在すること
 が必要とされる場合である。そしてこれは不可能である。なぜなら、もしこ
 のようであれば、あるものが無数のものに依存することになり、無数のものの
 中を通過することは不可能だから、それの生成はけっして完成しないだろうか
 ら。

\\

Per accidens autem dicitur multitudo infinita, quando non requiritur ad
 aliquid infinitas multitudinis, sed accidit ita esse. Et hoc sic
 manifestari potest in operatione fabri, ad quam quaedam multitudo
 requiritur per se, scilicet quod sit ars in anima, et manus movens, et
 martellus . Et si haec in infinitum multiplicarentur, nunquam opus
 fabrile compleretur, quia dependeret ex infinitis causis. Sed multitudo
 martellorum quae accidit ex hoc quod unum frangitur et accipitur aliud,
 est multitudo per accidens, accidit enim quod multis martellis
 operetur; et nihil differt utrum uno vel duobus vel pluribus operetur,
 vel infinitis, si infinito tempore operaretur. Per hunc igitur modum,
 posuerunt quod possibile est esse actu multitudinem infinitam per
 accidens.

&

これに対して、多が付帯的に無限と言われるのは、なにかのために多の無限
 性が必要とされるわけではないが、たまたま無数のものが存在するようになる
 場合である。これは職人の仕事において次のように明らかにされうる。仕事
 には、一種の多が自体的に必要とされる。たとえば、魂の中の技術知、動か
 す手、槌といったものが。もしこれらが無限に多数化されたならば、職人の
 仕事はけっして完成しなかったであろう。なぜなら、無限の原因に依存したか
 らである。しかし一つの槌が壊れて別の槌を受け取るということから生じる
 槌の多数性は付帯的な多である。なぜなら多くの槌を用いて働くというこ
 とはありうることであり、一つの槌で働くことも二つの槌で働くことも、複
 数の槌で、そして無限の時間働くのであれば無数の槌で働くことも、違いは
 ないからである。ゆえにこのしかたで、彼らは付帯的に無限な多が現実に
 あることは可能だとしたのである。

\\


Sed hoc est impossibile. Quia omnem multitudinem oportet esse in aliqua
 specie multitudinis. Species autem multitudinis sunt secundum species
 numerorum. Nulla autem species numeri est infinita, quia quilibet
 numerus est multitudo mensurata per unum. Unde impossibile est esse
 multitudinem infinitam actu, sive per se, sive per accidens.

&

しかしこれは不可能である。なぜなら、すべて多は多のなんらかの種の中にな
 ければならない。しかるに多の種は数の種にしたがってある。ところが、ど
 の数の種も無限でない。なぜならどの数も一によって測られる多だからであ
 る。したがって自体的にであれ付帯的にであれ、現実に無限の多が存在する
 ことは不可能である。


\\

Item, multitudo in rerum natura existens est creata, et omne creatum sub
 aliqua certa intentione creantis comprehenditur, non enim in vanum
 agens aliquod operatur. Unde necesse est quod sub certo numero omnia
 creata comprehendantur. Impossibile est ergo esse multitudinem
 infinitam in actu, etiam per accidens.

&

さらに、諸事物の本性において存在する多は創造されたものであり、すべて創
 造されたものは創造者のあるなんらかの特定の意図のもとに包含される。働
 くものは、無駄に働くことがないからである。したがって、ある特定の数のも
 とに、すべて創造されたものは包含されていることが必然である。ゆえに、付
 帯的であっても、現実に無限の多が存在することは不可能である。


\\

Sed esse multitudinem infinitam in potentia, possibile est. Quia
 augmentum multitudinis consequitur divisionem magnitudinis, quanto enim
 aliquid plus dividitur, tanto plura secundum numerum resultant. Unde,
 sicut infinitum invenitur in potentia in divisione continui, quia
 proceditur ad materiam, ut supra ostensum est; eadem ratione etiam
 infinitum invenitur in potentia in additione multitudinis.

&

しかし、可能態において無限の多が存在することは可能である。なぜなら大
 きさの分割には多数性の増大が伴う、つまり何かがより多く分割されるだけ数
 においてより多くのものが結果として生じるからである。したがって上で示された通り、
 連続体の分割において可能態において無限が見出されるように、というのは
 これは質料の方向へ進められるからなのだが、同じ理由で多数性の付加にお
 いて、可能態において無限が見出される。


\\


{\scshape Ad primum ergo dicendum} quod unumquodque quod est in potentia,
 reducitur in actum secundum modum sui esse, dies enim non reducitur in
 actum ut sit tota simul, sed successive. Et similiter infinitum
 multitudinis non reducitur in actum ut sit totum simul, sed successive,
 quia post quamlibet multitudinem, potest sumi alia multitudo in
 infinitum.

&

第一異論に対しては、それゆえ、次のように言われるべきである。可能態にあるもの
 はなんであれ、自らの存在のあり方に従って現実態ヘもたらされる。たとえ
 ば一日は、全体が同時にというかたちではなく、継起的に現実態ヘもたらさ
 れる。同様に、多の無限は全体が同時にというかたちで現実態ヘもたらされ
 るのではなく、継起的にである。なぜなら、どんな多のあとにも別の多が無
 限に取られうるからである。


\\

{\scshape Ad secundum dicendum} quod species figurarum habent infinitatem ex
 infinitate numeri, sunt enim species figurarum, trilaterum,
 quadrilaterum, et sic inde. Unde, sicut multitudo infinita numerabilis
 non reducitur in actum quod sit tota simul, ita nec multitudo
 figurarum.

&

第二異論に対しては次のように言われるべきである。形の種は、数の無限性に基づい
 て無限性をもつ。なぜなら形の種とは、三角形、四角形など、そのような
 ものだからである。したがって、ちょうど数えられうる無限の多が全体が同
 時にであるような現実態ヘもたらされることがないように、形の多もそいう
 いうことがない。




\\

{\scshape Ad tertium dicendum} quod, licet, quibusdam positis, alia poni non
 sit eis oppositum; tamen infinita poni opponitur cuilibet speciei
 multitudinis. Unde non est possibile esse aliquam multitudinem actu
 infinitam.

&

第三異論に対しては次のように言われるべきである。あるものが置かれたとき、他
 のものが置かれることはそれらに対立しないとしても、しかし、無限のもの
 が置かれることは、多のどんな種にも対立する。したがって、現実に無限で
 あるなんらかの多が存在することは不可能である。


\end{longtable}


\end{document}