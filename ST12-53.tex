\documentclass[10pt]{jsarticle} % use larger type; default would be 10pt
\usepackage[utf8]{inputenc} % set input encoding (not needed with XeLaTeX)
\usepackage[T1]{fontenc}
%\usepackage[round,comma,authoryear]{natbib}
%\usepackage{nruby}
\usepackage{okumacro}
\usepackage{longtable}
%\usepqckage{tablefootnote}
\usepackage[polutonikogreek,english,japanese]{babel}
%\usepackage{amsmath}
\usepackage{latexsym}
\usepackage{color}
\usepackage{otf}
\usepackage{schemata}
%----- header -------
\usepackage{fancyhdr}
\pagestyle{fancy}
\lhead{{\it Summa Theologiae} I-II, q.53}
%--------------------


\bibliographystyle{jplain}


\title{{\bf Prima Secundae}\\{\HUGE Summae Theologiae}\\Sancti Thomae
Aquinatis\\{\sffamily QUAESTIO QUINQUAGESIMATERTIA}\\{\bf DE CORRUPTIONE ET DIMINUTIONE
HABITUUM}}
\author{Japanese translation\\by Yoshinori {\sc Ueeda}}
\date{Last modified \today}

%%%% コピペ用
%\rhead{a.~}
%\begin{center}
% {\Large {\bf }}\\
% {\large }\\
% {\footnotesize }\\
% {\Large \\}
%\end{center}
%
%\begin{longtable}{p{21em}p{21em}}
%
%&
%
%\\
%\end{longtable}
%\newpage

\begin{document}

\maketitle
\thispagestyle{empty}
\begin{center}
{\Large 『神学大全』第二部の一\\第五十三問\\習慣の消滅と減衰について}
\end{center}


\begin{longtable}{p{21em}p{21em}}
Deinde considerandum est de corruptione et diminutione habituum. Et circa hoc quaeruntur tria. 



\begin{enumerate}
 \item utrum habitus corrumpi possit.
 \item utrum possit diminui.
 \item de modo corruptionis et diminutionis.
\end{enumerate}

&

次に、習慣の消滅と減衰について考察されるべきである。これを巡って三つのことが問われる。
\begin{enumerate}
 \item 習慣は消滅しうるか。
 \item それは減衰しうるか。
 \item 消滅と減衰のあり方について。
\end{enumerate}
\end{longtable}
\newpage
\rhead{a.~1}
\begin{center}
{\Large {\bf ARTICULUS PRIMUS}}\\
{\large UTRUM HABITUS CORRUMPI POSSIT}\\
{\footnotesize Part.I, q.89, a.5.}\\
{\Large 第一項\\習慣は消滅しうるか}
\end{center}

\begin{longtable}{p{21em}p{21em}}


{\scshape Ad primum sic proceditur}. Videtur quod habitus corrumpi non
possit. Habitus enim inest sicut natura quaedam, unde operationes
secundum habitum sunt delectabiles. Sed natura non corrumpitur,
manente eo cuius est natura. Ergo neque habitus corrumpi potest,
manente subiecto.

&

第一項の問題へ議論は以下のように進められる。習慣は消滅しえないと思われ
る。理由は以下の通り。習慣はある種の本性のようなものとして内在する。し
かるに本性はそれが属するものが存続しているときには消滅しない。ゆえに習
慣も基体が存続しているときには消滅しえない。

\\



2. {\scshape Praeterea}, omnis corruptio formae vel est per
corruptionem subiecti, vel est a contrario, sicut aegritudo
corrumpitur corrupto animali, vel etiam superveniente sanitate. Sed
scientia, quae est quidam habitus, non potest corrumpi per
corruptionem subiecti, quia intellectus, qui est subiectum eius, est
substantia quaedam, et non corrumpitur, ut dicitur in I de
anima. Similiter etiam non potest corrumpi a contrario, nam species
intelligibiles non sunt ad invicem contrariae, ut dicitur in VII
Metaphys. Ergo habitus scientiae nullo modo corrumpi potest.


&

さらに、すべての形相の消滅は、たとえば病気が動物の消滅によって消滅する
場合のように、基体の消滅によるか、あるいは、たとえば健康が到来すること
によって消滅する場合のように、反対のものによってであるかのどちらかであ
る。しかるにある種の習慣である知は、基体の消滅によって消滅しえない。な
ぜなら知の基体である知性は、『デ・アニマ』第1巻で言われるように、ある
種の実体であり消滅しないからである。同様に、反対のものによっても消滅し
ない。なぜなら『形而上学』第7巻で言われるように、可知的形象は相互に反
対のものでないからである。ゆえに知という習慣はけっして消滅しえない。



\\



3. Praeterea, omnis corruptio est per aliquem motum. Sed habitus
scientiae, qui est in anima, non potest corrumpi per motum per se
ipsius animae, quia anima per se non movetur. Movetur autem per
accidens per motum corporis.


&

さらに、すべての消滅は何らかの運動によってある。しかるに知の習慣は魂の
中にあるが、魂自身の自体的な運動によっては消滅しえない。なぜなら、魂そ
れ自体は動かないからである。ただし、身体の運動によって附帯的に動く。


\\


Nulla autem transmutatio corporalis videtur posse corrumpere species
intelligibiles existentes in intellectu, cum intellectus sit per se
locus specierum, sine corpore, unde ponitur quod nec per senium nec
per mortem corrumpuntur habitus.

&

しかし、どんな物体的変化も知性の中にある可知的形象を消滅させえないと思
われる。なぜなら、知性は身体なしに、自体的に種の場所だからである。この
ことから、老齢によっても死によっても習慣は消滅しないとされる。

\\


Ergo scientia corrumpi non potest. Et per consequens, nec habitus
virtutis, qui etiam est in anima rationali, et, sicut philosophus
dicit in I {\itshape Ethic}., {\itshape virtutes sunt permanentiores
disciplinis}.


&


ゆえに、知は消滅しえない。したがって、理性的な魂の中にある徳の習慣もま
た(消滅しえない)。そして、哲学者が『ニコマコス倫理学』第1巻で言うよ
うに、「徳は学習よりも永続する」。

\\



{\scshape Sed contra est} quod philosophus dicit, in libro de
longitudine et brevitate vitae, quod {\itshape scientiae corruptio est
oblivio et deceptio}. Peccando etiam aliquis habitum virtutis
amittit. Et ex contrariis actibus virtutes generantur et corrumpuntur,
ut dicitur II {\itshape Ethic}.


&

しかし反対に、哲学者が『生命の長さと短さについて』という書物の中で、知
の消滅は忘却と欺瞞である、と述べている。しかるに、人は罪を犯すことによっ
て徳の習慣を捨てる。そして、『ニコマコス倫理学』第2巻で言われるように、
反対の行為から徳は生成し消滅する。


\\

{\scshape Respondeo dicendum quod} secundum se dicitur aliqua forma
corrumpi per contrarium suum, per accidens autem, per corruptionem sui
subiecti. Si igitur fuerit aliquis habitus cuius subiectum est
corruptibile, et cuius causa habet contrarium, utroque modo corrumpi
poterit, sicut patet de habitibus corporalibus, scilicet sanitate et
aegritudine.


&

解答する。以下のように言われるべきである。それ自体に即して語られるなら
ば、ある形相は、自らに反対のものによって消滅する。しかし附帯的に語られ
るならば、自らの基体の消滅によっても消滅する。ゆえに、もしその基体が可
滅的であるような習慣や、それの原因が反対のものを持つような習慣があった
ならば、どちらのしかたによっても消滅しえたであろう。ちょうど身体的な習
慣について明らかであるように。つまりそれは健康や病気によって(消滅しう
るのだから)。

\\


Illi vero habitus quorum subiectum est incorruptibile, non possunt
corrumpi per accidens. Sunt tamen habitus quidam qui, etsi
principaliter sint in subiecto incorruptibili, secundario tamen sunt
in subiecto corruptibili, sicut habitus scientiae, qui principaliter
est quidem in intellectu possibili, secundario autem in viribus
apprehensivis sensitivis, ut supra dictum est.


&

他方で、その基体が不滅であるような習慣は、附帯的には消滅しえない。しか
し、主要には不滅の基体の中に在るとしても、副次的には可滅的な基体の中に
在るような習慣がある。たとえば知という習慣は、主要には可能知性の中に在
るが、副次的には、上で述べられたとおり、感覚的な把握力の中にある。


\\


Et ideo ex parte intellectus possibilis, habitus scientiae non potest
corrumpi per accidens; sed solum ex parte inferiorum virium
sensitivarum.



&

ゆえに、可能知性の側から知の習慣が附帯的に消滅することはありえず、ただ
内部感覚の力の側からそのことが起こりうる。

\\


Est igitur considerandum si possunt huiusmodi habitus per se
corrumpi. Si igitur fuerit aliquis habitus qui habeat aliquod
contrarium, vel ex parte sua vel ex parte suae causae, poterit per se
corrumpi, si vero non habet contrarium, non poterit per se corrumpi.


&

ゆえに、そのような習慣が、それ自体によって消滅しうるかどうかが考察され
るべきである。ゆえに、自分の側からか、あるいは自分の原因の側から、何ら
かの反対物を持つような習慣があったならば、それ自体によって消滅しえただ
ろう。そうでなくもし反対物を持たないならば、それ自身によって消滅するこ
とはありえなかっただろう。


\\


Manifestum est autem quod species intelligibilis in intellectu
possibili existens, non habet aliquid contrarium. Neque iterum
intellectui agenti, qui est causa eius, potest aliquid esse
contrarium.


&

ところで、可能知性の中に存在する可知的形象が、何らの反対物も持たないこ
とは明らかである。さらに、それ(=可知的形象)の原因である能動知性もま
た、反対のものではありえない。

\\


Unde si aliquis habitus sit in intellectu possibili immediate ab
intellectu agente causatus, talis habitus est incorruptibilis et per
se et per accidens. Huiusmodi autem sunt habitus primorum
principiorum, tam speculabilium quam practicorum, qui nulla oblivione
vel deceptione corrumpi possunt, sicut philosophus dicit, in VI
Ethic., de prudentia, quod non perditur per oblivionem.


&

したがって、もし能動知性によって直接的に原因される何らかの習慣が可能知
性の中に在るならば、そのような習慣は、それ自体によっても附帯的にも消滅
しえない。しかるに、観照的なものも実践的なものも第一諸原理についての習
慣はそのようなものであり、どんな忘却や欺瞞によっても消滅しえない。それ
は哲学者が『ニコマコス倫理学』第6巻で、思慮について、「忘却によって失
われない」と述べているとおりである。

\\


Aliquis vero habitus est in intellectu possibili ex ratione causatus,
scilicet habitus conclusionum, qui dicitur scientia, cuius causae
dupliciter potest aliquid contrarium esse.


&

他方で、可能知性の中のある習慣は、理性によって原因されたものであり、す
なわちそれは、知(学知)と呼ばれる結論の習慣である。その原因には、二通
りのしかたで、何かが反対のものでありうる。

\\


Uno modo, ex parte ipsarum propositionum ex quibus ratio procedit,
etenim enuntiationi quae est, bonum est bonum, contraria est ea quae
est, bonum non est bonum, secundum philosophum, in II Periherm.


&

一つには、理性がそれから出発する命題の側からであり、たとえば『命題論』
第2巻の哲学者によれば、「善は善でない」という命題は「善は善である」と
いう命題に反対する。


\\


Alio modo, quantum ad ipsum processum rationis; prout syllogismus
sophisticus opponitur syllogismo dialectico vel demonstrativo. Sic
igitur patet quod per falsam rationem potest corrumpi habitus verae
opinionis, aut etiam scientiae. Unde philosophus dicit quod deceptio
est corruptio scientiae, sicut supra dictum est.


&

もう一つには、理性の進行(推論)それ自体にかんしてであり、たとえば、詭
弁的三段論法は、対話的あるいは論証的三段論法に対立する。それゆえ、この
ように、偽の理性によって、真の意見やさらには学知の習慣が消滅しうる。こ
のことから哲学者は、上述のように、欺瞞は知の消滅であると述べている。

\\


Virtutum vero quaedam sunt intellectuales, quae sunt in ipsa ratione,
ut dicitur in VI Ethic., de quibus est eadem ratio quae est de
scientia vel opinione.


&

他方、徳のうちのあるものは知性的であり、それらは『ニコマコス倫理学』第
6巻で言われるように、理性自身の中にあり、それらについては、学知や意見
についてと同じことが言える。


\\


Quaedam vero sunt in parte animae appetitiva, quae sunt virtutes
morales, et eadem ratio est de vitiis oppositis. Habitus autem
appetitivae partis causantur per hoc quod ratio nata est appetitivam
partem movere. Unde per iudicium rationis in contrarium moventis
quocumque modo, scilicet sive ex ignorantia, sive ex passione, vel
etiam ex electione, corrumpitur habitus virtutis vel vitii.


&

また他方で、ある徳は欲求的魂の側にあり、それらは道徳的な徳であり、反対
する悪徳についても同じことが言える。しかし、欲求的部分の習慣は、理性が
欲求的部分を動かす本性があることによって原因される。したがって、どんな
しかたでも反対へと動かす理性の判断によって、すなわち、無知によって、あ
るいは情念によって、あるいは選択によっても、徳あるいは悪徳の習慣が消滅
する。


\\



{\scshape Ad primum ergo dicendum} quod, sicut dicitur in VII
{\itshape Ethic}., habitus similitudinem habet naturae, deficit tamen
ab ipsa. Et ideo, cum natura rei nullo modo removeatur ab ipsa,
habitus difficile removetur.


&

第一異論に対しては、それゆえ、以下のように言われるべきである。『ニコマ
コス倫理学』第7巻で言われるように、習慣は本性への類似を持つが、しかし
本性には欠ける。ゆえに、事物の本性はけっしてその事物から除去されないが、
習慣は、除去されるのが困難であるにすぎない。

\\



{\scshape Ad secundum dicendum} quod, etsi speciebus intelligibilibus
non sit aliquid contrarium, enuntiationibus tamen et processui
rationis potest aliquid esse contrarium, ut dictum est.


&

第二異論に対しては以下のように言われるべきである。可知的形象には何も反
対のものがないにしても、命題と、理性の推論には、すでに述べられたとおり、
反対のものがある。

\\



{\scshape Ad tertium dicendum} quod scientia non removetur per motum
corporalem quantum ad ipsam radicem habitus, sed solum quantum ad
impedimentum actus; inquantum intellectus indiget in suo actu viribus
sensitivis, quibus impedimentum affertur per corporalem
transmutationem.


&

第三異論に対しては以下のように言われるべきである。学知は、習慣の根拠そ
れ自体にかんしては、身体的な運動によって除去されることはなく、ただ作用
の妨害に関してのみ除去される。すなわち、知性が自らの作用において感覚的
な力を必要とするが、身体的な変容によって、それに対する妨害がもたらされ
る。

\\


Sed per intelligibilem motum rationis potest corrumpi habitus
scientiae, etiam quantum ad ipsam radicem habitus. Et similiter etiam
potest corrumpi habitus virtutis. Tamen quod dicitur, virtutes esse
permanentiores disciplinis, intelligendum est non ex parte subiecti
vel causae, sed ex parte actus, nam virtutum usus est continuus per
totam vitam, non autem usus disciplinarum.


&

しかし、理性の可知的な運動によって、学知の習慣は、習慣の根拠それ自体に
かんしてもまた消滅する。同様に、徳の習慣も消滅しうる。しかし、「徳は学
習よりも永続する」と言われていることは、基体や原因の側からではなく、作
用の側から理解されるべきである。すなわち、徳の使用は全生涯を通して連続
するが、学習の使用はそうでない、という意味で。

\end{longtable}

\newpage



\rhead{a.~2}
\begin{center}
{\Large {\bf ARTICULUS SECUNDUS}}\\ {\large UTRUM HABITUS POSSIT
DIMINUI}\\ {\Large 第二項\\習慣は減衰するか}
\end{center}

\begin{longtable}{p{21em}p{21em}}

{\scshape Ad secundum sic proceditur}. Videtur quod habitus diminui
non possit. Habitus enim est quaedam qualitas et forma
simplex. Simplex autem aut totum habetur, aut totum amittitur. Ergo
habitus, etsi corrumpi possit, diminui non potest.

&

第二項の問題へ、議論は以下のように進められる。習慣は減衰しえないと思わ
れる。理由は以下の通り。習慣はある種の単純な性質であり形相である。しか
るに単純なものは、全体として所有されるか全体として捨てられるかのどちら
かである。ゆえに、習慣は消滅しうるが減衰はしえない。


\\



2. {\scshape Praeterea}, omne quod convenit accidenti, convenit eidem
secundum se, vel ratione sui subiecti. Habitus autem secundum seipsum
non intenditur et remittitur, alioquin sequeretur quod aliqua species
de suis individuis praedicaretur secundum magis et minus.

&

さらに、すべて附帯性に適合するものは、それ自体に即して適合するか、それ
の基体の性格によって適合するかのどちらかである。しかるに習慣は、それ自
体に即して強められたり弱められたりはしない。さもなければ、ある種が、そ
のもとに含まれる個体について、より多く、より少なくというかたちで述語付
けられたであろう。

\\


Si igitur secundum participationem subiecti diminui possit, sequeretur
quod aliquid accidat habitui proprium, quod non sit commune ei et
subiecto. Cuicumque autem formae convenit aliquid proprium praeter
suum subiectum, illa forma est separabilis, ut dicitur in I de
anima. Sequitur ergo quod habitus sit forma separabilis, quod est
impossibile.

&

ゆえに、もし基体の分有に即して減衰されえるとすれば、当の習慣と基体に共
通しない何か固有のものが、習慣に附帯することになろう。しかるに、『デ・
アニマ』第1巻で言われているように、どんな形相についても、ある固有性が
その基体を離れてその形相に適合するならば、その形相は分離可能である。ゆ
えに、習慣が分離可能な形相だということになるが、これは不可能である。

\\



3. {\scshape Praeterea}, ratio et natura habitus, sicut et cuiuslibet
accidentis, consistit in concretione ad subiectum, unde et quodlibet
accidens definitur per suum subiectum. Si igitur habitus secundum
seipsum non intenditur neque remittitur, neque etiam secundum
concretionem sui ad subiectum diminui poterit. Et ita nullo modo
diminuetur.

&

さらに、習慣の性格や本性には、どんな附帯性とも同じく、基体へ固着すると
いうことが含まれ、それゆえ、どんな附帯性もその基体によって定義される。
ゆえに、もし習慣がそれ自身に即して強められも弱められもしないのであれば、
自らの基体への固着に即してもまた弱められえないだろう。ゆえに、どんなし
かたによっても弱められない。

\\



{\scshape Sed contra est} quod contraria nata sunt fieri circa
idem. Augmentum autem et diminutio sunt contraria. Cum igitur habitus
possit augeri, videtur quod etiam possit diminui.

&

しかし反対に、相反するものは同一のものを巡って生じる本性がある。しかる
に、増大と減衰は相反する。ゆえに、習慣は増大しうるので、減衰もしうる。

\\



{\scshape Respondeo dicendum quod} habitus dupliciter diminuuntur,
sicut et augentur, ut ex supradictis patet. Et sicut ex eadem causa
augentur ex qua generantur, ita ex eadem causa diminuuntur ex qua
corrumpuntur, nam diminutio habitus est quaedam via ad corruptionem,
sicut e converso generatio habitus est quoddam fundamentum augmenti
ipsius.

&

解答する。以下のように言われるべきである。前に述べられたことから明らか
なとおり、増大と同様に、習慣は二つのしかたで減衰する。そして、それによっ
て生成するのと同じ原因によって増大し、それによって消滅するのと同じ原因
によって減衰する。すなわち、習慣の減衰は、ある種の消滅への道である。こ
れは逆に、習慣の生成がある種の習慣の増大の基礎であるのと同様である。

\\



{\scshape Ad primum ergo dicendum} quod habitus secundum se
consideratus, est forma simplex, et secundum hoc non accidit ei
diminutio, sed secundum diversum modum participandi, qui provenit ex
indeterminatione potentiae ipsius participantis, quae scilicet
diversimode potest unam formam participare, vel quae potest ad plura
vel ad pauciora extendi.

&

第一異論に対しては、それゆえ以下のように言われるべきである。それ自体に
即して考察される習慣は、単純な形相であり、その限りにおいてはそれに減衰
は生じないが、分有のさまざまなありかたに即しては(減衰がありうる)。こ
のようなあり方は、分有するもの自身の能力の非限定性に由来する。すなわち
その能力は、さまざまなしかたで一つの形相を分有しうるので、より多くのも
のへ、あるいはより少しのものへ及びうる。

\\



{\scshape Ad secundum dicendum} quod ratio illa procederet, si ipsa
essentia habitus nullo modo diminueretur. Hoc autem non ponimus, sed
quod quaedam diminutio essentiae habitus non habet principium ab
habitu, sed a participante.

&

第二異論に対しては以下のように言われるべきである。この異論は、習慣の本
質がどんなしかたによっても減衰しなかったならば、成立したであろう。しか
し、私たちはそのような措定をせず、むしろ、習慣の本性角ある種の減衰が、
習慣から根源を持つのではなく、分有するものから根源を持つことを措定する。

\\



3. {\scshape Ad tertium dicendum} quod, quocumque modo significetur
accidens, habet dependentiam ad subiectum secundum suam rationem,
aliter tamen et aliter. Nam accidens significatum in abstracto,
importat habitudinem ad subiectum quae incipit ab accidente, et
terminatur ad subiectum, nam albedo dicitur {\itshape qua aliquid est
album}.


&

第三異論に対しては以下のように言われるべきである。附帯性は、どのような
しかたで表示されても、自らの性格に即して基体への依存性をもつ。しかし、
そのしかたはさまざまである。すなわち、抽象的に表示された附帯性は、附帯
性から始まり基体へ終わるような基体への関係を意味する。たとえば、「白性」
は、「それによって何かが白いところのもの」である。


\\


Et ideo in definitione accidentis abstracti non ponitur subiectum
quasi prima pars definitionis, quae est genus; sed quasi secunda, quae
est differentia; dicimus enim quod simitas est {\itshape curvitas
nasi}.


&

それゆえ、抽象的な附帯性の定義において、基体は定義の第一の部分すなわち
類として置かれず、むしろ第二の部分すなわち種差として置かれる。例えば私
たちは、獅子鼻性とは「鼻の曲がり」であると言う。

\\

Sed in concretis incipit habitudo a subiecto, et terminatur ad
accidens, dicitur enim album {\itshape quod habet albedinem}.  Propter
quod in definitione huiusmodi accidentis ponitur subiectum tanquam
genus, quod est prima pars definitionis, dicimus enim quod simum est
{\itshape nasus curvus}. Sic igitur id quod convenit accidentibus ex
parte subiecti, non autem ex ipsa ratione accidentis, non attribuitur
accidenti in abstracto, sed in concreto.

&

しかし具体的なものにおいては、基体から関係が始まり附帯性へと終局する。
たとえば、白いものとは「白性をもつもの」である。このため、このような附
帯性の定義においては、基体があたかも類、つまり定義の第一の部分であるか
のように置かれる。例えば私たちは、獅子鼻とは「曲がった鼻」であると言う。
ゆえに、このように、附帯性に基体の側から適合し、附帯性の性格自体から適
合するのでないものは、抽象的に附帯性に帰せられず、むしろ具体的に帰せら
れる。

\\


Et huiusmodi est intensio et remissio in quibusdam accidentibus, unde
albedo non dicitur magis et minus, sed album. Et eadem ratio est in
habitibus et aliis qualitatibus, nisi quod quidam habitus augentur vel
diminuuntur per quandam additionem, ut ex supradictis patet.

&

そして、ある附帯性における強化や減衰はこのようなものである。したがって、
白性がより多く、より少なくとは言われず、そう言われるのは白いものである。
習慣やその他の性質においても同じ道理である。ただし、前に述べられたこと
から明らかなとおり\footnote{Q.52, a.2.}、ある習慣はある種の付加によっ
て増大または減衰する。




\end{longtable}
\newpage



\rhead{a.~3}
\begin{center}
{\Large {\bf ARTICULUS TERTIUS}}\\ {\large UTRUM HABITUS CORRUMPATUR
VEL DIMINUATUR\\PER SOLAM CESSATIONEM AB OPERE }\\ {\footnotesize
II$^{a}$II$^{ae}$, q.23, a.10; I {\itshape Sent.}, d.17, q.2, a.5.}\\
{\Large 第三項\\習慣は働きの休止によってのみ消滅または減衰するか}
\end{center}

\begin{longtable}{p{21em}p{21em}}


{\scshape Ad tertium sic proceditur}. Videtur quod habitus non
corrumpatur aut diminuatur per solam cessationem ab opere. Habitus
enim permanentiores sunt quam passibiles qualitates, ut ex supradictis
apparet. Sed passibiles qualitates non corrumpuntur neque diminuuntur
per cessationem ab actu, non enim albedo diminuitur si visum non
immutet, neque calor si non calefaciat. Ergo neque habitus diminuuntur
vel corrumpuntur per cessationem ab actu.

&

第三項の問題へ、議論は以下のように進められる。習慣は働きの休止によって
のみ消滅または減衰するのではないと思われる。理由は以下の通り。前に述べ
られたことから明らかなとおり、習慣は受動的性質よりも持続するものである。
しかるに受動的性質は、働きの休止によって消滅も減衰もしない。たとえば白
性は、視覚を動かさないならば減衰する、ということはないし、熱は、熱しな
いならば減衰する、ということもない。ゆえに習慣も、働きの休止によって減
衰したり消滅したりすることはない。

\\



2. {\scshape Praeterea}, corruptio et diminutio sunt quaedam
mutationes. Sed nihil mutatur absque aliqua causa movente. Cum igitur
cessatio ab actu non importet aliquam causam moventem, non videtur
quod per cessationem ab actu possit esse diminutio vel corruptio
habitus.

&

さらに、消滅と減衰は、ある種の変化である。しかるに、なにものも何らかの
運動因がなければ、変化することはない。ゆえに、働きの休止がなんらかの運
動員を含意することはないので、働きの休止によって習慣の減衰や消滅があり
うるとは思われない。

\\



3. {\scshape Praeterea}, habitus scientiae et virtutis sunt in anima
intellectiva, quae est supra tempus. Ea vero quae sunt supra tempus,
non corrumpuntur neque diminuuntur per temporis diuturnitatem. Ergo
neque huiusmodi habitus corrumpuntur vel diminuuntur per temporis
diuturnitatem, si diu aliquis absque exercitio permaneat.

&

さらに、学知や徳の習慣が知性的魂の中にあり、それは時間を越えている。他
方、時間を越えているものは、時間の持続のなかで消滅したり減衰したりしな
い。ゆえに、ある人が(そのような習慣を)遂行することなしに長く留まると
しても、そのような習慣もまた時間の持続の中で消滅したり減衰したりしない。

\\



{\scshape Sed contra est} quod philosophus, in libro {\itshape de
Longit.~et Brevit.~vitae}, dicit quod corruptio scientiae non solum
est deceptio, sed etiam oblivio. Et in VIII {\itshape Ethic}. dicitur
quod multas amicitias inappellatio dissolvit. Et eadem ratione, alii
habitus virtutum per cessationem ab actu diminuuntur vel tolluntur.

&

しかし反対に、哲学者は『寿命の長さと短さについて』という書物の中で、欺
きだけでなく忘却も学知の消滅である、と述べている。また、『ニコマコス倫
理学』第8巻では、音信不通は多くの友情を壊す、と述べられている。そして
同じ理由で、徳の他の習慣が働きの休止によって減衰し消滅する。

\\



{\scshape Respondeo dicendum} quod, sicut dicitur in VIII {\itshape
Physic}., aliquid potest esse movens dupliciter, uno modo, per se,
quod scilicet movet secundum rationem propriae formae, sicut ignis
calefacit; alio modo, per accidens, sicut id quod removet prohibens.


&

解答する。以下のように言われるべきである。『自然学』第8巻で述べられて
いるとおり、あるものは二つの仕方で動かされる。一つには自体的にであり、
ちょうど火が熱するように、固有の形相の性格にしたがって動かす場合である。
またもう一つには、附帯的にであり、たとえば障害を取り除くもの(が動かす
と言われる場合)である。


\\

Et hoc modo cessatio ab actu causat corruptionem vel diminutionem
habituum, inquantum scilicet removetur actus qui prohibebat causas
corrumpentes vel diminuentes habitum.


&


そしてこの仕方で、働きの休止は、習慣の消滅や減衰の原因となるのであって、
それは、習慣を消滅させ減衰させる諸原因を阻止する働きを取り除く限りにお
いてである。

\\


Dictum est enim quod habitus per se corrumpuntur vel diminuuntur ex
contrario agente. Unde quorumcumque habituum contraria subcrescunt per
temporis tractum, quae oportet subtrahi per actum ab habitu
procedentem; huiusmodi habitus diminuuntur, vel etiam tolluntur
totaliter, per diuturnam cessationem ab actu; ut patet et in scientia
et in virtute.


&

じっさい、習慣は、相反する作用者によって、自体的に消滅したり減衰すると
すでに述べられた。このことから、どんな習慣でも、それに相反するものが時
間を通して増長するが、それは習慣から出てくる働きによって除去されなけれ
ばならないが、そのような習慣は、働きの長い休止によって減衰し、場合によっ
て完全になくなる。これは学知や徳において明らかなとおりである。

\\


Manifestum est enim quod habitus virtutis moralis facit hominem
promptum ad eligendum medium in operationibus et passionibus.


&

じっさい、道徳的徳の習慣は、人間を、働きや感情において中庸を選択しやす
くさせる。


\\


Cum autem aliquis non utitur habitu virtutis ad moderandas passiones
vel operationes proprias, necesse est quod proveniant multae passiones
et operationes praeter modum virtutis, ex inclinatione appetitus
sensitivi, et aliorum quae exterius movent. Unde corrumpitur virtus,
vel diminuitur, per cessationem ab actu.


&

しかし、ある人が、感情や固有の働きを適切に行うための徳の習慣を用いない
ならば、感覚的欲求の傾向性や、外から動かす他のものの傾向性によって、徳
の限度を外れて多くの感情や働きが到来することが必然である。このことから、
徳は、働きの休止によって消滅したり減衰したりする。

\\

Similiter etiam est ex parte habituum intellectualium, secundum quos
est homo promptus ad recte iudicandum de imaginatis.  Cum igitur homo
cessat ab usu intellectualis habitus, insurgunt imaginationes
extraneae, et quandoque ad contrarium ducentes; ita quod, nisi per
frequentem usum intellectualis habitus, quodammodo succidantur vel
comprimantur, redditur homo minus aptus ad recte iudicandum, et
quandoque totaliter disponitur ad contrarium.


&

人間がそれによって想像されたものについて容易に正しく判断する知性的な習
慣の側からも同様である。人間が知性的な習慣の使用を止めると、外的な表象
像が浮かび上がってきて、時に反対の方向へと導く。したがって、知性的な習
慣を頻繁に用いることによって、ある意味でそれらが切り取られ制約されない
かぎり、人間は、正しく判断するのに相応しくない状態とされ、時として、全
く反対のものへと態勢付けられる。



\\


Et sic per cessationem ab actu diminuitur, vel etiam corrumpitur
intellectualis habitus.

&

このようにして、働きの休止によって、知性的な習慣は減衰し、また消滅もす
る。


\\





{\scshape Ad primum ergo dicendum} quod ita etiam calor per
cessationem a calefaciendo corrumperetur, si per hoc incresceret
frigidum, quod est calidi corruptivum.

&


第一異論に対しては、それゆえ、以下のように言われるべきである。もし、そ
のことによって、火を消滅させる冷が増大したならば、熱もまた、熱すること
を止めることで消滅したであろう。

\\



{\scshape Ad secundum dicendum} quod cessatio ab actu est movens ad
corruptionem vel diminutionem, sicut removens prohibens, ut dictum
est.

&

第二異論に対しては以下のように言われるべきである。働きの休止は、すでに
述べられたとおり、障害を取り除くものとして、消滅や減衰へと動かす。

\\



{\scshape Ad tertium dicendum} quod pars intellectiva animae secundum
se est supra tempus, sed pars sensitiva subiacet tempori. Et ideo per
temporis cursum, transmutatur quantum ad passiones appetitivae partis
et etiam quantum ad vires apprehensivas. Unde philosophus dicit, in IV
{\itshape Physic}., quod tempus est causa oblivionis.

&

第三異論に対しては以下のように言われるべきである。知性的魂の部分は、そ
れ自体に即しては時間を越えているが、感覚的部分は時間のもとにある。ゆえ
に、時間の流れにしたがって、欲求的部分の情念に関して、さらに把握能力に
関して、変化する。このことから哲学者は『自然学』第4巻で、時間が忘却の
原因であると述べている。

\end{longtable}
\end{document}