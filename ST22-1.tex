\documentclass[10pt]{jsarticle} % use larger type; default would be 10pt
%\usepackage[utf8]{inputenc} % set input encoding (not needed with XeLaTeX)
%\usepackage[round,comma,authoryear]{natbib}
%\usepackage{nruby}
\usepackage{okumacro}
\usepackage{longtable}
%\usepqckage{tablefootnote}
\usepackage[polutonikogreek,english,japanese]{babel}
%\usepackage{amsmath}
\usepackage{latexsym}
\usepackage{color}
\usepackage{otf}

%----- header -------
\usepackage{fancyhdr}
\lhead{{\it Summa Theologiae} II-IIae, q.~1}
%--------------------

\bibliographystyle{jplain}

\title{{\bf SECUNDA SECUNDAE}\\{\HUGE Summae Theologiae}\\Sancti Thomae
Aquinatis\\{\sffamily QUAESTIO PRIMA}\\DE OBIECTO FIDEI}
\author{Japanese translation\\by Yoshinori {\sc Ueeda}}
\date{Last modified \today}


%%%% コピペ用
%\rhead{a.~}
%\begin{center}
% {\Large {\bf }}\\
% {\large }\\
% {\footnotesize }\\
% {\Large \\}
%\end{center}
%
%\begin{longtable}{p{21em}p{21em}}
%
%&
%
%
%\\
%\end{longtable}
%\newpage



\begin{document}
\maketitle
\pagestyle{fancy}

\begin{center}
{\Large 第一問\\信仰の対象について}
\end{center}

\begin{longtable}{p{21em}p{21em}}
{\huge C}{\scshape irca} virtutes igitur theologicas primo erit
 considerandum de fide; secundo, de spe; tertio, de caritate. Circa
 fidem vero quadruplex consideratio occurrit, prima quidem de ipsa fide;
 secunda de donis intellectus et scientiae sibi correspondentibus;
 tertia de vitiis oppositis; quarta de praeceptis ad hanc virtutem
 pertinentibus. Circa fidem vero primo erit considerandum de eius
 obiecto; secundo, de eius actu; tertio, de ipso habitu fidei. Circa
 primum quaeruntur decem. 

&

さて、神学的徳をめぐって、第一に、信仰について、第二に、希望について、第
 三に、愛徳について考察されるべきであろう。
信仰については、考察が四通りになされる。第一に、信仰それ自体について、第
 二に、知性と学知のそれぞれに対応する賜物について、第三に、対立する悪徳
 について、そして第四に、この徳に属する教訓についてである。
また、信仰について、第一に、その対象について、第二に、その作用について、
 第三に、信仰という習態それ自身について考察されるべきであろう。
第一について、十のことが問われる。
\\


\begin{enumerate}
 \item utrum obiectum fidei sit veritas prima.
 \item utrum obiectum fidei sit aliquid complexum vel
 incomplexum, idest res aut enuntiabile.
 \item utrum fidei possit subesse falsum.
 \item utrum obiectum fidei possit esse aliquid
 visum.
 \item utrum possit esse aliquid scitum.
 \item utrum credibilia debeant distingui per certos articulos.
 \item utrum iidem articuli subsint fidei secundum omne tempus.
 \item de numero articulorum. 
 \item de modo tradendi articulos in symbolo. 
 \item cuius sit fidei symbolum constituere.
\end{enumerate}



&

\begin{enumerate}
 \item 信仰の対象は第一真理か。
 \item 信仰の対象は、複合的なものか複合的でないものか、つまり、事物か、それ
       とも命題的なものか。
 \item 信仰に偽であるものが入りうるか。
 \item 信仰の対象が何か見られたものでありうるか。
 \item [信仰の対象が]知られたものでありうるか。
 \item 信じられるべき事柄が、一定の項目によって区別されるべきか。
 \item あらゆる時にわたって、同一の項目が信仰に属するか。
 \item その項目の数について。
 \item 項目を象徴のなかに伝える方法について。
 \item 信仰の象徴をつくるのは、だれに属するか。
\end{enumerate}

\end{longtable}

\newpage


\rhead{a.~1}
\begin{center}
 {\Large {\bf ARTICULUS PRIMUS}}\\
 {\large UTRUM OBIECTUM FIDEI SIT VERITAS PRIMA}\\
 {\footnotesize III {\itshape Sent.}, d.~24, a.~1, qu$^a$.~1; {\itshape
 De Vesit.}, q.~14, a.~8; {\itshape De Virtut.}, q.~4, a.~1.}\\
 {\Large 第一項\\信仰の対象は第一真理か}
\end{center}


\begin{longtable}{p{21em}p{21em}}

{\huge A}{\scshape d primum sic proceditur}. Videtur quod
obiectum fidei non sit veritas prima. Illud enim videtur esse obiectum
fidei quod nobis proponitur ad credendum. Sed non solum proponuntur
nobis ad credendum ea quae pertinent ad divinitatem, quae est veritas
prima; sed etiam ea quae pertinent ad humanitatem Christi et Ecclesiae
sacramenta et creaturarum conditionem. Ergo non solum veritas prima est
fidei obiectum.


&

第一項の問題へ、議論は以下のように進められる。
信仰の対象は第一真理でないと思われる。理由は以下の通り。
信仰の対象とは、信じるように私たちに提示されるものであると思われる。
ところが、神性つまり第一真理に属する事柄が、信じるように私たちに提示さ
 れるだけでなく、キリストの人間性、教会の秘蹟、被造物の状態に属する事柄
 もまた、提示される。ゆえに、第一真理だけが信仰の対象ではない。


\\


{\scshape 2 Praeterea}, fides et infidelitas sunt
circa idem, cum sint opposita. Sed circa omnia quae in sacra Scriptura
continentur potest esse infidelitas, quidquid enim horum homo negaverit,
infidelis reputatur. Ergo etiam fides est circa omnia quae in sacra
Scriptura continentur. Sed ibi multa continentur de hominibus et de
aliis rebus creatis. Ergo obiectum fidei non solum est veritas prima,
sed etiam veritas creata.


&

さらに、信仰と不信仰は同一のものにかかわる。それらは対立するからである。
ところが、聖書に含まれるすべての事柄をめぐって不信仰がありうる。なぜなら、
人がそれらのうちのどれを否定しても、その人は不信仰とみなされるからである。
ゆえに、信仰もまた、聖書に含まれるすべてのことにかかわる。しかし、そこに
は、人間や他の被造物について、多くのことが含まれている。ゆえに、信仰の対
象は、第一真理だけでなく、被造の真理でもある。

\\


{\scshape 3 Praeterea}, fides caritati
condividitur, ut supra dictum est. Sed caritate non solum diligimus
Deum, qui est summa bonitas, sed etiam diligimus proximum. Ergo fidei
obiectum non est solum veritas prima.


&

さらに、信仰は、以前に述べられたとおり、愛徳と同じレベルで分けられる。と
ころが、愛徳によって、私たちは最高善である神を愛するだけでなく、隣人も愛
する。ゆえに、信仰の対象は、第一真理だけではない。
\\


{\scshape Sed contra est} quod Dionysius dicit, {\scshape vii} cap.~{\itshape de
Div.~Nom}., quod {\itshape fides est circa simplicem et semper
existentem veritatem}. Haec autem est veritas prima. Ergo obiectum fidei
est veritas prima.


&

しかし反対に、ディオニュシウスは『神名論』七章で、「信仰は、単純で常に真
理であるものにかかわる」と述べている。ところが、これは第一真理である。ゆ
えに、信仰の対象は第一真理である。

\\

{\scshape Respondeo dicendum} quod cuiuslibet
cognoscitivi habitus obiectum duo habet, scilicet id quod materialiter
cognoscitur, quod est sicut materiale obiectum; et id per quod
cognoscitur, quod est formalis ratio obiecti. 


&

解答する。以下のように言われるべきである。
どのような認識的習態の対象も、二つのものをもつ。すなわち、質料的に認識さ
 れるもの、つまり、いわば質料的対象と、それをとおして認識されるところの
 もの、つまり、対象の形相的な根拠(ratio)とである。

\\

Sicut in scientia
geometriae materialiter scita sunt conclusiones; formalis vero ratio
sciendi sunt media demonstrationis, per quae conclusiones
cognoscuntur. 

&

たとえば、幾何学で、質料的に知られるのは結論だが、それを知ることの形相的
な根拠は論証の中項である。なぜなら、結論は中項を通して認識されるのだか
ら。

\\

Sic igitur in fide, si consideremus formalem rationem
obiecti, nihil est aliud quam veritas prima, non enim fides de qua
loquimur assentit alicui nisi quia est a Deo revelatum; unde ipsi
veritati divinae innititur tanquam medio. 


&

それゆえ、信仰においても同様に、私たちが、もし対象の形相的な根拠を考察す
るのであれば、それは第一真理以外にはない。なぜなら、私たちが語っている信
仰は、それが神によって啓示されたという理由以外では、何事にも同意しないか
らである。したがって、[私たちの信仰は]中項としての、神の真理自体に依拠
する。

\\


Si vero consideremus
materialiter ea quibus fides assentit, non solum est ipse Deus, sed
etiam multa alia. Quae tamen sub assensu fidei non cadunt nisi secundum
quod habent aliquem ordinem ad Deum, prout scilicet per aliquos
divinitatis effectus homo adiuvatur ad tendendum in divinam
fruitionem. Et ideo etiam ex hac parte obiectum fidei est quodammodo
veritas prima, inquantum nihil cadit sub fide nisi in ordine ad Deum,
sicut etiam obiectum medicinae est sanitas, quia nihil medicina
considerat nisi in ordine ad sanitatem.

&

他方、信仰が同意するものを、質料的に考察するならば、神自体だけでなく、他
 の多くのものも[対象である]。しかし、これらのものは、神性のなんらかの
 結果を通して、人間が神を享受することへと向かうのを助けられるかぎりで、
 神へのなんらかの秩序をもつという点でのみ、信仰の同意のもとに入る。
ゆえに、この側面からでも、信仰の対象は、なんらかの意味で、第一真理である。
 つまり、神への秩序においてでなければ、なにものも信仰のもとに入らないと
 いう意味で。ちょうどそれは、医学の対象は健康だが、それは、医学が、健康
 への秩序のにおいてでなければ何ものも考察しないのと同じである。


\\


{\scshape Ad primum ergo dicendum} quod ea quae
pertinent ad humanitatem Christi et ad sacramenta Ecclesiae vel ad
quascumque creaturas cadunt sub fide inquantum per haec ordinamur ad
Deum. Et eis etiam assentimus propter divinam veritatem.


&

第一異論に対しては、それゆえ、以下のように言われるべきである。キリストの
人間性と教会の秘蹟に属するもの、あるいは、なんであれ被造物に属するものど
もは、それらをとおして私たちが神へと秩序づけられるかぎりで、信仰のもとに
入る。そして、私たちは、神の真理のために、それらにも同意する。

\\


Et similiter dicendum est ad secundum,
de omnibus illis quae in sacra Scriptura traduntur.


&
第二異論に対しても同様に、聖書の中に伝えられるすべてのものについて言われ
 るべきである。



\\


Ad tertium dicendum quod etiam caritas
diligit proximum propter Deum; et sic obiectum eius proprie est ipse
Deus, ut infra dicetur.


&

第三異論に対しては、以下のように言われるべきである。
愛徳もまた、神のために隣人を愛する。その意味で、後に述べられるように、そ
 の対象は、固有には、神自体である。


\end{longtable}

\newpage

\rhead{a.~2}
\begin{center}
 {\Large {\bf ARTICULUS SECUNDUS}}\\
 {\large UTRUM OBIECTUM FIDEI SIT ALIQUID COMPLEXUM PER MODUM ENUNTIABILIS}\\
 {\footnotesize I {\itshape Sent.}, d.~16, Expos.~litt.; III, d.~24,
 a.~1, qu$^a$ 2; {\itshape De Verit.}, q.~14, a.~8, ad 5, 12; a.~12.}\\
 {\Large 第二項\\信仰の対象は命題的なもののように複合されたものか}
\end{center}

\begin{longtable}{p{21em}p{21em}}
{\huge A}{\scshape d secundum sic proceditur}. Videtur
quod obiectum fidei non sit aliquid complexum per modum
enuntiabilis. Obiectum enim fidei est veritas prima, sicut dictum
est. Sed prima veritas est aliquid incomplexum. Ergo obiectum fidei non
est aliquid complexum.

&

第二の問題へ、議論は以下のように進められる。
信仰の対象は、命題的なもののように複合されたものではないと思われる。
理由は以下の通り。
今言われたとおり、信仰の対象は第一真理である。ところが、第一真理は、複合
 的でないものである。ゆえに、信仰の対象は複合的なものでない。


\\


{\scshape 2 Praeterea}, expositio fidei in symbolo
continetur. Sed in symbolo non ponuntur enuntiabilia, sed res, non enim
dicitur ibi quod Deus sit omnipotens, sed, {\itshape credo in Deum
omnipotentem}. Ergo obiectum fidei non est enuntiabile, sed res.

&

さらに、信仰の説明は信経の中に含まれる。ところが、信経の中に置かれるのは
 命題的なものではなく事物である。たとえば、そこでは、「神は全能者である」
 とは言われず、「私は全能である神を信じる」と言われる。ゆえに、信仰の対
 象は命題的なものではなく、事物である。


\\


{\scshape 3 Praeterea}, fidei succedit visio, secundum illud I {\itshape
ad Cor}.~{\scshape xiii}, {\itshape videmus nunc per speculum in
aenigmate, tunc autem facie ad faciem}. Sed visio patriae est de
incomplexo, cum sit ipsius divinae essentiae. Ergo etiam fides viae.

&

さらに、かの『コリント人への第一の手紙』「私たちは、今、鏡をとおして謎の
 中に見るが、そのときには顔と顔を合わせて見るだろう」によれば、信仰には
 直視が続く。ところが、天国での直視は複合的でないものにかかわる。なぜな
 ら、それは神自身の本質の直視だからである。ゆえに、道中の信仰もまた[複
 合的でないものにかかわる]。

\\


{\scshape Sed contra}, fides est media inter
scientiam et opinionem. Medium autem et extrema sunt eiusdem
generis. Cum igitur scientia et opinio sint circa enuntiabilia, videtur
quod similiter fides sit circa enuntiabilia. Et ita obiectum fidei, cum
fides sit circa enuntiabilia, est aliquid complexum.

&


しかし反対に、信仰は、学知と意見の中間にある。
ところが、中間のものは、両端のものと同じ類に属する。
ゆえに、学知と意見は命題的なものにかかわるから、信仰も同様に命題的なもの
 にかかわると思われる。したがって、信仰が命題的なことにかかわるのだから、
 信仰の対象は、複合的なものである。

\\


{\scshape Respondeo dicendum} quod cognita sunt in
cognoscente secundum modum cognoscentis. Est autem modus proprius humani
intellectus ut componendo et dividendo veritatem cognoscat, sicut in
primo dictum est.

&

解答する。以下のように言われるべきである。
認識されたものは、認識するものの中に、認識するもののあり方にしたがって存
 在する。ところで、第一部で述べられたとおり、複合分割することで真理を認
 識することが、人間知性に固有のありかたである。

\\


 Et ideo ea quae sunt secundum se simplicia intellectus
humanus cognoscit secundum quandam complexionem, sicut e converso
intellectus divinus incomplexe cognoscit ea quae sunt secundum se
complexa.


&

ゆえに、それ自体において単純なものを、人間知性は、なんらかの複合にしたがっ
 て認識する。ちょうどそれは、この逆に、神の知性が、それ自体において複合
 的なものを、複合的でないかたちで認識するのと同様である。

\\

 Sic igitur obiectum fidei dupliciter considerari potest. Uno
modo, ex parte ipsius rei creditae, et sic obiectum fidei est aliquid
incomplexum, scilicet res ipsa de qua fides habetur.

&

それゆえ以上のような意味で、信仰の対象は二通りに考察されうる。一つには、
 信じられる事物自体の側からであり、この場合、信仰の対象は、複合的でない
 何か、つまり、それについて信仰が持たれるところの事物そのものである。

\\

 Alio modo, ex parte
credentis, et secundum hoc obiectum fidei est aliquid complexum per
modum enuntiabilis. Et ideo utrumque vere opinatum fuit apud antiquos,
et secundum aliquid utrumque est verum.

&

もう一つには、信じるものの側からであり、このかぎりでは、信仰の対象は、命
 題的なもののように複合的なものである。ゆえに、昔の人々のもとで、どち
 らも正しく考えられたのであり、どちらもなんらかの点で真である。

\\


{\scshape Ad primum ergo dicendum} quod ratio illa
procedit de obiecto fidei ex parte ipsius rei creditae.

&

第一異論に対しては、それゆえ、以下のように言われるべきである。
この論は、信じられた事物自体の側から、信仰の対象について論じ進められてい
 る。

\\


{\scshape Ad secundum dicendum} quod in symbolo
tanguntur ea de quibus est fides inquantum ad ea terminatur actus
credentis, ut ex ipso modo loquendi apparet. Actus autem credentis non
terminatur ad enuntiabile, sed ad rem, non enim formamus enuntiabilia
nisi ut per ea de rebus cognitionem habeamus, sicut in scientia, ita et
in fide.

&

第二異論に対しては、以下のように言われるべきである。
語り方自体からあきらかなとおり、信経においては、信仰がかかわるものが、信
 仰する者の行為がそれに到達するかぎりで、触れられている。ところが、信仰
 する者の行為は、命題的なものに達するのではなく、事物に達する。なぜなら、
 学知においても信仰においても、私たちは、それによって事物についての認識
 を持つために、命題的なものを作るからである。


\\


{\scshape Ad tertium dicendum} quod visio patriae erit veritatis primae
secundum quod in se est, secundum illud I {\itshape Ioan}.~{\scshape
iii}, {\itshape cum apparuerit, similes ei erimus et videbimus eum
sicuti est}. Et ideo visio illa erit non per modum enuntiabilis, sed per
modum simplicis intelligentiae. Sed per fidem non apprehendimus
veritatem primam sicut in se est. Unde non est similis ratio.

&

第三異論に対しては、以下のように言われるべきである。かの『ヨハネの手紙一』
三章、「彼が現れたとき、私たちは彼に似たものとなり、あるがままに彼を見る」
\footnote{「御子が現れるとき、御子に似たものとなるということを知っていま
す。なぜなら、そのとき御子をありのままに見るからです」(3:2)}によれば、天
国での直視は、それ自体において在るかぎりでの第一真理にかかわる。ゆえに、
かの直視は、命題的なものの在り方によらず、単純な知性認識のありかたによる。
しかし、私たちは信仰によって、それ自体においてあるがままに第一真理を捉え
るのではない。ゆえに、同じことは言えない。




\end{longtable}
\newpage

\rhead{a.~3}
\begin{center}
 {\Large {\bf ARTICULUS TERTIUS}}\\
 {\large UTRUM FIDEI POSSIT SUBESSE FALSUM}\\
 {\footnotesize III {\itshape Sent.}, d.~24, a.~1, qu$^a$ 3.}\\
 {\Large 第三項\\信仰の中に偽が入りうるか}
\end{center}

\begin{longtable}{p{21em}p{21em}}

{\huge A}{\scshape d tertium sic proceditur}. Videtur
quod fidei possit subesse falsum. Fides enim condividitur spei et
caritati. Sed spei potest aliquid subesse falsum, multi enim sperant se
habituros vitam aeternam qui non habebunt. Similiter etiam et caritati,
multi enim diliguntur tanquam boni qui tamen boni non sunt. Ergo etiam
fidei potest aliquid subesse falsum.


&

第三項の問題へ、議論は以下のように進められる。
信仰の中に偽が入りうると思われる。理由は以下の通り。
信仰は、希望、愛徳と同じレベルで分割される。
ところが、希望の中には、何らかの偽が入りうる。たとえば、永遠の生命をもつ
 ことにならない多くの人々が、自分は永遠の生命をもつであろうことを希望し
 ている。同様に、愛徳の中にも偽が入りうるのであって、たとえば、善人でな
 い多くの人々が、善人として愛されている。ゆえに、信仰の中にも、何らかの
 偽が入りうる。

\\


{\scshape 2 Praeterea}, Abraham credidit Christum nasciturum, secundum
illud Ioan.~{\scshape viii}, {\itshape Abraham, pater vester, exultavit
ut videret diem meum}. Sed post tempus Abrahae Deus poterat non
incarnari, sola enim sua voluntate carnem accepit, et ita esset falsum
quod Abraham de Christo credidit. Ergo fidei potest subesse falsum.


&

さらに、かの『ヨハネによる福音書』8章の「あなたたちの父であるアブラハム
 は、私の日を見て喜んだ」\footnote{「あなたたちの父アブラハムは、わたし
 の日を見るのを楽しみにしていた。そして、それを見て、喜んだのである」
 (8:56)}
によれば、アブラハムはキリストが将来生まれることを信じていた。
ところが、アブラハムの時代のあと、神が受肉しないことは可能であった。な
 ぜなら、神は、自らの意志のみによって、肉を取ったからである。したがって、
 その場合、アブラハムがキリストについて信じていたことは偽であっただ
 ろう。ゆえに、信仰の中に偽が入ることは可能である。

\\


{\scshape 3 Praeterea}, fides antiquorum fuit quod
Christus esset nasciturus, et haec fides duravit in multis usque ad
praedicationem Evangelii. Sed Christo iam nato, antequam praedicare
inciperet, falsum erat Christum nasciturum. Ergo fidei potest subesse
falsum.


&

さらに、キリストが生まれるだろうという古代の人々の信仰が存在し、そして、
 この信仰は、福音が述べ伝えられるまで、多くの人々の中で続いていた。とこ
 ろが、すでにキリストが生まれていて、しかも、福音が述べ伝えられ始める前、
[その間、]「キリストが生まれるだろう」ということは偽であった。ゆえに、信仰
 の中に偽が入りうる。

\\


{\scshape 4 Praeterea}, unum de pertinentibus ad
fidem est ut aliquis credat sub sacramento altaris verum corpus Christi
contineri. Potest autem contingere, quando non recte consecratur, quod
non est ibi verum corpus Christi, sed solum panis. Ergo fidei potest
subesse falsum.


&

さらに、信仰に属する事柄のひとつに、「人は、聖餐の中にキリストの真の身体が含
 まれている」と信じることがある。ところが、正しく聖別されない場合には、
 そこにキリストの真の身体はなく、ただパンがある、ということが起こりうる。
 ゆえに、信仰の中に、偽が入りうる。

\\


{\scshape  Sed contra}, nulla virtus perficiens
intellectum se habet ad falsum secundum quod est malum intellectus, ut
patet per philosophum, in VI {\itshape Ethic}. Sed fides est quaedam virtus
perficiens intellectum, ut infra patebit. Ergo ei non potest subesse
falsum.


&

しかし反対に、哲学者によって『ニコマコス倫理学』の中で明らかとなっている
 とおり、知性を完成させるどんな徳も、知性における悪である偽に関係しない。
 ところが、信仰は、あとで明らかになるとおり、知性を完成させる一種の徳で
 ある。ゆえに、それに偽が入ることはありえない。

\\


{\scshape Respondeo dicendum} quod nihil subest
alicui potentiae vel habitui aut etiam actui, nisi mediante ratione
formali obiecti, sicut color videri non potest nisi per lucem, et
conclusio sciri non potest nisi per medium demonstrationis. Dictum est
autem quod ratio formalis obiecti fidei est veritas prima. Unde nihil
potest cadere sub fide nisi inquantum stat sub veritate prima. Sub qua
nullum falsum stare potest, sicut nec non ens sub ente, nec malum sub
bonitate. Unde relinquitur quod fidei non potest subesse aliquod falsum.

& 


解答する。以下のように言われるべきである。光なしに色が見られえず、論証
の中項なしに結論が知られえないように、どんな能力ないし習態の中にも、ある
いは働きであっても、その中に、形相的対象の性格が媒介することなく、何かが
入ることはない。ところで、信仰の形相的対象の根拠(ratio)が第一真理であるこ
とは、すでに述べられた。したがって、第一真理の中にないかぎり、何ものも信
仰の中に入りえない。そしてちょうど、非有が有の中に入らず、悪が善の中に入
らないように、その中にはどんな偽も入りえない。したがって、信仰にはどんな
偽も入りえないことが帰結する。

\\


{\scshape Ad primum ergo dicendum} quod, quia verum
est bonum intellectus, non autem est bonum appetitivae virtutis, ideo
omnes virtutes quae perficiunt intellectum excludunt totaliter falsum,
quia de ratione virtutis est quod se habeat solum ad bonum. Virtutes
autem perficientes partem appetitivam non excludunt totaliter falsum,
potest enim aliquis secundum iustitiam aut temperantiam agere aliquam
falsam opinionem habens de eo circa quod agit. 
Et ita, cum fides
perficiat intellectum, spes autem et caritas appetitivam partem, non est
similis ratio de eis.

&

第一異論に対しては、それゆえ、以下のように言われるべきである。
真は知性の善であるが、欲求的徳の善ではない。ゆえに、知性を完成させるすべ
 ての徳は、全面的に偽を排除する。なぜなら、徳の性格には、ただ
 善のみに関係することが属するからである。
ところが、欲求的部分を完成させる徳は、必ずしも全面的に偽を排除するわけで
 はない。たとえば、自分が行っていることについて誤った意見をもっている人
 が、正義と節制にしたがって行為する、ということが可能である。
したがって、信仰は知性を完成させるが、希望と愛徳は欲求的部分を完成させる
 ので、これらを同列には扱えない。



\\

 Et tamen neque etiam spei subest falsum. Non enim
aliquis sperat se habiturum vitam aeternam secundum propriam potestatem
(hoc enim esset praesumptionis), sed secundum auxilium gratiae, in qua
si perseveraverit, omnino infallibiliter vitam aeternam
consequetur. Similiter etiam ad caritatem pertinet diligere Deum in
quocumque fuerit. Unde non refert ad caritatem utrum in isto sit Deus
qui propter Deum diligitur.

&

とは言っても、希望の中にも偽が入るわけではない。
というのも、人は、自分が永遠の生命をもつだろうということを、自分固有の力
 によって希望するのではなく(もしそんなことがあったらそれは傲慢である)、
 恩恵の助けによって希望するのだから、その恩恵のなかにしっかりと留まった
 ならば、まったく誤りなく、永遠の生命を獲得するだろうからである。
同様に、愛徳にも、神がどこに存在するとしても、神を愛することが属している。し
 たがって、神のために愛される人の中に神がいるかどうかというは、愛徳には関係し
 ない。


\\

{\scshape Ad secundum dicendum} quod Deum non
incarnari, secundum se consideratum, fuit possibile etiam post tempus
Abrahae. Sed secundum quod cadit sub praescientia divina, habet quandam
necessitatem infallibilitatis, ut in primo dictum est. Et hoc modo cadit
sub fide. Unde prout cadit sub fide, non potest esse falsum.

&


第二異論に対しては、以下のように言われるべきである。それ自体で考察される
ならば、「神が受肉しない」ということは、アブラハムの時代のあとであっても
可能であった。しかし、第一部\footnote{14問13項「神は未来の非必然的な事柄
を知るか」}で述べられたとおり、神の予知の中に含まれるかぎりでは、[「神が
受肉する」ということは]一種の不可謬の必然性をもっている。したがって、信
仰の中に入るものとしては、偽でありえない。

\\


{\scshape Ad tertium dicendum} quod hoc ad fidem
credentis pertinebat post Christi nativitatem quod crederet eum
quandoque nasci. Sed illa determinatio temporis, in qua decipiebatur,
non erat ex fide, sed ex coniectura humana. Possibile est enim hominem
fidelem ex coniectura humana falsum aliquid aestimare. Sed quod ex fide
falsum aestimet, hoc est impossibile.

& 

第三異論に対しては、以下のように言われるべきである。キリストの生誕以後に、
信者の信仰に属していたのは、「彼が何らかの時点で生まれる」と信じることで
あった。しかし、この時間の限定、ここで彼[=この信者]は誤ったのだが、は、
信仰ではなく、人間の推測によるものであった。じっさい、信心深い人が、
人間の推測にもとづいて、何かを誤って判断することはありうる。しかし、[そ
の人が]信仰にもとづいて誤って何かを判断すること、これはありえない。

\\


{\scshape Ad quartum dicendum} quod fides credentis
non refertur ad has species panis vel illas, sed ad hoc quod verum
corpus Christi sit sub speciebus panis sensibilis quando recte fuerit
consecratum. Unde si non sit recte consecratum, fidei non suberit
propter hoc falsum.

&

第四異論に対しては、以下のように言われるべきである。
信者の信仰は、パンのこの形象やあの形象に関係するのではなく、「正しく聖別さ
 れたときには、キリストの真の身体が、可感的なパンの形象の中にある」という
 ことに関係する。したがって、正しく聖別されていなくても、このことのため
 に、信仰の中に偽が入るわけではない。



\end{longtable}
\newpage


\rhead{a.~4}
\begin{center}
 {\Large {\bf ARTICULUS QUARTUS}}\\
 {\large UTRUM OBIECTUM FIDEI POSSIT ESSE ALIQUID VISUM}\\
 {\footnotesize III {\itshape Sent.}, d.~24, a.~2, qu$^a$ 1; {\itshape
 De Verit.}, q.~19, a.~9; {\itshape Ad Hebr.}, c.~11, lect.~1.}\\
 {\Large 第四項\\信仰の対象は、何か見られたものでありうるか}
\end{center}

\begin{longtable}{p{21em}p{21em}}


{\huge A}{\scshape d quartum sic proceditur}. Videtur quod obiectum
fidei sit aliquid visum. Dicit enim dominus Thomae, Ioan.~{\scshape xx},
{\itshape quia vidisti me, credidisti}. Ergo et de eodem est visio et
fides.

&

第四項の問題へ、議論は以下のように進められる。
信仰の対象は、何か見られたものであるように思われる。
理由は以下の通り。
『ヨハネによる福音書』20章で、主はトマスに「あなたはわたしを見たから信じ
 た」\footnote{「イエスはトマスに言われた。「わたしを見たから信じたの
 か。見ないのに信じる人は幸いである。」(20:29)}と述べている。ゆえに、同一のものについて、見ることと信じることとが
 ある。

\\


{\scshape 2 Praeterea}, apostolus, I {\itshape ad Cor}.~{\scshape xiii},
dicit, {\itshape videmus nunc per speculum in aenigmate}. Et loquitur de
cognitione fidei. Ergo id quod creditur videtur.

&

さらに、『コリントの信徒への手紙一』13章で、使徒は「私たちは、今は、鏡をとおして謎の
 中に見る」\footnote{「わたしたちは、今は、鏡におぼろに映ったものを見て
 いる。」(13:12)}と述べている。そして、彼は信仰の認識について語っている。ゆえ
 に、信じられていることは、見られる。

\\


{\scshape 3 Praeterea}, fides est quoddam spirituale lumen. Sed quolibet
lumine aliquid videtur. Ergo fides est de rebus visis.

&

さらに、信仰は、ある種の霊的な光である。ところが、どんな光によっても、何
 かが見られる。ゆえに、信仰は、見られる諸事物にかかわる。

\\


{\scshape 4 Praeterea}, quilibet sensus visus nominatur, ut Augustinus
dicit, in libro de Verb. Dom. Sed fides est de auditis, secundum illud
{\itshape ad Rom}.~{\scshape x}, {\itshape fides ex auditu}. Ergo fides
est de rebus visis.

&

さらに、アウグスティヌスが、『主の言葉について』という書物で述べているよ
 うに、どんな感覚も視覚と名付けられる。ところが、『ローマの信徒への手紙』
 10章「聞かれたものから信仰が」\footnote{「実に、信仰は聞くことにより、し
 かも、キリストの言葉を聞くことによって始まるのです」(10:17)}によれば、信
 仰は聞かれたものにかかわる。ゆえに、信仰は、見られる諸事物にかかわる。

\\


{\scshape Sed contra est} quod apostolus dicit, {\itshape ad
 Heb}.~{\scshape xi}, quod {\itshape fides est argumentum non apparentium}.

&

しかし反対に、『ヘブライ人への手紙』11章で、使徒は、「信仰は、現れていな
 いものどもの議論である」\footnote{「信仰とは、望んでいる事柄を確信し、
 見えない事実を確認することです」(11:1)}と述べている。

\\


{\scshape Respondeo dicendum} quod fides importat assensum intellectus ad
id quod creditur. Assentit autem alicui intellectus dupliciter. Uno
modo, quia ad hoc movetur ab ipso obiecto, quod est vel per seipsum
cognitum, sicut patet in principiis primis, quorum est intellectus; vel
est per aliud cognitum, sicut patet de conclusionibus, quarum est
scientia. 

&

解答する。以下のように言われるべきである。
信仰は、信じられる事柄への知性の同意を含意する。
ところで、知性は、二通りのしかたで何かに同意する。
一つには、対象それ自体によってそれへ動かされるからである。その対象は、
 それ自体によって認識される場合と、他のものによって認識される場合とがあ
 る。前者はたとえば第一の諸原理において明らかなとおりであり、それらにつ
 いては「知性」がある。後者は、たとえば、結論にかんして明らかなとおりであり、
 それらについては「学・学知(scientia)」がある。


\\

Alio modo intellectus assentit alicui non quia sufficienter
moveatur ab obiecto proprio, sed per quandam electionem voluntarie
declinans in unam partem magis quam in aliam. Et si quidem hoc fit cum
dubitatione et formidine alterius partis, erit opinio, si autem fit cum
certitudine absque tali formidine, erit fides. Illa autem videri
dicuntur quae per seipsa movent intellectum nostrum vel sensum ad sui
cognitionem. Unde manifestum est quod nec fides nec opinio potest esse
de visis aut secundum sensum aut secundum intellectum.

&

もう一つのしかたでは、知性は、固有の対象によって十分に動かされるからでは
 なく、一方よりも他方へと傾く意志的なある種の選択によって、何かに同意す
 る。そして、これが他方に対する疑いや恐れを伴ってなされる場合には、「意
 見・臆測(opinio)」が、他方に対するそのような恐れがなく、確かさを伴ってなされる場合に
 は、「信仰」がある。
ところで、それ自身によって私たちの知性や感覚を、自らにかんする認識へと動
 かすものが、「見られる」と言われる。したがって、感覚においても知性にお
 いても、見られるものについては、信仰も意見もありえないことが明らかであ
 る。


\\


{\scshape Ad primum ergo dicendum} quod Thomas {\itshape aliud vidit et aliud
credidit. Hominem vidit et Deum credens confessus est, cum dixit,
dominus meus et Deus meus}.

&

第一異論に対しては、それゆえ、以下のように言われるべきである。
トマスが見たものと、信じたものとは異なる。彼が「わたしの主、わたしの神」
 と言ったとき、人間を見て、神を信じることを告白した。

\\


{\scshape Ad secundum dicendum} quod ea quae subsunt fidei dupliciter
considerari possunt. Uno modo, in speciali, et sic non possunt esse
simul visa et credita, sicut dictum est. Alio modo, in generali,
scilicet sub communi ratione credibilis. Et sic sunt visa ab eo qui
credit, non enim crederet nisi videret ea esse credenda, vel propter
evidentiam signorum vel propter aliquid huiusmodi.

&

第二異論に対しては、次のように言われるべきである。
信仰の中に入るものとは、二通りに考えられうる。
一つには、特殊的にであり、この場合には、すでに述べられたとおり、同時に、
 見られかつ信じられることはありえない。もう一つには、一般的に、すなわち、信
 じられうるもの、という共通の性格のもとに考えられる。この場合には、それ
 らは、信じる人によって見られたものである。なぜなら、しるしの明証や、何かそのようなもののために、もし、それらが信じられるべきものだということが見られな
 かったならば、信じなかったであろうからである。


\\


{\scshape Ad tertium dicendum} quod lumen fidei facit videre ea quae
creduntur. Sicut enim per alios habitus virtutum homo videt illud quod
est sibi conveniens secundum habitum illum, ita etiam per habitum fidei
inclinatur mens hominis ad assentiendum his quae conveniunt rectae fidei
et non aliis.

&

第三異論に対しては、以下のように言われるべきである。
信仰の光は、信じられているものを見るようにする。
ちょうど、徳の他の習態によって、人が、その習態に即して自らに適合的なもの
 を見るように、信仰の習態によっても、人間の精神は、他のものにではなく、
 正しい信仰に適合するものに同意する事へ傾く。

\\


{\scshape Ad quartum dicendum} quod auditus est verborum significantium
ea quae sunt fidei, non autem est ipsarum rerum de quibus est fides. Et
sic non oportet ut huiusmodi res sint visae.

&

第四異論に対しては、以下のように言われるべきである。
聞かれたものは、信仰に属することを表示する言葉にかかわるが、それについて
 信仰があるところの事物そのものにかかわるのではない。この意味で、そのよ
 うな事物が見られたものである必要はない。



\\


\end{longtable}
\newpage

\rhead{a.~5}
\begin{center}
 {\Large {\bf ARTICULUS QUINTUS}}\\
 {\large UTRUM EA QUAE SUNT FIDEI POSSINT ESSE SCITA}\\
 {\footnotesize I$^a$II$^{ae}$, q.~67, a.~3; III {\itshape Sent.}, d.~24,
 a.~2, qu$^a$ 2; {\itshape De Verit.}, q.~14, a.~9; {\itshape Ad Hebr.},
 c.~9, lect.~1.}\\
 {\Large 第五項\\信仰に属することが知られたものでありうるか}
\end{center}

\begin{longtable}{p{21em}p{21em}}
{\huge A}{\scshape d quintum sic proceditur}. Videtur quod ea quae sunt
fidei possint esse scita. Ea enim quae non sciuntur videntur esse
ignorata, quia ignorantia scientiae opponitur. Sed ea quae sunt fidei
non sunt ignorata, horum enim ignorantia ad infidelitatem pertinet,
secundum illud I {\itshape ad Tim}.~{\scshape i}: {\itshape Ignorans
feci in incredulitate mea}. Ergo ea quae sunt fidei possunt esse scita.

&


第五項の問題へ向けて、議論は以下のように進められる。信仰に属することが
知られたものでありうると思われる。理由は以下の通り。無知は知に対立するの
で、知られていないものは、無知の対象だと思われる。ところが、信仰に属する
ことは、無知の対象ではない。なぜなら、かの『テモテへの手紙一』「わたしの
不信心において知らずに行った」\footnote{「以前、わたしは神を冒\UTF{7006}する者、
迫害する者、暴力を振るう者でした。しかし、信じていないとき知らずに行っ
たことなので、憐れみを受けました。」(1:13)}によれば、それらについて無知で
あることは不信仰に属するからである。ゆえに、信仰に属することが知られたも
のでありうる。



\\




{\scshape 2 Praeterea}, scientia per rationes
acquiritur. Sed ad ea quae sunt fidei a sacris auctoribus rationes
inducuntur. Ergo ea quae sunt fidei possunt esse scita.

&

さらに、知は理性をとおして獲得される。ところが、理性は、信仰に属すること
 へ、聖なる著作者たちによって導かれる。ゆえに、信仰に属することが知られたも
 のでありうる。

\\


{\scshape 3 Praeterea}, ea quae demonstrative
probantur sunt scita, quia demonstratio est syllogismus faciens
scire. Sed quaedam quae in fide continentur sunt demonstrative probata a
philosophis, sicut Deum esse, et Deum esse unum, et alia huiusmodi. Ergo
ea quae sunt fidei possunt esse scita.

&

さらに、論証は、知をもたらす三段論法であるから、論証的に証明されるものは
 知られたものである。ところが、「神が存在する」や「神は一である」や、そ
 の他このようなもののように、哲学者たちによって論証的に証明されたものが
 信仰の中に含まれる。ゆえに、信仰に属することが知られたものでありうる。

\\


{\scshape 4 Praeterea}, opinio plus distat a
scientia quam fides, cum fides dicatur esse media inter opinionem et
scientiam. Sed opinio et scientia possunt esse aliquo modo de eodem, ut
dicitur in I {\itshape Poster}. Ergo etiam fides et scientia.

&

さらに、信仰は、意見と知の中間にあると言われるので、意見は、信仰よりも、
 知から離れている。ところが、『分析論後書』1巻で言われているとおり、意見
 と知は、ある意味で同じものについてありうる。ゆえに、信仰と知もまた[同
 じものについてありうる]。

\\


{\scshape Sed contra est} quod Gregorius dicit, quod {\itshape
apparentia non habent fidem, sed agnitionem}. Ea ergo de quibus est
fides agnitionem non habent. Sed ea quae sunt scita habent
agnitionem. Ergo de his quae sunt scita non potest esse fides.

&

しかし反対に、グレゴリウスは「明らかな事柄は信仰ではなく認識をもつ」と言
 う。ゆえに、それについて信仰があるところのものは、認識を持たない。とこ
 ろが、知られたものは認識をもつ。ゆえに、知られたものについて信仰はあり
 えない。

\\


{\scshape Respondeo dicendum} quod omnis scientia habetur per aliqua
principia per se nota, et per consequens visa. Et ideo oportet
quaecumque sunt scita aliquo modo esse visa. Non autem est possibile
quod idem ab eodem sit creditum et visum, sicut supra dictum est. Unde
etiam impossibile est quod ab eodem idem sit scitum et creditum. Potest
tamen contingere ut id quod est visum vel scitum ab uno, sit creditum ab
alio. Ea enim quae de Trinitate credimus nos visuros speramus, secundum
illud I {\itshape ad Cor}.~{\scshape xiii}: {\itshape Videmus nunc per
speculum in aenigmate, tunc autem facie ad faciem}, quam quidem visionem
iam Angeli habent, unde quod nos credimus illi vident. Et similiter
potest contingere ut id quod est visum vel scitum ab uno homine, etiam
in statu viae, sit ab alio creditum, qui hoc demonstrative non novit. Id
tamen quod communiter omnibus hominibus proponitur ut credendum est
communiter non scitum. Et ista sunt quae simpliciter fidei subsunt. Et
ideo fides et scientia non sunt de eodem.

&

解答する。以下のように言われるべきである。
すべて知は何らかの自明な原理によって持たれ、その結果、見られたものとなる。
ゆえに、なんであれ知られたものは、なんらかの意味で見られたものである。
 ところが、前に述べられたとおり、同一のものが同じ人によって、信じられか
 つ見られることは不可能である。したがって、同一のものが同じ人によって、
 知られかつ信じられることもまた不可能である。
しかし、ある人によって見られ、あるいは知られたものが、別の人によって信じ
 られることはありうる。たとえば、かの『コリントの信徒への手紙一』13章「私た
 ちは、今、鏡の中に謎において見るが、そのときは、顔と顔を合わせて見る」
 によれば、私たちが三位一体について信じていることを、将来、見ることを希
 望しているが、その直視を、すでに天使たちはもっている。したがって、私た
 ちが信じていることを、天使たちは見ている。
同様に、現生の状態においても、ある人によって見られ、あるいは知られたもの
が、論証的にそれを知ることがない別の人によって、信じられることはありうる。
しかし、すべての人間に共通に信じられるべきこととして示されているものは、
 共通に、知られていないものである。そしてこれらが、端的に、信仰の中に入
 る。ゆえに、信仰と知は、同一のものにかかわらない。


\\


{\scshape Ad primum ergo dicendum} quod infideles
eorum quae sunt fidei ignorantiam habent, quia nec vident aut sciunt ea
in seipsis, nec cognoscunt ea esse credibilia. Sed per hunc modum
fideles habent eorum notitiam, non quasi demonstrative, sed inquantum
per lumen fidei videntur esse credenda, ut dictum est.

&

第一異論に対しては、以下のように言われるべきである。
不信仰者が信仰に属することについて無知をもつのは、それらを、それら自体に
 おいて見も知りもせず、また、それらが信じられるべきであることも認識しな
 いからである。これに対して、信仰者はそれらについての知(notitia)を、いわ
 ば論証的にではなく、すでに述べられたとおり、信仰の光によって信じられる
 べきものと見られるかぎりで、もつ。

\\


{\scshape Ad secundum dicendum} quod rationes quae
inducuntur a sanctis ad probandum ea quae sunt fidei non sunt
demonstrativae, sed persuasiones quaedam manifestantes non esse
impossibile quod in fide proponitur. Vel procedunt ex principiis fidei,
scilicet ex auctoritatibus sacrae Scripturae, sicut Dionysius dicit,
 {\scshape ii} cap.~{\itshape de Div.~Nom}. Ex his autem principiis ita probatur aliquid apud
fideles sicut etiam ex principiis naturaliter notis probatur aliquid
apud omnes. Unde etiam theologia scientia est, ut in principio operis
dictum est.

&

第二異論に対しては、以下のように言われるべきである。聖者たちによって、信
仰に属するものを証明することへ導かれる理性・理論(ratio)は、論証的なもので
はなく、信仰の中に提示されていることが不可能でないことを明示する、一種の
説得的なものである。あるいは[別の答え方をすると]、ディオニュシウスが
『神名論』2章で言うように、それらは信仰の原理から、すなわち、聖書の権威か
ら出てくる。ところが、ちょうど、すべての人にとって、自然本性的に知られた
諸原理から何かが証明されるように、信仰者にとって、これらの[聖書の]諸原
理から、何かが証明される。したがって、この書の初めに言われたとおり、神学
もまた学知である。


\\


{\scshape Ad tertium dicendum} quod ea quae
demonstrative probari possunt inter credenda numerantur, non quia de
ipsis sit simpliciter fides apud omnes, sed quia praeexiguntur ad ea
quae sunt fidei, et oportet ea saltem per fidem praesupponi ab his qui
horum demonstrationem non habent.

&

第三異論に対しては、以下のように言われるべきである。
論証的に証明されうる事柄が、信仰されるべき事柄の中に数えられうるのは、す
 べての人が、そ れらについて端的に信仰をもつからではなく、それらが、信仰に属する
 ことに前もって必要とされるからであり、これらの論証をもたない人々によっ
 ては、それらが少なくとも信仰をとおして前提とされるべきだからである。

\\


{\scshape Ad quartum dicendum} quod, sicut
philosophus ibidem dicit, a diversis hominibus de eodem omnino potest
haberi scientia et opinio, sicut et nunc dictum est de scientia et
fide. Sed ab uno et eodem potest quidem haberi fides et scientia de
eodem secundum quid, scilicet subiecto, sed non secundum idem : potest
enim esse quod de una et eadem re aliquis aliquid sciat et aliquid aliud
opinetur; et similiter de Deo potest aliquis demonstrative scire quod
sit unus, et credere quod sit trinus. Sed de eodem secundum idem non
potest esse simul in uno homine scientia nec cum opinione nec cum fide,
alia tamen et alia ratione. Scientia enim cum opinione simul esse non
potest simpliciter de eodem, quia de ratione scientiae est quod id quod
scitur existimetur esse impossibile aliter se habere; de ratione autem
opinionis est quod id quod quis existimat, existimet possibile aliter se
habere. Sed id quod fide tenetur, propter fidei certitudinem,
existimatur etiam impossibile aliter se habere, sed ea ratione non
potest simul idem et secundum idem esse scitum et creditum, quia scitum
est visum et creditum est non visum, ut dictum est.

&

第四異論に対しては、以下のように言われるべきである。
哲学者はその箇所で、あらゆる点で同じことについて、さまざまな人々によって、知
 と意見・臆測が持たれうることを語っている。これは今、知と信仰について言われた
 ことと同じである。
しかし、ある点で同じことについては、つまり、同一の主題であるということで
 あって、同一の観点からではない場合には、同じ一人の人によっても、信仰と
 知が持たれうる。つまり、同じ一つの事物について、ある人が何かを知り、別
 の何かを臆測することがあるように、神について、ある人が、論証的にそれが
 一であることを知り、三であることを信じるというように。
しかし、同じことについて、同一の観点から、同時に、一人の人のなかに、知は
 意見とともに、あるいは信仰とともに、存在することはできない。そしてこれ
 らは別々の理由による。
知が意見と同時に、端的に同じ事柄について存在しえないのは、「知」の性格に、知
 られているものが、他のあり方でありえないと判断されていることが含まれる
 が、「意見・臆測」の性格には、だれかが判断することが、他のあり方でもあ
 りうると判断していることが含まれるからである。
信仰によって保たれているものもまた、信仰の確かさのために、他の
 あり方でありえないと判断されているが、しかし、これを根拠として、
 同時に、同じものが、同じ観点のもとで、知られたものでありかつ信じられた
 ものであることにはならない。すでに述べられたとおり、
 知られたものは見られたものだが、信じられたものは見られたものでないから
 である。



\end{longtable}

\newpage


\rhead{a.~6}
\begin{center}
 {\Large {\bf ARTICULUS SEXTUS}}\\
 {\large UTRUM CREDIBILIA SINT PER CERTOS ARTICULOS DISTINGUENDA}\\
 {\footnotesize III {\itshape Sent.}, d.~25, q.~1, a.~1, qu$^a$ 1; a.~2,
 ad 6; I {\itshape Cor.}, cap.~15, lect. 1.}\\
 {\Large 第六項\\信じられうる事柄は、特定の箇条によって区別されるべきか}
\end{center}

\begin{longtable}{p{21em}p{21em}}

{\huge A}{\scshape d sextum sic proceditur}. Videtur quod
credibilia non sint per certos articulos distinguenda. Eorum enim omnium
quae in sacra Scriptura continentur est fides habenda. Sed illa non
possunt reduci ad aliquem certum numerum, propter sui multitudinem. Ergo
superfluum videtur articulos fidei distinguere.

&

第六項の問題へ、議論は以下のように進められる。
信じられうる事柄が、特定の箇条によって区別されるべきでないと思われる。
理由は以下のとおり。
聖書に含まれるすべての事柄についての信仰が持たれるべきである。
ところが、これらの事柄は、その多さのために、なんらかの特定の数に還元され
 ることができない。ゆえに、信仰箇条を区別することは不要だと思われる。

\\


{\scshape 2 Praeterea}, materialis distinctio, cum
in infinitum fieri possit, est ab arte praetermittenda. Sed formalis
ratio obiecti credibilis est una et indivisibilis, ut supra dictum est,
scilicet veritas prima, et sic secundum rationem formalem credibilia
distingui non possunt. Ergo praetermittenda est credibilium materialis
distinctio per articulos.

&
さらに、質料的区別は、無限になされうるので、学芸によって無視されるべきであ
 る。ところが、信じられうる対象の形相的性格は、前に述べられたとおり、ひ
 とつで不可分のもの、すなわち第一真理であるので、形相的性格にしたがって
 は、信じられうる事柄は区別されえない。ゆえに、箇条によって、信じられう
 る事柄を質料的に区別することは無視されるべきである。

\\


{\scshape 3 Praeterea}, sicut a quibusdam dicitur, articulus est
{\itshape indivisibilis veritas de Deo arctans nos ad credendum}. Sed
credere est voluntarium, quia, sicut Augustinus dicit, {\itshape nullus
credit nisi volens}. Ergo videtur quod inconvenienter distinguantur
credibilia per articulos.

&

さらに、ある人によって語られるように、箇条は「私たちを信じることへ強いる、
 神についての不可分な真理である」。ところが、アウグスティヌスが「意志し
 ないかぎりだれも信じない」と言うように、信じることは意志的な事柄である。
 ゆえに、箇条によって信じられる事柄が区別されるのは不適切だと思われる。

\\


{\scshape Sed contra est quod} Isidorus dicit, {\itshape articulus est
perceptio divinae veritatis tendens in ipsam}. Sed perceptio divinae
veritatis competit nobis secundum distinctionem quandam, quae enim in
Deo unum sunt in nostro intellectu multiplicantur. Ergo credibilia
debent per articulos distingui.

&
しかし反対に、イシドルスは「箇条は、神の真理へと向かう、神の真理の把握で
 ある」と述べる。ところが、神の真理の把握は、ある種の区別にしたがって、
 私たちに適合する。なぜなら、神において一つのものが、私たちの知性におい
 ては多数化されるからである。ゆえに、信じられる事柄は、箇条によって区別
 されるべきである。

\\


{\scshape Respondeo dicendum} quod nomen {\itshape articuli} ex Graeco
videtur esse derivatum. {\itshape Arthron} enim in Graeco, quod in
Latino {\itshape articulus} dicitur, significat quandam coaptationem
aliquarum partium distinctarum. Et ideo particulae corporis sibi invicem
coaptatae dicuntur membrorum articuli. Et similiter in grammatica apud
Graecos dicuntur articuli quaedam partes orationis coaptatae aliis
dictionibus ad exprimendum earum genus, numerum vel casum. Et similiter
in rhetorica articuli dicuntur quaedam partium coaptationes, dicit enim
Tullius, in IV {\itshape Rhet}., {\itshape quod articulus dicitur cum
singula verba intervallis distinguuntur caesa oratione, hoc modo,
acrimonia, voce, vultu adversarios perterruisti}. Unde et credibilia
fidei Christianae dicuntur per articulos distingui inquantum in quasdam
partes dividuntur habentes aliquam coaptationem ad invicem.


&

解答する。以下のように言われるべきである。「箇条(articulus)」という言葉は、
ギリシア語に由来すると思われる。つまり、ギリシア語のarthronが、ラテン語で
はarticulusと言われるが、これは、区別されたなんらかの諸部分の結びつきを意
味する。ゆえに、相互に結びつけられた身体の諸部分は、肢体のarticulus(関節)
と言われる。また、同様に、ギリシア人たちの文法で、性、数、格を表示するた
めに、他の言葉に結びつけられる言明の部分が、articulus(冠詞)と言われる。
また、同様に、弁論術において、部分の一種の結びつきがarticulus(連語)と言わ
れる。たとえば、キケロは『弁論術』4巻で、「『あなたは、峻烈さ、声、
 表情で、敵たちを恐れさせた』のように、ひとつひとつの言葉が、接続詞を使
 わずに、空白で区別されているとき、articulusと言われる」と述べている。
したがって、キリスト教信仰に属する信じられるべき事柄も、なんらかの相互の
 結びつきをもって、ある種の部分に分かたれるかぎりで、箇条(articulus)によって区別されると言われる。


\\



Est autem obiectum fidei aliquid non
visum circa divina, ut supra dictum est. Et ideo ubi occurrit aliquid
speciali ratione non visum, ibi est specialis articulus, ubi autem multa
secundum eandem rationem sunt incognita, ibi non sunt articuli
distinguendi. Sicut aliam difficultatem habet ad videndum quod Deus sit
passus, et aliam quod mortuus resurrexerit, et ideo distinguitur
articulus resurrectionis ab articulo passionis. Sed quod sit passus,
mortuus et sepultus, unam et eandem difficultatem habent, ita quod, uno
suscepto, non est difficile alia suscipere, et propter hoc omnia haec
pertinent ad unum articulum.


&


ところで、信仰の対象は、前に述べられたとおり、神的なことがらをめぐって見
 られない何かである。ゆえに、種的な観点において、見られない何かが生じてい
 るところに、種的な箇条があるが、同一の観点において、知られていない多数の
 ものがある場合には、そこで箇条が区別されるべきでない。たとえば、神が受難
 したことと、死者が蘇ったことを見るためには、異なる困難さがあるので、それ
 ゆえ、復活についての箇条と、受難についての箇条は区別される。しかし、受難
 したこと、死んだこと、埋葬されたことは、一つで同一の困難さを持つので、一
 つが受け入れられると、その他を受け入れることは難しくない。そのため、これ
 らすべては、一つの箇条に属する。

\\


{\scshape Ad primum ergo dicendum} quod aliqua sunt
credibilia de quibus est fides secundum se; aliqua vero sunt credibilia
de quibus non est fides secundum se, sed solum in ordine ad alia, sicut
etiam in aliis scientiis quaedam proponuntur ut per se intenta, et
quaedam ad manifestationem aliorum. Quia vero fides principaliter est de
his quae videnda speramus in patria, secundum illud {\itshape Heb}.~{\scshape xi}, {\itshape fides est
substantia sperandarum rerum}; ideo per se ad fidem pertinent illa quae
directe nos ordinant ad vitam aeternam, sicut sunt tres personae,
omnipotentia Dei, mysterium incarnationis Christi, et alia huiusmodi. Et
secundum ista distinguuntur articuli fidei. Quaedam vero proponuntur in
sacra Scriptura ut credenda non quasi principaliter intenta, sed ad
praedictorum manifestationem, sicut quod Abraham habuit duos filios,
quod ad tactum ossium Elisaei suscitatus est mortuus, et alia huiusmodi,
quae narrantur in sacra Scriptura in ordine ad manifestationem divinae
maiestatis vel incarnationis Christi. Et secundum talia non oportet
articulos distinguere.

&


第一異論に対しては、それゆえ、以下のように言われるべきである。
あるものは、それについて、それ自体において信仰があるから、信じられうるものであるが、
 別のあるものは、それ自体において信仰があるのではなく、むしろ、他のもの
 への秩序においてのみ信仰がある。たとえば、他の諸学においても、ある事柄
 は自体的に意図されたものとして提示されるが、他の事柄を明示するために提
 示されるものもある。さて、信仰は、かの『ヘブライ人への手紙』11章の「信
 仰とは、希望されるべき事物の実体である」\footnote{「信仰とは、望んでい
 る事柄を確信し、見えない事実を確認することです」(11:1)}によれば、私たち
 が、天国で見られるべきだと希望する事柄に、主要にかかわるので、それゆえ、
 私たちを、直接的に、永遠の生命へと秩序づけるものが、自体的に信仰に属す
 る。たとえば、三つのペルソナや、神の全能、キリストの受肉の秘蹟、その他
 このようなものである。そして、これらに基づいて、信仰箇条は区別される。
 他方、ある事柄は、主要に意図されたものとしてではなく、公言されたことを
 明示するために、聖書の中に、信じられるべき事柄として提示されている。た
 とえば、アブラハムには二人の息子がいたことや、エリシャの骨に触れて死人
 が生き返ったこと\footnote{「エリシャは死んで葬られた。その後、モアブの
 部隊が毎年この地に侵入してきた。人々がある人を葬ろうとしていたとき、そ
 の部隊を見たので、彼をエリシャの墓に投げ込んで立ち去った。その人はエリ
 シャの骨に触れると生き返り、自分の足で立ち上がった。『列王記下』
 (13:20-21)}や、その他このようなことである。これらは、神の力やキリストの
 受肉を明示することへ秩序づけられて、聖書の中で語られている。そして、こ
 れらに対応して、箇条を区別する必要はない。

\\


{\scshape Ad secundum dicendum} quod ratio formalis
obiecti fidei potest accipi dupliciter. Uno modo, ex parte ipsius rei
creditae. Et sic ratio formalis omnium credibilium est una, scilicet
veritas prima. Et ex hac parte articuli non distinguuntur. Alio modo
potest accipi formalis ratio credibilium ex parte nostra. Et sic ratio
formalis credibilis est ut sit non visum. Et ex hac parte articuli fidei
distinguuntur, ut visum est.

&

第二異論に対しては、以下のように言われるべきである。
信仰の形相的対象の性格は、二通りに理解されうる。
一つには、信じられる事柄自体の側からであり、この場合、信じられうるすべて
 の事柄の形相的性格は一つ、すなわち、第一真理である。そして、この側面か
 ら箇条が区別されることはない。もう一つには、信じられうる事柄の形相的性
 格が、私たちの側から理解されうる。この場合、信じられうるものの形相的性
 格は、それが見られたものでないことである。この側面から、すでに見られた
 ように、信仰箇条は区別される。
\\


{\scshape Ad tertium dicendum} quod illa definitio
datur de articulo magis secundum quandam etymologiam nominis prout habet
derivationem Latinam, quam secundum eius veram significationem prout a
Graeco derivatur. Unde non est magni ponderis. Potest tamen dici quod,
licet ad credendum necessitate coactionis nullus arctetur, cum credere
sit voluntarium; arctatur tamen necessitate finis, quia {\itshape accedentem ad
Deum oportet credere}, et {\itshape sine fide impossibile est placere Deo}, ut
apostolus dicit, {\itshape Heb}.~{\scshape xi}.


&

第三異論に対しては、以下のように言われるべきである。
かの定義は、箇条について、ギリシャ語に由来するものとして、その本当の意味
 にしたがって与えられていると言うよりはむしろ、ラテン語の派生語をもつも
 のとして、名称のある種の語源にしたがって与えられている。それゆえ、あま
 り重要ではない。しかし、信じることは意志的なことだから、だれも、強制の
 必然性によって信じることへと強いられることはないが、しかし、目的の必然
 性によっては強いられる。なぜなら、『ヘブライ人への手紙』11章
 \footnote{「信仰がなければ、神に喜ばれることはできません。神に近づく者
 は、神が存在しておられること、また、神はご自身を求める者たちに報いてく
 ださる方であることを、信じていなければならないからです」(11:6)}で使徒が述
 べるように、「神へ近づく人は信じなければなら」ず、
 「信仰なしに、神に気にいられることは不可能」だからである。


\end{longtable}
\newpage


\rhead{a.~7}
\begin{center}
 {\Large {\bf ARTICULUS SEPTIMUS}}\\
 {\large UTRUM ARTICULI FIDEI SECUNDUM SUCCESSIONEM TEMPORUM CREVERINT}\\
 {\footnotesize Infra, q.~2, a.~7; q.~174, a.~6; III {\itshape Sent.},
 d.~25, q.~2, a.~2.~qu$^a$ 1.}\\
 {\Large 第七項\\信仰箇条は時間の経過とともに増えたか}
\end{center}

\begin{longtable}{p{21em}p{21em}}

{\huge A}{\scshape d septimum sic proceditur}. Videtur
quod articuli fidei non creverint secundum temporum successionem. Quia,
ut apostolus dicit, ad {\itshape Heb}.~{\scshape xi}, {\itshape fides est substantia sperandarum
rerum}. Sed omni tempore sunt eadem speranda. Ergo omni tempore sunt
eadem credenda.

&

第七項の問題へ、議論は以下のように進められる。信仰箇条は、時間の経過と
ともに増えたのではないと思われる。理由は以下の通り。使徒が『ヘブライ人
への手紙』11章で言うように、「信仰は希望されるべき事物の実体である」。
ところが、あらゆる時において、同一のことが希望されるべきである。ゆえに、
あらゆる時において、同一のことが信じられるべきである。


\\


{\scshape 2 Praeterea}, in scientiis humanitus ordinatis per
successionem temporum augmentum factum est propter defectum cognitionis
in primis qui scientias invenerunt, ut patet per philosophum, in II
{\itshape Metaphys}. Sed doctrina fidei non est inventa humanitus, sed
tradita a Deo. {\itshape Dei enim donum est}, ut dicitur {\itshape
Ephes}.~{\scshape ii}. Cum igitur in Deum nullus defectus scientiae
cadat, videtur quod a principio cognitio credibilium fuerit perfecta, et
quod non creverit secundum successionem temporum.


&


さらに、

\\


{\scshape 3 Praeterea}, operatio gratiae non minus
ordinate procedit quam operatio naturae. Sed natura semper initium sumit
a perfectis ut Boetius dicit, in libro de Consol. Ergo etiam videtur
quod operatio gratiae a perfectis initium sumpserit, ita quod illi qui
primo tradiderunt fidem perfectissime eam cognoverunt.

&


\\


{\scshape 4 Praeterea}, sicut per apostolos ad nos
fides Christi pervenit, ita etiam in veteri testamento per priores
patres ad posteriores devenit cognitio fidei, secundum illud
Deut. XXXII, interroga patrem tuum et annuntiabit tibi. Sed apostoli
plenissime fuerunt instructi de mysteriis, acceperunt enim, sicut
tempore prius, ita et ceteris abundantius, ut dicit Glossa, super illud
Rom. VIII, nos ipsi primitias spiritus habentes. Ergo videtur quod
cognitio credibilium non creverit per temporum successionem.

&


\\


{\scshape  Sed contra est quod} Gregorius dicit,
quod secundum incrementa temporum crevit scientia sanctorum patrum, et
quanto viciniores adventui salvatoris fuerunt, tanto sacramenta salutis
plenius perceperunt.

&


\\


{\scshape Respondeo dicendum} quod ita se habent in
doctrina fidei articuli fidei sicut principia per se nota in doctrina
quae per rationem naturalem habetur. In quibus principiis ordo quidam
invenitur, ut quaedam in aliis implicite contineantur, sicut omnia
principia reducuntur ad hoc sicut ad primum, impossibile est simul
affirmare et negare, ut patet per philosophum, in IV Metaphys. Et
similiter omnes articuli implicite continentur in aliquibus primis
credibilibus, scilicet ut credatur Deus esse et providentiam habere
circa hominum salutem, secundum illud ad Heb. XI, accedentem ad Deum
oportet credere quia est, et quod inquirentibus se remunerator sit. In
esse enim divino includuntur omnia quae credimus in Deo aeternaliter
existere, in quibus nostra beatitudo consistit, in fide autem
providentiae includuntur omnia quae temporaliter a Deo dispensantur ad
hominum salutem, quae sunt via in beatitudinem. Et per hunc etiam modum
aliorum subsequentium articulorum quidam in aliis continentur, sicut in
fide redemptionis humanae implicite continetur et incarnatio Christi et
eius passio et omnia huiusmodi. Sic igitur dicendum est quod, quantum ad
substantiam articulorum fidei, non est factum eorum augmentum per
temporum successionem, quia quaecumque posteriores crediderunt
continebantur in fide praecedentium patrum, licet implicite. Sed quantum
ad explicationem, crevit numerus articulorum, quia quaedam explicite
cognita sunt a posterioribus quae a prioribus non cognoscebantur
explicite. Unde dominus Moysi dicit, Exod. VI, ego sum Deus Abraham,
Deus Isaac, Deus Iacob, et nomen meum Adonai non indicavi eis. Et David
dicit, super senes intellexi. Et apostolus dicit, ad Ephes. III, aliis
generationibus non est agnitum mysterium Christi sicut nunc revelatum
est sanctis apostolis eius et prophetis.

&


\\



{\scshape Ad primum ergo dicendum} quod semper
fuerunt eadem speranda apud omnes. Quia tamen ad haec speranda homines
non pervenerunt nisi per Christum, quanto a Christo fuerunt remotiores
secundum tempus, tanto a consecutione sperandorum longinquiores, unde
apostolus dicit, ad Heb. XI, iuxta fidem defuncti sunt omnes isti, non
acceptis repromissionibus, sed a longe eas respicientes. Quanto autem
aliquid a longinquioribus videtur, tanto minus distincte videtur. Et
ideo bona speranda distinctius cognoverunt qui fuerunt adventui Christi
vicini.
&


\\



{\scshape Ad secundum dicendum} quod profectus
cognitionis dupliciter contingit. Uno modo, ex parte docentis, qui in
cognitione proficit, sive unus sive plures, per temporum
successionem. Et ista est ratio augmenti in scientiis per rationem
humanam inventis. Alio modo, ex parte addiscentis, sicut magister qui
novit totam artem non statim a principio tradit eam discipulo, quia
capere non posset, sed paulatim, condescendens eius capacitati. Et hac
ratione profecerunt homines in cognitione fidei per temporum
successionem. Unde apostolus, ad Gal. III, comparat statum veteris
testamenti pueritiae.

&


\\



{\scshape Ad tertium dicendum} quod ad generationem
naturalem duae causae praeexiguntur, scilicet agens et materia. Secundum
igitur ordinem causae agentis, naturaliter prius est quod est
perfectius, et sic natura a perfectis sumit exordium, quia imperfecta
non ducuntur ad perfectionem nisi per aliqua perfecta
praeexistentia. Secundum vero ordinem causae materialis, prius est quod
est imperfectius, et secundum hoc natura procedit ab imperfecto ad
perfectum. In manifestatione autem fidei Deus est sicut agens, qui habet
perfectam scientiam ab aeterno, homo autem est sicut materia recipiens
influxum Dei agentis. Et ideo oportuit quod ab imperfectis ad perfectum
procederet cognitio fidei in hominibus. Et licet in hominibus quidam se
habuerint per modum causae agentis, quia fuerunt fidei doctores; tamen
manifestatio spiritus datur talibus ad utilitatem communem, ut dicitur I
ad Cor. XII. Et ideo tantum dabatur patribus qui erant instructores
fidei de cognitione fidei, quantum oportebat pro tempore illo populo
tradi vel nude vel in figura.

&


\\



{\scshape Ad quartum dicendum} quod ultima
consummatio gratiae facta est per Christum, unde et tempus eius dicitur
tempus plenitudinis, ad Gal. IV. Et ideo illi qui fuerunt propinquiores
Christo vel ante, sicut Ioannes Baptista, vel post, sicut apostoli,
plenius mysteria fidei cognoverunt. Quia et circa statum hominis hoc
videmus, quod perfectio est in iuventute, et tanto habet homo
perfectiorem statum vel ante vel post, quanto est iuventuti propinquior.

&


\end{longtable}
\newpage




\end{document}


%\rhead{a.~}
%\begin{center}
% {\Large {\bf }}\\
% {\large }\\
% {\footnotesize }\\
% {\Large \\}
%\end{center}
%
%\begin{longtable}{p{21em}p{21em}}
%
%&
%
%
%\\
%\end{longtable}
%\newpage



Articulus 8

[38802] IIª-IIae q. 1 a. 8 arg. 1 Ad octavum sic proceditur. Videtur quod inconvenienter articuli fidei enumerentur. Ea enim quae possunt ratione demonstrativa sciri non pertinent ad fidem ut apud omnes sint credibilia, sicut supra dictum est. Sed Deum esse unum potest esse scitum per demonstrationem, unde et philosophus hoc in XII Metaphys. probat, et multi alii philosophi ad hoc demonstrationes induxerunt. Ergo Deum esse unum non debet poni unus articulus fidei.

[38803] IIª-IIae q. 1 a. 8 arg. 2 Praeterea, sicut de necessitate fidei est quod credamus Deum omnipotentem, ita etiam quod credamus eum omnia scientem et omnibus providentem; et circa utrumque eorum aliqui erraverunt. Debuit ergo inter articulos fidei fieri mentio de sapientia et providentia divina, sicut et de omnipotentia.

[38804] IIª-IIae q. 1 a. 8 arg. 3 Praeterea, eadem est notitia patris et filii, secundum illud Ioan. XIV, qui videt me videt et patrem. Ergo unus tantum articulus debet esse de patre et filio; et, eadem ratione, de spiritu sancto.

[38805] IIª-IIae q. 1 a. 8 arg. 4 Praeterea, persona patris non est minor quam filii et spiritus sancti. Sed plures articuli ponuntur circa personam spiritus sancti, et similiter circa personam filii. Ergo plures articuli debent poni circa personam patris.

[38806] IIª-IIae q. 1 a. 8 arg. 5 Praeterea, sicuti personae patris et personae spiritus sancti aliquid appropriatur, ita et personae filii secundum divinitatem. Sed in articulis ponitur aliquod opus appropriatum patri, scilicet opus creationis; et similiter aliquod opus appropriatum spiritui sancto, scilicet quod locutus est per prophetas. Ergo etiam inter articulos fidei debet aliquod opus appropriari filio secundum divinitatem.

[38807] IIª-IIae q. 1 a. 8 arg. 6 Praeterea, sacramentum Eucharistiae specialem habet difficultatem prae multis articulis. Ergo de ea debuit poni specialis articulus. Non videtur ergo quod articuli sufficienter enumerentur.

[38808] IIª-IIae q. 1 a. 8 s. c. Sed in contrarium est auctoritas Ecclesiae sic enumerantis.

[38809] IIª-IIae q. 1 a. 8 co. Respondeo dicendum quod, sicut dictum est, illa per se pertinent ad fidem quorum visione in vita aeterna perfruemur, et per quae ducemur in vitam aeternam. Duo autem nobis ibi videnda proponuntur, scilicet occultum divinitatis, cuius visio nos beatos facit; et mysterium humanitatis Christi, per quem in gloriam filiorum Dei accessum habemus, ut dicitur ad Rom. V. Unde dicitur Ioan. XVII, haec est vita aeterna, ut cognoscant te, Deum verum, et quem misisti Iesum Christum. Et ideo prima distinctio credibilium est quod quaedam pertinent ad maiestatem divinitatis; quaedam vero pertinent ad mysterium humanitatis Christi, quod est pietatis sacramentum, ut dicitur I ad Tim. III. Circa maiestatem autem divinitatis tria nobis credenda proponuntur. Primo quidem, unitas divinitatis, et ad hoc pertinet primus articulus. Secundo, Trinitas personarum, et de hoc sunt tres articuli secundum tres personas. Tertio vero proponuntur nobis opera divinitatis propria. Quorum primum pertinet ad esse naturae, et sic proponitur nobis articulus creationis. Secundum vero pertinet ad esse gratiae, et sic proponuntur nobis sub uno articulo omnia pertinentia ad sanctificationem humanam. Tertium vero pertinet ad esse gloriae, et sic ponitur alius articulus de resurrectione carnis et de vita aeterna. Et ita sunt septem articuli ad divinitatem pertinentes. Similiter etiam circa humanitatem Christi ponuntur septem articuli. Quorum primus est de incarnatione sive de conceptione Christi; secundus de nativitate eius ex virgine; tertius de passione eius et morte et sepultura; quartus est de descensu ad Inferos; quintus est de resurrectione; sextus de ascensione; septimus de adventu ad iudicium. Et sic in universo sunt quatuordecim. Quidam tamen distinguunt duodecim articulos fidei, sex pertinentes ad divinitatem et sex pertinentes ad humanitatem. Tres enim articulos trium personarum comprehendunt sub uno, quia eadem est cognitio trium personarum. Articulum vero de opere glorificationis distinguunt in duos, scilicet in resurrectionem carnis et gloriam animae. Similiter articulum conceptionis et nativitatis coniungunt in unum.

[38810] IIª-IIae q. 1 a. 8 ad 1 Ad primum ergo dicendum quod multa per fidem tenemus de Deo quae naturali ratione investigare philosophi non potuerunt, puta circa providentiam eius et omnipotentiam, et quod ipse solus sit colendus. Quae omnia continentur sub articulo unitatis Dei.

[38811] IIª-IIae q. 1 a. 8 ad 2 Ad secundum dicendum quod ipsum nomen divinitatis importat provisionem quandam, ut in primo libro dictum est. Potentia autem in habentibus intellectum non operatur nisi secundum voluntatem et cognitionem. Et ideo omnipotentia Dei includit quodammodo omnium scientiam et providentiam, non enim posset omnia quae vellet in istis inferioribus agere nisi ea cognosceret et eorum providentiam haberet.

[38812] IIª-IIae q. 1 a. 8 ad 3 Ad tertium dicendum quod patris et filii et spiritus sancti est una cognitio quantum ad unitatem essentiae, quae pertinet ad primum articulum. Quantum vero ad distinctionem personarum, quae est per relationes originis, quodammodo in cognitione patris includitur cognitio filii, non enim esset pater si filium non haberet, quorum nexus est spiritus sanctus. Et quantum ad hoc bene moti sunt qui posuerunt unum articulum trium personarum. Sed quia circa singulas personas sunt aliqua attendenda circa quae contingit esse errorem, quantum ad hoc de tribus personis possunt poni tres articuli. Arius enim credidit patrem omnipotentem et aeternum, sed non credidit filium coaequalem et consubstantialem patri, et ideo necessarium fuit apponere articulum de persona filii ad hoc determinandum. Et eadem ratione contra Macedonium necesse fuit ponere articulum tertium de persona spiritus sancti. Et similiter etiam conceptio Christi et nativitas, et etiam resurrectio et vita aeterna, secundum unam rationem possunt comprehendi sub uno articulo, inquantum ad unum ordinantur, et secundum aliam rationem possunt distingui, inquantum seorsum habent speciales difficultates.

[38813] IIª-IIae q. 1 a. 8 ad 4 Ad quartum dicendum quod filio et spiritui sancto convenit mitti ad sanctificandam creaturam, circa quod plura credenda occurrunt. Et ideo circa personam filii et spiritus sancti plures articuli multiplicantur quam circa personam patris, qui nunquam mittitur, ut in primo dictum est.

[38814] IIª-IIae q. 1 a. 8 ad 5 Ad quintum dicendum quod sanctificatio creaturae per gratiam et consummatio per gloriam fit etiam per donum caritatis, quod appropriatur spiritui sancto, et per donum sapientiae, quod appropriatur filio. Et ideo utrumque opus pertinet et ad filium et ad spiritum sanctum per appropriationem secundum rationes diversas.

[38815] IIª-IIae q. 1 a. 8 ad 6 Ad sextum dicendum quod in sacramento Eucharistiae duo possunt considerari. Unum scilicet quod sacramentum est, et hoc habet eandem rationem cum aliis effectibus gratiae sanctificantis. Aliud est quod miraculose ibi corpus Christi continetur, et sic concluditur sub omnipotentia, sicut et omnia alia miracula, quae omnipotentiae attribuuntur.



%\rhead{a.~}
%\begin{center}
% {\Large {\bf }}\\
% {\large }\\
% {\footnotesize }\\
% {\Large \\}
%\end{center}
%
%\begin{longtable}{p{21em}p{21em}}
%
%&
%
%
%\\
%\end{longtable}
%\newpage


Articulus 9

[38816] IIª-IIae q. 1 a. 9 arg. 1 Ad nonum sic proceditur. Videtur quod inconvenienter articuli fidei in symbolo ponantur. Sacra enim Scriptura est regula fidei, cui nec addere nec subtrahere licet, dicitur enim Deut. IV, non addetis ad verbum quod vobis loquor, neque auferetis ab eo. Ergo illicitum fuit aliquod symbolum constituere quasi regulam fidei, post sacram Scripturam editam.

[38817] IIª-IIae q. 1 a. 9 arg. 2 Praeterea, sicut apostolus dicit, ad Ephes. IV, una est fides. Sed symbolum est professio fidei. Ergo inconvenienter traditur multiplex symbolum.

[38818] IIª-IIae q. 1 a. 9 arg. 3 Praeterea, confessio fidei quae in symbolo continetur pertinet ad omnes fideles. Sed non omnibus fidelibus convenit credere in Deum, sed solum illis qui habent fidem formatam. Ergo inconvenienter symbolum fidei traditur sub hac forma verborum, credo in unum Deum.

[38819] IIª-IIae q. 1 a. 9 arg. 4 Praeterea, descensus ad Inferos est unus de articulis fidei, sicut supra dictum est. Sed in symbolo patrum non fit mentio de descensu ad Inferos. Ergo videtur insufficienter collectum.

[38820] IIª-IIae q. 1 a. 9 arg. 5 Praeterea, sicut Augustinus dicit, exponens illud Ioan. XIV, creditis in Deum, et in me credite, Petro aut Paulo credimus, sed non dicimur credere nisi in Deum. Cum igitur Ecclesia Catholica sit pure aliquid creatum, videtur quod inconvenienter dicatur, in unam sanctam, Catholicam et apostolicam Ecclesiam.

[38821] IIª-IIae q. 1 a. 9 arg. 6 Praeterea, symbolum ad hoc traditur ut sit regula fidei. Sed regula fidei debet omnibus proponi et publice. Quodlibet igitur symbolum deberet in Missa cantari, sicut symbolum patrum. Non videtur ergo esse conveniens editio articulorum fidei in symbolo.

[38822] IIª-IIae q. 1 a. 9 s. c. Sed contra est quod Ecclesia universalis non potest errare, quia spiritu sancto gubernatur, qui est spiritus veritatis, hoc enim promisit dominus discipulis, Ioan. XVI, dicens, cum venerit ille spiritus veritatis, docebit vos omnem veritatem. Sed symbolum est auctoritate universalis Ecclesiae editum. Nihil ergo inconveniens in eo continetur.

[38823] IIª-IIae q. 1 a. 9 co. Respondeo dicendum quod, sicut apostolus dicit, ad Heb. XI, accedentem ad Deum oportet credere. Credere autem non potest aliquis nisi ei veritas quam credat proponatur. Et ideo necessarium fuit veritatem fidei in unum colligi, ut facilius posset omnibus proponi, ne aliquis per ignorantiam a fidei veritate deficeret. Et ab huiusmodi collectione sententiarum fidei nomen symboli est acceptum.

[38824] IIª-IIae q. 1 a. 9 ad 1 Ad primum ergo dicendum quod veritas fidei in sacra Scriptura diffuse continetur et variis modis, et in quibusdam obscure; ita quod ad eliciendum fidei veritatem ex sacra Scriptura requiritur longum studium et exercitium, ad quod non possunt pervenire omnes illi quibus necessarium est cognoscere fidei veritatem, quorum plerique, aliis negotiis occupati, studio vacare non possunt. Et ideo fuit necessarium ut ex sententiis sacrae Scripturae aliquid manifestum summarie colligeretur quod proponeretur omnibus ad credendum. Quod quidem non est additum sacrae Scripturae, sed potius ex sacra Scriptura assumptum.

[38825] IIª-IIae q. 1 a. 9 ad 2 Ad secundum dicendum quod in omnibus symbolis eadem fidei veritas docetur. Sed ibi oportet populum diligentius instrui de fidei veritate ubi errores insurgunt, ne fides simplicium per haereticos corrumpatur. Et haec fuit causa quare necesse fuit edere plura symbola. Quae in nullo alio differunt nisi quod in uno plenius explicantur quae in alio continentur implicite, secundum quod exigebat haereticorum instantia.

[38826] IIª-IIae q. 1 a. 9 ad 3 Ad tertium dicendum quod confessio fidei traditur in symbolo quasi ex persona totius Ecclesiae, quae per fidem unitur. Fides autem Ecclesiae est fides formata, talis enim fides invenitur in omnibus illis qui sunt numero et merito de Ecclesia. Et ideo confessio fidei in symbolo traditur secundum quod convenit fidei formatae, ut etiam si qui fideles fidem formatam non habent, ad hanc formam pertingere studeant.

[38827] IIª-IIae q. 1 a. 9 ad 4 Ad quartum dicendum quod de descensu ad Inferos nullus error erat exortus apud haereticos, et ideo non fuit necessarium aliquam explicationem circa hoc fieri. Et propter hoc non reiteratur in symbolo patrum, sed supponitur tanquam praedeterminatum in symbolo apostolorum. Non enim symbolum sequens abolet praecedens, sed potius illud exponit, ut dictum est.

[38828] IIª-IIae q. 1 a. 9 ad 5 Ad quintum dicendum quod, si dicatur in sanctam Ecclesiam Catholicam, est hoc intelligendum secundum quod fides nostra refertur ad spiritum sanctum, qui sanctificat Ecclesiam, ut sit sensus, credo in spiritum sanctum sanctificantem Ecclesiam. Sed melius est et secundum communiorem usum, ut non ponatur ibi in, sed simpliciter dicatur sanctam Ecclesiam Catholicam, sicut etiam Leo Papa dicit.

[38829] IIª-IIae q. 1 a. 9 ad 6 Ad sextum dicendum quod, quia symbolum patrum est declarativum symboli apostolorum, et etiam fuit conditum fide iam manifestata et Ecclesia pacem habente, propter hoc publice in Missa cantatur. Symbolum autem apostolorum, quod tempore persecutionis editum fuit, fide nondum publicata, occulte dicitur in prima et in completorio, quasi contra tenebras errorum praeteritorum et futurorum.


%\rhead{a.~}
%\begin{center}
% {\Large {\bf }}\\
% {\large }\\
% {\footnotesize }\\
% {\Large \\}
%\end{center}
%
%\begin{longtable}{p{21em}p{21em}}
%
%&
%
%
%\\
%\end{longtable}
%\newpage


Articulus 10

[38830] IIª-IIae q. 1 a. 10 arg. 1 Ad decimum sic proceditur. Videtur quod non pertineat ad summum pontificem fidei symbolum ordinare. Nova enim editio symboli necessaria est propter explicationem articulorum fidei, sicut dictum est. Sed in veteri testamento articuli fidei magis ac magis explicabantur secundum temporum successionem propter hoc quod veritas fidei magis manifestabatur secundum maiorem propinquitatem ad Christum, ut supra dictum est. Cessante ergo tali causa in nova lege, non debet fieri maior ac maior explicatio articulorum fidei. Ergo non videtur ad auctoritatem summi pontificis pertinere nova symboli editio.

[38831] IIª-IIae q. 1 a. 10 arg. 2 Praeterea, illud quod est sub anathemate interdictum ab universali Ecclesia non subest potestati alicuius hominis. Sed nova symboli editio interdicta est sub anathemate auctoritate universalis Ecclesiae. Dicitur enim in gestis primae Ephesinae synodi quod, perlecto symbolo Nicaenae synodi, decrevit sancta synodus aliam fidem nulli licere proferre vel conscribere vel componere praeter definitam a sanctis patribus qui in Nicaea congregati sunt cum spiritu sancto, et subditur anathematis poena; et idem etiam reiteratur in gestis Chalcedonensis synodi. Ergo videtur quod non pertineat ad auctoritatem summi pontificis nova editio symboli.

[38832] IIª-IIae q. 1 a. 10 arg. 3 Praeterea, Athanasius non fuit summus pontifex, sed Alexandrinus patriarcha. Et tamen symbolum constituit quod in Ecclesia cantatur. Ergo non magis videtur pertinere editio symboli ad summum pontificem quam ad alios.

[38833] IIª-IIae q. 1 a. 10 s. c. Sed contra est quod editio symboli facta est in synodo generali. Sed huiusmodi synodus auctoritate solius summi pontificis potest congregari, ut habetur in decretis, dist. XVII. Ergo editio symboli ad auctoritatem summi pontificis pertinet.

[38834] IIª-IIae q. 1 a. 10 co. Respondeo dicendum quod, sicut supra dictum est, nova editio symboli necessaria est ad vitandum insurgentes errores. Ad illius ergo auctoritatem pertinet editio symboli ad cuius auctoritatem pertinet sententialiter determinare ea quae sunt fidei, ut ab omnibus inconcussa fide teneantur. Hoc autem pertinet ad auctoritatem summi pontificis, ad quem maiores et difficiliores Ecclesiae quaestiones referuntur ut dicitur in decretis, dist. XVII. Unde et dominus, Luc. XXII, Petro dixit, quem summum pontificem constituit, ego pro te rogavi, Petre, ut non deficiat fides tua, et tu aliquando conversus confirma fratres tuos. Et huius ratio est quia una fides debet esse totius Ecclesiae, secundum illud I ad Cor. I, idipsum dicatis omnes, et non sint in vobis schismata. Quod servari non posset nisi quaestio fidei de fide exorta determinaretur per eum qui toti Ecclesiae praeest, ut sic eius sententia a tota Ecclesia firmiter teneatur. Et ideo ad solam auctoritatem summi pontificis pertinet nova editio symboli, sicut et omnia alia quae pertinent ad totam Ecclesiam, ut congregare synodum generalem et alia huiusmodi.

[38835] IIª-IIae q. 1 a. 10 ad 1 Ad primum ergo dicendum quod in doctrina Christi et apostolorum veritas fidei est sufficienter explicata. Sed quia perversi homines apostolicam doctrinam et ceteras Scripturas pervertunt ad sui ipsorum perditionem, sicut dicitur II Pet. ult.; ideo necessaria est, temporibus procedentibus, explanatio fidei contra insurgentes errores.

[38836] IIª-IIae q. 1 a. 10 ad 2 Ad secundum dicendum quod prohibitio et sententia synodi se extendit ad privatas personas, quarum non est determinare de fide. Non enim per huiusmodi sententiam synodi generalis ablata est potestas sequenti synodo novam editionem symboli facere, non quidem aliam fidem continentem, sed eandem magis expositam. Sic enim quaelibet synodus observavit, ut sequens synodus aliquid exponeret supra id quod praecedens synodus exposuerat, propter necessitatem alicuius haeresis insurgentis. Unde pertinet ad summum pontificem, cuius auctoritate synodus congregatur et eius sententia confirmatur.

[38837] IIª-IIae q. 1 a. 10 ad 3 Ad tertium dicendum quod Athanasius non composuit manifestationem fidei per modum symboli, sed magis per modum cuiusdam doctrinae, ut ex ipso modo loquendi apparet. Sed quia integram fidei veritatem eius doctrina breviter continebat, auctoritate summi pontificis est recepta, ut quasi regula fidei habeatur.


