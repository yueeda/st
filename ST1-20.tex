\documentclass[10pt]{jsarticle} % use larger type; default would be 10pt
%\usepackage[utf8]{inputenc} % set input encoding (not needed with XeLaTeX)
%\usepackage[round,comma,authoryear]{natbib}
%\usepackage{nruby}
\usepackage{okumacro}
\usepackage{longtable}
%\usepqckage{tablefootnote}
\usepackage[polutonikogreek,english,japanese]{babel}
%\usepackage{amsmath}
\usepackage{latexsym}
\usepackage{color}

%----- header -------
\usepackage{fancyhdr}
\lhead{{\it Summa Theologiae} I, q.~20}
%--------------------

\bibliographystyle{jplain}

\title{{\bf PRIMA PARS}\\{\HUGE Summae Theologiae}\\Sancti Thomae
Aquinatis\\{\sffamily QUEAESTIO VIGESIMA}\\DE AMORE DEI}
\author{Japanese translation\\by Yoshinori {\sc Ueeda}}
\date{Last modified \today}


%%%% コピペ用
%\rhead{a.~}
%\begin{center}
% {\Large {\bf }}\\
% {\large }\\
% {\footnotesize }\\
% {\Large \\}
%\end{center}
%
%\begin{longtable}{p{21em}p{21em}}
%
%&
%
%
%\\
%\end{longtable}
%\newpage



\begin{document}
\maketitle
\pagestyle{fancy}

\begin{center}
{\Large 第二十問\\神の愛について}
\end{center}


\begin{longtable}{p{21em}p{21em}}

{\Huge D}{\scshape einde} considerandum est de his quae absolute ad voluntatem Dei
 pertinent. In parte autem appetitiva inveniuntur in nobis et passiones
 animae, ut gaudium, amor, et huiusmodi; et habitus moralium virtutum,
 ut iustitia, fortitudo, et huiusmodi. Unde primo considerabimus de
 amore Dei; secundo, de iustitia Dei, et misericordia eius. 

&

次に、無条件的に神の意志に属する事柄について考察されるべきである。
ところで、欲求的部分の中には、私たちにおいて、喜びや愛などの、魂の情念と、
 正義や強さなどの、倫理的な徳の習態とが見出される。したがって、第一に、
 神の愛について、第二に、神の正義について、そして神の憐れみについて考察
 するであろう。


\\

Circa primum
 quaeruntur quatuor. 

\begin{enumerate}
 \item utrum in Deo sit amor.
 \item utrum amet omnia.
 \item utrum magis amet unum quam aliud.
 \item utrum meliora magis amet.
\end{enumerate}

&

第一の問題ををめぐって、四つのことが問われる。
\begin{enumerate}
 \item 神の中に愛があるか。
 \item 神は万物を愛するか。
 \item 神は一つのものを他のものより愛するか。
 \item 神はより善いものをより愛するか。
\end{enumerate}

\\

\end{longtable}

\newpage

\rhead{a.~1}
\begin{center}
 {\Large {\bf ARTICULUS PRIMUS}}\\
 {\large UTRUM AMOR SIT IN DEO}\\
 {\footnotesize Infra, q.82, a.5, ad 1; III {\itshape Sent.}, d.32, a.1,
 ad 1; I {\itshape SCG}, cap.91; IV, cap.19; {\itshape De Div.~Nom.},
 cap.4, lect.9.}\\
 {\Large 第一項\\愛が神の中にあるか}
\end{center}

\begin{longtable}{p{21em}p{21em}}

{\Huge A}{\scshape d primum sic proceditur}. Videtur quod amor
non sit in Deo. Nulla enim passio est in Deo. Amor est passio. Ergo amor
non est in Deo.


&

第一の問題へ、議論は以下のように進められる。
愛は神の中にないと思われる。理由は以下の通り。神の中にはどんな情念もない。
 ところが、愛は情念である。ゆえに、愛は神の中にない。


\\


{\scshape 2 Praeterea}, amor, ira, tristitia, et
huiusmodi, contra se dividuntur. Sed tristitia et ira non dicuntur de
Deo nisi metaphorice. Ergo nec amor.


&

さらに、愛、怒り、悲しみ、そしてこのようなものは、相互に分かたれる。
ところが、悲しみと怒りは、比喩的にでなければ、神について言われない。ゆえ
 に、愛もまたそうである。



\\


{\scshape 3 Praeterea}, Dionysius dicit, IV cap.~{\itshape de
Div.~Nom.}, {\itshape amor est vis unitiva et concretiva}. Hoc autem in Deo locum
habere non potest, cum sit simplex. Ergo in Deo non est amor.


&

さらに、ディオニュシウスは『神名論』第4章で、「愛は、合一させ、結びつけ
 る力である」と述べている。ところが、神は単純なので、神の中でこのようなことは
 ありえない。ゆえに、神の中に愛はない。

\\


{\scshape Sed contra est} quod dicitur I Ioan.~{\scshape iv}, {\itshape Deus
caritas est}.


&

しかし反対に、『ヨハネの手紙一』第4章で「神は愛(caritas)である」と言われている。


\\


{\scshape Respondeo dicendum} quod necesse est ponere
amorem in Deo. Primus enim motus voluntatis, et cuiuslibet appetitivae
virtutis, est amor. Cum enim actus voluntatis, et cuiuslibet appetitivae
virtutis tendat in bonum et malum, sicut in propria obiecta; 



&

解答する。以下のように言われるべきである。
神の中に愛を措定することが必要である。理由は以下の通り。
意志の第一の運動、そして、どんな欲求的な力の第一の運動も、愛である。
実際、意志の作用、または何であれ欲求的な力の作用は、固有対象として善と悪に向か
 う。

\\

bonum autem
principalius et per se est obiectum voluntatis et appetitus, malum autem
secundario et per aliud, inquantum scilicet opponitur bono, oportet
naturaliter esse priores actus voluntatis et appetitus qui respiciunt
bonum, his qui respiciunt malum; ut gaudium quam tristitia, et amor quam
odium. Semper enim quod est per se, prius est eo quod est per
aliud. 

&

しかし善は、より主要なものであり、自体的に意志と欲求の対象であるのに対
 し、悪は第二義的で、他のものによって、すなわち、善に対置される限りで
 [欲求の対象]であるから、本性的に、善にかかわる意志ないし欲求の作用は、
 悪にかかわるそれよりも先でなければならない。たとえば、喜びは悲しみより
 も、愛は憎しみよりも[先であるように]。というのも、自体的であるものは、
 他によってあるものよりも、常に先だからである。


\\



Rursus, quod est communius, naturaliter est prius, unde et
intellectus per prius habet ordinem ad verum commune, quam ad
particularia quaedam vera. Sunt autem quidam actus voluntatis et
appetitus, respicientes bonum sub aliqua speciali conditione, sicut
gaudium et delectatio est de bono praesenti et habito; desiderium autem
et spes, de bono nondum adepto. Amor autem respicit bonum in communi,
sive sit habitum, sive non habitum. Unde amor naturaliter est primus
actus voluntatis et appetitus. 



&

さらに、より共通的なものは、本性的に、より先である。したがって、知性もま
 た、個別具体的な諸々の真よりも、共通な真に対して、より先なる秩序をもつ。
 ところで、ある特殊な条件の下で善に関係する意志ないし欲求の作用が存在す
 る。たとえば、喜びや快は、現前し、所有されている善についてあるが、ま
 だ獲得されてない善については、欲求や希望があるというように。これに対し
 て、愛は、所有されていようがいまいが、共通に、善に関係する。したがって、
 愛は、本性的に、意志ないし欲求の第一の作用である。


\\



Et propter hoc, omnes alii motus
appetitivi praesupponunt amorem, quasi primam radicem. Nullus enim
desiderat aliquid, nisi bonum amatum, neque aliquis gaudet, nisi de bono
amato. Odium etiam non est nisi de eo quod contrariatur rei amatae. Et
similiter tristitiam, et cetera huiusmodi, manifestum est in amorem
referri, sicut in primum principium. Unde in quocumque est voluntas vel
appetitus, oportet esse amorem, remoto enim primo, removentur
alia. Ostensum est autem in Deo esse voluntatem. Unde necesse est in eo
ponere amorem.



&

そしてこのため、欲求の他のすべての運動は、第一の根拠(radix)として、愛を前提する。
実際、愛された善を欲求するのでなければ、だれも何も欲求せず、愛された善について
喜ぶのでなければ、だれも何についても喜ばない。さらに憎しみは、愛される事物に反
 対するものについてでなければならない。同様に悲しみや、その他このようなものも、
 第一原理へ関係づけられるかのようにして、愛へ関係づけられることが明らか
 である。
したがって、そこにおいて意志や欲求があるところにはどこでも、愛がなければ
 ならない。第一のものが取り除かれれば、その他のものも取り除かれるからで
 ある。
さて、神の中に意志があることが示された。したがって、神の中に愛を措定する
 ことが必要である。


\\



{\scshape Ad primum ergo dicendum} quod vis cognitiva
non movet, nisi mediante appetitiva. Et sicut in nobis ratio universalis
movet mediante ratione particulari, ut dicitur in III {\itshape de Anima}; ita
appetitus intellectivus, qui dicitur voluntas, movet in nobis mediante
appetitu sensitivo. Unde proximum motivum corporis in nobis est
appetitus sensitivus. 


&

 第一異論に対しては、それゆえ、以下のように言われるべきである。
 認識する力は、欲求する力の媒介がなければ動かさない。
 そして、『デ・アニマ』第3巻で言われるように、私たちにおいて、普遍的理性
 は個別的理性を媒介として動かすように、知性的欲求、これが意志と言われる
 のだが、も、私たちにおいて、感覚的欲求を媒介として動かす。したがって、
 私たちにおいて、身体の近接動因は、感覚的欲求である。

\\


Unde semper actum appetitus sensitivi concomitatur
aliqua transmutatio corporis; et maxime circa cor, quod est primum
principium motus in animali. Sic igitur actus appetitus sensitivi,
inquantum habent transmutationem corporalem annexam, passiones dicuntur,
non autem actus voluntatis. 


&

 したがって、常に、身体の何らかの変容は、感覚的欲求の作用を伴う。そして、
 これは最大限に、動物における運動の第一根源である心臓をめぐってある。
このようにして、それゆえ、感覚的欲求の作用は、それに結びついた身体的変容
 を持つ限りで、情念(受動)と言われる。しかし、意志の作用はそうでない。
 
\\

Amor igitur et gaudium et delectatio,
secundum quod significant actus appetitus sensitivi, passiones sunt, non
autem secundum quod significant actus appetitus intellectivi. Et sic
ponuntur in Deo. Unde dicit philosophus, in VII {\itshape Ethic}., quod {\itshape Deus una et
 simplici operatione gaudet}.\footnote{ ``una et simplici operatione''
 は、gaudeoの目的語に取る。\textgreek{di\`o <o je\`os >ae\`i m\'ian ka\`i <apl\~hn qa'irei <hdon'hn;}
({\itshape EN} VII, 1154$^b$26.)
}
Et eadem ratione, sine passione amat.



 &

ゆえに、愛、喜び、快楽は、感覚的欲求の作用を表示する限りでは、情念だが、
 知性的能力の作用を表示する限りでは、そうでない。そしてこの意味で、神に
 おいて[愛が]措定される。したがって、哲学者は『ニコマコス倫理学』第7巻
 で、「神は一つの単純な働きを楽しむ」と述べている。
そして、同じ理由で、神は情念なしに愛する。 

\\


{\scshape Ad secundum dicendum} quod in passionibus
sensitivi appetitus, est considerare aliquid quasi materiale, scilicet
corporalem transmutationem; et aliquid quasi formale, quod est ex parte
appetitus. Sicut in ira, ut dicitur in I {\itshape de Anima}, materiale est
accensio sanguinis circa cor, vel aliquid huiusmodi; formale vero,
appetitus vindictae. Sed rursus, ex parte eius quod est formale, in
quibusdam horum designatur aliqua imperfectio; sicut in desiderio, quod
est boni non habiti; et in tristitia, quae est mali habiti. Et eadem
ratio est de ira, quae tristitiam supponit. Quaedam vero nullam
imperfectionem designant, ut amor et gaudium. 

&

第二異論に対しては、以下のように言われるべきである。
感覚的欲求の情念においては、あるものを質料的なものとして、またあるものを
 形相的なものとして考察することができる。前者は、身体的変容であり、後者
 は欲求という側面からの考察である。たとえば、怒りにおいて、『デ・アニマ』
 第1巻で言われるように、質料的なものは、心臓の周りの血の燃焼や、何かその
 ようなものであり、他方、形相的なものは、復讐への欲求である。しかしさら
 に、形相的なものの側から、これらのうちのあるものにおいては、なんらかの
 不完全さが指定されている。たとえば、「願望」においては、所有さ
 れていない善へのものであることが、また、「悲しみ」においては、所有され
 ている悪についてであることが、指定されている。そして、「怒り」について
 も同じことが言えて、それは、「悲しみ」を前提とする。
これに対して、ある情念は、どんな不完全性も指定しないのであり、たとえば
 「愛」や「喜び」がそうである。


\\

Cum igitur nihil horum Deo
conveniat secundum illud quod est materiale in eis, ut dictum est; illa
quae imperfectionem important etiam formaliter, Deo convenire non
possunt nisi metaphorice, propter similitudinem effectus, ut supra
dictum est. Quae autem imperfectionem non important, de Deo proprie
dicuntur, ut amor et gaudium, tamen sine passione, ut dictum est.

&


ゆえに、すでに述べられたとおり\footnote{第一異論解答。}、こういったもの
 のどれも、それらにおいて質料的であるものに即しては、神に適合しないから、
さらに形相的にも不完全性を含意するものは、神に適合しえない。
 ただし、前に述べられたとおり\footnote{前項。}、結果の類似によって比喩的に語られるのでな
 い限り。これに対して、「愛」や「喜び」のように不完全性を含意しないもの
 は、神について固有に語られる。ただし、すでに述べられたとおり、情念なし
 にであるが。







\\


{\scshape Ad tertium dicendum} quod actus amoris semper
tendit in duo, scilicet in bonum quod quis vult alicui; et in eum cui
vult bonum. Hoc enim est proprie amare aliquem, velle ei bonum. Unde in
eo quod aliquis amat se, vult bonum sibi. Et sic illud bonum quaerit
sibi unire, inquantum potest. Et pro tanto dicitur amor vis unitiva,
etiam in Deo, sed absque compositione, quia illud bonum quod vult sibi,
non est aliud quam ipse, qui est per suam essentiam bonus, ut supra
ostensum est. 

&


第三異論に対しては、以下のように言われるべきである。
愛の作用は、常に、二つのものへ向かう。すなわち、ある人が、ある人のために
 意志する善と、その人のために善を意志するその人とである。
というのも、固有の意味で「ある人を愛する」のは、「その人にとっての善を意志す
 る」ことだからである。したがって、ある人は自分を愛することにおいて、自
 分にとっての善を意志する。この意味で、その善を、できる限り、自分と一つ
 にすることを求める。そしてこの限りで、神においても、愛は合一させる力と
 言われるが、そこに複合は意味されていない。なぜなら、神が自らのために意志す
 る善とは、神自身と別でないからである。そして、前に示されたとおり、神は
 自らの本質によって善である。

\\

In hoc vero quod aliquis amat alium, vult bonum illi. Et
sic utitur eo tanquam seipso, referens bonum ad illum, sicut ad
seipsum. Et pro tanto dicitur amor vis concretiva, quia alium aggregat
sibi habens se ad eum sicut ad seipsum. Et sic etiam amor divinus est
vis concretiva, absque compositione quae sit in Deo, inquantum aliis
bona vult.


&

他方、ある人が他人を愛することにおいては、
 その人は、その他人にとっての善を意志する。この意味で、その人は他人を自
 分自身のように扱っている。自分自身へであるかのように、他人へと善
 をもたらすからである。そしてこの限りで、愛は結びつける力と言われる。なぜなら、他人を自分
 へと集め、自分自身に関係するかのように、その人に関係するからである。
 そしてこの意味でも、他のものどもにとっての善を意志する限りで、神の愛は
 集める力である。ただし、神の中に複合が意味されるのではない。



\end{longtable}
\newpage




\rhead{a.~2}
\begin{center}
 {\Large {\bf ARRICULUS SECUNDUS}}\\
 {\large UTRUM DEUS OMNIA AMET}\\
 {\footnotesize Infra, q.23, a.3, ad 1; I$^a$II$^{ae}$, q.110, a.1; II
 {\itshape Sent.}, d.26, a.1; III, d.32, a.1, 2; I {\itshape SCG},
 cap.91, cap.150; {\itshape De Verit.}, q.27, a.1; {\itshape De
 Virtut.}, q.2, a.7, ad 2; in {\itshape Ioan.}, cap.5, lect.3; {\itshape
 De Div.~Nom.}, cap.4, lect.9.}\\
 {\Large 第二項\\神は万物を愛するか}
\end{center}

\begin{longtable}{p{21em}p{21em}}


{\Huge A}{\scshape d secundum sic proceditur}. Videtur quod
Deus non omnia amet. Quia, secundum Dionysium, {\scshape iv} cap.~{\itshape de Div.~Nom}.,
amor amantem extra se ponit, et eum quodammodo in amatum
transfert. Inconveniens autem est dicere quod Deus, extra se positus, in
alia transferatur. Ergo inconveniens est dicere quod Deus alia a se
amet.


&

第二項の問題へ、議論は以下のように進められる。
神は万物を愛するわけではないと思われる。理由は以下の通り。
ディオニュシウスの『神名論』第4章によれば、愛は愛する者を自分の外に出し、
それを、ある意味で愛される者へと運ぶ。しかし、神が、自分の外に置
 かれるとか、他のものへと運ばれると語ることは不適切である。
ゆえに、神が自分以外の他のものを愛すると語ることは不適切である。


\\


{\scshape 2 Praeterea}, amor Dei aeternus est. Sed ea
quae sunt alia a Deo, non sunt ab aeterno nisi in Deo. Ergo Deus non
amat ea nisi in seipso. Sed secundum quod sunt in eo, non sunt aliud ab
eo. Ergo Deus non amat alia a seipso.


&

さらに、神の愛は永遠である。
しかし、神でないものどもは、神の中でのみ、永遠から存在する。
ゆえに、神は、それらを、自分の中でのみ愛する。
しかし、それらは、神の中にある限りにおいて、神以外のものではない。
ゆえに、神が、自分以外のものを愛することはない。


\\


{\scshape 3 Praeterea}, duplex est amor, scilicet
concupiscentiae, et amicitiae. Sed Deus creaturas irrationales non amat
amore concupiscentiae, quia nullius extra se eget, nec etiam amore
amicitiae, quia non potest ad res irrationales haberi, ut patet per
philosophum, in VIII {\itshape Ethic}. Ergo Deus non omnia amat.


&

さらに、二通りの愛がある。すなわち、「求める愛」と「友愛」である。
しかし、神が非理性的被造物を「求める愛」によって愛することはない。なぜな
 ら、神は自分の外になにものも必要としないからである。
また、「友愛」によって愛することもない。なぜなら、『ニコマコス倫理学』第
 8巻の哲学者によって明らかなとおり、そのような愛が非理性的事物にたいして
 所有されることはありえないからである。
ゆえに、神が万物を愛することはない。


\\


{\scshape 4 Praeterea}, in Psalmo dicitur, {\itshape odisti omnes
qui operantur iniquitatem}. Nihil autem simul odio habetur et
amatur. Ergo Deus non omnia amat.


&

さらに、『詩編』で、「あなたは不正を行うすべての人々を憎む」
 \footnote{「主よ、あなたは偽って語る者を滅ぼし、流血の罪を犯す者、欺く
 者をいとわれます。」(5:7)}と言われて
 いる。しかし、なにものも、同時に、憎まれ、かつ愛されることはない。
ゆえに、神は万物を愛するわけではない。

\\


{\scshape Sed contra est} quod dicitur {\itshape Sap}.~{\scshape xi}, {\itshape diligis
omnia quae sunt, et nihil odisti eorum quae fecisti}.


&

しかし反対に、『知恵の書』9章「あなたは存在するすべてのものを愛し、作っ
 たものどものどれ一つも憎んでいない」と言われている。

\\


{\scshape Respondeo dicendum} quod Deus omnia existentia
amat. Nam omnia existentia, inquantum sunt, bona sunt, ipsum enim esse
cuiuslibet rei quoddam bonum est, et similiter quaelibet perfectio
ipsius. Ostensum est autem supra quod voluntas Dei est causa omnium
rerum et sic oportet quod intantum habeat aliquid esse, aut quodcumque
bonum, inquantum est volitum a Deo. Cuilibet igitur existenti Deus vult
aliquod bonum. Unde, cum amare nil aliud sit quam velle bonum alicui,
manifestum est quod Deus omnia quae sunt, amat. 



&

解答する。以下のように言われるべきである。
神はすべて存在するものを愛する。理由は以下の通り。
すべて存在するものは、存在するかぎり、善である。
なぜなら、どの事物の存在(esse)も、ある種の善であり、また同様に、事物のど
 んな完成もまた、善だからである。
ところで、神の意志はすべての事物の原因であり、したがって、[事物は]神に
 よって意志されているだけ、何かの存在や、何であれ善をもつのでなければな
 らない、ということが以前に示された\footnote{{\itshape ST} I, q.19, a.4.}。
それゆえ、存在するどんなものにも、神はなんらかの善を意志している。
したがって、愛するとは、何かにとっての善を意志することに他ならないから、
 神が、存在するすべてのものを愛することは明らかである。


\\



Non tamen eo modo sicut
nos. Quia enim voluntas nostra non est causa bonitatis rerum, sed ab ea
movetur sicut ab obiecto, amor noster, quo bonum alicui volumus, non est
causa bonitatis ipsius, sed e converso bonitas eius, vel vera vel
aestimata, provocat amorem, quo ei volumus et bonum conservari quod
habet, et addi quod non habet, et ad hoc operamur. Sed amor Dei est
infundens et creans bonitatem in rebus.


&

しかし、私たちが愛するのとは違ったしかたによる。
というのも、私たちの意志は、諸事物の善性の原因でなく、むしろ、対象によっ
 て動かされるようにして、それら諸事物の善性によって動かさ
 れるから、私たちの愛、すなわちその愛によって、私たちがある人にとっての
 善を意志するその愛は、その善性の原因でなく、むしろ逆に、その人の善性が、そ
 れが真の善性であれ、善だと評価されたものであれ、愛を呼び起こす。そして
 その愛によって、その人のために、その人が所有している善が保たれることや、
 所有していない善が加えられることを意志し、それに向けて私たちは働く。し
 かし、神の愛は、諸事物の中に善性を注ぎ入れ、創造するものである。




\\


{\scshape Ad primum ergo dicendum} quod amans sic fit
extra se in amatum translatus, inquantum vult amato bonum, et operatur
per suam providentiam, sicut et sibi. Unde et Dionysius dicit, {\scshape iv}
cap.~{\itshape de Div.~Nom}., {\itshape audendum est autem et hoc pro veritate dicere, quod
et ipse omnium causa, per abundantiam amativae bonitatis, extra seipsum
fit ad omnia existentia providentiis}.


&

第一異論に対しては、それゆえ、以下のように言われるべきである。
愛する者が、自分から外に出て愛される者へと移されるのは、自分にとってと同
 じように愛される者にとっての善を意志し、予知(配慮)によって働くからである。
それゆえに、ディオニュシウスは『神名論』4章で以下のように述べる。「万物
 の原因であるそれもまた、愛する善性の充溢を通して、自らの外に、存在する
 すべてのものへと、予知[=摂理]によってあると、真理のために、敢えて言
 わなければならない」。


\\


{\scshape Ad secundum dicendum} quod, licet creaturae ab
aeterno non fuerint nisi in Deo, tamen per hoc quod ab aeterno in Deo
fuerunt, ab aeterno Deus cognovit res in propriis naturis, et eadem
ratione amavit. Sicut et nos per similitudines rerum, quae in nobis
sunt, cognoscimus res in seipsis existentes.


&

第二異論に対しては、以下のように言われるべきである。
被造物が、神の中でしか永遠から存在しなかったのはたしかだが、しかし、永遠
 から神の中にあったということによって、神は永遠からそれらの事物を固有の本性
 において認識し、同じ観点から愛していた。
それはちょうど、私たちもまた、私たちの中にある諸事物の類似を通して、それ
 自身において存在している事物を認識するのと同様である。

\\


{\scshape Ad tertium dicendum} quod amicitia non potest
haberi nisi ad rationales creaturas, in quibus contingit esse
redamationem, et communicationem in operibus vitae, et quibus contingit
bene evenire vel male, secundum fortunam et felicitatem, sicut et ad eas
proprie benevolentia est. Creaturae autem irrationales non possunt
pertingere ad amandum Deum, neque ad communicationem intellectualis et
beatae vitae, qua Deus vivit. Sic igitur Deus, proprie loquendo, non
amat creaturas irrationales amore amicitiae, sed amore quasi
concupiscentiae; inquantum ordinat eas ad rationales creaturas, et etiam
ad seipsum; non quasi eis indigeat, sed propter suam bonitatem et
nostram utilitatem. Concupiscimus enim aliquid et nobis et aliis.


&

第三異論に対しては、以下のように言われるべきである。
友愛は、理性的被造物に対してしか持たれえない。理性的被造物の中では、愛し
 返してくれたり、日々の生活の中で心を伝えたりすることがありうるし、また、
 運や幸福という点で、うまく行ったり行かなかったりということがある。そ
 れはちょうど、それらの被造物に対して、固有の意味で好意というものがある
 のと同様である。
これに対して、非理性的被造物は、神を愛することや、知的な伝達、また、神
 がそれによって生きているところの至福なる生に到達することができない。
それゆえ、神は、厳密に言えば、友愛によってではなく、いわば「求める愛」によっ
 て、非理性的被造物を愛する。ただしそれは、それらを理性的被造物へ、さら
 には自分自身へと秩序づける限りにおいてであり、それらを必要とするという
 意味ではなく、自らの善性のために、そして私たちの役に立つようにという意
 味である。なぜなら、私たちは、自分自身のためも他の人々のためにも、
 何かを求めるからである。


\\


{\scshape Ad quartum dicendum} quod nihil prohibet unum
et idem secundum aliquid amari, et secundum aliquid odio haberi. Deus
autem peccatores, inquantum sunt naturae quaedam, amat, sic enim et
sunt, et ab ipso sunt. Inquantum vero peccatores sunt, non sunt, sed ab
esse deficiunt, et hoc in eis a Deo non est. Unde secundum hoc ab ipso
odio habentur.


&


第四異論に対しては、以下のように言われるべきである。
一つの同一のものが、ある点で愛され、別の点で憎まれることは、なんら問題な
 い。
ところで神は、罪人たちを、ある種の本性・自然物(natura)である点で愛する。なぜなら、
 彼らはそのようにして[=神が愛するから]存在するのであり、そして存在す
 るのは神によってだからである。
他方、罪人たちは、罪人であるという点で、存在するのではなく、むしろ存在を欠いている。そし
 てこのことが彼らの中にあるのは、神によってではない。
したがって、この点で、彼らは神によって憎まれている。




\end{longtable}
\newpage


\rhead{a.~3}
\begin{center}
 {\Large {\bf ARTICULUS TERTIUS}}\\
 {\large UTRUM DEUS AEQUALITER DILIGAT OMNIA}\\
 {\footnotesize II {\itshape Sent.}, d.26, a.1, ad 2; III, d.19, a.5,
 qu$^a$ 1; d.32, a.4; I {\itshape SCG} cap.95.}\\
 {\Large 第三項\\神は万物を等しく愛するか}
\end{center}

\begin{longtable}{p{21em}p{21em}}

{\Huge A}{\scshape d tertium sic proceditur}. Videtur quod
Deus aequaliter diligat omnia. Dicitur enim {\itshape Sap}.~{\scshape vi}, {\itshape aequaliter est ei
cura de omnibus}. Sed providentia Dei, quam habet de rebus, est ex amore
quo amat res. Ergo aequaliter amat omnia.


&

第三項の問題へ、議論は以下のように進められる。
神は万物を等しく愛すると思われる。理由は以下の通り。
『知恵の書』第6章で「万物についての配慮は、神にとって等しくある」
 \footnote{「万物を公平に計らっておられるからだ」(6:7)}と言わ
 れている。
しかし、神が諸事物についてもつ神の摂理は、それによって事物を愛するところ
 の愛に由来する。ゆえに、神は等しく万物を愛する。



\\


{\scshape 2 Praeterea}, amor Dei est eius essentia. Sed
essentia Dei magis et minus non recipit. Ergo nec amor eius. Non igitur
quaedam aliis magis amat.


&

さらに、神の愛は神の本質である。
しかし、神の本質は、より多く、より少なくということを受け入れない。
ゆえに、神の愛もまた受け入れない。
ゆえに、あるものを他のものよりも多く愛するということはない。


\\


{\scshape 3 Praeterea}, sicut amor Dei se extendit ad
res creatas, ita et scientia et voluntas. Sed Deus non dicitur scire
quaedam magis quam alia, neque magis velle. Ergo nec magis quaedam aliis
diligit.


&

さらに、神の愛が被造物に及ぶように、知と意志もまたそうである。
しかし、神は、あるものを他のものよりも多く知るとか、多く意志するとかと言
 われない。
ゆえに、あるものを他のものよりも多く愛するということもない。



\\


{\scshape Sed contra est} quod dicit Augustinus, {\itshape super
Ioann}., {\itshape omnia diligit Deus quae fecit; et inter ea magis diligit
creaturas rationales; et de illis eas amplius, quae sunt membra
unigeniti sui; et multo magis ipsum unigenitum suum}.


&

しかし反対に、アウグスティヌスは『ヨハネ福音書注解』で言う。「神は、作っ
 たすべてのものを愛する。そしてそれらのうちで、理性的被造物をより愛する。
 そして、それらについて、自らの独り子の四肢である(成員である)被造物を、
 そして、さらに自らの独り子そのものを、愛する」。


\\


{\scshape Respondeo dicendum} quod, cum amare sit velle
bonum alicui, duplici ratione potest aliquid magis vel minus amari. Uno
modo, ex parte ipsius actus voluntatis, qui est magis vel minus
intensus. Et sic Deus non magis quaedam aliis amat, quia omnia amat uno
et simplici actu voluntatis, et semper eodem modo se habente. 


&

解答する。以下のように言われるべきである。
「愛する」とは「ある人にとっての善を意志する」ことだから、二つの観点で、
あるものが、より多くあるいはより少なく愛されるということがありうる。
一つには、より強く、あるい
 はより弱く作用する意志の作用それ自体の側からである。
そしてこの観点では、神があるものを他のものよりもよ
 り多く愛するということはない。なぜなら、神は一つの単純な意志の作用によっ
 て万物を愛するからであり、また、常に同じ状態にあるからである。


\\

Alio modo,
ex parte ipsius boni quod aliquis vult amato. Et sic dicimur aliquem
magis alio amare, cui volumus maius bonum; quamvis non magis intensa
voluntate. Et hoc modo necesse est dicere quod Deus quaedam aliis magis
amat. Cum enim amor Dei sit causa bonitatis rerum, ut dictum est, non
esset aliquid alio melius, si Deus non vellet uni maius bonum quam
alteri.

&



もう一つの観点では、ある人が愛されたもののために欲する善の側からである。
そしてこの観点から、私たちが、AよりもBにより大きな善を欲するとき、その意
 志の強さは同じであっても、AよりもBをより多く愛すると言われる。
そして、この意味では、神があるものを他のものより多く愛すると言われること
 が必然である。
なぜなら、すでに述べられたとおり、神の愛は事物の善性の原因なので、もし神
 が、あるものに対して他のものよりも大きな善を意志しなかったならば、ある
 ものが他のものよりも善い、ということはなかったであろうから。


\\


{\scshape Ad primum ergo dicendum} quod dicitur Deo
aequaliter esse cura de omnibus, non quia aequalia bona sua cura omnibus
dispenset; sed quia ex aequali sapientia et bonitate omnia administrat.


&

第一異論に対しては、それゆえ、以下のように言われるべきである。
神にとって、万物への配慮は等しくあると言われるのは、自ら
 の配慮によって、神が等しい善を万物に配分するからではなく、等しい知恵と
 善性に基づいて、万物を統率するからである。


\\


{\scshape Ad secundum dicendum} quod ratio illa procedit
de intensione amoris ex parte actus voluntatis, qui est divina
essentia. Bonum autem quod Deus creaturae vult, non est divina
essentia. Unde nihil prohibet illud intendi vel remitti.


&

第二異論に対しては、以下のように言われるべきである。
かの異論は、意志の作用の側からの愛の強さについて論じているが、それは神の
 本質である。
しかし、神が被造物のために意志する善は、神の本質ではない。
したがって、それが意図されたり取り上げられたりしても問題ない。


\\


{\scshape Ad tertium dicendum} quod intelligere et velle
significant solum actus, non autem in sua significatione includunt
aliqua obiecta, ex quorum diversitate possit dici Deus magis vel minus
scire aut velle; sicut circa amorem dictum est.


&

第三異論に対しては、以下のように言われるべきである。
「知性認識すること」や「意志すること」は、作用だけを表示するが、その意味
 内容の中に、何らの対象も含まない。しかし対象の多様性に基づいて、神がよ
 り多くまたはより少なく知ったり意志したりすると言われうる。ちょうど、愛
 について述べられたように。


\end{longtable}
\newpage



\rhead{a.~4}
\begin{center}
 {\Large {\bf ARTICULUS QUARTUS}}\\
 {\large AN DEUS SEMPER MAGIS DILIGAT MELIORA}\\
 {\footnotesize III {\itshape Sent.}, d.31, q.2, a.3, qu$^a$3; d.32, a.5.}\\
 {\Large 第四項\\神は常により善いものをより愛するか}
\end{center}

\begin{longtable}{p{21em}p{21em}}

{\Huge A}{\scshape d quartum sic proceditur}. Videtur quod
 Deus non semper magis diligat meliora. Manifestum est enim quod
 Christus est melior toto genere humano, cum sit Deus et homo. Sed Deus
 magis dilexit genus humanum quam Christum, quia dicitur {\itshape Rom}.~{\scshape viii},
 {\itshape proprio filio suo non pepercit, sed pro nobis omnibus tradidit
 illum}. Ergo Deus non semper magis diligit meliora.


&


第四の問題へ、議論は以下のように進められる。神は、常により善いものをより
 愛するとは限らないと思われる。理由は以下の通り。
キリストが、全人類よりも善いことは明らかである。なぜなら、彼は神でありか
 つ人間なのだから。しかし、神は、キリストよりも人類を愛した。なぜなら、
『「ローマの信徒への手紙』8章で「自分自身の息子を惜しまず、私たちすべての
 ために、彼を遣わした」\footnote{「わたしたちすべてのために、その御子を
 さえ惜しまず死に渡された方は、御子と一緒にすべてのものをわたしたちに賜
 らないはずがありましょうか。」(8:32)}と言われているからである。ゆえに、神は、常により
 善いものをより愛するとは限らない。

\\


{\scshape 2 Praeterea}, Angelus est melior homine, unde in Psalmo {\scshape viii}
 dicitur de homine, {\itshape minuisti eum paulo minus ab Angelis}. Sed Deus plus
 dilexit hominem quam Angelum, dicitur enim {\itshape Hebr}.~{\scshape ii}, {\itshape nusquam Angelos
 apprehendit, sed semen Abrahae apprehendit}. Ergo Deus non semper magis
 diligit meliora.


&


さらに、天使は人間よりも善い。だから『詩編』8章で、人間について「あなた
 は彼を、天使より少し小さくした」\footnote{「神に僅かに劣るものとして人
 を造り」(8:3)}と言われる。しかし、神は人間を天使より
 も愛した。なぜなら、『ヘブライ人への手紙』2章で「何処にも天使たちを保護
 せず、かえってアブラハムの種子はそれを保護する」\footnote{「たしかに、
 イエスは天使たちを助けず、アブラハムの子孫を助けられるのです」(2:16)}と言われているからである。
 ゆえに、神が常に善いものをより愛するとは限らない。

\\


{\scshape 3 Praeterea}, Petrus fuit melior Ioanne, quia plus Christum
 diligebat. Unde dominus, sciens hoc esse verum, interrogavit Petrum,
 dicens, {\itshape Simon Ioannis, diligis me plus his}? Sed tamen Christus plus
 dilexit Ioannem quam Petrum, ut enim dicit Augustinus, super illud
 Ioan.{\scshape xxi}, {\itshape Simon Ioannis diligis me}? ``Hoc ipso signo Ioannes a ceteris
 discipulis discernitur; non quod solum eum, sed quod plus eum ceteris
 diligebat.'' Non ergo semper magis diligit meliora.


&

さらに、ペトロはヨハネよりも善かった。なぜなら、キリストを、より愛したから
 である。それで主は、次のことが真であることを知りながら、ペトロに、
 「ヨハネの子シモンよ、あなたは彼ら以上に私を愛するか」\footnote{「食事が
 終わると、イエスはシモン・ペトロに、「ヨハネの子シモン、この人たち以上
 に私を愛しているか」と言われた。」『ヨハネによる福音書』(21:15)}と問うた。しかし、キリスト
 はペトロよりもヨハネをより愛した。それは、アウグスティヌスが『ヨハネ福
 音書注解』で「ヨハネの子シモン、私を愛するか」について次のように言うとおりである。「このしるし自体によって、ヨハネは他の弟子たちから区別され
る。すなわち、(神が)彼を愛しただけでなく、他の者たちよりも彼を愛したこ
 とによって」。ゆえに、より善いものをより愛するとは限らない。

\\


{\scshape 4 Praeterea}, melior est innocens poenitente; cum poenitentia
 sit {\itshape secunda tabula post naufragium}, ut dicit Hieronymus. Sed Deus plus
 diligit poenitentem quam innocentem, quia plus de eo gaudet, dicitur
 enim Luc.~{\scshape xv}, {\itshape dico vobis quod maius gaudium erit in caelo super uno
 peccatore poenitentiam agente, quam super nonaginta novem iustis, qui
 non indigent poenitentia}. Ergo Deus non semper magis diligit meliora.


&

さらに、罪のない人は悔いている人よりも善い。なぜなら、悔いている人は、ヒエロ
 ニュムスが言うとおり、「海難の後の(船に次ぐ)二枚目の板」だからである。
しかし神は、罪のない人よりも悔いている人を、より愛する。なぜなら、『ルカ
 による福音書』15章「あなたたちに言う。悔悛を行う一人の罪人に対して、悔
 悛を行う必要がない99人の正しい人々よりも、大きな喜びが天にあるだろう」
 \footnote{「言っておくが、このように、悔い改める一人の罪人については、
 悔い改める必要のない九十九人の正しい人についてよりも大きな喜びが天にあ
 る」(15:7)}
 と言われているからである。ゆえに、神は常に、より善い人をより愛するとは
 限らない。


\\


{\scshape 5 Praeterea}, melior est iustus praescitus, quam peccator
 praedestinatus. Sed Deus plus diligit peccatorem praedestinatum, quia
 vult ei maius bonum, scilicet vitam aeternam. Ergo Deus non semper
 magis diligit meliora.

&

さらに、あらかじめ知られた義人は、予定された罪人よりも善い。しかし、神は、
 予定された罪人をより愛する。なぜなら、より大きな善、すなわち、永遠の生
 命を、彼のために意志するからである。ゆえに、神は、常に、より善いものを
 より愛するとは限らない。


\\


{\scshape Sed contra}, unumquodque diligit sibi simile; ut patet per
 illud quod habetur {\itshape Eccli}.~{\scshape xiii}, {\itshape omne
 animal diligit sibi simile}. Sed intantum aliquid est melius, inquantum
 est Deo similius. Ergo meliora magis diliguntur a Deo.


&

しかし反対に、各々のものは、自分に似たものを愛する。それはちょうど、『シ
 ラ書』13章「すべての動物は自分に似たものを愛する」\footnote{「生き物は
 すべて、その同類を愛し、人間もすべて、自分に近い者を愛する」(13:15)}と書かれていることか
 ら明らかなとおりである。ところで、或るものは、神に似ているほど、より善
 い。ゆえに、より善いものは、より神に愛される。

\\


{\scshape Respondeo dicendum} quod necesse est dicere, secundum
 praedicta, quod Deus magis diligat meliora. Dictum est enim quod Deum
 diligere magis aliquid, nihil aliud est quam ei maius bonum velle,
 voluntas enim Dei est causa bonitatis in rebus. Et sic, ex hoc sunt
 aliqua meliora, quod Deus eis maius bonum vult. Unde sequitur quod
 meliora plus amet.


&

解答する。以下のように言われるべきである。既述のことに即するならば、神は、
 より善いものをより愛すると言うことが必要である。理由は以下の通り。
神が何かをより愛するのは、それに対してより大きな善を意志することに他なら
 ない。神の意志は、諸事物における善性の原因だからである。この意味で、
神がそれにより大きな善を意志することに基づいて、何かがより善いということ
 がある。したがって、(神が)より善いものをより愛することが帰結する。


\\


{\scshape Ad primum ergo dicendum} quod Deus Christum diligit, non solum
 plus quam totum humanum genus, sed etiam magis quam totam universitatem
 creaturarum, quia scilicet ei maius bonum voluit, quia {\itshape dedit ei nomen,
 quod est super omne nomen}, ut verus Deus esset. Nec eius excellentiae
 deperiit ex hoc quod Deus dedit eum in mortem pro salute humani
 generis, quinimo ex hoc factus est victor gloriosus; {\itshape factus enim est
 principatus super humerum eius}, ut dicitur Isaiae {\scshape ix}.


&

 第一異論に対しては、それゆえ、以下のように言われるべきである。
 神がキリストを愛するのは、人類全体以上であるだけでなく、被造物からなる
 全宇宙以上にでもある。なぜなら、真の神であるように、「彼にすべての名前
 に優る名前を与えた」\footnote{「このため、神はキリストを高く上げ、あら
 ゆる名にまさる名をお与えになりました」『フィリピの信徒への手紙』(2:9)。}の
 だから、彼に、より大きな善を欲したからである。また、神が彼を、人類の救
 済のために死なせたことで、彼の卓越性が損なわれるわけではない。実際、
 『イザヤ書』9章「彼の肩の上に、至高の権威が作られた」\footnote{「ひとり
 のみどりごがわたしたちのために生まれた。ひとりの男の子がわたしたちに与
 えられた。権威が彼の肩にある」(9:5)}と言われるように、
 このことのために、彼は栄光ある勝利者とされたのである。

\\


{\scshape Ad secundum dicendum} quod naturam humanam assumptam a Dei
 verbo in persona Christi, secundum praedicta, Deus plus amat quam omnes
 Angelos, et melior est, maxime ratione unionis. Sed loquendo de humana
 natura communiter, eam angelicae comparando, secundum ordinem ad
 gratiam et gloriam, aequalitas invenitur; cum {\itshape eadem sit mensura hominis
 et Angeli}, ut dicitur {\itshape Apoc}.~{\scshape xxi}; ita tamen quod quidam Angeli quibusdam
 hominibus, et quidam homines quibusdam Angelis, quantum ad hoc,
 potiores inveniuntur. Sed quantum ad conditionem naturae, Angelus est
 melior homine. Nec ideo naturam humanam assumpsit Deus, quia hominem
 absolute plus diligeret, sed quia plus indigebat. Sicut bonus
 paterfamilias aliquid pretiosius dat servo aegrotanti, quod non dat
 filio sano.


&

第二異論に対しては、以下のように言われるべきである。
前述のことにしたがえば、キリストのペルソナにおいて神の御言葉によって取ら
 れた人間本性を、神はすべての天使たちよりも愛するし、また、その本性は、
最大限に一性という根拠で\footnote{キリストにおいて、神性と人間性とが同一
 であるという教義。}、[天使たちよりも]善い。
しかし、人間本性について共通に語り、それを天使の本性と比較するならば、恩
 恵と栄光に対する秩序という点で、等しさが見出される。なぜなら、『ヨハネ
 の黙示録』21章で言われるとおり、「人間と天使の尺度は同じ」
 \footnote{「これは人間の物差しによって測ったもので、天使が用いたのもこ
 れである。」(21:17)}だからである。しかしこれは、ある天使たちはある人間
 たちよりも、また、ある人間たちはある天使たちよりも、この点にかんして
 \footnote{i.e. 恩恵と栄光への秩序において。}、
 より強いのが見出される、というしかたによってである。
しかし、本性の状態という点では、天使は人間よりも善い。それゆえ、人間本性
 を神が取ったのは、人間を無条件的により愛したからではなく、より必要とし
 たからである。それはたとえば、善い家長は、病気の奴隷に、健康な息子に与えない
 貴重なものを与えるようにである。

\\


{\scshape Ad tertium dicendum} quod haec dubitatio de Petro et Ioanne
 multipliciter solvitur. Augustinus namque refert hoc ad mysterium,
 dicens quod vita activa, quae significatur per Petrum, plus diligit
 Deum quam vita contemplativa, quae significatur per Ioannem, quia magis
 sentit praesentis vitae angustias, et aestuantius ab eis liberari
 desiderat, et ad Deum ire. Contemplativam vero vitam Deus plus diligit,
 quia magis eam conservat; non enim finitur simul cum vita corporis,
 sicut vita activa. 


&

第三異論に対しては、以下のように言われるべきである。
ペトロとヨハネにかんするこの疑問は、多くのしかたで解決される。
すなわち、アウグスティヌスは、以下のように述べて、これを神秘だとしている。
ペトロによって示される活動的生は、ヨハネによって示される観照的生よりも、
 神を愛する。なぜなら、より現在の生の苦難を感じ、より情熱的にそれらから解放されるこ
 と、そして神へと向かうことを願うからである。他方、神は、観照的生をより
 愛する。なぜなら、それをより保存するからである。つまり、活動的生のよう
 に、身体の生命とともに終わることがないからである。

\\

Quidam vero dicunt quod Petrus plus dilexit Christum
 in membris; et sic etiam a Christo plus fuit dilectus; unde ei
 Ecclesiam commendavit. Ioannes vero plus dilexit Christum in seipso; et
 sic etiam plus ab eo fuit dilectus; unde ei commendavit matrem. 


&

またある人々は以下のように言う。ペトロはキリストを、メンバーの中で愛した。
そのようなかたちで、キリストからもより愛された。それゆえ、彼に教会を彼に
 託した。他方、ヨハネは、キリストを彼自身において愛した。そしてそのよう
 なかたちで、彼によってより愛された。したがって、彼に、母(マリア)を託
 した。


\\

Alii
 vero dicunt quod incertum est quis horum plus Christum dilexerit amore
 caritatis, et similiter quem Deus plus dilexerit in ordine ad maiorem
 gloriam vitae aeternae. Sed Petrus dicitur plus dilexisse, quantum ad
 quandam promptitudinem vel fervorem, Ioannes vero plus dilectus,
 quantum ad quaedam familiaritatis indicia, quae Christus ei magis
 demonstrabat, propter eius iuventutem et puritatem. 


&

また、他の人々は次のように言う。彼らの中で、だれが愛徳の愛によってキリス
 トをより愛したかは不確かであり、同様にまた、神が、永遠の生命とい
 うより大きな栄光への秩序において、だれをより愛したかも不確かである。し
 かし、ペトロが[神を]より愛したと言われるのは、ある種の敏速さと情熱の
 点においてであり、他方ヨハネがより愛されたのは、彼の若さと純粋さのため
 に、キリストが彼に示したより大きな親しさのしるしにかんしてである。


\\

Alii vero dicunt
 quod Christus plus dilexit Petrum, quantum ad excellentius donum
 caritatis, Ioannem vero plus, quantum ad donum intellectus. Unde
 simpliciter Petrus fuit melior, et magis dilectus, sed Ioannes secundum
 quid. Praesumptuosum tamen videtur hoc diiudicare, quia, ut dicitur
 {\itshape Prov}.~{\scshape xvi}, {\itshape spirituum ponderator est dominus}, et non alius.


&

また他の人々は次のように言う。キリストがペトロをより愛したのは、より卓越
 した愛徳の贈り物にかんしてであり、他方、ヨハネがより[愛されたのは]、
 知性の贈り物にかんしてである。したがって、端的には、ペトロがより善く、
 より愛されたが、しかし、ヨハネは、ある意味において、そうであった。
しかし、このことを判断するのは僭越なことである。なぜなら、『箴言』16章で
 言われるとおり、「霊を計る者は主」\footnote{「人間の道は自分の目に清く
 見えるが、主はその精神を調べられる。」(16:2)}であって他のものではないからである。


\\


{\scshape Ad quartum dicendum} quod poenitentes et innocentes se habent
 sicut excedentia et excessa. Nam sive sint innocentes, sive
 poenitentes, illi sunt meliores et magis dilecti, qui plus habent de
 gratia. Ceteris tamen paribus, innocentia dignior est et magis
 dilecta. Dicitur tamen Deus plus gaudere de poenitente quam de
 innocente, quia plerumque poenitentes cautiores, humiliores et
 ferventiores resurgunt. Unde Gregorius dicit ibidem, quod {\itshape dux in
 praelio eum militem plus diligit, qui post fugam conversus, fortiter
 hostem premit, quam qui nunquam fugit, nec unquam fortiter fecit}. 

&

第四異論に対しては、以下のように言われるべきである。
 悔いている者と罪のない者は、いい勝負をしている\footnote{ ``sicut
 excedentia et excessa''は、文字通りの意味ではなく、「どちらが上とは決め
 られない」という意味の熟語。}。
 というのも、罪がなかろうと悔いていようと、より多くの恩恵を持つ者が、
 より善く、より愛されているからである。しかし、他の点が同じであれば、罪
 のない者は、より価値がありより愛されている。しかし、神が、罪のない者よ
 りも悔いる者をより喜ぶと言われるのは、しばしば、悔いる者は、より注意深
 く、より慎み深く、より熱心な者となって戻ってくるからである。
このことから、グレゴリウスは、同じ箇所で次のように述べている。「将軍は、
 戦場で、逃げたあとに戻ってきて、より勇敢に敵を打ちのめす兵士の方を、一
 度も逃げたことがないが、一度も勇敢に戦わなかった兵士よりも愛する」。


\\

Vel,
 alia ratione, quia aequale donum gratiae plus est, comparatum
 poenitenti, qui meruit poenam, quam innocenti, qui non meruit. Sicut
 centum marcae maius donum est, si dentur pauperi, quam si dentur regi.


&

あるいは、別の理由によれば、等しい恩恵の贈り物は、罰を受けるのがふさわしい悔いる者にもたらされる
 と、罰を受けるのがふさわしくない罪のない者にもたらされるよりも、より大
 きい。ちょうど、100マルクが、貧しい者に与えられるならば、王に与えられる
 よりも、大きな贈り物であるように。



\\


{\scshape Ad quintum dicendum} quod, cum voluntas Dei sit causa
 bonitatis in rebus, secundum illud tempus pensanda est bonitas eius qui
 amatur a Deo, secundum quod dandum est ei ex bonitate divina aliquod
 bonum. Secundum ergo illud tempus quo praedestinato peccatori dandum
 est ex divina voluntate maius bonum, melior est; licet secundum aliquod
 aliud tempus, sit peior; quia et secundum aliquod tempus, non est nec
 bonus neque malus.


&

第五異論に対しては、以下のように言われるべきである。
神の意志は、諸事物における善性の原因だから、神に愛される人の善性は、神の
 善性に基づいて何らかの善が彼に与えられるべきその時に、評価されるべきで
 ある。ゆえに、予定された罪人に、神の意志にもとづいてより大きな善が与え
 られるべき時には、それはより善いが、しかし、別の時には、より悪い。なぜ
 なら、また別の時には、彼は善くも悪くもないからである。




\end{longtable}
\newpage

\end{document}
