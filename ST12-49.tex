\documentclass[10pt]{jsarticle} % use larger type; default would be 10pt
\usepackage[utf8]{inputenc} % set input encoding (not needed with XeLaTeX)
\usepackage[T1]{fontenc}
%\usepackage[round,comma,authoryear]{natbib}
%\usepackage{nruby}
\usepackage{okumacro}
\usepackage{longtable}
%\usepqckage{tablefootnote}
\usepackage[polutonikogreek,english,japanese]{babel}
%\usepackage{amsmath}
\usepackage{latexsym}
\usepackage{color}
\usepackage{otf}
\usepackage{schemata}
%----- header -------
\usepackage{fancyhdr}
\pagestyle{fancy}
\lhead{{\it Summa Theologiae} I-II, q.49}
%--------------------


\bibliographystyle{jplain}


\title{{\bf Prima Secundae}\\{\HUGE Summae Theologiae}\\Sancti Thomae
Aquinatis\\{\sffamily Quaestio Quadragesimanona}\\{\bf DE HABITIBUS IN GENERALI,\\QUOAD EORUM SUBSTANTIAM}}
\author{Japanese translation\\by Yoshinori {\sc Ueeda}}
\date{Last modified \today}

%%%% コピペ用
%\rhead{a.~}
%\begin{center}
% {\Large {\bf }}\\
% {\large }\\
% {\footnotesize }\\
% {\Large \\}
%\end{center}
%
%\begin{longtable}{p{21em}p{21em}}
%
%&
%
%\\
%\end{longtable}
%\newpage

\begin{document}

\maketitle

\begin{center}
{\Large 『神学大全』第二部の一\\第四十九問\\
一般的に習慣について、その実体に即して}
\end{center}

\newpage
\begin{longtable}{p{21em}p{21em}}

Post actus et passiones, considerandum est de principiis humanorum
actuum. Et primo, de principiis intrinsecis; secundo, de principiis
extrinsecis. Principium autem intrinsecum est potentia et habitus; sed
quia de potentiis in prima parte dictum est, nunc restat de habitibus
considerandum. Et primo quidem, in generali; secundo vero, de
virtutibus et vitiis, et aliis huiusmodi habitibus, qui sunt humanorum
actuum principia. Circa ipsos autem habitus in generali, quatuor
consideranda sunt, primo quidem, de ipsa substantia habituum, secundo,
de subiecto eorum; tertio, de causa generationis, augmenti et
corruptionis ipsorum; quarto, de distinctione ipsorum. Circa primum
quaeruntur quatuor. 

\begin{enumerate}
 \item utrum habitus sit qualitas. 
 \item utrum sit determinata species qualitatis.
 \item utrum habitus importet ordinem ad actum.
 \item de necessitate habitus.
\end{enumerate}


&

行為と情念の後に人間的行為の根源について考察されるべきである。そしてそ
れは第一に内的な根源について、第二に外的な根源についてである。ところで
内的根源は能力と習慣\footnote{故稲垣良典氏の業績に敬意を表してhatibus
を「習慣」と訳す。この「習慣」は「習態」とも訳される特殊な性質であり、
日本語の通常の意味での「習慣」よりもかなり広い外延を持つことに注意され
たい。とくに、それが「性質」のカテゴリーに位置づけられることからもわか
るように、習慣や徳の存在論に目が向けられていることにも注意。}である。
しかし能力については第一部で語られたので、今、習慣について考察されるべ
きことが残されている。そして第一に一般的に、第二に徳と悪徳について、そ
して人間的行為の根源である他のそのような習慣についてである。さて、その
習慣について一般的に、四つのことが考察されるべきである。第一に習慣の実
体それ自体について、第二に習慣の基体について、第三に習慣の発生、増加、
消滅について、第四に習慣の区別についてである。第一のことをめぐって四つ
のことが問われる。

\begin{enumerate}
 \item 習慣は性質か。
 \item 習慣は性質の限定された種か。
 \item 習慣は行為への秩序づけを含意するか。
 \item 習慣の必要性について。
\end{enumerate}




\end{longtable}

\newpage
\rhead{a.~1}
\begin{center}
{\Large {\bf ARTICULUS PRIMUS}}\\
{\large UTRUM HABITUS SIT QUALITAS}\\
{\footnotesize III {\itshape Sent.}, d.23, q.1, a.1; V {\itshape
 Metaphys.}, lect.20.}\\
{\Large 第一項\\習慣は性質か}
\end{center}

\begin{longtable}{p{21em}p{21em}}

{\scshape Ad primum sic proceditur}. Videtur quod habitus non sit
qualitas. Dicit enim Augustinus, in libro {\itshape Octoginta Trium Quaest}.,
quod {\itshape hoc nomen habitus dictum est ab hoc verbo quod est habere}. Sed
habere non solum pertinet ad qualitatem, sed ad alia genera, dicimur
enim habere etiam quantitatem, et pecuniam, et alia huiusmodi. Ergo
habitus non est qualitas.

&

第一項の問題へ議論は以下のように進められる。習慣は性質でないと思われる。
理由は以下の通り。アウグスティヌスは『八十三問題集』で「この「習慣」と
いう名は「持つ」という動詞から語られている」と言っている。しかるに持つ
ことは性質だけでなく他の類にも属する。例えば私たちは、量やお金やその他
そのようなものを持つと言われる。ゆえに習慣は性質でない。

\\



2. {\scshape Praeterea}, habitus ponitur unum praedicamentum; ut patet in libro
{\itshape Praedicamentorum}. Sed unum praedicamentum non continetur sub
alio. Ergo habitus non est qualitas.

&

さらに、習慣は『カテゴリー論』という書物の中で明らかなとおり、諸範疇の
 うちのひとつである。しかるに範疇の中のあるものが別の範疇に含まれるこ
 とはない。ゆえに習慣は性質でない。

\\



3. {\scshape Praeterea}, omnis habitus est dispositio, ut dicitur in
{\itshape Praedicamentis}. Sed dispositio est ordo habentis partes, ut dicitur in
V {\itshape Metaphys}. Hoc autem pertinet ad praedicamentum situs. Ergo habitus
non est qualitas.

&

さらに、『カテゴリー論』で述べられているように、全て習慣は態勢(傾向性)
である。しかるに態勢は、『形而上学』第5巻で言われているように、部分を
持つものの秩序である。しかしこれは「体位」の範疇に属する。ゆえに習慣は
 性質でない。


\\



{\scshape Sed contra est} quod philosophus dicit, in praedicamentis, quod habitus
est {\itshape qualitas de difficili mobilis}.

&

しかし反対に、哲学者は『カテゴリー論』の中で、習慣は変わりにくい性
質であると述べている。

\\



{\scshape Respondeo dicendum} quod hoc nomen habitus ab habendo est sumptum. A
quo quidem nomen habitus dupliciter derivatur, uno quidem modo,
secundum quod homo, vel quaecumque alia res, dicitur aliquid habere;
alio modo, secundum quod aliqua res aliquo modo se habet in seipsa vel
ad aliquid aliud. 

&

解答する。以下のように言われるべきである。「習慣・ハビトゥス(habitus)」
というこの名称は持つこと(habere)から取られている。ハビトゥスはそこから
二つのかたちで派生するのであり、一つのしかたでは、人間やなんであれ他の
事物が何かを持つと言われる限りにおいてであり、もう一つのしかたでは、あ
る事物が何らかのしかたでそれ自身において、あるいは他の何かに対して
\kenten{ある}(se habere)限りにおいてである。


\\


Circa primum autem, considerandum est quod habere,
secundum quod dicitur respectu cuiuscumque quod habetur, commune est
ad diversa genera. Unde philosophus inter post praedicamenta habere
ponit, quae scilicet diversa rerum genera consequuntur; sicut sunt
opposita, et prius et posterius, et alia huiusmodi. 

&

第一のしかたをめぐって以下のことが考察されるべきである。持つことは、何
であれ持たれるものにかんして言われる限りにおいて、さまざまな類に共通で
ある。そのことから哲学者は、「対立する」「より先・より後」その他そのよ
うなもののような、事物のさまざまな類に伴うものである「超範疇的なものど
も(postpraedicamenta)」の中に、「持つこと」を置いている。

\\


Sed inter ea quae
habentur, talis videtur esse distinctio, quod quaedam sunt in quibus
nihil est medium inter habens et id quod habetur, sicut inter
subiectum et qualitatem vel quantitatem nihil est medium. Quaedam vero
sunt in quibus est aliquid medium inter utrumque, sed sola relatio,
sicut dicitur aliquis habere socium vel amicum. 

&

しかし、持たれるものどもの中には次のような区別がある。すなわち、あるも
のどもは、持つものと持たれるものの間に媒介するものが何もない。たとえば
基体と性質あるいは量の間に何も媒体はない。他方で、あるものどもは、持つ
ものと持たれるものの間に何らかの媒介があるが、その媒介が関係だけである。
たとえばある人が同朋や友人を持つと言われる場合のように。

\\


Quaedam vero sunt inter quae est aliquid medium, non quidem actio vel
passio, sed aliquid per modum actionis vel passionis, prout scilicet
unum est ornans vel tegens, et aliud ornatum aut tectum, unde
philosophus dicit, in V {\itshape Metaphys}., quod {\itshape habitus
dicitur tanquam actio quaedam habentis et habiti}, sicut est in illis
quae circa nos habemus.

&

しかし他方で、あるものどもは、その間に媒介物があり、それは能動や受動で
なく、能動や受動のしかたによる何か、すなわち、一方が飾るものや覆うもの
であり、他方が飾られたものまたは覆われたものである、というような何かで
ある。このことから哲学者は『形而上学』第5巻で、「ハビトゥスは、持つも
のの持たれたものへの作用の一種であるかのように語られる」と言う。ちょう
ど私たちの周りで私たちが持つものどもにおいてそうであるように。


\\


Et ideo in his constituitur unum speciale genus rerum, quod dicitur
praedicamentum habitus, de quo dicit philosophus, in V {\itshape
Metaphys}., quod {\itshape inter habentem indumentum, et indumentum
quod habetur, est habitus medius}.

&


それゆえ、これらにおいて諸事物の一つの特殊な類が構成され、これが『カテ
ゴリー論』において「ハビトゥス」と言われている。哲学者はこれについて
『形而上学』第5巻で「衣服を持つものと持たれている衣服との間には、媒介
となるハビトゥスがある」と述べている。


\\



Si autem sumatur habere prout res aliqua dicitur quodam modo se habere
in seipsa vel ad aliud; cum iste modus se habendi sit secundum aliquam
qualitatem, hoc modo habitus quaedam qualitas est, de quo philosophus,
in V {\itshape Metaphys}., dicit quod habitus dicitur dispositio
secundum quam bene vel male disponitur dispositum, et aut secundum se
aut ad aliud, ut sanitas habitus quidam est. Et sic loquimur nunc de
habitu. Unde dicendum est quod habitus est qualitas.

&

しかし、持つことが、ある事物が何らかのしかたで自らにおいてあるいは他に
対して\kenten{ある}と言われるものとして理解されるならば、その
\kenten{ある}しかたは何らかの性質にしたがってあるのだから、この意味で
のハビトゥスは、一種の性質である。そしてアリストテレスは『形而上学』第
5巻で、「健康がなんらかのハビトゥスであるようにして、傾向性・態勢が、
それにしたがって自らにおいてあるいは他に対してよくあるいは悪く態勢付け
られるかぎりにおいて、ハビトゥスと言われる、と述べている。そしてこの意
味で今私たちはハビトゥスについて語っている。したがって、ハビトゥスは性
質だと言われるべきである。


\end{longtable}



\schema
{
  \schemabox{habitus}
}
{
  \schema
  {
    \schemabox{S habet aliquid.}
  }
  {
    \schemabox{nihil est medium (subjectum habet qualitatem.)\\medium
    est sola relatio (S habet amicum.)\\est medium tanquam actio (={\scshape habitus} ut praedicamentum)}
  }
  {\schemabox{\\S se habet ad ipsum/ad aliud. (=dispositio, {\scshape qualitas})}}
}


\begin{longtable}{p{21em}p{21em}}


{\scshape Ad primum ergo dicendum} quod obiectio illa procedit de habere
communiter sumpto, sic enim est commune ad multa genera, ut dictum
est.

&

第一異論に対しては、それゆえ以下のように言われるべきである。この異論は
共通的に取られた「持つこと」について論じていて、すでに述べられたとおり、
その意味であれば多くの類に共通する。

\\



{\scshape Ad secundum dicendum} quod ratio illa procedit de habitu
secundum quod intelligitur aliquid medium inter habens et id quod
habetur, sic enim est quoddam praedicamentum, ut dictum est.

&

第二異論に対しては以下のように言われるべきである。この論は、持つものと
持たれるものとの間に何か媒介物が理解される限りにおいてのハビトゥス(習
慣)について論じている。そしてこの意味であれば、すでに述べられたとおり、
それは何らかのカテゴリーである。


\\

 {\scshape Ad tertium dicendum} quod dispositio quidem semper importat
ordinem alicuius habentis partes, sed hoc contingit tripliciter, ut
statim ibidem philosophus subdit, scilicet {\itshape aut secundum
locum, aut secundum potentiam, aut secundum speciem}. {\itshape In
quo}, ut Simplicius dicit in {\itshape Commento Praedicamentorum},
{\itshape comprehendit omnes dispositiones. Corporales quidem, in eo
quod dicit <<secundum locum>>}: et hoc pertinet ad praedicamentum
situs, qui est ordo partium in loco. {\itshape Quod autem dicit
<<secundum potentiam>>, includit illas dispositiones quae sunt in
praeparatione et idoneitate nondum perfecte}, sicut scientia et virtus
inchoata. {\itshape Quod autem dicit <<secundum speciem>>, includit
perfectas dispositiones, quae dicuntur habitus}, sicut scientia et
virtus complete.

&

第三異論に対しては以下のように言われるべきである。傾向性・態勢はたしか
に常に、部分を持つ何らかのものの秩序を含意するが、すぐ後に哲学者が述べ
るように、これは三通りに生じる。すなわち、「場所に即して、あるいは能力
に即して、あるいは種に即して」である。そしてシンプリキオスが『カテゴリー
論注解』で言うように「この中に全ての傾向性・態勢を包含する」。すなわち
身体的なものは「場所に即して」と言うものの中にこれを包含し、そしてこれ
は場所における部分の秩序である「体位」の範疇に属する。「能力に即して」
と言うことは、始まったばかりの知や徳のように、準備や適切さにおいてまだ
完全でないものを包含する。他方で「種に即して」と言うことは、「ハビトゥ
ス・習慣」と言われる、完全に知や徳であるもののような、完全な傾向性・態
勢を包含する。


\\


\end{longtable}
\newpage


\rhead{a.2}
\begin{center}
{\Large {\bf ARTICULUS SECUNDUS}}\\
{\large UTRUM HABITUS SIT DETERMINATA SPECIES QUALITATIS}\\
{\footnotesize {\itshape De Virtut.}, q.1, a.1.}\\
{\Large 第二項\\習慣は性質の限定された種か}
\end{center}

\begin{longtable}{p{21em}p{21em}}

{\scshape Ad secundum sic proceditur}. Videtur quod habitus non sit determinata
species qualitatis. Quia, ut dictum est, habitus, secundum quod est
qualitas, dicitur {\itshape dispositio secundum quam bene aut male disponitur
dispositum}. Sed hoc contingit secundum quamlibet qualitatem, nam et
secundum figuram contingit aliquid bene vel male esse dispositum, et
similiter secundum calorem et frigus, et secundum omnia
huiusmodi. Ergo habitus non est determinata species qualitatis.

&

第二項の問題へ議論は以下のように進められる。
習慣は性質の限定された種ではないと思われる。理由は以下の通り。
すでに述べられたとおり、性質であるかぎりにおける習慣は「それにしたがっ
て態勢がよくまたは悪く態勢付けられるところの態勢」である。
しかしこのことはどのような性質においても生じる。すなわち、形態において
も何かが良くあるいは悪く態勢付けられるし、同様に熱さと冷たさにおいても、
その他そのようなすべてのものにおいてもそうである。
ゆえに習慣は性質の限定された種ではない。



\\



2.~{\scshape Praeterea}, philosophus, in {\itshape Praedicamentis}, caliditatem et frigiditatem
dicit esse dispositiones vel habitus, sicut aegritudinem et
sanitatem. Sed calor et frigus sunt in tertia specie qualitatis. Ergo
habitus vel dispositio non distinguuntur ab aliis speciebus
qualitatis.

&

さらに、哲学者は『カテゴリー論』の中で、熱と冷を、病気と健康と同じよう
に、態勢ないし習慣であると言っている。しかし熱さと冷たさは、性質の第三
の種である。ゆえに習慣ないし態勢は性質の他の種から区別されない。

\\



3.~{\scshape Praeterea}, {\itshape difficile mobile} non est differentia pertinens ad genus
qualitatis, sed magis pertinet ad motum vel passionem. Nullum autem
genus determinatur ad speciem per differentiam alterius generis; sed
oportet differentias per se advenire generi, ut philosophus dicit, in
VII {\itshape Metaphys}. Ergo, cum habitus dicatur esse {\itshape qualitas difficile
mobilis}, videtur quod non sit determinata species qualitatis.

&

さらに、「変わりにくい」は性質の類に属する種差ではなく、むしろ運動ない
し受動に属する。しかるに、どんな類も、他の類に属する種差によって種に限
定されることはない。そうではなく、哲学者が『形而上学』第7巻で述べるよ
うに、自体的に類に到来する種差によってなければならない。ゆえに、習慣は
「変わりにくい性質」であると言われるのだから、性質の限定された種ではな
いと思われる。

\\



{\scshape Sed contra} est quod philosophus dicit, in {\itshape Praedicamentis}, quod {\itshape una
species qualitatis est habitus et dispositio}.

&

しかし反対に、哲学者は『カテゴリー論』の中で「性質の一つの種が、習慣と
態勢である」と述べている。

\\



{\scshape Respondeo dicendum} quod philosophus, in {\itshape Praedicamentis}, ponit inter
quatuor species qualitatis primam, dispositionem et habitum. Quarum
quidem specierum differentias sic assignat Simplicius, in {\itshape Commento
Praedicamentorum}, dicens quod {\itshape qualitatum quaedam sunt naturales, quae
secundum naturam insunt, et semper, quaedam autem sunt adventitiae,
quae ab extrinseco efficiuntur, et possunt amitti. Et haec quidem,
quae sunt adventitiae, sunt habitus et dispositiones, secundum facile
et difficile amissibile differentes}. 


&

解答する。以下のように言われるべきである。哲学者は『カテゴリー論』で、
性質の四つの種の中で第一のものとして態勢と習慣を置いている。シンプリキ
オスはこれらの種の違いを割り当てて『カテゴリー論注解』で以下のように言
う。「性質のうちのあるものは自然的なものであり、本性にしたがって内在し、
そして常にある。他方あるものは到来するものであり、外のものによって作り
出され、失われることが可能である。そしてこれ(すなわち到来するもの)
が、習慣と態勢であり、これらは失われやすさと失われにくさによって異なる。



\\



{\itshape Naturalium autem qualitatum
quaedam sunt secundum id quod aliquid est in potentia, et sic est
secunda species qualitatis. Quaedam vero secundum quod aliquid est in
actu, et hoc vel in profundum, vel secundum superficiem. Si in
profundum quidem, sic est tertia species qualitatis, secundum vero
superficiem, est quarta species qualitatis, sicut figura et forma,
quae est figura animati.} 


&

さらに、自然的な性質の中のあるものは、能力の中にあるかぎりでのそれであ
り、この意味で、性質の第二の種がある。他方で、あるものが現実態において
ある限りでの性質があり、これは深くへ達するか、あるいは表面においてある
かである。深くに達するものである場合には、それは性質の第三の種であり、
表面においてある場合には、第四の種である。ちょうど形態と容貌のように。
後者は魂あるものの形態である」。


\\



-- Sed ista distinctio specierum qualitatis
inconveniens videtur. Sunt enim multae figurae et qualitates
passibiles non naturales, sed adventitiae, et multae dispositiones non
adventitiae, sed naturales, sicut sanitas et pulchritudo et
huiusmodi. 
Et praeterea hoc non convenit ordini specierum, semper enim
quod naturalius est, prius est. 


&

しかし性質の種のこの区別はうまくいっていないと思われる。なぜなら、多く
の形態が自然的でなく受動的な性質であるし、多くの態勢が、たとえば健康や
美しさなどのように、到来するものではなく自然的だからである。そしてさら
に、これは(性質の種として挙げられる)種の順序に適合しない。なぜなら、
自然的なものほど先にくるのが常だからである。


\\


Et ideo aliter accipienda est
distinctio dispositionum et habituum ab aliis qualitatibus. Proprie
enim qualitas importat quendam modum substantiae. Modus autem est, ut
dicit Augustinus, {\itshape super Gen.~ad litteram}, {\itshape quem mensura praefigit}, unde
importat quandam determinationem secundum aliquam mensuram. 


&

ゆえに態勢と習慣の他の性質からの区別は別様に理解されるべきである。すな
わち、性質は、それに固有の意味で、実体のある様態を意味する。しかるに様
態は、アウグスティヌスが『創世記逐語注解』で言うように、「尺度があらかじ
めそれを定める」ものである。このことから、様態は、ある尺度におけるなんら
かの限定を意味する。

\\


Et ideo
sicut id secundum quod determinatur potentia materiae secundum esse
substantiale dicitur qualitas quae est differentia substantiae; ita id
secundum quod determinatur potentia subiecti secundum esse
accidentale, dicitur qualitas accidentalis, quae est etiam quaedam
differentia, ut patet per philosophum in V {\itshape Metaphys}. 


&

それゆえ、ちょうど、それにしたがって質料の能力が実体的存在において限定
されるところのものが実体の差異である性質と言われるように、それにしたがっ
て基体の能力が附帯的存在において限定されるところのものが、附帯的な性質
と言われる。そして、『形而上学』第5巻の哲学者によって明らかなとおり、
これもまた何らかの差異である。


\\



Modus autem sive
determinatio subiecti secundum esse accidentale, potest accipi vel in
ordine ad ipsam naturam subiecti; vel secundum actionem et passionem
quae consequuntur principia naturae, quae sunt materia et forma; vel
secundum quantitatem. 


&

さて、附帯的な存在における基体の様態ないし限定は、基体の本性それ自体へ
の秩序においてか、あるいは、質料と形相である本性の諸根源から帰結する能
動と受動において、あるいは、量において、理解されうる。


\\



Si autem accipiatur modus vel determinatio
subiecti secundum quantitatem, sic est quarta species qualitatis. Et
quia quantitas, secundum sui rationem, est sine motu, et sine ratione
boni et mali; ideo ad quartam speciem qualitatis non pertinet quod
aliquid sit bene vel male, cito vel tarde transiens. 



&


基体の様態や限定が量に即して理解されるならば、その場合には、性質の第四
の種がある。そして量は、その性格上、運動の性格や善悪の性格をもたないの
で、性質の第四の種には、あるものが善くあるいは悪くあるとか、早くあるい
は遅く移り変わることは適合しない。


\\


Modus autem sive
determinatio subiecti secundum actionem et passionem, attenditur in
secunda et tertia specie qualitatis. Et ideo in utraque consideratur
quod aliquid facile vel difficile fiat, vel quod sit cito transiens
aut diuturnum. Non autem consideratur in his aliquid pertinens ad
rationem boni vel mali, quia motus et passiones non habent rationem
finis, bonum autem et malum dicitur per respectum ad finem. 


&

他方、能動と受動に即した基体の様態や限定は、性質の第二と第三の種におい
て見出される。それゆえ、どちらにおいても、あることが容易にあるいは困難
を伴って生じるとか、あるいは、早く変化するとか長く続くとかいうことが考
察される。しかしこらにおいて、善悪の性格に属することは何も考察されない。
なぜなら、運動や受動は目的の性格を持たないが、善と悪は目的への関係によっ
て語られるからである。


\\


Sed modus et determinatio subiecti in ordine ad naturam rei, pertinet
ad primam speciem qualitatis, quae est habitus et dispositio, dicit
enim philosophus, in VII {\itshape Physic}., loquens de habitibus animae et
corporis, quod sunt {\itshape dispositiones quaedam perfecti ad optimum; dico
autem perfecti, quod est dispositum secundum naturam}.


&

しかし、事物の本性への秩序における基体の様態や限定は、性質の第一の種に
属し、それが習慣と態勢である。実際哲学者は、『自然学』第7巻で、魂と身
体の習慣について語る中で、それらが完成されたものの最善のものへの態勢だ
と述べている。私が「完成されたものの」と言うのは、本性にしたがって態勢
付けられたもののことである。


\\


Et quia ipsa forma et
natura rei est finis et cuius causa fit aliquid, ut dicitur in II
{\itshape Physic}. ideo in prima specie consideratur et bonum et malum; et etiam
facile et difficile mobile, secundum quod aliqua natura est finis
generationis et motus. Unde in V {\itshape Metaphys}. philosophus definit
habitum, quod est {\itshape dispositio secundum quam aliquis disponitur bene vel
male}. Et in II {\itshape Ethic}. dicit quod {\itshape habitus sunt secundum quos ad
passiones nos habemus bene vel male}. 



&

そして、『自然学』第2巻で述べられるように、事物の形相と本性は目的であ
り、それのために何かが生じるところのものなので、この第一の種においては
善悪が考察される。そしてまた、ある本性が生成や運動の目的である限りにお
いて、容易であるとか、変わりにくいとかいうことが考察される。このことか
ら『形而上学』第5巻で、哲学者は習慣を「それに従って何かがよくあるいは
悪く態勢付けられるところの態勢」と定義する。また、『ニコマコス倫理学』
第2巻で、習慣は、「それに従って、私たちがよくあるいは悪く情念へ関係す
るところのもの」と述べている。

\\


Quando enim est modus conveniens
naturae rei, tunc habet rationem boni, quando autem non convenit, tunc
habet rationem mali. Et quia natura est id quod primum consideratur in
re, ideo habitus ponitur prima species qualitatis.

&

なぜなら、様態が事物の本性に適合するときに善の性格を持ち、適合しないと
きに悪の性格を持つからである。そして本性は事物において第一に考察される
ものなので、習慣は性質の第一の種に置かれる。

\\



{\scshape Ad primum ergo dicendum} quod dispositio ordinem quendam importat, ut
dictum est. Unde non dicitur aliquis disponi per qualitatem, nisi in
ordine ad aliquid. Et si addatur bene vel male, quod pertinet ad
rationem habitus, oportet quod attendatur ordo ad naturam, quae est
finis. Unde secundum figuram, vel secundum calorem vel frigus, non
dicitur aliquis disponi bene vel male, nisi secundum ordinem ad
naturam rei, secundum quod est conveniens vel non conveniens. 


&

第一異論に対しては、それゆえ、以下のように言われるべきである。すでに述
べられたとおり、態勢はなんらかの秩序を意味する。このことから、何かへの
秩序においてでなければ、何かが性質によって態勢付けられるとは言われない。
そしてもし、習慣の性格に属する「よくあるいは悪く」が加えられるならば、
目的であるところの本性への秩序が見出される。したがって、形態において、
あるいは熱と冷において、それが適合的かそうでないかに即して、事物の本性
への秩序においてでなければ、誰かがよくあるいは悪く態勢付けられるとは言
われない。


\\


Unde et
ipsae figurae et passibiles qualitates, secundum quod considerantur ut
convenientes vel non convenientes naturae rei, pertinent ad habitus
vel dispositiones, nam figura, prout convenit naturae rei, et color,
pertinent ad pulchritudinem; calor autem et frigus, secundum quod
conveniunt naturae rei, pertinent ad sanitatem. Et hoc modo caliditas
et frigiditas ponuntur a philosopho in prima specie qualitatis.

&

したがって、この形態や受動的性質自体もまた、事物の本性に適合的かあるい
はそうでないかに即するならば、習慣や態勢に属する。たとえば形態は事物の
本性に適合するものとして、色は美しさに適合するものとして(そのようなも
のである)。また熱と冷は事物の本性に適合する限りにおいて健康に属する。
この意味で、熱さと冷たさは、哲学者によって、性質の第一の種の中に置かれ
ている。

\\



Unde patet solutio ad secundum. Licet a quibusdam aliter solvatur, ut
Simplicius dicit, in {\itshape Commento Praedicamentorum}.

&

このことから、第二異論への解答は明らかである。ただし、シンプリキオスが
『カテゴリー論注解』の中で言うように、ある人々によっては別のしかたで解
答されている。

\\



{\scshape Ad tertium dicendum} quod ista differentia, difficile mobile, non
diversificat habitum ab aliis speciebus qualitatis, sed a
dispositione. Dispositio autem dupliciter accipitur, uno modo,
secundum quod est genus habitus, nam in V Metaphys. dispositio ponitur
in definitione habitus; alio modo, secundum quod est aliquid contra
habitum divisum. Et potest intelligi dispositio proprie dicta
condividi contra habitum, dupliciter. 


&

第三異論に対しては以下のように言われるべきである。「変わりにくい」とい
うこの差異は、習慣を性質の他の種から分けるのではなく、態勢から分ける。
ところで態勢は二通りに理解できる。一つには、習慣の類である限りにおいて
であり、というのも『形而上学』第5巻で、態勢が習慣の定義の中に置かれて
いるからである。もう一つには、習慣に反対して分けられる何かとしてである。
そして、固有の意味で習慣に反対して分けられると言われる態勢は二通りに理
解できる。

\\


Uno modo, sicut perfectum et
imperfectum in eadem specie, ut scilicet dispositio dicatur, retinens
nomen commune, quando imperfecte inest, ita quod de facile amittitur;
habitus autem, quando perfecte inest, ut non de facili amittatur. Et
sic dispositio fit habitus, sicut puer fit vir. 


&

一つには、同一の種における完全なものと不完全なものとしてであり、それは
つまり、不完全に内在し、したがって失われやすいときに、共通の名前を保持
しつつ態勢と言われ、他方、完全に内在し、したがって変わりにくいときに習
慣と言われる。この意味では、態勢は、ちょうど少年が大人になるように、習
慣になる。

\\

Alio modo possunt
distingui sicut diversae species unius generis subalterni, ut dicantur
dispositiones illae qualitates primae speciei, quibus convenit
secundum propriam rationem ut de facili amittantur, quia habent causas
transmutabiles, ut aegritudo et sanitas; habitus vero dicuntur illae
qualitates quae secundum suam rationem habent quod non de facili
transmutentur, quia habent causas immobiles, sicut scientiae et
virtutes. 


&

もう一つのしかたでは、(態勢と習慣は)一つの類に属する従属する異なる種
として区別されうる。この場合、この態勢は第一の種に属する性質だと言われ、
病気と健康のように、変化しうる原因をもつがゆえに、固有の性格に即してそ
れに容易に失われるという性格が属する。他方で習慣と言われるのは、知識や
徳のように、不動の原因をもつがゆえに、自らの性格において変わりにくいと
いうことをもつ性質である。

\\


Et secundum hoc dispositio non fit habitus. Et hoc videtur
magis consonum intentioni Aristotelis. Unde ad huius distinctionis
probationem inducit communem loquendi consuetudinem, secundum quam
qualitates quae secundum rationem suam sunt facile mobiles, si ex
aliquo accidenti difficile mobiles reddantur, habitus dicuntur, et e
converso est de qualitatibus quae secundum suam rationem sunt
difficile mobiles; nam si aliquis imperfecte habeat scientiam, ut de
facili possit ipsam amittere, magis dicitur disponi ad scientiam quam
scientiam habere. 
Ex quo patet quod nomen habitus diuturnitatem
quandam importat; non autem nomen dispositionis. 


&

そしてこちらの意味では、態勢は習慣でない。そしてこちらの方が、よりアリ
ストテレスの意図と調和すると思われる。したがって、この区別を証明するた
めに、彼は共通の習慣的な言い方を導入して、それによれば、自らの性格にお
いて変化しやすい性質は、もし何らかの偶然によって変化しにくくなると、習
慣と言われ、またそれ自身の性格において変化しにくい性質についてもまた逆
のことが言われる。すなわち、もしある人が不完全に知識をもち、容易にその
知識を失うならば、彼は知識をもっているのではなく、むしろ知識へと態勢付
けられていると言われる。
このことから、習慣という名称は、なんらかの長く続くことを含意するが、態
勢という名称はそうでないことが明らかである。

\\




&


\\

Nec impeditur quin
secundum hoc facile et difficile mobile sint specificae differentiae,
propter hoc quod ista pertinent ad passionem et motum, et non ad genus
qualitatis. 
Nam istae differentiae, quamvis per accidens videantur se
habere ad qualitatem, designant tamen proprias et per se differentias
qualitatum. Sicut etiam in genere substantiae frequenter accipiuntur
differentiae accidentales loco substantialium, inquantum per eas
designantur principia essentialia.

&

また、変化しやすい・変化しにくいということが受動や運動に属し性質の類に
属さないために、これらが種を区別するものでないということも妨げにならな
い。なぜなら、これらの差異は、性質に対して附帯的に関係しているように見
えるけれども、性質の固有で自体的な差異を示しているからである。ちょうど
それは、実体の類においても、本質的な諸根源が示されるかぎりで、附帯的な
差異が実体的なものの場所に理解されることが頻繁にあるのと同様である。

\\
\end{longtable}



\schema
{
  \schemabox{modus sive \\determinatio subiecti \\secundum esse accidentale}
}
{
  \schema
  {
    \schemabox{in ordine ad \\ipsam naturam subiecti\\ --- prima species qualitatis}
  }
  {
    \schemabox{facile mobilis --- dispositio\\difficile mobilis --- habitus}
  }
  \schema
  {\schemabox{sec.~actionem et passionem}}
  {\schemabox{secunda species qualitatis\\tertia species qualitatis}}
  {\schemabox{sec.~quantitatem --- quarta species qualitatis}}
}



\newpage



\rhead{a.~3}
\begin{center}
{\Large {\bf ARTICULUS TERTIUS}}\\
{\large UTRUM HABITUS IMPORTET ORDINEM AD ACTUM}\\
{\Large 第三項\\習慣は働きへの秩序を含意するか}
\end{center}

\begin{longtable}{p{21em}p{21em}}
{\scshape Ad tertium sic proceditur}. Videtur quod habitus non importet ordinem
ad actum. Unumquodque enim agit secundum quod est actu. Sed
philosophus dicit, in III {\itshape de Anima}, quod {\itshape cum aliquis fit sciens
secundum habitum, est etiam tunc in potentia, aliter tamen quam ante
addiscere}. Ergo habitus non importat habitudinem principii ad actum.

&

第三項の問題へ、議論は以下のように進められる。習慣は働きへの秩序を含意
しないと思われる。理由は以下の通り。各々のものは、それが現実態において
ある限りにおいて働く。しかし哲学者は『デ・アニマ』第3巻で「ある人が習
態において知る人になるとき、そのときにはさらに可能態においてある。しか
し学ぶ前とは違うかたちで」と述べている。ゆえに習慣は根源の働きへの関係
を含意しない。

\\



2.~{\scshape Praeterea}, illud quod ponitur in definitione alicuius, per se convenit
illi. Sed esse principium actionis ponitur in definitione potentiae;
ut patet in V {\itshape Metaphys}. Ergo esse principium actus per se convenit
potentiae. Quod autem est per se, est primum in unoquoque genere. Si
ergo etiam habitus sit principium actus, sequitur quod sit posterior
quam potentia. Et sic non erit prima species qualitatis habitus vel
dispositio.

&

第二に、あるものの定義の中に置かれるものは、それに自体的に適合する。し
かるに、『形而上学』第5巻で明らかなとおり、働きの根源であることは、能
力の定義の中に置かれる。ゆえに働きの根源であることは自体的に能力に適合
する。しかるに、自体的にそうであるものは、各々の類において第一のもので
ある。ゆえにもし、習慣もまた働きの根源であるならば、それは能力よりも後
であることが帰結する。このようにして、習慣あるいは態勢は、性質の第一の
種ではないことになるだろう。


\\



3.~{\scshape Praeterea}, sanitas quandoque est habitus, et similiter macies et
pulchritudo. Sed ista non dicuntur per ordinem ad actum. Non ergo est
de ratione habitus quod sit principium actus.

&

さらに、健康は、ときどき、習慣である。同様に、弱さや美しさもまたそうである。
しかるにこれらは働きへの秩序によっては言われない。
ゆえに働きの根源であることは習慣の性格に属さない。

\\



{\scshape Sed contra est} quod Augustinus dicit, in libro {\itshape de Bono Coniugali}, quod
{\itshape habitus est quo aliquid agitur cum opus est}. Et Commentator dicit, in
III {\itshape de Anima}, quod {\itshape habitus est quo quis agit cum voluerit}.

&

しかし反対に、アウグスティヌスは『婚姻の善について』という書物の中で
「習慣とはそれによってあるものが必要なときに働くものである」と述べてい
る。また注解者は『デ・アニマ』の中で「習慣は、欲するときにそれによって
働くところのものである」と述べている。



\\



{\scshape Respondeo dicendum} quod habere ordinem ad actum potest competere
habitui et secundum rationem habitus; et secundum rationem subiecti in
quo est habitus. Secundum quidem rationem habitus, convenit omni
habitui aliquo modo habere ordinem ad actum. Est enim de ratione
habitus ut importet habitudinem quandam in ordine ad naturam rei,
secundum quod convenit vel non convenit. 



&

解答する。以下のように言われるべきである。
働きへの秩序を持つことは、習慣の性格に即して、また、習慣がその中にある基体の性格に即して、習慣に適合しうる。
実際、習慣の性格に即して言えば、働きへの秩序を持つことは全ての習慣に何らかのしかたで適合する。
なぜなら習慣の性格には、適合するかしないかに即して、事物の本性への秩序における何らかの関係を意味することが属するからである。


\\

Sed natura rei, quae est
finis generationis, ulterius etiam ordinatur ad alium finem, qui vel
est operatio, vel aliquod operatum, ad quod quis pervenit per
operationem. Unde habitus non solum importat ordinem ad ipsam naturam
rei, sed etiam consequenter ad operationem, inquantum est finis
naturae, vel perducens ad finem. Unde et in V {\itshape Metaphys}. dicitur in
definitione habitus, quod est {\itshape dispositio secundum quam bene vel male
disponitur dispositum aut secundum se}, idest secundum suam naturam,
{\itshape aut ad aliud}, idest in ordine ad finem. 



&

しかるに生成の目的である事物の本性は、さらに他の目的へと秩序づけられる。
それは働きだったり、その働きを通して人が到達する働きの結果だったりする。
したがって習慣は事物の本性自体への秩序を含意するだけでなく、結果的に、
本性の目的である限りにおける、あるいは目的へと導くものである限りにおけ
る働きへの秩序をも含意する。このことから『形而上学』第5巻で、習慣の定
義の中で「それにしたがって、それ自体において(すなわち自らの本性におい
て)、または他のものに対して(すなわち目的への秩序において)よくあるい
は悪く態勢付けられたものが態勢付けられるところのもの」と述べられている。


\\


Sed sunt quidam habitus qui
etiam ex parte subiecti in quo sunt, primo et principaliter important
ordinem ad actum. Quia ut dictum est, habitus primo et per se importat
habitudinem ad naturam rei. Si igitur natura rei in qua est habitus,
consistat in ipso ordine ad actum, sequitur quod habitus principaliter
importet ordinem ad actum. Manifestum est autem quod natura et ratio
potentiae est ut sit principium actus. Unde omnis habitus qui est
alicuius potentiae ut subiecti, principaliter importat ordinem ad
actum.

&

しかし、習慣がそこにおいて存在するところの基体の側からも、第一にそして
主要に、働きへの秩序を含意するような習慣がある。なぜなら、すでに述べら
れたとおり、習慣は第一にそして自体的に事物の本性への秩序を意味する。ゆ
えにもし習慣がその中に存在するところの事物の本性が、働きへの秩序自体に
おいて成立するならば、その習慣は主要に働きへの秩序を意味することになる。
しかるに、能力の本性と性格は働きの根源であることである。したがって、全
て何らかの能力を基体としてそれに属する習慣は、主要に、働きへの秩序を意
味する。

\\



{\scshape Ad primum} ergo dicendum quod habitus est actus quidam, inquantum est
qualitas, et secundum hoc potest esse principium operationis. Sed est
in potentia per respectum ad operationem. Unde habitus dicitur {\itshape actus
primus}, et operatio {\itshape actus secundus}; ut patet in II {\itshape de Anima}.

&

第一異論に対しては、それゆえ、以下のように言われるべきである。習慣は、
性質である限りにおいてなんらかの現実態であり、その限りにおいて働き(operatio)の根
源でありうる。しかし、働きに対しては可能態においてある。したがって、習
態は「第一現実態」、働きは「第二現実態」と言われる。これは『デ・アニマ』
第2巻で明らかである。

\\



{\scshape Ad secundum dicendum} quod non est de ratione habitus quod respiciat
potentiam, sed quod respiciat naturam. Et quia natura praecedit
actionem, quam respicit potentia; ideo prior species qualitatis
ponitur habitus quam potentia.

&

第二異論に対しては以下のように言われるべきである。習慣の性格には可能態
に関係することは含まれず、本性に関係することが含まれる。そして、本性は
働きに先行し、働きは能力に関係するので、習慣は、能力よりも先に性質の種
として措定される。

\\



{\scshape Ad tertium dicendum} quod sanitas dicitur habitus, vel habitualis
dispositio, in ordine ad naturam, sicut dictum est. Inquantum tamen
natura est principium actus, ex consequenti importat ordinem ad
actum. Unde philosophus dicit, in X {\itshape de historia Animal}., quod homo
dicitur sanus, vel membrum aliquod, {\itshape quando potest facere operationem
sani}. Et est simile in aliis.

&

第三異論に対しては以下のように言われるべきである。すでに述べられたとお
り、健康が習慣あるいは習慣的な態勢と呼ばれるのは本性への秩序に即してで
ある。しかし本性が働きの根源である限りにおいては、結果的に働きへの秩序
を意味する。このことから哲学者は『動物誌』第10缶の中で、人間やその四肢
のどれかが健康と言われるのは、「健康な人の働きを行うことができるとき」
であると述べている。そしてこれはその他のものにおいても同様である。

\\

\end{longtable}
\newpage


\rhead{a.4}
\begin{center}
{\Large {\bf ARTICULUS QUARTUS}}\\
{\large UTRUM SIT NECESSARIUM ESSE HABITUM}\\
{\footnotesize III {\itshape Sent.}, d.23, q.1, a.1; {\itshape De
 Verit.}, q.20, a.2; {\itshape De Virtut.}, q.1, a.1.}\\
{\Large 第四項\\習慣があることは必要か}
\end{center}

\begin{longtable}{p{21em}p{21em}}
{\scshape Ad quartum sic proceditur}. Videtur quod non sit necessarium esse
 habitus. Habitus enim sunt quibus aliquid disponitur bene vel male ad
 aliquid, sicut dictum est. Sed per suam formam aliquid bene vel male
 disponitur, nam secundum formam aliquid est bonum, sicut et ens. Ergo
 nulla necessitas est habituum.

&

第四項の問題へ、議論は以下のように進められる。
習慣があることは必要でないと思われる。理由は以下の通り。
すでに述べられたとおり、習慣とは何か或るものがよくあるいは悪く何かへと態勢付けられるところのものである。
しかし、ものは自らの形相によってよくあるいは悪く態勢付けられる。
なぜなら、ものは形相にしたがって善いもの、すなわち有だからである。
ゆえに習慣にはいかなる必要性も属さない。

\\



2.~{\scshape Praeterea}, habitus importat ordinem ad actum. Sed potentia importat
 principium actus sufficienter, nam et potentiae naturales absque
 habitibus sunt principia actuum. Ergo non fuit necessarium habitus
 esse.

&

さらに、習慣は働きへの秩序を意味する。
しかるに能力は、十分に、働きの根源を意味する。
なぜなら自然的な能力は習慣なしに働きの根源だからである。
ゆえに習慣があることは必要なかった。

\\



3.~{\scshape Praeterea}, sicut potentia se habet ad bonum et malum, ita et habitus,
 et sicut potentia non semper agit, ita nec habitus. Existentibus
 igitur potentiis, superfluum fuit habitum esse.

&

さらに、能力が善と悪に関係するように、習慣も善悪に関係する。
また、能力が常に働かないように、習慣も常に働かない。
ゆえに、能力があれば、習慣があることは余計なことであった。

\\



{\scshape Sed contra est} quod habitus sunt perfectiones quaedam, ut dicitur in
 VII Physic. Sed perfectio est maxime necessaria rei, cum habeat
 rationem finis. Ergo necessarium fuit habitus esse.

&

しかし反対に、『自然学』第7巻で言われているように、習慣はある種の完成である。
しかるに完成は、目的の性格を持つので、最大限に事物にとって必要なものである。
ゆえに習慣があることは必要であった。

\\



{\scshape Respondeo dicendum} quod, sicut supra dictum est, habitus importat
dispositionem quandam in ordine ad naturam rei, et ad operationem vel
finem eius, secundum quam bene vel male aliquid ad hoc disponitur. Ad
hoc autem quod aliquid indigeat disponi ad alterum, tria
requiruntur. 

&

解答する。以下のように言われるべきである。
前に述べられたとおり、習慣は、事物の本性、働き、目的への秩序において、或るものがそれにしたがってよくあるいは悪くそれへと態勢付けられる限りでの態勢を意味する。
ところで、ものが別のものへ態勢付けられる必要があるために、三つのことが必要とされる。

\\

Primo quidem, ut id quod disponitur, sit alterum ab eo ad
quod disponitur; et sic se habeat ad ipsum ut potentia ad actum. Unde
si aliquid sit cuius natura non sit composita ex potentia et actu, et
cuius substantia sit sua operatio, et ipsum sit propter seipsum; ibi
habitus vel dispositio locum non habet, sicut patet in Deo. 

&

第一に、態勢付けられるもの(A)が、それへと態勢付けられるもの(B)と異なっ
ていることであり、その意味で、前者(A)は後者(B)に対して、可能態が現実態
に対するように関係する。このため、もし、そのものの本性が可能態と現実態
から複合されてなく、その本性が働きであり、それがそれ自身のために在るよ
うなものがあるならば、そこに習慣や態勢の場所はない。これは神において明
らかであるように。

\\


Secundo
requiritur quod id quod est in potentia ad alterum, possit pluribus
modis determinari, et ad diversa. Unde si aliquid sit in potentia ad
alterum, ita tamen quod non sit in potentia nisi ad ipsum, ibi
dispositio et habitus locum non habet, quia tale subiectum ex sua
natura habet debitam habitudinem ad talem actum. 

&

第二に、別のものに対して可能態にあるものが、複数のしかたで様々なものへ
限定されうることが必要である。このことから、もし、他のもの(C)に対して
可能態にあるが、しかしそれ(C)に対してしか可能態にないようなものがある
ならば、そこにも習慣や態勢の場所はない。なぜならそのような基体は、自ら
の本性に基づいて、そのような現実態へのしかるべき関係をもつからである。


\\

Unde si corpus
caeleste sit compositum ex materia et forma, cum illa materia non sit
in potentia ad aliam formam, ut in primo dictum est, non habet ibi
locum dispositio vel habitus ad formam; aut etiam ad operationem, quia
natura caelestis corporis non est in potentia nisi ad unum motum
determinatum. 

&

したがって、もし天体が質料と形相から複合されているならば、第一部で述べ
られたように、その質料は別の形相への可能態にはないので、そこでも形相へ
の態勢や習慣の場所がない。あるいは働きに対する耐性や習慣もない。なぜな
ら、天体の本性は一つの限定された運動に対してしか可能態にないからである。

\\

Tertio requiritur quod plura concurrant ad disponendum
subiectum ad unum eorum ad quae est in potentia, quae diversis modis
commensurari possunt, ut sic disponatur bene vel male ad formam vel ad
operationem. 

&

第三に、基体を、基体がそれらに対して可能態にあるものどものうちのひとつ
へと態勢付けるために、複数のものが集まることが必要である。(基体がそれ
らに対して可能態にあるものどもは)さまざまなしかたで共有されうるので、
形相や働きに対して、よくあるいは悪く態勢付けられる。

\\

Unde qualitates simplices elementorum, quae secundum unum
modum determinatum naturis elementorum conveniunt, non dicimus
dispositiones vel habitus, sed simplices qualitates, dicimus autem
dispositiones vel habitus sanitatem, pulchritudinem et alia huiusmodi,
quae important quandam commensurationem plurium quae diversis modis
commensurari possunt. 

&

このため、ある元素の単純な性質は、一つの限定されたしかたでその元素の本
性に適合するので、私たちはそれを態勢や習慣とは呼ばず、「単純な性質」と
呼ぶ。これに対して、健康や美しさや、その他そのようなものを、私たちは態
勢や習慣と呼ぶが、それはそれらが、さまざまなしかたで共有されうる複数の
ものの共有を意味するからである。

\\

Propter quod philosophus dicit, in V {\itshape Metaphys}.,
quod {\itshape habitus est dispositio}, et dispositio est {\itshape ordo habentis partes
vel secundum locum, vel secundum potentiam, vel secundum speciem}; ut
supra dictum est. 

&

このために哲学者は、すでに述べられたとおり、『形而上学』第5巻で「習慣
は態勢であり」そして態勢は「場所において、あるいは能力において、あるい
は種において部分を持つものの秩序である」と述べている。

\\

Quia igitur multa sunt entium ad quorum naturas et
operationes necesse est plura concurrere quae diversis modis
commensurari possunt, ideo necesse est habitus esse.

&

ゆえに、存在するものの中の多くのものが、その本性と働きのために、さまざ
まなしかたで共有されうる複数のものが集まる必要があるので、習慣が存在す
ることは必要である。

\\


{\scshape Ad primum ergo dicendum} quod per formam perficitur natura rei, sed
oportet quod in ordine ad ipsam formam disponatur subiectum aliqua
dispositione. -- Ipsa tamen forma ordinatur ulterius ad operationem, quae
vel est finis, vel via in finem. Et si quidem habeat forma determinate
unam tantum operationem determinatam, nulla alia dispositio requiritur
ad operationem praeter ipsam formam. Si autem sit talis forma quae
possit diversimode operari, sicut est anima; oportet quod disponatur
ad suas operationes per aliquos habitus.

&

第一異論に対しては、それゆえ、以下のように言われるべきである。事物の本
性は形相によって完成される。しかし基体は、その形相に対する秩序において、
なんらかの態勢によって態勢付けられなければならない。しかし、その形相は
さらに働きへと秩序づけられる。その働きは目的である場合もあれば、目的へ
の途である場合もある。そしてもしかりに、形相が限定的にただ一つの限定さ
れた働きしか持たないならば、その形相以外に、働きのための他の態勢が必要
とされることはない。しかしもし、魂のようにさまざまなしかたで働きうる形
相であるならば、自らの働きへ、何らかの習慣によって態勢付けられなければ
ならない。

\\



{\scshape Ad secundum dicendum} quod potentia
quandoque se habet ad multa, et ideo oportet quod aliquo alio
determinetur. Si vero sit aliqua potentia quae non se habeat ad multa,
non indiget habitu determinante, ut dictum est. Et propter hoc vires
naturales non agunt operationes suas mediantibus aliquibus habitibus,
quia secundum seipsas sunt determinatae ad unum.

&

第二異論に対しては以下のように言われるべきである。能力は、多くのものへ
関係することがある。それゆえ、他の何かによって限定されなければならない。
他方もし、多くのものへ関係しない能力があれば、すでに述べられたとおり、
習慣によって限定される必要はない。このため、自然本性的な力は、何らの媒
介する習慣なしに自らの働きを行う。なぜなら、それらはそれら自身において
一つのものへ限定されているからである。

\\



{\scshape Ad tertium dicendum} quod non idem habitus se habet ad bonum et malum,
sicut infra patebit. Eadem autem potentia se habet ad bonum et
malum. Et ideo necessarii sunt habitus ut potentiae determinentur ad
bonum.

&

第三異論に対しては以下のように言われるべきである。後で明らかになるが、
同一の習慣が善と悪に関係するのではない。しかるに同一の能力が善と悪に関
係する。ゆえに、能力が善へ限定されるために、習慣が必要である。


\end{longtable}

\end{document}