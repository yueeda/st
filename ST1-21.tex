\documentclass[10pt]{jsarticle} % use larger type; default would be 10pt
%\usepackage[utf8]{inputenc} % set input encoding (not needed with XeLaTeX)
%\usepackage[round,comma,authoryear]{natbib}
%\usepackage{nruby}
\usepackage{okumacro}
\usepackage{longtable}
%\usepqckage{tablefootnote}
\usepackage[polutonikogreek,english,japanese]{babel}
%\usepackage{amsmath}
\usepackage{latexsym}
\usepackage{color}

%----- header -------
\usepackage{fancyhdr}
\lhead{{\it Summa Theologiae} I, q.~21}
%--------------------

\bibliographystyle{jplain}

\title{{\bf PRIMA PARS}\\{\HUGE Summae Theologiae}\\Sancti Thomae
Aquinatis\\{\sffamily QUEAESTIO VIGESIMAPRIMA}\\DE IUSTITIA ET
MISERICORDIA DEI}
\author{Japanese translation\\by Yoshinori {\sc Ueeda}}
\date{Last modified \today}


%%%% コピペ用
%\rhead{a.~}
%\begin{center}
% {\Large {\bf }}\\
% {\large }\\
% {\footnotesize }\\
% {\Large \\}
%\end{center}
%
%\begin{longtable}{p{21em}p{21em}}
%
%&
%
%
%\\
%\end{longtable}
%\newpage



\begin{document}
\maketitle
\pagestyle{fancy}

\begin{center}
{\Large 第二十一問\\神の正義と憐れみについて}
\end{center}


\begin{longtable}{p{21em}p{21em}}

Post considerationem divini amoris, de iustitia et misericordia eius
 agendum est. Et circa hoc quaeruntur quatuor. 
\begin{enumerate}
 \item utrum in Deo sit iustitia.
 \item utrum iustitia eius veritas dici possit.
 \item utrum in Deo sit misericordia.
 \item utrum in omni opere Dei sit iustitia et misericordia.
\end{enumerate}

&

神の愛の考察のあとに、神の正義と憐れみについての考察が為されるべきである。
これをめぐっては、四つのことが問われる。
\begin{enumerate}
 \item 神の中に正義があるか。
 \item 神の正義は真理と言われうるか。
 \item 神の中に憐れみがあるか。
 \item 神のすべての業の中に、正義と憐れみがあるか。
\end{enumerate}
\end{longtable}


\newpage

{\rhead{a.~1}
\begin{center}
 {\Large {\bf ARTICULUS PRIMUS}}\\
 {\large UTRUM IN DEO SIT IUSTITIA}\\
 {\footnotesize IV {\itshape Sent.}, dist.46, q.1, a.1, qu$^a$1; I
 {\itshape SCG} cap.93, {\itshape De Div.~Nom.}, cap.8, lect.4.}\\
 {\Large 第一項\\神の中に正義があるか}
\end{center}

\begin{longtable}{p{21em}p{21em}}

{\Huge A}{\scshape d primum sic proceditur}. Videtur quod in
Deo non sit iustitia. Iustitia enim contra temperantiam
dividitur. Temperantia autem non est in Deo. Ergo nec iustitia.

&

第一の問題へ、議論は以下のように進められる。
神の中に正義はないと思われる。理由は以下の通り。
正義は節制に対立して分けられる\footnote{「知慮」「勇気」「節制」「正義」
 は、相互に区別され、四つの枢要徳(virtutes cardinales)と呼ばれる。}。
ところが、神の中に節制はない。
ゆえに、正義もない。


\\


{\scshape 2 Praeterea}, quicumque facit omnia pro libito
suae voluntatis, non secundum iustitiam operatur. Sed, sicut dicit
apostolus, {\itshape ad Ephes}.~{\scshape i}, Deus {\itshape operatur omnia secundum consilium suae
voluntatis}. Non ergo ei iustitia debet attribui.

&

さらに、すべてを自分の意志の思いのままに行う人はだれであれ、正義にしたがっ
 て働かない。ところが、使徒が『エフェソの信徒への手紙』で言うように、神
 は「すべてを自分の意志の思量にしたがってなす」\footnote{「キリストにお
 いてわたしたちは、御心のままにすべてのことを行われる方の御計画によって
 前もって定められ、約束されたものの相続者とされました。」(1:11)}。
ゆえに、神に正義が帰せられるべきではない。


\\


{\scshape 3 Praeterea}, actus iustitiae est reddere
debitum. Sed Deus nulli est debitor. Ergo Deo non competit iustitia.

&

さらに、正義の働きは、債務を履行することにある。
ところが、神は、だれに対しても債務者でない。
ゆえに、神に正義は適合しない。


\\


{\scshape 4 Praeterea}, quidquid est in Deo, est eius
essentia. Sed hoc non competit iustitiae, dicit enim Boetius, in libro
{\itshape de Hebdomad}., quod {\itshape bonum essentiam, iustum vero actum respicit}. Ergo
iustitia non competit Deo.

&


さらに、何であれ神の中にあるものは、神の本質である。ところが、これ[神の
 本質であること]は、正義に適合しない。なぜなら、ボエティウスが『デ・ヘ
 ブドマディブス』という書物で述べるように、「善は本質に、正義は働きに関
 係する」からである。ゆえに、正義は神に適合しない。

\\


{\scshape Sed contra est} quod dicitur in Psalmo {\scshape x},
{\itshape iustus dominus, et iustitias dilexit}.

&

しかし反対に、『詩編』10章で、「主は正しく正義を愛する」\footnote{「主は
 正しくいまし、恵みの業を愛し、御顔を心のまっすぐな人に向けてくださる。」(11:7)}と言われている。

\\


{\scshape Respondeo dicendum} quod duplex est species
iustitiae. Una, quae consistit in mutua datione et acceptione, ut puta
quae consistit in emptione et venditione, et aliis huiusmodi
communicationibus vel commutationibus. Et haec dicitur a philosopho, in
V {\itshape Ethic}., {\itshape iustitia commutativa}, vel directiva commutationum sive
communicationum. Et haec non competit Deo, quia, ut dicit apostolus,
Rom.~{\scshape xi}, {\itshape quis prior dedit illi, et retribuetur ei?} 

&

解答する。以下のように言われるべきである。正義には二つの種がある。
ひとつは、売買やその他そのような交易や交換のような、相互の授受において成
 り立つ。そしてこの正義は、『ニコマコス倫理学』第五巻で、哲学者によって、
 交換的正義、すなわち、交換や交易を正しく導く正義と言われている。
そして、この正義は神に適合しない。なぜなら、『ローマの信徒への手紙』11章
 で使徒が言うように、「だれが、先に彼に与え、彼に戻してもらうのか」
 \footnote{「だれがまず主に与えて、その報いを受けるであろうか。」(11:35)}。


\\

Alia, quae consistit
in distribuendo, et dicitur {\itshape distributiva iustitia}, secundum quam aliquis
gubernator vel dispensator dat unicuique secundum suam dignitatem. Sicut
igitur ordo congruus familiae, vel cuiuscumque multitudinis gubernatae,
demonstrat huiusmodi iustitiam in gubernante; ita ordo universi, qui
apparet tam in rebus naturalibus quam in rebus voluntariis, demonstrat
Dei iustitiam. Unde dicit Dionysius, {\scshape viii} cap.~{\itshape de Div.~Nom.}, {\itshape oportet
videre in hoc veram Dei esse iustitiam, quod omnibus tribuit propria,
secundum uniuscuiusque existentium dignitatem; et uniuscuiusque naturam
in proprio salvat ordine et virtute}.

&

もう一つの正義は、分配において成り立つので、分配的正義と言われる。それに
 したがって、統治者や管理者が、各々の者に、そのふさわしさに応じたものを与える。
それゆえ、ちょうど、家族や何であれ統治された多数のものの調和のとれた秩序
 が、統治する者の中にあるそのような正義を示すように、宇宙の秩序、これは、
 自然的事物においてだけでなく、意志的事物においても現れているが、は、神
 の正義を示している。このことから、ディオニュシウスは、『神名論』8章で、
 次のように述べる。
「万物に、存在する各々のもののふさわしさに応じて固有のものを与え、各々のもの
 の本性を、その固有の秩序と力において保っていることの中に、神の真の正義があるのを見るべきである」。


\\


{\scshape Ad primum ergo dicendum} quod virtutum
moralium quaedam sunt circa passiones; sicut temperantia circa
concupiscentias, fortitudo circa timores et audacias, mansuetudo circa
iram. Et huiusmodi virtutes Deo attribui non possunt, nisi secundum
metaphoram, quia in Deo neque passiones sunt, ut supra dictum est; neque
appetitus sensitivus, in quo sunt huiusmodi virtutes sicut in subiecto,
ut dicit philosophus in III Ethic. 

&

第一の異論に対しては、それゆえ、以下のように言われるべきである。
倫理的徳のあるものは、情念に関係する。たとえば、節制は欲望に、勇気は恐怖
 と大胆さに、柔和さは怒りに関係する。そして、比喩でない限り、このような徳が神に帰せられ
 ることはありえない。なぜなら、前に述べられたとおり、神の中に情念はなく、
 また、感覚的欲求もないからである。これらの徳は、哲学者が『ニコマコス倫
 理学』第3巻の中で述べるように、この感覚的欲求をあたかも基体として、存在
 する。

\\


Quaedam vero virtutes morales sunt
circa operationes; ut puta circa dationes et sumptus, ut iustitia et
liberalitas et magnificentia; quae etiam non sunt in parte sensitiva,
sed in voluntate. Unde nihil prohibet huiusmodi virtutes in Deo ponere,
non tamen circa actiones civiles sed circa actiones Deo
convenientes. Ridiculum est enim secundum virtutes politicas Deum
laudare, ut dicit philosophus in X {\itshape Ethic}.

&

他方で、ある倫理徳は、働きに関係する。たとえば、正義、寛容、人間の大きさ
 は、与えることや支払うことに関係する。これらは、感覚的部分ではなく、意
 志の中にある。したがって、このような徳を神の中に置くことを妨げるものは
 ない。ただし、市民的な活動にかんしてではなく、神に属する活動にかんし
 てである。というのも、哲学者が『ニコマコス倫理学』第10巻で言うように、
 政治的な力に即して神を褒めるのは、馬鹿げたことだからである。


\\


{\scshape Ad secundum dicendum} quod, cum bonum
intellectum sit obiectum voluntatis, impossibile est Deum velle nisi
quod ratio suae sapientiae habet. Quae quidem est sicut lex iustitiae,
secundum quam eius voluntas recta et iusta est. Unde quod secundum suam
voluntatem facit, iuste facit, sicut et nos quod secundum legem facimus,
iuste facimus. Sed nos quidem secundum legem alicuius superioris, Deus
autem sibi ipsi est lex.

&


第二異論に対しては、以下のように言われるべきである。
意志の対象は、知性認識された善[=知性によって善だと判断されたもの]だか
 ら、神の知恵の性格にかなうものでなければ
 \footnote{直訳は、「神の知恵のratio(根拠、理拠、性格)が持つものでなけれ
 ば」。}、神はそれを意志しない。この性格は、いわば正義の法であり、それに従っ
 て、神の意志は正しく、正義である。それゆえ、神が自らの意志に従って
 為すことは、正しく為すのであり、それはちょうど、私たちもまた、法に従っ
 て為すときに、正しく為すのと同様である。しかし、私たちは、ある上位のも
 のの法に従ってであるが、神は、自分自身にとって、法である。

\\


{\scshape Ad tertium dicendum} quod unicuique debetur
quod suum est. Dicitur autem esse suum alicuius, quod ad ipsum
ordinatur; sicut servus est domini, et non e converso; nam liberum est
quod sui causa est. In nomine ergo debiti, importatur quidam ordo
exigentiae vel necessitatis alicuius ad quod ordinatur. 

&

第三異論に対しては、以下のように言われるべきである。
各々のものには、「[それぞれにとって]自分のもの」である
 ものが、帰せられるべきである\footnote{ ``S debet
 dat. acc."の構文で、Sは債務者、dat.は債権者、acc.は債務。それゆえ受動態
 では、 ``債務 debetur 債権者(dat.) ab 債務者."のかたちになる。それを当て
 はめると、unicuiqueは債権者、suum estは債務。債務者は省略されているが、
 第二異論解答で言われた、「正義の法」と考えられる。これは神と同じだが、後で述べ
 られるように、神を債務者とは呼ばない。}。
ところで、「自分のもの」が、あるものに所属すると言われるのは、それがそれ
 自身へと秩序づけられているからである。
たとえば、[奴隷が主人に秩序づけられているから]奴隷は主人に所属するよう
 に。
そしてその逆でないのは、自由であることと
は、自らの原因だということだからである。
それゆえ、「債務」という名称においては、それへと秩序づけられている何かへの要
求や必要の秩序が意味される。



\\

Est autem duplex
ordo considerandus in rebus. 
Unus, quo aliquid creatum ordinatur ad
aliud creatum, sicut partes ordinantur ad totum, et accidentia ad
substantias, et unaquaeque res ad suum finem. Alius ordo, quo omnia
creata ordinantur in Deum. Sic igitur et debitum attendi potest
dupliciter in operatione divina, aut secundum quod aliquid debetur Deo;
aut secundum quod aliquid debetur rei creatae. 


&

ところで、諸事物において、二通りの秩序が考察され
るべきである。
一つは、それによって、創造されたあるものが、他の創造されたものへと秩序づけられるような秩
 序である。たとえば、部分が全体へ、附帯性が実体へ、各々の事物がその事物
 の目的へ、という具合に。
もう一つは、それによって、創造されたすべてのものが神へと秩序づけられるような秩序
 である。
このことから、それゆえ、債務もまた、神の働きの中に、二通りのかたちで
 見出されうる。神にたいして何らかの債務があるかぎりでと、創造された事物
 にたいしてなんらかの債務があるかぎりでと。


\\

Et utroque modo Deus
debitum reddit. Debitum enim est Deo, ut impleatur in rebus id quod eius
sapientia et voluntas habet, et quod suam bonitatem manifestat, et
secundum hoc iustitia Dei respicit decentiam\footnote{d\u{e}cet , cuit, 2,
I.v. impers. It is seemly, comely, becoming; it beseems, behooves, is fitting, suitable, proper (for syn. v. debeo init.):} ipsius, secundum quam
reddit sibi quod sibi debetur. 


&

そして、どちらのしかたでも、神は、債務を履行している。
つまり、神にとって、諸事物において、神の知恵と意志がもつこと
 がらと、自らの善性を示すことがらとが満たされることが、債務である。
 そして、自らにたいする債務を自らに履行することによって、神の正義は、それ
 自身のふさわしさを得る。


\\

Debitum etiam est alicui rei creatae,
quod habeat id quod ad ipsam ordinatur, sicut homini, quod habeat manus,
et quod ei alia animalia serviant. 
Et sic etiam Deus operatur iustitiam,
quando dat unicuique quod ei debetur secundum rationem suae naturae et
conditionis.

&

また、創造された何らかの被造物に対しては、その事物自身へ秩序づけられたも
 のを、その事物がもつことが、債務である。
たとえば、人間にたいしては、人間が手をもつことや、人間に他の動物が仕える
 ことが、債務である。
このようなかたちでもまた、神は、各々のものに、それ自身の本性や条件の性格
 に即して、それにたいする債務を与えるとき、正義を行う。


\\


 Sed hoc debitum dependet ex primo, quia hoc unicuique
debetur, quod est ordinatum ad ipsum secundum ordinem divinae
sapientiae. Et licet Deus hoc modo debitum alicui det, non tamen ipse
est debitor, quia ipse ad alia non ordinatur, sed potius alia in
ipsum. 


&



しかし、この後者の債務は、第一のものに依存する。
なぜなら、神の知恵の秩序に即して、各々のものに秩序づけら
 れたものが、それぞれにたいする債務となるからである。
そして、神は、このような意味で、債務をあるものへ与えるが、神自身が
債務者であるわけではない。なぜなら、神自身は他のものへ秩序づけられず、
 むしろ、他のものを自分自身へと秩序づけるからである。

\\


Et ideo iustitia quandoque dicitur in Deo condecentia suae
bonitatis; quandoque vero retributio pro meritis. Et utrumque modum
tangit Anselmus, dicens, cum punis malos, iustum est, quia illorum
meritis convenit; cum vero parcis malis, iustum est, quia bonitati tuae
condecens est.

&

ゆえに、神において、ある時には、神自身の善性のふさわしさが正義と言われ、
 またある時には、功績に対する報いが正義と言われる。
そしてこのどちらのしかたにも、アンセルムスは以下のように述べて、触れてい
 る。「あなたが悪人を罰する時、それは正しい。なぜなら、彼らの報いに合致
 するからである。また、あなたが悪人を赦す時、それは正しい。なぜなら、あ
 なたの善性にふさわしいからである」。


\\


{\scshape Ad quartum dicendum} quod, licet iustitia
respiciat actum, non tamen per hoc excluditur quin sit essentia Dei,
quia etiam id quod est de essentia rei, potest esse principium
actionis. Sed bonum non semper respicit actum, quia aliquid dicitur esse
bonum, non solum secundum quod agit, sed etiam secundum quod in sua
essentia perfectum est. Et propter hoc ibidem dicitur quod bonum
comparatur ad iustum, sicut generale ad speciale.

&

第四異論に対しては、以下のように言われるべきである。
正義は行為に関係するが、それによって、それが神の本質であることが排除され
 るわけではない。なぜなら、事物の本質に属するものであっても、行為の根源
 でありうるからである。しかし、善は、必ずしも常に行為に関係しない。なぜ
 なら、あるものが善と言われるのは、それが行為する限りにおいてだけでなく、
 その本質において完全である限りにおいてもだから。このために、同じ箇所で、
 類的なものが種的なものに関係するように、善は正義に関係すると言われてい
 る。




\end{longtable}
\newpage








\rhead{a.~2}
\begin{center}
 {\Large {\bf ARTICULUS SECUNDUS}}\\
 {\large UTRUM IUSTITIA DEI SIT VERITAS}\\
 {\footnotesize IV {\itshape Sent.}, d.46, q.1, a.1, qu$^a$3.}\\
 {\Large 第二項\\神の正義は真理か}
\end{center}

\begin{longtable}{p{21em}p{21em}}







{\Huge A}{\scshape d secundum sic proceditur}. Videtur quod
iustitia Dei non sit veritas. Iustitia enim est in voluntate, est enim
{\itshape rectitudo voluntatis}, ut dicit Anselmus. Veritas autem est in
intellectu, secundum philosophum in VI {\itshape Metaphys}. et in VI {\itshape Ethic}. Ergo
iustitia non pertinet ad veritatem.


&

第二項の問題へ、議論は以下のように進められる。
神の正義は真理でないと思われる。理由は以下の通り。
正義は意志の中にある。アンセルムスが言うように、それは「意志の正しさ」だ
 からである。他方、『形而上学』第6巻と『ニコマコス倫理学』第6巻の哲学者
 によれば、真理は知性の中にある。
ゆえに、正義は真理に属さない。

\\


{\scshape 2 Praeterea}, veritas, secundum philosophum in
IV {\itshape Ethic}., est quaedam alia virtus a iustitia. Non ergo veritas pertinet
ad rationem iustitiae.


&


さらに、『ニコマコス倫理学』第4巻の哲学者によれば、真理は、正義とは違う
 別の徳である。ゆえに、真理は、正義の性格に属さない。

\\


{\scshape Sed contra est} quod in Psalmo {\scshape lxxxiv}
dicitur, {\itshape misericordia et veritas obviaverunt sibi}; et ponitur ibi
veritas pro iustitia.


&


しかし反対に、『詩編』84章で「憐れみと真理が出会った」\footnote{「慈しみとまことは出会い、正義と平和は口づけし」(85:11)}と言われており、
 ここで、真理は正義の意味で使われている。

\\


{\scshape Respondeo dicendum} quod veritas consistit in
adaequatione intellectus et rei, sicut supra dictum est. Intellectus
autem qui est causa rei, comparatur ad ipsam sicut regula et mensura, e
converso autem est de intellectu qui accipit scientiam a rebus. Quando
igitur res sunt mensura et regula intellectus, veritas consistit in hoc,
quod intellectus adaequatur rei, ut in nobis accidit, ex eo enim quod
res est vel non est, opinio nostra et oratio vera vel falsa est. 

&


解答する。以下のように言われるべきである。
前に述べられたとおり、真理は、知性と事物の対等において成立する。
ところで、事物の原因である知性は、事物自体に対して、基準や尺度として関係
 するが、事物から知を受け取る知性についてはその逆である。
ゆえに、事物が知性の尺度や基準である場合には、私たちにおいてそうであるよ
 うに、真理は、知性が事物に対当することにおいて成立する。じっさい、事物
 がそうあるかないかに基づいて、私たちの信念や言明が真であったりなかった
 りするのだから。

\\

Sed
quando intellectus est regula vel mensura rerum, veritas consistit in
hoc, quod res adaequantur intellectui, sicut dicitur artifex facere
verum opus, quando concordat arti. Sicut autem se habent artificiata ad
artem, ita se habent opera iusta ad legem cui concordant. Iustitia
igitur Dei, quae constituit ordinem in rebus conformem rationi
sapientiae suae, quae est lex eius, convenienter veritas nominatur. Et
sic etiam dicitur in nobis veritas iustitiae.


&

しかし、知性が事物の基準や尺度である場合には、真理は、事物が知性に対当す
 ることにおいて成立する。ちょうど、制作知に一致するとき、制作者が、真の
 作品を作ると言われるように。ところで、ちょうど制作物が制作知に関係する
 ように、正義の業は、それに一致するところの法に関係する。
ゆえに、神の正義は、神自身の法である、神の知恵の性格に一致した秩序を事物の中に打ち立て
 るのだから、適切に真理と言われる。この意味で、私たちにおいても、正義の
 真理ということが言われる。


\\


{\scshape Ad primum ergo dicendum} quod iustitia,
quantum ad legem regulantem, est in ratione vel intellectu, sed quantum
ad imperium, quo opera regulantur secundum legem, est in voluntate.


&

第一異論にたいしては、それゆえ、以下のように言われるべきである。
正義は、規制する法にかんする限り、理性ないし知性の中にあり、それによって
 業が法に従って規制されるところの命令にかんする限り、意志の中にある。


\\


{\scshape Ad secundum dicendum} quod veritas illa de qua
loquitur philosophus ibi, est quaedam virtus per quam aliquis demonstrat
se talem in dictis vel factis, qualis est. Et sic consistit in
conformitate signi ad significatum, non autem in conformitate effectus
ad causam et regulam, sicut de veritate iustitiae dictum est.


&

第二異論に対しては、以下のように言われるべきである。
哲学者がそこで語っている真理は、ある人が、発言や行動において、ありのまま
 の自分を表現するある種の徳である。
その意味では、それは「しるし」と「しるしによって表現されたもの」との一致におい
 て成立するが、正義の真理について言われたように、結果の、原因や基準にた
 いする一致において成立するのではない。


\end{longtable}
\newpage




\rhead{a.~3}
\begin{center}
 {\Large {\bf ARTICULUS TERTIUS}}\\
 {\large UTRUM MISERICORDIA COMPETAT DEO}\\
 {\footnotesize II$^a$ II$^ae$, q.30, a.4; IV {\itshape Sent.}, d.46,
 q.2, a.1, qu$^a$ 1; I {\itshape SCG}, cap.91; {\itshape Psalm.~24.}}\\
 {\Large 第三項\\憐れみは神に適合するか}
\end{center}

\begin{longtable}{p{21em}p{21em}}

{\Huge A}{\scshape d tertium sic proceditur}. Videtur quod
misericordia Deo non competat. Misericordia enim est species tristitiae,
ut dicit Damascenus. Sed tristitia non est in Deo. Ergo nec
misericordia.


&


第三項の問題へ、議論は以下のように進められる。
憐れみは神に適合しないと思われる。理由は以下の通り。
ダマスケヌスが言うように、憐れみは、悲しみの種である。
ところが、神の中に悲しみはない。
ゆえに、憐れみもない。

\\


{\scshape 2 Praeterea}, misericordia est relaxatio
iustitiae. Sed Deus non potest praetermittere id quod ad iustitiam suam
pertinet. Dicitur enim II {\itshape ad Tim}.~{\scshape ii}, {\itshape si non credimus, ille fidelis
permanet, seipsum negare non potest}, negaret autem seipsum, ut dicit
Glossa ibidem, si dicta sua negaret. Ergo misericordia Deo non competit.


&

さらに、憐れみは、正義の緩和である。
ところが、神は、自らの正義に属することを不問にすることがありえない。なぜ
 なら、『テモテへの手紙2』第2章「もし私たちが信じなくても、彼は信義の人
 にとどまる。自分を否定することができないから」\footnote{「わたしたちが誠実でなくても、キリストは常に真実であられる。キリストは御自身を否むことができないからである。」(2:13)}と言われているが、
そこの『解説』が言うように、もし自分が言ったことを否定したならば、自分を否定した
 であろうから。ゆえに、憐れみは神に適合しない。

\\


{\scshape Sed contra est} quod dicitur in Psalmo {\scshape cx},
{\itshape miserator et misericors Dominus}.


&


しかし反対に、『詩編』110「憐れみ憐れみ深い主」\footnote{「主は恵み深く憐れみに富み」(111:4)}と言われている。

\\


{\scshape Respondeo dicendum} quod misericordia est Deo
maxime attribuenda, tamen secundum effectum, non secundum passionis
affectum. Ad cuius evidentiam, considerandum est quod misericors dicitur
aliquis quasi habens {\itshape miserum cor}, quia scilicet afficitur ex miseria
alterius per tristitiam, ac si esset eius propria miseria. 
Et ex hoc
sequitur quod operetur ad depellendam miseriam alterius, sicut miseriam
propriam, et hic est misericordiae effectus.

&

解答する。以下のように言われるべきである。
憐れみは、最大限に、神に帰せられるべきである。
ただし、それは結果の点においてであり、情念の感情の点においてではない。
これを明らかにするために、以下のことが考察されるべきである。
ある人が「憐れみ深い」と言われるのは、いわば、「不幸な心」を持っている人
 としてである。というのも、他人の不幸な出来事から、あたかも
 自分自身の不幸であるかのように、悲しみを受けとるからである。
そしてこのことから、他人の不幸を、あたかも自分の不幸であるかのように取り
 払おうと働くが、これが憐れみがもたらす結果である。



\\



 Tristari ergo de miseria
alterius non competit Deo, sed repellere miseriam alterius, hoc maxime
ei competit, ut per miseriam quemcumque defectum intelligamus. Defectus
autem non tolluntur, nisi per alicuius bonitatis perfectionem, prima
autem origo bonitatis Deus est, ut supra ostensum est. 



&

ゆえに、他人の不幸を悲しむことは神に適合しないが、他人の不幸を取り除くこ
 と、これは、憐れみによって、どんな欠陥を私たちが理解する場合でも、神に
 適合する。ところで、欠陥は、何らかの善性の完全性によらなければなくなら
 ないが、善性の第一の起源は、前に示されたとおり、神である。


\\




Sed considerandum
est quod elargiri\footnote{\={e}larg\u{\i}or, \={\i}ri, v. dep. a., to give out, distribute, bestow.} perfectiones rebus, pertinet quidem et ad bonitatem
divinam, et ad iustitiam, et ad liberalitatem, et misericordiam, tamen
secundum aliam et aliam rationem. Communicatio enim perfectionum,
absolute considerata, pertinet ad bonitatem, ut supra ostensum est. Sed
inquantum perfectiones rebus a Deo dantur secundum earum proportionem,
pertinet ad iustitiam, ut dictum est supra. 



&

しかし、以下のことが考察されるべきである。事物に完全性を与えることは、神
 の善性、正義、寛容、憐れみに属するが、しかし、それぞれ別の性格において
 である。すなわち、無条件的に考察された、完全性の分与は、前に示されたと
 おり、善性に属する。しかし、神によって、諸事物に、諸事物の比例に応じて
 完全性が与えられるかぎりでは、それは前に述べられたとおり、正義に属する。


\\



Inquantum vero non attribuit
rebus perfectiones propter utilitatem suam, sed solum propter suam
bonitatem, pertinet ad liberalitatem. Inquantum vero perfectiones datae
rebus a Deo, omnem defectum expellunt, pertinet ad misericordiam.


&

また、完全性が諸事物へ、その有用性にためにではなく、ただその善性のために
 与えられるかぎりでは、寛容に属する。また、神によって諸事物に与えられた
 完全性が、すべての欠陥を取り除くかぎりで、憐れみに属する。


\\


{\scshape Ad primum igitur dicendum} quod obiectio illa
procedit de misericordia, quantum ad passionis affectum.

&

ゆえに、第一異論に対しては、かの反論は、情念の感情にかんして進められてい
 ると言われるべきである。


\\


{\scshape Ad secundum dicendum} quod Deus misericorditer
agit, non quidem contra iustitiam suam faciendo, sed aliquid supra
iustitiam operando, sicut si alicui cui debentur centum denarii, aliquis
ducentos det de suo, tamen non contra iustitiam facit, sed liberaliter
vel misericorditer operatur. 
&


第二異論に対しては、以下のように言われるべきである。
神が憐れみ深く働くのは、自らの正義に反することをなすことによってではなく、
 正義を越えて何かを働くことによる。
たとえば、ある人に100デナリの借金があるとき、その人に、200デナリを懐から
 与えるような場合である。これは正義に反して行うのではなく、寛容に、ある
 いは、憐れみから働くことである。

\\


Et similiter si aliquis offensam in se
commissam remittat. 
Qui enim aliquid remittit, quodammodo donat illud,
unde apostolus remissionem donationem vocat, {\itshape Ephes}.~{\scshape v}, {\itshape donate invicem,
sicut et Christus vobis donavit}.


&

また、自分に向けてなされた攻撃を赦す場合も同様である。なぜなら、何かを赦
 す人は、あるしかたで、それを与えているからである。
このことから、使徒は『エフェソの信徒への手紙』第5章「お互いに与え合い
 なさい。キリストもあなたたちに与えたように」\footnote{「互いに親切にし、憐れみの心で接し、神がキリストによってあなた方を赦してくださったように、赦し合いなさい」(4:32)}と、許しを贈り物と呼んでい
 る。



\\

 Ex quo patet quod misericordia non
tollit iustitiam, sed est quaedam iustitiae plenitudo. Unde dicitur
Iac.~{\scshape ii}, quod {\itshape misericordia superexaltat iudicium}.


&


このことから、憐れみは正義を廃さず、むしろ、ある意味で正義の充満である。
 このことから、『ヤコブの手紙2』で「憐れみは、判断よりも高く上がる」\footnote{「人に憐れみをかけない者には、憐れみのない裁きが下されます。憐れみは裁きに打ち勝つのです。」}と言
 われている。



\end{longtable}
\newpage





\rhead{a.~4}
\begin{center}
 {\Large {\bf ARTICULUS QUARTUS}}\\
 {\large UTRUM IN OMNIBUS OPERIBUS DEI SIT MISERICORDIA ET IUSTITIA}\\
 {\footnotesize IV {\itshape Sent.}, d.46, q.2, a.2, qu$^a$ 2; II
 {\itshape SCG}, cap.28; {\itshape De Verit.}, q.28, a.1, ad 8;
 {\itshape Psalm}.~24; {\itshape Rom.}, cap.15, lect.~1.}\\
 {\Large 第四項\\神のすべての業の中に憐れみと正義があるか}
\end{center}

\begin{longtable}{p{21em}p{21em}}

{\Huge A}{\scshape d quartum sic proceditur}. Videtur quod non
in omnibus Dei operibus sit misericordia et iustitia. Quaedam enim opera
Dei attribuuntur misericordiae, ut iustificatio impii, quaedam vero
iustitiae, ut damnatio impiorum. Unde dicitur Iac.~{\scshape ii}, {\itshape iudicium sine
misericordia fiet ei qui non fecerit misericordiam}. Non ergo in omni
opere Dei apparet misericordia et iustitia.


&

第四項の問題へ、議論は以下のように進められる。
神のすべての業の中に、憐れみと正義があるわけではないと思われる。
理由は以下の通り。
不敬虔な者を正しくすることのように、ある神の業は憐れみに帰せられ、また、
 不敬虔な者を断罪することのように、別のある神の業は正義に帰せられる。こ
 のことから、『ヤコブの手紙』第2章で、「憐れみをかけなかった者には、憐れ
 みのない判断が行われる」\footnote{「人に憐れみをかけない者には、憐れみ
 のない裁きが下されます。」(2:13)}と言われる。
ゆえに、すべての神の業の中に、憐れみと正義が現れるわけではない。

\\


{\scshape 2 Praeterea}, apostolus, {\itshape ad Rom}.~{\scshape xv},
conversionem Iudaeorum attribuit iustitiae et veritati; conversionem
autem gentium, misericordiae. Ergo non in quolibet opere Dei est
iustitia et misericordia.


&

さらに、使徒は『ローマの信徒への手紙』第15章で、ユダヤ人たちの回心を、正
 義と真理に帰している。他方、異教徒の回心は、憐れみに帰している。
ゆえに、神のどの業の中にも、正義と憐れみがあるわけではない。


\\


{\scshape 3 Praeterea}, multi iusti in hoc mundo
affliguntur. Hoc autem est iniustum. Non ergo in omni opere Dei est
iustitia et misericordia.


&


さらに、この世界の中で、多くの正しい人々が打ちひしがれている。
ところで、このことは、正しくない。
ゆえに、神のあらゆる業の中に、正義と憐れみがあるわけではない。

\\


{\scshape 4 Praeterea}, iustitiae est reddere debitum,
misericordiae autem sublevare miseriam, et sic tam iustitia quam
misericordia aliquid praesupponit in suo opere. Sed creatio nihil
praesupponit. Ergo in creatione neque misericordia est, neque iustitia.


&

さらに、正義は債務を履行することであり、憐れみは悲惨を和らげることである。
この意味で、憐れみと同様、正義もまた、その業において何かを前提とする。
ところが、創造は何も前提にしない。
ゆえに、創造においては、憐れみも正義もない。


\\


{\scshape Sed contra est} quod dicitur in Psalmo {\scshape xxiv},
{\itshape omnes viae domini misericordia et veritas}.


&

しかし反対に、『詩編』第24節で「主のすべての道は、憐れみと真理である」\footnote{「その契約と定めを守る人にとって、主の道はすべて、慈しみとまこと。」(25:10)}と
 言われている。

\\


{\scshape Respondeo dicendum} quod necesse est quod in
quolibet opere Dei misericordia et veritas inveniantur; si tamen
misericordia pro remotione cuiuscumque defectus accipiatur; quamvis non
omnis defectus proprie possit dici miseria, sed solum defectus
rationalis naturae, quam contingit esse felicem; nam miseria felicitati
opponitur. 

&

解答する。以下のように言われるべきである。
神のどの業の中にも、憐れみと真理が見出されることが必然である。
ただし、「憐れみ」(misericordia)が、なんの欠陥であれそれを取り除くことと理解される場合
 にである。悲惨は幸福に対立するので、厳密には、あらゆる欠陥が悲惨と言わ
 れうることはなく、ただ幸福でありうる、理性的本性の欠陥だけが、悲惨と言
 われうるのだが。

\\

Huius autem necessitatis ratio est, quia, cum debitum quod ex
divina iustitia redditur, sit vel debitum Deo, vel debitum alicui
creaturae, neutrum potest in aliquo opere Dei praetermitti. Non enim
potest facere aliquid Deus, quod non sit conveniens sapientiae et
bonitati ipsius; secundum quem modum diximus aliquid esse debitum
Deo. 
Similiter etiam quidquid in rebus creatis facit, secundum
convenientem ordinem et proportionem facit; in quo consistit ratio
iustitiae. Et sic oportet in omni opere Dei esse iustitiam.

&

ところで、この必然性の根拠は以下の通りである。すなわち、神の正義に基づい
 て履行される債務は、神にた
 いする債務か、あるいは、何らかの被造物にたいする債務かのどちらかである
 が、どちらも、いかなる神の業においてもおろそかにされえない。
じっさい、神にとっては、自らの知恵と善性に適合することをなすことが債務で
 あると私たちは述べたが、神がそうしないことはありえない。
同様にまた、何であれ被造の諸事物のなかになすことは、[神はそれを]適切な
 秩序と比率にしたがってなす。そしてこのことにおいて、正義という性格が成
 立している。したがって、神のすべての業の中に、正義が存在しなければならない。


\\

 Opus autem
divinae iustitiae semper praesupponit opus misericordiae, et in eo
fundatur. Creaturae enim non debetur aliquid, nisi propter aliquid in eo
praeexistens, vel praeconsideratum, et rursus, si illud creaturae
debetur, hoc erit propter aliquid prius. 


&


ところで、神の正義の業は、常に、憐れみの業を前提とし、それに基づく。
理由は以下。[神が]、被造物にたいして、何かを債務とするのは、その造物の中にあらか
 じめ存在する何かのため、あるいは、あらかじめ考えられたもののために他な
 らない。また、もし或るものが被造物にたいする債務となるならば、
 それは、先行する何かのためでである。


\\


Et cum non sit procedere in
infinitum, oportet devenire ad aliquid quod ex sola bonitate divinae
voluntatis dependeat, quae est ultimus finis. Utpote si dicamus quod
habere manus debitum est homini propter animam rationalem; animam vero
rationalem habere, ad hoc quod sit homo; hominem vero esse, propter
divinam bonitatem. Et sic in quolibet opere Dei apparet misericordia,
quantum ad primam radicem eius. 


&

そして、無限に先行することは不可能だから、何かに行き着かなくてはならず、
 それは、ただ、究極の目的である神の意志の善性にのみ依存する。
たとえば、私たちが以下のように言う場合のように。「手を持つことは、理性的な魂の
 ために、人間にたいする債務である。しかし、理性的魂を持つことは、人間で
 あるためである。そして、人間であることは、神の善性のためである」。
このように、神のどんな業においても、その最初の根拠にかんして、憐れみが現
 れている。


\\

Cuius virtus salvatur in omnibus
consequentibus; et etiam vehementius in eis operatur, sicut causa
primaria vehementius influit quam causa secunda. Et propter hoc etiam ea
quae alicui creaturae debentur, Deus, ex abundantia suae bonitatis,
largius dispensat quam exigat proportio rei. Minus enim est quod
sufficeret ad conservandum ordinem iustitiae, quam quod divina bonitas
confert, quae omnem proportionem creaturae excedit.


&

この力は、後続するすべてのものの中に保たれ、それらにおいていっそう力強く
 働く。ちょうど、第一の原因が、第二の原因よりも、力強く流れ込むように。
そしてこのために、ある被造物にたいする債務を、神は、自らの善性の充満に基
 づいて、事物の比率が必要とするよりも余計に分け与える。
じっさい、被造物のあらゆる比率を超える神の善性が与える
 ものよりも少なくても、正義の秩序を保つには十分だっただろう。
 



\\


{\scshape Ad primum ergo dicendum} quod quaedam opera
attribuuntur iustitiae et quaedam misericordiae, quia in quibusdam
vehementius apparet iustitia, in quibusdam misericordia. Et tamen in
damnatione reproborum apparet misericordia, non quidem totaliter
relaxans, sed aliqualiter allevians, dum punit citra condignum. Et in
iustificatione impii apparet iustitia, dum culpas relaxat propter
dilectionem, quam tamen ipse misericorditer infundit, sicut de Magdalena
legitur, Luc.~{\scshape vii}, {\itshape dimissa sunt ei peccata multa, quoniam dilexit
multum}.


&

第一異論に対しては、それゆえ、以下のように言われるべきである。
ある業は正義に、ある業は憐れみに帰せられる。なぜなら、ある業においては、
 正義がより強く現れ、ある業においては、憐れみがより強く現れるからである。
しかし、神に見捨てられた人々の断罪において憐れみが現れるのは、すべての罪
 を許すことによってではなく、何らかのしかたで罪を軽くする、つまり、罰せ
 られるに値するより少なく罰することによってである。
また、不敬虔な者を正しくすることにおいて、愛のために罪を許すとき、正義が
 現れるが、その愛を、神は憐れみによって注ぎ入れる。それはちょうど、『ルカによ
 る福音書』第7章で、マグダラのマリアについて、「多くの罪が彼女に許された。
 多く愛したからである」\footnote{「この人が多くの罪を赦されたことは、私
 に示した愛の大きさでわかる。赦されることの少ない者は、愛することも少な
 い」(7:47)}と書かれているとおりである。



\\


{\scshape Ad secundum dicendum} quod iustitia et
misericordia Dei apparet in conversione Iudaeorum et gentium, sed aliqua
ratio iustitiae apparet in conversione Iudaeorum, quae non apparet in
conversione gentium, sicut quod salvati sunt propter promissiones
patribus factas.


&

第二異論に対しては、以下のように言われるべきである。
ユダヤ人と異教徒の回心において、神の正義と憐れみが現れるが、しかし、異教
 徒の回心においては現れない何らかの性格が、ユダヤ人の回心において現れて
 いる。たとえば、彼らが、父祖たちへの約束のために救われたこと、のように。

\\


{\scshape Ad tertium dicendum} quod in hoc etiam quod
iusti puniuntur in hoc mundo, apparet iustitia et misericordia;
inquantum per huiusmodi afflictiones aliqua levia in eis purgantur, et
ab affectu terrenorum in Deum magis eriguntur; secundum illud Gregorii,
{\itshape mala quae in hoc mundo nos premunt, ad Deum nos ire compellunt}.


&

第三異論に対しては、以下のように言われるべきである。
この世で、正しい人々が罰せられることにおいても、正義と憐れみが現れている。
それは、グレゴリウスの、かの「この世で私たちを押さえつける諸悪は、私たちを
 神へと駆り立てる」によれば、そのような災難を通して、軽い罪が彼らの中で
 浄化され、地上的な情念から、神へとより上昇させられるかぎりにおいてであ
 る。



\\


{\scshape Ad quartum dicendum} quod, licet creationi non
praesupponatur aliquid in rerum natura, praesupponitur tamen aliquid in
Dei cognitione. Et secundum hoc etiam salvatur ibi ratio iustitiae,
inquantum res in esse producitur, secundum quod convenit divinae
sapientiae et bonitati. Et salvatur quodammodo ratio misericordiae,
inquantum res de non esse in esse mutatur.


&

第四異論に対しては、以下のように言われるべきである。
たしかに、諸事物の本性におけるどんなものも、創造に前提とされないが、しか
 し、神の思惟の中で、何かが前提とされる。
そしてその意味で、神の知恵と善性に適合するように事物が存在へと産出される
 かぎりにおいて、正義という性格が成り立つ。また、諸事物を非存在から存在
 へと変えるかぎりで、ある意味で、憐れみの性格が成り立つ。



\end{longtable}
\end{document}