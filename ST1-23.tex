\documentclass[10pt]{jsarticle} % use larger type; default would be 10pt
%\usepackage[utf8]{inputenc} % set input encoding (not needed with XeLaTeX)
%\usepackage[round,comma,authoryear]{natbib}
%\usepackage{nruby}
\usepackage{okumacro}
\usepackage{longtable}
%\usepqckage{tablefootnote}
\usepackage[polutonikogreek,english,japanese]{babel}
%\usepackage{amsmath}
\usepackage{latexsym}
\usepackage{color}

%----- header -------
\usepackage{fancyhdr}
\lhead{{\it Summa Theologiae} I, q.~23}
%--------------------

\bibliographystyle{jplain}

\title{{\bf PRIMA PARS}\\{\HUGE Summae Theologiae}\\Sancti Thomae
Aquinatis\\{\sffamily QUEAESTIO VIGESIMATERTIA}\\DE PRAEDESTINATIONE}
\author{Japanese translation\\by Yoshinori {\sc Ueeda}}
\date{Last modified \today}


%%%% コピペ用
%\rhead{a.~}
%\begin{center}
% {\Large {\bf }}\\
% {\large }\\
% {\footnotesize }\\
% {\Large \\}
%\end{center}
%
%\begin{longtable}{p{21em}p{21em}}
%
%&
%
%
%\\
%\end{longtable}
%\newpage



\begin{document}
\maketitle
\pagestyle{fancy}


\begin{center}
{\Large 第二十三問\\予定について}
\end{center}


\begin{longtable}{p{21em}p{21em}}


Post considerationem divinae providentiae, agendum est de
 praedestinatione, et de libro vitae. 

&

 神の摂理の考察の後、予定について、そして生命の書について論じられるべき
 である。
 

\\


 Et circa praedestinationem
 quaeruntur octo.
\begin{enumerate}
 \item utrum Deo conveniat praedestinatio.
 \item quid sit praedestinatio; et utrum ponat aliquid in  praedestinato.
 \item utrum Deo competat reprobatio aliquorum hominum.
 \item de comparatione praedestinationis ad electionem; utrum scilicet
       praedestinati eligantur.
 \item utrum merita sint causa vel ratio praedestinationis, vel
       reprobationis, aut electionis.
 \item de certitudine praedestinationis; utrum scilicet praedestinati
       infallibiliter salventur.
 \item utrum numerus praedestinatorum sit certus.
 \item utrum praedestinatio possit iuvari precibus sanctorum.
\end{enumerate}

&

そして、予定について、八つのことが問われる。

\begin{enumerate}
 \item 予定は神に適合するか。
 \item 予定とは何か。それは予定された人の中に何かを置くか。
 \item ある人々を見捨てることが神に属するか。
 \item 予定の選択にたいする関係について。すなわち、予定された人々は選ば
       れているか。
 \item 功績は、予定や見捨てや選択の、原因または根拠か。
 \item 予定の確実性について。予定された者は、間違いなく救われるか。
 \item 予定された人々の数は確定しているか。
 \item 予定は、聖人たちの祈りによって助けられることが可能か。
\end{enumerate}


\end{longtable}


\newpage


\rhead{a.~1}
\begin{center}
 {\Large {\bf ARTICULUS PRIMUS}}\\
 {\large UTRUM HOMINES PRAEDESTINENTUR A DEO}\\
 {\footnotesize I {\itshape Sent.}, d.11, q.1, a.2; III {\itshape SCG},
 cap.163; {\itshape De Verit.}, q.6, a.1; {\itshape ad Rom.}, cap.1, lect.3.}\\
 {\Large 第一項\\人々は神によって予定されるか}
\end{center}

\begin{longtable}{p{21em}p{21em}}



{\Huge A}{\scshape d primum sic proceditur}. Videtur quod
homines non praedestinentur a Deo. Dicit enim Damascenus, in II libro,
{\itshape oportet cognoscere quod omnia quidem praecognoscit Deus, non autem omnia
praedeterminat. Praecognoscit enim ea quae in nobis sunt; non autem
praedeterminat ea}. Sed merita et demerita humana sunt in nobis,
inquantum sumus nostrorum actuum domini per liberum arbitrium. Ea ergo
quae pertinent ad meritum vel demeritum, non praedestinantur a Deo. Et
sic hominum praedestinatio tollitur.


 
&

第一項の問題へ、議論は以下のように進められる。人々が神によって予定され
ることはないと思われる。理由は以下の通り。ダマスケヌスは、第二巻
\footnote{『正統信仰論』第2巻第3章}で以下のように述べている。「次のこ
とを知らなければならない。神は実際にすべてのことを予め知るが、すべての
ことを予め限定するわけではない。なぜなら、私たちの中にある事柄を、神は
予め知るが、それらを予め限定しているわけではないから」。ところで、人間
の功績や罪過が私たちの中にあるのは、私たちが、自由裁量を通して私たちの
行為の主であるかぎりにおいてである。ゆえに、功績や罪過に属する事柄は、
神によって予定されていない。このようにして、人々の予定は取り除かれる。
 

\\


{\scshape 2 Praeterea}, omnes creaturae ordinantur ad
suos fines per divinam providentiam, ut supra dictum est. Sed aliae
creaturae non dicuntur praedestinari a Deo. Ergo nec homines.


 
&

さらに、前に述べられたとおり、すべての被造物は神の摂理によって自らの目
的へ秩序付けられている。ところが、他の被造物は、神によって予定されてい
るとは言われない。ゆえに、人間もまた[神によって予定されているとは言わ
れない]。
 

\\


{\scshape 3 Praeterea}, Angeli sunt capaces
beatitudinis, sicut et homines. Sed Angelis non competit praedestinari,
ut videtur, cum in eis nunquam fuerit miseria; praedestinatio autem est
{\itshape propositum miserendi}, ut dicit Augustinus. Ergo homines non
praedestinantur.


 
&

さらに、天使たちが至福を受けうるものであるように、人間もまた[至福を受
けうるものである]。ところで、あとで見られるように、天使たちには、予定
されることが属さない。なぜなら、天使たちの中には決して悲惨がなかったが、
アウグスティヌスが言うように、予定は、悲惨なことの計画だからである。ゆ
えに、人々が予定されることはない。

 
\\


{\scshape 4 Praeterea}, beneficia hominibus a Deo
collata, per spiritum sanctum viris sanctis revelantur, secundum illud
apostoli, I {\itshape Cor}.~{\scshape ii}, {\itshape nos autem non spiritum huius mundi accepimus, sed
spiritum qui ex Deo est, ut sciamus quae a Deo donata sunt nobis}. Si
ergo homines praedestinarentur a Deo, cum praedestinatio sit Dei
beneficium, esset praedestinatis nota sua praedestinatio. Quod patet
esse falsum.


 
&

さらに、『コリントの信徒への手紙一』第2章の使徒のかの言葉「私たちはこ
の世界の霊ではなく、神から来る聖霊を受け取ったので、神から私たちに何が
与えられたかを知っている」\footnote{「わたしたちは、世の霊ではなく、神
からの霊を受けました。それでわたしたちは、神から恵みとして与えられたも
のを知るようになったのです」(2:12)}によれば、神によって人間にもたらさ
れる善いことは、聖霊を通して聖人たちに啓示される。ゆえに、もし人々が神
によって予定されるなら、予定は神に属する善いことなので、予定されている
人々には自分の予定が知らされたであろう。しかしこれは明らかに偽である。
 

\\


{\scshape Sed contra est} quod dicitur {\itshape Rom}.~{\scshape viii}, {\itshape quos
praedestinavit, hos et vocavit}.


 
&

しかし反対に、『ローマの信徒への手紙』第8章で「彼が予定した人々を、彼
は呼んだ」\footnote{「神は予め定められた者たちを召し出し、」(8:30)}と
言われている。
 

\\


{\scshape Respondeo dicendum} quod Deo conveniens est
homines praedestinare. Omnia enim divinae providentiae subiacent, ut
supra ostensum est. Ad providentiam autem pertinet res in finem
ordinare, ut dictum est. Finis autem ad quem res creatae ordinantur a
Deo, est duplex. Unus, qui excedit proportionem naturae creatae et
facultatem, et hic finis est vita aeterna, quae in divina visione
consistit, quae est supra naturam cuiuslibet creaturae, ut supra habitum
est. Alius autem finis est naturae creatae proportionatus, quem scilicet
res creata potest attingere secundum virtutem suae naturae. 

 
&

解答する。以下のように言われるべきである。人々を予定することは、神に適
合する。理由は以下の通り。前に示されたとおり、万物は神の摂理のもとにあ
る。ところで、すでに述べられたとおり、摂理には、事物を目的へと秩序付け
ることが属する。ところで、神によって、被造の諸事物が、それへと秩序付け
られるところの目的は、二通りある。一つは、被造の本性の比(proportio)や
力(facultas)を越えるものであり、それは、永遠の生命である。これは、神を
見ることにおいて成り立ち、前に論じられたとおり、どんな被造物の本性をも
越えることである。他方、もう一つの目的は、被造の本性に比例したものであ
り、被造の諸事物は、自分の本性の能力(virtus)にしたがって、これに到達す
ることができる。




\\


Ad illud
autem ad quod non potest aliquid virtute suae naturae pervenire, oportet
quod ab alio transmittatur; sicut sagitta a sagittante mittitur ad
signum. Unde, proprie loquendo, rationalis creatura, quae est capax
vitae aeternae, perducitur in ipsam quasi a Deo transmissa. 

 
&

他方、自分の本性の能力で到達できない目的へは、他者によって運ばれなけれ
ばならない。ちょうど、矢が、的へ向けて射手によって放たれるように。した
がって、厳密に言えば、理性的被造物は永遠の生命を受け取りうるものなので、
それへと、神によって言わば運ばれて、到達しなければならない。


\\


Cuius quidem
transmissionis ratio in Deo praeexistit; sicut et in eo est ratio
ordinis omnium in finem, quam diximus esse providentiam. Ratio autem
alicuius fiendi in mente actoris existens, est quaedam praeexistentia
rei fiendae in eo. Unde ratio praedictae transmissionis creaturae
rationalis in finem vitae aeternae, praedestinatio nominatur, nam
destinare est mittere. Et sic patet quod praedestinatio, quantum ad
obiecta, est quaedam pars providentiae.


 
&

実に、この「運び」の理念が神の中に先在する。ちょうどそれは、神の中に、
万物の目的への秩序の理念があり、それを私たちが、摂理と言ったように。と
ころで、ある生じる事柄について、作者の精神の中にある理念は、ある意味で、
その生じる事柄の、精神における先在である。このことから、上述の、理性的
被造物を永遠の生命という目的へ運ぶ、という理念は、「予定」と名付けられ
る。なぜなら、「運ぶ」ことは「定める」ことだから。以上のことから、予定
は、その対象にかんして、摂理の一部であることが明らかである。


\\


{\scshape Ad primum ergo dicendum} quod Damascenus
nominat praedeterminationem impositionem necessitatis; sicut est in
rebus naturalibus, quae sunt praedeterminatae ad unum. Quod patet ex eo
quod subdit, {\itshape non enim vult malitiam, neque compellit virtutem}. Unde
praedestinatio non excluditur.


 
&


第一異論に対しては、それゆえ、以下のように言われるべきである。ダマスケ
ヌスは、必然性の付与を、予限定(praedeterminatio)と呼んでいる。ちょうど、
一つのことへとあらかじめ限定されている、自然的事物においてそうであるよ
うに。このことは、彼がそのあとに、「なぜなら、悪を意志せず、徳を集める
こともないから」と述べていることから明らかである。したがって、予定が排
除されているわけではない。

\\


{\scshape Ad secundum dicendum} quod creaturae
irrationales non sunt capaces illius finis qui facultatem humanae
naturae excedit. Unde non proprie dicuntur praedestinari, etsi aliquando
abusive praedestinatio nominetur respectu cuiuscumque alterius finis.


 
&

第二異論に対しては、以下のように言われるべきである。非理性的被造物は、
人間本性の機能を越えるかの目的を受けうるものではない。したがって、厳密
に言えば、それらは「予定される」とは言われない。ただし、誤用されて、他
のどんな目的にかんしても、「予定」と呼ばれることはある。


\\


{\scshape Ad tertium dicendum} quod praedestinari
convenit Angelis, sicut et hominibus, licet nunquam fuerint miseri. Nam
motus non accipit speciem a termino a quo, sed a termino ad quem, nihil
enim refert, quantum ad rationem dealbationis, utrum ille qui dealbatur,
fuerit niger aut pallidus vel rubeus. Et similiter nihil refert ad
rationem praedestinationis, utrum aliquis praedestinetur in vitam
aeternam a statu miseriae, vel non. -- Quamvis dici possit quod omnis
collatio boni supra debitum eius cui confertur, ad misericordiam
pertineat, ut supra dictum est.

 
&

第三異論に対しては、以下のように言われるべきである。予定されることは、
人間と同様に、天使にも適合する。ただし、天使は決して悲惨であったことは
ない。ところで、運動は、出発点ではなく終着点から種を受け取る。なぜなら、
「白化」の概念について、白くされるものが黒だったか、黄緑だったか、赤だっ
たかということは関係ないからである。同様に、予定の概念にとって、永遠の
生命へ予定される者が、悲惨の状態にあったかどうかは関係ない。ただし、前
に述べられたとおり、善が、もたらされる者のふさわしさを越えて、もたらさ
れることは、すべて、憐れみに属すると言われることは可能である。



\\


{\scshape Ad quartum dicendum} quod, etiam si aliquibus
ex speciali privilegio sua praedestinatio reveletur, non tamen convenit
ut reveletur omnibus, quia sic illi qui non sunt praedestinati,
desperarent; et securitas in praedestinatis negligentiam pareret.

 
&

第四異論に対しては、以下のように言われるべきである。ある人々に、特権的
に、自分が予定されていることが啓示されるということはあるにしても、すべ
ての人々にそれが啓示されることはふさわしいことではない。なぜなら、もし
そうなると、予定されていない人は絶望するだろうし、予定されている人々に
おいては、安心が怠惰を呼ぶだろうから。



\end{longtable}
\newpage

\rhead{a.~2}
\begin{center}
 {\Large {\bf ARTICULUS SECUNDUS}}\\
 {\large UTRUM PRAEDESTINATIO ALIQUID PONAT IN PRAEDESTINATO}\\
 {\footnotesize I {\itshape Sent.}, d.15, q.1, a.1}\\
 {\Large 第二項\\予定は予定された者の中に何かを置くか\footnote{「何かを置
 く」(aliquid ponere)とは、「実在的な変化を伴う」というほどの意味か。}}
\end{center}


\begin{longtable}{p{21em}p{21em}}


{\Huge A}{\scshape d secundum sic proceditur}. Videtur quod
praedestinatio ponat aliquid in praedestinato. Omnis enim actio ex se
passionem infert. Si ergo praedestinatio actio est in Deo, oportet quod
praedestinatio passio sit in praedestinatis.


&

第二項の問題へ、議論は以下のように進められる。予定は、予定された者の中
に何かを置くと思われる。理由は以下の通り。すべて能動は、それ自体から、
受動を引き起こす。ゆえに、能動としての予定が神の中にあるのだから、受動
としての予定が予定された者の中にあるはずである。

 
\\


{\scshape 2 Praeterea}, Origenes dicit, super illud
{\itshape Rom}.~{\scshape i}, {\itshape qui praedestinatus est} etc., {\itshape praedestinatio est eius qui non
est, sed destinatio eius est qui est}. Sed Augustinus dicit, in libro {\itshape de
praedestinatione sanctorum}, {\itshape quid est praedestinatio, nisi destinatio
alicuius?} Ergo praedestinatio non est nisi alicuius
 existentis\footnote{praeexistentis ABCDEsF.高田訳(創文社)はこの読みを
 取っているが、注はなく、理由は不明。}. Et ita
ponit aliquid in praedestinato.


&

さらに、オリゲネスが、かの『ローマの信徒への手紙』第1章「予定されてい
る人は」\footnote{「聖なる霊によれば、死者の中からの復活によって力ある
神の子と定められたのです。」(1:4)}について「予定(予め定めること)は、
いない人属するが、定めることは、いる人に属する」と言っている。しかしア
ウグスティヌスは、『聖人たちの予定について』という書物の中で、「予定j
は、だれかを定めることでなければ何であろう」と述べている。ゆえに、予定
は、だれか存在する者に属する以外にない。その意味で、それは何かを予定さ
れた者の中に置く。

 
\\


{\scshape 3 Praeterea}, praeparatio est aliquid in
praeparato. Sed praedestinatio est {\itshape praeparatio beneficiorum Dei}, ut
dicit Augustinus, in libro {\itshape  de Praedest. Sanct}. Ergo praedestinatio est
aliquid in praedestinatis.


&

さらに、準備は、準備された者の中にある何かである。ところが、アウグスティ
ヌスが『聖人たちの予定』という書物で述べるように、予定は、神の恩典の準
備である。ゆえに、予定は、予定された者の中に何かを置く。

\\


{\scshape  4 Praeterea}, temporale non ponitur in
definitione aeterni. Sed gratia, quae est aliquid temporale, ponitur in
definitione praedestinationis, nam praedestinatio dicitur esse
{\itshape praeparatio gratiae in praesenti, et gloriae in futuro}. Ergo
praedestinatio non est aliquid aeternum. Et ita oportet quod non sit in
Deo, sed in praedestinatis, nam quidquid est in Deo, est aeternum.


&


さらに、時間的なものは、永遠的なものの定義のなかに置かれない。ところが、
恩恵は、何か時間的なものだが、予定の定義の中に置かれる。すなわち、予定
は、「現在における恩恵の、また未来における栄光の準備」であると言われる。
ゆえに、予定は、何か永遠的なものではない。そしてまた、神の中にあるので
はなく、予定された者たちの中にあるのでなければならない。なぜなら、何で
あれ神の中にあるものは、永遠だからである。

 

\\


{\scshape  Sed contra est} quod Augustinus dicit, quod
praedestinatio est {\itshape praescientia beneficiorum Dei}. Sed praescientia non
est in praescitis, sed in praesciente. Ergo nec praedestinatio est in
praedestinatis, sed in praedestinante.


&

しかし反対に、アウグスティヌスは、予定は「神の恩典の予知」であると述べ
ている。ところが、予知は、予知されるものの中ではなく、予知する者の中に
ある。ゆえに、予定も、予定される者たちの中ではなく、予定する者の中にあ
る。
 

\\


{\scshape Respondeo dicendum} quod praedestinatio non est
aliquid in praedestinatis, sed in praedestinante tantum. Dictum est enim
quod praedestinatio est quaedam pars providentiae. Providentia autem non
est in rebus provisis; sed est quaedam ratio in intellectu provisoris,
ut supra dictum est. Sed executio providentiae, quae {\itshape gubernatio} dicitur,
 passive quidem est in gubernatis; active autem est in gubernante.


 &

 解答する。以下のように言われるべきである。予定は、予定される者たちの
 中にある何かではなく、予定する者の中だけにある何かである。理由は以下
 の通り。予定は、摂理の一部であると述べられた。ところで、摂理は、摂理
 される諸事物の中にはなく、前に述べられたとおり\footnote{第22問第1項。}、
 摂理する者の知性の中にある一種の理拠(ratio)である。しかし、摂理の遂行、
 これは「統治」と呼ばれるが、は、統治されるものどもの中に、ある種受動
 的に存在する。能動的には、統治する者の中にあるのだが。

 \\

 Unde
manifestum est quod praedestinatio est quaedam ratio ordinis aliquorum
in salutem aeternam, in mente divina existens. Executio autem huius
ordinis est passive quidem in praedestinatis; active autem est in
Deo. Est autem executio praedestinationis {\itshape vocatio} et {\itshape magnificatio},
secundum illud apostoli, {\itshape ad Rom}.~{\scshape viii}, {\itshape quos praedestinavit, hos et
vocavit; et quos vocavit, hos et magnificavit}.


&

このことから、予定は、ある人々の、永遠の救済への秩序の、一種の理拠であ
り、これが神の精神の中に存在することは明らかである。他方、この秩序の遂
行は、受動的に、予定された者たちの中にあり、能動的に、神の中にある。か
の使徒の、『ローマの信徒への手紙』第8章、「予定した者たちをまた呼び、
呼んだ者たちをまた大きくした」\footnote{「神はあらかじめ定められた者た
ちを召し出し、召し出した者たちを義とし、義とされた者たちに栄光をお与え
になったのです。」(8:30)}によれば、予定の遂行は、「呼ぶこと」「大きく
すること」である。
 
\\


{\scshape Ad primum ergo dicendum} quod actiones in
exteriorem materiam transeuntes, inferunt ex se passionem, ut calefactio
et secatio, non autem actiones in agente manentes, ut sunt intelligere
et velle, ut supra dictum est. Et talis actio est praedestinatio. Unde
praedestinatio non ponit aliquid in praedestinato. Sed executio eius,
quae transit in exteriores res, ponit in eis aliquem effectum.


&

第一異論に対しては、それゆえ、以下のように言われるべきである。前に述べ
られたとおり、燃やすことや切ることのように、外の質料へ出ていく能動は、
自分から受動をもたらすが、知性認識することや意志することのように、作用
者の中に留まる能動は、そうでない。そして、この後者のような能動が予定で
ある。したがって、予定は、予定される者の中に何も置かない。しかし、それ
の遂行は、外の事物へと出ていくから、それらの事物の中に、何らかの結果を
措定する。

\\


{\scshape Ad secundum dicendum} quod {\itshape destinatio}
aliquando sumitur pro reali missione alicuius ad aliquem terminum, et
sic destinatio non est nisi eius quod est. Alio modo sumitur destinatio
pro missione quam aliquis mente concipit, secundum quod dicimur
{\itshape destinare}, quod mente firmiter proponimus, et hoc secundo modo dicitur
II {\itshape Machab}.~cap.~{\scshape vi}, Eleazarus {\itshape destinavit non admittere illicita propter
vitae amorem}. Et sic destinatio potest esse eius quod non est. Tamen
{\itshape praedestinatio}, ratione antecessionis quam importat, potest esse eius
quod non est, qualitercumque destinatio sumatur.


&

第二異論に対しては、以下のように言われるべきである。「定めること」は、
時として、現実に何かをある終着点へ送ることとして理解されるが、その場合、
「定めることは」、存在するものにしか属さない。別の意味で、「定めること」
は、ある人が、心の中に抱いている使命、という意味で理解され、これに従え
ば、私たちが精神によって堅く提示することが、「定める」と言われる。そし
てこの後者の意味で、『マカバイ記二』第6章で、エレアザルスが「生命への
愛のために、不法な者たちを許さないように定めた」\footnote{「これこそ、
生命への愛着があるとはいえ、口にしてはならないものは断固として退けねば
ならない人々の取るべき態度である」(6:20)}と言われている。そしてこの意
味で、予定は、存在しないものに属しうる。しかし、「予定」は、予めという
意味を含むので、そのために、どのように「定めること」が理解されようとも、
存在しないものに属することができる。
 

\\


{\scshape Ad tertium dicendum} quod duplex est
praeparatio. Quaedam patientis, ut patiatur, et haec praeparatio est in
praeparato. Quaedam alia est agentis, ut agat, et haec est in agente. Et
talis praeparatio est praedestinatio; prout aliquod agens per
intellectum dicitur se praeparare ad agendum, inquantum praeconcipit
rationem operis fiendi. Et sic Deus ab aeterno praeparavit
praedestinando, concipiens rationem ordinis aliquorum in salutem.


&

 第三異論に対しては、以下のように言われるべきである。「準備」には二通
りある。一つは、受け取るための、受け取るものの準備であり、この準備は、
準備されるものの中にある。もう一つの準備は、作用するための、作用するも
のの準備であり、これは作用者の中にある。そして、このような準備が予定で
ある。つまり、生じるべき業の理拠を予め捉えるかぎりで、ある作用者が、知
性を通して、作用することへと自らを準備すると言われるようにである。。そ
してこの意味で、神は、永遠から、ある人々の救いへの秩序の理拠を捉えつつ、
予定することで、準備をしていた。
 

\\


{\scshape Ad quartum dicendum} quod gratia non ponitur
in definitione praedestinationis, quasi aliquid existens de essentia
eius, sed inquantum praedestinatio importat respectum ad gratiam, ut
causae ad effectum, et actus ad obiectum. Unde non sequitur quod
praedestinatio sit aliquid temporale.


&

第四異論に対しては、以下のように言われるべきである。恩恵は、その本質に
属するものとして、予定の定義の中に置かれるのではなく、予定が、結果にた
いする原因として、そして、対象にたいする働きとして、恩恵への関係を含意
する限りでである。したがって、予定が何か時間的なものであることは帰結し
ない。


\end{longtable}
\newpage





\rhead{a.~3}
\begin{center}
 {\Large {\bf ARTICULUS TERTIUS}}\\
 {\large UTRUM DEUS ALIQUEM HOMINEM REPROBET}\\
 {\footnotesize I {\itshape Sent.}, d.11, q.4, a.1; III {\itshape SCG.},
 cap.163; {\itshape ad Rom.}, cap.9, lect.2}\\
 {\Large 第三項\\神はある人間を見捨てるか}
\end{center}

\begin{longtable}{p{21em}p{21em}}

{\Huge A}{\scshape d tertium sic proceditur}. Videtur quod
Deus nullum hominem reprobet. Nullus enim reprobat quem diligit. Sed
Deus omnem hominem diligit, secundum illud {\itshape Sap}.~{\scshape xi}, {\itshape diligis omnia quae
sunt, et nihil odisti eorum quae fecisti}. Ergo Deus nullum hominem
reprobat.


&

第三の問題へ、議論は以下のように進められる。神はどんな人間も見捨てない
と思われる。理由は以下の通り。誰も、愛している人を見捨てない。ところが、
『知恵の書』第9章「あなたは存在する万物を愛し、あなたが作ったすべての
ものの何ものも憎まなかった」によれば、神はすべての人間を愛する。ゆえに、
神はどんな人も見捨てない。
 

 
\\


{\scshape 2 Praeterea}, si Deus aliquem hominem
reprobat, oportet quod sic se habeat reprobatio ad reprobatos, sicut
praedestinatio ad praedestinatos. Sed praedestinatio est causa salutis
praedestinatorum. Ergo reprobatio erit causa perditionis reproborum. Hoc
autem est falsum, dicitur enim Osee {\scshape xiii}, {\itshape perditio tua, Israel, ex te
 est; tantummodo ex me auxilium tuum}. Non ergo Deus aliquem reprobat.

 &

さらに、もし神がだれかを見捨てるとしたら、見捨てることと見捨てられる人
との関係は、予定することと予定される人との関係と同じはずである。ところ
が、予定することは、予定される人々の救いの原因である。ゆえに、見捨てる
ことは、見捨てられた人々の破滅の原因だということになる。しかしこれは誤
りである。なぜなら、『ホセア書』第13章「イスラエルよ、あなたの破滅は、
あなたから来る。私から来るのは、あなたの助けだけである」\footnote{「イ
スラエルよ、おまえの破滅が来る。わたしに背いたからだ。おまえの助けであ
るわたしに背いたからだ」(13:9)}と言われているからである。それゆえ、神
は誰も見捨てない。
 
 

\\


{\scshape  3 Praeterea}, nulli debet imputari quod vitare
non potest. Sed si Deus aliquem reprobat, non potest vitare quin ipse
pereat, dicitur enim {\itshape Eccle}.~{\scshape vii}, {\itshape considera opera Dei, quod nemo possit
corrigere quem ipse despexerit}. Ergo non esset hominibus imputandum quod
pereunt. Hoc autem est falsum. Non ergo Deus aliquem reprobat.


&

さらに、誰にも、避けることができないことの責任が問われることはない。と
ころが、もし神がだれかを見捨てるとすると、その人が滅びることは避けられ
ないことである。なぜなら、『コヘレトの言葉』第7章に「神の業を考えよ。
神自身が見下した人を、だれも矯正することはできない」\footnote{「神の御
業を見よ。神が曲げたものを、誰が直しえようか」(7:13)}。ゆえに、滅びる
ことを、人々の責任にすべきではないことになったであろう。しかしこれは誤
りである。ゆえに、神がだれかを見捨てることはない。
 

\\


{\scshape Sed contra est} quod dicitur {\itshape Malach}.~{\scshape i},
 {\itshape Iacob dilexi, Esau autem odio habui}.


&

 しかし反対に、『マラキ書』第1章で「わたしはヤコブを愛した。しかし、エサ
 ウを憎んだ」\footnote{「しかし、わたしはヤコブを愛し、
 エサウを憎んだ。」(1:2-3)}と言われている。

\\


{\scshape Respondeo dicendum} quod Deus aliquos
reprobat. Dictum enim est supra quod praedestinatio est pars
providentiae. Ad providentiam autem pertinet permittere aliquem defectum
in rebus quae providentiae subduntur, ut supra dictum est. Unde, cum per
divinam providentiam homines in vitam aeternam ordinentur, pertinet
etiam ad divinam providentiam, ut permittat aliquos ab isto fine
 deficere. Et hoc dicitur {\itshape reprobare}.

 &

解答する。以下のように言われるべきである。神はある人々を見捨てる。理由
は以下の通り。予定は摂理の一部だと前に述べられた。ところで、前に言われ
たように、摂理には、摂理のもとに服する諸事物において、何らかの欠陥を容
認することが属する。したがって、神の摂理によって、人間は、永遠の生命へ
と秩序付けられるのだから、神の摂理には、ある人々がこの目的に到達しない
ことを容認することもまた属する。そしてこれが、「見捨てる」と言われる。

 \\


 Sic igitur, sicut praedestinatio est
pars providentiae respectu eorum qui divinitus ordinantur in aeternam
salutem; ita reprobatio est pars providentiae respectu illorum qui ab
hoc fine decidunt. Unde reprobatio non nominat praescientiam tantum, sed
aliquid addit secundum rationem, sicut et providentia, ut supra dictum
est. Sicut enim praedestinatio includit voluntatem conferendi gratiam et
gloriam, ita reprobatio includit voluntatem permittendi aliquem cadere
in culpam, et inferendi damnationis poenam pro culpa.


&

ゆえに、神によって永遠の救済へ秩序付けられている人々にかんして、予定が
摂理の一部であるように、この目的に達しない人々にかんして、見捨ては摂理
の一部である。したがって、見捨ては、ただ予知をそう呼ぶのに留まらず、概
念上、それに何かを加える。これは、前に述べられたとおり、摂理の場合と同
様である。つまり、予定が、恩恵と栄光をもたらす意志を含むように、見捨て
は、ある人が罪へ落ちることを容認する意志と、その罪に対して、罰として永
遠の地獄行きをもたらす意志を含む。
 

\\


{\scshape Ad primum ergo dicendum} quod Deus omnes
homines diligit, et etiam omnes creaturas, inquantum omnibus vult
aliquod bonum, non tamen quodcumque bonum vult omnibus. Inquantum igitur
quibusdam non vult hoc bonum quod est vita aeterna, dicitur eos habere
odio, vel reprobare.


&

第一異論に対しては、それゆえ、以下のように言われるべきである。神は、万
物に何らかの善を欲するかぎりで、すべての人を愛し、すべての被造物を愛す
る。しかし、万物に、あらゆる善を意志するわけではない。ゆえに、ある人々
に、永遠の生命という善を意志しないかぎりで、彼らを憎み、見捨てると言わ
れる。

 
\\


{\scshape Ad secundum dicendum} quod aliter se habet
reprobatio in causando, quam praedestinatio. Nam praedestinatio est
causa et eius quod expectatur in futura vita a praedestinatis, scilicet
gloriae; et eius quod percipitur in praesenti, scilicet
gratiae. Reprobatio vero non est causa eius quod est in praesenti,
scilicet culpae; sed est causa derelictionis a Deo. Est tamen causa eius
quod redditur in futuro, scilicet poenae aeternae. Sed culpa provenit ex
libero arbitrio eius qui reprobatur et a gratia deseritur. Et secundum
hoc verificatur dictum prophetae, scilicet, {\itshape perditio tua, Israel, ex te}.


&

第二異論に対しては、以下のように言われるべきである。原因することにおい
て、見捨ては予定と異なる関係にある。すなわち、予定は、永遠の生命におい
て、予定された人々によって期待されているもの、つまり栄光と、現在知覚さ
れているもの、つまり恩恵の原因である。これに対して、見捨ては、現在ある
もの、つまり罪の原因ではなく、神による放置の原因である。また、未来に履
行されるもの、つまり永遠の罰の原因である。しかし、罪は、見捨てられ、恩
恵を取り上げられた人の自由裁量に由来する。この点で、預言者の言葉、すな
わち「イスラエルよ、あなたの滅びはあなたから来る」は正しい。

 
\\


{\scshape Ad tertium dicendum} quod reprobatio Dei non
subtrahit aliquid de potentia reprobati. Unde, cum dicitur quod
reprobatus non potest gratiam adipisci, non est hoc intelligendum
secundum impossibilitatem absolutam, sed secundum impossibilitatem
conditionatam, sicut supra dictum est quod praedestinatum necesse est
salvari, necessitate conditionata, quae non tollit libertatem
arbitrii. Unde, licet aliquis non possit gratiam adipisci qui reprobatur
a Deo, tamen quod in hoc peccatum vel illud labatur, ex eius libero
arbitrio contingit. Unde et merito sibi imputatur in culpam.


&

第三異論に対しては、以下のように言われるべきである。神の見捨ては、見捨
てられた人の能力から何かを取り去るものではない。したがって、見捨てられ
た人は恩恵を獲得できないと言われるとき、これは無条件的な不可能性におい
て理解されるべきではなく、条件付きの不可能性において理解されるべきであ
る。それはちょうど、前に\footnote{第19問第8項第1異論解答。}、予定され
た人が救われることは必然的である、ということが、自由選択を否定しない、
条件付きの必然性によって言われたように。したがって、神に見捨てられた人
は恩恵を獲得することができないが、これやあれの罪に陥るのは、その人の自
由裁量から生じる。ゆえに、自らに罪の責任が問われるのは当然のことである。

 
\end{longtable}
\newpage





\rhead{a.~4}
\begin{center}
 {\Large {\bf ARITICULUS QUARTUS}}\\
 {\large UTRUM PRAEDESTINATI ELIGATUR A DEO}\\
 {\footnotesize I {\itshape Sent.}, d.41, a.2; {\itshape De Verit.},
 q.6, a.2; {\itshape ad Rom.}, cap.9, lect.2.}\\
 {\Large 第四項\\予定された人たちは神に選ばれているか}
\end{center}

\begin{longtable}{p{21em}p{21em}}

{\Huge A}{\scshape d quartum sic proceditur}. Videtur quod
 praedestinati non eligantur a Deo. Dicit enim Dionysius, {\scshape iv} cap.~{\itshape de Div.~Nom}., quod, sicut sol corporeus non eligendo omnibus corporibus
 lumen immittit, ita et Deus suam bonitatem. Sed bonitas divina
 communicatur praecipue aliquibus secundum participationem gratiae et
 gloriae. Ergo Deus absque electione gratiam et gloriam communicat. Quod
 ad praedestinationem pertinet.


&

第四項の問題へ、議論は以下のように進められる。予定された人たちは、神に
よって選ばれるのではないと思われる。理由は以下の通り。ディオニュシウス
は『神名論』第4章で、ちょうど、物体的な太陽が、選ぶことなく、すべての
物体に光を放出するように、神もまた、自らの善性を[選ぶことなく、すべて
のものに放出する]。ところで、神の善性は、特に、恩恵と栄光の分有という
点で、あるものどもに伝達される。ゆえに、神は、選択することなく、恩恵と
栄光を伝達する。このことは、予定に属する。


\\



{\scshape 2 Praeterea}, electio est eorum quae sunt. Sed
 praedestinatio ab aeterno est etiam eorum quae non sunt. Ergo
 praedestinantur aliqui absque electione.


&

さらに、選択は、存在するものどもにかかわる。ところが、予定は、永遠から、
存在しないものどもについてもある。ゆえに、選択なしに、ある人たちは予定
される。


\\



{\scshape 3 Praeterea}, electio quandam discretionem
 importat. Sed Deus {\itshape vult omnes homines salvos fieri}, ut dicitur I
 {\itshape Tim}.~{\scshape ii}. Ergo praedestinatio, quae praeordinat homines in salutem, est
 absque electione.


&

さらに、選択は、ある種の区別を含意する。しかし神は、『テモテへの手紙一』
第2章で言われるように、「すべての人々が救われた者となることを意志する」
\footnote{神はすべての人々が救われて真理を知るようになることを望んでお
られます」(2:4)}。ゆえに、予定、これは人々を救いへとあらかじめ秩序付け
ることであるが、には選択がない。

\\



{\scshape Sed contra est} quod dicitur {\itshape Ephes}.~{\scshape i}, {\itshape Elegit
 nos in ipso ante mundi constitutionem}.


&

しかし反対に、『エフェソの信徒への手紙』第1章では「世界の建設以前に、
それ自身において私たちを選んだ」\footnote{「天地創造の前に、神は私たち
を愛して、御自分の前で聖なる者、汚れのない者にしようと、キリストにおい
てお選びになりました。」(1:4)}と言われている。

\\



{\scshape Respondeo dicendum} quod praedestinatio,
 secundum rationem, praesupponit electionem; et electio
 dilectionem. Cuius ratio est, quia praedestinatio, ut dictum est, est
 pars providentiae. Providentia autem, sicut et prudentia, est ratio in
 intellectu existens, praeceptiva ordinationis aliquorum in finem, ut
 supra dictum est. 


&

解答する。以下のように言われるべきである。予定は、概念上、選択を前提と
する。そして、選択は、愛を前提とする。その理由は以下の通りである。すな
わち、予定は、すでに述べられたように、摂理の一部である。ところが、前に
述べられたとおり、摂理は、思慮もまたそうであるように、あるものどもの目
的への秩序付けを指令しうる、知性の中に存在する理拠である。


\\


Non autem praecipitur aliquid ordinandum in finem,
 nisi praeexistente voluntate finis. Unde praedestinatio aliquorum in
 salutem aeternam, praesupponit, secundum rationem, quod Deus illorum
 velit salutem. Ad quod pertinet electio et dilectio. Dilectio quidem,
 inquantum vult eis hoc bonum salutis aeternae, nam diligere est velle
 alicui bonum, ut supra dictum est. 



&


ところで、目的へと秩序付けられたどんなものも、目的への意志の中に先在し
ないならば、指令されることはない。したがって、ある人々を永遠の救済へ予
定することは、概念上、神が、その人々の救いを意志することを前提とする。
これに、選択と愛とが属する。愛が属するのは、愛が、前に述べられたとおり、
ある人に善を意志することだから、永遠の救済という善をその人たちに欲する
かぎりにおいてである。

\\


Electio autem, inquantum hoc bonum
 aliquibus prae aliis vult, cum quosdam reprobet, ut supra dictum
 est. Electio tamen et dilectio aliter ordinantur in nobis et in Deo, eo
 quod in nobis voluntas diligendo non causat bonum; sed ex bono
 praeexistente incitamur ad diligendum. Et ideo eligimus aliquem, quem
 diligamus, et sic electio dilectionem praecedit in nobis. 


&

他方、選択が属するのは、ある人々を見捨てるのだから、この善を、他の人々
ではなくその人々に意志するかぎりでである。しかし、選択と愛は、私たちに
おいてと、神においてとでは、異なるかたちで秩序付けられている。なぜなら、
私たちにおいて、意志は、愛することによって善の原因となることはなく、む
しろ、先に存在する善にもとづいて、愛することへと駆り立てられる。それゆ
え、私たちは、ある人を選び、その人を愛する。このように、私たちにおいて
は、選択は愛に先行する。


\\


In Deo autem
 est e converso. Nam voluntas eius, qua vult bonum alicui diligendo, est
 causa quod illud bonum ab eo prae aliis habeatur. Et sic patet quod
 dilectio praesupponitur electioni, secundum rationem; et electio
 praedestinationi. Unde omnes praedestinati sunt electi et dilecti.


&

他方、神においては逆である。すなわち、神の意志は、それによって、愛され
るべき人に善を意志するのだが、その善が、他の人々ではなくその人に持たれ
ることの原因である。このようにして、概念上、愛が、選択にとって、前提と
され、選択が、予定にとって前提とされることが明らかである。したがって、
すべて予定されている人々は、選ばれ、愛される。


\\



{\scshape Ad primum ergo dicendum} quod, si consideretur
 communicatio bonitatis divinae in communi, absque electione bonitatem
 suam communicat; inquantum scilicet nihil est, quod non participet
 aliquid de bonitate eius, ut supra dictum est. Sed si consideretur
 communicatio istius vel illius boni, non absque electione tribuit, quia
 quaedam bona dat aliquibus, quae non dat aliis. Et sic in collatione
 gratiae et gloriae attenditur electio.


&

第一異論に対しては、それゆえ、以下のように言われるべきである。もし、神
の善性の伝達が、共通的に語られるのであれば、神は選択なしに自らの善性を
伝達する。だがその意味は、前に述べられたとおり、神の善性を少しも分有し
ないものは、存在しないということにすぎない。しかし、この善やあの善の伝
達が考察されるならば、選択なしに与えるわけではない。なぜなら、ある善を
あるものどもに与えるが、他のものどもには与えないからである。このように
して、恩恵と栄光の分配において、選択が見出される。


\\



{\scshape Ad secundum dicendum} quod, quando voluntas
 eligentis provocatur ad eligendum a bono in re praeexistente, tunc
 oportet quod electio sit eorum quae sunt; sicut accidit in electione
 nostra. Sed in Deo est aliter, ut dictum est. Et ideo, sicut dicit
 Augustinus, {\itshape eliguntur a Deo qui non sunt, neque tamen errat qui eligit}.


&

第二異論に対しては、以下のように言われるべきである。選択する者の意志が、
現実に前もって存在する善によって、選択することへと呼び出されるとき、そ
の選択は存在するものどもについてでなければならない。ちょうど私たちの選
択においてそうであるように。しかし、神においては、すでに述べられたとお
り、異なっている。ゆえに、アウグスティヌスが言うとおり、「存在していな
い人々が神によって選ばれるが、しかし、選ぶ者は誤らない」。


\\



{\scshape Ad tertium dicendum} quod, sicut supra dictum
 est, Deus vult omnes homines salvos fieri antecedenter, quod non est
 simpliciter velle, sed secundum quid, non autem consequenter, quod est
 simpliciter velle.


&

第三異論に対しては、以下のように言われるべきである。前に述べられたとお
り\footnote{第19問第6項}、神は、「先行するかたちで」すべての人間が救わ
れた者となることを意志するが、「後続するかたちで」そう意志するわけでは
ない。前者は端的にではなく、ある意味で意志することだが、後者は端的に意
志することである。





\end{longtable}
\newpage




\rhead{a.~5}
\begin{center}
 {\Large {\bf ARTICULUS QUINTUS}}\\
 {\large UTRUM PRAESCIENTIA MERITORUM SIT CAUSA PRAEDESTINATIONIS}\\
 {\footnotesize I {\itshape Sent.}, d.41, a.3; III {\itshape
 SCG}. cap.163; {\itshape De Verit.}, q.6, a.2; in {\itshape Ioan.},
 cap.15, lect.3; {\itshape ad Rom.}, cap.1, lect.3; cap.8, lect.6;
 cap.9, lect.3; {\itshape ad Ephes.}, cap.1, lect.1, 4.}\\
 {\Large 第五項\\功績の予知が予定の原因か}\footnote{「この世で善行を積むこ
 とがわかっているから、神はその人を救いに予定するのか」ということ。}
\end{center}

\begin{longtable}{p{21em}p{21em}}


{\Huge A}{\scshape d quintum sic proceditur}. Videtur quod
praescientia meritorum sit causa praedestinationis. Dicit enim
apostolus, {\itshape Rom}.~{\scshape viii}, {\itshape quos praescivit, hos et praedestinavit}. Et Glossa
Ambrosii, super illud {\itshape Rom}.~{\scshape ix}, {\itshape miserebor cui miserebor etc., dicit,
misericordiam illi dabo, quem praescio toto corde reversurum ad me}. Ergo
videtur quod praescientia meritorum sit causa praedestinationis.


&

第五項の問題へ、議論は以下のように進められる。功績の予知が予定の原因で
あると思われる。理由は以下の通り。使徒は、『ローマの信徒への手紙』8章
で「予知した人々、この人々を、予定しもした」\footnote{「神は前もって知っ
ておられた者たちを、御子の姿に似たものとしようとあらかじめ定められまし
た」(8:29)}と言う。そしてアンブロシウスの『注釈』は、かの『ローマの信
徒への手紙』9章「私は憐れもうと思うものを憐れむ、云々」\footnote{神は
モーセに「わたしは自分が憐れもうと思う者を憐れみ、慈しもうと思う者を慈
しむ」と言っておられます。(9:14)}について、「私は、憐れみを、心のすべ
てで私に向かうだろうと予知している人に与えるだろう」と述べている。ゆえ
に、功績ある人々についての予知が、予定の原因であると思われる。



\\


{\scshape 2 Praeterea}, praedestinatio divina includit
divinam voluntatem, quae irrationabilis esse non potest, cum
praedestinatio sit {\itshape propositum miserendi}, ut Augustinus dicit. Sed nulla
alia ratio potest esse praedestinationis nisi praescientia
meritorum. Ergo praescientia meritorum est causa vel ratio
praedestinationis.


&

さらに、アウグスティヌスが述べるように、予定とは「憐れまれるべきことの
計画」なのだから、神の予定は、神の意志を含み、そしてその意志は不合理で
はありえない。ところが、予定ということの根拠は、功績の予知以外にはあり
えない。ゆえに、功績の予知は、予定の原因、あるいは根拠である。


\\


{\scshape 3 Praeterea}, {\itshape non est iniquitas apud Deum}, ut
dicitur {\itshape Rom}.~{\scshape ix}. Iniquum autem esse videtur, ut aequalibus inaequalia
dentur. Omnes autem homines sunt aequales et secundum naturam, et
secundum peccatum originale, attenditur autem in eis inaequalitas
secundum merita vel demerita propriorum actuum. Non igitur inaequalia
praeparat Deus hominibus, praedestinando et reprobando, nisi propter
differentium meritorum praescientiam.


&

さらに、『ローマの信徒への手紙』9章で述べられるように、「神のもとに不
義はない」\footnote{「神に不義があるのか。決してそうではない」(9:14)}。
しかし、等しいものに等しくないものが与えられるならば、それは不義である。
ところで、すべての人間は、本性においても原罪においても等しい。ところが、
それぞれの行為の功績と罪過に即して、人間の中には不等さが見出される。ゆ
えに、予定し、劫罰をくだすことにおいて、神が人間たちに等しくない事柄を
準備するのは、異なる功績の予知のために他ならない。


\\


{\scshape Sed contra est} quod dicit Apostolus, {\itshape ad
Tit}.~{\scshape iii}, {\itshape non ex operibus iustitiae, quae fecimus nos, sed secundum suam
misericordiam salvos nos fecit}. Sicut autem salvos nos fecit, ita et
praedestinavit nos salvos fieri. Non ergo praescientia meritorum est
causa vel ratio praedestinationis.


&

しかし反対に、使徒は『テトスへの手紙』3章で「私たちが行った正義の業に
よってではなく、自身の憐れみに従って、私たちを救われたものにした」
\footnote{「神は、わたしたちが行った義の業によってではなく、御自分の憐
れみによって、私たちを救ってくださいました」(3:5)}と述べている。ところ
で、神は私たちを救われたものにしたように、私たちが救われたものとなるよ
うに予定した。ゆえに、功績の予知は、予定の原因や根拠ではない。

\\


{\scshape Respondeo dicendum} quod, cum praedestinatio
includat voluntatem, ut supra dictum est, sic inquirenda est ratio
praedestinationis, sicut inquiritur ratio divinae voluntatis. Dictum est
autem supra quod non est assignare causam divinae voluntatis ex parte
actus volendi; sed potest assignari ratio ex parte volitorum, inquantum
scilicet Deus vult esse aliquid propter aliud. Nullus ergo fuit ita
insanae mentis, qui diceret merita esse causam divinae
praedestinationis, ex parte actus praedestinantis. Sed hoc sub
quaestione vertitur, utrum ex parte effectus, praedestinatio habeat
aliquam causam. Et hoc est quaerere, utrum Deus praeordinaverit se
daturum effectum praedestinationis alicui, propter merita
aliqua. 



&


解答する。以下のように言われるべきである。前に述べられたとおり、予定は
意志を含むので、予定の根拠は、神の意志の根拠が探究されるようにして、探
究されるべきである。さて、神の意志の原因を、意志する働きの側から指定す
ることはできないが、神が「あるものが他のもののために存在する」ことを意
志するかぎりで、意志されるものどもの側から、根拠が指定されることは可能
である、ということが、前に\footnote{第19問第5項。}述べられた。ゆえに、
功績が、予定する者の側から、神の予定の原因である、と語るほど精神が不健
康な者はいなかった。しかし、結果の側から、予定が何らかの原因をもつかど
うかということは、問題とされる。そしてこのことは、神が、何らかの功績の
ために、ある人に予定の結果を与えるように、自らをあらかじめ秩序付けたか
どうか、と問うことでもある。


\\



Fuerunt igitur quidam, qui dixerunt quod effectus
praedestinationis praeordinatur alicui propter merita praeexistentia in
alia vita. Et haec fuit positio Origenis, qui posuit animas humanas ab
initio creatas, et secundum diversitatem suorum operum, diversos status
eas sortiri in hoc mundo corporibus un\={\i}tas. 



&

そういうわけで、予定の結果は、他の生(前世)ですでにあった功績のために、
ある人にあらかじめ定められていると言った人々がいた。そしてこれはオリゲ
ネスの立場であり、彼は、人間の魂を、原初から創造され、それらの業の多様
さに従って、様々な境遇(status)が割り当てられ、この世で身体に結びつけ
られている、と論じた。



\\



-- Sed hanc opinionem excludit
apostolus, {\itshape Rom}.~{\scshape ix}, dicens, {\itshape cum nondum nati fuissent, aut aliquid
egissent boni vel mali, non ex operibus, sed ex vocante dictum est, quia
maior serviet minori}. 


&

しかし、使徒は『ローマの信徒への手紙』第9章で、この意見を排除している。
彼は言う。「まだ彼らが生まれておらず、善や悪のなにも行わなかったときに、
業の側ではなく、呼ぶ者の側から、大きいものが小さいものに仕えるだろう、
と語られた」\footnote{「その子供たちがまだ生まれもせず、善いことも悪い
こともしていないのに「兄は弟に仕えるであろう」とリベカに告げられまし
た。」(9:11-12)}。


\\


Fuerunt ergo alii, qui dixerunt quod merita
praeexistentia in hac vita sunt ratio et causa effectus
praedestinationis. Posuerunt enim Pelagiani quod initium benefaciendi
sit ex nobis, consummatio autem a Deo. Et sic, ex hoc contingit quod
alicui datur praedestinationis effectus, et non alteri, quia unus
initium dedit se praeparando, et non alius. 


&

それで、この生で、あらかじめ存在している功績が、予定の結果の根拠であり
原因である、と語った他の人々がいた。すなわち、ペラギウス派の人々は、善
行の端緒は私たちからだが、その成就は神によってある、と論じた。こうして、
予定の結果がある人には与えられ、別の人に与えられないのは、一方が、端緒
を与えたのに、他方はそうしなかったからだ、ということになる。


\\

-- Sed contra hoc est quod
dicit apostolus, II {\itshape Cor}.~{\scshape iii}, quod {\itshape non sumus sufficientes cogitare
aliquid a nobis, quasi ex nobis}. Nullum autem anterius principium
inveniri potest quam cogitatio. Unde non potest dici quod aliquod in
nobis initium existat, quod sit ratio effectus praedestinationis. 




&

しかし、使徒が『コリントの信徒への手紙二』第3章で「私たちは、何かを、
あたかも私たちに基づいてであるかのように、私たちによって思惟するには十
分でない」\footnote{「もちろん、独りで何かできるなどと思う資格が、自分
にあるということではありません」(3:5)}と語ることは、これに反する。とこ
ろで、思惟以上に先行する根源は見出されえない。したがって、私たちの中に、
何らかの端緒が存在し、それが、予定の結果の根拠だと言われることはできな
い。

\\



Unde
fuerunt alii, qui dixerunt quod merita sequentia praedestinationis
effectum, sunt ratio praedestinationis, ut intelligatur quod ideo Deus
dat gratiam alicui, et praeordinavit se ei daturum, quia praescivit eum
bene usurum gratia; sicut si rex det alicui militi equum, quem scit eo
bene usurum. 


&

そこで、予定の結果に伴う功績が、予定の根拠だと言った他の人々がいた。つ
まり、神は、ある人が善く恩恵を用いるだろうことを予知したので、その人に
恩恵を与え、また、自らがその人に[恩恵を]与えるだろうことを予め定めた、
というわけである。ちょうど、王が、だれかの兵士に馬を与える場合、王が、
この兵士は馬を上手に使うだろうということを知っているような、その兵士に
与える、というように。馬を与えるかのように。

\\

-- Sed isti videntur distinxisse inter id quod est ex gratia,
et id quod est ex libero arbitrio, quasi non possit esse idem ex
utroque. Manifestum est autem quod id quod est gratiae, est
praedestinationis effectus, et hoc non potest poni ut ratio
praedestinationis, cum hoc sub praedestinatione concludatur. Si igitur
aliquid aliud ex parte nostra sit ratio praedestinationis, hoc erit
praeter effectum praedestinationis. Non est autem distinctum quod est ex
libero arbitrio, et ex praedestinatione; sicut nec est distinctum quod
est ex causa secunda, et causa prima, divina enim providentia producit
effectus per operationes causarum secundarum, ut supra dictum est. Unde
et id quod est per liberum arbitrium, est ex praedestinatione. 



&

しかし、この人々は、恩恵に基づくものと、自由裁量に基づくものとを、あた
かも両者から同一のことが生じることがありえないかのように、区別していた
と思われる。しかし、恩恵に属するものは、予定の結果であり、それは予定に
含まれるものだから、予定の根拠とされることができないのは明らかである。
ゆえに、もし、何か別のものが、私たちの側から、予定の根拠であるならば、
それは、予定の結果以外のものであるだろう。しかし、自由裁量に基づくもの
と、予定に基づくものとは、区別されない。ちょうど、第二原因から生じるも
のと、第一原因から生じるものとが区別されないように。じっさい、神の摂理
は、前に述べられたとおり、諸々の第二原因の働きを通して、結果を生み出す
のである。したがって、自由裁量によってあるものもまた、予定に基づいてい
る。



\\


Dicendum
est ergo quod effectum praedestinationis considerare possumus
dupliciter. Uno modo, in particulari. Et sic nihil prohibet aliquem
effectum praedestinationis esse causam et rationem alterius, posteriorem
quidem prioris, secundum rationem causae finalis; priorem vero
posterioris, secundum rationem causae meritoriae, quae reducitur ad
dispositionem materiae. Sicut si dicamus quod Deus praeordinavit se
daturum alicui gloriam ex meritis; et quod praeordinavit se daturum
alicui gratiam, ut mereretur gloriam. 



&

それゆえ、以下のように言われるべきである。私たちは、予定の結果を、二通
りに考察することができる。一つには、個別的にである。この場合、予定の何
らかの結果が、他のものの原因や根拠であることを妨げるものはない。じっさ
い、目的因の観点では、後のものは先のもの根拠であり、質料を整えること
\footnote{功績は恩恵に対して質料が形相に対するように関係する。}に帰せ
られる功績因(causa meritoria)の観点では、先のものが後のものの根拠であ
る、と言って差し支えない。ちょうど、私たちが、「神は、自らが、功績に基
づいて或る人に恩恵を与えるだろうことを予め定めた」と言い、また、「神は
或る人に、その人が栄光を獲得するように、恩恵を与えるだろうことを予め定
めた」と言うように。



\\



-- Alio modo potest considerari
praedestinationis effectus in communi. Et sic impossibile est quod totus
praedestinationis effectus in communi habeat aliquam causam ex parte
nostra. Quia quidquid est in homine ordinans ipsum in salutem,
comprehenditur totum sub effectu praedestinationis, etiam ipsa
praeparatio ad gratiam, neque enim hoc fit nisi per auxilium divinum,
secundum illud {\itshape Thren}.~ultimi, {\itshape converte nos, domine, ad te, et
convertemur}. Habet tamen hoc modo praedestinatio, ex parte effectus, pro
ratione divinam bonitatem; ad quam totus effectus praedestinationis
ordinatur ut in finem, et ex qua procedit sicut ex principio primo
movente.


&

もう一つには、予定の結果が、共通的に考察されうる。この場合、予定の結果
全体が、共通に、私たちの側から何らかの原因をもつことは不可能である。な
ぜなら、人間の中にあって、人間を救済へと秩序付けるものは何であれ、その
全体が、予定の結果のもとに含まれているからである。それは、恩恵への準備
それ自体ですらそうである。なぜなら、『哀歌』最後の「私たちを向き直らせ
よ、主よ。あなたへと。そうすれば、私たちは向き直るだろう」
\footnote{「主よ、御もとに立ち帰らせてください、私たちは立ち帰ります」
(5:21)}によれば、このこともまた、神の助けがなければ起こらないからであ
る。しかし、この意味、つまり結果の側から見れば、予定は、神の善性を根拠
としてもつのであり、予定の結果全体は、それへと、目的へ秩序付けられるよ
うにして秩序付けられ、また、そこから、第一の動者としての根源からである
ようにして、発出する。


\\


{\scshape Ad primum ergo dicendum} quod usus gratiae
praescitus, non est ratio collationis gratiae, nisi secundum rationem
causae finalis, ut dictum est.


&

第一異論に対しては、それゆえ、以下のように言われるべきである。予知され
ている恩恵の使用は、すでに述べられたとおり、目的因の観点においてでなけ
れば、恩恵を与える根拠でない。



\\


{\scshape Ad secundum dicendum} quod praedestinatio
habet rationem ex parte effectus, in communi, ipsam divinam
bonitatem. In particulari autem, unus effectus est ratio alterius, ut
dictum est.


&

第二異論に対しては、以下のように言われるべきである。すでに述べられたと
おり、予定は、結果の側から、共通的に[考察された場合に]、神の善性その
ものという根拠をもつ。他方、個別的には、ある一つの結果が、別の結果の根
拠である。


\\


{\scshape Ad tertium dicendum} quod ex ipsa bonitate
divina ratio sumi potest praedestinationis aliquorum, et reprobationis
aliorum. Sic enim Deus dicitur omnia propter suam bonitatem fecisse, ut
in rebus divina bonitas repraesentetur. Necesse est autem quod divina
bonitas, quae in se est una et simplex, multiformiter repraesentetur in
rebus; propter hoc quod res creatae ad simplicitatem divinam attingere
non possunt. Et inde est quod ad completionem universi requiruntur
diversi gradus rerum, quarum quaedam altum, et quaedam infimum locum
teneant in universo. Et ut multiformitas graduum conservetur in rebus,
Deus permittit aliqua mala fieri, ne multa bona impediantur, ut supra
dictum est. 


&

第三異論に対しては、以下のように言われるべきである。神の善性そのものか
ら、ある人々を予定に、ある人々を劫罰に付する根拠が取られうる。理由は以
下の通り。神が万物を自らの善性のために作ったのは、諸事物の中に神の善性
が表現されるためにである。ところで、それ自体において一つで単純なもので
ある神の善性は、創造された諸事物が、神の単純性に到達しえないために、諸
事物の中に、さまざまな形をとって表現さえる必要がある。このことから、宇
宙の完成のためには、諸事物の様々な階級があり、そのうちのあるものは、宇
宙の中で高い階級を、あるものは最も低い階級を持つことが必要とされる。そ
して、諸階層のさまざまな形が諸事物の中に保持されるように、前に述べられ
たとおり、神は、多くの善が妨げられないように、ある悪が生じることを許す。



\\



Sic igitur consideremus totum genus humanum, sicut totam
rerum universitatem. Voluit igitur Deus in hominibus, quantum ad
aliquos, quos praedestinat, suam repraesentare bonitatem per modum
misericordiae, parcendo; et quantum ad aliquos, quos reprobat, per modum
iustitiae, puniendo. Et haec est ratio quare Deus quosdam eligit, et
quosdam reprobat. Et hanc causam assignat apostolus, {\itshape ad Rom}.~{\scshape ix}, dicens,
{\itshape volens Deus ostendere iram} (idest vindictam iustitiae), {\itshape et notam facere
potentiam suam, sustinuit} (idest permisit) {\itshape in multa patientia, vasa irae
apta in interitum, ut ostenderet divitias gloriae suae in vasa
misericordiae, quae praeparavit in gloriam}. Et II {\itshape Tim}.~{\scshape ii} dicit, {\itshape in
magna autem domo non solum sunt vasa aurea et argentea, sed etiam lignea
et fictilia; et quaedam quidem in honorem, quaedam in contumeliam}. 


&

ゆえに、私たちは、人類全体を、諸事物の宇宙全体のように考察することがで
きる。つまり神は、人間において、ある人々にかんしては、彼らを予定し、自
らの善性を、憐れみというかたちで、つまり許すことによって表現し、別のあ
る人々にかんしては、彼らを見捨て、正義というかたちで、つまり罰すること
によって表現することを意志した。そしてこれが、神がある人々を選び、ある
人々を見捨てる理由である。使徒は、『ローマの信徒への手紙』第9章で、次
のように述べて、この原因を指定している。「怒り(すなわち正義の主張)を
示し、自らの力を知らせることを欲する神が、大いなる受難の中で、滅びやす
い怒りの器を、栄光へと準備した憐れみの器の中に自らの栄光の富を示すため
に、耐えた(すなわち、許した)」\footnote{「神はその怒りを示し、その力
を知らせようとしておられたが、怒りの器として滅びることになっていた者た
ちを寛大な心で耐え忍ばれたとすれば、それも、憐れみの器として栄光を与え
ようと準備しておられた者たちに、御自分の豊かな栄光をお示しになるためで
あったとすれば、どうでしょう」(9:22-23)}。また、『テモテへの手紙二』第
2章で、次のように言う。「大きな家の中には、金の器や銀の器だけでなく、
木や土の器もある。そして、あるものは大切に、あるものは雑に扱われる」
\footnote{「さて、大きな家には金や銀の器だけでなく、木や土の器もありま
す。一方は貴いことに、他方は普通のことに用いられます」(2:20)}。


\\



Sed
quare hos elegit in gloriam, et illos reprobavit, non habet rationem
nisi divinam voluntatem. Unde Augustinus dicit, {\itshape super Ioannem}, {\itshape quare
hunc trahat, et illum non trahat, noli velle diiudicare, si non vis
errare}. 


&

しかし、なぜ、この人々を栄光へと選び、かの人々を見捨てたのかは、神の意
志以外の根拠がない。このことから、アウグスティヌスは、『ヨハネに夜福音
書注解』で、「なぜこれを取り、あれを取らないか、判断しようとするな。も
しあなたが誤りたくないのなら」と述べている。

\\


Sicut etiam in rebus naturalibus potest assignari ratio, cum
prima materia tota sit in se uniformis, quare una pars eius est sub
forma ignis, et alia sub forma terrae, a Deo in principio condita, ut
scilicet sit diversitas specierum in rebus naturalibus. 

&



これはまた、自然的諸事物の中でも、根拠が指定されうる。第一質料全体は、
それ自体一つのかたちなのに、なぜ、初めに神によって、その一つの部分は火
の形相のもとに、別の部分は土の形相のもとに作られたのか。それは、自然的
諸事物の中に種の多様性があるためにである。


\\



Sed quare haec
pars materiae est sub ista forma, et illa sub alia, dependet ex simplici
divina voluntate. Sicut ex simplici voluntate artificis dependet, quod
ille lapis est in ista parte parietis, et ille in alia, quamvis ratio
artis habeat quod aliqui sint in hac, et aliqui sint in illa. 


&

しかし、なぜ、質料のこの部分がこの形相のもとに、かの部分があの形相のも
とに作られたかは、たんに神の意志に依存する。それはちょうど、技術の理論
が、ある程度の石がここに、ある程度の石があそこにあることを示していると
しても、この石が壁のこの部分に、あの石があの部分にあることは、技術者の
単純な意志に依存するのと同様である。


\\


Neque
tamen propter hoc est iniquitas apud Deum, si inaequalia non
inaequalibus praeparat. Hoc enim esset contra iustitiae rationem, si
praedestinationis effectus ex debito redderetur, et non daretur ex
gratia. In his enim quae ex gratia dantur, potest aliquis pro libito suo
dare cui vult, plus vel minus, dummodo nulli subtrahat debitum, absque
praeiudicio iustitiae. Et hoc est quod dicit paterfamilias, Matt.~{\scshape xx},
{\itshape tolle quod tuum est, et vade. An non licet mihi quod volo, facere?}

&

しかし、不等でないものどもに、等しくないものを準備するからといって、神
のもとに選り好みがあるわけではない。なぜなら、もし、予定の結果が、恩恵
によって与えられるのではなく、債務を返済することであったならば、それは
正義の性格に反したであろうからである。というのも、恩恵から与えられるも
のどもにおいては、誰からも債務を奪わないかぎり\footnote{i.e.債務を履行
するかぎり}、正義を傷つけることなく、或る人が、自らの欲するままに、欲
する人に、多くまたは少なく、与えることができるからである。そしてこのこ
とが、『マタイによる福音書』第20章で、家長が「あなたのものを取って去れ。
私がしたいことをすることを、私には許されないのか」\footnote{「自分の分
を受け取って帰りなさい。(中略)自分のものを自分のしたいようにしては、
いけないか。」(20:14-15)}と語っていることである。




\end{longtable}
\newpage







\rhead{a.~6}
\begin{center}
 {\Large {\bf ARTICULUS SEXTUS}}\\
 {\large UTRUM PRAEDESTINATIO SIT CERTA}\\
 {\footnotesize I {\itshape Sent.}, d.11, q.3; {\itshape De Verit.},
 q.6, a.3; {\itshape Quodl.}, XI, q.3; XII, q.3.}\\
 {\Large 第六項\\予定は確定的か}
\end{center}

\begin{longtable}{p{21em}p{21em}}



{\Huge A}{\scshape d sextum sic proceditur}. Videtur quod
praedestinatio non sit certa. Quia super illud {\itshape Apoc}.~{\scshape iii}, {\itshape tene quod
habes, ne alius accipiat coronam tuam}, dicit Augustinus, quod {\itshape alius non
est accepturus, nisi iste perdiderit}. Potest ergo et acquiri et perdi
corona, quae est praedestinationis effectus. Non est igitur
praedestinatio certa.


&

第六項の問題へ、議論は以下のように進められる。予定は確定的でないと思わ
れる。理由は以下の通り。『ヨハネの黙示録』「あなたが持っているものを保
ちなさい。他の人が、あなたの冠を受け取らないように」\footnote{「わたし
は、すぐに来る。あなたの栄冠をだれにも奪われないように、持っているもの
を固く守りなさい」(3:11)}を注解して、アウグスティヌスは「他の人は、そ
の人が失わないかぎり、受け取らないだろう」と述べている。ゆえに、冠、こ
れは予定の結果だが、は、受け取られることも失われることもありうる。ゆえ
に、予定は確定的でない。


\\


{\scshape 2 Praeterea}, posito possibili, nullum
sequitur impossibile. Possibile est autem aliquem praedestinatum, ut
Petrum, peccare, et tunc occidi. Hoc autem posito, sequitur
praedestinationis effectum frustrari. Hoc igitur non est
impossibile. Non ergo est praedestinatio certa.


&


さらに、可能なものが措定されても、不可能なものは帰結しない。ところで、
ある予定されている人、たとえばペトロが罪を犯し、その時に殺される、とい
うことは可能である。しかし、このことが措定されるならば、予定の結果が無
効になるということが帰結する。ゆえに、このようなことは不可能でない。ゆ
えに、予定は確定的でない。

\\


{\scshape 3 Praeterea}, quidquid Deus potuit,
potest. Sed potuit non praedestinare quem praedestinavit. Ergo nunc
potest non praedestinare. Ergo praedestinatio non est certa.


&

さらに、何であれ神が可能であったことは、現に可能なことである。ところで、
神は、予定した人を予定しないことが可能であった。ゆえに、今、予定しない
ことが可能である。ゆえに、予定は確定的でない。

\\


{\scshape Sed contra est} quod super illud {\itshape Rom}.~{\scshape viii},
{\itshape quos praescivit, et praedestinavit etc., dicit Glossa, praedestinatio
est praescientia et praeparatio beneficiorum Dei, qua certissime
liberantur quicumque liberantur.}


&

しかし反対に、『ローマの信徒への手紙』第8章の「予知した人々を、予定し
もした云々」\footnote{「神は前もって知っておられた者たちを、御子の姿に
似たものにしようとあらかじめさだめられました」(8:29)}について、グレゴ
リウスは「予定は、神の善き業の予知であり準備であり、それによって、この
上なく確実に、自由にされる人は誰でも自由にされる」と述べている。


\\


{\scshape Respondeo dicendum} quod praedestinatio
certissime et infallibiliter consequitur suum effectum, nec tamen
imponit necessitatem, ut scilicet effectus eius ex necessitate
proveniat. Dictum est enim supra quod praedestinatio est pars
providentiae. Sed non omnia quae providentiae subduntur, necessaria
sunt, sed quaedam contingenter eveniunt, secundum conditionem causarum
proximarum, quas ad tales effectus divina providentia ordinavit. Et
tamen providentiae ordo est infallibilis, ut supra ostensum est. Sic
igitur et ordo praedestinationis est certus; et tamen libertas arbitrii
non tollitur, ex qua contingenter provenit praedestinationis
effectus. 

\hspace{2em}Ad hoc etiam consideranda sunt quae supra dicta sunt de divina
scientia et de divina voluntate, quae contingentiam a rebus non tollunt,
licet certissima et infallibilia sint.


&


解答する。以下のように言われるべきである。予定には、この上なく確実に、
誤りなく、その結果が伴う。しかし、その結果が必然的に到来するというよう
にして、必然性を付与することはない。理由は以下の通り。前に\footnote{第
1項。}、予定は摂理の一部だと述べられた。ところで、摂理のもとに服するす
べてのものが、必然的なものではなく、あるものは近接原因の状態に応じて偶
然的に生じる。神の摂理は、そのような原因を、そのような諸結果へ、秩序付
けた。しかし、前に\footnote{第22問第4項。}示されたとおり、摂理の秩序は
誤ることがない。ゆえに、同じように、予定の秩序もまた確定的である。しか
し、このことで、そこから偶然的に予定の結果が生じるところの裁量の自由が
否定されることはない。

\hspace{1zw}このためには、前に神の知\footnote{第13問第14項。}と神の意
 志\footnote{第19問第8項。}について語られたこともまた考察されるべきで
 ある。すなわち、どちらもこの上なく確定的で誤りえないにもかかわらず、
 諸事物から偶然性を除去しない。


\\


{\scshape Ad primum ergo dicendum} quod corona dicitur
esse alicuius, dupliciter. Uno modo, ex praedestinatione divina, et sic
nullus coronam suam amittit. Alio modo, ex merito gratiae, quod enim
meremur, quodammodo nostrum est. Et sic suam coronam aliquis amittere
potest per peccatum mortale sequens. (seq.)


&

第一異論に対しては、それゆえ、以下のように言われるべきである。冠がだれ
かに属するというのは、二通りの意味で言われる。一つには、神の予定にもと
づいてであり、この場合、誰も、自分の冠を失うことはない。もう一つには、
恩恵の功績にもとづいてである。というのも、私たちにふさわしいとされた事
柄は、ある意味で、私たちのものだからである。そしてこの意味では、ある人
が、死に至る罪をあとで犯すことによって、自らの冠を失うことがありうる。


\\

Alius autem illam coronam amissam
accipit, inquantum loco eius subrogatur. Non enim permittit Deus aliquos
cadere, quin alios erigat, secundum illud {\itshape Iob} {\scshape xxxiv}, {\itshape conteret multos et
innumerabiles, et stare faciet alios pro eis}.
Sic enim in locum
Angelorum cadentium substituti sunt homines; et in locum Iudaeorum,
gentiles. Substitutus autem in statum gratiae, etiam quantum ad hoc
coronam cadentis accipit, quod de bonis quae alius fecit, in aeterna
vita gaudebit, in qua unusquisque gaudebit de bonis tam a se quam ab
aliis factis.


&

しかし、他の人が、その人の場所を埋める限りにおいて、この失われた冠を受
け取る。なぜなら、神は、『ヨブ記』第34章「多くの数しれない者たちを打ち
砕き、彼らの代わりに他の人たちを立たせる」\footnote{「数知れない権力者
を打ち倒し、彼らに代えて他の人々を立てられる」(34:24)}によれば、他の者
たちを上げることなしに、ある人たちを落とすことはないからである。じっさ
いこのようにして、墜ちた天使たちの場所に人間が、ユダヤ人たちの場所に異
教徒たちが置かれた。また、恩恵の状態へ置かれた者は、他の者がなした善に
ついて、永遠の生において喜ぶという点でも、墜ちた者の冠を受け取る。永遠
の生においては、各々の者が、自分によってなされた善も、他者によってなさ
れた善も、等しく喜ぶであろう。


\\


{\scshape Ad secundum dicendum} quod, licet sit
possibile eum qui est praedestinatus, mori in peccato mortali, secundum
se consideratum; tamen hoc est impossibile, posito (prout scilicet
ponitur) eum esse praedestinatum. Unde non sequitur quod praedestinatio
falli possit.

&


第二異論に対しては、以下のように言われるべきである。予定された人が、そ
れ自体において考察されるかぎり\footnote{「予定されている」という条件を
無視して考えるかぎり。}、死に至る罪において死ぬことは可能である。しか
し、これは、彼が予定されているという措定のもとでは(つまり、現に措定さ
れているかぎり)、不可能である。したがって、予定が誤りうることは帰結し
ない。

\\


{\scshape Ad tertium dicendum} quod, cum praedestinatio
includat divinam voluntatem, sicut supra dictum est quod Deum velle
aliquid creatum est necessarium ex suppositione, propter immutabilitatem
divinae voluntatis, non tamen absolute; ita dicendum est hic de
praedestinatione. Unde non oportet dicere quod Deus possit non
praedestinare quem praedestinavit, in sensu composito accipiendo; licet,
absolute considerando, Deus possit praedestinare vel non
praedestinare. Sed ex hoc non tollitur praedestinationis certitudo.

&

第三異論に対しては、以下のように言われるべきである。予定は神の意志を含
むので、前に\footnote{第19問第3項。}、神がある被造物を意志することは、
神の意志の普遍性のために、前提によって必然的だが、無条件的には必然的で
ない、と述べられたとおり、ここで、予定についても述べられるべきである。
したがって、複合された意味で理解される場合には\footnote{第14問第13項第
3異論解答、および、このサイトにあるST1-14.pdfのその箇所の注を参照。}、
「神は予定した者を予定しないことができる」と言うべきではない。ただし、
無条件的に考えれば、神は、予定することもしないことも可能であるが。しか
し、このことから、予定が確定的であることが否定されるわけではない。


\end{longtable}
\newpage

\rhead{a.~7}
\begin{center}
 {\Large {\bf ARTICULUS SEPTIMUS}}\\
 {\large UTRUM NUMERUS PRAEDESTINATORUM SIT CERTUS}\\
 {\footnotesize I {\itshape Sent.}, d.11, q.3; {\itshape De Verit.},
 q.6, a.4.}\\
 {\Large 第七項\\予定された者の数は確定的か}
\end{center}

\begin{longtable}{p{21em}p{21em}}


{\Huge A}{\scshape d septimum sic proceditur}. Videtur quod
numerus praedestinatorum non sit certus. Numerus enim cui potest fieri
additio, non est certus. Sed numero praedestinatorum potest fieri
additio, ut videtur, dicitur enim {\itshape Deut}.~{\scshape i}, {\itshape dominus Deus noster addat ad
hunc numerum multa millia}; Glossa, {\itshape idest definitum apud Deum, qui novit
qui sunt eius}. Ergo numerus praedestinatorum non est certus.


&

第七項の問題へ、議論は以下のように進められる。予定された者の数は確定的
でないと思われる。理由は以下の通り。加算されうる数は、確定的でない。と
ころが、予定された者の数は、加算可能だと思われる。なぜなら、『申命記』
第1章で「私たちの主である神は、この数に、幾百万を加える」
\footnote{「あなたたちの神、主が人数を増やされたので、今やあなたたちは
空の星のように数多くなった。あなたたちの先祖の神、主が約束されたとおり、
更に、あなたたちを千倍にも増やして祝福されるように。」(1:10-11)}と言わ
れ、『注釈』は、「すなわち、誰が彼に属しているかを知る神のもとで決まっ
ている[数に]」と述べる。ゆえに、予定された者の数は確定的でない。


\\


{\scshape 2 Praeterea}, non potest assignari ratio quare
magis in hoc numero quam in alio, Deus homines praeordinet ad
salutem. Sed nihil a Deo sine ratione disponitur. Ergo non est certus
numerus salvandorum praeordinatus a Deo.


&


さらに、なぜ神が、この数ではなく別の数の人間を救いへと予め秩序付けるか
についての根拠を示すことはできない。ところが、何ものも、神によって、根
拠なしに為されることはない。ゆえに、神によって予め秩序付けられた、救わ
れる者たちの数が確定しているわけではない。

\\


{\scshape 3 Praeterea}, operatio Dei est perfectior quam
operatio naturae. Sed in operibus naturae bonum invenitur ut in
pluribus, defectus autem et malum ut in paucioribus. Si igitur a Deo
institueretur numerus salvandorum, plures essent salvandi quam
damnandi. Cuius contrarium ostenditur Matt.~{\scshape vii}, ubi dicitur, {\itshape lata et
spatiosa est via quae ducit ad perditionem, et multi sunt qui intrant
per eam, angusta est porta, et arcta via, quae ducit ad vitam, et pauci
sunt qui inveniunt eam}. Non ergo est praeordinatus a Deo numerus
salvandorum.


&

さらに、神の働きは、自然の働きよりも完全である。ところが、自然の業にお
いては、善がより多くのものに、欠陥や悪はより少ないものに見出される。ゆ
えに、もし、神によって、救われる者たちの数が設定されるならば、地獄行き
の者よりも、救われる者の方が多かったであろう。しかし、『マタイによる福
音書』第7章では、この反対のことが示されている。そこでは次のように言わ
れている。「破滅に至る道は幅があり広々として、多くの人々がそれを通って
中に入る。しかし生命に至る門は狭く、道は細く、それを見出す人は少ない」
\footnote{「狭い門から入りなさい。滅びに通じる門は広く、その道も広々と
して、そこから入る者が多い。しかし、命に通じる門はなんと狭く、その道も
細いことか。それを見出すものは少ない。」(7:13-14)}。それゆえ、救われる
者たちの数は、神によって予定されていない。




\\


{\scshape Sed contra est quod} Augustinus dicit, in
libro {\itshape de Correptione et Gratia}, {\itshape certus est praedestinatorum numerus, qui
neque augeri potest, neque minui}.


&

しかし反対に、アウグスティヌスは『懲戒と恩寵について』という本の中で
「予定された者の数は確定的で、増やすことも減らすこともできない」と述べ
ている。


\\


{\scshape Respondeo dicendum} quod numerus
praedestinatorum est certus. Sed quidam dixerunt eum esse certum
formaliter, sed non materialiter, ut puta si diceremus certum esse quod
centum vel mille salventur, non autem quod hi vel illi. Sed hoc tollit
certitudinem praedestinationis, de qua iam diximus. Et ideo oportet
dicere quod numerus praedestinatorum sit certus Deo non solum
formaliter, sed etiam materialiter. 


&


解答する。以下のように言われるべきである。予定された者の数は確定的であ
る。しかし、ある人々は、それが形相的には確定的だが、質料的には確定して
いない、と言った。つまり、百人、あるいは千人が救われることは確定してい
るが、これらの人々がか、あるいは、あれらの人々が救われるということは確
定していないと言うかのように。しかしこれは、私たちがすでに論じた、予定
の確かさを棄却する。ゆえに、予定された者の数は、たんに形相的にだけでな
く、質料的にも、神にとって確定的だと言わなければならない。


\\

Sed advertendum est quod numerus
praedestinatorum certus Deo dicitur, non solum ratione cognitionis, quia
scilicet scit quot sunt salvandi (sic enim Deo certus est etiam numerus
guttarum pluviae, et arenae maris); sed ratione electionis et
definitionis cuiusdam. 


&

しかし、予定された者の数が神にとって確定的だと言われるのは、認識という
観点において、つまり、救われるべき者がどれだけいるかを知っている、とい
うだけでなく(その意味でなら、神にとって、雨粒の数や、海の砂粒の数も確
定的である)、ある種の選択と決定という観点においてもそうであることに注
意されるべきである。

\\


Ad cuius evidentiam, est sciendum quod omne agens
intendit facere aliquid finitum, ut ex supradictis de infinito
apparet. Quicumque autem intendit aliquam determinatam mensuram in suo
effectu, excogitat aliquem numerum in partibus essentialibus eius, quae
per se requiruntur ad perfectionem totius. Non enim per se eligit
aliquem numerum in his quae non principaliter requiruntur, sed solum
propter aliud, sed in tanto numero accipit huiusmodi, inquantum sunt
necessaria propter aliud. 


&

このことを明らかにするために、以下のことが知られるべきである。前に
\footnote{第7問第4項。}、無限について論じられたことから明らかなとおり、
すべて働くものは、ある有限なものを意図する。ところで、自らの結果におい
て、ある限定された限度を意図する者はだれでも、全体の完成に自体的に要求
される、その本質的諸部分において、ある数を考え出す。じっさい、主要に必
要とされず、たんに他のもののために必要とされるものにおいては、それ自体
で何らかの数を選ぶことはなく、むしろ、他のものに必要とされるだけの数で、
それらを受け取る。

\\


Sicut aedificator excogitat determinatam
mensuram domus, et etiam determinatum numerum mansionum quas vult facere
in domo, et determinatum numerum mensurarum parietis vel tecti, non
autem eligit determinatum numerum lapidum, sed accipit tot, quot
sufficiunt ad explendam tantam mensuram parietis. 


&

ちょうど、建築家が、家の限定された大きさ、さらに、その家の中に作ろうと
する部屋の限定された数、そして、壁や屋根の寸法の、限定された数を考え出
すが、しかし、限定された石の数を選ぶことはなく、むしろ、それだけの大き
さの壁を埋めるのに必要とされるだけの数を受け入れる。

\\


Sic igitur
considerandum est in Deo, respectu totius universitatis quae est eius
effectus. Praeordinavit enim in qua mensura deberet esse totum
universum, et quis numerus esset conveniens essentialibus partibus
universi, quae scilicet habent aliquo modo ordinem ad perpetuitatem;
quot scilicet sphaerae, quot stellae, quot elementa, quot species
rerum. 


&

それゆえ、神においても同様に、神の結果である宇宙全体にかんして考えられ
るべきである。すなわち、神は、宇宙全体がどのような大きさであるべきか、
宇宙の本質的諸部分、つまり、何らかのしかたで永続性にたいする秩序を持つ
部分に、どの数が適しているか、星は何個、元素はいくつ、諸事物の種はいく
つ、ということを、予め定めた。


\\


Individua vero corruptibilia non ordinantur ad bonum universi
quasi principaliter, sed quasi secundario, inquantum in eis salvatur
bonum speciei. Unde, licet Deus sciat numerum omnium individuorum, non
tamen numerus vel boum vel culicum, vel aliorum huiusmodi, est per se
praeordinatus a Deo, sed tot ex huiusmodi divina providentia produxit,
quot sufficiunt ad specierum conservationem. 

&

他方、可滅的な個体は、主要なものとしてではなく、二次的なものとして、す
なわち、それらの中で、種の善が保たれるように、宇宙の善へ秩序付けられて
いる。したがって、神はすべての個物の数を知っているが、牛やアブや、その
他そのようなものの数は、それ自体で、神によって予め定められたのではなく、
種の保存のために十分なだけの数を、そのような神の摂理に基づいて生み出し
た。


\\

Inter omnes autem
creaturas, principalius ordinantur ad bonum universi creaturae
rationales, quae, inquantum huiusmodi, incorruptibiles sunt; et
potissime illae quae beatitudinem consequuntur, quae immediatius
attingunt ultimum finem. Unde certus est Deo numerus praedestinatorum,
non solum per modum cognitionis, sed etiam per modum cuiusdam
principalis praefinitionis. -- Non sic autem omnino est de numero
reproborum; qui videntur esse praeordinati a Deo in bonum electorum,
quibus omnia cooperantur in bonum. 



&

ところで、すべての被造物の中で、より主要なかたちで宇宙の善へと秩序付け
られているのは理性的被造物であるが、それらは、そのようなものである限り
において、不可滅的である。そして、至福を獲得し、より直接的に究極目的に
到達するものは、特にそうである。したがって、予定された者の数は、神にとっ
て、認識というしかたによってだけでなく、一種の主要な先行決定というしか
たによっても、確定している。これに対して、見捨てられた者の数については、
まったくこの通りでない。これらの者たちは、選ばれた者たち、つまり、万物
が善へ向けてこの人たちと協働する人たちの善へ向けて、神によって予め定め
られたと思われる。



\\

De numero autem omnium
praedestinatorum hominum, quis sit, dicunt quidam quod tot ex hominibus
salvabuntur, quot Angeli ceciderunt. Quidam vero, quod tot salvabuntur,
quot Angeli remanserunt. Quidam vero, quod tot ex hominibus salvabuntur,
quot Angeli ceciderunt, et insuper tot, quot fuerunt Angeli creati. Sed
melius dicitur quod {\itshape soli Deo est cognitus numerus electorum in superna
felicitate locandus}.


&


ところで、すべての予定された人間の数について、それがいくつであるかを、
ある人々は、堕落した天使の数だけ、人間の中から救われるだろう、と言って
いる。またある人々は、[堕落せずに]留まった天使の数だけ、救われるだろ
う、と言う。またある人々は、堕落した天使の数と、それに、創造された天使
の数を加えただけ、人間の中から救われるだろう、と言う。しかし、「高い幸
福のなかに置かれる、選ばれた者たちの数は、ただ神にのみ知られている」と
語られるのが、より善い。


\\


{\scshape Ad primum ergo dicendum} quod verbum illud
{\itshape Deuteronomii} est intelligendum de illis qui sunt praenotati a Deo
respectu praesentis iustitiae. Horum enim numerus et augetur et
minuitur, et non numerus praedestinatorum.


&

第一異論に対しては、それゆえ、以下のように言われるべきである。『申命記』
のかの言葉は、現在の正義にかんして、神によって予知された人々について理
解されるべきである。この人々の数は、増えたり減ったりするが、予定された
者の数は、そうでない。



\\


{\scshape Ad secundum dicendum} quod ratio quantitatis
alicuius partis, accipienda est ex proportione illius partis ad
totum. Sic enim est apud Deum ratio quare tot stellas fecerit, vel tot
rerum species, et quare tot praedestinavit, ex proportione partium
principalium ad bonum universi.


&


第二異論に対しては、以下のように言われるべきである。ある部分の数の根拠
は、その部分の全体への関係に基づいて理解されるべきである。実際、このよ
うにして、なぜこれだけの数の星と諸事物の種を造ったか、そして、なぜこれ
だけの数を予定したか、ということの根拠が、主要な諸部分の、宇宙の善への
比例関係に基づいて、神のもとにある。

\\


{\scshape Ad tertium dicendum} quod bonum proportionatum
communi statui naturae, accidit ut in pluribus; et defectus ab hoc bono,
ut in paucioribus. Sed bonum quod excedit communem statum naturae,
invenitur ut in paucioribus; et defectus ab hoc bono, ut in
pluribus. Sicut patet quod plures homines sunt qui habent sufficientem
scientiam ad regimen vitae suae, pauciores autem qui hac scientia
carent, qui moriones vel stulti dicuntur, sed paucissimi sunt, respectu
aliorum, qui attingunt ad habendam profundam scientiam intelligibilium
rerum. Cum igitur beatitudo aeterna, in visione Dei consistens, excedat
communem statum naturae, et praecipue secundum quod est gratia destituta
per corruptionem originalis peccati, pauciores sunt qui salvantur. Et in
hoc etiam maxime misericordia Dei apparet, quod aliquos in illam salutem
erigit, a qua plurimi deficiunt secundum communem cursum et
inclinationem naturae.


&

第三異論に対しては、以下のように言われるべきである。自然本性の共通な段
階に比例した善は、多くのものに生じるが、この善を欠くことは、より少ない
ものに生る。他方、自然本性の共通な段階を越える善は、より少ないものに生
じ、そのような善を欠くことは、多くのものに生じる。たとえば、自分の生活
を導くのに十分な知をもつ人は多く、この知を欠くものは少ない。彼らは、痴
呆や愚者と言われる。しかし、可知的諸事物についての豊かな知を所有するこ
とに到達する人は、他に比べて、最も少ない。それゆえ、神を見ることにおい
て成立する永遠の至福は、自然本性の共通な段階を越えているので、また特に、
原罪の堕落によって、恩恵を喪失しているかぎりでは、救われる人はさらに少
ない。そして、多くの人々が、自然本性の共通の進路と傾向性のために、そこ
に到達できないかの救いへと、ある人々を上げることにおいてもまた、最大限
に神の憐れみが現れている。


\end{longtable}
\newpage





\rhead{a.~8}
\begin{center}
 {\Large {\bf ARTICULUS OCTAVUS}}\\
 {\large UTRUM PRAEDESTINATIO POSSIT IUVARI PRECIBUS SANCTORUM}\\
 {\footnotesize I {\itshape Sent.}, d.41, a.4; III, d.42, a.3, qu$^a$ 1,
 a.3; iV, d.45, q.3, a.3, ad 5; {\itshape De Verit.}, q.6, a.6.}\\
 {\Large 第八項\\予定は聖人たちの祈りによって助けられうるか}
\end{center}

\begin{longtable}{p{21em}p{21em}}

{\Huge A}{\scshape d octavum sic proceditur}. Videtur quod
praedestinatio non possit iuvari precibus\footnote
{prex, pr\u{e}cis (nom. and
I.gen. sing. not in use; dat. and acc. sing. only ante-class.; most freq. in plur.), f. precor, a prayer, request, entreaty (class.). 
} sanctorum. Nullum enim
aeternum praeceditur ab aliquo temporali, et per consequens non potest
temporale iuvare ad hoc quod aliquod aeternum sit. Sed praedestinatio
est aeterna. Cum igitur preces sanctorum sint temporales, non possunt
iuvare ad hoc quod aliquis praedestinetur. Non ergo praedestinatio
iuvatur precibus sanctorum.


&

第八項の問題へ、議論は以下のように進められる。予定は、聖人たちの祈りに
よって、助けられえないと思われる。理由は以下の通り。どんな永遠のものも、
時間的なものによって先行されず、その結果、時間的なものは、何かが永遠で
あることのために、助けることができない。ところが、予定は永遠である。ゆ
えに、聖人たちの祈りは時間的なので、ある人が予定されることの助けとはな
らない。ゆえに、予定が、聖人たちの祈りによって、助けられることはない。


\\


{\scshape 2 Praeterea}, sicut nihil indiget consilio
nisi propter defectum cognitionis, ita nihil indiget auxilio nisi
propter defectum virtutis. Sed neutrum horum competit Deo
praedestinanti, unde dicitur {\itshape Rom}.~{\scshape xi}, {\itshape quis adiuvit spiritum domini? Aut
quis consiliarius eius fuit?} Ergo praedestinatio non iuvatur precibus
sanctorum.


&

さらに、ちょうど、認識の欠陥のためにでなければ、思量を必要としないよう
 に、力の欠陥のためにでなければ、援助を必要としない。ところが、このどち
 らも、予定する神に適合しない。このことから『ローマの信徒への手紙』第11章
 で「誰が神の霊を助けるか。だれが、彼の相談相手だったか」
 \footnote{「いったいだれが主の心を知っていたであろうか。だれが主の相談
 相手であっただろうか」(11:34)}と言われている。ゆえに、予定は、聖人たち
 の祈りによって助けられない。


\\


{\scshape 3 Praeterea}, eiusdem est adiuvari et
impediri. Sed praedestinatio non potest aliquo impediri. Ergo non potest
aliquo iuvari.


&

さらに、助けられることと妨げられることは同じことに属する。ところが、予
定は、何によっても妨げられえない。ゆえに、何によっても助けられえない。

\\


{\scshape Sed contra est} quod dicitur {\itshape Genes}.~{\scshape xxv}, quod
{\itshape Isaac rogavit Deum pro Rebecca uxore sua, et dedit conceptum
Rebeccae}. Ex illo autem conceptu natus est Iacob, qui praedestinatus
fuit. Non autem fuisset impleta praedestinatio, si natus non
fuisset. Ergo praedestinatio iuvatur precibus sanctorum.


&


しかし反対に、『創世記』第25章に「イサクは、自分の妻であるレベッカに代
わって神に願った。そしてレベッカに懐胎を与えた」\footnote{「イサクは、
妻に子供ができなかったので、妻のために主に祈った。その祈りは主に聞き入
れられ、妻リベカは身ごもった」(25:21)}と言われている。ところが、この懐
胎から、ヤコブが生まれ、彼は予定されていた。しかしもし彼が生まれていな
かったならば、予定は満たされなかったであろう。ゆえに、予定は、聖人たち
の祈りによって助けられる。

\\


{\scshape Respondeo dicendum} quod circa hanc quaestionem
diversi errores fuerunt. Quidam enim, attendentes certitudinem divinae
praedestinationis, dixerunt superfluas esse orationes, vel quidquid
aliud fiat ad salutem aeternam consequendam, quia his factis vel non
factis, praedestinati consequuntur, reprobati non consequuntur. Sed
contra hoc sunt omnes admonitiones sacrae Scripturae, exhortantes ad
orationem, et ad alia bona opera. 


&

解答する。以下のように言われるべきである。この問題をめぐっては、さまざ
まな誤謬があった。ある人々は、神の予定が確定的であることに注目して次の
ように言った。祈ることや、その他何であれ、永遠の救済を獲得するためにな
されることは、不必要である。なぜなら、これらのことがなされようと、なさ
れまいと、予定された者たちはそれを獲得し、見捨てられた者たちは獲得しな
いのだから、と。しかし、聖書のあらゆる勧告は、これに反して、祈りやその
他の善い業を勧めている。

\\

Alii vero dixerunt quod per orationes
mutatur divina praedestinatio. Et haec dicitur fuisse opinio
Aegyptiorum, qui ponebant ordinationem divinam, quam fatum appellabant,
aliquibus sacrificiis et orationibus impediri posse. -- Sed contra hoc
etiam est auctoritas sacrae Scripturae. Dicitur enim I {\itshape Reg}.~{\scshape xv}, {\itshape porro
triumphator in Israel non parcet, neque poenitudine flectetur}. Et
{\itshape Rom}.~{\scshape xi} dicitur quod {\itshape sine poenitentia
 sunt dona Dei et vocatio}. 

&

また他の人々は、祈りによって、神の予定が変化を受けると言った。これは、
エジプトの人々の意見だと言われている。彼らは、彼らが「運命」と呼んだ神
の秩序付けが、何らかの犠牲や祈りによって、妨げられうると考えていた。し
かし、聖書の権威はこれにも反している。すなわち、『サムエル記上』第15章
で「それにまた、イスラエルの勝者は、容赦せず、悔恨によって曲げられない」
\footnote{「イスラエルの栄光である神は、偽ったり気が変わったりすること
のない方だ」(15:29)}と言われている。また、『ローマの信徒への手紙』第11
章では「神の贈り物と呼び出しは、悔恨なしにある」\footnote{「神の賜物と
招きとは取り消されないものなのです。」(11:29)}と言われている。


\\


Et ideo
aliter dicendum, quod in praedestinatione duo sunt consideranda,
scilicet ipsa praeordinatio divina, et effectus eius. Quantum igitur ad
primum, nullo modo praedestinatio iuvatur precibus sanctorum, non enim
precibus sanctorum fit, quod aliquis praedestinetur a Deo. Quantum vero
ad secundum, dicitur praedestinatio iuvari precibus sanctorum, et aliis
bonis operibus, quia providentia, cuius praedestinatio est pars, non
subtrahit causas secundas, sed sic providet effectus, ut etiam ordo
causarum secundarum subiaceat providentiae. 


&

それゆえ、これらとは別に、以下のように言われるべきである。予定において、
二つのことが考察されるべきである。すなわち、神の予定それ自体と、その結
果とである。前者にかんしては、それゆえ、予定が聖人たちの祈りによって助
けられることは決してない。なぜなら、聖人たちの祈りによって、だれかが神
によって予定されるということが生じるわけではないからである。他方、後者
にかんしては、予定が聖人たちの祈りによって、そして他の善い業によって助
けられると言われる。なぜなら、予定がその部分であるところの摂理は、第二
原因を取り除かず、第二原因の秩序が摂理のもとにあるというかたちで、結果
を摂理するからである。


\\



Sicut igitur sic providentur
naturales effectus, ut etiam causae naturales ad illos naturales
effectus ordinentur, sine quibus illi effectus non provenirent; ita
praedestinatur a Deo salus alicuius, ut etiam sub ordine
praedestinationis cadat quidquid hominem promovet in salutem, vel
orationes propriae, vel aliorum, vel alia bona, vel quidquid huiusmodi,
sine quibus aliquis salutem non consequitur. 



&

ゆえに、ちょうど、自然的な結果が、それらがなければその結果が生じなかっ
たであろうような、自然的な諸原因も、それらの自然的な諸結果に秩序付けら
れるというかたちで摂理されるように、ある人の救済も、予定の秩序のもとに、
その人自身の祈りや、他人の祈り、あるいは他の善、何であれそのような、そ
れなしにはその人が救済を獲得できない、人間を救済へと動かすあらゆるもの
が入っている。


\\


Unde praedestinatis
conandum est ad bene operandum et orandum, quia per huiusmodi
praedestinationis effectus certitudinaliter impletur. Propter quod
dicitur II {\itshape Petr}.~{\scshape i}, {\itshape satagite, ut per bona opera certam vestram
vocationem et electionem faciatis}.


&


したがって、予定された者たちは、よく働き、善く祈るように努力するべきで
ある。なぜなら、そのようなことを通して、予定の結果は、確実に満たされる
からである。このために、『ペトロの手紙二』第1章で、「精一杯がんばりな
さい。善い業を通して、あなたたちの呼び出しと選びを確かにするために」
\footnote{「だから兄弟たち、召されていること、選ばれていることを確かな
ものとするように、いっそう努めなさい」(1:10)}と言われている。


\\


{\scshape Ad primum ergo dicendum} quod ratio illa
ostendit quod praedestinatio non iuvatur precibus sanctorum, quantum ad
ipsam praeordinationem.


&

第一異論に対しては、それゆえ、以下のように言われるべきである。かの論は、
予定それ自体にかんして、予定が聖人たちの祈りによって助けられないことを
示している。


\\


{\scshape Ad secundum dicendum} quod aliquis dicitur
adiuvari per alium, dupliciter. Uno modo, inquantum ab eo accipit
virtutem, et sic adiuvari infirmi est, unde Deo non competit. Et sic
intelligitur illud, {\itshape quis adiuvit spiritum domini?} Alio modo dicitur quis
adiuvari per aliquem, per quem exequitur suam operationem, sicut dominus
per ministrum. Et hoc modo Deus adiuvatur per nos, inquantum exequimur
suam ordinationem, secundum illud I {\itshape ad Cor}.~{\scshape iii}, {\itshape Dei enim adiutores
sumus}. Neque hoc est propter defectum divinae virtutis, sed quia utitur
causis mediis, ut ordinis pulchritudo servetur in rebus, et ut etiam
creaturis dignitatem causalitatis communicet.


&

第二異論に対しては、以下のように言われるべきである。二通りの意味で、
「ある人が他の人によって助けられる」と言われる。一つには、その人から力
をもらうことによる。この意味で助けられるのは、弱い者であり、したがって
神には適合しない。かの「主の霊を誰が助けるか」はこの意味で言われている。
「ある人がだれかに助けられる」と言われるもう一つの意味は、たとえば、主
人が召使いによって助けられるように、その人によって、自分の働きが遂行さ
れる場合である。そしてこの意味で、かの『コリントの信徒への手紙一』第3
章「なぜなら私たちは神を助ける者たちだから」\footnote{「わたしたちは神
のために力を合わせて働くものであり、」(3:9)}によれば、私たちが神が秩序
付けたものを遂行するかぎりで、神は私たちによって助けられる。また、これ
は神の力の欠陥のためではない。むしろ、秩序の美しさが諸事物の中に保たれ
るように、また、被造物に、原因性という威厳を伝達するように、中間の諸原
因を用いるからである。



\\


{\scshape Ad tertium dicendum} quod secundae causae non
possunt egredi ordinem causae primae universalis, ut supra dictum est;
sed ipsum exequuntur. Et ideo praedestinatio per creaturas potest
adiuvari, sed non impediri.

&


第三異論に対しては、以下のように言われるべきである。前に述べられたとお
り\footnote{第19問第6項、第22問第2項第二異論解答。}、第二原因は、普遍
的な第一原因の秩序から逃れられず、むしろその秩序を遂行する。ゆえに、予
定は、被造物によって助けられうるが、妨げられることはできない。

\end{longtable}
\end{document}