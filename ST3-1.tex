\documentclass[10pt]{jsarticle} % use larger type; default would be 10pt
%\usepackage[utf8]{inputenc} % set input encoding (not needed with XeLaTeX)
%\usepackage[round,comma,authoryear]{natbib}
%\usepackage{nruby}
\usepackage{okumacro}
\usepackage{longtable}
%\usepqckage{tablefootnote}
\usepackage[polutonikogreek,english,japanese]{babel}
%\usepackage{amsmath}
\usepackage{latexsym}
\usepackage{color}

%----- header -------
\usepackage{fancyhdr}
\lhead{{\it Summa Theologiae} III, q.~1}
%--------------------

\bibliographystyle{jplain}

\title{{\bf TERTIA PARS}\\{\HUGE Summae Theologiae}\\Sancti Thomae
Aquinatis\\{\sffamily QUEAESTIO PRIMA}\\DE CONVENIENTIA INCARNATIONIS}
\author{Japanese translation\\by Yoshinori {\sc Ueeda}}
\date{Last modified \today}


%%%% コピペ用
%\rhead{a.~}
%\begin{center}
% {\Large {\bf }}\\
% {\large }\\
% {\footnotesize }\\
% {\Large \\}
%\end{center}
%
%\begin{longtable}{p{21em}p{21em}}
%
%&
%
%
%\\
%\end{longtable}
%\newpage



\begin{document}
\maketitle
\pagestyle{fancy}

\begin{center}
{\Large 第一問\\受肉の適切さについて}
\end{center}



\begin{longtable}{p{21em}p{21em}}

{\Huge C}{\scshape irca} primum tria consideranda occurrunt, primo quidem, de convenientia
incarnationis ipsius; secundo, de modo unionis verbi incarnati; tertio,
de his quae consequuntur ad hanc unionem. Circa primum quaeruntur
sex. 

\begin{enumerate}
 \item utrum conveniens fuerit Deum incarnari.
 \item utrum fuerit necessarium ad reparationem humani generis.
 \item utrum, si non fuisset peccatum, Deus incarnatus fuisset.
 \item utrum principalius sit incarnatus ad tollendum originale peccatum
       quam actuale.
 \item utrum conveniens fuerit Deum incarnari a principio mundi.
 \item utrum
eius incarnatio differri debuerit usque in finem mundi.
\end{enumerate}

&

第一について、三つの考察されるべきことが生じる。第一に、受肉自体の適切さ
 について、第二に、受肉した言葉の合一性のあり方について、第三に、この合
 一に伴うものどもについて、である。

第一の問題を巡って、六つのことが問われる。
\begin{enumerate}
 \item 神が受肉することは適切だったか。
 \item それは人類の回復のために必要だったか。
 \item もし人間が罪を犯さなかったとしても、神は受肉したか。
 \item 神は、現実の罪よりも、原罪を消すために、より根源的に、受肉したの
       か。
 \item 神が世の始まりから受肉することが、適切だったのではないか。
 \item 神の受肉は、世の最後まで延期されるべきではなかったのか。
\end{enumerate}
\end{longtable}


\newpage


\rhead{a.~1}
\begin{center}
 {\Large {\bf ARTICULUS PRIMUS}}\\
 {\large UTRUM FUERIT CONVENIENS DEUM INCARNRI}\\
 {\footnotesize III {\itshape Sent.}, d.1, q.1, a.2; IV {\itshape SCG},
 cap.40, 49, 53, 54, 55; {\itshape Compend.~Theol.}, cap.200, 201.}\\
 {\Large 第一項\\神が受肉するのは適切だったか}
\end{center}


\begin{longtable}{p{21em}p{21em}}
{\Huge A}{\scshape d primum sic proceditur}. Videtur quod non fuerit conveniens Deum
incarnari. Cum enim Deus ab aeterno sit ipsa essentia bonitatis, sic
optimum est ipsum esse sicut ab aeterno fuit. Sed Deus ab aeterno fuit
absque omni carne. Ergo convenientissimum est ipsum non esse carni
unitum. Non ergo fuit conveniens Deum incarnari.

&

第一項の問題へ、議論は以下のように進められる。
神が受肉することは適切でなかったと思われる。理由は以下の通り。
神は永遠から善性の本質そのものであるから、神は、永遠からそうであったよう
 にあることが最善である。
しかるに、神は永遠から、あらゆる肉なしにあった。
ゆえに、神が肉に合一していないことが、もっとも適切である。
ゆえに、神が受肉することは適切でなかった。

\\

2 {\scshape Praeterea}, quae sunt in infinitum distantia, inconvenienter iunguntur,
sicut inconveniens esset iunctura si quis pingeret imaginem in qua
{\itshape humano capiti cervix iungeretur equina}. Sed Deus et caro in infinitum
distant, cum Deus sit simplicissimus caro autem composita, et praecipue
humana. Ergo inconveniens fuit quod Deus carni uniretur humanae.

&

さらに、無限に隔たっているものどもは、不適切に結びあわされる。たとえば、
 だれかが、「人間の頭に馬の首が結びつけられている」絵を描いたならば、その結
 びあわされたものは不適切であっただろう。
しかるに、神と肉は無限に隔たっている。なぜなら、神はもっとも単純だが、肉
 は複合されていて、とくに、人間の肉はそうである。
ゆえに、神は人間の肉に統一されるのは不適切であった。

\\


3 {\scshape Praeterea}, sic distat corpus a summo spiritu sicut malitia a summa
bonitate. Sed omnino esset inconveniens quod Deus, qui est summa
bonitas, malitiam assumeret. Ergo non est conveniens quod summus
spiritus increatus corpus assumeret.

&

さらに、肉が最高の霊から隔たるのは、ちょうど悪が最高の善性から隔たるのと
 同じである。
しかるに、最高の善性である神が、悪を取ったりすれば、あらゆる点で不適切で
 あっただろう。
ゆえに、創造されない最高の霊が肉をとったとしたら、それは不適切である。


\\

4 {\scshape Praeterea}, inconveniens est ut qui excedit magna, contineatur in
minimo; et cui imminet
 cura magnorum, ad parva se transferat. Sed Deum,
qui totius mundi curam gerit, tota universitas capere non sufficit. Ergo
videtur inconveniens quod {\itshape intra corpusculum vagientis
 infantiae lateat
cui parum putatur universitas; et tandiu a sedibus suis absit ille
regnator, atque ad unum corpusculum totius mundi cura transferatur;} ut
Volusianus scribit ad Augustinum.

&

さらに、大きなものどもを越える者が、もっとも小さいものに含まれるのは不適
切であり、大きなものどもへの配慮がそれに身近であるようなものが、小さいも
のどもへ自ら向かうのも不適切である。
しかし、神は、全世界についての配慮をもつが、全宇宙が神をつかむのには充分
 でない。ゆえに、ウォルシアヌスがアウグスティヌスに書いているように、「それに比べれば宇宙が小さいと考えられる者が、金切り声を上げる幼
 児の中に隠れていること、そして、かくも長い間、かの統治者が自らの場所か
 ら離れ、一つの小さい肉体に全宇宙の配慮を向けること
 は、不適切である」。


\\


{\scshape Sed contra}, illud videtur esse convenientissimum ut per visibilia
monstrentur invisibilia Dei, ad hoc enim totus mundus est factus, ut
patet per illud Apostoli, {\itshape Rom}.~{\scshape i}: {\itshape invisibilia Dei per ea quae facta
sunt, intellecta, conspiciuntur}. Sed sicut Damascenus dicit, in
principio III libri, per incarnationis mysterium {\itshape monstratur simul
bonitas et sapientia et iustitia et potentia Dei vel virtus, bonitas
quidem, quoniam non despexit
 proprii plasmatis
 infirmitatem; iustitia
vero, quoniam non alium facit vincere tyrannum, neque vi
 eripit ex morte
hominem; sapientia vero, quoniam invenit difficillimi decentissimam
solutionem; potentia vero, sive virtus, infinita, quia nihil est maius
quam Deum fieri hominem}. Ergo conveniens fuit Deum incarnari.

&
しかし反対に、かの使徒の『ローマの信徒への手紙』第1章「神の見えないところは、作ら
 れたものをとおして知られ、見られる」\footnote{「世界が造られたときから、目に見えない神の性質、つまり神の永遠の力と神性は被造物に現れており、これを通して神を知ることができます。従って、彼らには弁解の余地がありません。 」(1:20)}によって明らかなとおり、全世界がそ
 れへ向けて作られたところの神の見えざるところが、見えるものを通して示さ
 れるのは、この上なく適切であると思われる。
しかるに、ちょうどダマスケヌスが第三巻の初めで言うように、「受肉の秘儀によっ
 て、神の善性と知恵と正義と、能力または力が同時に示される。善性は、自ら
 が生み出したものの弱さを見下さないからであり、正義は、他のものを独裁者
に勝たさず、死から人間を力ずくで奪わないからであり、他方、知恵は、もっと
 も困難で、もっともすばらしい解決法を見つけるからであり、また無限の能力または
 力は、神が人間になること以上に偉大なことはないからである」。ゆえに、神が
 受肉するのは適切であった。

\\

{\scshape Respondeo dicendum} quod unicuique rei conveniens est illud quod competit
sibi secundum rationem propriae naturae, sicut homini conveniens est
ratiocinari quia hoc convenit sibi inquantum est rationalis secundum
suam naturam. Ipsa autem natura Dei est bonitas, ut patet per Dionysium,
I cap. {\itshape de Div.~Nom}. Unde quidquid pertinet ad rationem boni, conveniens
est Deo. 


&

解答する。以下のように言われるべきである。
各々の事物に適切なものとは、固有の本性の性格に
 したがって、その事物に適合する事柄である。
たとえば、人間にとって推論することが適切であ
 るのは、このことが、自らの本性にしたがって、理性的である限りにおける人
 間に適合するからである。
ところで、ディオニュシウスの『神名論』第1章によって明らかなとおり、神の本
 性そのものは、善性である。したがって、善の性格に属するものは何であれ、
 神にとって適切である。

\\


Pertinet autem ad rationem boni ut se aliis communicet, ut
patet per Dionysium, {\scshape iv} cap.~{\itshape de Div.~Nom}. Unde ad rationem summi boni
pertinet quod summo modo se creaturae communicet. Quod quidem maxime fit
per hoc quod {\itshape naturam creatam sic sibi coniungit ut una persona fiat ex
tribus, verbo, anima et carne}, sicut dicit Augustinus, XIII {\itshape de
Trin}. Unde manifestum est quod conveniens fuit Deum incarnari.

&


さらに、ディオニュシウス『神名論』第4章によって明らか
 なとおり、善の性格には、自らを他に伝えることが属している。したがって、
 最高善の性格には、最高のかたちで自らを被造物へ伝えることが属する。アウ
 グスティヌスが『三位一体論』第13巻で言うように、このことは、「創造された
 本性を、言葉、魂、肉の三つから一つのペルソナが生じるように、自らと結び
 つける」ことによって、最大限に生じる。このことから、神が受肉することは適切で
 あったことが明らかである。


\\

{\scshape Ad primum ergo dicendum} quod incarnationis mysterium non est impletum
per hoc quod Deus sit aliquo modo a suo statu immutatus in quo ab
aeterno non fuit, sed per hoc quod novo modo creaturae se univit, vel
potius eam sibi. Est autem conveniens ut creatura, quae secundum
rationem sui mutabilis est, non semper eodem modo se habeat. Et ideo,
sicut creatura, cum prius non esset, in esse producta est, convenienter,
cum prius non esset unita Deo, postmodum fuit ei unita.

&

第一異論に対しては、それゆえ、次のように言われるべきである。
受肉の秘儀は、神が、なんらかのかたちで、永遠からそうあったのではないよう
 な、自らの状態から変化したことによってではなく、あらたなかたちで、自ら
 を被造物に合一した、あるいは、被造物を自らに合一させたことによって
 完成した。
しかるに、その性格において可変的である被造物は、常に同一の状態にあるわけ
 ではないということが適合的である。ゆえに、ちょうど、被造物が、より先に
 は存在しなかったのだから、存在へ生み出されたように、より先には神に合一
 していなかったのだから、その後、神に合一したことは適切である。
\\

{\scshape Ad secundum dicendum} quod uniri Deo in unitate personae non fuit
conveniens carni humanae secundum conditionem suae naturae, quia hoc
erat supra dignitatem ipsius. Conveniens tamen fuit Deo, secundum
infinitam excellentiam bonitatis eius, ut sibi eam uniret pro salute
humana.

&

第二異論に対しては、次のように言われるべきである。ペルソナの一性において神に
 合一されることは、肉の本性の状態において人間の肉に適合的ではなかった。
 なぜなら、このことは、肉の本性の値打ちを超えていたからである。
しかし、人類の救済のために、被造物を自らに合
 一させることは、神の善性の無限の卓越において、神に適切であった。


\\

{\scshape Ad tertium dicendum} quod quaelibet alia conditio secundum quam
quaecumque creatura differt a creatore, a Dei sapientia est instituta,
et ad Dei bonitatem ordinata, Deus enim propter suam bonitatem, cum sit
increatus, immobilis, incorporeus, produxit creaturas mobiles et
corporeas; et similiter malum poenae a Dei iustitia est introductum
propter gloriam Dei. Malum vero culpae committitur per recessum ab arte
divinae sapientiae et ab ordine divinae bonitatis. Et ideo conveniens
esse potuit assumere naturam creatam, mutabilem, corpoream et
poenalitati subiectam, non autem fuit conveniens ei assumere malum
culpae.

&

第三異論に対しては、次のように言われるべきである。
どんな被造物であれ、それが創造者と異なるどんな他の状態も、神の知恵によっ
 て作られ、神の善性へ秩序づけられている。なぜなら、神は、創造されず、不
 変で、非物体的だから、自らの善性のために、被造物を可変的で物質的に作っ
 たからである。同様に、罰の悪も、神の栄光のために、神の正義によって導入
 された。他方、罪の悪は、神の知恵の技と、神の善性の秩序からの後退によっ
 てなされている。ゆえに、創造され、可変で、物質的で、罰に服する本性をと
 ることは、適切でありえたが、罰の悪を取ることは、神に適切ではなかった。

\\

{\scshape Ad quartum dicendum} quod, sicut Augustinus respondet, in Epistola {\itshape ad
Volusianum}, {\itshape non habet hoc Christiana doctrina, quod ita sit Deus infusus
carni humanae ut curam gubernandae universitatis vel deseruerit vel
amiserit, vel ad illud corpusculum quasi contractam transtulerit,
hominum est iste sensus nihil nisi corpus valentium cogitare. Deus autem
non mole, sed virtute magnus est, unde magnitudo virtutis eius nullas in
angusto sentit angustias. Non est ergo incredibile, ut verbum hominis
transiens simul auditur a multis et a singulis totum, quod verbum Dei
permanens simul ubique sit totum}. Unde nullum inconveniens sequitur, Deo
incarnato.
&

第四に対しては、次のように言われるべきである。ウォルシアヌス宛書簡でアウ
 グスティヌスが答えているように、「神が人間の肉の中に注がれ、その結果、統
 治すべき世界への配慮がなくなったり捨てられたりするとか、あるいは、かの
 肉体へ結びつけられた配慮へと変容したりするということは、キリスト教の教
 義に含まれていない。このような理解は、物体しか考えることができない人間
 に属する。しかし神は、嵩によってではなく力によって大きい。したがって、
 神の力の大きさは、狭いところでもまったく狭さを感じない。ゆえに、越えて
 いく人間の言葉が同時に多くの人々によって聞かれ、一人一人によって全体が聞かれ
 るように、常に留まる神の言葉が、同時に、至る所に、全体としてあることも
 信じられないことはない」。ゆえに、神が受肉したとしても、どんな不都合も帰
 結しない。
\end{longtable}



\newpage

\rhead{a.~2}
\begin{center}
 {\Large {\bf ARTIULUS SECUNDUS}}\\
 {\large UTRUM FUERIT NECESSARIUM AD REPARATIONEM\\HUMANI GENERIS VERBUM
 DEI INCARNARI}\\
 {\footnotesize III {\itshape Sent.}, d.4, q.3, a.1, ad 3; IV d.10, a.1,
 ad 3; IV {\itshape SCG}, cap.64; {\itshape Cont.~Graec.~Armen.} etc.,
 cap.5; {\itshape Compend.~Theol.}, cap.200, 201; {\itshape In Psalm}.~45.}\\
 {\Large 第二項\\人類の回復のためには神の言葉が受肉することが必要だったか}
\end{center}

\begin{longtable}{p{21em}p{21em}}

{\Huge A}{\scshape d secundum sic proceditur}. Videtur quod non fuerit
 necessarium ad reparationem humani generis verbum Dei incarnari. Verbo
 enim Dei, cum sit Deus perfectus, ut in primo habitum est, nihil
 virtutis per carnem assumptam accrevit. Si ergo verbum Dei incarnatum
 naturam reparavit, etiam absque carnis assumptione eam potuit reparare.


&

第二項の問題へ、議論は以下のように進められる。
人類の回復のために、神の言葉が受肉することは必要なかったと思われる。理由
 は以下の通り。
第一部で論じられたとおり、神は完全なので、肉を取ることによって神の言葉に
 力が加わることはなかった。ゆえに、もし、受肉した神の言葉が本性を回復し
 たのであれば、受肉しなくても、それを回復することができた。



\\

{\scshape 2 Praeterea}, ad reparationem humanae naturae, quae per
 peccatum collapsa erat, nihil aliud requiri videbatur quam quod homo
 satisfaceret pro peccato. Non enim Deus ab homine requirere plus debet
 quam possit, et, cum pronior sit ad miserendum quam ad puniendum, sicut
 homini imputat actum peccati, ita etiam videtur quod ei imputet ad
 deletionem peccati actum contrarium. Non ergo fuit necessarium ad
 reparationem humanae naturae verbum Dei incarnari.

&

さらに、人間本性は、罪によって堕落していたのだが、それを回復するために要
 求されるものは、人間が罪を犯さなかったならば満たしていたであろうものに
 他ならないと思われる。なぜなら、神は人間に、人間ができる以上のことを要
 求すべきでないし、また、神は罰するよりも憐れむ方へ向かいやすいので、ちょ
 うど人間に、罪に属する行為を課するように、罪を取り除くために、反対の行
 為を課すと思われる。ゆえに、人間本性の回復のために神の言葉が受肉する必要はなかっ
 た。


\\

{\scshape 3 Praeterea}, ad salutem hominis praecipue pertinet ut Deum
 revereatur, unde dicitur Malach. {\scshape i}, {\itshape si ego dominus, ubi timor meus? Si
 pater, ubi honor meus?} Sed ex hoc ipso homines Deum magis reverentur
 quod eum considerant super omnia elevatum, et ab hominum sensibus
 remotum, unde in Psalmo dicitur, {\itshape excelsus super omnes gentes dominus,
 et super caelos gloria eius}; et postea subditur,  {\itshape quis sicut dominus
 Deus noster?} Quod ad reverentiam pertinet. Ergo videtur non convenire
 humanae saluti quod Deus nobis similis fieret per carnis assumptionem.

&

さらに、人間の救済のためには、とくに、神を敬うことが属する。このことから、
 『マラキ書』第1章で「もし、私が主人なら、私への畏れはどこにあるか。もし
 父ならば、どこに私の名誉があるか」\footnote{「子は父を、僕は主人を敬う
 ものだ。しかし、わたしが父であるなら/わたしに対する尊敬はどこにあるの
 か。わたしが主人であるなら/わたしに対する畏れはどこにあるのかと/万軍
 の主はあなたたちに言われる。わたしの名を軽んずる祭司たちよ/あなたたち
 は言う/我々はどのようにして御名を軽んじましたか、と。 」(1:6)}と言われ
 る。しかし、人々は、神を全てのものの上に高められ、人々の感覚から遠く離れたものと考えることによっ
 て、より神を敬う。このことから、『詩編』で、「全ての部族の上に高められ
 た主、その栄光は諸天の上に高められ」\footnote{「主はすべての国を超えて高くいまし/主の栄光は天を超えて輝く。」(113:4)}と言われ、そしてそのあとに、「私た
 ちの主である神のようなものは何かあるだろうか」\footnote{「わたしたちの
 神、主に並ぶものがあろうか。主は御座を高く置き 」(113:5)}と言われている
 が、これは、敬うことに属する。ゆえに、人間の救済に、神が、肉を取ること
 によって、私たちに似たものになることは、適切ではないと思われる。


\\

{\scshape Sed contra}, illud per quod humanum genus liberatur a
 perditione, est necessarium ad humanam salutem. Sed mysterium divinae
 incarnationis est huiusmodi, secundum illud Ioan.~{\scshape iii}, {\itshape sic Deus dilexit
 mundum ut filium suum unigenitum daret, ut omnis qui credit in ipsum
 non pereat, sed habeat vitam aeternam}. Ergo necesse fuit ad humanam
 salutem Deum incarnari.

&

しかし反対に、何かあるものによって、人類が破滅から解放されるならば、その
 あるものは、人間の救済に必要である\footnote{論理的には間違い。AがBする
 からといって、BのためにAが必要とは限らない。A以外のものもBするかもしれ
 ない。こ
 こでは、「何かあるものによって、そしてそれによってのみ」という含意があ
 ると解釈すべきか。}。しかし、『ヨハネによる福音書』第3章
 「神は、自分の一人だけの息子を与えるほど、世界を愛した。彼を信じる全ての人が、
 滅びず、英会陰の生命をもつために」\footnote{「神は、その独り子をお与え
 になったほどに、世を愛された。独り子を信じる者が一人も滅びないで、永遠
 の命を得るためである。」(3:16)}によれば、神の受肉の秘蹟は、そのようなも
 のである。

\\

{\scshape Respondeo dicendum quod} ad finem aliquem dicitur aliquid esse
 necessarium dupliciter, uno modo, sine quo aliquid esse non potest,
 sicut cibus est necessarius ad conservationem humanae vitae; alio modo,
 per quod melius et convenientius pervenitur ad finem, sicut equus
 necessarius est ad iter. 

&

解答する。以下のように言われるべきである。
ある目的のために、あるものが必要である、ということは、二通りの意味で言わ
 れる。一つには、それが無いと、あるものがあることができない場合であり、
 たとえば、食物は、人間の生命を保つために必要である。もう一つには、それ
 によって、よりよく、より適切に、目的に到達する場合であり、たとえば、馬
 が旅に必要である場合である。


\\



Primo modo Deum incarnari non fuit necessarium
 ad reparationem humanae naturae, Deus enim per suam omnipotentem
 virtutem poterat humanam naturam multis aliis modis reparare. Secundo
 autem modo necessarium fuit Deum incarnari ad humanae naturae
 reparationem. 

&

第一の意味では、神が受肉することは、人間本性の回復のために必要でなかった。
 なぜなら、神は、自らの全能の力によって、人間本性を、他の多くのしかたで
 回復することができたからである。しかし第二の意味では、神が人間本性の回
 復のために受肉することは必要であった。


\\


Unde dicit Augustinus, XIII {\itshape de Trin}., {\itshape ostendamus non
 alium modum possibilem Deo defuisse, cuius potestati omnia aequaliter
 subiacent, sed sanandae miseriae nostrae convenientiorem alium modum
 non fuisse}. Et hoc quidem considerari potest quantum ad promotionem
 hominis in bono. 


&

このため、アウグスティヌスは、『三位一体論』第13巻で次のように言っている。
 「その権能のもとに、万物が等しく服する神に、他の可能なしかたがなかった
 のではなく、私たちの悲惨を癒す、それ以上適切な手段がなかったということ
 を示そうではないか」。そしてこのことは、善における人間のレベルアップに
 関して考察されうる。


\\


Primo quidem, quantum ad fidem, quae magis
 certificatur ex hoc quod ipsi Deo loquenti credit. Unde Augustinus
 dicit, XI {\itshape de Civ.~Dei}, {\itshape ut homo fidentius ambularet ad veritatem, ipsa
 veritas, Dei filius, homine assumpto, constituit atque fundavit
 fidem}. 


&

第一に、信仰に関してであり、語る神自身を信じることによって、より確実なも
 のとされる。このことから、アウグスティヌスは『神の国』第11巻で、次のよ
 うに言う。「人間
 が、より信仰を持って、真理へと歩いていくように、真理自身が、神の息子が、
 人間を採って、信仰を建て、基礎付けた」。


\\

Secundo, quantum ad spem, quae per hoc maxime erigitur. 
Unde
 Augustinus dicit, XIII {\itshape de Trin}., {\itshape nihil tam necessarium fuit ad
 erigendam spem nostram quam ut demonstraretur nobis quantum diligeret
 nos Deus. Quid vero huius rei isto indicio manifestius, quam ut Dei
 filius naturae nostrae dignatus est inire consortium? }


&

第二に、希望に関してであり、それは、これによって、最大限に鼓舞される。
このことから、アウグスティヌスは、『三位一体論』第13巻で次のように述べる。
「神が私たちをどれほど愛しているかを、私たちに示すことほど、私たちの希望
 を鼓舞するために必要なことはなかった。神の息子が、私たちの本性の仲間に
 なる価値があるとしたという、この判断以上に、このことについて、何が、より明らか
 だろうか」。

\\


Tertio, quantum
 ad caritatem, quae maxime per hoc excitatur. Unde Augustinus dicit, in
 libro {\itshape de Catechizandis Rudibus}, {\itshape quae maior causa est adventus domini,
 nisi ut ostenderet Deus dilectionem suam in nobis?} Et postea subdit, {\itshape si
 amare pigebat, saltem reamare non pigeat}. 



&

第三に、愛に関してであり、このことによって、それは最大限に駆り立てられる。
 このことから、アウグスティヌスは、『教えの手ほどき』という書物で、「神
 が自分の愛を私たちの中に示すためでなければ、それ以上に大きな何の原因が、
 主にやって来ただろうか」と述べ、そのあとで、「愛するのが嫌だったとして
 も、愛し返すことは嫌がらないように」と述べる。


\\


Quarto, quantum ad rectam
 operationem, in qua nobis exemplum se praebuit. Unde Augustinus dicit,
 in quodam sermone {\itshape de nativitate domini}, {\itshape homo sequendus non erat, qui
 videri poterat, Deus sequendus erat, qui videri non poterat. Ut ergo
 exhiberetur homini et qui ab homine videretur, et quem homo sequeretur,
 Deus factus est homo}. 


&

第四に、正しい働きにかんしてであり、それにおいて、私たちに手本を示した。
 このことからアウグスティヌスは、『主の誕生について』というある説教で、
 以下のように述べている。「目に見える人間には従うべきでなく、目に見えな
 い神に従うべきだった。それゆえ、人間に見える者が人間に示されるように、
 また、人間がその者に従うように、神は人となった」。


\\


Quinto, quantum ad plenam participationem
 divinitatis, quae vere est hominis beatitudo, et finis humanae
 vitae. Et hoc collatum est nobis per Christi humanitatem, dicit enim
 Augustinus, in quodam sermone {\itshape de Nativ. domini}, {\itshape factus est Deus homo,
 ut homo fieret Deus}. 


&

第五に、神性への十分な参加に関してであり、これは真に、人間の至福であり、
 人間の生の目的である。そしてこれは、キリストの人間性を通して私たちにも
 たらされた。なぜなら、アウグスティヌスは、『主の誕生について』というあ
 る説教で、「人間が神になるために、神は人間になった」と言うからである。


\\


Similiter etiam hoc utile fuit ad remotionem
 mali. Primo enim per hoc homo instruitur ne sibi Diabolum praeferat, et
 eum veneretur, qui est auctor peccati. Unde dicit Augustinus, XIII {\itshape de
 Trin}., {\itshape quando sic Deo coniungi potuit humana natura ut fieret una
 persona, superbi illi maligni spiritus non ideo se audeant homini
 praeponere quia non habent carnem}. 


&

さらに同様に、これは、悪を取り続ために有益であった。第一に、このことによっ
 て、人間は、罪の作者である悪霊を、自分より選好したり敬ったりないように、
 教えられる。このことからアウグスティヌスは、『三位一体論』第13巻で、以
 下のように言う。「人間本性が、一つのペルソナとなるように神に結びつけら
 れることができたとき、かの高慢で邪悪な霊たちは、肉を持っていないために、
 あえて人間より自らを選好することがない」。


\\


Secundo, quia per hoc instruimur
 quanta sit dignitas humanae naturae, ne eam inquinemus peccando. Unde
 dicit Augustinus, in libro {\itshape de Vera Religione}, {\itshape demonstravit nobis Deus
 quam excelsum locum inter creaturas habeat humana natura, in hoc quod
 hominibus in vero homine apparuit}. 


&

第二に、このことによって、私たちは、罪を犯すことによってそれを腐敗させな
 いように、人間本性の偉大さがどれだけ大きいかを教えられる。このことから、
 アウグスティヌスは『真の宗教について』で次のように述べる。「神が
 真の人間において人間たちに現れたことによって、神は、人間本性が、被造物
 の中でどれほど高い場所を持っているかを示した」。


\\


Et Leo Papa dicit, in sermone {\itshape de
 nativitate}, {\itshape agnosce, o Christiane, dignitatem tuam, et divinae consors
 factus naturae, noli in veterem vilitatem degeneri conversatione
 redire}. 


&

そして、教皇レオも『誕生について』という説教で言っている。「キリスト教徒よ、あな
 たの偉大さを知りなさい。神の本性の仲間となったのだから、下らな
 いおしゃべりをして、もといた低級なところに逆戻りしてはいけません」。

\\



Tertio quia, ad praesumptionem hominis tollendam, {\itshape gratia Dei,
 nullis meritis praecedentibus, in homine Christo nobis commendatur}, ut
 dicitur XIII {\itshape de Trinitate}. 


&

第三に、人間の高慢をくじくためにである。『三位一体論』第13巻で述べられる
 ように、「神の恩恵は、どんな先行する功績もなく、人間キリストにおいて、
 私たちに委ねられている」。


\\


Quarto, quia {\itshape superbia hominis, quae maximum
 impedimentum est ne inhaereatur Deo per tantam Dei humilitatem redargui
 potest atque sanari}, ut Augustinus dicit ibidem. 


&

第四に、アウグスティヌスが同じ箇所で言うように、「人間の傲慢は、神を継ぐ
 最大限の障害だが、神のこれほどの慎みによって、論駁され、癒されうる」。

\\



Quinto, ad liberandum
 hominem a servitute. Quod quidem, ut Augustinus dicit, XIII {\itshape de Trin}.,
 {\itshape fieri debuit sic ut Diabolus iustitia hominis Iesu Christi superaretur},
 quod factum est Christo satisfaciente pro nobis. Homo autem purus
 satisfacere non poterat pro toto humano genere; Deus autem satisfacere
 non debebat; unde oportebat Deum et hominem esse Iesum Christum. 


&

第五に、人間を隷属から解放するためである。アウグスティヌスが『三位一体論』
 第13巻で言うように、このことは「悪霊が、人間であるイエス・キリストの正
 義によって征服されるために、なされるべきであった。そしてこれは、私たち
のために満足を与えるキリストによってなされた。しかし、純粋な人間は、人類
 全体のために満足を与えることができなかったが、他方で、神は満足を与える
 べきでなかったので、イエス・キリストは、神でありかつ人間である必要があっ
 た。


\\



Unde
 et Leo Papa dicit, in sermone {\itshape de Nativ}., {\itshape suscipitur a virtute
 infirmitas, a maiestate humilitas, ut, quod nostris remediis
 congruebat, unus atque idem Dei et hominum mediator et mori ex uno, et
 resurgere posset ex altero. Nisi enim esset verus Deus, non afferret
 remedium, nisi esset homo verus, non praeberet exemplum}. 



&

このことから、教皇レオも、説教『誕生について』で、以下のように言う。「弱
 さが力によって、低さが偉大さによって採られるのは、私たちの救済に適する
 ように、一つの同一の、神と人々の仲介者が、一方から死に、他方から復活す
 ることができるようにである。というのも、もし真の神でなかったならば、救
 済をもたらすことはできなかったし、もし真の人間でなかったならば、手本を
 示すことがなかっただろうから」。

\\


Sunt autem et
 aliae plurimae utilitates quae consecutae sunt, supra comprehensionem
 sensus humani.


&

しかし、これらに続く他の多くの有益さがあるが、それらは人間の認識能力の把
 握を超えている。

\\



{\scshape Ad primum ergo dicendum} quod ratio illa procedit secundum primum modum necessarii, sine quo ad finem perveniri non potest.


&


第一異論に対しては、それゆえ、以下のように言われるべきである。
この論は、それがないと目的へ到達できないという、第一の意味での「必要」に
 即して進められている。


\\

{\scshape Ad secundum dicendum} quod aliqua satisfactio potest dici
 sufficiens dupliciter. Uno modo, perfecte, quia est condigna per
 quandam adaequationem ad recompensationem commissae culpae. Et sic
 hominis puri satisfactio sufficiens esse non potuit, quia tota natura
 humana erat per peccatum corrupta; nec bonum alicuius personae, vel
 etiam plurium, poterat per aequiparantiam totius naturae detrimentum
 recompensare. 


&

第二異論に対しては、以下のように言われるべきである。
ある賠償が十分であるということは、二つの意味で言われうる。一つには、完全
 にであり、犯された罪の補償が、ある種の等しさによって、同等である場合で
 ある。この意味では、純粋な人間の賠償が、十分であることは不可能である。
 なぜなら、人間の全本性は、罪によって腐敗したからである。また、ある人物
 の善や、あるいは複数の人の善であっても、本性全体の腐敗を、同等さという
 かたちで償うことは不可能であった。



\\

Tum etiam quia peccatum contra Deum commissum quandam
 infinitatem habet ex infinitate divinae maiestatis, tanto enim offensa
 est gravior, quanto maior est ille in quem delinquitur. Unde oportuit,
 ad condignam satisfactionem, ut actio satisfacientis haberet efficaciam
 infinitam, ut puta Dei et hominis existens. 



&

さらに、神に対してなされた罪は、神の偉大さの無限性から、ある種の無限性を
 もっているので、攻撃されたものが重大であるほど、それを失っているものも
 大きい。したがって、失われた偉大さと等しい者を賠償するためには、賠償す
 る者の作用が無限の効力を持っていなければならなかった。たとえば、神と人
 間である者のように。


\\

Alio modo potest dici
 satisfactio sufficiens imperfecte, scilicet secundum acceptationem eius
 qui est ea contentus, quamvis non sit condigna. Et hoc modo satisfactio
 puri hominis est sufficiens. Et quia omne imperfectum praesupponit
 aliquid perfectum, a quo sustentetur, inde est quod omnis puri hominis
 satisfactio efficaciam habet a satisfactione Christi.

&

もう一つの意味では、不完全に十分である賠償が、そう言われる。すなわち、同
 じ価値でなくても、それを請求する人が容認するかぎりにおいてである。この
 意味では、純粋な人間の賠償は、十分である。そして、すべて不完全なものは完
 全なものを前提とし、それに支えられるから、全ての、純粋な人間の賠償は、
 キリストの賠償から効力を持つ。


\\

{\scshape Ad tertium dicendum} quod Deus, assumendo carnem, suam
 maiestatem non minuit, et per consequens non minuitur ratio reverentiae
 ad ipsum. Quae augetur per augmentum cognitionis ipsius. Ex hoc autem
 quod nobis appropinquare voluit per carnis assumptionem, magis nos ad
 se cognoscendum attraxit.


&


第三異論に対しては、以下のように言われるべきである。
神は、肉を取ることによって、自らの偉大さを減少させず、結果的に、自分自身
 への畏敬の根拠を減少させない。この根拠は、神の認識が増えるほど、増加す
 る。しかし、肉を取ることによって私たちに近づこうと意志したことによって、
 私たちを、神自身を認識することへ、より引きつけた。




\end{longtable}

\newpage



\rhead{a.~3}
\begin{center}
 {\Large {\bf ARTICULUS TERTIUS}}\\
 {\large UTRUM, SI HOMO NON PECCASSET,\\ NIHILOMINUS DEUS INCARNATUS FUISSET}\\
 {\footnotesize III {\itshape Sent.}, d.1, q.1, a.3; I {\itshape ad
 Tim.}, cap.1, lect.4.}\\
 {\Large 第三項\\もし人間が罪を犯さなかったとしても、それでも神は受肉し
 たか}
\end{center}

\begin{longtable}{p{21em}p{21em}}

{\Huge A}{\scshape d tertium sic proceditur}. Videtur quod,
si homo non peccasset, nihilominus Deus incarnatus fuisset. Manente enim
causa, manet effectus. Sed sicut Augustinus dicit, XIII {\itshape de Trin}., {\itshape alia
multa sunt cogitanda in Christi incarnatione} praeter absolutionem a
peccato, de quibus dictum est. Ergo, etiam si homo non peccasset, Deus
incarnatus fuisset.


&

第三項の問題へ、議論は以下のように進められる。
もし人間が罪を犯さなかったとしても、それでも神は受肉したと思われる。
理由は以下の通り。
原因に変化がなければ、結果にも変化がない。しかし、アウグスティヌスが『三
 位一体論』第13巻で言うように、すでに語られた\footnote{前項。}罪からの解
 放以外にも、「他の多くのことが、キリストの受肉において考えられるべきで
 ある」。ゆえに、人間が罪を犯さなかったとしても、神は受肉したであろう。


\\


2 {\scshape Praeterea}, ad omnipotentiam divinae
virtutis pertinet ut opera sua perficiat, et se manifestet per aliquem
infinitum effectum. Sed nulla pura creatura potest dici infinitus
effectus, cum sit finita per suam essentiam. In solo autem opere
incarnationis videtur praecipue manifestari infinitus effectus divinae
potentiae, per hoc quod in infinitum distantia coniunguntur, inquantum
factum est quod homo esset Deus. In quo etiam opere maxime videtur
perfici universum, per hoc quod ultima creatura, scilicet homo, primo
principio coniungitur, scilicet Deo. Ergo, etiam si homo non peccasset,
Deus incarnatus fuisset.


&

さらに、神の力の全能性には、自らの業を完成させることと、ある無限の結果に
 よって自分自身を明示することが属する。
しかし、どんな純粋な被造物も、自分自身の本質によって限られるので、無限な結果とは言われえない。
しかし、受肉の業においてだけは、神の能力の無限な結果が、特に明示されると
 思われる。人間が神であるとされるかぎりで、無限に隔たった者が結びつけら
 れることによって。また、その業において、宇宙が最大限に完成されると思わ
 れる。すなわち、究極の被造物である人間が、第一の根源である神に結びつけ
 られることによって。ゆえに、人間が罪を犯さなかったとしても、神は受肉し
 たであろう。

\\


3 {\scshape Praeterea}, humana natura per peccatum non
est facta capacior gratiae. Sed post peccatum capax est gratiae unionis,
quae est maxima gratia. Ergo, si homo non peccasset, humana natura huius
gratiae capax fuisset. Nec Deus subtraxisset naturae humanae bonum cuius
capax erat. Ergo, si homo non peccasset, Deus incarnatus fuisset.


&

さらに、人間本性は、罪によって、恩恵を受けられないものとされた。しかし、
 罪のあとに、最大の恩恵である、合一の恩恵を受けうる。ゆえに、もし人間が
 罪を犯さなかったとしても、その本性は、この恩恵を受けうるものだっただろ
 う。神もまた、人間が受けうる善を、人間本性から取り除きはしなかっただろ
 う。ゆえに、もし人間が罪を犯さなかったとしても、神は受肉したであろう。



\\


4 {\scshape Praeterea}, praedestinatio Dei est
aeterna. Sed dicitur, {\itshape Rom}.~{\scshape i}, de Christo, quod {\itshape praedestinatus est filius
Dei in virtute}. Ergo etiam ante peccatum necessarium erat filium Dei
incarnari, ad hoc quod Dei praedestinatio impleretur.


&


さらに、神の述語付けは永遠である。しかし、『ローマの信徒への手紙』第1章
 で、キリストについて「その力において、神の息子であることが予定されてい
 た」\footnote{「聖なる霊によれば、死者の中からの復活によって力ある神の
 子と定められたのです。この方が、わたしたちの主イエス・キリストです。」
 (1:4)}と言われている。ゆえに、罪の前でも、神の予定が満たされるために、
神の息子が受肉することは必然であった。

\\


5 {\scshape Praeterea}, incarnationis mysterium est
primo homini revelatum, ut patet per hoc quod dixit: {\itshape Hoc nunc os ex
ossibus meis}, etc., quod apostolus dicit esse {\itshape magnum sacramentum in
Christo et Ecclesia}, ut patet {\itshape Ephes}.~{\scshape v}. Sed homo non potuit esse
praescius sui casus, eadem ratione qua nec Angelus, ut Augustinus
probat, {\itshape super Gen.~ad Litt}. Ergo, etiam si homo non peccasset, Deus
incarnatus fuisset.


&

さらに、「今や、この骨が、私の骨から云々」\footnote{「人は言った。「ついに、これこそ/わたしの骨の骨/わたしの肉の
 肉。これをこそ、女(イシャー)と呼ぼう/まさに、男(イシュ)から取られ
 たものだから。」『創世記』(2:23)}
と言ったことを、使徒が、
『エフェソの信徒への手紙』第5章で「キリストと教会における大きな秘蹟」\footnote{「それゆえ、人は父と母を離れてその妻と結ばれ、二人は一体となる。この神秘は偉大です。わたしは、キリストと教会について述べているのです。」(5:31-2)}
と言っていることから明らかなとおり、受肉の秘蹟は、第一の人間(=アダム)
に啓示されたものである。
しかしその人間は、自分の堕落を予め知ることができず、また、アウグスティヌス
 が『創世記逐語注解』で証明しているとおり、同じ理由で天使もそれができな
 い。ゆえに、もし人間が罪を犯さなかったとしても、神は受肉したであろう。



\\


{\scshape Sed contra est} quofd Augustinus dicit, in
libro {\itshape de Verbis Dom}., exponens illud quod habetur Luc.~{\scshape xix}, {\itshape Venit filius
hominis quaerere et salvum facere quod perierat, si homo non peccasset,
filius hominis non venisset}. Et I {\itshape ad Tim}.~{\scshape i}, super illud verbum,
{\itshape Christus venit in hunc mundum ut peccatores salvos faceret}, dicit
Glossa, {\itshape nulla causa veniendi fuit Christo domino, nisi peccatores salvos
facere. Tolle morbos, tolle vulnera, et nulla medicinae est causa}.


&


しかし反対に、アウグスティヌスは『主の言葉について』という書物で、『ルカ
 による福音書』第19章で述べられていること\footnote{「人の子は、失われた
 ものを捜して救うために来たのである。」(19:10)}を説明して、こう述べてい
 る。「人間の息子は、なくなったものを探し、救うためにやって来る。もし人
 間が罪を犯さなかったならば、人間の息子は来なかったであろう」。そして、
 『テモテへの手紙一』第1章「キリストは、罪人たちを救われたものにするため
 にこの世界へやって来た」\footnote{「「キリスト・イエスは、罪人を救うた
 めに世に来られた」という言葉は真実であり、そのまま受け入れるに値します。
 わたしは、その罪人の中で最たる者です。」(1:15)}というかの言葉について、
 注解は次のように述べている。「主キリストにとって、罪人を救われたものに
 すること以外に、やって来る理由はなかった。病をなくし、傷をなくせば、薬
 を必要とする理由はない」。


\\


{\scshape Respondeo dicendum} quod aliqui circa hoc
diversimode opinantur. Quidam enim dicunt quod, etiam si homo non
peccasset, Dei filius fuisset incarnatus. Alii vero contrarium
asserunt. Quorum assertioni magis assentiendum videtur. 


&

解答する。人々はこれについてさまざまな意見をもっている。ある人々は、人間
 が罪を犯さなかったとしても、神の息子は受肉しただろうと言う。しかし別の
 人々は、反対のことを主張する。そして、こちらの主張の方に、より同意すべきだと思われ
 る。

\\

Ea enim quae ex
sola Dei voluntate proveniunt, supra omne debitum creaturae, nobis
innotescere non possunt nisi quatenus in sacra Scriptura traduntur, per
quam divina voluntas innotescit. 


&

理由は以下の通り。
神の意志だけに基づいて生じる事柄は、被造物のあらゆる債務を超えていて、聖
 書の中に伝えられる以外のしかたで、私たちに知らされることはできない。聖
 書によって、神の意志が知らされるからである。


\\

Unde, cum in sacra Scriptura ubique
incarnationis ratio ex peccato primi hominis assignetur, convenientius
dicitur incarnationis opus ordinatum esse a Deo in remedium peccati, ita
quod, peccato non existente, incarnatio non fuisset. Quamvis potentia
Dei ad hoc non limitetur, potuisset enim, etiam peccato non existente,
Deus incarnari.


&

したがって、聖書のあらゆる箇所で、受肉の根拠が、第一の人間の罪から指定さ
 れているのだから、受肉の業は、罪の償いへ、神によって秩序付けられていて、
 したがって、罪が存在しなかったならば、受肉はなかったであろう、と言われ
 ることがより適切である。ただし、神の能力は、このことに制限されていない
 のであって、罪が存在しなかったとしても、神が受肉することは可能ではあっ
 た。


\\


{\scshape Ad primum ergo dicendum} quod omnes aliae
causae quae sunt assignatae, pertinent ad remedium peccati. Si enim homo
non peccasset, perfusus fuisset lumine divinae sapientiae, et iustitiae
rectitudine perfectus a Deo, ad omnia necessaria cognoscenda. Sed quia
homo, deserto Deo, ad corporalia collapsus erat, conveniens fuit ut
Deus, carne assumpta, etiam per corporalia ei salutis remedium
exhiberet. Unde dicit Augustinus, super illud Ioan.~{\scshape i} cap., {\itshape Verbum caro
factum est, caro te obcaecaverat, caro te sanat, quoniam sic venit
Christus ut de carne vitia carnis exstingueret}.


&

第一異論に対しては、それゆえ、以下のように言われるべきである。
指定された全ての他の原因は、罪の救済方法に属する。もし、人間が罪を犯さなかっ
 たならば、全ての必要なことを認識するように、神の知恵の光に満たされ、神
 によって正義の正しさにおいて完成されていたであろう。しかし、人間は、神
 に見捨てられ、物体的なものへと堕落したので、神は、肉を取って、物体的な
 ものを通してもまた、人間に救いの方法を示すことが適切であった。このこと
 から、アウグスティヌスは、『ヨハネによる福音書』第1章のかの箇所について、
 以下のように述べている。「言葉は肉になった:肉はあなたを盲目にしたが、肉
 があなたを癒す。なぜなら、キリストは、肉から肉の悪徳を消し去るようにやっ
 て来たからである」。


\\


{\scshape Ad secundum dicendum} quod in ipso modo
productionis rerum ex nihilo divina virtus infinita ostenditur. Ad
perfectionem etiam universi sufficit quod naturali modo creatura
ordinetur sic in Deum sicut in finem. Hoc autem excedit limites
perfectionis naturae, ut creatura uniatur Deo in persona.


&

第二異論に対しては、以下のように言われるべきである。
諸事物を無から生み出すというこのしかたの中に、神の無限の力が示されている。
さらに、世界の完成のためには、被造物が、自然本性的なしかたで、目的として
 の神へ秩序付けられていることで十分である。しかし、被造物が神にペルソナ
 において合一されるとういことは、自然の完全性の限界を超えている。


\\


{\scshape Ad tertium dicendum} quod duplex capacitas
attendi potest in humana natura. Una quidem secundum ordinem potentiae
naturalis. Quae a Deo semper impletur, qui dat unicuique rei secundum
suam capacitatem naturalem. Alia vero secundum ordinem divinae
potentiae, cui omnis creatura obedit ad nutum. 

&

第三異論に対しては、以下のように言われるべきである。
人間本性の中には、二通りの受容性が見出されうる。
一つは、自然本性的能力の秩序に即してであり、これは、神によって、常に満た
 されている。神は、各々の事物に、それの自然本性的な受容性に即して与える
 からである。もう一つは、神の能力の秩序に即してであり、全ての被造物は、
 即座にそれに従う。

\\

Et ad hoc pertinet ista
capacitas. Non autem Deus omnem talem capacitatem naturae replet,
alioquin, Deus non posset facere in creatura nisi quod facit; quod
falsum est, ut in primo habitum est. 


&

そして、問題になっている受容性は、こちらに属する。
しかし、神が、このようなすべての本性の受容性を満たすことはない。さもなけ
 れば、神は、現に行っている以外のことを被造物において行うことができない
 ことになるが、第一部で論じられたように、これは偽である。


\\


Nihil autem prohibet ad aliquid
maius humanam naturam productam esse post peccatum, Deus enim permittit
mala fieri ut inde aliquid melius eliciat. Unde dicitur {\itshape Rom}.~{\scshape v}, {\itshape ubi
abundavit iniquitas, superabundavit et gratia}. Unde et in benedictione
Cerei Paschalis dicitur, {\itshape o felix culpa, quae talem ac tantum meruit
habere redemptorem!}


&

しかし、罪のあとに、人間本性が、より大きな何かへと生み出されることは、な
 んら差し支えない。じっさい、神は、何かよりよいものが出てくるように、悪
 が生じることを許す。このことから、『ローマの信徒への手紙』第5章で「不公
 正が満ちているところに、恩恵がさらに満ちていた」\footnote{「律法が入り
 込んで来たのは、罪が増し加わるためでありました。しかし、罪が増したとこ
 ろには、恵みはなおいっそう満ちあふれました。」(5:20) }と言われる。この
 ことから、イースターのロウソクの祈祷でも「幸いな罪よ。それは、かくも素晴らしく大きな償う者を
 もつという褒美を得るのだから」。


\\


{\scshape Ad quartum dicendum} quod praedestinatio
praesupponit praescientiam futurorum. Et ideo, sicut Deus praedestinat
salutem alicuius hominis per orationem aliorum implendam, ita etiam
praedestinavit opus incarnationis in remedium humani peccati.


&

第四異論に対しては、以下のように言われるべきである。
予定は、未来の予知を前提とする。
ゆえに、神が、他の人々の祈りが成就するように、ある人の救済を予定するのと
 同じかたちで、受肉の業も、人間の罪の救済策へと予定した。


\\


{\scshape Ad quintum dicendum} quod nihil prohibet
alicui revelari effectus cui non revelatur causa. Potuit ergo primo
homini revelari incarnationis mysterium sine hoc quod esset praescius
sui casus, non enim quicumque cognoscit effectum, cognoscit et causam.


&

第五異論に対しては、以下のように言われるべきである。
ある人に、原因が啓示されないが、結果が啓示されるということがあってもよい。
 ゆえに、第一の人間に、自分の堕落を予め知ることなしに、受肉の秘蹟が啓示
 されることは可能であった。じっさい、結果を認識する人が皆、原因を認識す
 るわけではない。


\end{longtable}
\newpage


\rhead{a.~4}
\begin{center}
 {\Large {\bf ARTICULUS QUARTUS}}\\
 {\large UTRUM DEUS PRINCIPALIUS INCARNATUS FUERIT IN REMEDIUM ACTUALIUM
 PECCATORUM QUAM IN REMEDIUM ORIGINALIS PECCATI}\\
 {\footnotesize III {\itshape Sent.}, d.1, q.1, a.2, ad 6; {\itshape De
 Articulis XLII, a.28; {\itshape De Articulis XXXVI}, a.23.}}\\
 {\Large 第四項\\神は、主に、原罪よりも現実的な罪の救済のために、受肉し
 たか。}
\end{center}

\begin{longtable}{p{21em}p{21em}}



{\Huge A}{\scshape d quartum sic proceditur}. Videtur quod
Deus principalius incarnatus fuerit in remedium actualium peccatorum
quam in remedium originalis peccati. Quanto enim peccatum est gravius,
tanto magis humanae saluti adversatur, propter quam Deus est
incarnatus. Sed peccatum actuale est gravius quam originale peccatum,
minima enim poena debetur originali peccato, ut Augustinus dicit, {\itshape Contra
Iulianum}. Ergo principalius incarnatio Christi ordinatur ad deletionem
actualium peccatorum.


&

第四項の問題へ、議論は以下のように進められる。
神は、主に、原罪の救済よりも現実的な罪の救済のために受肉したと思われる。
 理由は以下の通り。
罪が重いほど、人間の救済は妨げられる。その救済のために、神は受肉した。
しかし、現実的な罪は、原罪よりも重い。なぜなら、アウグスティヌスが『ユリ
 アヌス駁論』で述べるように、原罪に対する債務は最少だからである。
ゆえに、キリストの受肉は、現実的な罪を取り除くことへ、主に、秩序付けられ
 ている。


\\


2 {\scshape Praeterea}, peccato originali non debetur
poena sensus, sed solum poena damni, ut in secundo habitum est. Sed
Christus venit pro satisfactione peccatorum poenam sensus pati in cruce,
non autem poenam damni, quia nullum defectum habuit divinae visionis aut
fruitionis. Ergo principalius venit ad deletionem peccati actualis quam
originalis.


&


さらに、第二部で論じられたとおり\footnote{STI-IIae, q.87, a.5, arg.2;
 II{\itshape Sent.}, d.33, q.2, a.1.; ``et ideo poenae Deus est auctor;
 diversimode vero diversarum. Quaedam enim est poena damni, ut
 subtractio gratiae, et hujusmodi; et harum poenarum Deus causa est, non
 quidem agendo aliquid, sed potius non agendo: ex eo enim quod Deus
 gratiam non influit, consequitur in isto gratiae privatio. Quaedam vero
 poena sensus est, quae per aliquam actionem infligitur; et hujus etiam
 agendo Deus auctor est.'' (II Sent., d.37, q.3, a.1, c.)
}、原罪に対する債務は感覚の罪ではなく、喪失の罪である。
しかし、キリストは、罪人たちが十字架上で受けた感覚の罪を償うために来たの
 であって、喪失の罪のためではない。なぜなら、キリストは、神を見ること、
 享受することに対して、何の欠陥もなかったからである。ゆえに、主要には、
 原罪ではなく現実の罪を取り除くためにやって来た。

\\


3 {\scshape Praeterea}, sicut Chrysostomus dicit, in II
{\itshape de Compunctione Cordis}, {\itshape hic est affectus servi fidelis, ut beneficia
domini sui quae communiter omnibus data sunt, quasi sibi soli praestita
reputet, quasi enim de se solo loquens Paulus ita scribit, {\itshape ad Galat}.~{\scshape ii},
dilexit me, et tradidit semetipsum pro me}. Sed propria peccata nostra
sunt actualia, originale enim est {\itshape commune peccatum}. Ergo hunc affectum
debemus habere, ut aestimemus eum principaliter propter actualia peccata
venisse.


&

さらに、クリュソストムスは、『心の痛み』の第2巻で次のように言う。「忠実
 な奴隷の愛は以下のようなものである。すなわち、自分の主人の優しさが、そ
 れがみんなに与えられているのに、自分だけに示されていると思う。あたかも、
 パウロが、『ガラテヤの信徒への手紙』第2章で、自分一人について語り以下の
 ように述べているようにである。「私を愛し、自分自身を私のために与えた」
 \footnote{「生きているのは、もはやわたしではありません。キリストがわた
 しの内に生きておられるのです。わたしが今、肉において生きているのは、わ
 たしを愛し、わたしのために身を献げられた神の子に対する信仰によるもので
 す。」(2:20) }。しかし、私たちに固有の罪は現実的なものであり、原罪は、
 「共通の罪」である。ゆえに、彼が主要に、現実的な罪のために来たというこ
 の愛を、私たちは持たなければならない。


\\


{\scshape Sed contra est} quod Ioan.~{\scshape i} dicitur, {\itshape ecce
agnus Dei, ecce qui tollit peccata mundi}.


&

しかし反対に、『ヨハネによる福音書』第1章で「見よ、神の子羊だ。世界の罪を消
 す者だ」\footnote{「その翌日、ヨハネは、自分の方へイエスが来られるのを
 見て言った。「見よ、世の罪を取り除く神の小羊だ。 」」(1:29)}と言われて
 いる。

\\


{\scshape Respondeo dicendum} quod certum est Christum
venisse in hunc mundum non solum ad delendum illud peccatum quod
traductum est originaliter in posteros, sed etiam ad deletionem omnium
peccatorum quae postmodum superaddita sunt, non quod omnia deleantur
(quod est propter defectum hominum, qui Christo non inhaerent, secundum
illud Ioan.~{\scshape iii}, {\itshape venit lux in mundum, et dilexerunt homines magis
tenebras quam lucem}), sed quia ipse exhibuit quod sufficiens fuit ad
omnem deletionem. 
Unde dicitur {\itshape Rom}.~{\scshape v}, {\itshape non sicut delictum, sic et donum,
nam iudicium ex uno in condemnationem, gratia autem ex multis delictis
in iustificationem}. 


&

解答する。以下のように言われるべきである。
たしかに、キリストは、起源から末裔へと伝えられているかの罪を取り除くため
 だけでなく、そのあとに付け加えられた全ての罪を取り除くために、この世に
 やって来た。ただし、こう言われるのは、すべての罪が取り除かれたからではなく、(これは、『ヨ
 ハネによる福音書』第3章「光が世界に来、人間たちは光よりも闇を愛した」
 \footnote{「光が世に来たのに、人々はその行いが悪いので、光よりも闇の方
 を好んだ。それが、もう裁きになっている。 」(3:19)})によれば、キリストに
 はない人間の欠陥のためである)むしろ、全ての除去のために十分なものを示し
 たからである。このことから、『ローマの信徒への手紙』第5章で、次のように
 言われる。「罪
としてではなく、贈り物として。すなわち、判決は、一つの罪のために
 有罪とされるが、恩恵は、多くの罪からでも正しいとされる」\footnote{「しかし、恵みの賜物は罪とは比較になりません。一人の罪によって多くの人が死ぬことになったとすれば、なおさら、神の恵みと一人の人イエス・キリストの恵みの賜物とは、多くの人に豊かに注がれるのです。 この賜物は、罪を犯した一人によってもたらされたようなものではありません。裁きの場合は、一つの罪でも有罪の判決が下されますが、恵みが働くときには、いかに多くの罪があっても、無罪の判決が下されるからです。」(5:15-16)}。





\\




Tanto autem principalius ad alicuius peccati
deletionem Christus venit, quanto illud peccatum maius est. 
Dicitur
autem maius aliquid dupliciter. Uno modo, intensive, sicut est maior
albedo quae est intensior. Et per hunc modum maius est peccatum actuale
quam originale, quia plus habet de ratione voluntarii, ut in secundo
dictum est. 


&

しかし、キリストがある罪の除去のためにどの程度
 主要にやってきたかは、その罪の大きさに比例する。
そして、何かが大きいということは、二通りの意味で言われる。
一つには、内包的にであり、白さがより白いが、より大きい白である、という場
 合である。
そして、この意味では、現実的な罪が、原罪よりも大きい。なぜなら、第二部で
 言われたように、意志的なものの性格をより多くもつからである。



\\


Alio modo dicitur aliquid maius extensive, sicut dicitur
maior albedo quae est in maiori superficie. Et hoc modo peccatum
originale, per quod totum genus humanum inficitur, est maius quolibet
peccato actuali, quod est proprium singularis personae. 


&

もう一つの意味では、外延的に、あるものが大きいと言われる。たとえば、より広
い表面にある白が、より大きい白と言われるように。そして、この意味では、原
 罪は、人類全体が被るのだから、個々の人物に固有である現実的な罪よりも大
 きい。


\\


Et quantum ad
hoc, Christus principalius venit ad tollendum originale peccatum,
inquantum {\itshape bonum gentis divinius est quam bonum unius}, ut dicitur in I
{\itshape Ethic}.

&

そしてこの点にかんして、キリストは、原罪をなくすためにやって来た。それは、『ニコ
 マコス倫理学』第1巻で述べられるとおり、「部族の善は、一人の人の善よりも
 神的である」という観点においてである。


\\



{\scshape Ad primum ergo dicendum} quod ratio illa
procedit de intensiva magnitudine peccati.


&

第一異論に対しては、それゆえ、以下のように言われるべきである。
かの論は、罪の内包的な大きさについて論じられている。


\\


{\scshape Ad secundum dicendum} quod peccato originali
in futura retributione non debetur poena sensus, poenalitates tamen quas
sensibiliter in hac vita patimur, sicut famem, sitim, mortem et alia
huiusmodi, ex peccato originali procedunt. Et ideo Christus, ut plene
pro peccato originali satisfaceret, voluit sensibilem dolorem pati, ut
mortem et alia huiusmodi in seipso consummaret.


&

第二異論に対しては、以下のように言われるべきである。
将来の応報において、原罪にとって、感覚の罪が債務とはならない(感覚の罪が
 罰として下されることはない)が、飢え、渇き、死などの、私たちがこの世で感
 覚的に受ける罪性は、原罪からでてくる。それゆえ、キリストは、十分に原罪
 を償うため、死やその他そのようなものを自らにおいて味わうために、感覚
 的な悲しみを受けることを欲した。



\\


{\scshape Ad tertium dicendum} quod, sicut Chrysostomus
ibidem inducit, verba illa dicebat apostolus, {\itshape non quasi diminuere volens
amplissima et per orbem terrarum diffusa Christi munera, sed ut pro
omnibus se solum indicaret obnoxium. 
Quid enim interest si et aliis
praestitit, cum quae tibi sunt praestita ita integra sunt et ita
perfecta quasi nulli alii ex his aliquid fuerit praestitum? }


&

第三異論に対しては、以下のように言われるべきである。
クリュソストムスが同じ箇所で言うように、使徒が言ったかの言葉は、「キリス
 トの贈り物がこの上なく十分に、地上に行き渡って広がっているのを狭めよう
 として言われたのではなく、全ての人を前に、自分自身だけに、罪が示されて
 いるという意味で言われた。なぜなら、他の誰にも何も与えられなかったかの
 ように、あなたに与えられたものが、素晴らしく完全であるとしても、他の人々
 にもそれが与えられていたら、それが何だろうか」。

\\

Ex hoc ergo
quod aliquis debet sibi reputare beneficia Christi praestita esse, non
debet existimare quod non sint praestita aliis. 
Et ideo non excluditur
quin principalius venerit abolere peccatum totius naturae quam peccatum
unius personae.


&

ゆえに、ある人が、自分にキリストの恩恵が与えられたと考えなければならない
 ということから、それが他の人に与えられないと考えるべきだということには
 ならない。ゆえに、
より主要に、 一人の人物の罪よりも、全本性の罪を排するために来たことは排
 除されない。


\\

 Sed illud peccatum commune ita perfecte curatum est in
unoquoque ac si in eo solo esset curatum. Et praeterea, propter unionem
caritatis, totum quod omnibus est impensum, unusquisque debet sibi
adscribere.


&

しかし、かの共通の罪は、あたかも彼だけの中で癒されたかのように、各々の人
 において完全に癒された。さらに、愛の合一のため、万人に与えられたもの
 全体を、一人一人が、自分に帰属させるべきである。



\end{longtable}
\newpage


\rhead{a.~5}
\begin{center}
 {\Large {\bf ARTICULUS QUINTUS}}\\
 {\large UTRUM CONVENIENS FUISSET DEUM INCARNARI A PRINCIPIO HUMANI GENERIS}\\
 {\footnotesize III {\itshape Sent.}, d.1, a.1, a.4; IV {\itshape SCG},
 cap.53, 55; {\itshape In Isaiam}, cap.2; {\itshape Ad Galat.}, cap.4, lect.2.}\\
 {\Large 第五項\\神が人類の最初から受肉したとしたらそれは適切だったか}
\end{center}

\begin{longtable}{p{21em}p{21em}}


{\Huge A}{\scshape d quintum sic proceditur}. Videtur quod
conveniens fuisset Deum incarnari a principio humani
generis. Incarnationis enim opus ex immensitate divinae caritatis
processit, secundum illud {\itshape Ephes}.~{\scshape ii}, {\itshape Deus, qui dives est in
misericordia, propter nimiam caritatem suam qua dilexit nos, cum essemus
mortui peccatis, convivificavit nos in Christo}. Sed caritas non tardat
subvenire amico necessitatem patienti, secundum illud {\itshape Prov}.~{\scshape iii}, {\itshape ne
dicas amico tuo, vade et revertere, cras dabo tibi; cum statim possis
dare}. Ergo Deus incarnationis opus differre non debuit, sed statim a
principio per suam incarnationem humano generi subvenire.


&

第五項の問題へ、議論は以下のように進められる。
神が人類の最初から受肉したとしたら、それは適切だったと思われる。理由は以
 下の通り。『エフェソの信徒への手紙』第2章「神は、私たちを愛するその計り
 がたい愛のために、憐れみにおいて豊かであり、私たちが罪のために死んでい
 たので、キリストにおいて私たちを共に活かした」\footnote{「しかし、憐れ
 み豊かな神は、わたしたちをこの上なく愛してくださり、その愛によって、罪
 のために死んでいたわたしたちをキリストと共に生かし、――あなたがたの救
 われたのは恵みによるのです―― 」(2:4-5)}によれば、受肉の業は、神の無尽
 蔵な愛から出てきた。しかし、『箴言』第3章「直ちに与えることができ
 るのに、あなたの友に、「急いで行け、そしてもどってこい。明日あなたに与
 えるから」と言うな。」\footnote{「出直してくれ、明日あげよう、と友に言
 うな/あなたが今持っているなら。」(3:28)}によれば、愛は、必要に迫られて
 いる友に遅れてやって来ることがない。ゆえに神は、受肉の業を遅らせるべき
 でなく、初めからすぐに、自らの受肉を通して人類にやって来るべきであった。


\\


2 {\scshape Praeterea}, I {\itshape Tim.}~{\scshape i} dicitur, {\itshape Christus
venit in hunc mundum peccatores salvos facere}. Sed plures salvati
fuissent si a principio humani generis Deus incarnatus fuisset, plurimi
enim, ignorantes Deum, in suo peccato perierunt in diversis
saeculis. Ergo convenientius fuisset quod a principio humani generis
Deus incarnatus fuisset.


&

さらに、『テモテへの手紙一』第1章で、「キリストはこの世に罪人を救うためにやっ
 て来た」と言われている。しかし、もし初めから人類に受肉した神がいたなら
 ば、より多くの人々が救われていたであろう。というのも、神を知らない多く
 の人々が、さまざまな時代に、自らの罪において死んだからである。ゆえに、
 初めから、人類に、受肉したか磨いたら、その方がより適切であっただろう。

\\


3 {\scshape Praeterea}, opus gratiae non est minus
ordinatum quam opus naturae. Sed {\itshape natura initium sumit a perfectis}, ut
dicit Boetius, in libro {\itshape de Consolatione}. Ergo opus gratiae debuit a
principio esse perfectum. Sed in opere incarnationis consideratur
perfectio gratiae, secundum illud, {\itshape verbum caro factum est}, et postea
subditur, {\itshape plenum gratiae et veritatis}. Ergo Christus a principio humani
generis debuit incarnari.


&

さらに、恩恵の業は、自然の業に劣らず秩序付けられている。
しかし、ボエティウスが『哲学の慰め』で述べるように「自然は、完全なものか
 ら端緒を採る」。ゆえに、恩恵の業は、最初から完全なものでなければならな
 かった。しかし、受肉の業において、「言葉が肉となった」とあり、そのあとに、「恩恵と真理の充満」とある\footnote{「言は
 肉となって、わたしたちの間に宿られた。わたしたちはその栄光を見た。それ
 は父の独り子としての栄光であって、恵みと真理とに満ちていた。 」(『ヨハ
 ネによる福音書』1:14)}ことによれば、恩恵の
 完成が考えられている。ゆえに、キリストは、最初から人類のために受肉され
 るべきであった。


\\


{\scshape Sed contra est} quod dicitur {\itshape Galat}.~{\scshape iv}, {\itshape at
ubi venit plenitudo temporis, misit Deus filium suum, factum ex muliere},
ubi dicit Glossa quod {\itshape plenitudo temporis est quod praefinitum fuit a Deo
patre quando mitteret filium suum}. Sed Deus sua sapientia omnia
definivit. Ergo convenientissimo tempore Deus est incarnatus. Et sic non
fuit conveniens quod a principio humani generis Deus incarnaretur.


&

しかし反対に、『ガラテヤの信徒への手紙』第4章で、「そして時が満ちると、神は自分の息子を派遣し、女性から生まれたも
 のとした」\footnote{「しかし、時が満ちると、神は、その御子を女から、し
 かも律法の下に生まれた者としてお遣わしになりました。」 }と述べられて
 いて、注解はこれについて、「時が満ちるとは、自分の息子をいつ派遣するか
 が、父である神によって、決定されていたということである」と述べている。
しかし、神は、その知恵によって、全てのことを決定している。ゆえに、この上
 なく適切なときに、神は受肉した。この意味で、人類の初めから神が受肉して
 いたら、適切ではなかった。


\\


{\scshape Respondeo dicendum} quod, cum opus
incarnationis principaliter ordinetur ad reparationem naturae humanae
per peccati abolitionem, manifestum est quod non fuit conveniens a
principio humani generis, ante peccatum, Deum incarnatum fuisse, non
enim datur medicina nisi iam infirmis. Unde ipse dominus dicit,
Matth.~{\scshape ix}, {\itshape non est opus valentibus medicus, sed male habentibus, nonenim veni vocare iustos, sed peccatores}. 


&

解答する。以下のように言われるべきである。
受肉の業は、主要に、罪を取り除くことによって、人間本性を回復することに秩
 序付けられているから、人類の最初から、罪の前に、神が受肉することが適切
 ではなかったことは明らかである。すでに病気の人にでなければ、薬は与えら
 れないからである。それゆえ、主自身も『マタイによる福音書』第9章で「医者
 は、健康な人には必要なく、病気の人に必要である。私は正しい人々ではなく、
 罪人たちを呼ぶために来た」\footnote{「イエスはこれを聞いて言われた。
 「医者を必要とするのは、丈夫な人ではなく病人である。『わたしが求めるの
 は憐れみであって、いけにえではない』とはどういう意味か、行って学びなさ
 い。わたしが来たのは、正しい人を招くためではなく、罪人を招くためである。」
 (9:12-13)}と言っている。

\\

Sed non etiam statim post
peccatum conveniens fuit Deum incarnari. 
Primo quidem, propter
conditionem humani peccati, quod ex superbia provenerat, unde eo modo
erat homo liberandus ut, humiliatus, recognosceret se liberatore
indigere. Unde super illud Galat.~{\scshape iii}, {\itshape ordinata per Angelos in manu
mediatoris}, dicit Glossa, {\itshape magno consilio factum est ut, post hominis
casum, non illico Dei filius mitteretur. Reliquit enim Deus prius
hominem in libertate arbitrii, in lege naturali, ut sic vires naturae
suae cognosceret. Ubi cum deficeret, legem accepit. Qua data, invaluit
morbus, non legis, sed naturae vitio, ut ita, cognita sua infirmitate,
clamaret ad medicum, et gratiae quaereret auxilium}. 


&

しかし、罪の直後に神が受肉することもまた、適切ではなかった。
第一に、人間の罪の条件のためにであり、つまりそれは傲慢から出てきたので、
 自分を低くして、解放者を必要とすることを改めて認識するというかたちで、
 人間は自由にされるべきっであった。このことから、かの『ガラテヤの信徒へ
 の手紙』第3章「天使たちによって秩序付けられたものどもが、仲介者の手の中
 に」\footnote{「では、律法とはいったい何か。律法は、約束を与えられたあ
 の子孫が来られるときまで、違犯を明らかにするために付け加えられたもので、
 天使たちを通し、仲介者の手を経て制定されたものです。」(3:19)}について、
 『注解』は次のように述べている。「人間の堕落のあと、すぐに神の子が派遣
 されなかったのは、偉大な思慮による。つまり神は、先に、人間を自由裁量の
 中に、自然の法の中に残し、そうして自分の本性の力を認識するようにした。
 そこで失敗をしたので、法を受け入れた。法が与えられたので、法ではなく本
 性の悪徳によって弱くなった者が強くなり、自分の低さを認識し、薬を求め、
 恩恵の助けを求めるようになった」。


\\


Secundo, propter
ordinem promotionis in bonum, secundum quem ab imperfecto ad perfectum
proceditur. Unde apostolus dicit, I {\itshape ad Cor}.~{\scshape xv}, {\itshape non prius quod
spirituale est, sed quod animale, deinde quod spirituale. Primus homo de
terra, terrenus, secundus homo de caelo, caelestis}. 



&

第二に、不完全なものから完全なものへ進むという、善への進行の秩序のために
 である。このことから、使徒は『コリントの信徒への手紙一』第15章で次のよ
 うに述べる。「霊的であることが先なのではなくて、動物であることが先であ
 る。そして次に霊的となる。最初の人間は土からなので土的であり、第二の人
 間は、天からであり、天的である」\footnote{「最初に霊の体があったのではありません。自然の命の体があり、次いで霊の体があるのです。最初の人は土ででき、地に属する者であり、第二の人は天に属する者です。」(15:46-47)}。


\\


Tertio, propter
dignitatem ipsius verbi incarnati. Quia super illud Galat.~{\scshape iv}, {\itshape ubi venit
plenitudo temporis}, dicit Glossa, {\itshape quanto maior iudex veniebat, tanto
praeconum series longior praecedere debebat}. 



&

第三に、受肉した言葉の偉大さのためである。なぜなら、かの『ガラテヤの信徒
 への手紙』第4章「そのとき時は満ちた」について、『注解』は「偉大な裁き人
 が来るほど、その先触れは長く続かなければならない」と述べるからである。


\\


Quarto, ne fervor fidei
temporis prolixitate tepesceret. Quia circa finem mundi {\itshape refrigescet
caritas multorum}, et Luc.~{\scshape xviii} dicitur, {\itshape cum filius hominis veniet,
putasne inveniet fidem super terram?}


&

第四に、信仰の熱が、時の長さのために冷めないためである。なぜなら、世の終
 わりについて、「多くの人々の愛が冷める」\footnote{「不法がはびこるので、
 多くの人の愛が冷える」『マタイによる福音書』(24:12)}、また、『ルカによ
 る福音書』第18章「人の息子が来る
 とき、地上に信仰を見出すと思うか」\footnote{「言っておくが、神は速やかに裁いてくださる。しかし、人の子が来るとき、果たして地上に信仰を見いだすだろうか。」(18:8)}とあるからである。



\\


{\scshape Ad primum ergo dicendum} quod caritas non
differt amico subvenire, salva tamen negotiorum opportunitate et
personarum conditione. Si enim medicus statim a principio aegritudinis
medicinam daret infirmo, minus proficeret, vel magis laederet quam
iuvaret. Et ideo etiam dominus non statim incarnationis remedium humano
generi exhibuit, ne illud contemneret ex superbia, si prius suam
infirmitatem non cognosceret.


&

第一異論に対しては、以下のように言われるべきである。
愛は、友人に、遅れてやって来ることがない。しかし、事柄の適切さと人々の状
 況を常におろそかにしない。たとえば、もし医者が病気の最初から病人に薬を
 与えたら、益にならず、助けるより傷つけるであろうように。ゆえに、主もま
 た、直ちに受肉という救済策を人類に示さなかったのは、先に自分の低さを認
 識していなかったならば、傲慢によってそれを見下していただろうが、そうさ
 せないためにである。


\\


{\scshape Ad secundum dicendum} quod Augustinus ad hoc
respondet, in libro 
{\itshape de Sex Quaestionibus Paganorum}, dicens, qu.~{\scshape ii}, quod
{\itshape tunc voluit Christus hominibus apparere, et apud eos praedicari suam
doctrinam, quando et ubi sciebat esse qui in eum fuerant credituri. His
enim temporibus, et his in locis, tales homines in eius praedicatione
futuros esse sciebat quales, non quidem omnes, sed tamen multi in eius
corporali praesentia fuerunt, qui nec in eum, suscitatis mortuis,
credere voluerunt}. 


&

第二異論に対しては、以下のように言われるべきである。
アウグスティヌスは、これに答えて、『異教徒たちの六つの質問について』とい
 う書物の第二問で、以下のように述べている。「キリストが、人間たちに現れ、彼
 らのもとで自分の教説を述べることを意志したのは、彼を信じる人々がいるこ
 とを知っていたときと場所にであった。つまり、そのときその場所では、全て
 ではないが多くの人々が、彼の肉体的な現前にあって、死者が蘇っても彼を信
 じようとしないよな人々が、彼の説教において、将来いるであろうことを知っ
 ていた」。




\\

Sed hanc responsionem reprobans idem Augustinus
dicit, in libro {\itshape de Perseverantia: nunquid possumus dicere Tyrios aut
Sidonios, talibus apud se virtutibus factis, credere noluisse, aut
credituros non fuisse si fierent, cum ipse dominus eis attestetur quod
acturi essent magnae humilitatis poenitentiam, si in eis facta essent
divinarum illa signa virtutum?} 



&

しかし、この応答に反論して、同じアウグスティヌスが、『忍耐について』とい
 う書物で次のように述べている。「ティルスとシドンの人々が、そのような力
 によってなされた事柄が彼らのもとにあったときに、信じようとしないとか信
 じるものとならなかったと、私たちは言うことができるだろうか。
 主自身が、彼らの中に、神の力のしるしがなされたならば、大きな謙遜による
 悔悛を行うだろうことを、彼らに証言しているのに」。


\\

{\itshape Proinde}, ut ipse solvens subdit, {\itshape sicut
apostolus ait, non est volentis neque currentis, sed miserentis Dei, qui
his quos praevidit, si apud eos facta essent, suis miraculis credituros,
quibus voluit subvenit, aliis autem non subvenit, de quibus in sua
praedestinatione, occulte quidem sed iuste, aliud iudicavit. Ita
misericordiam eius in his qui liberantur, et veritatem in his qui
puniuntur sine dubitatione credamus}.


&

そしてアウグスティヌスは、問題を解いて次のように述べる。「したがって、ちょ
 うど使徒が「意志する人や走る人ではなく、憐れむ神の」\footnote{「従って、
 これは、人の意志や努力ではなく、神の憐れみによるものです。」(9:16)}と言
 うように、神は、神が助ける人々については、彼らのもとでなされたならば、自分たちの奇跡
 を信じたであろうと予見するが、神が助けない他の人々には、自
 らの予定において、隠れてしかし正しく、他のことを命じた。このように、神
 の憐れみを、自由にされる人々において、その真理を、罰せられる人々におい
 て、私たちは疑いなく信じる」。\footnote{難読。全体の文意はやや不明。異
 論解答で、このように長い引用があるのはまれ。おそらく何らかの特殊事情が
 あることが推測される。}


\\


{\scshape Ad tertium dicendum} quod perfectum est prius
imperfecto, in diversis quidem, tempore et natura, oportet enim quod
perfectum sit quod alia ad perfectionem adducit, sed in uno et eodem
imperfectum est prius tempore, etsi sit posterius natura. Sic ergo
imperfectionem naturae humanae duratione praecedit aeterna Dei
perfectio, sed sequitur ipsam consummata perfectio in unione ad Deum.


&

第三異論に対しては、以下のように言われるべきである。
たしかに、さまざまなものどもにおいては、時間と本性において、完全なものが
 不完全なものよりも先である。他のものどもを、完成へ導くのは、完全なもの
 だからである。しかし、一つの同一のものにおいては、不完全なものが、本性
 において後であるにして、時間において先である。この意味で、人間本性の不
 完全性は、神の永遠の完全性に、持続において先行する。しかし、神への合一
 において、その最終の完全性が続く。


\end{longtable}
\newpage




\rhead{a.~6}
\begin{center}
 {\Large {\bf ARTICULUS SEXTUS}}\\
 {\large UTRUM INCARNATIONIS OPUS DIFFERRI DEBUERIT USQUE IN FINEM MUNDI}\\
 {\footnotesize III {\itshape Sent.}, d.1, q.1, a.4.}\\
 {\Large 第六項\\受肉の業は世界の終わりまで遅らされるべきであったか}
\end{center}

\begin{longtable}{p{21em}p{21em}}

{\Huge A}{\scshape d sextum sic proceditur}. Videtur quod
incarnationis opus differri debuerit usque in finem mundi. Dicitur enim
in Psalmo, {\itshape senectus mea in misericordia uberi}, idest, {\itshape in novissimo}, ut
Glossa dicit. Sed tempus incarnationis est maxime tempus misericordiae,
secundum illud Psalmi, quoniam venit tempus miserendi eius. Ergo
incarnatio debuit differri usque in finem mundi.


&

第六項の問題へ、議論は以下のように進められる。
受肉の業は、世界の終わりまで延期されるべきだったと思われる。理由は以下の
 通り。
『詩編』に「私の老年は豊かな憐れみに」とあり、注解はそれを、「最
 後の日々に」と述べている。しかし、かの『詩編』の「彼の憐れみの時が来た
 ので」\footnote{「どうか、立ち上がって/シオンを憐れんでください。恵みのとき、定められたときが来ました。」(102:14)}によれば、受肉の時は、最大限に、憐れみの時である。ゆえに、受肉は、
 世界の終わりまで延期されるべきであった。


\\


{\scshape 2 Praeterea}, sicut dictum est, perfectum, in
eodem, tempore est posterius imperfecto. Ergo id quod est maxime
perfectum, debet esse ultimo in tempore. Sed summa perfectio humanae
naturae est in unione ad verbum, quia in Christo complacuit omnem
plenitudinem divinitatis inhabitare, ut apostolus dicit, Coloss. I. Ergo
incarnatio debuit differri usque in finem mundi.


&

さらに、すでに述べられたとおり、同一のものにおいて、完全なものは、時間的
 に、不完全なものよりあとにある。ゆえに、最大限に完全なものは、時間にお
 いてもっとも後でなければならない。しかし、人間本性の最高の完成は、言葉
 への合一である。なぜなら、使徒が『コロサイの信徒への手紙』第1章で
 「キリストの中に、神性の全ての充満が宿ることを喜び」\footnote{「神は、御心のままに、満ちあふれるものを余すところなく御子
 の内に宿らせ、」(1:19)}とあるからである。ゆえに、受肉は世界の終わりまで
 延期されるべきであった。


\\


{\scshape 3 Praeterea}, non est conveniens fieri per
duo quod per unum fieri potest. Sed unus Christi adventus sufficere
poterat ad salutem humanae naturae, qui erit in fine mundi. Ergo non
oportuit quod antea veniret per incarnationem. Et ita incarnatio
differri debuit usque in finem mundi.


&

さらに、一つのものから生じうるものが二つのものから生じるのは適切でない。
しかし、世界の終わりに、一度キリストが到来すれば、人間本性の救済には十分で
 あった。ゆえに、その前に、受肉によって彼が来る必要はなかった。したがっ
 て、受肉は世界の終わりまで延期されるべきであった。


\\


{\scshape Sed contra est} quod dicitur Habacuc III, in
medio annorum notum facies. Non ergo debuit incarnationis mysterium, per
quod mundo innotuit, usque in finem mundi differri.


&


しかし反対に、『ハバクク書』第3章「あなたは世々の中間に知らせ」\footnote{「主よ、あなたの名声をわたしは聞きました。主よ、わたしはあなたの御業に畏れを抱きます。数年のうちにも、それを生き返らせ/数年のうちにも、それを示してください。怒りのうちにも、憐れみを忘れないでください。」(3:2)}
と言われている。ゆえに、それによって世界に知らせたところの受肉の秘蹟は、
 世界の終わりまで延期されるべきでなかった。


\\


{\scshape Respondeo dicendum} quod, sicut non fuit
conveniens Deum incarnari a principio mundi, ita non fuit conveniens
quod incarnatio differretur usque in finem mundi. Quod quidem apparet,
primo, ex unione divinae et humanae naturae. 



&

解答する。以下のように言われるべきである。
神が世界の最初に受肉することが適切でなかったのと同様に、受肉が世界の終わりまで
 延期されることも適切でなかった。このことは第一に、神と人間の本性の合一
 から明らかである。


\\


Sicut enim dictum est,
perfectum uno modo tempore praecedit imperfectum, in eo enim quod de
imperfecto fit perfectum, imperfectum tempore praecedit perfectum; in eo
vero quod est perfectionis causa efficiens, perfectum tempore praecedit
imperfectum. In opere autem incarnationis utrumque concurrit. 




&

すでに述べられたとおり、完全なものは、一つのかたちでは、時間的に不完全な
 ものに先行する。不完全なものから完全なものが生じるものにおいては、不完
 全なののが時間的に完全なものに先行するからである。しかし、完全性の作出
 因であるものにおいては、完全なものが、時間的に、不完全なものに先行する。
受肉の業においては、どちらも生じる。


\\


Quia
natura humana in ipsa incarnatione est perducta ad summam perfectionem,
et ideo non decuit quod a principio humani generis incarnatio facta
fuisset. Sed ipsum verbum incarnatum est perfectionis humanae causa
efficiens, secundum illud Ioan.~{\scshape i}, {\itshape de plenitudine eius omnes accepimus},
et ideo non debuit incarnationis opus usque in finem mundi differri. Sed
perfectio gloriae, ad quam perducenda est ultimo natura humana per
verbum incarnatum, erit in fine mundi. 


&

というのも、人間本性は、受肉それ自体において、最高の完全性へと導かれるの
 で、それゆえ、人類の最初から受肉がなされることは適切でなかった。しかし、
 かの『ヨハネによる福音書』第1章「その充満から、私たちは全てのことを受け
 た」\footnote{「わたしたちは皆、この方の満ちあふれる豊かさの中から、恵
 みの上に、更に恵みを受けた。」(1:16)}によれば、 受肉される言葉そのもの
 は、人間の完全性の作出因である。ゆえに、受肉の業は、世界の終わりまで延
 期されるべきでなかった。しかし、受肉した言葉によって人間本性が最終的に
 それへと導かれるべき栄光の完全性は、世界の終わりにあるだろう。


\\

Secundo, ex effectu humanae
salutis. Ut enim dicitur in libro {\itshape de quaest.~Nov.~et Vet.~Test}., {\itshape in
potestate dantis est quando vel quantum velit misereri. Venit ergo
quando et subveniri debere scivit, et gratum futurum beneficiumi}.



&

第二に、人間の救済の結果によって明らかである。
『新約と旧約の問いについて』という書物で以下のように言われてる。「与える
 者の権能の中に、いつ、あるいはどれだけ憐れむことを欲するかということが
 ある。ゆえに、助けるべきだと知るときに、そして、有益なものが感謝されるで
 あろうときにやって来る。

\\


{\itshape Cum
enim languore quodam humani generis obsolescere coepisset cognitio Dei
inter homines et mores immutarentur, eligere dignatus est Abraham, in
quo forma esset renovatae notitiae Dei et morum. Et cum adhuc reverentia
segnior esset, postea per Moysen legem litteris dedit. 
}


&

たとえば、人類の弱さによって、人々の間に神につ
 いて認識が廃れ始め、習俗が変わりつつあったときに、アブラハムを選び、彼
 において、神についての知と習俗の刷新が形を取るようにする価値があると考
 えた。そして、崇敬が弱くなったときに、モーセをとおして文字によって法を
 与えた。
\\


{\itshape Et quia eam
gentes spreverunt non se subiicientes ei, neque hi qui acceperunt
servaverunt, motus misericordia dominus misit filium suum, qui, data
omnibus remissione peccatorum, Deo patri illos iustificatos offerret}. Si
autem hoc remedium differretur usque in finem mundi, totaliter Dei
notitia et reverentia et morum honestas abolita fuisset in
terris. 



&

そして部族たちがその法を蔑ろにし、自らをそれに従属させなかった
 ので、また、それを受け入れた者たちは、それを保存しなかったので、主は憐れみ
 に動かされて自分の息子を送った。その息子は、万人に罪人たちの許しが与えられ、
 父なる神に、彼らが義とされることを申し出た」。しかし、この救済が世界の
 終わりまで延期されたならば、神についての知、神への尊敬、そして正しい習俗は、完全
 にこの地においてなくなっていたであろう。

\\


Tertio apparet quod hoc non fuisset conveniens ad
manifestationem divinae virtutis, quae pluribus modis homines salvavit,
non solum per fidem futuri, sed etiam per fidem praesentis et
praeteriti.


&


第三に、これは、将来への信仰だけでなく、現在と過去に
 ついての信仰によっても、複数のしかたで人々を救った、神の力を明示するた
 めにも、適切でなかったことが明らかである。


\\


{\scshape Ad primum ergo dicendum} quod Glossa illa
exponit de misericordia perducente ad gloriam. Si tamen referatur ad
misericordiam exhibitam humano generi per incarnationem Christi,
sciendum est quod, sicut Augustinus dicit, in libro {\itshape Retractationum},
tempus incarnationis potest comparari iuventuti humani generis, {\itshape propter
vigorem fervoremque fidei, quae per dilectionem operatur}, senectuti
autem, quae est sexta aetas, {\itshape propter numerum temporum, quia Christus
venit in sexta aetate}. 


&

第一異論に対しては、それゆえ、以下のように言われるべきである。
かの注解は、栄光へ導く憐れみについて説明している。しかし、もし、キリスト
 の受肉によって人類に示された憐れみに関係づけられるならば、以下のことが
 知られるべきである。
アウグスティヌスが『再考録』という書物で述べるように、受肉の時は、「愛に
 よって働く信仰の強さと熱意のために」人類の青年時代に比較され、また、
 「時間の数のために、なぜならキリストは第六の時代に来たので」老年に比較
 される。老年は、第六の時代だから。


\\

Et {\itshape quamvis in corpore non possit esse simul
iuventus et senectus, potest tamen simul esse in anima, illa propter
alacritatem, ista propter gravitatem}. Et ideo in libro {\itshape Octogintatrium
quaest}., alicubi dixit Augustinus quod {\itshape non oportuit divinitus venire
magistrum, cuius imitatione humanum genus in mores optimos formaretur,
nisi tempore iuventutis} alibi autem dixit Christum in sexta aetate
humani generis, tanquam in senectute, venisse.


&

また、「身体において、青年と老年は同時にありえないが、魂においては同時に
 ありえる。前者は活発さのために、後者は重厚さのために」。ゆえに、『八十
 三問題集』という書物のどこかで、アウグスティヌスは「青年の時代でなけれ
 ば、それを真似て人類が最善の習俗へと形成されるところの神の教師がやって
 来る必要はなかった」と述べ、しかし別のところでは、キリストは、人類の第
 六の時代に、つまり老年においてやって来たと述べた。


\\


{\scshape Ad secundum dicendum} quod opus incarnationis
non solum est considerandum ut terminus motus de imperfecto ad
perfectum, sed ut principium perfectionis in humana natura, ut dictum
est.


&

第二異論に対しては、以下のように言われるべきである。
受肉の業は、不完全から完全への運動の終局としてのみ考えられるべきでなく、
 すでに述べられたとおり、人間本性における完成の始まりとしても考えられる
 べきである。


\\


{\scshape Ad tertium dicendum} quod, sicut Chrysostomus
dicit, super illud Ioan., ``non misit Deus filium suum in mundum ut
iudicet mundum,'' {\itshape duo sunt Christi adventus, primus quidem, ut remittat
peccata; secundus, ut iudicet. Si enim hoc non fecisset, universi simul
perditi essent, omnes enim peccaverunt, et egent gloria Dei}. Unde patet
quod non debuit adventum misericordiae differre usque in finem mundi.


&

第三異論に対しては、以下のように言われるべきである。
クリュソストムスは、かの『ヨハネによる福音書』「神は自分の息子を、世界を
 助けるために世界に送った」\footnote{「神が御子を世に遣わされたのは、世
 を裁くためではなく、御子によって世が救われるためである。」(3:17)}を注解
 して以下のように述べている。「キリストの到来は二つある。一つは、罪を赦
 すためにであり、もう一つは、裁くためである。もし、これをしなかったなら
 ば、宇宙は一度に消えていたであろう。すべての人が罪を犯し、神の栄光を必
 要とするからである」。従って、憐れみが世界の終わりまで延期されるべきで
 なかったことは明らかである。





\end{longtable}

\end{document}