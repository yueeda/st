\documentclass[10pt]{jsarticle} % use larger type; default would be 10pt
\usepackage[utf8]{inputenc} % set input encoding (not needed with XeLaTeX)
\usepackage[T1]{fontenc}
%\usepackage[round,comma,authoryear]{natbib}
%\usepackage{nruby}
\usepackage{okumacro}
\usepackage{longtable}
%\usepqckage{tablefootnote}
\usepackage[polutonikogreek,english,japanese]{babel}
%\usepackage{amsmath}
\usepackage{latexsym}
\usepackage{color}
\usepackage{otf}
\usepackage{schemata}
%----- header -------
\usepackage{fancyhdr}
\pagestyle{fancy}
\lhead{{\it Summa Theologiae} I-II, q.51}
%--------------------


\bibliographystyle{jplain}


\title{{\bf Prima Secundae}\\{\HUGE Summae Theologiae}\\Sancti Thomae
Aquinatis\\{\sffamily QUAESTIO QUINQUAGESIMAPRIMA}\\{\bf DE CAUSA
HABITUUM QUANTUM AD GENERATIONEM IPSORUM}}
\author{Japanese translation\\by Yoshinori {\sc Ueeda}}
\date{Last modified \today}

%%%% コピペ用
%\rhead{a.~}
%\begin{center}
% {\Large {\bf }}\\
% {\large }\\
% {\footnotesize }\\
% {\Large \\}
%\end{center}
%
%\begin{longtable}{p{21em}p{21em}}
%
%&
%
%\\
%\end{longtable}
%\newpage

\begin{document}

\maketitle
\thispagestyle{empty}
\begin{center}
{\Large 『神学大全』第二部の一\\第五十一問\\習慣の原因について\\その生
 成にかんして}
\end{center}


\begin{longtable}{p{21em}p{21em}}

Deinde considerandum est de causa habituum. Et primo, quantum ad
generationem ipsorum; secundo, quantum ad augmentum; tertio, quantum
ad diminutionem et corruptionem. Circa primum quaeruntur
quatuor. 


\begin{enumerate}
 \item utrum aliquis habitus sit a natura.
 \item utrum aliquis habitus ex actibus causetur.
 \item utrum per unum actum possit generari habitus.
 \item utrum aliqui habitus sint in hominibus infusi a Deo.
\end{enumerate}

&

次に習慣の原因について考察されるべきである。そして第一に、習慣の生成に
 ついて、第二に習慣の増大について、第三に減衰と消滅について考察される
 べきである。第一をめぐって四つのことが問われる。

\begin{enumerate}
 \item ある習慣は本性によって生じるか。
 \item ある習慣は作用に基づいて原因されるか。
 \item 一つの作用から習慣が生成するか。
 \item ある習慣は神から人間へ注入されたものか。
\end{enumerate}
\end{longtable}
\newpage
\rhead{a.~1}
\begin{center}
{\Large {\bf ARTICULUS PRIMUS}}\\
{\large UTRUM ALIQUIS HABITUS SIT A NATURA}\\
{\footnotesize Infra, q.63, a.1.}\\
{\Large 第一項\\何らかの習慣は本性によって生じるか}
\end{center}

\begin{longtable}{p{21em}p{21em}}


{\scshape Ad primum sic proceditur}. Videtur quod nullus habitus sit a
natura. Eorum enim quae sunt a natura, usus non subiacet
voluntati. Sed {\itshape habitus est quo quis utitur cum voluerit}, ut
dicit Commentator, in III {\itshape de Anima}. Ergo habitus non est a
natura.

&

第一項の問題へ議論は以下のように進められる。どんな習慣も本性からではな
いと思われる。理由は以下の通り。本性によってあるものの使用は意志のもと
にない。しかるに、『デ・アニマ』第3巻において注解者は「習慣は、ある人
が意志したときに用いるもの」である。ゆえに習慣は本性によってあるのでは
ない。

\\



2. {\scshape Praeterea}, natura non facit per duo quod per unum potest
facere. Sed potentiae animae sunt a natura. Si igitur habitus
potentiarum a natura essent, habitus et potentia essent unum.


&

さらに、本性は、一つのものによってできることを二つのものによってしたり
はしない。しかるに魂の能力は本性によってある。ゆえに、もし能力の習慣が
本性によってあったならば、習慣と能力が一つのものだっただろう。

\\





3. {\scshape Praeterea}, natura non deficit in necessariis. Sed
habitus sunt necessarii ad bene operandum, ut supra dictum est. Si
igitur habitus aliqui essent a natura, videtur quod natura non
deficeret quin omnes habitus necessarios causaret. Patet autem hoc
esse falsum. Ergo habitus non sunt a natura.


&

さらに、本性は必要なものにおいて欠けることがない。しかるに習慣は、前に
述べられたとおり、よく働くために必要である。ゆえにもしある習慣が本性に
よってあったならば、全ての必要な習慣を原因することから欠けることがなかっ
たであろう。しかるにこれが偽であることは明らかである。ゆえに習慣は本性
によってあるのではない。

\\





{\scshape Sed contra est} quod in VI {\itshape Ethic}., inter alios
habitus ponitur intellectus principiorum, qui est a natura, unde et
principia prima dicuntur naturaliter cognita.

&

しかし反対に、『ニコマコス倫理学』第6巻で、他の習慣の中に、諸原理につ
いての直知が置かれているが、これは本性によってある。したがって、第一諸
原理も、自然本性的に知られると言われる。

\\


{\scshape Respondeo dicendum} quod aliquid potest esse naturale alicui
dupliciter. Uno modo, secundum naturam speciei, sicut naturale est
homini esse risibile, et igni ferri sursum.


&

解答する。以下のように言われるべきである。何かが何かにとって、二つのし
かたで自然本性的でありうる。一つには、種の本性にしたがってであり、たと
えば人間にとって笑いうることが、火にとって上方に運ばれることが自然本性
的である。

\\



Alio modo, secundum naturam individui, sicut naturale est Socrati vel
Platoni esse aegrotativum vel sanativum, secundum propriam
complexionem.


&

もう一つには、個体の本性にしたがってであり、たとえばソクラテスやプラト
ンにとって、固有の体質に即して、病気になり得ることや健康になり得ること
が本性的である。

\\



Rursus, secundum utramque naturam potest dici aliquid naturale
dupliciter, uno modo, quia totum est a natura; alio modo, quia
secundum aliquid est a natura, et secundum aliquid est ab exteriori
principio.


&

さらに、どちらの本性に即しても、二通りのしかたであるものが自然本性的で
あると言われうる。一つには、全体が本性によってある場合であり、もう一つ
には、あるものに即してそれが本性によってあり、別のものに即しては外的な
根源によってある場合である。

\\



Sicut cum aliquis sanatur per seipsum, tota sanitas est a natura, cum
autem aliquis sanatur auxilio medicinae, sanitas partim est a natura,
partim ab exteriori principio.


&

たとえば、ある人が自分の力で健康になるとき、健康全体が本性によってある
が、ある人が薬の助けによって健康になるときは、健康は部分的に本性によっ
て、部分的に外的な根源によってある。

\\



Sic igitur si loquamur de habitu secundum quod est dispositio subiecti
in ordine ad formam vel naturam, quolibet praedictorum modorum
contingit habitum esse naturalem.


&

ゆえに、このようにして、もし私たちが形相や本性への秩序における基体の態
勢であるかぎりにおける習慣について語るならば、前述のしかたのどれによっ
ても、習慣は本性的でありうる。

\\



Est enim aliqua dispositio naturalis quae debetur humanae speciei,
extra quam nullus homo invenitur. Et haec est naturalis secundum
naturam speciei.



&

すなわち、人間の種に必要であり、それの外には人間は見出されないような本
性的な態勢が存在する。これは種の本性にしたがって本性的である。

\\



Sed quia talis dispositio quandam latitudinem habet, contingit
diversos gradus huiusmodi dispositionis convenire diversis hominibus
secundum naturam individui.


&

しかしこのような態勢は、ある種の広さを持っていて、個体の本性に応じてさ
まざまな人間に、さまざまな程度でこのような態勢が生じる。

\\



Et huiusmodi dispositio potest esse vel totaliter a natura, vel partim
a natura et partim ab exteriori principio, sicut dictum est de his qui
sanantur per artem.


&

そしてこのような態勢は、全体的に本性によるか、あるいは、技術を通して健
康になる人について述べられたように、部分的に本性により、部分的に外的な
根源によるかのいずれかである。

\\



Sed habitus qui est dispositio ad operationem, cuius subiectum est
potentia animae, ut dictum est, potest quidem esse naturalis et
secundum naturam speciei, et secundum naturam individui.


&

しかし働きへの態勢である習慣は、すでに述べられたとおりその基体は魂の能
力であるが、種の本性にしたがっても、個体の本性にしたがっても、本性的で
ありうる。

\\



Secundum quidem naturam speciei, secundum quod se tenet ex parte
ipsius animae, quae, cum sit forma corporis, est principium
specificum. Secundum autem naturam individui, ex parte corporis, quod
est materiale principium.


&

種の本性にしたがってある場合とは、魂の側から成立する限りにおいてである。
魂は身体の形相であるから、種を分ける根源だからである。他方、個体の本性
にしたがってある場合とは、身体の側から成立する限りにおいてであり、それ
は身体が質料的な根源だからである。

\\




Sed tamen neutro modo contingit in hominibus esse habitus naturales
ita quod sint totaliter a natura. In Angelis siquidem contingit, eo
quod habent species intelligibiles naturaliter inditas, quod non
competit animae humanae, ut in primo dictum est.


&

しかしどちらの場合も、人間において、全体的に本性によってあるようなしか
たでの本性的な習慣は存在しえない。これは天使においては生じる。なぜなら、
天使は本性的に与えられた可知的形象を持っているからであるが、第一部で述
べられたように、これは人間の魂にはあてはまらない。

\\



Sunt ergo in hominibus aliqui habitus naturales, tanquam partim a
natura existentes et partim ab exteriori principio; aliter quidem in
apprehensivis potentiis, et aliter in appetitivis.

&

ゆえに、人間の中には、部分的に本性によってあるが、部分的に外的な根源に
よってあるような自然本性的な習慣が存在する。ただし、把握能力においてと
欲求能力においては違うしかたによってであるが。

\\



In apprehensivis enim potentiis potest esse habitus naturalis secundum
inchoationem, et secundum naturam speciei, et secundum naturam
individui.


&

把握能力の中には、自然本性的な習慣は、端緒において、種の本性において、
そして個人の本性においてありうる。

\\


Secundum quidem naturam speciei, ex parte ipsius animae, sicut
intellectus principiorum dicitur esse habitus naturalis. Ex ipsa enim
natura animae intellectualis, convenit homini quod statim, cognito
quid est totum et quid est pars, cognoscat quod omne totum est maius
sua parte, et simile est in ceteris.

&

種の本性においては、魂自身の側に基づいてであり、たとえば諸原理について
の直知が、自然本性的な習慣と言われる。というのも、知性的魂の本性自体に
基づいて、人間には、全体が何か、部分が何かを知ると、ただちに、全ての全
体は部分よりも大きいということを認識するということが適合する。その他、
似たようなものにおいてもそうである。


\\



Sed quid sit totum, et quid sit pars, cognoscere non potest nisi per
species intelligibiles a phantasmatibus acceptas. Et propter hoc
philosophus, in fine {\itshape Posteriorum}, ostendit quod cognitio
principiorum provenit nobis ex sensu.

&

しかし、何が全体か、何が部分かは、表象像から受け取られた可知的形象を通
してでなければ認識することができない。このために哲学者は『分析論後書』
の最後で、諸原理の認識は、感覚に基づいて私たちにやって来ることを示して
いる。

\\



Secundum vero naturam individui, est aliquis habitus cognoscitivus
secundum inchoationem naturalis, inquantum unus homo, ex dispositione
organorum, est magis aptus ad bene intelligendum quam alius, inquantum
ad operationem intellectus indigemus virtutibus sensitivis.

&

他方、個人の本性にしたがっては、ある認識的な習慣が本性的な始まりに即し
て、一人の人が、諸器官の態勢に基づいて、他の人よりよく知性認識すること
に適している限りにおいて見出される。それは、知性の働きにかんして私たち
が感覚の力を必要とする限りにおいてである。

\\



In appetitivis autem potentiis non est aliquis habitus naturalis
secundum inchoationem, ex parte ipsius animae, quantum ad ipsam
substantiam habitus, sed solum quantum ad principia quaedam ipsius,
sicut principia iuris communis dicuntur esse seminalia virtutum.

&

また、欲求的な能力においては、ある本性的な習慣が、始まりに即して、魂自
身の側から、習慣の実体自体にかんしてあるということはなく、たとえば共通
の法の諸原理が、諸徳の種子であると言われるように、ただそれの一種の原理
にかんしてのみある。

\\



Et hoc ideo, quia inclinatio ad obiecta propria, quae videtur esse
inchoatio habitus, non pertinet ad habitum, sed magis pertinet ad
ipsam rationem potentiarum.


&

そしてこれは、固有の対象への傾向性、これが習慣の始まりであると思われる
が、は、習慣にではなくむしろ諸能力の性格に属するからである。

\\


Sed ex parte corporis, secundum naturam individui, sunt aliqui habitus
appetitivi secundum inchoationes naturales. Sunt enim quidam dispositi
ex propria corporis complexione ad castitatem vel mansuetudinem, vel
ad aliquid huiusmodi.

&

しかし身体の側からは、個体の本性に即して、何らかの欲求的な習慣が、本性
的な始まりにおいてある。たとえば、ある人々は、固有の身体の自然的な体質
に基づいて、潔癖さや穏やかさや何かそのようなものに態勢付けられているよ
うに。

\\


{\scshape Ad primum ergo dicendum} quod obiectio illa procedit de
natura secundum quod dividitur contra rationem et voluntatem, cum
tamen ipsa ratio et voluntas ad naturam hominis pertineant.

&

第一異論に対しては、それゆえ、以下のように言われるべきである。この異論
は、理性と意志に対立して分けられるかぎりでの本性について論じられている
が、理性も意志も人間の本性に属している。

\\




{\scshape Ad secundum dicendum} quod aliquid etiam naturaliter potest
superaddi potentiae, quod tamen ad ipsam potentiam pertinere non
potest. Sicut in Angelis non potest pertinere ad ipsam potentiam
intellectivam quod sit per se cognoscitiva omnium, quia oporteret quod
esset actus omnium, quod solius Dei est.

&

第二異論に対しては以下のように言われるべきである。能力自体に属すること
ができないあるものが、本性的に能力に付け加えられるということはありうる。
たとえば天使において、それ自身によって万物を認識しうるものであることが、
知性的な能力それ自体に属することはできない。なぜなら、そのためには万物
の現実態でなければならなかっただろうが、それは神だけに属することだから。

\\



Id enim quo aliquid cognoscitur, oportet esse actualem similitudinem
eius quod cognoscitur, unde sequeretur, si potentia Angeli per seipsam
cognosceret omnia, quod esset similitudo et actus omnium.

&

それによって何かが認識されるところのものは、認識されるものの現実的な類
似である必要があるので、もし天使の能力がそれ自体によって万物を認識した
ならば、天使は万物の類似であり現実態であっただろう。

\\


Unde oportet quod superaddantur potentiae intellectivae ipsius aliquae
species intelligibiles, quae sunt similitudines rerum intellectarum,
quia per participationem divinae sapientiae, et non per essentiam
propriam, possunt intellectus eorum esse actu ea quae intelligunt. Et
sic patet quod non omne id quod pertinet ad habitum naturalem, potest
ad potentiam pertinere.

&

このことから、天使の知的能力に、何らかの可知的形象が加えられなければな
らず、それは知性認識された諸事物の類似である。なぜなら、固有の本質によっ
てではなく、神の知恵の分有によって、天使たちの知性は、現実態において、
彼らが認識するところのものでありうるからである。このようにして、自然本
性的な習慣に属する全てのものが、能力に属するわけではないことが明らかで
ある。

\\





{\scshape Ad tertium dicendum} quod natura non aequaliter se habet ad
causandas omnes diversitates habituum, quia quidam possunt causari a
natura, quidam non, ut supra dictum est. Et ideo non sequitur, si
aliqui habitus sint naturales, quod omnes sint naturales.

&

第三異論に対しては以下のように言われるべきである。自然本性が、習慣の全
ての差異を原因することに等しく関係するわけではない。なぜなら、前に述べ
られたとおり、ある習慣は本性によって原因されうるが、別の習慣はそうでな
いからである。ゆえに、もしある習慣が本性的であれば全ての習慣が本性的で
あるということは帰結しない。

\end{longtable}
\newpage




\rhead{a.~2}
\begin{center}
{\Large {\bf ARTICULUS SECUNDUS}}\\ {\large UTRUM ALIQUIS HABITUS
CAUSETUR EX ACTIBUS}\\ {\footnotesize {\itshape De Malo}, q.11, a.2,
ad 4; {\itshape De Virtut.}, q.1, a.9.}\\ {\Large 第二項\\何らかの習慣
は作用から原因されるか}
\end{center}

\begin{longtable}{p{21em}p{21em}}

{\scshape Ad secundum sic proceditur}. Videtur quod nullus habitus
possit ex actu causari. Habitus enim est qualitas quaedam, ut supra
dictum est. Omnis autem qualitas causatur in aliquo subiecto,
inquantum est alicuius receptivum. Cum igitur agens ex hoc quod agit,
non recipiat aliquid, sed magis ex se emittat; videtur quod non possit
aliquis habitus in agente ex propriis actibus generari.

&

第二項の問題へ、議論は以下のように進められる。どんな習慣も作用
\footnote{actusとその派生語は「現実態」「作用」「行為」「能動」などの
意味を含み、日本語から見ると多義的である。一つの訳語を割り当てることは
不可能なので、これらの言葉が出てきたら原文を参照して確認することが望ま
しい。}から原因されることはできないと思われる。理由は以下の通り。習慣
は、前に述べられたとおり、ある種の性質である。しかるに全ての性質は、何
かを受け取りうるかぎりにおいて、ある基体において原因される。ゆえに、作
用者は、作用することに基づいて、何かを受け取ることがなく、むしろ自分か
ら(何かを)出すので、何らかの習慣が作用者の中で、その固有の作用に基づ
いて生み出されることはないと思われる。


\\




2. {\scshape Praeterea}, illud in quo causatur aliqua qualitas,
movetur ad qualitatem illam, sicut patet in re calefacta vel
infrigidata, quod autem producit actum causantem qualitatem, movet, ut
patet de calefaciente vel infrigidante. Si igitur in aliquo causaretur
habitus per actum sui ipsius, sequeretur quod idem esset movens et
motum, agens et patiens. Quod est impossibile, ut dicitur in VII
{\itshape Physic}.

&

さらに、aの中で何らかの性質が原因されるとき、aはその性質へと動かされる。
たとえば熱せられるものや冷やされるものにおいて明らかなように。他方、性
質を原因する作用を生み出すものは、動かす。それは熱するものや冷やすもの
において明らかなとおりである。ゆえに、もしあるものにおいて、習慣が自ら
自身の作用によって原因されるならば、同一のものが動かすものでありかつ動
かされるもの、能動者であるとともに受動者であることになる。これは『自然
学』第7巻で言われるように、不可能である。


\\




3. {\scshape Praeterea}, effectus non potest esse nobilior sua
causa. Sed habitus est nobilior quam actus praecedens habitum, quod
patet ex hoc, quod nobiliores actus reddit. Ergo habitus non potest
causari ab actu praecedente habitum.

&

さらに、結果がその原因より高貴であることはありえない。しかるに習慣は、
その習慣に先行する作用よりも高貴である。このことは、より高貴な作用を行
うことから明らかである。ゆえに習慣は、その習慣に先行する作用によって原
因されることはできない。

\\




{\scshape Sed contra est} quod philosophus, in II {\itshape Ethic}.,
docet habitus virtutum et vitiorum ex actibus causari.

&

しかし反対に、哲学者は『ニコマコス倫理学』第2巻で、美徳と悪徳の習慣は、
作用から原因されることを教えている。

\\




{\scshape Respondeo dicendum} quod in agente quandoque est solum
activum principium sui actus, sicut in igne est solum principium
activum calefaciendi.


&

解答する。以下のように言われるべきである。時として、作用者の中には、自
らの作用の作用的(能動的)根源だけがある。たとえば火の中には、ただ熱す
る作用の根源だけがある。

\\


Et in tali agente non potest aliquis habitus causari ex proprio actu,
et inde est quod res naturales non possunt aliquid consuescere vel
dissuescere, ut dicitur in II {\itshape Ethic}.


&

そしてそのような作用者の中に、固有の作用に基づいて、何らかの習慣が原因
されることはありえない。このことから、『ニコマコス倫理学』第2巻で言わ
れるように、自然的事物が何かに慣れたり慣れなかったりすることはありえな
い。

\\


Invenitur autem aliquod agens in quo est principium activum et
passivum sui actus, sicut patet in actibus humanis.


&

しかし、人間の作用(行為)において明らかであるように、その中に能動的根
源と受動的根源の両方があるような作用者も見出される。

\\

Nam actus appetitivae virtutis procedunt a vi appetitiva secundum quod
movetur a vi apprehensiva repraesentante obiectum, et ulterius vis
intellectiva, secundum quod ratiocinatur de conclusionibus, habet
sicut principium activum propositionem per se notam.

&

すなわち、欲求的力の作用は、対象を表象する把握力によって動かされる限り
において欲求的力から発出し、さらには結論について推論する限りにおいて、
知性的力は、自明の諸命題をあたかも能動的根源であるかのようにもつ。

\\


Unde ex talibus actibus possunt in agentibus aliqui habitus causari,
non quidem quantum ad primum activum principium, sed quantum ad
principium actus quod movet motum.


&

したがって、そのような作用に基づいて、作用者の中に何らかの習慣が原因さ
れうる。しかしそれは第一の能動的根源にかんしてではなく、運動を動かす作
用の根源にかんしてである。

\\

Nam omne quod patitur et movetur ab alio, disponitur per actum
agentis, unde ex multiplicatis actibus generatur quaedam qualitas in
potentia passiva et mota, quae nominatur habitus.


&

すなわち、全て他のものから受動し他のものによって動かされるものは、作用
者の作用を通して態勢付けられる。したがって、多様な作用に基づいて、ある
種の性質が受動的能力、動かされる能力の中に生み出され、それが「習慣」と
名付けられる。

\\

Sicut habitus virtutum moralium causantur in appetitivis potentiis,
secundum quod moventur a ratione, et habitus scientiarum causantur in
intellectu, secundum quod movetur a primis propositionibus.

&

たとえば、理性に動かされることに即して道徳的徳の習慣が欲求的能力の中に
原因され、第一の諸命題によって動かされる限りにおいて知の習慣が知性の中
に原因される。

\\




{\scshape Ad primum ergo dicendum} quod agens, inquantum est agens,
non recipit aliquid. Sed inquantum agit motum ab alio, sic recipit
aliquid a movente, et sic causatur habitus.

&

第一異論に対しては、それゆえ、以下のように言われるべきである。作用者は、
作用者である限りにおいて、何かを受け取ることがない。しかし他のものによっ
て動かされて作用する限りにおいて、動かすものから何かを受け取り、そのよ
うにして習慣が原因される。

\\




{\scshape Ad secundum dicendum} quod idem, secundum idem, non potest
esse movens et motum. Nihil autem prohibet idem a seipso moveri
secundum diversa, ut in VIII {\itshape Physic}.~probatur.

&

第二異論に対しては以下のように言われるべきである。同一のものが同一の点
に即して動かすものでありかつ動かされるものであることは不可能である。し
かし、『自然学』第8巻で証明されているように、異なる点において、同一の
ものが自分自身によって動かされるのは何ら差し支えない。

\\




{\scshape Ad tertium dicendum} quod actus praecedens habitum inquantum
procedit a principio activo, procedit a nobiliori principio quam sit
habitus generatus, sicut ipsa ratio est nobilius principium quam sit
habitus virtutis moralis in vi appetitiva per actuum consuetudines
generatus; et intellectus principiorum est nobilius principium quam
scientia conclusionum.

&

第三異論に対しては以下のように言われるべきである。能動的根源から発出す
る限りにおいて、習慣に先行する作用は、生み出された習慣よりも高貴な根源
から発出する。たとえば、理性自体は、欲求的力の中に慣習的な行為によって
生み出された道徳的徳の習慣よりも高貴な根源であり、また、諸原理について
の知性は結論についての学知よりも高貴な根源である。

\\



\end{longtable}
\newpage

\rhead{a.~3}
\begin{center}
{\Large {\bf ARTICULUD TERTIUS}}\\ {\large UTRUM PER UNUM ACTUM POSSIT
GENERARI HABITUS}\\ {\footnotesize I {\itshape Sent.}, d.17, q.2, a.3,
ad 4; {\itshape De Virtut.}, q.1, a.9, ad 11.}\\ {\Large 第三項\\一つ
の作用が習慣を生み出すか}
\end{center}

\begin{longtable}{p{21em}p{21em}}


{\scshape Ad tertium sic proceditur}. Videtur quod per unum actum
possit habitus generari. Demonstratio enim actus rationis est. Sed per
unam demonstrationem causatur scientia quae est habitus conclusionis
unius. Ergo habitus potest causari ex uno actu.


&

第三項の問題へ、議論は以下のように進められる。一つの作用によって習慣が
生み出されうると思われる。理由は以下の通り。論証は理性の作用である。し
かるに、一つの論証によって、一つの結論の習慣である学知が原因される。ゆ
えに習慣は一つの作用によって原因されうる。

\\



2. {\scshape Praeterea}, sicut contingit actus crescere per
multiplicationem, ita contingit actum crescere per intensionem. Sed
multiplicatis actibus, generatur habitus. Ergo etiam si multum
intendatur unus actus, poterit esse causa generativa habitus.

&

さらに、何度も行うことで作用が増大するように、強度を強めることによって
も作用は増大する。しかるに作用が何度も行われると、習慣が生み出される。
ゆえに、一つの作用が強く意図されると、それは習慣を生み出す原因でありう
る。

\\


3. {\scshape Praeterea}, sanitas et aegritudo sunt habitus quidam. Sed
ex uno actu contingit hominem vel sanari vel infirmari. Ergo unus
actus potest habitum causare.

&

さらに、健康と病気はある種の習慣である。しかるに一つの作用に基づいて人
間は健康になったり病気になったりする。ゆえに一つの作用が習慣を原因しう
る。

\\



{\scshape Sed contra est} quod philosophus dicit, in I {\itshape
Ethic}., quod {\itshape una hirundo ver non facit, nec una dies, ita
utique nec beatum nec felicem una dies, nec paucum tempus}. Sed
{\itshape beatitudo est operatio secundum habitum perfectae virtutis},
ut dicitur in I {\itshape Ethic}. Ergo habitus virtutis, et eadem
ratione alius habitus, non causatur per unum actum.


&

しかし反対に、哲学者は『ニコマコス倫理学』第1巻で「一羽の燕や一日が春
をもたらすことはないように、一日や短い時間が至福で幸福な人を作り出すこ
ともない」と述べている。しかし『ニコマコス倫理学』第1巻で言われるよう
に「至福は完成された徳の習慣にしたがって働きである」。ゆえに、徳の習慣
は、同じ理由で他の習慣も、一つの作用によって原因されない。

\\



{\scshape Respondeo dicendum} quod, sicut iam dictum est, habitus per
actum generatur inquantum potentia passiva movetur ab aliquo principio
activo.

&

解答する。以下のように言われるべきである。すでに述べられたとおり、習慣
は、受動的能力がある能動的根源によって動かされる限りにおいて、作用を通
して生み出される。

\\


Ad hoc autem quod aliqua qualitas causetur in passivo, oportet quod
activum totaliter vincat passivum. Unde videmus quod, quia ignis non
potest statim vincere suum combustibile, non statim inflammat ipsum,
sed paulatim abiicit contrarias dispositiones, ut sic totaliter
vincens ipsum, similitudinem suam ipsi imprimat.


&

ところで、何らかの性質が受動するものにおいて原因されるためには、能動的
なものが完全に受動的なものに勝たなければならない。このことから、火は可
燃物をただちに負かすことができないので、ただちにそれを燃やすことはなく、
少しずつ反対する態勢を除去して、そのようにして全体的にそれを負かし、自
らの類似を可燃物に刻印するのを私たちは見る。

\\

Manifestum est autem quod principium activum quod est ratio, non
totaliter potest supervincere appetitivam potentiam in uno actu, eo
quod appetitiva potentia se habet diversimode et ad multa; iudicatur
autem per rationem, in uno actu, aliquid appetendum secundum
determinatas rationes et circumstantias.


&

また、理性という能動的根源が、一つの行為において欲求的能力に完全に打ち
克つことがないのは明らかである。なぜなら、欲求的能力はさまざまなしかた
で多くのものに関わるが、一つの行為においては、理性によって、限定された
性格や環境に即した欲求されるべきあるものが指し示されるからである。

\\

Unde ex hoc non totaliter vincitur appetitiva potentia, ut feratur in
idem ut in pluribus, per modum naturae, quod pertinet ad habitum
virtutis. Et ideo habitus virtutis non potest causari per unum actum,
sed per multos.

&

したがって、このことから全体的に欲求的能力が負かされて、複数の場合に一
つのものへ本性のあり方によって動かされるというようにはならない。しかし
これが徳の習慣に属する。ゆえに、徳の習慣は一つの行為によって原因され得
ず、多くの行為によって原因される。

\\

In apprehensivis autem potentiis considerandum est quod duplex est
passivum, unum quidem ipse intellectus possibilis; aliud autem
intellectus quem vocat Aristoteles {\itshape passivum}, qui est
{\itshape ratio particularis}, idest vis cogitativa cum memorativa et
imaginativa.


&

他方、把握能力においては、受動的であることが二通りに考察されるべきであ
る。一つには、可能知性自体であり、もう一つは、アリストテレスが受動的と
呼んでいる知性であり、それは個別的理性、すなわち記憶力と想像力を伴った
思考力とである。

\\

Respectu igitur primi passivi, potest esse aliquod activum quod uno
actu totaliter vincit potentiam sui passivi, sicut una propositio per
se nota convincit intellectum ad assentiendum firmiter conclusioni;
quod quidem non facit propositio probabilis.


&

第一の受動にかんしては、一つの作用によって全体的にそれに対応する受動的
能力を負かすような能動的なものが存在しうる。たとえば、自明の一つの命題
が、結論に堅く同意するように知性を打ち負かす場合のように。しかし蓋然的
な命題はこのようなことはしない。



\\

Unde ex multis actibus rationis oportet causari habitum opinativum,
etiam ex parte intellectus possibilis, habitum autem scientiae
possibile est causari ex uno rationis actu, quantum ad intellectum
possibilem.


&

したがって、臆見の習慣は、多くの理性の作用に基づいて、可能知性の側から
も原因されなければならない。対するに学知の習慣は、可能知性に関して、一
つの作用に基づいて原因される。

\\

Sed quantum ad inferiores vires apprehensivas, necessarium est eosdem
actus pluries reiterari, ut aliquid firmiter memoriae imprimatur.
Unde philosophus, in libro {\itshape de Memoria et Reminiscentia},
dicit quod {\itshape meditatio confirmat memoriam}.

&

しかし下位の把握力にかんしては、何かが堅く記憶に刻まれるためには、同一
の作用が複数回繰り返される必要がある。このことから哲学者は『記憶と想起
について』という書物で、「熟考は記憶を固める」と述べている。

\\

Habitus autem corporales possibile est causari ex uno actu, si activum
fuerit magnae virtutis, sicut quandoque medicina fortis statim inducit
sanitatem.



&

これに対して身体的な習慣は、たとえば、強い薬がただちに健康をもたらすよ
うに、もし作用が大きな力をもっていたら、一つの作用によって原因されうる。

\\



Et per hoc patet responsio ad obiecta.

&

そしてこれによって異論に対する解答は明らかである。

\end{longtable}
\newpage


\rhead{a.~4}
\begin{center}
{\Large {\bf ARTICULUS QUARTUS}}\\ {\large UTRUM ALIQUI HABITUS SINT
HOMINIBUS INFUSI A DEO}\\ {\footnotesize Infra, q.63, a.3.}\\ {\Large
第四項\\何らかの習慣が神から人間に注入されるか}
\end{center}

\begin{longtable}{p{21em}p{21em}}

{\scshape Ad quartum sic proceditur}. Videtur quod nullus habitus
hominibus infundatur a Deo. Deus enim aequaliter se habet ad omnes. Si
igitur quibusdam infundit habitus aliquos, omnibus eos
infunderet. Quod patet esse falsum.

&

第四項の問題へ、議論は以下のように進められる。どんな習慣も人間に神から
注入されないと思われる。理由は以下の通り。神は万人に等しく関係する。ゆ
えに、もしある人々にある習慣を注入すれば、それを全て人にそれを注入した
だろう。これは偽であると思われる。

\\

2. {\scshape Praeterea}, Deus operatur in omnibus secundum modum qui
convenit naturae ipsorum, quia {\itshape divinae providentiae est
naturam salvare}, ut dicit Dionysius, {\scshape iv} cap.~{\itshape de
Div.~Nom}. Sed habitus in homine naturaliter causantur ex actibus, ut
dictum est. Non ergo causat Deus in hominibus aliquos habitus absque
actibus.

&

さらに、神はすべての人々に、その人の本性に適合するしかたで働く。なぜな
ら、ディオニュシウスが『神名論』第4章で述べるように、「神の摂理には本
性を救うことが属する」からである。しかるに人間において習慣は、すでに述
べられたとおり、作用に基づいて原因される。ゆえに神が人間の中に、作用な
しに、何らかの習慣を原因することはない。

\\



3. {\scshape Praeterea}, si aliquis habitus a Deo infunditur, per
illum habitum homo potest multos actus producere. Sed {\itshape ex
illis actibus causatur similis habitus}, ut in II {\itshape
Ethic}. dicitur. Sequitur ergo duos habitus eiusdem speciei esse in
eodem, unum acquisitum, et alterum infusum. Quod videtur esse
impossibile, non enim duae formae unius speciei possunt esse in eodem
subiecto. Non ergo habitus aliquis infunditur homini a Deo.

&

さらに、もしある習慣が神から注入されるなら、その習慣によって人間は多く
の作用を生み出すことができる。しかし、『ニコマコス倫理学』第2巻で言わ
れるように、「それらの作用から同様な習慣が原因される」。ゆえに、一つは
獲得されもう一つは注入された、一つの種に属する二つの習慣が、同じ人の中
にあることになる。これは不可能であるように思われる。なぜなら、同一の基
体の中に、一つの種に属する二つの形相があることは不可能だからである。

\\



{\scshape Sed contra est} quod dicitur {\itshape Eccli}.~{\scshape
xv}, {\itshape implevit eum dominus spiritu sapientiae et
intellectus}. Sed sapientia et intellectus quidam habitus sunt. Ergo
aliqui habitus homini a Deo infunduntur.

&

しかし反対に、『集会の書』第15章で「神は彼を知恵と知性の霊で満たした」
と言われている。しかるに知恵と知性はある種の習慣である。ゆえにある習慣
が人間に神から注がれる。

\\



{\scshape Respondeo dicendum} quod duplici ratione aliqui habitus
homini a Deo infunduntur. Prima ratio est, quia aliqui habitus sunt
quibus homo bene disponitur ad finem excedentem facultatem humanae
naturae, qui est ultima et perfecta hominis beatitudo, ut supra dictum
est.

&


解答する。以下のように言われるべきである。二つの理由で、神から人間に何
らかの習慣が注がれる。第一の理由は、ある習慣が、それによって人間が、人
間本性の機能を越える目的へとよく秩序づけられるところのものだからである。
その目的とは、前に述べられたとおり、人間の究極で完全な至福である。

\\

Et quia habitus oportet esse proportionatos ei ad quod homo disponitur
secundum ipsos, ideo necesse est quod etiam habitus ad huiusmodi finem
disponentes, excedant facultatem humanae naturae.

&

そして習慣は、それにしたがって人間がそれへと態勢付けられるところのもの
に比例していなければならないので、そのような目的へ態勢付ける習慣もまた
人間本性の機能を越えていることが必要である。

\\

Unde tales habitus nunquam possunt homini inesse nisi ex infusione
divina, sicut est de omnibus gratuitis virtutibus.

&

したがって、そのような習慣は、神からの注入によらない限り、人間に内在す
ることができない。これは全ての恩恵による美徳と同様である。


\\

Alia ratio est, quia Deus potest producere effectus causarum
secundarum absque ipsis causis secundis, ut in primo dictum est.


&

もう一つの理由は、第一部で述べられたとおり、神が、第二原因の結果を、第
二原因なしに生み出すことができるからである。

\\

Sicut igitur quandoque, ad ostensionem suae virtutis, producit
sanitatem absque naturali causa, quae tamen per naturam posset
causari; ita etiam quandoque, ad ostendendam suam virtutem, infundit
homini illos etiam habitus qui naturali virtute possunt causari.


&

ゆえにちょうど、時として自分の力を示すために、自然的な原因なしに、自然
が原因しうる健康を生み出すように、時として、自らの力を示すために、自然
的な力によって原因されうる習慣をも、人間に注入することがある。

\\

Sicut apostolis dedit scientiam Scripturarum et omnium linguarum, quam
homines per studium vel consuetudinem acquirere possunt, licet non ita
perfecte.

&

たとえば、使徒たちに、聖書や全ての言語の知識を与えたが、人間は勉強や習
慣(consuetudo)によってこれを獲得することができる。ただし、そこまで完全
にではないが。

\\



{\scshape Ad primum ergo dicendum} quod Deus, quantum ad suam naturam,
aequaliter se habet ad omnes, sed secundum ordinem suae sapientiae
certa ratione quaedam tribuit aliquibus, quae non tribuit aliis.

&

第一異論に対しては、それゆえ、以下のように言われるべきである。神は自ら
の本性にかんして万人に等しく関係するが、自らの知恵の秩序においては、あ
る人々に与えるものを別の人々には与えない。

\\




{\scshape Ad secundum dicendum} quod hoc quod Deus in omnibus operatur
secundum modum eorum, non excludit quin Deus quaedam operetur quae
natura operari non potest, sed ex hoc sequitur quod nihil operatur
contra id quod naturae convenit.


&

第二異論に対しては以下のように言われるべきである。神が万物においてそれ
らのあり方に従って働くことは、神が、本性によって働くことができないこと
を神が働くことを排除せず、むしろこのことから帰結するのは、何も本性に適
合するものに反して働くことがないということである。


\\



{\scshape Ad tertium dicendum} quod actus qui producuntur ex habitu
infuso, non causant aliquem habitum, sed confirmant habitum
praeexistentem, sicut medicinalia remedia adhibita homini sano per
naturam, non causant aliquam sanitatem, sed sanitatem prius habitam
corroborant.


&

第三異論に対しては以下のように言われるべきである。注入された習慣によっ
て生み出される作用は、どんな習慣も原因せず、先在する習慣を固める。たと
えば健康な人に使われた医療的な治療は、何らの健康も原因せず、むしろ先に
所有されていた健康を強化するように。



\end{longtable}

\end{document}