\documentclass[10pt]{jsarticle} % use larger type; default would be 10pt
%\usepackage[utf8]{inputenc} % set input encoding (not needed with XeLaTeX)
%\usepackage[round,comma,authoryear]{natbib}
%\usepackage{nruby}
\usepackage{okumacro}
\usepackage{longtable}
%\usepqckage{tablefootnote}
\usepackage[polutonikogreek,english,japanese]{babel}
%\usepackage{amsmath}
\usepackage{latexsym}
\usepackage{color}

%----- header -------
\usepackage{fancyhdr}
\pagestyle{fancy}
\lhead{{\it Summa Theologiae} I, q.~11}
%--------------------

\bibliographystyle{jplain}
\title{{\bf Prima Pars}\\{\HUGE Summae Theologiae}\\Sancti Thomae
Aquinatis\\Quaestio Undecima\\{\bf De Unitate Dei}}
\author{Japanese translation\\by Yoshinori {\sc Ueeda}}
\date{Last modified \today}


%%%% コピペ用
%\rhead{a.~}
%\begin{center}
% {\Large {\bf }}\\
% {\large }\\
% {\footnotesize }\\
% {\Large \\}
%\end{center}
%
%\begin{longtable}{p{21em}p{21em}}
%
%&
%
%\\
%\end{longtable}
%\newpage


\begin{document}
\maketitle
\begin{center}
 {\Large 第十一問\\神の一性について}
\end{center}



\begin{longtable}{p{21em}p{21em}}
Post praemissa, considerandum est de divina unitate. Et circa hoc
quaeruntur quatuor. 
\begin{enumerate}
 \item Primo, utrum unum addat aliquid supra ens.
 \item Secundo, utrum opponantur unum et multa.
 \item Tertio, utrum Deus sit unus.
 \item Quarto,utrum sit maxime unus.
\end{enumerate}

&

上で述べられたことのあとに、神の一性について考察されるべきである。
これについては、四つのことが問われる。
\begin{enumerate}
 \item 一は有に何かを加えるか。
 \item 一と多は対立するか。
 \item 神は一人であるか。
 \item 神は最大限に一人であるか。
\end{enumerate}

\end{longtable}
\newpage
\rhead{a.~1}

\begin{center}
 {\Large {\bf ARTICULUS PRIMUS}}\\
 {\large UTRUM UNUM ADDAT ALIQUID SUPRA ENS}\\
 {\footnotesize Infra, q.~30, a.~3; I {\itshape Sent.}, d.~19, a.~1, ad
 2; d.~24, a.~3; {\itshape De Pot.}, q.~9, a.~7; {\itshape Quodl.}~X,
 q.~1, a.~1; IV {\itshape Metaphys.}, lect.~2; X, Lect.~3.}\\
 {\Large 第一項\\一は有に何かを付加するか}
\end{center}



\begin{longtable}{p{21em}p{21em}}
{\huge A}{\scshape d primum sic proceditur}. Videtur quod unum addat aliquid supra
ens. Omne enim quod est in aliquo genere determinato, se habet ex
additione ad ens, quod circuit omnia genera. Sed unum est in genere
determinato, est enim principium numeri, qui est species
quantitatis. Ergo unum addit aliquid supra ens.

&

第一項の問題へ議論は以下のように進められる。一は、有に何かを加えると思
われる。理由は以下の通り。すべてなんらかの限定された類の中にあるものは、
有に何かを加えることによってある。そして有はすべての類を包含する。しか
るに、一は限定された類の中にある。なぜなら、一は数の根源であり、数は量
の種だからである。ゆえに、一は有に何かを加える。

\\

2.~{\scshape Praeterea}, quod dividit aliquod commune, se habet ex additione ad
illud. Sed ens dividitur per unum et multa. Ergo unum addit aliquid
supra ens.

& 

さらに、ある共通なものを分けるものは、それ(=共通なもの)への付加によっ
てある。しかるに、有は、一と多によって分けられる。ゆえに、一は有に何か
を加える。

\\

3.~{\scshape Praeterea}, si unum non addit supra ens, idem esset dicere
{\itshape unum} et {\itshape ens}. Sed nugatorie dicitur {\itshape ens
ens}. Ergo nugatio esset dicere {\itshape ens unum}, quod falsum
est. Addit igitur unum supra ens.

&

さらに、もし一が有に何も加えないならば、一と言うのも有と言うのも同じこ
とだっただろう。しかるに、「有は有である」と語ることは無意味である。ゆ
 えに(もし一が有に何も加えないならば)「有は一である」と語ることも無
 意味だっただろうが、それは偽である(無意味ではない)。ゆえに、一は有
に何かを加える。

\\

{\sc Sed contra est} quod dicit Dionysius, ult.~cap.~{\it de Div.~Nom.}, {\it nihil est
existentium non participans uno}, quod non esset, si unum adderet supra
ens quod contraheret ipsum. Ergo unum non habet se ex additione ad ens.

&

しかし反対に、ディオニュシウスは、『神名論』最終章で、「存在するものの
中で一を分有していないものはない」と述べている。もし、一が有を制限する
ような何かを有に付加していたら、こういうことはなかったであろう。ゆえに、
一は有に何かを加えるようなものではない。

\\

{\sc Respondeo dicendum} quod unum non addit supra ens rem aliquam,
sed tantum negationem divisionis, unum enim nihil aliud significat
quam ens indivisum. Et ex hoc ipso apparet quod unum convertitur cum
ente.

&

解答する。以下のように言われるべきである。一は有になんらかの事物(=実
在的な何か)を付加するのではなく、たんに分割の否定を付加する。なぜな
ら、「一」はまさに「分割されていない有」を表示するからである。また、
このことから一と有は置換されることが明らかである。


\\

Nam omne ens aut est simplex, aut compositum. Quod autem est
simplex, est indivisum et actu et potentia. Quod autem est compositum,
non habet esse quandiu partes eius sunt divisae, sed postquam
constituunt et componunt ipsum compositum. Unde manifestum est quod
esse cuiuslibet rei consistit in indivisione. Et inde est quod
unumquodque, sicut custodit suum esse, ita custodit suam unitatem.

&

すなわち、すべて
有は単純か複合されているかのどちらかである。単純である有は、現実態に
おいても可能態においても分割されていない。これに対して、複合されている
有はその部分がバラバラであるあいだは存在を持たず、(それらの部分が)
複合体そのものを構成して複合したあとに存在を持つ。したがって、どんな事
物の存在も分割されていないことにおいて成立することが明らかである。各々
のものが自らの存在を保つように、自らの一性を保つのもこのためである。

\\


{\scshape Ad primum igitur dicendum} quod quidam, putantes idem esse
unum quod convertitur cum ente, et quod est principium numeri, divisi
sunt in contrarias positiones. Pythagoras enim et Plato, videntes quod
unum quod convertitur cum ente, non addit aliquam rem supra ens, sed
significat substantiam entis prout est indivisa, existimaverunt sic se
habere de uno quod est principium numeri. Et quia numerus componitur ex
unitatibus, crediderunt quod numeri essent substantiae omnium rerum.

&

第一異論に対しては、それゆえ、次のように言われるべきである。有と置換さ
れる一と、数の根源(始まり)の一とが同じであると考える人々は、反対の立
場に分かれている。すなわち、ピュタゴラスとプラトンは、有と置換される一
が有になんら実在的なものを加えず、かえって分割されていないものとしての
有の実体を意味する点を見て、数の始まりである一もそれと同じようにあると
考えた。そして数は一性から複合されるので、数がすべての事物の実体である
と信じた。

\\

-- E contrario autem Avicenna, considerans quod unum quod est principium
numeri, addit aliquam rem supra substantiam entis (alias numerus ex
unitatibus compositus non esset species quantitatis), credidit quod unum
quod convertitur cum ente, addat rem aliquam supra substantiam entis,
sicut album supra hominem.  

&

反対にアヴィセンナは、数の始まりである一が有の実体に何か実在的なもの
を加えることを見て(さもなければ、一性から複合される数が量の種であること
はなかったであろう)、有と置換される一が、ちょうど白が人間に何か実在的な
ものを加えるように、有の実体に何か実在的なものを加えると信じた。


\\


Sed hoc manifeste falsum est, quia quaelibet res est una per suam
substantiam. Si enim per aliquid aliud esset una quaelibet res, cum
illud iterum sit unum, si esset iterum unum per aliquid aliud, esset
abire in infinitum. Unde standum est in primo.

&

しかしこれは明らかに偽である。なぜなら、どんな事物も自らの実体によっ
て一であるから。つまり、もしもどんな事物も何か別のものによって一であ
るならば、その「別のもの」もまた一であるので、それもまた別の何かによっ
て一であるとすれば、無限に進むことになる。したがって、最初の段階で留ま
るべきである。

\\


-- Sic igitur dicendum est quod unum quod convertitur cum ente, non addit
aliquam rem supra ens, sed unum quod est principium numeri, addit
aliquid supra ens, ad genus quantitatis pertinens.

&

ゆえに、次のように言われるべきである。有と置換される一は、有に何か実在
的なものを加えるのではないが、数の始まりである一は、有に何か(実在的な
もの)を加え、量の類に属する。

\\



{\scshape Ad secundum dicendum} quod nihil prohibet id quod est uno modo
divisum, esse alio modo indivisum; sicut quod est divisum numero, est
indivisum secundum speciem, et sic contingit aliquid esse uno modo unum,
alio modo multa.

&

第二異論に対しては次のように言われるべきである。あるしかたで分割されて
いるものが、べつのしかたで分割されていないとしても、それは差し支えない。
たとえば、数的に分割されているものが、種において分割されていない場合が
それであり、このように、あるものが、あるしかたで一であり、別のしかたで
多であることが起こる。

\\

Sed tamen si sit indivisum simpliciter; vel quia est indivisum
secundum id quod pertinet ad essentiam rei, licet sit divisum quantum ad
ea quae sunt extra essentiam rei, sicut quod est unum subiecto et multa
secundum accidentia; vel quia est indivisum in actu, et divisum in
potentia, sicut quod est unum toto et multa secundum partes, huiusmodi
erit unum simpliciter, et multa secundum quid. 

&

しかしもし、ある事物が端的に分割されていないならば、そのようなものは端的
に一であり、ある意味において多である。それは、基体において一であり附帯性
において多である場合のように、事物の本質の外にあるものにかんして分割され
ているとしても、事物の本質に属するものにしたがって分割されていない場合か、
あるいは、全体において一であるが部分において多である場合のように、現実態
において分割されていないが、可能態において分割されている場合である。


\\

Si vero aliquid e
converso sit indivisum secundum quid, et divisum simpliciter; utpote
quia est divisum secundum essentiam, et indivisum secundum rationem, vel
secundum principium sive causam, erit multa simpliciter, et unum
secundum quid; ut quae sunt multa numero et unum specie, vel unum
principio. 

&

逆に、あるものがある意味において分割されてなく、端的に分割されているなら
ば、たとえば、本質において分割されていて、概念において、あるいは根源や原
因において分割されていないならば、それは端的に多であり、ある意味において
一である。たとえば、数において多であるものが、種や一つの根源において一で
ある場合のように。


\\


Sic igitur ens dividitur per unum et multa, quasi per unum simpliciter,
et multa secundum quid. Nam et ipsa multitudo non contineretur sub ente,
nisi contineretur aliquo modo sub uno. Dicit enim Dionysius,
ult.~cap.~{\it de Div.~Nom.}, quod {\it non est multitudo non
participans uno, sed quae sunt multa partibus, sunt unum toto; et quae
sunt multa accidentibus, sunt unum subiecto; et quae sunt multa numero,
sunt unum specie; et quae sunt speciebus multa, sunt unum genere; et
quae sunt multa processibus, sunt unum principio.}

&


それゆえ、このようにして、有は一と多に分割される。すなわち、端的な一と、
ある意味における多によって。というのも、多性そのものもまた、なんらかの
かたちで一に含まれない限り、有のもとに含まれなかっただろうから。じっさ
い、ディオニュシウスは『神名論』最終章で、次のように言う。「一を分有し
ないような多はない。部分において多であるようなものも、全体において一で
ある。附帯性において多であるものも基体において一であり、数において多で
あるものも種において一である。種において多であるものは、類において一で
あり、発出において多であるものも、根源において一である」。


\\


{\scshape Ad tertium dicendum} quod ideo non est nugatio cum dicitur ens
unum, quia unum addit aliquid secundum rationem supra ens.

&


第三異論に対しては次のように言われるべきである。有は一であると言われる
ときにそれが無意味でないのは、一が有に、概念において何かを加えるからで
ある。


\end{longtable}


\newpage
\rhead{a.~2}

\begin{center}
 {\Large {\bf ARTICULUS SECUNDUS}}\\
 {\large UTRUM UNUM ET MULTA OPPONANTUR}\\
 {\footnotesize I {\itshape Sent.}, d.~24, q.~1, a.~3, ad 4; {\itshape
 De Pot.}, q.~3, a.~16, ad 3; q.~9, a.~7, ad 14 sqq.; X {\itshape
 Metaphys.}, lect.~4, 8.}\\
 {\Large 第二項\\一と多は対立するか}
\end{center}


\begin{longtable}{p{21em}p{21em}}

{\huge A}{\scshape d secundum sic proceditur}. Videtur quod unum et multa non
opponantur. Nullum enim oppositum praedicatur de suo opposito. Sed omnis
multitudo est quodammodo unum, ut ex praedictis patet. Ergo unum non
opponitur multitudini.

&

第二項の問題へ、議論は以下のように進められる。一と多は対立しないと思わ
れる。理由は以下の通り。どんな対立するものも、自分が対立している相手の
述語とはならない。しかし、すでに述べられたことから明らかなとおり、すべ
て多はある意味で一である。ゆえに、一は多に対立しない。

\\

2.~{\scshape Praeterea}, nullum oppositum constituitur ex suo opposito. Sed unum
constituit multitudinem. Ergo non opponitur multitudini.

&

さらに、どんな対立するものも、自分が対立している相手から構成されたりはし
ない。しかるに、一は多を構成する。ゆえに、一は多に対立しない。

\\

3.~{\scshape Praeterea}, unum uni est oppositum. Sed multo opponitur paucum. Ergo non opponitur ei unum.

&

さらに、一は一に対立する。しかるに、多には少が対立する。ゆえに、多に一が
対立するわけではない。

\\

4.~{\scshape Praeterea}, si unum opponitur multitudini, opponitur ei sicut indivisum
diviso, et sic opponetur ei ut privatio habitui. Hoc autem videtur
inconveniens, quia sequeretur quod unum sit posterius multitudine, et
definiatur per eam; cum tamen multitudo definiatur per unum. Unde erit
circulus in definitione, quod est inconveniens. Non ergo unum et multa
sunt opposita.

&

さらに、もし一が多に対立するならば、分割されていないものが分割されてい
るものに対するようなかたちで対立する。しかし、これは不都合であるように
思われる。なぜなら、もしそうだとすると、一が多よりもあとに来て、多によっ
て定義されることになるだろうから。しかし、じっさいには一によって多が定
義されるのだから。こうなると、定義において循環があることになるが、これ
は不都合である。ゆえに、一と多は対立するのではない。

\\

{\scshape Sed contra}, quorum rationes sunt oppositae, ipsa sunt opposita. Sed
ratio unius consistit in indivisibilitate, ratio vero multitudinis
divisionem continet. Ergo unum et multa sunt opposita.

&

しかし反対に、あるものどもの概念規定が対立するならば、それらのものは対
立する。しかるに、一の概念規定は不可分性において成立し、多の概念規定は
分割を包含する。ゆえに、一と多は対立する。



\\

{\scshape Respondeo dicendum} quod unum opponitur multis, sed diversimode. Nam unum
quod est principium numeri, opponitur multitudini quae est numerus, ut
mensura mensurato, unum enim habet rationem primae mensurae, et numerus
est multitudo mensurata per unum, ut patet ex {\it X Metaphys}. Unum vero quod
convertitur cum ente, opponitur multitudini per modum privationis, ut
indivisum diviso.

&

解答する。以下のように言われるべきである。一は多に対立する。しかし、対
立のしかたはさまざまである。すなわち、数の始まりである一は、数である多
に対して、尺度が、尺度によって測られるものに対するように対立する。なぜ
なら、一は第一の尺度という性格を持ち、『形而上学』10巻から明らかなよう
に、数は一によって測られた多だからである。これに対して、有と置換される
一は、分割されていないものが分割されているものに対するように、欠如とい
うかたちによって、多に対立する。




\\


{\scshape Ad primum ergo dicendum} quod nulla privatio tollit totaliter esse, quia
privatio est negatio in subiecto, secundum philosophum. Sed tamen omnis
privatio tollit aliquod esse. Et ideo in ente, ratione suae
communitatis, accidit quod privatio entis fundatur in ente, quod non
accidit in privationibus formarum specialium, ut visus vel albedinis,
vel alicuius huiusmodi. Et sicut est de ente, ita est de uno et bono,
quae convertuntur cum ente, nam privatio boni fundatur in aliquo bono,
et similiter remotio unitatis fundatur in aliquo uno. Et exinde
contingit quod multitudo est quoddam unum, et malum est quoddam bonum,
et non ens est quoddam ens. 

&

第一異論に対しては、それゆえ、次のように言われるべきである。哲学者によ
れば、欠如は基体における否定だから、どんな欠如も、存在をまったく取り除
くことはない。むしろ、すべての欠如は、なんらかの存在を取り除く。それゆ
え、有において、その共通性のために、有の欠如が有に基礎を持つということ
が起こる。これは、視覚や白性やその他そのような、特殊的な形相の欠如にお
いては起こらない。そして、有について起こることは、有と置換される一や善
にも起こる。すなわち、善の欠如がなんらかの善を基礎とし、同様に、一性の
除去が、なんらかの一に基礎を持つ。このことから、多がなんらかの一であり、
悪がなんらかの善であり、非有がなんらかの有である、ということが起こる。


\\

Non tamen oppositum praedicatur de opposito,
quia alterum horum est simpliciter, et alterum secundum quid. Quod enim
secundum quid est ens, ut in potentia, est non ens simpliciter, idest
actu, vel quod est ens simpliciter in genere substantiae, est non ens
secundum quid, quantum ad aliquod esse accidentale.

&

しかし、対立するものが、自分に対立する相手に述語されることはない。なぜ
なら、それらの一方が\kenten{端的に}、他方が、\kenten{ある意味において}
だから。たとえば、可能態における有のように、ある意味における有は、端的
に、すなわち現実態においては非有であり、あるいは、実体の類において端的
に有であるものは、ある意味において、つまりなんらかの附帯的存在にかんし
ては、非有である。

\\

 Similiter ergo quod
est bonum secundum quid, est malum simpliciter; vel e converso. Et
similiter quod est unum simpliciter, est multa secundum quid; et e
converso.

&

ゆえに、同様に、ある意味における善は、端的な悪であり、逆も又しかりである。
同様に、端的な一は、ある意味における多であり、逆も又しかりである。

\\

{\scshape Ad secundum dicendum} quod duplex est totum, quoddam homogeneum, quod
componitur ex similibus partibus; quoddam vero heterogeneum, quod
componitur ex dissimilibus partibus. In quolibet autem toto homogeneo,
totum constituitur ex partibus habentibus formam totius, sicut quaelibet
pars aquae est aqua, et talis est constitutio continui ex suis
partibus. 

&

第二異論に対しては、次のように言われるべきである。全体には2種類ある。
ある全体は同質的であり、類似した部分から複合される。また別の全体は異質
的であり、類似していない部分から複合される。どんな同質的全体においても、
全体は、全体の形相を持つ部分から構成される。たとえば、水のどの部分も水
である。

\\


In quolibet autem toto heterogeneo, quaelibet pars caret forma
totius, nulla enim pars domus est domus, nec aliqua pars hominis est
homo. Et tale totum est multitudo. 

&

これに対して、どの異質的全体においても、いかなる部分も全体の形相を欠く。
たとえば、家のどの部分も家ではないし、人間のどの部分も人間ではない。
多とはこのような全体である。

\\

Inquantum ergo pars eius non habet formam multitudinis, componitur
multitudo ex unitatibus, sicut domus ex non domibus, non quod unitates
constituant multitudinem secundum id quod habent de ratione
indivisionis, prout opponuntur multitudini; sed secundum hoc quod
habent de entitate, sicut et partes domus constituunt domum per hoc
quod sunt quaedam corpora, non per hoc quod sunt non domus.

&

ゆえに、多の部分が多の形相を持たない限りにおいて、ちょうど家が家でないも
のから構成されるように、多は一性から構成される。それは、諸々の一性が、多
と対立する限りでの、非分割性という性格を持つかぎりで、多を構成するという
のではなく、むしろ、一性が存在性を持つ限りにおいてである。ちょうど、家の
部分が、家でないことによってではなく、なんらかの物体であることによって、
家を構成するように。

\\

{\scshape Ad tertium dicendum} quod multum accipitur dupliciter. Uno modo,
absolute, et sic opponitur uni. Alio modo, secundum quod importat
excessum quendam, et sic opponitur pauco. Unde primo modo duo sunt
multa; non autem secundo.

&

第三異論に対しては次のように言われるべきである。多は二通りに理解される。
一つには無条件的にであり、その場合には一に対立する。もう一つは、ある種
の過剰を意味する限りで理解される。この意味では、少に対立する。したがっ
て、2は、第一の意味では多だが、第二の意味では多でない。

\\

{\scshape Ad quartum dicendum} quod unum opponitur privative multis,
inquantum in ratione multorum est quod sint divisa. Unde oportet quod
divisio sit prius unitate, non simpliciter, sed secundum rationem
nostrae apprehensionis. Apprehendimus enim simplicia per composita, unde
definimus punctum, cuius pars non est, vel principium lineae. 

&

第四異論に対しては、次のように言われるべきである。一は、多の概念に「分
割されている」ということが含まれている限りにおいて、多に対して欠如とし
て対立する。このことから、分割は一性に先行しなければならないが、それは、
端的にではなく、私たちの理解という点においてである。すなわち私たちは、
単純なものを複合されたものを通して捉えるのであり、それゆえ、点を「その
部分がないもの」や「線の始まり」と定義する。

\\

Sed
multitudo, etiam secundum rationem, consequenter se habet ad unum, quia
divisa non intelligimus habere rationem multitudinis, nisi per hoc quod
utrique divisorum attribuimus unitatem. Unde unum ponitur in definitione
multitudinis, non autem multitudo in definitione unius. Sed divisio
cadit in intellectu ex ipsa negatione entis. Ita quod primo cadit in
intellectu ens; secundo, quod hoc ens non est illud ens, et sic secundo
apprehendimus divisionem; tertio, unum; quarto, multitudinem.

&

しかし多は、概念においてさえ、一に後続する。なぜなら、私たちが、分割さ
れたもののそれぞれに一性を帰することによらない限り、私たちが、分割され
たものは多の性格を持っていると理解することはないからである。それゆえ、
一が多の定義の中に置かれるが、多が一の定義におかれることはない。しかし、
分割は、有の否定から、知性の中に入ってくる。そのように、第一に知性に入
るのは有であり、第二に入るのは、この有はあの有でないということである。
このようにして、私たちは第二に分割を捉え、第三に一を、第四に多をとらえ
る。

\end{longtable}



\newpage
\rhead{a.~3}

\begin{center}
 {\Large {\bf ARTICULUS TERTIUS}}\\
 {\large UTRUM DEUS SIT UNUS}\\
 {\footnotesize Infra, q.~103, a.~3; I {\itshape Sent.}, d.~2, a.~1; II,
 d.~1, q.~1, a.~1; I {\itshape SCG.}, c.~42; {\itshape De Pot.}, q.~3,
 a.~6; {\itshape Compend.~Theol.}, c.~15; {\itshape de Div.~Nom.},
 c.~13, lect.~2, 3; VIII {\itshape Physic.}, lect.~12; XII {\itshape
 Metaphys.}, lect.~10.}\\
 {\Large 第三項\\神は一人か}
\end{center}



\begin{longtable}{p{21em}p{21em}}
{\huge A}{\scshape d tertium sic proceditur}. Videtur quod Deus non sit
 unus. Dicitur enim I {\it ad Cor}.~{\scshape viii}, siquidem sunt dii multi et domini multi.

&

第三項の問題へ、議論は以下のように進められる。神は一人\footnote{男性単
 数のunusと中性単数のunumを、「一人」と「一」と訳し分けてみる。}でないと思われる。
 理由は以下の通り。『コリントの信徒への手紙一』8章では、「じっさい多く
 の神々や、多くの主が存在する」\footnote{「世の中に偶像の神などはなく、
 また、唯一の神以外にいかなる神もいないことを、私たちは知っています。
 現に多くの神々、多くの主がいると思われているように、たとえ天や地に神々
 と呼ばれるものがいても、私たちにとっては、唯一の神、父である神がおら
 れ、万物はこの神から出、私たちはこの神へ帰っていくのです」(8:5-6)}と
 述べられている。

\\

2.~{\scshape Praeterea}, unum quod est principium numeri, non potest praedicari de
Deo, cum nulla quantitas de Deo praedicetur. Similiter nec unum quod
convertitur cum ente, quia importat privationem, et omnis privatio
imperfectio est, quae Deo non competit. Non est igitur dicendum quod
Deus sit unus.

&

さらに、数の始めである一は、神についての述語とならない。なぜなら、いか
なる量も神について述語とならないからである。同様に、有と置換される一も
(神の述語とならない)。なぜなら、それは欠如を含意するが、あらゆる欠如
は不完全であり、不完全さは神に適合しないからである。ゆえに、神が一であ
ると言われるべきではない。

\\

Sed contra est quod dicitur {\it Deut}.~{\scshape vi}, {\itshape Audi,
 Israel, Dominus Deus tuus unus est}.

&

しかし反対に、『申命記』六章では「イスラエルよ、聞け。あなたの主なる神
は一人である」\footnote{「聞け、イスラエルよ。我らの神、主は唯一の主であ
る。」(6:4)}と言われている。


\\

{\scshape Respondeo dicendum} quod Deum esse unum, ex tribus
demonstratur. Primo quidem ex eius simplicitate. Manifestum est enim
quod illud unde aliquod singulare est {\itshape hoc aliquid}, nullo modo est multis
communicabile. Illud enim unde Socrates est homo, multis communicari
potest, sed id unde est hic homo, non potest communicari nisi uni
tantum. Si ergo Socrates per id esset homo, per quod est hic homo, sicut
non possunt esse plures Socrates, ita non possent esse plures
homines. 

&

解答する。以下のように言われるべきである。神が一人であることは、三つのこ
とから証明される。第一に、神の単純性からである。すなわち、ある個物が、
そこから\kenten{この}或るものであるところのもの(「\kenten{これ}である
根拠)は、けっして、多くのものに伝えられえない。たとえば、ソクラテスが
人間である根拠は多くの人に伝えられうるが、しかし\kenten{この}人間であ
る根拠は、一人の人にしか伝えられえない。ゆえに、もしソクラテスが、
\kenten{この}人間である根拠によって人間であったならば、ちょうど複数の
ソクラテスが存在しえないのと同様に、複数の人間も存在しえなかったであろ
う。

\\

Hoc autem convenit Deo, nam ipse Deus est sua natura, ut supra
ostensum est. Secundum igitur idem est Deus, et hic Deus. Impossibile
est igitur esse plures deos. 

&

しかし、このことが神には適合するのであり、前に示されたように
 \footnote{STI, q.3, a.3, c.}、神は自ら
の本性である。したがって、同一のものによって、神は神であり、かつ、
\kenten{この}神である。ゆえに、複数の神が存在することは不可能である。


\\

Secundo vero, ex infinitate eius
perfectionis. Ostensum est enim supra quod Deus comprehendit in se totam
perfectionem essendi. Si ergo essent plures dii, oporteret eos
differre. Aliquid ergo conveniret uni, quod non alteri. Et si hoc esset
privatio, non esset simpliciter perfectus, si autem hoc esset perfectio,
alteri eorum deesset. Impossibile est ergo esse plures deos. Unde et
antiqui philosophi, quasi ab ipsa coacti veritate, ponentes principium
infinitum, posuerunt unum tantum principium. 

&

第二に、神の無限の完全性からである。神が自らのうちに存在のすべての完
全性を包含することが上で示された\footnote{STI, q.4, a.2, c.}。さて、もし複数の神々がいたとすれば、
それらは異なったはずである。ゆえに、ある神には適合するが、他の神には適
合しない何かがあっただろう。もし、それが欠如だったならば、(前者の神は)
端的に完全でなかっただろう。もし、それが完全性だったならば、それらの他
方の神には、その完全性が欠けていることになっただろう。ゆえに、複数の神々
が存在することは不可能である。このことから、古代の哲学者たちは、あたか
も真理そのものに強いられるかのようにして、無限の根源を措定する場合には、
ただ一つの根源を措定したのである。

\\

Tertio, ab unitate mundi. Omnia enim quae sunt, inveniuntur esse
ordinata ad invicem, dum quaedam quibusdam deserviunt.  Quae autem
diversa sunt, in unum ordinem non convenirent, nisi ab aliquo uno
ordinarentur. Melius enim multa reducuntur in unum ordinem per unum,
quam per multa, quia per se unius unum est causa, et multa non sunt
causa unius nisi per accidens, inquantum scilicet sunt aliquo modo
unum.

&

第三に、世界の一性によってである。存在するすべてのものは、或るものが他の
ものに仕えながら、相互に秩序づけられていることが見いだされる。しかるに、
相異なるものどもは、なんらかの一つのものによって秩序づけられなかったなら
ば、一つの秩序へと一致しなかったであろう。じっさい、多くのものによってよ
りも一つのものによって、多くのものが一つの秩序へと、よりうまく導かれる。
なぜなら、一は自体的に一つのものの原因だが、多は、附帯的に、すなわち、な
んらかのかたちで一つである限りにおいてでなければ、一の原因でないからであ
る。

\\

Cum igitur illud quod est
primum, sit perfectissimum et per se, non per accidens, oportet quod
primum reducens omnia in unum ordinem, sit unum tantum. Et hoc est Deus.

&

ゆえに、第一であるものは、もっとも完全であり、附帯的でなく自体的である
ので、万物を一つの秩序へ導く第一のものは、ただ一つでなければならない。
そしてこれが神である。

\\


{\scshape Ad primum ergo dicendum} quod dicuntur dii multi secundum errorem
quorundam qui multos deos colebant, existimantes planetas et alias
stellas esse deos, vel etiam singulas partes mundi. Unde subdit, {\itshape nobis
autem unus Deus}, et cetera.

&

第一異論に対しては、それゆえ、次のように言われるべきである。多くの神々と言
われるのは、惑星や他の星々が神々だと考えて、あるいは、世界の各々の部分
が神々だと考えて、多くの神々を崇拝していた人々の誤りにしたがってのこと
である。それゆえ、「私たちには一人の神が云々」とそのあとで言われている。

\\

{\scshape Ad secundum dicendum} quod unum secundum quod est principium
numeri, non praedicatur de Deo; sed solum de his quae habent esse in
materia. Unum enim quod est principium numeri, est de genere
mathematicorum; quae habent esse in materia, sed sunt secundum rationem
a materia abstracta. Unum vero quod convertitur cum ente, est quoddam
metaphysicum, quod secundum esse non dependet a materia. Et licet in Deo
non sit aliqua privatio, tamen, secundum modum apprehensionis nostrae,
non cognoscitur a nobis nisi per modum privationis et remotionis. Et sic
nihil prohibet aliqua privative dicta de Deo praedicari; sicut quod est
incorporeus, infinitus. Et similiter de Deo dicitur quod sit unus.

&

第二異論に対しては、次のように言われるべきである。数の始めである一は、神の述
語とならない。その一が述語となるのは、ただ質料において存在を持つものども
についてである。なぜなら、数の始めの一は、数学的なものどもの類に属するが、
これらは質料において存在を持ち、概念において、質料から抽象されているか
らである。他方、有と置換される一は、ある種の形而上学的なものであり、存在
において質料に依存しない。そして、たしかに神において欠如はないが、しかし、
私たちの理解のあり方に応じて、欠如や除去というかたちによってでなければ認
識されない。だから、欠如というかたちで語られたものが神の述語となることは、
差し支えない。たとえば、神が\kenten{非}物体的であるとか、\kenten{無}限で
あるとか。同じ様にして、神について、一人であると言われる。

\end{longtable}


\newpage
\rhead{a.~4}

\begin{center}
 {\Large {\bf ARTICULUS QUARTUS}}\\
 {\large UTRUM DEUS SIT MAXIME UNUS}\\
 {\footnotesize I {\itshape Sent.}, d.~24, q.~1, a.~1; {\itshape de
 Div.~Nom.}, c.~13, lect.~3.}\\
 {\Large 第四項\\神は最大に一人か}
\end{center}



\begin{longtable}{p{21em}p{21em}}
{\huge A}{\scshape d quartum sic proceditur}. Videtur quod Deus non sit
 maxime unus. Unum enim dicitur secundum privationem divisionis. Sed
 privatio non recipit magis et minus. Ergo Deus non dicitur magis unus
 quam alia quae sunt unum.

&

第四項の問題へ、議論は以下のように進められる。神は最大に一人ではないと思
われる。理由は以下の通り。一は、分割の欠如にしたがって言われる。しかる
に、欠如は、「より多く、より少なく」ということを受け入れない。ゆえに、
神は、他の一であるものよりも一であるとは言われえない。

\\

2.~{\scshape Praeterea}, nihil videtur esse magis indivisibile quam id quod est
indivisibile actu et potentia, cuiusmodi est punctus et unitas. Sed
intantum dicitur aliquid magis unum, inquantum est indivisibile. Ergo
Deus non est magis unum quam unitas et punctus.

&

さらに、現実態においても可能態においても不可分なもの以上に不可分なもの
はないと思われる。点と一性はそのようなものに属する。しかるに、あるもの
は、不可分であればあるほど、それだけいっそう一と言われる。ゆえに、神は
一性と点以上に一であるわけではない

\\

3.~{\scshape Praeterea}, quod est per essentiam bonum, est maxime bonum, ergo quod
est per essentiam suam unum, est maxime unum. Sed omne ens est unum per
suam essentiam, ut patet per philosophum in IV {\it Metaphys}. Ergo omne ens
est maxime unum. Deus igitur non est magis unum quam alia entia.

&

さらに、本質によって善であるものは、最大限に善である。ゆえに、自らの本
質によって一であるものは、最大限に一である。しかるに、『形而上学』第四
巻の哲学者によれば、すべて存在するものは、自らの本質によって一である。
ゆえに、すべて存在するものは、最大限に一である。ゆえに、神が、他の存在
するものよりも一であるわけではない。

\\

{\bf Sed contra} est quod dicit Bernardus, quod {\itshape inter omnia quae unum
 dicuntur, arcem tenet unitas divinae Trinitatis}.

&

しかし反対に、ベルナルドゥスは「一と言われるすべてのものの中で、神の三位
一体の一性が頂点である」と述べている。

\\


{\scshape Respondeo dicendum quod}, cum unum sit ens indivisum, ad hoc quod
aliquid sit maxime unum, oportet quod sit et maxime ens et maxime
indivisum. Utrumque autem competit Deo. Est enim maxime ens, inquantum
est non habens aliquod esse determinatum per aliquam naturam cui
adveniat, sed est ipsum esse subsistens, omnibus modis
indeterminatum. Est autem maxime indivisum, inquantum neque dividitur
actu neque potentia, secundum quemcunque modum divisionis, cum sit
omnibus modis simplex, ut supra ostensum est. Unde manifestum est quod
Deus est maxime unus.

&

解答する。以下のように言われるべきである。一とは分割されていない有であ
るから、あるものが最大限に一であるためには、最大限に有でありかつ最大限
に分割されていないのでなければならない。ところで、この両方が、神に適合
する。神は、神に到来する何らかの本性によって限定されるようないかなる存
 在も持たず、あらゆる意味において限定されていない自存する存在そのもの
である限りにおいて、最大限に有である。さらにまた、上で示されたように、
神はあらゆるしかたにおいて単純だから、どんな分割のあり方によっても、現
実態においても可能態においても分割されない限りで、最大限に分割されない
ものである。したがって、神が最大限に一人であることが明らかである。



\\



Ad primum ergo dicendum{\scshape } quod, licet privatio secundum se non
recipiat magis et minus, tamen secundum quod eius oppositum recipit
magis et minus, etiam ipsa privativa dicuntur secundum magis et
minus. Secundum igitur quod aliquid est magis divisum vel divisibile,
vel minus, vel nullo modo, secundum hoc aliquid dicitur magis et minus
vel maxime unum.

&

第一異論に対しては、それゆえ、次のように言われるべきである。欠如が、自
体的に、「より多く・少なく」ということを受け取らないにしても、その反対
物が「より多く・少なく」を受け取るのに応じて、その欠如的なものどもも、
「より多く・少なく」によって語られる。ゆえに、あるものが、より多く分割
されている、あるいは、分割されうる、あるいは、より少なくそうであるとか、
あるいはけっしてそうでない、というのに応じて、あるものが、より多くより
少なく、あるいは最大限に、一であると言われる。

\\

{\scshape Ad secundum dicendum} quod punctus et unitas quae est principium
numeri, non sunt maxime entia, cum non habeant esse nisi in subiecto
aliquo. Unde neutrum eorum est maxime unum. Sicut enim subiectum non est
maxime unum, propter diversitatem accidentis et subiecti, ita nec
accidens.

&

第二異論に対しては、次のように言われるべきである。点と、数の始めである
一性は、なんらかの基体においてでなければ存在をもたないので、最大限の
有ではない。したがって、このどちらも最大限に一ではない。じっさい、附帯
性と基体の違いのために、基体が最大限に一でないように、附帯性もまた最大
限に一でない。

\\

{\scshape Ad tertium dicendum} quod, licet omne ens sit unum per suam
substantiam, non tamen se habet aequaliter substantia cuiuslibet ad
causandam unitatem, quia substantia quorundam est ex multis composita,
quorundam vero non.

&

第三異に論対しては、次のように言われるべきである。たしかにすべての有は
自らの実体によって一であるが、どんなものの実体も「一性を原因する」
ということに等しく関係するわけではない。
なぜなら、或るものどもの実体は、多くのものから複合されているが、別のも
のどもの実体はそうでないからである。


\end{longtable}

\end{document}