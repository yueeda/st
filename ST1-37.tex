\documentclass[10pt]{jsarticle} % use larger type; default would be 10pt
%\usepackage[utf8]{inputenc} % set input encoding (not needed with XeLaTeX)
%\usepackage[round,comma,authoryear]{natbib}
%\usepackage{nruby}
\usepackage{okumacro}
\usepackage{longtable}
%\usepqckage{tablefootnote}
\usepackage[polutonikogreek,english,japanese]{babel}
%\usepackage{amsmath}
\usepackage{latexsym}
\usepackage{color}
%\usepackage{tikz}

%----- header -------
\usepackage{fancyhdr}
\pagestyle{fancy}
\lhead{{\it Summa Theologiae} I, q.~37}
%--------------------


\title{{\bf PRIMA PARS}\\{\Huge Summa Theologiae}\\Sancti Thomae
Aquinatis\\{\sffamily QUEAESTIO TRIGESTIMASEPTIMA}\\DE NOMINE SPIRITUS
SANCTI QUOD EST AMOR}
\author{Japanese translation\\by Yoshinori {\scshape Ueeda}}
\date{Last modified \today}

%%%% コピペ用
%\rhead{a.~}
%\begin{center}
% {\Large {\bf }}\\
% {\large }\\
% {\footnotesize }\\
% {\Large \\}
%\end{center}
%
%\begin{longtable}{p{21em}p{21em}}
%
%&
%
%\\
%\end{longtable}
%\newpage



\begin{document}

\maketitle

\begin{center}
{\Large 聖トマス・アクィナスの神学大全の第一部\\第三十七問\\愛という聖
 霊の名について}
\end{center}

\thispagestyle{empty}
\begin{longtable}{p{21em}p{21em}}
Deinde quaeritur de nomine amoris. Et circa hoc quaeruntur duo. 

\begin{enumerate}
 \item utrum sit proprium nomen spiritus sancti.
 \item utrum pater et filius diligant se spiritu sancto.
\end{enumerate}

&

次に愛という名について問われる。これをめぐっては二つのことが問われる。

\begin{enumerate}
 \item それは聖霊に固有の名か。
 \item 父と息子は聖霊によって自らを愛するか。
\end{enumerate}

\end{longtable}



\newpage


\rhead{a.~1}
\begin{center}
{\Large {\bf ARTICULUS PRIMUS}}\\
{\large UTRUM AMOR SIT PROPRIUM NOMEN SPIRITUS SANCTI}\\
{\footnotesize I {\itshape Sent.}, d.10, a.1, ad 4; d.27, q.2, a.2, qu$^{a}$2.}\\
{\Large 第一項\\愛は聖霊に固有の名か}
\end{center}

\begin{longtable}{p{21em}p{21em}}

    {\Large A}{\scshape d primum sic proceditur}. Videtur quod amor non sit proprium nomen
spiritus sancti. Dicit enim Augustinus, XV {\itshape de Trin}., {\itshape nescio cur, sicut
sapientia dicitur et pater et filius et spiritus sanctus, et simul
omnes non tres sed una sapientia, non ita et caritas dicatur pater et
filius et spiritus sanctus, et simul omnes una caritas}. Sed nullum
nomen quod de singulis personis praedicatur et de omnibus in communi
singulariter, est nomen proprium alicuius personae. Ergo hoc nomen
amor non est proprium spiritus sancti.


&

第一項の問題へ議論は以下のように進められる。愛は聖霊に固有の名ではない
と思われる。理由は以下の通り。アウグスティヌスは『三位一体論』第15巻で
以下のように述べている。「私はどうして、ちょうど父も息子も聖霊も知恵と
言われ、またすべてが同時に三つではなく一つの知恵であるように、父も息子
も聖霊も愛であり、同時にすべてが一つの愛であるのはどうしてかということ
を知らない」。しかるに、各々のペルソナに述語され、かつすべてのペルソナ
に共通に単数で述語されるどんな名も、何らかのペルソナに固有の名ではない。
ゆえに「愛」というこの名は聖霊に固有ではない。


\\



2.~{\scshape Praeterea}, spiritus sanctus est persona subsistens. Sed amor non
significatur ut persona subsistens, sed ut actio quaedam ab amante
transiens in amatum. Ergo amor non est proprium nomen spiritus sancti.


&

さらに聖霊は自存するペルソナである。しかるに愛は自存するペルソナとして
ではなく愛する人から愛される者へ越えていくある種の作用として表示されて
いる。ゆえに愛は聖霊に固有の名ではない。

\\



3.~{\scshape Praeterea}, amor est nexus amantium, quia secundum Dionysium, {\scshape iv}
cap.~{\itshape de Div.~Nom}., est {\itshape quaedam vis unitiva}. Sed nexus est medium inter
ea quae connectit, non autem aliquid ab eis procedens. Cum igitur
spiritus sanctus procedat a patre et filio, sicut ostensum est,
videtur quod non sit amor aut nexus patris et filii.


&

さらに愛は愛する人たちの紐帯である。なぜなら『神名論』第4章のディオニュ
シウスによれば愛は「ある種の結び付ける力」だからである。しかるに紐帯は
結び付けられたものの間の媒介であり、それらから発出するものではない。ゆ
えに、聖霊はすでに示されたとおり父と息子から発出するのだから、父と息子
の愛あるいは紐帯ではないと思われる。

\\



4.~{\scshape Praeterea}, cuiuslibet amantis est aliquis amor. Sed spiritus sanctus
est amans. Ergo eius est aliquis amor. Si igitur spiritus sanctus est
amor, erit amor amoris, et spiritus a spiritu. Quod est inconveniens.


&

さらにどんな愛する人に何らかの愛がある。しかるに聖霊は愛する者である。
 ゆえに聖霊には何らかの愛がある。ゆえにもし聖霊が愛であれば、愛の愛が
 あることになり、霊からの霊があることになる。これは不都合である。

\\



{\scshape Sed contra est} quod Gregorius dicit, in homilia Pentecostes, {\itshape ipse
spiritus sanctus est amor}.


&

しかし反対にグレゴリウスは五旬節講話において「この聖霊は愛である」と述
べている。

\\



{\scshape Respondeo dicendum} quod nomen amoris in divinis sumi potest et
essentialiter et personaliter. Et secundum quod personaliter sumitur,
est proprium nomen spiritus sancti; sicut verbum est proprium nomen
filii. Ad cuius evidentiam, sciendum est quod, cum in divinis, ut
supra ostensum est, sint duae processiones, una per modum intellectus,
quae est processio verbi; alia per modum voluntatis, quae est
processio amoris, quia prima est nobis magis nota, ad singula
significanda quae in ea considerari possunt, sunt magis propria nomina
adinventa; non autem in processione voluntatis. 


&

解答する。以下のように言われるべきである。
愛という名は神において同時に本質的にまたペルソナ的に解釈されうる。そし
 てペルソナ的に解釈される限りにおいて、それは聖霊に固有の名である。そ
 れはちょうど言葉が息子に固有の名であるのと同様である。
このことを明らかにするためには以下のことが知られるべきである。
神においては、前に示されたとおり、二つの発出があり、一つは知性のあり方
 によるもので言葉の発出である。もう一つは意志のあり方によるものであり
 愛の発出である。第一の発出の方が私たちにはよく知られているので、その
 発出において考察されうる各々の事柄を表示するために固有の名がより多く
 考案されたが、意志の発出においてはそうでない。

\\



Unde et quibusdam
circumlocutionibus utimur ad significandam personam procedentem, et
relationes etiam quae accipiuntur secundum hanc processionem, et
processionis et spirationis nominibus nominantur, ut supra dictum est,
quae tamen sunt magis nomina originis quam relationis, secundum
proprietatem vocabuli. 


&

このことから、発出するペルソナを表示するために私たちはある種の回りくど
 い表現を用いるし、この発出に応じて理解される関係もまた「発出」や「霊
 発」という名によって名付けられる。これは前に述べられたとおりである。
 しかしこれらは語彙の固有性においては関係の名というよりは起源の名であ
 る。

\\


Et tamen similiter utramque processionem
considerari oportet. Sicut enim ex hoc quod aliquis rem aliquam
intelligit, provenit quaedam intellectualis conceptio rei intellectae
in intelligente, quae dicitur verbum; ita ex hoc quod aliquis rem
aliquam amat, provenit quaedam impressio, ut ita loquar, rei amatae in
affectu amantis, secundum quam amatum dicitur esse in amante, sicut et
intellectum in intelligente. Ita quod, cum aliquis seipsum intelligit
et amat, est in seipso non solum per identitatem rei, sed etiam ut
intellectum in intelligente, et amatum in amante. 


&

しかし、両方の発出が同じように考察されなければならない。すなわち、ある
 人が何らかの事物を知性認識することから、知性認識された事物のある種の
 知的な懐念が知性認識する人の中に到来し、それが言葉と言われる。ちょうどそのように、ある人
 が何らかの事物を愛することから、愛された事物の何らかのいわば印象が愛
 する人の中に到来し、その印象に即して愛されるものが愛する人の中にある
 と言われる。それはちょうど、知性認識されたものが知性認識するものの中
 にあるのと同様である。このようにして、ある人が自分自身を知性認識し、
 そして愛するとき、その人は、事物の同一性に即してだけで
 なく、知性認識されたものが知性認識するものの中にあり、愛されたものが
 愛するものの中にあるかたちで、自分自身の中にある。

\\


Sed ex parte
intellectus, sunt vocabula adinventa ad significandum respectum
intelligentis ad rem intellectam, ut patet in hoc quod dico
{\itshape intelligere}, et sunt etiam alia vocabula adinventa ad significandum
processum intellectualis conceptionis, scilicet ipsum {\itshape dicere}, et
{\itshape verbum}. Unde in divinis intelligere solum essentialiter dicitur, quia
non importat habitudinem ad verbum procedens, sed {\itshape verbum} personaliter
dicitur, quia significat id quod procedit, ipsum vero {\itshape dicere} dicitur
notionaliter, quia importat habitudinem principii verbi ad verbum
ipsum. 


&

しかし知性の側からは、知性認識するものの知性認識された事物への関係を表
示するために用語が考案され、これは私が「知性認識する」と言うことにおい
て明らかだが、さらにまた、知性的懐念の発出を表示するための用語もまた考
案され、それはすなわち、この「語る」や「言葉」がそれである。したがって、
神においては「知性認識する」だけが本質的に語られる。なぜなら、それは発
出する言葉への関係を含意しないからである。しかし「言葉」はペルソナ的に
語られ、それは発出するものを表示するからである。さらに「語る」自体は識
標的に語られ、それは言葉の根源の言葉自身への関係を含意するからである。

\\


Ex parte autem voluntatis, praeter {\itshape diligere} et {\itshape
amare}, quae important habitudinem amantis ad rem amatam, non sunt
aliqua vocabula imposita, quae importent habitudinem ipsius
impressionis vel affectionis rei amatae, quae provenit in amante ex
hoc quod amat, ad suum principium, aut e converso. Et ideo, propter
vocabulorum inopiam, huiusmodi habitudines significamus vocabulis
{\itshape amoris} et {\itshape dilectionis}; sicut si verbum
nominaremus {\itshape intelligentiam conceptam}, vel {\itshape
sapientiam genitam}.


&

他方意志の側からは、愛するものの愛された事物への関係を意味する「愛好す
る」や「愛する」以外に、愛することから愛するものの中に到来するこの印象
(impressio)自体あるいは愛された事物の感情(affectus)の、自らの起源への
関係あるいはその逆の関係を意味する用語は名付けられていない。それゆえ、
用語の不足のために、このような関係を私たちは愛や愛好という用語で表示す
る。それは、もし私たちが「懐念された知性認識」や「生み出された知恵」を
「言葉」と名付けたならば、同じことになったであろうようにである。


\\


Sic igitur, inquantum in amore vel dilectione non importatur
nisi habitudo amantis ad rem amatam, amor et diligere essentialiter
dicuntur, sicut intelligentia et intelligere. Inquantum vero his
vocabulis utimur ad exprimendam habitudinem eius rei quae procedit per
modum amoris, ad suum principium, et e converso; ita quod per amorem
intelligatur amor procedens, et per diligere intelligatur spirare
amorem procedentem, sic amor est nomen personae, et diligere vel amare
est verbum notionale, sicut dicere vel generare.


&

このようにして、愛や愛好の中に愛するものの愛された事物への関係しか含意
されていない限りにおいて、愛や愛好することは本質的に語られる。それは知
性認識や知性認識することが本質的に語られるのと同様である。他方、これら
の用語を愛のしかたで発出する事物の自らの根源への関係やその逆の関係を表
現するために私たちが用いる限りにおいて、愛によって発出する愛が、「愛好
する」によって発出する愛の霊発が理解され、その意味で愛はペルソナの名で
あり「愛好する」や「愛する」は、「語る」や「生む」と同様に識標的な言葉
となる。

\\



{\scshape Ad primum ergo dicendum} quod Augustinus loquitur de caritate, secundum
quod essentialiter sumitur in divinis, ut dictum est.


&

第一異論についてそれゆえ以下のように言われるべきである。
すでに述べられたとおり、アウグスティヌスは神において本質的に解釈される
かぎりで愛徳について語っている。

\\



{\scshape Ad secundum dicendum} quod intelligere et velle et amare, licet
significentur per modum actionum transeuntium in obiecta, sunt tamen
actiones manentes in agentibus, ut supra dictum est; ita tamen quod in
ipso agente important habitudinem quandam ad obiectum. 

&

第二異論に対しては以下のように言われるべきである。
知性認識することと意志することと愛することは、対象へと越えていく作用の
 しかたで表示されているが、前に述べられたとおり作用者の中に留まる作用
 である。しかし作用者において対象へのある種の関係を含意している。

\\



Unde amor,
etiam in nobis, est aliquid manens in amante, et verbum cordis manens
in dicente; tamen cum habitudine ad rem verbo expressam, vel
amatam. Sed in Deo, in quo nullum est accidens, plus habet, quia tam
verbum quam amor est subsistens. 

&

このことから私たちにおいてもまた、愛は愛するものの中に留まるものであり、
また心の言葉は語るものの中に留まるものでありながら、表現された事物や愛
 された事物への関係を伴っている。しかし神においては、そこに何らの附帯
 性もないので、いっそうそうである。なぜなら言葉も愛も自存するものだか
 ら。

\\



Cum ergo dicitur quod spiritus
sanctus est amor patris in filium, vel in quidquam aliud, non
significatur aliquid transiens in alium; sed solum habitudo amoris ad
rem amatam; sicut et in verbo importatur habitudo verbi ad rem verbo
expressam.


&

ゆえに、聖霊が父の息子への愛であるとか、なんであれ他のものへの愛である
 と言われるとき、他のものへと越えていく何かが表示されているのではなく、
 愛の愛された事物への関係だけが表示されている。それはちょうど、言葉に
 おいてもまた、言葉の、言葉によって表現された事物への関係が含意される
 のと同様である。


\\



{\scshape Ad tertium dicendum} quod spiritus sanctus dicitur esse nexus patris et
filii, inquantum est amor, quia, cum pater amet unica dilectione se et
filium, et e converso, importatur in spiritu sancto, prout est amor,
habitudo patris ad filium, et e converso, ut amantis ad amatum. 



&

第三異論に対しては以下のように言われるべきである。
聖霊が父と息子の紐帯であると言われるのは、聖霊が愛であるかぎりにおいて
 である。というのも、父は一つの愛と愛好によって自らと息子とを愛し、逆
 もまた同じなので、愛であるかぎりにおける聖霊において、愛するものの愛
 されるものへの関係として、父の息子へのまたその逆の関係が意味されるか
 らである。


\\



Sed ex
hoc ipso quod pater et filius se mutuo amant, oportet quod mutuus
amor, qui est spiritus sanctus, ab utroque procedat. Secundum igitur
originem, spiritus sanctus non est medius, sed tertia in Trinitate
persona. Secundum vero praedictam habitudinem, est medius nexus
duorum, ab utroque procedens.


&

しかし父と息子が相互に愛し合うこと自体からは、聖霊であ
 る相互の愛が両者から発出しなければならない。ゆえに起源において聖霊は
 媒介ではなく三性における第三のペルソナである。他方上述の関係において
 は、聖霊は両者から発出するものとして媒介であり二人の紐帯である。

\\



{\scshape Ad quartum dicendum} quod, sicut filio, licet intelligat, non tamen
sibi competit producere verbum, quia intelligere convenit ei ut verbo
procedenti; ita, licet spiritus sanctus amet, essentialiter
accipiendo, non tamen convenit ei quod spiret amorem, quod est
diligere notionaliter sumptum; quia sic diligit essentialiter ut amor
procedens, non ut a quo procedit amor.


&

第四異論に対しては以下のように言われるべきである。息子は知性認識するが、
言葉を生み出すことは適合しない。なぜなら知性認識することは、発出する言
葉として息子に適合するからである。同じように、本質的に理解するときに聖
霊は愛するが、愛を霊発すること、これは愛好することが識標的に理解された
ものであるが、は聖霊に適合しない。なぜなら本質的に愛好するのは発出する
愛としてであって、\kenten{そこ}から愛が発出するところの\kenten{それ}
としてではないからである。

\\

\end{longtable}
\newpage




\rhead{a.~2}
\begin{center}
{\Large {\bf ARTICULUS SECUNDUS}}\\
{\large UTRUM PATER ET FILIUS DILIGANT SE SPRITU SANCTO}\\
{\footnotesize I {\itshape Sent.}, d.32, q.1; {\itshape De Pot.}, q.2, a.9, ad 13.}\\
{\Large 第二項\\父と息子は聖霊によって自らを愛するか}
\end{center}

\begin{longtable}{p{21em}p{21em}}


{\scshape Ad secundum sic proceditur}. Videtur quod pater et filius
 non diligant se spiritu sancto. Augustinus enim, in VII {\itshape de
 Trin}., probat quod pater non est sapiens sapientia genita. Sed sicut
 filius est sapientia genita, ita spiritus sanctus est amor procedens,
 ut dictum est. Ergo pater et filius non diligunt se amore procedente,
 qui est spiritus sanctus.


&

第二項の問題へ議論は以下のように進められる。父と息子は聖霊によって自己
を愛するのではないと思われる。理由は以下の通り。アウグスティヌスは『三
位一体論』第7巻で父は生み出された知恵によって知者であるのではないこと
を証明している。しかるに、すでに述べられたとおり、息子が生み出された知
恵であるように、聖霊は発出する愛である。ゆえに父と息子は聖霊である発出
する愛によって自らを愛するのではない。

\\


2.~{\scshape Praeterea}, cum dicitur, pater et filius diligunt se
 spiritu sancto, hoc verbum diligere aut sumitur essentialiter, aut
 notionaliter. Sed non potest esse vera secundum quod sumitur
 essentialiter, quia pari ratione posset dici quod pater intelligit
 filio. 

&

さらに、「父と息子が聖霊によって自らを愛する」と言われるとき、この「愛
 する」という言葉は本質的に解釈されるか識標的に解釈されるかのどち
 らかである。しかし本質的に解釈されるかぎりでは真でありえない。なぜな
 ら、同じ理由で父が息子によって知性認識するといわれうるだろうから。

\\


Neque etiam secundum quod sumitur notionaliter, quia pari
 ratione posset dici quod pater et filius spirant spiritu sancto, vel
 quod pater generat filio. Ergo nullo modo haec est vera, pater et
 filius diligunt se spiritu sancto.


&

また識標的に解釈される限りにおいても真ではない。なぜなら同じ理由で父と
 息子が聖霊によって霊発するとか、父は息子によって生むとかと言われうる
 だろうから。ゆえにいかなる意味においても「父と息子が聖霊によって自ら
 を愛する」は真でない。
 

 
\\

3.~{\scshape Praeterea}, eodem amore pater diligit filium, et se, et
 nos. Sed pater non diligit se spiritu sancto. Quia nullus actus
 notionalis reflectitur super principium actus, non enim potest dici
 quod pater generat se, vel spirat se. 


&

さらに、同じ愛によって父は息子と自己を愛し私たちを愛する。しかるに父は
 聖霊によって自己を愛するのではない。なぜならどんな識標的な作用も作用
 の根源へと反射することがないからである。じっさい、父が自己を生むとか自己を霊発
 するとは言われえない。

\\


Ergo etiam non potest dici quod
 diligat se spiritu sancto, secundum quod diligere sumitur
 notionaliter. Item, amor quo diligit nos, non videtur esse spiritus
 sanctus, quia importatur respectus ad creaturam, et ita ad essentiam
 pertinet. Ergo et haec est falsa, pater diligit filium spiritu
 sancto.


&

ゆえに「愛する」が識標的に解釈されるかぎりで自らを聖霊によって愛すると
 は言われえない。さらに、それによって私たちを愛するところの愛は聖霊で
 はないと思われる。なぜなら、それは被造物への関係が含意されているので
 本質に属するからである。ゆえに「父は聖霊によって息子を愛する」もまた
 偽である。

\\

{\scshape Sed contra est} quod Augustinus dicit, VI {\itshape de Trin}., quod
 spiritus sanctus est {\itshape quo Genitus a Generante diligitur, Genitoremque
 suum diligit}.


&

しかし反対に、アウグスティヌスは『三位一体論』第4巻で「聖霊は、それに
 よって生まれたものが生むものによって愛されるところのものであり、自ら
 の生むものをも愛する」と述べている。

\\

{\scshape Respondeo dicendum} quod circa hanc quaestionem
 difficultatem affert quod, cum dicitur, pater diligit filium spiritu
 sancto, cum ablativus construatur in habitudine alicuius causae,
 videtur quod spiritus sanctus sit principium diligendi patri et
 filio; quod est omnino impossibile. 

&

解答する。以下のように言われるべきである。
この問題をめぐって困難をもたらすのは「父と息子が聖霊によって愛する」と
 言われるとき、奪格\footnote{「聖霊によって」はSpiritu Sanctoと奪格で
 表されている。}は何らかの原因の関係において用いられるので、聖霊が
 父と息子が愛することの根源であるように見えるが、それはあらゆる点で不
 可能だからである。

\\


Et ideo quidam dixerunt hanc esse
 falsam, pater et filius diligunt se spiritu sancto. Et dicunt hanc
 esse retractatam ab Augustino in suo simili, cum scilicet retractavit
 istam, pater est sapiens sapientia genita. 

&

それゆえ、ある人々は「父と息子が聖霊によって自らを愛する」というこの命
 題が偽であると言った。そしてこれはアウグスティヌスが「父は生み出され
 た知恵によって知者である」を取り下げたときに、彼によって彼の似た表現
 において撤回されたと言う。


\\



Quidam vero dicunt quod
 est propositio impropria; et est sic exponenda, pater diligit filium
 spiritu sancto, idest amore essentiali, qui appropriatur spiritui
 sancto. 

&

他方、ある人々はこれが適切な命題でなく、「父は聖霊によって」すなわち
 「本質的な愛によって息子を愛する」と説明されるべきだという。この愛は
 聖霊に固有化されるものである。

\\


Quidam vero dixerunt quod ablativus iste construitur in
 habitudine signi, ut sit sensus, spiritus sanctus est signum quod
 pater diligat filium, inquantum scilicet procedit ab eis ut
 amor. 

&

また別の人々は、この奪格はしるしの関係において構成されていて、その意味
 は「聖霊は父が息子を愛することのしるしである」すなわちそれらから愛と
 して発出するということだという。

\\



Quidam vero dixerunt quod ablativus iste construitur in
 habitudine causae formalis, quia spiritus sanctus est amor, quo
 formaliter pater et filius se invicem diligunt. 

&

また別の人々は、この奪格は形相因の関係において構成されていると言う。な
 ぜなら聖霊は、それによって父と子が形相的に互いを愛するところの愛だか
 らである。
\\


Quidam vero dixerunt
 quod construitur in habitudine effectus formalis. Et isti propinquius
 ad veritatem accesserunt. 

&

また別の人々は、形相的な結果の関係において構成されていると言った。そし
 てこの人々がより真理に近づいていた。

\\



Unde ad huius evidentiam, sciendum est quod, cum res communiter
 denominentur a suis formis, sicut album ab albedine, et homo ab
 humanitate; omne illud a quo aliquid denominatur, quantum ad hoc
 habet habitudinem formae. 

&

したがって、これを明らかにするために以下のことが知られるべきである。諸
 事物は自らの形相によって共通に名付けられる。たとえば白いものは白性に
 よって人間は人間性によって名付けられる。すべて、それによって何かが名
 付けられるところのものは、その点に関して形相の関係をもつ。

\\


Ut si dicam, {\itshape iste est indutus vestimento},
 iste ablativus construitur in habitudine causae formalis, quamvis non
 sit forma. Contingit autem aliquid denominari per id quod ab ipso
 procedit, non solum sicut agens actione; sed etiam sicut ipso termino
 actionis, qui est effectus, quando ipse effectus in intellectu
 actionis includitur. 


&

たとえば私が「彼は服で着衣している」と言うならば、この奪格は形相ではないにしても形相因の関係において構成され
 ている。ところで、或るものはそれから発出するものによって名付けられる
 ことがあるが、それは作用者が作用によって名付けられるというかたちだけでなく、結果
 が作用の理解に含まれているときには、作用の端、つまり結果によっても名付
 けられることがある。

\\


Dicimus enim quod ignis est calefaciens
 calefactione, quamvis calefactio non sit calor, qui est forma ignis,
 sed actio ab igne procedens, et dicimus quod arbor est florens
 floribus, quamvis flores non sint forma arboris, sed quidam effectus
 ab ipsa procedentes. 


&

たとえば私たちは火は熱することにによって熱するものであると言うが、熱す
ることは火の形相である熱ではなく火から発出する作用である。また私たちは
 木が花によって花咲くと言うが、花は木の形相ではなく木から発出するある
 種の結果である。

\\


Secundum hoc ergo dicendum quod, cum diligere in divinis dupliciter
 sumatur, essentialiter scilicet et notionaliter; secundum quod
 essentialiter sumitur, sic pater et filius non diligunt se spiritu
 sancto, sed essentia sua. 


&

ゆえにこれに従って以下のように言われるべきである。
愛することは神において二通りに解釈されるのであり、それは本質的にと識標
 的にである。本質的に解釈されるときには、父と息子は聖霊によって自らを
 愛するのではなく自らの本質によって愛する。

\\


Unde Augustinus dicit, in XV {\itshape de Trin}., {\itshape quis
 audet dicere patrem nec se nec filium nec spiritum sanctum diligere
 nisi per spiritum sanctum?} Et secundum hoc procedunt primae
 opiniones. 

&

このことからアウグスティヌスは『三位一体論』第15巻で「父が自らを、息子
 を、聖霊を、聖霊以外のものによって愛するなどと敢えてだれが言うだろう
 か」と述べている。そしてこの理解に沿って第一の意見は進んでいる。

\\


Secundum vero quod notionaliter sumitur, sic diligere
 nihil est aliud quam spirare amorem; sicut dicere est producere
 verbum, et florere est producere flores. Sicut ergo dicitur arbor
 florens floribus, ita dicitur pater dicens verbo vel filio, se et
 creaturam, et pater et filius dicuntur diligentes spiritu sancto, vel
 amore procedente, et se et nos.



&

他方、識標的に理解される限りにおいて、ちょうど語ることが言葉を発出させ
ることであり、花咲くことが花を生み出すことであるように、愛することは愛
 を霊発することに他ならない。それゆえ、木が花によって花咲くと言われる
 ように、父は言葉ないし息子によって自らと被造物を語り、父と息子は聖霊
 によって、あるいは発出する愛によって自らと私たちを愛すると言われる。

\\

{\scshape Ad primum ergo dicendum} quod esse {\itshape sapientem} vel {\itshape intelligentem} in
 divinis non sumitur nisi essentialiter, et ideo non potest dici quod
 {\itshape Pater sit sapiens vel intelligens filio}. Sed {\itshape diligere} sumitur non
 solum essentialiter, sed etiam notionaliter. Et secundum hoc,
 possumus dicere quod pater et filius diligunt se spiritu sancto, ut
 dictum est.


&

第一異論に対してはそれゆえ以下のように言われるべきである。
神において「知者」であるや「知性認識する者」であるということは本質的にしか解釈されない。それゆえ「父は息子によって知者あるいは知性認識する者である」とは言われえない。しかし「愛する」は本質的にだけでなく識標的にも解釈される。そしてこの意味で、私たちは父と息子が自分自身を聖霊によって愛すると言うことができる。これはすでに述べられたとおりである。

\\

{\scshape Ad secundum dicendum} quod, quando in intellectu alicuius actionis
 importatur determinatus effectus, potest denominari principium
 actionis et ab actione et ab effectu; sicut possumus dicere quod
 arbor est florens floritione, et floribus. 

&

第二異論に対しては以下のように言われるべきである。
ある作用の理解の中に限定された結果が含意されるとき、作用の根源は作用によって、また結果によって名付けられうる。
たとえば私たちは「木が開花によって花咲く」と言い「木が花によって花咲く」と言うことができる。

\\


Sed quando in actione non
 includitur determinatus effectus, tunc non potest principium actionis
 denominari ab effectu, sed solum ab actione, non enim dicimus quod
 arbor producit florem flore, sed productione floris. In hoc igitur
 quod dico spirat vel generat, importatur actus notionalis tantum. 

&

しかし作用の中に限定された結果が含まれていないときには、作用の根源が結果から名付けられることはできず、ただ作用によってのみ名付けられる。
実際、私たちは「木が花によって花を生む」とは言わず、「花の生産によって花を生む」と言う。
ゆえに私が「霊発する」や「生む」と言うことにおいては識標的な作用だけが含意される。

\\


Unde non possumus dicere quod pater spiret spiritu sancto, vel generet
 filio. Possumus autem dicere quod pater dicit verbo, tanquam persona
 procedente, et dicit dictione, tanquam actu notionali, quia dicere
 importat determinatam personam procedentem, cum dicere sit producere
 verbum. 

&

このことから、「父が聖霊によって霊発する」や「父が息子によって生む」と言うことはできない。
しかし「父が言葉によって語る」は、「言葉」が発出するペルソナという意味で、また「父が語りによって語る」は「語り」が識標的な作用という意味で言うことが可能である。
なぜなら語ることは言葉を生み出すことなので、発出する限定されたペルソナを含意するからである。


\\


Et similiter diligere, prout notionaliter sumitur, est
 producere amorem. Et ideo potest dici quod pater diligit filium
 spiritu sancto, tanquam persona procedente, et ipsa dilectione,
 tanquam actu notionali.


&

そして同様に愛することも、識標的に解されるかぎり、愛を生み出すことである。
ゆえに「父が息子を聖霊によって愛する」と言うことができ、それは聖霊が発出するペルソナという意味であり、また「愛によって愛する」と言うこともできるが、その愛は識標的な作用という意味である。


\\

{\scshape Ad tertium dicendum} quod pater non solum filium, sed etiam se et nos
 diligit spiritu sancto. Quia, ut dictum est diligere, prout
 notionaliter sumitur, non solum importat productionem divinae
 personae, sed etiam personam productam per modum amoris, qui habet
 habitudinem ad rem dilectam. 

&

第三異論に対しては以下のように言われるべきである。
父は息子だけでなく自分自身と私たちも、聖霊によって愛する。
なぜなら、すでに述べられたとおり愛することは、識標的に解されるかぎり、神のペルソナの産出だけでなく、愛のしかたで生み出されたペルソナも意味するからである。そしてそのペルソナが愛された事物への関係をもつ。



\\


Unde, sicut pater dicit se et omnem
 creaturam verbo quod genuit, inquantum verbum genitum sufficienter
 repraesentat patrem et omnem creaturam; ita diligit se et omnem
 creaturam spiritu sancto, inquantum spiritus sanctus procedit ut amor
 bonitatis primae, secundum quam pater amat se et omnem creaturam. 

&

このことから、父が、生んだ言葉によって自分自身とすべての被造物を語るが、それは生まれた言葉が父とすべての被造物を十分に表現する限りにおいてであるように、ちょうどそのように、自らとすべての被造物を聖霊によって愛するが、それは聖霊が、父がそれにおいて自分自身とすべての被造物を愛するところの、第一の善性への愛として発出する限りにおいてである。


\\


Et
 sic etiam patet quod respectus importatur ad creaturam et in verbo et
 in amore procedente, quasi secundario; inquantum scilicet veritas et
 bonitas divina est principium intelligendi et amandi omnem creaturam.


&

そしてこのようにして、被造物への関係が発出する言葉においても愛においてもいわば第二義的に含意されることもまた明らかである。すなわち、神の真理と善性がすべての被造物を知性認識試合する根源である限りにおいて、という意味である。


\\
\end{longtable}
\end{document}
