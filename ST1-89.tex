\documentclass[10pt]{jsarticle} % use larger type; default would be 10pt
%\usepackage[utf8]{inputenc} % set input encoding (not needed with XeLaTeX)
%\usepackage[round,comma,authoryear]{natbib}
%\usepackage{nruby}
\usepackage{okumacro}
\usepackage{longtable}
%\usepqckage{tablefootnote}
\usepackage[polutonikogreek,english,japanese]{babel}
%\usepackage{amsmath}
\usepackage{latexsym}
\usepackage{color}

%----- header -------
\usepackage{fancyhdr}
\pagestyle{fancy}
\lhead{{\it Summa Theologiae} I, q.~89}
%--------------------


\bibliographystyle{jplain}


\title{{\bf Prima Pars}\\{\HUGE Summae Theologiae}\\Sancti Thomae
Aquinatis\\{\sffamily Quaestio Octogesimanona}\\{\bf De Cognitione Animae Separatae}}
\author{Japanese translation\\by Yoshinori {\sc Ueeda}}
\date{Last modified \today}

%%%% コピペ用
%\rhead{a.~}
%\begin{center}
% {\Large {\bf }}\\
% {\large }\\
% {\footnotesize }\\
% {\Large \\}
%\end{center}
%
%\begin{longtable}{p{21em}p{21em}}
%
%&
%
%\\
%\end{longtable}
%\newpage

\begin{document}

\maketitle

\begin{center}
{\Large 第八十九問\\
分離した魂の認識について}
\end{center}


\begin{longtable}{p{21em}p{21em}}
Deinde considerandum est de cognitione animae separatae. Et circa hoc
quaeruntur octo.
 
&

 次に、分離した魂の認識について考察されるべきである。これについて、8つの
 ことが問われる。
\\

\begin{enumerate}
 \item Utrum anima separata a corpore possit
intelligere.
 \item Utrum intelligat substantias separatas. 
 \item Utrum intelligat omnia naturalia. 
 \item Utrum cognoscat
singularia. 
 \item Utrum habitus scientiae hic acquisitae remaneat in
anima separata. 
 \item Utrum possit uti habitu scientiae hic
acquisitae. 
 \item Utrum distantia localis impediat cognitionem animae
separatae. 
 \item Utrum animae separatae a corporibus cognoscant ea
quae hic aguntur.
\end{enumerate}

&

\begin{enumerate}
 \item 身体から分離した魂は知性認識しうるか。
 \vspace{4ex}
 \item 離在実体を知性認識しうるか。
 \item すべての自然的なものを認識しうるか。
 \item 個物を認識しうるか。
 \item この世で獲得された知識の習態が分離した魂の中に残るか。
 \item この世で獲得された知識の習態を用いることができるか。
 \item 場所の隔たりが分離した魂の認識を阻害するか。
 \item 身体から分離した魂は、この世で為されていることを認識するか。
\end{enumerate}

\end{longtable}

\newpage
\rhead{a.~1}

\begin{center}
{\bf \Large ARTICULUS PRIMUS}\\
{\large UTRUM ANIMA SEPARATA ALIQUID INTELLIGERE POSSIT}\\
{\footnotesize III {\it Sent.}, d.~31, q.~2, a.~4; IV, d.~1, q.~1, a.~1; II
 {\it SCG.},
 c.~81; {\it De Verit.}, q.~19, a.~1; {\it de Anima}, a.~15; {\it Quodl.}~III, q.~9,
 a.~1.}
{\Large 第1項\\分離した魂が何かを知性認識できるか}

\end{center}

\begin{longtable}{p{21em}p{21em}}

{\huge A}{\sc d primum sic proceditur}. Videtur quod anima separata nihil omnino
 intelligere possit. Dicit enim philosophus, in I {\it de
 anima}\footnote{Cap.~4, n.~14. S.Th.~lect.~x.}, quod
 {\it intelligere corrumpitur, interius quodam corrupto}. Sed omnia interiora
 hominis corrumpuntur per mortem. Ergo et ipsum intelligere corrumpitur.

&

第一に対しては次のように進められる。分離した魂は、まったく何も知性認識
できないと思われる。なぜなら、哲学者は『デ・アニマ』第1巻で「何かが内
側で消滅すると、知性認識も消滅する」と述べている。しかるに、死によって、
人間のすべての内側が消滅する。ゆえに、知性認識もまた消滅する。


\\

2.~{\sc Praeterea}, anima humana impeditur ab intelligendo per
ligamentum
%\footnote{l\u{\i}g\={a}mentum , i, n.~id.,
%I.a band, tie, bandage (post - Aug.)}
sensus, et perturbata imaginatione, sicut supra\footnote{q.~84, a.~7
``Utrum intellectus possit actu intelligere per species intelligibiles
quas penes se habet, non convertendo se ad phantasmata.''; a.~8
``utrum iudicium intellectus impediatur per impedimentum sensitivarum
virtutum.''} dictum est. Sed morte totaliter sensus et imaginatio
corrumpuntur, ut ex supra\footnote{q.~77, a.~8 ``Utrum omnes potentiae
remaneant in anima a corpore separata.''} dictis patet. Ergo anima
post mortem nihil intelligit.

&

さらに、上で述べられたとおり、人間の魂は、感覚の束縛や想像力の混乱によっ
て、知性認識から阻害される。しかるに、上で述べられたことから明らかなと
おり、死によって、感覚と想像力は完全に消滅する。ゆえに、魂は死後何も知
性認識しない。


\\



3.~{\sc Praeterea}, si anima separata intelligit, oportet quod per
aliquas species intelligat. Sed non intelligit per species innatas,
quia a principio est sicut {\it tabula in qua nihil est
scriptum}\footnote{Arist.~{\it de Anima}, III, cap.~4,
n.11. S.Th.~lect.~9.  }. Neque per species quas abstrahat a rebus,
quia non habet organa sensus et imaginationis, quibus mediantibus
species intelligibiles abstrahuntur a rebus. Neque etiam per species
prius abstractas, et in anima conservatas, quia sic anima pueri nihil
intelligeret post mortem. Neque etiam per species intelligibiles
divinitus influxas, haec enim cognitio non esset naturalis, de qua
nunc agitur, sed gratiae. Ergo anima separata a corpore nihil
intelligit.


&

さらに、もし分離した魂が知性認識するならば、なんらかの形象によって知性
認識しなければならない。しかるに、生得的な形象によって知性認識するので
はない。なぜなら、初めは「何も書かれていない板」のようであるから。また、
諸事物から抽象する形象によってでもない。なぜなら、感覚と想像力の器官を
媒介として、可知的形象を諸事物から抽象するが、そのような器官を持たない
からである。さらにまた、以前に抽象され、魂の内に保存された形象によって
でもない。なぜなら、もしそうなら、子供の魂は、死後、何も知性認識しない
であろうから。さらにまた、神から注がれた可知的形象によってでもない。な
ぜなら、そのような認識は、今論じられている自然的な認識ではなく、恩恵に
よる認識だから。ゆえに、身体から分離した魂は、何も知性認識しない。

\\



{\sc Sed contra est quod} philosophus dicit, in I {\it de
anima}\footnote{Cap.~1, n.~20. S.Th.~lect.~2.}, quod si non est aliqua
operationum animae propria, non contingit ipsam separari. Contingit
autem ipsam separari. Ergo habet aliquam operationem propriam; et
maxime eam quae est intelligere. Intelligit ergo sine corpore
existens.


&

しかし反対に、哲学者は『デ・アニマ』第1巻で「もし働きの中で魂に固有な
なんらかの働きがなければ、魂が分離することは起こらない」と述べている。
しかるに、魂が分離するということが生じている。ゆえに、なんらかの固有の
働きがあり、知性認識する働きが最大限にそれである。ゆえに、身体なしに存
在する魂は知性認識する。


\\


{\sc Respondeo dicendum quod} ista quaestio difficultatem habet ex hoc
quod anima, quandiu est corpori coniuncta, non potest aliquid
intelligere nisi convertendo se ad phantasmata, ut per experimentum
patet. Si autem hoc non est ex natura animae, sed per accidens hoc
convenit ei ex eo quod corpori alligatur,
%\footnote
%{al-l\u{\i}go (adl- ), \={a}vi, \={a}tum, 1, v.~a.
%I.~A..~Lit., to bind to something:
%B.~In gen., to bind, to bind up, bind round:
%II.~Trop., to bind, to hold fast, to hinder, detain; or in a moral sense, to bind, to oblige, lay under obligation
%} 
sicut Platonici posuerunt, de facili quaestio solvi posset. Nam remoto
impedimento corporis, rediret anima ad suam naturam, ut intelligeret
intelligibilia simpliciter, non convertendo se ad phantasmata, sicut
est de aliis substantiis separatis. Sed secundum hoc, non esset anima
corpori unita propter melius animae, si peius intelligeret corpori
unita quam separata; sed hoc esset solum propter melius corporis, quod
est irrationabile, cum materia sit propter formam, et non e converso.

& この問題が難しいのは、経験によって明らかなとおり、魂が、身体に結合し
ているあいだ、表象像の方を向くことによらない限り何も知性認識しえないこ
とによる。もし、プラトン派の人々が考えたように、これが魂の本性に基づく
のでなく、魂が身体に結びつけられていることによって、附帯的に、魂に適合
しているのであれば、この問題は容易に解かれえたであろう。すなわち、身体
という障害が取り除かれると、魂は自らの本性に戻り、他の分離実体について
そうであるように、表象像の方に向くことなく、端的に可知的なものを知性認
識したであろう、と。しかし、この考えによれば、もし身体に合一した魂が分
離した魂よりも悪く知性認識したならば、魂が身体に合一されたのは、魂のよ
り善い善のためではなく、むしろ、非理性的なものである身体の、より善い善
のためであっただろう。なぜなら、質料は形相のためにあるのであって、その
逆ではないからである。

\\

Si autem ponamus quod anima ex sua natura habeat ut intelligat
convertendo se ad phantasmata, cum natura animae per mortem corporis
non mutetur, videtur quod anima naturaliter nihil possit intelligere,
cum non sint ei praesto
%\footnote
%{praest\={o} (old collat.~form praest\={u} , acc.~to Curtius Valerianus in Cassiod.~p.~2289 P.: qui praestu sunt, Inscr.~Carina Via Appia, 1, p.~217.~In later time as adj.: prae-stus , a, um:
%I.“bonorum officio praestus fui,” Inscr.~Grut.~669, 4), adv.~dat.~from praestus, a sup.~form from prae, so that praesto esse alicui = to be or stand in the foremost place for or as respects one, at hand, ready, present, here; usually with esse (very freq.~and class.).
%I.~Lit.:
%II.~In partic: praesto esse or adire
%A.~To be at hand, to attend or wait upon, to serve, aid:
%B.~With esse, to present one's self in a hostile manner, to resist, oppose
%}
phantasmata ad quae convertatur. Et ideo ad hanc difficultatem
tollendam, considerandum est quod, cum nihil operetur nisi inquantum
est actu, modus operandi uniuscuiusque rei sequitur modum essendi
ipsius.

&

しかし、もしわたしたちが、魂が、自らの本性に基づいて、表象像の方を向く
ことによって知性認識すると考えるなら、魂の本性は身体の死によって変化を
受けないので、魂がその方を向く表象像が手元にないのだから、魂は本性的に
何も知性認識できないように思われる。それゆえ、この困難を克服するために、
以下のことが考察されるべきである。すなわち、なにものも、現実態にある限
りにおいてでなければ働かないので、各々の事物の働き方は、そのもの在り方
に従う。

\\

Habet autem anima alium modum essendi cum unitur corpori, et cum
fuerit a corpore separata, manente tamen eadem animae natura; non ita
quod uniri corpori sit ei accidentale, sed per rationem suae naturae
corpori unitur; sicut nec levis natura mutatur cum est in loco
proprio, quod est ei naturale, et cum est extra proprium locum, quod
est ei praeter naturam.

&

しかるに、魂は、身体に合一しているときと、身体から分離しているときとで
は、異なる在り方をするが、魂の本性は同一にとどまる。つまり、身体に合一
されることは、魂にとって附帯的でなく、その本性の性格によって、身体に合
一される。ちょうど、軽いものの本性が、それに本性的である固有の場所にあ
るときと、それにとって本性的でない、固有の場所の外にあるときとで変化し
ないように。

\\

Animae igitur secundum illum modum essendi quo corpori est unita,
competit modus intelligendi per conversionem ad phantasmata corporum,
quae in corporeis organis sunt, cum autem fuerit a corpore separata,
competit ei modus intelligendi per conversionem ad ea quae sunt
intelligibilia simpliciter, sicut et aliis substantiis separatis. Unde
modus intelligendi per conversionem ad phantasmata est animae
naturalis, sicut et corpori uniri, sed esse separatum a corpore est
praeter rationem suae naturae, et similiter intelligere sine
conversione ad phantasmata est ei praeter naturam. Et ideo ad hoc
unitur corpori, ut sit et operetur secundum naturam suam.

&ゆえに、魂には、それによって身体に合一されるところの在り方にしたがっ
て、身体器官の中にある、諸物体の表象像の方を向くことによって知性認識す
るというあり方が適合する。他方、身体から分離しているときには、他の離在
実体と同じように、端的に可知的なものの方を向くことによって知性認識する
というありかたが適合する。したがって、表象像の方を向くことによって知性
認識するというあり方は、身体に合一するということと同様に、魂にとって本
性的である。しかし、身体から分離されるということは、その本性の性格を外
れているし、同様に、表象像の方を向かないで知性認識することも、魂にとっ
て、その本性から外れている。ゆえに、魂は、自らの本性に従って存在し、ま
た働くように、身体に合一される。

\\

Sed hoc rursus habet dubitationem. Cum enim natura semper ordinetur ad
id quod melius est; est autem melior modus intelligendi per
conversionem ad intelligibilia simpliciter, quam per conversionem ad
phantasmata, debuit sic a Deo institui animae natura, ut modus
intelligendi nobilior ei esset naturalis, et non indigeret corpori
propter hoc uniri.

&

しかし、これでもまだ疑問がある。なぜなら、本性は常により善いものへ秩序
づけられるが、表象像の方を向くことによる知性認識よりも、端的に可知的な
ものの方を向くことによる知性認識の方が、より善い知性認識のあり方なので、
魂の本性は、神によって、より高貴な知性認識のあり方が、魂にとって自然本
性的であるように創られるべきだったのであり、このためには、身体に合一さ
れる必要はなかった(のではないか、という疑問である)。


\\


Considerandum est igitur quod, etsi intelligere per conversionem ad
superiora sit simpliciter nobilius quam intelligere per conversionem
ad phantasmata; tamen ille modus intelligendi, prout erat possibilis
animae, erat imperfectior. Quod sic patet. In omnibus enim substantiis
intellectualibus invenitur virtus intellectiva per influentiam divini
luminis. Quod quidem in primo principio est unum et simplex; et quanto
magis creaturae intellectuales distant a primo principio, tanto magis
dividitur illud lumen et diversificatur, sicut accidit in lineis a
centro egredientibus.

&

それゆえ、上位のものどもの方を向くことによって知性認識することは、端的
に、表象像の方を向くことによって知性認識することよりも高貴であるとして
も、前者の知性認識のしかたは、魂にとって可能であることとしては、より不
完全であったということが考察されるべきである。これは以下のようにして明
らかである。すなわち、すべての知性的実体において、神の光の流入による知
的な力が見出される。これは、第一根源においては一つで単純だが、知性的被
造物が第一根源から遠ざかるほど、かの光はより多く分割され多様化される。
ちょうど、中心から出て行く線において生じるように。


\\



Et inde est quod Deus per unam suam essentiam omnia intelligit;
superiores autem intellectualium substantiarum, etsi per plures formas
intelligant, tamen intelligunt per pauciores, et magis universales, et
virtuosiores ad comprehensionem rerum, propter efficaciam virtutis
intellectivae quae est in eis; in inferioribus autem sunt formae
plures, et minus universales, et minus efficaces ad comprehensionem
rerum, inquantum deficiunt a virtute intellectiva superiorum.


& したがって、神は、一つの自らの本質によって万物を知性認識し、また、知
性的実体の内上位の者たちは、複数の形相によって知性認識するが、それらの
中にある知性的力の強さのために、より少なく、より普遍的で、諸事物の把握
により力のある形相によって知性認識する。他方、下位の者たちのなかには、
それらが上位の者たちの知性的な力から劣っている限りにおいて、より多くの、
より普遍性の少ない、事物の把握のためにより力のない形相がある。

\\



Si ergo inferiores substantiae haberent formas in illa universalitate
in qua habent superiores, quia non sunt tantae efficaciae in
intelligendo, non acciperent per eas perfectam cognitionem de rebus,
sed in quadam communitate et confusione. Quod aliqualiter apparet in
hominibus, nam qui sunt debilioris intellectus, per universales
conceptiones magis intelligentium non accipiunt perfectam cognitionem,
nisi eis singula in speciali explicentur.


&

ゆえに、下位の[知的]諸実体が、上位の者たちが持っている普遍性において
それらの形相を持ったならば、それらは上位の者たちほどの知性認識における
強さを持たないので、それらの形相によって、諸事物についての完全な認識を
受け取らず、ある共通性と混乱のもとに、認識を受け取ったであろう。このこ
とは、ある程度、人間においても明らかである。すなわち、知性が弱い人々は、
彼らに個々の事柄が個別的に説明されない限り、より多く認識する者たちの普
遍的な概念によっては、完全な認識を受け取らない。

\\


Manifestum est autem inter substantias intellectuales, secundum
naturae ordinem, infimas esse animas humanas. Hoc autem perfectio
universi exigebat, ut diversi gradus in rebus essent. Si igitur animae
humanae sic essent institutae a Deo ut intelligerent per modum qui
competit substantiis separatis, non haberent cognitionem perfectam,
sed confusam in communi.


&

ところで、知的諸実体の中で、人間の魂が、本性の秩序において最下位である
ことは明らかである。しかるに、諸事物においてさまざまな階級が存在するこ
とを、世界の完全性は必要とする。ゆえに、もし人間の魂が、離在実体に適合
するしかたで知性認識するように、神によって創られていたならば、完全な認
識ではなく、混乱し、共通的な認識を持っていたであろう。

\\



Ad hoc ergo quod perfectam et propriam cognitionem de rebus habere
possent, sic naturaliter sunt institutae ut corporibus uniantur, et
sic ab ipsis rebus sensibilibus propriam de eis cognitionem accipiant;
sicut homines rudes
%\footnote
%{r\u{u}dis , e, adj.~cf.~crudus,
%I.unwrought, untilled, unformed, unused, rough, raw, wild (cf.~crudus):
% omnis fere materia non deformata, rudis appellatur, sicut vestimentum
% rude, non perpolitum: sic aes infectum rudusculum, Cincius
% ap.~Fest.~p.~265 M\"{u}ll. (class.; esp.~freq.~in the trop.~signif.).
%I.~Lit.:
%B.~Poet., transf., young, new (cf.~integer):
%II.~Trop., rude, unpolished, uncultivated, unskilled, awkward, clumsy, ignorant; hence (like ignarus), with gen., unacquainted with, inexperienced in, etc. (cf.~imperitus). 
%}
ad scientiam induci non possunt nisi per sensibilia exempla. Sic ergo
patet quod propter melius animae est ut corpori uniatur, et intelligat
per conversionem ad phantasmata; et tamen esse potest separata, et
alium modum intelligendi habere.


&ゆえに、諸事物について完全で固有の認識をもつことができるために、[人
間の魂は]身体に合一されるというかたちで本性的に創られ、そのようにして、
その可感的諸事物から、それらについての固有の認識を受け取る。ちょうど、
教育を受けていない人が、可感的な例によらなければ、学知へと導かれないよ
うに。ゆえに、このようにして、[人間の魂が]身体に合一されること、そし
て、表象像の方を向くことによって知性認識することは、魂の善のためである
こと、また、そうではあるが、[身体から]分離されうること、そして、他の
知性認識のしかたを持つことが明らかである。


\\



Ad primum ergo dicendum quod, si diligenter verba philosophi
discutiantur,
%\footnote
%{dis-c\u{u}t\u{\i}o , cussi, cussum, 3, v. a. quatio,
%I.to strike asunder, dash to pieces, shatter, etc.
%I. In gen.:
%II. In partic.
%A. In medic. lang., to scatter, disperse ( = digerere)
%B. Pregn., to break up, scatter, disperse, dissipate.
%1. Lit. (rarely):
%2. Trop.:
%\textcolor{red}{For the meaning to investigate, discuss (lit., to separate mentally,
% distinguish, as in disputare, discernere, etc.), which prevails in the
% post-class. derivatives: discussio, discussor, and discusse; as also in
% the Romance: discutere, discussare, discussione; discuter, discussion,
% etc., there appear to be no examples in the literary language.}
%} 
philosophus hoc dixit ex quadam suppositione prius facta, scilicet
quod intelligere sit quidam motus coniuncti, sicut et sentire, nondum
enim differentiam ostenderat inter intellectum et sensum. Vel potest
dici quod loquitur de illo modo intelligendi qui est per conversionem
ad phantasmata.

&

第一に対しては、それゆえ、次のように言われるべきである。もし哲学者の言
葉が注意深く分析されるならば、哲学者がこれを言ったのは、以前に為された
ある前提、すなわち、知性認識することが、感覚もそうであるように、結合体
のなんらかの運動であるという前提のもとでであり、つまり、彼はまだ知性と
感覚の違いを明らかにしていない。あるいは、次のように言うこともできる。
彼は、表象像の方を向くことによる知性認識のしかたについて語っている、と。

\\



De quo etiam procedit secunda ratio.


&

第二の異論もこれについて論じている。

\\



Ad tertium dicendum quod anima separata non intelligit per species
innatas; nec per species quas tunc abstrahit; nec solum per species
conservatas, ut obiectio probat, sed per species ex influentia divini
luminis participatas, quarum anima fit particeps sicut et aliae
substantiae separatae, quamvis inferiori modo. Unde tam cito cessante
conversione ad corpus, ad superiora convertitur. Nec tamen propter hoc
cognitio non est naturalis, quia Deus est auctor non solum influentiae
gratuiti luminis, sed etiam naturalis.


&

第三に対しては、次のように言われるべきである。異論が証明するとおり、分
離した魂は、生得的な形象によって知性認識しないし、そのときに抽象する形
象によっても、保存された形象のみによっても、知性認識しない。しかし、神
の光の流入によって分有された形象によっては、知性認識する。魂は、より劣っ
たしかたではあるが、他の離在実体のように、それらを分有するものとなる。
したがって、身体の方を向くことをやめるとすぐに、[魂は]上位のものの方
を向く。しかし、このことのために、認識が本性的でないことにはならない。
なぜなら、神は恩恵の光だけでなく、自然の光を流入させる者だからである。


\end{longtable}
%
%\begin{itemize}
% \item[◎] 表象像とは何か。表象像とは「現象」である、という仮説を検証す
%	  る作業中。
% \item[◎] 主文中にある、``... ad phantasmata corporum, quae in
%	   corporeis organis sunt,''という表現から、表象像はなんらかの身
%	   体器官のうちにあるとされている。どのような器官か。脳?
%
%\end{itemize}
%

\newpage
\rhead{a.~2}

\begin{center}
{\Large {\bf ARTICULUS SECUNDUS}}\\ {\large UTRUM ANIMA SEPARATA
INTELLIGAT SUBSTANTIAS SEPARATAS}\\ {\footnotesize III {\it SCG},
c.~14; {\it Qu.~de Anima}, a.~17; {\it Quodl}.~III, q.~9, a.~1.}\\
{\Large 第2項\\分離した魂は離在実体を知性認識するか}

\end{center}

\begin{longtable}{p{21em}p{21em}}

{\huge A}{\sc d secundum sic proceditur}. Videtur quod anima separata
non intelligat substantias separatas. Perfectior enim est anima
corpori coniuncta, quam a corpore separata, cum anima sit naturaliter
pars humanae naturae; quaelibet autem pars perfectior est in suo
toto. Sed anima coniuncta corpori non intelligit substantias
separatas, ut supra habitum est. Ergo multo minus cum fuerit a corpore
separata.

&

第二に対しては次のように進められる。分離した魂は離在実体を知性認識しな
いように思われる。魂は、自然本性的に、人間本性の部分だから、身体に結合
された魂の方が、身体から分離された魂よりも完全であるいかなる部分も、そ
の全体の中にある方が、より完全なのだから。しかるに、身体に結合された魂
は、上で述べられたとおり\footnote{{\it ST} I, q.~88, a.~1}、離在実体を
知性認識しない。ゆえに、身体から分離されたとき、さらにいっそう認識しな
い。

\\


2 {\sc Praeterea}, omne quod cognoscitur, vel cognoscitur per sui
praesentiam, vel per suam speciem. Sed substantiae separatae non
possunt cognosci ab anima per suam praesentiam, quia nihil illabitur
%\footnote{ill\={a}bor (inl- ), psus, 3,
%I.v.~dep.~n.~[inlabor], to fall, slip, slide, glide, or flow into; to
% fall down, sink down (rare but class.). } 
animae nisi solus Deus. Neque etiam per aliquas species quas anima ab
Angelo abstrahere possit, quia Angelus simplicior est quam anima. Ergo
nullo modo anima separata potest cognoscere substantias separatas.

&

さらに、すべて認識されるものは、自らが現在することによってか、あるいは、
自らの形象によって認識される。しかるに、離在実体は、その現在によって魂
に認識されえない。なぜなら、ただ神以外の何ものも、魂の中に入ってこない
であろうから。また、魂が天使から抽象することができるなんらかの形象によ
るのでもない。なぜなら、天使は魂よりも単純だから。ゆえに、分離した魂は、
いかなるしかたによっても、離在実体を認識できない。

\\


3 {\sc Praeterea}, quidam philosophi posuerunt in cognitione
separatarum substantiarum consistere ultimam hominis felicitatem. Si
ergo anima separata potest intelligere substantias separatas, ex sola
sua separatione consequitur felicitatem. Quod est inconveniens.

&

さらに、ある哲学者たちは\footnote{Averroes. cf. {\it ST} I, q.~88,
a.~1, c.}、離在実体を認識することにおいて、人間の究極の幸福が成立する
とした。ゆえに、もし分離した魂が離在実体を知性認識することができたら、
自らが分離することだけから、幸福が伴うことになる。これは不都合である。

\\


{\sc Sed contra est} quod animae separatae cognoscunt alias animas
separatas; sicut dives in Inferno positus vidit Lazarum et Abraham,
Luc. XVI. Ergo vident etiam et Daemones et Angelos animae separatae.

&

しかし反対に、分離した魂は、他の分離した魂を認識する。ちょうど、地獄に
置かれた金持ちが、ラザロやアブラハムを見たように(『ルカによる福音書』
16章)。ゆえに、分離した魂は、悪魔や天使たちを見る。

\\


{\sc Respondeo dicendum} quod, sicut Augustinus dicit in IX {\it de
Trin}.: {\it Mens nostra cognitionem rerum incorporearum per seipsam
accipit}, idest cognoscendo seipsam, sicut supra dictum est. Per hoc
ergo quod anima separata cognoscit seipsam, accipere possumus qualiter
cognoscit alias substantias separatas. Dictum est autem quod quandiu
anima corpori est unita, intelligit convertendo se ad phantasmata. Et
ideo nec seipsam potest intelligere nisi inquantum fit actu
intelligens per speciem a phantasmatibus abstractam, sic enim per
actum suum intelligit seipsam, ut supra dictum est.

&

答えて言わなければならない。アウグスティヌスが『三位一体論』9巻で述べ
ているように、「私たちの精神は、非物体的な事物の認識を、自分自身によっ
て獲得する」。すなわち、上で述べられたとおり、自分自身を認識することに
よって[獲得する]。ゆえに、私たちは、分離した魂が他の離在実体をどのよ
うに認識するかを、それが自己を認識するということを通して、理解すること
ができる。さて、魂が身体に合一しているあいだ、魂は表象像の方を向くこと
によって知性認識すると言われた。ゆえに、表象像から抽象された形象によっ
て、現実に知性認識するものになる限りにおいてでなければ、魂は自らを知性
認識することができない。このように、上で述べられたとおり、自らの働きを
通して自己を知性認識する。

\\


Sed cum fuerit a corpore separata, intelliget non convertendo se ad
phantasmata, sed ad ea quae sunt secundum se intelligibilia, unde
seipsam per seipsam intelliget. Est autem commune omni substantiae
separatae quod intelligat {\it id quod est supra se, et id quod est
infra se, per modum suae substantiae}, sic enim intelligitur aliquid
secundum quod est in intelligente; est autem aliquid in altero per
modum eius in quo est.

&しかし、[魂が]身体から分離したとき、表象像の方を向くことによってで
はなく、それ自体において可知的なものどもの方を向くことで知性認識するで
あろう。したがって、自己を自己によって知性認識するであろう。ところで、
すべての離在実体には、「自己の上にあるものや自己の下にあるものを、自ら
の実体のあり方を通して」知性認識することが共通する。なぜなら、あるもの
は、それが知性認識するものの中にあることにしたがって、そのように知性認
識されるからである。また、あるものは他のものの中に、そこにおいて在ると
ころのもののあり方によって在る。
\\


Modus autem substantiae animae separatae est infra modum substantiae
angelicae, sed est conformis modo aliarum animarum separatarum. Et
ideo de aliis animabus separatis perfectam cognitionem habet; de
Angelis autem imperfectam et deficientem, loquendo de cognitione
naturali animae separatae. De cognitione autem gloriae est alia ratio.

&

しかるに、分離した魂の実体のあり方は、天使の実体のあり方より下だが、し
かし、他の分離した魂とは一致したあり方にある。ゆえに、分離した魂の本性
的な認識について語るならば、他の分離した魂については、完全な認識を持つ
が、天使たちについては、不完全で欠陥のある認識しかもたない。ただし、栄
光の認識については、別の理論がある。


\\



{\sc Ad primum ergo dicendum} quod anima separata est quidem
imperfectior, si consideretur natura qua communicat cum natura
corporis, sed tamen quodammodo est liberior ad intelligendum,
inquantum per gravedinem et occupationem corporis a puritate
intelligentiae impeditur.

&

第一に対しては、それゆえ、次のように言われるべきである。分離した魂は、
身体の本性とそれによって共通するところの本性が考察されるならば、[身体
と合一した魂よりも]たしかに不完全だが、しかし[身体と合一した魂が]重
さによって、あるいは、身体をとることによって、知性認識の純粋さから妨げ
られるかぎりにおいて、ある意味で、知性認識に対してより自由である。

\\


{\sc Ad secundum dicendum} quod anima separata intelligit Angelos per
similitudines divinitus impressas. Quae tamen deficiunt a perfecta
repraesentatione eorum, propter hoc quod animae natura est inferior
quam Angeli.

&

第二に対しては次のように言われるべきである。分離した魂は、神によって刻
印された類似を通して天使を知性認識する。しかし、この類似は、それらの完
全な表現からは欠けているのであり、それは、魂の本性が天使の本性よりも下
位にあるためである。



\\


{\sc Ad tertium dicendum} quod in cognitione substantiarum separatarum
non quarumcumque, consistit ultima hominis felicitas, sed solius Dei,
qui non potest videri nisi per gratiam. In cognitione vero aliarum
substantiarum separatarum est magna felicitas, etsi non ultima, si
tamen perfecte intelligantur. Sed anima separata naturali cognitione
non perfecte eas intelligit, ut dictum est.

&

第三に対しては、次のように言われるべきである。どんな離在実体の認識にお
いても、人間の究極の幸福が成立するのではなく、ただ神の認識において成立
する。この神は、恩恵によらなければ見られえない。他方、他の離在実体の認
識においては、究極ではないが、大きな幸福がある。ただし、完全に知性認識
されるならば。しかし、すでに述べられたように、分離した魂は、自然本性的
な認識によって、それらを完全に知性認識することはできない。



\end{longtable}

\newpage
\rhead{a.~3}

\begin{center}
{\Large {\bf ARTICULUS TERTIUS}}\\ {\large UTRUM ANIMA SEPARATA OMNIA
NATURALIA COGNOSCAT}\\ {\footnotesize {\it Q.~de Anima}, a.~18.}\\
{\Large 第3項\\分離した魂はすべての自然的なものを認識するか}
\end{center}

\begin{longtable}{p{21em}p{21em}}


{\huge A}{\sc d tertium sic proceditur}. Videtur quod anima separata
omnia naturalia cognoscat. In substantiis enim separatis sunt rationes
omnium rerum naturalium. Sed animae separatae cognoscunt substantias
separatas. Ergo cognoscunt omnia naturalia.

&

第三に対しては次のように進められる。分離した魂は、すべての自然本性的な
ものを認識すると思われる。なぜなら、離在実体の中には、すべての自然的諸
事物の理念が存在する。しかるに、分離した魂は、離在実体を認識する。ゆえ
に、すべての自然本性的なものを認識する。

\\




2 {\sc Praeterea}, qui intelligit magis intelligibile, multo magis
potest intelligere minus intelligibile. Sed anima separata intelligit
substantias separatas, quae sunt maxima intelligibilium. Ergo multo
magis potest intelligere omnia naturalia, quae sunt minus
intelligibilia.

&

さらに、より可知的なものを知性認識するものは、より可知的でないものを、
優れてよく知性認識できる。しかるに、分離した魂は離在実体を知性認識する
が、それは可知的なものの中でもっとも可知的なものである。ゆえに、より可
知的でないものであるすべての自然本性的なものを知性認識できる。


\\




{\sc Sed contra}, in Daemonibus magis viget
%\footnote{v\u{\i}g\u{e}o , \={e}re, v.~n.~Sanscr.~ugras, mighty; Gr.~\textgreek{<ugi'hs}, sound; cf.~Lat.~vegeo, vigil, augeo,
%I.to be lively or vigorous; to thrive, flourish, bloom; to be in honor, esteem, repute, etc.~(class.; mostly of things, concrete and abstract; cf.~valeo).
%I.~In gen.:
%II. Of persons, to live, be alive:
%}
naturalis cognitio quam in anima separata. Sed Daemones non omnia
naturalia cognoscunt; sed multa addiscunt per longi temporis
experientiam, ut Isidorus dicit.\footnote{{\it Sententiarum} (al. {\it
de Summo Bono}) lib. I, cap.~10, al.~19. Cf. August. {\it de
Divinat.~Daemon.}, cap.~3} Ergo neque animae separatae omnia naturalia
cognoscunt.

&

しかし反対に、悪魔の中には、分離した魂の中よりも大きな自然的認識がある。
しかるに、イシドルスが言うように、悪魔たちはすべての自然的なものを認識
せず、長い時間の経験を通して多くを学ぶ。ゆえに、分離した魂も、すべての
自然的なものを認識するわけではない。


\\




2 {\sc Praeterea}, si anima statim cum est separata, omnia naturalia
cognosceret, frustra homines studerent ad rerum scientiam capessendam.
%\footnote
%{c\u{a}pesso (c\u{a}pisso , Pac.~ap.~Non.~p.~227, 1), \={\i}vi (Sall.~H.~3, 68 Dietsch; Tac.~A.~15, 49), or ii (Tac.~A.~12, 30: capessi, given by Diom.~p.~367 P., and by Charis.~ap.~Prisc.~p.~902 ib., but apparently erroneously; cf.~Struve, p.~198, and lacesso), \={\i}tum (acc.~to Prisc.~l.~l.
%I.part.~fut.~capessiturus, Tac.~A.~6, 48), 3, v.~desid.~a.~[capio].
%I.~Lit., to seize, take, or catch at eagerly, to snatch at, lay hold of (capesso = desidero capere, Prisc.~l.~l.; “rare but class.):
%B.~Of relations of place, to strive to reach a place or limit, to betake one's self to, to go to, to repair or resort to; constr.~usu.~with acc.; ante-class.~also capere se in or ad aliquem locum.
%II.~Trop.
%A.~To take hold of any thing with zeal, to take upon one's self, take in
% hand, to undertake, enter upon, engage in, execute, manage (the most
% usu.~signif.; cf.~I.~A.):
% so, capessere rem
% publicam, to undertake affairs of state, to engage in public affairs,
% administer (differing, by the idea of zealous co-operation and
% activity, from accedere ad rem publicam, which designates merely the
% entering upon a public office or duty)
%}
Hoc autem est inconveniens. Non ergo anima separata omnia naturalia
cognoscit.

&

さらに、もし魂が、分離されるやいなやすべての自然的なものを認識したとす
ると、人間が諸事物のが口を獲得しようと努力するのは無駄であることになる。
ゆえに、分離した魂がすべての自然的なものを認識するわけではない。

\\



{\sc Respondeo dicendum} quod, sicut supra dictum est, anima separata
intelligit per species quas recipit ex influentia divini luminis,
sicut et Angeli, sed tamen, quia natura animae est infra naturam
Angeli, cui iste modus cognoscendi est connaturalis, anima separata
per huiusmodi species non accipit perfectam rerum cognitionem, sed
quasi in communi et confusam. Sicut igitur se habent Angeli ad
perfectam cognitionem rerum naturalium per huiusmodi species, ita
animae separatae ad imperfectam et confusam. Angeli autem per
huiusmodi species cognoscunt cognitione perfecta omnia naturalia, quia
omnia quae Deus fecit in propriis naturis, fecit in intelligentia
angelica, ut dicit Augustinus, {\it super Gen.~ad Litt}. Unde et
animae separatae de omnibus naturalibus cognitionem habent, non certam
et propriam, sed communem et confusam.

&

答えて言わなければならない。上で述べられたとおり、分離した魂は、天使の
ように、神の光の流入から受け取る形象を通して知性認識するが、しかし、魂
の本性は、天使の本性の下位にあるので、天使はこのような認識のしかたが本
性に適っているが、分離した魂は、このような形象によって、完全な事物の認
識を受け取らず、いわば共通的に、混乱した認識を受け取る。ゆえに、天使た
ちが、このような形象によって、自然的諸事物の完全な認識へ関係するように、
分離した魂は、不完全で混乱した認識へ関係する。ところで、アウグスティヌ
スが『創世記逐語注解』で述べるように、神が固有の本性において作ったすべ
てのものを、知性的実体である天使の中にも作ったのだから、天使はこのよう
な形象をとおして、完全な認識によって、すべての自然的なものを認識する。
したがって、分離した魂も、すべての自然的なものについての認識を持つが、
しかし、共通的で混乱した認識を持つ。


\\




{\sc Ad primum ergo dicendum} quod nec ipse Angelus per suam
substantiam cognoscit omnia naturalia, sed per species quasdam, ut
supra dictum est. Et ideo non propter hoc sequitur quod anima
cognoscat omnia naturalia, quia cognoscit quoquo modo substantiam
separatam.

&

第一に対しては、それゆえ、次のように言われるべきである。天使も、自らの
実体によって、すべての自然的なものを認識するのではなく、上述のように、
なんらかの形象によってである。ゆえに、[分離した魂が]各々のかたちで離
在実体を認識するからという理由で、魂がすべての自然的なものを認識するこ
とは帰結しない。

\\




{\sc Ad secundum dicendum} quod, sicut anima separata non perfecte
intelligit substantias separatas ita nec omnia naturalia perfecte
cognoscit, sed sub quadam confusione, ut dictum est.

&

第二に対しては次のように言われるべきである。分離した魂が、離在実体を完
全には認識しないように、すべての自然的なものを完全には認識せず、上述の
ように、なんらかの混乱のもとに認識する。

\\




{\sc Ad tertium dicendum} quod Isidorus loquitur de cognitione
futurorum; quae nec Angeli nec Daemones nec animae separatae
cognoscunt, nisi vel in suis causis, vel per revelationem divinam. Nos
autem loquimur de cognitione naturalium.

&第三に対しては、次のように言われるべきである。イシドルスは、未来につ
いての認識について語っているのであり、これは、天使も悪魔も分離した魂も
認識しない。認識するとすれば、自らがもつ原因においてか、神の啓示によっ
てである。しかし私たちは、自然的なものの認識について語っている。


\\




{\sc Ad quartum dicendum} quod cognitio quae acquiritur hic per
studium, est propria et perfecta; illa autem est confusa. Unde non
sequitur quod studium addiscendi sit frustra.

&

第四に対しては、次のように言われるべきである。この世で研究によって獲得
される認識は、固有で完全であるが、かの認識は、混乱したものである。ゆえ
に、学ぶ研究が無駄だということにはならない。


\end{longtable}

\newpage
\rhead{a.~4}


\begin{center}
{\Large {\bf ARTICULUS QUARTUS}}\\ {\large UTRUM ANIMA SEPARATA
COGNOSCAT SINGULARIA}\\ {\footnotesize IV {\it Sent}., d.~50, q.~1,
a.~3; {\it De Verit}., q.~19, a.~2; {\it Qu.~de Anima}, a.~20.}\\
{\Large 第4項\\分離した魂は個物を認識するか}
\end{center}

\begin{longtable}{p{21em}p{21em}}
{\huge A}{\sc d quartum sic proceditur}. Videtur quod anima separata
non cognoscat singularia. Nulla enim potentia cognoscitiva remanet in
anima separata nisi intellectus, ut ex supra dictis patet. Sed
intellectus non est cognoscitivus singularium, ut supra habitum
est. Ergo anima separata singularia non cognoscit.


&

第四に対しては、次のように進められる。分離した魂は個物を認識しないと思
われる。なぜなら、上述のことから明らかなとおり、知性以外のどんな認識能
力も、分離した魂に残らない。しかるに、すでに論じられたとおり、知性は、
個物を認識しうるものではない。ゆえに、分離した魂は個物を認識しない。


\\




2 {\sc Praeterea}, magis est determinata cognitio qua cognoscitur
aliquid in singulari, quam illa qua cognoscitur aliquid in
universali. Sed anima separata non habet determinatam cognitionem de
speciebus rerum naturalium. Multo igitur minus cognoscit singularia.


&


さらに、あるものが、普遍的に認識される認識よりも、個別的に認識される認
識の方が、より限定された認識である。しかるに、分離した魂は、自然的諸事
物の形象について、限定された認識を持たない。ゆえに、ましてや、個物を認
識しない。

\\





3 {\sc Praeterea}, si cognoscit singularia, et non per sensum, pari
ratione omnia singularia cognosceret. Sed non cognoscit omnia
singularia. Ergo nulla cognoscit.


&


さらに、もし個物を認識し、かつ、感覚によって認識するのでないとすれば、
同じ理由で、すべての個物を認識することになっただろう。しかし、すべての
個物を認識するわけではない。ゆえに、どの個物も認識しない。
\\





{\sc Sed contra est} quod dives in inferno positus dixit: {\it Habeo
quinque fratres}, ut habetur Luc.~{\sc xvi}.


&

しかし反対に、『ルカによる福音書』16章で述べられているように、地獄に置
かれた金持ちは、「私には五人の兄弟がいる」と言った。

\\



{\sc Respondeo dicendum} quod animae separatae aliqua singularia
cognoscunt, sed non omnia, etiam quae sunt praesentia. Ad cuius
evidentiam, considerandum est quod duplex est modus intelligendi. Unus
per abstractionem a phantasmatibus, et secundum istum modum singularia
per intellectum cognosci non possunt directe, sed indirecte, sicut
supra dictum est.


&

答えて言わなければならない。分離した魂は、なんらかの個物を認識するが、
すべての個物を認識するわけではなく、現在のものですら
%\footnote{天使は、自
%らに与えられた形象によって、現在の個物をすべて認識する。\\
%cf.~({\it ST} I,
%q.~57, a.~3, arg.~3) Praeterea, Angelus non cognoscit per species
%acceptas a rebus, sed per species innatas universales. Sed species
%universales aequaliter se habent ad praesens, praeteritum et
%futurum. Ergo videtur quod Angeli indifferenter cognoscant praeterita et
%praesentia et futura.\\
%  (ad 3) Ad tertium dicendum quod, licet species
%quae sunt in intellectu Angeli, quantum est de se, aequaliter se habeant
%ad praesentia, praeterita et futura; tamen praesentia, praeterita et
%futura non aequaliter se habent ad rationes. Quia ea quae praesentia
%sunt, habent naturam per quam assimilantur speciebus quae sunt in mente
%Angeli, et sic per eas cognosci possunt. Sed quae futura sunt, nondum
%habent naturam per quam illis assimilentur, unde per eas cognosci non
%possunt.}
すべてを認識するとは限らない。これを明らかにするためには、知性認識のし
かたには二通りあることが考察されるべきである一つは、表象像からの抽象に
よるものであり、このしかたによっては、個物は、知性によって直接には認識
されえず、上述のごとく、
%\footnote{{\it ST} I, q.~86, a.~1.
%Respondeo dicendum quod singulare in rebus materialibus intellectus
%noster directe et primo cognoscere non potest. Cuius ratio est, quia
%principium singularitatis in rebus materialibus est materia
%individualis, intellectus autem noster, sicut supra dictum est,
%intelligit abstrahendo speciem intelligibilem ab huiusmodi materia. Quod
%autem a materia individuali abstrahitur, est universale. Unde
%intellectus noster directe non est cognoscitivus nisi
%universalium. Indirecte autem, et quasi per quandam reflexionem, potest
%cognoscere singulare, quia, sicut supra dictum est, etiam postquam
%species intelligibiles abstraxit, non potest secundum eas actu
%intelligere nisi convertendo se ad phantasmata, in quibus species
%intelligibiles intelligit, ut dicitur in III {\it de Anima}. Sic igitur ipsum
%universale per speciem intelligibilem directe intelligit; indirecte
%autem singularia, quorum sunt phantasmata. Et hoc modo format hanc
%propositionem, Socrates est homo.}
間接的に、認識される。


\\




Alius modus intelligendi est per influentiam specierum a Deo, et per
istum modum intellectus potest singularia cognoscere. Sicut enim ipse
Deus per suam essentiam, inquantum est causa universalium et
individualium principiorum, cognoscit omnia et universalia et
singularia, ut supra dictum est; ita substantiae separatae per
species, quae sunt quaedam participatae similitudines illius divinae
essentiae, possunt singularia cognoscere.


&

もうひとつの知性認識のしかたは、神からの形象の流入によるものであり、こ
のしかたによっても、知性は個物を認識しうる。上で述べられたように、神自
身もまた、普遍的なものと個別的根源の原因であるかぎりにおいて、普遍的な
ものも個的なものもすべて認識する。ちょうどそのように、離在実体も、かの
神の本質のある種の分有された類似である形象によって、個物を認識しうる。

\\


In hoc tamen est differentia inter Angelos et animas separatas, quia
Angeli per huiusmodi species habent perfectam et propriam cognitionem
de rebus, animae vero separatae confusam. Unde Angeli, propter
efficaciam sui intellectus per huiusmodi species non solum naturas
rerum in speciali cognoscere possunt, sed etiam singularia sub
speciebus contenta.


&

しかし、次の点が、天使と分離した魂とでは異なる。すなわち、天使たちは、
このような形象によって、事物についての完全で固有の認識を持つが、しかし
分離した魂は、混乱した認識しか持たない。したがって、天使たちは、自らの
知性の強力さのために、そのような形象によって、事物の本性を種において認
識できるだけでなく、種に含まれる個物もまた認識できる。

\\

Animae vero separatae non possunt cognoscere per huiusmodi species
nisi solum singularia illa ad quae quodammodo determinantur, vel per
praecedentem cognitionem, vel per aliquam affectionem, vel per
naturalem habitudinem, vel per divinam ordinationem, quia omne quod
recipitur in aliquo, determinatur in eo secundum modum recipientis.

&

しかし、分離した魂は、このような形象によっては、なんらかのかたちでそれ
へと限定されるような個物しか認識できない。それは、先行する認識によって
か、あるいは、なんらかの受動状態によってか、あるいは、本性的な関係によっ
てか、あるいは、神による秩序付けによってかである。何かの中に受け取られ
るものは、受け取るもののありかたに従って、そのものにおいて限定されるか
らである。

\\

{\sc Ad primum ergo dicendum} quod intellectus per viam abstractionis
non est cognoscitivus singularium. Sic autem anima separata non
intelligit, sed sicut dictum est.


&

第一に対しては、それゆえ、次のように言われるべきである。知性は、抽象の
道によっては個物を認識しえない。しかし、そのようなかたちで分離した魂が
知性認識するのではなく、上で言われたようなかたちで認識する。

\\


Ad secundum dicendum quod ad illarum rerum species vel individua
cognitio animae separatae determinatur, ad quae anima separata habet
aliquam determinatam habitudinem, sicut dictum est.

&

第二に対しては次のように言われるべきである。すでに言われたように、分離
した魂の認識は、その分離した魂がなんらかの限定された関係をもつような事
物の形象や個体に限定される。


\\



Ad tertium dicendum quod anima separata non se habet aequaliter ad
omnia singularia, sed ad quaedam habet aliquam habitudinem quam non
habet ad alia. Et ideo non est aequalis ratio ut omnia singularia
cognoscat.

&

第三に対しては、次のように言われるべきである。分離した魂は、すべての個
物に等しく関係するのではなく、あるものに対して、他のものに対しては持た
ないような関係を持つ。ゆえに、すべての個物を認識することになるような、
等しい根拠はない。





\end{longtable}

\newpage
\rhead{a.~5}

\begin{center}
{\Large {\bf ARTICULUS QUINTUS}}\\ {\large UTRUM HABITUS SCIENTIAE HIC
ACQUISITAE\\ REMANEAT IN ANIMA SEPARATA}\\ {\footnotesize {\it ST}
I-IIae, q.~67, a.~2; {\it IV Sent.}, d.~1, q.~1, a.~2; {\it
Quodl.}~XII, q.~9, a.~1; {\it I Cor.}, c.~13, l.~3.}\\ {\Large 第5項\\
この世で獲得された知の習態が分離した魂の中に残るか}
\end{center}

\begin{longtable}{p{21em}p{21em}}



{\huge A}{\sc d quintum sic proceditur}. Videtur quod habitus
scientiae hic acquisitae non remaneat in anima separata. Dicit enim
apostolus, I {\it ad Cor}.~{\sc xiii}: {\it Scientia destruetur}.


&

第5に対しては、次のように進められる。この世で獲得された知の習態が、分
離した魂のなかに残ることはない、と思われる。なぜなら、使徒は「知が破壊
されるだろう」\footnote{「愛は決して滅びない。預言は廃れ、異言はやみ、
知識は廃れよう、」(13:8)}と述べている。

\\




2. {\sc Praeterea}, quidam minus boni in hoc mundo scientia pollent,
%\footnote {poll\u{e}o (polet, pollet: I.“quia nondum geminabant
%antiqui consonantes,” Fest.~``p.~205 M\"{u}ll.), \={e}re,
%v.~n.~potis-valeo.~ I.~Lit.~ A.~In gen., to be strong, powerful, or
%potent, to be able, to prevail, avail (class.; cf.: valeo, possum): (β).
%With abl.: }
aliis magis bonis carentibus scientia. Si ergo habitus scientiae
permaneret etiam post mortem in anima, sequeretur quod aliqui minus
boni etiam in futuro statu essent potiores aliquibus magis bonis. Quod
videtur inconveniens.


&

さらに、この世で、より少なく善である人々が知において優れ、より善である
人たちが知を欠いている。ゆえに、知の習態が死後も魂の中に残ったならば、
より少なく善である者が、将来も、より善である人々よりも力のある状態にあ
ることになるだろう。これは不都合であるように思われる。

\\


3. {\sc Praeterea}, animae separatae habebunt scientiam per
influentiam divini luminis. Si igitur scientia hic acquisita in anima
separata remaneat, sequetur quod duae erunt formae unius speciei in
eodem subiecto. Quod est impossibile.


&

さらに、分離した魂は、神の光の流入によって、知をもつだろう。ゆえに、も
し、この世で獲得された知が分離した魂のなかに残るなら、同じ基体のなかに、
一つの種に属する二つの形相があることになる。これは不可能である。

\\


4. {\sc Praeterea}, philosophus dicit, in libro {\it Praedicament}.,
quod {\it habitus est qualitas difficile mobilis; sed ab aegritudine,
vel ab aliquo huiusmodi, quandoque corrumpitur scientia}. Sed nulla
est ita fortis immutatio in hac vita, sicut immutatio quae est per
mortem. Ergo videtur quod habitus scientiae per mortem corrumpatur.


&

さらに、哲学者は、『カテゴリー論』という書物の中で、次のように述べてい
る。「習態は、動かされることが困難な性質である。しかし、病気やその他そ
のようなものによって、知が消滅することが時にある」。しかし、この世での
変化の中で、死による変化ほど強力なものはない。ゆえに、知の習態は、死に
よって消滅すると思われる。

\\


{\sc Sed contra est} quod Hieronymus dicit, in Epistola {\it ad
Paulinum}: {\it discamus in terris, quorum scientia nobis perseveret
in caelo}.

&

しかし反対に、ヒエロニュムスは、パウリナ宛『書簡』で「その知が天におい
て私たちに保存されるようなことを、地において学びましょう」と述べている。

\\


{\sc Respondeo dicendum} quod quidam posuerunt habitum scientiae non
esse in ipso intellectu, sed in viribus sensitivis, scilicet
imaginativa, cogitativa et memorativa; et quod species intelligibiles
non conservantur in intellectu possibili. Et si haec opinio vera
esset, sequeretur quod, destructo corpore, totaliter habitus scientiae
hic acquisitae destrueretur.

&


答えて言わなければならない。ある人々\footnote{cf. Avicenna {\it de
Anima}, part.~5, cap.~6.}は、知の習態が、知性自身の中でなく、感覚能力
の中に、すなわち、想像力、思考力、記憶力の中にあるとし、また、可知的形
象は、可能知性の中に保存されないとした。もしこの意見が真であったならば、
身体が破壊されると、この世で獲得された知の習態は、全面的に破壊されると
いうことが帰結しただろう。


\\



Sed quia scientia est in intellectu, qui est {\it locus specierum}, ut
dicitur in III de {\it Anima}; oportet quod habitus scientiae hic
acquisitae partim sit in praedictis viribus sensitivis, et partim in
ipso intellectu. Et hoc potest considerari ex ipsis actibus ex quibus
habitus scientiae acquiritur, nam {\it habitus sunt similes actibus ex
quibus acquiruntur}, ut dicitur in II {\it Ethic}. Actus autem
intellectus ex quibus in praesenti vita scientia acquiritur, sunt per
conversionem intellectus ad phantasmata, quae sunt in praedictis
viribus sensitivis.


&

しかし、『デ・アニマ』第3巻で述べられるように、知性は「形象の場所」で
あるから、知は知性においてある。したがって、この世で獲得された知の習態
は、一部は前述の感覚能力の中にあり、一部は知性自身の中にある。そしてこ
のことは、『ニコマコス倫理学』第2巻で「習態は、そこから習態が獲得され
るところの働きに似ている」ので、そこから知の習態が獲得されるところの働
きに基づいて考察されうる。ところで、現在の生において、知は知性の働きか
ら獲得されるが、その知性の働きは、知性が表象像の方へ向くことによってあ
る。そして、この表象像は、前述の感覚能力の中にある。

\\


Unde per tales actus et ipsi intellectui possibili acquiritur facultas
quaedam ad considerandum per species susceptas; et in praedictis
inferioribus viribus acquiritur quaedam habilitas
%\footnote{h\u{a}b\u{\i}l\u{\i}tas, \={a}tis, f.~habilis,
%I.aptitude, ability:}
ut facilius per conversionem ad ipsas intellectus possit
intelligibilia speculari. Sed sicut actus intellectus principaliter
quidem et formaliter est in ipso intellectu, materialiter autem et
dispositive in inferioribus viribus, idem etiam dicendum est de
habitu.

& したがって、このような働きによって、可能知性には、受け取った形象によっ
て考察する働きが獲得され、また、前述の下位の能力においては、知性がそれ
らの方を向くことによって、より容易に、可知的なものが観照されうるように
なる、一種の適性(habilitas)が獲得される。しかし、ちょうど、知性の働き
は、主要かつ形相的に、知性そのものの中にあるように、質料的、態勢的に、
下位の能力の中にある。同じことが、習態についても言われるべきである。



\\


Quantum ergo ad id quod aliquis praesentis scientiae habet in
inferioribus viribus, non remanebit in anima separata, sed quantum ad
id quod habet in ipso intellectu, necesse est ut remaneat. Quia, ut
dicitur in libro de {\it Longitudine et Brevitate Vitae}, dupliciter
corrumpitur aliqua forma, uno modo, per se, quando corrumpitur a suo
contrario, ut calidum a frigido; alio modo, per accidens, scilicet per
corruptionem subiecti.

&

ゆえに、ある人が下位の能力の中にもっている、現在の知の部分に関しては、
分離された魂の中に残らないだろうが、知性自身の中に持っているものに関し
ては、それが残ることが必然である。なぜなら、『生の長さと短さについて』
という本\footnote{アリストテレスの著作。}で言われているように、ある形
相は二通りのしかたで消滅する。一つは、自体的にであり、たとえば、熱いも
のが冷たいものによって消滅するように、自らに反対のものによって消滅する
場合である。もう一つは、附帯的に、すなわち、基体が消滅することによって
である。
%\footnote
%{形相の消滅について語る興味深いテキスト。ここでは消滅のしかたが二つ挙げ
% られているが、三通りに区別されることもある。cf. {\it SCG} II, c.~79, n.~10. ``Item. Nulla forma corrumpitur nisi vel ex actione
% contrarii, vel per corruptionem sui subiecti, vel per defectum suae
% causae: per actionem quidem contrarii, sicut calor destruitur per
% actionem frigidi; per corruptionem autem sui subiecti, sicut, destructo
% oculo destruitur vis visiva; per defectum autem causae, sicut lumen
% aeris deficit deficiente solis praesentia, quae erat ipsius causa. Sed
% anima humana non potest corrumpi per actionem contrarii: non est enim
% ei aliquid contrarium; cum per intellectum possibilem ipsa sit
% cognoscitiva et receptiva omnium contrariorum.''
%}

\\



Manifestum est autem quod per corruptionem subiecti, scientia quae est
in intellectu humano, corrumpi non potest, cum intellectus sit
incorruptibilis, ut supra ostensum est. Similiter etiam nec per
contrarium corrumpi possunt species intelligibiles quae sunt in
intellectu possibili, quia intentioni intelligibili nihil est
contrarium; et praecipue quantum ad simplicem intelligentiam, qua
intelligitur quod quid est.

&

しかし、基体が消滅することによって、人間知性の中にある知が滅びることが
ありえないのは明らかである。なぜなら、知性は、上で示された通り、不滅だ
からである。同様に、可能知性の中にある可知的形象が、反対のものによって
消滅することもありえない。なぜなら、可知的なインテンチオには、何も反対
のものがないからである。とくに、単純な知性認識、すなわち、それによって
何であるかが知性認識されるところのものは、反対物をもたない。

\\

Sed quantum ad operationem qua intellectus componit et dividit, vel
etiam ratiocinatur, sic invenitur contrarietas in intellectu, secundum
quod falsum in propositione vel in argumentatione est contrarium
vero. Et hoc modo interdum scientia corrumpitur per contrarium, dum
scilicet aliquis per falsam argumentationem abducitur a scientia
veritatis.


&

しかし、知性が複合分割する働きにかんして、あるいは、推論の働きにかんし
ては、命題や推論の中の偽が真に反対するかぎりで、知性において反対性が見
出される。そしてこのしかたで、知が、反対のものによって消滅することがあ
る。すなわち、ある人が、偽の推論によって、真理に属する知から離されてい
くように。


\\


Et ideo philosophus, in libro praedicto, ponit duos modos quibus
scientia per se corrumpitur, scilicet oblivionem, ex parte
memorativae, et deceptionem, ex parte argumentationis falsae. Sed hoc
non habet locum in anima separata. Unde dicendum est quod habitus
scientiae, secundum quod est in intellectu manet in anima separata.

&

それゆえ、哲学者は、前述の書物で、知が自体的に消滅する二つのかたちを述
べている。すなわち、記憶力の側からは忘却であり、偽の推論の側からは、欺
瞞である。しかし、このようなことは、分離した魂にはない。したがって、知
の習態は、それが知性においてある限り、分離した魂の中に留まる。

\\


{\sc Ad primum ergo dicendum} quod apostolus non loquitur ibi de
scientia quantum ad habitum, sed quantum ad cognitionis actum. Unde ad
huius probationem inducit, {\it Nunc cognosco ex parte}.


&

第一に対して、それゆえ、次のように言われるべきである。使徒はそこで、習
態にかんする限りで知について述べているのでなく、認識の働きにかんして述
べている。したがって、この証明のために「いま私は部分に基づいて認識する」
ということを引いている。

\\


{\sc Ad secundum dicendum} quod, sicut secundum staturam
%\footnote{st\u{a}t\={u}ra , ae, f.~status, from sto; prop.~a
%standing upright, an upright posture; hence, I.height or size of the
%body, stature.}
corporis aliquis minus bonus erit maior aliquo magis bono; ita nihil
prohibet aliquem minus bonum habere aliquem scientiae habitum in
futuro, quem non habet aliquis magis bonus. Sed tamen hoc quasi
nullius momenti est in comparatione ad alias praerogativas quas
meliores habebunt.


&

第二に対しては、次のように言われるべきである。ちょうど、より少なく善で
ある人が、より多く善である人よりも、背が高いことがあるように、より少な
く善である人が、より多く善である人がもたない知の習態を将来においてもつ
としても、何ら差し支えはない。しかし、これは、より善い人たちがもつであ
ろう他の特権に比べれば、いわばまったく無意味なことである。

\\


Ad tertium dicendum quod utraque scientia non est unius rationis. Unde
nullum inconveniens sequitur.


&

第三に対しては、次のように言われるべきである。両方の知が、ひとつの
ratioに属するわけではない。ゆえに、なんの不都合も生じない。

\\


{\sc Ad quartum dicendum} quod ratio illa procedit de corruptione
scientiae quantum ad id quod habet ex parte sensitivarum virium.

&

かの異論は、感覚能力の側からもつ知の消滅について論じている。



\end{longtable}


\newpage
\rhead{a.~6}

\begin{center}
{\Large {\bf ARTICULUS SEXTUS}}\\ {\large UTRUM ACTUS SCIENTIAE HIC
ACQUISITAE MANEAT IN ANIMA SEPARATA}\\ {\footnotesize III {\it Sent.},
d.~31, q.~2, a.~4; IV, d.~1, q.~1, a.~2.}\\ {\Large 第6項\\この世で獲
得された知の働きが分離した魂の中に残るか}
\end{center}

\begin{longtable}{p{21em}p{21em}}

{\huge A}{\sc d sextum sic proceditur}. Videtur quod actus scientiae
hic acquisitae non maneat in anima separata. Dicit enim philosophus,
in I {\it de Anima}, quod corrupto corpore, anima {\it neque
reminiscitur neque amat}. Sed considerare ea quae prius aliquis novit,
est reminisci. Ergo anima separata non potest habere actum scientiae
quam hic acquisivit.

&

第6に対しては、次のように進められる。この世で獲得された知の働きは、分
離した魂の中に残らないと思われる。なぜなら、哲学者は『デ・アニマ』第1
巻\footnote{Cap.~4, n.~14. S.Th.~lect.~10.}で、身体が消滅すると、魂は
「思い出すことも、愛することもない」と述べている。しかるに、思い出すこ
ととは、ある人が、以前に知ったことを考察することである。ゆえに、分離し
た魂は、この世で獲得した知の働きをもつことができないと思われる。

\\

2. {\sc Praeterea}, species intelligibiles non erunt potentiores in
anima separata quam sint in anima corpori unita. Sed per species
intelligibiles non possumus modo intelligere, nisi convertendo nos
super phantasmata, sicut supra habitum est. Ergo nec anima separata
hoc poterit. Et ita nullo modo per species intelligibiles hic
acquisitas anima separata intelligere poterit.

&

さらに、可知的形象は、身体に合一された魂の内にあるときよりも、分離した
魂の内にあるときの方が、力が落ちるであろう。しかるに、上で述べられた通
り、いま、私たちは、私たちを表象像へと向けることなしに、知性認識するこ
とができない。ゆえに、分離した魂も、これ[私たちを表象像へと向けること
なしに知性認識すること]ができないであろう。したがって、この世で獲得さ
れた可知的形象によって、分離した魂が知性認識することは、どのようなかた
ちによっても不可能であろう。

\\

3. {\sc Praeterea}, philosophus dicit, in II {\it Ethic}., quod {\it
habitus similes actus reddunt actibus per quos acquiruntur}. Sed
habitus scientiae hic acquiritur per actus intellectus convertentis se
supra phantasmata. Ergo non potest alios actus reddere. Sed tales
actus non competunt animae separatae. Ergo anima separata non habebit
aliquem actum scientiae hic acquisitae.

& さらに、哲学者は『ニコマコス倫理学』第2巻で、「習態は、それによって
その習態が獲得されたところの働きに似た働きを生み出す」と述べる。しかる
に、知の習態は、この世で、自らを表象像へ向ける知性の働きによって獲得さ
れる。ゆえに、別の働きを与えることができない。しかし、このような働きは、
分離した魂には適合しない。ゆえに、分離した魂は、この世で獲得されたいか
なる知の働きももたないであろう。



\\

{\it Sed contra est quod} Luc.~{\sc xvi}, dicitur ad divitem in
Inferno positum, {\it Recordare quia recepisti bona in vita tua}.

&しかし反対に、『ルカによる福音書』16章で、地獄に置かれた金持ちに対し、
「あなたの人生で、善いものどもを受け取ったことを思い出せ」
\footnote{「子よ、思い出してみるがよい。おまえは生きているあいだに良い
ものをもらっていたが、」(16:25)}と言われている。

\\


{\sc Respondeo dicendum} quod in actu est duo considerare, scilicet
speciem actus, et modum ipsius. Et species quidem actus consideratur
ex obiecto in quod actus cognoscitivae virtutis dirigitur
%\footnote{d\={\i}-r\u{\i}go or d\={e}r\u{\i}go (the latter
% form preferred by Roby, L.~G.~2, p.~387; cf.~Rib.~Proleg.~ad
% Verg.~p.~401 sq.; so Liv.~21, 19, 1; 21, 47, 8; 22, 28 Weissenb.;
% id.~22, 47, 2 Drak.; Lach.~ad Lucr.~4, 609; Tac.~A.~6, 40 Ritter;
% acc.~to Brambach, s.~v., the two forms are different words, de-rigo
% meaning I.to give a particular direction to; di-rigo, to arrange in
% distinct lines, set or move different ways; cf.~describo and
% discribo.~But the distinction is not observed in the MSS.~and
% edd.~generally), rexi, rectum, 3 (perf.~sync.~direxti, Verg.~A.~6, 57),
% v.~a.~dis-rego, to lay straight, set in a straight line, to arrange,
% draw up (class.; cf.: guberno, collineo, teneo).}
per speciem, quae est obiecti similitudo, sed modus actus pensatur
%\footnote
%{penso , \={a}vi, \={a}tum, 1, v.~freq.~a.~pendo,
%I.to weigh or weigh out carefully (not freq.~till after the Aug.~per.; not in Cic., for in Off.~2, 19, 68, conpensandum is the correct reading.~Neither is it found in Plaut., Ter., Lucr., or Cæs.; syn.: pendo, expendo).
%I.~Lit.:
%II.~Transf.
%A.~To counterbalance with any thing, to compensate, recompense, repay, make good, requite; for the usual compensare:
%B.~To pay, repay, punish with any thing:
%2.~To purchase with any thing:
%C.~To exchange for any thing:
%D.~To allay, quench:
%E.~To weigh, ponder, examine, consider: 
%}
ex virtute agentis. Sicut quod aliquis videat lapidem, contingit ex
specie lapidis quae est in oculo, sed quod acute videat, contingit ex
virtute visiva oculi.


&

答えて言わなければならない。働きにおいて、二つのことを考察することがで
きる。すなわち、働きの形象と、働きのあり方である。働きの形象は、認識能
力の働きが、形象によってそちらへと向けられている対象から考察される。こ
の形象は、対象の類似である。他方、働きのあり方は、働くものの力から測ら
れる。たとえば、ある人が石を見ることは、目の中にある石の形象から生じる
が、しかし、鋭くそれを見ることは、目の見る力から生じる。


\\



Cum igitur species intelligibiles maneant in anima separata, sicut
dictum est; status autem animae separatae non sit idem sicut modo est,
sequitur quod secundum species intelligibiles hic acquisitas, anima
separata intelligere possit quae prius intellexit; non tamen eodem
modo, scilicet per conversionem ad phantasmata, sed per modum
convenientem animae separatae. Et ita manet quidem in anima separata
actus scientiae hic acquisitae, sed non secundum eundem modum.


&

ゆえに、すでに述べられた通り、可知的形象が分離した魂の中に残り、そして、
分離した魂の状態は、今のそれと同じではないのだから、分離した魂は、この
世で獲得された可知的形象において、以前に知性認識したことを知性認識する
ことができるということ、しかし、同じしかたによって、すなわち、表象像の
方を向くことによってではなく、分離した魂に適合するしかたによってである、
ということが帰結する。そして、このようにして、分離した魂の中には、この
世で獲得された知の働きが残るが、しかし、[今と]同じあり方においてでは
ない。

\\



{\sc Ad primum ergo dicendum} quod philosophus loquitur de
reminiscentia, secundum quod memoria pertinet ad partem sensitivam,
non autem secundum quod memoria est quodammodo in intellectu, ut
dictum est.

&

第一に対しては、それゆえ、次のように言われるべきである。哲学者は、記憶
が感覚能力の側に属するかぎりで、記憶力について述べているのであり、すで
に述べられた通り、記憶があるかたちで知性の中にあるかぎりで述べているの
ではない。


\\


{\sc Ad secundum dicendum} quod diversus modus intelligendi non
provenit ex diversa virtute specierum, sed ex diverso statu animae
intelligentis.

&

知性認識のさまざまなあり方は、形象のさまざまな力から由来するのではなく、
知性認識する者の魂のさまざまな状態に由来する。

\\



{\sc Ad tertium dicendum} quod actus per quos acquiritur habitus, sunt
similes actibus quos habitus causant, quantum ad speciem actus, non
autem quantum ad modum agendi. Nam operari iusta, sed non iuste, idest
delectabiliter, causat habitum iustitiae politicae, per quem
delectabiliter operamur.

&

第三に対しては、次のように言われるべきである。ある習態が、ある働きによっ
て得られたとき、その働きが、その習態が原因となって生じた働きに似ている
のは、働きの形象においてであり、働きの様態[働き方]においてではない。
たとえば、正しいことを行いながら、正しく行わない、すなわち、楽しみのた
めに行うことは、政治的な正義の習態の原因となる。私たちは、この習態によっ
て、楽しみのために働く。\footnote{Cf.~Arist.~{\it Ethic.}, lib.~5,
c.~8, n.~1. (s.Th.~lect. 13); {\it Magn.Moral}., lib.~1, cap.~34,
n.~23.}


\end{longtable}

\newpage
\rhead{a.~7}

\begin{center}
{\Large {\bf ARTICULUS SEPTIMUS}}\\ {\large UTRUM DISTANTIA LOCALIS
IMPEDIAT COGNITIONEM ANIMAE SEPARATAE}\\ {\footnotesize IV {\it
Sent.}, d.~1, q.~1, a.~4.}\\ {\Large 第7項\\場所の隔たりが分離した魂の
認識を妨げるか}
\end{center}

\begin{longtable}{p{21em}p{21em}}

{\huge A}{\sc d septimum sic proceditur}. Videtur quod distantia
localis impediat cognitionem animae separatae. Dicit enim Augustinus,
in libro {\it de Cura pro Mortuis agenda}, quod {\it animae mortuorum
ibi sunt, ubi ea quae hic fiunt scire non possunt}. Sciunt autem ea
quae apud eos aguntur. Ergo distantia localis impedit cognitionem
animae separatae.

&

第7に対しては、次のように進められる。場所の隔たりは、分離した魂の認識
を妨げると思われる。なぜなら、アウグスティヌスは、『死者たちに払われる
べき配慮について』で、「死者たちの魂は、この世で為されることを知ること
ができないところにいる」と述べている。しかるに、彼らは、彼らのもとで為
されることを知る。ゆえに、場所の隔たりは、分離した魂の認識を妨げると思
われる。

\\



2. {\sc Praeterea}, Augustinus dicit, in libro {\it de Divinatione
Daemonum}, quod {\it daemones, propter celeritatem motus, aliqua nobis
ignota denuntiant}. Sed agilitas motus ad hoc nihil faceret, si
distantia localis cognitionem Daemonis non impediret. Multo igitur
magis distantia localis impedit cognitionem animae separatae, quae est
inferior secundum naturam quam Daemon.

&

さらに、アウグスティヌスは『悪魔の予知について』で「悪魔たちは、運動の
速さのために、私たちが知らないことを知らせる」と述べている。しかし、も
し、場所の隔たりが悪魔の認識を妨げるのでなかったら、運動の速さは、この
ため[=私たちが知らないことを知らせるため]に、何もなさなかったであろ
う。ゆえに、ましてや、場所の隔たりは分離した魂の認識を妨げる。それは、
本性において、悪魔よりも下なのだから。

\\



3. {\sc Praeterea}, sicut distat aliquis secundum locum, ita secundum
tempus sed distantia temporis impedit cognitionem animae separatae,
non enim cognoscunt futura. Ergo videtur quod etiam distantia secundum
locum animae separatae cognitionem impediat.

&

さらに、ある人は、場所において隔たるように、時間において隔たる。しかる
に、時間の隔たりは、分離した魂の認識を妨げる。なぜなら、未来を認識しな
いから。ゆえに、場所における隔たりもまた、分離した魂の認識を妨げると思
われる。

\\



{\sc Sed contra est} quod dicitur Luc.~{\sc xvi}, quod {\it dives cum
esset in tormentis, elevans oculos suos, vidit Abraham a longe}. Ergo
distantia localis non impedit animae separatae cognitionem.

&

しかし反対に、『ルカによる福音書』16章で、「金持ちたちが、責め苦の中に
いたとき、自らの目を上げ、アブラハムを遠くから見た」\footnote{「そして、
金持ちは陰府でさいなまれながら目を上げると、宴席でアブラハムとそのすぐ
そばにいるラザロとが、はるかかなたに見えた。」(16:23)}と言われている。
ゆえに、場所の隔たりは、分離した魂の認識を妨げない。

\\



{\sc Respondeo dicendum} quod quidam posuerunt quod anima separata
cognosceret singularia abstrahendo a sensibilibus. Quod si esset
verum, posset dici quod distantia localis impediret animae separatae
cognitionem, requireretur enim quod vel sensibilia agerent in animam
separatam, vel anima separata in sensibilia; et quantum ad utrumque,
requireretur distantia determinata.


&


答えて言わなければならない。ある人々は、分離した魂が、個物を、可感的な
ものどもから抽象することによって認識すると考えた。かりにこれが真であっ
たとしたら、場所の隔たりが分離した魂の認識を妨げると言われることができ
たであろう。なぜなら、その場合には、可感的なものが分離した魂へ働きかけ
ることか、または、分離した魂が可感的なものへ働きかけることが要求された
だろうから。


\\
Sed praedicta positio est impossibilis, quia abstractio specierum a
sensibilibus fit mediantibus sensibus et aliis potentiis sensitivis,
quae in anima separata actu non manent. Intelligit autem anima
separata singularia per influxum specierum ex divino lumine, quod
quidem lumen aequaliter se habet ad propinquum et distans. Unde
distantia localis nullo modo impedit animae separatae cognitionem.

&


しかし、このような立場は不可能である。なぜなら、可感的なものから形象を
抽象することは、感覚その他の感覚能力の媒介によって為されるが、これら
[=感覚能力]は、分離した魂の中に、現実態において、留まっていないから
である。しかし、分離した魂は、神の光から形象が流入することによって、個
物を認識するが、この光は、近くにあるものと離れているものとに、等しく関
係する。従って、場所の隔たりは、どのようなかたちでも、分離した魂の認識
を妨げない。

\\


{\sc Ad primum ergo dicendum} quod Augustinus non dicit quod propter
hoc quod ibi sunt animae mortuorum, ea quae hic sunt videre non
possunt, ut localis distantia huius ignorantiae causa esse credatur,
sed hoc potest propter aliquid aliud contingere, ut infra dicetur.


&

第一に対しては、それゆえ、次のように言われるべきである。アウグスティヌ
スは、場所の隔たりがこの無知の原因であると信じられるようなかたちで、死
者たちの魂が、「そこに在る」ということのために、ここに在るものどもを見
ることができないと言っているのではない。そうではなく、こういったことは、
下で\footnote{次の項。}述べられるように、何か別のことのために生じうる

\\



{\sc Ad secundum dicendum} quod Augustinus ibi loquitur secundum
opinionem illam qua aliqui posuerunt quod Daemones habent corpora
naturaliter sibi unita, secundum quam positionem, etiam potentias
sensitivas habere possunt, ad quarum cognitionem requiritur
determinata distantia. Et hanc opinionem etiam in eodem libro
Augustinus expresse tangit, licet eam magis recitando quam asserendo
tangere videatur, ut patet per ea quae dicit XXI libro {\it de
Civ.~Dei}.

&


第二に対しては、次のように言われるべきである。アウグスティヌスはそこで、
悪魔たちが、自らに合一した身体を本性的にもっていると考えた人たちの意見
に沿って論じている。この考えによれば、悪魔らは感覚能力も持つことができ
るし、それらの能力が認識を行うには、一定の距離が必要とされる。そして、
この意見を、アウグスティヌスは同じ書物の中でもはっきりと扱っている。た
だし、『神の国』第21巻で語ることを通して明らかな通り、主張するようにで
はなく、引用するように、扱っている。


\\


{\sc Ad tertium dicendum} quod futura, quae distant secundum tempus,
non sunt entia in actu. Unde in seipsis non sunt cognoscibilia, quia
sicut deficit aliquid ab entitate, ita deficit a cognoscibilitate. Sed
ea quae sunt distantia secundum locum, sunt entia in actu, et secundum
se cognoscibilia. Unde non est eadem ratio de distantia locali, et de
distantia temporis.


&

第三に対しては、次のように言われるべきである。未来は、時間において隔たっ
ているが、現実態における有ではない。したがって、それ自身において認識さ
れうるものではない。なぜなら、有であることentitasから欠ける程度に、認
識されうるものであることからも欠けるからである。これに対して、場所にお
いて隔たっているものは、現実態における有であり、それ自体において認識さ
れうるものである。したがって、場所的な隔たりと、時間的な隔たりについて、
同じ議論はできない。



\end{longtable}

\newpage
\rhead{a.~8}

\begin{center}
{\Large {\bf ARTICULUS OCTAVUS}}\\ {\large UTRUM ANIMAE SEPARATAE
COGNOSCAT EA QUAE HIC AGUNTUR}\\ {\footnotesize {\it ST} II-II, q.~83,
a.~4, ad 2; IV {\it Sent.}, d.~45, q.~3, a.~1, ad 1, 2; d.~1, q.~1,
a.~4, ad 1; {\it De Verit.}, q.~8, a.~2, ad 12; q.~9, a.~6, ad 5;
Qu.~{\it de Anima.}, a.~20, ad 3.}\\ {\Large 第8項\\分離した魂は、この
世で為されていることを認識するか}
\end{center}

\begin{longtable}{p{21em}p{21em}}


{\huge A}{\sc d octavum sic proceditur}. Videtur quod animae separatae
cognoscant ea quae hic aguntur. Nisi enim ea cognoscerent, de eis
curam non haberent. Sed habent curam de his quae hic aguntur; secundum
illud Luc. {\sc xvi}: {\it Habeo quinque fratres, ut testificetur
illis, ne et ipsi veniant in hunc locum tormentorum}. Ergo animae
separatae cognoscunt ea quae hic aguntur.

&


第8に対しては、次のように進められる。分離した魂は、この世で為されるこ
とを認識すると思われる。なぜなら、もしそういった事柄を認識しなかったら、
それらについて心配することもなかっただろう。しかし、『ルカによる福音書』
第16章「私には5人の兄弟がいます。彼らもまた、この責め苦の場所に来るこ
とがないように、証言してください」\footnote{「わたしには兄弟が五人いま
す。あの者たちまで、こんな苦しい場所に来ることのないように、よく言い聞
かせてください。」(16:28)}によれば、この世で為されている事柄について心
配をしている。ゆえに、分離した魂は、この世で為されていることを認識する。

\\




2.{\sc Praeterea}, frequenter mortui vivis apparent, vel dormientibus
vel vigilantibus, et eos admonent de iis quae hic aguntur; sicut
Samuel apparuit Sauli, ut habetur I {\it Reg}.~{\sc xxviii}. Sed hoc
non esset si ea quae hic sunt non cognoscerent. Ergo ea quae hic
aguntur cognoscunt.

&

さらに、たとえばサムエルがサウルに現れたと『サムエル記上』28章で述べら
れているように、しばしば、死者たちは、眠っているときであれ目覚めている
ときであれ、生きている人々に現れ、この世で為されていることについて、彼
らに忠告する。しかし、もし、この世にある事柄を認識していなかったならば、
このようなことはなかったであろう。ゆえに、この世で為されていることを認
識している。

\\




3. {\sc Praeterea}, animae separatae cognoscunt ea quae apud eas
aguntur. Si ergo quae apud nos aguntur non cognoscerent, impediretur
earum cognitio per localem distantiam. Quod supra negatum est.

&


さらに、分離した魂は、彼らのもとで為されていることを認識している。ゆえ
に、もし、私たちのもとで為されていることを認識しなかったならば、彼らの
認識は、場所の隔たりによって妨げらることになるが、これは、上で否定され
た。

\\




{\sc Sed contra est} quod dicitur {\it Iob} {\sc xiv} : {\it Sive
fuerint filii eius nobiles, sive ignobiles, non intelliget}.

&


しかし反対に、『ヨブ記』14章「彼の息子たちが高貴になるか、高貴でなくな
るか、彼は知性認識しないだろう」\footnote{「その子らが名誉を得ても、彼
は知ることなく、彼らが不幸になっても、もう悟らない。」(14:21)}と言われ
ている。


\\


{\sc Respondeo dicendum} quod, secundum naturalem cognitionem, de qua
nunc hic agitur, animae mortuorum nesciunt quae hic aguntur. Et huius
ratio ex dictis accipi potest. Quia anima separata cognoscit
singularia per hoc quod quodammodo determinata est ad illa, vel per
vestigium alicuius praecedentis cognitionis seu affectionis, vel per
ordinationem divinam. Animae autem mortuorum, secundum ordinationem
divinam, et secundum modum essendi, segregatae sunt a conversatione
viventium, et coniunctae conversationi spiritualium substantiarum quae
sunt a corpore separatae. Unde ea quae apud nos aguntur ignorant.

&


答えて言わなければならない。今ここで論じられている、自然本性的な認識に
おいては、死者たちの魂が、この世で為されることを認識することはない。こ
の理由は、すでに述べられたことから理解できる。分離した魂は、なんらかの
先行する認識や受動の痕跡によって、または、神の命令(秩序付け)によって、
あるかたちでそれへと限定されることをとおして、個物を認識する。しかるに、
死者の魂は、神の命令においても、存在のあり方においても、生きている者た
ちとの会話か切り離され、また、身体から分離している霊的実体との会話に結
びつけられている。したがって、私たちのもとで為されていることを、彼らは
知らない。


\\



Et hanc rationem assignat Gregorius in XII {\it Moralium}, dicens:
{\it Mmortui vita in carne viventium post eos, qualiter disponatur,
nesciunt, quia vita spiritus longe est a vita carnis; et sicut
corporea atque incorporea diversa sunt genere, ita sunt distincta
cognitione}. Et hoc etiam Augustinus videtur tangere in libro {\it de
Cura pro Mortuis agenda}, dicens {\it quod animae mortuorum rebus
viventium non intersunt}.
%\footnote
%{inter-sum , f\u{u}i, esse (interf\u{u}t\={u}rus, Cic.~Div.~in Caecil.~11, 35;
%I.“in tmesi: interque esse desiderat pugnis,” Arn.~7, 255), v.~n., to be between, lie between (class.; syn.~interjaceo).
%I.~In gen.
%A.~Of space:
%B.~Of time:
%II.~Transf.
%A.~To be apart; with abl.~of distance (syn.~disto):
%B.~To be different, to differ:
%C.~To be present at, take part in, attend; constr.~absol., with dat.~or in and abl.
%D.~\textcolor{red}{To interest, be of interest to one} (very rare as pers.~verb):
%}


&


そして、グレゴリウスは、『道徳論』第12巻で、次のように述べてその理由を
指定している。「死者たちは、彼らのあとに肉において生きる者たちの生がど
のような状態にあるかを知らない。なぜなら、霊の生は、肉の生から遠く隔た
るからである。ちょうど、物体的なものと非物体的なものとが類において異な
るように、認識においても区別される」。また、このことは、アウグスティヌ
スが『死者たちのために払われるべき配慮』という書物で「死者たちの魂は、
生きている者たちの事柄に関心がない」と述べるときに触れているように思わ
れる。

\\


Sed quantum ad animas beatorum, videtur esse differentia inter
Gregorium et Augustinum. Nam Gregorius ibidem subdit: {\it Quod tamen
de animabus sanctis sentiendum non est, quia quae intus omnipotentis
Dei claritatem vident, nullo modo credendum est quod sit foris aliquid
quod ignorent}. -- Augustinus vero, in libro {\it de Cura pro Mortuis
agenda}, expresse dicit quod {\it nesciunt mortui, etiam sancti, quid
agant vivi et eorum filii}, ut habetur in Glossa, super illud, {\it
Abraham nescivit nos}, Isaiae {\sc lxiii}.

& しかし、至福者たちの魂にかんする限り、グレゴリウスとアウグスティヌス
には違いがあるように思われる。すなわち、グレゴリウスは、同じ箇所で次の
ように続けている。「しかしこのことが、聖なる魂について考えられるべきで
はない。なぜなら、それら[=聖なる魂]は全能である神の光を見ているのだ
から、彼らの知らないなにかが外にあるとは、決して信じられるべきでないか
ら」。これに対して、アウグスティヌスは、『死者たちのために払われるべき
配慮について』という書物の中で、かの『イザヤ書』63章の「アブラハムは私
たちを知らない」\footnote{「あなたはわたしたちの父です。アブラハムがわ
たしたちを見知らず、イスラエルがわたしたちを認めなくても、主よ、あなた
はわたしたちの父です。」(63:16)}を注解して「死者たちは、聖者であっても、
生きている者たちとその子孫たちが何をしているかを知らない」とはっきり述
べている。



\\


Quod quidem confirmat per hoc quod a matre sua non visitabatur, nec in
tristitiis consolabatur, sicut quando vivebat; nec est probabile ut
sit facta vita feliciore crudelior. Et per hoc quod dominus promisit
Iosiae regi quod prius moreretur ne videret mala quae erant populo
superventura, ut habetur IV {\it Reg}.~{\sc xxii}. Sed Augustinus hoc
dubitando dicit, unde praemittit, {\it ut volet, accipiat quisque quod
dicam}. Gregorius autem assertive, quod patet per hoc quod dicit, {\it
nullo modo credendum est}.


&

彼[=アウグスティヌス]はこれを、次のことによって、確かなものとしてい
る。すなわち、[母親が亡くなったあと]彼の母親が訪ねてきたことがないし、
生きていたときのように、悲しいときに慰めてくれたこともない。しかし、よ
り幸福な生において、[死んだ母が]より無慈悲になったとは考えられない。
そして、このことによって、『列王記下』22章に書かれているように、主がヨ
シアの王に、民に訪れるであろう害悪を見ないように、先に死ぬことを約束し
た。\footnote{「それゆえ、見よ、わたしはあなたを先祖の数に加える。あな
たは安らかに息を引き取って墓に葬られるであろう。私がこの所にくだす災い
のどれも、その目で見ることがない。」(22:20)} しかし、アウグスティヌス
は、これを疑いながら述べており、それゆえ「もしそうしたいなら、なんであ
れ私が言うことを受け取りなさい」、と前で言う。これに対してグレゴリウス
は、「決して信じられるべきでない」と述べることから明らかなとおり、はっ
きりと主張している。
\\




Magis tamen videtur, secundum sententiam Gregorii, quod animae
sanctorum Deum videntes, omnia praesentia quae hic aguntur
cognoscant. Sunt enim Angelis aequales, de quibus etiam Augustinus
asserit quod ea quae apud vivos aguntur non ignorant. Sed quia
sanctorum animae sunt perfectissime iustitiae divinae coniunctae, nec
tristantur, nec rebus viventium se ingerunt, nisi secundum quod
iustitiae divinae dispositio exigit.


&

しかし、グレゴリウスの議論に従って、神を見る聖者たちの魂が、この世で為
されているすべての現在のことを認識するという方が、より確からしいと思わ
れる。なぜなら、彼らは天使に等しいのであり、天使たちについてはアウグス
ティヌスもまた、生きている者たちのもとで為されることについて知らないこ
とがないと述べている。しかし、聖者たちの魂は、この上なく完全に、神の正
義に結びつけられているので、神の正義の状態が必要とするのでない限り、悲
しむことも、生きている者たちの事柄に介入することもない。

\\




{\sc Ad primum ergo dicendum} quod animae mortuorum possunt habere
curam de rebus viventium, etiam si ignorent eorum statum; sicut nos
curam habemus de mortuis, eis suffragia impendendo, quamvis eorum
statum ignoremus. Possunt etiam facta viventium non per seipsos
cognoscere, sed vel per animas eorum qui hinc ad eos accedunt; vel per
Angelos seu Daemones; vel etiam {\it spiritu Dei revelante}, sicut
Augustinus in eodem libro dicit.

&

第一に対しては、それゆえ、次のように言われるべきである。死者たちの魂は、
生きている者たちの事柄について、かりにそれらの状態を知らなくても、心配
することができる。ちょうど、わたしたちが、死者たちの状態を知らなくても、
彼らに取りなしの祈りを捧げることで、死者たちについて心配するように。ま
た、生きている者たちが為したことを、それ自体によってではなく、この世か
ら彼らのもとへとやって来る者たちの魂を通して、あるいは、天使や悪魔によっ
て、あるいは、アウグスティヌスが同じ書物で述べるように「神の啓示する霊
によって」、認識できる。

\\




{\sc Ad secundum dicendum} quod hoc quod mortui viventibus apparent
qualitercumque, vel contingit per specialem Dei dispensationem, ut
animae mortuorum rebus viventium intersint, et est inter divina
miracula computandum.  Vel huiusmodi apparitiones fiunt per
operationes Angelorum bonorum vel malorum, etiam ignorantibus mortuis,
sicut etiam vivi ignorantes aliis viventibus apparent in somnis, ut
Augustinus dicit in libro praedicto.  Unde et de Samuele dici potest
quod ipse apparuit per revelationem divinam; secundum hoc quod dicitur
{\it Eccli}.~{\sc xlvi}, quod {\it dormivit, et notum fecit regi finem
vitae suae}.  Vel illa apparitio fuit procurata per Daemones, si tamen
Ecclesiastici auctoritas non recipiatur, propter hoc quod inter
canonicas Scripturas apud Hebraeos non habetur.

&


第二に対しては、次のように言われるべきである。どのようなかたちであれ、
死者たちが生きている者たちに現れることや、あるいは、神の特別な計らいに
よって、死者たちの魂が、生きている者たちの事柄に現れることは、神の奇跡
に数えられるべきである。あるいは、アウグスティヌスが上述の本で述べてい
るように、ちょうど、生きている人でも、知らないうちに、夢の中で、他の生
きている人に現れるように、このような出現は、良い、あるいは悪い天使の働
きによって、死者たちが知らないままに生じている。したがって、かの『知恵
の書』46節「彼は眠った。そして、自分の人生の目的を王に知らせた」によれ
ば、サミュエルについても、彼が、神の啓示によって出現したと言われること
ができる。あるいは、かの出現は、悪魔たちによって行われた[とも解釈でき
る]。かりに、ヘブライ人たちのもとで、聖典の中に入っていないため、『知
恵の書』の権威が受け取られないとしても。


\\




{\sc Ad tertium dicendum} quod ignorantia huiusmodi non contingit ex
locali distantia, sed propter causam praedictam.


&

第三に対しては、次のように言われるべきである。そのような無知は、場所の
隔たりによってではなく、上述の原因のために生じる。

\end{longtable}
\end{document}