\documentclass[10pt]{jsarticle} % use larger type; default would be 10pt
%\usepackage[utf8]{inputenc} % set input encoding (not needed with XeLaTeX)
%\usepackage[round,comma,authoryear]{natbib}
%\usepackage{nruby}
\usepackage{okumacro}
\usepackage{longtable}
%\usepqckage{tablefootnote}
\usepackage[polutonikogreek,english,japanese]{babel}
%\usepackage{amsmath}
\usepackage{latexsym}
\usepackage{color}

%----- header -------
\usepackage{fancyhdr}
\pagestyle{fancy}
\lhead{{\it Summa Theologiae} I, q.~89}
%--------------------


\bibliographystyle{jplain}


\title{{\bf Prima Pars}\\{\HUGE Summae Theologiae}\\Sancti Thomae
Aquinatis\\{\sffamily Quaestio Octogesimanona}\\{\bf De Cognitione Animae Separatae}}
\author{Japanese translation\\by Yoshinori {\sc Ueeda}}
\date{Last modified \today}

%%%% コピペ用
%\rhead{a.~}
%\begin{center}
% {\Large {\bf }}\\
% {\large }\\
% {\footnotesize }\\
% {\Large \\}
%\end{center}
%
%\begin{longtable}{p{21em}p{21em}}
%
%&
%
%\\
%\end{longtable}
%\newpage

\begin{document}

\maketitle

\begin{center}
{\Large 第八十九問\\
分離した魂の認識について}
\end{center}


\begin{longtable}{p{21em}p{21em}}
Deinde considerandum est de cognitione animae separatae. Et circa hoc
quaeruntur octo. 
&
次に、分離した魂の認識について考察されるべきである。これについて、8つの
 ことが問われる。
\\

\begin{enumerate}
 \item Utrum anima separata a corpore possit
intelligere.
 \item Utrum intelligat substantias separatas. 
 \item Utrum intelligat omnia naturalia. 
 \item Utrum cognoscat
singularia. 
 \item Utrum habitus scientiae hic acquisitae remaneat in
anima separata. 
 \item Utrum possit uti habitu scientiae hic
acquisitae. 
 \item Utrum distantia localis impediat cognitionem animae
separatae. 
 \item Utrum animae separatae a corporibus cognoscant ea
quae hic aguntur.
\end{enumerate}

&

\begin{enumerate}
 \item 身体から分離した魂は知性認識しうるか。
 \vspace{4ex}
 \item 離在実体を知性認識しうるか。
 \item すべての自然的なものを認識しうるか。
 \item 個物を認識しうるか。
 \item この世で獲得された知識の習態が分離した魂の中に残るか。
 \item この世で獲得された知識の習態を用いることができるか。
 \item 場所の隔たりが分離した魂の認識を阻害するか。
 \item 身体から分離した魂は、この世で為されていることを認識するか。
\end{enumerate}

\end{longtable}

\newpage
\rhead{a.~1}

\begin{center}
{\bf \Large ARTICULUS PRIMUS}\\
{\large UTRUM ANIMA SEPARATA ALIQUID INTELLIGERE POSSIT}\\
{\footnotesize III {\it Sent.}, d.~31, q.~2, a.~4; IV, d.~1, q.~1, a.~1; II
 {\it SCG.},
 c.~81; {\it De Verit.}, q.~19, a.~1; {\it de Anima}, a.~15; {\it Quodl.}~III, q.~9,
 a.~1.}
{\Large 第1項\\分離した魂が何かを知性認識できるか}

\end{center}

\begin{longtable}{p{21em}p{21em}}

{\huge A}{\sc d primum sic proceditur}. Videtur quod anima separata nihil omnino
 intelligere possit. Dicit enim philosophus, in I {\it de
 anima}\footnote{Cap.~4, n.~14. S.Th.~lect.~x.}, quod
 {\it intelligere corrumpitur, interius quodam corrupto}. Sed omnia interiora
 hominis corrumpuntur per mortem. Ergo et ipsum intelligere corrumpitur.

&
第一に対しては次のように進められる。
分離した魂は、まったく何も知性認識できないと思われる。
なぜなら、哲学者は『デ・アニマ』第1巻で「何かが内側で消滅すると、知性認
 識も消滅する」と述べている。しかるに、死によって、人間のすべての内側が
 消滅する。ゆえに、知性認識もまた消滅する。


\\

2.~{\sc Praeterea}, anima humana impeditur ab
 intelligendo per ligamentum
%\footnote{l\u{\i}g\={a}mentum , i, n.~id.,
%I.a band, tie, bandage (post - Aug.)}
 sensus, et perturbata imaginatione, sicut
 supra\footnote{q.~84, a.~7 ``Utrum intellectus possit actu intelligere
 per species intelligibiles quas penes se habet, non convertendo se ad
 phantasmata.''; a.~8 ``utrum iudicium intellectus impediatur per
 impedimentum sensitivarum virtutum.''} dictum est. Sed morte totaliter sensus et imaginatio
 corrumpuntur, ut ex supra\footnote{q.~77, a.~8 ``Utrum omnes potentiae
 remaneant in anima a corpore separata.''} dictis patet. Ergo anima post mortem nihil
 intelligit.

&
さらに、上で述べられたとおり、人間の魂は、感覚の束縛や想像力の混乱によっ
 て、知性認識から阻害される。
しかるに、上で述べられたことから明らかなとおり、死によって、感覚と想像力
 は完全に消滅する。ゆえに、魂は死後何も知性認識しない。


\\



3.~{\sc Praeterea}, si anima separata intelligit,
 oportet quod per aliquas species intelligat. Sed non intelligit per
 species innatas, quia a principio est sicut {\it tabula in qua nihil est
 scriptum}\footnote{Arist.~{\it de Anima}, III, cap.~4, n.11. S.Th.~lect.~9.
}. Neque per species quas abstrahat a rebus, quia non habet
 organa sensus et imaginationis, quibus mediantibus species
 intelligibiles abstrahuntur a rebus. Neque etiam per species prius
 abstractas, et in anima conservatas, quia sic anima pueri nihil
 intelligeret post mortem. Neque etiam per species intelligibiles
 divinitus influxas, haec enim cognitio non esset naturalis, de qua nunc
 agitur, sed gratiae. Ergo anima separata a corpore nihil intelligit.


&

さらに、もし分離した魂が知性認識するならば、なんらかの形象によって知性認
 識しなければならない。
しかるに、生得的な形象によって知性認識するのではない。なぜなら、初めは
 「何も書かれていない板」のようであるから。
また、諸事物から抽象する形象によってでもない。なぜなら、感覚と想像力の器
 官を媒介として、可知的形象を諸事物から抽象するが、そのような器官を持た
 ないからである。
さらにまた、以前に抽象され、魂の内に保存された形象によってでもない。なぜ
 なら、もしそうなら、子供の魂は、死後、何も知性認識しないであろうから。
さらにまた、神から注がれた可知的形象によってでもない。なぜなら、そのよう
 な認識は、今論じられている自然的な認識ではなく、恩恵による認識だから。
ゆえに、身体から分離した魂は、何も知性認識しない。

\\



{\sc Sed contra est quod} philosophus dicit, in I
 {\it de anima}\footnote{Cap.~1, n.~20. S.Th.~lect.~2.}, quod si non est aliqua operationum animae propria, non
 contingit ipsam separari. Contingit autem ipsam separari. Ergo habet
 aliquam operationem propriam; et maxime eam quae est
 intelligere. Intelligit ergo sine corpore existens.


&

しかし反対に、哲学者は『デ・アニマ』第1巻で「もし働きの中で魂に固有なな
 んらかの働きがなければ、魂が分離することは起こらない」と述べている。
しかるに、魂が分離するということが生じている。ゆえに、なんらかの固有の働
 きがあり、知性認識する働きが最大限にそれである。ゆえに、身体なしに存在
 する魂は知性認識する。


\\


{\sc Respondeo dicendum quod} ista quaestio
 difficultatem habet ex hoc quod anima, quandiu est corpori coniuncta,
 non potest aliquid intelligere nisi convertendo se ad phantasmata, ut
 per experimentum patet. Si autem hoc non est ex natura animae, sed per
 accidens hoc convenit ei ex eo quod corpori alligatur,
%\footnote
%{al-l\u{\i}go (adl- ), \={a}vi, \={a}tum, 1, v.~a.
%I.~A..~Lit., to bind to something:
%B.~In gen., to bind, to bind up, bind round:
%II.~Trop., to bind, to hold fast, to hinder, detain; or in a moral sense, to bind, to oblige, lay under obligation
%} 
sicut Platonici
 posuerunt, de facili quaestio solvi posset. Nam remoto impedimento
 corporis, rediret anima ad suam naturam, ut intelligeret intelligibilia
 simpliciter, non convertendo se ad phantasmata, sicut est de aliis
 substantiis separatis. Sed secundum hoc, non esset anima corpori unita
 propter melius animae, si peius intelligeret corpori unita quam
 separata; sed hoc esset solum propter melius corporis, quod est
 irrationabile, cum materia sit propter formam, et non e converso. 

& この問題が難しいのは、経験によって明らかなとおり、魂が、身体に結合して
いるあいだ、表象像の方を向くことによらない限り何も知性認識しえないこ
とによる。もし、プラトン派の人々が考えたように、これが魂の本性に基づくの
でなく、魂が身体に結びつけられていることによって、附帯的に、魂に適合して
いるのであれば、この問題は容易に解かれえたであろう。すなわち、身体という
障害が取り除かれると、魂は自らの本性に戻り、他の分離実体についてそうであ
るように、表象像の方に向くことなく、端的に可知的なものを知性認識したであ
ろう、と。しかし、この考えによれば、もし身体に合一した魂が分離した魂より
も悪く知性認識したならば、魂が身体に合一されたのは、魂のより善い善のため
ではなく、むしろ、非理性的なものである身体の、より善い善のためであっただ
ろう。なぜなら、質料は形相のためにあるのであって、その逆ではないからであ
る。

\\

Si autem ponamus quod anima ex sua natura habeat ut intelligat convertendo
 se ad phantasmata, cum natura animae per mortem corporis non mutetur,
 videtur quod anima naturaliter nihil possit intelligere, cum non sint
 ei praesto
%\footnote
%{praest\={o} (old collat.~form praest\={u} , acc.~to Curtius Valerianus in Cassiod.~p.~2289 P.: qui praestu sunt, Inscr.~Carina Via Appia, 1, p.~217.~In later time as adj.: prae-stus , a, um:
%I.“bonorum officio praestus fui,” Inscr.~Grut.~669, 4), adv.~dat.~from praestus, a sup.~form from prae, so that praesto esse alicui = to be or stand in the foremost place for or as respects one, at hand, ready, present, here; usually with esse (very freq.~and class.).
%I.~Lit.:
%II.~In partic: praesto esse or adire
%A.~To be at hand, to attend or wait upon, to serve, aid:
%B.~With esse, to present one's self in a hostile manner, to resist, oppose
%}
 phantasmata ad quae convertatur. Et ideo ad hanc
 difficultatem tollendam, considerandum est quod, cum nihil operetur
 nisi inquantum est actu, modus operandi uniuscuiusque rei sequitur
 modum essendi ipsius.

&

しかし、もしわたしたちが、魂が、自らの本性に基づいて、表象像の方を向くこ
 とによって知性認識すると考えるなら、魂の本性は身体の死によって変化を受
 けないので、魂がその方を向く表象像が手元にないのだから、魂は本性的
 に何も知性認識できないように思われる。それゆえ、この困難を克服するため
 に、以下のことが考察されるべきである。すなわち、なにものも、現実態にあ
 る限りにおいてでなければ働かないので、各々の事物の働き方は、そのもの在り
 方に従う。

\\

 Habet autem anima alium modum essendi cum unitur
 corpori, et cum fuerit a corpore separata, manente tamen eadem animae
 natura; non ita quod uniri corpori sit ei accidentale, sed per rationem
 suae naturae corpori unitur; sicut nec levis natura mutatur cum est in
 loco proprio, quod est ei naturale, et cum est extra proprium locum,
 quod est ei praeter naturam.

&

しかるに、魂は、身体に合一しているときと、身体から分離してい
 るときとでは、異なる在り方をするが、魂の本性は同一にとどまる。つまり、
 身体に合一されることは、魂にとって附帯的でなく、その本性の性格によって、
 身体に合一される。ちょうど、軽いものの本性が、それに本性的である固有の
 場所にあるときと、それにとって本性的でない、固有の場所の外にあるときと
 で変化しないように。

\\

Animae igitur secundum illum modum essendi
 quo corpori est unita, competit modus intelligendi per conversionem ad
 phantasmata corporum, quae in corporeis organis sunt, cum autem fuerit
 a corpore separata, competit ei modus intelligendi per conversionem ad
 ea quae sunt intelligibilia simpliciter, sicut et aliis substantiis
 separatis. Unde modus intelligendi per conversionem ad phantasmata est
 animae naturalis, sicut et corpori uniri, sed esse separatum a corpore
 est praeter rationem suae naturae, et similiter intelligere sine
 conversione ad phantasmata est ei praeter naturam. Et ideo ad hoc
 unitur corpori, ut sit et operetur secundum naturam suam. 

&
ゆえに、魂には、それによって身体に合一されるところの在り方
 にしたがって、身体器官の中にある、諸物体の表象像の方を向くことによって
 知性認識するというあり方が適合する。他方、身体から分離しているときには、
 他の離在実体と同じように、端的に可知的なものの方を向くことによって知性
 認識するというありかたが適合する。
したがって、表象像の方を向くことによって知性認識するというあり方は、身体
 に合一するということと同様に、魂にとって本性的である。しかし、身体から
 分離されるということは、その本性の性格を外れているし、同様に、表象像の
 方を向かないで知性認識することも、魂にとって、その本性から外れて
 いる。ゆえに、魂は、自らの本性に従って存在し、また働くように、身体に合
 一される。

\\

Sed hoc
 rursus habet dubitationem. Cum enim natura semper ordinetur ad id quod
 melius est; est autem melior modus intelligendi per conversionem ad
 intelligibilia simpliciter, quam per conversionem ad phantasmata,
 debuit sic a Deo institui animae natura, ut modus intelligendi nobilior
 ei esset naturalis, et non indigeret corpori propter hoc
 uniri. 

& しかし、これでもまだ疑問がある。なぜなら、本性は常により善いものへ秩序
づけられるが、表象像の方を向くことによる知性認識よりも、端的に可知的なも
のの方を向くことによる知性認識の方が、より善い知性認識のあり方なので、魂
の本性は、神によって、より高貴な知性認識のあり方が、魂にとって自然本性的
であるように創られるべきだったのであり、このためには、身体に合一される必
要はなかった(のではないか、という疑問である)。


\\


Considerandum est igitur quod, etsi intelligere per conversionem
 ad superiora sit simpliciter nobilius quam intelligere per conversionem
 ad phantasmata; tamen ille modus intelligendi, prout erat possibilis
 animae, erat imperfectior. Quod sic patet. In omnibus enim substantiis
 intellectualibus invenitur virtus intellectiva per influentiam divini
 luminis. Quod quidem in primo principio est unum et simplex; et quanto
 magis creaturae intellectuales distant a primo principio, tanto magis
 dividitur illud lumen et diversificatur, sicut accidit in lineis a
 centro egredientibus.

&
それゆえ、上位のものどもの方を向くことによって知性認識することは、端的に、
 表象像の方を向くことによって知性認識することよりも高貴であるとしても、
 前者の知性認識のしかたは、魂にとって可能であることとしては、より不完全
 であったということが考察されるべきである。
これは以下のようにして明らかである。
すなわち、すべての知性的実体において、神の光の流入による知的な力が見出さ
 れる。これは、第一根源においては一つで単純だが、知性的被造物が第一根源
 から遠ざかるほど、かの光はより多く分割され多様化される。ちょうど、中心
 から出て行く線において生じるように。


\\



 Et inde est quod Deus per unam suam essentiam
 omnia intelligit; superiores autem intellectualium substantiarum, etsi
 per plures formas intelligant, tamen intelligunt per pauciores, et
 magis universales, et virtuosiores ad comprehensionem rerum, propter
 efficaciam virtutis intellectivae quae est in eis; in inferioribus
 autem sunt formae plures, et minus universales, et minus efficaces ad
 comprehensionem rerum, inquantum deficiunt a virtute intellectiva
 superiorum.


& したがって、神は、一つの自らの本質によって万物を知性認識し、また、知性
的実体の内上位の者たちは、複数の形相によって知性認識するが、それらの中に
ある知性的力の強さのために、より少なく、より普遍的で、諸事物の把握により
力のある形相によって知性認識する。他方、下位の者たちのなかには、それらが
 上位の者たちの知性的な力から劣っている限りにおいて、より多く
 の、より普遍性の少ない、事物の把握のためにより力のない形相がある。

\\



 Si ergo inferiores substantiae haberent formas in illa
 universalitate in qua habent superiores, quia non sunt tantae
 efficaciae in intelligendo, non acciperent per eas perfectam
 cognitionem de rebus, sed in quadam communitate et confusione. Quod
 aliqualiter apparet in hominibus, nam qui sunt debilioris intellectus,
 per universales conceptiones magis intelligentium non accipiunt
 perfectam cognitionem, nisi eis singula in speciali
 explicentur.


&

ゆえに、下位の[知的]諸実体が、上位の者たちが持っている普遍性においてそ
 れらの形相を持ったならば、それらは上位の者たちほどの知性認識における強
 さを持たないので、それらの形相によって、諸事物についての完全な認識を受
 け取らず、ある共通性と混乱のもとに、認識を受け取ったであろう。
このことは、ある程度、人間においても明らかである。
すなわち、知性が弱い人々は、彼らに個々の事柄が個別的に説明されない限
 り、より多く認識する者たちの普遍的な概念によっては、完全な認識を受け取
 らない。

\\


 Manifestum est autem inter substantias intellectuales,
 secundum naturae ordinem, infimas esse animas humanas. Hoc autem
 perfectio universi exigebat, ut diversi gradus in rebus essent. Si
 igitur animae humanae sic essent institutae a Deo ut intelligerent per
 modum qui competit substantiis separatis, non haberent cognitionem
 perfectam, sed confusam in communi.


&

ところで、知的諸実体の中で、人間の魂が、本性の秩序において最下位であるこ
 とは明らかである。
しかるに、諸事物においてさまざまな階級が存在することを、世界の完全性は必
 要とする。
ゆえに、もし人間の魂が、離在実体に適合するしかたで知性認識するように、神
 によって創られていたならば、完全な認識ではなく、混乱し、共通的な認識を
 持っていたであろう。

\\



 Ad hoc ergo quod perfectam et
 propriam cognitionem de rebus habere possent, sic naturaliter sunt
 institutae ut corporibus uniantur, et sic ab ipsis rebus sensibilibus
 propriam de eis cognitionem accipiant; sicut homines rudes
%\footnote
%{r\u{u}dis , e, adj.~cf.~crudus,
%I.unwrought, untilled, unformed, unused, rough, raw, wild (cf.~crudus):
% omnis fere materia non deformata, rudis appellatur, sicut vestimentum
% rude, non perpolitum: sic aes infectum rudusculum, Cincius
% ap.~Fest.~p.~265 M\"{u}ll. (class.; esp.~freq.~in the trop.~signif.).
%I.~Lit.:
%B.~Poet., transf., young, new (cf.~integer):
%II.~Trop., rude, unpolished, uncultivated, unskilled, awkward, clumsy, ignorant; hence (like ignarus), with gen., unacquainted with, inexperienced in, etc. (cf.~imperitus). 
%}
 ad scientiam
 induci non possunt nisi per sensibilia exempla. Sic ergo patet quod
 propter melius animae est ut corpori uniatur, et intelligat per
 conversionem ad phantasmata; et tamen esse potest separata, et alium
 modum intelligendi habere.


&
ゆえに、諸事物について完全で固有の認識をもつことができるために、[人間の
 魂は]身体に合一されるというかたちで本性的に創られ、そのようにして、そ
 の可感的諸事物から、それらについての固有の認識を受け取る。
ちょうど、教育を受けていない人が、可感的な例によらなければ、学知へと導か
 れないように。
ゆえに、このようにして、[人間の魂が]身体に合一されること、そして、表象像の方を向くこ
 とによって知性認識することは、魂の善のためであること、また、そうではあ
 るが、[身体から]分離されうること、そして、他の知性認識のしかたを持つ
 ことが明らかである。


\\



Ad primum ergo dicendum quod, si diligenter
 verba philosophi discutiantur,
%\footnote
%{dis-c\u{u}t\u{\i}o , cussi, cussum, 3, v. a. quatio,
%I.to strike asunder, dash to pieces, shatter, etc.
%I. In gen.:
%II. In partic.
%A. In medic. lang., to scatter, disperse ( = digerere)
%B. Pregn., to break up, scatter, disperse, dissipate.
%1. Lit. (rarely):
%2. Trop.:
%\textcolor{red}{For the meaning to investigate, discuss (lit., to separate mentally,
% distinguish, as in disputare, discernere, etc.), which prevails in the
% post-class. derivatives: discussio, discussor, and discusse; as also in
% the Romance: discutere, discussare, discussione; discuter, discussion,
% etc., there appear to be no examples in the literary language.}
%} 
philosophus hoc dixit ex quadam
 suppositione prius facta, scilicet quod intelligere sit quidam motus
 coniuncti, sicut et sentire, nondum enim differentiam ostenderat inter
 intellectum et sensum. Vel potest dici quod loquitur de illo modo
 intelligendi qui est per conversionem ad phantasmata.

&

第一に対しては、それゆえ、次のように言われるべきである。
もし哲学者の言葉が注意深く分析されるならば、哲学者がこれを言ったのは、以
 前に為されたある前提、すなわち、知性認識することが、感覚もそうであるよ
 うに、結合体のなんらかの運動であるという前提のもとでであり、つまり、彼
 はまだ知性と感覚の違いを明らかにしていない。
あるいは、次のように言うこともできる。
彼は、表象像の方を向くことによる知性認識のしかたについて語っている、と。

\\



De quo etiam procedit secunda ratio.


&

第二の異論もこれについて論じている。

\\



Ad tertium dicendum quod anima separata non
 intelligit per species innatas; nec per species quas tunc abstrahit;
 nec solum per species conservatas, ut obiectio probat, sed per species
 ex influentia divini luminis participatas, quarum anima fit particeps
 sicut et aliae substantiae separatae, quamvis inferiori modo. Unde tam
 cito cessante conversione ad corpus, ad superiora convertitur. Nec
 tamen propter hoc cognitio non est naturalis, quia Deus est auctor non
 solum influentiae gratuiti luminis, sed etiam naturalis.


&

第三に対しては、次のように言われるべきである。
異論が証明するとおり、分離した魂は、生得的な形象によって知性認識しないし、
 そのときに抽象する形象によっても、保存された形象のみによっても、知性認
 識しない。しかし、神の光の流入によって分有された形象によっては、知性認
 識する。魂は、より劣ったしかたではあるが、他の離在実体のように、それら
 を分有するものとなる。
したがって、身体の方を向くことをやめるとすぐに、[魂は]上位のものの方を
 向く。しかし、このことのために、認識が本性的でないことにはならない。な
 ぜなら、神は恩恵の光だけでなく、自然の光を流入させる者だからである。


\end{longtable}
%
%\begin{itemize}
% \item[◎] 表象像とは何か。表象像とは「現象」である、という仮説を検証す
%	  る作業中。
% \item[◎] 主文中にある、``... ad phantasmata corporum, quae in
%	   corporeis organis sunt,''という表現から、表象像はなんらかの身
%	   体器官のうちにあるとされている。どのような器官か。脳?
%
%\end{itemize}
%

\newpage
\rhead{a.~2}

\begin{center}
 {\Large {\bf ARTICULUS SECUNDUS}}\\
 {\large UTRUM ANIMA SEPARATA INTELLIGAT SUBSTANTIAS SEPARATAS}\\
{\footnotesize III {\it SCG}, c.~14; {\it Qu.~de Anima}, a.~17; {\it
 Quodl}.~III, q.~9, a.~1.}\\
{\Large 第2項\\分離した魂は離在実体を知性認識するか}

\end{center}

\begin{longtable}{p{21em}p{21em}}

{\huge A}{\sc d secundum sic proceditur}. Videtur quod
anima separata non intelligat substantias separatas. Perfectior enim est
anima corpori coniuncta, quam a corpore separata, cum anima sit
naturaliter pars humanae naturae; quaelibet autem pars perfectior est in
suo toto. Sed anima coniuncta corpori non intelligit substantias
separatas, ut supra habitum est. Ergo multo minus cum fuerit a corpore
separata.

&

第二に対しては次のように進められる。
分離した魂は離在実体を知性認識しないように思われる。
魂は、自然本性的に、人間本性の部分だから、身体に結合された魂の方が、身体
 から分離された魂よりも完全である
いかなる部分も、その全体の中にある方が、より完全なのだから。
しかるに、身体に結合された魂は、上で述べられたとおり\footnote{{\it ST}
 I, q.~88, a.~1}、離在実体を知性認識
 しない。ゆえに、身体から分離されたとき、さらにいっそう認識しない。

\\


2 {\sc Praeterea}, omne quod cognoscitur, vel
cognoscitur per sui praesentiam, vel per suam speciem. Sed substantiae
separatae non possunt cognosci ab anima per suam praesentiam, quia nihil
illabitur
%\footnote{ill\={a}bor (inl- ), psus, 3,
%I.v.~dep.~n.~[inlabor], to fall, slip, slide, glide, or flow into; to
% fall down, sink down (rare but class.). } 
animae nisi solus Deus. Neque etiam per aliquas species quas
anima ab Angelo abstrahere possit, quia Angelus simplicior est quam
anima. Ergo nullo modo anima separata potest cognoscere substantias
separatas.

&

さらに、すべて認識されるものは、自らが現在することによってか、あるいは、
 自らの形象によって認識される。
しかるに、離在実体は、その現在によって魂に認識されえない。なぜなら、ただ神以
 外の何ものも、魂の中に入ってこないであろうから。
また、魂が天使から抽象することができるなんらかの形象によるのでもない。な
 ぜなら、天使は魂よりも単純だから。
ゆえに、分離した魂は、いかなるしかたによっても、離在実体を認識できない。

\\


3 {\sc Praeterea}, quidam philosophi posuerunt in
cognitione separatarum substantiarum consistere ultimam hominis
felicitatem. Si ergo anima separata potest intelligere substantias
separatas, ex sola sua separatione consequitur felicitatem. Quod est
inconveniens.

&

さらに、ある哲学者たちは\footnote{Averroes. cf. {\it ST} I, q.~88, a.~1, c.}、離在実体を認識することにおいて、人間の究極の幸
 福が成立するとした。ゆえに、もし分離した魂が離在実体を知性認識すること
 ができたら、自らが分離することだけから、幸福が伴うことになる。これは不
 都合である。

\\


{\sc Sed contra est} quod animae separatae
cognoscunt alias animas separatas; sicut dives in Inferno positus vidit
Lazarum et Abraham, Luc. XVI. Ergo vident etiam et Daemones et Angelos
animae separatae.

&

しかし反対に、分離した魂は、他の分離した魂を認識する。ちょうど、地獄に置
 かれた金持ちが、ラザロやアブラハムを見たように(『ルカによる福音書』16
 章)。ゆえに、分離した魂は、悪魔や天使たちを見る。

\\


{\sc Respondeo dicendum} quod, sicut Augustinus dicit in IX {\it de
Trin}.: {\it Mens nostra cognitionem rerum incorporearum per seipsam
accipit}, idest cognoscendo seipsam, sicut supra dictum est. Per hoc
ergo quod anima separata cognoscit seipsam, accipere possumus qualiter
cognoscit alias substantias separatas. Dictum est autem quod quandiu
anima corpori est unita, intelligit convertendo se ad phantasmata. Et
ideo nec seipsam potest intelligere nisi inquantum fit actu intelligens
per speciem a phantasmatibus abstractam, sic enim per actum suum
intelligit seipsam, ut supra dictum est. 

&

答えて言わなければならない。アウグスティヌスが『三位一体論』9巻で述べてい
 るように、「私たちの精神は、非物体的な事物の認識を、自分自身によって獲得
 する」。すなわち、上で述べられたとおり、自分自身を認識することによって
 [獲得する]。ゆえに、私たちは、分離した魂が他の離在実体をどのように認識
 するかを、それが自己を認識するということを通して、理解することができる。
さて、魂が身体に合一しているあいだ、魂は表象像の方を向くことによって知性
 認識すると言われた。ゆえに、表象像から抽象された形象によって、現実に知
 性認識するものになる限りにおいてでなければ、魂は自らを知性認識すること
 ができない。このように、上で述べられたとおり、自らの働きを通して自己を
 知性認識する。

\\


Sed cum fuerit a corpore separata, intelliget non convertendo se ad
phantasmata, sed ad ea quae sunt secundum se intelligibilia, unde
seipsam per seipsam intelliget. Est autem commune omni substantiae
separatae quod intelligat {\it id quod est supra se, et id quod est
infra se, per modum suae substantiae}, sic enim intelligitur aliquid
secundum quod est in intelligente; est autem aliquid in altero per modum
eius in quo est. 

&
しかし、[魂が]身体から分離したとき、表象像の方を向くことによってではな
 く、それ自体において可知的なものどもの方を向くことで知性認識するであろ
 う。したがって、自己を自己によって知性認識するであろう。
ところで、すべての離在実体には、「自己の上にあるものや自己の下にあるものを、
 自らの実体のあり方を通して」知性認識することが共通する。
なぜなら、あるものは、それが知性認識するものの中にあることにしたがって、
 そのように知性認識されるからである。また、あるものは他のものの中に、そ
 こにおいて在るところのもののあり方によって在る。
\\


Modus autem substantiae animae separatae est infra
modum substantiae angelicae, sed est conformis modo aliarum animarum
separatarum. Et ideo de aliis animabus separatis perfectam cognitionem
habet; de Angelis autem imperfectam et deficientem, loquendo de
cognitione naturali animae separatae. De cognitione autem gloriae est
alia ratio.

&

しかるに、分離した魂の実体のあり方は、天使の実体のあり方より下だが、しか
 し、他の分離した魂とは一致したあり方にある。
ゆえに、分離した魂の本性的な認識について語るならば、他の分離した魂につい
 ては、完全な認識を持つが、天使たちについては、不完全で欠陥のある認識し
 かもたない。
ただし、栄光の認識については、別の理論がある。


\\



{\sc Ad primum ergo dicendum} quod anima separata est quidem imperfectior, si
consideretur natura qua communicat cum natura corporis, sed tamen
quodammodo est liberior ad intelligendum, inquantum per gravedinem et
occupationem corporis a puritate intelligentiae impeditur.

& 第一に対しては、それゆえ、次のように言われるべきである。分離した魂は、
身体の本性とそれによって共通するところの本性が考察されるならば、[身体と
合一した魂よりも]たしかに不完全だが、しかし[身体と合一した魂が]重さに
 よって、あるいは、身体をとることによって、知性認識の純粋さから妨げられ
 るかぎりにおいて、ある意味で、知性認識に対してより自由である。

\\


{\sc Ad secundum dicendum} quod anima separata
intelligit Angelos per similitudines divinitus impressas. Quae tamen
deficiunt a perfecta repraesentatione eorum, propter hoc quod animae
natura est inferior quam Angeli.

&
第二に対しては次のように言われるべきである。
分離した魂は、神によって刻印された類似を通して天使を知性認識する。
しかし、この類似は、それらの完全な表現からは欠けているのであり、それは、
 魂の本性が天使の本性よりも下位にあるためである。



\\


{\sc Ad tertium dicendum} quod in cognitione
substantiarum separatarum non quarumcumque, consistit ultima hominis
felicitas, sed solius Dei, qui non potest videri nisi per gratiam. In
cognitione vero aliarum substantiarum separatarum est magna felicitas,
etsi non ultima, si tamen perfecte intelligantur. Sed anima separata
naturali cognitione non perfecte eas intelligit, ut dictum est.

& 第三に対しては、次のように言われるべきである。どんな離在実体の認識にお
いても、人間の究極の幸福が成立するのではなく、ただ神の認識において成立す
る。この神は、恩恵によらなければ見られえない。他方、他の離在実体の認識に
おいては、究極ではないが、大きな幸福がある。ただし、完全に知性認識される
ならば。しかし、すでに述べられたように、分離した魂は、自然本性的な認識に
よって、それらを完全に知性認識することはできない。



\end{longtable}

\newpage
\rhead{a.~3}

\begin{center}
 {\Large {\bf ARTICULUS TERTIUS}}\\
 {\large UTRUM ANIMA SEPARATA OMNIA NATURALIA COGNOSCAT}\\
 {\footnotesize {\it Q.~de Anima}, a.~18.}\\
 {\Large 第3項\\分離した魂はすべての自然的なものを認識するか}
\end{center}

\begin{longtable}{p{21em}p{21em}}


{\huge A}{\sc d tertium sic proceditur}. Videtur quod anima separata
 omnia naturalia cognoscat. In substantiis enim separatis sunt rationes
 omnium rerum naturalium. Sed animae separatae cognoscunt substantias
 separatas. Ergo cognoscunt omnia naturalia.

&

第三に対しては次のように進められる。
分離した魂は、すべての自然本性的なものを認識すると思われる。
なぜなら、離在実体の中には、すべての自然的諸事物の理念が存在する。
しかるに、分離した魂は、離在実体を認識する。ゆえに、すべての自然本性的な
 ものを認識する。

\\




2 {\sc Praeterea}, qui intelligit magis intelligibile, multo magis
 potest intelligere minus intelligibile. Sed anima separata intelligit
 substantias separatas, quae sunt maxima intelligibilium. Ergo multo
 magis potest intelligere omnia naturalia, quae sunt minus
 intelligibilia.

&

さらに、より可知的なものを知性認識するものは、より可知的でないものを、優
 れてよく知性認識できる。
しかるに、分離した魂は離在実体を知性認識するが、それは可知的なものの中で
 もっとも可知的なものである。ゆえに、より可知的でないものであるすべての
 自然本性的なものを知性認識できる。


\\




{\sc Sed contra}, in Daemonibus magis viget
%\footnote{v\u{\i}g\u{e}o , \={e}re, v.~n.~Sanscr.~ugras, mighty; Gr.~\textgreek{<ugi'hs}, sound; cf.~Lat.~vegeo, vigil, augeo,
%I.to be lively or vigorous; to thrive, flourish, bloom; to be in honor, esteem, repute, etc.~(class.; mostly of things, concrete and abstract; cf.~valeo).
%I.~In gen.:
%II. Of persons, to live, be alive:
%}
 naturalis cognitio quam in
 anima separata. Sed Daemones non omnia naturalia cognoscunt; sed multa
 addiscunt per longi temporis experientiam, ut Isidorus
 dicit.\footnote{{\it Sententiarum} (al. {\it de Summo Bono}) lib. I,
 cap.~10, al.~19. Cf. August. {\it de Divinat.~Daemon.}, cap.~3} Ergo
 neque animae separatae omnia naturalia cognoscunt.

&

しかし反対に、悪魔の中には、分離した魂の中よりも大きな自然的認識がある。
 しかるに、イシドルスが言うように、悪魔たちはすべての自然的なものを認識
 せず、長い時間の経験を通して多くを学ぶ。
ゆえに、分離した魂も、すべての自然的なものを認識するわけではない。


\\




2 {\sc Praeterea}, si anima statim cum est
 separata, omnia naturalia cognosceret, frustra homines studerent ad
 rerum scientiam capessendam.
%\footnote
%{c\u{a}pesso (c\u{a}pisso , Pac.~ap.~Non.~p.~227, 1), \={\i}vi (Sall.~H.~3, 68 Dietsch; Tac.~A.~15, 49), or ii (Tac.~A.~12, 30: capessi, given by Diom.~p.~367 P., and by Charis.~ap.~Prisc.~p.~902 ib., but apparently erroneously; cf.~Struve, p.~198, and lacesso), \={\i}tum (acc.~to Prisc.~l.~l.
%I.part.~fut.~capessiturus, Tac.~A.~6, 48), 3, v.~desid.~a.~[capio].
%I.~Lit., to seize, take, or catch at eagerly, to snatch at, lay hold of (capesso = desidero capere, Prisc.~l.~l.; “rare but class.):
%B.~Of relations of place, to strive to reach a place or limit, to betake one's self to, to go to, to repair or resort to; constr.~usu.~with acc.; ante-class.~also capere se in or ad aliquem locum.
%II.~Trop.
%A.~To take hold of any thing with zeal, to take upon one's self, take in
% hand, to undertake, enter upon, engage in, execute, manage (the most
% usu.~signif.; cf.~I.~A.):
% so, capessere rem
% publicam, to undertake affairs of state, to engage in public affairs,
% administer (differing, by the idea of zealous co-operation and
% activity, from accedere ad rem publicam, which designates merely the
% entering upon a public office or duty)
%}
 Hoc autem est inconveniens. Non ergo anima
 separata omnia naturalia cognoscit.

&

さらに、もし魂が、分離されるやいなやすべての自然的なものを認識したとする
 と、人間が諸事物のが口を獲得しようと努力するのは無駄であることになる。
 ゆえに、分離した魂がすべての自然的なものを認識するわけではない。

\\



{\sc Respondeo dicendum} quod, sicut supra dictum
 est, anima separata intelligit per species quas recipit ex influentia
 divini luminis, sicut et Angeli, sed tamen, quia natura animae est
 infra naturam Angeli, cui iste modus cognoscendi est connaturalis,
 anima separata per huiusmodi species non accipit perfectam rerum
 cognitionem, sed quasi in communi et confusam. Sicut igitur se habent
 Angeli ad perfectam cognitionem rerum naturalium per huiusmodi species,
 ita animae separatae ad imperfectam et confusam. Angeli autem per
 huiusmodi species cognoscunt cognitione perfecta omnia naturalia, quia
 omnia quae Deus fecit in propriis naturis, fecit in intelligentia
 angelica, ut dicit Augustinus, {\it super Gen.~ad Litt}. Unde et animae
 separatae de omnibus naturalibus cognitionem habent, non certam et
 propriam, sed communem et confusam.

&

答えて言わなければならない。
上で述べられたとおり、分離した魂は、天使のように、神の光の流入から受
 け取る形象を通して知性認識するが、しかし、魂の本性は、天使の本性の下位
 にあるので、天使はこのような認識のしかたが本性に適っているが、分離した
 魂は、このような形象によって、完全な事物の認識を受け取らず、いわば共通
 的に、混乱した認識を受け取る。
ゆえに、天使たちが、このような形象によって、自然的諸事物の完全な認識へ関
 係するように、分離した魂は、不完全で混乱した認識へ関係する。
ところで、アウグスティヌスが『創世記逐語注解』で述べるように、神が固有の
 本性において作ったすべてのものを、知性的実体である天使の中にも作ったの
 だから、天使はこのような形象をとおして、完全な認識によって、すべての自
 然的なものを認識する。
したがって、分離した魂も、すべての自然的なものについての認識を持つが、し
 かし、共通的で混乱した認識を持つ。


\\




{\sc Ad primum ergo dicendum} quod nec ipse
 Angelus per suam substantiam cognoscit omnia naturalia, sed per species
 quasdam, ut supra dictum est. Et ideo non propter hoc sequitur quod
 anima cognoscat omnia naturalia, quia cognoscit quoquo modo substantiam
 separatam.

&

第一に対しては、それゆえ、次のように言われるべきである。
天使も、自らの実体によって、すべての自然的なものを認識するのではなく、上
 述のように、なんらかの形象によってである。
ゆえに、[分離した魂が]各々のかたちで離在実体を認識するからという理由で、
 魂がすべての自然的なものを認識することは帰結しない。

\\




{\sc Ad secundum dicendum} quod, sicut anima
 separata non perfecte intelligit substantias separatas ita nec omnia
 naturalia perfecte cognoscit, sed sub quadam confusione, ut dictum est.

&

第二に対しては次のように言われるべきである。
分離した魂が、離在実体を完全には認識しないように、すべての自然的なものを
 完全には認識せず、上述のように、なんらかの混乱のもとに認識する。

\\




{\sc Ad tertium dicendum} quod Isidorus loquitur
 de cognitione futurorum; quae nec Angeli nec Daemones nec animae
 separatae cognoscunt, nisi vel in suis causis, vel per revelationem
 divinam. Nos autem loquimur de cognitione naturalium.

&
第三に対しては、次のように言われるべきである。
イシドルスは、未来についての認識について語っているのであり、これは、天使
 も悪魔も分離した魂も認識しない。認識するとすれば、自らがもつ原因におい
 てか、神の啓示によってである。しかし私たちは、自然的なものの認識につい
 て語っている。


\\




{\sc Ad quartum dicendum} quod cognitio quae
 acquiritur hic per studium, est propria et perfecta; illa autem est
 confusa. Unde non sequitur quod studium addiscendi sit frustra.

&
第四に対しては、次のように言われるべきである。
この世で研究によって獲得される認識は、固有で完全であるが、かの認識は、混
 乱したものである。ゆえに、学ぶ研究が無駄だということにはならない。


\end{longtable}

\newpage
\rhead{a.~4}


\begin{center}
 {\Large {\bf ARTICULUS QUARTUS}}\\
 {\large UTRUM ANIMA SEPARATA COGNOSCAT SINGULARIA}\\
 {\footnotesize IV {\it Sent}., d.~50, q.~1, a.~3; {\it De Verit}.,
 q.~19, a.~2; {\it Qu.~de Anima}, a.~20.}\\
 {\Large 第4項\\分離した魂は個物を認識するか}
\end{center}

\begin{longtable}{p{21em}p{21em}}
{\huge A}{\sc d quartum sic proceditur}. Videtur quod
 anima separata non cognoscat singularia. Nulla enim potentia
 cognoscitiva remanet in anima separata nisi intellectus, ut ex supra
 dictis patet. Sed intellectus non est cognoscitivus singularium, ut
 supra habitum est. Ergo anima separata singularia non cognoscit.


&
第四に対しては、次のように進められる。
分離した魂は個物を認識しないと思われる。
なぜなら、上述のことから明らかなとおり、知性以外のどんな認識能力も、分離
 した魂に残らない。
しかるに、すでに論じられたとおり、知性は、個物を認識しうるものではない。
ゆえに、分離した魂は個物を認識しない。


\\




2 {\sc Praeterea}, magis est determinata cognitio
 qua cognoscitur aliquid in singulari, quam illa qua cognoscitur aliquid
 in universali. Sed anima separata non habet determinatam cognitionem de
 speciebus rerum naturalium. Multo igitur minus cognoscit singularia.


&


さらに、あるものが、普遍的に認識される認識よりも、個別的に認識される認識
 の方が、より限定された認識である。
しかるに、分離した魂は、自然的諸事物の形象について、限定された認識を持た
 ない。
ゆえに、ましてや、個物を認識しない。

\\





3 {\sc Praeterea}, si cognoscit singularia, et non
 per sensum, pari ratione omnia singularia cognosceret. Sed non
 cognoscit omnia singularia. Ergo nulla cognoscit.


&


さらに、もし個物を認識し、かつ、感覚によって認識するのでないとすれば、同
 じ理由で、すべての個物を認識することになっただろう。しかし、すべての個物
 を認識するわけではない。ゆえに、どの個物も認識しない。
\\





{\sc Sed contra est} quod dives in inferno
 positus dixit: {\it Habeo quinque fratres}, ut habetur Luc.~{\sc xvi}.


&

しかし反対に、『ルカによる福音書』16章で述べられているように、地獄に置か
 れた金持ちは、「私には五人の兄弟がいる」と言った。

\\



{\sc Respondeo dicendum} quod animae separatae
 aliqua singularia cognoscunt, sed non omnia, etiam quae sunt
 praesentia. Ad cuius evidentiam, considerandum est quod duplex est
 modus intelligendi. Unus per abstractionem a phantasmatibus, et
 secundum istum modum singularia per intellectum cognosci non possunt
 directe, sed indirecte, sicut supra dictum est. 


& 答えて言わなければならない。分離した魂は、なんらかの個物を認識するが、
すべての個物を認識するわけではなく、現在のものですら
%\footnote{天使は、自
%らに与えられた形象によって、現在の個物をすべて認識する。\\
%cf.~({\it ST} I,
%q.~57, a.~3, arg.~3) Praeterea, Angelus non cognoscit per species
%acceptas a rebus, sed per species innatas universales. Sed species
%universales aequaliter se habent ad praesens, praeteritum et
%futurum. Ergo videtur quod Angeli indifferenter cognoscant praeterita et
%praesentia et futura.\\
%  (ad 3) Ad tertium dicendum quod, licet species
%quae sunt in intellectu Angeli, quantum est de se, aequaliter se habeant
%ad praesentia, praeterita et futura; tamen praesentia, praeterita et
%futura non aequaliter se habent ad rationes. Quia ea quae praesentia
%sunt, habent naturam per quam assimilantur speciebus quae sunt in mente
%Angeli, et sic per eas cognosci possunt. Sed quae futura sunt, nondum
%habent naturam per quam illis assimilentur, unde per eas cognosci non
%possunt.}
すべてを認識するとは限らない。これを明らかにするためには、知
性認識のしかたには二通りあることが考察されるべきである一つは、表象像から
の抽象によるものであり、このしかたによっては、個物は、知性によって直接に
は認識されえず、上述のごとく、
%\footnote{{\it ST} I, q.~86, a.~1.
%Respondeo dicendum quod singulare in rebus materialibus intellectus
%noster directe et primo cognoscere non potest. Cuius ratio est, quia
%principium singularitatis in rebus materialibus est materia
%individualis, intellectus autem noster, sicut supra dictum est,
%intelligit abstrahendo speciem intelligibilem ab huiusmodi materia. Quod
%autem a materia individuali abstrahitur, est universale. Unde
%intellectus noster directe non est cognoscitivus nisi
%universalium. Indirecte autem, et quasi per quandam reflexionem, potest
%cognoscere singulare, quia, sicut supra dictum est, etiam postquam
%species intelligibiles abstraxit, non potest secundum eas actu
%intelligere nisi convertendo se ad phantasmata, in quibus species
%intelligibiles intelligit, ut dicitur in III {\it de Anima}. Sic igitur ipsum
%universale per speciem intelligibilem directe intelligit; indirecte
%autem singularia, quorum sunt phantasmata. Et hoc modo format hanc
%propositionem, Socrates est homo.}
間接的に、認識される。


\\




Alius modus
 intelligendi est per influentiam specierum a Deo, et per istum modum
 intellectus potest singularia cognoscere. Sicut enim ipse Deus per suam
 essentiam, inquantum est causa universalium et individualium
 principiorum, cognoscit omnia et universalia et singularia, ut supra
 dictum est; ita substantiae separatae per species, quae sunt quaedam
 participatae similitudines illius divinae essentiae, possunt singularia
 cognoscere.


&

もうひとつの知性認識のしかたは、神からの形象の流入によるものであり、この
 しかたによっても、知性は個物を認識しうる。
上で述べられたように、神自身もまた、普遍的なものと個別的根源の原因であるかぎりに
 おいて、普遍的なものも個的なものもすべて認識する。
ちょうどそのように、離在実体も、かの神の本質のある種の分有された類似であ
 る形象によって、個物を認識しうる。

\\


 In hoc tamen est differentia inter Angelos et animas
 separatas, quia Angeli per huiusmodi species habent perfectam et
 propriam cognitionem de rebus, animae vero separatae confusam. Unde
 Angeli, propter efficaciam sui intellectus per huiusmodi species non
 solum naturas rerum in speciali cognoscere possunt, sed etiam
 singularia sub speciebus contenta. 


&

しかし、次の点が、天使と分離した魂とでは異なる。
すなわち、天使たちは、このような形象によって、事物についての完全で固有の
 認識を持つが、しかし分離した魂は、混乱した認識しか持たない。
したがって、天使たちは、自らの知性の強力さのために、そのような形象によっ
 て、事物の本性を種において認識できるだけでなく、種に含まれる個物もまた
 認識できる。

\\

Animae vero separatae non possunt
 cognoscere per huiusmodi species nisi solum singularia illa ad quae
 quodammodo determinantur, vel per praecedentem cognitionem, vel per
 aliquam affectionem, vel per naturalem habitudinem, vel per divinam
 ordinationem, quia omne quod recipitur in aliquo, determinatur in eo
 secundum modum recipientis.

&

しかし、分離した魂は、このような形象によっては、なんらかのかたちでそれへ
 と限定されるような個物しか認識できない。それは、先行する認識によってか、
 あるいは、なんらかの受動状態によってか、あるいは、本性的な関係によって
 か、あるいは、神による秩序付けによってかである。
何かの中に受け取られるものは、受け取るもののありかたに従って、そのものに
 おいて限定されるからである。

\\

{\sc Ad primum ergo dicendum} quod intellectus per
 viam abstractionis non est cognoscitivus singularium. Sic autem anima
 separata non intelligit, sed sicut dictum est.


&

第一に対しては、それゆえ、次のように言われるべきである。
知性は、抽象の道によっては個物を認識しえない。
しかし、そのようなかたちで分離した魂が知性認識するのではなく、上で言われ
 たようなかたちで認識する。

\\





Ad secundum dicendum quod ad illarum rerum
 species vel individua cognitio animae separatae determinatur, ad quae
 anima separata habet aliquam determinatam habitudinem, sicut dictum
 est.

&

第二に対しては次のように言われるべきである。
すでに言われたように、分離した魂の認識は、その分離した魂がなんらかの限定
 された関係をもつような事物の形象や個体に限定される。


\\



Ad tertium dicendum quod anima separata non
 se habet aequaliter ad omnia singularia, sed ad quaedam habet aliquam
 habitudinem quam non habet ad alia. Et ideo non est aequalis ratio ut
 omnia singularia cognoscat.

&

第三に対しては、次のように言われるべきである。
分離した魂は、すべての個物に等しく関係するのではなく、あるものに対して、
 他のものに対しては持たないような関係を持つ。
ゆえに、すべての個物を認識することになるような、等しい根拠はない。





\end{longtable}

\newpage
\rhead{a.~5}

\begin{center}
 {\Large {\bf ARTICULUS QUINTUS}}\\
 {\large UTRUM HABITUS SCIENTIAE HIC ACQUISITAE\\ REMANEAT IN ANIMA
 SEPARATA}\\
 {\footnotesize {\it ST} I-IIae, q.~67, a.~2; {\it IV Sent.}, d.~1, q.~1,
 a.~2; {\it Quodl.}~XII, q.~9, a.~1; {\it I Cor.}, c.~13, l.~3.}\\
 {\Large 第5項\\この世で獲得された知の習態が分離した魂の中に残るか}
\end{center}

\begin{longtable}{p{21em}p{21em}}



{\huge A}{\sc d quintum sic proceditur}. Videtur quod
habitus scientiae hic acquisitae non remaneat in anima separata. Dicit
enim apostolus, I {\it ad Cor}.~{\sc xiii}: {\it Scientia destruetur}.


&

第5に対しては、次のように進められる。
この世で獲得された知の習態が、分離した魂のなかに残ることはない、と思われ
 る。なぜなら、使徒は「知が破壊されるだろう」\footnote{「愛は決して滅び
 ない。預言は廃れ、異言はやみ、知識は廃れよう、」(13:8)}と述べている。

\\




2. {\sc Praeterea}, quidam minus boni in hoc mundo scientia
pollent,
%\footnote {poll\u{e}o (polet, pollet: I.“quia nondum geminabant
%antiqui consonantes,” Fest.~``p.~205 M\"{u}ll.), \={e}re,
%v.~n.~potis-valeo.~ I.~Lit.~ A.~In gen., to be strong, powerful, or
%potent, to be able, to prevail, avail (class.; cf.: valeo, possum): (β).
%With abl.: }
 aliis magis bonis carentibus scientia. Si ergo habitus
scientiae permaneret etiam post mortem in anima, sequeretur quod aliqui
minus boni etiam in futuro statu essent potiores aliquibus magis
bonis. Quod videtur inconveniens.


&

さらに、この世で、より少なく善である人々が知において優れ、より善である人
 たちが知を欠いている。ゆえに、知の習態が死後も魂の中に残ったならば、よ
 り少なく善である者が、将来も、より善である人々よりも力のある状態にある
 ことになるだろう。これは不都合であるように思われる。

\\


3. {\sc Praeterea}, animae separatae habebunt
scientiam per influentiam divini luminis. Si igitur scientia hic
acquisita in anima separata remaneat, sequetur quod duae erunt formae
unius speciei in eodem subiecto. Quod est impossibile.


&

さらに、分離した魂は、神の光の流入によって、知をもつだろう。
ゆえに、もし、この世で獲得された知が分離した魂のなかに残るなら、同じ基体
 のなかに、一つの種に属する二つの形相があることになる。これは不可能であ
 る。

\\


4. {\sc Praeterea}, philosophus dicit, in libro {\it Praedicament}.,
quod {\it habitus est qualitas difficile mobilis; sed ab aegritudine,
vel ab aliquo huiusmodi, quandoque corrumpitur scientia}. Sed nulla est
ita fortis immutatio in hac vita, sicut immutatio quae est per
mortem. Ergo videtur quod habitus scientiae per mortem corrumpatur.


&

さらに、哲学者は、『カテゴリー論』という書物の中で、次のように述べている。
 「習態は、動かされることが困難な性質である。しかし、病気やその他そのよ
 うなものによって、知が消滅することが時にある」。
しかし、この世での変化の中で、死による変化ほど強力なものはない。
ゆえに、知の習態は、死によって消滅すると思われる。

\\


{\sc Sed contra est} quod Hieronymus dicit, in Epistola {\it ad
Paulinum}: {\it discamus in terris, quorum scientia nobis perseveret in
caelo}.

&

しかし反対に、ヒエロニュムスは、パウリナ宛『書簡』で「その知が天において
 私たちに保存されるようなことを、地において学びましょう」と述べている。

\\


{\sc Respondeo dicendum} quod quidam posuerunt
habitum scientiae non esse in ipso intellectu, sed in viribus
sensitivis, scilicet imaginativa, cogitativa et memorativa; et quod
species intelligibiles non conservantur in intellectu possibili. Et si
haec opinio vera esset, sequeretur quod, destructo corpore, totaliter
habitus scientiae hic acquisitae destrueretur. 

&


答えて言わなければならない。ある人々\footnote{cf. Avicenna {\it de Anima},
part.~5, cap.~6.}は、知の習態が、知性自身の中でなく、感覚能力の中に、すな
わち、想像力、思考力、記憶力の中にあるとし、また、可知的形象は、可能知性
の中に保存されないとした。
もしこの意見が真であったならば、身体が破壊されると、この世で獲得された知
 の習態は、全面的に破壊されるということが帰結しただろう。


\\



Sed quia scientia est in intellectu, qui est {\it locus specierum}, ut
dicitur in III de {\it Anima}; oportet quod habitus scientiae hic
acquisitae partim sit in praedictis viribus sensitivis, et partim in
ipso intellectu. Et hoc potest considerari ex ipsis actibus ex quibus
habitus scientiae acquiritur, nam {\it habitus sunt similes actibus ex
quibus acquiruntur}, ut dicitur in II {\it Ethic}. Actus autem
intellectus ex quibus in praesenti vita scientia acquiritur, sunt per
conversionem intellectus ad phantasmata, quae sunt in praedictis viribus
sensitivis.


&

しかし、『デ・アニマ』第3巻で述べられるように、知性は「形象の場所」である
 から、知は知性においてある。したがって、この世で獲得された知の習態は、一
 部は前述の感覚能力の中にあり、一部は知性自身の中にある。そしてこのことは、
 『ニコマコス倫理学』第2巻で「習態は、そこから習態が獲得されるところの働
 きに似ている」ので、そこから知の習態が獲得されるところの働きに基づいて考
 察されうる。ところで、現在の生において、知は知性の働きから獲得されるが、
 その知性の働きは、知性が表象像の方へ向くことによってある。そして、この表
 象像は、前述の感覚能力の中にある。

\\


 Unde per tales actus et ipsi intellectui possibili acquiritur facultas
quaedam ad considerandum per species susceptas; et in praedictis
inferioribus viribus acquiritur quaedam
habilitas
%\footnote{h\u{a}b\u{\i}l\u{\i}tas, \={a}tis, f.~habilis,
%I.aptitude, ability:}
 ut facilius per conversionem ad ipsas intellectus
possit intelligibilia speculari. Sed sicut actus intellectus
principaliter quidem et formaliter est in ipso intellectu, materialiter
autem et dispositive in inferioribus viribus, idem etiam dicendum est de
habitu.

& したがって、このような働きによって、可能知性には、
受け取った形象によって考察する働きが獲得され、また、前述の下位の能力にお
いては、知性がそれらの方を向くことによって、より容易に、可知的なものが観
照されうるようになる、一種の適性(habilitas)が獲得される。しかし、ちょうど、
知性の働きは、主要かつ形相的に、知性そのものの中にあるように、質料的、態
勢的に、下位の能力の中にある。同じことが、習態についても言われるべきであ
る。



\\


Quantum ergo ad
id quod aliquis praesentis scientiae habet in inferioribus viribus, non
remanebit in anima separata, sed quantum ad id quod habet in ipso
intellectu, necesse est ut remaneat. Quia, ut dicitur in libro de
{\it Longitudine et Brevitate Vitae}, dupliciter corrumpitur aliqua forma, uno
modo, per se, quando corrumpitur a suo contrario, ut calidum a frigido;
alio modo, per accidens, scilicet per corruptionem subiecti. 
&

ゆえに、ある人が下位の能力の中にもっている、現在の知の部分に関しては、分
 離された魂の中に残らないだろうが、知性自身の中に持っているものに関して
 は、それが残ることが必然である。
なぜなら、『生の長さと短さについて』という本\footnote{アリストテレスの
 著作。}で言われているように、ある
 形相は二通りのしかたで消滅する。一つは、自体的にであり、たとえば、熱い
 ものが冷たいものによって消滅するように、自らに反対のものによって消滅す
 る場合である。もう一つは、附帯的に、すなわち、基体が消滅することによっ
 てである。
%\footnote
%{形相の消滅について語る興味深いテキスト。ここでは消滅のしかたが二つ挙げ
% られているが、三通りに区別されることもある。cf. {\it SCG} II, c.~79, n.~10. ``Item. Nulla forma corrumpitur nisi vel ex actione
% contrarii, vel per corruptionem sui subiecti, vel per defectum suae
% causae: per actionem quidem contrarii, sicut calor destruitur per
% actionem frigidi; per corruptionem autem sui subiecti, sicut, destructo
% oculo destruitur vis visiva; per defectum autem causae, sicut lumen
% aeris deficit deficiente solis praesentia, quae erat ipsius causa. Sed
% anima humana non potest corrumpi per actionem contrarii: non est enim
% ei aliquid contrarium; cum per intellectum possibilem ipsa sit
% cognoscitiva et receptiva omnium contrariorum.''
%}

\\



Manifestum
est autem quod per corruptionem subiecti, scientia quae est in
intellectu humano, corrumpi non potest, cum intellectus sit
incorruptibilis, ut supra ostensum est. Similiter etiam nec per
contrarium corrumpi possunt species intelligibiles quae sunt in
intellectu possibili, quia intentioni intelligibili nihil est
contrarium; et praecipue quantum ad simplicem intelligentiam, qua
intelligitur quod quid est. 

&

しかし、基体が消滅することによって、人間知性の中にある知が滅びることがあ
 りえないのは明らかである。なぜなら、知性は、上で示された通り、不滅だか
 らである。
同様に、可能知性の中にある可知的形象が、反対のものによって消滅することも
 ありえない。なぜなら、可知的なインテンチオには、何も反対のものがないか
 らである。とくに、単純な知性認識、すなわち、それによって何であるかが知
 性認識されるところのものは、反対物をもたない。

\\

Sed quantum ad operationem qua intellectus
componit et dividit, vel etiam ratiocinatur, sic invenitur contrarietas
in intellectu, secundum quod falsum in propositione vel in
argumentatione est contrarium vero. Et hoc modo interdum scientia
corrumpitur per contrarium, dum scilicet aliquis per falsam
argumentationem abducitur a scientia veritatis. 


&

しかし、知性が複合分割する働きにかんして、あるいは、推論の働きにかん
しては、命題や推論の中の偽が真に反対するかぎりで、知性において反対性が見
 出される。
そしてこのしかたで、知が、反対のものによって消滅することがある。すなわち、
 ある人が、偽の推論によって、真理に属する知から離されていくように。


\\


Et ideo philosophus, in
libro praedicto, ponit duos modos quibus scientia per se corrumpitur,
scilicet oblivionem, ex parte memorativae, et deceptionem, ex parte
argumentationis falsae. Sed hoc non habet locum in anima separata. Unde
dicendum est quod habitus scientiae, secundum quod est in intellectu
manet in anima separata.

&

それゆえ、哲学者は、前述の書物で、知が自体的に消滅する二つのかたちを述べ
 ている。すなわち、記憶力の側からは忘却であり、偽の推論の側からは、欺瞞
 である。
しかし、このようなことは、分離した魂にはない。
したがって、知の習態は、それが知性においてある限り、分離した魂の中に留ま
 る。

\\


{\sc Ad primum ergo dicendum} quod apostolus non
loquitur ibi de scientia quantum ad habitum, sed quantum ad cognitionis
actum. Unde ad huius probationem inducit, {\it Nunc cognosco ex parte}.


&

第一に対して、それゆえ、次のように言われるべきである。
使徒はそこで、習態にかんする限りで知について述べているのでなく、認識の働
 きにかんして述べている。したがって、この証明のために「いま私は部分に基づ
 いて認識する」ということを引いている。

\\


{\sc Ad secundum dicendum} quod, sicut secundum
staturam
%\footnote{st\u{a}t\={u}ra , ae, f.~status, from sto; prop.~a
%standing upright, an upright posture; hence, I.height or size of the
%body, stature.}
 corporis aliquis minus bonus erit maior aliquo magis
bono; ita nihil prohibet aliquem minus bonum habere aliquem scientiae
habitum in futuro, quem non habet aliquis magis bonus. Sed tamen hoc
quasi nullius momenti est in comparatione ad alias praerogativas quas
meliores habebunt.


&

第二に対しては、次のように言われるべきである。
ちょうど、より少なく善である人が、より多く善である人よりも、背が高いこと
 があるように、より少なく善である人が、より多く善である人がもたない知の
 習態を将来においてもつとしても、何ら差し支えはない。
しかし、これは、より善い人たちがもつであろう他の特権に比べれば、いわばまっ
 たく無意味なことである。

\\


Ad tertium dicendum quod utraque scientia
non est unius rationis. Unde nullum inconveniens sequitur.


&

第三に対しては、次のように言われるべきである。
両方の知が、ひとつのratioに属するわけではない。ゆえに、なんの不都合も生
 じない。

\\


{\sc Ad quartum dicendum} quod ratio illa procedit
de corruptione scientiae quantum ad id quod habet ex parte sensitivarum
virium.

&

かの異論は、感覚能力の側からもつ知の消滅について論じている。



\end{longtable}


\newpage
\rhead{a.~6}

\begin{center}
 {\Large {\bf ARTICULUS SEXTUS}}\\
 {\large UTRUM ACTUS SCIENTIAE HIC ACQUISITAE MANEAT IN ANIMA SEPARATA}\\
 {\footnotesize III {\it Sent.}, d.~31, q.~2, a.~4; IV, d.~1, q.~1,
 a.~2.}\\
 {\Large 第6項\\この世で獲得された知の働きが分離した魂の中に残るか}
\end{center}

\begin{longtable}{p{21em}p{21em}}

{\huge A}{\sc d sextum sic proceditur}. Videtur quod actus scientiae hic
 acquisitae non maneat in anima separata. Dicit enim philosophus, in I
 {\it de Anima}, quod corrupto corpore, anima {\it neque reminiscitur
 neque amat}. Sed considerare ea quae prius aliquis novit, est
 reminisci. Ergo anima separata non potest habere actum scientiae quam
 hic acquisivit.

&

第6に対しては、次のように進められる。この世で獲得された知の働きは、分離し
 た魂の中に残らないと思われる。なぜなら、哲学者は『デ・アニマ』第1巻
 \footnote{Cap.~4, n.~14. S.Th.~lect.~10.}で、身体が消滅すると、魂は「思
 い出すことも、愛することもない」と述べている。しかるに、思い出すこととは、
 ある人が、以前に知ったことを考察することである。ゆえに、分離した魂は、こ
 の世で獲得した知の働きをもつことができないと思われる。

\\

2. {\sc Praeterea}, species intelligibiles non
 erunt potentiores in anima separata quam sint in anima corpori
 unita. Sed per species intelligibiles non possumus modo intelligere,
 nisi convertendo nos super phantasmata, sicut supra habitum est. Ergo
 nec anima separata hoc poterit. Et ita nullo modo per species
 intelligibiles hic acquisitas anima separata intelligere poterit.

& さらに、可知的形象は、身体に合一された魂の内にあるときよりも、分離した
魂の内にあるときの方が、力が落ちるであろう。しかるに、上で述べられた通り、
いま、私たちは、私たちを表象像へと向けることなしに、知性認識することがで
きない。
ゆえに、分離した魂も、これ[私たちを表象像へと向けることなしに知性認識
 すること]ができないであろう。
したがって、この世で獲得された可知的形象によって、分離した魂が知性認識す
 ることは、どのようなかたちによっても不可能であろう。

\\

3. {\sc Praeterea}, philosophus dicit, in II {\it Ethic}., quod {\it
 habitus similes actus reddunt actibus per quos acquiruntur}. Sed
 habitus scientiae hic acquiritur per actus intellectus convertentis se
 supra phantasmata. Ergo non potest alios actus reddere. Sed tales actus
 non competunt animae separatae. Ergo anima separata non habebit aliquem
 actum scientiae hic acquisitae.

& さらに、哲学者は『ニコマコス倫理学』第2巻で、「習態は、それによってその
習態が獲得されたところの働きに似た働きを生み出す」と述べる。しかるに、知
の習態は、この世で、自らを表象像へ向ける知性の働きによって獲得される。ゆ
えに、別の働きを与えることができない。しかし、このような働きは、分離した
 魂には適合しない。ゆえに、分離した魂は、この世で獲得されたいかなる知の
 働きももたないであろう。



\\

{\it Sed contra est quod} Luc.~{\sc xvi}, dicitur ad
 divitem in Inferno positum, {\it Recordare quia recepisti bona in vita tua}.

&
しかし反対に、『ルカによる福音書』16章で、地獄に置かれた金持ちに対し、
 「あなたの人生で、善いものどもを受け取ったことを思い出せ」
 \footnote{「子よ、思い出してみるがよい。おまえは生きているあいだに良い
 ものをもらっていたが、」(16:25)}と言われている。

\\


{\sc Respondeo dicendum} quod in actu est duo considerare, scilicet
 speciem actus, et modum ipsius. Et species quidem actus consideratur ex
 obiecto in quod actus cognoscitivae virtutis
 dirigitur
%\footnote{d\={\i}-r\u{\i}go or d\={e}r\u{\i}go (the latter
% form preferred by Roby, L.~G.~2, p.~387; cf.~Rib.~Proleg.~ad
% Verg.~p.~401 sq.; so Liv.~21, 19, 1; 21, 47, 8; 22, 28 Weissenb.;
% id.~22, 47, 2 Drak.; Lach.~ad Lucr.~4, 609; Tac.~A.~6, 40 Ritter;
% acc.~to Brambach, s.~v., the two forms are different words, de-rigo
% meaning I.to give a particular direction to; di-rigo, to arrange in
% distinct lines, set or move different ways; cf.~describo and
% discribo.~But the distinction is not observed in the MSS.~and
% edd.~generally), rexi, rectum, 3 (perf.~sync.~direxti, Verg.~A.~6, 57),
% v.~a.~dis-rego, to lay straight, set in a straight line, to arrange,
% draw up (class.; cf.: guberno, collineo, teneo).}
 per speciem, quae
 est obiecti similitudo, sed modus actus pensatur
%\footnote
%{penso , \={a}vi, \={a}tum, 1, v.~freq.~a.~pendo,
%I.to weigh or weigh out carefully (not freq.~till after the Aug.~per.; not in Cic., for in Off.~2, 19, 68, conpensandum is the correct reading.~Neither is it found in Plaut., Ter., Lucr., or Cæs.; syn.: pendo, expendo).
%I.~Lit.:
%II.~Transf.
%A.~To counterbalance with any thing, to compensate, recompense, repay, make good, requite; for the usual compensare:
%B.~To pay, repay, punish with any thing:
%2.~To purchase with any thing:
%C.~To exchange for any thing:
%D.~To allay, quench:
%E.~To weigh, ponder, examine, consider: 
%}
 ex virtute
 agentis. Sicut quod aliquis videat lapidem, contingit ex specie lapidis
 quae est in oculo, sed quod acute videat, contingit ex virtute visiva
 oculi.


&
答えて言わなければならない。
働きにおいて、二つのことを考察することができる。すなわち、働きの形象と、働
 きのあり方である。
働きの形象は、認識能力の働きが、形象によってそちらへと向けられている対象か
 ら考察される。この形象は、対象の類似である。
他方、働きのあり方は、働くものの力から測られる。
たとえば、ある人が石を見ることは、目の中にある石の形象から生じるが、しか
 し、鋭くそれを見ることは、目の見る力から生じる。


\\



Cum igitur species intelligibiles
 maneant in anima separata, sicut dictum est; status autem animae
 separatae non sit idem sicut modo est, sequitur quod secundum species
 intelligibiles hic acquisitas, anima separata intelligere possit quae
 prius intellexit; non tamen eodem modo, scilicet per conversionem ad
 phantasmata, sed per modum convenientem animae separatae. Et ita manet
 quidem in anima separata actus scientiae hic acquisitae, sed non
 secundum eundem modum.


&

ゆえに、すでに述べられた通り、可知的形象が分離した魂の中に残り、そして、
 分離した魂の状態は、今のそれと同じではないのだから、分離した魂は、この世
 で獲得された可知的形象において、以前に知性認識したことを知性認識すること
 ができるということ、しかし、同じしかたによって、すなわち、表象像の方を向
 くことによってではなく、分離した魂に適合するしかたによってである、という
 ことが帰結する。
そして、このようにして、分離した魂の中には、この世で獲得された知の働きが
 残るが、しかし、[今と]同じあり方においてではない。

\\



{\sc Ad primum ergo dicendum} quod philosophus
 loquitur de reminiscentia, secundum quod memoria pertinet ad partem
 sensitivam, non autem secundum quod memoria est quodammodo in
 intellectu, ut dictum est.

&
第一に対しては、それゆえ、次のように言われるべきである。
哲学者は、記憶が感覚能力の側に属するかぎりで、記憶力について述べているの
 であり、すでに述べられた通り、記憶があるかたちで知性の中にあるかぎりで
 述べているのではない。


\\


{\sc Ad secundum dicendum} quod diversus modus
 intelligendi non provenit ex diversa virtute specierum, sed ex diverso
 statu animae intelligentis.

&

知性認識のさまざまなあり方は、形象のさまざまな力から由来するのではなく、
 知性認識する者の魂のさまざまな状態に由来する。

\\



{\sc Ad tertium dicendum} quod actus per quos
 acquiritur habitus, sunt similes actibus quos habitus causant, quantum
 ad speciem actus, non autem quantum ad modum agendi. Nam operari iusta,
 sed non iuste, idest delectabiliter, causat habitum iustitiae
 politicae, per quem delectabiliter operamur.

&

第三に対しては、次のように言われるべきである。ある習態が、ある働きによっ
て得られたとき、その働きが、その習態が原因となって生じた働きに似ているの
は、働きの形象においてであり、働きの様態[働き方]においてではない。たと
えば、正しいことを行いながら、正しく行わない、すなわち、楽しみのために行
うことは、政治的な正義の習態の原因となる。私たちは、この習態によって、楽
しみのために働く。\footnote{Cf.~Arist.~{\it Ethic.}, lib.~5, c.~8,
n.~1. (s.Th.~lect. 13); {\it Magn.Moral}., lib.~1, cap.~34, n.~23.}


\end{longtable}

\newpage
\rhead{a.~7}

\begin{center}
 {\Large {\bf ARTICULUS SEPTIMUS}}\\
 {\large UTRUM DISTANTIA LOCALIS IMPEDIAT COGNITIONEM ANIMAE SEPARATAE}\\
 {\footnotesize IV {\it Sent.}, d.~1, q.~1, a.~4.}\\
 {\Large 第7項\\場所の隔たりが分離した魂の認識を妨げるか}
\end{center}

\begin{longtable}{p{21em}p{21em}}

{\huge A}{\sc d septimum sic proceditur}. Videtur quod distantia localis
 impediat cognitionem animae separatae. Dicit enim Augustinus, in libro
 {\it de Cura pro Mortuis agenda}, quod {\it animae mortuorum ibi sunt,
 ubi ea quae hic fiunt scire non possunt}. Sciunt autem ea quae apud eos
 aguntur. Ergo distantia localis impedit cognitionem animae separatae.

&

第7に対しては、次のように進められる。
場所の隔たりは、分離した魂の認識を妨げると思われる。なぜなら、
アウグスティヌスは、『死者たちに払われるべき配慮について』で、「死者たち
 の魂は、この世で為されることを知ることができないところにいる」と述べて
 いる。しかるに、彼らは、彼らのもとで為されることを知る。ゆえに、場所の
 隔たりは、分離した魂の認識を妨げると思われる。

\\



2. {\sc Praeterea}, Augustinus dicit, in libro {\it de Divinatione
 Daemonum}, quod {\it daemones, propter celeritatem motus, aliqua nobis
 ignota denuntiant}. Sed agilitas motus ad hoc nihil faceret, si
 distantia localis cognitionem Daemonis non impediret. Multo igitur
 magis distantia localis impedit cognitionem animae separatae, quae est
 inferior secundum naturam quam Daemon.

&

さらに、アウグスティヌスは『悪魔の予知について』で「悪魔たちは、運動の速
 さのために、私たちが知らないことを知らせる」と述べている。しかし、もし、
 場所の隔たりが悪魔の認識を妨げるのでなかったら、運動の速さは、このため
 [=私たちが知らないことを知らせるため]に、何もなさなかったであろう。ゆえ
 に、ましてや、場所の隔たりは分離した魂の認識を妨げる。それは、本性におい
 て、悪魔よりも下なのだから。

\\



3. {\sc Praeterea}, sicut distat aliquis secundum
 locum, ita secundum tempus sed distantia temporis impedit cognitionem
 animae separatae, non enim cognoscunt futura. Ergo videtur quod etiam
 distantia secundum locum animae separatae cognitionem impediat.

&
さらに、ある人は、場所において隔たるように、時間において隔たる。しかるに、
 時間の隔たりは、分離した魂の認識を妨げる。なぜなら、未来を認識しないか
 ら。ゆえに、場所における隔たりもまた、分離した魂の認識を妨げると思われ
 る。

\\



{\sc Sed contra est} quod dicitur Luc.~{\sc xvi}, quod
 {\it dives cum esset in tormentis, elevans oculos suos, vidit Abraham a
 longe}. Ergo distantia localis non impedit animae separatae cognitionem.

& しかし反対に、『ルカによる福音書』16章で、「金持ちたちが、責め苦の中に
いたとき、自らの目を上げ、アブラハムを遠くから見た」\footnote{「そして、
金持ちは陰府でさいなまれながら目を上げると、宴席でアブラハムとそのすぐそ
ばにいるラザロとが、はるかかなたに見えた。」(16:23)}と言われている。ゆえ
に、場所の隔たりは、分離した魂の認識を妨げない。

\\



{\sc Respondeo dicendum} quod quidam posuerunt quod
 anima separata cognosceret singularia abstrahendo a sensibilibus. Quod
 si esset verum, posset dici quod distantia localis impediret animae
 separatae cognitionem, requireretur enim quod vel sensibilia agerent in
 animam separatam, vel anima separata in sensibilia; et quantum ad
 utrumque, requireretur distantia determinata. 


&
答えて言わなければならない。
ある人々は、分離した魂が、個物を、可感的なものどもから抽象することによっ
 て認識すると考えた。かりにこれが真であったとしたら、場所の隔たりが分離
 した魂の認識を妨げると言われることができたであろう。なぜなら、その場合
 には、可感的なものが分離した魂へ働きかけることか、または、分離した魂が
 可感的なものへ働きかけることが要求されただろうから。


\\
Sed praedicta positio est
 impossibilis, quia abstractio specierum a sensibilibus fit mediantibus
 sensibus et aliis potentiis sensitivis, quae in anima separata actu non
 manent. Intelligit autem anima separata singularia per influxum
 specierum ex divino lumine, quod quidem lumen aequaliter se habet ad
 propinquum et distans. Unde distantia localis nullo modo impedit animae
 separatae cognitionem.

&
しかし、このような立場は不可能である。なぜなら、可感的なものから形象を
 抽象することは、感覚その他の感覚能力の媒介によって為されるが、こ
 れら[=感覚能力]は、分離した魂の中に、現実態において、留まっていない
 からである。しかし、分離した魂は、神の光から形象が流入することに
 よって、個物を認識するが、この光は、近くにあるものと離れているものとに、
 等しく関係する。従って、場所の隔たりは、どのようなかたちでも、分離した
 魂の認識を妨げない。

\\


{\sc Ad primum ergo dicendum} quod Augustinus non
 dicit quod propter hoc quod ibi sunt animae mortuorum, ea quae hic sunt
 videre non possunt, ut localis distantia huius ignorantiae causa esse
 credatur, sed hoc potest propter aliquid aliud contingere, ut infra
 dicetur.


&

第一に対しては、それゆえ、次のように言われるべきである。アウグスティヌス
は、場所の隔たりがこの無知の原因であると信じられるようなかたちで、死者た
ちの魂が、「そこに在る」ということのために、ここに在るものどもを見ること
ができないと言っているのではない。そうではなく、こういったことは、下で
\footnote{次の項。}述べられるように、何か別のことのために生じうる

\\



{\sc Ad secundum dicendum} quod Augustinus ibi
 loquitur secundum opinionem illam qua aliqui posuerunt quod Daemones
 habent corpora naturaliter sibi unita, secundum quam positionem, etiam
 potentias sensitivas habere possunt, ad quarum cognitionem requiritur
 determinata distantia. Et hanc opinionem etiam in eodem libro
 Augustinus expresse tangit, licet eam magis recitando quam asserendo
 tangere videatur, ut patet per ea quae dicit XXI libro {\it de Civ.~Dei}.

&
第二に対しては、次のように言われるべきである。
アウグスティヌスはそこで、悪魔たちが、自らに合一した身体を本性的にもって
 いると考えた人たちの意見に沿って論じている。
この考えによれば、悪魔らは感覚能力も持つことができるし、それらの能力が認
 識を行うには、一定の距離が必要とされる。
そして、この意見を、アウグスティヌスは同じ書物の中でもはっきりと扱ってい
 る。ただし、『神の国』第21巻で語ることを通して明らかな通り、主張するよ
 うにではなく、引用するように、扱っている。


\\


{\sc Ad tertium dicendum} quod futura, quae
 distant secundum tempus, non sunt entia in actu. Unde in seipsis non
 sunt cognoscibilia, quia sicut deficit aliquid ab entitate, ita deficit
 a cognoscibilitate. Sed ea quae sunt distantia secundum locum, sunt
 entia in actu, et secundum se cognoscibilia. Unde non est eadem ratio
 de distantia locali, et de distantia temporis.


&

第三に対しては、次のように言われるべきである。
未来は、時間において隔たっているが、現実態における有ではない。
したがって、それ自身において認識されうるものではない。
なぜなら、有であることentitasから欠ける程度に、認識されうるものであるこ
 とからも欠けるからである。
これに対して、場所において隔たっているものは、現実態における有であり、そ
 れ自体において認識されうるものである。
したがって、場所的な隔たりと、時間的な隔たりについて、同じ議論はできない。



\end{longtable}

\newpage
\rhead{a.~8}

\begin{center}
 {\Large {\bf ARTICULUS OCTAVUS}}\\
 {\large UTRUM ANIMAE SEPARATAE COGNOSCAT EA QUAE HIC AGUNTUR}\\
 {\footnotesize {\it ST} II-II, q.~83, a.~4, ad 2; IV {\it Sent.},
 d.~45, q.~3, a.~1, ad 1, 2; d.~1, q.~1, a.~4, ad 1; {\it De Verit.},
 q.~8, a.~2, ad 12; q.~9, a.~6, ad 5; Qu.~{\it de Anima.}, a.~20, ad 3.}\\
 {\Large 第8項\\分離した魂は、この世で為されていることを認識するか}
\end{center}

\begin{longtable}{p{21em}p{21em}}


{\huge A}{\sc d octavum sic proceditur}. Videtur quod animae separatae
 cognoscant ea quae hic aguntur. Nisi enim ea cognoscerent, de eis curam
 non haberent. Sed habent curam de his quae hic aguntur; secundum illud
 Luc. {\sc xvi}: {\it Habeo quinque fratres, ut testificetur illis, ne
 et ipsi veniant in hunc locum tormentorum}. Ergo animae separatae
 cognoscunt ea quae hic aguntur.

&
第8に対しては、次のように進められる。
分離した魂は、この世で為されることを認識すると思われる。なぜなら、
もしそういった事柄を認識しなかったら、それらについて心配することもなかっ
 ただろう。しかし、『ルカによる福音書』第16章「私には5人の兄弟がいます。
 彼らもまた、この責め苦の場所に来ることがないように、証言してください」
 \footnote{「わたしには兄弟が五人います。あの者たちまで、こんな苦しい場
 所に来ることのないように、よく言い聞かせてください。」(16:28)}によれば、
 この世で為されている事柄について心配をしている。ゆえに、分離した魂は、
 この世で為されていることを認識する。

\\




2.{\sc Praeterea}, frequenter mortui vivis
 apparent, vel dormientibus vel vigilantibus, et eos admonent de iis
 quae hic aguntur; sicut Samuel apparuit Sauli, ut habetur I
 {\it Reg}.~{\sc xxviii}. Sed hoc non esset si ea quae hic sunt non
 cognoscerent. Ergo ea quae hic aguntur cognoscunt.

&

さらに、たとえばサムエルがサウルに現れたと『サムエル記上』28章で述べられ
 ているように、しばしば、死者たちは、眠っているときであれ目覚めていると
 きであれ、生きている人々に現れ、この世で為されていることについて、彼ら
 に忠告する。しかし、もし、この世にある事柄を認識していなかったならば、
 このようなことはなかったであろう。ゆえに、この世で為されていることを認
 識している。

\\




3. {\sc Praeterea}, animae separatae cognoscunt ea
 quae apud eas aguntur. Si ergo quae apud nos aguntur non cognoscerent,
 impediretur earum cognitio per localem distantiam. Quod supra negatum
 est.

&
さらに、分離した魂は、彼らのもとで為されていることを認識している。ゆえに、
 もし、私たちのもとで為されていることを認識しなかったならば、彼らの認識
 は、場所の隔たりによって妨げらることになるが、これは、上で否定された。

\\




{\sc Sed contra est} quod dicitur {\it Iob} {\sc xiv} : {\it Sive fuerint
 filii eius nobiles, sive ignobiles, non intelliget}.

&
しかし反対に、『ヨブ記』14章「彼の息子たちが高貴になるか、高貴でなくなる
 か、彼は知性認識しないだろう」\footnote{「その子らが名誉を得ても、彼は
 知ることなく、彼らが不幸になっても、もう悟らない。
」(14:21)}と言われている。


\\


{\sc Respondeo dicendum} quod, secundum naturalem
 cognitionem, de qua nunc hic agitur, animae mortuorum nesciunt quae hic
 aguntur. Et huius ratio ex dictis accipi potest. Quia anima separata
 cognoscit singularia per hoc quod quodammodo determinata est ad illa,
 vel per vestigium alicuius praecedentis cognitionis seu affectionis,
 vel per ordinationem divinam. Animae autem mortuorum, secundum
 ordinationem divinam, et secundum modum essendi, segregatae sunt a
 conversatione viventium, et coniunctae conversationi spiritualium
 substantiarum quae sunt a corpore separatae. Unde ea quae apud nos
 aguntur ignorant. 

&
答えて言わなければならない。
今ここで論じられている、自然本性的な認識においては、死者たちの魂が、この
 世で為されることを認識することはない。この理由は、すでに述べられたこと
 から理解できる。
分離した魂は、なんらかの先行する認識や受動の痕跡によって、または、神の命
 令(秩序付け)によって、あるかたちでそれへと限定されることをとおして、個物を認識する。
しかるに、死者の魂は、神の命令においても、存在のあり方においても、生きて
 いる者たちとの会話か切り離され、また、身体から分離している霊的実体との
 会話に結びつけられている。
したがって、私たちのもとで為されていることを、彼らは知らない。


\\



Et hanc rationem assignat Gregorius in XII {\it Moralium}, dicens: {\it
 Mmortui vita in carne viventium post eos, qualiter disponatur,
 nesciunt, quia vita spiritus longe est a vita carnis; et sicut corporea
 atque incorporea diversa sunt genere, ita sunt distincta
 cognitione}. Et hoc etiam Augustinus videtur tangere in libro {\it de
 Cura pro Mortuis agenda}, dicens {\it quod animae mortuorum rebus
 viventium non intersunt}.
%\footnote
%{inter-sum , f\u{u}i, esse (interf\u{u}t\={u}rus, Cic.~Div.~in Caecil.~11, 35;
%I.“in tmesi: interque esse desiderat pugnis,” Arn.~7, 255), v.~n., to be between, lie between (class.; syn.~interjaceo).
%I.~In gen.
%A.~Of space:
%B.~Of time:
%II.~Transf.
%A.~To be apart; with abl.~of distance (syn.~disto):
%B.~To be different, to differ:
%C.~To be present at, take part in, attend; constr.~absol., with dat.~or in and abl.
%D.~\textcolor{red}{To interest, be of interest to one} (very rare as pers.~verb):
%}


&
そして、グレゴリウスは、『道徳論』第12巻で、次のように述べてその理由を指
 定している。「死者たちは、彼らのあとに肉において生きる者たちの生がどの
 ような状態にあるかを知らない。なぜなら、霊の生は、肉の生から遠く隔たる
 からである。ちょうど、物体的なものと非物体的なものとが類において異なる
 ように、認識においても区別される」。また、このことは、アウグスティヌス
 が『死者たちのために払われるべき配慮』という書物で「死者たちの魂は、生
 きている者たちの事柄に関心がない」と述べるときに触れているように思われ
 る。

\\


Sed
 quantum ad animas beatorum, videtur esse differentia inter Gregorium et
 Augustinum. Nam Gregorius ibidem subdit: {\it Quod tamen de animabus sanctis
 sentiendum non est, quia quae intus omnipotentis Dei claritatem vident,
 nullo modo credendum est quod sit foris aliquid quod
 ignorent}. -- Augustinus vero, in libro {\it de Cura pro Mortuis agenda},
 expresse dicit quod {\it nesciunt mortui, etiam sancti, quid agant vivi et
 eorum filii}, ut habetur in Glossa, super illud, {\it Abraham nescivit nos},
 Isaiae {\sc lxiii}. 

& しかし、至福者たちの魂にかんする限り、グレゴリウスとアウグスティヌスに
は違いがあるように思われる。すなわち、グレゴリウスは、同じ箇所で次のよう
に続けている。「しかしこのことが、聖なる魂について考えられるべきではない。
なぜなら、それら[=聖なる魂]は全能である神の光を見ているのだから、彼ら
の知らないなにかが外にあるとは、決して信じられるべきでないから」。これに
対して、アウグスティヌスは、『死者たちのために払われるべき配慮について』
という書物の中で、かの『イザヤ書』63章の「アブラハムは私たちを知らない」
\footnote{「あなたはわたしたちの父です。アブラハムがわたしたちを見知らず、
イスラエルがわたしたちを認めなくても、主よ、あなたはわたしたちの父です。」
(63:16)}を注解して「死者たちは、聖者であっても、生きている者たちとその子
孫たちが何をしているかを知らない」とはっきり述べている。



\\


Quod quidem confirmat per hoc quod a matre sua non visitabatur, nec in
 tristitiis consolabatur, sicut quando vivebat; nec est probabile ut sit
 facta vita feliciore crudelior. Et per hoc quod dominus promisit Iosiae
 regi quod prius moreretur ne videret mala quae erant populo
 superventura, ut habetur IV {\it Reg}.~{\sc xxii}. Sed Augustinus hoc
 dubitando dicit, unde praemittit, {\it ut volet, accipiat quisque quod
 dicam}. Gregorius autem assertive, quod patet per hoc quod dicit, {\it
 nullo modo credendum est}.


& 彼[=アウグスティヌス]はこれを、次のことによって、確かなものとしてい
る。すなわち、[母親が亡くなったあと]彼の母親が訪ねてきたことがないし、
生きていたときのように、悲しいときに慰めてくれたこともない。しかし、より
幸福な生において、[死んだ母が]より無慈悲になったとは考えられない。そし
て、このことによって、『列王記下』22章に書かれているように、主がヨシアの
王に、民に訪れるであろう害悪を見ないように、先に死ぬことを約束した。
\footnote{「それゆえ、見よ、わたしはあなたを先祖の数に加える。あなたは安
らかに息を引き取って墓に葬られるであろう。私がこの所にくだす災いのどれも、
その目で見ることがない。」(22:20)} しかし、アウグスティヌスは、これを疑い
ながら述べており、それゆえ「もしそうしたいなら、なんであれ私が言うことを
受け取りなさい」、と前で言う。これに対してグレゴリウスは、「決して信じら
れるべきでない」と述べることから明らかなとおり、はっきりと主張している。
\\




Magis tamen videtur, secundum sententiam Gregorii,
 quod animae sanctorum Deum videntes, omnia praesentia quae hic aguntur
 cognoscant. Sunt enim Angelis aequales, de quibus etiam Augustinus
 asserit quod ea quae apud vivos aguntur non ignorant. Sed quia
 sanctorum animae sunt perfectissime iustitiae divinae coniunctae, nec
 tristantur, nec rebus viventium se ingerunt, nisi secundum quod
 iustitiae divinae dispositio exigit.


&

しかし、グレゴリウスの議論に従って、神を見る聖者たちの魂が、この世で為さ
 れているすべての現在のことを認識するという方が、より確からしいと思われ
 る。なぜなら、彼らは天使に等しいのであり、天使たちについてはアウグスティヌ
 スもまた、生きている者たちのもとで為されることについて知らないことがな
 いと述べている。
しかし、聖者たちの魂は、この上なく完全に、神の正義に結びつけられているの
 で、神の正義の状態が必要とするのでない限り、悲しむことも、生きている者
 たちの事柄に介入することもない。

\\




{\sc Ad primum ergo dicendum} quod animae
 mortuorum possunt habere curam de rebus viventium, etiam si ignorent
 eorum statum; sicut nos curam habemus de mortuis, eis suffragia
 impendendo, quamvis eorum statum ignoremus. Possunt etiam facta
 viventium non per seipsos cognoscere, sed vel per animas eorum qui hinc
 ad eos accedunt; vel per Angelos seu Daemones; vel etiam {\it spiritu Dei
 revelante}, sicut Augustinus in eodem libro dicit.

&

第一に対しては、それゆえ、次のように言われるべきである。死者たちの魂は、
生きている者たちの事柄について、かりにそれらの状態を知らなくても、心配す
ることができる。ちょうど、わたしたちが、死者たちの状態を知らなくても、彼
らに取りなしの祈りを捧げることで、死者たちについて心配するように。また、
生きている者たちが為したことを、それ自体によってではなく、この世から彼ら
のもとへとやって来る者たちの魂を通して、あるいは、天使や悪魔によって、あ
るいは、アウグスティヌスが同じ書物で述べるように「神の啓示する霊によっ
て」、認識できる。

\\




{\sc Ad secundum dicendum} quod hoc quod mortui viventibus apparent
 qualitercumque, vel contingit per specialem Dei dispensationem, ut
 animae mortuorum rebus viventium intersint, et est inter divina
 miracula computandum. 
Vel huiusmodi apparitiones fiunt per operationes
 Angelorum bonorum vel malorum, etiam ignorantibus mortuis, sicut etiam
 vivi ignorantes aliis viventibus apparent in somnis, ut Augustinus
 dicit in libro praedicto. 
Unde et de Samuele dici potest quod ipse
 apparuit per revelationem divinam; secundum hoc quod dicitur {\it
 Eccli}.~{\sc xlvi}, quod {\it dormivit, et notum fecit regi finem vitae
 suae}. 
Vel illa apparitio fuit procurata per Daemones, si tamen
 Ecclesiastici auctoritas non recipiatur, propter hoc quod inter
 canonicas Scripturas apud Hebraeos non habetur.

&
第二に対しては、次のように言われるべきである。
どのようなかたちであれ、死者たちが生きている者たちに現れることや、あるい
 は、神の特別な計らいによって、死者たちの魂が、生きている者たちの事柄に
 現れることは、神の奇跡に数えられるべきである。
あるいは、アウグスティヌスが上述の本で述べているように、ちょうど、生きて
 いる人でも、知らないうちに、夢の中で、他の生きている人に現れるように、こ
 のような出現は、良い、あるいは悪い天使の働きによって、死者たちが知らな
 いままに生じている。
したがって、かの『知恵の書』46節「彼は眠った。そして、自分の人生の目的を
 王に知らせた」によれば、サミュエルについても、彼が、神の啓示によって出
 現したと言われることができる。
あるいは、かの出現は、悪魔たちによって行われた[とも解釈できる]。かりに、
 ヘブライ人たちのもとで、聖典の中に入っていないため、『知恵の書』の権威
 が受け取られないとしても。


\\




{\sc Ad tertium dicendum} quod ignorantia
 huiusmodi non contingit ex locali distantia, sed propter causam
 praedictam.


&

第三に対しては、次のように言われるべきである。
そのような無知は、場所の隔たりによってではなく、上述の原因のために生じる。



\end{longtable}

\end{document}