\documentclass[10pt]{jsarticle} % use larger type; default would be 10pt
%\usepackage[utf8]{inputenc} % set input encoding (not needed with XeLaTeX)
%\usepackage[round,comma,authoryear]{natbib}
%\usepackage{nruby}
\usepackage{okumacro}
\usepackage{longtable}
%\usepqckage{tablefootnote}
\usepackage[polutonikogreek,english,japanese]{babel}
%\usepackage{amsmath}
\usepackage{latexsym}
\usepackage{color}

%----- header -------
\usepackage{fancyhdr}
\lhead{{\it Summa Theologiae} I, q.~67}
%--------------------

\bibliographystyle{jplain}

\title{{\bf PRIMA PARS}\\{\HUGE Summae Theologiae}\\Sancti Thomae
Aquinatis\\{\sffamily QUEAESTIO SEXAGESIMASEPTIMA}\\DE OPERE
DISTINCTIONIS SECUNDUM SE}
\author{Japanese translation\\by Yoshinori {\sc Ueeda}}
\date{Last modified \today}


%%%% コピペ用
%\rhead{a.~}
%\begin{center}
% {\Large {\bf }}\\
% {\large }\\
% {\footnotesize }\\
% {\Large \\}
%\end{center}
%
%\begin{longtable}{p{21em}p{21em}}
%
%&
%
%
%\\
%\end{longtable}
%\newpage



\begin{document}
\maketitle
\pagestyle{fancy}

\begin{center}
{\Large 第六十七問\\区別の業それ自体について}
\end{center}

\begin{longtable}{p{21em}p{21em}}
{\scshape Consequenter} considerandum est de opere distinctionis
 secundum se et primo, de opere primae diei; secundo, de opere secundae
 diei; tertio, de opere tertiae. Circa primum quaeruntur quatuor. 

\begin{enumerate}
 \item utrum lux proprie in spiritualibus dici possit.
 \item utrum lux corporalis sit corpus.
 \item utrum sit qualitas.
 \item utrum conveniens fuit prima die fieri lucem.
\end{enumerate}

&

続いて、区別の業それ自体について考察されるべきである。第一に、第一日目の
 業について、第二に、二日目の業について、第三に、三日目の業について。第
 一にかんして、四つのことが問われる。

\begin{enumerate}
 \item 光は固有に霊的諸事物の中で語られうるか。
 \item 物体的光は物体か。
 \item それは性質か。
 \item 第一日目に光が作られることは適切だったか。
\end{enumerate}


\end{longtable}
\newpage

\rhead{a.~1}
\begin{center}
 {\Large {\bf ARTICULUS PRIMUS}}\\
 {\large UTRUM LUX PROPRIE IN SPIRITUALIBUS DICATUR}\\
 {\footnotesize II {\itshape Sent.}, d.~13, a.~1; {\itshape In Ioan.},
 cap.~1, lect.~3.}\\
 {\Large 第一項\\光は固有に霊的諸事物の中で語られるか}
\end{center}

\begin{longtable}{p{21em}p{21em}}




{\huge A}{\scshape d primum sic proceditur}. Videtur quod lux proprie in
spiritualibus dicatur. Dicit enim Augustinus, IV {\itshape super Gen.~ad
Litt}., quod in spiritualibus {\itshape melior et certior lux est}, et
quod {\itshape Christus non sic dicitur lux quo modo lapis, sed illud
proprie, hoc figurative}.

&


第一項の問題へ、議論は以下のように進められる。
光は、固有に、霊的諸事物の中で語られると思われる。理由は以下の通り。
アウグスティヌスは『創世記逐語注解』4巻で、霊的諸事物の中で、「より善くよ
 り確かなのは光である」と述べ、また「キリストは「石」と言われるのと同じ
 しかたで「光」と言われるのではない。「光」は固有に言われるが、「石」は
 比喩的に言われる」と述べている。

\\


{\scshape 2 Praeterea}, Dionysius, {\scshape iv} cap.~{\itshape de
Div.~Nom}., ponit {\itshape Lumen} inter nomina intelligibilia
Dei. Nomina autem intelligibilia proprie dicuntur in spiritualibus. Ergo
lux proprie dicitur in spiritualibus.

&

さらに、ディオニュシウスは『神名論』4章で、「光」を、神の可知的な名前の
 一つに数えている。ところが、可知的な名前は、霊的諸事物の中で固有に語ら
 れる。ゆえに、光は、固有に、霊的諸事物の中で語られる。

\\


{\scshape 3 Praeterea}, apostolus dicit, {\itshape ad Ephes}.~{\scshape
v}, {\itshape omne quod manifestatur, lumen est}. Sed manifestatio magis
proprie est in spiritualibus quam in corporalibus. Ergo et lux.

&

さらに、使徒は『エフェソの信徒への手紙』5章で「すべて明らかにされるもの
 は光である」\footnote{「明らかにされるものはみな、光となるのです。」(3:14)}と述べている。ところが、「明らかにすること」は、物体的諸事
 物よりも、霊的諸事物の中に、より固有である。ゆえに光も云々。


\\


{\scshape Sed contra est} quod Ambrosius, in libro {\itshape de Fide},
ponit {\itshape splendorem} inter ea quae de Deo metaphorice dicuntur.

&

しかし反対に、アンブロシウスは『信仰について』という書物の中で、神につい
 て比喩的に語られるものどもの中に「輝き」を置いている。

\\


{\scshape Respondeo dicendum} quod de aliquo nomine dupliciter convenit
loqui, uno modo, secundum primam eius impositionem; alio modo, secundum
usum nominis. Sicut patet in nomine visionis, quod primo impositum est
ad significandum actum sensus visus; sed propter dignitatem et
certitudinem huius sensus, extensum est hoc nomen, secundum usum
loquentium, ad omnem cognitionem aliorum sensuum (dicimus enim, {\itshape vide
quomodo sapit}, vel {\itshape quomodo redolet}, vel {\itshape quomodo est calidum}); et
ulterius etiam ad cognitionem intellectus, secundum illud Matth.~{\scshape v},
{\itshape beati mundo corde\footnote{ ``mundo corde''は、性質の奪格。
 「人々」に当たる言葉(主格)が省略されている。 }, quoniam ipsi Deum videbunt}. 


&


解答する。以下のように言われるべきである。
任意の名前について、二通りのかたちで適切に語ることができる。一つには、そ
 の最初の名付けにしたがってであり、もう一つは、その名前の使用にしたがっ
 てである。たとえばそれは、「見」という名前において、最初は視覚の作用を
 表示するために名付けられたが、この感覚の品格と確実さのために、語る人々
 の使用にしたがって、他の感覚のすべての認識へと名前が拡張されていること
 が明らかなようにである(じっさい、私たちは「どんな味がするか見よ」、
 「どのような香りがするか見よ」、「どれだけ熱いか見よ」と言う)。そして
 さらに、かの『マタイによる福音書』5章「幸いである、清らかな心の人々は。彼らは
 神を見るだろうから」\footnote{「心の清い人々は、幸いである。その人たち
 は神を見る。」(5:8)}によれば、知性の認識にまで拡張されている。



\\


Et similiter dicendum est de nomine {\itshape lucis}. Nam primo quidem
est institutum ad significandum id quod facit manifestationem in sensu
visus, postmodum autem extensum est ad significandum omne illud quod
facit manifestationem secundum quamcumque cognitionem. --- Si ergo
accipiatur nomen {\itshape luminis} secundum suam primam impositionem,
metaphorice in spiritualibus dicitur, ut Ambrosius dicit. Si autem
accipiatur secundum quod est in usu loquentium ad omnem manifestationem
extensum, sic proprie in spiritualibus dicitur.

&


同様に、「光」という名前についても言われるべきである。
すなわち、最初に、視覚において明示するものを表示するために名付けられ、そ
 のあとに、どんな認識においてであれ、明示するすべてのものを表示すること
 へと拡張された。ゆえに、もし、「光」という名前が、最初の名付けにしたがっ
 て理解されるならば、アンブロシウスが言うように、霊的諸事物の中では比喩
 的に語られる。しかし、語る人々の使用において、すべての明示へと拡張され
 ていることにしたがって理解されるならば、その場合には、霊的諸事物の中で、
 固有に語られる。


\\


\noindent
Et per hoc patet responsio ad obiecta.

&

そして、これによって、異論への解答は明らかである。



\end{longtable}
\newpage





\rhead{a.~2}
\begin{center}
 {\Large {\bf ARTICULUS SECUNDUS}}\\
 {\large UTRUM LUX SIT CORPUS}\\
 {\footnotesize II {\itshape Sent.}, d.~13, a.~3; II {\itshape de
 Anima}, lect.~14.}\\
 {\Large 第二項\\光は物体か}
\end{center}

\begin{longtable}{p{21em}p{21em}}



{\huge A}{\scshape d secundum sic proceditur}. Videtur quod lux sit
corpus. Dicit enim Augustinus, in libro {\itshape de Lib.~Arbit}., quod
{\itshape lux in corporibus primum tenet locum}. Ergo lux est corpus.

&


第二項の問題へ、議論は以下のように進められる。
光は物体だと思われる。理由は以下の通り。
アウグスティヌスは『自由意志』という書物の中で、「光は物体的諸事物の中で
 第一の位置を占める」と述べている。ゆえに、光は物体である。

\\


{\scshape 2. Praeterea}, philosophus dicit quod lumen est
species ignis. Sed ignis est corpus. Ergo lumen est corpus.

&

さらに、哲学者は、光(lumen)は火の種であると言っている。ところが、火は物体である。
 ゆえに、光は物体である。

\\


{\scshape 3. Praeterea}, ferri, intersecari, et reflecti
est proprie corporum, haec autem omnia attribuuntur lumini vel
radio. Coniunguntur etiam diversi radii et separantur, ut Dionysius
dicit, {\scshape ii} cap.~{\itshape de Div.~Nom}., quod etiam videtur non nisi corporibus
convenire posse. Ergo lumen est corpus.

&

さらに、「運ばれる」「分けられる」「反射される」は、固有に物体的諸事物に
 属するが、これらすべては、光(lumen)や光線(radius)に帰属する。さらに、ディ
 オニュシウスが『神名論』2章で述べるように、さまざまな光線が結びつけられ
 たり分離されたりしているが、これらは、物体的諸事物にしか適合しないと思
 われる。ゆえに、光(lumen)は物体である。


\\


{\scshape Sed contra}, duo corpora non possunt simul
esse in eodem loco. Sed lumen est simul cum aere. Ergo lumen non est
corpus.

&
しかし反対に、二つの物体が、同一の場所に、同時に存在することはできない。
 ところが、光(lumen)は、空気(aer)と同時に存在する。ゆえに、光は物体でない。

\\


{\scshape Rpondeo dicendum} quod impossibile est lumen
esse corpus. Quod quidem apparet tripliciter. Primo quidem, ex parte
loci. Nam locus cuiuslibet corporis est alius a loco alterius corporis,
nec est possibile, secundum naturam, duo corpora esse simul in eodem
loco, qualiacumque corpora sint; quia contiguum requirit distinctionem
in situ. 


&

解答する。以下のように言われるべきである。
光が物体であることは不可能である。これは、三通りの仕方で明らかとなる。
第一に、場所の側からである。
どんな物体の場所も、他の物体の場所と異なる。また、本性的に、二つの物体が、
 それらがどのような物体であっても、同時に同じ場所にあることは不可能であ
 る。なぜなら、接触は、位置における区別を必要とするからである。

\\



Secundo, apparet idem ex ratione motus. Si enim lumen esset
corpus, illuminatio esset motus localis corporis. Nullus autem motus
localis corporis potest esse in instanti, quia omne quod movetur
localiter, necesse est quod prius perveniat ad medium magnitudinis quam
ad extremum. Illuminatio autem fit in instanti. Nec potest dici quod
fiat in tempore imperceptibili. Quia in parvo spatio posset tempus
latere, in magno autem spatio, puta ab oriente in occidentem, tempus
latere non posset, statim enim cum sol est in puncto orientis,
illuminatur totum hemisphaerium usque ad punctum oppositum. 


&

第二に、同じことが運動の性格から明らかである。
もし光が物体であったならば、照明は物体の場所的運動であっただろう。ところ
 が、物体のどんな場所的運動も、瞬間的でない。なぜなら、場所的に動くもの
 はすべて、ある大きさの終端よりも中間に、先に到達する必要があるからであ
 る。ところが、照明は瞬間的に生じる。また、知覚できないほど短い時間に生
 じると言うこともできない。なぜなら、短い距離なら時間を感じさせないこと
 が可能かも知れないが、大きな距離で、たとえば、東の空から西の空まで、時
 間を感じさせないことは不可能である。じっさい、太陽が東の一点に昇ると、
 反対の一点にいたるまで、全半球が瞬間的に照明される。

\\



Est etiam
aliud considerandum ex parte motus. Quia omne corpus habet motum
naturalem determinatum, motus autem illuminationis est ad omnem partem,
nec magis secundum circulum quam secundum rectitudinem. Unde manifestum
est quod illuminatio non est motus localis alicuius corporis. 


&


さらに、運動の側から別のことが考察されるべきである。すべての物体は、限定
 された自然本性的運動をもつ。ところが、照明の運動は、あらゆる部分におよ
 び、円運動よりも直線運動、というようなことがない。したがって、照明は、
 なんらかの物体の場所的運動ではない。

\\


Tertio,
apparet idem ex parte generationis et corruptionis. Si enim lumen esset
corpus, quando aer obtenebrescit per absentiam luminaris, sequeretur
quod corpus luminis corrumperetur, et quod materia eius acciperet aliam
formam. Quod non apparet, nisi aliquis dicat etiam tenebras esse
corpus. Nec etiam apparet ex qua materia tantum corpus, quod replet
medium hemisphaerium, quotidie generetur. 



&


第三に、同じことが生成消滅の側から明らかである。
もし光が物体であったならば、発光体の不在によって空気が暗くなるとき、光と
 いう物体が消滅し、それ[=光]の質料が、他の形相を受け取る、ということ
 が、そのあとに続くであろう。このことは、もし、ある人が、暗闇もまた物体だと言
 わないかぎり、明らかなことではない。また、半球という媒体を満たすこれだ
 け大きな物体が、どのような質料から毎日生成するかということもまた、明ら
 かでない。

\\


Ridiculum est etiam dicere
quod ad solam absentiam luminaris, tantum corpus corrumpatur. Si quis
etiam dicat quod non corrumpitur, sed simul cum sole accedit et
circumfertur, quid dici poterit de hoc, quod ad interpositionem alicuius
corporis circa candelam, tota domus obscuratur? Nec videtur quod lumen
congregetur circa candelam, quia non apparet ibi maior claritas post
quam ante. Quia ergo omnia haec non solum rationi, sed sensui etiam
repugnant, dicendum est quod impossibile est lumen esse corpus.


&

また、たんに発光体が不在であることに即して、これほど大きな物体が消滅する
 と言うのも馬鹿げている。もし、消滅するのではなく、太陽とともに近づき、
 巡るのだとさらに言う人がいるならば、ロウソクの周りを何かの物体で遮断す
 れば、家全体が暗くなることについてどう言いうるだろうか。また、光がロウ
 ソクの周りに集められているようにも見えない。なぜなら、そこに、以前より
 もより大きな明るさが表れているわけではないからである。ゆえに、これらす
 べてのことは、理性だけでなく、感覚にも反するから、光(lumen)が物体であることは
 不可能だと言われるべきである。

\\


{\scshape  Ad primum ergo dicendum} quod Augustinus
accipit lucem pro corpore lucido in actu, scilicet pro igne, quod inter
quatuor elementa nobilissimum est.

&

第一異論に対しては、それゆえ、以下のように言われるべきである。アウグスティ
ヌスは[そこで]、光を、現実に明るい物体、すなわち火として理解している。
これは、四つの元素の中で、もっとも高貴なものである。

\\


{\scshape Ad secundum dicendum} quod Aristoteles {\itshape lumen}
nominat ignem in propria materia, sicut ignis in materia aerea dicitur
{\itshape flamma}, et in materia terrea dicitur {\itshape carbo}. Non
tamen est multum curandum de eis exemplis quae Aristoteles inducit in
libris logicalibus, quia inducit ea ut probabilia secundum opinionem
aliorum.

&

第二異論に対しては、以下のように言われるべきである。
アリストテレスは、固有の質料の中にある火を「光」(lumen)と名付けている。
 それは、空気を質料としてある火を「炎」(flamma)、土を質料としてある火を
 「炭火」(carbo)と名付けるのと同様である。しかし、アリストテレスが論理学
 的著作の中に導入するこれらの例をあまり気遣うべきではない。なぜなら、彼
 はそれらを他の人々の意見に従って、蓋然的なものとして導入しているからで
 ある。

\\


{\scshape Ad tertium dicendum} quod omnia illa
attribuuntur lumini metaphorice, sicut etiam possent attribui
calori. Quia enim motus localis est naturaliter primus motuum, ut
probatur in VIII {\itshape Physic}., utimur nominibus pertinentibus ad motum
localem, in alteratione et in omnibus motibus, sicut etiam nomen
{\itshape distantiae} derivatum est a loco ad omnia contraria, ut dicitur in X
{\itshape Metaphys}.

&

第三異論に対しては、以下のように言われるべきである。
これらすべては、ちょうど、それらが熱にも適用できるように、比喩的に光に適
 用されている。なぜなら、『自然学』8巻で証明されるように、場所的運動が、
 自然本性的に、運動の中で第一のものなので、私たちは、場所的運動に属する
 名称を、変化や、すべての運動において用いるからである。ちょうど、『形而
 上学』10巻で言われるように、「距離」という名称は場所に由来するけれども、
 あらゆる反対のものにたいして[使われるように]である。



\end{longtable}
\newpage






\rhead{a.~3}
\begin{center}
 {\Large {\bf ARTICULUS TERTIUS}}\\
 {\large UTRUM LUX SIT QUALITAS}\\
 {\footnotesize II {\itshape Sent.}, d.~13, a.~3; II {\itshape de
 Anima}, lect.~14.}\\
 {\Large 第三項\\光は性質か}
\end{center}

\begin{longtable}{p{21em}p{21em}}



{\huge A}{\scshape d tertium sic proceditur}. Videtur quod lux
non sit qualitas. Omnis enim qualitas permanet in subiecto etiam
postquam agens discesserit; sicut calor in aqua postquam removetur ab
igne. Sed lumen non remanet in aere recedente luminari. Ergo lumen non
est qualitas.

&

第三項の問題へ、議論は以下のように進められる。
光は性質でないと思われる。理由は以下の通り。
すべて性質は、作用者が去ったあとも、基体の中に留まる。たとえば、火が取り
 除かれたあとも、熱は水の中に留まる。ところが、光源がなくなると、空気の
 中に光は留まらない。ゆえに、光は性質でない。

\\


{\scshape 2 Praeterea}, omnis qualitas sensibilis habet
contrarium; sicut calido contrariatur frigidum, et albo nigrum. Sed
lumini nihil est contrarium, tenebra enim est privatio luminis. Ergo
lumen non est qualitas sensibilis.

&
さらに、すべて可感的性質は反対のものをもつ。たとえば、熱には冷が、白には
 黒が反対のものとなる。ところが、光には何も反対のものがない。というのも、
 闇は光の欠如だからである。ゆえに、光は可感的性質でない。

\\


{\scshape 3 Praeterea}, causa est potior effectu. Sed
lux caelestium corporum causat formas substantiales in istis
inferioribus. Dat etiam esse spirituale coloribus, quia facit eos
visibiles actu. Ergo lux non est aliqua qualitas sensibilis, sed magis
substantialis forma, aut spiritualis.

&


さらに、原因は結果よりも力がある。ところが、天体の光は、これら月下の諸事
 物における実体的形相の原因である。また、その光は、諸々の色に霊的存在を
 与える。なぜなら、それらの色を、現実に可視的にするからである。ゆえに、
 光は、可感的ななんらかの性質ではなく、実体形相、あるいは、霊的な形相で
 ある。

\\


{\scshape  Sed contra est} quod Damascenus dicit, in
libro I, quod lux est quaedam qualitas.

&

しかし反対に、ダマスケヌスは、[『正統信仰論』の]第1巻で、「光はなんら
 かの性質である」と述べている。

\\


{\scshape Respondeo dicendum} quod quidam dixerunt quod
lumen in aere non habet esse naturale, sicut color in pariete; sed esse
intentionale, sicut similitudo coloris in aere. Sed hoc non potest esse,
propter duo. Primo quidem, quia lumen denominat aerem, fit enim aer
luminosus in actu. Color vero non denominat ipsum, non enim dicitur aer
coloratus. Secundo, quia lumen habet effectum in natura, quia per radios
solis calefiunt corpora. Intentiones autem non causant transmutationes
naturales. 


&
解答する。以下のように言われるべきである。
ある人々は、光は、空気の中で、ちょうど色が壁においてもつような自然的存在
 をもたず、ちょうど色の類似が空気の中でもつような志向的存在をもつと述べ
 た。しかし、二つの理由でこれはありえない。第一に、空気は現実に光に満ち
 たものとなるから、光は空気の名となるが、色は空気の名にならない。空気が
 色づくとは言われないからである。第二に、太陽光線によって物体は熱せられ
 るので、光は自然において結果をもつが、志向的なものは自然的な転化の原因
 とならない。

\\



Alii vero dixerunt quod lux est forma substantialis solis. --- Sed hoc
etiam apparet impossibile, propter duo. Primo quidem, quia nulla forma
substantialis est per se sensibilis, quia {\itshape quod quid est} est
obiectum intellectus, ut dicitur in III {\itshape de Anima}. Lux autem
est secundum se visibilis. Secundo, quia impossibile est ut id quod est
forma substantialis in uno, sit forma accidentalis in alio, quia formae
substantiali per se convenit constituere in specie; unde semper et in
omnibus adest ei. Lux autem non est forma substantialis aeris, alioquin,
ea recedente, corrumperetur. Unde non potest esse forma substantialis
solis.


&

他方、別の人々は、光は太陽の実体形相だと言った。しかしこれも、二つの理由
 で不可能であることが明らかとなる。第一に、どんな実体形相も、それ自体に
 よって可感的ではない。なぜなら『デ・アニマ』3巻で言われるとおり、「何で
 あるか」は知性の対象だから。ところが、光は、それ自体において可視的であ
 る。第二に、あるものにおいて実体形相であるものが、別のものにおいて附帯
 形相であることは不可能である。なぜなら、種において構成するということが、
 自体的に実体形相に適合するのであり、それゆえに、常に、また[その種に属する]
 すべてのものにおいて、それ[=種]とともにあるからである。しかし
 光は空気の実体形相でない。もしそうならば、光がなくなると、空気は消滅し
 たであろう。[ゆえに、光は空気の附帯形相である。]したがって、空気は太陽の実体形相でありえない。




\\


Dicendum est ergo quod, sicut calor est qualitas activa
consequens formam substantialem ignis, ita lux est qualitas activa
consequens formam substantialem solis, vel cuiuscumque alterius corporis
a se lucentis, si aliquod aliud tale est. Cuius signum est, quod radii
diversarum stellarum habent diversos effectus, secundum diversas naturas
corporum.

&

それゆえ、以下のように言われるべきである。ちょうど熱が火の実体形相に伴う
作用的性質であるように、光は、太陽、あるいはもしそのようなものがあるなら
ば、その他なんであれ自ら光る物体の実体形相に伴う作用的性質である。そのし
 るしは、さまざまな星の光線が、諸物体のさまざまな本性においてさまざまな
 結果をもたらしていることにある。

\\


{\scshape Ad primum ergo dicendum} quod, cum qualitas
consequatur formam substantialem, diversimode se habet subiectum ad
receptionem qualitatis, sicut se habet ad receptionem formae. Cum enim
materia perfecte recipit formam, firmiter stabilitur etiam qualitas
consequens formam; sicut si aqua convertatur in ignem. Cum vero forma
substantialis recipitur imperfecte, secundum inchoationem quandam,
qualitas consequens manet quidem aliquandiu, sed non semper; sicut patet
in aqua calefacta, quae redit ad suam naturam. Sed illuminatio non fit
per aliquam transmutationem materiae ad susceptionem formae
substantialis, ut fiat quasi inchoatio aliqua formae. Et ideo lumen non
remanet nisi ad praesentiam agentis.

&

第一異論に対しては、それゆえ、以下のように言われるべきである。
性質は実体形相に伴うので、基体が性質の受容にたいする関係は、[実体]形相
 の受容にたいする関係がさまざまであるのと同様、さまざまである。
つまり、質料が形相を完全に受け取るとき、その形相に伴う性質もまた、堅く固
 定化される。たとえば、水が火に転化されるような場合である。他方、初期
 であるなどの理由で、実体形相が不完全に受け取られる場合には、それに伴う
 性質は、ある一定時間留まるが、常に留まるわけではない。たとえば、熱せら
 れた水において明らかなとおりであり、それは[まもなく]自分の[冷たい水
 という]本性へ戻る。
しかし、照明は、なんらかの形相の始まりとして生じるように、実体形相の受容
 へ質料が転化することによって生じるのではない。ゆえに、光は、作用者が現
 在しているときでなければ留まらない。

\\


{\scshape Ad secundum dicendum} quod accidit luci quod
non habeat contrarium, inquantum est qualitas naturalis primi corporis
alterantis, quod est a contrarietate elongatum.

&


第二異論に対しては、以下のように言われるべきである。
反対のものをもたないということが光に起こるのは、それが、反対性から離れて
 いる、転化させる第一の物体の自然本性的性質である限りにおいてである。


\\


{\sc Ad tertium dicendum} quod, sicut calor agit ad
formam ignis quasi instrumentaliter in virtute formae substantialis, ita
lumen agit quasi instrumentaliter in virtute corporum caelestium ad
producendas formas substantiales, et ad hoc quod faciat colores
visibiles actu, inquantum est qualitas primi corporis sensibilis.

&

第三異論に対しては、以下のように言われるべきである。
ちょうど、熱が、火の形相にたいして、実体形相の力において道具的に働くよう
 に、光も、実体形相を生み出すために、天体の力において道具的に働く。また、
 可感的な第一物体の性質であるかぎりで、色を現実に可視的にするために、[道
 具的に働
 く]。



\end{longtable}
\newpage





\rhead{a.~4}
\begin{center}
 {\Large {\bf ARTICULUS QUARTUS}}\\
 {\large UTRUM CONVENIENTER LUCIS PRODUCTIO IN PRIMA DIE PONATUR}\\
 {\footnotesize Infra, q.~69, a.~1; II {\itshape Sent.}, d.~13, a.~4.}\\
 {\Large 第四項\\光の産出が第一日目に置かれているのは適切か}
\end{center}

\begin{longtable}{p{21em}p{21em}}

{\huge A}{\scshape d quartum sic proceditur}. Videtur quod
inconvenienter lucis productio in prima die ponatur. Est enim lux
qualitas quaedam, ut dictum est. Qualitas autem, cum sit accidens, non
habet rationem primi, sed magis rationem postremi. Non ergo prima die
debet poni productio lucis.

&

第四項の問題へ、議論は以下のように進められる。
光の産出が第一日目に置かれているのは不適切だと思われる。理由は以下の通り。
すでに述べられたように、光はある種の性質である。ところが、性質は附帯性だ
 から、「第一」という性格をもたず、むしろ、「最後」という性格をもつ。ゆ
 えに、第一日目に光の産出が置かれるべきではなかった。

\\


{\scshape 2 Praeterea}, per lucem distinguitur nox a
die. Hoc autem fit per solem, qui ponitur factus die quarta. Ergo non
debuit poni productio lucis prima die.

&

さらに、光によって、夜は昼から区別される。しかし、このことは太陽によって
なされるのであり、太陽は、第四日目に作られたと書かれている。ゆえに、光の
 産出が第一日目に置かれるべきではなかった。

\\


{\scshape 3 Praeterea}, nox et dies fit per circularem
motum corporis lucidi. Sed circularis motus est proprius firmamenti,
quod legitur factum die secunda. Ergo non debuit poni in prima die
productio lucis distinguentis noctem et diem.

&
さらに、夜と昼は、輝く物体の円運動によって生じる。ところが、円運動は、天
 球に固有であり、天球は第二日目に作られたと読める。ゆえに、夜と昼を分け
 る光の産出が第一日目に置かれるべきではなかった。

\\


{\scshape 4 Si dicatur} quod intelligitur de luce
spirituali, contra, lux quae legitur facta prima die, facit
distinctionem a tenebris. Sed non erant in principio spirituales
tenebrae, quia etiam Daemones fuerunt a principio boni, ut supra dictum
est. Non ergo prima die debuit poni productio lucis.

&

もし、霊的な光について考えられると言われるならば、反対して以下のように言
 う。第一の日に作られたと読める光は、闇からの区別を行っている。ところが、
 初めには、闇の霊は存在しなかった。というのも、以前に述べられたように、
 悪魔たちさえも、初めには善かったのだから。ゆえに、光の産出は、第一の日
 に置かれるべきではなかった。

\\


{\scshape  Sed contra}, id sine quo non potest esse
dies, oportuit fieri in prima die. Sed sine luce non potest esse
dies. Ergo oportuit lucem fieri prima die.

&

しかし反対に、それがなくては日がありえないようなものは、第一日目に作られ
 るべきであった。ところが、光なしに日はありえない。ゆえに、第一日目に光
 が作られるべきであった。

\\


{\scshape Respondeo dicendum} quod de productione lucis
est duplex opinio. Augustino enim videtur quod non fuerit conveniens
Moysen praetermisisse spiritualis creaturae productionem. Et ideo dicit
quod, cum dicitur, {\itshape in principio creavit Deus caelum et terram}, per
{\itshape caelum} intelligitur spiritualis natura adhuc informis, per {\itshape terram} autem
intelligitur materia informis corporalis creaturae. Et quia natura
spiritualis dignior est quam corporalis, fuit prius formanda. Formatio
igitur spiritualis naturae significatur in productione lucis, ut
intelligatur de luce spirituali: formatio enim naturae spiritualis est
per hoc quod illuminatur ut adhaereat verbo Dei. 


&

解答する。以下のように言われるべきである。
光の産出については二通りの意見がある。アウグスティヌスには、モーセが霊的
 被造物の産出を省略したということは適切でないように思われた。それゆえ、
 「初めに神は天と地を創造した」と言われるとき、「天」で、まだ無形の霊的
 本性が、「地」で物体的被造物の無形の質料が理解されるという。そして、霊
 的本性は物体的本性よりも上位なので、先に作られるべきだった、と。ゆえに、
 霊的本性の形成が、光、すなわち霊的光について理解される光の産出において
 表示される。なぜなら、霊的本性の形成は、神の言葉に固着するように照明されることによるからである。


\\



Aliis autem videtur
quod sit praetermissa a Moyse productio spiritualis creaturae. Sed huius
rationem diversimode assignant. Basilius enim dicit quod Moyses
principium narrationis suae fecit a principio quod ad tempus pertinet
sensibilium rerum; sed spiritualis natura, idest angelica,
praetermittitur, quia fuit ante creata. 

&

他方、他の人々には、霊的被造物の産出が、モーゼによって省略されていると見
 られている。しかし、彼らはその理由をさまざまに示している。たとえばバシ
 リウスは、モーセは彼の話しを、可感的諸事物の時間に属する始まりから始め
 たと言う。しかし霊的本性、つまり天使たちは、それ以前に創造されたから、
 省略されているのだ、と。


\\


Chrysostomus autem assignat
aliam rationem. Quia Moyses loquebatur rudi populo, qui nihil nisi
corporalia poterat capere; quem etiam ab idololatria revocare
volebat. Assumpsissent autem idololatriae occasionem, si propositae
fuissent eis aliquae substantiae supra omnes corporeas creaturas, eas
enim reputassent deos, cum etiam proni essent ad hoc quod solem et lunam
et stellas colerent tanquam deos; quod eis inhibetur {\itshape Deut}.~{\scshape iv}. 


&


また、クリュソストムスは、別の説明をしている。つまり、モーセは、物体的事
 物以外に捉えることができない粗野な民衆に話していたので、彼らを偶像崇拝
 から呼び戻そうと欲していた。しかし、もし、彼らにすべての被造の物体以上
 のなんらかの実体が提示されていたら、彼らはそれらを神々とみなしたであろ
 う。なぜなら、彼らは太陽や月や星を神々として崇拝する傾向があったのだか
 ら。これは『申命記』4章\footnote{「また目を上げて天を仰ぎ、太陽、月、星と
 いった天の万象を見て、これらに惑わされ、ひれ伏し仕えてはならない。」(4:19)}で、彼らに禁じられている。



\\

Praemissa autem fuerat circa creaturam corporalem multiplex informitas,
una quidem in hoc quod dicebatur, {\itshape terra erat inanis et vacua};
alia vero in hoc quod dicebatur, {\itshape tenebrae erant super faciem
abyssi}. 


&

ところで、物体的被造物を巡って、多様な無形性が前もって示された。一つは、
 「地は空っぽで空虚であり」と言われたところにあり、もう一つは、「闇が深
 淵の面にあった」と言われているところにある。

\\




Necessarium autem fuit ut informitas tenebrarum primo
removeretur per lucis productionem, propter duo. Primo quidem, quia lux,
ut dictum est, est qualitas primi corporis, unde secundum eam primo fuit
mundus formandus. Secundo, propter communitatem lucis, communicant enim
in ea inferiora corpora cum superioribus. Sicut autem in cognitione
proceditur a communioribus, ita etiam in operatione, nam prius generatur
vivum quam animal, et animal quam homo, ut dicitur in libro {\itshape de
Gener.~Animal}.


&


さて、二つの理由で、最初に、光の産出によって闇の無形性が取り除かれること
 が必要であった。第一に、光は、すでに述べられたように、第一の物体の性質
 なので、それにしたがって第一に世界が形成されるべきであった。第二に、光
 の共通性のためにであり、つまり、光において、下位の諸物体は上位の諸物
 体と共有する。
ちょうど、認識において、より共通的なものから始められるように、働きにおい
 てもまたそうである。たとえば、『動物発生論』という書物で言われる
 ように、動物よりも生きるものが生成され、人間よりも先に動物が生成される。

\\


 Sic ergo oportuit ordinem divinae sapientiae
manifestari, ut primo inter opera distinctionis produceretur lux,
tanquam primi corporis forma, et tanquam communior. Basilius autem ponit
tertiam rationem, quia per lucem omnia alia manifestantur. Potest et
quarta ratio addi, quae in obiiciendo est tacta, quia dies non potest
esse sine luce; unde oportuit in prima die fieri lucem.

&


ゆえに、区別の業の最初に、第一の物体の形相として、そして、より共通なもの
 として、光が生み出されるというかたちで、神の知恵の秩序が明示されなければならな
 かった。バシリウスはさらに、三つめの理由を示している。すなわち、光によっ
 て、他のすべてのものが明示される、と。また第四の理由が加えられうる。こ
 れは反対異論が触れていたことだが、日は光なしにありえないので、第一の日
 に光が作られるべきであった、と。


\\


{\scshape Ad primum ergo dicendum} quod, secundum
opinionem quae ponit informitatem materiae duratione praecedere
formationem, oportet dicere quod materia a principio fuerit creata sub
formis substantialibus; postmodum vero fuerit formata secundum aliquas
conditiones accidentales, inter quas primum locum obtinet lux.

&

第一異論に対しては、それゆえ、以下のように言われるべきである。
質料の無形性が、持続において形成に先行するとする意見に従うならば、質料は、
 初めに、実体形相の下で創造されたと言わなければならない。他方で、それは
 なんらかの附帯的条件の下で形成されたのであり、光は、それらの中で第一の
 場所を占める。

\\


{\scshape Ad secundum dicendum} quod quidam dicunt lucem
illam fuisse quandam nubem lucidam, quae postmodum, facto sole, in
materiam praeiacentem rediit. Sed istud non est conveniens. Quia
Scriptura in principio {\itshape Genesis} commemorat institutionem naturae, quae
postmodum perseverat, unde non debet dici quod aliquid tunc factum
fuerit, quod postmodum esse desierit. Et ideo alii dicunt quod illa
nubes lucida adhuc remanet, et est coniuncta soli, ut ab eo discerni non
possit. Sed secundum hoc, illa nubes superflua remaneret, nihil autem
est vanum in operibus Dei. Et ideo alii dicunt quod ex illa nube
formatum est corpus solis. Sed hoc etiam dici non potest, si ponatur
corpus solis non esse de natura quatuor elementorum, sed esse
incorruptibile per naturam quia secundum hoc, materia eius non potest
esse sub alia forma. 



&


第二異論に対しては、以下のように言われるべきである。
ある人々は、かの「光」は、一種の輝く雲であり、それは後に、太陽が作られる
 ときに、そのもとになる質料に戻った、と言う。しかし、これは適切でない。
 なぜなら、聖書は『創世記』の最初に、自然の構築を記しているのであり、そ
 れはそのあとも持続するから、したがって、そのときに作られたものが後に存在を
 やめると言われるべきではないからである。このため、別の人々は、かの輝く
 雲はなお残っていて、太陽と区別されないかたちで太陽に結合されたと言う。
 しかし、これによれば、かの雲は無駄なものに留まったであろう。しかし、神
 の業に空虚なものはない。それゆえ、また別の人々は、かの雲から、物体とし
 ての太陽が形成されたと言う。しかし、もし、物体
 としての太陽が四元素の本性をもたず、その質料が他の形相のもとにあることが
 不可能であるために、本性的に不滅であるとするならば、これもまた不可能である。

\\



Et ideo est dicendum, ut Dionysius dicit {\scshape iv} cap.~{\itshape de
Div.~Nom}., quod illa lux fuit lux solis, sed adhuc informis, quantum ad
hoc, quod iam erat substantia solis, et habebat virtutem illuminativam
in communi; sed postmodum data est ei specialis et determinata virtus ad
particulares effectus. Et secundum hoc, in productione huius lucis
distincta est lux a tenebris, quantum ad tria. 



&

それゆえ、ディオニュシウスが『神名論』4章で言うように、以下のように言わ
 れるべきである。かの光は太陽の光であった。しかし、まだ以下の点で無形で
 あった。すなわち、すでに太陽の実体が存在し、共通的に、照らしうる力を持っ
 ていたが、その後、太陽に、個別的な結果にたいする種的で限定された力が与
 えられた。そしてこれに従えば、この光の産出において、光は以下の三つの点
 で闇から区別される。


\\



Primo quidem, quantum ad
causam, secundum quod in substantia solis erat causa luminis, in
opacitate autem terrae causa tenebrarum. Secundo, quantum ad locum, quia
in uno hemisphaerio erat lumen, in alio tenebrae. Tertio, quantum ad
tempus, quia in eodem hemisphaerio secundum unam partem temporis erat
lumen, secundum aliam tenebrae. Et hoc est quod dicitur, {\itshape lucem vocavit
diem, et tenebras noctem}.


&

第一に、原因にかんしてであり、光の原因は太陽の実体にあり、他方、闇の原因
 は大地の不透明さにある。第二に、場所にかんしてであり、それは、一つの半球に光が
 あり、もう一つの半球に闇があったからである。第三に、時間にかんしてであ
 り、それは、同一の半球において、時間の一方の部分において光があり、他方
 の部分において闇があったからである。これが、「光を昼と呼び、闇を夜と呼
 んだ」と言われる意味である。


\\


{\scshape Ad tertium dicendum} quod Basilius dicit lucem
et tenebras tunc fuisse per emissionem et contractionem luminis, et non
per motum. Sed contra hoc obiicit Augustinus quod nulla ratio esset
huius vicissitudinis emittendi et retrahendi luminis; cum homines et
animalia non essent, quorum usibus hoc deserviret. Et praeterea hoc non
habet natura corporis lucidi, ut retrahat lumen in sui praesentia, sed
miraculose potest hoc fieri, in prima autem institutione naturae non
quaeritur miraculum, sed quid natura rerum habeat, ut Augustinus
dicit. 



&


第三異論に対しては、以下のように言われるべきである。
バシリウスは、そのとき、光と闇が、光の拡張と収縮によってあったのであり、
 運動によってではなかった、と言っている。しかし、これに反対して、アウグ
 スティヌスは、[もしそうだとしたら]光の拡張と収縮の移り変わりには、ど
 んな理由もなかったであろう、と反論している。というのも、それは人間や動
 物のために有益だっただろうが、そのときには人間も動物もいなかったからで
 ある。またさらに、輝く物体の本性は、自らが現前しているときに光を収
 縮させるような本性をもたず、奇跡によってのみ、これがなされるが、しかし
アウグスティヌスが言うように、 自然の第一の構成において、奇跡が探究さられることはなく、諸事物の本性が何をもつかが探究される。

\\



Et ideo dicendum est quod duplex est motus in caelo. Unus
communis toti caelo, qui facit diem et noctem, et iste videtur
institutus primo die. Alius autem est, qui diversificatur per diversa
corpora; secundum quos motus fit diversitas dierum ad invicem, et
mensium et annorum. Et ideo in prima die fit mentio de sola distinctione
noctis et diei, quae fit per motum communem. In quarta autem die fit
mentio de diversitate dierum et temporum et annorum, cum dicitur, {\itshape ut
sint in tempora et dies et annos}; quae quidem diversitas fit per motus
proprios.

&

それゆえ、天における運動は二通りあると言われるべきである。
一つは、天全体に共通の運動であり、これが昼と夜を作る。そして、この運動が、
 第一日目に作られたと思われる。他方、もう一つの、さまざまな物体に
 よって多様化される運動があり、これらの運動にしたがって、日々、月々、年々
 相互の差異が生じる。ゆえに、最初の日に、夜と昼の区別だけについて言及さ
 れ、この区別は共通の運動によって生じる。他方、第四日目に、「時間や日々
 や年々」\footnote{「天の大空に光る物があって、昼と夜を分け、季節のしる
 し、日や年のしるしとなれ」(1:14)}と言われているときに、日々や時間や年々の差異が言及され、この際
 は、固有の諸運動によって生じる。

\\


{\scshape Ad quartum dicendum} quod, secundum
Augustinum, informitas non praecedit formationem duratione. Unde oportet
dicere quod per lucis productionem intelligatur formatio spiritualis
creaturae non quae est per gloriam perfecta, cum qua creata non fuit;
sed quae est per gratiam perfecta, cum qua creata fuit, ut dictum
est. Per hanc ergo lucem facta est divisio a tenebris, idest ab
informitate alterius creaturae non formatae. Vel, si tota creatura simul
formata fuit, facta est distinctio a tenebris spiritualibus, non quae
tunc essent (quia Diabolus non fuit creatus malus); sed quas Deus
futuras praevidit.

&

第四異論に対しては、以下のように言われるべきである。
アウグスティヌスによれば、無形性は、持続の点で、形成に先行するのではない。
 したがって、光の産出によって、霊的被造物の形成が理解されるとき、その霊
 的被造物は、栄光によって完成されたものではなく、というのも、天使は栄光に
 よって創造されたのではないからだが、むしろ、恩恵によって完成されたもの
 である。すでに述べられたように、天使は恩恵によって創造されたのだから。
 ゆえに、この光によって、闇からの分割、すなわち、形成されていない他の被
 造物の無形性からの分割がなされた。あるいは[別解を示すと]、もし全被造物が同時に形成さ
 れたのであれば、霊的な闇からの区別が作られた。ただし、それはそのとき
 に存在していたのではなく(なぜなら、悪魔は悪として創造されたのではないから)、
 神が、それらが未来に存在することを予見したのである。




\end{longtable}
\end{document}