\documentclass[10pt]{jsarticle} % use larger type; default would be 10pt
%\usepackage[utf8]{inputenc} % set input encoding (not needed with XeLaTeX)
%\usepackage[round,comma,authoryear]{natbib}
%\usepackage{nruby}
\usepackage{okumacro}
\usepackage{longtable}
%\usepqckage{tablefootnote}
\usepackage[polutonikogreek,english,japanese]{babel}
%\usepackage{amsmath}
\usepackage{latexsym}
\usepackage{color}
%\usepackage{tikz}

%----- header -------
\usepackage{fancyhdr}
\pagestyle{fancy}
\lhead{{\it Summa Theologiae} I, q.~35}
%--------------------


\title{{\bf PRIMA PARS}\\{\HUGE Summae Theologiae}\\Sancti Thomae
Aquinatis\\{\sffamily QUEAESTIO TRIGESTIMAQUINTA}\\DE IMAGINE}
\author{Japanese translation\\by Yoshinori {\sc Ueeda}}
\date{Last modified \today}

%%%% コピペ用
%\rhead{a.~}
%\begin{center}
% {\Large {\bf }}\\
% {\large }\\
% {\footnotesize }\\
% {\Large \\}
%\end{center}
%
%\begin{longtable}{p{21em}p{21em}}
%
%&
%
%\\
%\end{longtable}
%\newpage



\begin{document}

\maketitle
\thispagestyle{empty}
\begin{center}
{\Large 聖トマス・アクィナスの神学大全の第一部\\第三十五問\\似像について}
\end{center}



\begin{longtable}{p{21em}p{21em}}
{\Huge D}einde quaeritur de imagine. Et circa hoc quaeruntur
 duo. 
 

\begin{enumerate}
 \item utrum imago in divinis dicatur personaliter. 
 \item utrum sit proprium filii.
\end{enumerate}

&

次に似像について問われる。これをめぐっては二つのことが問われる。

\begin{enumerate}
 \item 似像は神においてペルソナ的に語られるか。
 \item それは息子に固有か。
\end{enumerate}

\end{longtable}

\newpage


\rhead{a.~1}
\begin{center}
{\Large {\bf ARTICULUS PRIMUS}}\\
{\large UTRUM IMAGO IN DIVINIS DICATUR PERSONALITER}\\
{\footnotesize Infra, q.93, a.5, ad 4; I {\itshape Sent.}, d.28, q.2,
 a.2.}\\
{\Large 第一項\\似像は神においてペルソナ的に語られるか}
\end{center}

\begin{longtable}{p{21em}p{21em}}


{\scshape Ad primum sic proceditur}. Videtur quod imago non dicatur personaliter
in divinis. Dicit enim Augustinus, in libro {\itshape de fide ad
 Petrum} : {\itshape Una est
sanctae Trinitatis divinitas et imago, ad quam factus est homo}. Igitur
imago dicitur essentialiter, et non personaliter.

&

第一項の問題へ議論は以下のように進められる。
似像は神においてペルソナ的に語られるのではないと思われる。理由は以下の
 通り。
アウグスティヌスは『信仰についてペトルスへ』という本で「聖なる三性の一
 つの神性と似像、それへ向けて人間は造られた」と述べている。ゆえに似像
 は本質的に語られるのでありペルソナ的にではない。


\\



2. {\scshape Praeterea}, Hilarius dicit, in libro {\itshape de Synod}., quod {\itshape imago est eius rei
ad quam imaginatur, species indifferens}. Sed species, sive forma, in
divinis dicitur essentialiter. Ergo et imago.

&

さらにヒラリウスは『教会会議について』で「似像とは、それへ向けて像が造
 られる事物の形象と違わない」と述べている。しかるに形象ないし形相は神
 において本質的に語られる。ゆえに似像もまた同じである。

\\



3. {\scshape Praeterea}, imago ab imitando dicitur, in quo importatur prius et
posterius. Sed in divinis personis nihil est prius et posterius ergo
imago non potest esse nomen personale in divinis.

&

さらに、似像は真似ることから言われ、そこにおいてより先とより先が含意さ
 れている。しかるに神のペルソナにおいてより先やより後ということはない。
 ゆえに似像が神においてペルソナ的な名称ではありえない。

\\



{\scshape Sed contra est} quod dicit Augustinus, {\itshape Quid est absurdius quam imaginem
ad se dici?} Ergo imago in divinis relative dicitur. Et sic est nomen
personale.

&

しかし反対にアウグスティヌスは「似像が自分に対して言われること以上に馬
 鹿げたことがあろうか」と述べている。ゆえに似像は神において関係的に言
 われる。そしてこの意味で、それはペルソナ的な名称である。

\\



{\scshape Respondeo dicendum} quod de ratione imaginis est similitudo. Non tamen
quaecumque similitudo sufficit ad rationem imaginis; sed similitudo
quae est in specie rei, vel saltem in aliquo signo speciei. Signum
autem speciei in rebus corporeis maxime videtur esse figura, videmus
enim quod diversorum animalium secundum speciem, sunt diversae
figurae, non autem diversi colores. 


&

解答する。以下のように言われるべきである。
似像の性格には類似が含まれる。しかしどんな類似でもこの似像の性格に十分
 というわけではなく、事物の種、あるいはせめて種の何らかのしるしにおけ
 る類似であることが必要である。
ところで、物体的事物における種のしるしは、最大限に、その形状だと思われ
 る。私たちは種においてさまざまな動物にはさまざまな形状が属するのを見
 るのであって、さまざまな色が属するのを見るわけではない。

\\


Unde, si depingatur color alicuius
rei in pariete, non dicitur esse imago, nisi depingatur figura. Sed
neque ipsa similitudo speciei sufficit vel figurae; sed requiritur ad
rationem imaginis origo, quia, ut Augustinus dicit in libro octoginta
trium quaest., unum ovum non est imago alterius, quia non est de illo
expressum. 

&

このことから、もしある事物の色が壁に描かれていても、形状が描かれていな
 いならばそれは似像ではない。しかし、種や形状の類似自体では十分でなく、
 似像の性格には起源ということが必要である。なぜならアウグスティヌスが『八十三問
 題集』という書物で述べているように、一つの卵は他の卵の似像ではないが、
 それは、その卵は他の卵から産み落とされたわけではないからである。


\\



Ad hoc ergo quod vere aliquid sit imago, requiritur quod ex
alio procedat simile ei in specie, vel saltem in signo speciei. Ea
vero quae processionem sive originem important in divinis, sunt
personalia. Unde hoc nomen imago est nomen personale.

&

ゆえに何かが真に似像であるためには、他のものから種においてあるいはせめ
 て種のしるしにおいてそれに似ているものが発出することが必要である。
他方、神において発出や起源を含意するものはペルソナ的なものである。ゆえ
 に似像というこの名称はペルソナ的な名称である。

\\



{\scshape Ad primum ergo dicendum} quod imago proprie dicitur quod procedit ad
similitudinem alterius. Illud autem ad cuius similitudinem aliquid
procedit, proprie dicitur exemplar, improprie vero imago. Sic tamen
Augustinus utitur nomine imaginis, cum dicit divinitatem sanctae
Trinitatis esse imaginem ad quam factus est homo.

&

第一異論に対してはそれゆえ以下のように言われるべきである。
固有には他のものの類似へと発出するものが似像と言われる。何かが
 \kenten{それ}の類似へと発出する\kenten{それ}は、固有には「範型」と言
 われるが、非固有的には似像と言われる。アウグスティヌスが聖なる三性の
 神性を人間がそれへ向けて作られたところの似像であると言うとき、このよ
 うな意味で似像という名称を用いている。

\\



{\scshape Ad secundum dicendum} quod species, prout ponitur ab Hilario in
definitione imaginis, importat formam deductam in aliquo ab alio. Hoc
enim modo imago dicitur esse species alicuius, sicuti id quod
assimilatur alicui, dicitur forma eius, inquantum habet formam illi
similem.

&

第二異論に対しては以下のように言われるべきである。
ヒラリウスによって似像の定義の中に置かれた「形象」は、他のものによって
 何かへと引き出された形相を意味している。この意味で在るものの形象は似
 像と言われる。ちょうど、何かへ類似化されるものは、そのものにとって似
 た形相をもつ限りにおいて、そのものの形相であると言われるように。


\\



{\scshape Ad tertium dicendum} quod imitatio in divinis personis non significat
posterioritatem, sed solam assimilationem.

&

第三異論に対しては以下のように言われるべきである。
神のペルソナにおける真似は後続性を意味せずたんに類似化だけを意味する。

\end{longtable}
\newpage


\rhead{a.~2}
\begin{center}
{\Large {\bf ARTICULUS SECUNDUS}}\\
{\large UTRUM NOMEN IMAGINIS SIT PROPRIUM FILII}\\
{\Large 第二項\\似像という名称は息子に固有か}
\end{center}

\begin{longtable}{p{21em}p{21em}}
{\scshape Ad secundum sic proceditur}. Videtur quod nomen imaginis non sit
proprium filio. Quia, ut dicit Damascenus, spiritus sanctus est {\itshape imago
filii}. Non est ergo proprium filii.


&

第二項の問題へ議論は以下のように進められる。似像という名称は息子に固有
 ではないと思われる。理由は以下の通り。
ダマスケヌスが言うように聖霊は「息子の似像」と言われる。ゆえに息子に固
 有ではない。


\\



2. {\scshape Praeterea}, de ratione imaginis est similitudo cum expressione, ut
Augustinus dicit, in libro {\itshape Octoginta trium Quaest}. Sed hoc convenit
spiritui sancto, procedit enim ab alio secundum modum
similitudinis. Ergo spiritus sanctus est imago. Et ita non est
proprium filii quod sit imago.


&

さらにアウグスティヌスが『八十三問題集』という本で言うように似像の性格
 には表現を伴う類似が含まれる。しかるにこのことは聖霊に適合する。なぜ
 なら、類似のしかたによって他のものから発出するからである。ゆえに聖霊
 は似像である。したがって似像であることは息子に固有ではない。

\\



3. {\scshape Praeterea}, homo etiam dicitur imago Dei, secundum illud
 I {\itshape ad Cor}.~{\scshape xi} : 
{\itshape vir non debet velare caput suum, quoniam imago et gloria Dei est}. Ergo
non est proprium filio.


&

さらにかの『コリントの信徒への手紙1』第11章「男はその頭を覆わなくても
 よい。なぜなら神の似姿であり栄光だから」\footnote{「男は神のかたちであ
 り栄光を映す者ですから、頭にかぶり物を着けるべきではありません。女は
 男の栄光を映す者です。」(11:7)}によれば、人間もまた神の似姿と言われる。
 ゆえに息子に固有ではない。

\\



{\scshape Sed contra est} quod Augustinus dicit, VI {\itshape de Trin}., quod s{\itshape olus filius
est imago patris}.


&

しかし反対に、アウグスティヌスは『三位一体論』第6巻で「ただ息子だけが
 父の似像である」と言っている。
\\



{\scshape Respondeo dicendum} quod doctores Graecorum communiter dicunt spiritum
sanctum esse imaginem patris et filii. Sed doctores Latini soli filio
attribuunt nomen imaginis, non enim invenitur in canonica Scriptura
nisi de filio. Dicitur enim {\itshape Coloss}.~{\scshape i}, {\itshape qui est imago Dei invisibilis,
primogenitus creaturae}; et {\itshape ad Hebr}.~{\scshape i}, {\itshape qui cum sit splendor gloriae,
et figura substantiae eius}. 


&

解答する。以下のように言われるべきである。
ギリシアの教師たちはそろって聖霊が父と息子の似像であると言う。しかしラ
 テンの教師たちは息子だけに似像という名称を帰す。なぜなら聖なる経典の
 中には、息子に点いてしか見出されないからである。たとえば『コロサイの
 信徒への手紙』第1章「見えざる神の似姿であり被造物の長子」\footnote{「御子は、見えない神のかたちであり/すべてのものが造られる前に/最初に生まれた方です。」(1:15)}と言われ、
 『ヘブライ人への手紙』第1章で「栄光の輝きであり彼の実体のかたちである
 人」\footnote{「御子は神の栄光の輝きであり、神の本質の現れであって、万物をその力ある言葉によって支えておられます。そして、罪の清めを成し遂げて、天の高い所におられる大いなる方の右の座に着かれました。」(1:3)}と言われている。

\\


Huius autem rationem assignant quidam ex
hoc, quod filius convenit cum patre non solum in natura, sed etiam in
notione principii, spiritus autem sanctus non convenit cum filio nec
cum patre in aliqua notione. Sed hoc non videtur sufficere. Quia sicut
secundum relationes non attenditur in divinis neque aequalitas neque
inaequalitas, ut Augustinus dicit; ita neque similitudo, quae
requiritur ad rationem imaginis. 


&

この理由をある人々は次のことに基づいて指摘している。すなわち、息子は父
 と本性においてだけでなく根源という識標においても一致するが、聖霊は息
 子にも父にもどんな識標においても一致しない。しかしこれは十分だと思わ
 れない。なぜなら、アウグスティヌスが言うように神のなかには関係に即し
 て等しさも不等性も見出されないので、似像の性格に要求される類
 似も見出されないからである。


\\

Unde alii dicunt quod spiritus
sanctus non potest dici imago filii, quia imaginis non est
imago. Neque etiam imago patris, quia etiam imago refertur immediate
ad id cuius est imago; spiritus sanctus autem refertur ad patrem per
filium. Neque etiam est imago patris et filii, quia sic esset una
imago duorum, quod videtur impossibile. Unde relinquitur quod spiritus
sanctus nullo modo sit imago.

&

このことから、他の人々は聖霊が息子の似像と言われえないのは、似像の似像
 はないからだと述べる。また(聖霊が)父の似像でもないのは、似像は
 \kenten{それ}の似像である\kenten{それ}へ媒介なしに関係づけられるが、
 聖霊は息子を通して父へ関係づけられるからだとする。また、(聖霊は)父
 と息子の似像でなく、それはもしそうなら二つのものに一つの似像があるこ
 とになり、これは不可能に思えるからだとする。
したがって残るのは、聖霊がいかなる意味においても似像でないということだ
 (と彼らは論じている)。


\\



 Sed hoc nihil est. Quia pater et filius
sunt unum principium spiritus sancti, ut infra dicetur, unde nihil
prohibet sic patris et filii, inquantum sunt unum, esse unam imaginem;
cum etiam homo totius Trinitatis sit una imago. 


&

しかしこれは証明になっていない。なぜなら父と息子は、後で述べられるよう
 に、聖霊の一つの根源であるから、父と息子が一つのものであるかぎりにお
 いて一つの似像であることは何ら差し支えない。人間もまた三性全体の一つ
 の似像なのだから。


\\

Et ideo aliter
dicendum est quod, sicut spiritus sanctus, quamvis sua processione
accipiat naturam patris, sicut et filius, non tamen dicitur natus;
ita, licet accipiat speciem similem patris, non dicitur imago. Quia
filius procedit ut verbum, de cuius ratione est similitudo speciei ad
id a quo procedit; non autem de ratione amoris; quamvis hoc conveniat
amori qui est spiritus sanctus, inquantum est amor divinus.


&

それゆえ、別のしかたで以下のように言われるべきである。
聖霊は息子と同じように自らの発出によって父の本性を受け取るが「生まれ
 た」と言われない。それと同じように、父に似た形象を受け取るが似像と言
 われない。なぜなら息子は言葉として発出するが、言葉の性格には
 \kenten{それ}から発出する\kenten{それ}への形象の類似が含まれるのに対
 して、愛の性格にはこのことが含まれないからである。聖霊である愛が、神
 の愛である限りにおいては、それが含まれるとしても。


\\



{\scshape Ad primum ergo dicendum} quod Damascenus et alii doctores Graecorum
communiter utuntur nomine imaginis pro perfecta similitudine.


&

第一異論に対してはそれゆえ以下のように言われるべきである。
ダマスケヌスや他のギリシアの教師たちは、共通に、似像という言葉を完全な
 類似という意味で用いている。

\\



{\scshape Ad secundum dicendum} quod, licet spiritus sanctus sit similis patri et
filio, non tamen sequitur quod sit imago, ratione iam dicta.


&

第二異論に対しては以下のように言われるべきである。
聖霊が父と息子に似ているとしても、すでに述べられた理由によって、それが
 似像であることは帰結しない。

\\



{\scshape Ad tertium dicendum} quod imago alicuius dupliciter in aliquo
invenitur. Uno modo, in re eiusdem naturae secundum speciem, ut imago
regis invenitur in filio suo. Alio modo, in re alterius naturae, sicut
imago regis invenitur in denario. Primo autem modo, filius est imago
patris, secundo autem modo dicitur homo imago Dei. Et ideo ad
designandam in homine imperfectionem imaginis, homo non solum dicitur
imago, sed {\itshape ad imaginem}, per quod motus quidam tendentis in
perfectionem designatur. Sed de filio Dei non potest dici quod sit {\itshape ad
imaginem}, quia est perfecta patris imago.


&

第三異論に対しては以下のように言われるべきである。
あるものの類似は二つのしかたであるものにおいて見出される。
一つには、王の類似が彼の息子の中に見出されるように、種において同じ本性
 をもつ事物において見出される場合である。もう一つは、王の似像が貨幣
 において見出される場合のように、別の本性の事物において見出される場合
 である。
第一のしかたで、息子は父の似像であり、第二のしかたで人間は神の似姿であ
 る。ゆえに人間における似像の不完全性を示すために、人間は似像と言われ
 るだけでなく「似像に向けて」と言われ、これによって、完全性へ向かうあ
 る種の運動が示される。しかし神の息子について「似像に向けて」とは言わ
 れえない。なぜなら父の完全な似像であるから。


\end{longtable}
\newpage




\end{document}