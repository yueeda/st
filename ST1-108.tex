\documentclass[10pt]{jsarticle} % use larger type; default would be 10pt
%\usepackage[utf8]{inputenc} % set input encoding (not needed with XeLaTeX)
%\usepackage[round,comma,authoryear]{natbib}
%\usepackage{nruby}
\usepackage{okumacro}
\usepackage{longtable}
%\usepqckage{tablefootnote}
\usepackage[polutonikogreek,english,japanese]{babel}
\usepackage[polutonikogreek]{betababel}
%\usepackage{amsmath}
\usepackage{latexsym}
\usepackage{color}

%----- header -------
\usepackage{fancyhdr}
\lhead{{\it Summa Theologiae} I, q.~108}
%--------------------

\bibliographystyle{jplain}

\title{{\bf PRIMA PARS}\\{\HUGE Summae Theologiae}\\Sancti Thomae
Aquinatis\\{\sffamily QUEAESTIO CENTESIMAOCTAVA}\\DE ORDINATIONE ANGELORUM\\SECUNDUM HIERARCHIAS ET ORDINES}
\author{Japanese translation\\by Yoshinori {\sc Ueeda}}
\date{Last modified \today}


%%%% コピペ用
%\rhead{a.~}
%\begin{center}
% {\Large {\bf }}\\
% {\large }\\
% {\footnotesize }\\
% {\Large \\}
%\end{center}
%
%\begin{longtable}{p{21em}p{21em}}
%
%&
%
%
%\\
%\end{longtable}
%\newpage



\begin{document}
\maketitle
\pagestyle{fancy}

\begin{center}
{\Large 第108問\\位階と階層における天使の配置について}\\
\end{center}

\begin{longtable}{p{21em}p{21em}}
Deinde considerandum est de ordinatione Angelorum secundum hierarchias
et ordines, dictum est enim quod superiores inferiores illuminant, et
non e converso. Et circa hoc quaeruntur octo. 

\begin{enumerate}
 \item utrum omnes Angeli sint unius hierarchiae. 
 \item utrum in una hierarchia sit unus tantum ordo.
 \item utrum in uno ordine sint plures Angeli.
 \item utrum distinctio hierarchiarum et ordinum sit a natura.
 \item de nominibus et proprietatibus singulorum ordinum.
 \item de comparatione ordinum ad invicem.
 \item utrum ordines durent post diem iudicii.
 \item utrum homines assumantur ad ordines Angelorum.
\end{enumerate}

&

上位の天使は下位の天使を照明するが、その逆はしないと語られたので、次に、
位階\footnote{「位階」と訳しているhierarchiaは、ギリシア語 \bcode{i(/eros}(聖なる)+ \bcode{a)rxo/s}(指導者)に由来する抽象名詞で、「聖なる指導権」「聖なる指導力」という意味。「ヒエラルキー」「位階」という言葉がもつ「階層構造」「階級」という意味はない。偽ディオニュシウスの『天上階級論』で天使の階級を論じるときに、階級を指すのにこの言葉が使われて意味が変化した。本問の論述でもhierarchiaの「指導力」と「階級」の二つの意味が混在している。}と階層における天使の配置について考察されるべきである。
そしてこれを巡って八つのことが問われる。

\begin{enumerate}
 \item すべての天使は一つの位階に属するか。
 \item 一つの位階の中にただ一つの階層があるか。
 \item 一つの階層の中に複数の天使がいるか。
 \item 位階と階層の区別は本性によるか。
 \item それぞれの階層の名称と固有性について。
 \item 階層相互の比較について。
 \item 階層は審判の日のあとも続くか。
 \item 人間が天使たちの階層へ迎えられるか。
\end{enumerate}

\end{longtable}
\newpage


\rhead{a.~1}
\begin{center}
{\Large {\bf ARTICULUS PRIMUS}}\\
{\large UTRUM OMNES ANGELI SINT UNIUS HIERARCHIAE}\\
{\footnotesize Part.III, q.8, a.4; II {\itshape Sent.}, d.9, a.3; IV,
 d.24, q.2, a.1, qu$^{a}$. 2, ad 4; {\itshape Ephes.}, cap.1, lect.7.}\\
{\Large 第一項\\すべての天使が一つの位階に属するか}
\end{center}

\begin{longtable}{p{21em}p{21em}}

{\scshape Ad primum sic proceditur}. Videtur quod omnes Angeli sint unius
hierarchiae. Cum enim Angeli sint supremi inter creaturas, oportet
dicere quod sint optime dispositi. Sed optima dispositio est
multitudinis secundum quod continetur sub uno principatu; ut patet per
philosophum, XII {\itshape Metaphys}., et in III {\itshape Politic}. Cum ergo hierarchia
nihil sit aliud quam sacer principatus, videtur quod omnes Angeli sint
unius hierarchiae.

&

第一項の問題へ、議論は以下のように進められる。すべての天使は一つの位階
に属すると思われる。理由は以下の通り。天使たちは被造物の中で最上位であ
るので、最もよく配置されていると言うべきである。しかるに最高の配置は、
一人の指導者のもとに含まれるかぎりにおいて、多数のものに属する。これは
『形而上学』第12巻と『自然学』第3巻の哲学者によって明らかなとおりであ
る。ゆえに、位階とは聖なる指導力に他ならないので、すべての天使は一つの位
階に属すると思われる。

\\



2.~{\scshape Praeterea}, Dionysius dicit, in {\scshape iii} cap.~{\itshape Cael.~Hier}.,
quod {\itshape hierarchia est ordo, scientia et actio}. Sed omnes
Angeli conveniunt in uno ordine ad Deum, quem cognoscunt, et a quo in
suis actionibus regulantur. Ergo omnes Angeli sunt unius hierarchiae.

&

さらに、ディオニュシウスは『天上階級論』第3章で、「位階(指導力)とは秩序、知、働き
である」と述べている。しかしすべての天使は神への一つの秩序の中にあり、そ
して神によって自らの働きにおいて規定されている。ゆえにすべての天使は一
つの位階に属する。


\\


3. {\scshape Praeterea}, sacer principatus, qui dicitur hierarchia, invenitur in
hominibus et Angelis. Sed omnes homines sunt unius hierarchiae. Ergo
etiam omnes Angeli sunt unius hierarchiae.

&

さらに、「位階」と呼ばれる聖なる指導力は、人間にも天使にも見出される。し
かるに、すべての人間は一つの位階に属する。ゆえにすべての天使も一つの
位階に属する。

\\



{\scshape Sed contra est} quod Dionysius, {\scshape vi} cap.~{\itshape Cael.~Hier}., distinguit tres
hierarchias Angelorum.

&

しかし反対に、ディオニュシウスは『天上階級論』第6章で、天使たちの三つ
の位階を区別している。

\\


{\scshape Respondeo dicendum} quod hierarchia est sacer principatus, ut dictum
est. In nomine autem principatus duo intelliguntur, scilicet ipse
princeps, et multitudo ordinata sub principe. Quia igitur unus est
Deus princeps non solum omnium Angelorum, sed etiam hominum, et totius
creaturae; ideo non solum omnium Angelorum, sed etiam totius
rationalis creaturae, quae sacrorum particeps esse potest, una est
hierarchia, secundum quod Augustinus dicit, in XII de Civ. 


&

解答する。以下のように言われるべきである。すでに述べられたとおり、位階
は聖なる指導力である。しかるに指導力という名称においては、二つのことが
知解される。すなわち、指導者自身と、その指導者の下に秩序付けられた多数
のものである。ゆえに、指導者である神は、すべての天使だけでなく人間たちや
全被造物に一人なので、すべての天使だけでなく、聖者たちのメンバーであり
うる全理性的被造物に、一つの位階が属する。これはアウグスティヌスが
『神の国』第12巻で述べていることによる。



\\


Dei duas esse civitates, hoc est societates, unam in bonis Angelis et
hominibus, alteram in malis. Sed si consideretur principatus ex parte
multitudinis ordinatae sub principe, sic unus principatus dicitur
secundum quod multitudo uno et eodem modo potest gubernationem
principis recipere.


&


\\


Quae vero non possunt secundum eundem modum
gubernari a principe, ad diversos principatus pertinent, sicut sub uno
rege sunt diversae civitates, quae diversis reguntur legibus et
ministris. 


&


\\

Manifestum est autem quod homines alio modo divinas
illuminationes percipiunt quam Angeli, nam Angeli percipiunt eas in
intelligibili puritate, homines vero percipiunt eas sub sensibilium
similitudinibus, ut Dionysius dicit I cap. Cael. Hier. 



&


\\

Et ideo
oportuit distingui humanam hierarchiam ab angelica. Et per eundem
modum in Angelis tres hierarchiae distinguuntur. Dictum est enim
supra, dum de cognitione Angelorum ageretur, quod superiores Angeli
habent universaliorem cognitionem veritatis quam inferiores. 


&


\\


Huiusmodi
autem universalis acceptio cognitionis secundum tres gradus in Angelis
distingui potest. Possunt enim rationes rerum de quibus Angeli
illuminantur, considerari tripliciter. 

&


\\


Primo quidem, secundum quod
procedunt a primo principio universali, quod est Deus, et iste modus
convenit primae hierarchiae, quae immediate ad Deum extenditur, et
quasi in vestibulis Dei collocatur, ut Dionysius dicit VII
cap. Cael. Hier. 


&


\\


Secundo vero, prout huiusmodi rationes dependent ab
universalibus causis creatis, quae iam aliquo modo multiplicantur, et
hic modus convenit secundae hierarchiae. 



&


\\

Tertio autem modo, secundum
quod huiusmodi rationes applicantur singulis rebus, et prout dependent
a propriis causis, et hic modus convenit infimae hierarchiae. Quod
plenius patebit, cum de singulis ordinibus agetur. 



&


\\


Sic igitur
distinguuntur hierarchiae ex parte multitudinis subiectae. Unde
manifestum est eos errare, et contra intentionem Dionysii loqui, qui
ponunt in divinis personis hierarchiam quam vocant supercaelestem. 



&


\\


In
divinis enim personis est quidam ordo naturae, sed non
hierarchiae. Nam, ut Dionysius dicit III cap. Cael. Hier., ordo
hierarchiae est alios quidem purgari et illuminari et perfici, alios
autem purgare et illuminare et perficere. Quod absit ut in divinis
personis ponamus.

&


\\


[32927] Iª q. 108 a. 1 ad 1
Ad primum ergo dicendum quod ratio illa procedit de principatu ex
parte principis, quia optimum est quod multitudo regatur ab uno
principe, ut philosophus in praedictis locis intendit.

&


\\


[32928] Iª q. 108 a. 1 ad 2
Ad secundum dicendum quod, quantum ad cognitionem ipsius Dei, quem
omnes uno modo, scilicet per essentiam, vident, non distinguuntur in
Angelis hierarchiae, sed quantum ad rationes rerum creatarum, ut
dictum est.

&


\\


[32929] Iª q. 108 a. 1 ad 3
Ad tertium dicendum quod omnes homines sunt unius speciei, et unus
modus intelligendi est eis connaturalis, non sic autem est in
Angelis. Unde non est similis ratio.

&




\end{longtable}
\newpage

\end{document}



%\rhead{a.~}
%\begin{center}
% {\Large {\bf }}\\
% {\large }\\
% {\footnotesize }\\
% {\Large \\}
%\end{center}
%
%\begin{longtable}{p{21em}p{21em}}
%
%&
%
%
%\\
%\end{longtable}
%\newpage

ARTICULUS 2
[32930] Iª q. 108 a. 2 arg. 1
Ad secundum sic proceditur. Videtur quod in una hierarchia non sint plures ordines. Multiplicata enim definitione, multiplicatur et definitum. Sed hierarchia, ut Dionysius dicit, est ordo. Si ergo sunt multi ordines, non erit una hierarchia, sed multae.

[32931] Iª q. 108 a. 2 arg. 2
Praeterea, diversi ordines sunt diversi gradus. Sed gradus in spiritualibus constituuntur secundum diversa dona spiritualia. Sed in Angelis omnia dona spiritualia sunt communia, quia nihil ibi singulariter possidetur. Ergo non sunt diversi ordines Angelorum.

[32932] Iª q. 108 a. 2 arg. 3
Praeterea, in ecclesiastica hierarchia distinguuntur ordines secundum purgare, illuminare et perficere, nam ordo diaconorum est purgativus, sacerdotum illuminativus, episcoporum perfectivus, ut Dionysius dicit V cap. Eccles. Hier. Sed quilibet Angelus purgat, illuminat et perficit. Non ergo est distinctio ordinum in Angelis.

[32933] Iª q. 108 a. 2 s. c.
Sed contra est quod apostolus dicit ad Ephes. I, quod Deus constituit Christum hominem supra omnem principatum et potestatem et virtutem et dominationem; qui sunt diversi ordines Angelorum, et quidam eorum ad unam hierarchiam pertinent, ut infra patebit.

[32934] Iª q. 108 a. 2 co.
Respondeo dicendum quod, sicut dictum est, una hierarchia est unus principatus, idest una multitudo ordinata uno modo sub principis gubernatione. Non autem esset multitudo ordinata, sed confusa, si in multitudine diversi ordines non essent. Ipsa ergo ratio hierarchiae requirit ordinum diversitatem. Quae quidem diversitas ordinum secundum diversa officia et actus consideratur. Sicut patet quod in una civitate sunt diversi ordines secundum diversos actus, nam alius est ordo iudicantium, alius pugnantium, alius laborantium in agris, et sic de aliis. Sed quamvis multi sint unius civitatis ordines, omnes tamen ad tres possunt reduci, secundum quod quaelibet multitudo perfecta habet principium, medium et finem. Unde et in civitatibus triplex ordo hominum invenitur, quidam enim sunt supremi, ut optimates; quidam autem sunt infimi, ut vilis populus; quidam autem sunt medii, ut populus honorabilis. Sic igitur et in qualibet hierarchia angelica ordines distinguuntur secundum diversos actus et officia; et omnis ista diversitas ad tria reducitur, scilicet ad summum, medium et infimum. Et propter hoc in qualibet hierarchia Dionysius ponit tres ordines.

[32935] Iª q. 108 a. 2 ad 1
Ad primum ergo dicendum quod ordo dupliciter dicitur. Uno modo, ipsa ordinatio comprehendens sub se diversos gradus, et hoc modo hierarchia dicitur ordo. Alio modo dicitur ordo gradus unus, et sic dicuntur plures ordines unius hierarchiae.

[32936] Iª q. 108 a. 2 ad 2
Ad secundum dicendum quod in societate Angelorum omnia possidentur communiter; sed tamen quaedam excellentius habentur a quibusdam quam ab aliis. Unumquodque autem perfectius habetur ab eo qui potest illud communicare, quam ab eo qui non potest, sicut perfectius est calidum quod potest calefacere, quam quod non potest; et perfectius scit qui potest docere, quam qui non potest. Et quanto perfectius donum aliquis communicare potest, tanto in perfectiori gradu est, sicut in perfectiori gradu magisterii est qui potest docere altiorem scientiam. Et secundum hanc similitudinem consideranda est diversitas graduum vel ordinum in Angelis, secundum diversa officia et actus.

[32937] Iª q. 108 a. 2 ad 3
Ad tertium dicendum quod inferior Angelus est superior supremo homine nostrae hierarchiae; secundum illud Matth. XI, qui minor est in regno caelorum, maior est illo, scilicet Ioanne Baptista, quo nullus maior inter natos mulierum surrexit. Unde minor Angelus caelestis hierarchiae potest non solum purgare sed illuminare et perficere, et altiori modo quam ordines nostrae hierarchiae. Et sic secundum distinctionem harum actionum non distinguuntur caelestes ordines; sed secundum alias differentias actionum.


%\rhead{a.~}
%\begin{center}
% {\Large {\bf }}\\
% {\large }\\
% {\footnotesize }\\
% {\Large \\}
%\end{center}
%
%\begin{longtable}{p{21em}p{21em}}
%
%&
%
%
%\\
%\end{longtable}
%\newpage


ARTICULUS 3
[32938] Iª q. 108 a. 3 arg. 1
Ad tertium sic proceditur. Videtur quod in uno ordine non sint plures Angeli. Dictum est enim supra omnes Angelos inaequales esse ad invicem. Sed unius ordinis esse dicuntur quae sunt aequalia. Ergo plures Angeli non sunt unius ordinis.

[32939] Iª q. 108 a. 3 arg. 2
Praeterea, quod potest sufficienter fieri per unum, superfluum est quod fiat per multa. Sed illud quod pertinet ad unum officium angelicum, sufficienter potest fieri per unum Angelum; multo magis quam per unum solem sufficienter fit quod pertinet ad officium solis, quanto perfectior est Angelus caelesti corpore. Si ergo ordines distinguuntur secundum officia, ut dictum est, superfluum est quod sint plures Angeli unius ordinis.

[32940] Iª q. 108 a. 3 arg. 3
Praeterea, supra dictum est quod omnes Angeli sunt inaequales. Si ergo plures Angeli sint unius ordinis, puta tres vel quatuor, infimus superioris ordinis magis conveniet cum supremo inferioris quam cum supremo sui ordinis. Et sic non videtur quod magis sit unius ordinis cum hoc, quam cum illo. Non igitur sunt plures Angeli unius ordinis.

[32941] Iª q. 108 a. 3 s. c.
Sed contra est quod Isaiae VI dicitur, quod Seraphim clamabant alter ad alterum. Sunt ergo plures Angeli in uno ordine Seraphim.

[32942] Iª q. 108 a. 3 co.
Respondeo dicendum quod ille qui perfecte cognoscit res aliquas, potest usque ad minima et actus et virtutes et naturas earum distinguere. Qui autem cognoscit eas imperfecte, non potest distinguere nisi in universali, quae quidem distinctio fit per pauciora. Sicut qui imperfecte cognoscit res naturales, distinguit earum ordines in universali, ponens in uno ordine caelestia corpora, in alio corpora inferiora inanimata, in alio plantas, in alio animalia, qui autem perfectius cognosceret res naturales, posset distinguere et in ipsis corporibus caelestibus diversos ordines, et in singulis aliorum. Nos autem imperfecte Angelos cognoscimus, et eorum officia, ut Dionysius dicit VI cap. Cael. Hier. Unde non possumus distinguere officia et ordines Angelorum, nisi in communi; secundum quem modum, multi Angeli sub uno ordine continentur. Si autem perfecte cognosceremus officia Angelorum, et eorum distinctiones, perfecte sciremus quod quilibet Angelus habet suum proprium officium et suum proprium ordinem in rebus, multo magis quam quaelibet stella, etsi nos lateat.

[32943] Iª q. 108 a. 3 ad 1
Ad primum ergo dicendum quod omnes Angeli unius ordinis sunt aliquo modo aequales, quantum ad communem similitudinem secundum quam constituuntur in uno ordine, sed simpliciter non sunt aequales. Unde Dionysius dicit, X cap. Cael. Hier., quod in uno et eodem ordine Angelorum, est accipere primos, medios et ultimos.

[32944] Iª q. 108 a. 3 ad 2
Ad secundum dicendum quod illa specialis distinctio ordinum et officiorum secundum quam quilibet Angelus habet proprium officium et ordinem, est nobis ignota.

[32945] Iª q. 108 a. 3 ad 3
Ad tertium dicendum quod, sicut in superficie quae partim est alba et partim nigra, duae partes quae sunt in confinio albi et nigri, magis conveniunt secundum situm quam aliquae duae partes albae, minus tamen secundum qualitatem; ita duo Angeli qui sunt in terminis duorum ordinum, magis secum conveniunt secundum propinquitatem naturae, quam unus eorum cum aliquibus aliis sui ordinis; minus autem secundum idoneitatem ad similia officia, quae quidem idoneitas usque ad aliquem certum terminum protenditur.


%\rhead{a.~}
%\begin{center}
% {\Large {\bf }}\\
% {\large }\\
% {\footnotesize }\\
% {\Large \\}
%\end{center}
%
%\begin{longtable}{p{21em}p{21em}}
%
%&
%
%
%\\
%\end{longtable}
%\newpage


ARTICULUS 4
[32946] Iª q. 108 a. 4 arg. 1
Ad quartum sic proceditur. Videtur quod distinctio hierarchiarum et ordinum non sit a natura in Angelis. Hierarchia enim dicitur sacer principatus, et in definitione eius Dionysius ponit quod deiforme, quantum possibile est, similat. Sed sanctitas et deiformitas est in Angelis per gratiam, non per naturam. Ergo distinctio hierarchiarum et ordinum in Angelis est per gratiam, non per naturam.

[32947] Iª q. 108 a. 4 arg. 2
Praeterea, Seraphim dicuntur ardentes, vel incendentes, ut Dionysius dicit VII cap. Cael. Hier. Hoc autem videtur ad caritatem pertinere, quae non est a natura, sed a gratia, diffunditur enim in cordibus nostris per spiritum sanctum, qui datus est nobis, ut dicitur ad Rom. V. Quod non solum ad sanctos homines pertinet, sed etiam de sanctis Angelis dici potest, ut Augustinus dicit XII de Civ. Dei. Ergo ordines in Angelis non sunt a natura, sed a gratia.

[32948] Iª q. 108 a. 4 arg. 3
Praeterea, hierarchia ecclesiastica exemplatur a caelesti. Sed ordines in hominibus non sunt per naturam, sed per donum gratiae, non enim est a natura quod unus est episcopus, et alius est sacerdos, et alius diaconus. Ergo neque in Angelis sunt ordines a natura, sed a gratia tantum.

[32949] Iª q. 108 a. 4 s. c.
Sed contra est quod Magister dicit, IX dist. II Sent., quod ordo Angelorum dicitur multitudo caelestium spirituum, qui inter se aliquo munere gratiae similantur, sicut et naturalium datorum participatione conveniunt. Distinctio ergo ordinum in Angelis est non solum secundum dona gratuita, sed etiam secundum dona naturalia.

[32950] Iª q. 108 a. 4 co.
Respondeo dicendum quod ordo gubernationis, qui est ordo multitudinis sub principatu existentis, attenditur per respectum ad finem. Finis autem Angelorum potest accipi dupliciter. Uno modo, secundum facultatem suae naturae, ut scilicet cognoscant et ament Deum naturali cognitione et amore. Et secundum respectum ad hunc finem, distinguuntur ordines Angelorum secundum naturalia dona. Alio modo potest accipi finis angelicae multitudinis supra naturalem facultatem eorum, qui consistit in visione divinae essentiae, et in immobili fruitione bonitatis ipsius; ad quem finem pertingere non possunt nisi per gratiam. Unde secundum respectum ad hunc finem, ordines distinguuntur in Angelis completive quidem secundum dona gratuita, dispositive autem secundum dona naturalia, quia Angelis data sunt dona gratuita secundum capacitatem naturalium, quod non est in hominibus, ut supra dictum est. Unde in hominibus distinguuntur ordines secundum dona gratuita tantum, et non secundum naturam.

[32951] Iª q. 108 a. 4 ad arg.
Et per hoc patet responsio ad obiecta.


%\rhead{a.~}
%\begin{center}
% {\Large {\bf }}\\
% {\large }\\
% {\footnotesize }\\
% {\Large \\}
%\end{center}
%
%\begin{longtable}{p{21em}p{21em}}
%
%&
%
%
%\\
%\end{longtable}
%\newpage


ARTICULUS 5
[32952] Iª q. 108 a. 5 arg. 1
Ad quintum sic proceditur. Videtur quod ordines Angelorum non convenienter nominentur. Omnes enim caelestes spiritus dicuntur et Angeli et virtutes caelestes. Sed nomina communia inconvenienter aliquibus appropriantur. Ergo inconvenienter nominatur unus ordo Angelorum, et alius virtutum.

[32953] Iª q. 108 a. 5 arg. 2
Praeterea, esse dominum est proprium Dei; secundum illud Psal. XCIX scitote quoniam dominus ipse est Deus. Ergo inconvenienter unus ordo caelestium spirituum dominationes vocatur.

[32954] Iª q. 108 a. 5 arg. 3
Praeterea, nomen dominationis ad gubernationem pertinere videtur. Similiter autem et nomen principatuum, et potestatum. Inconvenienter ergo tribus ordinibus haec tria nomina imponuntur.

[32955] Iª q. 108 a. 5 arg. 4
Praeterea, Archangeli dicuntur quasi principes Angeli. Non ergo hoc nomen debet imponi alii ordini quam ordini principatuum.

[32956] Iª q. 108 a. 5 arg. 5
Praeterea, nomen Seraphim imponitur ab ardore qui ad caritatem pertinet, nomen autem Cherubim imponitur a scientia. Caritas autem et scientia sunt dona communia omnibus Angelis. Non ergo debent esse nomina specialium ordinum.

[32957] Iª q. 108 a. 5 arg. 6
Praeterea, throni dicuntur sedes. Sed ex hoc ipso Deus in creatura rationali sedere dicitur, quod ipsum cognoscit et amat. Non ergo debet esse alius ordo thronorum ab ordine Cherubim et Seraphim. Sic igitur videtur quod inconvenienter ordines Angelorum nominentur.

[32958] Iª q. 108 a. 5 s. c.
Sed contra est auctoritas sacrae Scripturae, quae sic eos nominat. Nomen enim Seraphim ponitur Isaiae VI; nomen Cherubim Ezech. I; nomen thronorum, Coloss. I; dominationes autem et virtutes et potestates et principatus ponuntur Ephes. I; nomen autem Archangeli ponitur in canonica Iudae, nomina autem Angelorum in pluribus Scripturae locis.

[32959] Iª q. 108 a. 5 co.
Respondeo dicendum quod in nominatione angelicorum ordinum, considerare oportet quod propria nomina singulorum ordinum proprietates eorum designant, ut Dionysius dicit VII cap. Cael. Hier. Ad videndum autem quae sit proprietas cuiuslibet ordinis, considerare oportet quod in rebus ordinatis tripliciter aliquid esse contingit, scilicet per proprietatem, per excessum, et per participationem. Per proprietatem autem dicitur esse aliquid in re aliqua, quod adaequatur et proportionatur naturae ipsius. Per excessum autem, quando illud quod attribuitur alicui, est minus quam res cui attribuitur, sed tamen convenit illi rei per quendam excessum; sicut dictum est de omnibus nominibus quae attribuuntur Deo. Per participationem autem, quando illud quod attribuitur alicui, non plenarie invenitur in eo, sed deficienter; sicut sancti homines participative dicuntur dii. Si ergo aliquid nominari debeat nomine designante proprietatem ipsius, non debet nominari ab eo quod imperfecte participat, neque ab eo quod excedenter habet; sed ab eo quod est sibi quasi coaequatum. Sicut si quis velit proprie nominare hominem, dicet eum substantiam rationalem, non autem substantiam intellectualem, quod est proprium nomen Angeli, quia simplex intelligentia convenit Angelo per proprietatem, homini vero per participationem; neque substantiam sensibilem, quod est nomen bruti proprium, quia sensus est minus quam id quod est proprium homini, et convenit homini excedenter prae aliis animalibus. Sic igitur considerandum est in ordinibus Angelorum, quod omnes spirituales perfectiones sunt omnibus Angelis communes et quod omnes abundantius existunt in superioribus quam in inferioribus. Sed cum in ipsis etiam perfectionibus sit quidam gradus, superior perfectio attribuitur superiori ordini per proprietatem, inferiori vero per participationem, e converso autem inferior attribuitur inferiori per proprietatem, superiori autem per excessum. Et ita superior ordo a superiori perfectione nominatur. Sic igitur Dionysius exponit ordinum nomina secundum convenientiam ad spirituales perfectiones eorum. Gregorius vero, in expositione horum nominum, magis attendere videtur exteriora ministeria. Dicit enim, quod Angeli dicuntur qui minima nuntiant; Archangeli, qui summa; virtutes per quas miracula fiunt; potestates quibus adversae potestates repelluntur; principatus, qui ipsis bonis spiritibus praesunt.

[32960] Iª q. 108 a. 5 ad 1
Ad primum ergo dicendum quod Angelus nuntius dicitur. Omnes ergo caelestes spiritus, inquantum sunt manifestatores divinorum, Angeli vocantur. Sed superiores Angeli habent quandam excellentiam in hac manifestatione, a qua superiores ordines nominantur. Infimus autem Angelorum ordo nullam excellentiam supra communem manifestationem addit, et ideo a simplici manifestatione nominatur. Et sic nomen commune remanet infimo ordini quasi proprium, ut dicit Dionysius V cap. Cael. Hier. Vel potest dici quod infimus ordo specialiter dicitur ordo Angelorum, quia immediate nobis annuntiant. Virtus autem dupliciter accipi potest. Uno modo, communiter, secundum quod est media inter essentiam et operationem, et sic omnes caelestes spiritus nominantur caelestes virtutes, sicut et caelestes essentiae. Alio modo, secundum quod importat quendam excessum fortitudinis, et sic est proprium nomen ordinis. Unde Dionysius dicit, VIII cap. Cael. Hier., quod nomen virtutum significat quandam virilem et inconcussam fortitudinem, primo quidem ad omnes operationes divinas eis convenientes; secundo, ad suscipiendum divina. Et ita significat quod sine aliquo timore aggrediuntur divina quae ad eos pertinent, quod videtur ad fortitudinem animi pertinere.

[32961] Iª q. 108 a. 5 ad 2
Ad secundum dicendum quod, sicut dicit Dionysius XII cap. de Div. Nom., dominatio laudatur in Deo singulariter per quendam excessum, sed per participationem, divina eloquia vocant dominos principaliores ornatus, per quos inferiores ex donis eius accipiunt. Unde et Dionysius dicit in VIII cap. Cael. Hier., quod nomen dominationum primo quidem significat quandam libertatem, quae est a servili conditione et pedestri subiectione, sicut plebs subiicitur, et a tyrannica oppressione, quam interdum etiam maiores patiuntur. Secundo significat quandam rigidam et inflexibilem gubernationem, quae ad nullum servilem actum inclinatur, neque ad aliquem actum subiectorum vel oppressorum a tyrannis. Tertio significat appetitum et participationem veri dominii, quod est in Deo. Et similiter nomen cuiuslibet ordinis significat participationem eius quod est in Deo; sicut nomen virtutum significat participationem divinae virtutis; et sic de aliis.

[32962] Iª q. 108 a. 5 ad 3
Ad tertium dicendum quod nomen dominationis, et potestatis, et principatus, diversimode ad gubernationem pertinet. Nam domini est solummodo praecipere de agendis. Et ideo Gregorius dicit quod quaedam Angelorum agmina, pro eo quod eis cetera ad obediendum subiecta sunt, dominationes vocantur. Nomen vero potestatis ordinationem quandam designat; secundum illud apostoli ad Rom. XIII, qui potestati resistit, Dei ordinationi resistit. Et ideo Dionysius dicit quod nomen potestatis significat quandam ordinationem et circa susceptionem divinorum, et circa actiones divinas quas superiores in inferiores agunt, eas sursum ducendo. Ad ordinem ergo potestatum pertinet ordinare quae a subditis sint agenda. Principari vero, ut Gregorius dicit, est inter reliquos priorem existere, quasi primi sint in executione eorum quae imperantur. Et ideo Dionysius dicit, IX cap. Cael. Hier., quod nomen principatuum significat ductivum cum ordine sacro. Illi enim qui alios ducunt, primi inter eos existentes, principes proprie vocantur secundum illud Psalmi LXVII, praevenerunt principes coniuncti psallentibus.

[32963] Iª q. 108 a. 5 ad 4
Ad quartum dicendum quod Archangeli, secundum Dionysium, medii sunt inter principatus et Angelos. Medium autem comparatum uni extremo, videtur alterum, inquantum participat naturam utriusque, sicut tepidum respectu calidi est frigidum, respectu vero frigidi est calidum. Sic et Archangeli dicuntur quasi principes Angeli, quia respectu Angelorum sunt principes, respectu vero principatuum sunt Angeli. Secundum Gregorium autem, dicuntur Archangeli ex eo quod principantur soli ordini Angelorum, quasi magna nuntiantes. Principatus autem dicuntur ex eo quod principantur omnibus caelestibus virtutibus divinas iussiones explentibus.

[32964] Iª q. 108 a. 5 ad 5
Ad quintum dicendum quod nomen Seraphim non imponitur tantum a caritate, sed a caritatis excessu, quem importat nomen ardoris vel incendii. Unde Dionysius, VII cap. Cael. Hier., exponit nomen Seraphim secundum proprietates ignis, in quo est excessus caliditatis. In igne autem tria possumus considerare. Primo quidem, motum, qui est sursum, et qui est continuus. Per quod significatur quod indeclinabiliter moventur in Deum. Secundo vero, virtutem activam eius, quae est calidum. Quod quidem non simpliciter invenitur in igne, sed cum quadam acuitate, quia maxime est penetrativus in agendo, et pertingit usque ad minima; et iterum cum quodam superexcedenti fervore. Et per hoc significatur actio huiusmodi Angelorum, quam in subditos potenter exercent, eos in similem fervorem excitantes, et totaliter eos per incendium purgantes. Tertio consideratur in igne claritas eius. Et hoc significat quod huiusmodi Angeli in seipsis habent inextinguibilem lucem, et quod alios perfecte illuminant. Similiter etiam nomen Cherubim imponitur a quodam excessu scientiae, unde interpretatur plenitudo scientiae. Quod Dionysius exponit quantum ad quatuor, primo quidem, quantum ad perfectam Dei visionem; secundo, quantum ad plenam susceptionem divini luminis; tertio, quantum ad hoc, quod in ipso Deo contemplantur pulchritudinem ordinis rerum a Deo derivatam; quarto, quantum ad hoc, quod ipsi pleni existentes huiusmodi cognitione, eam copiose in alios effundunt.

[32965] Iª q. 108 a. 5 ad 6
Ad sextum dicendum quod ordo thronorum habet excellentiam prae inferioribus ordinibus in hoc, quod immediate in Deo rationes divinorum operum cognoscere possunt. Sed Cherubim habent excellentiam scientiae; Seraphim vero excellentiam ardoris. Et licet in his duabus excellentiis includatur tertia, non tamen in illa quae est thronorum, includuntur aliae duae. Et ideo ordo thronorum distinguitur ab ordine Cherubim et Seraphim. Hoc enim est commune in omnibus, quod excellentia inferioris continetur in excellentia superioris, et non e converso. Exponit autem Dionysius nomen thronorum, per convenientiam ad materiales sedes. In quibus est quatuor considerare. Primo quidem, situm, quia sedes supra terram elevantur. Et sic ipsi Angeli qui throni dicuntur, elevantur usque ad hoc, quod in Deo immediate rationes rerum cognoscant. Secundo in materialibus sedibus consideratur firmitas, quia in ipsis aliquis firmiter sedet. Hic autem est e converso, nam ipsi Angeli firmantur per Deum. Tertio, quia sedes suscipit sedentem, et in ea deferri potest. Sic et isti Angeli suscipiunt Deum in seipsis, et eum quodammodo ad inferiores ferunt. Quarto, ex figura, quia sedes ex una parte est aperta ad suscipiendum sedentem. Ita et isti Angeli sunt per promptitudinem aperti ad suscipiendum Deum, et famulandum ipsi.


%\rhead{a.~}
%\begin{center}
% {\Large {\bf }}\\
% {\large }\\
% {\footnotesize }\\
% {\Large \\}
%\end{center}
%
%\begin{longtable}{p{21em}p{21em}}
%
%&
%
%
%\\
%\end{longtable}
%\newpage


ARTICULUS 6
[32966] Iª q. 108 a. 6 arg. 1
Ad sextum sic proceditur. Videtur quod inconvenienter gradus ordinum assignentur. Ordo enim praelatorum videtur esse supremus. Sed dominationes, principatus et potestates ex ipsis nominibus praelationem quandam habent. Ergo isti ordines debent esse inter omnes supremi.

[32967] Iª q. 108 a. 6 arg. 2
Praeterea, quanto aliquis ordo est Deo propinquior, tanto est superior. Sed ordo thronorum videtur esse Deo propinquissimus, nihil enim coniungitur propinquius sedenti, quam sua sedes. Ergo ordo thronorum est altissimus.

[32968] Iª q. 108 a. 6 arg. 3
Praeterea, scientia est prior quam amor; et intellectus videtur esse altior quam voluntas. Ergo et ordo Cherubim videtur esse altior quam ordo Seraphim.

[32969] Iª q. 108 a. 6 arg. 4
Praeterea, Gregorius ponit principatus supra potestates. Non ergo collocantur immediate supra Archangelos, ut Dionysius dicit.

[32970] Iª q. 108 a. 6 s. c.
Sed contra est quod Dionysius ponit, in prima quidem hierarchia. Seraphim ut primos, Cherubim ut medios, thronos ut ultimos; in media vero, dominationes ut primos, virtutes ut medios, potestates ut ultimos; in ultima, principatus ut primos, Archangelos ut medios, Angelos ut ultimos.

[32971] Iª q. 108 a. 6 co.
Respondeo dicendum quod gradus angelicorum ordinum assignant et Gregorius et Dionysius, quantum ad alia quidem convenienter, sed quantum ad principatus et virtutes differenter. Nam Dionysius collocat virtutes sub dominationibus et supra potestates, principatus autem sub potestatibus et supra Archangelos, Gregorius autem ponit principatus in medio dominationum et potestatum, virtutes vero in medio potestatum et Archangelorum. Et utraque assignatio fulcimentum habere potest ex auctoritate apostoli. Qui, medios ordines ascendendo enumerans, dicit, Ephes. I, quod Deus constituit illum, scilicet Christum, ad dexteram suam in caelestibus, supra omnem principatum et potestatem et virtutem et dominationem, ubi virtutem ponit inter potestatem et dominationem, secundum assignationem Dionysii. Sed ad Coloss. I, enumerans eosdem ordines descendendo, dicit, sive throni, sive dominationes, sive principatus, sive potestates, omnia per ipsum et in ipso creata sunt, ubi principatus ponit medios inter dominationes et potestates, secundum assignationem Gregorii. Primo igitur videamus rationem assignationis Dionysii. In qua considerandum est quod, sicut supra dictum est, prima hierarchia accipit rationes rerum in ipso Deo; secunda vero in causis universalibus; tertia vero secundum determinationem ad speciales effectus. Et quia Deus est finis non solum angelicorum ministeriorum, sed etiam totius creaturae, ad primam hierarchiam pertinet consideratio finis; ad mediam vero dispositio universalis de agendis; ad ultimam autem applicatio dispositionis ad effectum, quae est operis executio; haec enim tria manifestum est in qualibet operatione inveniri. Et ideo Dionysius, ex nominibus ordinum proprietates illorum considerans, illos ordines in prima hierarchia posuit, quorum nomina imponuntur per respectum ad Deum, scilicet Seraphim et Cherubim et thronos. Illos vero ordines posuit in media hierarchia, quorum nomina designant communem quandam gubernationem sive dispositionem, scilicet dominationes, virtutes et potestates. Illos vero ordines posuit in tertia hierarchia, quorum nomina designant operis executionem, scilicet principatus, Angelos et Archangelos. In respectu autem ad finem, tria considerari possunt, nam primo, aliquis considerat finem; secundo vero, perfectam finis cognitionem accipit; tertio vero, intentionem suam in ipso defigit; quorum secundum ex additione se habet ad primum, et tertium ad utrumque. Et quia Deus est finis creaturarum sicut dux est finis exercitus, ut dicitur in XII Metaphys., potest aliquid simile huius ordinis considerari in rebus humanis, nam quidam sunt qui hoc habent dignitatis, ut per seipsos familiariter accedere possunt ad regem vel ducem; quidam vero super hoc habent, ut etiam secreta eius cognoscant; alii vero insuper circa ipsum semper inhaerent, quasi ei coniuncti. Et secundum hanc similitudinem accipere possumus dispositionem ordinum primae hierarchiae. Nam throni elevantur ad hoc, quod Deum familiariter in seipsis recipiant, secundum quod rationes rerum in ipso immediate cognoscere possunt, quod est proprium totius primae hierarchiae. Cherubim vero supereminenter divina secreta cognoscunt. Seraphim vero excellunt in hoc quod est omnium supremum, scilicet Deo ipsi uniri. Ut sic ab eo quod est commune toti hierarchiae, denominetur ordo thronorum; sicut ab eo quod est commune omnibus caelestibus spiritibus, denominatur ordo Angelorum. Ad gubernationis autem rationem tria pertinent. Quorum primum est definitio eorum quae agenda sunt, quod est proprium dominationum. Secundum autem est praebere facultatem ad implendum, quod pertinet ad virtutes. Tertium autem est ordinare qualiter ea quae praecepta vel definita sunt, impleri possint, ut aliqui exequantur, et hoc pertinet ad potestates. Executio autem angelicorum ministeriorum consistit in annuntiando divina. In executione autem cuiuslibet actus, sunt quidam quasi incipientes actionem et alios ducentes, sicut in cantu praecentores, et in bello illi qui alios ducunt et dirigunt, et hoc pertinet ad principatus. Alii vero sunt qui simpliciter exequuntur, et hoc pertinet ad Angelos. Alii vero medio modo se habent, quod ad Archangelos pertinet, ut supra dictum est. Invenitur autem congrua haec ordinum assignatio. Nam semper summum inferioris ordinis affinitatem habet cum ultimo superioris; sicut infima animalia parum distant a plantis. Primus autem ordo est divinarum personarum, qui terminatur ad spiritum sanctum, qui est amor procedens, cum quo affinitatem habet supremus ordo primae hierarchiae, ab incendio amoris denominatus. Infimus autem ordo primae hierarchiae est thronorum, qui ex suo nomine habent quandam affinitatem cum dominationibus, nam throni dicuntur, secundum Gregorium, per quos Deus sua iudicia exercet; accipiunt enim divinas illuminationes per convenientiam ad immediate illuminandum secundam hierarchiam, ad quam pertinet dispositio divinorum ministeriorum. Ordo vero potestatum affinitatem habet cum ordine principatuum, nam cum potestatum sit ordinationem subiectis imponere, haec ordinatio statim in nomine principatuum designatur, qui sunt primi in executione divinorum ministeriorum, utpote praesidentes gubernationi gentium et regnorum, quod est primum et praecipuum in divinis ministeriis; nam bonum gentis est divinius quam bonum unius hominis. Unde dicitur Dan. X, princeps regni Persarum restitit mihi. Dispositio etiam ordinum quam Gregorius ponit, congruitatem habet. Nam cum dominationes sint definientes et praecipientes ea quae ad divina ministeria pertinent, ordines eis subiecti disponuntur secundum dispositionem eorum in quos divina ministeria exercentur ut autem Augustinus dicit in III de Trin., corpora quodam ordine reguntur, inferiora per superiora, et omnia per spiritualem creaturam; et spiritus malus per spiritum bonum. Primus ergo ordo post dominationes dicitur principatuum, qui etiam bonis spiritibus principantur. Deinde potestates, per quas arcentur mali spiritus, sicut per potestates terrenas arcentur malefactores, ut habetur Rom. XIII. Post quas sunt virtutes, quae habent potestatem super corporalem naturam in operatione miraculorum. Post quas sunt Archangeli et Angeli, qui nuntiant hominibus vel magna, quae sunt supra rationem; vel parva, ad quae ratio se extendere potest.

[32972] Iª q. 108 a. 6 ad 1
Ad primum ergo dicendum quod in Angelis potius est quod subiiciuntur Deo, quam quod inferioribus praesident, et hoc derivatur ex illo. Et ideo ordines nominati a praelatione non sunt supremi, sed magis ordines nominati a conversione ad Deum.

[32973] Iª q. 108 a. 6 ad 2
Ad secundum dicendum quod illa propinquitas ad Deum quae designatur nomine thronorum, convenit etiam Cherubim et Seraphim, et excellentius, ut dictum est.

[32974] Iª q. 108 a. 6 ad 3
Ad tertium dicendum quod, sicut supra dictum est, cognitio est secundum quod cognita sunt in cognoscente; amor autem secundum quod amans unitur rei amatae. Superiora autem nobiliori modo sunt in seipsis quam in inferioribus, inferiora vero nobiliori modo in superioribus quam in seipsis. Et ideo inferiorum quidem cognitio praeeminet dilectioni, superiorum autem dilectio, et praecipue Dei, praeeminet cognitioni.

[32975] Iª q. 108 a. 6 ad 4
Ad quartum dicendum quod, si quis diligenter consideret dispositiones ordinum secundum Dionysium et Gregorium, parum vel nihil differunt, si ad rem referantur. Exponit enim Gregorius principatuum nomen ex hoc, quod bonis spiritibus praesunt, et hoc convenit virtutibus, secundum quod in nomine virtutum intelligitur quaedam fortitudo dans efficaciam inferioribus spiritibus ad exequenda divina ministeria. Rursus virtutes, secundum Gregorium, videntur esse idem quod principatus secundum Dionysium. Nam hoc est primum in divinis ministeriis, miracula facere, per hoc enim paratur via Annuntiationi Archangelorum et Angelorum.


%\rhead{a.~}
%\begin{center}
% {\Large {\bf }}\\
% {\large }\\
% {\footnotesize }\\
% {\Large \\}
%\end{center}
%
%\begin{longtable}{p{21em}p{21em}}
%
%&
%
%
%\\
%\end{longtable}
%\newpage


ARTICULUS 7
[32976] Iª q. 108 a. 7 arg. 1
Ad septimum sic proceditur. Videtur quod ordines non remanebunt post diem iudicii. Dicit enim apostolus, I ad Cor. XV, quod Christus evacuabit omnem principatum et potestatem, cum tradiderit regnum Deo et patri, quod erit in ultima consummatione. Pari ergo ratione, in illo statu omnes alii ordines evacuabuntur.

[32977] Iª q. 108 a. 7 arg. 2
Praeterea, ad officium angelicorum ordinum pertinet purgare, illuminare et perficere. Sed post diem iudicii unus Angelus non purgabit aut illuminabit aut perficiet alium, quia non proficient amplius in scientia. Ergo frustra ordines angelici remanerent.

[32978] Iª q. 108 a. 7 arg. 3
Praeterea, apostolus dicit, ad Heb. I, de Angelis, quod omnes sunt administratorii spiritus, in ministerium missi propter eos qui haereditatem capiunt salutis, ex quo patet quod officia Angelorum ordinantur ad hoc, quod homines ad salutem adducantur. Sed omnes electi usque ad diem iudicii salutem consequuntur. Non ergo post diem iudicii remanebunt officia et ordines Angelorum.

[32979] Iª q. 108 a. 7 s. c.
Sed contra est quod dicitur Iudic. V, stellae manentes in ordine et cursu suo, quod exponitur de Angelis. Ergo Angeli semper in suis ordinibus remanebunt.

[32980] Iª q. 108 a. 7 co.
Respondeo dicendum quod in ordinibus angelicis duo possunt considerari, scilicet distinctio graduum, et executio officiorum. Distinctio autem graduum est in Angelis secundum differentiam gratiae et naturae, ut supra dictum est. Et utraque differentia semper in Angelis remanebit. Non enim posset naturarum differentia ab eis auferri, nisi eis corruptis, differentia etiam gloriae erit in eis semper, secundum differentiam meriti praecedentis. Executio autem officiorum angelicorum aliquo modo remanebit post diem iudicii, et aliquo modo cessabit. Cessabit quidem, secundum quod eorum officia ordinantur ad perducendum aliquos ad finem, remanebit autem, secundum quod convenit in ultima finis consecutione. Sicut etiam alia sunt officia militarium ordinum in pugna, et in triumpho.

[32981] Iª q. 108 a. 7 ad 1
Ad primum ergo dicendum quod principatus et potestates evacuabuntur in illa finali consummatione quantum ad hoc, quod alios ad finem perducant, quia consecuto iam fine, non est necessarium tendere in finem. Et haec ratio intelligitur ex verbis apostoli, dicentis, cum tradiderit regnum Deo et patri, idest, cum perduxerit fideles ad fruendum ipso Deo.

[32982] Iª q. 108 a. 7 ad 2
Ad secundum dicendum quod actiones Angelorum super alios Angelos considerandae sunt secundum similitudinem actionum intelligibilium quae sunt in nobis. Inveniuntur autem in nobis multae intelligibiles actiones quae sunt ordinatae secundum ordinem causae et causati; sicut cum per multa media gradatim in unam conclusionem devenimus. Manifestum est autem quod cognitio conclusionis dependet ex omnibus mediis praecedentibus, non solum quantum ad novam acquisitionem scientiae, sed etiam quantum ad scientiae conservationem. Cuius signum est quod, si quis oblivisceretur aliquod praecedentium mediorum, opinionem quidem vel fidem de conclusione posset habere, sed non scientiam, ordine causarum ignorato. Sic igitur, cum inferiores Angeli rationes divinorum operum cognoscant per lumen superiorum Angelorum, dependet eorum cognitio ex lumine superiorum, non solum quantum ad novam acquisitionem scientiae, sed etiam quantum ad cognitionis conservationem. Licet ergo post iudicium non proficiant inferiores Angeli in cognitione aliquarum rerum, non tamen propter hoc excluditur quin a superioribus illuminentur.

[32983] Iª q. 108 a. 7 ad 3
Ad tertium dicendum quod, etsi post diem iudicii homines non sint ulterius ad salutem adducendi per ministerium Angelorum; tamen illi qui iam salutem erunt consecuti, aliquam illustrationem habebunt per Angelorum officia.


%\rhead{a.~}
%\begin{center}
% {\Large {\bf }}\\
% {\large }\\
% {\footnotesize }\\
% {\Large \\}
%\end{center}
%
%\begin{longtable}{p{21em}p{21em}}
%
%&
%
%
%\\
%\end{longtable}
%\newpage


ARTICULUS 8
[32984] Iª q. 108 a. 8 arg. 1
Ad octavum sic proceditur. Videtur quod homines non assumantur ad ordines Angelorum. Hierarchia enim humana continetur sub infima hierarchiarum caelestium, sicut infima sub media, et media sub prima. Sed Angeli infimae hierarchiae nunquam transferentur in mediam aut in primam. Ergo neque homines transferentur ad ordines Angelorum.

[32985] Iª q. 108 a. 8 arg. 2
Praeterea, ordinibus Angelorum aliqua officia competunt, utpote custodire, miracula facere, Daemones arcere, et huiusmodi, quae non videntur convenire animabus sanctorum. Ergo non transferentur ad ordines Angelorum.

[32986] Iª q. 108 a. 8 arg. 3
Praeterea, sicut boni Angeli inducunt ad bonum, ita Daemones inducunt ad malum. Sed erroneum est dicere quod animae hominum malorum convertantur in Daemones, hoc enim Chrysostomus reprobat, super Matth. Ergo non videtur quod animae sanctorum transferantur ad ordines Angelorum.

[32987] Iª q. 108 a. 8 s. c.
Sed contra est quod dominus dicit, Matth. XXII, de sanctis, quod erunt sicut Angeli Dei in caelo.

[32988] Iª q. 108 a. 8 co.
Respondeo dicendum quod, sicut supra dictum est, ordines Angelorum distinguuntur et secundum conditionem naturae, et secundum dona gratiae. Si ergo considerentur Angelorum ordines solum quantum ad gradum naturae, sic homines nullo modo assumi possunt ad ordines Angelorum, quia semper remanebit naturarum distinctio. Quam quidam considerantes, posuerunt quod nullo modo homines transferri possunt ad aequalitatem Angelorum. Quod est erroneum, repugnat enim promissioni Christi, dicentis, Lucae XX, quod filii resurrectionis erunt aequales Angelis in caelis. Illud enim quod est ex parte naturae, se habet ut materiale in ratione ordinis, completivum vero est quod est ex dono gratiae, quae dependet ex liberalitate Dei, non ex ordine naturae. Et ideo per donum gratiae homines mereri possunt tantam gloriam, ut Angelis aequentur secundum singulos Angelorum gradus. Quod est homines ad ordines Angelorum assumi. Quidam tamen dicunt quod ad ordines Angelorum non assumuntur omnes qui salvantur, sed soli virgines vel perfecti; alii vero suum ordinem constituent, quasi condivisum toti societati Angelorum. Sed hoc est contra Augustinum, qui dicit XII de Civ. Dei, quod non erunt duae societates hominum et Angelorum, sed una, quia omnium beatitudo est adhaerere uni Deo.

[32989] Iª q. 108 a. 8 ad 1
Ad primum ergo dicendum quod gratia Angelis datur secundum proportionem naturalium; non autem sic est de hominibus, ut supra dictum est. Et ideo sicut inferiores Angeli non possunt transferri ad naturalem gradum superiorum, ita nec ad gratuitum. Homines vero possunt ad gratuitum conscendere, sed non ad naturalem.

[32990] Iª q. 108 a. 8 ad 2
Ad secundum dicendum quod Angeli, secundum naturae ordinem, medii sunt inter nos et Deum. Et ideo, secundum legem communem, per eos administrantur non solum res humanae, sed etiam omnia corporalia. Homines autem sancti, etiam post hanc vitam, sunt eiusdem naturae nobiscum. Unde secundum legem communem, non administrant humana, nec rebus vivorum intersunt, ut Augustinus dicit in libro de cura pro mortuis agenda. Ex quadam tamen speciali dispensatione interdum aliquibus sanctis conceditur, vel vivis vel mortuis, huiusmodi officia exercere, vel miracula faciendo, vel Daemones arcendo, vel aliquid huiusmodi; sicut Augustinus in eodem libro dicit.

[32991] Iª q. 108 a. 8 ad 3
Ad tertium dicendum quod homines ad poenam Daemonum transferri, non est erroneum, sed quidam erronee posuerunt Daemones nihil aliud esse quam animas defunctorum. Et hoc Chrysostomus reprobat.

