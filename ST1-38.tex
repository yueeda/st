\documentclass[10pt]{jsarticle} % use larger type; default would be 10pt
%\usepackage[utf8]{inputenc} % set input encoding (not needed with XeLaTeX)
%\usepackage[round,comma,authoryear]{natbib}
%\usepackage{nruby}
\usepackage{okumacro}
\usepackage{longtable}
%\usepqckage{tablefootnote}
\usepackage[polutonikogreek,english,japanese]{babel}
%\usepackage{amsmath}
\usepackage{latexsym}
\usepackage{color}
%\usepackage{tikz}

%----- header -------
\usepackage{fancyhdr}
\pagestyle{fancy}
\lhead{{\it Summa Theologiae} I, q.~38}
%--------------------

\title{{\bf PRIMA PARS}\\{\Huge Summa Theologiae}\\Sancti Thomae
    Aquinatis\\{\sffamily QUEAESTIO TRIGESTIMAOCTAVA}\\DE NOMINE SPIRITUS
SANCTI QUOD EST DONUM}
\author{Japanese translation\\by Yoshinori {\scshape Ueeda}}
\date{Last modified \today}

%%%% コピペ用
%\rhead{a.~}
%\begin{center}
% {\Large {\bf }}\\
% {\large }\\
% {\footnotesize }\\
% {\Large \\}
%\end{center}
%
%\begin{longtable}{p{21em}p{21em}}
%
%&
%
%\\
%\end{longtable}
%\newpage
\begin{document}
\maketitle
\begin{center}
    {\Large 聖トマス・アクィナスの神学大全の第一部\\第三十八問\\贈物という聖
    霊の名について}
\end{center}

\thispagestyle{empty}
\begin{longtable}{p{21em}p{21em}}

    Consequenter quaeritur de dono. Et circa hoc quaeruntur duo. 

    \begin{enumerate}
        \item utrum donum possit esse nomen personale.
        \item utrum sit proprium spiritus sancti.
    \end{enumerate}

&

続いて贈物について問われる。これについては二つのことが問われる。
\begin{enumerate}
    \item 贈物はペルソナ的名称でありうるか。
    \item それは聖霊に固有か。
\end{enumerate}
\end{longtable}

\newpage

\rhead{a.~1}
\begin{center}
    {\Large {\bf ARTICULUS PRIMUS}}\\
    {\large UTRUM DONUM SIT NOMEN PERSONALE}\\
    {\footnotesize I \textit{Sent.}, d.18, a.1.}\\
    {\Large 第一項\\贈物はペルソナ的名称か}
\end{center}

\begin{longtable}{p{21em}p{21em}}

\textsc{Ad primum sic proceditur}. Videtur quod donum non sit nomen
    personale. Omne enim nomen personale importat aliquam
    distinctionem in divinis. Sed nomen doni non importat aliquam
    distinctionem in divinis, dicit enim Augustinus, XV \textit{de
    Trin.} : \textit{quod spiritus sanctus ita datur sicut Dei donum,
    ut etiam seipsum det sicut Deus}. Ergo donum non est nomen
    personale.


&

第一異論に対してはそれゆえ以下のように言われるべきである。
贈物はペルソナ的名称でないと思われる。理由は以下の通り。
すべてペルソナ的名称は神における何らかの区別を含意する。
しかるに贈物という名称は神におけるどんな区別も含意しない。
なぜならアウグスティヌスは『三位一体論』第15巻で「聖霊は神の贈物とし
て、ちょうど自分自身を神として与えるように、与えられる」と言っているからである。
ゆえに贈物はペルソナ的名称ではない。

\\


2. \textsc{Praeterea}, nullum nomen personale convenit essentiae divinae. Sed
essentia divina est donum quod pater dat filio, ut patet per Hilarium, IX
\textit{de Trin.} Ergo donum non est nomen personale.


&

さらに、どんなペルソナ的名称も神の本質に適合しない。しかるに神の本質は、『三位一体論』第9巻のでヒラリウスによって明らかなとおり、父から息子への贈物である。ゆえに贈物はペルソナの名称ではない。

\\

3. \textsc{Praeterea}, secundum Damascenum, nihil est subiectum aut serviens in
divinis personis. Sed donum importat quandam subiectionem et ad eum cui datur,
et ad eum a quo datur. Ergo donum non est nomen personale.


&

さらにダマスケヌスによれば、何ものも神の中で従属したり仕える者ではない。しかるに贈物は送り先や送り主へのある種の従属を含意する。ゆえに贈物はペルソナ的名称ではない。

\\

4. \textsc{Praeterea}, donum importat respectum ad creaturam, et ita videtur de Deo dici ex tempore. Sed nomina personalia dicuntur de Deo ab aeterno, ut pater et filius. Ergo donum non est nomen personale.


&

さらに贈物は被造物への関係を含意し、したがって神について時間的に語られると思われる。しかるにペルソナ的名称は、父や息子のように、神について永遠から語られる。ゆえに贈物はペルソナ的名称ではない。

\\

\textsc{Sed contra est} quod Augustinus dicit, XV \textit{de Trin}., \textit{sicut corpus carnis nihil aliud est quam caro, sic donum spiritus sancti nihil aliud est quam spiritus sanctus.} Sed spiritus sanctus est nomen personale. Ergo et donum.


&

しかし反対にアウグスティヌスは『三位一体論』第15巻でこう述べている。「肉の体が肉以外のものでないように聖霊の贈物は聖霊以外のものではない」。しかるに聖霊はペルソナ的名称である。ゆに贈物もまたペルソナ的名称である。

\\

\textsc{Respondeo dicendum} quod in nomine doni importatur aptitudo ad hoc quod donetur.
Quod autem donatur, habet habitudinem et ad id a quo datur, et ad id cui datur,
non enim daretur ab aliquo nisi esset eius; et ad hoc alicui datur, ut eius
sit. Persona autem divina dicitur esse alicuius, vel secundum originem, sicut
filius est patris; vel inquantum ab aliquo habetur. 


&

解答する。以下のように言われるべきである。
贈物という名称においては贈られるものに対する適合性が含意されている。
しかるに贈られるものは、贈り主と贈り相手への関係をもつ。
なぜなら贈り主のものであったのでなければ贈られなかっただろうし、
また贈り相手の所有となるように贈られるからである。
さて、神のペルソナが誰かのものであると言われるのは、息子が父のものであるように、期限に即してか、あるいは、何かによって所有される限りにおいてかのいずれかである。

\\

Habere autem dicimur id quo
libere possumus uti vel frui, ut volumus. Et per hunc modum divina persona non
potest haberi nisi a rationali creatura Deo coniuncta. Aliae autem creaturae
moveri quidem possunt a divina persona; non tamen sic quod in potestate earum
sit frui divina persona, et uti effectu eius. 


&

さらに、所有すると私たちに言われるのは、私たちが意志するとおりに用いたり享受したりできるものである。
この意味で神のペルソナは神に結び付けられた理性的な被造物によってしか持たれえない。
これに対して他の被造物は神のペルソナによって動かされうるが、それらの権能の中に神のペルソナを享受しその結果を用いることはない。

\\

Ad quod quandoque pertingit
rationalis creatura; ut puta cum sic fit particeps divini verbi et procedentis
amoris, ut possit libere Deum vere cognoscere et recte amare. Unde sola
creatura rationalis potest habere divinam personam. Sed ad hoc quod sic eam
habeat, non potest propria virtute pervenire, unde oportet quod hoc ei desuper
detur; hoc enim dari nobis dicitur, quod aliunde habemus. Et sic divinae
personae competit dari, et esse donum.


&

理性的被造物は、たとえば自由に神を真に認識し正しく愛することができるようなかたちで、神の言葉や発出する愛に参加するものであるときのような状態まで到達することがある。
このことから、理性的被造物だけが神のペルソナを所有することができる。
しかし、このようなかたちで神のペルソナをもつことへ、固有の力によって到達することはできないので、このことは理性的被造物にさらに与えられなければならない。じっさい私たちに与えられると言われるのは、どこかから私たちが持つものである。
この意味で神のペルソナに、「与えられる」や「贈物である」ことが適合する。

\\

\textsc{Ad primum ergo dicendum} quod nomen doni importat distinctionem personalem,
secundum quod donum dicitur esse alicuius per originem. Et tamen spiritus
sanctus dat seipsum, inquantum est sui ipsius, ut potens se uti, vel potius
frui; sicut et homo liber dicitur esse sui ipsius. 



&

第一異論に対してはそれゆえ以下のように言われるべきである。
贈物という名称がペルソナ的区別を含意するのは、贈物が起源によって何かに属する限りにおいてである。
しかし聖霊は、ちょうど人間が自分自身のものであると言われるのと同じように、自分自身を使う、あるいはむしろ自分自身を享受する者として、自分自身のものであるか切りにおいて自ら自身を与える。

\\



Et hoc est quod Augustinus
dicit, \textit{super Ioan}., \textit{quid tam tuum est quam tu?} Vel dicendum, et melius, quod
donum oportet esse aliquo modo dantis. Sed hoc esse huius dicitur
multipliciter. 



&

そしてこのことが、アウグスティヌスが『ヨハネ伝注解』で「何があなたよりもあなたのものであるでしょうか」と述べていることである。あるいは次に用に言われるべきであり、こちらの方がよい。すなわち、贈物はある意味で贈り主のものでなければならない。しかし誰かのものであるということは多くの意味で語られる。

\\


Uno modo, per modum identitatis, sicut dicit Augustinus super
Ioan., et sic donum non distinguitur a dante, sed ab eo cui datur. Et sic
dicitur quod spiritus sanctus dat se. Alio modo dicitur aliquid esse alicuius
ut possessio vel servus, et sic oportet quod donum essentialiter distinguatur a
dante. Et sic donum Dei est aliquid creatum. 



&

一つには、同一性のありかたによってであり、この意味でアウグスティヌスは『ヨハネ伝注解』で述べている。またこの意味で贈物は贈り主と区別されず、贈り相手から区別される。
そしてこの意味で聖霊は自らを与えると言われる。
別の意味では、或るものが所有物や下僕としてある人のものである。そしてこの場合には贈物は本質的に贈り主から区別される。
そしてこの意味で、ある被造物が神の贈り物である。

\\


Tertio modo dicitur hoc esse huius
per originem tantum, et sic filius est patris, et spiritus sanctus utriusque.
Inquantum ergo donum hoc modo dicitur esse dantis, sic distinguitur a dante
personaliter, et est nomen personale.


&

第三の意味でこれがそれのものであるのは起源によってのみであり、この意味で息子は父のものであり聖霊は両者のものである。
ゆえにこの意味で贈り主のものである贈物はペルソナ的に贈り主から区別され、そしてペルソナ的名称である。

\\

\textsc{Ad secundum dicendum} quod essentia dicitur esse donum patris primo modo, quia essentia est patris per modum identitatis.


&

第二異論に対しては以下のように言われるべきである。
本質が父の贈物であるということは第一の意味で言われている。なぜなら、本質は同一性というしかたによって父のものだからである。

\\

\textsc{Ad tertium dicendum} quod donum, secundum quod est nomen personale in divinis, non importat subiectionem, sed originem tantum, in comparatione ad dantem. In comparatione vero ad eum cui datur, importat liberum usum vel fruitionem, ut dictum est.


&

第三異論に対しては以下のように言われるべきである。
ペルソナ的名称が神においてあるという点で従属を含意することはなく、むしろ贈り主との比較における起源だけを含意する。
また、すでに述べられたとおり、贈り相手との関係においては、自由な使用や享受を含意する。

\\

\textsc{Ad quartum dicendum} quod donum non dicitur ex eo quod actu datur, sed inquantum habet aptitudinem ut possit dari. Unde ab aeterno divina persona dicitur donum, licet ex tempore detur. Nec tamen per hoc quod importatur respectus ad creaturam, oportet quod sit essentiale, sed quod aliquid essentiale in suo intellectu includatur, sicut essentia includitur in intellectu personae, ut supra dictum est.



&

第四異論に対しては以下のように言われるべきである。
贈物は現実に与えられることから言われるのではなく、それが与えられうる適合性をもつかぎりで言われる。
それゆえ、神のペルソナは時間的に与えられるにもかかわらず永遠から贈物と言われる。
また被造物への関係が含意されるということによって、それが本質的でなければならないというわけではない。
むしろ先述のごとく、本質がペルソナの理解の中に含まれるように、何か本質的なものがその理解に含まれるということである。

\\


\end{longtable}

\newpage
\rhead{a.~2}
\begin{center}
 {\Large {\bf ARTICULUS SECUNDUS}}\\
 {\large UTRUM DONUM SIT PROPRIUM NOMEN SPIRITU SANCTI}\\
 {\footnotesize I \textit{Sent.}, d.18, a.2.}\\
 {\Large 第二項\\贈物は聖霊に固有の名称か}
\end{center}

\begin{longtable}{p{21em}p{21em}}

\textsc{Ad secundum sic proceditur.} Videtur quod donum non sit proprium nomen
spiritus sancti. Donum enim dicitur ex eo quod datur. Sed, sicut dicitur
Isa.~\textsc{ix}, \textit{filius datus est nobis}. Ergo esse donum convenit
filio, sicut spiritui sancto.


&

第二項の問題へ議論は以下のように進められる。
贈物は聖霊に固有の名称でないと思われる。理由は以下の通り。
贈物は与えられることから言われる。
しかるに『イザヤ書』第9章で言われるように「息子は私たちに与えられた」\footnote{一人のみどりごが私たちのために生まれた。/一人の男の子が私たちに与えられた。/主権がその肩にあり、その名は/「驚くべき指導者、力ある神/永遠の父、平和の君」と呼ばれる(9:5)}。
ゆえに贈物であることは聖霊と同じように息子にも適合する。

\\

2. \textsc{Praeterea}, omne nomen proprium alicuius personae significat aliquam
eius proprietatem. Sed hoc nomen donum non significat proprietatem aliquam
spiritus sancti. Ergo donum non est proprium nomen spiritus sancti.


&

さらに、あるペルソナに固有な名称はすべて彼の何らかの固有性を表示する。
しかるに「贈物」というこの名称は聖霊の何らかの固有性を表示していない。
ゆえに贈物は聖霊の固有の名称でない。

\\

3. \textsc{Praeterea}, spiritus sanctus potest dici spiritus alicuius hominis, ut supra dictum est. Sed non potest dici donum alicuius hominis, sed solum donum Dei. Ergo donum non est proprium nomen spiritus sancti.

&

さらに、前述のように\footnote{STI, q.36, a.1, arg.3}、聖霊はある人間の霊とも言われうる。しかしある人間の贈物とは言われえず、ただ神の贈り物とだけ言われる。ゆえに贈物は聖霊に固有の名称でない。
\\

\textsc{Sed contra est} quod Augustinus dicit, in IV \textit{de Trin}., \textit{sicut natum esse est filio a patre esse, ita spiritui sancto donum Dei esse est a patre et filio procedere}. Sed spiritus sanctus sortitur proprium nomen inquantum procedit a patre et filio. Ergo et donum est proprium nomen spiritus sancti.


&

しかし反対にアウグスティヌスは『三位一体論』で次のように述べている。「息子にとって生み出されたものであることが父からであることであるように、聖霊にとって神の贈物であるということが父と息子から発出するということである」。
しかるに聖霊は父と息子から発出する限りにおいて固有の名称をあてがわれる。
ゆえに贈物もまた聖霊に固有の名称である。

\\

\textsc{Respondeo dicendum} quod donum, secundum quod personaliter sumitur in
divinis, est proprium nomen spiritus sancti. Ad cuius evidentiam, sciendum est
quod donum proprie est datio irreddibilis, secundum philosophum, idest quod non
datur intentione retributionis, et sic importat gratuitam donationem. Ratio
autem gratuitae donationis est amor, ideo enim damus gratis alicui aliquid,
quia volumus ei bonum. 

&

解答する。以下のように言われるべきである。
贈物は神においてペルソナ的に解されるかぎりでは聖霊に固有の名称である。
これを明らかにするためには以下のことが知られるべきである。
哲学者によれば、贈物は厳密にはお返しができない贈与すなわちお返しを意図して与えられるものではなく、その意味で、無償の贈与を含意する。
しかるに無償の贈与は愛であるから、私たちはその人にとっての善を欲するので、ある人に無償で何かを与える。
\\

Primum ergo quod damus ei, est amor quo volumus ei
bonum. Unde manifestum est quod amor habet rationem primi doni, per quod omnia
dona gratuita donantur. Unde, cum spiritus sanctus procedat ut amor, sicut iam
dictum est, procedit in ratione doni primi. Unde dicit Augustinus, XV \textit{de Trin}.,
quod \textit{per donum quod est spiritus sanctus, multa propria dona dividuntur membris
Christi}.


&

それゆえ、私たちがその人に第一に与えるのはそれによって私たちが彼にとっての善を欲するところの愛である。
したがって、愛がそれによって無償の贈物が与えられるところの第一善の性格を持つことは明らかである。
したがって、すでに述べられたとおり聖霊は愛として発出するので、第一の贈物という性格において発出する。
このことからアウグスティヌスは『三位一体論』第15巻で「聖霊である贈物を通して、キリストの四肢に固有の贈物が分け与えられる」と述べる。

\\

\textsc{Ad primum ergo dicendum} quod, sicut filius, quia procedit per modum verbi, quod
de ratione sua habet quod sit similitudo sui principii dicitur proprie imago,
licet etiam spiritus sanctus sit similis patri; ita etiam spiritus sanctus,
quia a patre procedit ut amor, dicitur proprie donum, licet etiam filius detur.
Hoc enim ipsum quod filius datur, est ex patris amore, secundum illud Ioan.~
\textsc{ii}, \textit{sic Deus dilexit mundum, ut filium suum unigenitum daret}.


&

第一異論に対してはそれゆえ以下のように言われるべきである。
ちょうど息子が、言葉のしかたで発出し、言葉はその性格の中に自らの根源の類似であることが入っているので、聖霊も父に似ているにもかかわらず、固有に似像と言われるように、聖霊もまた、息子も与えられるにもかかわらず、父から愛として発出するために、固有に贈物と言われる。
なぜなら、『ヨハネによる福音書』第3章「神は自分の一人の息子を与えるほどにこの世を愛した」\footnote{神は、その独り子をお与えになったほどに、世を愛された。御子を信じる者が一人も滅びないで、永遠の命を得るためである。(3:16)}によれば、息子が与えられること自体が、父の愛に基づくからである。

\\

\textsc{Ad secundum dicendum} quod in nomine doni importatur quod sit dantis per
originem. Et sic importatur proprietas originis spiritus sancti, quae est
processio.


&

第二異論に対しては以下のように言われるべきである。贈物という名称の中に起源によって贈り主に属するということが含まれる。
この意味で聖霊の起源の固有性が含意されていて、それは「発出」である。
\\

\textsc{Ad tertium dicendum} quod donum, antequam detur, est tantum dantis, sed postquam
datur, est eius cui datur. Quia igitur donum non importat dationem in actu, non
potest dici quod sit donum hominis; sed donum Dei dantis. Cum autem iam datum
est, tunc hominis est vel spiritus vel datum.


&

第三異論に対しては以下のように言われるべきである。
贈物は与えられる前にはたんに贈り主のものだが、与えられた後は贈り相手のものである。
ゆえに、贈物は現実に与えることを含意しないので、人間の贈物とは言われえず、与える神のものと言われうる。
しかし、すでに与えられたとき、そのときには霊または贈物は人間のものである。
\\

\end{longtable}

\end{document}
