\documentclass[10pt]{jsarticle} % use larger type; default would be 10pt
%\usepackage[utf8]{inputenc} % set input encoding (not needed with XeLaTeX)
%\usepackage[round,comma,authoryear]{natbib}
%\usepackage{nruby}
\usepackage{okumacro}
\usepackage{longtable}
%\usepqckage{tablefootnote}
\usepackage[polutonikogreek,english,japanese]{babel}
%\usepackage{amsmath}
\usepackage{latexsym}
\usepackage{color}

%----- header -------
\usepackage{fancyhdr}
\lhead{{\it Summa Theologiae} I, q.~24}
%--------------------

\bibliographystyle{jplain}

\title{{\bf PRIMA PARS}\\{\HUGE Summae Theologiae}\\Sancti Thomae
Aquinatis\\{\sffamily QUEAESTIO VIGESIMAQUARTA}\\DE LIBRO VITAE}
\author{Japanese translation\\by Yoshinori {\sc Ueeda}}
\date{Last modified \today}


%%%% コピペ用
%\rhead{a.~}
%\begin{center}
% {\Large {\bf }}\\
% {\large }\\
% {\footnotesize }\\
% {\Large \\}
%\end{center}
%
%\begin{longtable}{p{21em}p{21em}}
%
%&
%
%
%\\
%\end{longtable}
%\newpage



\begin{document}
\maketitle
\pagestyle{fancy}


\begin{center}
{\Large 第二十四問\\生命の書について}
\end{center}


\begin{longtable}{p{21em}p{21em}}

Deinde considerandum est de libro vitae. Et circa hoc quaeruntur
 tria. 

\begin{enumerate}
 \item quid sit liber vitae.
 \item cuius vitae sit liber.
 \item utrum aliquis possit deleri de libro vitae.
\end{enumerate}

&

次に、生命の書について考察されるべきである。これについては三つのことが問
 われる。

\begin{enumerate}
 \item 生命の書とは何か。
 \item その書には、だれの生命のことが書かれているか。
 \item だれかが、生命の書から削除されうるか。
\end{enumerate}

\end{longtable}


\newpage


\rhead{a.~1}
\begin{center}
 {\Large {\bf ARTICULUS PRIMUS}}\\
 {\large UTRUM LIBER VITAE SIT IDEM QUOD PRAEDESTINATIO}\\
 {\footnotesize I {\itshape Sent.}, d.11, q.1, a.2, ad 5; III, d.31,
 q.1, a.2, qu$^a$ 2;  {\itshape De Verit.}, q.7, a.1, 4; {\itshape ad
 Philipp.}, cap.4, lect.1; {\itshape ad Hebr.}, cap.12, lect.4.}\\
 {\Large 第一項\\生命の書は予定と同じものか}
\end{center}

\begin{longtable}{p{21em}p{21em}}

{\Huge A}{\scshape d primum sic proceditur}. Videtur quod liber vitae
 non sit idem quod praedestinatio. Dicitur enim {\itshape Eccli}.~{\scshape xxiv}, {\itshape haec omnia
 liber vitae}; Glossa, {\itshape idest novum et vetus testamentum}. Hoc autem non
 est praedestinatio. Ergo liber vitae non est idem quod praedestinatio.

&

第一の問題へ、議論は以下のように進められる。
生命の書は予定と同じでないと思われる。理由は以下の通り。
『集会の書』24に「これらすべてが生命の書である」と言われていて、『注解』
 には、「すなわち新約と旧約」とある。これは予定ではない。ゆえに、生命の
 書は、予定と同一ではない。

\\

{\scshape 2 Praeterea}, Augustinus, in libro XX {\itshape de
 Civ.~Dei}, ait quod liber vitae est {\itshape quaedam vis divina, qua fiet ut
 cuique opera sua bona vel mala in memoriam reducantur}. Sed vis divina
 non videtur pertinere ad praedestinationem, sed magis ad attributum
 potentiae. Ergo liber vitae non est idem quod praedestinatio.


&

さらに、アウグスティヌスは『神の国』第20巻で、生命の書とは「ある種の神の
 力であり、それによって、自らの善い行いや悪い行いが、各々の人に記憶され
 る」と述べている。ところが、神の力は、予定ではなく、むしろ能力の特質に
 属していると思われる。ゆえに、生命の書は、予定と同一ではない。

\\



{\scshape 3 Praeterea}, praedestinationi opponitur
 reprobatio. Si igitur liber vitae esset praedestinatio, inveniretur
 liber mortis, sicut liber vitae.


&

さらに、予定には、見捨てが対立する。ゆえに、もし生命の書が予定であったなら
 ば、生命の書と同様に、死の書も見出されたであろう。

\\



{\scshape Sed contra est} quod dicitur in Glossa, super
 illud Psalmi {\scshape lxviii}, {\itshape deleantur de libro viventium, liber iste est
 notitia Dei, qua praedestinavit ad vitam, quos praescivit}.


&

しかし反対に、かの『詩編』68の『注解』では、「生きる者たちの書から抹消さ
 れよ。この書は神の知であり、それによって、予知した者たちを生命へと予定
 した」と言われている。


\\



{\scshape Respondeo dicendum} quod {\itshape liber vitae} in Deo
 dicitur metaphorice, secundum similitudinem a rebus humanis
 acceptam. Est enim consuetum apud homines, quod illi qui ad aliquid
 eliguntur, conscribuntur in libro; utpote milites vel consiliarii, qui
 olim dicebantur {\itshape patres conscripti}. Patet autem ex praemissis quod omnes
 praedestinati eliguntur a Deo ad habendum vitam aeternam. Ipsa ergo
 praedestinatorum conscriptio dicitur liber vitae. 



&

解答する。以下のように言われるべきである
「生命の書」は、神において、人間的事象から受け取られた類似に基づいて、比
 喩的に語られる。理由は以下の通り。人間たちのもとでは、何かのために選ば
 れる人々が、本に記されるのが習いである。たとえば、兵士や行政相談員などがそ
 うであるが、後者の人々は、かつて、「記された父たち」と呼ばれていた
 \footnote{ローマの元老院議員の通称。}。ところで、すでに述べられたことか
 ら、予定された人々はすべて、神によって、永遠の生命を持つように選ばれて
 いる。ゆえに、この予定された人々の記入それ自体が、生命の書と言
 われる。


\\

Dicitur autem
 metaphorice aliquid conscriptum in intellectu alicuius, quod firmiter
 in memoria tenet, secundum illud {\itshape Prov}.~{\scshape iii}, {\itshape ne obliviscaris legis meae,
 et praecepta mea cor tuum custodiat}; et post pauca sequitur, {\itshape describe
 illa in tabulis cordis tui}. Nam et in libris materialibus aliquid
 conscribitur ad succurrendum memoriae. Unde ipsa Dei notitia, qua
 firmiter retinet se aliquos praedestinasse ad vitam aeternam, dicitur
 liber vitae. Nam sicut scriptura libri est signum eorum quae fienda
 sunt ita Dei notitia est quoddam signum apud ipsum, eorum qui sunt
 perducendi ad vitam aeternam; secundum illud II {\itshape Tim}.~{\scshape ii}, {\itshape firmum
 fundamentum Dei stat, habens signaculum hoc, novit dominus qui sunt
 eius}.


&

さて、『箴言』3「私の法を忘れるな。あなたの心は私の命令をしっかりと保て」、
 そして少し後の「それらを、汝の心の板に記せ」
 によれば、記憶の中にしっかりと保持するものは、或る人の知性の中に記されると比
 喩的に言われる。じっさい、物質的な本においても、あるものが、記憶を助け
 るために記入される。したがって、神は知によって、
 ある人々を永遠の生命へ予定したことをかたく保つのだが、その神の知自体が、
 生命の書と言われる。じっさい、『テモテへの手紙』2「神の固い礎は、このしるし
 を持ちながら、立っている:主は、彼のもとにある人々を知っている。」によれ
 ば、本に書かれたものは、なされるべきことのし
 るしであるように、神の知は、永遠の生命へ至るべき人々についての、神のも
 とでの一種のしるしである。


\\



{\scshape Ad primum ergo dicendum} quod liber vitae
 potest dici dupliciter. Uno modo, conscriptio eorum qui sunt electi ad
 vitam, et sic loquimur nunc de libro vitae. Alio modo potest dici liber
 vitae, conscriptio eorum quae ducunt in vitam. Et hoc dupliciter. Vel
 sicut agendorum, et sic novum et vetus testamentum dicitur liber
 vitae. Vel sicut iam factorum, et sic illa vis divina, qua fiet ut
 cuilibet in memoriam reducantur facta sua, dicitur liber vitae. Sicut
 etiam liber militiae potest dici, vel in quo scribuntur electi ad
 militiam, vel in quo traditur ars militaris, vel in quo recitantur
 facta militum.


&

第一異論に対しては、それゆえ、以下のように言われるべきである。
「生命の書」は、二つの意味で言われる。一つは、生命へと選ばれた人々の記入であ
 り、今は、この意味で、生命の書について語られている。もう一つの意味で、
 生命へ導く事物の記入が「生命の書」と言われうる。そして、これは二つの意
 味で言われる。一つは、なされるべき事柄の記入であり、この意味で、新約と
 旧約が生命の書と言われる。あるいは、すでになされた事柄の記入であり、こ
 の意味では、神の力それ自体が生命の書と言われ。だれにとっても、自分のし
 たことが、神の力によって、記憶に留められるからである。これはたとえば、
「戦いの書」と言われうるものが、戦いへ選ばれた人々が書かれているもの、戦
 争の技術が伝えられているもの、兵士たちの手柄が書かれているものでありう
 るのと同様である。

\\



Unde patet solutio ad secundum.


&

第二異論に対する解答は、以上のことから明らかである。

\\



{\scshape Ad tertium dicendum} quod non est consuetum
 conscribi eos qui repudiantur, sed eos qui eliguntur. Unde reprobationi
 non respondet liber mortis, sicut praedestinationi liber vitae.


&

第三異論に対しては、以下のように言われるべきである。
非難される人々が記入されるのは習いではなく、選ばれる人々が記入される。し
 たがって、予定に「生命の書」が対応するからといって、見捨てに「死の書」が対応することはない。

\\



{\scshape Ad quartum dicendum} quod secundum rationem
 differt liber vitae a praedestinatione. Importat enim notitiam
 praedestinationis, sicut etiam ex Glossa inducta apparet.


&

第四の論\footnote{反対異論のこと。}に対しては、以下のように言われるべきである。
生命の書は、概念において、予定と異なる。引用された『注解』から明らかなと
 おり、それは予定についての知を意味する。


\end{longtable}
\newpage


\rhead{a.~2}
\begin{center}
 {\Large {\bf ARTICULUS SECUNDUS}}\\
 {\large UTRUM LIBER VITAE SIT SOLUM RESPECTU VITAE GLORIAE PRAEDESTINATORUM}\\
 {\footnotesize III {\itshape Sent.}, d.31, q.1, a.2, qu$^a$2; {\itshape
 De Verit.}, q.7, a.5, 6, 7.}\\
 {\Large 第二項\\生命の書は予定された人々の栄光の生だけにかかわるか}
\end{center}

\begin{longtable}{p{21em}p{21em}}


{\Huge A}{\scshape d secundum sic proceditur}. Videtur quod
 liber vitae non sit solum respectu vitae gloriae
 praedestinatorum. Liber enim vitae est notitia vitae. Sed Deus per
 vitam suam cognoscit omnem aliam vitam. Ergo liber vitae praecipue
 dicitur respectu vitae divinae; et non solum respectu vitae
 praedestinatorum.


&

第二項の問題へ、議論は以下のように進められる。
生命の書は、予定された人々の栄光の生だけにかかわるのではないと思われる。
 理由は以下の通り。
生命の書は、生命についての知である。
ところが、神は、自らの生命によって、他のすべての生命を知る。
ゆえに、生命の書は、とくに、神の生命にかんして言われるのであり、予定され
 た人々の生についてだけ言われるのではない。


\\


{\scshape 2 Praeterea}, sicut vita gloriae est a Deo,
 ita vita naturae. Si igitur notitia vitae gloriae dicitur liber vitae,
 etiam notitia vitae naturae dicetur liber vitae.


&

さらに、栄光の生命が神からあるように、自然の生命も神からある。
ゆえに、もし栄光の生命についての知が生命の書と言われるならば、自然の生命
 についての知もまた生命の書と言われるであろう。


\\


{\scshape 3 Praeterea}, aliqui eliguntur ad gratiam, qui
 non eliguntur ad vitam gloriae; ut patet per id quod dicitur Ioan.~{\scshape vi},
 {\itshape nonne duodecim vos elegi, et unus ex vobis Diabolus est?} Sed liber
 vitae est conscriptio electionis divinae, ut dictum est. Ergo etiam est
 respectu vitae gratiae.


&

さらに、「ヨハネによる福音書」6「私はあなたたち12人を選んだが、あなたた
 ちのうちの一人は悪魔ではないか」によれば、恩恵へと選ばれても、栄光
 の生へ選ばれない人がいる。ところが、生命の書は、すでに述べられたとおり、
 神の選択の記入である。ゆえに、それは恩恵の生にかんしてもある。


\\


{\scshape Sed contra est} quod liber vitae est notitia
 praedestinationis, ut dictum est. Sed praedestinatio non respicit vitam
 gratiae, nisi secundum quod ordinatur ad gloriam, non enim sunt
 praedestinati, qui habent gratiam et deficiunt a gloria. Liber igitur
 vitae non dicitur nisi respectu gloriae.


&

しかし反対に、
すでに述べられたとおり、生命の書は予定についての知である。
ところで、予定は、栄光へ秩序付けられている限りにおいてでない限り、恩恵の
 生に関してではない。なぜなら、予定された者で、恩恵を持ちかつ栄光を欠く
 ような者はいないからである。ゆえに、生命の書は栄光に関してのみ語られる。

\\


{\scshape Respondeo dicendum} quod liber vitae, ut dictum
 est, importat conscriptionem quandam sive notitiam electorum ad
 vitam. Eligitur autem aliquis ad id quod non competit sibi secundum
 suam naturam. Et iterum, id ad quod eligitur aliquis, habet rationem
 finis, non enim miles eligitur aut conscribitur ad hoc quod armetur,
 sed ad hoc quod pugnet; hoc enim est proprium officium ad quod militia
 ordinatur. Finis autem supra naturam existens, est vita gloriae, ut
 supra dictum est. Unde proprie liber vitae respicit vitam gloriae.

&

解答する。以下のように言われるべきである。
生命の書は、すでに述べられたとおり、生命へ選ばれた人々のある種の記入ない
 し知を意味する。
ところで、ある人が、自分の本性においては適合しないことへと選ばれることが
 ある。さらに、人がそれへと選ばれたところのものは、目的の性格を持つ。た
 とえば、兵士は武装するためにではなく戦うために選ばれ記入されるように。
 戦うことが、軍隊の固有の任務であり、それへと秩序付けられている。
ところで、自然本性を越えている目的は、前に述べられたことから、栄光の生で
 ある。したがって、生命の書は、厳密には、栄光の生に関係する。


\\


{\scshape Ad primum ergo dicendum} quod vita divina,
 etiam prout est vita gloriosa, est Deo naturalis. Unde respectu eius
 non est electio, et per consequens neque liber vitae. Non enim dicimus
 quod aliquis homo eligatur ad habendum sensum, vel aliquid eorum quae
 consequuntur naturam.


&

第一異論に対しては、それゆえ、以下のように言われるべきである。
神の生命は、栄光に満ちた生命としてさえも、神にとっては本性的である。
したがって、それについて選択はなく、その帰結として、生命の書もない。
じっさい、私たちは、ある人が、感覚その他自然本性に伴うものを持つように選
 ばれるとは言わない。



\\


Unde per hoc etiam patet solutio ad
 secundum. Respectu enim vitae naturalis non est electio, neque liber
 vitae.


&

したがって、このことによって、第二異論への解答も明らかである。
すなわち、自然的な生について選択はなく、生命の書もない。


\\


{\scshape Ad tertium dicendum} quod vita gratiae non
 habet rationem finis, sed rationem eius quod est ad finem. Unde ad
 vitam gratiae non dicitur aliquis eligi, nisi inquantum vita gratiae
 ordinatur ad gloriam. Et propter hoc, illi qui habent gratiam et
 excidunt a gloria, non dicuntur esse electi simpliciter, sed secundum
 quid. Et similiter non dicuntur esse scripti simpliciter in libro vitae
 sed secundum quid; prout scilicet de eis in ordinatione et notitia
 divina existit, quod sint habituri aliquem ordinem ad vitam aeternam,
 secundum participationem gratiae.


&

第三異論に対しては、以下のように言われるべきである。
恩恵の生命は目的の性格を持たず、目的のためにあるものの性格を持つ。したがっ
 て、恩恵の生命が栄光へ秩序付けられている限りにおいてでなければ、ある人
 が、恩恵の生命へ選ばれるとは言われない。このため、恩恵を持ちながら栄光
 から転落する者は、端的には選ばれていると言われず、ある意味においてのみ
 である。同様に、端的には、生命の書に書かれていると言われず、ある意味に
 おいて、すなわち、その人たちについて、神の秩序付けあるいは知の中に、
 彼らが、恩恵を分有して、永遠の生命へのある種の秩序を持つであろうという
 ことが、存在している限りにおいてである。

%6/1/2018





\end{longtable}
\newpage


\rhead{a.~3}
\begin{center}
 {\Large {\bf ARTICULUS TERTIUS}}\\
 {\large UTRUM ALIQUIS DELEATUR A LIBRO VITAE}\\
 {\footnotesize I {\itshape Sent.}, d.11, q.1, a.2, ad 5; q.3, ad 3;
 III, d.31, q.1, a.2, qu$^a$ 3; {\itshape ad Philipp.}, cap.4, lect.1.}\\
 {\Large 第三項\\だれかが生命の書から消されることがあるか}
\end{center}

\begin{longtable}{p{21em}p{21em}}


{\Huge A}{\scshape d tertium sic proceditur}. Videtur quod
nullus deleatur de libro vitae. Dicit enim Augustinus, in XX de
Civ. Dei, quod {\itshape praescientia Dei, quae non potest falli, liber vitae
est}. Sed a praescientia Dei non potest aliquid subtrahi, similiter neque
a praedestinatione. Ergo nec de libro vitae potest aliquis deleri.


&

第三の問題へ、議論は以下のように進められる。
だれかが生命の書から消されるということはない、と思われる。理由は以下の通
 り。
アウグスティヌスは、『神の国』第20巻で、神の予知、これは誤ることがありえ
 ないものだが、は、生命の書である」と言っている。ところで、神の予知から
 何かが取り去られることはありえず、同様に、予定からもそうである。ゆえに、
 生命の書からだれかが消されることはありえない。


\\


{\scshape 2 Praeterea}, quidquid est in aliquo, est in
eo per modum eius in quo est. Sed liber vitae est quid aeternum et
immutabile. Ergo quidquid est in eo, est ibi non temporaliter, sed
immobiliter et indelebiliter.


&

さらに、何であれ、何かの中にあるものは、その何かのあり方によって、その中
 にある。ところで、生命の書は、永遠で不変の何かである。ゆえに、その中に
 あるものは、そこに、時間的でなく、不変で取り消されえないものとしてある。

\\


{\scshape 3 Praeterea}, deletio Scripturae
opponitur. Sed aliquis non potest de novo scribi in libro vitae. Ergo
neque inde deleri potest.

&

さらに、抹消は記入に対置される。ところが、ある人が、新しく生命の書に記入さ
 れることはありえない。したがって、消去されることも不可能である。


\\


Sed contra est quod dicitur in Psalmo
\scshape{lxviii}, {\itshape deleantur de libro viventium}.


&

しかし反対に、『詩編』68に、「彼らは生きる人々の書から消されよ」とある。

\\


{\scshape Respondeo dicendum} quod quidam dicunt quod de
libro vitae nullus potest deleri secundum rei veritatem, potest tamen
aliquis deleri secundum opinionem hominum. Est enim consuetum in
Scripturis ut aliquid dicatur fieri, quando innotescit. Et secundum hoc,
aliqui dicuntur esse scripti in libro vitae, inquantum homines opinantur
eos ibi scriptos, propter praesentem iustitiam quam in eis vident. Sed
quando apparet, vel in hoc seculo vel in futuro, quod ab hac iustitia
exciderunt, dicuntur inde deleri. 
Et sic etiam exponitur in Glossa
deletio talis, super illud Psalmi {\scshape lxviii}, {\itshape deleantur de libro
viventium}.

&


解答する。以下のように言われるべきである。
ある人々は、生命の書から、事物の真理において、だれかが消されることはない
 が、人間の意見において、だれかが消されることはありうる、と述べている。
 つまり、聖書では、何かを知ったときに、それが生じると言われることがよく
 ある。この意味で、人々が、ある人たちの中に正義があることを見ることによっ
 て、彼らがそこに書かれていると思うかぎりで、ある人が生命の書に書かれた
 と言われる。逆に、現世であれ来世であれ、そのような正義から脱落するよう
 に見えるときには、そこから消されると言われる。
『注釈』においても、かの『詩編』68の「生きている者たちの書から消されよ」
 について、そのように説明されている。


\\


Sed quia non deleri de libro vitae ponitur inter praemia
iustorum, secundum illud {\itshape Apoc}.~{\scshape iii}, {\itshape qui vicerit, sic vestietur
vestimentis albis, et non delebo nomen eius de libro vitae}; quod autem
sanctis repromittitur, non est solum in hominum opinione; potest dici
quod deleri vel non deleri de libro vitae, non solum ad opinionem
hominum referendum est, sed etiam quantum ad rem. 


&

しかし、かの『ヨハネの黙示録』3「勝った者は白い服を着せられ、私は彼の名前を
 生命の書から消すことをしない」によれば、生命の書から消されないことは、
 義人の褒美の一つである。ところが、聖人たちに約束されるものは、人々の意見
 の中だけのものではないので、生命の書から消されるとか消されないとかは、
 人々の意見だけでなく、事実にも関係しているはずである。

\\


Est enim liber vitae
conscriptio ordinatorum in vitam aeternam. Ad quam ordinatur aliquis ex
duobus, scilicet ex praedestinatione divina, et haec ordinatio nunquam
deficit; et ex gratia. Quicumque enim gratiam habet, ex hoc ipso est
dignus vita aeterna. Et haec ordinatio deficit interdum, quia aliqui
ordinati sunt ex gratia habita ad habendum vitam aeternam, a qua tamen
deficiunt per peccatum mortale. 


&

じっさい、生命の書とは、永遠の生命へ秩序付けられた人々の記入である。永遠
 の生命へは、二通りのしかたで、人は秩序付けられる。すなわち、神の予定に
 よってと、恩恵によってである。恩恵をもつ者は、そのこと自体によって、永
 遠の生命にふさわしい者となるからである。前者から何かが欠けることはない
 が、後者の秩序づけは、完成しないことがある。なぜなら、ある人々は、恩恵
 を持つことによって永遠の生命を持つように秩序付けられていたのに、大罪を
 犯すことによって、そこから墜ちるからである。


\\


Illi igitur qui sunt ordinati ad
habendum vitam aeternam ex praedestinatione divina, sunt simpliciter
scripti in libro vitae, quia sunt ibi scripti ut habituri vitam aeternam
in seipsa. Et isti nunquam delentur de libro vitae. Sed illi qui sunt
ordinati ad habendum vitam aeternam, non ex praedestinatione divina, sed
solum ex gratia, dicuntur esse scripti in libro vitae, non simpliciter,
sed secundum quid, quia sunt ibi scripti ut habituri vitam aeternam, non
in seipsa, sed in sua causa. 


&

ゆえに、永遠の生命を持つことへ、神の予定に基づいて秩序付けられている人々
 は、端的に、生命の書に記されている。なぜなら、彼らは、永遠の生命を、そ
 れ自体において持つであろうことが、そこに書かれているからである。
これに対して、神の予定ではなく恩恵に基づいて、永遠の生命を持つことへ秩序
 付けられている人々は、端的にではなく、ある意味において、生命の書に書か
 れていると言われる。なぜなら、彼らはそこに、永遠の生命を、それ自体におい
 てではなく、原因において持つであろうと書かれているからである。


\\


Et tales possunt deleri de libro vitae, ut
deletio non referatur ad notitiam Dei, quasi Deus aliquid praesciat, et
postea nesciat; sed ad rem scitam, quia scilicet Deus scit aliquem prius
ordinari in vitam aeternam, et postea non ordinari, cum deficit a
gratia.


&

そしてそのような人々は、生命の書から消されうる。それはつまり、その抹消が、
神が何かを予知していて、その後にその知を失うというように、神の知に関係す
 るのではなく、知られる事物に関係する。というのも、神がある人が最初に永
 遠の生命に秩序付けられていて、その後に、恩恵を欠くことによって、秩序付
 けられなくなることを知るからである。


\\


{\scshape Ad primum ergo dicendum} quod deletio, ut
dictum est, non refertur ad librum vitae ex parte praescientiae, quasi
in Deo sit aliqua mutabilitas, sed ex parte praescitorum, quae sunt
mutabilia.


&

第一異論に対しては、それゆえ、以下のように言われるべきである。
すでに述べられたとおり、抹消は、神の中に何らかの可変性があるかのように、
 予定の側から生命の書に関係づけられるのではなく、予定され
 る事物の側から関係づけられ、この後者が可変的なのである。


\\


{\scshape Ad secundum dicendum} quod, licet res in Deo
sint immutabiliter, tamen in seipsis mutabiles sunt. Et ad hoc pertinet
deletio libri vitae.


&

第二異論に対しては、以下のように言われるべきである。
事物が神の中に不変なものとしてあるとしても、事物そのものは可変的である。
 そしてこのことに、生命の書の抹消はかかわる。


\\


{\scshape Ad tertium dicendum} quod eo modo quo aliquis
dicitur deleri de libro vitae, potest dici quod ibi scribatur de novo;
vel secundum opinionem hominum, vel secundum quod de novo incipit habere
ordinem ad vitam aeternam per gratiam. Quod etiam sub divina notitia
comprehenditur, licet non de novo.


&

第三異論に対しては、以下のように言われるべきである。
ある人が、生命の書から抹消されると言われるのは、新たにそこに書かれると言
 われうるのと同じしかたによる。それが、人間の意見においてであれ、恩恵を
 通して永遠の生命への秩序を新たに持ち始めることにおいてであれ。このこと
 もまた、新たにではないが、神の知に包含されている。



\end{longtable}

\end{document}

