\documentclass[10pt]{jsarticle} % use larger type; default would be 10pt
%\usepackage[utf8]{inputenc} % set input encoding (not needed with XeLaTeX)
%\usepackage[round,comma,authoryear]{natbib}
%\usepackage{nruby}
\usepackage{okumacro}
\usepackage{longtable}
%\usepqckage{tablefootnote}
\usepackage[polutonikogreek,english,japanese]{babel}
%\usepackage{amsmath}
\usepackage{latexsym}
\usepackage{color}

%----- header -------
\usepackage{fancyhdr}
\lhead{{\it Summa Theologiae} I, q.~56}
%--------------------

\bibliographystyle{jplain}

\title{{\bf PRIMA PARS}\\{\HUGE Summae Theologiae}\\Sancti Thomae
Aquinatis\\{\sffamily QUEAESTIO QUINQUAGESIMASEXTA}\\DE COGNITIONE
ANGELORUM EX PARTE RERUM IMMATERIALIUM}
\author{Japanese translation\\by Yoshinori {\sc Ueeda}}
\date{Last modified \today}


%%%% コピペ用
%\rhead{a.~}
%\begin{center}
% {\Large {\bf }}\\
% {\large }\\
% {\footnotesize }\\
% {\Large \\}
%\end{center}
%
%\begin{longtable}{p{21em}p{21em}}
%
%&
%
%
%\\
%\end{longtable}
%\newpage



\begin{document}
\maketitle
\pagestyle{fancy}

\begin{center}
{\Large 第五十六問\\天使の認識について:非質料的事物の側から}
\end{center}

\begin{longtable}{p{21em}p{21em}}
Deinde quaeritur de cognitione Angelorum ex parte rerum quas
cognoscunt. Et primo, de cognitione rerum immaterialium; secundo, de
cognitione rerum materialium. Circa primum quaeruntur tria.

\begin{enumerate}
\item utrum Angelus cognoscat seipsum.
\item utrum unus cognoscat alium.
\item utrum Angelus per sua naturalia cognoscat Deum.
\end{enumerate}


&

次に、天使が認識する事物の側から、天使の認識について探求される。第一に、
非質料的事物の認識について、第二に、質料的事物の認識についてである。第
一については三つのことが問われる。

\begin{enumerate}
 \item 天使は自分自身を認識するか。
 \item ある天使は別の天使を認識するか。
 \item 天使は、自分に本性的なものによって神を認識するか。
\end{enumerate}

\end{longtable}
\newpage

\rhead{a.~1}
\begin{center}
{\Large {\bf ARTICULUS PRIMUS}}\\ {\large UTRUM ANGELUS COGNOSCAT
SEIPSUM}\\ {\footnotesize II {\itshape SCG}, c.~98; {\itshape De
Verit.}, q.~8, a.~6; III {\itshape de Anima}, lect.~9; {\itshape De
Causis}, lect.~13.}\\ {\Large 第一項\\天使は自分自身を認識するか}
\end{center}

\begin{longtable}{p{21em}p{21em}}


{\huge A}{\scshape d primum sic proceditur}. Videtur quod Angelus
seipsum non cognoscat. Dicit enim Dionysius, VI cap.~{\itshape
Angel.~Hier}., quod Angeli {\itshape ignorant proprias
virtutes}. Cognita autem substantia, cognoscitur virtus. Ergo Angelus
non cognoscit suam essentiam.

&

第一の問題へ、議論は以下のように進められる。天使は自分自身を認識しない
と思われる。理由は以下の通り。ディオニュシウスは『天使階級論』6章で、
天使たちは「自らに固有の諸徳を知らない」と述べている。ところで、実体が
認識されると、徳も認識される。ゆえに、天使は自分の本質を認識しない。

\\


2. {\scshape Praeterea}, Angelus est quaedam substantia singularis,
alioquin non ageret, cum actus sint singularium subsistentium. Sed
nullum singulare est intelligibile. Ergo non potest intelligi. Et ita,
cum Angelo non adsit nisi intellectiva cognitio, non poterit aliquis
Angelus cognoscere seipsum.


&

さらに、天使は、ある種の個的実体である。さもなければ、天使は作用しなかっ
たであろう。作用は、個的実体に属するからである。ところが、どんな個的な
ものも可知的でない。ゆえに、[天使は]知性認識されえない。このようにし
て、天使のもとには知性的認識しかないのだから、何らかの天使が自分を認識
することはありえない。

\\


3. {\scshape Praeterea}, intellectus movetur ab intelligibili, quia
intelligere est {\itshape quoddam pati}, ut dicitur in III {\itshape
de Anima}. Sed nihil movetur aut patitur a seipso; ut in rebus
corporalibus apparet. Ergo Angelus non potest intelligere seipsum.


&

さらに、『デ・アニマ』第三巻で言われるように、知性認識は「ある種の受動」
であるから、知性は、可知的なものによって動かされる。ところが、物体的事
物において明らかなように、何ものも、自分自身によって動かされることはな
い。ゆえに、天使は自分自身を知性認識できない。

\\


{\scshape Sed contra est} quod Augustinus dicit, II {\itshape super
Gen.~ad Litt}., quod Angelus {\itshape in ipsa sua conformatione, hoc
est illustratione veritatis, cognovit seipsum}.


&

しかし反対に、アウグスティヌスは『創世記逐語注解』第二巻で、天使は「自
分自身の一致において、つまり、真理の照明において、自分自身を認識する」
と述べている。

\\


{\scshape Respondeo dicendum} quod, sicut ex supra dictis patet,
obiectum aliter se habet in actione quae manet in agente, et in
actione quae transit in aliquid exterius. Nam in actione quae transit
in aliquid exterius, obiectum sive materia in quam transit actus, est
separata ab agente, sicut calefactum a calefaciente, et aedificatum ab
aedificante. Sed in actione quae manet in agente, oportet ad hoc quod
procedat actio, quod obiectum uniatur agenti, sicut oportet quod
sensibile uniatur sensui, ad hoc quod sentiat actu. Et ita se habet
obiectum unitum potentiae ad huiusmodi actionem, sicut forma quae est
principium actionis in aliis agentibus, sicut enim calor est
principium formale calefactionis in igne, ita species rei visae est
principium formale visionis in oculo.


&

解答する。以下のように言われるべきである。前に言われたことから明らかな
とおり、対象は、作用者の中にとどまる作用と、外にある何かへ出て行く作用
とでは、異なるかたちで作用の中にある。すなわち、外にある何かへ出て行く
働きの中に、対象、あるいは作用がそれへと出て行く質料は、作用者から切り
離されたおのである。たとえば、熱せられるものが熱するものから、建設され
たものが建設するものから分離されているように。これに対して、作用者の中
にとどまる作用の中で、対象は、作用が遂行されるために、作用者と合一され
ていなければならない。たとえば、現実に感覚するためには、可感的なものが
感覚に合一されなければならないように。このような作用のために能力に合一
された対象は、他の諸作用者において、作用の根源である形相としてある。た
とえば、ちょうど、熱が、火における熱する働きの形相的根源であるように、
見られた事物の形象は、目における、見る働きの形相的根源である。

\\



Sed considerandum est quod huiusmodi species obiecti quandoque est in
potentia tantum in cognoscitiva virtute, et tunc est cognoscens in
potentia tantum; et ad hoc quod actu cognoscat, requiritur quod
potentia cognoscitiva reducatur in actum speciei. Si autem semper eam
actu habeat, nihilominus per eam cognoscere potest absque aliqua
mutatione vel receptione praecedenti.


&

しかし、以下のことが考察されるべきである。このような対象の形象は、認識
能力において、ただ可能態のみにある場合があって、その場合には、認識者は、
ただ可能態のみにある。そして、それが現実に認識するためには、認識能力が、
形象の現実態へ引き出される必要がある。ところが、もしそれが、その形象を
常に現実態においてもっているならば、その形象によって、専攻するなんらの
変化も受容もなしに、認識することができるだろう。


\\

Ex quo patet quod moveri ab obiecto non est de ratione cognoscentis
inquantum est cognoscens, sed inquantum est potentia cognoscens. Nihil
autem differt, ad hoc quod forma sit principium actionis, quod ipsa
forma sit alii inhaerens, et quod sit per se subsistens, non enim
minus calor calefaceret si esset per se subsistens, quam calefacit
inhaerens. Sic igitur et si aliquid in genere intelligibilium se
habeat ut forma intelligibilis subsistens, intelliget seipsum. Angelus
autem, cum sit immaterialis, est quaedam forma subsistens, et per hoc
intelligibilis actu. Unde sequitur quod per suam formam, quae est sua
substantia, seipsum intelligat.


&

このことから、以下のことが明らかである。対象によって動かされることは、
認識者である限りでの認識者の性格に属するのではなく、可能態における認識
者という性格に属する。しかし、形相が作用の根源であるためには、その形相
が他のものに内在するか、それ自体によって自存するかとうことは関係ない。
なぜなら、熱は、たとえそれが、それ自体によって自存したとしても、内在す
る熱よりも熱することが少ないことはなかっただろうから。それゆえ、もし、
何かが、可知的なものどもの類の中で、自存する可知的形相としてあったとし
ても、それは自分自身を知性認識する。ところで、天使は非質料的なので、一
種の自存形相であり、そのことによって、現実態において可知的である。した
がって、自分の実体であるところの自分の形相によって、自分自身を知性認識
することが帰結する。

\\


{\scshape Ad primum ergo dicendum} quod littera illa est antiquae
translationis, quae corrigitur per novam, in qua dicitur, {\itshape
praeterea et ipsos}, scilicet Angelos, {\itshape cognovisse proprias
virtutes}; loco cuius habebatur in alia translatione, {\itshape et
adhuc et eos ignorare proprias virtutes}. -- Quamvis etiam littera
antiquae translationis salvari possit quantum ad hoc, quod Angeli non
perfecte cognoscunt suam virtutem, secundum quod procedit ab ordine
divinae sapientiae, quae est Angelis incomprehensibilis.


&

第一異論に対しては、それゆえ、以下のように言われるべきである。かの文言
は昔の翻訳のものであり、それは、新しい翻訳によって訂正されている。そこ
では、「さらに、彼ら(つまり天使たち)は、固有の力を認識した」となって
いる。この箇所の別の翻訳では、「そしてさらに、彼らも固有の能力を知らな
い」となっていた。また、古い翻訳の文言も、以下の点では救済されうる。す
なわち、天使たちは、天使たちにとって把握不可能である神の知恵の秩序から
発出したという点については、自分の能力を完全に認識していない、という点
である。


\\


{\scshape Ad secundum dicendum} quod singularium quae sunt in rebus
corporalibus, non est intellectus, apud nos, non ratione
singularitatis, sed ratione materiae, quae est in eis individuationis
principium. Unde si aliqua singularia sunt sine materia subsistentia,
sicut sunt Angeli, illa nihil prohibet intelligibilia esse actu.


&

第二異論に対しては、以下のように言われるべきである。物体的事物における
個物について、私たちのものに知性的な認識が存在しないのは、それが個物だ
からではなく、それらにおける個体化の根源である質料のためである。したがっ
て、 天使のように、質料なしに自存する何らかの個物があれば、それらが現
実態において可知的であることを妨げるものはない。

\\


{\scshape Ad tertium dicendum} quod moveri et pati convenit
intellectui secundum quod est in potentia. Unde non habet locum in
intellectu angelico; maxime quantum ad hoc quod intelligit
seipsum. Actio etiam intellectus non est eiusdem rationis cum actione
quae in corporalibus invenitur, in aliam materiam transeunte.


&

第三異論に対しては、以下のように言われるべきである。動かされる、受動す
る、ということは、それが可能態においてある限りで知性に適合する。したがっ
て、それは天使の知性にあてはまらないし、最大限に、自分自身を知性認識す
ることには当てはまらない。また、知性の作用は、他の質料へと出て行く、物
体的なものどもの中に見いだされる作用とは、同じ性格を持たない。



\end{longtable}
\newpage



\rhead{a.~2}
\begin{center}
{\Large {\bf ARTICULUS SECUNDUS}}\\ {\large UTRUM UNUS ANGELUS ALIUM
COGNOSCIT}\\ {\footnotesize II {\itshape SCG.}, cap.~98; {\itshape De
Verit.}, q.~8, a.~7.}\\ {\Large 第二項\\天使は他の天使を認識するか}
\end{center}

\begin{longtable}{p{21em}p{21em}}





{\huge A}{\scshape d secundum sic proceditur}. Videtur quod unus
Angelus alium non cognoscat. Dicit enim philosophus, in III {\itshape
de Anima}, quod si intellectus humanus haberet in se aliquam naturam
de numero naturarum rerum sensibilium, illa natura interius existens
prohiberet apparere extranea, sicut etiam si pupilla esset colorata
aliquo colore, non posset videre omnem colorem. Sed sicut se habet
intellectus humanus ad cognoscendas res corporeas, ita se habet
intellectus angelicus ad cognoscendas res immateriales. Cum igitur
intellectus angelicus habeat in se aliquam naturam determinatam de
numero illarum naturarum, videtur quod alias cognoscere non possit.


&

第二項の問題へ、議論は以下のように進められる。天使は他の天使を認識しな
いと思われる。理由は以下の通り。哲学者は『デ・アニマ』第3巻で次のよう
に述べている。もし人間の知性が、可感的事物の本性に数えられる何らかの本
性を持っていたならば、知性の中に存在するその本性は、外的なものが現れる
のを妨げたであろう。ちょうど、もし瞳が、なんらかの色によって色づいてい
たならば、すべての色を見ることができなかったであろうように。ところが、
ちょうど人間の知性が物体的事物を認識することに関係するように、天使の知
性は非質料的事物を認識することへ関係する。ゆえに、天使の知性は、自らの
中に、その本性に数えられるなんらかの限定された本性を持つから、他の本性
を認識することができない。

\\


{\scshape 2 Praeterea}, in libro {\itshape de Causis} dicitur quod
{\itshape omnis intelligentia scit\footnote{Busa版の ``sit''はtypo.}
quod est supra se, inquantum est causata ab eo; et quod est sub se,
inquantum est causa eius}. Sed unus Angelus non est causa
alterius. Ergo unus Angelus non cognoscit alium.


&

さらに、『原因論』という書物の中で、「すべての知性体は、それに原因され
たかぎりで自らの上位のものを知り、それの原因であるかぎりで、自らの下位
のものを知る」と言われている。ところが、或る天使は他の天使の原因ではな
い。ゆえに、或る天使が他の天使を知ることはない。

\\


{\scshape 3 Praeterea}, unus Angelus non potest cognoscere alium per
essentiam ipsius Angeli cognoscentis, cum omnis cognitio sit secundum
rationem similitudinis, essentia autem Angeli cognoscentis non est
similis essentiae Angeli cogniti nisi in genere, ut ex supra dictis
patet; unde sequeretur quod unus Angelus non haberet de alio
cognitionem propriam, sed generalem tantum. Similiter etiam non potest
dici quod unus Angelus cognoscat alium per essentiam Angeli cogniti,
quia illud quo intellectus intelligit, est intrinsecum intellectui;
sola autem Trinitas illabitur menti. Similiter etiam dici non potest
quod unus cognoscat alium per speciem, quia illa species non differret
ab Angelo intellecto, cum utrumque sit immateriale. Nullo igitur modo
videtur quod unus Angelus possit intelligere alium.


&

さらに、或る天使が他の天使を、認識する天使自身の本質によって認識するこ
とはできない。理由は以下の通り。すべて認識は類似の性格に応じてある。と
ころが、認識する天使の本質は、前に述べられたことから明らかなとおり、類
においてでなければ、認識される天使の本質の類似ではない。したがって、或
る天使が他の天使について固有の認識を持つことはできず、ただ類的な認識だ
けを持つことが帰結しただろう。同様に、或る天使が、認識された天使の本質
によって他の天使を認識すると言われることもできない。なぜなら、知性がそ
れによって知性認識するところのものは、知性に内的なものだからである。こ
れに対して、三位一体だけが、精神を貫く。また同様に、或る天使が他の天使
を、形象によって認識すると言われることもできない。なぜなら、ともに非質
料的だから、その形象は、知性認識される天使と異ならないことになるだろう
から。ゆえに、どのようなかたちでも、或る天使が他の天使を知性認識するこ
とはできないと思われる。

\\


{\scshape 4 Praeterea}, si unus Angelus intelligit alium, aut hoc
esset per speciem innatam, et sic sequeretur quod, si Deus nunc de
novo crearet aliquem Angelum, quod non posset cognosci ab his qui nunc
sunt. Aut per speciem acquisitam a rebus, et sic sequeretur quod
Angeli superiores non possent cognoscere inferiores, a quibus nihil
accipiunt. Nullo igitur modo videtur quod unus Angelus alium
cognoscat.


&

さらに、もし或る天使が他の天使を知性認識するならば、それは生得的な形象
によってであるか、事物から受け取られた形象によってであるかのいずれかで
ある。前者の場合、もし神が、今、新たになんらかの天使を創造するならば、
今存在してる天使たちによっては、それが認識されえないだろう。後者の場合、
上位の天使たちは、下位の天使たちから何物も受け取らないので、上位の天使
たちは下位の天使たちを認識しえないことが帰結しただろう。ゆえに、どのよ
うにしても、或る天使が他の天使を認識するとは思えない。

\\


{\scshape Sed contra est} quod dicitur in libro {\itshape de Causis},
quod {\itshape omnis intelligentia scit res quae non corrumpuntur}.


&

しかし反対に、『原因論』という書物の中で、「すべての知性体は、消滅しな
い諸事物を知る」と言われている。


\\


{\scshape Respondeo dicendum} quod, sicut Augustinus dicit, II
{\itshape super Gen.~ad Litt}., ea quae in verbo Dei ab aeterno
praeextiterunt, dupliciter ab eo effluxerunt, uno modo, in intellectum
angelicum; alio modo, ut subsisterent in propriis naturis. In
intellectum autem angelicum processerunt per hoc, quod Deus menti
angelicae impressit rerum similitudines, quas in esse naturali
produxit. In verbo autem Dei ab aeterno extiterunt non solum rationes
rerum corporalium, sed etiam rationes omnium spiritualium
creaturarum. Sic igitur unicuique spiritualium creaturarum a verbo Dei
impressae sunt omnes rationes rerum omnium, tam corporalium quam
spiritualium. Ita tamen quod unicuique Angelo impressa est ratio suae
speciei secundum esse naturale et intelligibile simul, ita scilicet
quod in natura suae speciei subsisteret, et per eam se intelligeret,
aliarum vero naturarum, tam spiritualium quam corporalium, rationes
sunt ei impressae secundum esse intelligibile tantum, ut videlicet per
huiusmodi species impressas, tam creaturas corporales quam spirituales
cognosceret.



&

解答する。以下のように言われるべきである。アウグスティヌスが『創世記逐
語注解』2巻で言うように、神の言葉のなかに永遠から先在したものどもは、
二通りのかたちで、それから流出した。一つには、天使の知性の中へ、もう一
つには、それらが固有の本性において自存するようにである。ところで、天使
の知性へ発出したのは、神が、天使の精神に、自然的存在において産出した諸
事物の類似を、刻印したことによる。ところで、それらが神の言葉の中に永遠
から存在したのは、物体的諸事物の理念だけでなく、すべての霊的被造物の理
念もまた先在した。したがって、物体的なものも霊的なものも、すべての事物
のすべての理念が、神の言葉によって、霊的被造物の各々に刻印された。しか
し、自分の種・形象(species)の理念は、各々の天使に、自然本性的存在と、
可知的存在とで、同時に刻印された。つまり、自分の種の本性において自存す
るように、また、その理念によって、自らを知性認識するようにである。他方、
他の本性の理念は、霊的なものも物体的なものも、ただ可知的存在においてだ
け刻印され、そのような刻印された形象を通して、物体的な被造物も霊的な被
造物も、認識した。





\\


{\scshape Ad primum ergo dicendum} quod naturae spirituales Angelorum
ab invicem distinguuntur ordine quodam, sicut supra dictum est. Et sic
natura unius Angeli non prohibet intellectum ipsius a cognoscendis
aliis naturis Angelorum, cum tam superiores quam inferiores habeant
affinitatem cum natura eius, differentia existente tantum secundum
diversos gradus perfectionis.


&

第一異論に対しては、それゆえ、次のように言われるべきである。天使たちの
霊的本性は、前に述べられたとおり、ある種の秩序によって相互に区別される。
そのため、たんに完全性の異なる度合いに応じてのみ差異があるのだから、上
位の天使も下位の天使も、他の天使の本性との適性をもっているので、一人の
天使の本性が、それ自身の知性が天使たちの他の本性を認識することを妨げる
ことはない。

\\


{\scshape Ad secundum dicendum} quod ratio causae et causati non facit
ad hoc quod unus Angelus alium cognoscat, nisi ratione similitudinis,
inquantum causa et causatum sunt similia. Et ideo, si inter Angelos
ponatur similitudo absque causalitate, remanebit in uno cognitio
alterius.

&

第二異論に対しては、以下のように言われるべきである。原因と原因されたも
のという性格は、一人の天使が他の天使を認識することになんら役立たない。
原因と原因されたものとが似ているかぎりで、類似性の性格によるのでない限
り。ゆえに、もし天使たちの中に、因果性のない類似性が措定されるならば、
一人の天使の中に他の天使の認識があるということは留まるであろう。

\\


{\scshape Ad tertium dicendum} quod unus Angelus cognoscit alium per
speciem eius in intellectu suo existentem, quae differt ab Angelo
cuius similitudo est, non secundum esse materiale et immateriale, sed
secundum esse naturale et intentionale. Nam ipse Angelus est forma
subsistens in esse naturali, non autem species eius quae est in
intellectu alterius Angeli, sed habet ibi esse intelligibile
tantum. Sicut etiam et forma coloris in pariete habet esse naturale,
in medio autem deferente habet esse intentionale tantum.

&

或る天使Aが天使Bを認識するのは、A自身の知性の中に存在するBの形象によっ
てである。この形象は、それの類似であるところの天使(天使B)と、質料的
存在と非質料的存在という点で異なるのではなく、自然本性的存在と、非自然
的存在(esse intentionale)の点で異なる。つまり、天使自身は、自然本性的
存在において自存する形相であるが、他の天使の知性の中にある、彼の形象は
そうでなく、ただ可知的存在だけを持つ。ちょうど、色の形相が、壁において
は自然本性的存在を持ち、媒体においては、非自然的存在だけを持つのと同様
である。

\\


{\scshape Ad quartum dicendum} quod Deus unamquamque creaturam fecit
proportionatam universo quod facere disposuit. Et ideo, si Deus
instituisset facere plures Angelos vel plures naturas rerum, plures
species intelligibiles mentibus angelicis impressisset. Sicut si
aedificator voluisset facere maiorem domum, fecisset maius
fundamentum. Unde eiusdem rationis est quod Deus adderet aliquam
creaturam universo, et aliquam speciem intelligibilem Angelo.

&

第四異論に対しては、以下のように言われるべきである。神は、各々の被造物
を、神が作ると決めた世界に比例して作った。それゆえ、もし神が、より多く
の天使や諸事物の本性を作ることにしたのであったならば、それだけ多くの可
知的形象を、天使の精神に刻印したであろう。ちょうど、もし建築家がより大
きな家を作ることを意志していたならば、より大きな基礎を作ったであろうよ
うに。したがって、神がある被造物を世界に付加することと、ある可知的形象
を天使に付加することとは、同一の根拠に属する。



\end{longtable}
\newpage


\rhead{a.~3}
\begin{center}
{\Large {\bf ARTICULUS TERTIUS}}\\ {\large UTRUM ANGELI PER SUA
NATURALIA DEUM COGNOSCERE POSSIT}\\ {\footnotesize II {\itshape
Sent.}, d.~23, q.~2, a.~1; III {\itshape SCG.}, cap.~41, 49; {\itshape
De Verit.}, q.~8, a.~3.}\\ {\Large 第三項\\天使は、自分が持つ自然本性
的なものによって神を認識しうるか}
\end{center}

\begin{longtable}{p{21em}p{21em}}

{\huge A}{\scshape d tertium sic proceditur}. Videtur quod Angeli per
sua naturalia Deum cognoscere non possint. Dicit enim Dionysius, I
cap. {\itshape de Div. Nom}., quod Deus est {\itshape super omnes
caelestes mentes incomprehensibili virtute collocatus}. Et postea
subdit quod, {\itshape quia est supra omnem substantiam, ab omni
cognitione est segregatus}.


&


第三項の問題へ、議論は以下のように進められる。天使たちは、自分が持って
いる自然本性的なものを通して、神を認識できないと思われる。理由は以下の
通り。ディオニュシウスは、『神名論』第1章で、神は「すべての天の精神を
越え、把握され得ない力によって存立する」と言う。そしてそのあとで、「な
ぜなら、すべての実体を超え、あらゆる認識から遮断されているから」と述べ
ている。

\\



{\scshape 2 Praeterea}, Deus in infinitum distat ab intellectu
Angeli. Sed in infinitum distantia non possunt attingi. Ergo videtur
quod Angelus per sua naturalia non possit Deum cognoscere.


&

さらに、神は無限に天使の知性から隔たっている。ところで、無限に隔たるも
のは、到達され得ない。ゆえに、天使は、自分が持つ自然本性的なものによっ
て、神を認識できないと思われる。

\\


{\scshape 3 Praeterea}, I {\itshape Cor}.~{\scshape xiii} dicitur,
videmus nunc per speculum in aenigmate, tunc autem facie ad faciem. Ex
quo videtur quod sit duplex Dei cognitio, una, qua videtur per
sui\footnote{$=$suam} essentiam, secundum quam dicitur videri facie ad
faciem; alia, secundum quod videtur in speculo creaturarum. Sed primam
Dei cognitionem Angelus habere non potuit per sua naturalia, ut supra
ostensum est. Visio autem specularis Angelis non convenit, quia non
accipiunt divinam cognitionem e rebus sensibilibus, ut dicit
Dionysius, {\scshape vii} cap.~{\itshape de Div.~Nom}. Ergo Angeli per
sua naturalia Deum cognoscere non possunt.


&

さらに、『コリントの信徒への手紙一』13章で、「私たちは、今、鏡を通して
謎において見るが、その時は、顔と顔を合わせてみるであろう」と言われてい
る。このことから、神を認識することには二通りある思われる。一つは、それ
によって、神の本質を通して神が見られる場合であり、これに応じて、「顔と
顔を合わせて見られる」と言われる。もう一つは、被造物の鏡の中に見られる
かぎりにおいてである。ところで、神を認識する第一の認識は、前に示された
とおり、天使が、自分に本性的なものを通して、それを持つことができなかっ
た。また、鏡を通した認識は、天使に適合しない。なぜなら、ディオニュシウ
スが『神名論』7章で述べるように、神の認識を、可感的諸持物から受け取る
のではないからである。ゆえに、天使たちは、自分が持つ自然本性的なものに
よって、神を認識できない。


\\


{\scshape Sed contra}, Angeli sunt potentiores in cognoscendo quam
homines. Sed homines per sua naturalia Deum cognoscere possunt;
secundum illud {\itshape Rom}.~{\scshape i}, {\itshape quod notum est
Dei, manifestum est in illis}. Ergo multo magis Angeli.


&


しかし反対に、天使たちは、人間たちよりも、認識において力が強い。ところ
が、かの『ローマの信徒への手紙』1章「神に属する知られたものは、かのも
のにおいて明らかである」\footnote{「なぜなら、神について知りうる事柄は、
彼らにも明らかだからです」(1:19)}によれば、人間たちは、自分が持つ自然
本性的なものによって、神を認識できる。ゆえに、まして天使は[自分が持つ
自然本性的なものによって、神を認識できる]。

\\


{\scshape Respondeo dicendum} quod Angeli aliquam cognitionem de Deo
habere possunt per sua naturalia. Ad cuius evidentiam, considerandum
est quod aliquid tripliciter cognoscitur. Uno modo, per praesentiam
suae essentiae in cognoscente, sicut si lux videatur in oculo, et sic
dictum est quod Angelus intelligit seipsum. Alio modo, per praesentiam
suae similitudinis in potentia cognoscitiva, sicut lapis videtur ab
oculo per hoc quod similitudo eius resultat in oculo. Tertio modo, per
hoc quod similitudo rei cognitae non accipitur immediate ab ipsa re
cognita, sed a re alia, in qua resultat, sicut cum videmus hominem in
speculo.


&


解答する。以下のように言われるべきである。天使たちは、自分が持つ自然本
性的なものによって、神についてのなんらかの認識を持つことができる。これ
を明らかにするためには、以下のことが考察されるべきである。何かが認識さ
れるのに三通りの形がある。一つには、光が目において見られるように、対象
の本質が認識者の中に現在することによる。このかたちで、天使は自分自身を
知性認識すると言われた。もう一つには、対象の類似性が認識能力の中に現在
することによる。たとえば、石が、その類似性が目の中に結果として生じるこ
とを通して目によって見られる場合がそれである。第三には、認識される事物
の類似が、認識される事物自体から、直接に得られるのではなく、[その類似
が]結果としてそこに生じた他の事物から得られることによる。たとえば、私
たちが人間を鏡の中に見る場合がそれに当たる。


\\



Primae igitur cognitioni assimilatur divina cognitio, qua per
essentiam suam videtur. Et haec cognitio Dei non potest adesse
creaturae alicui per sua naturalia, ut supra dictum est. Tertiae autem
cognitioni assimilatur cognitio qua nos cognoscimus Deum in via, per
similitudinem eius in creaturis resultantem; secundum illud {\itshape
Rom}.~{\scshape i}, {\itshape invisibilia Dei per ea quae facta sunt,
intellecta, conspiciuntur}. Unde et dicimur Deum videre in
speculo. Cognitio autem qua Angelus per sua naturalia cognoscit Deum,
media est inter has duas; et similatur illi cognitioni qua videtur res
per speciem ab ea acceptam. Quia enim imago Dei est in ipsa natura
Angeli impressa, per suam essentiam Angelus Deum cognoscit, inquantum
est similitudo Dei. Non tamen ipsam essentiam Dei videt, quia nulla
similitudo creata est sufficiens ad repraesentandam divinam
essentiam. Unde magis ista cognitio tenet se cum speculari, quia et
ipsa natura angelica est quoddam speculum divinam similitudinem
repraesentans.


&


それゆえ、神の認識は、第一の認識に似ている。神は自らの本質によって[神
に]見られるからである。そして、神をこのように認識することは、前に述べ
られたとおり、自分が持つ自然本性的なものによっては、どんな被造物にもあ
りえない。他方、私たちが現世において神を認識するその認識は、第三の認識
に似ている。なぜなら、被造物の中に結果として生じている神の類似をとおし
ての認識だからである。これは、『ローマの信徒への手紙』1章の「神の見え
ない事柄は、作られたものを通して知性認識され、明らかに見られる」
\footnote{「世界が造られたときから、目に見えない神の性質、つまり神の永
遠の力と神性は被造物に現れており、これを通して神を知ることができます」
(1:20)}による。ところが、天使が、自分が持つ自然本性的なものによって神
を認識するその認識は、これら二つの中間にあり、ある事物が、その事物から
受け取られた形象によって認識される認識に似ている。なぜなら、神の像が、
天使の本性自身の中に刻印されているので、天使は、自分の本質によって
\footnote{創文社訳(Blackfriars版にしたがうという注がある)にしたがって、
Leo版ではper suam essentiamの後ろにあるコンマを前に移動し、per suam
essentiamを後ろの節にかけて読む。}、神の類似である限りで、神を認識する
からである。しかし、[天使が]神の本質そのものを見るわけではない。なぜ
なら、被造のどんな類似も、神の本質を表現するのに十分ではないからである。
したがって、その認識は、むしろ鏡による認識と同類である。なぜなら、天使
の本性自体が、神の類似を表現する一種の鏡だからである。

\\


{\scshape Ad primum ergo dicendum} quod Dionysius loquitur de
cognitione comprehensionis, ut expresse eius verba ostendunt. Et sic a
nullo intellectu creato cognoscitur.


&

第一異論に対しては、それゆえ、以下のように言われるべきである。ディオニュ
シウスは、彼の言葉がはっきりと示しているように、「把握」というレベルの
認識について語っている。その意味では、[神は]どんな被造の知性によって
も、認識されない。

\\


{\scshape Ad secundum dicendum} quod propter hoc quod intellectus et
essentia Angeli in infinitum distant a Deo, sequitur quod non possit
ipsum comprehendere, nec per suam naturam eius essentiam videre. Non
tamen sequitur propter hoc, quod nullam eius cognitionem habere
possit, quia sicut Deus in infinitum distat ab Angelo, ita cognitio
quam Deus habet de seipso, in infinitum distat a cognitione quam
Angelus habet de eo.


&

第二異論に対しては、以下のように言われるべきである。天使の知性と本質が
神から無限に隔たっていることのために、天使が神を把握できないことや、自
分の本質を通して神の本質を観ることができないことが帰結する。しかし、神
についてのどんな認識も持ち得ないことは帰結しない。なぜなら、神が天使か
ら無限に隔たっているように、神が自分自身について持つ認識もまた、天使が
神について持つ認識から無限に隔たっているからである。


\\


{\scshape Ad tertium dicendum} quod cognitio quam naturaliter Angelus
habet de Deo, est media inter utramque cognitionem, et tamen magis se
tenet cum una, ut supra dictum est.



&

第三異論に対しては、以下のように言われるべきである。天使が神について自
然本性的に持つ認識は、どちらの認識からも中間である。しかし、前に述べら
れたとおり、その一つの方により近い。


\end{longtable}

\end{document}