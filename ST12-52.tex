\documentclass[10pt]{jsarticle} % use larger type; default would be 10pt
\usepackage[utf8]{inputenc} % set input encoding (not needed with XeLaTeX)
\usepackage[T1]{fontenc}
%\usepackage[round,comma,authoryear]{natbib}
%\usepackage{nruby}
\usepackage{okumacro}
\usepackage{longtable}
%\usepqckage{tablefootnote}
\usepackage[polutonikogreek,english,japanese]{babel}
%\usepackage{amsmath}
\usepackage{latexsym}
\usepackage{color}
\usepackage{otf}
\usepackage{schemata}
%----- header -------
\usepackage{fancyhdr}
\pagestyle{fancy}
\lhead{{\it Summa Theologiae} I-II, q.52}
%--------------------


\bibliographystyle{jplain}


\title{{\bf Prima Secundae}\\{\HUGE Summae Theologiae}\\Sancti Thomae
Aquinatis\\{\sffamily QUAESTIO QUINQUAGESIMASECUNDA}\\{\bf DE AUGMENTO
HABITUUM}}
\author{Japanese translation\\by Yoshinori {\sc Ueeda}}
\date{Last modified \today}

%%%% コピペ用
%\rhead{a.~}
%\begin{center}
% {\Large {\bf }}\\
% {\large }\\
% {\footnotesize }\\
% {\Large \\}
%\end{center}
%
%\begin{longtable}{p{21em}p{21em}}
%
%&
%
%\\
%\end{longtable}
%\newpage

\begin{document}

\maketitle
\thispagestyle{empty}
\begin{center}
{\Large 『神学大全』第二部の一\\第五十二問\\習慣の増加について}
\end{center}


\begin{longtable}{p{21em}p{21em}}

Deinde considerandum est de augmento habituum. Et circa hoc quaeruntur
tria. 

\begin{enumerate}
 \item utrum habitus augeantur.
 \item utrum augeantur per additionem.
 \item utrum quilibet actus augeat habitum.
\end{enumerate}

&

次に習慣の増加について考察されるべきである。このことをめぐっては三つの
 ことが問われる。
\begin{enumerate}
 \item 習慣は増やされるか。
 \item 習慣は追加によって増やされるか。
 \item どんな作用も習慣を増やすか。
\end{enumerate}
\end{longtable}



\newpage
\rhead{a.~1}
\begin{center}
{\Large {\bf ARTICULUS PRIMUS}}\\
{\large UTRUM HABITUS AUGEANTUR}\\
{\footnotesize Infra, q.66, a.1; {\itshape De Virtut.}, q.1, a.11;
 q.5, a.3; X {\itshape Ethic.}, lect.3.}\\
{\Large 第一項\\習慣は増やされるか}
\end{center}

\begin{longtable}{p{21em}p{21em}}

{\scshape Ad primum sic proceditur}. Videtur quod habitus augeri non
possint. Augmentum enim est circa quantitatem, ut dicitur in V
{\itshape Physic}. Sed habitus non sunt in genere quantitatis, sed in genere
qualitatis. Ergo circa eos augmentum esse non potest.

&

第一項の問題へ、議論は以下のように進められる。
習慣が増やされることはできないと思われる。理由は以下の通り。
『自然学』第5巻で言われるように、増加は量にかかわる。しかるに習慣は量
 の類にはなく性質の類にある。ゆえにそれらをめぐって増加はありえない。

\\


2. {\scshape Praeterea}, habitus est perfectio quaedam, ut dicitur in VII
{\itshape Physic}. Sed perfectio, cum importet finem et terminum, non videtur
posse recipere magis et minus. Ergo habitus augeri non potest.

&

さらに、『自然学』第7巻で言われるように、習慣はある種の完成である。
しかし完成は、終わりや終端を意味するので、より大より小ということを受容
 できないと思われる。ゆえに習慣が増やされることはできない。

\\


3. {\scshape Praeterea}, in his quae recipiunt magis et minus, contingit esse
alterationem, alterari enim dicitur quod de minus calido fit magis
calidum. Sed in habitibus non est alteratio, ut probatur in VII
{\itshape Physic}. Ergo habitus augeri non possunt.

&

さらに、より大より小ということを受け取るものにおいては、変化がありうる。
なぜなら、より少なく熱いものがより多く熱いものになることは、変化すると
言われるからである。しかるに『自然学』第7巻で証明されているように、習
態において変化はない。ゆえに習慣が増やされることはありえない。

\\


{\scshape Sed contra est} quod fides est quidam habitus, et tamen augetur, unde
discipuli domino dicunt, domine, adauge nobis fidem, ut habetur
{\itshape Luc}.~{\scshape xvii}. Ergo habitus augentur.

&

しかし反対に、進行はある種の習慣であるが増やされる。それゆえ、『ルカに
よる福音書』第17章に書かれているように、弟子た
ちは主に「主よ、私たちに信仰を増やしてください」\footnote{「さて、使徒たちが、「私どもの信仰を増してください」と言ったとき、 」(17:5)}と言う。ゆえに習慣は増
やされる。

\\


{\scshape Respondeo dicendum quod} augmentum, sicut et alia ad quantitatem
pertinentia, a quantitatibus corporalibus ad res spirituales
intelligibiles transfertur; propter connaturalitatem intellectus
nostri ad res corporeas, quae sub imaginatione cadunt. 
&

解答する。以下のように言われるべきである。増大は、量に属する他のものと
同じように、物体的な量から可知的で霊的な事物へ転用される。それは私たち
の知性の物体的事物への共本性のためであり、それがあるのは物体的事物が想
像力の中に入ってくるからである。

\\


Dicitur autem
in quantitatibus corporeis aliquid magnum, secundum quod ad debitam
perfectionem quantitatis perducitur, unde aliqua quantitas reputatur
magna in homine, quae non reputatur magna in elephante. 

&

さて、物体的量の中で何かが大きいと言われるのは、量のしかるべき完全性へ導かれる限りにおいてである。したがって、ゾウにおいては大きいと見なされないようなある量が、人間においては大きいと見なされる。

\\


Unde et in
formis dicimus aliquid magnum, ex hoc quod est perfectum. Et quia
bonum habet rationem perfecti, propter hoc in his quae non mole magna
sunt, idem est esse maius quod melius, ut Augustinus dicit, in VI {\itshape de
Trin}. 


&

したがって、形相においても私たちはあるものが大きいことを、それが完全で
あることに基づいて言う。そして善は完全という性格を持つの、そのために、
アウグスティヌスが『三位一体論』第6巻で言うように「嵩においてという意
味ではなく大きいものにおいて、より大きいこととより善いことは同じであ
る」。

\\

Perfectio autem formae dupliciter potest considerari, uno modo,
secundum ipsam formam; alio modo, secundum quod subiectum participat
formam. 
Inquantum igitur attenditur perfectio formae secundum ipsam
formam, sic dicitur ipsa esse parva vel magna; puta magna vel parva
sanitas vel scientia. 


&

ところで、形相の完全性は二通りに考察されうる。一つには、その形相自体に
即してであり、もう一つには、形相を分有する基体に即してである。ゆえに、
 形相の完全性が形相自体に即して見出される限りにおいて、その意味で形相が小さ
 いとかい大きいといわれる。たとえば健康や学知が大きいとか小さいとかと
 言われるように。

\\


Inquantum vero attenditur perfectio formae
secundum participationem subiecti, dicitur magis et minus; puta magis
vel minus album vel sanum. 


&

他方、基体の分有に即して形相の完全性が見出される限りでは、たとえば「白
 い」や「健康である」ということがより大きい、小さいと言われる。

\\


Non autem ista distinctio procedit secundum
hoc, quod forma habeat esse praeter materiam aut subiectum, sed quia
alia est consideratio eius secundum rationem speciei suae, et alia
secundum quod participatur in subiecto. 


&

ところで、この区別が出てくるのは、形相が質料や基体の外に存在を持つから
 ではなく、自らの種の性格における考察と、基体において分有される限りで
 の考察が別々だからである。

\\


Secundum hoc igitur, circa
intensionem et remissionem habituum et formarum, fuerunt quatuor
opiniones apud philosophos, ut Simplicius narrat in {\itshape Commento
Praedicamentorum}. 


&

ゆえに、習慣と形相の強化と減衰について、シンプリキオスが『カテゴリー論
 注解』で語るように、哲学者たちの間に四つの意見があった。

\\


Plotinus enim et alii Platonici ponebant ipsas
qualitates et habitus suscipere magis et minus, propter hoc quod
materiales erant, et ex hoc habebant indeterminationem quandam,
propter materiae infinitatem. 


&

すなわち、プロティノスとその他のプラトン主義者たちは、性質や習慣自体が
より大より小ということを保持すると考えた。その理由は、それらは質料的な
 ものなので、そこから、質料の無限性のためにある種の非限定性を持つため
 である。

\\


--- Alii vero in contrarium ponebant quod
ipsae qualitates et habitus secundum se non recipiebant magis et
minus; sed qualia dicuntur magis et minus, secundum diversam
participationem; puta quod iustitia non dicatur magis et minus, sed
iustum. Et hanc opinionem tangit Aristoteles in praedicamentis.


&

一方、他の人たちは反対に、性質や習慣はそれ自体においてより大きいより小
さいということを容れず、むしろ「どのように」ということが、さまざまな分
有によって、より大きいより小さいと言われる、と考えた。そしてアリストテ
レスは『カテゴリー論』の中でこの意見に触れている。

\\


--- Tertia
fuit opinio Stoicorum, media inter has. Posuerunt enim quod aliqui
habitus secundum se recipiunt magis et minus, sicuti artes; quidam
autem non, sicut virtutes. 


&

第三に、これらの間にストア派の意見があった。すなわち彼らは、技術のよう
なある習慣はそれ自体に即してより大きいより小さいということを受け取るが、
徳のような別の習慣はそうでないと述べた。

\\



--- Quarta opinio fuit quorundam dicentium quod
qualitates et formae immateriales non recipiunt magis et minus,
materiales autem recipiunt. 


&

第四は、非質料的な性質と形相はより大きいより小さいということを受
け取らないが、質料的なそれは受け取ると述べた人たちの意見だった。

\\


Ut igitur huius rei veritas manifestetur,
considerandum est quod illud secundum quod sortitur aliquid speciem,
oportet esse fixum et stans, et quasi indivisibile, quaecumque enim ad
illud attingunt, sub specie continentur; quaecumque autem recedunt ab
illo, vel in plus vel in minus, pertinent ad aliam speciem, vel
perfectiorem vel imperfectiorem. 


&

ゆえに、この事柄の真理が明らかにされるためには、以下のことが考察されな
 ければならない。すなわち、あるものが、それにしたがって種に分けられる
 ところのものは、固定して安定して、言わば個的なものでなければならない。
なぜなら、それに到達するものは種に含まれるが、より多くであろうとより少
 なくであろうと、それから逸脱するものは、より完全なあるいはより不完全
 な他の種に属するからである。

\\


Unde philosophus dicit, in VIII
{\itshape Metaphys}., quod species rerum sunt sicut numeri, in quibus additio vel
diminutio variat speciem. 


&

このことから哲学者は『形而上学』第8巻で、諸事物の種は数のようなもので
 あり、そこにおいて足すことと引くことが、種を異なるものとする、と述べ
 ている。

\\


Si igitur aliqua forma, vel quaecumque res,
secundum seipsam vel secundum aliquid sui, sortiatur rationem speciei;
necesse est quod, secundum se considerata, habeat determinatam
rationem, quae neque in plus excedere, neque in minus deficere
possit. 


&

ゆえにもし、ある形相や何であれある事物が、自らに即して、あるいは自らに
属する何かに即して種の性格へと分けられるとすれば、それ自体において考察
されたそれは限定された性格を持つのであり、より多くへ越え出たりより少な
く足りなかったりすることはありえない。

\\


Et huiusmodi sunt calor et albedo, et aliae huiusmodi
qualitates quae non dicuntur in ordine ad aliud, et multo magis
substantia, quae est per se ens. 


&

そして熱や白さや、その他、他のものへの秩序において語られるのではない性
質、とりわけ自体的な有である実体は、このようなものである。

\\


Illa vero quae recipiunt speciem ex
aliquo ad quod ordinantur, possunt secundum seipsa diversificari in
plus vel in minus, et nihilominus sunt eadem specie, propter unitatem
eius ad quod ordinantur, ex quo recipiunt speciem. 


&

他方、種を、それへと秩序づけられているところのあるものから受け取るもの
は、それ自身に即して、より多くあるいはより少なくといったかたちで多様化
される。にもかかわらずそれは同一の種においてであり、それは、種をそれに基づい
て受け取るところの、それへと秩序づけられているところのものの一性のた
めである。

\\


Sicut motus
secundum se est intensior et remissior, et tamen remanet eadem
species, propter unitatem termini, ex quo specificatur. Et idem potest
considerari in sanitate, nam corpus pertingit ad rationem sanitatis,
secundum quod habet dispositionem convenientem naturae animalis, cui
possunt dispositiones diversae convenientes esse; unde potest variari
dispositio in plus vel in minus, et tamen semper remanet ratio
sanitatis. 


&

たとえば、運動はそれ自体において強度が高かったり低かったりするが、しか
 し、それに基づいて種に分けられるところの終極の一性のために同じ種に留
 まる。同じことは健康においても考察されうるのであり、すなわち身体が健
 康の性格に到達するのは動物の本性に適合する態勢をもつ限りにおいてであ
 るが、それにはさまざまな態勢が適合的でありうるので、より多くより少な
 くという点でその態勢は変化しうるが、しかし常に同一の健康の性格に留ま
 る。

\\


Unde philosophus dicit, in X {\itshape Ethic}., quod {\itshape sanitas ipsa
recipit magis et minus, non enim eadem est commensuratio in omnibus,
neque in uno et eodem semper; sed remissa permanet sanitas usque ad
aliquid}. 


&

このことから哲学者は『ニコマコス倫理学』第10巻で以下のように述べている。
 「健康自体はより大きいより小さいということを受け取るが、全てのものに
 おいて同一の尺度があるわけではなく、常に一つの同一のものに留まるわけ
 でもない。むしろ健康が失われてもある程度までは健康に留まる」。

\\


Huiusmodi autem diversae dispositiones vel commensurationes
sanitatis se habent secundum excedens et excessum, unde si nomen
sanitatis esset impositum soli perfectissimae commensurationi, tunc
ipsa sanitas non diceretur maior vel minor. 


&

しかしこういったさまざまな健康の態勢や共尺度性は、越えるとか越えられる
 ということにしたがってあるので、もし「健康」という名がこの上なく完全
 な共尺度性に付けられるのであったならば、その場合には健康はより大きい
 より小さいということが言われなかったであろう。

\\


Sic igitur patet qualiter
aliqua qualitas vel forma possit secundum seipsam augeri vel minui, et
qualiter non. 


&

ゆえに、このようにして、どのようにしてある性質や形相が、それ自体に即し
 て増えたり減ったりし、どのようにしてそうでないかが明らかである。

\\


Si vero consideremus qualitatem vel formam secundum
participationem subiecti, sic etiam inveniuntur quaedam qualitates et
formae recipere magis et minus, et quaedam non. 


&

しかし他方で、もし私たちが基体の分有に即して性質や形相を考察するならば、
 その場合にもある種の性質や形相がより大きいより小さいということを受け
 取り、別の種類のものはそうでないことが見出される。

\\


Huiusmodi autem
diversitatis causam Simplicius assignat ex hoc, quod substantia
secundum seipsam non potest recipere magis et minus, quia est ens per
se. 


&

ところで、このような差異の原因を、シンプリキオスは、実体が自体的な有で
 あるために、実体はそれ自体に即してより大きいより小さいということを容
 れないということに基づいて指定している。

\\


Et ideo omnis forma quae substantialiter participatur in subiecto,
caret intensione et remissione, unde in genere substantiae nihil
dicitur secundum magis et minus. 


&

そして、それゆえ、基体において実体的に分有される形相は強化と減衰を欠き、
 したがって実体の類においては、何もより大きいより小さいということにしたがっ
 て語られない。

\\


Et quia quantitas propinqua est
substantiae, et figura etiam consequitur quantitatem; inde est quod
neque etiam in istis dicitur aliquid secundum magis aut minus. 


&

そして量は実体に近く、また形は量に伴うので、これらにおいてもより大きい
 より小さいということに即して何事も語られない。

\\


Unde
philosophus dicit, in VII {\itshape Physic}., quod cum aliquid accipit formam et
figuram, non dicitur alterari, sed magis fieri. 


&

このことから哲学者は『自然学』第7巻で、何かが形相と形を受け取るとき、
 変化するとは言われず、むしろ成ると言われる、と述べる。

\\


--- Aliae vero qualitates,
quae sunt magis distantes a substantia, et coniunguntur passionibus et
actionibus, recipiunt magis et minus secundum participationem
subiecti. 


&

他方で、(形とは)別の性質は、実体からより離れていて、受動と能動に結び
 ついているので、基体の分有に即して、より大きいより小さいということを
 受容する。

\\


Potest autem et magis explicari huiusmodi diversitatis
ratio. Ut enim dictum est, id a quo aliquid habet speciem, oportet
manere fixum et stans in indivisibili. 


&

しかしこのような差異の根拠はより説明されることが可能である。
すなわち、すでに述べられたとおり、何かがそこから種を受け取るところのも
のは、個体において固定されて安定して留まらなければならない。

\\


Duobus igitur modis potest
contingere quod forma non participatur secundum magis et minus. 




&

ゆえに、二通りのしかたで、形相がより大きいより小さいということに即して
分有されないということが起こる。

\\

Uno modo, quia participans habet speciem secundum ipsam.  Et inde est
quod nulla forma substantialis participatur secundum magis et
minus. Et propter hoc philosophus dicit, in VIII {\itshape Metaphys}.,
quod, {\itshape sicut numerus non habet magis neque minus, sic neque
substantia quae est secundum speciem}, idest quantum ad
participationem formae specificae; {\itshape sed si quidem quae cum
materia}, idest, secundum materiales dispositiones invenitur magis et
minus in substantia.


&

一つには、分有するものが、それ自体に即して種を持っているからである。こ
のことから、どんな実体形相も、より大きいより小さいということにしたがっ
て受け取られない。このため、哲学者は『形而上学』第8巻で以下のように述
べる。「ちょうど数がより大きいより小さいということをもたないように、種
における実体もまたそれをもたない」、すなわち、種に分ける形相を分有する
ことにかんしてそうでないが、「もし質料と共にあるものであれば」、すなわ
ち質料的な態勢に即しては、より大きいより小さいということが実体において
見出される。

\\


--- Alio modo potest contingere ex
hoc quod ipsa indivisibilitas est de ratione formae, unde oportet
quod, si aliquid participet formam illam, quod participet illam
secundum rationem indivisibilitatis. 


&

もう一つのしかたでは、個体性それ自体が形相の性格に由来し、そのため、も
しあるものがその形相を分有すれば、それを個体という性格に即して分有しな
ければならない、ということに基づいてそれが生じうる。

\\


Et inde est quod species numeri
non dicuntur secundum magis et minus, quia unaquaeque species in eis
constituitur per indivisibilem unitatem. 


&

このことから、数の種はより大きいより小さいということに即して語られない。
なぜなら、それぞれの種がそれらにおいて、不可分な一性によって構成され
ているからである。

\\


Et eadem ratio est de
speciebus quantitatis continuae quae secundum numeros accipiuntur ut
bicubitum et tricubitum; et de relationibus, ut duplum et triplum; et
de figuris, ut trigonum et tetragonum. 


&

同じ理由が、たとえば2クビトゥス\footnote{cubitusは数の単位。}や3クビトゥ
 スのような、数にしたがって理解される連続量の種や、二倍、三倍といった
 関係、そして三角形、四角形といった形についても当てはまる。

\\


Et hanc rationem ponit
Aristoteles in {\itshape Praedicamentis}, ubi, assignans rationem quare figurae
non recipiunt magis et minus, dicit, {\itshape quae quidem enim recipiunt
trigoni rationem et circuli, similiter trigona vel circuli sunt}, quia
scilicet indivisibilitas est de ipsa eorum ratione, unde quaecumque
participant rationem eorum, oportet quod indivisibiliter
participent. 


&

そしてこの理由をアリストテレスは『カテゴリー論』の中で述べていて、そこ
で彼は、形がより大きいより小さいということを受け取らない理由を示して以
下のように述べる。「三角形の性格や円の性格を受け取るものは、同様に三角
形であり円である」。というのも、個体性がそれらの性格に属しているので、
それらの性格を分有するものはなんでも、個体的に分有しなければならないか
らである。

\\


Sic igitur patet quod, cum habitus et dispositiones
dicantur secundum ordinem ad aliquid, ut dicitur in VII {\itshape Physic}.,
dupliciter potest intensio et remissio in habitibus et dispositionibus
considerari. 


&

ゆえに、このようにして、『自然学』第7巻で言われるように、習慣と態勢は
他のものへの秩序において語られるので、二通りのしかたで、習慣と態勢に
おける増大と減衰について考察されうる。


\\


Uno modo, secundum se, prout dicitur maior vel minor
sanitas; vel maior vel minor scientia, quae ad plura vel pauciora se
extendit. 


&

一つにはそれ自体に即してであり、健康がより大きいより小さいと言われるよ
 うにである。あるいは、より多くあるいはより少なく及ぶということで、学
 知がより大きいより小さいと言われるように。

\\


Alio modo, secundum participationem subiecti, prout scilicet
aequalis scientia vel sanitas magis recipitur in uno quam in alio,
secundum diversam aptitudinem vel ex natura vel ex consuetudine. 


&

もう一つのしかたでは、基体の分有に即してであり、等しい学知や健康が、他のものにおいてよりもよ
 り多くあるものにおいて受け取られる場合である。これは、本性または慣習
 にもとづくさまざまな適応性のために生じる。

\\


Non
enim habitus et dispositio dat speciem subiecto, neque iterum in sui
ratione includit indivisibilitatem. Quomodo autem circa virtutes se
habeat, infra dicetur.

&

じっさい、習慣や態勢は基体に種を与えず、さらにその性格の中に個体性を含まない。
しかしそれらがどのように諸徳をめぐって関係するかは、後に述べられる。

\\


{\scshape Ad primum ergo dicendum} quod, sicut nomen magnitudinis derivatur a
quantitatibus corporalibus ad intelligibiles perfectiones formarum;
ita etiam et nomen augmenti, cuius terminus est magnum.

&

第一異論に対しては、それゆえ、以下のように言われるべきである。
「大きさ」という名が物体的な量から形相の可知的な完全性へ派生するように、
 「増加」という名もそうである。それの終極は「大きい」ことなので。

\\


{\scshape Ad secundum dicendum} quod habitus quidem perfectio est, non tamen
talis perfectio quae sit terminus sui subiecti, puta dans ei esse
specificum. Neque etiam in sui ratione terminum includit, sicut
species numerorum. Unde nihil prohibet quin recipiat magis et minus.

&

第二異論に対しては以下のように言われるべきである。
習慣は完全性だが、自らの基体の終極であるような完全性ではない。たとえば、
 基体に種に分ける存在を与えるような完全性は、そのような完全性であるが。
また、数の種のように、自らの性格の中に終極を含まない。したがって、より
 大きいより小さいということを容れても問題はない。

\\


{\scshape Ad tertium dicendum} quod alteratio primo quidem est in qualitatibus
tertiae speciei. In qualitatibus vero primae speciei potest esse
alteratio per posterius, facta enim alteratione secundum calidum et
frigidum, sequitur animal alterari secundum sanum et aegrum. Et
similiter, facta alteratione secundum passiones appetitus sensitivi,
vel secundum vires sensitivas apprehensivas, sequitur alteratio
secundum scientias et virtutes, ut dicitur in VII {\itshape Physic}.
&

第三異論に対しては以下のように言われるべきである。変化は第一に性質の第
三の種の中にある。他方、性質の第一の種の中には、より後なるしかたで変化
がありうる。というのも、熱い冷たいということに即した変化が生じた後、健
康と病気に即した変化が動物に起こるからである。同様に、感覚的欲求の情念
や、感覚的な把握力に即した則した変化が生じると、学知や徳における変化が
後続する。これは『自然学』第7巻で言われている通りである。


\end{longtable}
\newpage


\rhead{a.~2}
\begin{center}
{\Large {\bf ARTICULUS SECUNDUS}}\\
{\large UTRUM HABITUS AUGEATUR PER ADDITIONEM}\\
{\footnotesize II$^{a}$II$^{ae}$, q.24, a.5; {\itshape De Virtut.},
 q.1, a.11; q.5, a.3.}\\
{\Large 第二項\\習慣は付加によって増やされるか}
\end{center}

\begin{longtable}{p{21em}p{21em}}


{\scshape Ad secundum sic proceditur}. Videtur quod augmentum habituum fiat per
additionem. Nomen enim augmenti, ut dictum est, a quantitatibus
corporalibus transfertur ad formas. Sed in quantitatibus corporalibus
non fit augmentum sine additione, unde in I {\itshape de Generat}. dicitur quod
{\itshape augmentum est praeexistenti magnitudini additamentum}. Ergo et in
habitibus non fit augmentum nisi per additionem.


&

第二項の問題へ、議論は以下のように進められる。
習慣の増大は付加によって生じると思われる。理由は以下の通り。
すでに述べられたとおり、増大という名称は、物体的な量から形相へと転用さ
 れる。しかるに、物体的な量において増大は付加なしには生じない。
このことから『生成消滅論』第1巻で「増大は、先在する大きさに付加される
 ことである」と言われている。ゆえに習慣においても付加によらなければ増
 大はない。

\\



2. {\scshape Praeterea}, habitus non augetur nisi aliquo agente. Sed omne agens
aliquid facit in subiecto patiente, sicut calefaciens facit calorem in
ipso calefacto. Ergo non potest esse augmentum nisi aliqua fiat
additio.


&

さらに、習慣が増やされるのは何らかの作用者によってである。しかるに全て
 の作用者は受動する基体において何かを作る。たとえば熱するものは、熱せ
 られる物自体の中に熱を作る。ゆえに、何らかの付加がなければ増大はあり
 えない。

\\



3. {\scshape Praeterea}, sicut id quod non est album, est in potentia ad album; ita
id quod est minus album, est in potentia ad magis album. Sed id quod
non est album, non fit album nisi per adventum albedinis. Ergo id quod
est minus album, non fit magis album nisi per aliquam aliam albedinem
supervenientem.


&

さらに、白くないものは白への可能態にあるように、あまり白くないものは、
 より白いものへの可能態にある。しかるに白くないものは、白性の到来によ
 らなければ白くならない。ゆえに、あまり白くないものも、何らかの他の白
 性の到来によらなければ、より白くはならない。

\\



{\scshape Sed contra est} quod philosophus dicit, in IV {\itshape
Physic}., {\itshape ex calido fit magis calidum, nullo facto in
materia calido, quod non esset calidum quando erat minus
calidum}. Ergo, pari ratione, nec in aliis formis quae augentur, est
aliqua additio.


&

しかし反対に、哲学者は『自然学』第4巻で「あまり熱くないときに熱いもの
でなかったようなものを、質料の中で熱いものにすることなしに、熱いものか
らより熱いものが生じる」と述べている。同じ理由で、作用を受ける他の形相
においても、何かの付加があるわけではない。

\\



{\scshape Respondeo dicendum} quod huius quaestionis solutio dependet ex
praemissa. Dictum est enim supra quod augmentum et diminutio in formis
quae intenduntur et remittuntur, accidit uno modo non ex parte ipsius
formae secundum se consideratae, sed ex diversa participatione
subiecti. 


&

解答する。以下のように言われるべきである。
この問題の解答は、前項の解答に依存する。
すなわち、強化されたり減衰したりする形相における増大と減少は、一つのし
 かたでは、それ自身において考察された形相自体の側から生じるのではなく、
 基体のさまざまな分有に基づいて生じると言われた。

\\

Et ideo huiusmodi augmentum habituum et aliarum formarum,
non fit per additionem formae ad formam; sed fit per hoc quod
subiectum magis vel minus perfecte participat unam et eandem
formam. 


&

ゆえに、このような習慣や他の形相の増大は、形相を形相に付加することで生
 じるのではなく、むしろ基体がより多くあるいはより少なく完全に、一つの
 同一の形相を分有することから生じる。

\\

Et sicut per agens quod est actu, fit aliquid actu calidum,
quasi de novo incipiens participare formam, non quod fiat ipsa forma,
ut probatur VII {\itshape Metaphys}.; ita per actionem intensam ipsius agentis
efficitur magis calidum, tanquam perfectius participans formam, non
tanquam formae aliquid addatur. 


&

そして、『形而上学』第7巻で証明されているとおり、現実態における作用者によって、あたかも新たに形相を分有し始める
ものであるかのように、あるものは現実に熱いものになるのであり、形相自体
 が生じるのではないように、作用者の強力な作用によって、形相が付加され
 るようにしてではなく、より完全に形相を分有するようにして、より熱いも
 のが作り出される。

\\

Si enim per additionem intelligeretur
huiusmodi augmentum in formis, hoc non posset esse nisi vel ex parte
ipsius formae, vel ex parte subiecti. 


&

もしも、形相におけるこのような増大が付加によって理解されるとしたならば、
それは形相自体の側か、あるいは基体の側かのどちらかからでしかありえない。

\\


Si autem ex parte ipsius formae,
iam dictum est quod talis additio vel subtractio speciem variaret;
sicut variatur species coloris, quando de pallido fit album. 


&

もし形相自体の側からだとすると、すでに、そのような加減は種を変えてしま
うことが述べられた。ちょうど、淡い色から白が生じる場合に、色の種
が変わるように。

\\


Si vero
huiusmodi additio intelligatur ex parte subiecti, hoc non posset esse
nisi vel quia aliqua pars subiecti recipit formam quam prius non
habebat, ut si dicatur frigus crescere in homine qui prius frigebat in
una parte, quando iam in pluribus partibus friget, vel quia aliquod
aliud subiectum additur participans eandem formam, sicut si calidum
adiungatur calido, vel album albo. 


&

他方、もしそのような付加が基体の側から理解されるとしたならば、それは、
たとえば、一つの部位で冷えていた人が複数の部位で冷えた場合に冷たさが増
加したと言われる場合のように、基体のある部分が、以前にもっていなかった
形相を受け取るか、あるいは、たとえば熱いものに(別の)熱いものが、白い
ものに(別の)白いものが結合される場合のように、同じ形相を分有して他の
何かが基体に付加されるかのどちらかである。


\\


Sed secundum utrumque istorum
duorum modorum, non dicitur aliquid magis album vel calidum, sed
maius. 

&

しかし、このどちらのしかたによっても、あるものがより白いとかより熱いと
は言われず、より大きく(白い、熱い)と言われる。

\\

Sed quia quaedam accidentia augentur secundum seipsa, ut supra
dictum est, in quibusdam eorum fieri potest augmentum per
additionem. 

&

しかし、前に述べられたとおり、ある種の附帯性は、それ自体に即して増やさ
 れるので、それらのあるものにおいては、付加による増加が生じうる。


\\


Augetur enim motus per hoc quod ei aliquid additur vel
secundum tempus in quo est, vel secundum viam per quam est, et tamen
manet eadem species, propter unitatem termini. 


&

たとえば運動は、そこにおいてある時間やそれを通る道において、何かが付加
されることによって、増やされる。しかし、終極が一つなので、同一の種に
留まる。

\\

Augetur etiam
nihilominus motus per intensionem, secundum participationem subiecti,
inquantum scilicet idem motus potest vel magis vel minus expedite aut
prompte fieri. 


&

それにもかかわらず、運動は強度によってもまた増やされる。それは、基体の分有に即し
 て、それが同じ運動を、より多くあるいはより少なく容易に、あるいは迅速
 に行う限りにおいてである。


\\

Similiter etiam et scientia potest augeri secundum
seipsam per additionem, sicut cum aliquis plures conclusiones
geometriae addiscit, augetur in eo habitus eiusdem scientiae secundum
speciem. 


&

同様に、学知もまた、それ自体に即して付加によって増やされる。たとえば、
ある人が幾何学のより多くの結論を学ぶときに、種においては同一の学知に
属する習慣が、それにおいて増やされる。

\\

Augetur nihilominus scientia in aliquo, secundum
participationem subiecti, per intensionem, prout scilicet expeditius
et clarius unus homo se habet alio in eisdem conclusionibus
considerandis. 


&

それだけでなく、ある事柄における学知は、基体の分有に即して、強度によっ
て増やされる。すなわち、一人の人が、同じ結論を考察する他の人よりも、よ
り容易に明晰な状態にある場合がそれである。


\\


In habitibus autem corporalibus non multum videtur
fieri augmentum per additionem. Quia non dicitur animal sanum
simpliciter, aut pulchrum, nisi secundum omnes partes suas sit
tale. 


&

他方で、身体的な習慣においては、付加による増加はあまり多くは見られない。
 なぜなら、動物が端的に健康であるとか美しいということは、その全ての部
 分がそのようである限りにおいてでなければ言われないからである。

\\


Quod autem ad perfectiorem commensurationem perducatur, hoc
contingit secundum transmutationem simplicium qualitatum; quae non
augentur nisi secundum intensionem, ex parte subiecti
participantis. 


&

他方、より完全な共尺度性へ導かれるものは、単純な性質変容によって生じる
 が、これは、分有する基体の側から、強度においてしか増やされない。


\\

Quomodo autem se habeat circa virtutes, infra dicetur.

&

また、徳をめぐってこれがどのようであるかは、後に論じられる。

\\



{\scshape Ad primum ergo dicendum} quod etiam in magnitudine corporali contingit
dupliciter esse augmentum. Uno modo, per additionem subiecti ad
subiectum; sicut est in augmento viventium. Alio modo, per solam
intensionem, absque omni additione; sicut est in his quae rarefiunt,
ut dicitur in IV Physic.


&

第一異論に対しては、それゆえ、以下のように言われるべきである。
物体的な大きさにおいても、二通りのしかたで増大が生じうる。
一つには、基体を基体に加えることによってであり、たとえば生物の成長がそ
れである。もう一つは、あらゆる付加なしにたんに強度による増大であり、
『自然学』第4巻で言われるように、それはたとえば希薄化するものにおいて見られ
る。


\\



{\scshape Ad secundum dicendum} quod causa augens habitum, facit quidem semper
aliquid in subiecto, non autem novam formam. Sed facit quod subiectum
perfectius participet formam praeexistentem, aut quod amplius se
extendat.


&

第二異論に対しては以下のように言われるべきである。習慣を原因する作用者
は、たしかに常に何かを基体において作るが、しかし新しい形相を作るのでは
ない。むしろ基体がより完全に先在する形相を分有するようにしたり、あるい
はより広範囲に広がるようにしたりする。

\\



{\scshape Ad tertium dicendum} quod id quod nondum est album, est in potentia ad
formam ipsam, tanquam nondum habens formam, et ideo agens causat novam
formam in subiecto. Sed id quod est minus calidum aut album, non est
in potentia ad formam, cum iam actu formam habeat, sed est in potentia
ad perfectum participationis modum. Et hoc consequitur per actionem
agentis.


&

第三異論に対しては以下のように言われるべきである。
まだ白くないものは、その形相に対して、まだ形相をもっていないものとして
可能態にある。それゆえ、作用者は基体において新しい形相を原因する。
しかし、あまり熱くないものやあまり白くないものは、すでに現実に形相をもっ
 ているのだから、形相に対して可能態にあるのではなく、分有の完全なあり
 方に対して可能態にある。そしてこのことは作用者の作用に伴う。


\end{longtable}
\newpage

\rhead{a.~3}
\begin{center}
{\Large {\bf ARTICULUS TERTIUS}}\\
{\large UTRUM QUILIBET ACTUS AUGEAT HABITUM}\\
{\Large 第三項\\どんな作用も習慣を増やすか}
\end{center}

\begin{longtable}{p{21em}p{21em}}

{\scshape Ad tertium sic proceditur}. Videtur quod quilibet actus augeat
habitum. Multiplicata enim causa, multiplicatur effectus. Sed actus
sunt causa habituum aliquorum, ut supra dictum est. Ergo habitus
augetur, multiplicatis actibus.

&

第三項の問題へ、議論は以下のように進められる。
どんな作用も習慣を増やすと思われる。理由は以下の通り。
原因が多様化すると結果も多様化する。しかるに前に述べられたとおり、作用は何らかの習慣の原因で
 ある。ゆえに作用が多様化すると習慣も増やされる。

\\



2. {\scshape Praeterea}, de similibus idem est iudicium. Sed omnes actus ab eodem
habitu procedentes sunt similes, ut dicitur in II {\itshape Ethic}. Ergo, si
aliqui actus augeant habitum, quilibet actus augebit ipsum.

&

さらに、類似したものについて、判断は同一である。
しかるに、『ニコマコス倫理学』第2巻で言われるように、同じ習慣から出て
 くる全ての作用は類似する。ゆえに、もしある作用が習慣を増やすなら、ど
 んな作用もそれを増やすであろう。

\\



3. {\scshape Praeterea}, simile augetur suo simili. Sed quilibet actus est similis
habitui a quo procedit. Ergo quilibet actus auget habitum.

&

さらに、類似したものは、自らに類似したものによって作用を受ける。
しかるに、どんな作用も、そこから出てくる習慣に類似する。
ゆえにどんな作用も習慣を増やす。

\\



{\scshape Sed contra}, idem non est contrariorum causa. Sed, sicut dicitur in II
{\itshape Ethic}., aliqui actus ab habitu procedentes diminuunt ipsum; utpote cum
negligenter fiunt. Ergo non omnis actus habitum auget.

&

しかし反対に、同一のものは、相反するものの原因ではない。しかるに、『ニ
 コマコス倫理学』第2巻で言われているように、習慣から出てくるある作用は、
 その習慣を減衰させる。たとえば、それが怠惰になされる場合のように。ゆ
 えに全ての作用が習慣を増やすとは限らない。

\\



{\scshape Respondeo dicendum} quod similes actus similes habitus causant, ut
dicitur in II {\itshape Ethic}. Similitudo autem et dissimilitudo non solum
attenditur secundum qualitatem eandem vel diversam; sed etiam secundum
eundem vel diversum participationis modum. 


&

解答する。以下のように言われるべきである。
『ニコマコス倫理学』第2巻で言われるように、類似した作用は類似した習慣
 を原因する。しかるに、類似と不類似は、同一の性質あるいは異なる性質に
 おいて見出されるだけでなく、同一のあるいは異なる分有のあり方に即して
 もまた見出される。

\\


Est enim dissimile non
solum nigrum albo, sed etiam minus album magis albo, nam etiam motus
fit a minus albo in magis album, tanquam ex opposito in oppositum, ut
dicitur in V {\itshape Physic}. 



&

たとえば黒は白と似ていないが、あまり白くないものとより白いものもまた似
 ていない。なぜなら、『自然学』第5巻で言われるように、あまり白くないも
 のからよりしろ鋳物への運動があり、それはあたかも対立する一方から他方
 への運動であるかのようだからである。

\\

Quia vero usus habituum in voluntate hominis
consistit, ut ex supradictis patet; sicut contingit quod aliquis
habens habitum non utitur illo, vel etiam agit actum contrarium; ita
etiam potest contingere quod utitur habitu secundum actum non
respondentem proportionaliter intensioni habitus. 



&

他方、前に述べられたことから明らかなとおり、習慣の使用は人間の意志に存
 するので、習慣をもつ人がその習慣を用いないとか、さらに反対の作用を行
 うことがあるように、習慣の強度に比例しない形で対応する作用に即して、
 習慣を用いるということが起こりうる。

\\

Si igitur intensio
actus proportionaliter aequetur intensioni habitus, vel etiam
superexcedat; quilibet actus vel auget habitum, vel disponit ad
augmentum ipsius; ut loquamur de augmento habituum ad similitudinem
augmenti animalis. 



&

ゆえに、もし作用の強度が習慣の強度に比例するしかたで等しい、あるいはそ
 れを凌駕するならば、どんな作用も習慣を増やしたりその増大へと態勢付け
 る。それはちょうど私たちが、習慣の増大について、動物の成長になぞらえ
 るようにである。

\\

Non enim quodlibet alimentum assumptum actu auget
animal, sicut nec quaelibet gutta cavat lapidem, sed, multiplicato
alimento, tandem fit augmentum. 


&

すなわち、どんな栄養が摂取されても現実に動物を成長させるわけではないし、
 どんな滴も石をうがつわけではない。しかし栄養が多くなると、それだけ声
 調が生じる。

\\


Ita etiam, multiplicatis actibus,
crescit habitus. Si vero intensio actus proportionaliter deficiat ab
intensione habitus, talis actus non disponit ad augmentum habitus, sed
magis ad diminutionem ipsius.

&

そのようにまた、作用が多くなると習慣が増加する。しかし、もし作用の強度
が習慣の強度に比して少ない場合には、そのような作用は習慣の増大を態勢
付けず、むしろそれの減衰へと態勢付ける。

\\



Et per hoc patet responsio ad obiecta.

&

以上のことによって異論に対する解答は明らかである。

\end{longtable}
\end{document}