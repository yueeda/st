\documentclass[paper=a4paper,fontsize=10pt,jafontsize=9pt,titlepage]{jlreq}
\usepackage{pxrubrica} %ルビ、傍点
\usepackage{longtable}
\usepackage[polutonikogreek,english,japanese]{babel}
\usepackage{latexsym}
\usepackage{color}
\usepackage{url}

%----- header -------
\usepackage{fancyhdr}
\pagestyle{fancy}
\lhead{{\itshape Summa Theologiae} I, q.17}
%--------------------


\bibliographystyle{jplain}

\title{{\bfseries PRIMA PARS}\\{\Huge Summae Theologiae}\\Sancti Thomae
Aquinatis\\{\sffamily QUEAESTIO DECIMASEPTIMA}\\DE FALSITATE}
\author{Japanese translation\\by Yoshinori {\scshape Ueeda}}
\date{Last modified \today}


%%%% コピペ用
%\rhead{a.~}
%\begin{center}
% {\Large {\bfseries}}\\
% {\large }\\
% {\footnotesize }\\
% {\Large \\}
%\end{center}
%
%\begin{longtable}{p{21em}p{21em}}
%
%&
%
%
%\\
%\end{longtable}
%\newpage



\begin{document}
\maketitle
\pagestyle{fancy}

\begin{center}
{\Large 第十七問\\虚偽について}
\end{center}

\begin{longtable}{p{21em}p{21em}}

{\huge D}{\scshape einde} quaeritur de falsitate. Et circa hoc
 quaeruntur quatuor. 

\begin{enumerate}
 \item utrum falsitas sit in rebus.
 \item utrum sit in sensu.
 \item utrum sit in intellectu.
 \item de oppositione veri et falsi.
\end{enumerate}
 
&

次に、虚偽について問われる。これをめぐって、四つのことが問われる。
\begin{enumerate}
 \item 事物の中に虚偽があるか。
 \item 感覚の中にあるか。
 \item 知性の中にあるか。
 \item 真と偽の対立について。
\end{enumerate}


\end{longtable}

\newpage

\rhead{a.~17}
\begin{center}
 {\Large {\bfseries ARTICULUS PRIMUS}}\\
 {\large UTRUM FALSITAS SIT IN REBUS}\\
 {\footnotesize I {\itshape Sent.}, d.~19, q.~5, a.~1; {\itshape De
 Verit.}, q.~1, a.~10; V {\itshape Metaphys.}, lect.~22; VI, lect.~4.}\\
 {\Large 第一項\\事物の中に虚偽があるか}
\end{center}

\begin{longtable}{p{21em}p{21em}}

{\huge A}{\scshape d primum sic proceditur}. Videtur quod falsitas non
sit in rebus. Dicit enim Augustinus, in libro {\itshape Soliloq}.,
{\itshape si verum est id quod est, falsum non esse uspiam concludetur,
quovis repugnante}.

&
第一項の問題へ、議論は以下のように進められる。
事物の中に虚偽はないと思われる。理由は以下の通り。
アウグスティヌスは、『ソリロクィア』という書物の中で、「もし、真が、在る
 ところのものであるならば、どのような反論があろうとも、いずれにせよ偽は
 存在しないと結論されることになる」と述べている。

\\


{\scshape 2 Praeterea}, falsum dicitur a fallendo. Sed
res non fallunt, ut dicit Augustinus in libro de vera Relig., {\itshape quia non
ostendunt aliud quam suam speciem}. Ergo falsum in rebus non invenitur.

&

さらに、「偽」は、「欺く(fallere)」に由来して語られる。ところが、アウグ
 スティヌスが『真の宗教について』という書物の中で言うように、事物は
 欺かない。「なぜなら、自分がもつ形象以外のものを明示しないからである」。
 ゆえに、事物の中に偽は見出されない。

\\


{\scshape 3 Praeterea}, verum dicitur in rebus per
comparationem ad intellectum divinum, ut supra dictum est. Sed quaelibet
res, inquantum est, imitatur Deum. Ergo quaelibet res vera est, absque
falsitate. Et sic nulla res est falsa.

&

さらに、前に述べられたとおり、真は、神の知性との関連で、事物の中にあると
 言われる。ところが、どんな事物も、存在するかぎり、神を模倣する。ゆえに、
 どんな事物も、虚偽から離れて、真である。この意味で、どんな事物も偽ではな
 い。

\\


{\scshape Sed contra est} quod dicit Augustinus, in libro {\itshape de
vera Relig}., quod {\itshape omne corpus est verum corpus et falsa
unitas}; quia imitatur unitatem, et non est unitas. Sed quaelibet res
imitatur divinam bonitatem, et ab ea deficit. Ergo in omnibus rebus est
falsitas.

&
しかし反対に、アウグスティヌスは、『真の宗教について』という書物の中で、
 「すべて物体は、真の物体でありかつ偽の一性である」と述べるが、その理由
 は、物体は一性を真似るが、一性でないからだという。ところが、どんな事物
 も神の善性を模倣し、それに達していない。ゆえに、どんな事物の中にも虚偽
 がある。

\\


{\scshape Respondeo dicendum} quod, cum verum et falsum
opponantur; opposita autem sunt circa idem; necesse est ut ibi prius
quaeratur falsitas, ubi primo veritas invenitur, hoc est in
intellectu. In rebus autem neque veritas neque falsitas est, nisi per
ordinem ad intellectum. Et quia unumquodque secundum id quod convenit ei
per se, simpliciter nominatur; secundum autem id quod convenit ei per
accidens, non nominatur nisi secundum quid; res quidem simpliciter falsa
dici posset per comparationem ad intellectum a quo dependet, cui
comparatur per se; in ordine autem ad alium intellectum, cui comparatur
per accidens, non posset dici falsa nisi secundum quid. 


&


解答する。以下のように言われるべきである。
真と偽は対立するが、対立するものは同一のものにかかわるから、第一に真理が
 見出されるところで、より先に虚偽が探し求められることが必要である。そし
 てそれは知性である。他方、事物の中には、知性への秩序によってでなければ、
 真理も虚偽も存在しない。そして、各々のものは、それに自体的に適合するも
 のにしたがって、端的に名付けられ、附帯的に適合するものによっては、ある
 意味においてでなければ名付けられないので、事物は、それに依存し、それに自体
 的に関係する知性との関係によって、端的に偽と言われうる。しかし、附帯的
 にそれに関係する他の知性への秩序においては、ある意味においてでなければ
 偽と言われえない。

\\



Dependent autem
ab intellectu divino res naturales, sicut ab intellectu humano res
artificiales. Dicuntur igitur res artificiales falsae simpliciter et
secundum se, inquantum deficiunt a forma artis, unde dicitur aliquis
artifex opus falsum facere, quando deficit ab operatione artis. Sic
autem in rebus dependentibus a Deo, falsitas inveniri non potest per
comparationem ad intellectum divinum, cum quidquid in rebus accidit, ex
ordinatione divini intellectus procedat, nisi forte in voluntariis
agentibus tantum, in quorum potestate est subducere se ab ordinatione
divini intellectus; in quo malum culpae consistit, secundum quod ipsa
peccata {\itshape falsitates} et {\itshape mendacia} dicuntur in Scripturis, secundum illud
Psalmi {\scshape iv}, {\itshape ut quid diligitis vanitatem et quaeritis mendacium?} Sicut per
oppositum operatio virtuosa veritas vitae nominatur, inquantum subditur
ordini divini intellectus; sicut dicitur Ioan.~{\scshape iii}, {\itshape qui facit veritatem,
venit ad lucem}. 



&


さて、自然的事物は神の知性に依存し、それはちょうど、人工物が人間の知性に
 依存するのと同じである。それゆえ、人工物が、端的に、それ自体に即して偽と
 言われるのは、技能(ars)の形相に達してないかぎりでであり、したがって、あ
 る職人は、技能の働きに達していないとき、「偽の仕事をする」(opus falsum
 facere)と言われる。しかし、このようなかたちで、神に依存する事物の中に、
 神の知性との関連によって偽が見出されることはありえない。なぜなら、なんで
 あれ事物の中に生じることは、神の知性による秩序付けに基づくからである。た
 だし、意志的行為者は例外であり、その権能の中に、神の知性による秩序付けか
 ら自らを引き離すことが属している。そして、『詩編』4 章の、かの「なぜあな
 たたちは空虚を愛し、嘘を求めるのか」\footnote{「人の子らよ、いつまでわた
 しの名誉を辱めにさらすのか、むなしさを愛し、偽りを求めるのか。」(4:3)}に
 よれば、罪そのものが「虚偽」や「嘘」といわれているかぎりで、罪という悪は、
 このことにおいて成り立つ。反対に、ちょうど、『ヨハネによる福音書』3章
 「真理を作る人が、光へとやって来る」\footnote{「真理を行う者は光の方に来
 る。(3:21)}と言われるように、神の知性による秩序に服するかぎりで、有徳な
 働きが、生の真理と言われるのも同様である。

\\


Sed per ordinem ad intellectum nostrum, ad quem comparantur res
naturales per accidens, possunt dici falsae, non simpliciter, sed
secundum quid. Et hoc dupliciter. Uno modo, secundum rationem
significati, ut dicatur illud esse falsum in rebus, quod significatur
vel repraesentatur oratione vel intellectu falso. Secundum quem modum
quaelibet res potest dici esse falsa, quantum ad id quod ei non inest,
sicut si dicamus diametrum esse falsum commensurabile, ut dicit
philosophus in V {\itshape Metaphys}.; et sicut dicit Augustinus, in
libro {\itshape Soliloq}., quod {\itshape tragoedus est falsus
Hector}. Sicut e contrario potest unumquodque dici verum, secundum id
quod competit ei. 



&

しかし、自然的事物が附帯的に関係する私たちの知性への秩序によっては、それ
 らの事物は端的に偽と言われず、ある意味において言われるにすぎない。
そして、これには二通りの仕方がある。
一つには、表示されるものの性格に基づいてであり、偽の言明や知性認識内容に
 よって表示されたり表現されたりするものが、事物において偽と言われるよう
 な場合である。この仕方では、どんな事物も、それに内在していないものにか
 んして偽と言われうる。たとえば、哲学者が『形而上学』5巻で言うように、直
 径は偽の共約可能なものである、と私たちが言う場合や、アウグスティヌスが
 『ソリロクィア』で、「悲劇役者は偽のヘクトルである」と言う場合のように。
逆に、各々のものは、それに適合するものに即して、真と言われるのも同様であ
 る。


\\



--- Alio modo, per modum causae. Et sic dicitur res esse
falsa, quae nata est facere de se opinionem falsam. Et quia innatum est
nobis per ea quae exterius apparent de rebus iudicare, eo quod nostra
cognitio a sensu ortum habet, qui primo et per se est exteriorum
accidentium; ideo ea quae in exterioribus accidentibus habent
similitudinem aliarum rerum, dicuntur esse falsa secundum illas res;
sicut fel est falsum mel, et stannum est falsum argentum. Et secundum
hoc dicit Augustinus, in libro Soliloq., quod eas res falsas nominamus,
quae verisimilia apprehendimus. Et philosophus dicit, in V {\itshape
Metaphys}., quod falsa dicuntur {\itshape quaecumque apta nata sunt
apparere aut qualia non sunt, aut quae non sunt}. 

&

もうひとつには、原因という仕方によってである。この場合、自らにつ
 いて偽の信念を作り出す本性をもつ事物が偽であると言われる。
そして、私たちは、外に現れているものによって事物について判断するように生
 まれついているので、というのも、私たちの認識は、感覚に起源をもち、感覚は
 第一に、そして自体的に、外的な附帯性についてあるからだが、それゆえ、外的
 な附帯性において他の事物の類似性を持っているものが、その事物に即して偽で
 あると言われる。たとえば、胆汁は偽のハチミツであり、鉛は偽の銀である。そ
 して、これにしたがって、アウグスティヌスは『ソリロクィア』という書物で、
 私たちが本物に似たものととらえるものを、偽の事物と名付ける、と述べている。
 また、哲学者も『形而上学』5巻で、「なんであれ、そのような性質でない、あ
 るいはそれでないようなものに見えるように生まれてついているものが」偽と
 言われる、と述べている。

\\

Et per hunc modum
etiam dicitur homo falsus, inquantum est amativus falsarum opinionum vel
locutionum. Non autem ex hoc quod potest eas confingere, quia sic etiam
sapientes et scientes falsi dicerentur, ut dicitur in V {\itshape
Metaphys}.

&
そしてこの仕方で、人間もまた、偽の信念や語りを愛する者であるかぎりで、偽
 と言われる。しかしそれは、そういったもの[偽の信念や語り]を作り出しう
 るからではない。なぜなら、『形而上学』5巻で言われるように、もしそうだと
 したら、知恵や知識のある者たちもまた、偽と言われたであろうから。

\\


{\scshape Ad primum ergo dicendum} quod res comparata ad intellectum,
secundum id quod est, dicitur vera, secundum id quod non est, dicitur
falsa. Unde {\itshape verus tragoedus est falsus Hector}, ut dicitur in
II {\itshape Soliloq}. Sicut igitur in his quae sunt, invenitur quoddam
non esse; ita in his quae sunt, invenitur quaedam ratio falsitatis.

&

第一異論に対しては、それゆえ、以下のように言われるべきである。
知性に関係づけられた事物は、それが「そうである」ことについて真と言われ、
 「そうでない」ことについて偽と言われる。
このことから、『ソリロクィア』2巻で言われるように、「真の悲劇役者は偽な
 るヘクトルである」。ゆえに、存在するものどもの中に、ある種の非存在が見
 出されるように、存在するものどもの中に、ある種の虚偽の性格が見出される。

\\

{\scshape Ad secundum dicendum} quod res per se non
fallunt, sed per accidens. Dant enim occasionem falsitatis, eo quod
similitudinem eorum gerunt, quorum non habent existentiam.

&

第二異論に対しては、以下のように言われるべきである。事物は、自体的には欺
かないが、附帯的には欺く。なぜなら、[事物は]それの存在をもっていないと
ころのものの類似を生み出す点で、虚偽を引き起こすからである。




\\


{\scshape Ad tertium dicendum} quod per comparationem ad
intellectum divinum non dicuntur res falsae, quod esset eas esse falsas
simpliciter, sed per comparationem ad intellectum nostrum, quod est eas
esse falsas secundum quid.

&

第三異論に対しては、以下のように言われるべきである。
事物が、神の知性との関係によって偽と言われることはない。もしそんなことが
 あったなら、それらの事物が端的に偽であることになったであろう。しかし、
 私たちの知性との関係によっては[偽と言われることがあり]、それは、それ
 らの事物が、ある意味で偽だということである。

\\


{\scshape Ad quartum}, quod in oppositum obiicitur,
dicendum quod similitudo vel repraesentatio deficiens non inducit
rationem falsitatis, nisi inquantum praestat occasionem falsae
opinionis. Unde non ubicumque est similitudo, dicitur res falsa, sed
ubicumque est talis similitudo, quae nata est facere opinionem falsam,
non cuicumque, sed ut in pluribus.

&

第四の、反対異論に対しては、以下のように言われるべきである。
完全でない類似や表現は、それが偽なる信念を引き起こすかぎりでなければ、虚
 偽の性格を持ち込まない。したがって、類似があるところであればどこでも、
 事物が偽と言われるわけではなく、偽なる信念を生み出す本性をもつ類似があ
 るところであればどこでも[事物が偽と言われる]。この場合でも、だれに対し
 てもそうだというのではなく、多くの人に、[偽の信念を生み出す]という意
 味である。


\end{longtable}

\newpage

\rhead{a.~2}
\begin{center}
 {\Large {\bfseries ARTICULUS SECUNDUS}}\\
 {\large IN SENSU SIT FALSITAS}\\
 {\footnotesize Infra, q.~85, a.~6; {\itshape De Verit.}, q.~1, a.~2;
 III {\itshape De Anima}, lect.~6; IV {\itshape Metaphys.}, lect.~12.}\\
 {\Large 第二項\\感覚の中に虚偽があるか}
\end{center}

\begin{longtable}{p{21em}p{21em}}

{\huge A}{\scshape d secundum sic proceditur}. Videtur quod in sensu
non sit falsitas. Dicit enim Augustinus, in libro {\itshape de Vera Relig}., {\itshape si
omnes corporis sensus ita nuntiant ut afficiuntur, quid ab eis amplius
exigere debemus, ignoro}. Et sic videtur quod ex sensibus non
fallamur. Et sic falsitas in sensu non est.


&


第二項の問題へ、議論は以下のように進められる。
感覚の中に偽はないと思われる。理由は以下の通り。
アウグスティヌスは、『真の宗教について』という書物のなかで、「もし、身体
 の全ての感覚が、感受したとおりのことを知らせるのであれば、私たちが、そ
 れ以上それらから何を要求するべきか、私にはわかりません」と述べている。
 このように、感覚に基づいて、私たちは欺かれないと思われる。このような意
 味で、感覚の中に偽はない。 

\\


{\scshape 2 Praeterea}, philosophus dicit, in IV {\itshape Metaphys}., quod
{\itshape falsitas non est propria sensui, sed phantasiae}.


&
さらに、哲学者は『形而上学』4巻で、偽は感覚に固有のものでなく、むしろ表
 象力に固有である、と述べている。

\\


{\scshape 3 Praeterea}, in incomplexis non est verum nec falsum, sed
solum in complexis. Sed componere et dividere non pertinet ad
sensum. Ergo in sensu non est falsitas.


&

さらに、真偽は、非複合的なものの中になく、ただ複合的なもの中にのみある。
 ところが、複合分割することは、感覚に属さない。ゆえに、感覚の中に虚偽は
 ない。

\\


{\scshape  Sed contra est} quod dicit Augustinus, in libro {\itshape Soliloq}.,
{\itshape apparet nos in omnibus sensibus similitudine lenocinante falli}.


&
しかし反対に、アウグスティヌスは、『ソリロクィア』という書物の中で、「私
 たちが、全ての感覚において、だます類似によって欺かれるのは明らかである」
 と述べている。

\\


{\scshape Respondeo dicendum} quod falsitas non est quaerenda in sensu,
nisi sicut ibi est veritas. Veritas autem non sic est in sensu, ut
sensus cognoscat veritatem; sed inquantum veram apprehensionem habet de
sensibilibus, ut supra dictum est. Quod quidem contingit eo quod
apprehendit res ut sunt. Unde contingit falsitatem esse in sensu, ex hoc
quod apprehendit vel iudicat res aliter quam sint. 

&

解答する。以下のように言われるべきである。
虚偽は、そこに真理があるのと同じ意味においてでなければ、感覚の中に求めら
 れるべきでない。
ところで、感覚の中に真理があるのは、感覚が心理を認識するという意味ででは
 ない。むしろ、前に述べられたとおり、可感的諸事物について、真の把握をも
 つ限りにおいてである。これは、事物をあるがままに捉えることによって生じ
 る。したがって、事物をあるがままとは違って捉えたり判断したりすることか
 ら、虚偽が感覚の中にあるということが起こる。

\\



Sic autem se habet ad
cognoscendum res, inquantum similitudo rerum est in sensu. Similitudo
autem alicuius rei est in sensu tripliciter. Uno modo, primo et per se;
sicut in visu est similitudo colorum et aliorum propriorum
sensibilium. Alio modo, per se, sed non primo; sicut in visu est
similitudo figurae vel magnitudinis, et aliorum communium
sensibilium. Tertio modo, nec primo nec per se, sed per accidens; sicut
in visu est similitudo hominis, non inquantum est homo, sed inquantum
huic colorato accidit esse hominem. 

&

ところで、事物は、事物の類似が感覚の中にあるという点で、認識に関係する。
 そして、ある事物の類似は、三通りのかたちで感覚の中にある。
一つには、第一かつ自体的にであり、たとえば、色の類似が視覚の中にある場合
 や、その他の固有に可感的なものの[類似が、それぞれの感覚の中にある場合
 である]。
もう一つには、自体的だが、第一にではないかたちであり、たとえば、視覚の中
 に、形や大きさの類似がある場合や、その他の共通に可感的なものの[類似が、
 それぞれの感覚の中にある場合である]。
第三には、第一にでも自体的にでもなく、附帯的に、というかたちであり、たと
 えば、視覚の中に、人間、といっても人間であるかぎりでの人間ではなくて、
 この、「色があるもの」に、「人間である」ということが附帯しているかぎり
 でであるが、の類似がある場合である。


\\



Et circa propria sensibilia sensus
non habet falsam cognitionem, nisi per accidens, et ut in paucioribus,
ex eo scilicet quod, propter indispositionem organi, non convenienter
recipit formam sensibilem, sicut et alia passiva, propter suam
indispositionem, deficienter recipiunt impressionem agentium. Et inde
est quod, propter corruptionem linguae, infirmis dulcia amara esse
videntur. De sensibilibus vero communibus et per accidens, potest esse
falsum iudicium etiam in sensu recte disposito, quia sensus non directe
refertur ad illa, sed per accidens, vel ex consequenti, inquantum
refertur ad alia.


&

そして、固有に可感的なものについては、感覚が偽の認識をもつことはない。た
 だし、附帯的に、比較的少数の場合に、偽の認識をもつことがあるが、それは
 つまり、感覚器官の不調のせいで、可感的形相を適切に受け取っていないため
 にである。ちょうど、他の受動的能力も、その不調のせいで、能動者の刻印を、
 ちゃんと受け取らないようにである。舌の異常のため、病人には、甘い物が苦
 く感じられるのはこのためである。
他方、共通に可感的なものや、附帯的に可感的なものについては、正常な感覚の
 中にも偽の判断がありうる。なぜなら、感覚は、それらのものに、直接的にで
 なく、他のものに関係づけられるかぎりで附帯的に、あるいは結果的に、関係
 づけられるからである。


\\


{\scshape Ad primum ergo dicendum} quod sensum affici, est ipsum eius
sentire. Unde per hoc quod sensus ita nuntiant sicut afficiuntur,
sequitur quod non decipiamur in iudicio quo iudicamus nos sentire
aliquid. Sed ex eo quod sensus aliter afficitur interdum quam res sit,
sequitur quod nuntiet nobis aliquando rem aliter quam sit. Et ex hoc
fallimur per sensum circa rem, non circa ipsum sentire.


&


第一異論に対しては、それゆえ、以下のように言われるべきである。
「感覚が感受する」ということは、感覚が感覚することそのものである。したがっ
 て、感覚が感受したとおりに知らせるということによって、「私たちは、私たち
 が何かを感覚していると判断するその判断において欺かれることがない」とい
 うことが帰結する。しかし、感覚が、時として、事物があるがままにでなく感
 受するということから、感覚が私たちに、事物をそのあるがままにでなく知ら
 せることがある、ということが帰結する。このために、私たちは、事物にかん
 して、感覚によって欺かれる。しかし、「感覚している」ということにかんし
 ては、欺かれることがない。

\\


{\scshape Ad secundum dicendum} quod falsitas dicitur non esse propria
sensui, quia non decipitur circa proprium obiectum. Unde in alia
translatione planius dicitur, quod {\itshape sensus proprii sensibilis falsus non
est}. Phantasiae autem attribuitur falsitas, quia repraesentat
similitudinem rei etiam absentis; unde quando aliquis convertitur ad
similitudinem rei tanquam ad rem ipsam, provenit ex tali apprehensione
falsitas. Unde etiam philosophus, in V {\itshape Metaphys}., dicit quod umbrae et
picturae et somnia dicuntur falsa, inquantum non subsunt res quarum
habent similitudinem.


&

第二異論に対しては、以下のように言われるべきである。
虚偽は、感覚に固有でないと言われるのは、感覚が固有対象について欺かれない
 からである。したがって、別の翻訳では、より明快に、「固有に可感的なもの
 の感覚は偽でない」と言われている。
他方、表象力に虚偽が帰せられるのは、それが存在しない事物であってもその類
 似を表現するからである。したがって、ある人が、あたかも事物それ自体に向
 かうように、事物の類似へ向かうとき、そのような把握に基づいて虚偽が生じ
 る。このことから、哲学者も『形而上学』5巻で、影、絵、夢が偽と言われるの
 は、それの類似をもっている事物がそこにないからである、と述べている。

\\


{\scshape Ad tertium dicendum} quod ratio illa procedit, quod falsitas
non sit in sensu sicut in cognoscente verum et falsum.


&

第三異論に対しては、以下のように言われるべきである。
この論は、真偽を認識するものの中にあるようなかたちで、虚偽が感覚の中にあ
 るのではない、ということを論じている。




\end{longtable}
\newpage



\rhead{a.~3}
\begin{center}
 {\Large {\bfseries ARTICULUS TERTIUS}}\\
 {\large UTRUM FALSITAS SIT IN INTELLECTU}\\
 {\footnotesize Infra, q.~85, a.~6; I {\itshape Sent.}, d.~19, a.~5,
 a.~1; I {\itshape Cont.~Gent.}, c.~59; III.~c.~108; {\itshape De
 Verit.}, q.~1, a.~12; I {\itshape Periherm.}, lect.~{\scshape iii}; III
 {\itshape de Anima}, lect.~{\scshape xi}; VI {\itshape Metaphys.},
 lect.~{\scshape iv}; IX, lect.~{\scshape xi}.}\\
 {\Large 第三項\\虚偽が知性の中にあるか}
\end{center}

\begin{longtable}{p{21em}p{21em}}

{\huge A}{\scshape d tertium sic proceditur}. Videtur quod falsitas non
sit in intellectu. Dicit enim Augustinus, in libro {\itshape Octoginta
trium Quaest}., {\itshape omnis qui fallitur, id in quo fallitur, non
intelligit}. Sed falsum dicitur esse in aliqua cognitione, secundum quod
per eam fallimur. Ergo in intellectu non est falsitas.


&

 第三項の問題へ、議論は以下のように進められる。知性の中に虚偽はないと
思われる。理由は以下の通り。アウグスティヌスは、『八十三問題』という書
物の中で、「誤る人は皆、その誤っている事柄において、知性認識していない」
と述べている。ところで、偽は、なんらかの認識によって私たちが誤る場合に、
その認識のなかに在ると言われる。ゆえに、知性の中に虚偽はない。

\\


{\scshape 2 Praeterea}, philosophus dicit, in III {\itshape de Anima},
quod {\itshape intellectus semper est rectus}. Non ergo in intellectu
est falsitas.


&

さらに、哲学者は『デ・アニマ』第3巻で「知性は常に正しい」と述べている。
 ゆえに、知性の中に虚偽はない。


\\


{\scshape Sed contra est} quod dicitur in III {\itshape de Anima},
quod {\itshape ubi compositio intellectuum est, ibi verum et falsum
est}. Sed compositio intellectuum est in intellectu. Ergo verum et
falsum est in intellectu.


&

しかし反対に、『デ・アニマ』第3巻で、「知性認識されたものどもの複合が
 あるところに、真と偽がある」と述べられている。ところで、知性認識され
 たものどもの複合は知性の中にある。ゆえに、真と偽は知性の中にある。


\\


{\scshape Respondeo dicendum} quod, sicut res habet esse
per propriam formam, ita virtus cognoscitiva habet cognoscere per
similitudinem rei cognitae. Unde, sicut res naturalis non deficit ab
esse quod sibi competit secundum suam formam, potest autem deficere ab
aliquibus accidentalibus vel consequentibus; sicut homo ab hoc quod est
habere duos pedes, non autem ab hoc quod est esse hominem, ita virtus
cognoscitiva non deficit in cognoscendo respectu illius rei cuius
similitudine informatur; potest autem deficere circa aliquid consequens
ad ipsam, vel accidens ei. Sicut est dictum quod visus non decipitur
circa sensibile proprium, sed circa sensibilia communia, quae
consequenter se habent ad illud, et circa sensibilia per accidens. 

&

解答する。以下のように言われるべきである。ちょうど、事物が、固有の形相
によって「存在する」ということをもつように、認識能力は、認識された事物
の類似によって、「認識する」ということを持つ。したがって、ちょうど自然
的事物が、自分の形相にしたがって自分に適合する存在を欠くことはないが、
しかし、なんらかの附帯性や、結果として生じるものを欠くことはありうるよ
うに、たとえば、人間が、二本の足を持つということを欠くことはありうるが、
人間であることを欠くことはないように、認識能力も、その類似によって形相
づけられるところの事物にかんして、認識することを欠くことはないが、しか
し、何かその類似の結果として生じることや、類似に附帯することにかんして
は、欠くことがありうる。それはちょうど、視覚が、固有に可感的なものにか
んして欺かれないが、それに対して結果として関係する共通的に可感的なもの
や、附帯的に可感的なものについては欺かれる、と前に言われた\footnote{前
項。}のと同様である。

\\


Sicut autem sensus informatur directe similitudine propriorum
sensibilium, ita intellectus informatur similitudine quidditatis
rei. Unde circa {\itshape quod quid est} intellectus non decipitur,
sicut neque sensus circa sensibilia propria.  In componendo vero vel
dividendo potest decipi, dum attribuit rei cuius quidditatem
intelligit, aliquid quod eam non consequitur, vel quod ei opponitur.


&


また、感覚が、固有に可感的なものの類似によって直接的に形相づけられるよ
 うに、知性は、事物の何性の類似によって形相づけられる。ここから、ちょ
 うど感覚が固有に可感的なものについて欺かれないように、知性は、何であ
 るかについて欺かれないことになる。他方、複合分割することにおいては、
 欺かれうるが、それは、何性を知性認識している事物に、それに伴わないも
 のや、それに対立するものを帰す場合にである。



\\

Sic enim se habet intellectus ad iudicandum de huiusmodi, sicut sensus
ad iudicandum de sensibilibus communibus vel per accidens. Hac tamen
differentia servata, quae supra circa veritatem dicta est, quod
falsitas in intellectu esse potest, non solum quia cognitio
intellectus falsa est, sed quia intellectus eam cognoscit, sicut et
veritatem, in sensu autem falsitas non est ut cognita, ut dictum est.


&

じっさい、知性が、これらのものについて判断することにたいする関係は、感
 覚が、共通的に、あるいは附帯的に可感的なものについて判断することにた
 いする関係と同じである。ただし、前に真理を巡って言われた次のような違
 いがある。虚偽が知性の中にありうるのは、知性の認識が偽であるからだけ
 でなく、知性がその虚偽を、ちょうど真理を認識するように、認識するから
 でもある。他方、すでに述べられたとおり、虚偽は、認識されたものとして
 感覚の中にあることはない。

\\

Quia vero falsitas intellectus per se solum circa compositionem
intellectus est, per accidens etiam in operatione intellectus qua
cognoscit {\itshape quod quid est}, potest esse falsitas, inquantum
ibi compositio intellectus admiscetur. Quod potest esse
dupliciter. Uno modo, secundum quod intellectus definitionem unius
attribuit alteri; ut si definitionem circuli attribuat homini. Unde
definitio unius rei est falsa de altera.


&

ところで、知性の虚偽は、自体的に、知性の複合を巡ってのみあるから、付帯
的に、何であるかを認識する知性の働きの中にも[虚偽が]ある。つまり、そ
こに、知性の複合が混じり込むかぎりで、虚偽がありうる。これは二通りにあ
りうる。一つには、知性が、あるものの定義を別のものに帰する場合であり、
たとえば、円の定義を人間に帰するような場合である。このことから、ある一
つの事物の定義は、別の事物について偽である。

\\



Alio modo, secundum quod partes definitionis componit ad invicem, quae
simul sociari non possunt, sic enim definitio non est solum falsa
respectu alicuius rei, sed est falsa in se. Ut si formet talem
definitionem, {\itshape animal rationale quadrupes}, falsus est
intellectus sic definiendo, propterea quod falsus est in formando hanc
compositionem, aliquod animal rationale est quadrupes. Et propter hoc,
in cognoscendo quidditates simplices non potest esse intellectus
falsus, sed vel est verus, vel totaliter nihil intelligit.


&

もう一つには、同時に結びつけられえない定義の諸部分を、相互に複合するこ
とによる場合であり、この場合、実に、定義は、たんにある事物にかんして偽
であるだけでなく、それ自身において偽である。たとえば、「四本足の理性的
動物」という定義を作るならば、そのように定義することによって、知性は偽
である。それは、何らかの理性的動物が四本足であるというこの複合を作るこ
とにおいて、知性が偽であるからである。そしてこのため、単純な何性を認識
することにおいては、知性は偽でありえず、真であるか、あるいは、何も知性
認識しないかのどちらかである。


\\


{\scshape Ad primum ergo dicendum} quod, quia quidditas rei est
proprium obiectum intellectus, propter hoc tunc proprie dicimur
aliquid intelligere, quando, reducentes illud in {\itshape quod quid
est}, sic de eo iudicamus, sicut accidit in demonstrationibus, in
quibus non est falsitas. Et hoc modo intelligitur verbum Augustini,
quod {\itshape omnis qui fallitur, non intelligit id in quo fallitur},
non autem ita, quod in nulla operatione intellectus aliquis fallatur.

&

第一異論に対しては、それゆえ、以下のように言われるべきである。事物の何
性が知性の固有対象であるのは以下の理由による。すなわち、私たちが、何か
を知性認識すると言われるのは、それを「何であるか」へ還元し、そのように
して、それについて判断するときだからである。たとえば論証において、こう
いったことが起こるのであり、その中に虚偽はない。そして、アウグスティヌ
スの「誤る人は皆、その 誤っている事柄において、知性認識していない」と
いう言葉は、このように理解されるのであり、人は知性のどんな働きにおいて
も誤らない、というように理解されるのではない。

\\


{\scshape Ad secundum dicendum} quod intellectus semper est rectus,
secundum quod intellectus est principiorum, circa quae non decipitur,
ex eadem causa qua non decipitur circa {\itshape quod quid est}. Nam
principia per se nota sunt illa quae statim, intellectis terminis,
cognoscuntur, ex eo quod praedicatum ponitur in definitione subiecti.


&

第二異論に対しては、以下のように言われるべきである。知性が常に正しいの
は、知性が諸原理にかんするものであり、それらの原理を巡って欺かれること
がないからである。それは、「何であるか」を巡って欺かれることがないのと
同じ理由による。つまり、自明な諸原理とは、用語が理解されると、述語が主
語の定義の中に置かれていることから、直ちに認識されるものなのである。

\end{longtable}
\newpage


\rhead{a.~4}
\begin{center}
 {\Large {\bfseries ARTICULUS QUARTUS}}\\
 {\large UTRUM VERUM ET FALSUM SINT CONTRARIAE}\\

 {\Large 第四項\\真と偽は反対のもの\footnote { ``Primo ostendit quod
contrarietas non est contradictio; dicens, quod cum quatuor modis
aliqua alicui opponantur: uno modo ut contradictio, sicut sedens non
sedenti; alio modo ut privatio, sicut caecus videnti: tertio modo ut
contrarietas, sicut nigrum albo: quarto modo ad aliquid, sicut filius
patri;- inter ista quatuor genera oppositionis primum est
contradictio.'' (in {\itshape X Met.}~l.6, n.~5.)  }か}
\end{center}

\begin{longtable}{p{21em}p{21em}}

{\huge A}{\scshape d quartum sic proceditur}. Videtur quod verum et
falsum non sint contraria. Verum enim et falsum opponuntur sicut quod
est et quod non est, nam verum est id quod est, ut dicit
Augustinus. Sed quod est et quod non est, non opponuntur ut
contraria. Ergo verum et falsum non sunt contraria.

&

第四項の問題へ、議論は以下のように進められる。真と偽は反対のものでない
と思われる。理由は以下の通り。アウグスティヌスが言うように、真とは存在
するものだから、真と偽は、存在するものと存在しないものが対立するように
対立する。しかし、存在するものと存在しないものとは、反対のものとして対
立するのではない。ゆえに、真と偽は反対のものでない。

\\


{\scshape 2 Praeterea}, unum contrariorum non est in alio. Sed falsum
est in vero, quia, sicut dicit Augustinus in libro {\itshape
Soliloq}., {\itshape tragoedus non esset falsus Hector, si non esset
verus tragoedus}\footnote{ ``Quo pacto enim iste quem commemoravi,
verus tragoedus esset, si nollet esse falsus Hector, falsa Andromache,
falsus Hercules, et alia innumera? aut unde vera pictura esset, si
falsus equus non esset? unde in speculo vera hominis imago, si non
falsus homo?''  ({\itshape Soliloq}.~II, c.~10.)\\「どうして、さっき私
が述べたあの人が、もし偽のヘクトルや偽のアンドロマケや偽のヘラクレスや、
その他数限りないものであろうと欲しなかったならば、真の悲劇役者であった
だろうか。あるいは、もし偽の馬でなかったならば、どうやって真の絵であっ
ただろうか。もし偽の人間でなかったなら、どうやって、鏡の中で、真の人間
の像であるだろうか。」\\条件節と帰結節が逆になっているのは、同値関係と
いう了解からか。}. Ergo verum et falsum non sunt contraria.

&

さらに、反対のものの一方が、他方の中に在るというようなことはない。とこ
 ろが、真の中に偽が在る。たとえば、アウグスティヌスは『ソリロクィア』
 という書物の中で「もし、彼が真の悲劇役者でなかったとしたら、偽のヘク
 トルでもなかったであろう」からである。ゆえに、真と偽は反対のものでな
 い。


\\

{\scshape 3 Praeterea}, in Deo non est contrarietas aliqua, nihil enim
 divinae substantiae est contrarium, ut dicit Augustinus, XII
 {\itshape de Civit.~Dei}. Sed Deo opponitur falsitas, nam idolum in
 Scriptura {\itshape mendacium} nominatur, Ierem. {\scshape viii},
 {\itshape apprehenderunt mendacium}; Glossa, {\itshape idest
 idola}. Ergo verum et falsum non sunt contraria.



&

さらに、神の中にはいかなる反対性もない。なぜなら、アウグスティヌスが
 『神の国』12巻で述べるように、神の実体に対して反対のものはなにもない
 からである。しかし、偽は神に対立する。なぜなら、聖書の中で、偶像が
 「嘘」と名付けられているからである。『エレミア書』「彼らは嘘をつかん
 だ」\footnote{「偽りに固執して」(8:5)}について『注解』は「つまり、偶
 像に」と述べている。ゆえに、真と偽は反対のものでない。\footnote{推論
 の構造がわかりにくい。表面に現れているのは以下の通り。
\begin{enumerate}
 \item 神のなかにはいかなる反対性もない。
 \item 偽は神に対立する。
 \item ゆえに、真と偽は反対のものではない。
\end{enumerate}
実際には以下のような推論か。
\begin{enumerate}
 \item 真と偽が反対のものだと仮定する。
 \item 偽は神に対立する。(聖書より)
 \item 神は真である。(第16問第5項)
 \item ゆえに、神と偽が、反対のものとして対立する。
 \item これは、「神の中にいかなる対立性もない」というアウグスティヌスの
       言葉に反する。
 \item ゆえに、1の仮定が偽。
\end{enumerate}
}

 
\\


{\scshape Sed contra est} quod dicit philosophus, in II {\itshape
Periherm}., ponit enim falsam opinionem verae contrariam.

&

しかし反対に、哲学者は『命題論』2巻で、偽なる意見を、真なる意見に反対
のものとして措定するから、と述べている。

\\


{\scshape Respondeo dicendum} quod verum et falsum opponuntur ut
 contraria, et non sicut affirmatio et negatio, ut quidam dixerunt. Ad
 cuius evidentiam, sciendum est quod negatio neque ponit aliquid,
 neque determinat sibi aliquod subiectum. Et propter hoc, potest dici
 tam de ente quam de non ente; sicut {\itshape non videns}, et
 {\itshape non sedens}. Privatio autem non ponit aliquid, sed
 determinat sibi subiectum. Est enim negatio in subiecto, ut dicitur
 IV {\itshape Metaphys}., caecum enim non dicitur nisi de eo quod est
 natum videre. Contrarium vero et aliquid ponit, et subiectum
 determinat, nigrum enim est aliqua species coloris. ---Falsum autem
 aliquid ponit. Est enim falsum, ut dicit philosophus, IV Metaphys.,
 ex eo quod dicitur vel videtur aliquid esse quod non est, vel non
 esse quod est. Sicut enim verum ponit acceptionem adaequatam rei, ita
 falsum acceptionem rei non adaequatam. Unde manifestum est quod verum
 et falsum sunt contraria.


&

解答する。以下のように言われるべきである。真と偽は、反対のものとして対
立するのであって、ある人々が言ったように、肯定と否定として対立するので
はない。これを明らかにするためには、以下のことが知られるべきである。
「否定」は何も措定せず、また、自らにとって何らの基体も限定しない。この
ために、[否定は]存在するものについてだけでなく、「見ない人」や「座っ
ていない人」のように、存在しないものについても語られうる。これに対して、
「欠如」は、何も措定しないが、自らにとっての基体を限定する。じっさい、
『形而上学』4巻で言われるとおり、[欠如とは]基体における否定である。
たとえば、「盲目」は、見る本性をもつものについてでなければ語られない。
他方、「反対」は、何かを措定し、また、基体を限定する。たとえば「黒」は、
何らかの色の種である。ところで、偽は何かを措定する。なぜなら、哲学者が
『形而上学』4巻で述べるとおり、ないものがある、あるいは、あるものがな
い、と言われたり思われたりすることから、偽が生じるからである。じっさい、
真が、事物に対等した理解を措定するように、偽は、事物に対等しない理解を
措定する。これらのことから、真と偽が反対であることは明らかである。



\\


{\scshape Ad primum ergo dicendum} quod id quod est in rebus, est
 veritas rei sed id quod est ut apprehensum, est verum intellectus, in
 quo primo est veritas. Unde et falsum est id quod non est ut
 apprehensum. Apprehendere autem esse et non esse, contrarietatem
 habet, sicut probat philosophus, in II {\itshape Periherm}., quod
 huic opinioni, {\itshape bonum est bonum}, contraria est, {\itshape
 bonum non est bonum}.


&

第一異論に対しては、以下のように言われるべきである。事物における「ある」
は、事物の真理である。しかし、認識されたものとしての「ある」は、知性の
真であり、その知性の中に、第一に、真理がある。したがって、偽も、認識さ
れたものとしての「(あるので)ない」である。ところが、「ある」を把握す
ることと「(あるので)ない」を把握することは、対立性を有する。たとえば、
哲学者は『命題論』2巻で「善は善である」という意見に反対であるのは「善
は善でない」であることを証明している。

\\


{\scshape Ad secundum dicendum} quod falsum non fundatur in vero sibi
 contrario, sicut nec malum in bono sibi contrario; sed in eo quod
 sibi subiicitur. Et hoc ideo in utroque accidit, quia verum et bonum
 communia sunt, et convertuntur cum ente, unde, sicut omnis privatio
 fundatur in subiecto quod est ens, ita omne malum fundatur in aliquo
 bono, et omne falsum in aliquo vero.


&

第二異論に対しては、以下のように言われるべきである。ちょうど悪が、自ら
に反対する善に基づくことがないように、偽も、自らに反対する真に基づくこ
とはなく、むしろ、自らに従属する真に基づく。この両者[=真・偽、善・悪]
においてこのことが起こるのは、真と善が共通的なものであり、有と置換され
るからである。したがって、ちょうど欠如が全て、有である基体に基づくよう
に、悪はすべてなんらかの善に基づき、偽はすべてなんらかの真に基づく。
\footnote{偽のヘクトルであることは、真の悲劇役者であることに基づくが、
真のヘクトルであることに基づくわけではない。}

\\


{\scshape Ad tertium dicendum} quod, quia contraria et opposita
 privative nata sunt fieri circa idem, ideo Deo, prout in se
 consideratur, non est aliquid contrarium, neque ratione suae
 bonitatis, neque ratione suae veritatis, quia in intellectu eius non
 potest esse falsitas aliqua. Sed in apprehensione nostra habet
 aliquid contrarium, nam verae opinioni de ipso contrariatur falsa
 opinio. Et sic idola mendacia dicuntur opposita veritati divinae,
 inquantum falsa opinio de idolis contrariatur verae opinioni de
 unitate Dei.

&

第三異論に対しては、以下のように言われるべきである。反対するものども
\footnote{たとえば、白と黒}や、欠如というかたちで対立するものども
\footnote{たとえば、目が見える人と目が見えない人}は、本性上、同一のも
のをめぐって生じる。ゆえに、神がそれ自体において考えられる場合、その神
に、その善性という観点からも、真理という観点からも、何も反対のものはな
い。なぜなら、神の知性の中には、どんな虚偽も存在しえないからである。し
かし、私たちの把握において、[神は]なんらか反対するものをもつ。なぜな
ら、神についての真の意見が、偽の意見に反対するからである。このようにし
て、偶像についての偽の意見が、神の一性についての真の意見に反対であると
いう意味で、偶像が、神の真理に対立するものとして、嘘と言われている。

\end{longtable}
\end{document}