\documentclass[10pt]{jsarticle} % use larger type; default would be 10pt
%\usepackage[utf8]{inputenc} % set input encoding (not needed with XeLaTeX)
%\usepackage[round,comma,authoryear]{natbib}
%\usepackage{nruby}
\usepackage{okumacro}
\usepackage{longtable}
%\usepqckage{tablefootnote}
\usepackage[polutonikogreek,english,japanese]{babel}
%\usepackage{amsmath}
\usepackage{latexsym}
\usepackage{color}

%----- header -------
\usepackage{fancyhdr}
\pagestyle{fancy}
\lhead{{\it Summa Theologiae} I, q.~27}
%--------------------


\title{{\bf PRIMA PARS}\\{\HUGE Summae Theologiae}\\Sancti Thomae
Aquinatis\\{\sffamily QUEAESTIO VIGESIMASEPTIMA}\\DE PROCESSIONE
DIVINARUM PERSONARUM}
\author{Japanese translation\\by Yoshinori {\sc Ueeda}}
\date{Last modified \today}

%%%% コピペ用
%\rhead{a.~}
%\begin{center}
% {\Large {\bf }}\\
% {\large }\\
% {\footnotesize }\\
% {\Large \\}
%\end{center}
%
%\begin{longtable}{p{21em}p{21em}}
%
%&
%
%\\
%\end{longtable}
%\newpage



\begin{document}

\maketitle

\begin{center}
{\Large 第二十七問\\神のペルソナの発出について}
\end{center}


\begin{longtable}{p{21em}p{21em}}
{\Huge C}onsideratis autem his quae ad divinae essentiae unitatem pertinent,
 restat considerare de his quae pertinent ad Trinitatem personarum in
 divinis. Et quia personae divinae secundum relationes originis
 distinguuntur, secundum ordinem doctrinae prius considerandum est de
 origine, sive de processione, secundo, de relationibus originis;
 tertio, de personis. Circa processionem quaeruntur quinque. 

\begin{enumerate}
 \item utrum processio sit in divinis.
 \item utrum aliqua processio in divinis generatio dici possit.
 \item utrum praeter generationem aliqua alia processio possit esse in
       divinis.
 \item utrum illa alia processio possit dici generatio.
 \item utrum in divinis sint plures processiones quam duae.
\end{enumerate}

&

さて、神の本質の一性に属することが考察されたので、神におけるペルソナの
 三性\footnote{trinitasは日本語で「三位一体」と訳される慣例だが、本来の意味は「三であ
 ること」「三性」でありトマスの論述もそれでなければ意味をなさないとこ
 ろが多いので、とくに宗教的文脈でない部分では「三性」と訳す。}に属することについて考察することが残されている。
そして、神のペルソナは、起源の関係に即して区別されるので、教授の順番とし
 ては、先に起源ないし発出について、次に起源の関係について、第三にペルソ
 ナについて、考察されるべきである。発出については、五つのことが問われる。

\begin{enumerate}
 \item 神の中に\footnote{「中に」とはもちろん場所的な意味ではない。
       「おいて」と訳しても場所的な意味に解される怖れはあるので、あえ
       て「中に」と訳す。その場所的な含意が逆に場所的な意味でないこと
       を強調する効果に期待して。}発出があるか。
 \item 神の中にある発出が生成\footnote{三位一体論の文脈において、
       generatioは「産出」などと訳されるべきかもしれないが、自然学の文脈
       でgeneratioは「生成」と訳されるのが通常なので、それとの接続を意識
       して、あえて「生成」と訳すことにする。}と言われうるか。
 \item 神の中に、ある生成の他に何か別の発出がありうるか。
 \item この別の発出は生成と言われうるか。
 \item 神の中に二つ以上の発出があるか。
\end{enumerate}



\end{longtable}

\newpage


\rhead{a.~1}
\begin{center}
 {\Large {\bf ARTICULUS PRIMUS}}\\
 {\large UTRUM PROCESSIO SIT IN DIVINIS}\\
 {\footnotesize I {\itshape Sent.}, d.13, a.1; IV {\itshape SCG},
 cap.11; {\itshape De Pot.}, q.10, a.1.}\\
 {\Large 第一項\\神の中に発出があるか}
\end{center}

\begin{longtable}{p{21em}p{21em}}


{\Huge A}{\scshape d primum sic proceditur}. Videtur quod in Deo non possit esse aliqua
processio. Processio enim significat motum ad extra. Sed in divinis
nihil est mobile, neque extraneum. Ergo neque processio.


&

第一項の問題へ、議論は以下のように進められる。
神の中に発出はないとおもわれる。理由は以下の通り。
発出は外への運動を意味する。しかし神の中に動かされうるものも外的なも
 のもない。ゆえに発出もない。

\\



2 {\scshape Praeterea}, omne procedens est diversum ab eo a quo procedit. Sed in Deo
non est aliqua diversitas, sed summa simplicitas. Ergo in Deo non est
processio aliqua.


&

さらに、すべて発出するものは、そこから発出するもとのものとは異なる。しか
 し、神の中にはどんな異なりもなく最高の単純性がある。ゆえに神の中に
 はどんな発出もない。

\\



3 {\scshape Praeterea}, procedere ab alio videtur rationi primi principii
repugnare. Sed Deus est primum principium, ut supra ostensum est. Ergo
in Deo processio locum non habet.


&

さらに、他のものから発出することは、第一根源という性格に反すると思われる。
 ところが前に示されたとおり、神は第一根源である。ゆえに神の中に発出
 の場所はない。

\\


{\scshape Sed contra est} quod dicit dominus, {\itshape Ioan}.~{\scshape viii}, {\itshape ego ex Deo processi}.


&

しかし反対に『ヨハネによる福音書』第8章で、主が「私は神から発出した」
 \footnote{「イエスは言われた。「神があなたがたの父であれば、あなたが
 たは私を愛するはずである。なぜなら、私は神のもとから来て、ここにいる
 からだ。私は勝手に来たのではなく、神が私をお遣わしになったのである。」
 (8:42) ``dixit ergo eis Iesus si Deus pater vester esset diligeretis
 utique me ego enim ex Deo processi et veni neque enim a me ipso veni
 sed ille me misit.''} と言っている。


\\



{\scshape Respondeo dicendum} quod divina Scriptura, in rebus divinis, nominibus ad
processionem pertinentibus utitur. Hanc autem processionem diversi
diversimode acceperunt. Quidam enim acceperunt hanc processionem
secundum quod effectus procedit a causa. Et sic accepit Arius, dicens
filium procedere a patre sicut primam eius creaturam, et spiritum
sanctum procedere a patre et filio sicut creaturam utriusque. Et
secundum hoc, neque filius neque spiritus sanctus esset verus Deus. Quod
est contra id quod dicitur de filio, I {\itshape Ioan}. ult., {\itshape ut simus in vero
filio eius, hic est verus Deus}. Et de spiritu sancto dicitur, I {\itshape Cor}.~{\scshape vi},
{\itshape nescitis quia membra vestra templum sunt spiritus sancti?} Templum autem
habere solius Dei est. 



&

解答する。以下のように言われるべきである。
神の書物は、神的な事柄において発出に属する言葉を用いている。しかしこ
 の発出を、人々はさまざまに理解した。ある人々は、この発出を結果が原因
 から出てくることに即して理解した。こう理解したのはアリウスである。彼は、
 息子\footnote{filiusは、三位一体論の文脈で、「御子」「子」と訳されるが、
 あえて直訳で「息子」と訳す。}は父から神の被造物の第一のものとして発出し、聖霊は父と息子から両
 者の被造物として発出すると言っている。そしてこれによれば、息子も聖霊も
 どちらも真の神でないことになる。これは、『ヨハネの手紙一』最終章「私
 たちは彼の真の息子のなかにいよう。彼は真の神である」に反し、『コリントの
 信徒への手紙一』第6章「あなたたちの四肢は聖霊の寺院であることをあなたた
 ちは知らない」にも反する。「寺院」を持つことは神にのみ属するからであ
 る。


\\


Alii vero hanc processionem acceperunt secundum
quod causa dicitur procedere in effectum, inquantum vel movet ipsum, vel
similitudinem suam ipsi imprimit. Et sic accepit Sabellius, dicens ipsum
Deum patrem filium dici, secundum quod carnem assumpsit ex virgine. Et
eundem dicit spiritum sanctum, secundum quod creaturam rationalem
sanctificat, et ad vitam movet. Huic autem acceptioni repugnant verba
domini de se dicentis, {\itshape Ioan}.~{\scshape v}, {\itshape non potest facere a se filius quidquam};
et multa alia, per quae ostenditur quod non est ipse pater qui
filius. 



&

一方、他の人々は、この発出を自らが動く、あるいは自らの類似を結果へ刻
 印する限りにおいて、原因が結果へ出ていくことに即して理解した。こう理解
 したのはサベリウスである。彼は、父なる神が息子と言われるのは、処女から肉
 を取ったからだと言う。また神を聖霊と言うのは、理性的被造物を聖なるも
 のにし生命へ動かすからだと言う。しかしこの理解には『ヨハネによる福
 音書』第5章で、自分について語る主の言葉、「息子は、自分だけでは何
 も為すことができない」や、息子である者が父自身ではないことを示すその他
 多くの言葉がそれに反する。


\\


Si quis autem diligenter consideret, uterque accepit
processionem secundum quod est ad aliquid extra, unde neuter posuit
processionem in ipso Deo. Sed, cum omnis processio sit secundum aliquam
actionem, sicut secundum actionem quae tendit in exteriorem materiam,
est aliqua processio ad extra; ita secundum actionem quae manet in ipso
agente, attenditur processio quaedam ad intra. 

&


しかしもし熱心に考察するならば、どちらも発出を何か外へのものとして理解
 したのであり、したがってどちらも神自身における発出ではありえなかった。
むしろ全ての発出は何らかの作用に即してあるので、外の質料へ向かう作用に
 即して外への発出があるように、作用者自身の中に留まる作用に即しては内へ
 の発出が見出される。


\\

Et hoc maxime patet in
intellectu, cuius actio, scilicet intelligere, manet in
intelligente. Quicumque enim intelligit, ex hoc ipso quod intelligit,
procedit aliquid intra ipsum, quod est conceptio rei intellectae, ex vi
intellectiva proveniens, et ex eius notitia procedens. Quam quidem
conceptionem vox significat, et dicitur verbum cordis, significatum
verbo vocis. 


&


そしてこれは、その作用つまり知性認識が知性認識する者の中に留まる知性に
 おいてもっとも明らかである。つまり誰であれ知性認識する者は、知性認
 識することに基づいて、それ自身の中へ何かを発出させる。それは知性的能
 力から出てきて、その知から発出する知性認識された事物の懐念(conceptio)
 である。実にこの懐念を音声は表示し、そしてそれは心の言葉と呼ばれ
またそれが音声の言葉によって表示される。


\\


Cum autem Deus sit super omnia, ea quae in Deo dicuntur,
non sunt intelligenda secundum modum infimarum creaturarum, quae sunt
corpora; sed secundum similitudinem supremarum creaturarum, quae sunt
intellectuales substantiae; a quibus etiam similitudo accepta deficit a
repraesentatione divinorum. 



&

しかし、神は万物の上にあるので、神において語られることは、最下位の被造物
つまり物体のあり方で理解されるべきではなく、最上位の被造物すなわち知
 性的実体のあり方で理解されるべきである。ただしそこから受け取られた類似であっ
 ても、神にかんすることの表現からは欠けているのだが。

\\



Non ergo accipienda est processio secundum
quod est in corporalibus, vel per motum localem, vel per actionem
alicuius causae in exteriorem effectum, ut calor a calefaciente in
calefactum; sed secundum emanationem intelligibilem, utpote verbi
intelligibilis a dicente, quod manet in ipso. Et sic fides Catholica
processionem ponit in divinis.


&

ゆえに発出は、物体的なものの中にある限りにおいてや、場所的運動によって
 や、熱が熱するものから熱せられたものへ出ていくように、ある原因の外
 の結果への作用によって理解されるのではなく、可知的流出、すなわち語る
 者から可知的な言葉が流出し、それが語る者自身の中に留まるかたちで理解さ
 れるべきである。この意味で、カトリックの信仰は発出を神の中に措定する。


\\



{\scshape Ad primum ergo dicendum} quod obiectio illa procedit de processione quae
est motus localis, vel quae est secundum actionem tendentem in
exteriorem materiam, vel in exteriorem effectum, talis autem processio
non est in divinis, ut dictum est.


&

第一異論に対しては、それゆえ以下のように言われるべきである。
かの異論は、場所的運動や外の質料や外の結果へ向かう作用に即して論じられ
 ている。しかし、述べられたとおり、そのような発出は神の中にない。

\\



{\scshape Ad secundum dicendum} quod id quod procedit secundum processionem quae
est ad extra, oportet esse diversum ab eo a quo procedit. Sed id quod
procedit ad intra processu intelligibili, non oportet esse diversum,
imo, quanto perfectius procedit, tanto magis est unum cum eo a quo
procedit. Manifestum est enim quod quanto aliquid magis intelligitur,
tanto conceptio intellectualis est magis intima intelligenti, et magis
unum, nam intellectus secundum hoc quod actu intelligit, secundum hoc
fit unum cum intellecto. Unde, cum divinum intelligere sit in fine
perfectionis, ut supra dictum est, necesse est quod verbum divinum sit
perfecte unum cum eo a quo procedit, absque omni diversitate.


&

第二異論に対しては以下のように言われるべきである。
外へ向かう発出に即して発出するものは、そこから発出したものと異なっていな
 ければならない。しかし可知的発出によって内へと発出するものは、異なっ
 ている必要はなく、むしろより完全に発出するほどそこから発出するもの
 と一つである。なぜなら、あるものがよりよく知性認識されるほど、知的な概
 念は知性認識する者にとって内的になり、そしてより一つとなるからである。
 すなわち、知性は知性認識されたものと一つになることに従って、現実に知
 性認識する。したがって神の知性認識は、前に述べられたとおり、完全性の
 終端にあるから、神の言葉は完全にあらゆる差異もなく、そこから発出したところのものと一つ
 であることが必然である。


\\



{\scshape Ad tertium dicendum} quod procedere a principio ut extraneum et diversum,
repugnat rationi primi principii, sed procedere ut intimum et absque
diversitate, per modum intelligibilem, includitur in ratione primi
principii. Cum enim dicimus aedificatorem principium domus, in ratione
huius principii includitur conceptio suae artis, et includeretur in
ratione primi principii, si aedificator esset primum principium. Deus
autem, qui est primum principium rerum, comparatur ad res creatas ut
artifex ad artificiata.


&

第三異論に対しては、以下のように言われるべきである。
外的で異なるものとして根源から発出することは、第一根源の性格に反するが、
 もっとも内的で差異がなく、可知的なかたちで発出することは第一根源の性
 格に含まれる。じっさい私たちが、棟梁が家の根源であると言うとき、この
 根源の性格には、彼の技術の概念が含まれるし、またもし棟梁が第一根源で
 あったならば、第一根源の概念の中に含まれたであろう。しかし神は諸事物
 の第一根源なので、技術者が技術作品に関係するように被造の事物に関係す
 る。


\end{longtable}
\newpage




\rhead{a.~2}
\begin{center}
 {\Large {\bf ARTICULUS SECUNDUS}}\\
 {\large UTRUM ALIQUA PROCESSIO IN DIVINIS GENERATIO DICI POSSIT}\\
 {\footnotesize IV {\itshape SCG}, cap.10, 11; {\itshape De Pot.}, q.2,
 a.1; Opusc.~II, {\itshape Contra Graecos, Armenos}, etc., cap.3;
 {\itshape Compend.~Theol.}, cap.11, 18; {\itshape ad Coloss.}, cap.1, lect.4.}\\
 {\Large 第二項\\何らかの発出が生成と呼ばれうるか}
\end{center}

\begin{longtable}{p{21em}p{21em}}



{\scshape Ad secundum sic proceditur}. Videtur quod processio quae est in divinis,
non possit dici generatio. Generatio enim est mutatio de non esse in
esse, corruptioni opposita; et utriusque subiectum est materia. Sed
nihil horum competit divinis. Ergo non potest generatio esse in divinis.


&

第二項の問題へ、議論は以下のように進められる。
神の中にある発出は、生成と呼ばれえないと思われる。理由は以下の通り。
生成は、非存在から存在への変化であり、消滅に対置される。
そして、どちらの基体も質料である。しかし、これらのどれも神に適合しない。
ゆえに、生成は神の中にありえない。


\\



2. {\scshape Praeterea}, in Deo est processio secundum modum intelligibilem, ut dictum
est. Sed in nobis talis processio non dicitur generatio. Ergo neque in
Deo.


&

さらに、すでに述べられたとおり、神の中に、発出は可知的なありかたに即して
 ある。しかし、私たちの中で、そのような発出は生成と言われない。ゆえに、
 神においても言われない。

\\



3. {\scshape Praeterea}, omne genitum accipit esse a generante. Esse ergo cuiuslibet
geniti est esse receptum. Sed nullum esse receptum est per se
subsistens. Cum igitur esse divinum sit esse per se subsistens, ut supra
probatum est, sequitur quod nullius geniti esse sit esse divinum. Non
est ergo generatio in divinis.


&


さらに、全て生まれたものは、生むものから存在を受け取る。ゆえに、すべて生まれた
 ものの存在は、受け取られた存在である。しかし、受け取られた存在は、どれ
 も、それ自体で自存しない。ゆえに、前に証明されたとおり、神の存在はそれ
 自体で自存するので、どんな生まれたものの存在も神の存在でないことが帰結
 する。ゆえに、神の中に生成はない。

\\



{\scshape Sed contra est} quod dicitur in Psalmo {\scshape ii}, {\itshape ego hodie genui te}.


&

しかし反対に、『詩編』第2章「私は今日あなたを生んだ」\footnote{「主の定められたところに従ってわたしは述べよう。主はわたしに告げられた。「お前はわたしの子/今日、わたしはお前を生んだ。 」(2:7)}と言われている。

\\



{\scshape Respondeo dicendum} quod processio verbi in divinis dicitur generatio. Ad
cuius evidentiam, sciendum est quod nomine generationis dupliciter
utimur. Uno modo, communiter ad omnia generabilia et corruptibilia, et
sic generatio nihil aliud est quam mutatio de non esse ad esse. Alio
modo, proprie in viventibus, et sic generatio significat originem
alicuius viventis a principio vivente coniuncto. Et haec proprie dicitur
{\itshape nativitas}. 


&

解答する。以下のように言われるべきである。
言葉の発出は、神の中で生成と言われうる。これを明らかにするために、以下の
 ことが知られるべきである。私たちは、生成という言葉を、二つの意味で用い
 る。一つには、全ての生成消滅しうるものに共通的に用い、この場合、生成は、
 非存在から存在への変化に他ならない。もう一つには、生きているものにおい
 て固有に用い、この場合、生成は、ある生きるものの、それに結びついた生きる
 根源からの起源を意味する。そしてこの生成は、厳密には「誕生」(nativitas)と呼ばれる。



\\


Non tamen omne huiusmodi dicitur genitum, sed proprie quod
procedit secundum rationem similitudinis. Unde pilus vel capillus non
habet rationem geniti et filii, sed solum quod procedit secundum
rationem similitudinis, non cuiuscumque, nam vermes qui generantur in
animalibus, non habent rationem generationis et filiationis, licet sit
similitudo secundum genus, sed requiritur ad rationem talis
generationis, quod procedat secundum rationem similitudinis in natura
eiusdem speciei, sicut homo procedit ab homine, et equus ab equo. 



&

しかし、このようなものすべてが、生まれたものとは言われず、固有には、類似
 性を伴って発出するものが、そう言われる。したがって、髪や毛髪は、生まれ
 たものや息子という性格を持たない。しかし、類似性を伴って発出するものだけ
とは言っても、その全てが生まれたものと言われるわけではない。たとえば動物
 の中で生成されるウジは、類における類似があるにもかかわらず、生成や出生
 (filiatio)の性格を持たない。そのような生成の性格には、同一の種の本性に
 おける類似性に即して発出することが必要とされる。たとえば、人間が人間か
 ら発出し、馬が馬から発出するように。



\\


In
viventibus autem quae de potentia in actum vitae procedunt, sicut sunt
homines et animalia, generatio utramque generationem includit. Si autem
sit aliquod vivens cuius vita non exeat de potentia in actum, processio,
si qua in tali vivente invenitur, excludit omnino primam rationem
generationis; sed potest habere rationem generationis quae est propria
viventium. 



&

ところで、可能態から現実態へと出ていく生きるものにおいて、人間と動物のよう
 に、生成は、このどちらの生成も含む。しかし、その生命が可能態から現実態
 へ出ていかないような、何らかの生きるものが存在するならば、そして、その
 ような生きるものの中に、発出が見出されるならば、その発出は、生成の第一
 の性格を全面的に排除するが、しかし、生きるものに固有である生成の性格を
 持つことはできる。

\\



Sic igitur processio verbi in divinis habet rationem
generationis. Procedit enim per modum intelligibilis actionis, quae est
operatio vitae, et a principio coniuncto, ut supra iam dictum est, et
secundum rationem similitudinis, quia conceptio intellectus est
similitudo rei intellectae, et in eadem natura existens, quia in Deo
idem est intelligere et esse, ut supra ostensum est. Unde processio
verbi in divinis dicitur generatio, et ipsum verbum procedens dicitur
filius.


&

ゆえに、神における言葉の発出は、以下のように生成の性格を持つ。
すなわち、それは可知的作用のしかたで発出するが、その作用は生命の働
 きである。
そして、すでに述べられたとおり、結びついた根源からの発出である。
また、知性の懐念は、知性認識された事物の類似なので、類似の性格に即する。
前に示されたとおり、神において知性認識の働きと存在は同一なので、それは同一の本性においてある。
したがって、神における言葉の発出は、生成と呼ばれ、発出する言葉それ自体は
 息子と言われる。


\\



{\scshape Ad primum ergo dicendum} quod obiectio illa procedit de generatione
secundum rationem primam, prout importat exitum de potentia in actum. Et
sic non invenitur in divinis, ut supra dictum est.


&

第一異論に対しては、それゆえ、以下のように言われるべきである。
この反論は、第一の意味、つまり、可能態から現実へ出ていくことを含意する生
 成について論じている。この意味の生成は、前に述べられたとおり、見出され
 ない。

\\



{\scshape Ad secundum dicendum} quod intelligere in nobis non est ipsa substantia
intellectus, unde verbum quod secundum intelligibilem operationem
procedit in nobis, non est eiusdem naturae cum eo a quo procedit. Unde
non proprie et complete competit sibi ratio generationis. Sed
intelligere divinum est ipsa substantia intelligentis, ut supra ostensum
est, unde verbum procedens procedit ut eiusdem naturae subsistens. 

&

第二異論に対しては、以下のように言われるべきである。
私たちの中に、知性認識することは、知性の実体でない。それで、私たちの中
 で、可知的な働きに即して発出する言葉は、そこから発出するところのものと、
 同一の本性を持たない。したがって、厳密な意味で、完全に、生成という性格
 が適合するわけではない。しかし、前に示されたとおり、神の知性認識は、知
 性認識する者の実体そのものなので、発出する言葉は、同一の本性を持って自
 存するものとして、発出する。

\\

Et
propter hoc proprie dicitur genitum et filius. Unde et his quae
pertinent ad generationem viventium, utitur Scriptura ad significandam
processionem divinae sapientiae, scilicet {\itshape conceptione} et {\itshape partu}, dicitur
enim ex persona divinae sapientiae, {\itshape Proverb}.~{\scshape viii}, {\itshape nondum erant abyssi,
et ego iam concepta eram; ante colles ego parturiebar}. Sed in intellectu
nostro utimur nomine conceptionis, secundum quod in verbo nostri
intellectus invenitur similitudo rei intellectae, licet non inveniatur
naturae identitas.


&

そして、このために、厳密な意味で、それは「生まれたもの」「息子」と言われる。
このことから、聖書も、神の知恵の発出を表示するために、生物の生成に属する
 事柄、すなわち、「懐胎(懐念)」や「出産」を用いている。たとえば、神の
 知恵に基づいて、『箴
 言』第8章では、「まだ深淵がなかったとき、私はすでに懐胎され、丘ができる前に、
 私は産み出された」\footnote{「わたしは生み出されていた/深淵も水のみな
 ぎる源も、まだ存在しないとき。/わたしは生み出されていた/深淵も水のみ
 なぎる源も、まだ存在しないとき。 」(8:24-25)}と言われている。しかし、私たちの知性の中
 で、私たちの知性の言葉において、知性認識された事物の類似が見出されるか
 ぎりで、「懐胎(懐念)」という言葉を用いる。ただし、そこに本性の同一性
 は見出されない。


\\



{\scshape Ad tertium dicendum} quod non omne acceptum est receptum in aliquo
subiecto, alioquin non posset dici quod tota substantia rei creatae sit
accepta a Deo, cum totius substantiae non sit aliquod subiectum
receptivum. Sic igitur id quod est genitum in divinis, accipit esse a
generante, non tanquam illud esse sit receptum in aliqua materia vel
subiecto (quod repugnat subsistentiae divini esse); sed secundum hoc
dicitur esse acceptum, inquantum procedens ab alio habet esse divinum,
non quasi aliud ab esse divino existens. In ipsa enim perfectione divini
esse continetur et verbum intelligibiliter procedens, et principium
verbi; sicut et quaecumque ad eius perfectionem pertinent, ut supra
dictum est.


&

第三異論に対しては、以下のように言われるべきである。
必ずしも全ての「受け取られたもの」が、何らかの基体において受け取られてい
 るとは限らない。さもなければ、「被造の事物の全実体は神から受け取られた」
 と言えないことになる。実体全体を受け取りうるような基体は存在しないのだ
 から。それゆえ、この意味で、神において生まれたものは、生むものから存在
 を受け取るが、しかし、何らかの質料や基体に受け取られたものとしての存在
 ではない(これは神の存在の自存性に反する)。そうではなく、この存在が受け取ら
 れたものと言われるのは、他から発出するものが神の存在を持つかぎりにおい
 てであり、神の存在とは別のものとして存在するかぎりにおいてではない。
なぜなら、ちょうど、前に述べられたとおり、何であれ神の完全性に属するもの
 がそうであるように、神の存在の完全性自体の中に、可知的なかたちで発出す
 る言葉と、その言葉の根源は、含まれるからである。

\\


\end{longtable}
\newpage



\rhead{a.~3}
\begin{center}
 {\Large {\bf ARTICULUS TERTIUS}}\\
 {\large UTRUM SIT IN DIVINIS ALIA PROCESSIO A GENERATIONE VERBI}\\
 {\footnotesize I {\itshape Sent.}, d.13, a.2; IV {\itshape SCG},
 cap.19; {\itshape De Pot.}, q.10, a.1, 2; Opusc.~II, {\itshape Contra
 Graecos, Armenos}, etc., cap.3.}\\
 {\Large 第三項\\神の中に、言葉の生成以外の他の生成があるか}
\end{center}

\begin{longtable}{p{21em}p{21em}}



{\scshape Ad tertium sic proceditur}. Videtur quod non sit in divinis alia
processio a generatione verbi. Eadem enim ratione erit aliqua alia
processio ab illa alia processione, et sic procederetur in infinitum,
quod est inconveniens. Standum est igitur in primo, ut sit una tantum
processio in divinis.


&

第三項の問題へ、議論は以下のように進められる。
神の中に、言葉の生成以外の発出はないと思われる。
理由は以下の通り。
ある理由$R$で、言葉の生成以外の発出$A$があるとすると、その同じ理由$R$で、$A$とは別の発
 出Bがあることになる。これは無限に進むであろう。しかしそ
 れは不都合である。ゆえに、第一のところで止まるべきであり、その結果、神
 の中にはただ一つの発出があることになる。


\\



2. {\scshape Praeterea}, in omni natura invenitur tantum unus modus communicationis
illius naturae, et hoc ideo est, quia operationes secundum terminos
habent unitatem et diversitatem. Sed processio in divinis non est nisi
secundum communicationem divinae naturae. Cum igitur sit una tantum
natura divina, ut supra ostensum est, relinquitur quod una sit tantum
processio in divinis.


&

さらに、あらゆる本性において、その本性を伝達するしかたはただ一つだけ見出
 される。その理由は、働きが、その終極に即して一性と差異性を持つからであ
 る。しかし、神の中に発出があるのは、神性の伝達に即してに他ならない。
ゆえに、前に示されたとおり、神の本性は一つだけだから、神における発出は一
 つだけだということになる。

\\



3. {\scshape Praeterea}, si sit in divinis alia processio ab intelligibili processione
verbi, non erit nisi processio amoris, quae est secundum voluntatis
operationem. Sed talis processio non potest esse alia a processione
intellectus intelligibili, quia voluntas in Deo non est aliud ab
intellectu, ut supra ostensum est. Ergo in Deo non est alia processio
praeter processionem verbi.


&

さらに、もし神の中に、言葉の可知的な発出以外の発出があるならば、それは、意志の働きに即してある、愛
 の発出以外にはないだろう。しかし、そのような発出は、知性の可知的な発出
 と別ではありえない。なぜなら、前に示されたとおり、神の中に、意志は知性と別のものではない
 からである。ゆえに、神の中に、言葉の発出以外に他の発出はない。

\\



{\scshape Sed contra est} quod spiritus sanctus procedit a patre, ut dicitur
{\itshape Ioan}.~{\scshape xv}. Ipse autem est alius a filio, secundum illud Ioan.~{\scshape xiv}: {\itshape Rogabo
patrem meum, et alium Paracletum dabit vobis}. Ergo in divinis est alia
processio praeter processionem verbi.


&

しかし反対に、『ヨハネによる福音書』15章\footnote{「わたしが父のもとから
 あなたがたに遣わそうとしている弁護者、すなわち、父のもとから出る真理の
 霊が来るとき、その方がわたしについて証しをなさるはずである。 」(15:26)}
 で言われるように、聖霊は父から発出する。しかし、『ヨハネによる福音書』
 14章「私は、他の弁護者をあなたたちに与えるように、私の父に頼もう」
 \footnote{わたしは父にお願いしよう。父は別の弁護者を遣わして、永遠にあ
 なたがたと一緒にいるようにしてくださる。」(14:16) }によれば、これは息子
 と違う。ゆえに、神の中に、言葉の発出以外の発出がある。

\\



{\scshape Respondeo dicendum quod} in divinis sunt duae processiones, scilicet
processio verbi, et quaedam alia. Ad cuius evidentiam, considerandum est
quod in divinis non est processio nisi secundum actionem quae non tendit
in aliquid extrinsecum, sed manet in ipso agente. Huiusmodi autem actio
in intellectuali natura est actio intellectus et actio
voluntatis. Processio autem verbi attenditur secundum actionem
intelligibilem. Secundum autem operationem voluntatis invenitur in nobis
quaedam alia processio, scilicet processio amoris, secundum quam amatum
est in amante, sicut per conceptionem verbi res dicta vel intellecta,
est in intelligente. Unde et praeter processionem verbi, ponitur alia
processio in divinis, quae est processio amoris.


&


解答する。以下のように言われるべきである。
神の中には、二つの発出、すなわち、言葉の発出と、もう一つ別の発出がある。
これを明らかにするために、以下のことが考察されるべきである。
神の中に、発出は、何か外のものへ出ていくのではなくて、作用者の中に留まる
 作用に即してのみある。しかし、知性的本性において、そのような作
 用は、知性の作用と意志の作用である。
知性の作用に即して、言葉の発出が見出される。意志の作用に即しては、私たち
 の中で、何か別の発出、すなわち、愛の発出が見出される。
愛に即して、愛されるものが愛する者の中にある。ちょうど、言葉の懐胎によっ
 て、語られた、あるいは知性認識された事物が、知性認識する者の中にあるよ
 うに。したがって、言葉の発出の他に、他の発出、すなわち愛の発出が、神の
 中に認められる。


\\



Ad primum ergo dicendum quod non est necessarium procedere in divinis
processionibus in infinitum. Processio enim quae est ad intra in
intellectuali natura, terminatur in processione voluntatis.


&

第一異論に対しては、それゆえ、以下のように言われるべきである。
神の中で、発出が無限に進む必要はない。なぜなら、知性的本性において、内へ
 の発出は、意志の発出で止まるからである。


\\



{\scshape Ad secundum dicendum} quod quidquid est in Deo, est Deus, ut supra
ostensum est, quod non contingit in aliis rebus. Et ideo per quamlibet
processionem quae non est ad extra, communicatur divina natura, non
autem aliae naturae.


&

第二異論に対しては、以下のように言われるべきである。
前に示されたとおり、神の中にあるものは何であれ、神である。しかし、これは
 他の事物ではそうならない。ゆえに、外へ向かうのでないどの発出によっても、
 神の本性は伝達される。しかし、他の本性ではそうならない。


\\



{\scshape Ad tertium dicendum} quod, licet in Deo non sit aliud voluntas et
intellectus, tamen de ratione voluntatis et intellectus est, quod
processiones quae sunt secundum actionem utriusque, se habeant secundum
quendam ordinem. Non enim est processio amoris nisi in ordine ad
processionem verbi, nihil enim potest voluntate amari, nisi sit in
intellectu conceptum. Sicut igitur attenditur quidam ordo verbi ad
principium a quo procedit, licet in divinis sit eadem substantia
intellectus et conceptio intellectus; ita, licet in Deo sit idem
voluntas et intellectus, tamen, quia de ratione amoris est quod non
procedat nisi a conceptione intellectus, habet ordinis distinctionem
processio amoris a processione verbi in divinis.


&

第三異論に対しては、以下のように言われるべきである。
神において、意志と知性は別のものではないが、しかし、意志と知性の性格には、
 その両者の作用に即してある発出が、ある種の秩序のもとにあることが含まれ
 ている。すなわち、愛の発出は、言葉の発出への秩序においてでなければなら
 ない。なぜなら、知性に捉えられたものでなければ、何ものも意志によって愛
 されないからである。それゆえ、ちょうど、知性と知性の懐念とは神において
 同一の実体であるにもかかわらず、言葉には、言葉がそこから発出する
 ところの根源への秩序が見出されるように、神において、意志と知性は同一で
 あるにもかかわらず、愛の性格に、知性の懐念からでなければ発出しないと
 いうことが含まれるので、神における愛の発出は、言葉の発出とは区別される
 秩序をもつ。


\end{longtable}
\newpage


\rhead{a.~4}
\begin{center}
 {\Large {\bf ARTICULUS QUARTUS}}\\
 {\large UTRUM PROCESSIO AMORIS IN DIVINIS SIT GENERATIO}\\
 {\footnotesize Infra, q.30, a.2, ad 2; I {\itshape Sent.}, d.13, a.3, ad 3, 4; III, d.8, a.1, ad 8; IV {\itshape SCG}, cap.19; {\itshape De Pot.}, q.2, a.4, ad 7; q.10, a.2, ad 22; {\itshape Compend.~Theol.}, cap.46.}\\
 {\Large 神における愛の発出は生成か\\}
\end{center}

\begin{longtable}{p{21em}p{21em}}


{\Huge A}{\scshape d quartum sic proceditur}. Videtur quod processio
 amoris in divinis sit generatio. Quod enim procedit in similitudine
 naturae in viventibus, dicitur generatum et nascens. Sed id quod
 procedit in divinis per modum amoris, procedit in similitudine naturae,
 alias esset extraneum a natura divina, et sic esset processio ad
 extra. Ergo quod procedit in divinis per modum amoris, procedit ut
 genitum et nascens.

&

第四項の問題へ、議論は以下のように進められる。
神における愛の発出は生成であると思われる。理由は以下の通り。
生きているものどもの中で、本性の類似において発出するものは、生成したもの、
 生まれたもの、と言われる。しかし、神において、愛のかたちで発出するもの
 は、本性の類似において発出する。さもなければ、それは神の本性とは異質な
 ものであり、したがって外への発出だっただろう。ゆえに、神において愛のか
 たちで発出するものは、生成したもの、生まれたもの、として発出する。


\\



2. {\scshape Praeterea}, sicut similitudo est de ratione verbi, ita est
 etiam de ratione amoris, unde dicitur {\itshape Eccli}.~{\scshape xiii} , quod {\itshape omne animal
 diligit simile sibi}. Si igitur ratione similitudinis verbo procedenti
 convenit generari et nasci, videtur etiam quod amori procedenti
 convenit generari.

&

さらに、言葉の性格に類似が含まれるように、愛の性格にも類似が含まれる。こ
 のことから、『集会の書』第13節で「すべての動物は自分に似たものを愛する」\footnote{「生き物はすべて、その同類を愛し、/人間もすべて、
 自分に近い者を愛する。」(13:15)}と言われている。ゆえに、もし、類似が根
 拠となって、発出する言葉に生成することや生まれることが適合するのであれ
 ば、発出する愛にも、生成することが適合すると思われる。

\\



3. {\scshape Praeterea}, non est in genere quod non est in aliqua eius
 specie. Si igitur in divinis sit quaedam processio amoris, oportet
 quod, praeter hoc nomen commune, habeat aliquod nomen speciale. Sed non
 est aliud nomen dare nisi generatio. Ergo videtur quod processio amoris
 in divinis sit generatio.



&

さらに、種に属さないものが類に属することはない。ゆえに、もし神の中に、
 愛という何らかの発出があるならば、この共通の名前以外に、何か特殊な名前
 を持つはずである。しかし、「生成」以外の名前は与えられない。ゆえに、神
 における愛の発出は生成である。


\\



{\scshape Sed contra est} quia secundum hoc sequeretur quod spiritus
 sanctus, qui procedit ut amor, procederet ut genitus. Quod est contra
 illud Athanasii, {\itshape spiritus sanctus a patre et filio non factus nec
 creatus nec genitus, sed procedens}.



&

しかし反対に、これに従うならば、愛として発出する聖霊が、生成したものとし
 て発出したことになる。しかしこれは、かのアタナシウスの「聖霊は、父と子
 から、作られたものとしてでも創造されたものとしてでも生成したものとして
 でもなく、発出する」に反する。



\\



{\scshape Respondeo dicendum} quod processio amoris in divinis non debet dici
 generatio. Ad cuius evidentiam, sciendum est quod haec est differentia
 inter intellectum et voluntatem, quod intellectus fit in actu per hoc
 quod res intellecta est in intellectu secundum suam similitudinem,
 voluntas autem fit in actu, non per hoc quod aliqua similitudo voliti
 sit in voluntate, sed ex hoc quod voluntas habet quandam inclinationem
 in rem volitam. 


&

解答する。以下のように言われるべきである。
神における愛の発出は、生成と言われるべきでない。
これを明らかにするためには、知性と意志には、以下の違いがあることを知らなければならない。
すなわち、知性は、知性認識された事物が自らの類似に即して知性の中にあるこ
 とによって現実態になるが、意志は、意志されたものの何らかの類似が意志する
 ものの中にあることによってではなく、意志が意志された事物へ何らかの傾き
 を持つことによって、現実態になる。



\\

Processio igitur quae attenditur secundum rationem
 intellectus, est secundum rationem similitudinis, et intantum potest
 habere rationem generationis, quia omne generans generat sibi
 simile. Processio autem quae attenditur secundum rationem voluntatis,
 non consideratur secundum rationem similitudinis, sed magis secundum
 rationem impellentis et moventis in aliquid. 



&

それゆえ、知性の観点から見出される発出は、類似の観点によるものであり、そ
 の限りで生成の性格を持つ。すべて生むものは自らに似たものを生むからであ
 る。これに対して、意志の観点から見出される発出は、類似の観点から考察さ
 れるのではなく、何かへと駆り立て動かすものという観点による。


\\


Et ideo quod procedit in
 divinis per modum amoris, non procedit ut genitum vel ut filius, sed
 magis procedit ut {\itshape spiritus}, quo nomine quaedam vitalis motio et
 impulsio designatur, prout aliquis ex amore dicitur moveri vel impelli
 ad aliquid faciendum.



&

それゆえ、神において愛のかたちで発出するものは、生成したものや息子として
 ではなく、「霊」---この名前によって、ある種の生命の動きや
 強制が意味される---として発出する。ある人が、愛によって、何かをするために動かされ駆り立
 てられる、と言われるように。

\\



{\scshape Ad primum ergo dicendum} quod quidquid est in divinis, est
 unum cum divina natura. Unde ex parte huius unitatis non potest accipi
 propria ratio huius processionis vel illius, secundum quam una
 distinguatur ab alia, sed oportet quod propria ratio huius vel illius
 processionis accipiatur secundum ordinem unius processionis ad
 aliam. Huiusmodi autem ordo attenditur secundum rationem voluntatis et
 intellectus. Unde secundum horum propriam rationem sortitur in divinis
 nomen utraque processio, quod imponitur ad propriam rationem rei
 significandam. Et inde est quod procedens per modum amoris et divinam
 naturam accipit, et tamen non dicitur natum.



&

第一異論に対しては、それゆえ、以下のように言われるべきである。
神の中にあるものはなんであれ、神の本性と一つである。したがって、この一性
 の側からは、これやあれの発出の固有の性格が理解されることはできない。
その固有の性格によって、一つの発出が他の発出から区別される。むしろ、一つ
 の発出の他の発出への秩序に即して、これやあれの発出の固有の性格は理解さ
 れるべきである。
さて、このような秩序は、意志と知性の性格に即して見出される。
ゆえに、これらの固有の性格に即して、神における両方の発出に名前が与えられ、
 その名前は、事物の固有の性格を表示するために名付けられる。
愛のかたちで発出するものが、神の本性を受け取るが、生成した
 ものとは言われないのは、このためである。

\\



{\scshape Ad secundum dicendum} quod similitudo aliter pertinet ad
 verbum, et aliter ad amorem. Nam ad verbum pertinet inquantum ipsum est
 quaedam similitudo rei intellectae, sicut genitum est similitudo
 generantis, sed ad amorem pertinet, non quod ipse amor sit similitudo,
 sed inquantum similitudo est principium amandi. Unde non sequitur quod
 amor sit genitus, sed quod genitum sit principium amoris.



&


第二異論に対しては、以下のように言われるべきである。
言葉と愛とでは、それに類似が属するしかたが異なる。
言葉の場合、言葉自体が、知性認識された事物のある種の類似である。それはちょ
 うど、生まれたものが、生むものの類似であるように。
これに対して愛の場合、愛自体が類似なのではなく、類似が愛することの根源で
 ある。
したがって、愛が生まれたものであることは帰結しない。むしろ、生まれたもの
 が、愛の根源である。


\\



{\scshape Ad tertium dicendum} quod Deum nominare non possumus nisi ex
 creaturis, ut dictum est supra. Et quia in creaturis communicatio
 naturae non est nisi per generationem, processio in divinis non habet
 proprium vel speciale nomen nisi generationis. Unde processio quae non
 est generatio, remansit sine speciali nomine. Sed potest nominari
 spiratio, quia est processio spiritus.



&


第三異論に対しては、以下のように言われるべきである。
前に述べられたとおり、私たちは、被造物にもとづいてでなければ、神を名付け
 ることができない。そして、被造物において、本性の伝達は、生成による以外
 にないので、神における発出は、「生成」以外の固有な名称、あるいは種的な名称を持た
 ない。したがって、生成でない発出は、種的な名前を持たないままとなってい
 る。しかし、霊の発出なので、「霊発」と名付けることは可能である。




\end{longtable}
\newpage



\rhead{a.~5}
\begin{center}
 {\Large {\bf ARTICULUS QUINTUS}}\\
 {\large UTRUM SINT PLURES PROCESSIONES IN DIVINIS QUAM DUAE}\\
 {\footnotesize IV {\itshape SCG}, cap.26; {\itshape De Pot.}, q.9, a.9;
 q.10, a.2, ad argumenta {\itshape Sed contra.}}\\
 {\Large 第五項\\神の中に二つ以上の発出があるか}
\end{center}

\begin{longtable}{p{21em}p{21em}}



{\Huge A}{\scshape d quintum sic proceditur}. Videtur quod sint plures processiones in
 divinis quam duae. Sicut enim scientia et voluntas attribuitur Deo, ita
 et potentia. Si igitur secundum intellectum et voluntatem accipiuntur
 in Deo duae processiones, videtur quod tertia sit accipienda secundum
 potentiam.

&

第五項の問題へ議論は以下のように進められる。
神の中に二つ以上の発出があると思われる。理由は以下の通り。
知と意志が神に帰せられるように、能力もまた帰せられる。ゆえに、知性と意志
 に即して、神の中に二つの発出が理解されるならば、能力に即して第三の発出
 が理解されるべきであると思われる。

\\



2. {\scshape Praeterea}, bonitas maxime videtur esse principium processionis, cum
 bonum dicatur diffusivum sui esse. Videtur igitur quod secundum
 bonitatem aliqua processio in divinis accipi debeat.

&

さらに、善は自らの存在を普及させうるものと言われるので、善性は、最大限に、
 発出の根源だと思われる。ゆえに神の中に善性にしたがった何らかの発
 出が理解されるべきであると思われる。

\\



3. {\scshape Praeterea}, maior est fecunditatis virtus in Deo quam in nobis. Sed in
 nobis non est tantum una processio verbi, sed multae, quia ex uno verbo
 in nobis procedit aliud verbum; et similiter ex uno amore alius
 amor. Ergo et in Deo sunt plures processiones quam duae.

&

さらに、力の豊かさは、私たちにおいてよりも神においての方が大きい。しかし、
 私たちの中に、たんに言葉の発出だけでなく、多くの発出がある。なぜなら、
私たちの中には、一つの言葉から、他の言葉が発出するし、一つの愛から他の愛
 が発出するからである。ゆえに、神の中にも、二つ以上の複数の発出がある。


\\



{\scshape Sed contra est} quod in Deo non sunt nisi duo procedentes, scilicet
 filius et spiritus sanctus. Ergo sunt ibi tantum duae processiones.

&

しかし反対に、神の中には、発出するものは二つしかない。すなわち、息子と
 聖霊とである。ゆえに、そこには発出が二つだけある。

\\



{\scshape Respondeo dicendum} quod processiones in divinis accipi non possunt nisi
 secundum actiones quae in agente manent. Huiusmodi autem actiones in
 natura intellectuali et divina non sunt nisi duae, scilicet intelligere
 et velle. Nam sentire, quod etiam videtur esse operatio in sentiente,
 est extra naturam intellectualem, neque totaliter est remotum a genere
 actionum quae sunt ad extra; nam sentire perficitur per actionem
 sensibilis in sensum. Relinquitur igitur quod nulla alia processio
 possit esse in Deo, nisi verbi et amoris.

&

解答する。以下のように言われるべきである。
神における発出は、作用者の中にとどまる作用に即してのみ理解されうる。しか
 しそのような作用は、知性的本性、神的本性の中に、二つしかない。すなわ
 ち、知性認識作用と意志作用とである。
感覚作用は、たしかに感覚するものの中にある働きに見えるが、しかし、知性的
 本性の外にあるし、外へ向かう作用の類からまったく離れているわけでもない。
 なぜなら、感覚することは、可感的なものの感覚への働きかけによって完成さ
 れるからである。ゆえに、言葉と愛以外のどんな発出も神の中にないことが帰
 結する。

\\



{\scshape Ad primum ergo dicendum} quod potentia est principium agendi in aliud,
 unde secundum potentiam accipitur actio ad extra. Et sic secundum
 attributum potentiae non accipitur processio divinae personae, sed
 solum processio creaturarum.

&

第一異論に対しては、それゆえ、以下のように言われるべきである。
能力は、他のものへ働きかけることの根源である。したがって、能力に即しては、
 外への作用が理解される。この意味で、能力という属性にしたがっては、神の
 ペルソナの発出ではなく、ただ被造物の発出だけが理解される。

\\



{\scshape Ad secundum dicendum} quod bonum, sicut dicit Boetius in libro {\itshape de Hebd}.,
 pertinet ad essentiam et non ad operationem, nisi forte sicut obiectum
 voluntatis. Unde, cum processiones divinas secundum aliquas actiones
 necesse sit accipere, secundum bonitatem et huiusmodi alia attributa
 non accipiuntur aliae processiones nisi verbi et amoris, secundum quod
 Deus suam essentiam, veritatem et bonitatem intelligit et amat.

&

第二異論に対しては、以下のように言われるべきである。
ボエティウスが『デ・ヘブドマディブス』という書物で言うように、善は、働き
 ではなく本質に属する。ただし、働きが意志の対象である場合は除いて。
したがって、神の発出は、何らかの作用に即して理解する必要があるので、善性
 その他そのような属性に即しては、言葉と愛以外の他の発出は理解されない。
言葉と愛は、 神が、自らの本質と真理と善性を、知性認識し愛するかぎりにお
 いて理解される。


\\



{\scshape Ad tertium dicendum} quod, sicut supra habitum est, Deus uno simplici
 actu omnia intelligit, et similiter omnia vult. Unde in eo non potest
 esse processio verbi ex verbo, neque amoris ex amore, sed est in eo
 solum unum verbum perfectum, et unus amor perfectus. Et in hoc eius
 perfecta fecunditas manifestatur.

&

第三異論に対しては、以下のように言われるべきである。
前に述べられたとおり、神は、一つの単純な作用によって万物を知性認識し、同
 様にして、万物を意志する。したがって、そこでは、言葉から言葉が、愛から
 愛が発出することはありえない。そこには、ただ
一つの完全な言葉、一つの完全な愛がある。そしてそこにおいて、神の完全な豊かさが
 示されている。




\end{longtable}
\end{document}
