\documentclass[10pt]{jsarticle} % use larger type; default would be 10pt
%\usepackage[utf8]{inputenc} % set input encoding (not needed with XeLaTeX)
%\usepackage[round,comma,authoryear]{natbib}
%\usepackage{nruby}
\usepackage{okumacro}
\usepackage{longtable}
%\usepqckage{tablefootnote}
\usepackage[polutonikogreek,english,japanese]{babel}
%\usepackage{amsmath}
\usepackage{latexsym}
\usepackage{color}

%----- header -------
\usepackage{fancyhdr}
\lhead{{\it Summa Theologiae} I-IIae, q.~1}
%--------------------

\bibliographystyle{jplain}

\title{{\bf PRIMA SECUNDAE}\\{\HUGE Summae Theologiae}\\Sancti Thomae
Aquinatis}
\author{Japanese translation\\by Yoshinori {\sc Ueeda}}
\date{Last modified \today}


%%%% コピペ用
%\rhead{a.~}
%\begin{center}
% {\Large {\bf }}\\
% {\large }\\
% {\footnotesize }\\
% {\Large \\}
%\end{center}
%
%\begin{longtable}{p{21em}p{21em}}
%
%&
%
%
%\\
%\end{longtable}
%\newpage







\begin{document}
\maketitle
\pagestyle{fancy}

\rhead{prologus}
\begin{longtable}{p{21em}p{21em}}
{\bfseries PROLOGUS}

Quia, sicut Damascenus dicit, homo factus ad
imaginem Dei dicitur, secundum quod per imaginem significatur
intellectuale et arbitrio liberum et per se potestativum; postquam
praedictum est de exemplari, scilicet de Deo, et de his quae
processerunt ex divina potestate secundum eius voluntatem; restat ut
consideremus de eius imagine, idest de homine, secundum quod et ipse est
suorum operum principium, quasi liberum arbitrium habens et suorum
operum potestatem.


&
{\bfseries 序文}

ダマスケヌスが言うように、人間が神の像に向けて作られたと言われるのは、
 その「像」によって、知性的であることと、選択において自由であり、自ら
 によって権能があることが意味されることによるのだから、範型、すなわち神
 について、そして、神の意志に従って神の権能から出てくるものどもについて
 予め述べられたあと、神の像、すなわち人間について、自由選択をもち、自ら
 の業の権能を持つものとして、人間もまた自らの業の根源であるかぎりで、考
 察することが残されている。

\end{longtable}

\newpage
\rhead{a.~1}
\begin{center}
 {\Large {\bf QUAESTIO PRIMA}}\\
 {\large DE ULTIMO FINE HOMINIS}\\

 {\Large 第一問\\人間の究極目的について}
\end{center}

\begin{longtable}{p{21em}p{21em}}
Ubi primo considerandum occurrit de ultimo fine humanae vitae; et deinde
de his per quae homo ad hunc finem pervenire potest, vel ab eo deviare,
ex fine enim oportet accipere rationes eorum quae ordinantur ad
finem. Et quia ultimus finis humanae vitae ponitur esse beatitudo,
oportet primo considerare de ultimo fine in communi; deinde de
beatitudine. 


&


まず第一に、人間の生の究極目的について考察すべきことが生じる。次いで、人
 間がそれによってこの目的へ到達できたり、あるいは、それから逸脱しうると
 ころのものどもについて[考察すべきである]。というのも、目的へ秩序づけられるものどもの性格
 は、その目的から受け取らなければならないからである。
そして、人間の生の究極目的は、至福だとされるから、第一に、究極目的につい
 て共通的に、次いで、至福について考察しなければならない。


\\

Circa primum quaeruntur octo. 

\begin{enumerate}
 \item utrum hominis sit agere propter finem.
 \item utrum hoc sit proprium rationalis naturae.
 \item utrum actus hominis recipiant speciem a fine.
 \item utrum sit aliquis ultimus finis humanae vitae.
 \item utrum unius hominis possint esse plures ultimi fines.
 \item utrum homo ordinet omnia in ultimum finem.
 \item utrum idem sit finis ultimus omnium hominum.
 \item utrum in illo ultimo fine omnes aliae creaturae conveniant.
\end{enumerate}


&

第一のことをめぐって、八つのことが問われる。
\begin{enumerate}
 \item 人間には、目的のために働くことが属するか。
 \item このことは、理性的本性に固有か。
 \item 人間の働きは、目的から種を受け取るか。
 \item 人間の生には、何らかの究極目的があるか。
 \item 一人の人間に、複数の究極目的がありうるか。
 \item 人間は、すべてのことを究極目的に秩序づけるか。
 \item すべての人間の究極目的は同じか。
 \item かの究極目的において、他のすべての被造物は一致するか。
\end{enumerate}

\end{longtable}

\newpage
\rhead{a.~1}
\begin{center}
 {\Large {\bf ARTICULUS PRIMUS}}\\
 {\large UTRUM HOMINI CONVENIAT AGERE PROPTER FINEM}\\
 {\footnotesize Infra, a.2; q.6, a.1; III{\itshape SCG}, cap.2}\\
 {\Large 第一項\\目的のために働くことは人間に適合するか}
\end{center}

\begin{longtable}{p{21em}p{21em}}


{\Huge A}{\scshape d primum sic proceditur}. Videtur quod
homini non conveniat agere propter finem. Causa enim naturaliter prior
est. Sed finis habet rationem ultimi, ut ipsum nomen sonat. Ergo finis
non habet rationem causae. Sed propter illud agit homo, quod est causa
actionis, cum haec praepositio propter designet habitudinem causae. Ergo
homini non convenit agere propter finem.

&

第一の問題へ、議論は以下のように進められる。
人間に、目的のために働くことは適合しないと思われる。理由は以下の通り。
原因は、本性的に、より先のものである。
ところが、目的は、その名前自身の響きからして、「最後」という意味を持つ。
ゆえに、目的は、原因という性格を持たない。
ところで、「ために」(propter)という前置詞は、原因であるという関係を指す
 ので、人間が何かの「ために」働く時、その何かは、その働きの原因である。
ゆえに、人間に、目的のために働くことは適合しない。


\\


{\scshape 2 Praeterea}, illud quod est ultimus
finis, non est propter finem. Sed in quibusdam actiones sunt ultimus
finis; ut patet per philosophum in I {\itshape Ethic}. Ergo non omnia homo agit
propter finem.

&

さらに、究極目的であるものは、目的のためにあるのではない。
ところが、『ニコマコス倫理学』第1巻の哲学者によって明らかなとおり、
究極目的は、なんらかの働きの中にある。
ゆえに、人間は、すべての事柄を、目的のために行うのではない。



\\


{\scshape 3 Praeterea}, tunc videtur homo agere
propter finem, quando deliberat. Sed multa homo agit absque
deliberatione, de quibus etiam quandoque nihil cogitat; sicut cum
aliquis movet pedem vel manum aliis intentus, vel fricat barbam. Non
ergo homo omnia agit propter finem.

&

さらに、人間が目的のために働くのは、よく考える時である。
しかし、人間は、多くのことを、よく考えないで行うし、まったく考えないで行
 うこともある。たとえば、ある人が足や手を、別のことに注意を向けながら動
 かしたり、髭をなでたりするように。
ゆえに、人間はすべてのことを目的のために行うとは限らない。


\\


{\scshape Sed contra}, omnia quae sunt in aliquo
genere, derivantur a principio illius generis. Sed finis est principium
in operabilibus ab homine; ut patet per philosophum in II {\itshape Physic}. Ergo
homini convenit omnia agere propter finem.

&


しかし反対に、なんらかの類の中にあるものはすべて、その類の根源から出てく
 る。ところで、『自然学』第2巻の哲学者によって明らかなとおり、目的は、人
 間によって行われる事柄の中の根源である。ゆえに、すべてを目的のために行
 うことは人間に適合する。

\\


{\scshape Respondeo dicendum} quod actionum quae ab
homine aguntur, illae solae proprie dicuntur humanae, quae sunt propriae
hominis inquantum est homo. Differt autem homo ab aliis irrationalibus
creaturis in hoc, quod est suorum actuum dominus. Unde illae solae
actiones vocantur proprie humanae, quarum homo est dominus. 


&

解答する。以下のように言われるべきである。
人間によってなされる行為のうち、人間であるかぎりでの人間に固有の行為だけ
 が、固有に、人間的行為と言われる。
ところで、人間は、他の非理性的被造物から、自らの働きの主であるという点で
 異なっている。したがって、人間がそれの主であるような行為だけが、固有に、
 人間的と呼ばれる。


\\

Est autem
homo dominus suorum actuum per rationem et voluntatem, unde et liberum
arbitrium esse dicitur facultas voluntatis et rationis. Illae ergo
actiones proprie humanae dicuntur, quae ex voluntate deliberata
procedunt. Si quae autem aliae actiones homini conveniant, possunt dici
quidem hominis actiones; sed non proprie humanae, cum non sint hominis
inquantum est homo. 

&

ところで、人間が自分の働きの主であるのは、理性と意志を通してである。この
 ことから、自由選択もまた、意志と理性の機能と言われる。
ゆえに、行為が固有に人間的と言われるのは、それがよく考えられた意志にから
 出てくる場合である。
他方、他の何らかの行為が人間に属する時、たしかにそれは人間の行為と言われ
 うるが、しかし、それは人間であるかぎりでの人間の行為ではないので、固有
 に人間的なものではない。


\\

Manifestum est autem quod omnes actiones quae
procedunt ab aliqua potentia, causantur ab ea secundum rationem sui
obiecti. Obiectum autem voluntatis est finis et bonum. Unde oportet quod
omnes actiones humanae propter finem sint.

&

ところで、ある能力から出てくるすべての行為は、その対象の性格にしたがって、
 その能力から原因されることは明らかである。
ところで、意志の対象は、目的であり善である。
したがって、すべての人間的行為は目的のためにあるはずである。


\\



{\scshape Ad primum ergo dicendum} quod finis, etsi
sit postremus in executione, est tamen primus in intentione agentis. Et
hoc modo habet rationem causae.

&



第一異論に対しては、それゆえ、以下のように言われるべきである。
目的は、たしかに遂行において最後だが、働くものの意図においては、最初であ
 る。そしてこの後者の意味で、原因という性格を持つ。


\\



{\scshape Ad secundum dicendum} quod, si qua actio
humana sit ultimus finis, oportet eam esse voluntariam, alias non esset
humana, ut dictum est. Actio autem aliqua dupliciter dicitur voluntaria,
uno modo, quia imperatur a voluntate, sicut ambulare vel loqui; alio
modo, quia elicitur\footnote{\={e}-l\u{\i}c\u{\i}o , l\u{\i}c\u{u}i and lexi:
I. to draw out, entice out, to lure forth, to bring out, to elicit (class.). } a voluntate, sicut ipsum velle. 


&

第二異論に対しては、以下のように言われるべきである。
もし何らかの人間的行為が究極目的であれば、それは意志的なものでなければな
 らない。さもなければ、すでに述べられたとおり、人間的な行為でないから
 である。ところで、ある行為は、二通りのしかたで意志的と言われる。
一つには、歩くことや話すことのように、意志によって命じられるからであり、
 もう一つには、意志すること自体のように、意志から出てくるからであ
 る。


\\

Impossibile autem
est quod ipse actus a voluntate elicitus sit ultimus finis. Nam obiectum
voluntatis est finis, sicut obiectum visus est color, unde sicut
impossibile est quod primum visibile sit ipsum videre, quia omne videre
est alicuius obiecti visibilis; ita impossibile est quod primum
appetibile, quod est finis, sit ipsum velle. 



&

ところで、意志から出てきた働きそれ自体が究極目的であることは不可能で
 ある。
なぜなら、ちょうど視覚の対象が色であるように、意志の対象は目的である。
したがって、見られうる第一のものが見ることそれ自体であることが不可能であ
 るように、というのも、すべての見ることは、可視的な何らかの対象を見るこ
 とだからであるが、欲求されうる第一のもの、これが目的であるが、が、意志
 することそれ自体であることは不可能である。


\\

Unde relinquitur quod, si
qua actio humana sit ultimus finis, quod ipsa sit imperata a
voluntate. Et ita ibi aliqua actio hominis, ad minus ipsum velle, est
propter finem. Quidquid ergo homo faciat, verum est dicere quod homo
agit propter finem, etiam agendo actionem quae est ultimus finis.

&

したがって、残るのは、もし人間的行為が究極目的であるならば、それが意志に
 よって命じられたものだということである。このようにして、ここでもまた、何らかの人間の
 行為、少なくとも意志すること自体は、目的のためにある。
ゆえに、何であれ人間がすることは、究極目的であるところの作用をすることに
 よってであっても、「人間が目的のために働く」と言うことは、真である。


\\


{\scshape Ad tertium dicendum} quod huiusmodi
actiones non sunt proprie humanae, quia non procedunt ex deliberatione
rationis, quae est proprium principium humanorum actuum. Et ideo habent
quidem finem imaginatum, non autem per rationem praestitutum.

&

第三異論に対しては、以下のように言われるべきである。
このような作用は、固有に人間的なものでなない。なぜなら、人間的な働きの根
 源である、理性の熟慮から出てきてないからである。
ゆえに、想像された目的は持つとしても、それは理性を通して示されたものでは
 ない。







\end{longtable}
\newpage










\rhead{a.~2}
\begin{center}
 {\Large {\bf ARTICULUS SECUNDUS}}\\
 {\large UTRUM AGERE PROPTER FINEM SIT PROPRIUM RATIONALIS NATURAE}\\
 {\footnotesize Infra, q.12, a.5; II {\itshape SCG}, cap.23; III, cap.1,
 2, 16, 24; {\itshape De Pot.}, q.1, a.5; q.3, a.15; V {\itshape
 Metaphys.}, lect.16.}\\
 {\Large 第二項\\目的のために働くことは理性的本性に固有か}
\end{center}

\begin{longtable}{p{21em}p{21em}}



{\Huge A}{\scshape d secundum sic proceditur}. Videtur
quod agere propter finem sit proprium rationalis naturae. Homo enim,
cuius est agere propter finem, nunquam agit propter finem ignotum. Sed
multa sunt quae non cognoscunt finem, vel quia omnino carent cognitione,
sicut creaturae insensibiles; vel quia non apprehendunt rationem finis,
sicut bruta animalia. Videtur ergo proprium esse rationalis naturae
agere propter finem.


&


第二項の問題へ、議論は以下のように進められる。
目的のために働くことは、理性的本性に固有であると思われる。理由は以下の通
 り。
人間には、目的のために働くことが属するが、知られていない目的のために働く
 ことはけっしてない。ところが、目的を認識しないものが多くある。そしてそれは、
 非感覚的被造物のように、まったく認識を欠いているか、あるいは、非理性的
 動物のように、目的という性格を捉えないかのどちらかである。ゆえに、目的
 のために働くことは、理性的本性に固有であると思われる。


\\


{\scshape 2 Praeterea}, agere propter finem est
ordinare suam actionem ad finem. Sed hoc est rationis opus. Ergo non
convenit his quae ratione carent.


&


さらに、目的にために働くことは、自分の行為を目的へと秩序づけることである。
ところが、このことは、理性が行うことである。
ゆえに、[このことは]理性を欠くものどもに適合しない。

\\


{\scshape 3 Praeterea}, bonum et finis est obiectum
voluntatis. Sed voluntas in ratione est, ut dicitur in III {\itshape de
Anima}. Ergo agere propter finem non est nisi rationalis naturae.


&

さらに、善と目的は、意志の対象である。
ところで、『デ・アニマ』第3巻で言われるように、意志は理性の中にある。
ゆえに、目的のために働くことは、理性的本性だけに属する。



\\


{\scshape Sed contra} est quod philosophus probat
in II {\itshape Physic}., quod non solum intellectus, sed etiam natura agit propter
finem.


&

しかし反対に、哲学者は、『自然学』第2巻で、知性だけでなく、自然もまた、
 目的のために働くことを証明している。


\\


{\scshape Respondeo dicendum} quod omnia agentia
necesse est agere propter finem. Causarum enim ad invicem ordinatarum,
si prima subtrahatur, necesse est alias subtrahi. Prima autem inter
omnes causas est causa finalis. Cuius ratio est, quia materia non
consequitur formam nisi secundum quod movetur ab agente, nihil enim
reducit se de potentia in actum. 


&

解答する。以下のように言われるべきである。
すべて働くものは、目的のために働くことが必然である。理由は以下の通り。
相互に秩序づけられた諸原因のうち、第一の原因が取り除かれると、他も取り除
 かれることが必然である。ところで、すべての原因の中で、第一のものは目的
 因である。理由は以下。質料は作用者によって動かされる限りにおいてでなけ
 れば、形相を獲得しない。自分自身を、可能態から現実態へ引き出さないから
 である。



\\

Agens autem non movet nisi ex
intentione finis. Si enim agens non esset determinatum ad aliquem
effectum, non magis ageret hoc quam illud, ad hoc ergo quod determinatum
effectum producat, necesse est quod determinetur ad aliquid certum, quod
habet rationem finis. Haec autem determinatio, sicut in rationali natura
fit per rationalem appetitum, qui dicitur voluntas; ita in aliis fit per
inclinationem naturalem, quae dicitur appetitus naturalis. 


&

ところで、作用者は、目的への意図・方向性に基づいてでなければ、動かさない。
なぜなら、もし、作用者が何らかの結果へと決定されていなかったならば、あれ
 よりもこれを為すということがなかっただろうから。それゆえ、限定された結
 果を生み出すためには、何か一定のものへ限定されることが必要であるが、こ
 の一定のものが、目的という性格を持つ。
そしてこの限定は、ちょうど理性的本性において、意志と言われる理性的欲求
 によってなされるように、他のものどもにおいては、自然手本性的欲求と言わ
 れる、自然本性的傾向性によってなされる。



\\

Tamen
considerandum est quod aliquid sua actione vel motu tendit ad finem
dupliciter, uno modo, sicut seipsum ad finem movens, ut homo; alio modo,
sicut ab alio motum ad finem, sicut sagitta tendit ad determinatum finem
ex hoc quod movetur a sagittante, qui suam actionem dirigit in
finem. Illa ergo quae rationem habent, seipsa movent ad finem, quia
habent dominium suorum actuum per liberum arbitrium, quod est facultas
voluntatis et rationis. 


&

しかし、以下のことが考察されるべきである。
あるものが、自分の働きや運動によって、目的へ向かうのに二通りのかたちがあ
 る。
一つには、人間のように、自分自身を目的へ動かすようにしてであり、もう一つ
 には、他のものによって目的へ動かされるようにしてである。たとえば、矢が、
 限定された目的へ、射手によって動かされることによって向かうように。そし
 て射手は、自らの働きを目的へと導く。
ゆえに、理性を持つものどもは、自分を目的へ動かす。なぜなら、意志と理性の
 機能である自由選択によって、自分の行為の所有権(dominium)をもつからである。



\\

Illa vero quae ratione carent, tendunt in finem
per naturalem inclinationem, quasi ab alio mota, non autem a seipsis,
cum non cognoscant rationem finis, et ideo nihil in finem ordinare
possunt, sed solum in finem ab alio ordinantur. Nam tota irrationalis
natura comparatur ad Deum sicut instrumentum ad agens principale, ut
supra habitum est. 


&

他方、理性を持たないものどもは、他のものによって動かされるようにして、自
 然本性的傾向性によって、目的へ向かう。しかし、目的という性格を認識しな
 いので、何かを目的へ秩序づけることができず、むしろ他のものによって目的
 へ秩序づけられるだけなので、自分自身によってではない。
じっさい、前に述べられたとおり、非理性的本性全体が神に対する関係は、道具
 が、主要な作用者に対する関係と同様である。



\\


Et ideo proprium est naturae rationalis ut tendat in
finem quasi se agens vel ducens ad finem, naturae vero irrationalis,
quasi ab alio acta vel ducta, sive in finem apprehensum, sicut bruta
animalia, sive in finem non apprehensum, sicut ea quae omnino cognitione
carent.


&


ゆえに、自分を目的へと向かわせる、あるいは導くようにして、目的へと向かう
 ことは、理性的本性に固有であるが、非理性的本性には、他のものによって向
 けられ、導かれて、[目的へと向かうことが固有である]。ただし、非理性的
 動物の場合、その目的は捉えられているが、認識をまったく欠くものどもは、
 その目的が捉えられていない。

\\


{\scshape Ad primum ergo dicendum} quod homo, quando
per seipsum agit propter finem, cognoscit finem, sed quando ab alio
agitur vel ducitur, puta cum agit ad imperium alterius, vel cum movetur
altero impellente, non est necessarium quod cognoscat finem. Et ita est
in creaturis irrationalibus.


&


 第一異論に対しては、それゆえ、以下のように言われるべきである。
 人間は、自分自身を通して目的のために働くとき、目的を認識しているが、た
 とえば、他の人の命令に応じて働くときや、強制する他のものによって動かさ
 れるときのように、他のものによって働きを受けたり、導かれたりするときは、
 目的を認識している必要はない。これは、非理性的被造物においても同様である。

\\


{\scshape Ad secundum dicendum} quod ordinare in
finem est eius quod seipsum agit in finem. Eius vero quod ab alio in
finem agitur, est ordinari in finem. Quod potest esse irrationalis
naturae, sed ab aliquo rationem habente.


&


 第二異論に対しては、以下のように言われるべきである。
 目的へ秩序づけることは、自分自身が目的へ向けて作用するものに属する。
 他方、他のものによって、目的へ向けて作用されるものには、目的へ秩序づけ
 られることが属する。そしてこのことは、非理性的被造物にも属しうる。しか
 し、理性を持つ何かによって秩序づけられるのではあるが。

\\


{\scshape Ad tertium dicendum} quod obiectum
voluntatis est finis et bonum in universali. Unde non potest esse
voluntas in his quae carent ratione et intellectu, cum non possint
apprehendere universale, sed est in eis appetitus naturalis vel
sensitivus, determinatus ad aliquod bonum particulare. Manifestum autem
est quod particulares causae moventur a causa universali, sicut rector
civitatis, qui intendit bonum commune, movet suo imperio omnia
particularia officia civitatis. Et ideo necesse est quod omnia quae
carent ratione, moveantur in fines particulares ab aliqua voluntate
rationali, quae se extendit in bonum universale, scilicet a voluntate
divina.


&

第三異論に対しては、以下のように言われるべきである。
意志の対象は、普遍における目的と善である。
したがって、理性と知性を欠くものどもは普遍を捉えられないから、それらの中
 に、意志が存在することは不可能である。むしろ、それらの中には自然的、あ
 るいは感覚的な欲求が存在し、何らかの個別的な善へと限定されている。
ところで、個別的な原因が普遍的な原因によって動かされることは明らかである。
 たとえば、都市の支配者は、共通の善を意図するが、自分の命令によって、都
 市のすべての個別的な機能を動かす。ゆえに、理性を欠くものはすべて、個別
 的な目的へと、何らかの理性的な意志、
 すなわち、普遍的な善にまで及ぶ
神の意志によって動かされることが必要である。




\end{longtable}
\newpage






\rhead{a.~3}
\begin{center}
 {\Large {\bf ARTICULUS TERTIUS}}\\
 {\large UTRUM ACTUS HOMINIS RECIPIAT SPECIEM EX FINE}\\
 {\footnotesize Infra, q.43, a.6; q.72, a.3; II {\itshape Sent.}, d.11,
 a.1; {\itshape De Verit.}, q.1, a.2, ad 3; q.11, a.3.}\\
 {\Large 第三項\\人間の行為は目的から種を受け取るか}
\end{center}

\begin{longtable}{p{21em}p{21em}}


{\Huge A}{\scshape d tertium sic proceditur}. Videtur quod
actus humani non recipiant speciem a fine. Finis enim est causa
extrinseca. Sed unumquodque habet speciem ab aliquo principio
intrinseco. Ergo actus humani non recipiunt speciem a fine.

&

第三の問題へ、議論は以下のように進められる。
人間的行為は、目的から種を受け取るのではないと思われる。理由は以下の通り。
目的は、外在的な原因である。ところで、各々のものは、何らかの内在的な根源
 から種を所有する。ゆえに、人間的行為は、目的から種を受け取るわけではな
 い。

\\


{\scshape 2 Praeterea}, illud quod dat speciem,
oportet esse prius. Sed finis est posterior in esse. Ergo actus humanus
non habet speciem a fine.

&


さらに、種を与えるものは、より先なるものでなければならない。
ところが、目的は、存在において、より後のものである。
ゆえに、人間的行為は、目的から種をもつのではない。

\\


{\scshape 3 Praeterea}, idem non potest esse nisi in
una specie. Sed eundem numero actum contingit ordinari ad diversos
fines. Ergo finis non dat speciem actibus humanis.

&

さらに、同一のものは、同一の種の中にしかありえない。
ところが、数的に同一の行為が、さまざまな目的へ秩序付けられる、ということ
 がありうる\footnote{もし、目的から種を受け取るのであれば、この行為は複
 数の種を受け取ることになり、複数の種の中にあることになる。しかしこれは
 大前提に反する。}。ゆえに、目的が人間的行為に種を与えるのではない。


\\


{\scshape Sed contra est} quod dicit Augustinus, in
libro {\itshape de moribus Ecclesiae et Manichaeorum}: {\itshape secundum quod finis est
culpabilis vel laudabilis, secundum hoc sunt opera nostra culpabilia vel
laudabilia}.

&


しかし反対に、アウグスティヌスは、『教会とマニ教の習俗について』で「目的
 が咎められうるものか褒められうるものかに応じて、私たちの業が咎められう
 るものであったり褒められうるものであったりする」と述べている。

\\


{\scshape Respondeo dicendum} quod unumquodque
sortitur speciem secundum actum, et non secundum potentiam, unde ea quae
sunt composita ex materia et forma, constituuntur in suis speciebus per
proprias formas. 



&

解答する。以下のように言われるべきである。
各々のものは、可能態ではなく現実態に即して、種を獲得する。
したがって、質料と形相から複合されたものは、それに固有の形相によって、自らの種
 の中に構成される。


\\



Et hoc etiam considerandum est in motibus propriis. Cum
enim motus quodammodo distinguatur per actionem et passionem, utrumque
horum ab actu speciem sortitur, actio quidem ab actu qui est principium
agendi; passio vero ab actu qui est terminus motus. Unde calefactio
actio nihil aliud est quam motio quaedam a calore procedens, calefactio
vero passio nihil aliud est quam motus ad calorem, definitio autem
manifestat rationem speciei. 



&

このこと[=現実態に即して種へ分類されるということ]は、固有の運動におい
 ても考えられるべきである。
つまり、運動は、ある意味で能動と受動に区別されるが、そのどちらも、現実態
 によって種を獲得するからである。能動は、作用の始原である現実態
 から、他方、受動は、運動の終点である現実態から[種を獲得する]。
したがって、熱する働きとは、「熱から発する何らかの運動」に他ならず、他方、熱せ
 られる働きとは、「熱への運動」に他ならない。ところで、定義とは種の性格
 を明示する。\footnote{ゆえに、これらの定義は、能動と受動の性格を明示し
 ている。}


\\



Et utroque modo actus humani, sive
considerentur per modum actionum, sive per modum passionum, a fine
speciem sortiuntur. Utroque enim modo possunt considerari actus humani,
eo quod homo movet seipsum, et movetur a seipso. Dictum est autem supra
quod actus dicuntur humani, inquantum procedunt a voluntate
deliberata. Obiectum autem voluntatis est bonum et finis. Et ideo
manifestum est quod principium humanorum actuum, inquantum sunt humani,
est finis. 


&

そして人間的行為は、能動として考えられても受動として考えられても、どちら
 のしかたでも、目的から種を獲得する。じっさい、人間は自ら
 を動かし、また、自らによって動かされるがゆえに、人間的行為はこの両方のしかたで考え
 られうる。
ところで、行為が「人間的」と言われるのは、思量された意志から出てくる限り
 においてであることが、前に述べられた。
ところで、意志の対象は、善であり、目的である。
ゆえに、人間的行為の始原が、それが人間的である限りにおいて、目的であるこ
 とは明らかである。


\\



Et similiter est terminus eorundem, nam id ad quod terminatur
actus humanus, est id quod voluntas intendit tanquam finem; sicut in
agentibus naturalibus forma generati est conformis formae generantis. 

&


同様に、それら[=人間的行為]の終点も同様である。なぜなら、人間的行為が
 そこで終わるところとは、意志が目的として目指すところだからである。
ちょうどそれは、自然的作用者において、生み出されたものの形相が、生み出す
 ものの形相に一致するのと同様である。


\\

Et
quia, ut Ambrosius dicit, {\itshape super Lucam}: {\itshape mores proprie dicuntur humani},
actus morales proprie speciem sortiuntur ex fine, nam idem sunt actus
morales et actus humani.

&





また、アンブロシウスが『ルカによる福音書注解』で言うように「道徳的なものが固有の意味で人間的と言わ
 れる」から、道徳的行為は、固有の意味で、目的から種を獲得する。道徳的行
 為と人間的行為は同じだからである。


\\


{\scshape Ad primum ergo dicendum} quod finis non
est omnino aliquid extrinsecum ab actu, quia comparatur ad actum ut
principium vel terminus; et hoc ipsum est de ratione actus, ut scilicet
sit ab aliquo, quantum ad actionem, et ut sit ad aliquid, quantum ad
passionem.

&

第一異論に対しては、それゆえ、以下のように言われるべきである。
目的は、あらゆる点で、行為から外在的であるわけではない。なぜなら、行為に
 対して、始原や終点として関係するからである。
すなわち、能動にかんしては、「あるものから」ということが、受動にかんしては、
 「あるものへ」ということが、行為の性格には属している。



\\


{\scshape Ad secundum dicendum} quod finis secundum
quod est prior in intentione, ut dictum est, secundum hoc pertinet ad
voluntatem. Et hoc modo dat speciem actui humano sive morali.


&

第二異論に対しては、以下のように言われるべきである。
すでに述べられたとおり、目的は、意図においてより先であるかぎりで、意志に
 属する。そしてこのしかたで、人間的ないし道徳的行為に種を与える。


\\


{\scshape Ad tertium dicendum} quod idem actus
numero, secundum quod semel egreditur ab agente, non ordinatur nisi ad
unum finem proximum, a quo habet speciem, sed potest ordinari ad plures
fines remotos, quorum unus est finis alterius. 



&

第三異論に対しては、以下のように言われるべきである。
数的に同一の行為は、行為者から同時に出てくるかぎり、一つの最近接目的
 にしか秩序付けられず、その目的から、種をもつ。しかし、それらの一つは他
 の目的であるといった、複数の遠
 隔目的へ秩序付けられることは可能である。


\\

--Possibile tamen est quod
unus actus secundum speciem naturae, ordinetur ad diversos fines
voluntatis, sicut hoc ipsum quod est occidere hominem, quod est idem
secundum speciem naturae, potest ordinari sicut in finem ad
conservationem iustitiae, et ad satisfaciendum irae. Et ex hoc erunt
diversi actus secundum speciem moris, quia uno modo erit actus virtutis,
alio modo erit actus vitii. 


&


 しかし、自然的な種において一つである行為が、意志のさまざまな目的へ秩序
 づけられることは可能である。たとえば、人間を殺すということは、自然の種
 において同一でありながら、目的として、正義を保持することと、怒りを満足させることへ
 と、秩序づけられうる。そして、このことに基づいて、道徳の種に基づいて、
 さまざまな行為があることになる。なぜなら、一方のしかたでは、その行為は
 有徳だが、他方のしかたでは、その行為は悪徳だからである。



\\

Non enim motus recipit speciem ab eo quod
est terminus per accidens, sed solum ab eo quod est terminus per
se. Fines autem morales accidunt rei naturali; et e converso ratio
naturalis finis accidit morali. Et ideo nihil prohibet actus qui sunt
iidem secundum speciem naturae, esse diversos secundum speciem moris, et
e converso.

&

じっさい、 運動が、附帯的に終点であるものから種を獲得することはなく、ただ、自体的に終点
 であるものからだけである。ところで、道徳の目的は、自然物にとって附帯的
 である。また逆に、自然的な性格は、道徳的なことにとって附帯的である。ゆ
 えに、自然の種において同一である諸行為が、道徳の種においてさまざまであっ
 たり、その逆であったりすることはなんら差し支えない。



\end{longtable}
\newpage


\rhead{a.~4}
\begin{center}
 {\Large {\bf ARTICULUS QUARTUS}}\\
 {\large UTRUM SIT ALIQUIS ULTIMUS FINIS HUMANAE VITAE}\\
 {\footnotesize II {\itshape Metaphys.}, lect.~4; I {\itshape Ethic.}, lect.~2.}\\
 {\Large 第四項\\人間の生には何らかの究極目的があるか}
\end{center}

\begin{longtable}{p{21em}p{21em}}


{\Huge A}{\scshape d quartum sic proceditur}. Videtur quod non sit
aliquis ultimus finis humanae vitae, sed procedatur in finibus in
infinitum. Bonum enim, secundum suam rationem, est diffusivum sui; ut
patet per Dionysium, {\scshape iv} cap.~{\itshape de Div.~Nom}. Si ergo
quod procedit ex bono, ipsum etiam est bonum, oportet quod illud bonum
diffundat aliud bonum, et sic processus boni est in infinitum. Sed bonum
habet rationem finis. Ergo in finibus est processus in infinitum.

&

第四項の問題へ、議論は以下のように進められる。
人間の生には、なんら、究極目的はなく、目的の中を無限に進むと思われる。
理由は以下の通り。
ディオニュシウス『神名論』第4章によって明らかなとおり、善は、それ自身の性格に
 したがって、自らを他へ及ぼしうるものである。
ゆえに、もし何かが善から発出すれば、それ自身もまた善であり、その善は他の
 善を流出させるはずである。そして善の発出は無限に進む。
ところで、善は目的の性格を持つ。ゆえに、諸目的の中に、無限の進行がある。

\\


{\scshape 2 Praeterea}, ea quae sunt rationis, in
infinitum multiplicari possunt, unde et mathematicae quantitates in
infinitum augentur. Species etiam numerorum propter hoc sunt infinitae,
quia, dato quolibet numero, ratio alium maiorem excogitare potest. Sed
desiderium finis sequitur apprehensionem rationis. Ergo videtur quod
etiam in finibus procedatur in infinitum.

&


さらに、理性に属することは、無限に多数化されうる。
このことから、数学的量もまた、無限に増やされる。
また、数の種も、このために、無限である。というのも、どんな数が与えられて
 も、理性はより大きな他の数を考え出すことができる。
ところが、目的への願望は、理性の把握に後続する。ゆえに、諸目的においても、
 無限に進行されると思われる。

\\


{\scshape 3 Praeterea}, bonum et finis est obiectum
voluntatis. Sed voluntas infinities potest reflecti supra seipsam,
possum enim velle aliquid, et velle me velle illud, et sic in
infinitum. Ergo in finibus humanae voluntatis proceditur in infinitum,
et non est aliquis ultimus finis humanae voluntatis.

&


さらに、善と目的は、意志の対象である。
ところが、意志は、無限回、自分自身の上に反射されうる。
じっさい、わたしたちは、あることを意志し、私がそれを意志することを意志し、
 そのように無限に意志することができる。
ゆえに、人間の意志の諸目的において、無限に進行されるのであって、何らかの、人間の意
 志の究極目的が存在するのではない。

\\


{\scshape Sed contra est} quod Philosophus dicit,
II {\itshape Metaphys}., quod {\itshape qui infinitum faciunt, auferunt naturam boni}. Sed
bonum est quod habet rationem finis. Ergo contra rationem finis est quod
procedatur in infinitum. Necesse est ergo ponere unum ultimum finem.

&

しかし反対に、哲学者は、『形而上学』第2巻で、「無限に行う人は、善の本性
 を取り除く」と言っている。ところが、善とは目的の性格を持つものである。
 ゆえに、無限に進むことは目的の性格に反する。ゆえに、一つの究極目的を措
 定することが必要である。


\\


{\scshape Respondeo dicendum} quod, per se loquendo,
impossibile est in finibus procedere in infinitum, ex quacumque
parte. In omnibus enim quae per se habent ordinem ad invicem, oportet
quod, remoto primo, removeantur ea quae sunt ad primum. Unde philosophus
probat, in VIII {\itshape Physic}., quod non est possibile in causis moventibus
procedere in infinitum, quia iam non esset primum movens, quo subtracto
alia movere non possunt, cum non moveant nisi per hoc quod moventur a
primo movente. 


&

解答する。以下のように言われるべきである。
自体的に語るならば、どの側からも、諸目的の中を無限に進むことは不可能であ
 る。
理由は以下の通り。
自体的に、相互に秩序をもつものすべてのものにおいて、第一のものが取り除か
 れると、その第一のもののためにあるものどもも取り除かれる。
このことから、哲学者は『自然学』第8巻で、動因の中を無限に進行することが
 不可能であることを、以下のように証明している。すなわち、[もし無限に進
 行したら]その場合、第一の動者が存在しないことになり、それが存在しない
 と、他のものも動かすことができない。なぜなら、それらは、第一の動者によっ
 て動かされることを通してでなければ、動かさないからである。


\\



In finibus autem invenitur duplex ordo, scilicet ordo
intentionis, et ordo executionis, et in utroque ordine oportet esse
aliquid primum. Id enim quod est primum in ordine intentionis est quasi
principium movens appetitum, unde, subtracto principio, appetitus a
nullo moveretur. Id autem quod est principium in executione, est unde
incipit operatio, unde, isto principio subtracto, nullus inciperet
aliquid operari. 



&

さて、諸目的の中には、二通りの秩序が見出される。それは、意図の秩序と実行
 の秩序であり、そのどちらの中にも、何か第一のものがなければならない。
その理由は以下の通りである。意図の秩序の中で第一のものは、欲求を動かす根
 源のようなものであり、したがって、その根源が取り除かれると、欲求は何に
 よっても動かされない。他方、実行の中での根源は、働きをそこから始めると
 ころであり、それゆえ、その根源が取り除かれると、だれもなにかを働き始める
 ことがなかったであろう。


\\


Principium autem intentionis est ultimus finis,
principium autem executionis est primum eorum quae sunt ad finem. Sic
ergo ex neutra parte possibile est in infinitum procedere, quia si non
esset ultimus finis, nihil appeteretur, nec aliqua actio terminaretur,
nec etiam quiesceret intentio agentis; si autem non esset primum in his
quae sunt ad finem, nullus inciperet aliquid operari, nec terminaretur
consilium, sed in infinitum procederet. 



&

意図の根源は、究極目的であり、実行の根源は、目的のためにあるものどものう
 ちの第一のものである。それゆえ、このようにして、どちらの側からも無限に
 進むことは不可能である。なぜなら、もし究極目的がなかったら、なにも欲求
 されず、どの作用も終わらず、作用者の意図も鎮まることがなかっただろうか
 ら。また、もし、目的のためにあるものどものうち、第一のものがなかったな
 らば、だれも何かの働きを始めることがなく、思量は終わらず、無限に進んだ
 だろうからである。



\\



Ea vero quae non habent ordinem
per se, sed per accidens sibi invicem coniunguntur, nihil prohibet
infinitatem habere, causae enim per accidens indeterminatae sunt. Et hoc
etiam modo contingit esse infinitatem per accidens in finibus, et in his
quae sunt ad finem.

&

他方、自体的な秩序を持たず、附帯的に相互に結びつけられたものどものが、無
 限性を持つことはなんら差し障りはない。諸原因は、附帯的な意味であれば、無際限だから
 である。そしてこの意味でなら、諸目的の中でも、目的のためにあるものども
 の中でも、無限が生じうる。


\\


{\scshape Ad primum ergo dicendum} quod de ratione
boni est quod aliquid ab ipso effluat, non tamen quod ipsum ab alio
procedat. Et ideo, cum bonum habeat rationem finis, et primum bonum sit
ultimus finis, ratio ista non probat quod non sit ultimus finis; sed
quod a fine primo supposito procedatur in infinitum inferius versus ea
quae sunt ad finem. Et hoc quidem competeret, si consideraretur sola
virtus primi boni, quae est infinita. 

&

第一異論に対しては、それゆえ、以下のように言われるべきである。
善の性格には「その善から何かが出ていく」ことは含まれるが、「その善が他の
 ものから出てくる」ことが含まれるわけではない。
ゆえに、善は目的の性格を持ち、第一前は究極目的だから、その論[=第一異論]
 は、究極目的が存在しないことではなく、想定された第一の目的から、目的の
 ためにあるものどもの方へ、下に向かって、無限に進むことを証明している。
そして、このことは、もし第一善の力だけが考察されるのであれば、それは無限なの
 だから、不都合ではなかっただろう。


\\

Sed quia primum bonum habet
diffusionem secundum intellectum, cuius est secundum aliquam certam
formam profluere in causata; aliquis certus modus adhibetur bonorum
effluxui a primo bono, a quo omnia alia bona participant virtutem
diffusivam. Et ideo diffusio bonorum non procedit in infinitum, sed,
sicut dicitur {\itshape Sap}.~{\scshape xi}, Deus omnia disposuit {\itshape in numero, pondere et
mensura}.

&

しかし、第一善は、知性に従った流出を行う。そしてそれは、原因されたものへ、
 ある一定の形相に従って流れ出るということである。ある特定の限度が、第一
 善からの諸善の流出に加えられていて、その第一善から、他のすべての諸善は、
 他へと及ぶ力を分有する。ゆえに、諸善が他へと及ぶことは、無限に進行せず、
 むしろ、『知恵の書』11章で言われるように、神はすべてを「数、重さ、限度
 において」配分した。


\\


{\scshape Ad secundum dicendum} quod in his quae
sunt per se, ratio incipit a principiis naturaliter notis, et ad aliquem
terminum progreditur. Unde philosophus probat, in I {\itshape Poster}., quod in
demonstrationibus non est processus in infinitum, quia in
demonstrationibus attenditur ordo aliquorum per se ad invicem
connexorum, et non per accidens. In his autem quae per accidens
connectuntur, nihil prohibet rationem in infinitum procedere. Accidit
autem quantitati aut numero praeexistenti, inquantum huiusmodi, quod ei
addatur quantitas aut unitas. Unde in huiusmodi nihil prohibet rationem
procedere in infinitum.

&

第二異論に対しては、以下のように言われるべきである。
自体的であるものどもについて、理性は、自然本性的に知られた諸原理から始め、
 何らかの終点へと進んで行く。このことから、哲学者は『分析論後書』第1巻で、
 論証の中に、無限進行がないことを証明している。なぜなら、論証においては、
 附帯的にではなく、自体的に連結しあった、あるものどもの秩序が見
 出されるからである。
これに対して、附帯的に連結するものどもにおいては、理性が無限に進行するこ
 とをなにも妨げない。
ところで、先に与えられた量や数に、量や一が加えられることは、量や数である
 かぎりでのそれらにとって、附帯的なことである。
したがって、このようなものにおいては、理性が無限に進行することを、なにも
 妨げない。


\\


{\scshape Ad tertium dicendum} quod illa
multiplicatio actuum voluntatis reflexae supra seipsam, per accidens se
habet ad ordinem finium. Quod patet ex hoc, quod circa unum et eundem
finem indifferenter semel vel pluries supra seipsam voluntas
reflectitur.

&

第三異論に対しては、以下のように言われるべきである。
そのような、自分自身へ反射された意志の作用の多数化は、諸目的の秩序にたい
 しては、附帯的に関係する。このことは、一つの同一の目的をめぐって、一度
 だけなのか、あるいは複数回なのか、その差別なく、意志は自分自身を顧みる
ことから明らかである。




\end{longtable}
\newpage


\rhead{a.~5}
\begin{center}
 {\Large {\bf ARTICULUS QUINTUS}}\\
 {\large UTRUM UNIUS HOMINIS POSSINT ESSE PLURES ULTIMI FINES}\\
 {\Large 第五項\\一人の人間に複数の究極目的がありうるか}
\end{center}

\begin{longtable}{p{21em}p{21em}}

{\Huge A}{\scshape d quintum sic proceditur}. Videtur quod
possibile sit voluntatem unius hominis in plura ferri simul, sicut in
ultimos fines. Dicit enim Augustinus, XIX {\itshape de Civ.~Dei}, quod quidam
ultimum hominis finem posuerunt in quatuor, scilicet {\itshape in voluptate, in
quiete, in primis naturae, et in virtute}. Haec autem manifeste sunt
plura. Ergo unus homo potest constituere ultimum finem suae voluntatis
in multis.


&

第五項の問題へ、議論は以下のように進められる。
一人の人間の意志が、究極目的としての複数のものへ、同時に、動かされることは
 可能だと思われる。理由は以下の通り。
アウグスティヌスは、『神の国』19巻で、ある人々は、人間の究極目的を、四つ
 のもの、すなわち「快楽、安静、自然の第一のものども、徳に」置いたと言っ
 ている。ところが、これらは明らかに複数である。ゆえに、一人の人間が、多
 くのものの中に、自分の意志の究極目的を立てることは可能である。


\\


{\scshape 2 Praeterea}, ea quae non opponuntur ad
invicem, se invicem non excludunt. Sed multa inveniuntur in rebus quae
sibi invicem non opponuntur. Ergo si unum ponatur ultimus finis
voluntatis, non propter hoc alia excluduntur.


&

さらに、相互に対立しないものは、相互に他を排除しない。ところが、事物の中
 の多くのものは、相互に対立しない。ゆえに、もし一つのものが意志の究極目
 的だとされても、そのために、他のものが排除されるわけではない。


\\


{\scshape 3 Praeterea}, voluntas per hoc quod
constituit ultimum finem in aliquo, suam liberam potentiam non
amittit. Sed antequam constitueret ultimum finem suum in illo, puta in
voluptate, poterat constituere finem suum ultimum in alio, puta in
divitiis. Ergo etiam postquam constituit aliquis ultimum finem suae
voluntatis in voluptate, potest simul constituere ultimum finem in
divitiis. Ergo possibile est voluntatem unius hominis simul ferri in
diversa, sicut in ultimos fines.


&

さらに、意志は、何かを究極目的とすることによって、自分の自由な能力
 を捨てることはない。ところが、あるもの、たとえば快楽を自分の究極目的と
 する前、自分の究極目的を、別のもの、たとえば富とすることができた。
ゆえに、自分の意志の究極目的を快楽としたあとも、同時に、究極目的を富とす
 ることができる。ゆに、一人の人間の意志が、究極目的としてのさまざまなも
 のへと、同時に動かされることが可能である。


\\


{\scshape Sed contra}, illud in quo quiescit
aliquis sicut in ultimo fine, hominis affectui dominatur, quia ex eo
totius vitae suae regulas accipit. Unde de gulosis dicitur {\itshape Philipp}.~{\scshape iii}:
{\itshape Quorum Deus venter est}, quia scilicet constituunt ultimum finem in
deliciis ventris. Sed sicut dicitur {\itshape Matth}.~{\scshape vi}, {\itshape nemo potest duobus
dominis servire}, ad invicem scilicet non ordinatis. Ergo impossibile est
esse plures ultimos fines unius hominis ad invicem non ordinatos.


&


しかし反対に、ある人が、究極目的において憩うように、なにかにおいて憩うと
 き、それは人間の感情を支配する。なぜなら、彼はそこから、自分の人生全体の規
 範を受け取るからである。したがって、『フィリピの信徒への手紙』第3章で、
 贅沢な人々について「神は彼らの腹である」\footnote{「彼らは腹を神とし」
 (3:19)}と言われているが、それは、究極目的を、腹の喜びとしているからであ
 る。しかし、『マタイによる福音書』第6章で言われているとおり、「だれも二人の主人に仕えること
 はできない」\footnote{(6:13)}、すなわち、相互に秩序付けられていない二つ
 のものに仕えることはできない。ゆえに、相互に秩序付けられていない複数の
 究極目的が、一人の人間にあることは不可能である。

\\


{\scshape Respondeo dicendum} quod impossibile est
quod voluntas unius hominis simul se habeat ad diversa, sicut ad ultimos
fines. Cuius ratio potest triplex assignari. Prima est quia, cum
unumquodque appetat suam perfectionem, illud appetit aliquis ut ultimum
finem, quod appetit, ut bonum perfectum et completivum sui ipsius. Unde
Augustinus dicit, XIX {\itshape de Civ.~Dei}: {\itshape finem boni nunc dicimus, non quod
consumatur ut non sit, sed quod perficiatur ut plenum sit}. Oportet
igitur quod ultimus finis ita impleat totum hominis appetitum, quod
nihil extra ipsum appetendum relinquatur. Quod esse non potest, si
aliquid extraneum ad ipsius perfectionem requiratur. Unde non potest
esse quod in duo sic tendat appetitus, ac si utrumque sit bonum
perfectum ipsius. 


&

解答する。一人の人間の意志が、同時に、様々なものに、究極目的として関係す
 ることは不可能である。この理由は、三通りに指定されうる。第一の理由は以
 下の通りである。各々のものは、自らの完成を欲求するので、だれかが究極目
 的として欲求するものとは、完全な善、そしてその人自身を完成させうる善とし
 て、その人が欲求するものである。
 このことから、アウグスティヌスは『神の国』第19巻で、「私たちが今、善の
 終極と言うのは、燃え尽きて存在しなくなるからではなく、完成されて充満す
 るからである」と述べる。ゆえに、究極目的は、それ以外に欲求されうるもの
 がなにも残らないように人間の全欲求を満たすのでなければならない。もし、
 何か外的なものが彼の完成に必要とされるならば、こういうことはありえなかっ
 たであろう。したがって、欲求が、二つのものへ、あたかもその両方がそれの
 完全な善であるかのように向かうことは不可能である。
 

\\


Secunda ratio est quia, sicut in processu rationis
principium est id quod naturaliter cognoscitur, ita in processu
rationalis appetitus, qui est voluntas, oportet esse principium id quod
naturaliter desideratur. Hoc autem oportet esse unum, quia natura non
tendit nisi ad unum. Principium autem in processu rationalis appetitus
est ultimus finis. Unde oportet id in quod tendit voluntas sub ratione
ultimi finis, esse unum. 



&

第二の理由は以下の通りである。理性の進行において、始まりは、自然本性的に
 認識されるものであるように、理性的欲求、すなわち意志の進行においても、
始まりが存在し、それは自然本性的に願望されるのでなければならない。ところ
 で、それは、一つでなければならない。なぜなら、自然は一つのものにしか向
 かわないからである。ところで、理性的欲求の進行の始まりは、究極目的であ
 る。したがって、意志が、究極目的という性格のもとで向かうものは、一つで
 なければならない。

 
\\

Tertia ratio est quia, cum actiones voluntarie
ex fine speciem sortiantur, sicut supra habitum est, oportet quod a fine
ultimo, qui est communis, sortiantur rationem generis, sicut et
naturalia ponuntur in genere secundum formalem rationem communem. Cum
igitur omnia appetibilia voluntatis, inquantum huiusmodi, sint unius
generis, oportet ultimum finem esse unum. Et praecipue quia in quolibet
genere est unum primum principium, ultimus autem finis habet rationem
primi principii, ut dictum est. 


&

 第三の理由は以下の通りである。
前に論じられたとおり、意志の作用は目的から種を獲得するので、ちょうど、自
 然的諸事物が、共通な形相的性格にしたがって類の中に置かれるように、共通
 なものである究極目的によって、類の性格を獲得するのでなければならない。
ゆえに、意志によって欲求されうるすべてのものが、そのようなものである限り
 において、一つの類に属するから、究極目的は一つでなければならない。
 とくに、どんな類においても、一つの第一の根源があるから、前に述べられた
 とおり、究極目的は、第一の根源という性格を持つ。


\\

Sicut autem se habet ultimus finis
hominis simpliciter ad totum humanum genus, ita se habet ultimus finis
huius hominis ad hunc hominem. Unde oportet quod, sicut omnium hominum
est naturaliter unus finis ultimus, ita huius hominis voluntas in uno
ultimo fine statuatur.


&


 ところで、人間の究極目的が、端的に、全人類に対してもつ関係は、この人
 間の究極目的が、この人間に対してもつ関係に等しい。したがって、すべての
 人間に、自然本性的に一つの究極目的があるように、この人間の意志は、一つ
 の究極目的において設定されている。\footnote{このパラグラフは、第三の理
 由が、欲求されうるもの全体が、一つの類に属さなければならず、
それが、端的な「人間」や「人類」の究極目的であることを示すだけなので、そ
 れを一人の人間のレベルに下ろすための議論。}

 
\\


{\scshape Ad primum ergo dicendum} quod omnia illa
plura accipiebantur in ratione unius boni perfecti ex his constituti, ab
his qui in eis ultimum finem ponebant.


&


第一異論に対しては、それゆえ、以下のように言われるべきである。
それらの中に究極目的を置いた人々によって、それら複数のすべてのものは、そ
 れらから構成される一つの完全な善の性格において理解されている。

\\


{\scshape Ad secundum dicendum} quod, etsi plura
accipi possint quae ad invicem oppositionem non habeant, tamen bono
perfecto opponitur quod sit aliquid de perfectione rei extra ipsum.


&

第二異論に対しては、以下のように言われるべきである。
相互に対立性をもたない複数のものが理解されうるとしても、しかし、完全な善
 の外にある、事物の完全性に属する何かは、完全な善に対立する。


\\


{\scshape Ad tertium dicendum} quod potestas
voluntatis non habet ut faciat opposita esse simul. Quod contingeret, si
tenderet in plura disparata sicut in ultimos fines, ut ex dictis patet.


&


第三異論に対しては、以下のように言われるべきである。
意志の力では、対立するのが同時に存在するようにすることはできない。
すでに述べられたことから明らかなとおり、もし、意志が、複数のさまざまなもの
 に、究極目的として向かっていたならば、このことは生じたであろう。



\end{longtable}
\newpage




\rhead{a.~6}
\begin{center}
 {\Large {\bf ARTICULUS SEXTUS}}\\
 {\large UTRUM HOMO OMNIA QUAE VULT, VELIT PROPTER ULTIMUM FINEM}\\
 {\footnotesize IV {\itshape Sent.}, d.49, q.1, a.3, qu$^a$ 4; {\itshape
 SCG}, cap.101.}\\
 {\Large 第六項\\人間は意志するものをすべて、究極目的のために意志するか}
\end{center}

\begin{longtable}{p{21em}p{21em}}

{\Huge A}{\scshape d sextum sic proceditur}. Videtur quod
non omnia quaecumque homo vult, propter ultimum finem velit. Ea enim
quae ad finem ultimum ordinantur, seriosa dicuntur, quasi utilia. Sed
iocosa a seriis distinguuntur. Ergo ea quae homo iocose agit, non
ordinat in ultimum finem.


&

第六項の問題へ、議論は以下のように進められる。
人間は、何であれ意志する事柄のすべてを、究極目的のために意志するわけでは
 ないと思われる。理由は以下の通り。
究極目的のために秩序付けられるものは、有用なものとして、真面目なものと言
 われる。ところが、ふざけたものは真面目なものから区別される。ゆえに、人
 間がふざけて行うことは、究極目的へ秩序付けられていない。

\\


{\scshape 2 Praeterea}, philosophus dicit, in
principio {\itshape Metaphys}., quod scientiae speculativae propter seipsas
quaeruntur. Nec tamen potest dici quod quaelibet earum sit ultimus
finis. Ergo non omnia quae homo appetit, appetit propter ultimum finem.


&

さらに、哲学者は『形而上学』の初めで、観照的学は、それ自身のために探究さ
 れる、と述べている。しかし、それらのどれも、究極目的とは言われえない。
 ゆえに、人間は、が欲求するすべてのものを、究極目的のために欲求するわけ
 ではない。

\\


{\scshape 3 Praeterea}, quicumque ordinat aliquid in
finem aliquem, cogitat de illo fine. Sed non semper homo cogitat de
ultimo fine in omni eo quod appetit aut facit. Non ergo omnia homo
appetit aut facit propter ultimum finem.


&


さらに、だれでも、何かを何らかの目的へ秩序付ける人は、その目的について考
 えている。ところが、人間は、欲求したり行ったりするすべての事柄に
 おいて、常に、究極目的について考えているわけではない。ゆえに、すべての
 事柄を、人間は、究極目的のために欲求したり行ったりするわけではない。

\\


{\scshape Sed contra est} quod dicit Augustinus,
XIX {\itshape de Civ.~Dei}: {\itshape illud est finis boni nostri, propter quod amantur
cetera, illud autem propter seipsum}.


&


しかし反対に、アウグスティヌスは、『神の国』19巻で「それのために他のもの
 が愛されるが、それはそれ自身のために愛されるものが、私たちにとって、善
 の終極である」と述べている。


\\


{\scshape Respondeo dicendum} quod necesse est quod
omnia quae homo appetit, appetat propter ultimum finem. Et hoc apparet
duplici ratione. Primo quidem, quia quidquid homo appetit, appetit sub
ratione boni. Quod quidem si non appetitur ut bonum perfectum, quod est
ultimus finis, necesse est ut appetatur ut tendens in bonum perfectum,
quia semper inchoatio alicuius ordinatur ad consummationem ipsius; sicut
patet tam in his quae fiunt a natura, quam in his quae fiunt ab arte. Et
ideo omnis inchoatio perfectionis ordinatur in perfectionem consummatam,
quae est per ultimum finem. 


&

解答する。以下のように言われるべきである。
人間は、欲求するすべての事柄を、究極目的のために欲求すると言うことが必然
 である。このことは、二通りの理由で明らかである。
一つには、人間は、なんであれ欲求するものを、善の性格のもとに欲求する。
それが、完全な善、つまり究極目的として欲求されない場合、完全な善へ向かう
 ものとして欲求されることが必然である。なぜなら、技術に
 よって作られるものと同様に、自然によって生じるものにおいても明らかなと
 おり、何かの始まりは、常に、それの成就へ秩序付けられているからである。
ゆえに、すべての完成の始まりは、成就された完成へと秩序付けられているが、
 この成就された完成は究極目的による。



\\


Secundo, quia ultimus finis hoc modo se
habet in movendo appetitum, sicut se habet in aliis motionibus primum
movens. Manifestum est autem quod causae secundae moventes non movent
nisi secundum quod moventur a primo movente. Unde secunda appetibilia
non movent appetitum nisi in ordine ad primum appetibile, quod est
ultimus finis.

&

第二に、究極目的は、欲求を動かすことに対して、あたかも第一動者が他の運動
 においてあるのと同じ関係にある。ところで、第二動因は、第一動者に動かさ
 れることにしたがってでなければ動かさないことは明らかである。ゆえに、第
 二に欲求されうるものは、第一に欲求されうるもの、つまり究極目的への秩序
 においてでなければ、欲求を動かさない。


\\


{\scshape Ad primum ergo dicendum} quod actiones
ludicrae non ordinantur ad aliquem finem extrinsecum; sed tamen
ordinantur ad bonum ipsius ludentis, prout sunt delectantes vel requiem
praestantes. Bonum autem consummatum hominis est ultimus finis eius.


&


第一異論に対しては、それゆえ、以下のように言われるべきである。
遊びの活動は、何らかの外的な目的へ秩序付けられないが、しかし、喜んだり、
 休息をとったりする人としての、遊ぶ人自身の善に秩序付けられている。
しかし、人間の成就した善は、その人の究極目的である。

\\


Et similiter dicendum ad secundum, de
scientia speculativa; quae appetitur ut bonum quoddam speculantis, quod
comprehenditur sub bono completo et perfecto, quod est ultimus finis.


&

第二異論の観照的学に対しても同様に言われるべきである。
その学は、観照する者のある種の善として欲求されるが、その善は、究極目的で
 ある完全で完成された善のもとに含まれる。


\\


{\scshape Ad tertium dicendum} quod non oportet ut
semper aliquis cogitet de ultimo fine, quandocumque aliquid appetit vel
operatur, sed virtus primae intentionis, quae est respectu ultimi finis,
manet in quolibet appetitu cuiuscumque rei, etiam si de ultimo fine actu
non cogitetur. Sicut non oportet quod qui vadit per viam, in quolibet
passu cogitet de fine.


&


第三異論に対しては、以下のように言われるべきである。
ある人が何かを欲求したり、働いたりするとき、どんな時でも、常に究極目的に
 ついて考えていなくてはならない、ということはない。しかし、究極目的に関
 係する第一の意図の力は、究極目的について現実に考えていないとしても、ど
 んな事物へのどんな欲求の中にもとどまっている。それはちょうど、道を行く
 人が、どの一歩においても目的について考えていなくてもよいのと同様である。


\end{longtable}
\newpage



\rhead{a.~7}
\begin{center}
 {\Large {\bf ARTICULUS SEPTIMUS}}\\
 {\large UTRUM SIT UNUS ULTIMUS FINIS OMNIUM HOMINUM}\\
 {\footnotesize Supra, a.5; I {\itshape Ethic.}, lect.9.}\\
 {\Large 第七項\\すべての人に一つの究極目的があるか}
\end{center}

\begin{longtable}{p{21em}p{21em}}

{\Huge A}{\scshape d septimum sic proceditur}. Videtur
quod non omnium hominum sit unus finis ultimus. Maxime enim videtur
hominis ultimus finis esse incommutabile bonum. Sed quidam avertuntur ab
incommutabili bono, peccando. Non ergo omnium hominum est unus ultimus
finis.


&

第七項の問題へ、議論は以下のように進められる。
すべての人に一つの究極目的があるわけではないと思われる。理由は以下の通り。
最大限に人間の究極目的だと思われるのは、不変の善である。しかし、ある人々
 は、罪を犯すことによって、不変の善から逸れていく。ゆえに、すべての人が
 一つの究極目的をもつのではないと思われる。


\\


{\scshape 2 Praeterea}, secundum ultimum finem tota
vita hominis regulatur. Si igitur esset unus ultimus finis omnium
hominum, sequeretur quod in hominibus non essent diversa studia
vivendi. Quod patet esse falsum.


&


さらに、究極目的に従って、人間の全人生が規制される。ゆえに、もしすべての
 人が一つの究極目的を持つならば、人々の中に、様々な人生の営みはなかった
 だろう。これは明らかに偽である。

\\


{\scshape 3 Praeterea}, finis est actionis
terminus. Actiones autem sunt singularium. Homines autem, etsi
conveniant in natura speciei, tamen differunt secundum ea quae ad
individua pertinent. Non ergo omnium hominum est unus ultimus finis.


&


さらに、目的は作用の端である。ところが、作用は個物に属する。しかし、人間
 は、種の本性において一致するが、個人に属することにおいては異なっている。
 ゆえに、すべての人が一つの究極目的を持つのではない。

\\


{\scshape Sed contra} est quod Augustinus dicit,
XIII {\itshape de Trin}., quod omnes homines conveniunt in appetendo ultimum finem,
qui est beatitudo.


&

しかし反対に、アウグスティヌスは『三位一体論』第13巻で、すべての人々は、
 究極目的、すなわち至福を欲求することにおいて一致すると述べている。


\\


{\scshape Respondeo dicendum} quod de ultimo fine
possumus loqui dupliciter, uno modo, secundum rationem ultimi finis;
alio modo, secundum id in quo finis ultimi ratio invenitur. Quantum
igitur ad rationem ultimi finis, omnes conveniunt in appetitu finis
ultimi, quia omnes appetunt suam perfectionem adimpleri, quae est ratio
ultimi finis, ut dictum est. 



&

解答する。以下のように言われるべきである。
究極目的について、わたしたちは二通りに語ることができる。
一つには、究極目的という性格に即してであり、もう一つは、究極目的という性
 格が見出されるものに即してである。
それゆえ、究極目的という性格にかんしては、すべての人々は、究極目的への欲
 求において一致する。なぜなら、すべての人は自らの完成が成就することを欲
 求するが、これは、すでに述べられたとおり、究極目的である。



\\



Sed quantum ad id in quo ista ratio
invenitur, non omnes homines conveniunt in ultimo fine, nam quidam
appetunt divitias tanquam consummatum bonum, quidam autem voluptatem,
quidam vero quodcumque aliud. Sicut et omni gustui delectabile est
dulce, sed quibusdam maxime delectabilis est dulcedo vini, quibusdam
dulcedo mellis, aut alicuius talium. 



&

しかし、この性格が見出されるものにかんしては、すべての人が究極目的におい
 て一致するわけではない。なぜなら、ある人々は成就された善として富を欲求
 し、ある人々は快を、またある人々は他のものを欲求するからである。それは
 ちょうど、すべての舌にとって、甘さが心地よいが、ある人々にはワインの甘
 さが、他の人々には蜂蜜の甘さが、あるいはそのような何かの甘さが、最も心
 地よいのと同様である。


\\



Illud tamen dulce oportet esse
simpliciter melius delectabile, in quo maxime delectatur qui habet
optimum gustum. Et similiter illud bonum oportet esse completissimum,
quod tanquam ultimum finem appetit habens affectum bene dispositum.


&

しかし、最善の味覚をもつ人が、最も心地よい甘さが、端的に、より心地よいの
 でなければならない。同様に、よく整えられた欲求をもつ人が、究極目的とし
 て欲求する善が、もっとも完全なものでなければならない。


\\


{\scshape Ad primum ergo dicendum} quod illi qui
peccant, avertuntur ab eo in quo vere invenitur ratio ultimi finis, non
autem ab ipsa ultimi finis intentione, quam quaerunt falso in aliis
rebus.


&

第一異論に対しては、それゆえ、以下のように言われるべきである。
罪を犯す人々は、究極目的の性格が、真に見出されるものから逸れているが、究
 極目的の意図それ自体から逸れているのではなく、それを誤って他の事物の中
 に求めている。


\\


{\scshape Ad secundum dicendum} quod diversa studia
vivendi contingunt in hominibus propter diversas res in quibus quaeritur
ratio summi boni.


&

第二異論に対しては、以下のように言われるべきである。
様々な人生の営みが人々の中に生じているのは、最高善の性格が求められる様々
 な事物のためである。


\\


{\scshape Ad tertium dicendum} quod, etsi actiones
sint singularium, tamen primum principium agendi in eis est natura, quae
tendit ad unum, ut dictum est.


&


第三異論に対しては、以下のように言われるべきである。
作用は個物に属するが、しかし、それらにおける、作用の第一根源は、本性であ
 り、それは、すでに述べられたとおり、一つのものへ向かう。


\end{longtable}
\newpage





\rhead{a.~8}
\begin{center}
 {\Large {\bf ARTICULUS OCTAVUS}}\\
 {\large UTRUM IN ILLO ULTIMO FINE ALIAE CREATURAE COVENIANT}\\
 {\footnotesize Part.I, q.103, a.2; II {\itshape Sent.}, d.38, a.1, 2;
 III {\itshape SCG}, cap.17, 25; {\itshape De Verit.}, q.5, a.6, ad 4.}\\
 {\Large 第八項\\かの究極目的において、他の被造物は一致するか}
\end{center}

\begin{longtable}{p{21em}p{21em}}


{\Huge A}{\scshape d octavum sic proceditur}. Videtur quod
in ultimo fine hominis etiam omnia alia conveniant. Finis enim respondet
principio. Sed illud quod est principium hominum, scilicet Deus, est
etiam principium omnium aliorum. Ergo in ultimo fine hominis omnia alia
communicant.


&

第八項の問題へ、議論は以下のように進められる。
人間の究極目的において、他のすべてのものも一致すると思われる。
理由は以下の通り。
終わりは始めに対応する。ところが、人間の初めであるもの、つまり神は、他の
 すべてのものの始めでもある。ゆえに、人間の究極目的において、他のすべて
 のものは共通する。


\\


{\scshape 2 Praeterea}, Dionysius dicit, in libro {\itshape de
Div.~Nom}., quod {\itshape Deus convertit omnia ad seipsum, tanquam ad ultimum
finem}. Sed ipse est etiam ultimus finis hominis, quia solo ipso fruendum
est, ut Augustinus dicit. Ergo in fine ultimo hominis etiam alia
conveniunt.


&


さらに、ディオニュシウスは『神名論』という書物で「神は万物を、究極目的へ
 向けるように、自分の方へ向ける」と述べている。ところが、神は人間の究極
 目的でもある。なぜなら、アウグスティヌスが言うように、ただ神だけを享受
 すべきだからである。ゆえに、人間の究極目的において、他のものも一致する。

\\


{\scshape 3 Praeterea}, finis ultimus hominis est
obiectum voluntatis. Sed obiectum voluntatis est bonum universale, quod
est finis omnium. Ergo necesse est quod in ultimo fine hominis omnia
conveniant.


&

さらに、人間の究極目的は、意志の対象である。
ところが、意志の対象は普遍的な善であり、これは、万物の目的である。
ゆえに、人間の究極目的において、すべてのものが一致することは必然である。


\\


{\scshape Sed contra est} quod ultimus finis
hominum est beatitudo; quam omnes appetunt, ut Augustinus dicit. Sed {\itshape non
cadit in animalia rationis expertia ut beata sint}, sicut Augustinus
dicit in libro {\itshape Octoginta trium Quaest}. Non ergo in ultimo fine hominis
alia conveniunt.


&

しかし反対に、人間の究極目的は至福である。アウグスティヌスが言うように、
 すべての人間はそれを欲求する。しかし、アウグスティヌスが『八十三問題集』
 で述べるように、「理性を欠く動物の中に、至福であることは入ってこない」。
 ゆえに、人間の究極目的において、他のものが一致することはない。


\\


{\scshape Respondeo dicendum} quod, sicut philosophus
dicit in II {\itshape Physic}.~et in V {\itshape Metaphys}., finis dupliciter dicitur,
scilicet cuius, et quo, idest ipsa res in qua ratio boni invenitur, et
usus sive adeptio illius rei. Sicut si dicamus quod motus corporis
gravis finis est vel locus inferior ut res, vel hoc quod est esse in
loco inferiori, ut usus, et finis avari est vel pecunia ut res, vel
possessio pecuniae ut usus. 


&

解答する。以下のように言われるべきである。
哲学者が『自然学』第2巻と『形而上学』第5巻で述べているように、目的は二通
 りの意味で語られる。すなわち、cuiusとquo、つまり、善の性格が見出される
 事物それ自体と、その事物の使用あるいは適用とである。たとえば、私たちが、
重い物体の運動の目的は、事物としては、下方の場所であり、使用としては、下
 方の場所に在ることである、と言ったり、貪欲な人の目的は、事物としてはお
 金であり、使用としてはお金の所有である、と言うように。

\\



Si ergo loquamur de ultimo fine hominis
quantum ad ipsam rem quae est finis, sic in ultimo fine hominis omnia
alia conveniunt, quia Deus est ultimus finis hominis et omnium aliarum
rerum. 


&

ゆえに、もし、目的である事物自体にかんして、人間の究極目的について語るな
 らば、人間の究極目的において、他のすべてのものは一致する。なぜなら、人
 間の、そして他のすべての事物の究極目的は、神だからである。

\\





-- Si autem loquamur de ultimo fine hominis quantum ad consecutionem
finis, sic in hoc fine hominis non communicant creaturae
irrationales. Nam homo et aliae rationales creaturae consequuntur
ultimum finem cognoscendo et amando Deum, quod non competit aliis
creaturis, quae adipiscuntur ultimum finem inquantum participant aliquam
similitudinem Dei, secundum quod sunt, vel vivunt, vel etiam cognoscunt.


&

 しかし、目的の獲得にかんして、人間の究極目的について語るならば、その目
 的においては、人間は非理性的被造物と共通しない。なぜなら、人間と、その
 他の理性的被造物は、神を認識し、愛することで、究極目的を獲得するが、こ
のようなことは、その他の被造物に適合せず、それらの被造物は、存在するこ
 とや、生きることや、さらには認識するという点で、神の何らかの類似を分
 有する限りにおいて、究極目的を獲得するからである。

\\


Et per hoc patet responsio ad obiecta.


&

 以上のことによって、異論への解答は明らかである。

\end{longtable}

\end{document}