\documentclass[10pt]{jsarticle}
\usepackage{okumacro}
\usepackage{longtable}
\usepackage[polutonikogreek,english,japanese]{babel}
\usepackage{latexsym}
\usepackage{color}

%----- header -------
\usepackage{fancyhdr}
\pagestyle{fancy}
\lhead{{\it Summa Theologiae} I-II, q.59}
%--------------------

\bibliographystyle{jplain}

\title{{\bf PRIMA SECUNDAE}\\{\HUGE Summae Theologiae}\\Sancti Thomae
Aquinatis\\{\sffamily QUEAESTIO QUINQUAGESIMANONA}\\DE COMPARATIONE VIRTUTIS MORALIS AD PASSIONEM}
\author{Japanese translation\\by Yoshinori {\sc Ueeda}}
\date{Last modified \today}

%%%% コピペ用
%\rhead{a.~}
%\begin{center}
% {\Large {\bf }}\\
% {\large }\\
% {\footnotesize }\\
% {\Large \\}
%\end{center}
%
%\begin{longtable}{p{21em}p{21em}}
%
%&
%
%\\
%\end{longtable}
%\newpage

\begin{document}

\maketitle
\thispagestyle{empty}

\begin{center}
{\LARGE 『神学大全』第二部の一}\\
{\Large 第五十九問\\道徳的徳と情念の関係について}
\end{center}

\begin{longtable}{p{21em}p{21em}}
 Deinde considerandum est de distinctione moralium virtutum ad
 invicem. Et quia virtutes morales quae sunt circa passiones,
 distinguuntur secundum diversitatem passionum, oportet primo
 considerare comparationem virtutis ad passionem; secundo,
 distinctionem moralium virtutum secundum passiones.


 Circa primum quaeruntur quinque. 
  
\begin{enumerate}
 \item utrum virtus moralis sit passio.
 \item utrum virtus moralis possit esse cum passione.
 \item utrum possit esse cum tristitia.
 \item utrum omnis virtus moralis sit circa passionem.
 \item utrum aliqua virtus moralis possit esse sine passione.
\end{enumerate}

&

 次に道徳的徳相互の区別について考察されるべきである。そして情念を巡っ
てある道徳的徳は情念の異なりに即して区別されるので、第一に、徳の情念へ
の関係を、第二に、情念に即した道徳的徳の区別を考察しなければならない。
一をめぐって五つのことが問われる。

\begin{enumerate}
 \item 道徳的徳は情念か。
 \item 道徳的徳は情念と共にありうるか。
 \item それは悲しみと共にありうるか。
 \item すべての道徳的徳は情念を巡ってあるか。
 \item ある道徳的徳は情念なしにありうるか。
\end{enumerate}
\end{longtable}
\newpage
\rhead{a.~1}
\begin{center}
{\Large {\bf ARTICULUS PRIMUS}}\\
{\large UTRUM VIRTUS MORALIS SIT PASSIO}\\
{\footnotesize III {\itshape Sent.}, d.23, q.1, a.3, qu$^{a}$2; II {\itshape Ethic.}, lect.5.}\\
{\Large 第一項\\道徳的徳は情念か}
\end{center}

\begin{longtable}{p{21em}p{21em}}

{\scshape Ad primum sic proceditur}. Videtur quod virtus moralis sit
passio. Medium enim est eiusdem generis cum extremis. Sed virtus
moralis est medium inter passiones. Ergo virtus moralis est passio.

&

 第一項の問題へ、議論は以下のように進められる。道徳的徳は情念であると
 思われる。理由は以下の通り。中間は両端と同じ類に属する。しかるに道徳
 的徳は情念の間の中間である。ゆえに道徳的徳は情念である。
 
\\



2. {\scshape Praeterea}, virtus et vitium, cum sint contraria, sunt in
eodem genere. Sed quaedam passiones vitia esse dicuntur, ut invidia et
ira. Ergo etiam quaedam passiones sunt virtutes.

&

 さらに、徳と悪徳は相反するものなので同一の類の中にある。しかるに、う
 らやみや怒りなどのように、悪徳であると言われる情念がある。ゆえに徳で
 ある情念もある。
 
\\



3. {\scshape Praeterea}, misericordia quaedam passio est, est enim
tristitia de alienis malis, ut supra dictum est. {\itshape Hanc} autem
{\itshape Cicero, locutor egregius, non dubitavit appellare virtutem};
ut Augustinus dicit, in IX {\itshape de Civ.~Dei}. Ergo passio potest
esse virtus moralis.

&

 さらに、憐れみはある種の情念である。というのも、前に述べられたとおり
 それは他者の悪についての悲しみだからである。しかるに、アウグスティヌ
 スが『神の国』第9巻で言うように、「高名な弁舌家のキケロはこれを徳と呼
 ぶことを疑わなかった」。ゆえに、情念が道徳的な徳であることは可能であ
 る。

 
\\


{\scshape Sed contra est} quod dicitur in II {\itshape Ethic}., quod
{\itshape passiones neque virtutes sunt neque malitiae}.

&

しかし反対に、『ニコマコス倫理学』第2巻で「情念は徳でも悪意でもない」
と言われている。
 
\\



{\scshape Respondeo dicendum quod} virtus moralis non potest esse
passio. Et hoc patet triplici ratione. Primo quidem, quia passio est
motus quidam appetitus sensitivi, ut supra dictum est. Virtus autem
moralis non est motus aliquis, sed magis principium appetitivi motus,
habitus quidam existens.

&

解答する。以下のように言われるべきである。道徳的は情念でありえない。こ
のことは三重の理由によって明らかである。第一に、情念は、前に述べられた
とおり、感覚的欲求のある種の運動である。これに対して、道徳的徳はある種
の習慣なのだから、何らかの運動ではなく、むしろ欲求的運動の根源である。

 
\\

 Secundo quia passiones ex seipsis non habent rationem boni vel
mali. Bonum enim vel malum hominis est secundum rationem, unde
passiones, secundum se consideratae, se habent et ad bonum et ad
malum, secundum quod possunt convenire rationi vel non
convenire. Nihil autem tale potest esse virtus, cum virtus solum ad
bonum se habeat, ut supra dictum est.


&

第二に、情念は、それ自身に基づいて善や悪の性格を持たない。なぜなら人間
の善や悪は理性に基づくからである。したがって情念は、それ自身として考え
られるならば、理性に一致するか一致しないかに応じて善や悪に関係する。し
かるにそのようなものは徳ではありえない。なぜなら徳は、上述の如く、ただ
善にのみ関係するからである。
 
\\


 Tertio quia, dato quod aliqua passio se habeat solum ad bonum, vel
solum ad malum, secundum aliquem modum; tamen motus passionis,
inquantum passio est, principium habet in ipso appetitu, et terminum
in ratione, in cuius conformitatem appetitus tendit. Motus autem
virtutis est e converso, principium habens in ratione et terminum in
appetitu, secundum quod a ratione movetur. Unde in definitione
virtutis moralis dicitur, in II Ethic., quod est {\itshape habitus
electivus in medietate consistens determinata ratione, prout sapiens
determinabit}.

&

 第三に、かりに何らかの仕方で、ある情念が善だけに、あるいは悪だけに関
 係するとしても、情念の運動は、情念である限りにおいて、欲求それ自体に
 始まりを持ち、欲求がそれへの一致へ向かうところの理性において終わりを
 持つ。しかるに徳の運動はその逆で、理性の中に始まりをもち、理性によっ
 て動かされるかぎりで欲求の中に終わりをもつ。このことから、道徳的な徳
 の定義において、「知者がそうするであろうように、理性によって限定され
 た中庸において成立する選択的な習慣」と『ニコマコス倫理学』第2巻で言わ
 れている。

 
\\



{\scshape Ad primum ergo dicendum} quod virtus, secundum suam
essentiam, non est medium inter passiones, sed secundum suum effectum,
quia scilicet inter passiones medium constituit.

&

 第一異論に対しては、それゆえ、以下のように言われるべきである。徳は、
 自らの本質に即して、情念の間の中間にあるのではなく、むしろ、自らの結
 果に即してである。なぜなら、それは情念の間の中間に成立するからである。
 
\\



{\scshape Ad secundum dicendum} quod, si vitium dicatur habitus
secundum quem quis male operatur, manifestum est quod nulla passio est
vitium. Si vero dicatur vitium peccatum, quod est actus vitiosus, sic
nihil prohibet passionem esse vitium, et e contrario concurrere ad
actum virtutis; secundum quod passio vel contrariatur rationi, vel
sequitur actum rationis.

&

 第二異論に対しては以下のように言われるべきである。もし、人がそれに即
 して悪く働くところの習慣が悪徳と言われるならば、どんな情念も悪徳では
 ないことが明らかである。しかし、邪悪な行為であるところの罪が悪徳と言
 われるのであれば、この意味では、情念が悪徳であったり、逆に有徳な行為
 に一致したりすることを妨げるものはない。それは情念が理性に反対するか、
 理性の行為にしたがうかによる。

 
\\

{\scshape Ad tertium dicendum} quod misericordia dicitur esse virtus,
idest virtutis actus, secundum quod {\itshape motus ille animi rationi
servit, quando scilicet ita praebetur misericordia, ut iustitia
conservetur, sive cum indigenti tribuitur, sive cum ignoscitur
poenitenti}, ut Augustinus dicit ibidem. Si tamen misericordia dicatur
aliquis habitus quo homo perficitur ad rationabiliter miserendum,
nihil prohibet misericordiam sic dictam esse virtutem. Et eadem est
ratio de similibus passionibus.

&

 第三異論に対しては以下のように言われるべきである。憐れみが徳、すなわ
 ち徳の行為と言われるのは、アウグスティヌスが同じ箇所で述べているよう
 に、「その魂の運動が理性に仕えるかぎりにおいてであり、それはすなわち、
 不足するものに分配されたり、後悔する人が赦されたりする場合のように、
 正義が保たれるように憐れみが与えられるときである」。しかし、人がそれ
 によって理性的に憐れむように完成されるところの習慣が憐れみと言われる
 のであれば、その意味で言われる憐れみが徳であることを何も妨げない。似
 た情念についても同様のことが言える。
 
\\

\end{longtable}
\newpage

\rhead{a.~2}
\begin{center}
{\Large {\bf ARTICULUS SECUNDUS}}\\
{\large UTRUM VIRTUS MORALIS POSSIT ESSE CUM PASSIONE}\\
{\footnotesize II {\itshape Ethic.}, lect.3}\\
{\Large 第二項\\道徳的徳は情念と共にありうるか}
\end{center}

\begin{longtable}{p{21em}p{21em}}

{\scshape Ad secundum sic proceditur}. Videtur quod virtus moralis cum
passione esse non possit. Dicit enim philosophus, in IV {\itshape
Topic}., quod {\itshape mitis est qui non patitur, patiens autem qui
patitur et non deducitur}. Et eadem ratio est de omnibus virtutibus
moralibus. Ergo omnis virtus moralis est sine passione.


&

 第二項の問題へ、議論は以下のように進められる。道徳的徳が情念とともに
 あることはできないと思われる。理由は以下の通り。哲学者は『トピカ』第4
 巻で「穏やかな人とは情念が起こらない人のことであり、我慢強い人とは情
 念が起こってもそれに持って行かれない人のことである」と言う。すべての
 道徳的徳について同じことが言える。ゆえにすべての道徳的徳は情念なしに
 ある。
 
\\



2. {\scshape Praeterea}, virtus est quaedam recta habitudo animae,
sicut sanitas corporis, ut dicitur in VII {\itshape Physic}., unde
{\itshape virtus quaedam sanitas animae esse videtur}, ut Tullius
dicit, in IV {\itshape de Tuscul.~quaest}. Passiones autem animae
dicuntur {\itshape morbi quidam animae}, ut in eodem libro Tullius
dicit. Sanitas autem non compatitur secum morbum. Ergo neque virtus
compatitur animae passionem.

&

さらに『自然学』第7巻で言われるように、徳とは、ちょうど健康が身体の正
しい関係であるように、魂の正しい関係である。このことから、キケロは「徳
とは魂の一種の健康であるように思われる」と『トゥスクルム荘対談集』で言
う。これに対して、キケロが同じ本で言うように、魂の情念とは「魂の一種の
病気」と言われる。しかるに健康は病気と両立しない。ゆえに徳も魂の情念と
両立しない。
 
\\



3. {\scshape Praeterea}, virtus moralis requirit perfectum usum
rationis etiam in particularibus. Sed hoc impeditur per passiones,
dicit enim philosophus, in VI {\itshape Ethic}., quod {\itshape
delectationes corrumpunt existimationem prudentiae}; et Sallustius
dicit, in {\itshape Catilinario}, quod {\itshape non facile verum
perspicit animus, ubi illa officiunt}, scilicet animi
passiones. Virtus ergo moralis non potest esse cum passione.


&

さらに、道徳的徳は個別的な事柄においても理性の完全な使用を要求する。じっ
さい哲学者は『ニコマコス倫理学』第6巻で「快は思慮の評価をだめにする」
と言い、サッルスティウスは『カティリーナの陰謀』の中で「それら」つまり
魂の情念「が邪魔をするところで魂が真を見抜くのは容易でない」と言う。ゆ
えに道徳的徳が情念と共にあることはできない。
 
\\



{\scshape Sed contra est} quod Augustinus dicit, in XIV {\itshape de
Civ.~Dei}: {\itshape Si perversa est voluntas, perversos habebit hos
motus}, scilicet passionum, {\itshape si autem recta est, non solum
inculpabiles, verum etiam laudabiles erunt}. Sed nullum laudabile
excluditur per virtutem moralem. Virtus ergo moralis non excludit
passiones, sed potest esse cum ipsis.

&

 しかし反対に、アウグスティヌスは『神の国』第14巻で次のように言う。
 「もし意志が歪んでいると、歪んだこれらの」すなわち諸情念の「運動を持
 つであろう。これに対してもし意志が正しいならば、罪を犯しえないだけで
 なく、賞賛される人になるであろう」。しかるに、賞賛されるどんなものも、
 道徳的徳によって排除されない。ゆえに道徳的徳は情念を排除せず、それら
 と共にあることができる。

 
\\



 {\scshape Respondeo dicendum quod} circa hoc fuit discordia inter
 Stoicos et Peripateticos, sicut Augustinus dicit, IX {\itshape de
 Civ.~Dei}. Stoici enim posuerunt quod passiones animae non possunt
 esse in sapiente, sive virtuoso, Peripatetici vero, quorum sectam
 Aristoteles instituit, ut Augustinus dicit in IX {\itshape de
 Civ.~Dei}, posuerunt quod passiones simul cum virtute morali esse
 possunt, sed ad medium reductae.



&


 解答する。以下のように言われるべきである。アウグスティヌスが『神の国』
 第9巻で述べているように、これについてはストア派の人々とペリパトス派の
 人々の間で不一致があった。すなわちストア派の人々は、知者あるいは有徳
 者のなかに魂の情念は存在しえないと主張したが、ペリパトス派の人々は、
 アウグスティヌスが『神の国』第9巻で言うように彼らのセクトを作ったのは
 アリストテレスだが、情念が道徳的徳と同時にありうるが、その情念は中庸
 へもたらされているものだと主張した。
 
\\

 Haec autem diversitas, sicut Augustinus ibidem dicit, magis erat
 secundum verba, quam secundum eorum sententias. Quia enim Stoici non
 distinguebant inter appetitum intellectivum, qui est voluntas, et
 inter appetitum sensitivum, qui per irascibilem et concupiscibilem
 dividitur; non distinguebant in hoc passiones animae ab aliis
 affectionibus humanis, quod passiones animae sint motus appetitus
 sensitivi, aliae vero affectiones, quae non sunt passiones animae,
 sunt motus appetitus intellectivi, qui dicitur voluntas, sicut
 Peripatetici distinxerunt, sed solum quantum ad hoc quod passiones
 esse dicebant quascumque affectiones rationi repugnantes.

&

しかしこの違いは、アウグスティヌスが同じ箇所で言うように、彼らの学説と
いうよりは言葉における違いであった。というのも、ストア派の人々は、知的
な欲求すなわち意志と、感覚的な欲求すなわち怒情的部分と欲情的部分によっ
て分けられるものとを区別しなかった。すなわち彼らは、魂の情念を他の人間
的な情動(affectio)から、ペリパトス派の人々が区別したように、魂の情念は
感覚的欲求の運動であり、他方で魂の情念でない他の情動は意志と呼ばれる知
的欲求の運動である、という点で区別したのではなく、なんであれ理性に反対
する情動をすべて情念であると言う点でのみ区別した。
 
\\


 Quae si ex deliberatione oriantur, in sapiente, seu in virtuoso, esse
 non possunt. Si autem subito oriantur, hoc in virtuoso potest
 accidere, eo quod {\itshape animi visa quae appellant phantasias, non
 est in potestate nostra utrum aliquando incidant animo; et cum
 veniunt ex terribilibus rebus, necesse est ut sapientis animum
 moveant, ita ut paulisper vel pavescat metu, vel tristitia
 contrahatur, tanquam his passionibus praevenientibus rationis
 officium; nec tamen approbant ista, eisque consentiunt}; ut
 Augustinus narrat in IX {\itshape de Civ.~Dei}, ab Agellio\footnote{Aulo Gellio (P)} dictum.

&

それらの情動は、もし熟慮に基づいて生じる場合には、知者や有徳者の中にあ
ることができないが、しかし、突然生じる場合には、それは有徳者においても
起こりうる。なぜなら、これはアウグスティヌスが『神の国』第9巻でアウル
ス・ゲッリウス\footnote{Piana版の読みに従う。}によって言われたこととし
て語っていることだが、「表象と呼ぶ魂に見えるものどもがあるときに魂に生
じるかどうかは、私たちの権能の中にはなく、恐ろしい事柄に基づいてやって
来るときには、これらの情念が理性の働きを妨げるものとして、少しのあいだ
恐怖を感じ、悲しみを被るようにして知者の魂を動かすことが必然」だからで
ある。「ただしこれらを是認したり、それらに同意することはないのだが」。
 
\\


 Sic igitur, si passiones dicantur inordinatae affectiones, non
 possunt esse in virtuoso, ita quod post deliberationem eis
 consentiatur; ut Stoici posuerunt. Si vero passiones dicantur
 quicumque motus appetitus sensitivi, sic possunt esse in virtuoso,
 secundum quod sunt a ratione ordinati. Unde Aristoteles dicit, in II
 {\itshape Ethic}., quod {\itshape non bene quidam determinant
 virtutes impassibilitates quasdam et quietes, quoniam simpliciter
 dicunt, sed deberent dicere quod sunt quietes a passionibus quae sunt
 ut non oportet, et quando non oportet}.


&

 このようであるから、それゆえ、もし「情念」が秩序付けられていない情動
 を意味するのであれば、ストア派の人々が考えたように、熟慮の後にそれら
 に同意されるという仕方で、それらが有徳者の中にあることはできない。し
 かし、「情念」がなんであれ感覚的欲求の運動を意味するのであれば、その
 場合には、理性によって秩序付けられているかぎりにおいて、有徳者の中に
 もあることができる。このことからアリストテレスは『ニコマコス倫理学』
 第2巻で以下のように述べる。「ある人々が徳をある種の受動不可能性や休止
 と限定するのは正しくない。なぜなら、この人たちは単純に語っているから
 である。むしろ、あるべきでない情念、しかるべき時でない情念からの休止
 と言うべきであった」。
 
\\

{\scshape Ad primum ergo dicendum} quod philosophus exemplum illud
inducit, sicut et multa alia in libris logicalibus, non secundum
opinionem propriam, sed secundum opinionem aliorum. Haec autem fuit
opinio Stoicorum, quod virtutes essent sine passionibus animae. Quam
opinionem philosophus excludit in II {\itshape Ethic}., dicens
virtutes non esse impassibilitates. Potest tamen dici quod, cum
dicitur quod mitis non patitur, intelligendum est secundum passionem
inordinatam.

&

 第一異論に対しては、それゆえ、以下のように言われるべきである。哲学者
 がその例を出すのは、論理学的な著作においては他にも多くの例があるよう
 に、自分の意見に即してではなく、他の人々の意見に即してである。ところ
 で、徳が魂の情念なしにあるというこの意見はストア派の人々に属するが、
 哲学者は『ニコマコス倫理学』第2巻で、この意見を「徳は受動不可能性では
 ない」と述べて排除している。しかし、穏やかな人は受動しないと述べられ
 るときに、この受動は秩序付けられていない情念において理解されるべきだ
 とは言われうる。
  
\\

{\scshape Ad secundum dicendum} quod ratio illa, et omnes similes quas
Tullius ad hoc inducit in IV libro {\itshape de Tuscul.~Quaest}., procedit de
passionibus secundum quod significant inordinatas affectiones.

&

 第二異論に対しては以下のように言われるべきである。この異論や、キケロ
 が『トゥスクルム荘対談集』第4巻で論じている似た議論は、秩序付けられて
 いない情動を意味するかぎりでの情念について行われている。
 
\\

{\scshape Ad tertium dicendum} quod passio praeveniens iudicium
rationis, si in animo praevaleat ut ei consentiatur, impedit consilium
et iudicium rationis. Si vero sequatur, quasi ex ratione imperata,
adiuvat ad exequendum imperium rationis.

&

 第三異論に対しては以下のように言われるべきである。理性の判断を妨げる
 情念は、もしそれに同意するように魂の中であらかじめ評価されるならば、
 知性の思量と判断を妨げるが、しかし理性に基づいて命じられるようなかた
 ちで後続するならば、理性の命令を遂行することを助ける。

\end{longtable}
\newpage






\rhead{a.~3}
\begin{center}
{\Large {\bf ARTICULUS TERTIUS}}\\
{\large UTRUM VIRTUS MORALIS POSSIT ESSE CUM TRISTITIA}\\
{\Large 第三項\\徳は悲しみと共にありうるか}
\end{center}

\begin{longtable}{p{21em}p{21em}}
{\scshape Ad tertium sic proceditur}. Videtur quod virtus cum
tristitia esse non possit. Virtutes enim sunt sapientiae effectus,
secundum illud {\itshape Sap}. {\scshape viii}, {\itshape sobrietatem
et iustitiam docet}, scilicet divina sapientia, {\itshape prudentiam
et virtutem}. Sed {\itshape sapientiae convictus non habet
amaritudinem}, ut postea subditur. Ergo nec virtutes cum tristitia
esse possunt.

&

 第三項の問題へ議論は以下のように進められる。徳は悲しみと共にあること
 ができないと思われる。理由は以下の通り。かの『知恵の書』第8章、神の知恵が「節制と
 正義を」そして「思慮と徳とを教える」\footnote{「人が義を
 愛するのならば/知恵の働きこそが徳である。/節制と賢明、正義と勇気の
 徳を教えてくれるのは/知恵であり/人生においてこれらの徳よりも有益な
 ものはない。」(8:7)}によれば、徳は知恵の結果である。しかるにその後で
 語られているように、「知恵の交わりは苦さを持たない」\footnote{「家に
 戻れば知恵の傍らに憩う。/知恵との交わりには苦しみがなく/知恵との暮
 らしには悲しみがない。/楽しみと喜びのみがあるからである。」(8:16)}。
 ゆえに徳も悲しみと共にあることができない。

\\

2. {\scshape Praeterea}, tristitia est impedimentum operationis; ut
patet per philosophum, in VII et X {\itshape Ethic}. Sed impedimentum
bonae operationis repugnat virtuti. Ergo tristitia repugnat virtuti.

&

 さらに、『ニコマコス倫理学』第7巻と第10巻の哲学者によって明らかなとお
 り、悲しみは働きの妨げである。しかるによい働きの妨げは徳に反する。ゆ
 えに悲しみは徳に反する。

\\

{\scshape 3.~Praeterea}, tristitia est quaedam animi aegritudo; ut
Tullius eam vocat, in III {\itshape de Tuscul.~Quaest}. Sed aegritudo
animae contrariatur virtuti, quae est bona animae habitudo. Ergo
tristitia contrariatur virtuti, nec potest simul esse cum ea.

&

 さらに、キケロが『トゥスクル荘対談集』でそう呼んでいるように、悲しみ
 は魂の一種の病気である。しかるに魂の病気は、魂の善い習慣である特に反
 対する。ゆえに悲しみは徳に反対するのであり、徳と同時にあることができ
 ない。

\\


{\scshape Sed contra est} quod Christus fuit perfectus virtute. Sed in
eo fuit tristitia, dicit enim, ut habetur Matth.~{\scshape xxvi}:
{\itshape Tristis est anima mea usque ad mortem}. Ergo tristitia
potest esse cum virtute.


&

しかし反対に、キリストは徳において完全であった。しかるに彼の中には悲し
みがあった。というのも『マタイによる福音書』第26章に書かれているように、
「私の魂は死ぬほど悲しい」\footnote{「そして、彼らに言われた。「私は死
ぬほど苦しい。ここを離れず、私と共に目を覚ましていなさい。」(26:38)}と
彼は言うからである。ゆえに悲しみは徳と共にありうる。
 
\\



 {\scshape Respondeo dicendum quod} sicut dicit Augustinus, XIV
 {\itshape de Civ.~Dei},{\itshape  Stoici voluerunt, pro tribus
 perturbationibus, in animo sapientis esse tres eupathias}, idest tres
 bonas passiones, {\itshape pro cupiditate scilicet voluntatem; pro laetitia,
 gaudium; pro metu, cautionem}.

&

 解答する。以下のように言われるべきである。アウグスティヌスは『神の国』
 第14巻で以下のように述べている。「ストア派の人々は、三つの煩悩の代わりに知者の
 魂の中には三つの善情」すなわち三つの善い情念「があると言おうとした。すなわち欲
 情の代わりに意志が、悦楽の代わりに喜びが、怖れの代わりに注意深さが。

\\

 {\itshape Pro tristitia vero, negaverunt posse aliquid esse in animo sapientis},
 duplici ratione. Primo quidem, quia tristitia est de malo quod iam
 accidit. Nullum autem malum aestimant posse accidere sapienti,
 crediderunt enim quod, sicut solum hominis bonum est virtus, bona
 autem corporalia nulla bona hominis sunt; ita solum inhonestum est
 hominis malum, quod in virtuoso esse non potest.

 &

 しかし悲しみの代わりに、彼らは何かが知者の魂の中にありうることを否定
 した」。それは二つの理由による。第一に、悲しみは、すでに生じている悪
 にかかわるが、どんな悪も知者には生じえないと彼らは考える。というのも、
 ちょうど人間の唯一の善が徳であって、どんな物体的な善も人間の善ではな
 いように、ただ不正直だけが人間の悪であり、これは有徳者の中にはありえ
 ないと彼らが信じたからである。

\\


 Sed hoc irrationabiliter dicitur. Cum enim homo sit ex anima et
 corpore compositus, id quod confert ad vitam corporis conservandam,
 aliquod bonum hominis est, non tamen maximum, quia eo potest homo
 male uti. Unde et malum huic bono contrarium in sapiente esse potest,
 et tristitiam moderatam inducere.

&

 しかしこれは理性的に語られていない。というのも、人間は魂と身体とから
 複合されているので、身体の生命を保つために役立つものは何らかの人間の
 善であるが、しかし最大の善ではなく、というのも、人間はそれを悪く使い
 うるからである。このことから、この善に反対する悪も、知者の中にはあり
 うるのであり、それは穏やかな悲しみを引き起こしうる。

\\


 Praeterea, etsi virtuosus sine gravi peccato esse possit, nullus
 tamen invenitur qui absque levibus peccatis vitam ducat, secundum
 illud I Ioan.~{\scshape i}, {\itshape si dixerimus quia peccatum non habemus, nos ipsos
 seducimus}.


&

 さらに、『ヨハネの手紙一』第1章「もし私たちが、私たちは罪を持っていな
 いと言うならば、私たち自身を欺いている」\footnote{「自分に罪がないと
 言うなら、自らを欺いており、真理は私たちの内にありません」(1:8)}によ
 れば、かりに有徳者が重大な罪なしにありうるとしても、軽い罪なしに人生
 を送る人はだれもいない

\\

 Tertio, quia virtuosus, etsi peccatum non habeat, forte quandoque
 habuit. Et de hoc laudabiliter dolet; secundum illud II {\itshape ad
 Cor}.~{\scshape vii}, {\itshape quae secundum Deum est tristitia,
 poenitentiam in salutem stabilem operatur}.


&

 第三に、有徳者であるがゆえに彼は罪を犯さないとしても、おそらく彼は罪
 を犯したことがあったであろう。そして『コリントの信徒への手紙二』第7章
 「神に即して悲しみであるものは、安定した救いへ向かう後悔を働かせる」
 \footnote{「神の御心に適った悲しみは、悔いのない、救いに至る悔い改め
 を生じさせ、この世の悲しみは死をもたらします。」(7:10)}によれば、彼が
 それについて悲しむのは賞賛されるべきことである。

\\


 Quarto, quia potest etiam dolere laudabiliter de peccato
 alterius. Unde eo modo quo virtus moralis compatitur alias passiones
 ratione moderatas, compatitur etiam tristitiam.

&

 第四に、他人の罪について悲しむことですら、賞賛されるべきことでありう
 る。したがって、倫理的徳は、理性によって抑制された他の情念と両立する
 ようなしかたで、悲しみとも両立する。

\\


 Secundo, movebantur ex hoc, quod tristitia est de praesenti malo,
 timor autem de malo futuro, sicut delectatio de bono praesenti,
 desiderium vero de bono futuro. Potest autem hoc ad virtutem
 pertinere, quod aliquis bono habito fruatur, vel non habitum habere
 desideret, vel quod etiam malum futurum caveat. Sed quod malo
 praesenti animus hominis substernatur, quod fit per tristitiam,
 omnino videtur contrarium rationi, unde cum virtute esse non potest.

&

 (ストア派の人々が知者のうちに悲しみに代わる善情がないと考えた理由の)
 二番目は、悲しみが現在の悪についてあり、これに対して怖れが将来の悪に
 ついてあることによって、そしてこれはちょうど、快楽が現在の善について、
 願望が将来の善についてであるのと同様なのだが、その事実によって彼らが
 動かされていたからである。つまり、ある人が所有された善を享受すること
 や、所有されていない善を望むこと、あるいは将来の悪を心配することは徳
 に属しうる。しかし現在の悪に人間の魂が服すること、これが悲しみによっ
 て生じることだが、これは、全く理性に反するように見えるので、これは徳
 とともにはありえない(と彼らは考えた)。

\\


 Sed hoc irrationabiliter dicitur. Est enim aliquod malum quod potest
 esse virtuoso praesens, ut dictum est. Quod quidem malum ratio
 detestatur. Unde appetitus sensitivus in hoc sequitur detestationem
 rationis, quod de huiusmodi malo tristatur, moderate tamen, secundum
 rationis iudicium.

&

 しかしこれは合理的に語られていない。すでに述べられたとおり、有徳者に
 現在するような何らかの悪というものがありうる。理性はこの悪を憎む。こ
 のことから、感覚的欲求がこれにおいて理性の憎みに後続し、このような悪
 について悲しむが、それは理性の判断に従った穏やかなものである。
 

\\

 Hoc autem pertinet ad virtutem, ut appetitus sensitivus rationi
 conformetur, ut dictum est. Unde ad virtutem pertinet quod tristetur
 moderate in quibus tristandum est, sicut etiam philosophus dicit in
 II {\itshape Ethic}.


&

しかし、すでに述べられたとおり、感覚的欲求が理性に一致することは徳に属
する。したがって、哲学者も『ニコマコス倫理学』第2巻で述べているように、
悲しまれるべきことについて穏やかに悲しむことは徳に属する。
 

\\

 Et hoc etiam utile est ad fugiendum mala. Sicut
 enim bona propter delectationem promptius quaeruntur, ita mala
 propter tristitiam fortius fugiuntur. Sic igitur dicendum est quod
 tristitia de his quae conveniunt virtuti, non potest simul esse cum
 virtute, quia virtus in propriis delectatur. Sed de his quae
 quocumque modo repugnant virtuti, virtus moderate tristatur.

&

そしてこれはまた悪を避けるために有用である。すなわち善がその快適さのた
めにより迅速に求められるように、悪はその悲しみのためにより強く避けられ
る。ゆえに以上のことから、徳に一致する事柄についての悲しみは徳とともに
ありえない、と言われるべきである。なぜなら徳はその固有の事柄において喜
ばれるからである。しかしどのような仕方であれ徳に反する事柄については、
徳は穏やかに悲しむ。
 

\\



{\scshape Ad primum ergo dicendum} quod ex illa auctoritate habetur
quod de sapientia sapiens non tristetur. Tristatur tamen de his quae
sunt impeditiva sapientiae. Et ideo in beatis, in quibus nullum
impedimentum sapientiae esse potest, tristitia locum non habet.


&

 第一異論に対しては、それゆえ、以下のように言われるべきである。かの権
威によって言われているのは、知者が知恵を悲しむことはない、ということで
ある。しかし知恵を阻害しうる事柄については悲しむ。ゆえに至福者において
も、彼らの中には知恵を阻害するものはありえないので、悲しみの場所はない。

\\



{\scshape Ad secundum dicendum} quod tristitia impedit operationem de
qua tristamur, sed adiuvat ad ea promptius exequenda per quae
tristitia fugitur.

&

 第二異論に対しては以下のように言われるべきである。悲しみは、それにつ
 いて私たちが悲しむ働きを妨げるが、それによって悲しみが避けるところの
 ものをより迅速に遂行することを助ける。

\\


{\scshape Ad tertium dicendum} quod tristitia immoderata est animae
aegritudo, tristitia autem moderata ad bonam habitudinem animae
pertinet, secundum statum praesentis vitae.

&

 第三異論に対しては以下のように言われるべきである。穏やかでない悲しみ
 は魂の病に属するが、穏やかな悲しみは、現生の状態においては、魂の善
 い関係性に属する。


\end{longtable}
\newpage

\rhead{a.~4}
\begin{center}
{\Large {\bf ARTICULUS QUARTUS}}\\
{\large UTRUM OMNIS VIRTUS MORALIS SIT CIRCA PASSIONES}\\
{\Large 第四項\\すべての道徳的徳は情念にかかわるか}
\end{center}

\begin{longtable}{p{21em}p{21em}}

{\scshape Ad quartum sic proceditur}. Videtur quod omnis virtus
moralis sit circa passiones. Dicit enim philosophus, in II {\itshape
Ethic}., quod {\itshape circa voluptates et tristitias est moralis
virtus}. Sed delectatio et tristitia sunt passiones, ut supra dictum
est. Ergo omnis virtus moralis est circa passiones.

&

 第四項の問題へ、議論は以下のように進められる。すべての道徳的徳が情念
 にかかわると思われる。理由は以下の通り。哲学者は『ニコマコス倫理学』
 第2巻で「快楽と悲しみに関して道徳的徳がある」と述べている。しかるに前
 に述べられたとおり、快楽と悲しみは情念である。ゆえにすべての道徳的徳
 は情念に関してある。
 
\\


2.{\scshape Praeterea}, rationale per participationem est subiectum
moralium virtutum, ut dicitur in I {\itshape Ethic}. Sed huiusmodi
pars animae est in qua sunt passiones, ut supra dictum est. Ergo omnis
virtus moralis est circa passiones.

&

さらに、『ニコマコス倫理学』第1巻で述べられるように、分有によって理性
的であるものが、道徳的諸徳の基体である。しかるにそのような魂の部分は、前
に述べられたとおり、その中に情念があるところである。ゆえにすべての道徳
的徳は情念に関してある。
 
\\


3.{\scshape Praeterea}, in omni virtute morali est invenire aliquam
passionem. Aut ergo omnes sunt circa passiones, aut nulla. Sed aliquae
sunt circa passiones, ut fortitudo et temperantia, ut dicitur in III
{\itshape Ethic}. Ergo omnes virtutes morales sunt circa passiones.

&

さらに、すべての道徳的徳の中に何らかの情念を見出すことができる。ゆえに
\footnote{この「ゆえに」の意味が不明。}、すべての道徳的徳が情念にかか
わるか、あるいはどんな道徳的徳も情念にかかわらないかのどちらかである。
しかるに『ニコマコス倫理学』第2巻で言われるように、勇気や節制のように、
情念にかかわるものがある。それゆえ、すべての道徳的徳は情念にかかわる。
 
\\



{\scshape Sed contra est} quod iustitia, quae est virtus moralis, non
est circa passiones, ut dicitur in V Ethic.

&

しかし反対に、『ニコマコス倫理学』第5巻で言われているように、道徳的徳
である正義は情念にかかわらない。
 
\\



{\scshape Respondeo} dicendum quod virtus moralis perficit appetitivam
partem animae ordinando ipsam in bonum rationis. Est autem rationis
bonum id quod est secundum rationem moderatum seu ordinatum.  Unde
circa omne id quod contingit ratione ordinari et moderari, contingit
esse virtutem moralem.
 
&

 解答する。以下のように言われるべきである。道徳的徳は魂の欲求的部分を
 理性の善へと秩序付けることによってそれを完成する。しかるに理性の善と
 は、理性に従って穏やかな、言い換えれば秩序付けられた善のことである。
 このことから、理性によって秩序付けられたり穏やかにされたりすることが
 ありうるすべてのものを巡って、道徳的な徳がありうる。

 
\\

Ratio autem ordinat non solum
passiones appetitus sensitivi; sed etiam ordinat operationes appetitus
intellectivi, qui est voluntas, quae non est subiectum passionis, ut
supra dictum est. Et ideo non omnis virtus moralis est circa
passiones; sed quaedam circa passiones, quaedam circa operationes.

&

しかるに理性は感覚的欲求の情念だけを秩序付けるのではなく、知性的欲求、
すなわち意志の働きも秩序付けるが、この働きは、上述の如く、情念の基体で
はない。ゆえにすべての道徳的徳が情念に関するわけではなく、あるものは情
念に関わり、あるものは働きに関わる。
 
\\



{\scshape Ad primum ergo dicendum} quod non omnis virtus moralis est
circa delectationes et tristitias sicut circa propriam materiam, sed
sicut circa aliquid consequens proprium actum. Omnis enim virtuosus
delectatur in actu virtutis, et tristatur in contrario. Unde
philosophus post praemissa verba subdit quod, {\itshape si virtutes
sunt circa actus et passiones; omni autem passioni et omni actui
sequitur delectatio et tristitia; propter hoc virtus erit circa
delectationes et tristitias}, scilicet sicut circa aliquid consequens.

&

 第一異論に対しては、それゆえ、以下のように言われるべきである。すべて
 の道徳的徳が快楽と悲しみに関してあるのは、固有の質料としてではなく、
 固有の作用の帰結としてである。すなわち、すべての有徳者は徳の作用にお
 いて快さを感じ、それに反するものにおいて悲しみを感じる。このことから
 哲学者は、上で引用されたところ次に、以下のように書いている。「もし徳
 が作用と受動にかんしてであるならば、そしてさらに受動とすべての作用に
 は快楽と悲しみが後続するならば、このことのために、徳は快楽と悲しみに
 関することになるだろう」すなわち、何らかの結果として、ということであ
 る。
 
\\


{\scshape Ad secundum dicendum} quod rationale per participationem non
solum est appetitus sensitivus, qui est subiectum passionum; sed etiam
voluntas, in qua non sunt passiones, ut dictum est.

&

 第二異論に対しては以下のように言われるべきである。分有によって理性的
 であるものは感情の基体である感覚的欲求だけでなく、意志もまたそうであ
 る。すでに述べられたとおり、意志の中に感情はない。
 
\\


{\scshape Ad tertium dicendum} quod in quibusdam virtutibus sunt
passiones sicut propria materia, in quibusdam autem non. Unde non est
eadem ratio de omnibus, ut infra ostendetur.

&

 第三異論に対しては以下のように言われるべきである。ある徳の中には固有
 の質料のようなものとして感情があるが、別の徳の中では、そうではない。
 このことから、この後に述べられるとおり\footnote{Q.60, a.2.}、それらに
 ついて同じことは言えない。


\end{longtable}
\newpage


\rhead{a.~5}
\begin{center}
{\Large {\bf ARTICULUS QUINTUS}}\\
{\large UTRUM ALIQUA VIRTUS MORALIS POSSIT ESSE ABSQUE PASSIONE}\\
{\Large 第五項\\何らかの道徳的徳は情念なしにありうるか}
\end{center}

\begin{longtable}{p{21em}p{21em}}
{\scshape Ad quintum sic proceditur}. Videtur quod virtus moralis
possit esse absque passione. Quanto enim virtus moralis est
perfectior, tanto magis superat passiones. Ergo in suo perfectissimo
esse, est omnino absque passionibus.
 
&

 第五項の問題へ、議論は以下のように進められる。道徳的徳は情念なしにあ
 りうると思われる。理由は以下の通り。道徳的徳が完全であればあるほど、
 それだけ情念を凌駕する。ゆえにそのもっとも完全な存在においては、全く
 情念なしにある。

 \\

2.{\scshape Praeterea}, tunc unumquodque est perfectum, quando est
remotum a suo contrario, et ab his quae ad contrarium inclinant. Sed
passiones inclinant ad peccatum, quod virtuti contrariatur, unde
{\itshape Rom}.~{\scshape vii}, nominantur {\itshape passiones
peccatorum}. Ergo perfecta virtus est omnino absque passione.

 &

さらに、各々のものが完全なのは、自らに反対するものや、それによって反対
へと傾くところのものから離れているときである。しかるに情念は罪へと傾か
せるものであり徳に反対する。それゆえ『ローマの信徒への手紙』第7章では
「罪の情念」\footnote{「私たちが肉にあったときは、律法による罪の欲情が
五体の内に働き、死に至る実を結んでいました」(7:5)}と名付けられている。
ゆえに完全な徳は、あらゆる点で情念なしにある。
 
\\


3.{\scshape Praeterea}, secundum virtutem Deo conformamur; ut patet
per Augustinum, in libro {\itshape de moribus Eccles}. Sed Deus omnia operatur
sine passione. Ergo virtus perfectissima est absque omni passione.
 
&

さらに、『教会の習俗について』という書物におけるアウグスティヌスによれ
ば、私たちは徳において神に一致される。しかるに神はすべてを情念なしに行
う。ゆえに最高に完全な徳はすべての情念なしにある。

 
\\




 {\scshape Sed contra est} quod {\itshape nullus iustus est qui non
 gaudet iusta operatione}, ut dicitur in I {\itshape Ethic}. Sed
 gaudium est passio. Ergo iustitia non potest esse sine passione. Et
 multo minus aliae virtutes.
 
&

 しかし反対に、『ニコマコス倫理学』第1巻で言われるように、「なされた正
 義に喜ばないような正義の人はいない」。しかるに喜びは情念である。ゆえ
 に正義は情念なしにありえない。ましてや他の徳はそうである。
 
\\




 {\scshape Respondeo dicendum quod}, si passiones dicamus inordinatas
 affectiones, sicut Stoici posuerunt; sic manifestum est quod virtus
 perfecta est sine passionibus. Si vero passiones dicamus omnes motus
 appetitus sensitivi, sic planum est quod virtutes morales quae sunt
 circa passiones sicut circa propriam materiam, sine passionibus esse
 non possunt.
 
&

 解答する。以下のように言われるべきである。もし私たちが、ストア派の人々
 が考えたように、秩序付けられていない情動を情念と呼ぶならば、その場合
 に完全な徳が情念なしにあることは明らかである。しかしもし、感覚的欲求
 の運動をすべて情念と呼ぶならば、その場合に、固有の質料として情念にか
 かわる道徳的徳が情念なしにありえないのは明らかである。
 
\\


 Cuius ratio est, quia secundum hoc, sequeretur quod virtus moralis
 faceret appetitum sensitivum omnino otiosum. Non autem ad virtutem
 pertinet quod ea quae sunt subiecta rationi, a propriis actibus
 vacent, sed quod exequantur imperium rationis, proprios actus agendo.
 
&

 その理由は以下の通りである。すなわち、この場合には、道徳的徳が感覚的
 欲求をまったく働かないものにしてしまったであろう。しかし、理性に従属
 するものを、それの固有の作用から外してしまうことは徳に属さず、むしろ、
 なされるべき固有の作用を働かせることによって理性の命令を遂行すること
 が、徳に属する。
 
\\


 Unde sicut virtus membra corporis ordinat ad actus exteriores
 debitos, ita appetitum sensitivum ad motus proprios
 ordinatos.

 &
 
 したがって、徳が身体の四肢をしかるべき外的な作用へと秩序付けるように、
 感覚的欲求を秩序付けられた固有の運動へと秩序付ける。

 \\


 Virtutes vero morales quae non sunt circa passiones, sed
 circa operationes, possunt esse sine passionibus (et huiusmodi virtus
 est iustitia), quia per eas applicatur voluntas ad proprium actum,
 qui non est passio.
 
&

他方、情念ではなく働きにかかわる道徳的徳は、情念なしにありうる(そして
このような徳が正義である)。というのも、それらの徳によって正義は固有の
情念ではなく行為へと適用されるからである。

 
\\

 Sed tamen ad actum iustitiae sequitur gaudium, ad minus in voluntate,
 quod non est passio. Et si hoc gaudium multiplicetur per iustitiae
 perfectionem, fiet gaudii redundantia usque ad appetitum sensitivum;
 secundum quod vires inferiores sequuntur motum superiorum, ut supra
 dictum est. Et sic per redundantiam huiusmodi, quanto virtus fuerit
 perfectior, tanto magis passionem causat.
 
&

しかし、正義の行為には喜びがともなう。少なくとも意志の中には喜びがある
が、この喜びは情念ではない。そしてもし、この喜びが正義の完全性によって
多様化される場合には、前に述べられたとおり、下位の力が上位の運動に従う
限りにおいて、喜びの余剰が感覚的欲求までおよぶ。かくしてこのような余剰
を通して、徳が完全であればあるほど、より大きな情念を生み出す。

 
\\



 {\scshape Ad primum ergo dicendum} quod virtus passiones inordinatas
 superat, moderatas autem producit.
 
&

 第一異論に対しては、それゆえ、以下のように言われるべきである。徳は無
 秩序な情念はこれを凌駕するが、穏やかな情念はむしろこれを生み出す。
 
\\




 {\scshape Ad secundum dicendum} quod passiones inordinatae inducunt
 ad peccandum, non autem si sunt moderatae.
 
&

 第二異論に対しては以下のように言われるべきである。無秩序な情念は罪へ
 と導くが、穏やかな情念である場合にはそうでない。
 
\\




 {\scshape Ad tertium dicendum} quod bonum in unoquoque consideratur
 secundum conditionem suae naturae. In Deo autem et Angelis non est
 appetitus sensitivus, sicut est in homine. Et ideo bona operatio Dei
 et Angeli est omnino sine passione, sicut et sine corpore, bona autem
 operatio hominis est cum passione, sicut et cum corporis ministerio.
 
 &

 第三異論に対しては以下のように言われるべきである。各々のものにおける
 善は、それの本性の条件に即して考察される。しかるに神と天使の中には、
 人間のように感覚的欲求はない。ゆえに神と天使の善い働きは、身体なしに
 あるのと同様に全く情念なしにあるが、人間の善い働きは、身体の奉仕と共
 にあるのと同様に、感情と共にある。
 
\end{longtable}
\end{document}