\documentclass[10pt]{jsarticle} % use larger type; default would be 10pt
%\usepackage[utf8]{inputenc} % set input encoding (not needed with XeLaTeX)
%\usepackage[round,comma,authoryear]{natbib}
%\usepackage{nruby}
\usepackage{okumacro}
\usepackage{longtable}
%\usepqckage{tablefootnote}
\usepackage[greek.ancient,english,japanese]{babel}
%\usepackage{amsmath}
\usepackage{latexsym}
\usepackage{color}

%----- header -------
\usepackage{fancyhdr}
\pagestyle{fancy}
\lhead{{\it Summa Theologiae} I, q.~12}
%--------------------

\bibliographystyle{jplain}

\title{{\bf PRIMA PARS}\\{\HUGE Summae Theologiae}\\Sancti Thomae
Aquinatis\\{\sffamily QUEAESTIO DUODECIMA}\\QUOMODO DEUS A NOBIS COGNOSCATUR}
\author{Japanese translation\\by Yoshinori {\sc Ueeda}}
\date{Last modified \today}

%%%% コピペ用
%\rhead{a.~}
%\begin{center}
% {\Large {\bf }}\\
% {\large }\\
% {\footnotesize }\\
% {\Large \\}
%\end{center}
%
%\begin{longtable}{p{21em}p{21em}}
%
%&
%
%\\
%\end{longtable}
%\newpage

\begin{document}
\maketitle

\begin{center}
{\Large 第十二問\\
神は私たちによってどのように知られるか}
\end{center}

\newpage

\begin{longtable}{p{21em}p{21em}}

{\sc \Large Quia} in superioribus consideravimus qualiter Deus sit secundum seipsum,
 restat considerandum qualiter sit in cognitione nostra, idest quomodo
 cognoscatur a creaturis. Et circa hoc quaeruntur tredecim.

&

私たちは先に、神がそれ自体においてどのようであるかを考察したので、私た
ちの認識においてどのようであるか、つまり、被造物によってどのように認識
されるかということが、考察されるべく残されている。これについて、十三の
ことが問われる。

\\

\begin{enumerate}
 \item  utrum aliquis intellectus creatus possit videre essentiam
	Dei. 
 \item utrum Dei essentia videatur ab intellectu per aliquam
       speciem creatam.
 \item utrum oculo corporeo Dei essentia possit videri.
 \item utrum aliqua substantia intellectualis creata ex suis
	naturalibus sufficiens sit videre Dei essentiam. 
 \item utrum intellectus creatus ad videndam Dei essentiam
       indigeat aliquo lumine creato. 
 \item utrum videntium essentiam Dei unus alio perfectius
       videat. 
 \item utrum aliquis intellectus creatus possit comprehendere
       Dei essentiam. 
 \item utrum intellectus creatus videns Dei essentiam, omnia in
       ipsa cognoscat. 
 \item utrum ea quae ibi cognoscit, per aliquas similitudines
       cognoscat. 
 \item utrum simul cognoscat omnia quae in Deo videt. 
 \item utrum in statu huius vitae possit aliquis homo essentiam Dei
       videre. 
 \item utrum per rationem naturalem Deum in hac vita possimus
       cognoscere. 
 \item utrum, supra cognitionem naturalis rationis, sit in
       praesenti vita aliqua cognitio Dei per gratiam.

\end{enumerate}

&

\begin{enumerate}
 \item なんらかの被造の知性が神の本質を見ることができるか。
 \item 神の本質は、なんらかの被造の形象を通して知性によって見られるか。
 \item 神の本質は、身体的な目(肉眼)によって見られうるか
 \item なんらかの被造の知性的実体は、自らが持つ本性的なものによって神
       の本質を見るのに十分であるか。
 \item 被造の知性は、神の本質を見るために、なんらかの被造の光を必要とす
       るか。
 \item 神の本質を見るものどものうちで、あるものは他のものよりも、完全に
       神を見るか。
 \item なんらかの被造の知性が、神の本質を把握できるか。
 \item 神の本質を見る被造の知性は、その本質において、万物を認識するか。
 \item そこで認識するものどもを、なんらかの類似を通して認識するか。
 \item 神において見るすべてのものを一度に認識するか。
 \item 現生の状態において、だれかの人間が神の本質を見ることができるか。
 \item 自然本性的な理性によって、私たちは神をこの生において認識できるか。
 \item 自然本性的な理性の認識を越えて、恩恵によって、神のなんらかの認識が、
       現在の生においてあるか。
\end{enumerate}
\end{longtable}

\newpage

\rhead{a.~1}

\begin{center}
{\Large {\bfseries ARTICULUS PRIMUS}}\\
{\large UTRUM ALIQUIS INTELLECTUS CREATUS\\ POSSIT DEUM VIDERE PER
 ESSENTIAM}\\
{\footnotesize Infra, a.~4, ad 3; I-II, q.~3, a.~8, et q.~5, a.~1; IV
 {\itshape Sent.}, d.~49, q.~2, a.~1; III {\itshape SCG.}, c.~51, 54,
 57; {\itshape De Verit.}, q.~8, a.~1; {\itshape Quodl.}~X, q.~8;
 {\itshape Compend.~Theol.}, c.~104, et part.~II, c.~9, 10; in {\itshape
 Matt.}, c.~5; in {\itshape Ioan.}, c.~1, lect.~9.}\\
{\Large 第一項\\なんらかの被造の知性が神を本質によって見ることができるか}
\end{center}

\begin{longtable}{p{21em}p{21em}}

{\huge A}{\sc d primum sic proceditur}. Videtur quod nullus
intellectus creatus possit Deum per essentiam videre. Chrysostomus
enim, {\it super Ioannem}, exponens illud quod dicitur Ioan.\ I, {\it
Deum nemo vidit unquam}, sic dicit, {\it Ipsum quod est Deus, non
solum prophetae, sed nec Angeli viderunt nec Archangeli, quod enim
creabilis est naturae, qualiter videre poterit quod increabile est?}
Dionysius etiam, I cap.\ {\it de Div.\ Nom.}, loquens de Deo, dicit,
{\it neque sensus est eius, neque phantasia, neque opinio, nec ratio,
nec scientia}.

&

第一項の問題へ議論は以下のように進められる。どんな被造の知性も神を本質によっ
て見ることができないと思われる。クリュソストムス\footnote{c.344-407 コ
ンスタンティノープルの大主教。名説教で知られた。ギリシャ語で「クリュソ
ストモス」は、「金の口」という意味。}は、『ヨハネ伝注解』で、『ヨハネ
伝』第1章の「だれも神をけっして見なかった」\footnote{「いまだかつて、
神を見た者はいない」(1:8)}を解説して次のように述べている。「神であるそ
のものを、預言者たちだけでなく、天使たちも、大天使たちも見なかった。創
造されうる本性に属するものが、どのようにして創造されえないものを見るこ
とができたりするだろうか」。さらにディオニュシウスは、『神名論』第1章
で、神について述べながら「神については、感覚も、表象像も、意見も、理性
も、学知もない」と言っている。

\\

2.~{\sc Praeterea}, omne infinitum, inquantum huiusmodi, est ignotum. Sed Deus
 est infinitus, ut supra ostensum est. Ergo secundum se est ignotus.

&

さらに、無限なものは、無限である限り知られない。しかるに上で示された
ように\footnote{第7問第1項。}神は無限である。ゆえにそれ自体において知られない。

\\

3.~{\sc Praeterea}, intellectus creatus non est cognoscitivus nisi
existentium, primum enim quod cadit in apprehensione intellectus, est
ens. Sed Deus non est existens, sed {\it supra existentia}, ut dicit
Dionysius. Ergo non est intelligibilis; sed est supra omnem
intellectum.

&

さらに、被造の知性は、存在するもの以外のものを認識しえない。なぜなら、
知性の把握に第一に入ってくるのは有だからである。しかるに、神は存在する
ものではなく、ディオニュシウスが言うように、「存在を越えるもの」
\footnote{{\it De Divinis Nominibus} cap.~IV, S.~Th.~Lect.~II.}である。
ゆえに、神は知性認識されうるものではなく、あらゆる知性を越えている。

\\

4.~{\sc Praeterea}, cognoscentis ad cognitum oportet esse aliquam
proportionem, cum cognitum sit perfectio cognoscentis. Sed nulla est
proportio intellectus creati ad Deum, quia in infinitum distant. Ergo
intellectus creatus non potest videre essentiam Dei.

&

さらに、認識されたものは、認識するものの完成だから、認識するものには、
認識されたものにたいする、なんらかの比(比例関係)がなければならない。
しかるに、神は無限に隔たっているから、被造の知性には神に対する比がない。
ゆえに、被造の知性は神の本質を見ることができない。

\\

{\sc Sed contra est} quod dicitur I {\itshape Ioan}.~{\scshape iii},
{\it videbimus eum sicuti est}.

&

しかし反対に、『ヨハネの手紙1』3章で「私たちは彼をありのままに見るで
あろう」\footnote{「なぜなら、そのとき御子をありのままに見るからです」
第3章第2節}と言われている。

\\

{\scshape Respondeo dicendum} quod, cum unumquodque sit cognoscibile
secundum quod est in actu, Deus, qui est actus purus absque omni
permixtione potentiae, quantum in se est, maxime cognoscibilis
est. Sed quod est maxime cognoscibile in se, alicui intellectui
cognoscibile non est, propter excessum intelligibilis supra
intellectum, sicut sol, qui est maxime visibilis, videri non potest a
vespertilione, propter excessum luminis. Hoc igitur attendentes,
quidam posuerunt quod nullus intellectus creatus essentiam Dei videre
potest.

&

解答する。以下のように言われるべきである。おのおののものは、それが現実
態においてある限りで認識されうるものであるから、いかなる可能態の混合
もなく純粋現実態である神は、それ自体において最大限に認識されうる。し
かし、それ自体において最大限に認識されうるものが、知性にたいする可知性
の過剰のために、ある知性にとって、認識されうるものでない。たとえば、最
大限に可視的である太陽は、光の過剰のために、コウモリには見られえない。
このことに気づいて、ある人々は、どんな被造の知性も神の本質を見ることが
できないと考えた。

\\

Sed hoc inconvenienter dicitur. Cum enim ultima hominis beatitudo in
altissima eius operatione consistat, quae est operatio intellectus, si
nunquam essentiam Dei videre potest intellectus creatus, vel nunquam
beatitudinem obtinebit, vel in alio eius beatitudo consistet quam in
Deo. Quod est alienum a fide.  In ipso enim est ultima perfectio
rationalis creaturae, quia est ei principium essendi, intantum enim
unumquodque perfectum est, inquantum ad suum principium attingit.

&

しかし、これは不適切に言われている。なぜなら、人間の究極の至福は、人間
の最高の働き、すなわち知性の働きにおいて成立するのだから、もし、被造の
知性がけっして神の本質を見ることができないとすれば、人間はけっして至福
を獲得しないか、あるいは、神以外のものにおいて人間の至福が成立すること
になるだろう。これは信仰に反する。なぜなら、理性的被造物の究極の完成は、
それ(神)においてあるからである。というのも、(神は)それ(理性的被造
 物)にとっての存在の根源であり、そして各々のものは、自らの根源に達し
ている限りにおいて、完全だからである。

\\

Similiter etiam est praeter rationem. Inest enim homini
naturale desiderium cognoscendi causam, cum intuetur effectum; et ex
hoc admiratio in hominibus consurgit. Si igitur intellectus rationalis
creaturae pertingere non possit ad primam causam rerum, remanebit
inane desiderium naturae. Unde simpliciter concedendum est quod beati
Dei essentiam videant.

&

また同様に理性にも反している。なぜなら、人
間には結果を見たときに原因を認識する本性的な欲求がある。そしてこのこ
とから、驚きが人々の中に生じる。それゆえもし理性的被造物の知性が諸
事物の第一の原因に到達することができなければ、本性の欲求が空しいままに留
まることになる。したがって至福者たちが神の本質を見ることは、端的に認め
られるべきである。

\\

{\sc  Ad primum ergo dicendum} quod utraque auctoritas loquitur de
 visione comprehensionis. Unde praemittit Dionysius immediate ante verba
 proposita, dicens, {\it omnibus ipse est universaliter incomprehensibilis,
 et nec sensus est}, et cetera. Et Chrysostomus parum post verba
 praedicta subdit, {\it visionem hic dicit certissimam Patris considerationem
 et comprehensionem, tantam quantam Pater habet de Filio}.

&

第一異論に対してはそれゆえ次のように言われるべきである。どちらの権威も、
把握というかたちの直視について語っている。だから、ディオニュシウスは、
示された言葉の直前に次のように述べている。「万物にとって、それはあまね
く把握不可能であり、それについての感覚もない」。また、クリュソストムス
は、示された言葉の少し後で、次のように述べている。「ここで言う直視は、
御父が御子について持つような、御父についてのもっとも確実な考察や把握で
ある」。

\\

{\sc  Ad secundum dicendum} quod infinitum quod se tenet ex parte
 materiae non perfectae per formam, ignotum est secundum se, quia omnis
 cognitio est per formam. Sed infinitum quod se tenet ex parte formae
 non limitatae per materiam, est secundum se maxime notum. Sic autem
 Deus est infinitus, et non primo modo, ut ex superioribus patet.

&

第二異論に対しては、次のように言われるべきである。形相によって完成され
ていない質料の側からある無限は、それ自体に即しては知られない。なぜなら
すべて認識は形相によってあるからである。これに対して質料によって制限さ
れていない形相の側からある無限は、それ自体に即して最大限に知られる。前
述のことから明らかなとおり、この後者のかたちで神は無限なのであり、前者
のかたちにおいてではない。

\\

{\sc  Ad tertium dicendum} quod Deus non sic dicitur non existens,
 quasi nullo modo sit existens, sed quia est supra omne existens,
 inquantum est suum esse. Unde ex hoc non sequitur quod nullo modo
 possit cognosci, sed quod omnem cognitionem excedat, quod est ipsum non
 comprehendi.

&

第三異論に対しては以下のように言われるべきである。神は、あらゆるしかた
で存在しないという意味で存在しないのではなく、自らの存在である限りにお
いて、すべて存在するものを越えるという意味で、存在しない。したがって、
このことから、神がいかなるかたちでも認識されえないということが帰結する
のではなく、すべての認識を超過すること、すなわち神が把握されないこと
が帰結する。

\\

{\sc Ad quartum dicendum} quod proportio dicitur dupliciter. Uno modo,
certa habitudo unius quantitatis ad alteram; secundum quod duplum,
triplum et aequale sunt species proportionis. Alio modo, quaelibet
habitudo unius ad alterum proportio dicitur. Et sic potest esse
proportio creaturae ad Deum, inquantum se habet ad ipsum ut effectus
ad causam, et ut potentia ad actum. Et secundum hoc, intellectus
creatus proportionatus esse potest ad cognoscendum Deum.

&

第四異論に対しては、次のように言われるべきである。比は、二通りに語られ
る。一つには、ある量から他の量への一定の関係である。この意味で、二倍、
三倍、等しい、などは、比の種である。別の意味では、あるものの他のものへ
の関係が、その関係がどのようなものであれ、比と言われる。この意味で、被
造物の神への関係が、結果の原因に対する関係、あるいは、可能態の現実態に
対する関係であるかぎりにおいてありうる。そしてこの限りで、被造の知性
は、神を認識することに対して比例的に関係しうる。

\end{longtable}

\newpage
\rhead{a.~2}

\begin{center}
{\Large {\bf ARTICULUS SECUNDUS}}\\
{\large UTRUM ESSENTIA DEI AB INTELLECTU CREATO PER ALIQUAM
 SIMILITUDINEM VIDEATUR} 

{\large 第二項\\
神の本質は、被造の知性によって、なんらかの類似を通して見られるか}
\end{center}

\begin{longtable}{p{21em}p{21em}}
{\sc Ad secundum sic proceditur}. Videtur quod essentia Dei ab
 intellectu creato per aliquam similitudinem videatur. Dicitur enim I
 {\itshape Ioan}.~{\scshape iii}, {\it scimus quoniam, cum apparuerit,
 similes ei erimus, et videbimus eum sicuti est}.

&

第二項の問題へ議論は以下のように進められる。神の本質は、被造の知性によっ
て、なんらかの類似を通して見られると思われる。理由は以下の通り。『ヨハ
ネの手紙1』第3章で「彼が現れたとき、私たちは彼に似たものとなり、彼をあ
りのままに見るであろうということを知っている」\footnote{「しかし、御子
が現れるとき、御子に似た者となることを知っています。なぜなら、そのとき
御子をありのままに見るからです。」(3:2) }と言われているからである。

\\

2.~{\sc  Praeterea}, Augustinus dicit, IX {\it de Trin}., {\it cum
Deum novimus, fit aliqua Dei similitudo in nobis}.

&

さらに、アウグスティヌスは、『三位一体論』第9巻で、「私たちが神を知る
とき、神のなんらかの類似が、私たちの中に生じる」\footnote{第11章第16節}と
言っている。

\\

3.~{\sc Praeterea}, intellectus in actu est
intelligibile\footnote{{\it intellectum} ABDEG.} in actu, sicut sensus
in actu est sensibile in actu. Hoc autem non est nisi inquantum
informatur sensus similitudine rei sensibilis, et intellectus
similitudine rei intellectae. Ergo, si Deus ab intellectu creato
videtur in actu, oportet quod per aliquam similitudinem videatur.

&

さらに、現実態における知性は、現実態における可知的なもの(知性認識され
たもの)である。ちょうど、現実態における感覚が、現実態における可感的な
ものであるように。しかるに、このことは、感覚が可感的事物の類似によって、
また知性が知性認識される事物の類似によって、形成される(形相付けられる)
かぎりにおいてそうであるのに他ならない。ゆえに、もし神が被造の知性によっ
て現実態において見られるのであれば、なんらかの類似を通して見られるので
なければならない。

\\

{\sc Sed Contra est} quod dicit Augustinus, XV {\it de Trin}., quod
cum apostolus dicit \guillemotleft videmus nunc per speculum et in
aenigmate\guillemotright, {\it speculi et aenigmatis nomine,
quaecumque similitudines ab ipso significatae intelligi possunt, quae
accommodatae sunt ad intelligendum Deum}. Sed videre Deum per
essentiam non est visio aenigmatica vel specularis, sed contra eam
dividitur. Ergo divina essentia non videtur per similitudines.

&

しかし反対に、アウグスティヌスが『三位一体論』第15巻\footnote{第9章16
節}で、使徒が「私たちは、今、鏡を通して、謎において見る」
\footnote{「私たちは、今は、鏡におぼろに映ったものを見ていますが、その
時には、顔と顔とを合わせて見ることになります。私は、今は一部分しか知り
ませんが、その時には、私が神にはっきり知られているように、はっきり知る
ことになります。」『コリントの信徒への手紙一』(13:12)(この箇所は以下
の文語訳の表現で日本でもよく知られている。「今われらは鏡をもて見るごと
く見るところ朧なり。然れど、かの時には顔をあわせて相見ん。今わが知ると
ころ全からず、然れど、かの時には我が知られたる如く全く知るべし」『コリ
ント前書』)}と言うとき、「鏡や謎という名前によって、何
であれそれによって示される、神を見るのに適した類似が理解されうる」と述
べている。しかるに、本質によって神を見ることは、謎や鏡による直視ではな
く、それらに対立して分かたれる。ゆえに、神の本質は、類似を通して見られ
るのではない。\footnote{聖書の文脈で、「今」(現世)と「その時」(来世)
を対比させていることが前提となっている。}

\\

{\sc Respondeo dicendum} quod ad visionem, tam sensibilem quam
intellectualem, duo requiruntur, scilicet virtus visiva, et unio rei
visae cum visu, non enim fit visio in actu, nisi per hoc quod res visa
quodammodo est in vidente. 

&

解答する。以下のように言われるべきである。感覚的に見ることにも、知性的に見ること
(知的直観)にも、二つのことが必要とされる。すなわち、見るちからと、
見られる事物と視覚との合一とである。なぜなら、現実態において見ることは、
見られる事物が、なんらかのかたちで見るもののうちにあることにおいてでな
ければ生じないからである。

\\

Et in rebus quidem corporalibus, apparet
quod res visa non potest esse in vidente per suam essentiam, sed solum
per suam similitudinem, sicut similitudo lapidis est in oculo, per
quam fit visio in actu, non autem ipsa substantia lapidis. 

&

そして、たしかに物体的事物においては、見られる事物がその本質によって見
るものの中に存在することができず、ただ自らの類似によって見るものの中に存
在することができることが明らかである。たとえば、石の類似が目の中に存在
し、その類似によって、現実態において見ることが生じるのであって、石の実
体そのものが目の中に存在するというわけではない。

\\

Si autem
esset una et eadem res, quae esset principium visivae virtutis, et
quae esset res visa, oporteret videntem ab illa re et virtutem visivam
habere, et formam per quam videret.

&

しかしながら、もし、見るちからの根源でありながら見られる事物でもあるような、一
つの同一の事物があったならば、見る者は、見るちからと、それによって見
るところの形相とを、その事物から持たなければならなかったであろう。

\\

Manifestum est autem quod Deus et est auctor intellectivae virtutis,
et ab intellectu videri potest. Et cum ipsa intellectiva virtus
creaturae non sit Dei essentia, relinquitur quod sit aliqua
participata similitudo ipsius, qui est primus intellectus. 

&

ところで、神が、知性認識しうるちからを作り出すものであり、かつ、知性に
よって見られうるということは明らかである。そして、被造物の知性認識しう
るちからは神の本質でないので、第一知性である神のなんらかの分有された類
似であるということになる。

\\

Unde et virtus intellectualis creaturae lumen quoddam intelligibile
dicitur, quasi a prima luce derivatum, sive hoc intelligatur de
virtute naturali, sive de aliqua perfectione superaddita gratiae vel
gloriae. Requiritur ergo ad videndum Deum aliqua Dei similitudo ex
parte visivae potentiae, qua scilicet intellectus sit efficax ad
videndum Deum.

 &

このことから、第一の光から派生したものとして、被造物の知的なちからも、な
んらかの可知的な光と言われる。このことが自然本性的なちからについて理解
されるにせよ、なんらかの付加された恩恵あるいは栄光の完全性について理解
されるにせよ、そうである。ゆえに神を見るためには、神のなんらかの類似が
見る能力の側から必要とされる。すなわち、知性が\kenten{それ}によって神
を見ることができるようになるような\kenten{その}類似である。

\\

Sed ex parte visae rei, quam necesse est aliquo modo uniri videnti per
nullam similitudinem creatam Dei essentia videri potest. Primo quidem,
quia, sicut dicit Dionysius, {\scshape i} cap.\ {\itshape de Div.\
Nom.}, per similitudines inferioris ordinis rerum nullo modo superiora
possunt cognosci, sicut per speciem corporis non potest cognosci
essentia rei incorporeae. Multo igitur minus per speciem creatam
quamcumque potest essentia Dei videri.

&

しかし、見られる事物、これはなんらかのかたちで見るものに合一される必要
があるのだが、その事物の側から言えば、被造のどんな類似によっても、神の
本質は見られえない。理由は以下の通りである。第一に、ちょうどディオニュ
シウスが『神名論』第1章で述べるように、下位の秩序に属する事物の類似に
よって、上位のものはけっして認識されない。たとえば、物体の形象によって、
非物体的な事物の本質は認識されえない。ゆえに、被造のどんな形象によって
も、さらにも増して、神の本質は見られえない。

\\

---Secundo, quia essentia Dei est ipsum esse eius, ut supra ostensum
est, quod nulli formae creatae competere potest. Non potest igitur
aliqua forma creata esse similitudo repraesentans videnti Dei
essentiam.

&

第二に、前に示されたように、神の本質は神の存在そのものである。このこと
はどんな被造物の形相にも適合しえない。ゆえに、どんな被造の形相も、見
るものに神の本質を表現する類似ではありえない。

\\

---Tertio, quia divina essentia est aliquod incircumscriptum,
continens in se supereminenter quidquid potest significari vel
intelligi ab intellectu creato. Et hoc nullo modo per aliquam speciem
creatam repraesentari potest, quia omnis forma creata est determinata
secundum aliquam rationem vel sapientiae, vel virtutis, vel ipsius
esse, vel alicuius huiusmodi. Unde dicere Deum per similitudinem
videri, est dicere divinam essentiam non videri, quod est erroneum.

&

第三に、神の本質は、ある制限のないものであり、そのうちに、より優れたか
たちで、被造の知性によって表示されたり知性認識されたりしうるものをなん
であれすべて含んでいる。そしてこのことは、被造のいかなる形象によって
も表現されえない。なぜなら、被造の形相はすべて、知恵、ちから、存在、そ
のようななんらかの概念(性格)において、限定されているからである。した
がって、神が類似によって見られると言うことは、神の本質が見られないと言
うことだが、これは誤りである。

\\

Dicendum ergo quod ad videndum Dei essentiam requiritur aliqua
similitudo ex parte visivae potentiae, scilicet lumen gloriae,
confortans intellectum ad videndum Deum, de quo dicitur in {\itshape
Psalmo}: {\it in lumine tuo videbimus lumen}. Non autem per aliquam
similitudinem creatam Dei essentia videri potest, quae ipsam divinam
essentiam repraesentet ut in se est.

&

ゆえに、次のように言われるべきである。神の本質を見るために、なんらかの
類似が見る能力の側から必要とされる。すなわちそれは、神を見るために知
性を強化する栄光の光である。これについて、『詩編』で「あなたの光におい
て、私たちは光を見るでしょう」\footnote{「あなたの光に、私たちは光を見
る。」(36:10)}と言われている。しかし、神の本質を自らのうちにあるものと
して表現するようななんらかの被造の類似によって、神の本質が見られるこ
とは不可能である。

\\

{\sc  Ad primum ergo dicendum} quod auctoritas illa loquitur de
similitudine quae est per participationem luminis gloriae.  

&

第一異論に対しては、それゆえ、次のように言われるべきである。かの権威は、栄光の
光の分有によってある類似について語っている。

\\

{\sc  Ad secundum dicendum} quod Augustinus ibi loquitur de
cognitione Dei quae habetur in via.

&

第二異論に対しては、次のように言われるべきである。アウグスティヌスは、
そこで、現生においてもたれる神の認識について語っている。

\\

{\sc Ad tertium dicendum} quod divina essentia est ipsum esse. Unde,
sicut aliae formae intelligibiles quae non sunt suum esse, uniuntur
intellectui secundum aliquod esse quo informant ipsum intellectum et
faciunt ipsum in actu; ita divina essentia unitur intellectui creato
ut intellectum in actu, per seipsam faciens intellectum in actu.

&

第三異論に対しては、次のように言われるべきである。神の本質は存在そのも
のである。したがって、ちょうど、自らの存在でない他の可知的な形相が、何
らかの存在において知性に合一され、その存在によって、その知性に形相を与
え、知性を現実態にするように、そのように、神の本質は、現実態における知
性認識されたものとして合一され、自らによって、知性を現実態にする。

\end{longtable}
\newpage

\rhead{a.~3}
\begin{center}
{\Large {\bf ARTICULUS TERTIUS}}\\
{\large UTRUM ESSENTIA DEI VIDERI POSSIT OCULIS CORPORALIBUS}\\
{\large 第三項\\神の本質は、肉の目によって見られうるか}
\end{center}

\begin{longtable}{p{21em}p{21em}}

{\sc \large Ad tertium sic proceditur}. Videtur quod essentia Dei videri
possit oculo corporali. Dicitur enim {\it Iob} XIX, {\it in carne mea videbo Deum,
etc}.; et XLII, {\it auditu auris audivi te, nunc autem oculus meus videt te}.

&

第三項の問題へ議論は以下のように進められる。神の本質は、肉の目によって
見られうると思われる。なぜなら、『ヨブ記』19章で「私は私の肉において神
を見るだろう」\footnote{(19:26)「この身をもって私は神を仰ぎ見るであろ
う」(新共同訳)、「われ肉を離れて神を見ん」(文語訳)}と言われ、42章
で「耳の聴覚によって私はあなたを聞いた。しかし今、私の目があなたを見る」
\footnote{(42:5)「あなたのことを、耳にしてはおりました。しかし今、この
目であなたを仰ぎ見ます」(新共同訳)、「われ汝の事を耳にて聞きゐたりし
が、今は目をもて汝を見たてまつる」(文語訳)}と言われている。

\\

2.~{\sc Praeterea}, Augustinus dicit, ultimo {\it de Civitate Dei},
cap.\ {\scshape xxix}, {\it Vis itaque praepollentior erit illorum}
(scilicet glorificatorum), {\it non ut acutius videant quam quidam
perhibentur videre serpentes vel aquilae (quantalibet enim acrimonia
cernendi eadem animalia vigeant, nihil aliud possunt videre quam
corpora), sed ut videant et incorporalia}.  Quicumque autem potest
videre incorporalia, potest elevari ad videndum Deum. Ergo oculus
glorificatus potest videre Deum.

&

さらに、アウグスティヌスは『神の国』の最終巻、第29章で、次のように
言っている。「また、彼ら(すなわち、栄光を受けた人々)の目のちからはよ
り優れたものとなるであろう。その結果、鋭く見ると言われているある種のヘ
ビやワシよりも鋭く見るというわけではない(どんなにこれらの動物が鋭い識
別力を持つとしても、それらは物体以外のものを見ることはできない)。そう
ではなくて、彼らは、非物体的なものも見るようになる」。ところで、非物体
的なものを見ることができるものはだれであれ、神を見ることへ引き上げられ
うる。ゆえに、栄光を受けた目は、神を見ることができる。

\\

3.~{\sc Praeterea}, Deus potest videri ab homine visione imaginaria,
dicitur enim {\itshape Isaiae} {\scshape vi}, {\it vidi dominum sedentem
super solium}, et cetera. Sed visio imaginaria a sensu originem habet,
phantasia enim est {\it motus factus a sensu secundum actum}, ut dicitur
in III {\it de Anima}. Ergo Deus sensibili visione videri potest.

&

さらに、神は、人間によって、表象的な視(visio imaginaria)によって見られう
る。なぜなら、『イザヤ書』6章で、「私は主が席に座っているのを見た、云々」
と言われているからである。しかるに、表象的な視は、感覚に起源を持つ。なぜ
なら、『デ・アニマ』第3巻で言われるように、表象像とは「現実態において、感
覚によって作られた運動」だからである。ゆえに、神は、感覚的な視によって見
られうる。

\\

{\sc Sed contra est} quod dicit Augustinus, in libro {\it de Videndo
Deum} ad Paulinam, {\it Deum nemo vidit unquam, vel in hac vita, sicut
ipse est; vel in Angelorum vita, sicut visibilia ista quae corporali
visione cernuntur}.

&

しかし反対に、アウグスティヌスは、パウリナ宛書簡「神を見ることについて」
で、次のように述べている。「現生において、神をあるがままに見た人は、けっ
してだれもいない。天使の生においても、肉の目によって知られるこの可視的
なものどものようには、だれもけっして見なかった」。

\\

{\sc Respondeo dicendum} quod impossibile est Deum videri sensu visus,
 vel quocumque alio sensu aut potentia sensitivae partis. Omnis enim
 potentia huiusmodi est actus corporalis organi, ut infra
 dicetur. Actus autem proportionatur ei cuius est actus. Unde nulla
 huiusmodi potentia potest se extendere ultra corporalia. Deus autem
 incorporeus est, ut supra ostensum est. Unde nec sensu nec
 imaginatione videri potest, sed solo intellectu.

&

解答する。以下のように言われるべきである。神が、視覚や、なんであれ他の
感覚や感覚的部分の能力によって見られることは不可能である。なぜなら、そ
のような能力はすべて、後で述べられるように\footnote{『神学大全』第1部
第78問第1項}、身体的器官の現実態である。しかるに、現実態は、\kenten{そ
れ}の現実態であるところの\kenten{それ}に、比例する。したがって、このよ
うなどんな能力も、物体的なものを超えてちからを及ぼすことがない。しかる
に神は、前に示されたとおり、非物体的である。したがって、感覚によっても
表象によっても、神は見られえず、ただ知性によってのみ見られる。

\\

{\sc Ad primum ergo dicendum} quod, cum dicitur {\it in carne mea
videbo Deum, Salvatorem meum}, non intelligitur quod oculo carnis sit
Deum visurus, sed quod in carne existens, post resurrectionem, visurus
sit Deum. Similiter quod dicitur, {\it nunc oculus meus videt te},
intelligitur de oculo mentis, sicut {\it Ephes}.\ {\scshape i} dicit
Apostolus, {\it det vobis spiritum sapientiae in agnitione eius,
illuminatos oculos cordis vestri}.

&

第一に対しては、それゆえ、次のように言われるべきである。私の肉において、
神を、私を救う者を見るだろう」と言われるとき、肉の目によって神を見るだ
ろうということが理解されるのではなく、復活ののちに、肉において存在する
者が、神を見るだろう、と理解される。同様に、「今、私の目があなたを見る」
も、ちょうど『エフェソの信徒への手紙』第1章で、使徒が「あなた方に、彼
を知ることにおいて、知恵の霊を、照らされたあなた方の心の目を与えますよ
うに」\footnote{「あなた方に知恵と啓示との霊を与え、神を深く知ることが
できるようにし、心の目を開いてくださるように」(1:17-18)}と述べるよう
に、精神の目について理解される。

\\

{\sc Ad secundum dicendum} quod Augustinus loquitur inquirendo in
verbis illis, et sub conditione. Quod patet ex hoc quod praemittitur,
{\it Longe itaque potentiae alterius erunt} (scilicet oculi
glorificati), {\it si per eos videbitur incorporea illa natura}. Sed
postmodum hoc determinat, dicens: {\it Valde credibile est sic nos
visuros mundana tunc corpora caeli novi et terrae novae, ut Deum
ubique praesentem, et universa etiam corporalia gubernantem,
clarissima perspicuitate videamus; non sicut nunc invisibilia Dei per
ea quae facta sunt intellecta conspiciuntur; sed sicut homines, inter
quos viventes motusque vitales exerentes vivimus, mox ut aspicimus,
non credimus vivere, sed videmus}. (continues)

&

第二に対しては、次のように言われるべきである。アウグスティヌスは、それ
らの言葉において、探究しながら、条件付きで語っている。このことは、その
前で「もし彼らによって、かの非物体的な本性が見られるならば、それら(す
なわち、栄光を受けた目)は、きわめて異なる能力に属するだろう」と言われ
ていることからも明らかである。しかし、後に、この点を詳述して次のように
述べている。「次のことがきわめてよく信じられる。すなわち、私たちは、そ
のとき、新しい天と新しい地に属する世界の物体を、至る所に現前し、すべて
の物体的なものを統治する神をこの上なく明らかな透明さによって見るように
見るであろう。今は、神の見えないところが、創られたものを通して知られ、
観想されるが、そのようなかたちではなく、むしろ、私たちは人間たちのあい
だで生き、生命の運動を働かせながら生きているが、その人間たちを見るやい
なや、彼らが生きていることを信じるのではなく、彼らが生きていることを見
て取る」。

\\

Ex quo patet quod hoc modo intelligit oculos glorificatos Deum
visuros, sicut nunc oculi nostri vident alicuius vitam. Vita autem non
videtur oculo corporali, sicut per se visibile, sed sicut sensibile
per accidens, quod quidem a sensu non cognoscitur, sed statim cum
sensu ab aliqua alia virtute cognoscitiva. Quod autem statim, visis
corporibus, divina praesentia ex eis cognoscatur per intellectum, ex
duobus contingit, scilicet ex perspicacitate intellectus; et ex
refulgentia divinae claritatis in corporibus innovatis.

&

このことから、ちょうど私たちの目がある人の生命を見るように、栄光を受け
た目が神を見るであろうと、彼(アウグスティヌス)が理解していることが明
らかである。しかるに、生命は、自体的に見られうるもののようには肉の目に
よって見られず、むしろ、附帯的に感覚されうるものである。つまりそれは、
感覚によって認識されないが、感覚とともに、なんらかの他の認識のちからに
よって、直ちに認識される。ところで、物体が見られると、直ちに、それらに
もとづいて神の現在が知性によって認識されるということは、二つの点から生
じる。すなわち、一つには、知性の透徹さによってであり、もう一つは、新し
くされた物体に神の明るさが反射することによってである。

\\

{\sc Ad tertium dicendum} quod in visione imaginaria non videtur Dei
essentia, sed aliqua forma in imaginatione formatur, repraesentans
Deum secundum aliquem modum similitudinis, prout in Scripturis divinis
divina per res sensibiles metaphorice describuntur.

&

第三に対しては、次のように言われるべきである。表象的に見ることにおいて、
神の本質は見られえず、神をなんらかのかたちの類似において表現するような、
なんらかの形相が、表象において形成される。ちょうど、聖書において、神に
かんすることが、可感的事物によって比喩的に描写されるように。

\end{longtable}

\newpage
\rhead{a.~4}

 \begin{center}
  {\Large {\bf ARTICULUS QUARTUS}}\\
{\large UTRUM ALIQUIS INTELLECTUS CREATUS PER SUA NATURALIA\\
DIVINAM ESSENTIAM VIDERE POSSIT}\\
{\large 第四項\\なんらかの被造の知性が\\自らが持つ本性的なものによって神の本質を見ることができるか}
 \end{center}

\begin{longtable}{p{21em}p{21em}}

{\sc Ad quartum sic proceditur}. Videtur quod aliquis intellectus
creatus per sua naturalia divinam essentiam videre possit. Dicit enim
Dionysius, cap.\ IV {\it de Div.\ Nom}., quod Angelus est {\it
speculum purum, clarissimum, suscipiens totam, si fas est dicere,
pulchritudinem Dei}. Sed unumquodque videtur dum videtur eius
speculum. Cum igitur Angelus per sua naturalia intelligat seipsum,
videtur quod etiam per sua naturalia intelligat divinam essentiam.

&

第四項の問題へ議論は以下のように進められる。なんらかの被造の知性が、自
らが持つ本性的なもの\footnote{この項の「本性」(natura)は恩恵(gratia)に対立し、生まれながらそのものに備わっている特性が「本性的なもの」と呼ばれている。本項反対異論を参照。}によって神の本質を見ることができると思われる。なぜ
なら、ディオニュシウスは、『神名論』第4章で、天使は「純粋でこの上なく
明るい鏡であり、もし言ってよければ、神のすべての美しさを受け取っている」
と述べている。しかるに、各々のものは、その鏡が見られているあいだ、見ら
れている。ゆえに、天使は、自らが持つ本性的なものによって自己自身を認識する
のだから、自らが持つ本性的なものによって神の本質も認識すると思われる。

\\

2.~{\sc Praeterea}, illud quod est maxime visibile, fit minus visibile
nobis propter defectum nostri visus, vel corporalis vel
intellectualis. Sed intellectus Angeli non patitur aliquem
defectum. Cum ergo Deus secundum se sit maxime intelligibilis, videtur
quod ab Angelo sit maxime intelligibilis. Si igitur alia
intelligibilia per sua naturalia intelligere potest, multo magis Deum.

&

さらに、最大限に見られうるものは、私たちの視覚(身体的なものであれ知性
的なものであれ)の欠陥のために、私たちにとってはより少なく見られうるも
のとなる。しかるに、天使の知性は、何ら欠陥を被っていない。ゆえに、神は、
それ自体において最大限に可知的なものだから、天使によって、最大限に可知
的であると思われる。ゆえに、もし、(天使が、)自らが持つ本性的なものを通し
て他の可知的なものどもを認識できるのだとすれば、ましてや神も認識できる。

\\

3.~{\sc Praeterea}, sensus corporeus non potest elevari ad
intelligendam substantiam incorpoream, quia est supra eius naturam. Si
igitur videre Deum per essentiam sit supra naturam cuiuslibet
intellectus creati, videtur quod nullus intellectus creatus ad
videndum Dei essentiam pertingere possit, quod est erroneum, ut ex
supradictis patet. Videtur ergo quod intellectui creato sit naturale
divinam essentiam videre.

&

さらに、身体の感覚が、非物体的実体を知性認識することまで上げられること
ができないのは、その本性を越えるからである。ゆえに、神を本質によって見
ることが、どんな被造知性の本性も越えるのであれば、どんな被造知性も神の
本質を見ることまで至ることができないと思われる。しかし、これは、上で述
べられたことからわかるように、誤りである。ゆえに、被造知性にとって、神
の本質を見ることは、本性的であると思われる。

\\

{\sc Sed contra est} quod dicitur {\it Rom}.\ {\sc vi}: {\it gratia
Dei vita aeterna}. Sed vita aeterna consistit in visione divinae
essentiae, secundum illud Ioan.\ {\sc xvii}: {\it haec est vita
aeterna, ut cognoscant te solum verum Deum}, etc. Ergo videre Dei
essentiam convenit intellectui creato per gratiam, et non per naturam.

&

しかし反対に、『ローマの信徒への手紙』第6章で「神の恩恵は永遠の生であ
る」と言われている\footnote{「神の賜物は、私たちの主キリスト・イエスに
よる永遠の命なのです」(6:23)}。しかるに、永遠の生は、『ヨハネによる福
音書』17章「唯一の真なる神であるあなたを認識すること、これが永遠の生で
ある」\footnote{「永遠の命とは、唯一のまことの神であるあなたと、あなた
のお遣わしになったイエス・キリストを知ることです」(17:3)}によれば、神
を見ることにおいて成立する。ゆえに、神の本質を見ることは、恩恵によって、
被造の知性に適合するのであり、本性的にではない。

\\

{\sc Respondeo dicendum} quod impossibile est quod aliquis intellectus
creatus per sua naturalia essentiam Dei videat. Cognitio enim
contingit secundum quod cognitum est in cognoscente. Cognitum autem
est in cognoscente secundum modum cognoscentis. Unde cuiuslibet
cognoscentis cognitio est secundum modum suae naturae. Si igitur modus
essendi alicuius rei cognitae excedat modum naturae cognoscentis,
oportet quod cognitio illius rei sit supra naturam illius
cognoscentis.

&

解答する。以下のように言われるべきである。なんらかの被造知性が、自らが
持つ本性的なものによって神の本質を見ることは不可能である。理由は以下の
通り。認識は、認識されるものが認識するもののうちにあることにおいて生じ
る。ところで、認識されるものは、認識するものの様態\footnote{「様態」と訳されたmodusは、とりあえず「ある特定の制限を伴うありかた」という意味に理解しておく。}にしたがって、認識する
もののうちにある。このことから、どのような認識者の認識も、自らの本性の様態
にしたがってある。ゆえに、もし、任意の認識された事物の存在の様態(modus
essendi)が、認識するものの本性の様態(modus naturae)を越えるならば、そ
の事物を認識することは、その認識者の本性を越えるはずである。

\\

Est autem multiplex modus essendi rerum. Quaedam enim sunt, quorum
natura non habet esse nisi in hac materia individuali, et huiusmodi
sunt omnia corporalia. Quaedam vero sunt, quorum naturae sunt per se
subsistentes, non in materia aliqua, quae tamen non sunt suum esse,
sed sunt esse habentes, et huiusmodi sunt substantiae incorporeae,
quas Angelos dicimus. Solius autem Dei proprius modus essendi est, ut
sit suum esse subsistens.

&

ところで、諸事物の存在の様態は多様である。すなわち、ある事物は、この個
的な質料においてでなければ存在を持たないような本性を持つが、すべての物
体的なものはこのようなものである。またあるものは、その本性が、なんらか
の質料においてではなく、それ自体で自存するが、しかし、それは自らの存在
でなく、むしろ存在を持つものである。私たちが天使と呼ぶ、非物体的実体は
このようなものである。これらに対して、自存する自らの存在であるという存
在の様態は、ただ神にのみ属する固有の様態である。

\\

Ea igitur quae non habent esse nisi in materia individuali, cognoscere
est nobis connaturale, eo quod anima nostra, per quam cognoscimus, est
forma alicuius materiae. Quae tamen habet duas virtutes
cognoscitivas. Unam, quae est actus alicuius corporei organi. Et huic
connaturale est cognoscere res secundum quod sunt in materia
individuali, unde sensus non cognoscit nisi singularia. Alia vero
virtus cognoscitiva eius est intellectus, qui non est actus alicuius
organi corporalis. Unde per intellectum connaturale est nobis
cognoscere naturas, quae quidem non habent esse nisi in materia
individuali; non tamen secundum quod sunt in materia individuali, sed
secundum quod abstrahuntur ab ea per considerationem intellectus. Unde
secundum intellectum possumus cognoscere huiusmodi res in universali,
quod est supra facultatem sensus. 

&

ゆえに、個的な質料においてしか存在を持たないものどもを認識することが、
私たちの本性にとって親和的\footnote{「親和的な」と訳したconnaturaleは、
cum+naturaleであり「本性を同じくする」というのが語源的な意味。}である。
なぜなら、私たちは、魂によって認識をするが、その魂は、なんらかの質料の
形相だからである。しかし、それは二つの認識のちからを持っている。一つは、
なんらかの身体的器官の現実態であるようなちからである。そしてこれに親和
的なのは、個的な質料においてある限りにおける事物を認識することであり、
このことから、感覚は個物しか認識しない。他方、それ(人間の魂)のもう一
つの認識のちからは知性であり、それは、どんな身体器官の現実態でもない。
このことから、私たちに親和的なのは、知性によって、個的な質料においてし
か存在を持たなような本性を認識することだが、ただし、それらが個的な質料
においてある限りで認識するのではなく、知性の考察によって、個的な質料か
ら抽象されるかぎりにおいて認識することである。したがって、私たちは、知
性において、そのような事物を普遍において認識することができるが、これは
感覚の機能を越えている。

\\

Intellectui autem angelico
connaturale est cognoscere naturas non in materia existentes. Quod est
supra naturalem facultatem intellectus animae humanae, secundum statum
praesentis vitae, quo corpori unitur.

&

これに対して、天使の知性に親和的なのは、
質料において存在しない本性を認識することである。これは、身体に合一して
いる現生の状態においては、人間の魂の知性の機能を越える。

\\

Relinquitur ergo quod cognoscere ipsum esse subsistens, sit
connaturale soli intellectui divino, et quod sit supra facultatem
naturalem cuiuslibet intellectus creati, quia nulla creatura est suum
esse, sed habet esse participatum. Non igitur potest intellectus
creatus Deum per essentiam videre, nisi inquantum Deus per suam
gratiam se intellectui creato coniungit, ut intelligibile ab ipso.

&

ゆえに、自存する存在そのものを認識することは、ただ神の知性だけに親和的
であり、また、それがどんな被造の知性の本性的機能も越えることになる。な
ぜなら、どんな被造物も自らの存在ではなく、分有された存在を持つからであ
る。ゆえに、被造の知性が神を本質によって見ることはできず、できるとすれ
ば、神が、その恩恵によって、自らを被造の知性に結びつけ、被造物によって
可知的となるようにする限りにおいてでしかない。

\\

{\sc Ad primum ergo dicendum} quod iste modus cognoscendi Deum, est
Angelo connaturalis, ut scilicet cognoscat eum per similitudinem eius
in ipso Angelo refulgentem. Sed cognoscere Deum per aliquam
similitudinem creatam, non est cognoscere essentiam Dei, ut supra
ostensum est. Unde non sequitur quod Angelus per sua naturalia possit
cognoscere essentiam Dei.

&

第一異論に対しては、それゆえ、次のように言われるべきである。天使には、
天使自身の中に反射する神の類似をとおして神を認識するという、そういう神
認識のかたちが、親和的である。しかし、上で示されたように、被造のなんら
かの類似を通して神を認識することは、神の本質を認識することではない。し
たがって、天使が、自らが持つ本性的なものを通して神の本質を認識しうるという
ことは帰結しない。

\\

{\sc Ad secundum dicendum} quod intellectus Angeli non habet defectum,
si defectus accipiatur privative, ut scilicet careat eo quod habere
debet. Si vero accipiatur negative, sic quaelibet creatura invenitur
deficiens, Deo comparata, dum non habet illam excellentiam quae
invenitur in Deo.

&

第二異論に対しては以下のように言われるべきである。欠陥が、持つべきもの
を持っていないという意味で、欠如として理解されるならば、天使の知性は、
欠陥を持たない。他方、否定という意味で理解されるならば、どんな被造物も、
神に比べると、神において見出される卓越性を持っていない限りにおいて、欠
陥を持つものだと考えられる。

\\

{\sc Ad tertium dicendum} quod sensus visus, quia omnino materialis
est, nullo modo elevari potest ad aliquid immateriale. Sed intellectus
noster vel angelicus, quia secundum naturam a materia aliqualiter
elevatus est, potest ultra suam naturam per gratiam ad aliquid altius
elevari. 

&

第三異論に対しては、次のように言われるべきである。視覚は、まったく質料
的なので、なんらかの非質料的なものまで引き上げられることはけっしてでき
ない。しかし、私たちや天使の知性は、本性において、なんらかのかたちで質
料から引き上げられているので、恩恵によって、何かより高いものへ、自らの
本性を越えて引き上げられることができる。

\\

Et huius signum est, quia visus nullo modo potest in
abstractione cognoscere id quod in concretione cognoscit, nullo enim
modo potest percipere naturam, nisi ut {\it hanc}. Sed intellectus
noster potest in abstractione considerare quod in concretione
cognoscit. Etsi enim cognoscat res habentes formam in materia, tamen
resolvit compositum in utrumque, et considerat ipsam formam per se. 

&

そのしるしに、視覚は、けっして、
具体的なものにおいて認識するものを、抽象において認識することができない。
なぜなら、視覚は、「この本性」というかたちでしか、本性をとらえることが
できないからである。これに対して、私たちの知性は、具体的なものにおいて
認識するものを、抽象において考察することができる。じっさい、(知性は、)
質料において形相を持つ事物を考察するが、しかし、その複合体を両者に分解
し、形相そのものを自体的に考察する。

\\

Et
similiter intellectus Angeli, licet connaturale sit ei cognoscere esse
concretum in aliqua natura, tamen potest ipsum esse secernere per
intellectum, dum cognoscit quod aliud est ipse, et aliud est suum
esse. 

&

同様に、天使の知性は、何らかの本性
における具体的な存在を認識することが親和的だが、しかし、\kenten{それ}と
\kenten{それの存在}は別であることを認識する限りにおいて、その存在を知
性によって切り離すことができる。

\\

Et ideo, cum intellectus creatus per suam naturam natus sit
apprehendere formam concretam et esse concretum in abstractione, per
modum resolutionis cuiusdam, potest per gratiam elevari ut cognoscat
substantiam separatam subsistentem, et esse separatum subsistens.

&

ゆえに、被造の知性は、自己の本性によっ
て、生まれつき、具体的な形相と具体的な存在を、なんらかの分解という方法
によって抽象において把握するのだから、分離して自存する実体と、分離して
自存する存在を認識することへ、恩恵によって引き上げられることが可能であ
る。

\end{longtable}
\newpage
\rhead{a.~5}

\begin{center}
{\Large {\bf ARTICULUS QUINTUS}}\\
{\large UTRUM INTELLECTUS CREATUS AD VIDENDUM DEI ESSENTIAM\\
ALIQUO CREATO LUMINE INDIGEAT}\\
{\large 第五項\\
被造知性は、神の本質を見るために、なんらかの被造の光を必要とするか}
\end{center}

\begin{longtable}{p{21em}p{21em}}
{\huge A}{\sc  d quintum sic proceditur}. Videtur quod intellectus creatus ad
 videndum essentiam Dei aliquo lumine creato non indigeat. Illud enim
 quod est per se lucidum in rebus sensibilibus, alio lumine non indiget
 ut videatur, ergo nec in intelligibilibus. Sed Deus est lux
 intelligibilis. Ergo non videtur per aliquod lumen creatum.

&

第五項の問題へ議論は以下のように進められる。被造知性は、神の本質を見る
ために、どんな被造の光も必要としないと思われる。理由は以下の通り。可感
的事物の中で、自体的に(=それ自身によって)光っているものは、それが見られるために、他の光を
必要としない。ゆえに、可知的なものにおいても(他の光を)必要としない。しかるに、神は
可知的な光である。ゆえに、なんらかの被造の光によって見られるのではない。\footnote{lumenとluxはどちらも「光」だが、トマスは別の箇所で、前者を大気中にある光、後者を光源の中から発してくる光だと説明している。『命題集注解』第2巻第13区分第1問、『神学大全』第1部第67問を参照。}

\\

2.~{\sc Praeterea}, cum Deus videtur per medium, non videtur per suam
essentiam. Sed cum videtur per aliquod lumen creatum, videtur per
medium. Ergo non videtur per suam essentiam.

&

さらに、神が媒介によって見られるとき、本質によって見られていない。しか
るに、なんらかの被造の光によって見られるとき、媒介によって見られている。
ゆえに、(そのときには)神はその本質によって見られていない。

\\

3.~{\sc Praeterea}, illud quod est creatum, nihil prohibet alicui
creaturae esse naturale. Si ergo per aliquod lumen creatum Dei
essentia videtur, poterit illud lumen esse naturale alicui
creaturae. Et ita illa creatura non indigebit aliquo alio lumine ad
videndum Deum, quod est impossibile. Non est ergo necessarium quod
omnis creatura ad videndum Dei essentiam lumen superadditum requirat.

&

さらに、創造されたあるものが、何らかの被造物にとって本性的であることを妨げるものは何もない。
ゆえに、もし、なんらかの被造の光によって神の本質が見られるならば、その光は、なんらかの被造物にとって本性的であ
りうることになる\footnote{このpoteritの未来形は論理的帰結を表す。}。そして、その被造物は、神を見るために、他の光をなんら
必要としないことになる\footnote{このindigebitも(poteritと並んで)論理的帰結を表す。}が、これは不可能である。\footnote{``quod est inpossibile''に
ついて、山田訳では、「この「これは不可能である」quod est impossibileは、
議論の筋からいうと、ないほうがよいように思われるが、レオニナ、ピアナい
ずれのテキストにも在るから、そのまま訳しておく(p. 338)」とある。
Blackfriars版のMcCabeの訳も、そのまま訳出している。論点は、被造の光に
よって神の本質が見られるならば、その光が被造物であるかぎり、それを本性
的にもつ被造物がありうることになるが、それは、前項により不可能、という
こと。箇条書きで論旨を明らかにすると以下の通りとなる。
\begin{enumerate}
 \item あらゆる被造物は、神を見るために、被造の光を必要とする。(背理法の仮定)
 \item 任意の被造物$a$について、$a$を自然本性的にもつものが存在することが可能。
       (この異論の仮定)
 \item 被造の光は被造物である。(定義により)
 \item 被造の光を自然本性的にもつもの$b$が存在することが可能。(2, 3)
 \item 被造の光を持つものはすべて、神の本質を見る。(この異論の仮定)
 \item $b$が自然本性的に神を見ることが可能。(4, 5)
 \item これは本問第4項の結論に反する。
 \item ゆえに1は偽。すなわち、「あらゆる被造物は、神を見るために、被造の光
       を必要とする」は偽。
\end{enumerate}
}
ゆえに、「あらゆる被造物が、神の本質を見るために、付加された光を必要とする」は必要でない\footnote{Non est necessarium quodは、「〜は必然的でない」と訳すのが通例だが、ここでは必然性が問題になっているわけではないので、事実として要求される必要がないという意味で「必要ない」と訳す。}。

\\

{\sc  Sed contra est} quod dicitur in Psalmo, {\it in lumine tuo videbimus lumen}.

&

しかし反対に、『詩編』で「あなたの光の中で、私たちは光を見る」
\footnote{「あなたの光に、私たちは光を見る」(36:10)(新共同訳)}と言わ
れている。

\\

{\sc Respondeo dicendum} quod omne quod elevatur ad aliquid quod
excedit suam naturam, oportet quod disponatur aliqua dispositione quae
sit supra suam naturam, sicut, si aer debeat accipere formam ignis,
oportet quod disponatur aliqua dispositione ad talem formam. 

&

解答する。以下のように言われるべきである。自らの本性を越えるものへと引
き上げられるものはすべて、自らの本性の上にあるなんらかの態勢
\footnote{「態勢」はdispositioの訳語。英語のdispositionは「傾向性」と
訳されるが、トマスで「傾向性」と訳されるのはinclinatioなので、訳語が足
りなくなって苦し紛れに使われている。必ずしも日本語として自然ではない。
inclinatio, dispositio, habitusの順で安定的になっていく、ある種の状態
のことを指す。この文脈で、動詞disponoは「態勢付ける」と訳されるが、日本語としてはさらに苦しい。}にならなければならない。たとえば、空気が火の形相を受け
取る必要があるならば、空気がそのような形相に対応する態勢になる必要があ
る。

\\

Cum autem aliquis intellectus creatus videt Deum per essentiam, ipsa
essentia Dei fit forma intelligibilis intellectus. Unde oportet quod
aliqua dispositio supernaturalis ei superaddatur, ad hoc quod elevetur
in tantam sublimitatem.

&

さて、ある被造知性が神を本質によって見るとき、神の本質そのものが、知性
の可知的形相となる。したがって、それほどまでの高みに引き上げられる
ためには、その本性を越えたなんらかの態勢が、知性に与えられれる必要があ
る。

\\

Cum igitur virtus naturalis intellectus creati
non sufficiat ad Dei essentiam videndam, ut ostensum est, oportet quod
ex divina gratia superaccrescat ei virtus intelligendi. Et hoc
augmentum virtutis intellectivae illuminationem intellectus vocamus;
sicut et ipsum intelligibile vocatur lumen.  

&

ゆえに、すでに示されたように\footnote{前項。}、被造知性の本性的なちか
らは、神の本質を見るのに十分でないので、神の恩恵によって、被造物が
\footnote{eiはintelligendiの主体を示す与格(行為者の与格)と取る。}知性
認識するちからが増大することが必要である。そして、この知性認識しうるち
からの増大を、私たちは知性の照明と呼ぶ。ちょうど、可知的なもの自体が
「光」と呼ばれるように。\footnote{認識のちからの増大が「光」や「照明」
 と呼ばれるのは、視覚においても、照明が強くなることによって、あたかも
 視力が上がったかのように感じることに比せられるであろう。}

\\

Et istud est lumen de quo
dicitur {\it Apoc}.~{\sc xxi}, quod {\it claritas Dei illuminabit
eam}, scilicet societatem beatorum Deum videntium. Et secundum hoc
lumen efficiuntur deiformes, idest Deo similes; secundum illud I
Ioan.\ {\sc iii}, {\it cum apparuerit, similes ei erimus, et videbimus
eum sicuti est}.

&

そしてこれが、『ヨハネの黙示録』21節「神の明るさが、それ(すなわち、至
福者たちの社会)を照らす」\footnote{「神の栄光が都を照らしており」
(21:23)}と言われる光である。『ヨハネの手紙1』3章「彼が現れたとき、私
たちは彼に似たものとなり、彼をあるがままに見るだろう」\footnote{「御子
が現れるとき、御子に似たものとなるということを知っています。なぜなら、
そのとき御子をありのままに見るからです」(3:2)(新共同訳)}によれば、こ
の光によって、神の形(deiformes)、つまり神に似たものとされる。

\\

{\sc Ad primum ergo dicendum} quod lumen creatum est necessarium ad
videndum Dei essentiam, non quod per hoc lumen Dei essentia
intelligibilis fiat, quae secundum se intelligibilis est, sed ad hoc
quod intellectus fiat potens ad intelligendum, per modum quo potentia
fit potentior ad operandum per habitum, sicut etiam et lumen corporale
necessarium est in visu exteriori, inquantum facit medium transparens
in actu, ut possit moveri a colore.

&

第一異論に対しては、それゆえ、次のように言われるべきである。被造の光が
神の本質を見るために必要とされるのは、その光によって神の本質が可知的に
なるためではない。神の本質はそれ自体において可知的だからである。そうで
はなく、能力が、習態によって、働く力を持つようになるようなかたちで、知
性が知性認識する力を持つようになるためにである。これはちょうど、物体的
な光もまた、外界を見るときに、媒介を現実に透明にし、色によって動かされ
うるようにするために、必要とされるようにである。\footnote{この時代の光
学が背景にある。光は、空間を満たしている透明体(diaphaum)を現実に透明に
し、しかる後に対象が形象を通して視覚に作用する。}

\\

{\sc Ad secundum dicendum} quod lumen istud non requiritur ad videndum
Dei essentiam quasi similitudo in qua Deus videatur, sed quasi
perfectio quaedam intellectus, confortans ipsum ad videndum Deum. Et
ideo potest dici quod non est medium {\it in quo} Deus videatur, sed
{\it sub quo} videtur. Et hoc non tollit immediatam visionem Dei.

&

第二異論に対しては、次のように言われるべきである。この光は、神がそこに
おいて見られる類似として、神の本質を見るために必要とされるのではなく、
知性を、神を見るまで強めるような、知性の一種の完成として必要とされる。
ゆえに、そこに\kenten{おいて}ではなく、それの\kenten{もとに}神が見られ
るところの媒介と言われうる。そしてこのことは、直接的な(無媒介的な)見
神を排除しない。
 

\\

{\sc Ad tertium dicendum} quod dispositio ad formam ignis non potest
esse naturalis nisi habenti formam ignis. Unde lumen gloriae non
potest esse naturale creaturae, nisi creatura esset naturae divinae,
quod est impossibile. Per hoc enim lumen fit creatura rationalis
deiformis, ut dictum est.

&

第三異論に対しては、次のように言われるべきである。火の形相を受け取るの
に適した態勢は、火の形相を持つものにとってのみ、自然本性的である。した
がって、栄光の光が被造物にとって自然本性的であるのは、被造物が、神の本
性を持つ場合に限られるが、そんなことは不可能である。なぜなら、すでに述
べられたように、この光によって、理性的被造物は神の形になるのだから。

\end{longtable}

\newpage
\rhead{a.~6}

\begin{center}
 {\Large {\bf ARTICULUS SEXTUS}}\\
{\large UTRUM VIDENTIUM ESENTIAM DEI\\UNUS ALIO PERFECTIUS VIDEAT}\\
{\large 第六項\\
神の本質を見る者たちの中で、ある者は他の者よりも、より完全に見るか}
\end{center}

\begin{longtable}{p{21em}p{21em}}
{\sc Ad sextum sic proceditur}. Videtur quod videntium essentiam Dei
unus alio perfectius non videat. Dicitur enim I Ioan.\ {\sc iii}, {\it
videbimus eum sicuti est}. Sed ipse uno modo est. Ergo uno modo
videbitur ab omnibus. Non ergo perfectius et minus perfecte.

&

第六項の問題へ議論は以下のように進められる。神の本質を見る者たちの中で、
ある者が他の者よりも完全に見ることはないと思われる。なぜなら、『ヨハネ
の手紙1』3章で、「私たちは神をあるがままに見るだろう」と言われている。
しかし、神のあり方は一つだけである。ゆえに、その一つのあり方によって、
すべての者によって見られるだろう。ゆえに、より完全に、とか、より少なく
完全に、ということはない。

\\

2.~{\sc Praeterea}, Augustinus dicit, in libro {\it Octoginta trium
Qq.}, quod unam rem non potest unus alio plus intelligere. Sed omnes
videntes Deum per essentiam, intelligunt Dei essentiam, intellectu
enim videtur Deus, non sensu, ut supra habitum est. Ergo videntium
divinam essentiam unus alio non clarius videt.

&

さらに、アウグスティヌスは『八十三問題集』という書物の中で、一つの事物
を、ある人が他の人よりもより多く知性認識することはありえない、と述べて
いる。しかるに、すべて神を本質によって見る者は、神の本質を知性認識する。
なぜなら、前に述べられたように\footnote{第3項。}、神は感覚ではなく知性
によって見られるからである。ゆえに、神の本質を見る者たちの中で、ある者
が他の者よりも、より明らかに見ることはない。

\\

3.~{\sc Praeterea}, quod aliquid altero perfectius videatur, ex duobus
contingere potest, vel ex parte obiecti visibilis; vel ex parte
potentiae visivae videntis. Ex parte autem obiecti, per hoc quod
obiectum perfectius in vidente recipitur, scilicet secundum
perfectiorem similitudinem. Quod in proposito locum non habet, Deus
enim non per aliquam similitudinem, sed per eius essentiam praesens
est intellectui essentiam eius videnti. Relinquitur ergo quod, si unus
alio perfectius eum videat, quod hoc sit secundum differentiam
potentiae intellectivae. Et ita sequitur quod cuius potentia
intellectiva naturaliter est sublimior, clarius eum videat. Quod est
inconveniens, cum hominibus promittatur in beatitudine aequalitas
Angelorum.

&

さらに、あるものが他のものよりも完全に見られるということは、二つのことに
もとづいて生じうる。一つは、可視的な対象の側からであり、もう一つは、見
る者がもつ見る能力の側からである。ところで、対象の側からというのは、対象
が、より完全に見る者に受け取られる、すなわち、より完全な類似において
受け取られることによる。このことは、提出されたものにおいて場所を持たな
い(=今の問題には関係ない)。なぜなら、神は、なんらかの類似によってでは
なく自らの本質によって、神の本質を見る者の知性に現在するからである。
ゆえに、もしある者が他の者よりも完全に神を見るとすれば、残るのは、知性
的能力の違いによる、ということである。そして、その知性的能力が自然
本性的に高い者が、神をより明らかに見ることになる。これは不都合である。
なぜなら、人間たちには、至福において、天使たちと等しくなることが約束さ
れているからである。

\\

{\sc Sed contra est} quod vita aeterna in visione Dei consistit,
secundum illud Ioan.\ {\sc xvii}: {\it haec est vita aeterna},
etc. Ergo, si omnes aequaliter Dei essentiam vident, in vita aeterna
omnes erunt aequales. Cuius contrarium dicit apostolus, I {\it Cor}.\
{\sc xv}: {\it stella differt a stella in claritate}.

&

しかし反対に、『ヨハネによる福音書』17章「これは永遠の生である、云々」
\footnote{「永遠の命とは、唯一のまことの神であられるあなたと、あなたの
お遣わしになったイエス・キリストを知ることです」(17:3)(新共同訳)}に
よれば、永遠の生は、神を見ることにおいて成立する。ゆえに、もし、すべて
の者が神の本質を等しく見るならば、永遠の生において、すべての人が等しく
なるであろう。これと矛盾することを、使徒は述べている。『コリントの信徒
への手紙1』15章「星々のあいだに明るさの違いがある」\footnote{「星と星
との間の輝きにも違いがあります」(15:41)(新共同訳)}。

\\

{\sc Respondeo dicendum} quod videntium Deum per essentiam unus alio
perfectius eum videbit. Quod quidem non erit per aliquam Dei
similitudinem perfectiorem in uno quam in alio, cum illa visio non sit
futura per aliquam similitudinem, ut ostensum est. 

&

解答する。以下のように言われるべきである。神を本質によって見る者たちの
うちで、ある者は他の者よりも、より完全に神を見るであろう。しかし、すで
に示されたように\footnote{第2項。}、神を見ることは、なんらかの類似によっ
てあるのではないから、それは、ある者のうちに、他の者のうちによりも、よ
り完全な類似があることによるのではない。

\\

Sed hoc erit per
hoc, quod intellectus unius habebit maiorem virtutem seu facultatem ad
videndum Deum, quam alterius. Facultas autem videndi Deum non competit
intellectui creato secundum suam naturam, sed per lumen gloriae, quod
intellectum in quadam deiformitate constituit, ut ex superioribus
patet. Unde intellectus plus participans de lumine gloriae, perfectius
Deum videbit.

&

むしろそれは、ある者の知性が、
他の者の知性よりも、神を見るためのより大きなちから、ないし機能\footnote{facultasの訳語が不足しているので「機能」と訳すが、「物事を容易になすことができるちから」という意味であり、virtusとほぼ同義。}をもつこ
とによる。神を見る機能は、被造の知性に、その本性において適合するのでは
なく、栄光の光によって適合する。この光は、前に述べられたこと
\footnote{前項。}から明らかなとおり、知性をある種の「神の形」において構成する。
したがって、栄光の光をより多く分有する知性が、より完全に神
を見るであろう。

\\

Plus autem participabit de lumine gloriae, qui plus
habet de caritate, quia ubi est maior caritas, ibi est maius
desiderium; et desiderium quodammodo facit desiderantem aptum et
paratum ad susceptionem desiderati. Unde qui plus habebit de caritate,
perfectius Deum videbit, et beatior erit.

&

ところで、より多くの栄光の光を分有する者とは、より多く
の愛をもつ者である。なぜなら、より大きな愛があるところに、より大きな欲
求があり、そして欲求は、ある意味で、欲求する者を、欲求する対象を受ける
のに適したもの、準備ができたものにするからである。したがって、より多く
の愛を持つであろう者が、より完全に神を見、より至福になるであろう。

\\

{\sc Ad primum ergo dicendum} quod, cum dicitur {\it videbimus eum
sicuti est}, hoc adverbium {\it sicuti} determinat modum visionis ex
parte rei visae ut sit sensus, {\it videbimus eum ita esse sicuti
est}, quia ipsum esse eius videbimus, quod est eius essentia. Non
autem determinat modum visionis ex parte videntis, ut sit sensus, quod
ita erit perfectus modus videndi, sicut est in Deo perfectus modus
essendi.

&

第一異論に対しては、それゆえ、次のように言わなければならない。「私たち
は神をあるがままに(=あるのと同じように)見るであろう」と言われるとき、
この「ままに(のと同じように)」という副詞は、見られる事物の側からの見
え方を限定するのであり、その結果その意味は「私たちは神を、それが「存在
する(〜が在る)」のと同じように「ある(〜である)」ことを見るであろう」
となる。なぜなら、私たちは、神の存在そのものを見るであろうが、それは神
の本質だからである\footnote{神の本質は神の存在以外ではない(q.3, a.4)。}。
しかし、見る者の側からの見方を限定するのではない。その場合の意味は、
「ちょうど神において、存在の完全なあり方があるように、完全な見方がある
であろう」となるのだが。\footnote{このように、「あるがままに神を見る」
という聖書の言葉の意味を、トマスは「神は、〈存在する〉である、というこ
とを見る」と大胆に読み替えている。言い換えれば、「あるがままに神を見る」
ということは、もし「あるがまま」が、通常の「どこも隠れていない状態で」
という意味であれば、トマスにおいては認められない。神を隅々まで認識する
(=把握するcomprehendere)ことができるのは神だけである。}

\\

Et per hoc etiam patet solutio ad secundum. Cum enim dicitur quod rem
unam unus alio melius non intelligit, hoc habet veritatem si referatur
ad modum rei intellectae, quia quicumque intelligit rem esse aliter
quam sit, non vere intelligit. Non autem si referatur ad modum
intelligendi, quia intelligere unius est perfectius quam intelligere
alterius.

&

そして、これによって、第二異論にたいする解答も明らかである。「ある人が、
他の人よりも、一つの事物をよりよく知性認識することがない」ということは、
知性認識される事物のあり方にかんしては真である。なぜなら、事物をそれが
在るとおりに知性認識しない者はだれでも、真に知性認識していないからであ
る。しかし、知性認識のしかたに関係する限りでは、真ではない。なぜなら、
ある人の知性認識の働きが、他の人の知性認識の働きよりも、より完全だとい
うことがあるからである。

\\

{\sc Ad tertium dicendum} quod diversitas videndi non erit ex parte
obiecti, quia idem obiectum omnibus praesentabitur, scilicet Dei
essentia, nec ex diversa participatione obiecti per differentes
similitudines, sed erit per diversam facultatem intellectus, non
quidem naturalem, sed gloriosam, ut dictum est.

&

第三異論に対しては、次のように言われるべきである。見ることの多様性は、
対象の側からではないだろう。なぜなら、同一の対象、すなわち神の本質が、
すべての者に提示されるだろうから。また、異なる類似を通して、対象がさま
ざまに分有されることによるのでもない。そうではなく、すでに述べられたよ
うに、知性の機能が多様であることによる。ただしそれは自然本性的な機能で
はなく、栄光の機能のことである。

\end{longtable}
\newpage

\rhead{a.~7}
\begin{center}
 {\Large {\bf ARTICULUS SEPTIMUS}}\\
{\large UTRUM VIDENTES DEUM PER ESSENTIAM IPSUM COMPREHENDANT\\
第七項\\
神を本質によって見る者は、神を把握するか}
\end{center}
\begin{longtable}{p{21em}p{21em}}
{\sc \large Ad septimum sic proceditur}. Videtur quod videntes Deum
 per essentiam ipsum comprehendant. Dicit enim apostolus, {\it
 Philipp}.\ {\sc iii}: {\it sequor autem si quo modo
 comprehendam}. Non autem frustra sequebatur, dicit enim ipse, I {\it
 Cor}.\ {\sc ix}: {\it sic curro, non quasi in incertum}. Ergo ipse
 comprehendit, et eadem ratione alii, quos ad hoc invitat, dicens:
 {\it sic currite, ut comprehendatis}.

&

第七項の問題へ議論は以下のように進められる。神を本質によって見る者たち
は、神を把握すると思われる。なぜなら、使徒は『フィリピの信徒への手紙』
3章で「なんとかして把握するように、私は追いかけている」\footnote{「わ
たしは、既にそれを得たというわけではなく、既に完全なものとなっているの
わけでもありません。なんとかして捕らえようと努めているのです。自分がキ
リスト・イエスに捕らえられているからです」(3:12)(新共同訳)}と述べて
いる。しかるに、彼(パウロ)は『コリントの信徒への手紙1』で「不確かな
ものに向かってではなく、私は走る」\footnote{「だから、わたしとしては、
やみくもに走ったりしないし、空を打つような拳闘もしません」(9:26)(新共
同訳)}と言うのだから、彼は無駄に従ったのではない。ゆえに、彼は把握し
たのであり、同じ理由で、「把握するために、走りなさい」\footnote{「あな
たがたも賞を得るように走りなさい」(9:24)(新共同訳)}と言って彼が招く
人々もまた、把握したのである。

\\

2.~{\sc  Praeterea}, ut dicit Augustinus in libro {\it de videndo Deum}
 ad Paulinam, {\it illud comprehenditur, quod ita totum videtur, ut nihil
 eius lateat videntem}. Sed si Deus per essentiam videtur, totus videtur,
 et nihil eius latet videntem; cum Deus sit simplex. Ergo a quocumque
 videtur per essentiam, comprehenditur.

&

さらに、アウグスティヌスが、パウリナ宛書簡『神を見ることについて』で述
べるように、「把握されるものとは、そのなかのどこも見る者から隠れていな
いように、そのすべてが見られるものである」。しかし、もし神が本質によっ
て見られるならば、全体が見られ、彼のどこも見る者に隠れないだろう。なぜ
なら、神は単純なのだから。ゆえに、だれによって本質によって見られようと
も、神は、把握される。

\\

3.~Si dicatur quod videtur totus, sed non totaliter, contra, {\it
totaliter} vel dicit modum videntis, vel modum rei visae. Sed ille qui
videt Deum per essentiam, videt eum totaliter, si significetur modus
rei visae, quia videt eum sicuti est, ut dictum est. Similiter videt
eum totaliter, si significetur modus videntis, quia tota virtute sua
intellectus Dei essentiam videbit. Quilibet ergo videns Deum per
essentiam, totaliter eum videbit. Ergo eum comprehendet.

&

もし「全体が」見られるのであり「全体的に」見られるのではないと言われる
ならば、それに反対して次のように述べる。「全体的に」ということは、見る
者のありかたを言う場合と、見られる事物のあり方を言う場合がある。しかる
に、神を本質によって見る者は、見られる事物のあり方が意味されている場合
に、神を\kenten{全体的に}見る。すでに述べられたように、神をあるがまま
に見るからである。同様に、見る者のあり方が意味されている場合にも、神を
\kenten{全体的に}見る。なぜなら、自らの、知性の全能力によって、神を見
るだろうからである。ゆえに、神を本質によって見る誰もが、神を全体的に見
るであろう。ゆえに、神を把握するだろう。

\\

{\sc  Sed contra} est quod dicitur Ierem.\ {\sc xxxii}: {\it Fortissime,
 magne, potens, dominus exercituum nomen tibi; magnus consilio, et
 incomprehensibilis cogitatu}. Ergo comprehendi non potest.

&

しかし反対に、『エレミア書』32章で「もっとも強く、偉大で力強く、あなた
の名前は万軍の主。思慮よりも大きく、思考によって把握され得ない」
\footnote{「大いなる神、ちからある神、その御名は万軍の主、その謀は偉大
であり、御業は力強い。」(32:18-9)(新共同訳)}と言われている。

\\

{\sc Respondeo dicendum} quod comprehendere Deum impossibile est
cuicumque intellectui creato: {\it attingere vero mente Deum
qualitercumque, magna est beatitudo}, ut dicit Augustinus. Ad cuius
evidentiam, sciendum est quod illud comprehenditur, quod perfecte
cognoscitur. Perfecte autem cognoscitur, quod tantum cognoscitur,
quantum est cognoscibile. 

&

解答する。以下のように言われるべきである。神を把握することは、どんな被
造の知性にも不可能である。アウグスティヌスが言うように、「どのようなか
たちであれ精神によって神に到達することが\footnote{つまり、被造物の至福
を構成するのは神を把握することではなく、どんなかたちであっても神に到達
することである。「把握」と「到達」が対比されている点に注意。}、大きな
至福である」。これを明らかにするために、次のことが知られるべきである。
把握されるとは、完全に認識されるということである。さらに、完全に認識さ
れるとは、対象がもつ認識されうるという性質の限界まで、認識されていると
いうことである。

\\

Unde si id quod est cognoscibile per scientiam demonstrativam,
opinione teneatur ex aliqua ratione probabili concepta, non
comprehenditur. Puta, si hoc quod est triangulum habere tres angulos
aequales duobus rectis, aliquis sciat per demonstrationem,
comprehendit illud, si vero aliquis eius opinionem accipiat
probabiliter, per hoc quod a sapientibus vel pluribus ita dicitur, non
comprehendet ipsum, quia non pertingit ad illum perfectum modum
cognitionis, quo cognoscibilis est.

&

したがって、もし論証的学知によって認識されうるものが、蓋然的に捉えら
れたなんらかの根拠にもとづいて臆見によって保持されるなら、それは把握
されているのではない。たとえば、もし三角形は二直角に等しい三つの角を
持つということを、ある人が論証によって知っているなら、彼はそれを把握し
ている。しかし別の人が、賢い人々によって、あるいは多くの人々によっ
てそう言われているからという理由でその意見を蓋然的に受け入れてい
るのであれば、彼はそれを把握していない。なぜならこの人は、それが認識
可能である完全な認識のあり方にまで到達していないからである。

\\

Nullus autem intellectus creatus pertingere potest ad illum perfectum
modum cognitionis divinae essentiae, quo cognoscibilis est. Quod sic
patet. Unumquodque enim sic cognoscibile est, secundum quod est ens
actu. Deus igitur, cuius esse est infinitum, ut supra ostensum est,
infinite cognoscibilis est. Nullus autem intellectus creatus potest
Deum infinite cognoscere. Intantum enim intellectus creatus divinam
essentiam perfectius vel minus perfecte cognoscit, inquantum maiori
vel minori lumine gloriae perfunditur. Cum igitur lumen gloriae
creatum, in quocumque intellectu creato receptum, non possit esse
infinitum, impossibile est quod aliquis intellectus creatus Deum
infinite cognoscat. Unde impossibile est quod Deum comprehendat.

&

ところで、どんな被造の知性も、神の本質を、それが認識可能である限界まで
認識する完全なあり方にまで到達できない。このことは次のようにして明らか
である。各々のものは、現実態における有である限りにおいて、認識されうる
ものである。ゆえに、神は、既に示されたように、その存在が無限なので、無
限に認識されうるものである。しかるに被造の知性は、神を無限に認識する
ことができない。なぜなら被造の知性は、より多くの、あるいはより少しの
栄光の光が注がれるのに応じて、神の本質を、より完全に、あるいはより不
完全に認識する。ゆえにどんな被造の知性に受け取られたものであれ、被造
の栄光の光は無限の存在ではありえないのだから、なんらかの被造の知性が
神を無限に認識することはできない。したがって神を把握することはできな
い。

\\

{\sc Ad primum ergo dicendum} quod {\it comprehensio} dicitur
dupliciter. Uno modo, stricte et proprie, secundum quod aliquid
includitur in comprehendente. Et sic nullo modo Deus comprehenditur,
nec intellectu nec aliquo alio, quia, cum sit infinitus, nullo finito
includi potest, ut aliquid finitum eum infinite capiat, sicut ipse
infinite est. Et sic de comprehensione nunc quaeritur.

&

第一異論に対しては、それゆえ、次のように言われるべきである。「把握」は
二通りに言われる。一つは、厳密に、固有の意味においてであり、あるものが、
把握するものの中に包み込まれる限りにおいてそう言われる。この意味では、
知性によっても他の何かによっても、だれも神を把握しない。なぜなら、神は
無限だから、どんな有限のものも、それを包み込むことができないからである。
ある有限なものが、神が無限であるように、神を無限に捉えるようには(でき
ない)。

\\

---Alio modo {\it comprehensio} largius sumitur, secundum quod
comprehensio {\it insecutioni} opponitur. Qui enim attingit aliquem,
quando iam tenet ipsum, comprehendere eum dicitur. Et sic Deus
comprehenditur a beatis, secundum illud {\it Cant}.\ {\sc iii}, {\it
tenui eum, nec dimittam}. Et sic intelliguntur auctoritates apostoli
de comprehensione.

&

もう一つの意味で、「把握」は、より広い意味で取られるが、それは、把握が、
「追跡」に対置される限りにおいてである。たとえば、ある人のところにたど
りつく人が、その人を保持しているあいだ、彼を「把握している」と言う。こ
の意味で、『雅歌』3章「私は彼を捕まえた。もう放さない」\footnote{「つ
かまえました、もう離しません」(3:4)(新共同訳)}によれば、神は、至福者
たちによって「把握されて」いる。使徒の、「把握」についての諸権威は、こ
の意味で理解される。

\\

---Et hoc modo {\it comprehensio} est una de tribus dotibus animae,
quae respondet spei; sicut visio fidei, et fruitio caritati.

&

そしてこの意味で、「把握」は、魂の三つの賜物の一つである。ちょうど、直
観が信仰に、享受が愛に対応するように、これは希望に対応する。

\\

Non enim, apud nos, omne quod videtur, iam tenetur vel habetur, quia
videntur interdum distantia, vel quae non sunt in potestate
nostra. Neque iterum omnibus quae habemus, fruimur, vel quia non
delectamur in eis; vel quia non sunt ultimus finis desiderii nostri,
ut desiderium nostrum impleant et quietent.

&

じっさい、私たちのもとでは、すべて見られるものが、すでに保持され、持た
れているわけではない。なぜなら、離れているときでも、あるいは、私たちの
能力のうちにないものでも、見られるからである。また、私たちは、私たちが
持っているすべてのものを、享受するわけではない。それは、私たちがそれら
を喜ばないからであったり、私たちの欲求の究極目的でないので、私たちの欲
求を満たしたり静めたりしないからである。

\\

Sed haec tria habent beati in Deo, quia et vident ipsum; et videndo,
tenent sibi praesentem, in potestate habentes semper eum videre; et
tenentes, fruuntur sicut ultimo fine desiderium implente.

&

しかし、これら三つを、至福者たちは、神において持つ。なぜなら、彼らは神
を見、見ることによって、自らに現在しているものとして神を保持し、その能
力の中に、常に神を見ることを保ち、そして、神を保持する者たちは、欲求を
満たす究極目的として、神を享受するからである。

\\

{\sc Ad secundum dicendum} quod non propter hoc Deus
incomprehensibilis dicitur, quasi aliquid eius sit quod non videatur,
sed quia non ita perfecte videtur, sicut visibilis est. Sicut cum
aliqua demonstrabilis propositio per aliquam probabilem rationem
cognoscitur, non est aliquid eius quod non cognoscatur, nec subiectum,
nec praedicatum, nec compositio, sed tota non ita perfecte
cognoscitur, sicut cognoscibilis est. Unde Augustinus, definiendo
comprehensionem, dicit quod {\it totum comprehenditur videndo, quod
ita videtur, ut nihil eius lateat videntem; aut cuius fines
circumspici possunt}, tunc enim fines alicuius circumspiciuntur,
quando ad finem in modo cognoscendi illam rem pervenitur.

&

第二異論に対しては、次のように言われるべきである。神が把握不可能と言わ
れるのは、神のうちの何かが見えないからではなく、見られうる限界にまで完
全に見られていないからである。たとえば、論証されうるなんらかの命題が、
ある蓋然的な理由によって認識されるとき、主語や述語や、その複合が認識さ
れていないというように、その命題のどこかが認識されていないのではなく、
命題全体が、認識されうる限界にまで完全に認識されていない。このことから、
アウグスティヌスは、「把握」を定義して、「そのどこも見る人から隠れてい
ないようなかたちで見られる全体、あるいは、その限界が見渡されうるように
見られる全体は、把握されている」と述べている。じっさい、あるものの限界
が見渡されるのは、その事物を認識するあり方において、限界にまで到達され
たときである。

\\

{\sc Ad tertium dicendum} quod {\it totaliter} dicit modum obiecti,
non quidem ita quod totus modus obiecti non cadat sub cognitione; sed
quia modus obiecti non est modus cognoscentis. Qui igitur videt Deum
per essentiam, videt hoc in eo, quod infinite existit, et infinite
cognoscibilis est, sed hic infinitus modus non competit ei, ut
scilicet ipse infinite cognoscat, sicut aliquis probabiliter scire
potest aliquam propositionem esse demonstrabilem, licet ipse eam
demonstrative non cognoscat.

&

第三異論に対しては、次のように言われるべきである。「全体的に」は、対
象のあり方を言うが、対象のあり方全体が、認識の中に入ってくるわけではな
い、という意味ではなく、対象のあり方が、認識者のあり方でないという意味
である。ゆえに、神を本質によって見る者は、無限に存在すること、無限に認
識されうることにおいてそれを見るが、しかし、この無限のあり方は、彼が無
限に認識する、というかたちでは、その者に適合しない。ちょうど、ある人が、
彼自身は、ある命題を論証的に認識していないが、蓋然的に、その命題が論証
可能であることを知ることができるように。

\end{longtable}

\newpage
\rhead{a.~8}

\begin{center}
 {\Large {\bf ARTICULUS OCTAVUS}}\\
{\large UTRUM VIDENTES DEUM PER ESSENTIAM OMNIA IN DEO VIDEANT\\
第八項\\
神を本質によって見る者は、万物を神において見るか}
\end{center}

\begin{longtable}{p{21em}p{21em}}

{\sc Ad octavum sic proceditur}. Videtur quod videntes Deum per
essentiam omnia in Deo videant. Dicit enim Gregorius, in IV {\it
Dialog}: {\it Quid est quod non videant, qui videntem omnia vident?}
Sed Deus est videns omnia. Ergo qui vident Deum, omnia vident.

&

第八項の問題へ議論は以下のように進められる。神を本質によって見る者たち
は、万物を神において見ると思われる。理由は以下の通り。グレゴリウスは
『対話』4章で「「万物を見る者」を見る者が、見ないものはあるだろうか」
と述べている。しかるに、神は万物を見る者である。ゆえに、神を見る者は、
万物を見る。

\\

2.~{\sc Item}, quicumque videt speculum, videt ea quae in speculo
resplendent. Sed omnia quaecumque fiunt vel fieri possunt, in Deo
resplendent sicut in quodam speculo, ipse enim omnia in seipso
cognoscit. Ergo quicumque videt Deum, videt omnia quae sunt et quae
fieri possunt.

&

さらに、鏡を見る者は誰でも、鏡に反映するものを見る。しかるに、なんであ
れ創られる、あるいは創られうるものは、ある意味で鏡において反映するよう
に、神において反映している。神は万物を自らにおいて認識するからである。
ゆえに、神を見る者は誰でも、存在するものと、創られうるすべてのものを見
る。

\\

3.~{\sc Praeterea}, qui intelligit id quod est maius, potest
intelligere minima, ut dicitur III {\it de Anima}. Sed omnia quae Deus
facit vel facere potest, sunt minus quam eius essentia. Ergo quicumque
intelligit Deum, potest intelligere omnia quae Deus facit vel facere
potest.

&

さらに、『デ・アニマ』第三巻で言われるように、より大きなものを知性認識
する者は、最小のものを知性認識しうる。しかるに、神が造り、あるいは造り
うるすべてのものは、神の本質よりも小さい。ゆえに、神を知性認識するもの
は、神が創り、あるいは創りうるすべてのものを知性認識できる。

\\

4.~{\sc Praeterea}, rationalis creatura omnia naturaliter scire
desiderat. Si igitur videndo Deum non omnia sciat, non quietatur eius
naturale desiderium, et ita, videndo Deum non erit beata. Quod est
inconveniens. Videndo igitur Deum, omnia scit.

&

さらに、理性的被造物は、万物を知ることを、本性的に欲する。ゆえに、神を
見ることによって万物を知ることがなかったら、彼の自然本性的欲求は静まら
ず、したがって、神を見ることによって至福にならないであろう。これは不都
合である。ゆえに、神を見ることによって万物を知る。

\\

{\sc Sed contra est} quod Angeli vident Deum per essentiam, et tamen
non omnia sciunt. Inferiores enim Angeli purgantur a superioribus a
nescientia, ut dicit Dionysius, {\sc vii} cap.\ {\it Cael.\
Hier}. Ipsi etiam nesciunt futura contingentia et cogitationes
cordium, hoc enim solius Dei est. Non ergo quicumque vident Dei
essentiam, vident omnia.

&

しかし反対に、天使は、神を本質によって見る。しかし、すべてを知っている
わけではない。ディオニュシウスが『天上階級論』7章で述べるように、下位
の天使は上位の天使によって無知から浄化されるからである。さらに、天使た
ちは、未来の非必然的なことや、心の思惟を知らない。なぜなら、これを知る
のはただ神だけだからである。ゆえに、神を本質によって見る誰もが、万物を
見るわけではない。

\\

{\sc Respondeo dicendum} quod intellectus creatus, videndo divinam
essentiam, non videt in ipsa omnia quae facit Deus vel facere
potest. Manifestum est enim quod sic aliqua videntur in Deo, secundum
quod sunt in ipso. Omnia autem alia sunt in Deo, sicut effectus sunt
virtute in sua causa. Sic igitur videntur alia in Deo, sicut effectus
in sua causa. Sed manifestum est quod quanto aliqua causa perfectius
videtur, tanto plures eius effectus in ipsa videri possunt. Qui enim
habet intellectum elevatum, statim uno principio demonstrativo
proposito, ex ipso multarum conclusionum cognitionem accipit, quod non
convenit ei qui debilioris intellectus est, sed oportet quod ei
singula explanentur. (continues)

&

解答する。以下のように言われるべきである。被造知性は、神の本質を見るこ
とによって、神が創る、あるいは創ることができるすべてのものを、そこにお
いて見るわけではない。理由は以下の通り。何かが神において見られるのは、
その何かが神の中に在る限りにおいてであるのは明らかである。しかるに、神
以外のすべてのものが神の中に在るのは、結果が、その原因の中に、ちからに
おいて\footnote{virtute, 山田訳は「潜勢的に」。}在る限りでである。ゆえ
に、神以外のものが神において見られるのは、結果が原因において見られるよ
うにしてである。しかし、ある原因が、より完全に見られれば見られるほど、
より多くのそれの結果が、その原因において見られうることも明らかである。
じっさい、高められた知性を持つ者は、一つの論証原理が示されると、すぐに、
それに基づいて、多くの結論の認識を受け取るが、このようなことは、弱い知
性を持つ者には起こらない。むしろそのような人には、一つずつ、説明される
必要がある。(続く)

\\

Ille igitur intellectus potest in causa cognoscere omnes causae
effectus, et omnes rationes effectuum, qui causam totaliter
comprehendit. Nullus autem intellectus creatus totaliter Deum
comprehendere potest, ut ostensum est. Nullus igitur intellectus
creatus, videndo Deum, potest cognoscere omnia quae Deus facit vel
potest facere, hoc enim esset comprehendere eius virtutem. Sed horum
quae Deus facit vel facere potest, tanto aliquis intellectus plura
cognoscit, quanto perfectius Deum videt.

&

ゆえに、原因を完全に把握する知性は、その原因において、すべての原因の結
果と、すべての結果の理念を認識することができる。しかるに、既に示された
ように、どんな被造の知性も、神を完全に把握することができない。ゆえに、
どんな被造の知性も、神を見ることによって、神が創るものや創ることができ
るすべてのものを認識できるわけではない。なぜなら、(もしそれができた
としたら)それは神のちからを把握することになるからである。しかし、ある
知性は、より完全に神を見れば見るほど、神が創るものや創ることができるも
ののうち、より多くのものを認識する。

\\

{\sc Ad primum ergo dicendum} quod Gregorius loquitur quantum ad
sufficientiam obiecti, scilicet Dei, quod, quantum in se est,
sufficienter continet omnia et demonstrat. Non tamen sequitur quod
unusquisque videns Deum omnia cognoscat, quia non perfecte
comprehendit ipsum.

&

第一異論に対しては、それゆえ、次のように言われるべきである。グレゴリウ
スは、対象、すなわち神の充満について語っているのであり、つまりその対象
は、それ自体における限り、十分に、万物を含み、明示している。しかし、こ
のことから、誰であれ神を見る者が、万物を認識することは帰結しない。なぜ
なら、神を完全に把握するわけではないからである。

\\

{\sc  Ad secundum dicendum} quod videns speculum, non est
 necessarium quod omnia in speculo videat, nisi speculum visu suo
 comprehendat.

&

第二異論に対しては、次のように言われるべきである。鏡を見る者は、その見
ることにおいて、鏡を把握するのでなければ、必ずしもすべてを鏡において見
るわけではない。

\\

{\sc  Ad tertium dicendum} quod, licet maius sit videre Deum quam
 omnia alia, tamen maius est videre sic Deum quod omnia in eo
 cognoscantur, quam videre sic ipsum quod non omnia, sed pauciora vel
 plura cognoscantur in eo. Iam enim ostensum est quod multitudo
 cognitorum in Deo, consequitur modum videndi ipsum vel magis perfectum
 vel minus perfectum.

&

第三異論に対しては、次のように言われるべきである。たしかに、神を見るこ
とは、他のすべてを見ることよりも大きいが、しかし、神において万物が認識
されるように神を見ることは、万物ではなく、より少ないものが、あるいは、
より多くのものが、神において認識されるように神を見ることよりも大きい。
じっさい、神において認識される事柄の多さは、神を見るしかたがより多く完
全であるか、より少なく完全であるかに伴う。

\\

{\sc Ad quartum dicendum} quod naturale desiderium rationalis
creaturae est ad sciendum omnia illa quae pertinent ad perfectionem
intellectus; et haec sunt species et genera rerum, et rationes earum,
quae in Deo videbit quilibet videns essentiam divinam. Cognoscere
autem alia singularia, et cogitata et facta eorum, non est de
perfectione intellectus creati, nec ad hoc eius naturale desiderium
tendit, nec iterum cognoscere illa quae nondum sunt, sed fieri a Deo
possunt. Si tamen solus Deus videretur, qui est fons et principium
totius esse et veritatis, ita repleret naturale desiderium sciendi,
quod nihil aliud quaereretur, et beatus esset. Unde dicit Augustinus,
V {\it Confess.}: {\it infelix homo qui scit omnia illa} (scilicet
creaturas), {\it te autem nescit, beatus autem qui te scit, etiam si
illa nesciat. Qui vero te et illa novit, non propter illa beatior est,
sed propter te solum beatus}.

&

第四異論に対しては、次のように言われるべきである。理性的被造物の自然本
性的欲求は、知性の完全性に属するすべてのものを知ることへと向かう。その
完全性とは、諸事物の種と類、そして、それらの理念である。これらを、神の
本質を見る者は誰でも、神において見るであろう。しかし、他の個物や、それ
らの個物によって思惟されたことや為されたことは、被造知性の完全性に含ま
れず、こういうことに、被造知性の自然本性的欲求は向かわない。さらにまた、
まだ存在しないが、神によって作られうるものについても同様である。しかし、
すべての存在と真理の源泉であり根源である神が見られるならば、それだけで、
他の何ものも求められず、至福となるようなかたちで、知ることの自然本性的
欲求を満たす。このことから、アウグスティヌスは『告白』第五巻で、「かの
すべてのもの(被造物)を知っていても、あなたを知らない人は不幸であり、
それら(すべての被造物)を知らないとしても、あなたを知る人は、幸福であ
る。しかし、あなたと被造物を知る人は、被造物のためにより幸福になるので
はなくて、あなたを知ることによってのみ幸福なのである」と言っている。

\end{longtable}
\newpage

\rhead{a.~9}
\begin{center}
 {\Large {\bf ARTICULUS NONUS}}\\{\large UTRUM EA QUAE VIDENTUR IN
DEO A VIDENTIBUS DIVINAM ESSENTIAM\\PER ALIQUAS SIMILITUDINES
VIDEANTUR\\第九項\\神の本質を見る者によって神において見られるものは、
なんらかの類似を通して見られるか}

\end{center} 

\begin{longtable}{p{21em}p{21em}}

{\sc Ad nonum sic proceditur}. Videtur quod ea quae videntur in Deo, a
videntibus divinam essentiam per aliquas similitudines
videantur. Omnis enim cognitio est per assimilationem cognoscentis ad
cognitum, sic enim intellectus in actu fit intellectum in actu, et
sensus in actu sensibile in actu, inquantum eius similitudine
informatur, ut pupilla similitudine coloris. Si igitur intellectus
videntis Deum per essentiam intelligat in Deo aliquas creaturas,
oportet quod earum similitudinibus informetur.

&

第九項の問題へ議論は以下のように進められる。神において見られることは、
神の本質を見る者たちによって、なんらかの類似を通して見られると思われる。
なぜなら、すべて認識は、認識主体が認識対象へ類似化することによって成り
立つ。たとえば、目が色の類似によって形相づけられるように、対象の類似に
よって形相づけられる限りにおいて、現実態における知性は、現実態における
「知性認識されたもの」となり、現実態における感覚は、現実態における可感
的なものとなる。ゆえに、もし神を本質によって見る者の知性が、神において
なんらかの被造物を知性認識するならば、それらの類似によって形相づけられ
なければならない。

\\

2.~{\sc Praeterea}, ea quae prius vidimus, memoriter tenemus. Sed
 Paulus, videns in raptu essentiam Dei, ut dicit Augustinus XII {\it
 super Gen.\ ad litteram}, postquam desiit essentiam Dei videre,
 recordatus est multorum quae in illo raptu viderat, unde ipse dicit
 quod {\it audivit arcana verba, quae non licet homini loqui}, II {\it
 Cor}.\ {\sc xii}. Ergo oportet dicere quod aliquae similitudines
 eorum quae recordatus est, in eius intellectu remanserint. Et eadem
 ratione, quando praesentialiter videbat Dei essentiam, eorum quae in
 ipsa videbat, aliquas similitudines vel species habebat.

&

さらに、私たちは、以前に見たものを、記憶の中に保つ。しかるに、アウグス
ティヌスが『創世記逐語注解』12巻で言うように、パウロは、脱魂状態におい
て神の本質を見、そして神の本質を見ることをやめたのち、その脱魂状態にお
いて見たことの多くを記憶した。それゆえ、彼は『コリントの信徒への手紙2』
第2章で、「私は人間に語ることが許されない隠された言葉を聞いた」
\footnote{「彼は楽園にまで引き上げられ、人が口にするのを許されない、言
い表しえない言葉を耳にしたのです」(12:4)}と述べる。ゆえに、記憶したも
のどものなんらかの類似が、彼の知性の中に残っていたのでなければならない。
同じ理由で、神の本質を現に見ていた時にも、そこにおいて見ていたものども
のなんらかの類似、あるいは形象を、彼は持っていた。
\\

{\sc  Sed contra est} quod per unam speciem videtur speculum, et
 ea quae in speculo apparent. Sed omnia sic videntur in Deo sicut in
 quodam speculo intelligibili. Ergo, si ipse Deus non videtur per
 aliquam similitudinem, sed per suam essentiam; nec ea quae in ipso
 videntur, per aliquas similitudines sive species videntur.

&

しかし反対に、鏡と、その鏡に映るものどもとは、一つの形象によって見られ
る。しかるに、神においてそのように見られるすべてのものは、なんらかの可
知的な鏡において見られる。ゆえに、もし神自身が、どんな類似によっても見
られず、神の本質によって見られるのであれば、神において見られるものもま
た、どんな類似や形象によっても見られない。

\\

{\sc Respondeo dicendum} quod videntes Deum per essentiam, ea quae in
ipsa essentia Dei vident, non vident per aliquas species, sed per
ipsam essentiam divinam intellectui eorum unitam.

&

解答する。以下のように言われるべきである。神を本質によって見る者たちは、
神の本質において見るものどもを、なんらかの形象によって見るのではなく、
彼らの知性に合一された、神の本質自身によって見る。

\\

Sic enim cognoscitur unumquodque, secundum quod similitudo eius est in
cognoscente. Sed hoc contingit dupliciter. Cum enim quaecumque uni et
eidem sunt similia, sibi invicem sint similia, virtus cognoscitiva
dupliciter assimilari potest alicui cognoscibili. 

&

理由は以下の通り。各々のものが認識されるのは、それの類似が認識者の中に
在る限りにおいてである。しかるに、このことは二通りのかたちで生じる。す
なわち、なんであれ、一つで同一のものに似ているものどもは、相互にも似て
いるのだから\footnote{a=bかつa=cならばb=c。}、認識しうるちから(認識能
力)は、なんらかの認識されうるものに対して、二通りに類似化されうる。

\\

Uno modo, secundum se, quando directe eius similitudine informatur, et
tunc cognoscitur illud secundum se.

&

ひとつには、それ自体においてであり、直接に、それ(=認識されうる対象)
の類似によって形相づけられるときである。その場合には、対象は、それ自体
において認識される。

\\

Alio
modo, secundum quod informatur specie alicuius quod est ei simile, et
tunc non dicitur res cognosci in seipsa, sed in suo simili. Alia enim
est cognitio qua cognoscitur aliquis homo in seipso, et alia qua
cognoscitur in sua imagine.

&

もう一つには、それに似ているなんらかの形象によって
形相づけられることにおいてであり、この場合は、事物がそれ自体において認
識されるとは言われず、自らに似たものにおいて認識されると言われる。たと
えば、ある人が、その人自身において認識されるその認識と、その人の像にお
いて認識される認識とは異なる。

\\

Sic ergo, cognoscere res per earum similitudines in cognoscente
existentes, est cognoscere eas in seipsis, seu in propriis naturis,
sed cognoscere eas prout earum similitudines praeexistunt in Deo, est
videre eas in Deo. Et hae duae cognitiones differunt. 

&

ゆえに、事物を、認識者のうちに存在するそれらの事物の類似によって認識す
ることは、それらを、それ自身において、あるいは、固有の本性において認識
することであるが、他方、それらの事物を、それらの類似が神の中に先在していると
認識することは、それらを神において見ることである。そして、これら二
つの認識は異なる。

\\

Unde secundum illam cognitionem qua res cognoscuntur a videntibus Deum
per essentiam in ipso Deo, non videntur per aliquas similitudines
alias; sed per solam essentiam divinam intellectui praesentem, per
quam et Deus videtur.

&

このことから、神を本質によって見る者たちによって、神自身において事物が
認識されるその認識においては、諸事物は、神以外の何かの類似によって見ら
れるのではなく、知性に現前している神の本質だけによって見られる。そし
てこの認識によって、神もまた見られる。

\\

{\sc  Ad primum ergo dicendum} quod intellectus videntis Deum
 assimilatur rebus quae videntur in Deo, inquantum unitur essentiae
 divinae, in qua rerum omnium similitudines praeexistunt.

&

第一異論に対しては、それゆえ、次のように言われるべきである。神を見る者
の知性は、そこにおいて、すべての事物の類似が先在している神の本質に合一
されるかぎりにおいて、神において見られる諸事物に類似化される。

\\

{\sc Ad secundum dicendum} quod aliquae potentiae cognoscitivae sunt,
quae ex speciebus primo conceptis alias formare possunt. Sicut
imaginatio ex praeconceptis speciebus montis et auri, format speciem
montis aurei, et intellectus ex praeconceptis speciebus generis et
differentiae, format rationem speciei. Et similiter ex similitudine
imaginis formare possumus in nobis similitudinem eius cuius est
imago. Et sic Paulus, vel quicumque alius videns Deum, ex ipsa visione
essentiae divinae potest formare in se similitudines rerum quae in
essentia divina videntur, quae remanserunt in Paulo etiam postquam
desiit Dei essentiam videre. Ista tamen visio, qua videntur res per
huiusmodi species sic conceptas, est alia a visione qua videntur res
in Deo.

&

第二異論に対しては、次のように言われるべきである。ある認識能力は、はじ
めに捉えられた形象から、他の形象を形成することができる。たとえば、想像
力は、あらかじめ捉えられた金と山の形象から、黄金の山の形象を形成するし、
知性も、あらかじめ捉えられた類と種差の形象から、種の概念を形成する。同
様に、像の類似にもとづいて、私たちは、それの像であるものの類似に向けて、
私たちの中に、(表象像を)形成することができる。この意味で、パウロは、
あるいは、だれであれ神を見る人は、神の本質を見ること自体から、自分の中
に、神の本質において見られた諸事物の類似を形成することができ、この類似
が、神の本質を見ることをやめたあとにもパウロの中に残った。しかし、この
ようにして抱かれた形象によって諸事物が見られることと、諸事物が神におい
て見られることとは異なる。

\end{longtable}

\newpage

\rhead{a.~10}
\begin{center}
{\Large {\bf ARTICULUS DECIMUS}}\\ {\large UTRUM VIDENTES DEUM PER
ESSENTIAM SIMUL VIDEAT OMNIA\\QUAE IN IPSO VIDENT\\第十項\\神を本質に
よって見る者たちは、そこにおいて見るすべてのものを、同時に見るか}
\end{center}

\begin{longtable}{p{21em}p{21em}}

{\sc  Ad decimum sic proceditur}. Videtur quod videntes Deum per
 essentiam non simul videant omnia quae in ipso vident. Quia, secundum
 philosophum, contingit multa scire, intelligere vero unum. Sed ea quae
 videntur in Deo, intelliguntur, intellectu enim videtur Deus. Ergo non
 contingit a videntibus Deum simul multa videri in Deo.

&

第十項の問題へ議論は以下のように進められる。神を本質によって見る者たち
は、神において見るすべてのものを、同時に見るのではないと思われる。理由
は以下の通り。哲学者によれば、多くのものを\kenten{知る}ことはありうる
が、\kenten{知性認識する}のは一つだけである\footnote{『トピカ』第2巻第
10章。114$^{b}$34--35}。しかるに、神において見られるものは、知性認識され
る。なぜなら、神は知性によって見られるからである。ゆえに、神を見る者た
ちによって、多くのことが、同時に神において見られるということは起こらな
い。

\\

 

2.~{\sc Praeterea}, Augustinus dicit, VIII {\it super Gen.~ad
 litteram}, quod {\it Deus movet creaturam spiritualem per tempus},
 hoc est per intelligentiam et affectionem. Sed creatura spiritualis
 est Angelus, qui Deum videt. Ergo videntes Deum, successive
 intelligunt et afficiuntur, tempus enim successionem importat.

&

さらに、アウグスティヌスは、『創世記逐語注解』8巻で、「神は、霊的被造
物を時間によって動かす」と述べている。そしてこれは、知性活動と精神の変
化\footnote{山田訳に従って、affectioを「精神の変化」と訳す。}による。
しかし、霊的被造物とは、神を見る天使である。ゆえに、神を見る者たちは、
継起的に、知性認識し、精神の変化を受ける。なぜなら、時間は継起を含意す
るからである。

\\

{\sc Sed contra est} quod Augustinus dicit, ultimo {\it de Trin}.:
 {\it non erunt volubiles nostrae cogitationes, ab aliis in alia
 euntes atque redeuntes; sed omnem scientiam nostram uno simul
 conspectu videbimus}.

&

しかし反対に、アウグスティヌスは『三位一体論』最終巻\footnote{第15巻16
章}で、次のように述べている。「ある思惟から別の思惟へ行きつ戻りつする
ような転変する思惟はないだろう。私たちは、すべての私たちの知を、一つの
同時の一瞥によって見るだろう」。

\\

{\sc Respondeo dicendum} quod ea quae videntur in Verbo, non
successive, sed simul videntur. 

&

解答する。以下のように言われるべきである。言葉\footnote{この大文字の
Verbumは神のこと。キリスト教では「御言葉(みことば)」と訳されることが
多い。『ヨハネによる福音書』冒頭などを踏まえ、三位一体論における「息子」
のペルソナの別名とされ、これが世界に現れたのがイエス・キリストであると
信じられている。この意味での「言葉」については、『神学大全』第1部第34
問題を参照。}において見られるものは、継起的にではなく、同時に見られる。

\\

Ad cuius evidentiam considerandum est,
quod ideo nos simul non possumus multa intelligere, quia multa per
diversas species intelligimus; diversis autem speciebus non potest
intellectus unus simul actu informari ad intelligendum per eas, sicut
nec unum corpus potest simul diversis figuris figurari. 

&

このことを明らかにするためには以下のことが考察されるべきである。私たち
が、同時に多くのものを知性認識できないのは、多くのものを相異なる形象を
通して知性認識するからである。しかるに、一つの知性は、同時に、さまざま
な形象によって、それらを通して知性認識するように現実に形相づけられるこ
とはできない。それは、一つの物体が、同時にさまざまな形によって形作られ
ることができないのと同様である。

\\

Unde contingit
quod, quando aliqua multa una specie intelligi possunt, simul
intelliguntur, sicut diversae partes alicuius totius, si singulae
propriis speciebus intelligantur, successive intelliguntur, et non
simul; si autem omnes intelligantur una specie totius, simul
intelliguntur. 

&

したがって、ある多くのものが、一つの形
象によって知性認識されうるときには、それらが同時に認識されるということが起こる。それはちょうど、
ある全体のさまざまな部分が、ひとつひとつが固有の形象によって知性認識さ
れるときには、同時ではなく継起的に知性認識されるが、全体の一つの形象に
よって、すべての部分が知性認識される場合には、一度に知性認識されるよう
にである。

\\

Ostensum est autem quod ea quae videntur in Deo, non
videntur singula per suas similitudines, sed omnia per unam essentiam
Dei. Unde simul, et non successive videntur.

&

ところで、神において見られるものは、ひとつひとつがそれ自身の類似
によって見られるのではなく、神の一つの本質によってすべてが見られるとい
うことがすでに示された。ゆえに、それらは、継起的にではなく、同時に見ら
れる。

\\

{\sc Ad primum ergo dicendum} quod sic unum tantum intelligimus,
inquantum una specie intelligimus. Sed multa una specie intellecta
simul intelliguntur, sicut in specie hominis intelligimus animal et
rationale, et in specie domus parietem et tectum.

&

第一異論に対しては、それゆえ、次のように言われるべきである。私たちが一
つのものだけを知性認識するのは、一つの形象によって知性認識する限りにお
いてである。しかし、一つの形象によって知性認識された多くのものは、同時
に知性認識される。たとえば、「人間」の形象において、私たちは「動物」と
「理性的なもの」を認識するし、また「家」の形象において「壁」と「屋根」
を認識するように。

\\

{\sc Ad secundum dicendum} quod Angeli, quantum ad cognitionem
naturalem qua cognoscunt res per species diversas eis inditas, non
simul omnia cognoscunt, et sic moventur, secundum intelligentiam, per
tempus. Sed secundum quod vident res in Deo, simul eas vident.

&

第二異論に対しては次のように言わなければならない。天使は、自らに与えら
れたさまざまな形象によって諸事物を認識するその本性的な認識にかんしては、
同時にすべてのものを認識しない。その限りで、天使は時間を通して、知性認
識活動において動かされる。しかし、諸事物を神において見る限りにおいては、
それらを同時に見る。

\end{longtable}

\newpage
\rhead{a.~11}

\begin{center}
 {\Large {\bf ARTICULUS UNDECIMUS}}\\
 {\large UTRUM ALIQUIS IN HAC VITA POSSIT VIDERE DEUM PER ESSENTIAM\\
第十一項\\
だれかが、この生において、神を本質によって見ることができるか}
\end{center}

\begin{longtable}{p{21em}p{21em}}

{\sc Ad undecimum sic proceditur}. Videtur quod aliquis in hac vita
possit Deum per essentiam videre. Dicit enim Iacob, {\it Gen}.~{\sc
xxxii}: {\it vidi Deum facie ad faciem}. Sed videre facie ad faciem,
est videre per essentiam, ut patet per illud quod dicitur I {\it
Cor}.~{\sc xiii}: {\it videmus nunc per speculum et in aenigmate, tunc
autem facie ad faciem}. Ergo Deus in hac vita per essentiam videri
potest.

&

第十一項の問題へ議論は以下のように進められる。だれかが、この生において、
神を本質によって見ることができると思われる。なぜなら、『創世記』32章で、
ヤコブが「私は顔と顔を合わせて神を見た」\footnote{「私は顔と顔を合わせ
て神を見たのに、なお生きている」(32:31)}と言っている。しかし、顔と顔を
合わせてみることは、本質によって見ることである。それは、かの『コリント
の信徒への手紙1』13章で「今は鏡を通して謎の中に見るが、そのときは顔と
顔を合わせて見るだろう」によって明らかである。ゆえに、神は、この生にお
いて、本質的に見られうる。

\\

2.~{\sc Praeterea}, {\it Num}.~{\sc xii} dicit dominus de Moyse: {\it
 ore ad os loquor ei, et palam, et non per aenigmata et figuras, videt
 Deum}. Sed hoc est videre Deum per essentiam. Ergo aliquis in statu
 huius vitae potest Deum per essentiam videre.

&

さらに、『民数記』12章で、神がモーセについて次のように語っている。「私
は口から口へ彼に語る。そして、彼は明らかに、謎やかたちによってではなく、
神を見る」\footnote{「口から口へ、私は彼と語り合う。あらわに、謎によら
ずに。主の姿を彼は仰ぎ見る」(12:8)}。しかし、これは神を本質によって見
ることである。ゆえに、この生の状態において、だれかが神を本質によって見
ることができる。

\\

3.~{\sc Praeterea}, illud in quo alia omnia cognoscimus, et per quod
de aliis iudicamus, est nobis secundum se notum. Sed omnia etiam nunc
in Deo cognoscimus. Dicit enim Augustinus, XII {\it Conf}.: {\it Si
ambo videmus verum esse quod dicis, et ambo videmus verum esse quod
dico, ubi quaeso illud videmus? Nec ego in te, nec tu in me, sed ambo
in ipsa quae supra mentes nostras est, incommutabili
veritate}. (continues)

&

さらに、私たちが、そこにおいて他のすべてのものを認識し、それによって他
のものについて判断するところのものは、私たちに、それ自体において知られ
ている。しかるに、私たちは、今でも、すべてを神において認識する。なぜな
ら、アウグスティヌスが、『告白』12巻で次のように述べるからである。「も
し、私たちのどちらも、あなたがいうことが真であることを見、そして私たち
のどちらも、私が言うことが真であることを見るならば、私たちはそれをどこ
で見ているのか、と私は問います。私があなたにおいてでも、あなたが私にお
いてでもなく、両方が、私たちの精神を越えている、不変の真理において見る
のです」。(続く)

\\

Idem etiam, in libro {\it de Vera Religione}, dicit quod secundum
veritatem divinam de omnibus iudicamus. Et XII {\it de Trin}.~dicit
quod {\it rationis est iudicare de istis corporalibus secundum
rationes incorporales et sempiternas, quae nisi supra mentem essent,
incommutabiles profecto non essent}. Ergo et in hac vita ipsum Deum
videmus.

&

また、同じ彼が、『真の宗教について』 という書物の中で、神の真理にした
がって、すべてについて私たちは判断する、と言っている。また、『三位一体
論』12巻で「これらの物体的なものについて、非物体的で常なる理念にしたがっ
て判断することが、理性に属している。この理念は、もし精神を越えているの
でなかったら、けっして不変ではなかっただろう」と述べる。ゆえに、この生
において、神自身を私たちは見る。

\\

4.~{\sc Praeterea}, secundum Augustinum, XII {\it super Gen.~ad
Litt}., visione intellectuali videntur ea quae sunt in anima per suam
essentiam. Sed visio intellectualis est de rebus intelligibilibus, non
per aliquas similitudines, sed per suas essentias, ut ipse ibidem
dicit. Ergo, cum Deus sit per essentiam suam in anima nostra, per
essentiam suam videtur a nobis.

&

さらに、『創世記逐語注解』12巻のアウグスティヌスによれば、魂の中に在る
ものは、知的な直視によって、その本質によって見られる。しかし、彼が同じ
箇所で述べているように、知的な直視は、可知的事物について、なんらかの類
似によってではなく、その本質によってある。ゆえに、神は自らの本質によっ
てわたしたちの魂の中に在るのだから、その本質によって、私たちに見られる。

\\

{\sc Sed contra est} quod dicitur {\it Exod}.~{\sc xxxiii}: {\it non
videbit me homo et vivet}. Glossa: {\it quandiu hic mortaliter
vivitur, videri per quasdam imagines Deus potest; sed per ipsam
naturae suae speciem non potest}.

&

しかし反対に、『出エジプト記』33章「人間は、私を見、かつ生きることはな
い」\footnote{「人はわたしを見て、なお生きていることはできないからであ
る」(33:20)}と言われていて、この注解は「ここで死すべきものとして生きて
いるあいだは、神はなんらかの像によって見られうる。しかし、自らの本性の
形象そのものによって見られることはできない」とある。

\\

{\sc Respondeo dicendum} quod ab homine puro Deus videri per essentiam
non potest, nisi ab hac vita mortali separetur. 

&

解答する。以下のように言われるべきである。神が純然たる人間によって見ら
れることは、人間がこの死すべき生から切り離されない限り、不可能である。

\\

Cuius ratio est quia, sicut supra dictum est, modus cognitionis
sequitur modum naturae rei cognoscentis.  Anima autem nostra, quandiu
in hac vita vivimus, habet esse in materia corporali, unde naturaliter
non cognoscit aliqua nisi quae habent formam in materia, vel quae per
huiusmodi cognosci possunt.

&

その理由は以下の通りである。先に述べられたように、認識のあり方は、認識
する事物の本性のあり方に従う。しかるに、私たちの魂は、この世に生きてい
るあいだ、物体的質料において存在を持つので、本性的に、質料において形相
を持つものや、そういうものを通して認識されうるもの以外にはなにも認識し
ない。

\\

Manifestum est autem quod per naturas rerum materialium divina
essentia cognosci non potest. Ostensum est enim supra quod cognitio
Dei per quamcumque similitudinem creatam, non est visio essentiae
ipsius.

&

また、質料的諸事物の本性によって神の本質が認識されえないことは明らかで
ある。なぜなら、どんな被造の類似によって神が認識されても、それが神の本
質を見ることでないことは、上で示されたからである。

\\

Unde impossibile est animae hominis secundum hanc vitam viventis,
essentiam Dei videre. Et huius signum est, quod anima nostra, quanto
magis a corporalibus abstrahitur, tanto intelligibilium abstractorum
fit capacior.

&

したがって、この生において生きる人間の魂にとって、神の本質を見ることは
不可能である。そのしるしに、私たちの魂は、より多く物体的なものから離さ
れるほど、それだけいっそう、抽象された可知的なものを受け入れやすくなる。

\\

Unde in somniis et alienationibus a sensibus corporis, magis divinae
revelationes percipiuntur, et praevisiones futurorum. Quod ergo anima
elevetur usque ad supremum intelligibilium, quod est essentia divina,
esse non potest quandiu hac mortali vita utitur.

&

したがって、夢の中や、身体の感覚を遮断された状態において、神の啓示や将
来の予知は、より感受される。ゆえに、魂が、可知的なものの中で最高のもの、
すなわち神の本質へ引き上げられることは、この死すべき生を享受しているう
ちは、不可能である。

\\

{\sc Ad primum ergo dicendum} quod, secundum Dionysium, {\sc iv}
cap.~{\it Cael.~Hier}., sic in Scripturis dicitur aliquis Deum
vidisse, inquantum formatae sunt aliquae figurae, vel sensibiles vel
imaginariae, secundum aliquam similitudinem aliquod divinum
repraesentantes. Quod ergo dicit Iacob, {\it vidi Deum facie ad
faciem}, referendum est, non ad ipsam divinam essentiam, sed ad
figuram in qua repraesentabatur Deus. Et hoc ipsum ad quandam
Prophetiae eminentiam pertinet, ut videatur persona Dei loquentis,
licet imaginaria visione, ut infra patebit, cum de gradibus prophetiae
loquemur. ---Vel hoc dicit Iacob ad designandam quandam eminentiam
intelligibilis contemplationis, supra communem statum.

&

第一異論に対しては、それゆえ、次のように言われるべきである。ディオニュ
シウス『天上階級論』4章によれば、聖書の中で、だれかが神を見たと言われ
るのは、なんらかの類似によって神の何かを表現する、感覚的、ないし表象的
ななんらかの形象が作られる限りにおいてである。それゆえ、ヤコブが言う
「わたしは顔と顔を合わせて神を見た」は、神の本質そのものではなく、そこ
において神が表現された形象に関係づけられるべきである。そしてこのこと自
体は、あとで\footnote{{\it ST} II-IIae, q.~174, a.~3 ``Utrum Gradus
Prophetiae Possint Distingui Secundum Visionem Imaginariam''}、預言者
の階級について語るときに明らかになるとおり、表象的な直視であっても、語
る神のペルソナが見られるということは、預言者の一種の卓越性に属する。

 あるいは、ヤコブがこれを言うのは、通常の状態を越えた、可知的観照の一
 種の卓越性を示すためである。

\\

{\sc Ad secundum dicendum} quod, sicut Deus miraculose aliquid
supernaturaliter in rebus corporeis operatur, ita etiam et
supernaturaliter, et praeter communem ordinem, mentes aliquorum in hac
carne viventium, sed non sensibus carnis utentium, usque ad visionem
suae essentiae elevavit; ut dicit Augustinus, XII {\it super Genes. ad
Litt.}, et in libro {\it de Videndo Deum} de Moyse, qui fuit magister
Iudaeorum, et Paulo, qui fuit magister gentium. Et de hoc plenius
tractabitur, cum de raptu agemus.

&

第二異論に対しては、次のように言われるべきである。神が奇跡によって、物
体的事物の中で超自然的に何か働くように、超自然的に、通常の秩序を脱して、
この肉において生きているが、肉の感覚を用いないある人々の精神を、自らの
本質を見ることにまで引き上げるということがあった。それはちょうど、アウ
グスティヌスが『創世記逐語注解』12巻や、『神を見ることについて』で、ユ
ダヤの人々の師であったモーゼと、異教徒の師であったパウロについて語って
いるように。そして、これについては、私たちが脱魂について論じる際に
\footnote{{\it ST} II-IIae, q.~175, a.~3 ``Utrum Paulus In Raptu
Viderit Dei Essentiam''}、より十分に論じられるであろう。

\\

{\sc Ad tertium dicendum} quod omnia dicimur in Deo videre, et
secundum ipsum de omnibus iudicare, inquantum per participationem sui
luminis omnia cognoscimus et diiudicamus, nam et ipsum lumen naturale
rationis participatio quaedam est divini luminis; sicut etiam omnia
sensibilia dicimus videre et iudicare in sole, idest per lumen
solis. Unde dicit Augustinus, I {\it Soliloquiorum: Disciplinarum
spectamina videri non possunt, nisi aliquo velut suo sole
illustrentur}, videlicet Deo. Sicut ergo ad videndum aliquid
sensibiliter, non est necesse quod videatur substantia solis, ita ad
videndum aliquid intelligibiliter, non est necessarium quod videatur
essentia Dei.

&

第三異論に対しては、次のように言われるべきである。私たちが、全てを神に
おいて見るとか、神に従ってすべてについて判断するとか言われるのは、神の
光を分有することによって、すべてを認識し判断するかぎりにおいてである。
そして、理性の自然的光そのものも、神の光の一種の分有である。ちょうど、
私たちが、すべての可感的なものを、太陽において、すなわち、太陽の光によっ
て見、判断すると言うように。このことから、アウグスティヌスは、『ソリロ
クィア』1巻で、「なにか自らの太陽のようなもの(すなわち神)によって照
らされなければ、諸学の実相が見られることはできない」と述べる。ゆえに、
ちょうど何かを感覚的に見るために、太陽の実体が見られる必要がないように、
何かを知的に見るために、神の本質が見られることが必要とはならない。

\\

{\sc Ad quartum dicendum} quod visio intellectualis est eorum quae
sunt in anima per suam essentiam sicut intelligibilia in
intellectu. Sic autem Deus est in anima beatorum, non autem in anima
nostra; sed per praesentiam, essentiam, et potentiam.

&

第四異論に対しては、次のように言われるべきである。知的な直視は、ちょう
ど可知的なものが知性においてあるように、自らの本質によって魂の中にある
ものどもについてある。神は、そのように、至福者の魂の中にあるが、私達の
魂の中には、そのようでなく、現在、本質、能力によって存在する
\footnote{「それゆえ、このようにして、神は万物の中に、万物が神の権能に
従属するかぎりにおいて、能力によって存在し、万物が神の目から見て、裸で
あり露わであるかぎりにおいて、現在によって存在し、すでに述べられたよう
に、存在の原因として万物に臨在するかぎりにおいて、万物の中に本質によっ
て存在する。」({\it ST} I, q.8, a.3, c.)}。

\end{longtable}

\newpage
\rhead{a.~12}

\begin{center}
 {\Large {\bf ARTICULUS DUODECIMUS}}\\
{\large UTRUM PER RATIONEM NATURALEM DEUM IN HAC VITA\\COGNOSCERE POSSIMUS}\\
{\large 第十二項\\私たちは、自然理性によって神をこの生において認識できるか}

\end{center}

\begin{longtable}{p{21em}p{21em}}

{\sc  Ad duodecimum sic proceditur}. Videtur quod per naturalem rationem
Deum in hac vita cognoscere non possimus. Dicit enim Boetius, in libro
{\it de Consol}., quod {\it ratio non capit simplicem formam}. Deus
autem maxime est simplex forma, ut supra ostensum est. Ergo ad eius
cognitionem ratio naturalis pervenire non potest.

&

第十二項の問題へ議論は以下のように進められる。私たちは、自然理性によっ
て、神をこの生において認識できないと思われる。なぜなら、ボエティウスは
『哲学の慰め』の中で、「理性は単純形相を捉えない」と述べている。しかる
に神は、上で示されたように、最大限に単純形相である。ゆえに、神の認識へ、
自然理性は到達できない。

\\

2.~{\sc  Praeterea}, ratione naturali sine phantasmate nihil
intelligit anima, ut dicitur in III {\it de Anima}. Sed Dei, cum sit
incorporeus, phantasma in nobis esse non potest. Ergo cognosci non
potest a nobis cognitione naturali.

&

さらに、『デ・アニマ』第3巻で言われるように、表象像がなければ、自然理
性によって、魂はなにものも知性認識しない。しかるに、神は非物体的なので、
神の表象像が私たちのうちに存在することはありえない。ゆえに、自然的な認
識によって、私たちに神が認識されることはありえない。

\\

3.~{\sc Praeterea}, cognitio quae est per rationem naturalem, communis
est bonis et malis, sicut natura eis communis est. Sed cognitio Dei
competit tantum bonis, dicit enim Augustinus, I {\it de Trin}., quod
{\it mentis humanae acies in tam excellenti luce non figitur, nisi per
iustitiam fidei emundetur}. Ergo Deus per rationem naturalem cognosci
non potest.

&

さらに、自然理性による認識は、善人と悪人に共通する。それは、自然がこの
両者に共通であるのと同様である。しかるに、神の認識は、善人だけに適合す
る。じっさい、アウグスティヌスは『三位一体論』1巻で「人間の精神の眼差
しは、信仰の義によって清められないかぎり、かくも高い光の中で形成されな
い」と述べている。ゆえに、神は自然理性によって認識されない。

\\

{\sc Sed contra est} quod dicitur {\it Rom}. {\sc i}: {\it quod notum
est Dei, manifestum est in illis}, idest, quod cognoscibile est de Deo
per rationem naturalem.

&

しかし反対に、『ローマの信徒への手紙』1章で、「神について知られること
は、彼らの中で明らかである\footnote{「神について知りうる事柄は、彼らに
も明らかだからです。」(1:19)}」と言われているが、「神について知られる
こと」とは、神について自然理性によって認識可能なもののことである。

\\

{\sc Respondeo dicendum} quod naturalis nostra cognitio a sensu
principium sumit, unde tantum se nostra naturalis cognitio extendere
potest, inquantum manuduci potest per sensibilia. Ex sensibilibus
autem non potest usque ad hoc intellectus noster pertingere, quod
divinam essentiam videat, quia creaturae sensibiles sunt effectus Dei
virtutem causae non adaequantes. 

&

解答する。以下のように言われるべきである。私たちの自然本性的認識は、感
覚に端を発するので、私たちの自然本性的認識は、可感的事物に導かれうると
ころまでしか、達することができない。しかるに、私たちの知性は、可感的事
物によって、神の本質を見ることにまで到達することができない。可感的被造
物は、原因のちからに対当しない、神の結果だからである。

\\

Unde ex sensibilium cognitione non potest tota Dei virtus cognosci, et
per consequens nec eius essentia videri. Sed quia sunt eius effectus a
causa dependentes, ex eis in hoc perduci possumus, ut cognoscamus de
Deo {\it an est}; et ut cognoscamus de ipso ea quae necesse est ei
convenire secundum quod est prima omnium causa, excedens omnia sua
causata.

&

したがって、可感的事物の認識からは、神のすべてのちからが認識されること
はできず、その帰結として、神の本質が見られることもできない。しかし、神
の結果は、原因に依存しているから、それらの結果から、神について「存在す
るかどうか」を認識するところまで、そしてまた、神について、それが自らが
生み出したすべてのものを超える、万物の第一原因である限りにおいて、必然
的に神に適合する事柄を認識するところまで、私たちは導かれうる。

\\

Unde cognoscimus de ipso habitudinem ipsius ad creaturas, quod
scilicet omnium est causa; et differentiam creaturarum ab ipso, quod
scilicet ipse non est aliquid eorum quae ab eo causantur; et quod haec
non removentur ab eo propter eius defectum, sed quia superexcedit.

&

したがって私たちは神について、神の被造物への関係、すなわち万物の原因で
あることや、被造物の神との違い、すなわち神は原因である神から生み出され
たもののどれとも異なるということ、そしてそれら(被造物の性質)が神から
取り除かれるのは、神の欠陥のためではなく、卓越性のためであることなどを
認識する。

\\

{\sc Ad primum ergo dicendum} quod ratio ad formam simplicem
pertingere non potest, ut sciat de ea {\it quid est}, potest tamen de
ea cognoscere, ut sciat {\it an est}.

&

第一異論に対しては、それゆえ、次のように言われるべきである。(自然)理
性が単純形相へ到達することができないというのは、それについて「何である
か」を知ることはできないという意味である。しかしそれについて「存在する
か」を認識するということはできる。

\\

{\sc Ad secundum dicendum} quod Deus naturali cognitione cognoscitur
per phantasmata effectus sui.

&

第二異論に対しては、次のように言われるべきである。神は、自然的認識によっ
て、自らの結果の表象を通して、認識される。

\\

{\sc Ad tertium dicendum} quod cognitio Dei per essentiam, cum sit per
gratiam, non competit nisi bonis, sed cognitio eius quae est per
rationem naturalem, potest competere bonis et malis. Unde dicit
Augustinus, in libro {\it Retractationum}: {\it Non approbo quod in
oratione dixi, Deus, qui non nisi mundos verum scire voluisti,
responderi enim potest, multos etiam non mundos multa scire vera},
scilicet per rationem naturalem.

&

第三異論に対しては、次のように言われるべきである。本質によって神を認識
することは、恩恵によるから、善人のみに適合するが、自然理性によってある
神の認識は、善人にも悪人にも適合しうる。したがって、アウグスティヌスは、
『再論』で「わたしは祈りの中で、「清い者でなければ真を知ることを欲しな
い神よ」と言ったが、今はそれを認めない。なぜなら、清くない多くの者も、
多くの真を知っていると答えられるから」と述べている。つまり、自然理性に
よって多くのことを知っている、ということである。

\end{longtable}

\newpage
\rhead{a.~13}

\begin{center}
 {\Large {\bf ARTICULUS DECIMUSTERTIUS}}\\
{\large UTRUM PER GRATIAM HABEATUR ALTIOR COGNITIO DEI QUAM EA QUAE
 HABETUR PER RATIONEM NATURALEM}\\
{\large 第十三項\\恩恵によって、自然理性によって得られるよりも高い神の認識
 が得られるか}
\end{center}

\begin{longtable}{p{21em}p{21em}}

{\sc Ad decimumtertium sic proceditur}. Videtur quod per gratiam non
habeatur altior cognitio Dei, quam ea quae habetur per naturalem
rationem. Dicit enim Dionysius, in libro {\it de Mystica Theologia},
quod ille qui melius unitur Deo in hac vita, unitur ei sicut omnino
ignoto, quod etiam de Moyse dicit, qui tamen excellentiam quandam
obtinuit in gratiae cognitione. Sed coniungi Deo ignorando de eo {\it
quid est}, hoc contingit etiam per rationem naturalem. Ergo per
gratiam non plenius cognoscitur a nobis Deus, quam per rationem
naturalem.

&

第十三項の問題へ議論は以下のように進められる。恩恵によって、自然理性に
よって得られるよりも、より高い神の認識が得られるわけではないと思われる。
なぜなら、ディオニュシウスは『神秘神学』という書物で、この世で神により
よく合一される者は、まったく知られないものとしての神に合一されると言う。
このことは、モーゼについても言われているが、彼は、恩恵の認識において、
ある種の卓越性を獲得した。しかし、神について、その「何であるか」を知ら
ないで、神に結びつけられることは、自然理性によっても起こる。ゆえに、恩
恵によって、自然理性によるよりも十分に、神が私たちによって認識されるわ
けではない。

\\

2.~{\sc Praeterea}, per rationem naturalem in cognitionem divinorum
pervenire non possumus, nisi per phantasmata, sic etiam nec secundum
cognitionem gratiae. Dicit enim Dionysius, I cap. {\it de
Cael. Hier}., quod {\it impossibile est nobis aliter lucere divinum
radium, nisi varietate sacrorum velaminum\footnote{v\={e}l\={a}men,
\u{\i}nis, n.\ velo, I.~a cover, covering, clothing, robe, garment}
circumvelatum}. Ergo per gratiam non plenius cognoscimus Deum, quam
per rationem naturalem.

&

さらに、私たちは、表象像をとおしてでなければ、自然理性によって神的なも
のどもの認識へ到達することができないが、これは、恩恵による認識において
も同様である。なぜなら、ディオニュシウスは『天上階級論』1章で、「さま
ざまな聖なる覆いによって覆われているのでなければ、神的光線が私たちに輝
くことは不可能である」と述べているからである。ゆえに、恩恵によって、自
然理性によるよりも十分に、私たちが神を認識するわけではない。

\\

3.~{\sc Praeterea}, intellectus noster per gratiam fidei Deo
adhaeret. Fides autem non videtur esse cognitio, dicit enim Gregorius,
in Homil., quod ea quae non videntur {\it fidem habent, et non
agnitionem}. Ergo per gratiam non additur nobis aliqua excellentior
cognitio de Deo.

&

さらに、私たちの知性は、信仰の恩恵によって、神に固着する。しかるに、信
仰は認識であるようには思われない。じっさい、グレゴリウスは『教話』にお
いて、見えないものどもは、「信仰をもち、認識をもたない」、と述べている。
ゆえに、恩恵によって、私たちに、神についてのより優れたなんらかの認識が
加えられるのではない。

\\

{\sc Sed contra est} quod dicit apostolus, I {\it Cor}. {\sc ii}, {\it
nobis revelavit Deus per spiritum suum}, illa scilicet quae {\it nemo
principum\footnote{princeps , c\u{\i}pis, adj.\ and I.~subst.\ comm.\
[primus-capio], first in time or order} huius saeculi novit, idest
philosophorum}, ut exponit Glossa.

&

しかし反対に、使徒は『コリントの信徒への手紙1』2章で、「神は私たちに、
神の霊を通して啓示した\footnote{「わたしたちには、神が``霊''によってそ
のことを明らかに示してくださいました」(2:10)}」と言うが、それはすなわ
ち、「この世のどんな支配者もそれを知らないことを\footnote{「この世の支
配者たちはだれ一人、この知恵を理解しませんでした。」(2:8)}」啓示したと
述べている。注解はこれを解説して、この世というのは哲学者たちのことだと
いう。

\\

{\sc Respondeo dicendum} quod per gratiam perfectior cognitio de Deo
habetur a nobis, quam per rationem naturalem. Quod sic patet. 

&

解答する。以下のように言われるべきである。恩恵によって、自然理性による
よりも完全な神についての認識が、私たちによって持たれる。このことは、以
下のようにして明らかである。

\\

Cognitio enim quam per naturalem rationem habemus, duo requirit,
scilicet, phantasmata ex sensibilibus accepta, et lumen naturale
intelligibile, cuius virtute intelligibiles conceptiones ab eis
abstrahimus. Et quantum ad utrumque, iuvatur humana cognitio per
revelationem gratiae.

&

自然理性によって私たちが持つ認識は、二つのことを必要とする。すなわち、
可感的なものから受け取られた表象像と、自然的で可知的な光とである。この
光の力によって、私たちは、表象像から可知的な懐念を抽象する。そして、こ
のどちらにかんしても、人間の認識は、恩恵の啓示によって助けられる。

\\

Nam et lumen naturale intellectus confortatur per infusionem luminis
gratuiti. Et interdum etiam phantasmata in imaginatione hominis
formantur divinitus, magis exprimentia res divinas, quam ea quae
naturaliter a sensibilibus accipimus; sicut apparet in visionibus
prophetalibus.

&

すなわち、知性の自然的な光は、無償の光の流入によって強められる。また、
ときには、私たちが可感的なものから自然本性的に受け取る表象像よりも、よ
り神的なことがらを表現する表象像が、神によって人間の表彰力をとおして形
成されることがある。たとえばそれは、預言者たちの直視において明らかな通
りである。

\\

Et interdum etiam aliquae res sensibiles formantur divinitus, aut
etiam voces, ad aliquid divinum exprimendum; sicut in Baptismo visus
est spiritus sanctus in specie columbae, et vox patris audita est,
{\it Hic est Filius meus dilectus}.

&

またときには、なんらかの可感的事物や、さらには音声さえも、神的な何かを
表現するために、神によって形成されることがある。たとえば、洗礼において、
聖霊が鳩の形において見られたり、「これが私の愛する子である」という父の
声が聞かれたのはこの事例である。\footnote{「イエスは洗礼を受けると、す
ぐ水の中から上がられた。そのとき、天がイエスに向かって開いた。イエスは、
神の霊が鳩のように御自分の上に下ってくるのを御覧になった。そのとき、
「これは私の愛する子、わたしの心に敵う者」と言う声が、天から聞こえた。」
『マタイによる福音書』(3:16-17)}

\\

{\sc Ad primum ergo dicendum} quod, licet per revelationem gratiae in
hac vita non cognoscamus de Deo {\it quid est}, et sic ei quasi ignoto
coniungamur; tamen plenius ipsum cognoscimus, inquantum plures et
excellentiores effectus eius nobis demonstrantur; et inquantum ei
aliqua attribuimus ex revelatione divina, ad quae ratio naturalis non
pertingit, ut Deum esse trinum et unum.

&

第一異論に対しては、それゆえ、次のように言われるべきである。現生におけ
る恩恵の啓示によって、神について「何であるか」を私たちが認識せず、その
意味で、いわば知られていないものとしての神に結び付けられるとしても、し
かし、より多くの、より優れた神の結果が私たちに示される限りにおいて、ま
た、神は三にして一である、といった、自然理性では到達しないなんらかの属
性を、神の啓示に基づいて神に帰する限りにおいて、私たちは、より十分に、
神を認識する。

\\

{\sc Ad secundum dicendum} quod ex phantasmatibus, vel a sensu
acceptis secundum naturalem ordinem, vel divinitus in imaginatione
formatis, tanto excellentior cognitio intellectualis habetur, quanto
lumen intelligibile in homine fortius fuerit. Et sic per revelationem
ex phantasmatibus plenior cognitio accipitur, ex infusione divini
luminis.

&

第二異論に対しては、次のように言われるべきである。表象像が、自然の秩序
に従って感覚によって受け取られたにせよ、神によって表象力の中に形成され
たにせよ、人間の中の可知的な光が強ければ強いほど、それだけいっそう優れ
た知的認識が、表象像から得られる。したがって、啓示によって、神の光の流
入に基づき、表象像から、より豊かな認識が獲得される。

\\

{\sc Ad tertium dicendum} quod fides cognitio quaedam est, inquantum
intellectus determinatur per fidem ad aliquod cognoscibile. Sed haec
determinatio ad unum non procedit ex visione credentis, sed a visione
eius cui creditur. Et sic, inquantum deest visio, deficit a ratione
cognitionis quae est in scientia, nam scientia determinat intellectum
ad unum per visionem et intellectum primorum principiorum.

&

第三異論に対しては、次のように言われるべきである。信仰は、知性が信仰に
よってなんらかの認識可能なものへ限定されるかぎりにおいて、一種の認識で
ある。しかし、この限定が一つのものへと進むのは、信じる者の直視に基づく
のではなく、信じられている人の直視に基づく。このように、直視がない限り
において、信仰は、知の中にある認識の理念から欠けている。すなわち、知は
知性を、第一諸原理の直視と知性認識によって限定するのである。

\end{longtable}
\end{document}