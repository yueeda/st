\documentclass[10pt]{jsarticle} % use larger type; default would be 10pt
%\usepackage[utf8]{inputenc} % set input encoding (not needed with XeLaTeX)
%\usepackage[round,comma,authoryear]{natbib}
%\usepackage{nruby}
\usepackage{okumacro}
\usepackage{longtable}
%\usepqckage{tablefootnote}
\usepackage[polutonikogreek,english,japanese]{babel}
%\usepackage{amsmath}
\usepackage{latexsym}
\usepackage{color}
%\usepackage{tikz}

%----- header -------
\usepackage{fancyhdr}
\pagestyle{fancy}
\lhead{{\it Summa Theologiae} I, q.~30}
%--------------------


\title{{\bf PRIMA PARS}\\{\HUGE Summae Theologiae}\\Sancti Thomae
Aquinatis\\{\sffamily QUEAESTIO TRIGESTIMA}\\DE PLURALITATE PERSONARUM
IN DIVINIS}
\author{Japanese translation\\by Yoshinori {\sc Ueeda}}
\date{Last modified \today}

%%%% コピペ用
%\rhead{a.~}
%\begin{center}
% {\Large {\bf }}\\
% {\large }\\
% {\footnotesize }\\
% {\Large \\}
%\end{center}
%
%\begin{longtable}{p{21em}p{21em}}
%
%&
%
%\\
%\end{longtable}
%\newpage



\begin{document}

\maketitle

\begin{center}
{\Large 第三十問\\神におけるペルソナの複数性について}
\end{center}


\begin{longtable}{p{21em}p{21em}}

{\Huge D}{\scshape einde} de pluralitate Personarum. Et circa hoc quaeruntur
quatuor. 

\begin{enumerate}
 \item utrum sint plures personae in divinis.
 \item quot sunt.
 \item quid significent termini numerales in divinis.
 \item de communitate huius nominis persona.
\end{enumerate}


&



次にペルソナの複数性について。これについては四つのことが問われる。

\begin{enumerate}
 \item 神において複数のペルソナが存在するか。
 \item (神においペルソナは)いくつあるか。
 \item 神において数詞は何を表示するか。
 \item ペルソナというこの名称の共通性について。
\end{enumerate}

\end{longtable}


\newpage



\rhead{a.~1}
\begin{center}
{\Large {\bf ARTICULUS PRIMUS}}\\
{\large UTRUM SIT PONERE PLURES PERSONAS IN DIVINIS}\\
{\footnotesize {\itshape Sent.}, d.2, a.4; d.23, a.4; {\itshape De
 Pot.}, q.9, a.5; {\itshape Compend.~Theol.}, cap.50, 55; {\itshape
 Quodl.}~VII, q.3, a.1.}\\
{\Large 第一項\\神において複数のペルソナを措定することができるか}
\end{center}

\begin{longtable}{p{21em}p{21em}}

{\scshape Ad primum sic proceditur}. Videtur quod non sit ponere plures personas
in divinis. Persona enim est rationalis naturae individua
substantia. Si ergo sunt plures personae in divinis, sequitur quod
sint plures substantiae, quod videtur haereticum.


&

第一項の問題へ議論は以下のように進められる。神において複数のペルソナを
 措定することはできないと思われる。理由は以下の通り。ペルソナとは理性
 的本性をもつ個的実体である。ゆえにもし神において複数のペルソナが存在
 するならば複数の実体があることになるが、これは異端である。

\\



2. {\scshape Praeterea}, pluralitas proprietatum absolutarum non facit distinctionem
personarum, neque in Deo neque in nobis, multo igitur minus pluralitas
relationum. Sed in Deo non est alia pluralitas nisi relationum, ut
supra dictum est. Ergo non potest dici quod in Deo sint plures
personae.


&

さらに、神においても私たちにおいても、非関係的な固有性の複数性はペルソ
 ナの区別をもたらさない。ましてや関係の複数性はそのような区別をもたら
 さない。しかるに、前に述べられたとおり神において関係以外の他の複数性
 はない。ゆえに神の中に複数のペルソナが存在すると言われることはできな
 い。

\\



3. {\scshape Praeterea}, Boetius dicit, de Deo loquens, quod hoc vere unum est, in
quo nullus est numerus. Sed pluralitas importat numerum. Ergo non sunt
plures personae in divinis.


&

さらにボエティウスは神について語りながら、そこにどんな数もないものが真
 に一であると言っている。しかるに複数性は数を含意する。ゆえに神におい
 て複数のペルソナは存在しない。


\\



4. {\scshape Praeterea}, ubicumque est numerus, ibi est totum et pars. Si igitur in
Deo sit numerus personarum, erit in Deo ponere totum et partem, quod
simplicitati divinae repugnat.


&

さらに数があるところにはどこでも全体と部分がある。ゆえにもし神において
 ペルソナの数があるならば、神において全体と部分を措定することができる
 だろう。しかしこれは神の単純性に反する。

\\



{\scshape Sed contra est quod} dicit Athanasius : {\itshape Alia est persona patris, alia
filii, alia spiritus sancti}. Ergo pater et filius et spiritus sanctus
sunt plures personae.


&

しかし反対にアタナシウスは「父、息子、聖霊のペルソナは別々である」と言っ
 ている。ゆえに父、息子、聖霊は複数のペルソナである。


\\



{\scshape Respondeo dicendum} quod plures esse personas in divinis, sequitur ex
praemissis. Ostensum est enim supra quod hoc nomen {\itshape persona} significat
in divinis relationem, ut rem subsistentem in natura divina. Supra
autem habitum est quod sunt plures relationes reales in divinis. Unde
sequitur quod sint plures res subsistentes in divina natura. Et hoc
est esse plures personas in divinis.


&

解答する。以下のように言われるべきである。
すでに述べられたことから神に複数のペルソナが存在することが帰結する。理
 由は以下の通り。「ペルソナ」というこの名称が、神において神の本性にお
 いて自存する事物として関係を表示することが示された。さらに、神におい
 て複数の実在的関係が存在することが上で論じられた。したがって神の本性
 において複数の自存する事物が存在することが帰結する。そしてこれは、神
 において複数のペルソナが存在するということである。

\\



{\scshape Ad primum ergo dicendum} quod {\itshape substantia} non ponitur in definitione
personae secundum quod significat essentiam, sed secundum quod
significat suppositum, quod patet ex hoc quod additur {\itshape individua}. Ad
significandum autem substantiam sic dictam, habent Graeci nomen
{\itshape hypostasis}, unde sicut nos dicimus tres personas, ita ipsi dicunt tres
hypostases. Nos autem non consuevimus dicere tres substantias, ne
intelligerentur tres essentiae, propter nominis aequivocationem.


&

第一異論に対してはそれゆえ以下のように言われるべきである。
「実体」がペルソナの定義の中に置かれるのは本質を意味する限りにおいてで
 はなく個体を意味する限りにおいてである。これは、「個的」と付加されて
 いることから明らかである。しかるにギリシア人たちはこのように語られた
 実体を意味するために「ヒュポスタシス」という名称をもっているので、彼
 らが三つのヒュポスタシスと語るようなしかたで私たちは三つのペルソナと
 語る。しかし私たちは、名称の両義性のために三つの本質と理解されないよ
 うに、慣例として三つの実体とは語らない。


\\



{\scshape Ad secundum dicendum} quod proprietates absolutae in divinis, ut
bonitas et sapientia, non opponuntur ad invicem, unde neque realiter
distinguuntur. Quamvis ergo eis conveniat subsistere, non tamen sunt
plures res subsistentes, quod est esse plures personas. Proprietates
autem absolutae in rebus creatis non subsistunt, licet realiter ab
invicem distinguantur, ut albedo et dulcedo. Sed proprietates
relativae in Deo et subsistunt, et realiter ab invicem distinguuntur,
ut supra dictum est. Unde pluralitas talium proprietatum sufficit ad
pluralitatem personarum in divinis.


&

第二異論に対しては以下のように言われるべきである。
善性や智恵のような神における非関係的な固有性は相互に対立しない。したがっ
 て実在的にも区別されない。ゆえに、それらに自存することが適合するとし
 ても複数の事物が自存することはない。これは複数のペルソナが存在するこ
 とである。また、白性や甘さのような被造物における非関係的な固有性は、相互に実在的
 に区別されるとしても自存しない。しかし神における関係的な固有性は、前
 に述べられたとおり自存しかつ相互に実在的に区別される。したがってその
 ような固有性の複数性は、神におけるペルソナの複数性にとって十分である。


\\



{\scshape Ad tertium dicendum} quod a Deo, propter summam unitatem et
simplicitatem, excluditur omnis pluralitas absolute dictorum; non
autem pluralitas relationum. Quia relationes praedicantur de aliquo ut
ad alterum; et sic compositionem in ipso de quo dicuntur non
important, ut Boetius in eodem libro docet.


&

第三異論に対しては以下のように言われるべきである。
神からは、その最高の一性と単純性のために非関係的に語られるものどもの複
 数制は排除されるが、関係の複数性は排除されない。なぜなら関係はあるも
 のについて他のものへあるものとして述語されるので、ボエティウスが同じ
 書物で教えているように、それについて語られるところのものにおける複合
 を意味しないからである。


\\



{\scshape Ad quartum dicendum} quod numerus est duplex, scilicet numerus simplex
vel absolutus, ut duo et tria et quatuor; et numerus qui est in rebus
numeratis, ut duo homines et duo equi. Si igitur in divinis accipiatur
numerus absolute sive abstracte, nihil prohibet in eo esse totum et
partem, et sic non est nisi in acceptione intellectus nostri; non enim
numerus absolutus a rebus numeratis est nisi in intellectu. Si autem
accipiamus numerum prout est in rebus numeratis, sic in rebus quidem
creatis, unum est pars duorum, et duo trium, ut unus homo duorum, et
duo trium, sed non est sic in Deo, quia tantus est pater quanta tota
Trinitas, ut infra patebit.


&

第四異論に対しては以下のように言われるべきである。
数には二通りのものがある。一つは、2, 3, 4のような単純で非関係的な数で
 あり、もう一つは二人の人間や二頭の馬のように、数えられた事物において
 ある数である。ゆえにもし神において数が非関係的にあるいは抽象的に理解
 されるなら、神において部分と全体があることを妨げるものはない。この意
 味ではそれは私たちの知性の理解においてのみ存在する。なぜなら数えられ
 た事物から抽象された数は知性の中にのみ存在するからである。他方で、も
 し数えられた事物において存在する数を私たちが理解するならば、ちょうど
 一人の人は二人の人の部分であり、二人の人は三人の人の部分であるように、
 一は二の、二は三の部分である。しかし神においてはそうでない。なぜなら、
 後で明らかになるように、父は三性全体の大きさと等しいからである。


\end{longtable}
\newpage


\rhead{a.~2}
\begin{center}
{\Large {\bf ARTICULUS SECUNDUS}}\\
{\large UTRUM IN DEO SINT PLURES PERSONAE QUAM TRES}\\
 {\footnotesize I {\itshape Sent.}, d.10, a.5; d.33, a.2, ad 1; IV
 {\itshape SCG}, cap.26; {\itshape De Pot.}, q.9, a.9; {\itshape
 Compend.~Theol.}, cap.56, 60.}\\
{\Large 第二項\\神の中に三以上の複数のペルソナがあるか}
\end{center}

\begin{longtable}{p{21em}p{21em}}

{\scshape Ad secundum sic proceditur}. Videtur quod in Deo sint plures personae
quam tres. Pluralitas enim personarum in divinis est secundum
pluralitatem proprietatum relativarum, ut dictum est. Sed quatuor sunt
relationes in divinis, ut supra dictum est, scilicet paternitas,
filiatio, communis spiratio et processio. Ergo quatuor personae sunt
in divinis.

&

第二項の問題へ議論は以下のように進められる。
神の中に三以上の複数のペルソナがあると思われる。理由は以下の通り。
すでに述べられたように、神におけるペルソナの複数性は、関係的な固有性の
 複数性に応じてある。しかるに前に述べられたとおり神における関係は四つ、
 すなわち父性、息子性、共通の霊発、発出である。ゆえに神においては四つ
 のペルソナが存在する。


\\



2. {\scshape Praeterea}, non plus differt natura a voluntate in Deo, quam natura ab
intellectu. Sed in divinis est alia persona quae procedit per modum
voluntatis, ut amor; et alia quae procedit per modum naturae, ut
filius. Ergo est etiam alia quae procedit per modum intellectus, ut
verbum; et alia quae procedit per modum naturae, ut filius. Et sic
iterum sequitur quod non sunt tantum tres personae in divinis.

&

さらに神においては、本性が知性と異なる以上に本性が意志と異なることはな
 い。しかし神において意志のかたちで発出するペルソナすなわち愛と、本性
 のかたちで発出するペルソナすなわち息子とは別のペルソナである。ゆえに、
 知性のかたちで発出するペルソナすなわち言葉と、本性
 のかたちで発出するペルソナすなわち息子とは別のペルソナである。かくし
 てただ三つだけ神の中にペルソナが存在するのではないことがさらに帰結す
 る。

\\



3. {\scshape Praeterea}, in rebus creatis quod excellentius est, plures habet
operationes intrinsecas, sicut homo supra alia animalia habet
intelligere et velle. Sed Deus in infinitum excedit omnem
creaturam. Ergo non solum est ibi persona procedens per modum
voluntatis et per modum intellectus, sed infinitis aliis modis. Ergo
sunt infinitae personae in divinis.

&

さらに、被造物において、よりすぐれたものはより多くの内的な働きをもつ。
 たとえば人間は他の動物にまさって知性認識と意志とをもつ。しかし神はす
 べての被造物を無限に越えている。ゆえに、そこに意志と知性のしかたで発
 出するペルソナだけでなく無限に他のしかたで発出するペルソナがある。ゆ
 えに神の中には無限のペルソナが存在する。


\\



4. {\scshape Praeterea}, ex infinita bonitate patris est, quod infinite seipsum
communicet, producendo personam divinam. Sed etiam in spiritu sancto
est infinita bonitas. Ergo spiritus sanctus producit divinam personam,
et illa aliam, et sic in infinitum.

&

さらに、神のペルソナを生み出すことによって自らを無
 限に伝達することは父の無限の善性に基づく。しかし聖霊においても無限の
 善性がある。ゆえに聖霊は神のペルソナを生み出し、それはまた別のペルソ
 ナを生み出し、といったように無限に続く。

\\



5. {\scshape Praeterea,} omne quod continetur sub determinato numero, est
mensuratum, numerus enim mensura quaedam est. Sed personae divinae
sunt immensae, ut patet per Athanasium, immensus pater, immensus
filius, immensus spiritus sanctus. Non ergo sub numero ternario
continentur.

&

さらに、ある限定された数のもとに含まれるものはすべて測られている。数は何
 らかの尺度だかからである。しかるに神のペルソナはアタナシウスの「測ら
 れない父、測られない息子、測られない聖霊」によれば測られない。ゆえに
 三という数のもとに含まれていない。


\\



{\scshape Sed contra est} quod dicitur I Ioan. ult. : {\itshape Tres sunt qui testimonium
dant in caelo, pater, verbum et spiritus sanctus}. Quaerentibus autem,
quid tres? Respondetur, tres personae, ut Augustinus dicit, in VII de
Trin. Sunt igitur tres personae tantum in divinis.

&

しかし反対に、『ヨハネの手紙1』の最後に「天において証言を与えるのは父、
 言葉、聖霊の三人である」と述べられている。しかし、アウグスティヌスが
 『三位一体論』第7巻で言うように、「三とは何か」と問う人々には「三つの
 ペルソナ」と答えられている。ゆえに神においてペルソナは三つだけ存在す
 る。

\\



{\scshape Respondeo dicendum} quod, secundum praemissa, necesse est ponere tantum
tres personas in divinis. Ostensum est enim quod plures personae sunt
plures relationes subsistentes, ab invicem realiter distinctae. Realis
autem distinctio inter relationes divinas non est nisi in ratione
oppositionis relativae. 



&

解答する。以下のように言われるべきである。
すでに述べられたことに従えば、神においてペルソナを三つだけ措定することが
 必然である。理由は以下の通り。
複数のペルソナとは相互に実在的に区別された複数の自存する関係であること
 が示された。
ところで、神の関係の間の実在的な区別は関係的な対立という性格において以
 外にはない。


\\



Ergo oportet duas relationes oppositas ad duas
personas pertinere, si quae autem relationes oppositae non sunt, ad
eandem personam necesse est eas pertinere. Paternitas ergo et
filiatio, cum sint oppositae relationes, ad duas personas ex
necessitate pertinent. Paternitas igitur subsistens est persona
patris, et filiatio subsistens est persona filii.


&

ゆえに、二つの対立関係は二つのペルソナに属さなければならないが、もしこ
 れらが対立関係でないならばそれらは同一のペルソナに属することが必然で
 ある。ゆえに父性と息子性は対立関係なので必然的に二つのペルソナに属す
 る。それゆえ、自存する父性は父のペルソナであり、自存する息子性は息子のペルソ
 ナである。


\\




 Aliae autem duae
relationes ad neutram harum oppositionem habent, sed sibi invicem
opponuntur. Impossibile est igitur quod ambae uni personae
conveniant. 
Oportet ergo quod vel una earum conveniat utrique dictarum
personarum, aut quod una uni, et alia alii.

&

他方、他の二つの関係(spiratio, processio)は、これらのどちら
 (paternitas, filiatio)にも対立せず、それら同士で対立し
 ている。ゆえに両者が一つのペルソナに適合することは不可能である
 \footnote{spiratioとprocessioは対立する関係なので一つのペルソナには属さない。}。
それゆえ、これらのうちの一つが述べられたペルソナの両方に適合する
 \footnote{spiratioまたはprocessioがpaternitasとfiliatioの両方に属する。
 ((paternitas, spiratio)(filiatio, spiratio)(processio)) or
 ((paternitas, processio)(filiatio, processio)(spiratio))}か、あ
 るいは一方が片方に、他方が別の方に属するか\footnote{spiratioが
 paternitasに、processioがfiliatioに属する、あるいは、spiratioがfiliatioに、
 processioがpaternitasに属する。(paternitas, spiratio)(filiatio,
 processio) or (paternitas, processio)(filiatio, spiratio). 【訳者注】
 この場合にどうしてペルソナが三になるかは不明。}のいずれかである。

\\




 Non autem potest esse quod
processio conveniat patri et filio, vel alteri eorum, quia sic
sequeretur quod processio intellectus, quae est generatio in divinis,
secundum quam accipitur paternitas et filiatio, prodiret ex
processione amoris, secundum quam accipitur spiratio et processio, si
persona generans et genita procederent a spirante, quod est contra
praemissa. 


&

しかし発出が父と息子に、あるいはそれらの一方に適合することはありえない。
 なぜなら、もしそうだとすると、また、もし生むペルソナと生まれたペルソ
 ナが霊発させるものから発出したならば、それにおいて父性と息子性が理解されると
 ころの神における生成である知性の発出が、霊発と発出がそれにおいて理解
 される愛の発出から発出したことになるが、これは上述のことに反するからで
 ある。


\\




Relinquitur ergo quod spiratio conveniat et personae patris
et personae filii, utpote nullam habens oppositionem relativam nec ad
paternitatem nec ad filiationem. Et per consequens oportet quod
conveniat processio alteri personae, quae dicitur persona spiritus
sancti, quae per modum amoris procedit, ut supra habitum
est. Relinquitur ergo tantum tres personas esse in divinis, scilicet
patrem et filium et spiritum sanctum.

&

それゆえ残されるのは、霊発が、父性にも息子性にも対立関係にないものとし
 て父と息子のペルソナに適合することである。そしてその帰結として、発出
 は別のペルソナに適合しなければならず、このペルソナは聖霊と言われる。
 ゆえに、ただ三つのペルソナが神において存在する、すなわち、父と息子と
 聖霊が存在することが(消去法から)帰結する。\footnote{前の注にある組
 み合わせの中から、paternitasまたはfiliatioがprocessioと組むこ
 とはできないとすれば、残るのは((paternitas, spiratio)(filiatio,
 spiratio)(processio))の組だけである。これが三性における四つの実在的関係
 のありかたを示している。}

\\



{\scshape Ad primum ergo dicendum} quod, licet sint quatuor relationes in
divinis, tamen una earum, scilicet spiratio, non separatur a persona
patris et filii, sed convenit utrique. Et sic, licet sit relatio, non
tamen dicitur {\itshape proprietas}, quia non convenit uni tantum personae, neque
est {\itshape relatio personalis}, idest constituens personam. Sed hae tres
relationes, paternitas, filiatio et processio, dicuntur {\itshape proprietates
personales}, quasi personas constituentes, nam paternitas est persona
patris, filiatio persona filii, processio persona spiritus sancti
procedentis.

&

第一異論に対してはそれゆえ以下のように言われるべきである。
たしかに神において四つの関係があるが、そのうちの一つすなわち霊発は父と
 息子のペルソナから分離されず、その両方に属する。かくして、それは関係ではあ
 るが「固有性」とは言われない。なぜならただ一つのペルソナに属するので
 はないからである。また「ペルソナ的関係」すなわちペルソナを構成する関
 係でもない。これに対して、父性、息子性、発出のこれら三つの関係はペル
 ソナを構成するものとして「ペルソナ的固有性」と言われる。すなわち、父
 性は父のペルソナであり息子性は息子のペルソナであり発出は聖霊のペルソ
 ナである。


\\



{\scshape Ad secundum dicendum} quod id quod procedit per modum intellectus, ut
verbum, procedit secundum rationem similitudinis, sicut etiam id quod
procedit per modum naturae, et ideo supra dictum est quod processio
verbi divini est ipsa generatio per modum naturae. Amor autem,
inquantum huiusmodi, non procedit ut similitudo illius a quo procedit
(licet in divinis amor sit coessentialis inquantum est divinus), et
ideo processio amoris non dicitur generatio in divinis.

&

第二異論に対しては以下のように言われるべきである。
知性のあり方によって言葉として発出するものは、自然本性のあり方によって
 発出するものもそうであるように、類似という性格にしたがって発出する。
 ゆえに前に神の言葉の発出は自然本性による生成それ自体であると述べられ
 た。これに対して愛は、愛であるかぎりにおいて、そこから発出するところ
 のものの類似として発出するのではない(ただし、神において愛は、それが
 神の愛である限りにおいて神と本質を同じくする)。ゆえに、愛の発出は神
 において生成と言われない。

\\



{\scshape Ad tertium dicendum} quod homo, cum sit perfectior aliis animalibus,
habet plures operationes intrinsecas quam alia animalia, quia eius
perfectio est per modum compositionis. Unde in Angelis, qui sunt
perfectiores et simpliciores, sunt pauciores operationes intrinsecae
quam in homine, quia in eis non est imaginari, sentire, et
huiusmodi. Sed in Deo, secundum rem, non est nisi una operatio, quae
est sua essentia. Sed quomodo sunt duae processiones, supra ostensum
est.

&

第三異論に対しては以下のように言われるべきである。
人間は他の動物よりも完全なので、他の動物よりも多くの内的な働きをもつが、
 それは人間の完全性が複合というあり方によるからである。したがって天使
 において、天使たちはより完全でより単純なので、人間よりも少ない内的な
 働きしかない。なぜなら彼らにおいては想像すること、感覚することといったものがないから
 である。しかし神においては、実在的に、一つの働きすなわち神の本質しかな
 い。どのように二つの発出があるかは前に示された。



\\



{\scshape Ad quartum dicendum} quod ratio illa procederet, si spiritus sanctus
haberet aliam numero bonitatem a bonitate patris, oporteret enim quod,
sicut pater per suam bonitatem producit personam divinam, ita et
spiritus sanctus. Sed una et eadem bonitas patris est et spiritus
sancti. Neque etiam est distinctio nisi per relationes
personarum. Unde bonitas convenit spiritui sancto quasi habita ab
alio, patri autem, sicut a quo communicatur alteri. Oppositio autem
relationis non permittit ut cum relatione spiritus sancti sit relatio
principii respectu divinae personae, quia ipse procedit ab aliis
personis quae in divinis esse possunt.

&

第四異論に対しては以下のように言われるべきである。
この異論は、もし聖霊が父の善性とは数的に異なる善性をもっていたら成り立っ
 たであろう。なぜなら、父が自らの善性によって神のペルソナを生み出すよ
 うに、聖霊もまたそうだっただろうから。しかし、父と聖霊には一
 つで同一の善性が属する。さらにその区別はペルソナ的な関係による以外に
 はない。したがって善性は聖霊に、他によって持たれたものとして属するが、
 父へは、それによって他に伝達されるものとして属する。しかし、関係の対
 立は、聖霊の関係と共に、神のペルソナに対する根源の関係があることを許
 さない。なぜならそれ自身が神の中にありうる他の複数のペルソナから発出
 するからである。


\\



{\scshape Ad quintum dicendum} quod numerus determinatus, si accipiatur numerus
simplex, qui est tantum in acceptione intellectus, per unum
mensuratur. Si vero accipiatur numerus rerum in divinis personis, sic
non competit ibi ratio mensurati, quia eadem est magnitudo trium
personarum, ut infra patebit; idem autem non mensuratur per idem.

&


第五異論に対しては以下のように言われるべきである。
限定された数は、もしそれがただ知性の把握の中にだけ存在する単純な数だと
 理解されるならば、一によって測られる。他方、もし神のペルソナにおいて
 事物の数が理解されるならば、そこに測られたものという性格は適合しない。
 なぜなら、後に明らかになるように、三つのペルソナの大きさは同じだから
 である。しかし同じものが同じものによって測られるわけではない。

\end{longtable}
\newpage


\rhead{a.~3}
\begin{center}
{\Large {\bf ARTICULUS TERTIUS}}\\
{\large UTRUM TERMINI NUMERALES PONANT ALIUQID IN DIVINIS}\\
{\footnotesize I {\itshape Sent.}, d.24, a.3; {\itshape De Pot.}, q.9,
 a.7; {\itshape Quodl.}~X, q.1, a.1.}\\
{\Large 第三項\\神において数詞は何かを措定するか}
\end{center}

\begin{longtable}{p{21em}p{21em}}
{\scshape Ad tertium sic proceditur}. Videtur quod termini numerales ponant
aliquid in divinis. Unitas enim divina est eius essentia. Sed omnis
numerus est unitas repetita. Ergo omnis terminus numeralis in divinis
significat essentiam. Ergo ponit aliquid in Deo.

&

第三項の問題へ議論は以下のように進められる。数詞は神において何かを措定
 すると思われる。理由は以下の通り。
神の一性は神の本質である。しかるにすべての数は反復された一性である。ゆ
 えにすべての数詞は神において本質を表示する。ゆえに神において何かを措
 定する。

\\



2. {\scshape Praeterea}, quidquid dicitur de Deo et creaturis, eminentius convenit
Deo quam creaturis. Sed termini numerales in creaturis aliquid
ponunt. Ergo multo magis in Deo.

&

さらに、神と被造物とについて語られるものはなんでも、被造物よりも神によ
 り優れたしかたで適合する。しかるに数詞は被造物において何かを措定する。
 ゆえにいわんや神においても何かを措定する。

\\



3. {\scshape Praeterea}, si termini numerales non ponunt aliquid in divinis, sed
inducuntur ad removendum tantum, ut per pluralitatem removeatur
unitas, et per unitatem pluralitas; sequitur quod sit circulatio in
ratione, confundens intellectum et nihil certificans; quod est
inconveniens. Relinquitur ergo quod termini numerales aliquid ponunt
in divinis.

&

さらに、もし数詞が神において何も措定せず、複数性によって一性が除去され
 一性によって複数性が除去されるようにただ除去するためにだけ導入される
 ならば、概念の中に循環が発生し知性を混乱させ何も明らかにしないことが
 帰結する。ゆえに数詞は何かを神において措定することが帰結する。


\\



{\scshape Sed contra est} quod Hilarius dicit, in IV {\itshape de
 Trin}. : {\itshape Sustulit
singularitatis ac solitudinis intelligentiam professio consortii}, quod
est professio pluralitatis. Et Ambrosius dicit, in libro {\itshape de
 Fide} : {\itshape Cum
unum Deum dicimus, unitas pluralitatem excludit deorum, non
quantitatem in Deo ponimus}. Ex quibus videtur quod huiusmodi nomina
sunt inducta in divinis ad removendum, non ad ponendum aliquid.

&

しかし反対に、ヒラリウスは『三位一体論』第4巻で「集いを認めること
 (すなわち複数性を認めること)は単
 独や孤独という理解を排除する」と言っている。またアンブロシウスは『信
 仰について』という書物の中で「一なる神と私たちが言うとき、一性は神々の複数
 性を排除するのであり、神の中に私たちが量を措定しているのではない」と
 述べている。これらのことから、このような名称は、神において何かを措定
 するためではなく除去するために導入されたと思われる。


\\



{\scshape Respondeo dicendum} quod Magister, in sententiis, ponit quod termini
numerales non ponunt aliquid in divinis, sed removent tantum. Alii
vero dicunt contrarium. 


&

解答する。以下のように言われるべきである。
教師(ペトルス・ロンバルドゥス)は『命題集』の中で、数詞は神において何
 かを措定せず除去するだけだと主張している。他方で他の人々は反対のことを
 述べている。


\\

Ad evidentiam igitur huius, considerandum est
quod omnis pluralitas consequitur aliquam divisionem. 
Est autem duplex
divisio. Una materialis, quae fit secundum divisionem continui, et
hanc consequitur numerus qui est species quantitatis. Unde talis
numerus non est nisi in rebus materialibus habentibus
quantitatem. 


&

ゆえに、これを明らかにするためにはすべての数詞が何らかの分割に伴うこと
 が考察されるべきである。
さらに分割には二通りある。一つは質料的なものであり、連続体の分割に従っ
 て生じ、これに量の種である数が伴う。したがって、このような数は量をも
 つ質料的事物においてしかない。



\\

Alia est divisio formalis, quae fit per oppositas vel
diversas formas, et hanc divisionem sequitur multitudo quae non est in
aliquo genere, sed est de transcendentibus, secundum quod ens
dividitur per unum et multa. 
Et talem multitudinem solam contingit
esse in rebus immaterialibus. 


&

もう一つは形相的な分割であり、対立あるいは相違する形相によって生じ、こ
 の分割に伴う多は、なんらかの類の中にはなく、超越的なものに属し、これ
 によって有が一と多に分割される。そしてこのような多は非質料的事物にお
 いてのみ生じる。


\\


Quidam igitur, non considerantes nisi
multitudinem quae est species quantitatis discretae, quia videbant
quod quantitas discreta non habet locum in divinis, posuerunt quod
termini numerales non ponunt aliquid in Deo, sed removent tantum. Alii
vero, eandem multitudinem considerantes, dixerunt quod, sicut scientia
ponitur in Deo secundum rationem propriam scientiae, non autem
secundum rationem sui generis, quia in Deo nulla est qualitas; ita
numerus in Deo ponitur secundum propriam rationem numeri, non autem
secundum rationem sui generis, quod est quantitas. 


&

それゆえ、ある人々は量の分割された種である多しか考えず、
分割された量は神の中に場所をもたないので、神において数詞は何も措定せず
 除去するだけだと主張した。他方、他の人々は、同じ多を考えて、ちょうど
 知がその知が固有の性格に即して神の中に措定され、自らの類の性格に即し
 てではないように、というのも神の中に性質はないからだが、そのように、
 神の中に数が措定されるのは数の固有の性格に即してであり数の類の性格、
 すなわち量に即してではないと述べた。

\\

Nos autem dicimus
quod termini numerales, secundum quod veniunt in praedicationem
divinam, non sumuntur a numero qui est species quantitatis; quia sic
de Deo non dicerentur nisi metaphorice, sicut et aliae proprietates
corporalium, sicut latitudo, longitudo, et similia, sed sumuntur a
multitudine secundum quod est transcendens. 


&

しかし私たちは、数詞は、それが神の述語に入る限りにおいて、量の種である
 数から取られていないと述べる。なぜなら、もしそうであったら、広さや長
 さやそれに類似するもののような多の物体的な固有性のように神につい
 て比喩的にしか語られなかったであろうから。そうではなく、超越的である
 かぎりでの多から取られていると私たちは述べる。


\\


Multitudo autem sic
accepta hoc modo se habet ad multa de quibus praedicatur, sicut unum
quod convertitur cum ente ad ens. Huiusmodi autem unum, sicut supra
dictum est, cum de Dei unitate ageretur, non addit aliquid supra ens
nisi negationem divisionis tantum, unum enim significat ens
indivisum. Et ideo de quocumque dicatur {\itshape unum}, significatur illa res
indivisa, sicut {\itshape unum} dictum de homine, significat naturam vel
substantiam hominis non divisam. 



&

このように理解された多は、それについて述語される多に対して、ちょうど有
 と置換される一が有に対するように関係する。前に神の一性について論じら
 れたときに述べられたとおり、その
 ような一は分割の否定以外のなにものも有に加えない。一とは分割されない
 有だからである。ゆえに一が何について語られても、分割されていないかの
 事物が表示される。たとえば人間について語られた一は分割されていない人
 間の本性ないし実体を表示する。


\\



Et eadem ratione, cum dicuntur res
{\itshape multae}, multitudo sic accepta significat res illas cum indivisione
circa unamquamque earum. Numerus autem qui est species quantitatis,
ponit quoddam accidens additum supra ens, et similiter unum quod est
principium numeri. 



&

同じ理由で、多くの事物について語られるとき、そのように理解された多は、
 その各々の事物についての不分割とともにそれらの事物を表示する。これに
 対して量の種である数は有に加えられた何らかの附帯性を述べ、数の
 根源である一もまた同様である。


\\


Termini ergo numerales significant in divinis illa
de quibus dicuntur, et super hoc nihil addunt nisi negationem, ut
dictum est, et quantum ad hoc, veritatem dixit Magister in
{\itshape Sententiis}. Ut, cum dicimus, {\itshape essentia est una}, unum significat
essentiam indivisam, cum dicimus, {\itshape persona est una}, significat personam
indivisam, cum dicimus, {\itshape personae sunt plures}, significantur illae
personae, et indivisio circa unamquamque earum; quia de ratione
multitudinis est, quod ex unitatibus constet.

&

ゆえに、数詞は神において、それについて語られるものを表示し、すでに述べ
 られたようにそれに否定以外のものを加えない。そしてこれにかんする限り、
 教師は『命題集』において真理を語っている。すなわち私たちが「本質は一
 である」と語るとき、一は不可分の本質を表示しているし、「ペルソナは一
 である」と語るときも、不可分のペルソナを表示するからである。また「ペ
 ルソナは複数である」と語るとき、それらのペルソナと、それぞれのペルソ
 ナについては不可分を表示する。なぜなら多の性格には、複数の一から構成
 されるということが含まれるからである。





\\



{\scshape Ad primum ergo dicendum} quod unum, cum sit de transcendentibus, est
communius quam substantia et quam relatio, et similiter
multitudo. Unde potest stare in divinis et pro substantia et pro
relatione, secundum quod competit his quibus adiungitur. Et tamen per
huiusmodi nomina, supra essentiam vel relationem, additur, ex eorum
significatione propria, negatio quaedam divisionis, ut dictum est.

&

第一異論に対してはそれゆえ以下のように言われるべきである。
一は超越的なものに属するので実体や関係よりも共通であり、この点は多も同
 様である。したがって神において、それに結び付けられたものに適合するか
 ぎりで実体も関係も表すことができる。しかしこのような名称によって、そ
 の固有の意味から、本質や関係にある種の分割の否定しか加えられない。


\\



{\scshape Ad secundum dicendum} quod multitudo quae ponit aliquid in rebus
creatis, est species quantitatis; quae non transumitur in divinam
praedicationem; sed tantum multitudo transcendens, quae non addit
supra ea de quibus dicitur, nisi indivisionem circa singula. Et talis
multitudo dicitur de Deo.

&

第二異論に対しては以下のように言われるべきである。
被造物において何かを措定する多は量の種であり、それは神の述語へと転用さ
 れない。転用されるのは超越的な多だけであり、これは語られたものどもに
 それぞれについての不分割しか加えない。そしてこのような多が神について
 語られる。



\\



{\scshape Ad tertium dicendum} quod unum non est remotivum multitudinis, sed
divisionis, quae est prior, secundum rationem, quam unum vel
multitudo. Multitudo autem non removet unitatem, sed removet
divisionem circa unumquodque eorum ex quibus constat multitudo. Et
haec supra exposita sunt, cum de divina unitate ageretur. 

&

第三異論に対しては以下のように言われるべきである。
一は多を排除するのではなく分割を排除する。分割は概念において一や多に先
 行する。また多は一を排除せず、多がそれらから構成される各々のものにつ
 いての分割を排除する。そしてこのことは、以前に神の一性について論じら
 れたときに説明された。


\\

Sciendum
tamen est quod auctoritates in oppositum inductae, non probant
sufficienter propositum. Licet enim pluralitate excludatur solitudo,
et unitate deorum pluralitas, non tamen sequitur quod his nominibus
hoc solum significetur. Albedine enim excluditur nigredo, non tamen
nomine albedinis significatur sola nigredinis exclusio.

&

しかし反対異論で引用された権威たちは、この命題を十分に証明していない。すなわ
 ち、複数性によって孤独が排除されるとしても、また一性によって神の複数
 性が排除されるとしても、これらの名称によってそのことだけが表示される
 ことは帰結しないからである。たとえば白性によって黒性が排除されるが、
 白性という名称によって黒性の排除だけが意味されるわけではないように。\footnote{このパ
 ラグラフは反対異論解答に相当する。}



\end{longtable}
\newpage





\rhead{a.~4}
\begin{center}
{\Large {\bf ARTICULUS QUARTUS}}\\
{\large UTRUM HOC NOMEN {\itshape PERSONA} POSSIT ESSE COMMUNE TRIBUS PERSONIS}\\
{\footnotesize I {\itshape Sent.}, d.25, a.3; {\itshape De Pot.}, q.8,
 a.3, ad 11.}\\
{\Large 第四項\\「ペルソナ」というこの名称は三つのペルソナに共通であり
 うるか}
\end{center}

\begin{longtable}{p{21em}p{21em}}

{\scshape Ad quartum sic proceditur}. Videtur quod hoc nomen {\itshape persona} non possit
esse commune tribus personis. Nihil enim est commune tribus personis
nisi essentia. Sed hoc nomen {\itshape persona} non significat essentiam in
recto. Ergo non est commune tribus.

&

第四項の問題へ議論は以下のように進められる。
「ペルソナ」というこの名称は三つのペルソナに共通でありえないと思われる。
 理由は以下の通り。
三つのペルソナに共通なものは本質以外にはない。しかるにこの「ペルソナ」
 という名称は表の意味で(in recto)本質を表示しない。ゆえに三つに共通で
 はない。

\\



2. {\scshape Praeterea}, commune opponitur incommunicabili. Sed de ratione personae
est quod sit incommunicabilis, ut patet ex definitione Ricardi de
s. Victore supra posita. Ergo hoc nomen persona non est commune
tribus.

&

さらに、共通であることは共有されえないことに対立する。しかるにペルソナ
 の性格には、前に\footnote{STI, q.29, a.3, ad 4.}示された聖ヴィクトルのリカルドゥスの定義から明らかな
 ように、共有されえないことが属している。ゆえにこの「ペルソナ」という
 名称は三つに共通でない。


\\



3. {\scshape Praeterea}, si est commune tribus, aut ista communitas attenditur
secundum rem, aut secundum rationem. Sed non secundum rem, quia sic
tres personae essent una persona. Nec iterum secundum rationem tantum,
quia sic persona esset universale, in divinis autem non est universale
et particulare, neque genus neque species, ut supra ostensum est. Non
ergo hoc nomen persona est commune tribus.

&

さらに、もし三つに共通ならばこの共通性は事物に即して見出されるか、ある
 いは概念に即して見出されるかのどちらかである。しかるに事物に即してで
 はない。なぜなら、もしそうなら三つのペルソナは一つのペルソナだっただ
 ろうから。また概念だけに即してでもない。なぜなら、もしそうなら「ペル
 ソナ」は普遍だということになるが、前に示されたとおり\footnote{STI,
 q.3, a.5.}、神において不変
 も個別も類も種もないからである。ゆえに「ペルソナ」というこの名称は三
 つに共通ではない。

\\



{\scshape Sed contra est} quod dicit Augustinus, VII {\itshape de Trin}., quod cum
quaereretur, {\itshape Quid tres?} Responsum est, {\itshape Tres Personae}; quia commune est
eis id quod est persona.

&

しかし反対にアウグスティヌスは『三位一体論』第7巻で「三とは何か」と問
 われたとき「三つのペルソナ」と答えられたと述べている。なぜなら、ペル
 ソナであることが彼らに共通するからである。


\\



{\scshape Respondeo dicendum} quod ipse modus loquendi ostendit hoc nomen {\itshape persona}
tribus esse commune, cum dicimus {\itshape tres personas}, sicut cum dicimus {\itshape tres
homines}, ostendimus {\itshape hominem} esse commune tribus. Manifestum est autem
quod non est communitas rei, sicut una essentia communis est tribus,
quia sic sequeretur unam esse personam trium, sicut essentia est
una. 


&

解答する。以下のように言われるべきである。
ちょうど私たちが「三人の人間」と言うときに「人間」が三人に共通であるこ
 とを示すように、「三つのペルソナ」と私たちが言うとき「ペルソナ」とい
 うこの名称が三つに共通することを、語り方自体が示している。
しかし、ちょうど一つの本質が三つに共通であるというように、事物の共通性
 でないことは明らかである。なぜならもしそうならば、ちょうど本質が一つ
 であるように、一つのペルソナが三つに属することになっただろうから。


\\

Qualis autem sit communitas, investigantes diversimode locuti
sunt. Quidam enim dixerunt quod est communitas negationis; propter
hoc, quod in definitione personae ponitur {\itshape incommunicabile}. Quidam
autem dixerunt quod est communitas intentionis, eo quod in definitione
personae ponitur {\itshape individuum}; sicut si dicatur quod esse speciem est
commune equo et bovi. Sed utrumque horum excluditur per hoc, quod hoc
nomen {\itshape persona} non est nomen negationis neque intentionis, sed est
nomen rei. 


&

ではどのような共通性かということは、探求する人々がさまざまに語った。あ
 る人々はそれは否定の共通性であり、このため、ペルソナの定義の中に
 「共有されえない」ということが置かれたのだと述べた。またある人々は、
 ちょうど種であることが馬と牛に共通すると言われる場合のように、ペルソ
 ナの定義の中に「個的な」が置かれているので、それは概念の共通性だと述
 べた。しかしこのどちらも、「ペルソナ」というこの名称が、否定の名称で
 も概念の名称でもなく、事物の名称であることによって排除される。



\\


Et ideo dicendum est quod etiam in rebus humanis hoc nomen
{\itshape persona} est commune communitate rationis, non sicut genus vel species,
sed sicut {\itshape individuum vagum.} Nomina enim generum vel specierum, ut {\itshape homo}
vel {\itshape animal}, sunt imposita ad significandum ipsas naturas communes; non
autem intentiones naturarum communium, quae significantur his
nominibus {\itshape genus} vel {\itshape species}. Sed individuum vagum, ut {\itshape aliquis homo},
significat naturam communem cum determinato modo existendi qui
competit singularibus, ut scilicet sit per se subsistens distinctum ab
aliis. 
Sed in nomine {\itshape singularis designati}, significatur determinatum
distinguens, sicut in nomine Socratis haec caro et hoc os.

&

それゆえ、以下のように言われるべきである。「ペルソナ」というこの名称は人間の事柄においても概念の共通性
 によって共通だが、しかしそれは類や種としてではなく「漠然とした個」と
 してである。「人間」や「動物」のような類や種の名称は、共通する本性そ
 れ自体を表示するために付けられた。しかし「類」や「種」といった名称で
 表示される共通な本性の概念はそうでない。しかし「ある人間」のような漠
 然とした個は、他から区別されて自存するという、個に適合する限定された
 在り方とともに共通の本性を表示する。
しかし「指定された個」の名称においては「ソクラテス」「この肉」「この骨」
におけるように限定されたものを区別して表示される。

\\

 Hoc tamen
interest, quod {\itshape aliquis homo} significat naturam, vel individuum ex
parte naturae, cum modo existendi qui competit singularibus, hoc autem
nomen {\itshape persona} non est impositum ad significandum individuum ex parte
naturae, sed ad significandum rem subsistentem in tali natura. 

&

しかし次の点で異なる。
 「ある人間」は本性や個体を、本性の側から、個物に適合する存在のしかた
 を伴って表示するが、「ペルソナ」というこの名称は、本性の側から個体を
 表示するために名付けられたのではなくそのような本性において自存する事
 物を表示するために名付けられている。


\\


Hoc
autem est commune secundum rationem omnibus personis divinis, ut
unaquaeque earum subsistat in natura divina distincta ab aliis. Et sic
hoc nomen {\itshape persona}, secundum rationem, est commune tribus personis
divinis.

&

ところで、それらの各々が他から区別された神の本性において自存するという
 ことがすべての神のペルソナに概念において共通する。この意味で「ペルソ
 ナ」というこの名称は、概念において、神の三つのペルソナに共通する。

\\



{\scshape Ad primum ergo dicendum} quod ratio illa procedit de communitate rei.

&

第一異論に対してはそれゆえ以下のように言われるべきである。
この論は事物の共通性について論じている。

\\



{\scshape Ad secundum dicendum} quod, licet persona sit incommunicabilis, tamen
ipse modus existendi incommunicabiliter, potest esse pluribus
communis.

&

第二異論に対しては以下のように言われるべきである。
ペルソナが共有されえないとしても、共有されえないしかたがで存在するとい
 うその在り方は複数のものに共通しうる。


\\



{\scshape Ad tertium dicendum} quod, licet sit communitas rationis et non rei
tamen non sequitur quod in divinis sit universale et particulare, vel
genus vel species. Tum quia neque in rebus humanis communitas personae
est communitas generis vel speciei. Tum quia personae divinae habent
unum esse, genus autem et species, et quodlibet universale,
praedicatur de pluribus secundum esse differentibus.

&

第三異論に対しては以下のように言われるべきである。
概念の共通性であり事物の共通性でないことから、神において普遍と個別や類と
 種があることは帰結しない。それは一つには、人間的な事柄においてもペル
 ソナの共通性は類や種の共通性でないからであり、また一つには、神のペル
 ソナは一つの存在を持つからである。類や種、あるいはどんな普遍も、存在
 において異なる複数のものどもに述語されるのだから。




\end{longtable}
\newpage



\end{document}