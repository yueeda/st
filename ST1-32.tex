\documentclass[10pt]{jsarticle} % use larger type; default would be 10pt
%\usepackage[utf8]{inputenc} % set input encoding (not needed with XeLaTeX)
%\usepackage[round,comma,authoryear]{natbib}
%\usepackage{nruby}
\usepackage{okumacro}
\usepackage{longtable}
%\usepqckage{tablefootnote}
\usepackage[polutonikogreek,english,japanese]{babel}
%\usepackage{amsmath}
\usepackage{latexsym}
\usepackage{color}
%\usepackage{tikz}

%----- header -------
\usepackage{fancyhdr}
\pagestyle{fancy}
\lhead{{\it Summa Theologiae} I, q.~32}
%--------------------


\title{{\bf PRIMA PARS}\\{\HUGE Summae Theologiae}\\Sancti Thomae
Aquinatis\\{\sffamily QUEAESTIO TRIGESTIMASECUNDA}\\DE DIVINARUM
PERSONARUM COGNITIONE}
\author{Japanese translation\\by Yoshinori {\sc Ueeda}}
\date{Last modified \today}

%%%% コピペ用
%\rhead{a.~}
%\begin{center}
% {\Large {\bf }}\\
% {\large }\\
% {\footnotesize }\\
% {\Large \\}
%\end{center}
%
%\begin{longtable}{p{21em}p{21em}}
%
%&
%
%\\
%\end{longtable}
%\newpage



\begin{document}

\maketitle

\begin{center}
{\Large 第三十二問\\神のペルソナの認識について}
\end{center}


\begin{longtable}{p{21em}p{21em}}

{\Huge C}{\scshape onsequenter} inquirendum est de cognitione divinarum personarum. Et
 circa hoc quaeruntur quatuor. 

\begin{enumerate}
 \item utrum per rationem naturalem possint cognosci divinae personae. 
 \item utrum sint aliquae notiones divinis personis attribuendae.
 \item de numero notionum.
 \item utrum liceat diversimode circa notiones opinari.
\end{enumerate}


&

続いて神のペルソナの認識について探求されるべきである。これをめぐって四
 つのことが問われる。


\begin{enumerate}
 \item 理性的本性によって神のペルソナは認識されうるか。
 \item 神のペルソナには何らかの識標が帰せられるべきか。
 \item その識標の数について。
 \item それらの識標についてさまざまに意見することが許されるか。
\end{enumerate}

\end{longtable}



\newpage



\rhead{a.~1}
\begin{center}
{\Large {\bf ARTICULUS PRIMUS}}\\
{\large UTRUM TRINITAS DIVINARUM PERSONARUM POSSIT PER NATURALEM
 RATIONEM COGNOSCI}\\
{\footnotesize I {\itshape Sent.}, d.3, q.1, a.4; {\itshape De
 Verit.}, q.10, a.13; in Boet.~{\itshape De Trin.}, q.1, a.4;
 {\itshape ad Rom.}, cap.1, lect.6.}\\
{\Large 第一項\\神のペルソナの三性は自然本性的理性によって認識されうる
 か}
\end{center}

\begin{longtable}{p{21em}p{21em}}

{\scshape Ad primum sic proceditur}. Videtur quod Trinitas divinarum personarum
possit per naturalem rationem cognosci. Philosophi enim non devenerunt
in Dei cognitionem nisi per rationem naturalem, inveniuntur autem a
philosophis multa dicta de Trinitate personarum. Dicit enim
Aristoteles, in I {\itshape de Caelo et Mundo} : {\itshape Per hunc numerum}, scilicet
ternarium, {\itshape adhibuimus nos ipsos magnificare Deum unum, eminentem
proprietatibus eorum quae sunt creata}. 


&

第一項の問題へ議論は以下のように進められる。
神のペルソナの三性は自然本性的理性によって認識されうると思われる。理由
 は以下の通り。
哲学者たちは神の認識へ自然本性的理性だけによって到達したが、哲学者たち
 によって神のペルソナの三性について語られる多くのことが見出されている。
 たとえばアリストテレスは『天体論』第1巻で「この数」すなわち三「によっ
 て私たちは私たち自身を一つの神、創造されたものどもの固有性にまさるも
 のと崇めるようにした」と述べている。

\\


Augustinus etiam dicit, VII
{\itshape Confes}.~: {\itshape Ibi legi}, scilicet in libris Platonicorum, {\itshape non quidem his
verbis, sed hoc idem omnino, multis et multiplicibus suaderi
rationibus, quod in principio erat verbum, et verbum erat apud Deum,
et Deus erat verbum}, et huiusmodi quae ibi sequuntur, in quibus verbis
distinctio divinarum personarum traditur. 


&

さらにアウグスティヌスは『告白』第7巻で「ここで」すなわちプラトン派の
 人々の書物で「始めに言葉があった、言葉は神のもとにあっ
 た、神は言葉であったということ、そしてそこでそれに続く同様の事柄が、これらの言葉によってではないが、しかしこのまったく同じこ
 とが、多くの多様な根拠によって説得的に論じられている」と述べていて、
 それらの言葉において神のペルソナの区別が伝えられている。

\\


Dicitur etiam in Glossa
{\itshape Rom}.~{\scshape i}, et {\itshape Exod}.~{\scshape viii}, quod magi Pharaonis defecerunt in tertio signo,
idest in notitia tertiae personae, scilicet spiritus sancti, et sic ad
minus duas cognoverunt. 


&

さらに『ローマの信徒への手紙』第1章と『出エジプト記』第8章の『注解』に
 おいて、ファラオの魔術師たちは三番目のしるし、つまり第三のペルソナで
 ある聖霊の識標において誤った。つまりこのことは少なくとも二つは認識し
 ていたということになる。


\\

Trismegistus etiam dixit, {\itshape Monas genuit
monadem, et in se suum reflexit ardorem}, per quod videtur generatio
filii, et spiritus sancti processio intimari. Cognitio ergo divinarum
personarum potest per rationem naturalem haberi.

&

さらにトリスメギストスも「一なる者が一なる者を生み、そして自分自身の方
 へ燃える思いを反射させた」と述べたが、これによって息子の生成と聖霊の
 発出が表明されていると思われる。ゆえに神のペルソナの認識は自然本性的
 理性によって持たれうる。

\\



2. {\scshape Praeterea}, Ricardus de sancto Victore dicit, in libro
 {\itshape de Trin.~: Credo
sine dubio quod ad quamcumque explanationem veritatis, non modo
probabilia, imo etiam necessaria argumenta non desint}. Unde etiam ad
probandum Trinitatem personarum, aliqui induxerunt rationem ex
infinitate bonitatis divinae, quae seipsam infinite communicat in
processione divinarum personarum. 


&

さらに、聖ヴィクトルのリカルドゥスは『三位一体論』という書物の中でこう
 述べている。「私は疑いなく以下のことを信じる。真理のどの説明に対して
 もたんに蓋然的でなく必然的な議論が欠けていない」。このことからペルソナ
 の三性を証明するためにも、ある人々は神のペルソナの発出において無限に
 自らを伝達する神の無限の善性に基づいて推論を行った。

\\


Quidam vero per hoc, quod {\itshape nullius
boni sine consortio potest esse iucunda possessio}. Augustinus vero
procedit ad manifestandum Trinitatem personarum, ex processione verbi
et amoris in mente nostra, quam viam supra secuti sumus. Ergo per
rationem naturalem potest cognosci Trinitas personarum.


&


またある人々は、
 「交わりなくしては、どんな善にも楽しい所有はありえない」ということか
 ら(そう考えた)。他方アウグスティヌスは、私たちの精神における言葉と
 愛の発出に基づいてペルソナの三性を明示する事へ進んだが、私たちはその
 道に従った。ゆえに自然本性的理性によってペルソナの三性は認識されうる。


\\



3. {\scshape Praeterea}, superfluum videtur homini tradere quod humana ratione
cognosci non potest. Sed non est dicendum quod traditio divina de
cognitione Trinitatis sit superflua. Ergo Trinitas personarum ratione
humana cognosci potest.


&

さらに、人間の理性によって認識されえない事柄を人間に伝えることは無駄で
 あるように思われる。しかるに三性の認識について神が伝えることは無駄だ
 とはいわれえない。ゆえにペルソナの三性は人間理性によって認識されうる。


\\



{\scshape Sed contra est} quod Hilarius dicit, in libro II {\itshape de Trin., non putet
homo sua intelligentia generationis sacramentum posse
consequi}. Ambrosius etiam dicit, {\itshape impossibile est generationis scire
secretum, mens deficit, vox silet}. 


&

しかし反対にヒラリウスは『三位一体論』第2巻で「人間が自らの知性認識に
 よって生成の秘蹟を理解しうると思わないように」と言っている。アンブ
 ロシウスも「生成の秘密を知ることは不可能であり、精神は挫折し声は沈黙
 する」と述べている。


\\


Sed per originem generationis et
processionis distinguitur Trinitas in personis divinis, ut ex supra
dictis patet. Cum ergo illud homo non possit scire et intelligentia
consequi, ad quod ratio necessaria haberi non potest, sequitur quod
Trinitas personarum per rationem cognosci non possit.


&

しかし前に述べられたことから明らかなとおり、生成と発出の起源によって神
 のペルソナにおいて三性が区別される。ゆえに必要な理拠が持たれえない事
 柄を人間は知ることができず知性認識を獲得することはできないのだから、
 ペルソナの三性が理性によって認識されることができないということが帰結
 する。


\\



{\scshape Respondeo dicendum} quod impossibile est per rationem naturalem ad
cognitionem Trinitatis divinarum personarum pervenire. Ostensum est
enim supra quod homo per rationem naturalem in cognitionem Dei
pervenire non potest nisi ex creaturis. Creaturae autem ducunt in Dei
cognitionem, sicut effectus in causam. 


&

解答する。以下のように言われるべきである。
自然本性的理性によって神のペルソナの三性の認識に到達することは不可能で
 ある。理由は以下の通り。前に、人間が自然本性的理性によって神の認識へ
 到達するのは被造物に基づいてでしかないことが示された。しかるに被造物
 は、結果が原因へ導くようにして神の認識へ導く。




\\


Hoc igitur solum ratione
naturali de Deo cognosci potest, quod competere ei necesse est
secundum quod est omnium entium principium, et hoc fundamento usi
sumus supra in consideratione Dei. Virtus autem creativa Dei est
communis toti Trinitati, unde pertinet ad unitatem essentiae, non ad
distinctionem personarum. Per rationem igitur naturalem cognosci
possunt de Deo ea quae pertinent ad unitatem essentiae, non autem ea
quae pertinent ad distinctionem personarum. 


&

ゆえに、自然本性的理性だけによって神について認識されうるのはすべての存
 在者の根源である限りにおいて神に適合するものだけであり、この基本を私
 たちは神についての考察において前に用いた。しかし神の創造する力は三性
 全体に共通であり、したがってそれは本質の一性に属するのでありペルソナ
 の区別に属するのではない。ゆえに神について自然本性的理性によって本質
 の一性に属することは認識されうるが、ペルソナの区別に属することは認識
 されえない。


\\


Qui autem probare nititur
Trinitatem personarum naturali ratione, fidei dupliciter
derogat. Primo quidem, quantum ad dignitatem ipsius fidei, quae est ut
sit de rebus invisibilibus, quae rationem humanam excedunt. Unde
apostolus dicit, {\itshape ad Heb}.~{\scshape xi}, quod fides est de {\itshape non apparentibus}. 


&

しかし、自然本性的理性によってペルソナの三性を証明しようとする人は二通
 りのしかたで信仰を見下している。第一に信仰自体の権威にかんしてであり、
 信仰は人間理性を越える見えざるものにかんしてある。したがって使徒は
 『ヘブライ人への手紙』第11章で「信仰は見えないものにかかわる」と述べ
 ている。
 



\\


Et apostolus dicit, I {\itshape Cor}.~{\scshape ii} : {\itshape Sapientiam loquimur inter perfectos,
sapientiam vero non huius saeculi, neque principum huius saeculi; sed
loquimur Dei sapientiam in mysterio, quae abscondita est}. 

&

また使途は『コリントの信徒への手紙1』第2章で「私たちは完全なもののあい
 だで知恵を語るが、この世の知恵ではなく、この世の首長たちの知恵でもなく、むしろ
 私たちは神秘における神の知恵を語るのであり、それは隠されている」\footnote{「しかし、私たちは、成熟した人たちの間では、知恵を語ります。それはこの世の知恵ではなく、また、この世の無力な支配者たちの知恵でもありません。私たちが語るのは、隠された秘義としての神の知恵であって、神が私たちに栄光を与えるために、世界の始まる前から定めておられたものです。 」(2:6-7)}と述
 べている。

\\

Secundo,
quantum ad utilitatem trahendi alios ad fidem. Cum enim aliquis ad
probandam fidem inducit rationes quae non sunt cogentes, cedit in
irrisionem infidelium, credunt enim quod huiusmodi rationibus
innitamur, et propter eas credamus. 


&

第二に、他の人々を信仰へと導く有用性の観点からである。つまり、ある人が
 信仰を証明するために、十分にそれを証明しない論拠を導入するとき、その
 人は不信者たちの嘲りに陥る。なぜなら不信者たちは、私たちがそのような理由に依
 拠してそれらのために信じていると信じるからである。


\\


Quae igitur fidei sunt, non sunt
tentanda probare nisi per auctoritates, his qui auctoritates
suscipiunt. Apud alios vero, sufficit defendere non esse impossibile
quod praedicat fides. Unde Dionysius dicit, {\scshape ii} cap.~{\itshape de Div.~Nom}.: {\itshape Si
aliquis est qui totaliter eloquiis resistit, longe erit a nostra
philosophia; si autem ad veritatem eloquiorum}, scilicet sacrorum,
{\itshape respicit, hoc et nos canone utimur}.


&

ゆえに、信仰に属する事柄は、権威によって、その権威を保持する人々に対し
 て証明すべきであり、それ以外のしかたを試みてはならない。
他方で、他の人々のもとでは、信仰が表明することが不可能でないことを擁護
 することで十分である。したがってディオニュシウスは『神名論』第2章で次
 のように言う。「もしだれかが語られたことにまったく抵抗するならば、彼
 は私たちの哲学から遠く隔たっているだろう。しかし語られた(つまり聖な
 る)言葉の真理にかんしては、私たちもこのカノンを用いる」。


\\



{\scshape Ad primum ergo dicendum} quod philosophi non cognoverunt mysterium
Trinitatis divinarum personarum per propria, quae sunt paternitas,
filiatio et processio; secundum illud apostoli, I {\itshape ad
 Cor}.~{\scshape ii} : {\itshape Loquimur
Dei sapientiam, quam nemo principum huius saeculi cognovit}, idest
{\itshape philosophorum}, secundum Glossam. 

&

第一異論に対してはそれゆえ以下のように言われるべきである。
哲学者たちは神のペルソナの三性の神秘を固有性によって、すなわち父性、息子性、
 発出によって認識するのではない。このことは、使途の『コリントの信徒へ
 の手紙1』第2章「私たちはこの世のどの首長も認識しない神の知恵を語る」
 によるが、この「首長」とは注解によれば「哲学者」のことである。


\\


Cognoverunt tamen quaedam essentialia
attributa quae appropriantur personis, sicut potentia patri, sapientia
filio, bonitas spiritui sancto, ut infra patebit. Quod ergo
Aristoteles dicit, per hunc numerum adhibuimus nos ipsos etc., non est
sic intelligendum, quod ipse poneret ternarium numerum in divinis, sed
vult dicere quod antiqui utebantur ternario numero in sacrificiis et
orationibus, propter quandam ternarii numeri perfectionem. 


&

しかし、彼らは、たとえば後で明らかになるように、能力は父に、知恵は息子
 に、善性は聖霊にというようにペルソナに固有とされるある種の本質的な属
 性を認識した。ゆえにアリストテレスが「この数によって私たち自身を云々」
 と述べていることは、彼が神において三という数を措定したと理解されるべ
 きではなく、彼が言おうとしていることは、古代の人々は三という数を三という数のある種の完全性のため
 に犠牲や祈りにおいて用いているということである。



\\


In libris
etiam Platonicorum invenitur in principio erat verum, non secundum
quod verbum significat personam genitam in divinis, sed secundum quod
per verbum intelligitur ratio idealis, per quam Deus omnia condidit,
quae filio appropriatur. 


&

さらにプラトン派の人々の書物の中に「始めに言葉があった」が見出されるの
 は、神において生まれた言葉を意味する限りにおいてではなく、言葉が、
息子に固有とされ、それによって神が万物を作ったイデア的な理拠を意
 味するかぎりにおいてである。


\\


Et licet appropriata tribus personis
cognoscerent, dicuntur tamen in tertio signo defecisse, idest in
cognitione tertiae personae, quia a bonitate, quae spiritui sancto
appropriatur, deviaverunt, dum cognoscentes Deum, {\itshape non sicut Deum
glorificaverunt}, ut dicitur {\itshape Rom}.~{\scshape i}. 


&

また、彼らが三つのペルソナに固有化されたものを認識したとしても、しかし
彼らは第三のしるしつまり第三のペルソナの認識において失敗したと言われて
 いる。それは、聖霊に固有とされる善性から逸れたからである。『ローマの
 信徒への手紙』に言われるように「彼らは神としての栄光を与えなかった」\footnote{「なぜなら、彼らは神を知りながら、神として崇めることも感謝することもせず、かえって、空しい思いにふけり、心が鈍く暗くなったからです。 」(1:21)}
 からである。

\\


Vel, quia ponebant Platonici unum
primum ens, quod etiam dicebant esse patrem totius universitatis
rerum, consequenter ponebant aliam substantiam sub eo, quam vocabant
mentem vel paternum intellectum, in qua erant rationes omnium rerum,
sicut Macrobius recitat super somnium Scipionis, non autem ponebant
aliquam substantiam tertiam separatam, quae videretur spiritui sancto
respondere. 

&

あるいは、プラトン派の人々は一つの第一の有を措定し、またそれが全宇宙の
 事物の父であると言い、引き続いてそれに由来する他の実体を措定してそれ
 を「精神」や「父の知性」と呼んでそこに万物の理拠があるとした。たとえ
 ばマクロビウスが『スキピオの夢』で引用しているように。しかし彼らは聖
 霊に対応すると思われる第三の分離した実体を措定してはいなかった。

\\


Sic autem nos non ponimus patrem et filium, secundum
substantiam differentes, sed hoc fuit error Origenis et
Arii. Sequentium in hoc Platonicos. Quod vero Trismegistus dixit,
monas monadem genuit, et in se suum reflexit ardorem, non est
referendum ad generationem filii vel processionem spiritus sancti, sed
ad productionem mundi, nam unus Deus produxit unum mundum propter sui
ipsius amorem.


&

しかし私たちは父と息子を違う実体において措定したりはしない。むしろこれ
 はこの点においてプラトン派の人々に従ったオリゲネスとアリウスの誤謬で
 ある。他方、トリスメギストスが「一なる者が一なる者を生み、そして自分自身の方
 へ燃える思いを反射させた」と語ったことは、息子の生成や聖霊の発出に関
 連付けられるべきではなく、むしろ世界の産出に関係づけられるべきである。
 すなわち一つの神が一つの世界を自ら自身への愛のために生み出した、とい
 うように。


\\



{\scshape Ad secundum dicendum} quod ad aliquam rem dupliciter inducitur
ratio. Uno modo, ad probandum sufficienter aliquam radicem, sicut in
scientia naturali inducitur ratio sufficiens ad probandum quod motus
caeli semper sit uniformis velocitatis. 


&

第二異論に対しては以下のように言われるべきである。
ある事柄について、二通りの推論がある。一つにはある根拠を十分に証明する
 ための推論である。たとえば自然学において天の運動が常に一定の速さであ
 ることを十分に証明するための推論がおこなわれている。


\\


Alio modo inducitur ratio, non
quae sufficienter probet radicem, sed quae radici iam positae ostendat
congruere consequentes effectus, sicut in astrologia ponitur ratio
excentricorum et epicyclorum ex hoc quod, hac positione facta, possunt
salvari apparentia sensibilia circa motus caelestes, non tamen ratio
haec est sufficienter probans, quia etiam forte alia positione facta
salvari possent. 


&

もう一つは、十分
 に根拠を証明するのではなく、すでに示された根拠にそれから出てくる諸結
 果が調和することを示す推論が行われる。たとえば天文学において導円と周
 転円という根拠が措定されるが、それは、これを想定すると天体の運動にか
 んする感覚的な現れを説明することができるということに基づく。しかしこ
 の根拠はそれを十分に証明するものではない。なぜなら、別のことを想定し
 ても説明が可能だからである。

\\


Primo ergo modo potest induci ratio ad probandum Deum
esse unum, et similia. Sed secundo modo se habet ratio quae inducitur
ad manifestationem Trinitatis, quia scilicet, Trinitate posita,
congruunt huiusmodi rationes; non tamen ita quod per has rationes
sufficienter probetur Trinitas personarum. 


&

ゆえに、第一のしかたで、神が一つであることや、それと類似することを証明
 するために推論が行われる。しかし三性を示すための推論は第二のしかたで
行われている。なぜなら、三性を措定するとこれらの根拠の説明が付くからで
 ある。しかし、これらの根拠によって十分にペルソナの三性が証明されるわ
 けではない。


\\

Et hoc patet persingula. 
Bonitas enim infinita Dei manifestatur etiam in productione
creaturarum, quia infinitae virtutis est ex nihilo producere. Non enim
oportet, si infinita bonitate se communicat, quod aliquid infinitum a
Deo procedat, sed secundum modum suum recipiat divinam
bonitatem. 


&

そしてこのことは、それぞれを通して明らかである。すなわち、神の無限の善
 性は被造物の産出においても明示されるが、それは無から生み出すことは無
 限の力に属することだからである。もし無限の善性によって自らを伝達する
としても、何か無限なものが神から発出する必要はない。むしろ自らの在り方
 に応じて神の善性を受け取ることが必要である。


\\


Similiter etiam quod dicitur, quod sine consortio non
potest esse iucunda possessio alicuius boni, locum habet quando in una
persona non invenitur perfecta bonitas; unde indiget, ad plenam
iucunditatis bonitatem, bono alicuius alterius consociati
sibi. 


&

同様に、「交わりなくしては、どんな善にも楽しい所有はありえない」と言わ
 れていることは、一つのペルソナにおいて完全な善性が見出されないときに
 あてはまる。ゆえに十分な楽しさの善性のために、自らに交わる何らかの他
 の善を必要とする。


\\


Similitudo autem intellectus nostri non sufficienter probat
aliquid de Deo, propter hoc quod intellectus non univoce invenitur in
Deo et in nobis. Et inde est quod Augustinus, super Ioan., dicit quod
per fidem venitur ad cognitionem, et non e converso.


&

しかし私たちの知性の類似は、知性が神と私たちにおいて一義的に見出されな
 いことから、神について何らかのことを十分に証明することはない。それゆ
 え、アウグスティヌスは『ヨハネ伝注解』で信仰を通って認識へ至るのであ
 りその逆ではないと述べている。

\\



{\scshape Ad tertium dicendum} quod cognitio divinarum personarum fuit necessaria
nobis dupliciter. Uno modo, ad recte sentiendum de creatione
rerum. Per hoc enim quod dicimus Deum omnia fecisse verbo suo,
excluditur error ponentium Deum produxisse res ex necessitate
naturae. 


&

第三異論に対しては以下のように言われるべきである。
神のペルソナの認識は私たちにとって二つのしかたで必要であった。
一つには諸事物の創造について正しく理解するためである。
すなわち、神は自分の言葉によって万物を作ったと私たちが言うことによって、
 神が諸事物を本性の必然性から生み出したと主張する人々の誤謬が取り除か
 れる


\\


Per hoc autem quod ponimus in eo processionem amoris,
ostenditur quod Deus non propter aliquam indigentiam creaturas
produxit, neque propter aliquam aliam causam extrinsecam; sed propter
amorem suae bonitatis. 


&

さらに神の中に愛の発出を私たちが措定することによって、神が何らかの必要
 のためや何らかの外的な他の原因のために被造物を生み出したのではなく、
 自らの善性への愛のために生み出したことが示される。

\\


Unde et Moyses, postquam dixerat : {\itshape In principio
creavit Deus caelum et terram}, subdit, {\itshape dixit Deus, fiat lux}, ad
manifestationem divini verbi; et postea dixit, {\itshape vidit Deus lucem, quod
esset bona}, ad ostendendum approbationem divini amoris; et similiter
in aliis operibus. 


&

このことからモーセも、「初めに神は天と地を作った」\footnote{「初めに神は天と地を創造された。」(1:1)}と述べた後、神の言葉を明
 示するために、「神は「光あれ」と言った」\footnote{「神は言われた。「光あれ。」すると光があった。」(1:2)}と述べる。その後、「神は光を見、
 それは善かった」\footnote{「神は光を見て良しとされた。神は光と闇を分け、」(1:4)}と述べているが、これは神の愛の是認を示すためである。こ
 れは他の業においても同様である。


\\


Alio modo, et principalius, ad recte sentiendum de
salute generis humani, quae perficitur per filium incarnatum, et per
donum spiritus sancti.

&
もう一つには、そしてより重要なことに、正しく人類の救済について理解する
 ためにである。これは受肉した息子によって、また聖霊の賜物によって完成
 される。


\end{longtable}
\newpage


\rhead{a.~2}
\begin{center}
{\Large {\bf ARTICULUS SECUNDUS}}\\
{\large UTRUM SINT PONENDAE NOTIONES IN DIVINIS}\\
{\footnotesize I {\itshape Sent.}, d.33, a.2.}\\
{\Large 第二項\\神において識標が措定されるべきか}
\end{center}

\begin{longtable}{p{21em}p{21em}}

{\scshape Ad secundum sic proceditur}. Videtur quod non sint ponendae notiones in
divinis. Dicit enim Dionysius, in {\scshape i} cap.~{\itshape de Div.~Nom}., quod {\itshape non est
audendum dicere aliquid de Deo, praeter ea quae nobis ex sacris
eloquiis sunt expressa}. Sed de notionibus nulla fit mentio in eloquiis
sacrae Scripturae. Ergo non sunt ponendae notiones in divinis.


&

第二項の問題へ議論は以下のように進められる。
神において識標を措定すべきではないと思われる。理由は以下の通り。
ディオニュシウスは『神名論』第1章で次のように言っている。「神について
 私たちに聖なる言葉に基づいて表明されていること以外の何かを敢えて語ろ
 うとすべきでない」。しかるに識標については聖書の言葉の中に何の言及も
 ない。ゆえに神において識標が措定されるべきでない。




\\



2. {\scshape Praeterea}, quidquid ponitur in divinis, aut pertinet ad unitatem
essentiae, aut ad Trinitatem personarum. Sed notiones non pertinent ad
unitatem essentiae, nec ad Trinitatem personarum. De notionibus enim
neque praedicantur ea quae sunt essentiae, non enim dicimus quod
{\itshape paternitas sit sapiens} vel {\itshape creet}, neque etiam ea quae sunt personae;
non enim dicimus quod {\itshape paternitas generet} et {\itshape filiatio generetur}. Ergo
non sunt ponendae notiones in divinis.


&

さらに、神において措定されるものはなんであれ、本質の一性に属するか、ペ
 ルソナの三性に属するかのどちらかである。しかし識標は本質の一性にもペ
 ルソナの三性にも属さない。理由は以下の通り。識標について、本質に属す
 るものは述語されない。なぜなら、「父性は知恵である」とか「父性は創造
 する」と私たちは言わないからである。またペルソナに属するものも述語さ
 れない。なぜなら「父性は生む」とか「息子性が生み出される」とも私たち
 は言わないからである。ゆえに識標が神において措定されるべきでない。


\\



3. {\scshape Praeterea}, in simplicibus non sunt ponenda aliqua abstracta, quae sint
principia cognoscendi, quia cognoscuntur seipsis. Sed divinae personae
sunt simplicissimae. Ergo non sunt ponendae in divinis personis
notiones.


&

さらに、単純なものにおいて、何か抽象的なものが措定されるべきではない。
 抽象的なものは、それ自身によって認識されるので、認識の根源だからであ
 る。しかし神のペルソナは最も単純である。ゆえに神のペルソナにおいて識
 標が措定されるべきではない。


\\



{\scshape Sed contra est} quod dicit Ioannes Damascenus: {\itshape Differentiam
hypostaseon}, idest personarum, {\itshape in tribus proprietatibus, idest
paternali et filiali et processionali, recognoscimus}. Sunt ergo
ponendae proprietates et notiones in divinis.


&

しかし反対に、ヨハネス・ダマスケヌスは次のように言っている。「ヒュポス
 タシスの(つまりペルソナの)違いを、私たちは三つの固有性、すなわち父的、息子的、
 発出的な固有性において認識する」。ゆえに神において固有性や識標が措定
 されるべきである。


\\



{\scshape Respondeo dicendum} quod Praepositivus, attendens simplicitatem
personarum, dixit non esse ponendas proprietates et notiones in
divinis, et sicubi inveniantur, exponit abstractum pro concreto, sicut
enim consuevimus dicere: {\itshape Rogo benignitatem tuam, idest te benignum},
ita cum dicitur in divinis {\itshape paternitas}, intelligitur {\itshape Deus pater}. 

&

解答する。以下のように言われるべきである。
プラエポシティーウスはペルソナの単純性に注意して、神の中に固有性や識標
 が措定されるべきではないと言い、それが見出されるところではどこでも、
 抽象的なものを具体的なものとして説明した。それはちょうど、「私はあな
 たが親切であることを求める」ということを「私はあなたの親切を求める」
 と言う習わしであるように、神において「父性」が語られるときには「父で
 ある神」が理解される。


\\

Sed,
sicut ostensum est supra, divinae simplicitati non praeiudicat quod in
divinis utamur nominibus concretis et abstractis. Quia secundum quod
intelligimus, sic nominamus. Intellectus autem noster non potest
pertingere ad ipsam simplicitatem divinam, secundum quod in se est
consideranda, et ideo secundum modum suum divina apprehendit et
nominat, idest secundum quod invenitur in rebus sensibilibus, a quibus
cognitionem accipit. 

&

しかし、前に示されたとおり\footnote{Q.3, a.3, ad1; Q.13, a.1, ad1.}、
 神の単純性にとって、神において具体的名称や抽象的名称を私たちが用いる
 ことを先行して判断しない。なぜなら、私たちは知性認識するように名付け
 るからである。しかし私たちの知性はそれ自体において考察されるかぎりで
 の神の単純性自体へ到達することはできず、それゆえ、自らのあり方に従っ
 て、すなわちそこから認識を受け取る可感的事物において見出されるものに
 従って神の事柄をとらえ、名付ける。

\\

In quibus, ad significandum simplices formas,
nominibus abstractis utimur, ad significandum vero res subsistentes,
utimur nominibus concretis. Unde et divina, sicut supra dictum est,
ratione simplicitatis, per nomina abstracta significamus, ratione vero
subsistentiae et complementi, per nomina concreta. 

&

それらにおいて、私たちは単純な形相を表示するために抽象的な名称を用い、自存
 する事物を表示するために具体的な名称を用いる。したがって、前に述べら
 れたとおり、私たちは神の事柄をその単純性のゆえに抽象的な名称で表示し、
 自存性と完全性のゆえに、具体的な名称で表示する。



\\


Oportet autem non
solum nomina essentialia in abstracto et in concreto significare, ut
cum dicimus {\itshape deitatem} et {\itshape Deum}, vel {\itshape sapientiam} et {\itshape sapientem}; sed etiam
personalia, ut dicamus {\itshape paternitatem} et {\itshape patrem}. Ad quod duo praecipue
nos cogunt. 


&

しかし、私たちが「神性」と「神」あるいは「知恵」と「知者」と言う場合の
 ように、本質的な名称を抽象的にまた具体的に表示しなければならないだけで
 なく、ペルソナ的な名称も、たとえば「父性」と「父」と言う場合のように、
 抽象的にまた具体的に表示しなければならない。そしてこのことについて、
 二つのことがとくに私たちに制約を与えている。

\\

Primo quidem, haereticorum instantia. Cum enim confiteamur
patrem et filium et spiritum sanctum esse unum Deum et tres personas,
quaerentibus quo sunt unus Deus, et quo sunt tres personae, sicut
respondetur quod sunt essentia vel deitate unum, ita oportuit esse
aliqua nomina abstracta, quibus responderi possit personas
distingui. Et huiusmodi sunt proprietates vel notiones in abstracto
significatae, ut paternitas et filiatio. Et ideo essentia significatur
in divinis ut quid, persona vero ut quis, proprietas autem ut
quo. 

&

第一に、異端者たちの事例である。すなわち、私たちは父と息子と聖霊とが一
 つの神であり三つのペルソナであると告白するので、「何によって神は一つ
 か」「何によってペルソナは三か」と問う人々に対して、たとえば本質また
 は神性によって一であると答えられるように、それによってペルソナが区別
 されることが答えられうる抽象的な名称が存在しなければならなかった。
そしてこのようなものが、父性や息子性のような、抽象的に表示された固有性
 や識標であある。ゆえに、神において本質は「何」として、ペルソナは「だ
 れ」として、そして固有性は「なにによって」として表示される。




\\

Secundo, quia una persona invenitur in divinis referri ad duas
personas, scilicet persona patris ad personam filii et personam
spiritus sancti. Non autem una relatione, quia sic sequeretur quod
etiam filius et spiritus sanctus una et eadem relatione referrentur ad
patrem; et sic, cum sola relatio in divinis multiplicet Trinitatem,
sequeretur quod filius et spiritus sanctus non essent duae
personae. 


&

第二に、一つのペルソナが神において二つのペルソナへ関係づけられるのが見
 出される。すなわち、父のペルソナは息子のペルソナと聖霊のペルソナへ関
 係づけられる。しかしそれは一つの関係によってではない。なぜなら、もし
 一つの関係によってならば息子と聖霊も、一つで同一の関係によって父へ関
 係づけられることになるが、神においては関係だけが三性を作り出すので、
 息子と聖霊は二つのペルソナでないことになるからである。


\\


Neque potest dici, ut Praepositivus dicebat, quod sicut Deus
uno modo se habet ad creaturas, cum tamen creaturae diversimode se
habeant ad ipsum, sic pater una relatione refertur ad filium et ad
spiritum sanctum, cum tamen illi duo duabus relationibus referantur ad
patrem. 


&

また、プラエポシティーウスが言ったように以下のように言われることもでき
 ない。すなわち、神が一つのしかたで被造物に関
 係するが、被造物はさまざまなしかたで神に関係するように、父は一つの関
 係によって息子と聖霊に関係するが、この二つは二つの関係によって父に関
 係する、と。

\\

Quia cum ratio specifica relativi consistat in hoc quod ad
aliud se habet, necesse est dicere quod duae relationes non sunt
diversae secundum speciem, si ex opposito una relatio eis
correspondeat, oportet enim aliam speciem relationis esse domini et
patris, secundum diversitatem filiationis et servitutis. 

&

なぜなら、関係の種を分ける根拠は、他のものへ関係するということにあるの
で、
もし対立するものから一つの関係がそれらに対応するならば、
二つの関係は種において異なるのでないと言うことが必然である。
なぜなら、
息子性と奴隷性の異なりに対応して、種が異なる関係が主と父に属しなければなら
ないからである。


\\


Omnes autem
creaturae sub una specie relationis referuntur ad Deum, ut sunt
creaturae ipsius, filius autem et spiritus sanctus non secundum
relationes unius rationis referuntur ad patrem, unde non est
simile. 

&

しかしすべての被造物は、神の被造物として一つの種の関係によって神に関係
 するが、息子と聖霊は一つの性格に属する関係において父に関係
 づけられるのではないので、同じではない。

\\

Et iterum, in Deo non requiritur relatio realis ad creaturam,
ut supra dictum est, relationes autem rationis in Deo multiplicare non
est inconveniens. 

&

さらに、前に述べられた
 とおり\footnote{Q.28, a.1, ad3.}神において、被造物への実在的な関係は
 必要とされないが、概念的関係が神において複数生じることは不都合ではな
 い。

\\

Sed in patre oportet esse relationem realem qua
refertur ad filium et spiritum sanctum, unde secundum duas relationes
filii et spiritus sancti quibus referuntur ad patrem, oportet
intelligi duas relationes in patre, quibus referatur ad filium et
spiritum sanctum. Unde, cum non sit nisi una patris persona, necesse
fuit seorsum significari relationes in abstracto, quae dicuntur
{\itshape proprietates} et {\itshape notiones}.


&

しかし父において、それによって息子と聖霊へ関係づけられる実在的な
 関係がなければらなない。したがって、息子と聖霊が父へ関係する二つの関
 係において、父の中に、息子と聖霊へ関係する二つの関係が理解されなければ
 ならない。したがって、父には一つのペルソナしかないので、諸関係が抽象
 的に別々に表示される必要があったのであり、それが「固有性」や「識標」と言われ
 ている。



\\



{\scshape Ad primum ergo dicendum} quod, licet de notionibus non fiat mentio in
sacra Scriptura, fit tamen mentio de personis, in quibus intelliguntur
notiones, sicut abstractum in concreto.


&

第一異論に対してはそれゆえ以下のように言われるべきである。
聖書の中に識標についての言及はないがペルソナについての言及はあり、ちょ
 うど抽象的なものが具体的なものにおいて理解されるようにして、それ
 において識標が理解される。


\\



{\scshape Ad secundum dicendum} quod notiones significantur in divinis, non ut
res, sed ut rationes quaedam quibus cognoscuntur personae; licet ipsae
notiones vel relationes realiter sint in Deo, ut supra dictum est. Et
ideo ea quae habent ordinem aliquem ad actum aliquem essentialem vel
personalem, non possunt dici de notionibus, quia hoc repugnat modo
significandi ipsarum. 

&

第二異論に対しては以下のように言われるべきである。
識標は神において事物としてではなくそれによってペルソナが認識されるある
 種の根拠として表示されている。前に述べられたとおりそれらの識標や関係が実在的に神の中にあ
 るのではあるが。ゆえに、それがある本質的なあるいはペルソナ的な作用へ
 の何らかの秩序をもつものは、識標について語られない。なぜならそういう
 ことはそれらの表示様態に反するからである。

\\

Unde non possumus dicere quod {\itshape paternitas generet}
vel {\itshape creet}, {\itshape sit sapiens} vel {\itshape intelligens}. Essentialia vero quae non
habent ordinem ad aliquem actum, sed removent conditiones creaturae a
Deo possunt praedicari de notionibus, possumus enim dicere quod
{\itshape paternitas est aeterna} vel {\itshape immensa}, vel quodcumque huiusmodi. Et
similiter, propter identitatem rei, possunt substantiva personalia et
essentialia praedicari de notionibus, possumus enim dicere quod
{\itshape paternitas est Deus}, et {\itshape paternitas est Pater}.


&

したがって、私たちは「父性が生む」や「父性が創造する」「父性が知者である」「父
 性が知性認識するものである」とは言えない。他方、何らかの作用への秩序
 をもたず、神から被造物の条件を取り除く本質的なものは、識標について述
 語付けられる。たとえば私たちは「父性は永遠である」や「父性は無尽蔵で
 ある」やそのようなことを言うことができる。同様に、事物の同一性のため
 に、ペルソナ的な実体を表す名詞や本質的な実体を表す名詞が識標に述語付
 けられうる。たとえば私たちは、「父性は神である」や「父性は父である」
 と言うことができる。


\\



{\scshape Ad tertium dicendum} quod, licet personae sint simplices, tamen absque
praeiudicio simplicitatis possunt propriae rationes personarum in
abstracto significari, ut dictum est.


&

第三異論に対しては以下のように言われるべきである。
ペルソナは単純だが、すでに述べられたとおり、ペルソナの固有な性格は、単
 純性という予断なしに、抽象的に表示されうる。



\end{longtable}
\newpage


\rhead{a.~3}
\begin{center}
{\Large {\bf ARTICULUS TERTIUS}}\\
{\large UTRUM SINT QUINQUE NOTIONES}\\
{\footnotesize I {\itshape Sent.}, d.26, q.2, a.3; d.28, q.1, a.1;
 {\itshape De Pot.}, q.9, a.9, ad 21, 27; q.10, a.5, ad 12; {\itshape
 Compend.~Theol.}, cap.62, sqq.}\\
{\Large 第三項\\五つの識標があるか}
\end{center}

\begin{longtable}{p{21em}p{21em}}

{\scshape Ad tertium sic proceditur}. Videtur quod non sint quinque
notiones. Propriae enim notiones personarum sunt relationes quibus
distinguuntur. Sed relationes in divinis non sunt nisi quatuor, ut
supra dictum est. Ergo et notiones sunt tantum quatuor.

&

第三項の問題へ議論は以下のように進められる。
五つの識標があるのではないと思われる。理由は以下の通り。
ペルソナの固有の識標は、それらによって区別される関係である。
しかるに前に述べられたとおり\footnote{Q.28, a.4.}神において関係は四つ
しかない。ゆえに識標も四つだけである。

\\



2. {\scshape Praeterea}, propter hoc quod in divinis est una essentia, dicitur Deus
unus, propter hoc autem quod sunt tres personae, dicitur Deus
trinus. Si ergo in divinis sunt quinque notiones, dicetur quinus, quod
est inconveniens.

&

さらに、神において一つの本質があるために神は一人と言われ、三つのペルソ
ナがあるために神は三人と言われる。ゆえに神の中に五つの識標があるならば、
神は五人と言われるであろう。これは不適切である。


\\



3. {\scshape Praeterea}, si, tribus personis existentibus in divinis, sunt quinque
notiones, oportet quod in aliqua personarum sint aliquae notiones duae
vel plures; sicut in persona patris ponitur innascibilitas et
paternitas et communis spiratio. Aut igitur istae tres notiones
differunt re, aut non. Si differunt re, sequitur quod persona patris
sit composita ex pluribus rebus. Si autem differunt ratione tantum,
sequitur quod una earum possit de alia praedicari, ut dicamus quod,
sicut bonitas divina est eius sapientia propter indifferentiam rei,
ita communis spiratio sit paternitas, quod non conceditur. Igitur non
sunt quinque notiones.

&

さらに、もし神の中に三つのペルソナが存在しながら五つの識標が存在するな
らば、ペルソナのうちのどれかにおいて何らかの二つあるいは複数の識標があ
るのでなければならない。ちょうど父のペルソナの中に不生性と父性と共通霊
発が措定されるように。そして、これら三つの識標は実在的に異なるかそうで
ないかのどちらかである。もし実在的に異なるならば父のペルソナは複数の事
物から複合されていることになる。またもしたんに概念的に異なるならば、そ
れらのうちのひとつが他のものについて述語されうることになる。ちょうど事
物の差異がないために、私たちが「神の善性は神の知恵である」と言うように。
同じように共通霊発は父性であると言えることになるが、これは認められない。
ゆえに五つの識標があるのではない。

\\



{\scshape Sed contra}, videtur quod sint plures. Quia sicut pater a nullo est, et
secundum hoc accipitur notio quae dicitur innascibilitas, ita a
spiritu sancto non est alia persona. Et secundum hoc oportebit
accipere sextam notionem.

&

しかし反対に(五つよりも)多く存在すると思われる\footnote{この反対異論
の結論はトマスの主張と一致しない。}。なぜなら、父は何かによって存在する
のではないので、これに従って不生性と呼ばれる識標が理解されるが、同じように
聖霊によって他のペルソナがあるのではない。するとそれに従って第六の識標
を理解するべきである。


\\



{\scshape Praeterea}\footnote{第二反対異論。}, sicut patri et filio commune est quod ab eis procedat
spiritus sanctus, ita commune est filio et spiritui sancto quod
procedant a patre. Ergo, sicut una notio ponitur communis patri et
filio, ita debet poni una notio communis filio et spiritui sancto.

&

さらに、ちょうど聖霊が発出することが父と息子に共通であるように、息子と
聖霊には父から発出することが共通する。ゆえに、一つの識標が父と息子に共
通に措定されるように、息子と聖霊に共通の識標が措定されるべきである。


\\



{\scshape Respondeo dicendum} quod notio dicitur id quod est propria ratio
cognoscendi divinam personam. Divinae autem personae multiplicantur
secundum originem. 


&

解答する。以下のように言われるべきである。
識標とは神のペルソナを認識する固有の理拠である。
しかるに神のペルソナは起源にしたがって多数化される。

\\


Ad originem autem pertinet {\itshape a quo alius}, et {\itshape qui ab
alio}, et secundum hos duos modos potest innotescere persona. Igitur
persona patris non potest innotescere per hoc quod sit ab alio, sed
per hoc quod a nullo est, et sic ex hac parte eius notio est
{\itshape innascibilitas}. 


&

ところで、起源には、「他のものがそれからある」ということと、「それが他の
ものからある」ということが属し、この二つのしかたでペルソナを識標するこ
とができる。それゆえ、父のペルソナは「それが他のものからある」によって
 識表されずむしろ「何ものからでもない」によって識表される。このように
 して、彼のこの側面からの識標は「不生性」である。

\\



Sed inquantum aliquis est ab eo, innotescit
dupliciter. Quia inquantum filius est ab eo, innotescit notione
{\itshape paternitatis}, inquantum autem spiritus sanctus est ab eo, innotescit
notione {\itshape communis spirationis}. 

&

しかしある者が他の者からであるかぎりにおいて、二つのしかたで識標する。
なぜなら、息子がそれからであるかぎりにおいて、父性という識標によって識
表するが、聖霊がそれからであるかぎりにおいては、共通霊発という識標によっ
て識表するからである。

\\


Filius autem potest innotescere per hoc
quod est ab alio nascendo, et sic innotescit per {\itshape filiationem}. Et per
hoc quod est alius ab eo, scilicet spiritus sanctus, et per hoc
innotescit eodem modo sicut et pater, scilicet {\itshape communi
spiratione}. 



&

他方、息子は他のものから生まれることによって識表することができるが、そ
のようにして、息子性によって識表する。そして、彼から他の者、すなわち聖
霊があるということによって、そして父と同じしかたで、つまり共通霊発によっ
て識表することによっても(識表する)。


\\



Spiritus sanctus autem innotescere potest per hoc quod est
ab alio vel ab aliis, et sic innotescit {\itshape processione}. Non autem per hoc
quod alius sit ab eo, quia nulla divina persona procedit ab eo. Sunt
igitur quinque notiones in divinis, scilicet {\itshape innascibilitas,
paternitas, filiatio, communis spiratio et processio}. 


&

さらに聖霊は他の者から、あるいは他の者たちからということによって識表す
ることができ、そのようにして発出によって識表する。しかし、彼から他の者
があるということによって識表しない。なぜなら、聖霊からはどの神のペルソ
ナも発出しないからである。ゆえに五つの識標が神の中にある。すなわち、不
生性、父性、息子性、共通霊発、発出である。

\\



Harum autem
tantum quatuor sunt relationes, nam innascibilitas non est relatio
nisi per reductionem, ut infra dicetur. Quatuor autem tantum
{\itshape proprietates} sunt, nam communis spiratio non est proprietas, quia
convenit duabus personis. 


&

しかしこれらのうちで四つだけが関係である。すなわち不生性は後で明らかに
なるように、還元によらないかぎり関係ではない。さらに四つだけが固有性で
ある。すなわち、共通霊発は二つのペルソナに適合するので固有性ではない。

\\


Tres autem sunt notiones personales, idest
constituentes personas, scilicet paternitas, filiatio et processio,
nam communis spiratio et innascibilitas dicuntur notiones personarum,
non autem personales, ut infra magis patebit.

&

また、父性、息子性、発出の三つがペルソナ的識標、すなわちペルソナを構成
する識標であり、共通霊発と不生性はペルソナの識標ではあるがペルソナ的識
標ではない。これもあとでさらに明らかになる。



\\




{\scshape Ad primum ergo dicendum} quod praeter quatuor relationes oportet ponere
aliam notionem, ut dictum est.

&

第一異論に対してはそれゆえ以下のように言われるべきである。
すでに述べられたとおり、四つの関係の他に別の識標を措定しなければならない。

\\



{\scshape Ad secundum dicendum} quod essentia in divinis significatur ut res
quaedam; et similiter personae significantur ut res quaedam sed
notiones significantur ut rationes notificantes personas. Et ideo,
licet dicatur Deus unus propter unitatem essentiae, et trinus propter
Trinitatem personarum; non tamen dicitur quinus propter quinque
notiones.

&

第二異論に対しては以下のように言われるべきである。
本質は神においてある種の事物として表示される。同様にペルソナもある種の
事物として表示される。しかし識標はペルソナを識表する理拠として表示され
る。ゆえに本質の一性のために神は一人であると言われ、ペルソナの三性のゆ
えに三人と言われるが、五つの識標のゆえに五人とは言われない。

\\



{\scshape Ad tertium dicendum} quod, cum sola oppositio relativa faciat
pluralitatem realem in divinis, plures proprietates unius personae,
cum non opponantur ad invicem relative, non differunt realiter. Nec
tamen de invicem praedicantur, quia significantur ut diversae rationes
personarum. Sicut etiam non dicimus quod attributum potentiae sit
attributum scientiae, licet dicamus quod scientia sit potentia.

&

第三異論に対しては以下のように言われるべきである。神においては関係的な
対立だけが実在的な複数性を生むので、一つのペルソナに属する複数の固有性
は、相互に関係的に対立しないので、実在的に違わない。しかし相互に述語さ
れることもない。なぜなら、ペルソナの異なる性格として表示されるからであ
る。それはちょうど私たちが、知は能力だと言うにしても、能力の属性は知の
属性だと言わないようなものである。


\\



{\scshape Ad quartum dicendum} quod, cum persona importet dignitatem, ut supra
dictum est, non potest accipi notio aliqua spiritus sancti ex hoc quod
nulla persona est ab ipso. Hoc enim non pertinet ad dignitatem ipsius;
sicut pertinet ad auctoritatem patris quod sit a nullo.

&

第四異論\footnote{反対異論のこと。}対しては、以下のように言われるべきである。
前に述べられたとおりペルソナは品格を含意するので、どのペルソナもそれか
らあるのではないことに基づいて聖霊の何らかの識標が理解されることはあり
えない。このことは聖霊の品格に属するのではないからであり、これは何者か
らでもないということが父の権威に属するのとは異なる。

\\



{\scshape Ad quintum dicendum} quod filius et spiritus sanctus non conveniunt in
uno speciali modo existendi a patre; sicut pater et filius conveniunt
in uno speciali modo producendi spiritum sanctum. Id autem quod est
principium innotescendi, oportet esse aliquid speciale. Et ideo non
est simile.

&

第五異論\footnote{第二反対異論のこと。}に対しては、以下のように言われるべきである。
息子と聖霊は神から存在する一つの特殊なしかたにおいて一致することはない。
これは父と息子が聖霊を生むという一つの特殊なしかたで一致するのとは異な
る。他方、不生の根源であるところのものは、何か特殊なものでなければなら
ない。それゆえ同じではない。
\end{longtable}
\newpage



\rhead{a.~4}
\begin{center}
{\Large {\bf ARTICULUS QUARTUS}}\\
{\large UTRUM LICEAT CONTRARIE OPINARI DE NOTIONIBUS}\\
{\footnotesize I {\itshape Sent.}, d.33, a.5.}\\
{\Large 第四項\\識標について反対して考えることが許されるか}
\end{center}

\begin{longtable}{p{21em}p{21em}}

{\scshape Ad quartum sic proceditur}. Videtur quod non liceat contrarie opinari
de notionibus. Dicit enim Augustinus, in I {\itshape de Trin}., quod {\itshape non erratur
alicubi periculosius} quam in materia Trinitatis, ad quam certum est
notiones pertinere. Sed contrariae opiniones non possunt esse absque
errore. Ergo contrarie opinari circa notiones non licet.

&

第四項の問題へ議論は以下のように進められる。
識標について反対して考えることは許されないと思われる。理由は以下の通り。
アウグスティヌスは『三位一体論』第1巻で、三性の事柄以上に「誤るのが危
 険なところはない」と言っているが、識標はたしかに三性についての事柄に
 属する。しかるに反対する意見は誤りなしにはありえない。ゆえに識標につ
 いて反対して考えることは許されない。

\\



2. {\scshape Praeterea}, per notiones cognoscuntur personae, ut dictum est. Sed
circa personas non licet contrarie opinari. Ergo nec circa notiones.

&

さらに、すでに述べられたとおり、識標によってペルソナが認識される。しか
 るにペルソナをめぐって反対して考えることは許されない。ゆえに識標をめ
 ぐっても許されない。

\\



{\scshape Sed contra}, articuli fidei non sunt de notionibus. Ergo circa notiones
licet sic vel aliter opinari.

&

しかし反対に、識標についての信仰箇条はない。ゆえに識標についてこのよう
 に、または別様に考えることが許される。

\\



{\scshape Respondeo dicendum} quod ad fidem pertinet aliquid dupliciter. Uno
modo, directe; sicut ea quae nobis sunt principaliter divinitus
tradita, ut Deum esse trinum et unum, filium Dei esse incarnatum, et
huiusmodi. Et circa haec opinari falsum, hoc ipso inducit haeresim,
maxime si pertinacia adiungatur. 


&

解答する。以下のように言われるべきである。
ある事柄が信仰に属するのに二通りのしかたがある。
一つには直接的に属するのであり、たとえば神が三でありかつ一であることや
 神の息子が受肉したことなどのように、私たちに主要に神から伝えられた
 事柄である。そしてこのような事柄をめぐって誤ったことを考えることは、
 それ自身によって異端が生じ、もし頑固さに結びついていれば最大限に異端
 である。

\\



Indirecte vero ad fidem pertinent ea
ex quibus consequitur aliquid contrarium fidei; sicut si quis diceret
Samuelem non fuisse filium Elcanae; ex hoc enim sequitur Scripturam
divinam esse falsam. Circa huiusmodi ergo absque periculo haeresis
aliquis falsum potest opinari, antequam consideretur, vel determinatum
sit, quod ex hoc sequitur aliquid contrarium fidei, et maxime si non
pertinaciter adhaereat. 



&

他方、間接的に信仰に属することとは、それらから信仰に反する何かが帰結するよ
 うな事柄である。たとえば誰かがサミュエルはエルカナの息子でなかったと
 言ったとすれば、このことから神の聖書が誤っていることが帰結する。ゆえ
 にこのような事柄については、なんらの異端の危険しにある人が偽を考えう
 る。ただしそれは、そこから信仰に反することが帰結することが考察され決
 定される以前のことであって、そして、頑固にそう考えられていない場合は
 最大限にそう認められる。

\\



Sed postquam manifestum est, et praecipue si
sit per Ecclesiam determinatum, quod ex hoc sequitur aliquid
contrarium fidei, in hoc errare non esset absque haeresi. Et propter
hoc, multa nunc reputantur haeretica, quae prius non reputabantur,
propter hoc quod nunc est magis manifestum quid ex eis sequatur. 



&

しかしそのことから信仰に反することが帰結することが明らかにされた後は、とくに
 教会によって決定されるならば、そのことにおいて誤ることは異端なしでは
 ない。このため、以前には痛んだと知られていなかった多くの事柄が、今や
 異端であることが知られている。それは、現在、それらから何が帰結するか
 ということがより明らかになっているからである。


\\


Sic
igitur dicendum est quod circa notiones aliqui absque periculo
haeresis contrarie sunt opinati, non intendentes sustinere aliquid
contrarium fidei. Sed si quis falsum opinaretur circa notiones,
considerans quod ex hoc sequatur aliquid contrarium fidei, in haeresim
laberetur.

&

それゆえ以上の理由で、識標をめぐってある人々が異端の危険なしに矛盾する
ことを考えているが、それは信仰の何らかの矛盾を保持することを意図しているわけ
 ではない。しかしもし誰かが、そこから信仰に反する何かが帰結することを考えてい
 ながらその誤ったことを識標について考えるならば、彼は異端に陥るであろ
 う。


\\



Et per hoc patet responsio ad obiecta.

&

以上のことから異論に対する解答は明らかである。

\end{longtable}
\end{document}