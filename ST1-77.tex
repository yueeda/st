\documentclass[paper=a4paper,fontsize=10pt,jafontsize=9pt,titlepage]{jlreq}
\usepackage{pxrubrica} %ルビ、傍点
\usepackage{longtable}
\usepackage[polutonikogreek,english,japanese]{babel}
\usepackage{latexsym}
\usepackage{color}
\usepackage{url}

%----- header -------
\usepackage{fancyhdr}
\pagestyle{fancy}
\lhead{{\itshape Summa Theologiae} I, q.77}
%--------------------


\bibliographystyle{jplain}


\title{{\bfseries Prima Pars}\\{\Huge Summae Theologiae}\\Sancti Thomae
Aquinatis\\{\sffamily Quaestio Septuagesimasexta}\\{\bfseries DE HIS QUAE PERTINENT AD POTENTIAS ANIMAE IN GENERALI}}
\author{Japanese translation\\by Yoshinori {\scshape Ueeda}}
\date{Last modified \today}

%%%% コピペ用
%\rhead{a.~}
%\begin{center}
% {\Large {\bfseries }}\\
% {\large }\\
% {\footnotesize }\\
% {\Large \\}
%\end{center}
%
%\begin{longtable}{p{21em}p{21em}}
%
%&
%
% 
%
%\\
%\end{longtable}
%\newpage

\begin{document}

\maketitle

\begin{center}
 {\Large 第七十七問\\魂の能力一般に属する事柄について}
\end{center}

\begin{longtable}{p{21em}p{21em}}
 Deinde considerandum est de his quae pertinent ad potentias animae. Et primo, in generali; secundo, in speciali. Circa primum quaeruntur octo.

 \begin{enumerate}
  \item utrum essentia animae sit eius potentia.
  \item utrum sit una tantum potentia animae, vel plures.
  \item quomodo potentiae animae distinguantur.
  \item de ordine ipsarum ad invicem.
  \item utrum anima sit subiectum omnium potentiarum.
  \item utrum potentiae fluant ab essentia animae.
  \item utrum potentia una oriatur ex alia.
  \item utrum omnes potentiae animae remaneant in ea post mortem.
 \end{enumerate}

 &

次に、魂の能力に属する事柄について、考察されるべきである。第一に、一般的に、そして第二に、特殊的に。第一のことをめぐって、八つのことが問われる。
 
 \begin{enumerate}
  \item 魂の本質は魂の能力か。
  \item 魂の能力は一つだけか、それとも複数か。
  \item 魂の能力はどのように区別されるか。
  \item それらの相互の秩序について。
  \item 魂はすべての能力の基体か。
  \item 能力は魂の本質から流れ出るか。
  \item ある能力が別の能力から出てくるか。
  \item 死後、魂のすべての能力が残存するか。
 \end{enumerate}

\end{longtable}
\newpage 
 

\rhead{a.~1}
\begin{center}
{\Large {\bfseries ARTICULUS PRIMUS}}\\
{\large UTRUM IPSA ESSENTIA ANIMAE SIT EIUS POTENTIA}\\
{\footnotesize Supra, q.54, a.3; I {\itshape Sent.}, d.3, a.4, a.2; {\itshape De Spirit.~Creat.}, a.11; {\itshape Quodl.} X, q.3, a.1; Qu.~{\itshape de Anima}, a.12.}\\
{\Large 第一項\\魂の本質それ自体がその能力か}
\end{center}

\begin{longtable}{p{21em}p{21em}}

{\scshape Ad primum sic proceditur}. Videtur quod ipsa essentia animae
sit eius potentia. Dicit enim Augustinus, in IX {\itshape de Trin}.,
quod {\itshape mens, notitia et amor sunt substantialiter in anima,
vel, ut idem dicam, essentialiter}. --- Et in X dicit quod {\itshape
memoria, intelligentia et voluntas sunt una vita, una mens, una
essentia}.

&

 第一項の問題へ、議論は以下のように進められる。魂の本質それ自体がその
 能力だと思われる。理由は以下の通り。アウグスティヌスは『三位一体論』
 第9巻で「精神と知と愛は実体的に、あるいは同じことだが、本質的に魂の中
 にある」と言う。また第10巻では「記憶と知性と意志は一つの生命であり、
 一つの精神であり、一つの本質である」と言っている。


\\

2. {\scshape Praeterea}, anima est nobilior quam materia prima. Sed
materia prima est sua potentia. Ergo multo magis anima.

&

 さらに、魂は第一質料より高貴である。しかるに第一質料はその能力(可能
 態)である。ゆえに、ましてや魂もまたそうである。

\\


3. {\scshape Praeterea}, forma substantialis est simplicior quam
accidentalis, cuius signum est, quod forma substantialis non
intenditur vel remittitur, sed in indivisibili consistit. Forma autem
accidentalis est ipsa sua virtus. Ergo multo magis forma
substantialis, quae est anima.

&

さらに、実体形相は附帯形相よりも単純である。そのしるしに、実体形相は強
められたり弱められたりせず、個体において存立している。しかし附帯形相は、
自らのちからそれ自体である。ゆえに、ましてや魂であるところの実体形相も
またそうである。

\\

4. {\scshape Praeterea}, potentia sensitiva est qua sentimus, et
potentia intellectiva qua intelligimus. Sed id quo primo sentimus et
intelligimus est anima, secundum philosophum, in II {\itshape de
Anima}. Ergo anima est suae potentiae.

&

さらに、感覚能力は、私たちがそれによって感覚するところのものであり、知
性能力は、私たちがそれによって知性認識するところのものである。しかるに、
『デ・アニマ』第2巻の哲学者によれば、それによって第一に感覚し知性認識
するところのものは魂である。ゆえに、魂は、自らの能力である。

\\


5. {\scshape Praeterea}, omne quod non est de essentia rei, est
accidens. Si ergo potentia animae est praeter essentiam eius, sequitur
quod sit accidens. Quod est contra Augustinum, in IX {\itshape de
Trin}., ubi dicit quod praedicta {\itshape non sunt in anima sicut in
subiecto, ut color aut figura in corpore, aut ulla alia qualitas aut
quantitas, quidquid enim tale est, non excedit subiectum in quo est;
mens autem potest etiam alia amare et cognoscere}.

&

さらに、すべて事物の本質に属さないものは、附帯性である。ゆえに、もし魂
の能力が魂の本質の外にあるならば、それは附帯性であることが帰結する。こ
れはアウグスティヌスが『三位一体論』第9巻で述べていることに反する。す
なわち彼はそこで以下のように言う。上述のものが「色や形や、その他どんな
他の性質や量が物体においてあるようなしかたで、基体としての魂においてあ
るのではない。なぜなら、何であれそのようなものは、そこにおいてある基体
を越えないからである。しかるに精神は、他の者どもをも愛し認識することが
できる」。

\\



6. {\scshape Praeterea}, forma simplex subiectum esse non
potest. Anima autem est forma simplex, cum non sit composita ex
materia et forma, ut supra dictum est. Non ergo potentia animae potest
esse in ipsa sicut in subiecto.

&

 さらに、単純形相は基体であることができない。しかるに魂は、上述のよう
 に、質料と形相から複合されていないので、単純形相である。ゆえに魂の能
 力は、基体としての魂においてあることはできない。


\\



7. {\scshape Praeterea}, accidens non est principium substantialis
differentiae. Sed sensibile et rationale sunt substantiales
differentiae, et sumuntur a sensu et ratione, quae sunt potentiae
animae. Ergo potentiae animae non sunt accidentia. Et ita videtur quod
potentia animae sit eius essentia.

&

 さらに、附帯性は、実体的な種差の根源ではない。しかるに、可感的である
 ことと理性的であることは、実体的な種差であるが、これらは魂の能力であ
 る感覚と理性から採られている。ゆえに魂の能力は附帯性ではない。従って、
 魂の能力はその本質であると思われる。


\\



{\scshape Sed contra est} quod Dionysius dicit, XI cap.~{\itshape
Caelest.~Hier}., quod caelestes spiritus dividuntur in essentiam,
virtutem et operationem. Multo igitur magis in anima aliud est
essentia, et aliud virtus sive potentia.

&

 しかし反対に、ディオニュシウスは『天上階級論』第11章で、天の霊たちは
 本質、ちから、働きへと分割されると言う。ゆえに、ましてや、魂において、
 本質は、ちからないし能力と別である。


\\



{\scshape Respondeo dicendum} quod impossibile est dicere quod
essentia animae sit eius potentia; licet hoc quidam posuerint. Et hoc
 dupliciter ostenditur, quantum ad praesens.


&

 解答する。以下のように言われるべきである。ある人々はそう言ったのだが、
 魂の本質が、それの能力であると語ることは不可能である。そして現在の問
 題にかんする限り、これは二通りに明示される。

\\


 Primo quia, cum potentia et actus dividant ens et quodlibet genus
entis, oportet quod ad idem genus referatur potentia et actus. Et
ideo, si actus non est in genere substantiae, potentia quae dicitur ad
illum actum, non potest esse in genere substantiae.

&

 第一に、可能態(potentia)と現実態は(actus)有と、何であれ有の類を分割す
 るので、可能態と現実態は、同一の類に還元されなければならない。ゆえに、
 もし現実態が実体の類にないならば、その現実態に対して語られる可能態も
 また、実体の類の中にはありえない。


\\


 Operatio autem animae non est in genere substantiae; sed in solo Deo,
cuius operatio est eius substantia. Unde Dei potentia, quae est
operationis principium, est ipsa Dei essentia. Quod non potest esse
verum neque in anima, neque in aliqua creatura; ut supra etiam de
Angelo dictum est.



&

しかるに魂の働きは実体の類の中にはない。そうであるのは、その働きがその
実体である神においてだけである。このことから、働きの根源である神の能力
は、神の本質それ自体である。このことは、魂において、またどんな被造物に
おいても真でありえない。これは以前に天使についても語られた。


\\


Secundo, hoc etiam impossibile apparet in anima.  Nam anima secundum
suam essentiam est actus. Si ergo ipsa essentia animae esset
immediatum operationis principium, semper habens animam actu haberet
opera vitae; sicut semper habens animam actu est vivum.


&

 第二に、このことは魂においては不可能であるように見える。理由は以下の
 通り。魂は、自らの本質に即して現実態・作用(actus)である。ゆえに、もし
 魂の本質自体が、働きの直接的な根源だとすると、魂を持つものは、常に現
 実態において生命の業を持ったであろう。ちょうど、魂を持つものは常に現
 実態において生きるものであるように。


\\


 Non enim, inquantum est forma, est actus ordinatus ad ulteriorem
actum, sed est ultimus terminus generationis. Unde quod sit in
potentia adhuc ad alium actum, hoc non competit ei secundum suam
essentiam, inquantum est forma; sed secundum suam potentiam.

&


 実際、魂は、それが形相である限りにおいては、究極の現実態へ秩序付けら
 れた現実態・作用ではなく、(魂自体が)生成の最終目的である。したがっ
 て、他の現実態に対して可能態にあることは、自らの本質に即して、形相で
 あるかぎりにおける魂に適合するのではなく、自らの能力に即してである。

\\

 Et sic ipsa anima, secundum quod subest suae potentiae, dicitur actus
primus, ordinatus ad actum secundum. Invenitur autem habens animam non
semper esse in actu operum vitae.

&

このように、魂自体は、自らの能力のもとにあるかぎりにおいて、第一現実態
と言われ、第二現実態へと秩序付けられている。ところが、魂を持つものが、
常に、生命の業を現実態において持つわけではない。

\\

Unde etiam in definitione animae dicitur quod est {\itshape actus
corporis potentia vitam habentis}, quae tamen potentia {\itshape non
abiicit animam}. Relinquitur ergo quod essentia animae non est eius
potentia. Nihil enim est in potentia secundum actum, inquantum est
actus.

&

したがって、魂の定義においても、「可能的に生命を持つ身体の現実態」と言
われるが、この可能態は「魂を捨てない」と言われる。ゆえに、魂の本質はそ
の能力ではないことが残される。なぜなら、何ものも、現実態であるかぎりに
おいて、現実態に即して可能態にあるものはないからである。
 

\\



{\scshape Ad primum ergo dicendum} quod Augustinus loquitur de mente
secundum quod noscit se et amat se. Sic ergo notitia et amor,
inquantum referuntur ad ipsam ut cognitam et amatam, substantialiter
vel essentialiter sunt in anima, quia ipsa substantia vel essentia
 animae cognoscitur et amatur.

&

 第一異論に対しては、それゆえ、以下のように言われるべきである。アウグ
 スティヌスは、自らを知り、自らを愛する限りにおける精神について語って
 いる。それゆえ、その意味で、知と愛は、知られたものや愛されたものとし
 ての自らへ関係づけられるかぎりにおいて、魂の中で実体的、ないし本質的
 なしかたである。というのも、魂の実体や本質自体が知られ愛されるからで
 ある。

\\


 Et similiter intelligendum est quod alibi dicit, quod sunt {\itshape
una vita, una mens, una essentia}. Vel, sicut quidam dicunt, haec
locutio verificatur secundum modum quo totum potestativum praedicatur
de suis partibus, quod medium est inter totum universale et totum
integrale.

&

また、ある箇所で「一つの生命、一つの精神、一つの本質」と言われることも、
同様に理解されるべきである。あるいは、ある人々が言うように、この表現は、
力の全体が自らの部分に述語付けられるしかたに即して真とされる。これは普
遍的全体と統合的全体の間にある。

\\

 Totum enim universale adest cuilibet parti secundum totam suam
essentiam et virtutem, ut animal homini et equo, et ideo proprie de
singulis partibus praedicatur. Totum vero integrale non est in
qualibet parte, neque secundum totam essentiam, neque secundum totam
virtutem.

&

 すなわち普遍的全体は、自らの本質とちからの全体に即して、どの部分のも
 とにもある。たとえば、動物が、人間と馬に対してそうであるように。それ
 ゆえ、固有の意味で、個々の部分について述語付けられる。他方で、統合的
 全体は、本質全体に即しても、ちから全体に即しても、どの部分の中にもな
 い。


\\


 Et ideo nullo modo de singulis partibus praedicatur; sed aliquo modo,
licet improprie, praedicatur de omnibus simul, ut si dicamus quod
paries, tectum et fundamentum sunt domus. 

&

 それゆえ、統合的全体は、個々の部分について決して述語付けられない。し
 かし、あるしかたで、固有にではないにしても、すべてのものについて同時
 に述語付けられる。たとえば、私たちが、壁と屋根と基礎が家である、と言
 う場合のように。


\\

Totum vero potentiale adest singulis partibus secundum totam suam
essentiam, sed non secundum totam virtutem.  Et ideo quodammodo potest
praedicari de qualibet parte; sed non ita proprie sicut totum
universale. Et per hunc modum Augustinus dicit quod memoria,
intelligentia et voluntas sunt una animae essentia.

&

 これに対して、能力的全体は、個々の部分に、自らの全本質に即してあるが、
 ちから全体に即してはない。ゆえに、どの部分についても、ある意味で述語
 付けられるが、普遍的全体のように固有に述語付けられることはない。そし
 てこのしかたで、アウグスティヌスは「記憶、知性、意志は、魂の一つの本
 質である」と言っている。


\\


{\scshape Ad secundum dicendum} quod actus ad quem est in potentia
materia prima, est substantialis forma. Et ideo potentia materiae non
est aliud quam eius essentia.

&

 第二異論に対しては、以下のように言われるべきである。第一質料が、それ
 に対して可能態にあるところの現実態は、実体形相である。ゆえに質料の可
 能態は、それの本質に他ならない。

\\



{\scshape Ad tertium dicendum} quod actio est compositi, sicut et
esse, existentis enim est agere. Compositum autem per formam
substantialem habet esse substantialiter; per virtutem autem quae
consequitur formam substantialem, operatur. Unde sic se habet forma
accidentalis activa ad formam substantialem agentis (ut calor ad
formam ignis), sicut se habet potentia animae ad animam.

&

 第三異論に対しては、以下のように言われるべきである。作用は、存在もそ
 うであるように、複合体に属する。作用することは、存在するものに属する
 からである。しかるに、複合体は、実体形相を通して、存在を実体的に有す
 る。これに対して、実体形相に伴うちから(virtus)を通して、働く
 (operatur)。このことから、(例えば熱が火の形相に対するように)能動的
 な附帯形相が、作用者の実体形相に対して関係するようにして、魂の能力は
 魂に関係する。

\\



{\scshape Ad quartum dicendum} quod hoc ipsum quod forma accidentalis
est actionis principium, habet a forma substantiali. Et ideo forma
substantialis est primum actionis principium, sed non proximum. Et
secundum hoc philosophus dicit quod id quo intelligimus et sentimus,
est anima.


&

 第四異論に対しては、以下のように言われるべきである。附帯形相が作用の
 根源であるということそれ自体を、実体形相によってもつ。ゆえに、実体形
 相は作用の第一根源だが、しかし、近接的な根源ではない。そしてこのこと
 に即して、哲学者は、私たちがそれによって知性認識し、感覚するところの
 ものは魂である、と述べている。

\\


{\scshape Ad quintum dicendum} quod, si accidens accipiatur secundum
quod dividitur contra substantiam, sic nihil potest esse medium inter
substantiam et accidens, quia dividuntur secundum affirmationem et
negationem, scilicet secundum esse in subiecto et non esse in
subiecto. Et hoc modo, cum potentia animae non sit eius essentia,
oportet quod sit accidens, et est in secunda specie qualitatis.

&

第五異論に対しては、以下のように言われるべきである。附帯性が、それが実
体に対して分割される限りにおいて理解されるならば、その場合には、実体と
附帯性の間には何もありえない。というのも、これらは肯定と否定、すなわち
基体においてあるか、基体においてあるのではないかということに即して分割
されるからである。そしてこの意味では、魂の能力は魂の本質でないので、そ
れは附帯性でなければならず、性質の第二の種の中にある。

\\



 Si vero accipiatur accidens secundum quod ponitur unum quinque
universalium, sic aliquid est medium inter substantiam et
accidens. Quia ad substantiam pertinet quidquid est essentiale rei,
non autem quidquid est extra essentiam, potest sic dici accidens, sed
solum id quod non causatur ex principiis essentialibus speciei.

&


 他方、附帯性が五つの普遍の一つとされるかぎりで理解されるならば、その
 場合には、実体と附帯性の中間に何かがある。というのも、事物にとって本
 質的なものは何であれ実体に属すが、本質の外にあるものがすべてこの意味
 で附帯性とは言われ得ず、ただ種の本質的根源から原因されないものだけが
 附帯性と言われるからである。

\\


 Proprium enim non est de essentia rei, sed ex principiis
essentialibus speciei causatur, unde medium est inter essentiam et
accidens sic dictum. Et hoc modo potentiae animae possunt dici mediae
inter substantiam et accidens, quasi proprietates animae naturales.

&

たとえば、固有性は事物の本質に属さないが、種の本質的根源から原因される。
このことから、この意味で言われる本質と附帯性の中間にある。この意味で、
魂の能力は、あたかも魂の本性的な固有性であるかのようにして、実体と附帯
性の中間にあると言われうる。
 

\\


 Quod autem Augustinus dicit, quod notitia et amor non sunt
in anima sicut accidentia in subiecto, intelligitur secundum modum
praedictum, prout comparantur ad animam, non sicut ad amantem et
cognoscentem; sed prout comparantur ad eam sicut ad amatam et
cognitam. Et hoc modo procedit sua probatio, quia si amor esset in
anima amata sicut in subiecto, sequeretur quod accidens transcenderet
suum subiectum; cum etiam alia sint amata per animam.

&

 さらに、アウグスティヌスが、知と愛は、附帯性が基体の中にあるように、
 魂の中にあるのではない、と言うことは、愛するものや知るものとしての魂
 に関係づけられるかぎりでではなく、愛されたものや知られたものとしての
 魂に関係づけられるかぎりで理解される。そしてこのしかたで、その証明は
 進行する。すなわち、もし愛が、愛される魂の中に、基体の中にあるように
 してあるならば、附帯性が自らの基体を越えていくことになっただろう。他
 のものもまた魂によって愛されるものだから。


\\

{\scshape Ad sextum dicendum} quod anima, licet non sit composita ex
materia et forma, habet tamen aliquid de potentialitate admixtum ut
supra dictum est. Et ideo potest esse subiectum accidentis. Propositio
autem inducta locum habet in Deo, qui est actus purus, in qua materia
Boetius eam introducit.

&


 第六異論に対しては、以下のように言われるべきである。魂は、質料と形相
 と辛複合されてはいないが、上述のように、何らかの可能態性との混合を持っ
 ている。ゆえに、附帯性の基体でありうる。異論で引用されている命題は、
 純粋現実態である神においては成立する。ボエティウスは、この問題におい
 て、この命題を導入している。

\\



{\scshape Ad septimum dicendum} quod rationale et sensibile, prout
sunt differentiae, non sumuntur a potentiis sensus et rationis; sed ab
ipsa anima sensitiva et rationali. Quia tamen formae substantiales,
quae secundum se sunt nobis ignotae, innotescunt per accidentia; nihil
prohibet interdum accidentia loco differentiarum substantialium poni.

&

 第七異論に対しては、以下のように言われるべきである。理性的であること、
 感覚的であることは、それらが種差であるかぎりにおいては、感覚や理性と
 いう能力からではなく、感覚的魂や理性的魂それ自体から採られている。し
 かし、実体形相は、それ自体に即しては私たちに知られないので、附帯性を
 通して知られる。実体形相の代わりに附帯性が置かれることがある、という
 ことは、なんら差し支えない。


\end{longtable}
\newpage

\rhead{a.~2}
\begin{center}
{\Large {\bfseries ARTICULUS SECUNDUS}}\\
{\large UTRUM SINT PLURES POTENTIAE ANIMAE}\\
{\Large 第二項\\複数の能力が魂にはあるか}
\end{center}

\begin{longtable}{p{21em}p{21em}}
 {\scshape Ad secundum sic proceditur}. Videtur quod non sint plures
 potentiae animae. Anima enim intellectiva maxime ad divinam
 similitudinem accedit. Sed in Deo est una et simplex potentia. Ergo
 et in anima intellectiva.

 &

 第二項の問題へは以下のように進められる。魂に複数の能力はないと思われ
 る。理由は以下の通り。知性的魂は最大限に神の類似へ近づいている。しか
 るに神の中にあるのは一つの単純な能力である。ゆえに知性的魂においても
 同じである。

\\

2. {\scshape Praeterea}, quanto virtus est superior, tanto est magis
unita. Sed anima intellectiva excedit omnes alias formas in
virtute. Ergo maxime debet habere unam virtutem seu potentiam.

 &

さらに、ちからが上位であるほど、それはより合一されている。しかるに知性
的魂は、ちからにおいて、他のすべての形相を越えている。ゆえに、最大限に、
一つのちからないし能力をもつはずである。

\\


3. {\scshape Praeterea}, operari est existentis in actu. Sed per
eandem essentiam animae homo habet esse secundum diversos gradus
perfectionis, ut supra habitum est. Ergo per eandem potentiam animae
operatur diversas operationes diversorum graduum.

 &

 さらに、働くことは、現実に存在するものに属する。しかるに、上述のよう
 に、人間は、魂の同一の本質によって、完全性のさまざまな段階に即して存
 在を持つ。ゆえに、魂の同一の能力によって、さまざまな段階に属するさま
 ざまな働きを行う。


\\


 {\scshape Sed contra est} quod philosophus, in II de anima ponit
 plures animae potentias.

 &

しかし反対に、哲学者は『デ・アニマ』第2巻で、魂の複数の能力を措定して
いる。

\\


 {\scshape Respondeo dicendum} quod necesse est ponere plures animae
 potentias. Ad cuius evidentiam, considerandum est quod, sicut
 philosophus dicit in II {\itshape de Caelo}, quae sunt in rebus
 infima, non possunt consequi perfectam bonitatem, sed aliquam
 imperfectam consequuntur paucis motibus; superiora vero his
 adipiscuntur perfectam bonitatem motibus multis; his autem superiora
 sunt quae adipiscuntur perfectam bonitatem motibus paucis; summa vero
 perfectio invenitur in his quae absque motu perfectam possident
 bonitatem.


 &

解答する。以下のように言われるべきである。魂の複数の能力を措定すること
は必要である。これを明らかにするためには、以下のことが考察されるべきで
ある。哲学者が『天体論』第2巻で言うように、諸事物において最下位のもの
どもは、完全な善性を獲得することができず、少ない運動によって何らかの不
完全な善性を獲得する。これに対して、それらより上位のものどもは、多くの
運動によって完全な善性に到達する。さらに、より少ない運動によって完全な
善性に到達するものは、それらより上位のものである。他方、最高に完全なも
のが、運動なしに完全な善性を所有するものにおいて見出される。

\\


 Sicut infime est ad sanitatem dispositus qui non potest
 perfectam consequi sanitatem, sed aliquam modicam consequitur paucis
 remediis; melius autem dispositus est qui potest perfectam consequi
 sanitatem, sed remediis multis; et adhuc melius, qui remediis paucis;
 optime autem, qui absque remedio perfectam sanitatem habet.
 
 &

 たとえば、完全な健康を獲得することができず、少しの薬で何らかの少しの
 健康を獲得する人は、健康に対して最低の態勢にあり、完全な健康を獲得で
 きるが、多くの薬によってである人は中間の態勢であり、そして、少しの薬
 によってそうである人はさらによく、そして、薬なしに完全の健康を持つ人
 は最善である。


\\

Dicendum est ergo quod res quae sunt infra hominem, quaedam
 particularia bona consequuntur, et ideo quasdam paucas et
 determinatas operationes habent et virtutes. Homo autem potest
 consequi universalem et perfectam bonitatem, quia potest adipisci
 beatitudinem.
 
 &

 それゆえ、以下のように言われるべきである。人間より下の事物は、ある個
 別的な善を獲得し、それゆえ何らかの少なく限定された働きやちからをもつ。
 これに対して人間は、普遍的で完全な善性を獲得できる。なぜなら、人間は
 至福に到達できるからである。


\\

 Est tamen in ultimo gradu, secundum naturam, eorum quibus competit
 beatitudo, et ideo multis et diversis operationibus et virtutibus
 indiget anima humana. Angelis vero minor diversitas potentiarum
 competit. In Deo vero non est aliqua potentia vel actio, praeter eius
 essentiam.
 
 &

 しかし、人間は、至福が適合するものどもの中で、本性において最下位にい
 るので、人間の魂は多くのさまざまなな働きとちからを必要とする。他方、
 天使には、能力のより少ない多様性が適合する。しかし神の中には、神の本
 質以外に、何の能力も作用もない。


\\

 Est et alia ratio quare anima humana abundat diversitate potentiarum,
 videlicet quia est in confinio spiritualium et corporalium
 creaturarum, et ideo concurrunt in ipsa virtutes utrarumque
 creaturarum.

 &

 なぜ人間の魂が能力の多様性にあふれているかを説明する他の理由がある。
 すなわち、人間の魂は、霊的被造物と物体的被造物の境界にあるので、魂に
 おいて、どちらの被造物のちからも集まるからである。

 

\\

 {\scshape Ad primum ergo dicendum} quod in hoc ipso magis ad
 similitudinem Dei accedit anima intellectiva quam creaturae
 inferiores, quod perfectam bonitatem consequi potest; licet per multa
 et diversa; in quo deficit a superioribus.

 &

 第一異論に対しては、それゆえ、以下のように言われるべきである。知性的
 魂は、完全な善性を獲得できるという点で、下位の被造物よりも神の類似に
 近づいている。ただし、多くの多様なものを通してであり、この点において、
 上位のものには及ばない。

\\


 {\scshape Ad secundum dicendum} quod virtus unita est superior, si ad
 aequalia se extendat. Sed virtus multiplicata est superior, si plura
 ei subiiciantur.

 &

 第二異論に対しては、以下のように言われるべきである。合一されたちから
 が上位であるのは、等しいものに及ぶ場合である。もし、複数のものがそれ
 に従属するならば、多数化されたちからの方が上位である。
 
\\

 {\scshape Ad tertium dicendum} quod unius rei est unum esse
 substantiale, sed possunt esse operationes plures. Et ideo est una
 essentia animae, sed potentiae plures.

 &

 第三異論に対しては、以下のように言われるべきである。一つの事物には、
 一つの実体的な存在が属するが、複数の働きが属しうる。ゆえに、魂の本質
 は一つだが、能力は複数である。


\end{longtable}
\newpage
 
\rhead{a.~3}
\begin{center}
{\Large {\bfseries ARTICULUS TERTIUS}}\\
{\large UTRUM POTENTIAE DISTINGUANTUR\\PER ACTUS ET OBIECTA}\\
{\footnotesize Qu.~{\itshape de Anima}, a.13; lib.~II {\itshape de Anima}, lect.~6.}\\
{\Large 第三項\\能力は作用と対象によって区別されるか}
\end{center}

\begin{longtable}{p{21em}p{21em}}
 {\scshape Ad tertium sic proceditur}. Videtur quod potentiae non
 distinguantur per actus et obiecta. Nihil enim determinatur ad
 speciem per illud quod posterius, vel extrinsecum est. Actus autem
 est posterior potentia; obiectum autem est extrinsecum. Ergo per ea
 potentiae non distinguuntur secundum speciem.
 
&

 第三項の問題へ、議論は以下のように進められる。能力は作用と対象によっ
 て区別されるのではないと思われる。理由は以下の通り。何ものも、より後
 のものや外的なものによって、種へと限定されることはない。しかるに、作
 用は能力より後のものであり、対象は外的なものである。ゆえにそれらによっ
 て能力が種において区別されることはない。


\\



2. {\scshape Praeterea}, contraria sunt quae maxime differunt. Si
igitur potentiae distinguerentur penes obiecta, sequeretur quod
contrariorum non esset eadem potentia. Quod patet esse falsum fere in
omnibus, nam potentia visiva eadem est albi et nigri, et gustus idem
est dulcis et amari.

 
&

さらに、対立するものとは最大限に異なるものである。ゆえに、もし能力が対
象に即して区別されるならば、同一の能力は反対のものに関わらなかったであ
ろう。これはほとんどすべての事柄において偽であることが明らかである。な
ぜなら、同一の見る能力が使徒ろくろに関わり、同一の味覚が甘いものと苦い
ものに関わるからである。

\\



3. {\scshape Praeterea}, remota causa, removetur effectus. Si igitur
potentiarum differentia esset ex differentia obiectorum, idem obiectum
non pertineret ad diversas potentias. Quod patet esse falsum, nam idem
est quod potentia cognoscitiva cognoscit, et appetitiva appetit.
 
 
&

さらに、原因が取り除かれると、結果もなくなる。ゆえに、もし能力の差異が
対象の差異に基づいてあるならば、同一の対象がさまざまな能力に属すること
がなかったであろう。これは偽であることが明らかである。というのも、認識
能力が認識し、欲求能力が欲求するのは同一のものだからである。


\\


4. {\scshape Praeterea}, id quod per se est causa alicuius, in omnibus
causat illud. Sed quaedam obiecta diversa, quae pertinent ad diversas
potentias, pertinent etiam ad aliquam unam potentiam, sicut sonus et
color pertinent ad visum et auditum, quae sunt diversae potentiae; et
tamen pertinent ad unam potentiam sensus communis. Non ergo potentiae
distinguuntur secundum differentiam obiectorum.

 
&

さらに、自体的に何かの原因であるものは、すべてのものにおいて、それを原
因する。しかるに、さまざまな能力に属するさまざまな対象が、さらに一つの
能力に属するということがある。たとえば、音と色は、異なる能力である視覚
と聴覚に属するが、しかし共通感覚という一つの能力にも属する。ゆえに能力
は対象の差異に即して区別されるのではない。

\\

 {\scshape Sed contra}, posteriora distinguuntur secundum priora. Sed
 philosophus dicit II {\itshape de Anima}, quod {\itshape priores
 potentiis actus et operationes secundum rationem sunt; et adhuc his
 priora sunt opposita, sive obiecta}. Ergo potentiae distinguuntur
 secundum actus et obiecta.
 
&

 しかし反対に、後のものは先のものにしたがって区別される。しかるに哲学
 者が『デ・アニマ』第2巻で「作用と働きは、その性格に即しては能力よりも
 先であり、さらにこれらよりも対立するもの(ないし対象)が先である」と
 言う。ゆえに能力は、作用と対象に即して区別される。


\\

 {\scshape Respondeo dicendum} quod potentia, secundum illud quod est
 potentia, ordinatur ad actum. Unde oportet rationem potentiae accipi
 ex actu ad quem ordinatur, et per consequens oportet quod ratio
 potentiae diversificetur, ut diversificatur ratio actus.

&

解答する。以下のように言われるべきである。能力・可能態(potentia)は、そ
れが能力である限りにおいて、作用・現実態(actus)に秩序付けられている。
このことから、ある能力の性格は、それが秩序付けられている先の作用から受
け取られなければならない。そして結果的に、能力の性格は、作用の性格が多
様化されるようにして多様化されなければならない。

\\


 Ratio autem actus diversificatur secundum diversam rationem
 obiecti. Omnis enim actio vel est potentiae activae, vel
 passivae. Obiectum autem comparatur ad actum potentiae passivae,
 sicut principium et causa movens, color enim inquantum movet visum,
 est principium visionis.
 
&

 ところで、作用の性格は、対象の多様な性格に即して多様化される。その理
 由は以下の通りである。すべての作用は能動的能力か受動的能力かのいずれ
 かに属する。しかるに対象は受動的能力の作用に対して、動かす根源や原因
 として関係する。たとえば色は視覚を動かす限りにおいて、視覚の根源であ
 る。

\\

 Ad actum autem potentiae activae comparatur obiectum ut terminus et
 finis, sicut augmentativae virtutis obiectum est quantum perfectum,
 quod est finis augmenti. 
 
 
&

他方で、対象は能動的能力の作用に対して、終極や目的として関係する。たと
えば、成長能力の対象は完成された量であり、これは成長の目的である。

\\

Ex his autem duobus actio speciem recipit, scilicet ex principio, vel
 ex fine seu termino, differt enim calefactio ab infrigidatione,
 secundum quod haec quidem a calido, scilicet activo, ad calidum; illa
 autem a frigido ad frigidum procedit. Unde necesse est quod potentiae
 diversificentur secundum actus et obiecta.



 &

 そしてこれら二つのものから、すなわち根源から、または目的ないし終極か
 ら、作用は種を受け取る。たとえば加熱作用が冷却作用と異なるのは、前者
 が作用的根源としての熱いものから熱いものへ、後者は冷たいものから冷た
 いものへと出ていくことに即してである。このことから、能力が、作用と対
 象に即して多様化されることが必要である。

 \\

  Sed tamen considerandum est quod ea quae sunt per accidens, non
 diversificant speciem. Quia enim coloratum accidit animali, non
 diversificantur species animalis per differentiam coloris, sed per
 differentiam eius quod per se accidit animali, per differentiam
 scilicet animae sensitivae, quae quandoque invenitur cum ratione,
 quandoque sine ratione. Unde rationale et irrationale sunt
 differentiae divisivae animalis, diversas eius species constituentes.
 
&

しかし、附帯的であるものは種を多様化しないことが考えられるべきである。
というのも、たとえば、色づいていることが、動物に附帯するが、動物の種は
色の差異によって多様化されず、むしろ自体的に動物に附帯するものの差異に
よって、すなわち、理性と共に見出されることもあれば、理性なしに見出され
ることもあるという感覚的魂の差異によって、多様化される。 このことから、
理性的であることと非理性的であることとは、動物を多様化する差異であり、
動物の多様な種を構成する。


\\

 Sic igitur non quaecumque diversitas obiectorum diversificat
 potentias animae; sed differentia eius ad quod per se potentia
 respicit. Sicut sensus per se respicit passibilem qualitatem, quae
 per se dividitur in colorem, sonum et huiusmodi, et ideo alia
 potentia sensitiva est coloris, scilicet visus, et alia soni,
 scilicet auditus.
 
&

ゆえに、このように、どんな対象の差異も魂の能力を多様化するというわけで
はなく、そうするのは、能力が自体的に関係するものの差異である。たとえば、
感覚は自体的に受動的性質に関係するが、この性質は、自体的に、色、音、そ
のようなものに多様化されるので、色を感覚する能力、すなわち視覚と、音を
感覚する能力、すなわち聴覚は、異なる能力である。

\\

 Sed passibili qualitati, ut colorato accidit esse musicum vel
 grammaticum, vel magnum et parvum, aut hominem vel lapidem. Et ideo
 penes huiusmodi differentias potentiae animae non distinguuntur.

 
&

しかし、たとえば色づいたものという受動的性質には、音楽的であることや、
文法的であること、あるいは大きいことや小さいこと、または人間であること
や石ころであることが附帯する。ゆえに、これらの差異に即しては、魂の能力
 は区別されない。
 
\\


 {\scshape Ad primum ergo dicendum} quod actus, licet sit posterior
 potentia in esse, est tamen prior in intentione et secundum rationem,
 sicut finis agente. Obiectum autem, licet sit extrinsecum, est tamen
 principium vel finis actionis. Principio autem et fini
 proportionantur ea quae sunt intrinseca rei.

 
&

 第一異論に対しては、それゆえ、以下のように言われるべきである。作用は、
 存在において、能力よりも後だが、意図においてや性格に即して、先である。
 ちょうど、目的が作用者よりも先であるように。これに対して対象は、たし
 かに外的なものだが、作用の根源または目的である。そして根源と目的な、
 事物に内在的な事柄に比例的に関係する。


\\


{\scshape Ad secundum dicendum} quod, si potentia aliqua per se
respiceret unum contrariorum sicut obiectum, oporteret quod contrarium
ad aliam potentiam pertineret. Sed potentia animae non per se respicit
propriam rationem contrarii, sed communem rationem utriusque
contrariorum, sicut visus non respicit per se rationem albi, sed
rationem coloris. Et hoc ideo, quia unum contrariorum est quodammodo
ratio alterius, cum se habeant sicut perfectum et imperfectum.
 
 
&

 第二異論に対しては、以下のように言われるべきである。もしある能力が、
 自体的に、相反するものどもの一方に、対象に関係するようにして関係した
 ならば、相反するもう片方は、別の能力に属さなければならなかっただろう。
 しかし魂の能力は、相反するものどもの固有の性格に自体的に関係するので
 はなく、相反するものどもの両方に共通する性格に関係する。たとえば視覚
 は、自体的に白の性格に関係せず、むしろ色の性格に関係する。そしてこれ
 は、相反するものどもの一方は、ある意味で他方の根拠(ratio)であることに
 よる。なぜなら、一方はもう片方に、完全なものが不完全なものに対するよ
 うな関係にあるからである。

\\


 {\scshape Ad tertium dicendum} quod nihil prohibet id quod est
 subiecto idem esse diversum secundum rationem. Et ideo potest ad
 diversas potentias animae pertinere.

 
&

 第三異論に対しては、以下のように言われるべきである。基体において同一
 のものが、性格において多様なものであることは差し支えない。ゆえに、そ
 れらは魂の多様な能力に属しうる。


\\

 {\scshape Ad quartum dicendum} quod potentia superior per se respicit
 universaliorem rationem obiecti, quam potentia inferior, quia quanto
 potentia est superior, tanto ad plura se extendit. Et ideo multa
 conveniunt in una ratione obiecti, quam per se respicit superior
 potentia, quae tamen differunt secundum rationes quas per se
 respiciunt inferiores potentiae. Et inde est quod diversa obiecta
 pertinent ad diversas inferiores potentias, quae tamen uni superiori
 potentiae subduntur.

  
&

第四異論に対しては、以下のように言われるべきである。上位の能力は、下位
の能力よりも、自体的に、より普遍的な対象の性格に関係する。なぜなら、能
力は上位であるほど、多くのものへ及ぶからである。ゆえに、多くのものが、
対象の一つの性格において一致していて、上位の能力はそれに自体的に関係す
るが、下位の諸能力が自体的に関係する性格に即しては異なっているというこ
とが起こる。このことから、多様な対象が多様な会の諸能力に属するが、しか
しそれらが、一つの上位の能力に従属するということがある。


\end{longtable}
\newpage


\rhead{a.~4}
\begin{center}
{\Large {\bfseries ARTICULUS QUARTUS}}\\
{\large UTRUM IN POTENTIIS ANIMAE SIT ORDO}\\
{\footnotesize Qu.~{\itshape de Anima}, art.13, ad 10.}\\
{\Large 第四項\\魂の諸能力の中に秩序があるか}
\end{center}

\begin{longtable}{p{21em}p{21em}}
{\scshape Ad quartum sic proceditur}. Videtur quod in potentiis animae
non sit ordo. In his enim quae cadunt sub una divisione, non est prius
et posterius, sed sunt naturaliter simul. Sed potentiae animae contra
se invicem dividuntur. Ergo inter eas non est ordo.

 
&

 第四項の問題へ、議論は以下のように進められる。魂の諸能力の中に秩序は
 ないと思われる。理由は以下の通り。一つの分割のもとにあるものに前後は
 なく、本性において同時である。しかるに魂の諸能力は、相互に対立して分
 割される。ゆえに、それらの間に秩序はない。


\\



2. {\scshape Praeterea}, potentiae animae comparantur ad obiecta, et
ad ipsam animam. Sed ex parte animae, inter eas non est ordo, quia
anima est una. Similiter etiam nec ex parte obiectorum, cum sint
diversa et penitus disparata, ut patet de colore et sono. In potentiis
ergo animae non est ordo.

 
&

さらに、魂の諸能力は、対象と魂自身に関係づけられる。しかるに、魂の側か
らは、それらの間に秩序はない。なぜなら魂は一つだから。同様に対象の側か
らもない。なぜなら、色と音について明らかであるように、対象は多様であり、
まったくばらばらだから。ゆえに、魂の諸能力の中に秩序はない。

\\




3. {\scshape Praeterea}, in potentiis ordinatis hoc invenitur, quod
operatio unius dependet ab operatione alterius. Sed actus unius
potentiae animae non dependet ab actu alterius, potest enim visus
exire in actum absque auditu, et e converso. Non ergo inter potentias
animae est ordo.
 
&

さらに、秩序付けられた諸能力においては、ある能力の働きが、他の能力の働
きに依存するということが見出される。しかるに、魂の一つの能力の作用は、
別の能力の作用に依存しない。なぜなら、視覚は聴覚なしに作用へと出ていく
ことができるし、その逆もできる。ゆえに魂の諸能力の中に秩序はない。
 

\\




 {\scshape Sed contra est} quod philosophus, in II {\itshape de
 Anima}, comparat partes sive potentias animae figuris. Sed figurae
 habent ordinem ad invicem. Ergo et potentiae animae.

 
&

しかし反対に、哲学者は『デ・アニマ』第2巻で、魂の諸部分ないし諸能力を、
形(図形)になぞらえている。しかるに、形は相互に秩序をもつ。ゆえに魂の
諸能力もそうである。

\\




 {\scshape Respondeo dicendum} quod, cum anima sit una, potentiae vero
 plures; ordine autem quodam ab uno in multitudinem procedatur;
 necesse est inter potentias animae ordinem esse.

 
&

解答する。以下のように言われるべきである。魂は一つだが能力は複数であり、一から多へ

\\


 Triplex autem ordo
 inter eas attenditur. Quorum duo considerantur secundum dependentiam
 unius potentiae ab altera, tertius autem accipitur secundum ordinem
 obiectorum.


 
&



\\


 Dependentia autem unius potentiae ab altera dupliciter
 accipi potest, uno modo, secundum naturae ordinem, prout perfecta
 sunt naturaliter imperfectis priora; alio modo, secundum ordinem
 generationis et temporis, prout ex imperfecto ad perfectum
 venitur.


 
&



\\


 Secundum igitur primum potentiarum ordinem, potentiae
 intellectivae sunt priores potentiis sensitivis, unde dirigunt eas et
 imperant eis. Et similiter potentiae sensitivae hoc ordine sunt
 priores potentiis animae nutritivae.


 
&



\\


 Secundum vero ordinem secundum,
 e converso se habet. Nam potentiae animae nutritivae sunt priores, in
 via generationis, potentiis animae sensitivae, unde ad earum actiones
 praeparant corpus.


 
&



\\


 Et similiter est de potentiis sensitivis respectu
 intellectivarum. Secundum autem ordinem tertium, ordinantur quaedam
 vires sensitivae ad invicem, scilicet visus, auditus et olfactus. Nam
 visibile est prius naturaliter, quia est commune superioribus et
 inferioribus corporibus. Sonus autem audibilis fit in aere, qui est
 naturaliter prior commixtione elementorum, quam consequitur odor.

 
&



\\



[31630] Iª q. 77 a. 4 ad 1
 Ad primum ergo dicendum quod alicuius generis species se habent secundum prius et posterius, sicut numeri et figurae, quantum ad esse; licet simul esse dicantur inquantum suscipiunt communis generis praedicationem.

 
&



\\



[31631] Iª q. 77 a. 4 ad 2
 Ad secundum dicendum quod ordo iste potentiarum animae est et ex parte animae, quae secundum ordinem quendam habet aptitudinem ad diversos actus, licet sit una secundum essentiam; et ex parte obiectorum; et etiam ex parte actuum, ut dictum est.

 
&



\\



[31632] Iª q. 77 a. 4 ad 3
 Ad tertium dicendum quod ratio illa procedit de illis potentiis in quibus attenditur ordo solum secundum tertium modum. Illae autem potentiae quae ordinantur secundum alios duos modos, ita se habent quod actus unius dependet ab altera.

 
&





\end{longtable}
\newpage

\end{document}

%\rhead{a.~}
%\begin{center}
% {\Large {\bfseries }}\\
% {\large }\\
% {\footnotesize }\\
% {\Large \\}
%\end{center}
%
%\begin{longtable}{p{21em}p{21em}}
%
%&
%
% 
%
%\\
%\end{longtable}
%\newpage


ARTICULUS 5
[31633] Iª q. 77 a. 5 arg. 1
Ad quintum sic proceditur. Videtur quod omnes potentiae animae sint in anima sicut in subiecto. Sicut enim se habent potentiae corporis ad corpus, ita se habent potentiae animae ad animam. Sed corpus est subiectum corporalium potentiarum. Ergo anima est subiectum potentiarum animae.

[31634] Iª q. 77 a. 5 arg. 2
Praeterea, operationes potentiarum animae attribuuntur corpori propter animam, quia, ut dicitur in II de anima, anima est quo sentimus et intelligimus primum. Sed propria principia operationum animae sunt potentiae. Ergo potentiae per prius sunt in anima.

[31635] Iª q. 77 a. 5 arg. 3
Praeterea, Augustinus dicit, XII super Gen. ad Litt., quod anima quaedam sentit non per corpus, immo sine corpore, ut est timor et huiusmodi; quaedam vero sentit per corpus. Sed si potentia sensitiva non esset in sola anima sicut in subiecto, nihil posset sine corpore sentire. Ergo anima est subiectum potentiae sensitivae; et pari ratione, omnium aliarum potentiarum.

[31636] Iª q. 77 a. 5 s. c.
Sed contra est quod philosophus dicit, in libro de somno et vigilia quod sentire non est proprium animae neque corporis, sed coniuncti. Potentia ergo sensitiva est in coniuncto sicut in subiecto. Non ergo sola anima est subiectum omnium potentiarum suarum.

[31637] Iª q. 77 a. 5 co.
Respondeo dicendum quod illud est subiectum operativae potentiae, quod est potens operari, omne enim accidens denominat proprium subiectum. Idem autem est quod potest operari, et quod operatur. Unde oportet quod eius sit potentia sicut subiecti, cuius est operatio; ut etiam philosophus dicit, in principio de somno et vigilia. Manifestum est autem ex supra dictis quod quaedam operationes sunt animae, quae exercentur sine organo corporali, ut intelligere et velle. Unde potentiae quae sunt harum operationum principia, sunt in anima sicut in subiecto. Quaedam vero operationes sunt animae, quae exercentur per organa corporalia; sicut visio per oculum, et auditus per aurem. Et simile est de omnibus aliis operationibus nutritivae et sensitivae partis. Et ideo potentiae quae sunt talium operationum principia, sunt in coniuncto sicut in subiecto, et non in anima sola.

[31638] Iª q. 77 a. 5 ad 1
Ad primum ergo dicendum quod omnes potentiae dicuntur esse animae, non sicut subiecti, sed sicut principii, quia per animam coniunctum habet quod tales operationes operari possit.

[31639] Iª q. 77 a. 5 ad 2
Ad secundum dicendum quod omnes huiusmodi potentiae per prius sunt in anima quam in coniuncto, non sicut in subiecto, sed sicut in principio.

[31640] Iª q. 77 a. 5 ad 3
Ad tertium dicendum quod opinio Platonis fuit quod sentire est operatio animae propria, sicut et intelligere. In multis autem quae ad philosophiam pertinent, Augustinus utitur opinionibus Platonis, non asserendo, sed recitando. Tamen, quantum ad praesens pertinet, hoc quod dicitur anima quaedam sentire cum corpore et quaedam sine corpore, dupliciter potest intelligi. Uno modo, quod hoc quod dico cum corpore vel sine corpore, determinet actum sentiendi secundum quod exit a sentiente. Et sic nihil sentit sine corpore, quia actio sentiendi non potest procedere ab anima nisi per organum corporale. Alio modo potest intelligi ita quod praedicta determinent actum sentiendi ex parte obiecti quod sentitur. Et sic quaedam sentit cum corpore, idest in corpore existentia, sicut cum sentit vulnus vel aliquid huiusmodi, quaedam vero sentit sine corpore, idest non existentia in corpore, sed solum in apprehensione animae, sicut cum sentit se tristari vel gaudere de aliquo audito.



%\rhead{a.~}
%\begin{center}
% {\Large {\bfseries }}\\
% {\large }\\
% {\footnotesize }\\
% {\Large \\}
%\end{center}
%
%\begin{longtable}{p{21em}p{21em}}
%
%&
%
% 
%
%\\
%\end{longtable}
%\newpage


ARTICULUS 6
[31641] Iª q. 77 a. 6 arg. 1
Ad sextum sic proceditur. Videtur quod potentiae animae non fluant ab eius essentia. Ab uno enim simplici non procedunt diversa. Essentia autem animae est una et simplex. Cum ergo potentiae animae sint multae et diversae, non possunt procedere ab eius essentia.

[31642] Iª q. 77 a. 6 arg. 2
Praeterea, illud a quo aliud procedit, est causa eius. Sed essentia animae non potest dici causa potentiarum; ut patet discurrenti per singula causarum genera. Ergo potentiae animae non fluunt ab eius essentia.

[31643] Iª q. 77 a. 6 arg. 3
Praeterea, emanatio quendam motum nominat. Sed nihil movetur a seipso, ut probatur in VII libro Physic.; nisi forte ratione partis, sicut animal dicitur moveri a seipso, quia una pars eius est movens et alia mota. Neque etiam anima movetur, ut probatur in I de anima. Non ergo anima causat in se suas potentias.

[31644] Iª q. 77 a. 6 s. c.
Sed contra, potentiae animae sunt quaedam proprietates naturales ipsius. Sed subiectum est causa propriorum accidentium, unde et ponitur in definitione accidentis, ut patet in VII Metaphys. Ergo potentiae animae procedunt ab eius essentia sicut a causa.

[31645] Iª q. 77 a. 6 co.
Respondeo dicendum quod forma substantialis et accidentalis partim conveniunt, et partim differunt. Conveniunt quidem in hoc, quod utraque est actus, et secundum utramque est aliquid quodammodo in actu. Differunt autem in duobus. Primo quidem, quia forma substantialis facit esse simpliciter, et eius subiectum est ens in potentia tantum. Forma autem accidentalis non facit esse simpliciter; sed esse tale, aut tantum, aut aliquo modo se habens, subiectum enim eius est ens in actu. Unde patet quod actualitas per prius invenitur in forma substantiali quam in eius subiecto, et quia primum est causa in quolibet genere, forma substantialis causat esse in actu in suo subiecto. Sed e converso, actualitas per prius invenitur in subiecto formae accidentalis, quam in forma accidentali, unde actualitas formae accidentalis causatur ab actualitate subiecti. Ita quod subiectum, inquantum est in potentia, est susceptivum formae accidentalis, inquantum autem est in actu, est eius productivum. Et hoc dico de proprio et per se accidente, nam respectu accidentis extranei, subiectum est susceptivum tantum; productivum vero talis accidentis est agens extrinsecum. Secundo autem differunt substantialis forma et accidentalis, quia, cum minus principale sit propter principalius, materia est propter formam substantialem; sed e converso, forma accidentalis est propter completionem subiecti. Manifestum est autem ex dictis quod potentiarum animae subiectum est vel ipsa anima sola, quae potest esse subiectum accidentis secundum quod habet aliquid potentialitatis, ut supra dictum est; vel compositum. Compositum autem est in actu per animam. Unde manifestum est quod omnes potentiae animae, sive subiectum earum sit anima sola, sive compositum, fluunt ab essentia animae sicut a principio, quia iam dictum est quod accidens causatur a subiecto secundum quod est actu, et recipitur in eo inquantum est in potentia.

[31646] Iª q. 77 a. 6 ad 1
Ad primum ergo dicendum quod ab uno simplici possunt naturaliter multa procedere ordine quodam. Et iterum propter diversitatem recipientium. Sic igitur ab una essentia animae procedunt multae et diversae potentiae, tum propter ordinem potentiarum, tum etiam secundum diversitatem organorum corporalium.

[31647] Iª q. 77 a. 6 ad 2
Ad secundum dicendum quod subiectum est causa proprii accidentis et finalis, et quodammodo activa; et etiam ut materialis, inquantum est susceptivum accidentis. Et ex hoc potest accipi quod essentia animae est causa omnium potentiarum sicut finis et sicut principium activum; quarundam autem sicut susceptivum.

[31648] Iª q. 77 a. 6 ad 3
Ad tertium dicendum quod emanatio propriorum accidentium a subiecto non est per aliquam transmutationem; sed per aliquam naturalem resultationem, sicut ex uno naturaliter aliud resultat, ut ex luce color.



%\rhead{a.~}
%\begin{center}
% {\Large {\bfseries }}\\
% {\large }\\
% {\footnotesize }\\
% {\Large \\}
%\end{center}
%
%\begin{longtable}{p{21em}p{21em}}
%
%&
%
% 
%
%\\
%\end{longtable}
%\newpage


ARTICULUS 7
[31649] Iª q. 77 a. 7 arg. 1
Ad septimum sic proceditur. Videtur quod una potentia animae non oriatur ab alia. Eorum enim quae simul esse incipiunt, unum non oritur ab alio. Sed omnes potentiae animae sunt simul animae concreatae. Ergo una earum ab alia non oritur.

[31650] Iª q. 77 a. 7 arg. 2
Praeterea, potentia animae oritur ab anima sicut accidens a subiecto. Sed una potentia animae non potest esse subiectum alterius, quia accidentis non est accidens. Ergo una potentia non oritur ab alia.

[31651] Iª q. 77 a. 7 arg. 3
Praeterea, oppositum non oritur a suo opposito, sed unumquodque oritur ex simili secundum speciem. Potentiae autem animae ex opposito dividuntur, sicut diversae species. Ergo una earum non procedit ab alia.

[31652] Iª q. 77 a. 7 s. c.
Sed contra, potentiae cognoscuntur per actus. Sed actus unius potentiae causatur ab alio; sicut actus phantasiae ab actu sensus. Ergo una potentia animae causatur ab alia.

[31653] Iª q. 77 a. 7 co.
Respondeo dicendum quod in his quae secundum ordinem naturalem procedunt ab uno, sicut primum est causa omnium, ita quod est primo propinquius, est quodammodo causa eorum quae sunt magis remota. Ostensum est autem supra quod inter potentias animae est multiplex ordo. Et ideo una potentia animae ab essentia animae procedit mediante alia. Sed quia essentia animae comparatur ad potentias et sicut principium activum et finale, et sicut principium susceptivum, vel seorsum per se vel simul cum corpore; agens autem et finis est perfectius, susceptivum autem principium, inquantum huiusmodi, est minus perfectum, consequens est quod potentiae animae quae sunt priores secundum ordinem perfectionis et naturae, sint principia aliarum per modum finis et activi principii. Videmus enim quod sensus est propter intellectum, et non e converso. Sensus etiam est quaedam deficiens participatio intellectus, unde secundum naturalem originem quodammodo est ab intellectu, sicut imperfectum a perfecto. Sed secundum viam susceptivi principii, e converso potentiae imperfectiores inveniuntur principia respectu aliarum, sicut anima, secundum quod habet potentiam sensitivam, consideratur sicut subiectum et materiale quoddam respectu intellectus. Et propter hoc, imperfectiores potentiae sunt priores in via generationis, prius enim animal generatur quam homo.

[31654] Iª q. 77 a. 7 ad 1
Ad primum ergo dicendum quod, sicut potentia animae ab essentia fluit, non per transmutationem, sed per naturalem quandam resultationem, et est simul cum anima; ita est etiam de una potentia respectu alterius.

[31655] Iª q. 77 a. 7 ad 2
Ad secundum dicendum quod accidens per se non potest esse subiectum accidentis; sed unum accidens per prius recipitur in substantia quam aliud, sicut quantitas quam qualitas. Et hoc modo unum accidens dicitur esse subiectum alterius, ut superficies coloris, inquantum substantia uno accidente mediante recipit aliud. Et similiter potest dici de potentiis animae.

[31656] Iª q. 77 a. 7 ad 3
Ad tertium dicendum quod potentiae animae opponuntur ad invicem oppositione perfecti et imperfecti; sicut etiam species numerorum et figurarum. Haec autem oppositio non impedit originem unius ab alio, quia imperfecta naturaliter a perfectis procedunt.



%\rhead{a.~}
%\begin{center}
% {\Large {\bfseries }}\\
% {\large }\\
% {\footnotesize }\\
% {\Large \\}
%\end{center}
%
%\begin{longtable}{p{21em}p{21em}}
%
%&
%
% 
%
%\\
%\end{longtable}
%\newpage


ARTICULUS 8
[31657] Iª q. 77 a. 8 arg. 1
Ad octavum sic proceditur. Videtur quod omnes potentiae animae remaneant in anima a corpore separata. Dicitur enim in libro de spiritu et anima, quod anima recedit a corpore, secum trahens sensum et imaginationem, rationem et intellectum et intelligentiam, concupiscibilitatem et irascibilitatem.

[31658] Iª q. 77 a. 8 arg. 2
Praeterea, potentiae animae sunt eius naturales proprietates. Sed proprium semper inest, et nunquam separatur ab eo cuius est proprium. Ergo potentiae animae sunt in ea etiam post mortem.

[31659] Iª q. 77 a. 8 arg. 3
Praeterea, potentiae animae, etiam sensitivae, non debilitantur debilitato corpore, quia, ut dicitur in I de anima, si senex accipiat oculum iuvenis, videbit utique sicut et iuvenis. Sed debilitas est via ad corruptionem. Ergo potentiae animae non corrumpuntur corrupto corpore, sed manent in anima separata.

[31660] Iª q. 77 a. 8 arg. 4
Praeterea, memoria est potentia animae sensitivae, ut philosophus probat. Sed memoria manet in anima separata, dicitur enim, Luc. XVI, diviti epuloni in Inferno secundum animam existenti, recordare quia recepisti bona in vita tua. Ergo memoria manet in anima separata; et per consequens aliae potentiae sensitivae partis.

[31661] Iª q. 77 a. 8 arg. 5
Praeterea, gaudium et tristitia sunt in concupiscibili, quae est potentia sensitivae partis. Manifestum est autem animas separatas tristari et gaudere de praemiis vel poenis quas habent. Ergo vis concupiscibilis manet in anima separata.

[31662] Iª q. 77 a. 8 arg. 6
Praeterea, Augustinus dicit, XII super Gen. ad Litt., quod sicut anima, cum corpus iacet sine sensu nondum penitus mortuum, videt quaedam secundum imaginariam visionem; ita cum fuerit a corpore penitus separata per mortem. Sed imaginatio est potentia sensitivae partis. Ergo potentia sensitivae partis manet in anima separata; et per consequens omnes aliae potentiae.

[31663] Iª q. 77 a. 8 s. c.
Sed contra est quod dicitur in libro de Eccles. Dogmat., ex duabus tantum substantiis constat homo, anima cum ratione sua, et carne cum sensibus suis. Ergo, defuncta carne, potentiae sensitivae non manent.

[31664] Iª q. 77 a. 8 co.
Respondeo dicendum quod, sicut iam dictum est, omnes potentiae animae comparantur ad animam solam sicut ad principium. Sed quaedam potentiae comparantur ad animam solam sicut ad subiectum, ut intellectus et voluntas. Et huiusmodi potentiae necesse est quod maneant in anima, corpore destructo. Quaedam vero potentiae sunt in coniuncto sicut in subiecto, sicut omnes potentiae sensitivae partis et nutritivae. Destructo autem subiecto, non potest accidens remanere. Unde, corrupto coniuncto, non manent huiusmodi potentiae actu; sed virtute tantum manent in anima, sicut in principio vel radice. Et sic falsum est, quod quidam dicunt huiusmodi potentias in anima remanere etiam corpore corrupto. Et multo falsius, quod dicunt etiam actus harum potentiarum remanere in anima separata, quia talium potentiarum nulla est actio nisi per organum corporeum.

[31665] Iª q. 77 a. 8 ad 1
Ad primum ergo dicendum quod liber ille auctoritatem non habet. Unde quod ibi scriptum est, eadem facilitate contemnitur, qua dicitur. Tamen potest dici quod trahit secum anima huiusmodi potentias, non actu, sed virtute.

[31666] Iª q. 77 a. 8 ad 2
Ad secundum dicendum quod huiusmodi potentiae quas dicimus actu in anima separata non manere, non sunt proprietates solius animae, sed coniuncti.

[31667] Iª q. 77 a. 8 ad 3
Ad tertium dicendum quod dicuntur non debilitari huiusmodi potentiae debilitato corpore, quia anima manet immutabilis, quae est virtuale principium huiusmodi potentiarum.

[31668] Iª q. 77 a. 8 ad 4
Ad quartum dicendum quod illa recordatio accipitur eo modo quo Augustinus ponit memoriam in mente; non eo modo quo ponitur pars animae sensitivae.

[31669] Iª q. 77 a. 8 ad 5
Ad quintum dicendum quod tristitia et gaudium sunt in anima separata, non secundum appetitum sensitivum, sed secundum appetitum intellectivum; sicut etiam in Angelis.

[31670] Iª q. 77 a. 8 ad 6
Ad sextum dicendum quod Augustinus loquitur ibi inquirendo, non asserendo. Unde quaedam ibi dicta retractat.


