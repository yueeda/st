\documentclass[10pt]{jsarticle} % use larger type; default would be 10pt
%\usepackage[utf8]{inputenc} % set input encoding (not needed with XeLaTeX)
%\usepackage[round,comma,authoryear]{natbib}
%\usepackage{nruby}
\usepackage{okumacro}
\usepackage{longtable}
%\usepqckage{tablefootnote}
\usepackage[polutonikogreek,english,japanese]{babel}
%\usepackage{amsmath}
\usepackage{latexsym}
\usepackage{color}

%----- header -------
\usepackage{fancyhdr}
\lhead{{\it Summa Theologiae} I, q.~18}
%--------------------

\bibliographystyle{jplain}

\title{{\bf PRIMA PARS}\\{\HUGE Summae Theologiae}\\Sancti Thomae
Aquinatis\\{\sffamily QUAESTIO DECIMAOCTAVA}\\DE VITA DEI}
\author{Japanese translation\\by Yoshinori {\sc Ueeda}}
\date{Last modified \today}


%%%% コピペ用
%\rhead{a.~}
%\begin{center}
% {\Large {\bf }}\\
% {\large }\\
% {\footnotesize }\\
% {\Large \\}
%\end{center}
%
%\begin{longtable}{p{21em}p{21em}}
%
%&
%
%
%\\
%\end{longtable}
%\newpage



\begin{document}
\maketitle
\pagestyle{fancy}

\begin{center}
{\Large 第十八問\\神の生命について}
\end{center}

\begin{longtable}{p{21em}p{21em}}

{\huge Q}{\scshape uoniam} autem intelligere viventium est, post
 considerationem de scientia et intellectu divino, considerandum est de
 vita ipsius. Et circa hoc quaeruntur quatuor.

\begin{enumerate}
 \item quorum sit vivere.
 \item quid sit vita.
 \item utrum vita Deo conveniat.
 \item utrum omnia in Deo sint vita.
\end{enumerate}
 
&

さて、知性認識することは生きているものに属するので、知と神の知性についての考
 察の後に、神の生命について、考察されるべきである。これをめぐって4つのこ
 とが問われる。
 
\begin{enumerate}
 \item 生きることは何に属するか。
 \item 生命とは何か。
 \item 生命は神に適合するか。
 \item 神の中のすべてのものは生命か。
\end{enumerate}


\end{longtable}

\newpage

\rhead{a.~1}
\begin{center}
 {\Large {\bf ARTICULUS PRIMUS}}\\
 {\large UTRUM OMNIUM NATURALIUM RERUM SIT VIVERE}\\
 {\footnotesize III {\itshape Sent.}, d.~35, q.~1, a.~1;IV, d.~14, q.~2, a.~3, qu$^a$ 2; d.~49, qu$^a$ 1, a.~2, qu$^a$ 3; I {\itshape SCG}, cap.~97; {\itshape De Verit.}, q.~4, a.~8; {\itshape De Pot.}, q.~10, a.~1; {\itshape De Div.~Nom.}, cap.~6, lect.~1; in {\itshape Ioan.}, cap.~17, lect.~1; I {\itshape De Anima}, lect.~14; II, lect.~1.}\\

 {\Large 第一項\\すべての自然的事物に生きることが属するか}
\end{center}

\begin{longtable}{p{21em}p{21em}}

{\huge A}{\scshape d primum sic proceditur}. Videtur quod omnium rerum
 naturalium sit vivere. Dicit enim philosophus, in VIII {\itshape Physic}., quod
 motus est {\itshape ut vita quaedam natura existentibus omnibus}. Sed omnes res
 naturales participant motum. Ergo omnes res naturales participant
 vitam.
 
&

第一の問題へ、議論は以下のように進められる。すべての自然的事物に生きるこ
 とが属すると思われる。理由は以下のとおり。哲学者は『自然学』8巻で「運
 動は、自然においてあるすべてのものどもにとって、ある種の生命としてある」
 と述べている\footnote{じっさいには、「生命のようなものであろうか」と疑
 問のかたちで述べられている。}。ところで、すべての自然的事物は運動を分有している。ゆえに、
 すべての自然的事物は生命を分有している。


\\



{\scshape 2 Praeterea}, plantae dicuntur vivere, inquantum habent in
 seipsis principium motus augmenti et decrementi. Sed motus localis est
 perfectior et prior secundum naturam quam motus augmenti et decrementi,
 ut probatur in VIII {\itshape Physic}. Cum igitur omnia corpora naturalia habeant
 aliquod principium motus localis, videtur quod omnia corpora naturalia
 vivant.
 
&


さらに、植物は、増大と減少という運動の根源を自らのうちに持つ限りで、生き
 ていると言われる。ところが、『自然学』8巻で証明されているように、場所
 的運動は、増大と減少という運動よりも、完全であり、本性上、それに先行す
 る。ゆえに、すべての自然的物体は、場所的運動の何らかの根源を持つので、
 すべての自然的物体は生きていると思われる。

\\



{\scshape 3 Praeterea}, inter corpora naturalia imperfectiora sunt
 elementa. Sed eis attribuitur vita, dicuntur enim {\itshape aquae vivae}. Ergo
 multo magis alia corpora naturalia vitam habent.
 
&

さらに、自然的物体の中で、元素はより不完全なものである。ところで、生命は
 それら元素に帰せられる。なぜなら、「生きている水」\footnote{流水のこと。}と言われるからである。ゆ
 えに、ましてや、元素以外の自然的物体は生命を持つ。

\\



{\scshape  Sed contra est} quod dicit Dionysius, {\scshape vi} cap.~{\itshape de Div.~Nom}.,
 quod {\itshape plantae secundum ultimam resonantiam vitae habent vivere}, ex quo
 potest accipi quod ultimum gradum vitae obtinent plantae. Sed corpora
 inanimata sunt infra plantas. Ergo eorum non est vivere.
 
&

しかし反対に、ディオニュシウスは『神名論』6章で「植物は、生命の最後の響
 きに従って生きることができている」と言っている。ここから、生命の最後の
 段階を植物が持っていることが理解されうる。ところが、魂のない物体は植物
 よりも下に位置する。ゆえに、それらには生きることが属さない。

\\



{\scshape Respondeo dicendum} quod ex his quae manifeste vivunt,
 accipere possumus quorum sit vivere, et quorum non sit vivere. Vivere
 autem manifeste animalibus convenit, dicitur enim in libro {\itshape de
 Vegetabilibus}, quod vita in animalibus manifesta est. Unde secundum
 illud oportet distinguere viventia a non viventibus secundum quod
 animalia dicuntur vivere. Hoc autem est in quo primo manifestatur vita,
 et in quo ultimo remanet.

 
&

解答する。以下のように言われるべきである。
私たちが、生きることが何に属し、何に属さないかを理解できるのは、明らかに
 生きているものどもに基づいてである\footnote{生命を定義するためには、ま
 ず、明らかに生きているものを認識することが必要。個別主義と言われる典型
 的な方法論。逆に、生きているものを見つけるには、まず、生きているとはど
 ういうことかがわかっていないといけない、と考えるのはmethodismと呼ばれる。}。ところで、生きることは、明らかに動物に適合す
 る。実際、『植物について』という書物の中で、「生命は、動物の中に
 明らかに示される」と言われている。したがって、動物が生きている
 と言われる点に即して、生きているものと生きていないものを区別しなければ
 ならない。そして、その点とは、そこにおいて生命が第一に明示され、また、そ
 こにおいて最後まで留まるところのものである。

\\



 Primo autem dicimus animal vivere, quando
 incipit ex se motum habere; et tandiu iudicatur animal vivere, quandiu
 talis motus in eo apparet; quando vero iam ex se non habet aliquem
 motum, sed movetur tantum ab alio tunc dicitur animal mortuum, per
 defectum vitae. 

 
&

さて、私たちが動物を生きていると言うのは、それが、自らに基づいて運動を持
 ち始めるときであり、そして、そのような運動がそのものに現れているあいだ、その動物は
 生きていると判断される。他方、自分からの運動を何も持たず、ただ
 他のものによって動かされるだけであるとき、そのときには、その動物は生命
 を欠くことによって死んだと言われる。


\\


Ex quo patet quod illa proprie sunt viventia, quae
 seipsa secundum aliquam speciem motus movent; sive accipiatur motus
 proprie, sicut motus dicitur actus imperfecti, idest existentis in
 potentia; sive motus accipiatur communiter, prout motus dicitur actus
 perfecti, prout intelligere et sentire dicitur moveri, ut dicitur in
 III {\itshape de Anima}. Ut sic viventia dicantur quaecumque se agunt ad motum vel
 operationem aliquam, ea vero in quorum natura non est ut se agant ad
 aliquem motum vel operationem, viventia dici non possunt, nisi per
 aliquam similitudinem.
 
&

以上のことから次のことが明らかである。
すなわち、自分自身を、運動の何らかの種に
 即して動かすものが、固有の意味で(proprie)[=厳密に]生きているものであ
 る。[ただしこれには二つの場合があり]運動が固有の意味で
 理解される場合と、共通的に理解される場合がある。前者は、不完全なもの、
 つまり可能態にあるものの現実態が運動と言われる場合であり、後者は、『デ・
 アニマ』第3巻で述べられるように、知性認識することや感覚することが、「動
 かされる」と言われるときのように、完全なものの現実態が運動と言われる場
 合である。その結果、この後者の意味で、自分自身を、運動やなんらかの働き
 へともたらすものが、何であれ、生きていると言われる。他方、自分の本性の
 中に、自分をなんらかの運動や働きへともたらすようなものがないものは、な
 んらかの類似によるのでない限り、生きているとは言われえない。




\\



{\scshape Ad primum ergo dicendum} quod verbum illud philosophi potest
 intelligi vel de motu primo, scilicet corporum caelestium; vel de motu
 communiter. Et utroque modo motus dicitur quasi vita corporum
 naturalium, per similitudinem; et non per proprietatem. Nam motus caeli
 est in universo corporalium naturarum, sicut motus cordis in animali,
 quo conservatur vita. Similiter etiam quicumque motus naturalis hoc
 modo se habet ad res naturales, ut quaedam similitudo vitalis
 operationis. Unde, si totum universum corporale esset unum animal, ita
 quod iste motus esset a movente intrinseco, ut quidam posuerunt,
 sequeretur quod motus esset vita omnium naturalium corporum.
 
&

第一異論に対しては、それゆえ、以下のように言われるべきである。
哲学者のあの言葉は、第一の運動、すなわち、諸天体の運動についてか、あるい
 は、運動について共通的にか、いずれかのかたちで理解可能である。そして、
 どちらの意味でも、運動は、類似によって、あたかも自然的諸物体の生命のよ
 うなものである、と語られているのであって、固有の意味によるのではない。
つまり、自然的諸物体の世界のなかで、天の運動は、ちょうど、生命がそれによっ
 て保たれるところの心臓の運動が、動物の中にあるのと同じようにある。
同様にまた、どんな自然的運動も、ある種の生命的な働きの類似というかたちで、自然的
 諸事物に関係する。したがって、ある人々が考えたように、もしも物体的世界
 の全体が一つの動物であり、その運動が、内在する動者によってあった
 ならば、運動は、すべての自然的諸事物の生命であることが帰結したであ
 ろう。


\\



{\scshape Ad secundum dicendum} quod corporibus gravibus et levibus non
 competit moveri, nisi secundum quod sunt extra dispositionem suae
 naturae, utpote cum sunt extra locum proprium, cum enim sunt in loco
 proprio et naturali, quiescunt. Sed plantae et aliae res viventes
 moventur motu vitali, secundum hoc quod sunt in sua dispositione
 naturali, non autem in accedendo ad eam vel in recedendo ab ea, imo
 secundum quod recedunt a tali motu, recedunt a naturali
 dispositione. Et praeterea, corpora gravia et levia moventur a motore
 extrinseco, vel generante, qui dat formam, vel removente prohibens, ut
 dicitur in VIII {\itshape Physic}., et ita non movent seipsa, sicut corpora
 viventia.
 
&

第二異論に対しては、以下のように言われるべきである。
重い物体や軽い物体に運動が属するのは、固有の場所の外にあるときなど、それらが自らの本性
 の状態の外にある場合に限る。なぜなら、固有で本性
 的な場所にあるとき、それらは静止しているからである。
これに対して、植物やその他の生物が生命の運動によって動かされるのは、自ら
 の本性的な状態のうちにあるかぎりにおいてであり、そのような状態へ近づく
 ことによって、あるいはそれから離れることによってではない。むしろそのよ
 うな運動から離れるかぎり、それらは本性的な状態から離れる。
さらにまた、重い物体や軽い物体は、『自然学』8巻で言われるように、外的な
 動者、つまり形相を与える生み出すものや、障害を取り除くものによって動か
 されているのであり、生きている物体のように、自分自身を動かしているので
 はない。
 

\\



{\scshape Ad tertium dicendum} quod aquae vivae dicuntur, quae habent
 continuum fluxum, aquae enim stantes, quae non continuantur ad
 principium continue fluens, dicuntur mortuae, ut aquae cisternarum et
 lacunarum. Et hoc dicitur per similitudinem, inquantum enim videntur se
 movere, habent similitudinem vitae. Sed tamen non est in eis vera ratio
 vitae, quia hunc motum non habent a seipsis, sed a causa generante eas;
 sicut accidit circa motum aliorum gravium et levium.
 
&

第三異論に対しては、以下のように言われるべきである。「生きている」と言わ
れる水は、継続的な流れをもつものである。なぜなら、貯水槽や水たまりの水のように、絶えず流れ出す水源につ
ながっていない止まった水は、「死んでいる」と言われるからである。そしてこれは、類似によって言われる。なぜな
ら、自分を動かしているように見える限りで、生命の類似を持っているからであ
る。しかし、それらの中に、真の生命という性格があるわけではない。なぜなら、こ
の運動を、自分自身からもつのではなく、それらを生み出す原因から持つからで
ある。これは、他の重いものや軽いものの運動に関して起こることと同様である。




\end{longtable}
\newpage


\rhead{a.~2}
\begin{center}
 {\Large {\bf ARTICULUS SECUNDUS}}\\
 {\large UTRUM VITA SIT QUAEDAM OPERATIO}\\
 {\footnotesize Infra, q.~54, a.~1, ad 2; III {\itshape Sent.}, d.~35,
 q.~1, a.~1, ad 1; IV, d.~49, q.~1, a.~2, qu$^a$ 3; I {\itshape SCG.},
 cap.~98; {\itshape De Div.~Nom.}, cap.~6, lect.~1.}\\
 {\Large 第二項\\生命はなんらかの働きか}
\end{center}

\begin{longtable}{p{21em}p{21em}}


{\huge A}{\scshape d secundum sic proceditur}. Videtur quod vita sit
 quaedam operatio. Nihil enim dividitur nisi per ea quae sunt sui
 generis. Sed vivere dividitur per operationes quasdam, ut patet per
 philosophum in II libro {\itshape De Anima}, qui distinguit vivere per quatuor,
 scilicet alimento uti, sentire, moveri secundum locum, et
 intelligere. Ergo vita est operatio quaedam.


&

第二項の問題へ、議論は以下のように進められる。
生命はなんらかの働きであると思われる。理由は以下のとおり。
なにものも、自分と同類のものによってでなければ分割されない。
ところが、『デ・アニマ』2巻の哲学者によって明らかなとおり、生きることは、
 ある種の働きによって分割される。つまり彼は、栄養を摂ること、感覚するこ
 と、場所的に動くこと、知性認識することという4つのことによって、生きるこ
 とを区別している。ゆえに、生命はなんらかの働きである。


\\



{\scshape 2 Praeterea}, vita activa dicitur alia esse a
 contemplativa. Sed contemplativi ab activis non diversificantur nisi
 secundum operationes quasdam. Ergo vita est quaedam operatio.


&

さらに、活動的生は、観照的生とは異なると言われる。ところが、観照的な人々
 は、活動的な人々から、なんらかの働きにおいてでなければ異なるものとされ
 ない。ゆえに、生命は、なんらかの働きである。

\\



{\scshape 3 Praeterea}, cognoscere Deum est operatio quaedam. Haec autem
 est vita, ut patet per illud Ioan. {\scshape xvii}, {\itshape haec est autem vita aeterna,
 ut cognoscant te solum verum Deum}. Ergo vita est operatio.


&

さらに、神を認識することは一種の働きである。ところが、かの『ヨハネによる
 福音書』17章「唯一で真なる神であるあなたを認識することが、永遠の生命で
 ある」\footnote{「永遠の命とは、唯一のまことの神であられるあなたと、あ
 なたのお遣わしになったイエス・キリストを知ることです」(17:3)}によれば、
 これは、生命である。ゆえに、生命は働きである。

\\



{\scshape  Sed contra est} quod dicit philosophus, in II {\itshape De Anima},
 {\itshape vivere viventibus est esse}.


&

しかし反対に、哲学者は『デ・アニマ』2巻で「生きることは、生きるものにとっ
 て、存在することである」と述べている。


\\



{\scshape Respondeo dicendum} quod, sicut ex dictis patet, intellectus
 noster, qui proprie est cognoscitivus quidditatis rei ut proprii
 obiecti, accipit a sensu, cuius propria obiecta sunt accidentia
 exteriora. Et inde est quod ex his quae exterius apparent de re,
 devenimus ad cognoscendam essentiam rei. Et quia sic nominamus aliquid
 sicut cognoscimus illud, ut ex supradictis patet, inde est quod
 plerumque a proprietatibus exterioribus imponuntur nomina ad
 significandas essentias rerum. Unde huiusmodi nomina quandoque
 accipiuntur proprie pro ipsis essentiis rerum, ad quas significandas
 principaliter sunt imposita, aliquando autem sumuntur pro
 proprietatibus a quibus imponuntur, et hoc minus proprie. Sicut patet
 quod hoc nomen {\itshape corpus} impositum est ad significandum quoddam
 genus substantiarum, ex eo quod in eis inveniuntur tres dimensiones, et
 ideo aliquando ponitur hoc nomen {\itshape corpus} ad significandas
 tres dimensiones, secundum quod corpus ponitur species quantitatis.


&

解答する。以下のように言われるべきである。
すでに述べられたことから明らかなとおり、私たちの知性は、固有対象として事
物の何性を認識しうるものなのだが、感覚から(認識内容を)を受け取るのであ
 り、そしてその感覚の固有対象は、外的な附帯性である。
したがって、私たちは、事物について外的に現れているものどもから出発して、
 事物の本質を認識することへ至る。また、前に述べられたことから明らかなと
 おり、私たちは、何かを名付けるのに、それを認識するとおりに名付ける。
したがって、多くの場合、諸事物の本質を表示するために、外的な固有性から名
 が付けられる。
このことから、そのような名称は、本来的に、もともとそれを表示するために付け
 られた、諸事物の本質それ自体[を表示するもの]として理解されることもあ
 れば、そこから付けられた固有性[を表示するもの]として受け取られること
 もある。この後者は本来的な用法でない。
たとえば、「物体」というこの名称は、その中に三つの次元が見出されることに
 基づいて、実体のある類を表示するために付けられたのだが、またそれゆえに、物
 体が量の種とされる場合には、この「物体」という名称が、三次元を表示する
 ために使われる。



\\

Sic ergo dicendum est et de vita. Nam vitae nomen sumitur ex quodam
 exterius apparenti circa rem, quod est movere seipsum, non tamen est
 impositum hoc nomen ad hoc significandum, sed ad significandam
 substantiam cui convenit secundum suam naturam movere seipsam, vel
 agere se quocumque modo ad operationem. Et secundum hoc, {\itshape
 vivere} nihil aliud est quam esse in tali natura, et {\itshape vita}
 significat hoc ipsum, sed in abstracto; sicut hoc nomen {\itshape
 cursus} significat ipsum {\itshape currere} in abstracto. Unde
 {\itshape vivum} non est praedicatum accidentale, sed
 substantiale. Quandoque tamen {\itshape vita} sumitur minus proprie pro
 operationibus vitae, a quibus nomen vitae assumitur; sicut dicit
 philosophus, IX {\itshape Ethic}., quod {\itshape vivere principaliter
 est sentire vel intelligere}.


&

それゆえ、生命についても同じように言われるべきである。
すなわち、「生命」という名称は、自分自身を動かすという、事物に関する外的
 な現れから取られているが、しかし、この名称は、それを表示するために付け
 られたのではなく、むしろ、自らの本性にしたがって自らを動かす(movere)こと、ある
 いは、どんなかたちであれ、自らを働きへと導く(agere)ことが適合する実体
 を表示するために付けられた。そしてこの限りでは、「生きること」は、その
 ような本性において「存在すること」に他ならず、「生命」も、このことを表
 示する。ただし、抽象的にであって、それはちょうど、「走り」という名称が、「走ること」を抽
 象的に表示するのと同様である。
したがって、「生きている」は、附帯的述語ではなく、実体的述語である。
しかし、時には、本来的ではない意味で、「生命」は、生命という名称がそこか
 ら取られている生命の働きを[表示する
 ものとして]受け取られる。ちょうど、哲学者が『ニコマコス倫理学』9巻で「生きること
 は、主要には、感覚することまたは知性認識することである」と述べるように。






\\



{\scshape Ad primum ergo dicendum} quod philosophus ibi accipit {\itshape vivere}
 pro operatione vitae. -- Vel dicendum est melius, quod sentire et
 intelligere, et huiusmodi, quandoque sumuntur pro quibusdam
 operationibus; quandoque autem pro ipso esse sic operantium. Dicitur
 enim IX {\itshape Ethic}., quod {\itshape esse\footnote{Piana版では vivere} est sentire vel intelligere}, idest habere
 naturam ad sentiendum vel intelligendum. Et hoc modo distinguit
 philosophus vivere per illa quatuor. Nam in istis inferioribus quatuor
 sunt genera viventium. Quorum quaedam habent naturam solum ad utendum
 alimento, et ad consequentia, quae sunt augmentum et generatio; quaedam
 ulterius ad sentiendum, ut patet in animalibus immobilibus, sicut sunt
 ostrea; quaedam vero, cum his, ulterius ad movendum se secundum locum,
 sicut animalia perfecta, ut quadrupedia et volatilia et huiusmodi;
 quaedam vero ulterius ad intelligendum, sicut homines.


&

第一異論に対しては、それゆえ、以下のように言われるべきである。
哲学者は、そこで、「生きること」を生命の働きとして解している。
あるいは、以下のように言う方がもっとよい。感覚すること、知性認識すること、
 その他このようなことは、なんらかの働きとして理解されることもあれば、そ
 のように働くものの存在そのものとして理解されることもある。たとえば、
 『ニコマコス倫理学』9巻で「存在することは、感覚すること、あるいは知性認
 識することである」と述べられているが、これは、感覚するための、あるいは
 知性認識するための本性をもつこと、という意味である。そして、哲学者はこ
 のようにして、生きることを、かの四つのものによって区別する。
すなわち、下位のものたちの中には、生きるものの四つの類がある。そのあるも
 のは、ただ栄養を摂取するため、そして、それに伴うこと、つまり、成長と生
 成のための本性だけをもつ。またあるものは、それに加えて感覚するための本
 性をもつ。これは、貝のように、動かない動物において明らかである。また
 あるものは、これらとともに、さらに自らを場所的に動かすための本性をもつ。
たとえば、四本足のものや羽があるものなどのような、完全な動物のように。ま
 たあるものは、人間のように、さらに知性認識するための本性をもつ。




\\



{\scshape Ad secundum dicendum} quod opera vitae dicuntur, quorum
 principia sunt in operantibus, ut seipsos inducant in tales
 operationes. Contingit autem aliquorum operum inesse hominibus non
 solum principia naturalia, ut sunt potentiae naturales; sed etiam
 quaedam superaddita, ut sunt habitus inclinantes ad quaedam operationum
 genera quasi per modum naturae, et facientes illas operationes esse
 delectabiles. Et ex hoc dicitur, quasi per quandam similitudinem, quod
 illa operatio quae est homini delectabilis, et ad quam inclinatur, et
 in qua conversatur, et ordinat vitam suam ad ipsam, dicitur vita
 hominis, unde quidam dicuntur agere vitam luxuriosam, quidam vitam
 honestam. Et per hunc modum vita contemplativa ab activa
 distinguitur. Et per hunc etiam modum cognoscere Deum dicitur vita
 aeterna.


&

第二異論に対しては、以下のように言われるべきである。
生命の業と言われるのは、それの根源が働くものの中にあって、自分自身をそ
 のような働きへと導くようなものである。
ところが、ある業の根源が人間に内在する場合に、自然本性的能力のような自然
 本性的な根源だけでなく、たとえば、あたかも本性のかたちによるかのように
 して、働きのある類へと傾かせ、その働きを快適なものにする、付加された根源である場合もある。習態(habitus)がそうであるように。そして、このことか
 ら、ある種の類似によって、人間にとって快適であり、それへと傾かされ、そ
 こに住まい、自分の生命をそれへと秩序付けるところの働きが、人間の生命
 (vita hominis)と言われる。このことから、ある人々は贅沢な生命を送ると言
 われ、ある人々は、誠実な生命を送ると言われる。そしてこ
 のようにして、観照的な生命が、活動的な生命から区別される。
 さらにまたこのことによって、神を認識することが、永遠の生命と言わ
 れる。

\\






Unde patet solutio ad tertium.


&

このことから、第三異論にたいする解答は明らかである。





\end{longtable}
\newpage

\rhead{a.~3}
\begin{center}
 {\Large {\bf ARTICULUS TERTIUS}}\\
 {\large UTRUM DEO CONVENIAT VITA}\\
 {\footnotesize I {\itshape SCG.}, cap.~97, 98; IV, cap.~11; in
 {\itshape Ioan}., cap.~14, lect.~2; XII {\itshape Metaphys.}, lect.~8.}\\
 {\Large 第三項\\神に生命は適合するか}
\end{center}

\begin{longtable}{p{21em}p{21em}}


{\huge A}{\scshape d tertium sic proceditur}. Videtur quod Deo non
conveniat vita. Vivere enim dicuntur aliqua secundum quod movent seipsa,
ut dictum est. Sed Deo non competit moveri. Ergo neque vivere.


&
第三項の問題へ、議論は以下のように進められる。
神に生命は適合しないと思われる。理由は以下の通り。
あるものが生きていると言われるのは、すでに述べられたとおり、自分自身を動
 かす点においてである。ところが、動かされることは神に適合しない。ゆえに、
 生きることも適合しない。

\\


{\scshape 2 Praeterea}, in omnibus quae vivunt, est accipere aliquod
vivendi principium, unde dicitur in II {\itshape de Anima}, quod
{\itshape anima est viventis corporis causa et principium}. Sed Deus non
habet aliquod principium. Ergo sibi non competit vivere.


&

さらに、生きているすべてのものにおいて、生きる根源である何かが理解される。
 このことから、『デ・アニマ』2巻で「魂とは、生きている身体の原因であり根
 源である」と言われる。ところが、神がなんらかの根源をもつことはない。ゆ
 えに、神に生きることは適合しない。

\\


{\scshape 3 Praeterea}, principium vitae in rebus viventibus quae apud
nos sunt, est anima vegetabilis, quae non est nisi in rebus
corporalibus. Ergo rebus incorporalibus non competit vivere.


&

さらに、私たちのもとにある生きている事物の生命の根源とは、植物的魂であり、
 これは、物体的事物の中にしかない。ゆえに、非物体的事物に生きることは適
 合しない。

\\


{\scshape Sed contra est} quod dicitur in Psalmo {\scshape lxxxiii},
{\itshape cor meum et caro mea exultaverunt in Deum vivum}.


&


しかし反対に、『詩編』83「私の心と肉は、生きている神へ向かって飛び上がっ
 た」\footnote{「命の神に向かって、私の身も心も叫びます」(84:3)}と言われている。

\\


{\scshape Respondeo dicendum} quod vita maxime proprie in Deo est. Ad
cuius evidentiam, considerandum est quod, cum vivere dicantur aliqua
secundum quod operantur ex seipsis, et non quasi ab aliis mota; quanto
perfectius competit hoc alicui, tanto perfectius in eo invenitur
vita. In moventibus autem et motis tria per ordinem inveniuntur. Nam
primo, finis movet agentem; agens vero principale est quod per suam
formam agit; et hoc interdum agit per aliquod instrumentum, quod non
agit ex virtute suae formae, sed ex virtute principalis agentis; cui
instrumento competit sola executio actionis. 


&
答えて言わなければならない。
生命は、最大限に固有に神の中にある。これを明らかにするためには、以下のこ
 とが考えられるべきである。あるものは、他のものども
 によって動かされているかのようにではなく、自分自身から働くという点で、
生きると言われるから、このことがより完全に適合するほど、より完全にそのも
 のの中に生命が見いだされる。
ところで、動くものや動かされるもののなかに、三つのことが、順番に見いださ
 れる。第一に、目的が作用者を動かす。そして、主要な作用者とは、自分の形相
 によって作用するものである。そして時として、これ[主要
 な作用者]は、何らかの道具によって作用する。それ[道具]は、自分の形相
 の力によってではなく、主要な作用者の力によって作用する。そしてこの道具
 には、作用の遂行だけが属する。


\\

Inveniuntur igitur quaedam,
quae movent seipsa, non habito respectu ad formam vel finem, quae inest
eis a natura, sed solum quantum ad executionem motus, sed forma per quam
agunt, et finis propter quem agunt, determinantur eis a natura. Et
huiusmodi sunt plantae, quae secundum formam inditam eis a natura,
movent seipsas secundum augmentum et decrementum. 

&

それゆえ、形相や目的が自然本性によって内在するため、それら[形相や目的]との関係をもた
 ないで自分を動かすようなもの、つまり、たんに運動の遂行にかんしてのみ自らを動かし、それによっ
 て作用するところの形相や、そのために作用するところの目的は、それらに自
 然本性によって決定されているようなものが見いだされる。
植物はそのようなものであり、自然本性によってそれらに与えられた形相にした
 がって、増大と減少という点で、自分を動かしている。

\\



Quaedam vero ulterius
movent seipsa, non solum habito respectu ad executionem motus, sed etiam
quantum ad formam quae est principium motus, quam per se acquirunt. Et
huiusmodi sunt animalia, quorum motus principium est forma non a natura
indita, sed per sensum accepta. Unde quanto perfectiorem sensum habent,
tanto perfectius movent seipsa. Nam ea quae non habent nisi sensum
tactus, movent solum seipsa motu dilatationis et constrictionis, ut
ostrea, parum excedentia motum plantae. Quae vero habent virtutem
sensitivam perfectam, non solum ad cognoscendum coniuncta et tangentia,
sed etiam ad cognoscendum distantia, movent seipsa in remotum motu
processivo. 


&

他方、あるものどもは、更に自分を動かし、運動の遂行への関係をもつだけでな
 く、運動の根源である形相を自分で獲得するため、その形相への関係の点でも、
 自分を動かす。そのようなものとは動物であり、それらの運動の根源は、自然
 本性によって与えられた形相ではなく、感覚を通して受け取られた形相である。
 このため、より完全な感覚を持つほど、より完全に自分を動かす。たとえば、
 触覚しか持たないもの、たとえば貝は、伸長と収縮の運動によっての
 み自分を動かし、植物の運動をほとんど超えることがない。
これに対して、結びついて接触しているものを認
 識するだけでなく、隔たっているものさえも認識するような完全な感覚能力を
 持つものは、遠くのものに向かって、前進運動によって自分を動かす。


\\


Sed quamvis huiusmodi animalia formam quae est principium
motus, per sensum accipiant, non tamen per seipsa praestituunt sibi
finem suae operationis, vel sui motus; sed est eis inditus a natura,
cuius instinctu ad aliquid agendum moventur per formam sensu
apprehensam. Unde supra talia animalia sunt illa quae movent seipsa,
etiam habito respectu ad finem, quem sibi praestituunt. Quod quidem non
fit nisi per rationem et intellectum, cuius est cognoscere proportionem
finis et eius quod est ad finem, et unum ordinare in alterum. Unde
perfectior modus vivendi est eorum quae habent intellectum, haec enim
perfectius movent seipsa. 

&

しかし、このような動物が、運動の根源である形相を自分で獲得するとしても、
 自分の働きの目的を、自分で自分に設定するわけではない。むしろそれは、自
 然本性によって、それに与えられるのであり、それへの本能によって、感覚に
 よって捉えられた形相を通して、何かをなすことへと動かされる。それゆえ、
 このような動物の上位に、目的にかんしても自らを動かし、目的を自分に設定
 するものが存在する。このことは、目的と、目的に関係するものとの関係を認
 識し、一方を他方に秩序付けることができる、理性と知性によらなくてはなさ
 れない。したがって、知性を持つものどもには、より完全な生き方が属する。
 それらは、より完全に自分を動かすからである。

\\


Et huius est signum, quod in uno et eodem
homine virtus intellectiva movet potentias sensitivas; et potentiae
sensitivae per suum imperium movent organa, quae exequuntur motum. Sicut
etiam in artibus, videmus quod ars ad quam pertinet usus navis, scilicet
ars gubernatoria, praecipit ei quae inducit formam navis, et haec
praecipit illi quae habet executionem tantum, in disponendo
materiam. 

&


そして、一人の同一の人間の
 中で、知性の力が感覚能力を動かし、感覚能力が、自らの命令によって諸器官
 を動かし、諸器官が運動を遂行するということは、以上のことの象徴である。
さらにまた、ちょうど技術においても、船の使用が属する技術、すなわち、航海
 術が、船の形[=形相]をもたらす技術に命令し、この技術が、材料[=質料]を配置する
 ことにおいて、たんに実行のみを行う技術に命令することを私たちは見る。

\\


Sed quamvis intellectus noster ad aliqua se agat, tamen aliqua
sunt ei praestituta a natura; sicut sunt prima principia, circa quae non
potest aliter se habere, et ultimus finis, quem non potest non
velle. Unde, licet quantum ad aliquid moveat se, tamen oportet quod
quantum ad aliqua ab alio moveatur. Illud igitur cuius sua natura est
ipsum eius intelligere, et cui id quod naturaliter habet, non
determinatur ab alio, hoc est quod obtinet summum gradum vitae. Tale
autem est Deus. Unde in Deo maxime est vita. Unde philosophus, in XII
{\itshape Metaphys}., ostenso quod Deus sit intelligens, concludit quod habeat
vitam perfectissimam et sempiternam, quia intellectus eius est
perfectissimus, et semper in actu.


&


しかし、私たちの知性は、あるものに対しては自ら作用するが、別のあるものは、
 自然本性によって、知性に設定されている。たとえば、第一の諸原理や究極目
 的がそのようなものであり、前者に関して、知性は別様に関係することができ
 ず、また後者を意志しないことができない。したがって、[私たちの知性は]
 あるものに関しては、自ら作用するが、別のあるものに関しては、他者によっ
 て動かされる。ゆえに、知性認識そのものが、自分の本性であるようなもの、
 そして、それが自然本性的に持つものが、他者によってそれに限定されているの
 でないようなものは、生命の最高の段階をもつものである。ところで、そのよ
 うなものは神である。したがって、神の中に、最大限に生命がある。このこと
 から哲学者は『形而上学』12巻で、神が知性認識するものを明示することによっ
 て、神が最も完全で永続する生命を持つことを結論している。なぜなら、神の
 知性は、最も完全で、常に現実態にあるからである。

\\


{\scshape Ad primum ergo dicendum} quod, sicut dicitur in IX {\itshape Metaphys}.,
duplex est actio, una, quae transit in exteriorem materiam, ut
calefacere et secare; alia, quae manet in agente, ut intelligere,
sentire et velle. Quarum haec est differentia, quia prima actio non est
perfectio agentis quod movet, sed ipsius moti; secunda autem actio est
perfectio agentis. Unde, quia motus est actus mobilis, secunda actio,
inquantum est actus operantis, dicitur motus eius; ex hac similitudine,
quod, sicut motus est actus mobilis, ita huiusmodi actio est actus
agentis; licet motus sit actus imperfecti, scilicet existentis in
potentia, huiusmodi autem actio est actus perfecti, idest existentis in
actu, ut dicitur in III {\itshape de Anima}. Hoc igitur modo quo intelligere est
motus, id quod se intelligit, dicitur se movere. Et per hunc modum etiam
Plato posuit quod Deus movet seipsum, non eo modo quo motus est actus
imperfecti.


&

第一異論に対しては、それゆえ、以下のように言われるべきである。
『形而上学』9巻で言われるように、作用には二通りある。一つは、たとえば熱
 することや切ることのように、外部の質料へと出て行くものであり、もう一つ
 は、知性認識すること、感覚すること、意志することのように、作用者の中に
 留まるものである。これらの違いは、一つ目の作用が、動かす作用者の完成で
 なく、動かされる物自体の完成であるのに対し、二つ目の作用は、作用者の完
 成である。したがって、運動は、動かされうるものの現実態だから、二つ目の
 作用は、働くものの現実態である限りで、それ[=働くもの]の運動と言われ
 る。この類似、すなわち、運動が動かされうるものの現実態であるように、そ
 のような作用が作用者の現実態であるという類似に基づいて、運動は不完全な
 もの、つまり、可能態にあるものの現実態ではあるが、このような作用は、
 『デ・アニマ』3巻で言われるように、完全
 なもの、すなわち現実態にあるものの現実態である。
ゆえに、知性認識することが運動であるようなこの意味で、自分を知性認識する
 ものは、自分を動かすと言われる。そしてこの意味で、プラトンもまた、神は
 自分自身を動かすと考えた。しかしこの運動は、不完全なものの現実態という
 意味ではない。

\\


{\scshape Ad secundum dicendum} quod, sicut Deus est ipsum suum esse et
suum intelligere, ita est suum vivere. Et propter hoc, sic vivit, quod
non habet vivendi principium.


&

第二異論に対しては、以下のように言われるべきである。
神は、自らの存在であり自らの知性認識であるように、自らの生きることでもあ
 る。このため、生きることの根源を持たないかたちで、神は生きている。


\\


{\scshape Ad tertium dicendum} quod vita in istis inferioribus recipitur
in natura corruptibili, quae indiget et generatione ad conservationem
speciei, et alimento ad conservationem individui. Et propter hoc, in
istis inferioribus non invenitur vita sine anima vegetabili. Sed hoc non
habet locum in rebus incorruptibilibus.


&

第三異論に対しては、以下のように言われるべきである。
生命は、この下位のものどものに、可滅的な本性の中に受け取られている。その
 ような本性は、種の保存のために生成を、個の保存のために食物を必要とする。
 このため、これら下位のものどもの中に、植物的魂がなければ見出されない。
 しかしこのことは、不滅的物体には当てはまらない。



\end{longtable}
\newpage






\rhead{a.~4}
\begin{center}
 {\Large {\bf ARTICULUS QUARTUS}}\\
 {\large UTRUM OMNIA SINT VITA IN DEO}\\
 {\footnotesize IV {\itshape SCG}, c.~13; {\itshape De Verit.}, q.~4,
 a.~8; in {\itshape Ioan.}, c.~1, lect.~2.}\\
 {\Large 第四項\\万物は神の中で生きているか}
\end{center}

\begin{longtable}{p{21em}p{21em}}

{\huge A}{\scshape d quartum sic proceditur}. Videtur quod non
 omnia sint vita in Deo. Dicitur enim {\itshape Act}.~{\scshape xvii}, {\itshape in ipso vivimus,
 movemur et sumus}. Sed non omnia in Deo sunt motus. Ergo non omnia in
 ipso sunt vita.


&

第四項の問題へ、議論は以下のように進められる。
万物が神の中で生きているわけではないと思われる。理由は以下のとおり。
『使徒言行録』17章で「私たちはそれの中で生き、動き、存在する」
 \footnote{「我らは神の中に生き、動き、存在する」(17:28)}と言われている。
 ところで、神の中で、すべてのものが運動しているわけではない。ゆえに、す
 べてがその中で生きているわけではない。

\\




{\scshape 2 Praeterea}, omnia sunt in Deo sicut in primo
 exemplari. Sed exemplata debent conformari exemplari. Cum igitur non
 omnia vivant in seipsis, videtur quod non omnia in Deo sint vita.


&

さらに、万物は、第一の範型の中に在るようにして神の中に在る。ところで、範
 型から生まれたものは、範型に一致しなければならない。ゆえに、すべてのも
 のがそれ自身において生きているわけではないのだから、すべてのものが神の
 中で生きているわけではないと思われる。


\\




{\scshape 3 Praeterea}, sicut Augustinus dicit in libro
 {\itshape de vera Relig}., substantia vivens est melior qualibet substantia non
 vivente. Si igitur ea quae in seipsis non vivunt, in Deo sunt vita,
 videtur quod verius sint res in Deo quam in seipsis. Quod tamen videtur
 esse falsum, cum in seipsis sint in actu, in Deo vero in potentia.


&


さらに、アウグスティヌスは『真の宗教について』という書物の中で、生きてい
 る実体は、生きていないどんな実体よりも善いと述べる。ゆえに、もし、それ
 自身において生きていないものが、神の中で生きているとすれば、事物は、そ
 れ自身においてよりも、神の中で、より真のものであることになる。しかしこ
 れは、それ自身においては現実態に、他方神の中には可能態にあるのだ
 から、偽だと思われる。

\\




{\scshape 4 Praeterea}, sicut sciuntur a Deo bona, et ea
 quae fiunt secundum aliquod tempus; ita mala, et ea quae Deus potest
 facere, sed nunquam fiunt. Si ergo omnia sunt vita in Deo, inquantum
 sunt scita ab ipso, videtur quod etiam mala, et quae nunquam fiunt,
 sunt vita in Deo, inquantum sunt scita ab eo. Quod videtur
 inconveniens.


&


さらに、善や、ある時刻に生じることが神によって知られるよ
 うに、悪や、神が作ることができるが決して作られないものもまた[神によっ
 て知られる]。ゆえに、もし万物が、神によって知られているという意味で神
 の中で生きているならば、悪もまた、そして、決して作られないものも、神に
 よって知られている限りで、神の中で生きていることになる。これは不都合だ
 と思われる。

\\




{\scshape  Sed contra est} quod dicitur Ioan. I, {\itshape quod
 factum est, in ipso vita erat}. Sed omnia praeter Deum facta sunt. Ergo
 omnia in Deo sunt vita.


&

しかし反対に、『ヨハネによる福音書』1章で「作られたものは、それの中で生
 命であった」\footnote{このままの文章はない。「万物は言によって成った。
 成ったもので、言によらずに成ったものは何一つなかった。言のうちに命
 があった」(1:3-4)}と言われている。ところが、神以外のすべてのものは、作られた。
 ゆえに、万物は神の中で生命である。

\\




{\scshape Respondeo dicendum} quod, sicut dictum est,
 vivere Dei est eius intelligere. In Deo autem est idem intellectus et
 quod intelligitur, et ipsum intelligere eius. Unde quidquid est in Deo
 ut intellectum, est ipsum vivere vel vita eius. Unde, cum omnia quae
 facta sunt a Deo, sint in ipso ut intellecta, sequitur quod omnia in
 ipso sunt ipsa vita divina.


&


解答する。以下のように言われるべきである。
すでに述べられたとおり、神の「生きること」は、その「知性認識すること」で
 ある。ところで、神の中で、知性と、知性認識されるものと、神の「知性認識
 すること」とは同一である。したがって、なんであれ知性認識された
 ものとして神の中に存在するものは、「生きること」そのものであり、神の生
 命である。したがって、神によって作られたすべてのものは、知性認識された
 ものとしてその中に存在するのだから、万物は、神の中で、神の生命そのもの
 であることが帰結する。

\\




{\scshape Ad primum ergo dicendum} quod creaturae in Deo
 esse dicuntur dupliciter. Uno modo, inquantum continentur et
 conservantur virtute divina, sicut dicimus ea esse in nobis, quae sunt
 in nostra potestate. Et sic creaturae dicuntur esse in Deo, etiam prout
 sunt in propriis naturis. Et hoc modo intelligendum est verbum apostoli
 dicentis, {\itshape in ipso vivimus, movemur et sumus}, quia et nostrum vivere, et
 nostrum esse, et nostrum moveri causantur a Deo. Alio modo dicuntur res
 esse in Deo sicut in cognoscente. Et sic sunt in Deo per proprias
 rationes, quae non sunt aliud in Deo ab essentia divina. Unde res,
 prout sic in Deo sunt, sunt essentia divina. Et quia essentia divina
 est vita, non autem motus, inde est quod res, hoc modo loquendi, in Deo
 non sunt motus, sed vita.


&

第一異論に対しては、それゆえ、以下のように言われるべきである。
被造物が神の中に存在するということは、二通りの意味で語られる。
一つには、ちょうど私たちの権能の中にあるものは、私たちの中にあると言うよ
 うに、神の力によって含まれ、保存されている限りで、そう言われる。
そしてこの意味では、被造物は、それ固有の本性において在るものとしてですら、
 神の中に存在すると言われる。そして、使徒の「私達はその中で生き、動き、
 存在する」という言葉は、この意味で理解されるべきである。なぜなら、私た
 ちが生きること、私たちが存在すること、私たちが動くことは、神が原因となっ
 て生じるからである。また別の意味では、諸事物が、認識者の中に在るという
 意味で、神の中に存在すると言われる。そしてこの意味で、[万物は]固有の
 性格(propriae rationes:それ自身の理念)によって神の中に存在するが、この固有の性格は、神
 の中で、神の本質と別のものではない。したがって、諸事物は、このようなかたちで神の
 中に存在するものとしては、神の本質である。そして、神の本質は生命であり、
 運動ではないので、この意味で言われる諸事物は、神の中で、運動ではなく生
 命である。


\\




{\scshape Ad secundum dicendum} quod exemplata oportet
 conformari exemplari secundum rationem formae, non autem secundum modum
 essendi. Nam alterius modi esse habet quandoque forma in exemplari et
 in exemplato, sicut forma domus in mente artificis habet esse
 immateriale et intelligibile, in domo autem quae est extra animam,
 habet esse materiale et sensibile. Unde et rationes rerum quae in
 seipsis non vivunt, in mente divina sunt vita, quia in mente divina
 habent esse divinum.


&

第二異論に対しては、以下のように言われるべきである。
範型から生じたものは、範型に、形相の性格に即して一致しなければならないが、
 存在の仕方に即して一致しなくてもよい。というのも、たとえば家の形相が、
 建築家の精神の中で、非質料的で可知的な存在を持つが、魂の外にある家にお
 いては、質料的で可感的な存在を持つように、形相は、時として、範型と範型
 から生じたものとで、異なる存在を持つことがあるからである。したがって、
 それ自身においては生きていない諸事物の性格も、神の精神の中では生命であ
 る。なぜなら、神の精神の中で、神の存在を持っているからである。


\\




{\scshape Ad tertium dicendum} quod, si de ratione rerum
 naturalium non esset materia, sed tantum forma, omnibus modis veriori
 modo essent res naturales in mente divina per suas ideas, quam in
 seipsis. Propter quod et Plato posuit quod homo separatus erat verus
 homo, homo autem materialis est homo per participationem. 
&


第三異論に対しては、以下のように言われるべきである。
仮に、自然的諸事物の性格に質料が属さず、ただ形相だけが属していたならば、
 自然的諸事物は、それ自身においてよりも、神の精神の中に、自らのイデアに
 よって、より真のあり方で、存在したであろう。
このために、プラトンも、[質料から]離在する人間が、真の人間であり、質料
 的な人間は、分有による人間である、と考えた。


\\


Sed quia de
 ratione rerum naturalium est materia, dicendum quod res naturales
 verius esse habent simpliciter in mente divina, quam in seipsis, quia
 in mente divina habent esse increatum, in seipsis autem esse
 creatum. Sed esse hoc, utpote homo vel equus, verius habent in propria
 natura quam in mente divina, quia ad veritatem hominis pertinet esse
 materiale, quod non habent in mente divina. Sicut domus nobilius esse
 habet in mente artificis, quam in materia, sed tamen verius dicitur
 domus quae est in materia, quam quae est in mente; quia haec est domus
 in actu, illa autem domus in potentia.



&

しかし、自然的諸事物の性格
 には質料が含まれるので\footnote{自然的諸事物の性格に質料が含まれるとい
 う点は、直後の部分を飛び越して、Sed esse hoc以下の論述(人間や馬につ
 いての論述)に影響する。}、次
 のように言われるべきである。すなわち、端的に[=もっている存在だけに注目すれば]、自然的諸事物は、神の精神の中で、それ自身においてより真である存
 在をもつ。なぜなら、神の中では非被造の存在を持つが、それ自身においては、
 被造の存在を持つからである。しかし、[自然的諸事物は、]人間[であること]や馬[であること]
 といった、この存在(esse hoc)[=これであること]を、神の精神の中よりも、固有の本性において、より真に持っている。な
 ぜなら、人間の真理には質料的であることが属しているが、このことを、神の
 精神の中では持たないからである。ちょうど、家が、材料[=質料]において
 よりも、建築家の精神の中で、より高貴な存在を持つが、しかし、精神の中に
 ある家よりも、材料の中にある家の方が、より真の家と言われる。なぜなら、
 後者は現実態における家だが、前者は可能態における家だからである。
 \footnote{興味深い議論。非質料的存在である天使の場合はどうなるかを考えてみると面白い。
 cf.~ST I, q.~56, a.~2.}




\\




{\scshape Ad quartum dicendum} quod, licet mala sint in
 Dei scientia, inquantum sub Dei scientia comprehenduntur, non tamen
 sunt in Deo sicut creata a Deo vel conservata ab ipso, neque sicut
 habentia rationem in Deo, cognoscuntur enim a Deo per rationes
 bonorum. Unde non potest dici quod mala sint vita in Deo. Ea vero quae
 secundum nullum tempus sunt, possunt dici esse vita in Deo, secundum
 quod vivere nominat intelligere tantum, inquantum intelliguntur a Deo,
 non autem secundum quod vivere importat principium operationis.


&


第四異論に対しては、以下のように言われるべきである。
悪は、神の知のもとに把握されている限りで、神の知の中にあるが、しかし、神
 によって作られたもの、神によって保存されたもの、として在るのではなく、
 神の中で何かの性格(ratio)をもつものとして在るのでもない。なぜなら、神によって、悪は善の性格を通
 して知られるからである。したがって、悪が神の中で生命であると言われるこ
 とはできない。他方、どんな時刻にも存在しないものどもは、神の中で生命で
 あると言われうる。それは、「生きること」が、「知性認識すること」を名付
 ける、ということにしたがって、それらが神に知性認識されている限りでであ
 る。しかし、生命が働きの根源を意味する限りではそう言われえない。


\end{longtable}

\end{document}