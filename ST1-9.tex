\documentclass[10pt]{jsarticle} % use larger type; default would be 10pt
%\usepackage[utf8]{inputenc} % set input encoding (not needed with XeLaTeX)
%\usepackage[round,comma,authoryear]{natbib}
%\usepackage{nruby}
\usepackage{okumacro}
\usepackage{longtable}
%\usepqckage{tablefootnote}
\usepackage[polutonikogreek,english,japanese]{babel}
%\usepackage{amsmath}
\usepackage{latexsym}
\usepackage{color}

%----- header -------
\usepackage{fancyhdr}
\pagestyle{fancy}
\lhead{{\it Summa Theologiae} I, q.~9}
%--------------------


\bibliographystyle{jplain}
\title{{\bf PRIMA PARS}\\{\HUGE Summae Theologiae}\\Sancti Thomae
Aquinatis\\{\sffamily QUAESTIO NONA}\\DE DEI IMMUTABILITATE}
\author{Japanese translation\\by Yoshinori {\sc Ueeda}}
\date{Last modified \today}



%%%% コピペ用
%\rhead{a.~}
%\begin{center}
% {\Large {\bf }}\\
% {\large }\\
% {\footnotesize }\\
% {\Large \\}
%\end{center}
%
%\begin{longtable}{p{21em}p{21em}}
%
%&
%
%\\
%\end{longtable}
%\newpage



\begin{document}
\maketitle

\begin{center}
 {\Large 第九問\\神の不変性について}
\end{center}

\begin{longtable}{p{21em}p{21em}}
Consequenter considerandum est de immutabilitate et aeternitate
divina, quae immutabilitatem consequitur. Circa immutabilitatem vero
quaeruntur duo.
\begin{enumerate}
 \item utrum Deus sit omnino immutabilis. 
 \item utrum esse immutabile sit proprium Dei.
\end{enumerate}

 &

 続いて、神の不変性と、不変性に伴う永遠性について考察されるべきである。
 不変性については二つのことが問われる。

\begin{enumerate}
\item 神は、あらゆる点で不変か。
\item 不変であることは、神に固有か。
\end{enumerate}
\end{longtable}
\newpage
\rhead{a.~1}
\begin{center}
 {\Large {\bf ARTICULUS PRIMUS}}\\
 {\large UTRUM DEUS SIT OMNINO IMMUTABILIS}\\
 {\footnotesize I {\itshape Sent.}, d.~8, q.~3, a.~1; I {\itshape SCG.},
 c.~13, 14; II, c.~25; {\itshape De Pot.}, q.~8, a.~1, ad 9; {\itshape
 Compend.~Theol.}, c.~4; in Boet.~{\itshape de Trin.}, q.~5, a.~4, ad 2.}\\
 {\Large 第一項\\神はあらゆる点で不変か}
\end{center}


\begin{longtable}{p{21em}p{21em}}
{\huge A}{\scshape d primum sic proceditur}. Videtur quod Deus non sit
omnino immutabilis.  Quidquid enim movet seipsum, est aliquo modo
mutabile. Sed, sicut dicit Augustinus, VIII {\it super Genesim ad
litteram}:  {\itshape spiritus creator movet se nec per tempus nec per
locum}. Ergo Deus est aliquo modo mutabilis.

& 

第一項の問題へ、議論は次のように進められる。神はあらゆる点で不変である
わけではない、と思われる。理由は以下の通り。自らを動かすものは何でも、
なんらかのかたちで変化を受けうる(可変的な)ものである。しかるにアウグ
スティヌスが『創世記逐語注解』第8巻で述べているように、「創造主である
霊は、時間によらず場所によらず、自らを動かす」。ゆえに神はなんらかの意
味で変化を受けうる。

\\

2.~{\scshape Praeterea}, {\it Sap}.~{\scshape vii} dicitur de
sapientia quod est {\itshape mobilior omnibus mobilibus}. Sed Deus est
ipsa sapientia. Ergo Deus est mobilis.

 &

さらに、『知恵の書』7章には、知恵について、あらゆる可変的なものよりも
可変的だと言われている。しかるに神は知恵そのものである。ゆえに、神は可
変的である。

\\

3.~{\scshape Praeterea}, appropinquari et elongari motum significant.
 Huiusmodi autem dicuntur de Deo in Scriptura, {\it Iac}.~{\scshape
 iv}, {\itshape appropinquate Deo, et appropinquabit vobis}. Ergo Deus
 est mutabilis.

&

さらに、近づくことや遠ざかることは、運動を表示する。しかるに、聖書の中で、
 このようなことが神について言われている。たとえば『ヤコブの手紙』4章「神
 に近づきなさい。そうすれば、(神もまた)あなたたちに近づくであろう」。
 \footnote{「神に近づきなさい。そうすれば、神は近づいてくださいます。」
 (4:8)}ゆえに、神は可変的である。

\\

Sed contra est quod dicitur Malach.~{\scshape iii}, {\itshape ego
 Deus, et non mutor}.

&

 しかし反対に、『マラキ書』3章に「私は神であり、変化させられない」
 \footnote{「主であるわたしは変わることがない」(3:6)}と言われている。


\\


{\scshape Respondeo dicendum} quod ex praemissis ostenditur Deum esse
 omnino immutabilem. Primo quidem, quia supra ostensum est esse
 aliquod primum ens, quod Deum dicimus, et quod huiusmodi primum ens
 oportet esse purum actum absque permixtione alicuius potentiae, eo
 quod potentia simpliciter est posterior actu. Omne autem quod
 quocumque modo mutatur, est aliquo modo in potentia. Ex quo patet
 quod impossibile est Deum aliquo modo mutari.


&

解答する。以下のように言われるべきである。これまでに示された事柄から、
 神があらゆる点で不変であることが示される。第一に、なにか、第一の(一
 番はじめの)存在者、それを私たちは神と呼ぶのだが、が存在すること、そ
 して、可能態は端的に現実態より後のものだから、そのような第一の存在者
 は、どんな可能態の混合もなく純粋な現実態でなければならないことが、上
 で示された。しかるに、すべて、どのようなかたちであれ変化を受けるもの
 は、なんらかの点で可能態にある。このことから、神がどんな意味でも変化
 を受けることが不可能であることが明らかである。

\\


Secundo, quia omne quod movetur, quantum ad aliquid manet, et quantum
 ad aliquid transit, sicut quod movetur de albedine in nigredinem,
 manet secundum substantiam. Et sic in omni eo quod movetur,
 attenditur aliqua compositio. Ostensum est autem supra quod in Deo
 nulla est compositio, sed est omnino simplex. Unde manifestum est
 quod Deus moveri non potest.


&

第二に、すべて動かされるものは、ある点においてはもとのままに留まるが、
 別の点において移動する。たとえば、白から黒へ動かされる(変化させられ
 る)ものは、実体においてはもとのままに留まる。このように、動かされる
 すべてのものにおいて、なんらかの複合が見いだされる。しかるに、上で、
 神においていかなる複合もなく、あらゆる点で単純であることが示された。
 ゆえに、神が動かされることは不可能であることが明らかである。


\\

Tertio, quia omne quod movetur, motu suo aliquid acquirit, et
 pertingit ad illud ad quod prius non pertingebat. Deus autem, cum sit
 infinitus, comprehendens in se omnem plenitudinem perfectionis totius
 esse, non potest aliquid acquirere, nec extendere se in aliquid ad
 quod prius non pertingebat. Unde nullo modo sibi competit motus. Et
 inde est quod quidam antiquorum, quasi ab ipsa veritate coacti,
 posuerunt primum principium esse immobile.


&

第三に、すべて動かされるものは、自らの運動によってなにかを獲得し、以前
 には到達していなかったものに到達する。しかるに神は、無限なので、自ら
 のうちに、エッセ全体の完全性のすべての充溢を包摂するのだから、なにか
 を獲得することも、自らを、以前に到達していなかったところへ拡張するこ
 とも不可能である。また、このため、古代の人々のなかのある人々は、あた
 かも真理そのものによって強制されたかのように、第一根源が不変であると
 考えたのである。

\\



{\scshape Ad primum ergo dicendum} quod Augustinus ibi loquitur
 secundum modum quo Plato dicebat primum movens movere seipsum, omnem
 operationem nominans motum; secundum quod etiam ipsum intelligere et
 velle et amare motus quidam dicuntur. Quia ergo Deus intelligit et
 amat seipsum, secundum hoc dixerunt quod Deus movet seipsum, non
 autem secundum quod motus et mutatio est existentis in potentia, ut
 nunc loquimur de mutatione et motu.

&

第一異論に対しては、それゆえ、次のように言われるべきである。アウグスティ
 ヌスはそこで、プラトンが、第一動者が自らを動かすと語ったかたちにした
 がって、つまりあらゆる働きを運動と名付けるかたちで、述べている。それ
 にしたがえば、知性認識すること、意志すること、愛することもまた、なん
 らかの運動と言われる。それゆえ彼ら[プラトンとアウグスティヌス]は、
 神が自らを知性認識し愛するがゆえに、それにしたがって、神が自らを動か
 すと言ったのであり、今私たちが変化と運動について語っているように、運
 動と変化が可能態に在るものに属するというかぎりで、そう言ったのではな
 い。


\\


{\scshape Ad secundum dicendum} quod sapientia dicitur mobilis esse
similitudinarie, secundum quod suam similitudinem diffundit usque ad
ultima rerum. Nihil enim esse potest, quod non procedat a divina
sapientia per quandam imitationem, sicut a primo principio effectivo
et formali; prout etiam artificiata procedunt a sapientia
artificis. Sic igitur inquantum similitudo divinae sapientiae gradatim
procedit a supremis, quae magis participant de eius similitudine,
usque ad infima rerum, quae minus participant dicitur esse quidam
processus et motus divinae sapientiae in res, sicut si dicamus solem
procedere usque ad terram, inquantum radius luminis eius usque ad
terram pertingit. Et hoc modo exponit Dionysius, cap. I {\it
Cael.~Hier.}, dicens quod {\itshape omnis processus divinae
manifestationis venit ad nos a patre luminum moto}.


&

第二異論に対しては次のように言われるべきである。知恵が「可変的である」
と言われるのは、類似的に、すなわち、自らの類似性を諸事物のなかで最も遠
いものまで及ぼす限りにおいてである。というのも、作出的、形相的な第一根
源としての神の知恵から、なんらかの類似によって発出しないものは、なにも
存在しえないからである。ちょうど、制作物が、制作者の知恵から出てくる場
合のように。それゆえ、このように、神の知恵の類似が、神の類似をより多く
分有する最高のものどもから、(神の類似を)より少なく分有する、諸事物の
うちで最低のものどもまで、徐々に発出する限りにおいて、神の知恵の一種の
発出と運動が存在すると言われる。ちょうど、太陽の光の光線が大地に到達す
る限りにおいて、太陽が大地にまで発出する、と私たちが言うとすれば、その
ようなことである。そしてこの意味でディオニュシウスは、『天上階級論』第
1章で、神の明示のあらゆる発出は、光の動く父から私たちへ来る、と述べる
のである。

\\


{\scshape Ad tertium dicendum} quod huiusmodi dicuntur de Deo in
 Scripturis metaphorice. Sicut enim dicitur sol intrare domum vel exire,
 inquantum radius eius pertingit ad domum; sic dicitur Deus
 appropinquare ad nos vel recedere a nobis, inquantum percipimus
 influentiam bonitatis ipsius, vel ab eo deficimus.

&


第三異論に対しては、次のように言われるべきである。聖書において、このよ
 うなことは、神について比喩的に言われている。つまり、太陽が家に入って
 いくとか出ていくとか言われるのは、太陽の光線が家に到達する限りにおい
 てである。同じように、神が私たちに近づくとか私たちから離れるとかは、
 私たちが神の善性の影響を感じるかぎりで、あるいはそういうことがない限
 りにおいて、言われるのである。


\end{longtable}
\newpage
\rhead{a.~2}


\begin{center}
 {\Large {\bf ARTICULUS SECUNDUS}}\\
 {\large UTRUM ESSE IMMUTABILE SIT DEI PROPRIUM}\\
 {\footnotesize Infra, q.~10, a.~3; q.~65, a.~1, ad 1; III, q.~57, a.~1;
 I {\itshape Sent.}, d.~8, q.~3, a.~2; d.~19, q.~5, a.~3; II d.~7, q.~1,
 a.~1; {\itshape De Malo}, q.~16, a.~2, ad 6; {\itshape Quodl.}~X, q.~2}\\
 {\Large 第二項\\不変であることは神に固有か}
\end{center}

\begin{longtable}{p{21em}p{21em}}

{\huge A}{\scshape d secundum sic proceditur}. Videtur quod esse
immutabile non sit proprium Dei. Dicit enim philosophus, in II {\it
Metaphys.}, quod materia est in omni eo quod movetur. Sed substantiae
quaedam creatae, sicut Angeli et animae, non habent materiam, ut
quibusdam videtur. Ergo esse immutabile non est proprium Dei.


&

第二項の問題へ、議論は以下のように進められる。不変であることは、神に固
有でないと思われる。理由は以下の通り。哲学者が『形而上学』第2巻で、す
べて動かされるもののうちには質料がある、と述べている。しかるに、ある人々
によってそう思われたように、天使や魂のように、被造物のある実体は質料を
もたない。ゆえに、不変であることは、神に固有ではない。

\\

2.~{\scshape Praeterea}, omne quod movetur, movetur propter aliquem
   finem, quod ergo iam pervenit ad ultimum finem, non movetur. Sed
   quaedam creaturae iam pervenerunt ad ultimum finem, sicut omnes
   beati. Ergo aliquae creaturae sunt immobiles.

&

さらに、すべて動かされるものは、ある目的のために動かされる。ゆえに、究
極目的にすでに到達したものは、動かされない。しかるに、すべての至福者の
ように、ある被造物は、すでに究極目的に到達している。ゆえに、ある被造物
は不変である。

\\


3.~{\scshape Praeterea}, omne quod est mutabile, est variabile. Sed
formae sunt invariabiles, dicitur enim in libro {\it Sex
Principiorum}, quod {\itshape forma est simplici et invariabili
essentia consistens}. Ergo non est solius Dei proprium esse
immutabile.


&

さらに、すべて可変的なものは、多様化されうる。しかるに、形相は多様化さ
れえない。なぜなら、『六つの根源』という書物において、「形相は単純で多
様化されえない本質によって成り立つ」と言われているからである。ゆえに、
不変であることは、神だけに固有なのではない。

\\


{\scshape Sed contra est} quod dicit Augustinus, in libro {\it de
Natura Boni} : {\itshape solus Deus immutabilis est; quae autem fecit,
quia ex nihilo sunt, mutabilia sunt}.

&

しかし反対に、アウグスティヌスは『善の本性について』という書物の中で、
「神だけが不変である。これに対して、神が作ったものは、無から存在するの
で、可変的である」、と述べている。


\\


[1] Respondeo dicendum quod solus Deus est omnino immutabilis, omnis
 autem creatura aliquo modo est mutabilis.  Sciendum est enim quod
 mutabile potest aliquid dici dupliciter, uno modo, per potentiam quae
 in ipso est; alio modo, per potentiam quae in altero est.


&

解答する。以下のように言われるべきである。神だけがあらゆる点において不
変であり、すべての被造物はなんらかの点で可変的である。理由は以下の通り。
あるものは、二通りのしかたで可変的と言われうることが知られるべきである。
ひとつにはそのもの自身のうちにある可能性\footnote{やや冒険だが、本項で
は通常「能力」と訳されるpotentiaを、明らかに「可能態」と訳すべきところ
を除いて「可能性」と訳してみた。「神のpotentia」と「受動的なpotentia」
の両方を訳出できる単語が他に見当たらないため。}によってであり、もう一
つには他のもののうちにある可能性によってである。

\\

[2] Omnes enim creaturae, antequam essent, non erant possibiles esse
per aliquam potentiam creatam, cum nullum creatum sit aeternum, sed
per solam potentiam divinam, inquantum Deus poterat eas in esse
producere. Sicut autem ex voluntate Dei dependet quod res in esse
producit, ita ex voluntate eius dependet quod res in esse conservat,
non enim aliter eas in esse conservat, quam semper eis esse dando;
unde si suam actionem eis subtraheret, omnia in nihilum redigerentur,
ut patet per Augustinum, {\it IV super Gen. ad Litt.} Sicut igitur in
potentia creatoris fuit ut res essent, antequam essent in seipsis, ita
in potentia creatoris est, postquam sunt in seipsis, ut non sint. Sic
igitur per potentiam quae est in altero, scilicet in Deo, sunt
mutabiles, inquantum ab ipso ex nihilo potuerunt produci in esse, et
de esse possunt reduci in non esse.

&

すべての被造物は、それらが存在する前には存在することが可能なものだった
 が、それはなんらかの被造の可能性によって可能だったわけではない。なぜ
 ならどんな創造されたものも永遠でないからである。むしろそれは神の可能
 性によってであり、神ががそれらを存在へと生み出すことができた限りにお
 いて(存在することが可能であったの)である。ところで、事物を存在へ生
 み出すことが神の意志に依存しているように、事物を存在のうちに保存する
 ことも、神の意志に依存する。なぜなら、被造物を存在のうちに保つには、
 常にそれらに存在を与えることによるしかないからである。したがって、
 『創世記逐語注解』のアウグスティヌスによって明らかなとおり、もしかり
 に、神が自分の作用を被造物から引き抜いたならば、万物は無へと帰したで
 あろう。ゆえに、ちょうど創造者の可能性の中に、諸事物がそれ自身におい
 て存在する以前、それらの事物を存在させることがあったように、創造者の
 可能性の中には、諸事物がそれ自身において存在した後、それらを存在させ
 ないことがある。ゆえに、このようにして、諸事物が神によって無から生み
 出されることができたし、存在から非存在へ帰せられることができる限りに
 おいて、諸事物は、他のものすなわち神のうちにある可能性によって可変的
 である。

\\

[3] Si autem dicatur aliquid mutabile per potentiam in ipso
existentem, sic etiam aliquo modo omnis creatura est mutabilis.  Est
enim in creatura duplex potentia, scilicet activa et passiva. Dico
autem potentiam passivam, secundum quam aliquid assequi potest suam
perfectionem, vel in essendo vel in consequendo finem. Si igitur
attendatur mutabilitas rei secundum potentiam ad esse, sic non in
omnibus creaturis est mutabilitas, sed in illis solum in quibus illud
quod est possibile in eis, potest stare cum non esse.


&

他方、あるものが、それ自身の中にある可能性によって可変的であると言われ
るならば、その場合にも、すべての被造物はなんらかの点で可変的である。理
由は以下の通り。被造物の中には、二通りの可能性、すなわち、能動的な可能
性と受動的な可能性がある。ところで、私が受動的な可能性というのは、存在
することにおいてであれ、目的を獲得することにおいてであれ、あるものが、
それにしたがって自らの完全性を獲得しうるような可能性である。ゆえに、存
在に対する可能性(=受容可能性)という点で事物の可変性が見られるならば、
その場合には、すべての被造物の中に可変性があるわけではなく、それらにお
いて「可能である」ことが「現にない」ことと両立しうるもの\footnote{つま
り「現実にはXでないが、Xであることが可能」という状態でありうるもの。}
に限られる。

\\

[4] Unde in corporibus inferioribus est mutabilitas et secundum esse
substantiale, quia materia eorum potest esse cum privatione formae
substantialis ipsorum, et quantum ad esse accidentale, si subiectum
compatiatur secum privationem accidentis; sicut hoc subiectum, homo,
compatitur secum non album, et ideo potest mutari de albo in non
album.  Si vero sit tale accidens quod consequatur principia
essentialia subiecti, privatio illius accidentis non potest stare cum
subiecto, unde subiectum non potest mutari secundum illud accidens,
sicut nix non potest fieri nigra.

&

したがって、下位の諸物体においては、実体的な存在にかんしても附帯的な存
在にかんしても可変性がある。前者については、それらの質料がそれらの実体
形相を欠いても存在しうるからであり、後者については、基体が附帯性の欠如
と両立する場合がそれである。たとえば、人間というこの基体は、白くないこ
とと両立するので、白いものから白くないものへ変化しうるように。他方、も
し基体の本質的根源から帰結するような附帯性が存在するならば、その附帯性
が欠如することは基体と両立しない。したがって、基体がその附帯性において
変化を受けることはありえない。ちょうど、雪が黒くなることが不可能である
ように。


\\

[5] In corporibus vero caelestibus, materia non compatitur secum
privationem formae, quia forma perficit totam potentialitatem
materiae, et ideo non sunt mutabilia secundum esse substantiale; sed
secundum esse locale, quia subiectum compatitur secum privationem
huius loci vel illius.

&


また天体において、質料が形相の欠如と両立することはない。なぜなら、(天
体の)形相は、質料の全ての可能性を完成しているので、(天体は)実体的な
存在において可変的でないからである。しかし、場所の存在においては可変的
である。なぜなら、基体が、この場所の欠如やあの場所の欠如と両立するから
である。

\\

[6] Substantiae vero incorporeae, quia sunt ipsae formae subsistentes,
quae tamen se habent ad esse ipsarum sicut potentia ad actum, non
compatiuntur secum privationem huius actus, quia esse consequitur
formam, et nihil corrumpitur nisi per hoc quod amittit formam. Unde in
ipsa forma non est potentia ad non esse, et ideo huiusmodi substantiae
sunt immutabiles et invariabiles secundum esse. Et hoc est quod dicit
Dionysius, {\it IV cap. de Div. Nom.}, quod substantiae intellectuales
creatae mundae sunt a generatione et ab omni variatione, sicut
incorporales et immateriales.

Sed tamen remanet in eis duplex mutabilitas. Una secundum quod sunt in
potentia ad finem, et sic est in eis mutabilitas secundum electionem
de bono in malum, ut Damascenus dicit.

Alia secundum locum, inquantum virtute sua finita possunt attingere
quaedam loca quae prius non attingebant, quod de Deo dici non potest,
qui sua infinitate omnia loca replet, ut supra dictum est.

&

また、非物体的実体は、自存する形相そのものだが、自らの存在に対して、可
能態が現実態に対するように関係するので、この現実態の欠如と両立すること
はない。なぜなら、存在は形相に伴い、形相を捨てることによってでないかぎ
り、なにも消滅しないからである。したがって、この形相のなかに非存在への
可能性はなく、それゆえ、このような実体は、存在において不変であり多様化
されえない。そしてこれが、ディオニュシウスが『神名論』第4章で、被造の
知的実体は、非物体的で非質料的なものとして、生成とすべての多様化から免
れていると述べることである。

しかし、それらには二通りの可変性が残っている。一つは、目的に対して可能
態にあるということにしたがってであり、このかぎりで、ダマスケヌスが言う
ように、それらの中に、善から悪へといった選択における可変性がある。

もう一つは、自らの有限な力(virtus)によって、以前には触れていなかった場
所に触れることがありうる。このことは、神については不可能である。神は、
上で述べられたように、自らの無限性によって、すべての場所を満たすからで
ある。

\\


\\

[7] Sic igitur in omni creatura est potentia ad mutationem, vel secundum
esse substantiale, sicut corpora corruptibilia; vel secundum esse
locale tantum, sicut corpora caelestia, vel secundum ordinem ad finem
et applicationem virtutis ad diversa, sicut in Angelis.
Et universaliter omnes creaturae communiter sunt mutabiles secundum
potentiam creantis, in cuius potestate est esse et non esse
earum. Unde, cum Deus nullo istorum modorum sit mutabilis, proprium
eius est omnino immutabilem esse.
&


ゆえにすべての被造物の中に変化への可能性がある。それは消滅しうる物体の
場合のように実体的な存在において、また天体の場合のように場所的な存在の
みにおいて、また天使の場合のように目的への秩序において、またさまざまな
ものへの力の適用においてである。そして、普遍的にすべての被造物は共通に、
創造者の可能性において可変的である。創造者の権能の中には、被造物の存在
と非存在が含まれるからである。したがって、神はこれらのどのしかたによっ
ても可変的でないので、あらゆる点で不変であることは神に固有である。

\\


%\footnote{Summary: 
%「あるものが~の点で可変的である」 \\
%△…自然物、○…天体、◎…天使 とする
% \begin{itemize}
%  \item 自分の中にある可能性によって △○◎
%   \begin{itemize}
%    \item 存在への可能性 △○
%      \begin{itemize}
%       \item 実体的存在への可能性 △
%       \item 附帯的存在への可能性 △○
%        \begin{itemize}
%         \item 場所的存在 △○
%         \item その他の附帯性の存在 △
%        \end{itemize}
%      \end{itemize}
%    \item 目的の選択 ◎
%    \item 場所との接触 ◎
%   \end{itemize}
%  \item 他者の中にある可能性によって △○◎
% \end{itemize}
%}


\\



{\scshape Ad primum ergo dicendum} quod obiectio illa procedit de eo
quod est mutabile secundum esse substantiale vel accidentale, de tali
enim motu philosophi tractaverunt.

&

第一異論に対しては、それゆえ、次のように言われるべきである。かの反論は、
実体的な存在あるいは付帯的な存在において可変的なものについて論じられて
いる。つまり哲学者たちは、そのような運動について論じたのである。

\\

{\scshape Ad secundum dicendum} quod Angeli boni, supra
immutabilitatem essendi, quae competit eis secundum naturam, habent
immutabilitatem electionis ex divina virtute, tamen remanet in eis
mutabilitas secundum locum.

&


第二異論に対しては、次のように言われるべきである。善い天使は、本性にお
いてそれらに適合する存在の不変性にくわえて、神の力にもとづく選択の不変
性ももっている。しかしそれらには場所における可変性が残っている。

\\

{\scshape Ad tertium dicendum} quod formae dicuntur invariabiles, quia
non possunt esse subiectum variationis, subiiciuntur tamen variationi,
inquantum subiectum secundum eas variatur. Unde patet quod secundum quod
sunt, sic variantur, non enim dicuntur entia quasi sint subiectum
essendi, sed quia eis aliquid est.

&

第三異論に対しては、次のように言われるべきである。形相が多様化されえな
いと言われるのは、多様化の基体になりえないからだが、しかし、基体が形相
に応じて多様化されるかぎりで、形相は多様性のもとにある。したがって、諸
形相が存在するかぎりにおいて、その意味で諸形相が多様であるのは明らかで
ある。なぜなら、形相が「存在するもの」と言われるのは、形相が存在の基体
だからではなく、形相によって何かが存在するからである。

\end{longtable}
\end{document}