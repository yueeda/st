\documentclass[10pt]{jsarticle} % use larger type; default would be 10pt
%\usepackage[utf8]{inputenc} % set input encoding (not needed with XeLaTeX)
%\usepackage[round,comma,authoryear]{natbib}
%\usepackage{nruby}
\usepackage{okumacro}
\usepackage{longtable}
%\usepqckage{tablefootnote}
\usepackage[polutonikogreek,english,japanese]{babel}
%\usepackage{amsmath}
\usepackage{latexsym}
\usepackage{color}

%----- header -------
\usepackage{fancyhdr}
\pagestyle{fancy}
\lhead{{\it Summa Theologiae} I, q.~8}
%--------------------


\bibliographystyle{jplain}
\title{{\bf Prima Pars}\\{\HUGE Summae Theologiae}\\Sancti Thomae
Aquinatis\\Quaestio Nona\\{\bf De Dei Immutabilitate}}
\author{Japanese translation\\by Yoshinori {\sc Ueeda}}
\date{Last modified \today}

%%%% コピペ用
%\rhead{a.~}
%\begin{center}
% {\Large {\bf }}\\
% {\large }\\
% {\footnotesize }\\
% {\Large \\}
%\end{center}
%
%\begin{longtable}{p{21em}p{21em}}
%
%&
%
%\\
%\end{longtable}
%\newpage



\begin{document}
\maketitle

\begin{center}
 {\Large 第九問\\神の不変性について}
\end{center}

\begin{longtable}{p{21em}p{21em}}
Consequenter considerandum est de immutabilitate et aeternitate divina, quae immutabilitatem consequitur. Circa immutabilitatem vero quaeruntur duo. 
\begin{enumerate}
 \item utrum Deus sit omnino immutabilis. 
 \item utrum esse immutabile sit proprium Dei.
\end{enumerate}
&
続いて、神の不変性と永遠性について考察されるべきである。後者は、不変性に
 伴う。
不変性については、二つのことが問われる。

\begin{enumerate}
\item 神は、あらゆる点で不変か。
\item 不変であることは、神に固有か。
\end{enumerate}
\end{longtable}
\newpage
\rhead{a.~1}


\begin{center}
 {\Large {\bf ARTICULUS PRIMUS}}\\
 {\large UTRUM DEUS SIT OMNINO IMMUTABILIS}\\
 {\footnotesize I {\itshape Sent.}, d.~8, q.~3, a.~1; I {\itshape SCG.},
 c.~13, 14; II, c.~25; {\itshape De Pot.}, q.~8, a.~1, ad 9; {\itshape
 Compend.~Theol.}, c.~4; in Boet.~{\itshape de Trin.}, q.~5, a.~4, ad 2.}\\
 {\Large 第一項\\神はあらゆる点で不変か}
\end{center}


\begin{longtable}{p{21em}p{21em}}
{\huge A}{\scshape d primum sic proceditur}. Videtur quod Deus non sit
omnino immutabilis.  Quidquid enim movet seipsum, est aliquo modo
mutabile. Sed, sicut dicit Augustinus, VIII {\it super Genesim ad
litteram}:  {\itshape spiritus creator movet se nec per tempus nec per
locum}. Ergo Deus est aliquo modo mutabilis.

& 

第一に対しては、次のように進められる。
神は、あらゆる点で不変であるわけではない、と思われる。なぜなら、自らを動
かすものは何でも、なんらかのかたちで、変化を受けうる(可変的な)ものであ
る。しかるに、アウグスティヌスが『創世記逐語注解』8巻で述べているように、
「創造主である霊は、時間によらず場所によらず、自らを動かす」 。ゆえに、神は、
なんらかの意味で、変化を受けうる。

\\

2.~{\scshape Praeterea}, {\it Sap}.~{\scshape vii} dicitur de sapientia quod est
{\itshape mobilior omnibus mobilibus}. Sed Deus est ipsa sapientia. Ergo
Deus est mobilis.  
&

さらに、『知恵の書』7章には、知恵について、あらゆる可変
的なものよりも可変的だと言われている。しかるに神は知恵そのものである。ゆ
えに、神は可変的である。

\\

3.~{\scshape Praeterea}, appropinquari et elongari motum
 significant. Huiusmodi autem dicuntur de Deo in Scriptura, {\it
 Iac}.~{\scshape iv}, {\itshape appropinquate Deo, et appropinquabit
 vobis}. Ergo Deus est mutabilis.

&

さらに、近づくことや遠ざかることは、運動を表示する。しかるに、聖書の中で、
 このようなことが神について言われている。たとえば『ヤコブの手紙』4章「神
 に近づきなさい。そうすれば、(神もまた)あなたたちに近づくであろう」。
 \footnote{「神に近づきなさい。そうすれば、神は近づいてくださいます。」
 (4:8)}ゆえに、神は可変的である。

\\

Sed contra est quod dicitur Malach.~{\scshape iii}, {\itshape ego Deus,
 et non mutor}.

&
しかし反対に、『マラキ書』3章に「私は神であり、変化させられない」
 \footnote{「主であるわたしは変わることがない」(3:6)}と
 言われている。


\\


{\scshape Respondeo dicendum} quod ex praemissis ostenditur Deum esse
 omnino immutabilem. Primo quidem, quia supra ostensum est esse aliquod
 primum ens, quod Deum dicimus, et quod huiusmodi primum ens oportet
 esse purum actum absque permixtione alicuius potentiae, eo quod
 potentia simpliciter est posterior actu. Omne autem quod quocumque modo
 mutatur, est aliquo modo in potentia. Ex quo patet quod impossibile est
 Deum aliquo modo mutari. 


&

答えて言わなければならない。これまでに示された事柄から、神があらゆる点で
 不変であることが示される。第一に、なにか、第一の(一番はじめの)存在者、
 それを私たちは神と呼ぶのだが、が存在すること、そして、可能態は端的に現
 実態より後のものだから、そのような第一の存在者は、どんな可能態の混合も
 なく純粋な現実態でなければならないことが、上で示された。しかるに、すべ
 て、どのようなかたちであれ変化を受けるものは、なんらかの点で可能態にあ
 る。このことから、神がどんな意味でも変化を受けることが不可能であること
 が明らかである。

\\


Secundo, quia omne quod movetur, quantum ad aliquid manet, et quantum ad
 aliquid transit, sicut quod movetur de albedine in nigredinem, manet
 secundum substantiam. Et sic in omni eo quod movetur, attenditur aliqua
 compositio. Ostensum est autem supra quod in Deo nulla est compositio,
 sed est omnino simplex. Unde manifestum est quod Deus moveri non
 potest.


&

第二に、すべて動かされるものは、ある点においてはもとのままに留まるが、別
 の点において移動する。たとえば、白から黒へ動かされる(変化させられる)
 ものは、実体においてはもとのままに留まる。このように、動かされるすべて
 のものにおいて、なんらかの複合が見いだされる。しかるに、上で、神におい
 ていかなる複合もなく、あらゆる点で単純であることが示された。ゆえに、神
 が動かされることは不可能であることが明らかである。


\\

Tertio, quia omne quod movetur, motu suo aliquid acquirit, et pertingit
 ad illud ad quod prius non pertingebat. Deus autem, cum sit infinitus,
 comprehendens in se omnem plenitudinem perfectionis totius esse, non
 potest aliquid acquirere, nec extendere se in aliquid ad quod prius non
 pertingebat. Unde nullo modo sibi competit motus. Et inde est quod
 quidam antiquorum, quasi ab ipsa veritate coacti, posuerunt primum
 principium esse immobile.


&

第三に、すべて動かされるものは、自らの運動によってなにかを獲得し、以前に
 は到達していなかったものに到達する。しかるに神は、無限なので、自らのうち
 に、エッセ全体の完全性のすべての充溢を包摂するのだから、なにかを獲得する
 ことも、自らを、以前に到達していなかったところへ拡張することも不可能であ
 る。また、このため、古代の人々のなかのある人々は、あたかも真理そのものに
 よって強制されたかのように、第一根源が不変であると考えたのである。

\\



{\scshape Ad primum ergo dicendum} quod Augustinus ibi loquitur secundum
 modum quo Plato dicebat primum movens movere seipsum, omnem operationem
 nominans motum; secundum quod etiam ipsum intelligere et velle et amare
 motus quidam dicuntur. Quia ergo Deus intelligit et amat seipsum,
 secundum hoc dixerunt quod Deus movet seipsum, non autem secundum quod
 motus et mutatio est existentis in potentia, ut nunc loquimur de
 mutatione et motu.

&

第一に対しては、それゆえ、次のように言われるべきである。アウグスティヌス
 はそこで、プラトンが、第一動者が自らを動かすと語ったかたちにしたがって、
 つまりあらゆる働きを運動と名付けるかたちで、述べている。それにしたがえ
 ば、知性認識すること、意志すること、愛することもまた、なんらかの運動と
 言われる。それゆえ彼ら[プラトンとアウグスティヌス]は、神が自らを知性
 認識し愛するがゆえに、それにしたがって、神が自らを動かすと言ったのであ
 り、今私たちが変化と運動について語っているように、運動と変化が可能態に
 在るものに属するというかぎりで、そう言ったのではない。


\\


{\scshape Ad secundum dicendum} quod sapientia dicitur mobilis esse
similitudinarie, secundum quod suam similitudinem diffundit usque ad
ultima rerum. Nihil enim esse potest, quod non procedat a divina
sapientia per quandam imitationem, sicut a primo principio effectivo et
formali; prout etiam artificiata procedunt a sapientia artificis. Sic
igitur inquantum similitudo divinae sapientiae gradatim procedit a
supremis, quae magis participant de eius similitudine, usque ad infima
rerum, quae minus participant dicitur esse quidam processus et motus
divinae sapientiae in res, sicut si dicamus solem procedere usque ad
terram, inquantum radius luminis eius usque ad terram pertingit. Et hoc
modo exponit Dionysius, cap. I {\it Cael.~Hier.}, dicens quod {\itshape omnis processus divinae manifestationis venit ad nos a patre luminum
moto}.


&

第二に対しては次のように言われるべきである。知恵が「可変的である」と言わ
れるのは、類似的に、すなわち、自らの類似性を諸事物のなかで最も遠いものま
で及ぼす限りにおいてである。というのも、作出的、形相的な第一根源としての
神の知恵から、なんらかの類似によって発出しないものは、なにも存在しえない
からである。ちょうど、制作物が、制作者の知恵から出てくる場合のように。そ
れゆえ、このように、神の知恵の類似が、神の類似をより多く分有する最高のも
のどもから、(神の類似を)より少なく分有する、諸事物のうちで最低のものど
もまで、徐々に発出する限りにおいて、神の知恵の一種の発出と運動が存在する
と言われる。ちょうど、太陽の光の光線が大地に到達する限りにおいて、太陽が
大地にまで発出する、と私たちが言うとすれば、そのようなことである。そして
この意味でディオニュシウスは、『天上階級論』第1章で、神の明示のあらゆる発
出は、光の動く父から私たちへ来る、と述べるのである。

\\


{\scshape Ad tertium dicendum} quod huiusmodi dicuntur de Deo in
 Scripturis metaphorice. Sicut enim dicitur sol intrare domum vel exire,
 inquantum radius eius pertingit ad domum; sic dicitur Deus
 appropinquare ad nos vel recedere a nobis, inquantum percipimus
 influentiam bonitatis ipsius, vel ab eo deficimus.

&


第三に対しては、次のように言われるべきである。聖書において、このようなこ
 とは、神について比喩的に言われている。つまり、太陽が家に入っていくとか
 出ていくとか言われるのは、太陽の光線が家に到達する限りにおいてである。
 同じように、神が私たちに近づくとか私たちから離れるとかは、私たちが神の
 善性の影響を感じるかぎりで、あるいはそういうことがない限りにおいて、言
 われるのである。


\end{longtable}
\newpage
\rhead{a.~2}


\begin{center}
 {\Large {\bf ARTICULUS SECUNDUS}}\\
 {\large UTRUM ESSE IMMUTABILE SIT DEI PROPRIUM}\\
 {\footnotesize Infra, q.~10, a.~3; q.~65, a.~1, ad 1; III, q.~57, a.~1;
 I {\itshape Sent.}, d.~8, q.~3, a.~2; d.~19, q.~5, a.~3; II d.~7, q.~1,
 a.~1; {\itshape De Malo}, q.~16, a.~2, ad 6; {\itshape Quodl.}~X, q.~2}\\
 {\Large 第二項\\不変であることは神に固有か}
\end{center}

\begin{longtable}{p{21em}p{21em}}

{\huge A}{\scshape d secundum sic proceditur}. Videtur quod esse
immutabile non sit proprium Dei. Dicit enim philosophus, in II {\it
Metaphys.}, quod materia est in omni eo quod movetur. Sed substantiae
quaedam creatae, sicut Angeli et animae, non habent materiam, ut
quibusdam videtur. Ergo esse immutabile non est proprium Dei.


&

第二にかんしては次のように進められる。不変であることは、神に固有でないと
思われる。なぜなら、哲学者が『形而上学』第二巻で、すべて動かされるものの
うちには質料がある、と述べている。しかるに、ある人々によってそう思われた
ように、天使や魂のように、被造物のある実体は質料をもたない。ゆえに、不変
であることは、神に固有ではない。

\\

2.~{\scshape Praeterea}, omne quod movetur, movetur propter aliquem
   finem, quod ergo iam pervenit ad ultimum finem, non
   movetur. Sed quaedam creaturae iam pervenerunt ad ultimum
   finem, sicut omnes beati. Ergo aliquae creaturae sunt
   immobiles.

&

さらに、すべて動かされるものは、ある目的のために動かされる。ゆえに、究極
目的にすでに到達したものは、動かされない。しかるに、すべての至福者のよう
に、ある被造物は、すでに究極目的に到達している。ゆえに、ある被造物は不変
である。

\\


3.~{\scshape Praeterea}, omne quod est mutabile, est variabile. Sed
formae sunt invariabiles, dicitur enim in libro {\it Sex Principiorum},
quod {\itshape forma est simplici et invariabili essentia
consistens}. Ergo non est solius Dei proprium esse immutabile.  


&

さらに、すべて可変的なものは、多様化されうる。しかるに、形相は多様化され
えない。なぜなら、『六つの根源』という書物において、「形相は単純で多様化
されえない本質によって成り立つ」と言われているからである。ゆえに、不変で
あることは、神だけに固有なのではない。

\\


{\scshape Sed contra est} quod dicit Augustinus, in libro {\it de Natura
Boni} : {\itshape solus Deus immutabilis est; quae autem fecit, quia ex
nihilo sunt, mutabilia sunt}.

&

しかし反対に、アウグスティヌスは『善の本性について』という書物の中で、
「神だけが不変である。これに対して、神が作ったものは、無から存在するので、
可変的である」、と述べている。


\\


[1] Respondeo dicendum quod solus Deus est omnino immutabilis,
 omnis autem creatura aliquo modo est mutabilis. 
Sciendum est enim quod mutabile potest aliquid dici dupliciter, uno
modo, per potentiam quae in ipso est; alio modo, per potentiam quae in
altero est.


&

答えて言わなければならない。神だけがあらゆる点において不変であり、すべ
ての被造物はなんらかの点で可変的である。理由は以下の通り。あるものは、
二通りのしかたで可変的と言われうることが知られるべきである。
ひとつにはそのもの自身のうちにある可能性\footnote{やや冒険だが、本項では通常「能力」と訳
 されるpotentiaを、明らかに「可能態」と訳すべきところ
 を除いて「可能性」と訳してみた。「神のpotentia」と「受動的なpotentia」の両
 方を訳出できる単語が他に見当たらないため。}によってであり、もう一つには他の
もののうちにある可能性によってである。

\\

[2] Omnes enim creaturae, antequam essent, non erant possibiles esse
per aliquam potentiam creatam, cum nullum creatum sit aeternum, sed
per solam potentiam divinam, inquantum Deus poterat eas in esse
producere. Sicut autem ex voluntate Dei dependet quod res in esse
producit, ita ex voluntate eius dependet quod res in esse conservat,
non enim aliter eas in esse conservat, quam semper eis esse dando;
unde si suam actionem eis subtraheret, omnia in nihilum redigerentur,
ut patet per Augustinum, {\it IV super Gen. ad Litt.} Sicut igitur in
potentia creatoris fuit ut res essent, antequam essent in seipsis, ita
in potentia creatoris est, postquam sunt in seipsis, ut non sint. Sic
igitur per potentiam quae est in altero, scilicet in Deo, sunt
mutabiles, inquantum ab ipso ex nihilo potuerunt produci in esse, et
de esse possunt reduci in non esse.

&

すべての被造物は、それらが存在する前には存在することが可能なものだった
 が、それはなんらかの被造の可能性によって可能だったわけではない。なぜなら
 どんな創造されたものも永遠でないからである。むしろそれは神の可能性によっ
 てであり、神ががそれらを存在へと生み出すことができた限りにおいて(存
 在することが可能であったの)である。
ところで、事物を存在へ生み出すことが神の意志に依存しているように、事物
を存在のうちに保存することも、神の意志に依存する。なぜなら、被造物を存
在のうちに保つには、常にそれらに存在を与えることによるしかないからであ
る。したがって、『創世記逐語注解』のアウグスティヌスによって明らかなと
おり、もしかりに、神が自分の作用を被造物から引き抜いたならば、万物は無
へと帰したであろう。ゆえに、ちょうど創造者の可能性の中に、諸事物がそれ自身
において存在する以前、それらの事物を存在させることがあったように、創造
者の可能性の中には、諸事物がそれ自身において存在した後、それらを存在させな
いことがある。ゆえに、このようにして、諸事物が神によって無から生み出さ
れることができたし、存在から非存在へ帰せられることができる限りにおいて、
諸事物は、他のものすなわち神のうちにある可能性によって可変的である。

\\

[3] Si autem dicatur aliquid mutabile per potentiam in ipso
existentem, sic etiam aliquo modo omnis creatura est mutabilis.  Est
enim in creatura duplex potentia, scilicet activa et passiva. Dico
autem potentiam passivam, secundum quam aliquid assequi potest suam
perfectionem, vel in essendo vel in consequendo finem. Si igitur
attendatur mutabilitas rei secundum potentiam ad esse, sic non in
omnibus creaturis est mutabilitas, sed in illis solum in quibus illud
quod est possibile in eis, potest stare cum non esse.


&

他方、あるものが、それ自身の中にある可能性によって可変的であると言われるな
らば、その場合にも、すべての被造物はなんらかの点で可変的である。理由は
以下の通り。被造物の中には、二通りの可能性、すなわち、能動的な可能性と受
動的な可能性がある。ところで、私が受動的な可能性というのは、存在することにお
いてであれ、目的を獲得することにおいてであれ、あるものが、それにしたがっ
て自らの完全性を獲得しうるような可能性である。ゆえに、存在に対する可能性(=受容
 可能性)という点で
事物の可変性が見られるならば、その場合には、すべての被造物の中に可変性
があるわけではなく、それらにおいて「可能である」ことが「現にない」こ
とと両立しうるもの\footnote{つまり「現実にはXでないが、Xであることが可能」と
 いう状態でありうるもの。}に限られる。

\\

[4] Unde in corporibus inferioribus est mutabilitas et secundum esse
substantiale, quia materia eorum potest esse cum privatione formae
substantialis ipsorum, et quantum ad esse accidentale, si subiectum
compatiatur secum privationem accidentis; sicut hoc subiectum, homo,
compatitur secum non album, et ideo potest mutari de albo in non
album.
Si vero sit tale accidens quod consequatur principia essentialia
subiecti, privatio illius accidentis non potest stare cum subiecto,
unde subiectum non potest mutari secundum illud accidens, sicut nix
non potest fieri nigra.

&

したがって、下位の諸物体においては、実体的な存在にかんしても附帯的な
存在にかんしても可変性がある。前者については、それらの質料がそれら
の実体形相を欠いても存在しうるからであり、後者については、基体が附帯
性の欠如と両立する場合がそれである。たとえば、人間というこの基体は、白くない
ことと両立するので、白いものから白くないものへ変化しうるように。
他方、もし基体の本質的根源から帰結するような附帯性が存在するならば、そ
 の附帯性が欠如することは基体と両立しない。したがって、基体がその附帯
 性において変化を受けることはありえない。ちょうど、雪が黒くなることが不可能であるよう
に。


\\

[5] In corporibus vero caelestibus, materia non compatitur secum
privationem formae, quia forma perficit totam potentialitatem
materiae, et ideo non sunt mutabilia secundum esse substantiale; sed
secundum esse locale, quia subiectum compatitur secum privationem
huius loci vel illius.

&


また天体において、質料が形相の欠如と両立することはない。なぜなら、
(天体の)形相は、質料の全ての可能性を完成しているので、(天体は)実体的な存在に
 おいて可変的でないからである。しかし、場所の存在においては可変的であ
る。なぜなら、基体が、この場所の欠如やあの場所の欠如と両立するからであ
る。

\\

[6] Substantiae vero incorporeae, quia sunt ipsae formae subsistentes,
quae tamen se habent ad esse ipsarum sicut potentia ad actum, non
compatiuntur secum privationem huius actus, quia esse consequitur
formam, et nihil corrumpitur nisi per hoc quod amittit formam. Unde in
ipsa forma non est potentia ad non esse, et ideo huiusmodi substantiae
sunt immutabiles et invariabiles secundum esse. Et hoc est quod dicit
Dionysius, {\it IV cap. de Div. Nom.}, quod substantiae intellectuales
creatae mundae sunt a generatione et ab omni variatione, sicut
incorporales et immateriales.

Sed tamen remanet in eis duplex mutabilitas. Una secundum quod sunt in
potentia ad finem, et sic est in eis mutabilitas secundum electionem
de bono in malum, ut Damascenus dicit.

Alia secundum locum, inquantum virtute sua finita possunt attingere
quaedam loca quae prius non attingebant, quod de Deo dici non potest,
qui sua infinitate omnia loca replet, ut supra dictum est.

&

また、非物体的実体は、自存する形相そのものだが、自らの存在に対して、可
能態が現実態に対するように関係するので、この現実態の欠如と両立すること
はない。なぜなら、存在は形相に伴い、形相を捨てることに
よってでないかぎり、なにも消滅しないからである。したがって、この形相の
 なかに非存在への可能性はなく、それゆえ、このような実体は、存在において不変であり多様化
されえない。そしてこれが、ディオニュシウスが『神名論』第4章で、被造の知的実体は、非
物体的で非質料的なものとして、生成とすべての多様化から免れていると述
べることである。

しかし、それらには二通りの可変性が残っている。一つは、目的に対して可能
態にあるということにしたがってであり、このかぎりで、ダマスケヌスが言う
ように、それらの中に、善から悪へといった選択における可変性がある。

もう一つは、自らの有限な力(virtus)によって、以前には触れていなかった場
所に触れることがありうる。このことは、神については不可能である。神は、
上で述べられたように、自らの無限性によって、すべての場所を満たすからで
ある。

\\


\\

[7] Sic igitur in omni creatura est potentia ad mutationem, vel secundum
esse substantiale, sicut corpora corruptibilia; vel secundum esse
locale tantum, sicut corpora caelestia, vel secundum ordinem ad finem
et applicationem virtutis ad diversa, sicut in Angelis.
Et universaliter omnes creaturae communiter sunt mutabiles secundum
potentiam creantis, in cuius potestate est esse et non esse
earum. Unde, cum Deus nullo istorum modorum sit mutabilis, proprium
eius est omnino immutabilem esse.
&


ゆえにすべての被造物の中に変化への可能性がある。それは消滅しうる物体の場
合のように実体的な存在において、また天体の場合のように場所的な存在のみ
において、また天使の場合のように目的への秩序において、またさまざまなも
のへの力の適用においてである。
そして、普遍的にすべての被造物は共通に、創造者の可能性において可
変的である。創造者の権能の中には、被造物の存在と非存在が含まれるからで
ある。したがって、神はこれらのどのしかたによっても可変的でないので、あ
らゆる点で不変であることは神に固有である。

\\


%\footnote{Summary: 
%「あるものが~の点で可変的である」 \\
%△…自然物、○…天体、◎…天使 とする
% \begin{itemize}
%  \item 自分の中にある可能性によって △○◎
%   \begin{itemize}
%    \item 存在への可能性 △○
%      \begin{itemize}
%       \item 実体的存在への可能性 △
%       \item 附帯的存在への可能性 △○
%        \begin{itemize}
%         \item 場所的存在 △○
%         \item その他の附帯性の存在 △
%        \end{itemize}
%      \end{itemize}
%    \item 目的の選択 ◎
%    \item 場所との接触 ◎
%   \end{itemize}
%  \item 他者の中にある可能性によって △○◎
% \end{itemize}
%}


\\



{\scshape Ad primum ergo dicendum} quod obiectio illa procedit de eo
quod est mutabile secundum esse substantiale vel accidentale, de tali
enim motu philosophi tractaverunt.

&

第一に対しては、それゆえ、次のように言われるべきである。かの反論は、実
体的な存在あるいは付帯的な存在において可変的なものについて論じられてい
る。つまり哲学者たちは、そのような運動について論じたのである。

\\

{\scshape Ad secundum dicendum} quod Angeli boni, supra
immutabilitatem essendi, quae competit eis secundum naturam, habent
immutabilitatem electionis ex divina virtute, tamen remanet in eis
mutabilitas secundum locum.

&


第二に対しては、次のように言われるべきである。善い天使は、本性において
それらに適合する存在の不変性にくわえて、神の力にもとづく選択の不変性も
もっている。しかしそれらには場所における可変性が残っている。

\\

{\scshape Ad tertium dicendum} quod formae dicuntur invariabiles, quia
non possunt esse subiectum variationis, subiiciuntur tamen variationi,
inquantum subiectum secundum eas variatur. Unde patet quod secundum quod
sunt, sic variantur, non enim dicuntur entia quasi sint subiectum
essendi, sed quia eis aliquid est.

&

第三に対しては、次のように言われるべきである。形相が多様化されえないと言
われるのは、多様化の基体になりえないからだが、しかし、基体が形相に応じて
多様化されるかぎりで、形相は多様性のもとにある。したがって、諸形相が存在するかぎりに
おいて、その意味で諸形相が多様であるのは明らかである。なぜなら、形相が「存在するもの」と言われ
るのは、形相が存在の基体だからではなく、形相によって何かが存在するから
 である。



\end{longtable}



\end{document}