\documentclass[10pt]{jsarticle} % use larger type; default would be 10pt
%\usepackage[utf8]{inputenc} % set input encoding (not needed with XeLaTeX)
%\usepackage[round,comma,authoryear]{natbib}
%\usepackage{nruby}
\usepackage{okumacro}
\usepackage{longtable}
%\usepqckage{tablefootnote}
\usepackage[polutonikogreek,english,japanese]{babel}
%\usepackage{amsmath}
\usepackage{latexsym}
\usepackage{color}

%----- header -------
\usepackage{fancyhdr}
\lhead{{\it Summa Theologiae} III, q.~17}
%--------------------

\bibliographystyle{jplain}

\title{{\bf TERTIA PARS}\\{\HUGE Summae Theologiae}\\Sancti Thomae
Aquinatis\\{\sffamily QUEAESTIO DECIMASEPTIMA}\\DE UNITATE CHRISTI QUANTUM AD ESSE}
\author{Japanese translation\\by Yoshinori {\sc Ueeda}}
\date{Last modified \today}


%%%% コピペ用
%\rhead{a.~}
%\begin{center}
% {\Large {\bf }}\\
% {\large }\\
% {\footnotesize }\\
% {\Large \\}
%\end{center}
%
%\begin{longtable}{p{21em}p{21em}}
%
%&
%
%
%\\
%\end{longtable}
%\newpage



\begin{document}
\maketitle
\pagestyle{fancy}

\begin{center}
{\Large 第十七問\\キリストの一性について:存在にかんして}
\end{center}

\begin{longtable}{p{21em}p{21em}}
Deinde considerandum est de his quae pertinent ad unitatem in Christo in
 communi. Nam de his quae pertinent ad unitatem vel pluralitatem in
 speciali, suis locis determinandum est, sicut supra determinatum est
 quod in Christo non una est tantum scientia; et infra determinabitur
 quod in Christo non una est tantum nativitas. Considerandum est ergo
 primo, de unitate Christi quantum ad esse; secundo, quantum ad velle;
 tertio, quantum ad operari. Circa primum quaeruntur duo. 

\begin{enumerate}
 \item utrum Christus sit unum vel duo.
 \item utrum in Christo sit tantum unum esse.
\end{enumerate}


&

次に、キリストにおける一性に属する事柄について、共通的に考察されるべきで
 ある。
というのも、一性や複数性に属する事柄について、特殊的には、前に、キリスト
 においてはただ一つだけの知があるのではない、と決定されたように、また、後
 で、キリストにおいてはただ一つの出生があるのではない、と決定されるであ
 ろうように、それぞれの場所において決定されるべきだからである。
ゆえに、第一に、キリストの一性について、存在にかんする限りで、第二に、意
 志にかんする限りで、第三に、働きにかんする限りで考察されるべきである。
第一にかんしては二つのことが問われる。

\begin{enumerate}
 \item キリストは一か二か。
 \item キリストにおいて、ただ一つの存在があるか。
\end{enumerate}


\end{longtable}

\newpage



\rhead{a.~1}
\begin{center}
 {\Large {\bf ARTICULUS PRIMUS}}\\
 {\large UTRUM CHRISTUS SIT UNUM VEL DUO}\\
 {\footnotesize III {\itshape Sent.}, d.4, q.2, a.1; IV {\itshape SCG.},
 cap.38, {\itshape De Unione Verb.}, a.2, ad 2, 7; a.3; {\itshape
 Cont.~Graec., Armen.~etc.}, cap.6}\\
 {\Large 第一項\\キリストは一か二か}
\end{center}

\begin{longtable}{p{21em}p{21em}}

{\Huge A}{\scshape d primum sic proceditur}. Videtur quod
 Christus non sit unum, sed duo. Dicit enim Augustinus, in I {\itshape de Trin}.,
 {\itshape quia forma Dei accepit formam servi, utrumque Deus, propter accipientem
 Deum, utrumque homo, propter acceptum hominem}. Sed {\itshape utrumque} dici non
 potest ubi non sunt duo. Ergo Christus est duo.


&

第一項の問題へ、議論は以下のように進められる。
キリストは一ではなく二であると思われる。理由は以下の通り。アウグスティヌ
 スは『三位一体論』で以下のように述べている。「神の形相が僕の形相を受け取ったので、受け取る神のためにどちらも神であり、受け取られた人間のために、どちらも人間である」。ところで、「どちらも」は、二でないところでは語られえない。ゆえに、キリストは二である。

\\


{\scshape 2 Praeterea}, ubicumque est {\itshape aliud et aliud},
 ibi sunt duo. Sed Christus est aliud et aliud, dicit enim Augustinus,
 in {\itshape Enchirid}., {\itshape cum in forma Dei esset, formam servi accepit, utrumque
 unus, sed aliud propter verbum, aliud propter hominem}. Ergo Christus
 est duo.


&

さらに、「一つは〜、もう一つは〜」ということがあるときにはいつも、そこには「二」がある。
ところが、キリストは「一つは〜、もう一つは〜」である。なぜなら、アウグスティヌスは『エンキリディオン』で、「神の形相においてあり、僕の形相を受け取ったので、どちらも一である。ただ、一つは言葉のために、もう一つは人間のために」と述べるからである。ゆえに、キリストは二である。


\\


{\scshape 3 Praeterea}, Christus non est tantum homo,
 quia, si purus homo esset, non esset Deus. Ergo est aliquid aliud quam
 homo. Et ita in Christo est {\itshape aliud et aliud}. Ergo Christus est duo.


&


さらに、キリストはたんに人間であるだけではない。なぜなら、もし純粋に人間であったならば、神ではなかったであろう。ゆえに、人間以外の何かである。したがって、キリストにおいて、あるものは何かであり別のものは別の何かである。ゆえに、キリストは二である。

\\


{\scshape 4 Praeterea}, Christus est aliquid quod est
 pater, et est aliquid quod non est pater. Ergo Christus est {\itshape aliquid et
 aliquid}. Ergo Christus est duo.


&

さらに、キリストは、父である何かであり、かつ、父ではない何かである。ゆえに、キリストは、何かであり、かつ、(それとは別の)何かである。ゆえに、キリストは二である。


\\


{\scshape 5 Praeterea}, sicut in mysterio Trinitatis
 sunt tres personae in una natura, ita in mysterio incarnationis sunt
 duae naturae in una persona. Sed propter unitatem naturae, non obstante
 distinctione personae, pater et filius sunt unum, secundum illud
 Ioan.~{\scshape x}, {\itshape ego et pater unum sumus}. Ergo, non obstante unitate personae,
 propter dualitatem naturarum Christus est duo.

&

さらに、三位一体の秘密において、三つのペルソナが一つの本性においてある
ように、受肉の秘密においては、二つの本性が一つのペルソナにおいてある。
ところで、本性の一性のために、ペルソナの区別を阻害することなく、父と息
子は一である。これは「ヨハネによる福音書」第10章「私と父は一つである」
による。ゆえに、ペルソナの一性を阻害することなく、本性の二性のために、
キリストは二である。



\\


{\scshape 6 Praeterea}, philosophus dicit, in III
 {\itshape Physic}., quod {\itshape unum} et {\itshape duo} denominative dicuntur. Sed Christus habet
 dualitatem naturarum. Ergo Christus est duo.


&

さらに、哲学者は『自然学』第三巻で「一」と「二」は、派生的に語られると
述べている。ところが、キリストは、本性の二性をもっている。ゆえに、キ
リストは二である。



\\


{\scshape 7 Praeterea}, sicut forma accidentalis facit
 {\itshape alterum}, ita forma substantialis {\itshape aliud}, ut Porphyrius dicit. Sed in
 Christo sunt duae naturae substantiales, humana scilicet et
 divina. Ergo Christus est aliud et aliud. Ergo Christus est duo.


&


さらに、ポルピュリウスが言うように、附帯的形相は他のようにするが、実体
的形相は他のものにする。ところで、キリストの中には二つの実体的本性、す
なわち人間の本性と神の本性とがある。ゆえに、キリストは、或るものであり、
かつ、(それとは違う)何か或るものである。ゆえに、キリストは二である。

\\


{\scshape Sed contra est} quod Boetius dicit, in
 libro {\itshape De Duabus Naturis}, {\itshape omne quod est, inquantum est, unum est}. Sed
 Christum esse confitemur. Ergo Christus est unum.


&

しかし反対に、ボエティウスが『二つの本性について』で述べるように、存在
するものはすべて、存在するかぎり、一である。ところで、キリストが存在す
ることを私たちは告白する。ゆえに、キリストは一である。


\\


{\scshape Respondeo dicendum} quod natura, secundum se
 considerata, prout in abstracto significatur, non vere potest
 praedicari de supposito seu persona nisi in Deo, in quo non differt
 quod est et quo est, ut in prima parte habitum est. 


&

解答する。以下のように言われるべきである。第一部で論じられたとおり、そ
れ自体に即して考察され、抽象的に表示されるかぎりでの本性は、「あるとこ
ろのもの」と「それによってあるところのそれ」が異ならない神以外において
は、個体やペルソナについて、真には述語されえない。

\\

In Christo autem
 cum sint duae naturae, divina scilicet et humana, altera earum,
 scilicet divina, potest de eo praedicari et in abstracto et in
 concreto, dicimus enim quod filius Dei, qui supponitur in hoc nomine
 Christus, est divina natura, et est Deus. 


&

ところで、キリストの中には、神と人間の二つの本性があるので、そのうちの
一方、すなわち神の本性は、抽象的にも具体的にも述語されうる。たとえば、
私たちは、「キリスト」という名で示される神の子が、神の本性であると言う
し、また、神であるとも言う。

\\

Sed humana natura non potest
 praedicari de Christo secundum se in abstracto, sed solum in concreto,
 prout scilicet significatur in supposito. Non enim vere potest dici
 quod Christus sit humana natura, quia natura humana non est nata
 praedicari de suo supposito, dicitur autem quod Christus est homo,
 sicut et quod Christus est Deus. Deus autem significat habentem
 deitatem, et homo significat habentem humanitatem. 


&

これに対して、人間の本性は、キリストそれ自体について、抽象的には述語さ
れえず、ただ具体的に、すなわち、個体において表示されるかぎりで、述語さ
れうる。じっさい、キリストが人間本性であるとは、真には語られえない。な
ぜなら、人間本性は、それの個体について述語されるようには作られていない
からである。他方、キリストは人間である、とは言われる。それは、キリスト
は神であると言われるのと同様である。神は、神の本性を持つものを表示し、
人間は、人間の本性を持つものを表示する。


\\

Aliter tamen habens
 humanitatem significatur per hoc nomen homo, et aliter per hoc nomen
 Iesus, vel Petrus. Nam hoc nomen homo importat habentem humanitatem
 indistincte, sicut et hoc nomen Deus indistincte importat habentem
 deitatem. Hoc tamen nomen Petrus, vel Iesus, importat distincte
 habentem humanitatem, scilicet sub determinatis individualibus
 proprietatibus, sicut et hoc nomen filius Dei importat habentem
 deitatem sub determinata proprietate personali. 

&

ところで、人間本性を持つものが「人間」という名で表示される場合と、「イ
エス」や「ペトロ」という名で表示される場合とでは異なる。すなわち、この
「人間」という名は、人間の本性を持つものを、不分明に意味する。それはちょ
うど、「神」という名が、神性をもつものを不分明に意味するのと同様である。
しかし、「ペトロ」や「イエス」という名は、人間の本性を持つものを、判明
に、すなわち、限定された諸々の個的な固有性のもとで意味する。それはちょ
うど、この「神の子」という名も、神性をもつものを、限定されたペルソナ的
固有性のもとに意味するのと同様である。



\\


Numerus autem
 dualitatis in Christo ponitur circa ipsas naturas. Et ideo, si ambae
 naturae in abstracto praedicarentur de Christo, sequeretur quod
 Christus esset duo. 

&


さて、キリストにおける二という数は、その本性を巡って措定される。ゆえに、
もし、両方の本性が、キリストについて抽象的に述語されたならば、キリスト
は二であることになっていたであろう。

\\

Sed quia duae naturae non praedicantur de Christo
 nisi prout significantur in supposito, oportet secundum rationem
 suppositi praedicari de Christo unum vel duo. 

&


しかし、二つの本性は、キリストについて、個体において表示されるようにで
なければ述語されないので、個体の性格に即して、一または二が、キリストに
ついて述語されなければならない。

\\

Quidam autem posuerunt in
 Christo duo supposita, sed unam personam, quae quidem videtur se
 habere, secundum eorum opinionem, tanquam suppositum completum ultima
 completione. 


&

さて、ある人々は、キリストの中に二つの個体を措定したが、ペルソナは一つ
措定した。彼らの意見によれば、ペルソナは、究極の完成によって、完成され
た個体として関係するように思われる。


\\

Et ideo, quia ponebant in Christo duo supposita, dicebant Christum
esse duo neutraliter; sed quia ponebant unam personam, dicebant
Christum esse unum masculine, nam neutrum genus designat quiddam
informe et imperfectum; genus autem masculinum designat quiddam
formatum et perfectum. Nestoriani autem, ponentes in Christo duas
personas, dicebant Christum non solum esse duo neutraliter, sed etiam
duos masculine.

&

ゆえに、彼らはキリストの中に二つの個体を措定したので、キリストは、中性
として二であるが、他方で一つのペルソナを措定したので、男性としては一で
あると言った。つまり、中性の類は、あるものを形相を欠き不完全なものとし
て表示するのに対して、男性の類は、形相をとり完全なものとして表示すると
いうわけである。他方、ネストリアヌス派の人々は、キリストの中に二つのペ
ルソナを措定するので、キリストは、中性として二であるだけでなく、男性と
しても二だと言った。



\\

Quia vero nos ponimus in Christo unam personam et unum suppositum, ut
ex praedictis patet, sequitur quod dicamus quod non solum Christus est
unus masculine, sed etiam est unum neutraliter.


&

さて、私たちは、前述のことから明らかなとおり\footnote{第二問第二項、第
 三項。}、キリストの中に一つのペルソナと一つの個体を措定するので、キリ
 ストは男性として一であるだけでなく、中性としても一であると語ることが
 帰結する。



\\


{\scshape Ad primum ergo dicendum} quod verbum illud Augustini non est
sic intelligendum quod ly utrumque teneatur ex parte praedicati, quasi
dicat quod Christus sit utrumque, sed tenetur ex parte subiecti. Et
tunc ly utrumque ponitur, non quasi pro duobus suppositis, sed pro
duobus nominibus significantibus duas naturas in concreto. Possum enim
dicere quod utrumque, scilicet Deus et homo, est Deus, propter
accipientem Deum, et utrumque scilicet Deus et homo, est homo, propter
acceptum hominem.


&

第一異論に対しては、それゆえ、以下のように言われるべきである。アウグス
ティヌスのかの言葉は、「キリストはどちらもである」というように、「どち
らも」が述語の側で理解されるべきではなく、主語の側で理解されるべきであ
る。その場合、「どちらも」は、二つの個体としてではなく、二つの本性を具
体的に表示する二つの名前として措定される。じっさい、私は、「どちらも」
すなわち神と人間が、受け入れる神のために神であり、「どちらも」すなわち
神と人間が、受け入れられた人間のために人間である、と言うことができる。

\\


{\scshape Ad secundum dicendum} quod, cum dicitur, Christus est aliud
et aliud locutio est exponenda ut sit sensus, habens aliam et aliam
naturam. Et hoc modo exponit Augustinus in libro contra Felicianum,
ubi, cum dixisset, {\itshape in mediatore Dei et hominum aliud Dei
filius, aliud hominis filius} subdit, {\itshape aliud, inquam, pro
discretione substantiae, non alius, pro unitate personae}. Et
Gregorius Nazianzenus, in epistola ad Chelidonium, si oportet
compendiose dicere, aliud quidem et aliud ea ex quibus salvator est,
siquidem non idem est invisibile visibili, et quod absque tempore ei
quod sub tempore. Non autem alius et alius: absit. Haec enim ambo
unum.


&

第二異論に対しては、以下のように言われるべきである。キリストが、「一つ
は〜、もう一つは〜」と言われるとき、この語り方は、「ひとつの本性と、そ
れとは別の本性をもつ」という意味になるように説明されるべきである。そし
て、アウグスティヌスは、『フェリキアヌス駁論』という書物の中で、「神と
人間の仲介者において、一つは神の子、もう一つは人間の子」と述べたときに、
「私が、「一つは」と〔中性で〕言うのは、実体を区別するためであり、「一
人は」と〔男性で〕言わないのは、ペルソナの一性のためである」と補足して
いる。また、ナジアンゾスのグレゴリウスは、ケルドニオス宛書簡で「もし簡
潔に語らなければならないならば、救い主である者の側から、「一つは〜、も
う一つは〜」である。目に見えないものは、目に見えるものに、時間なしにあ
るものが、時間のもとにあるものに、同じではないからである。しかし、「一
人は〜、もう一人は〜」ではない。まったく違う。これらはどちらも一つだか
らである。


\\


{\scshape Ad tertium dicendum} quod haec est falsa, Christus est
tantum homo, quia non excludit aliud suppositum, sed aliam naturam, eo
quod termini in praedicato positi tenentur formaliter. Si vero
adderetur aliquid per quod traheretur ad suppositum, esset vera
locutio, puta, Christus est tantum id quod est homo. Non tamen
sequeretur quod sit aliquid aliud quam homo, quia ly aliud, cum sit
relativum diversitatis substantiae, proprie refertur ad suppositum,
sicut et omnia relativa facientia personalem relationem. Sequitur
autem, ergo habet aliam naturam.


&

第三異論に対しては、以下のように言われるべきである。「キリストはたんな
る人間である」が偽であるのは、他の個体を排除しているからではなく、他の
本性を排除しているからである。なぜなら、述語の位置に置かれた語は、形相
的に理解されるからである。しかし、たとえば、「キリストは、たんに、人間
であるものである」のように、もしも、それによって個体へと注意が向けられ
るような何かが加えられていたならば、その表現は真であっただろう。しかし
このことから、人間以外の何かであることは帰結しない。なぜなら、この「何
か」は、実体の異なに関係するので、固有の意味では、個体に関係するからで
ある。それは、ペルソナの関係を作り出すすべての関係と同様である。しかし、
「それゆえ、他の本性をもつ」ということが帰結する。


\\


{\scshape Ad quartum dicendum} quod, cum dicitur,
 Christus est aliquid quod est pater, ly aliquid tenetur pro natura
 divina, quae etiam in abstracto praedicatur de patre et filio. Sed cum
 dicitur, Christus est aliquid quod non est pater, ly aliquid tenetur
 non pro ipsa natura humana secundum quod significatur in abstracto, sed
 secundum quod significatur in concreto; non quidem secundum suppositum
 distinctum, sed secundum suppositum indistinctum; prout scilicet
 substat naturae, non autem proprietatibus individuantibus. Et ideo non
 sequitur quod Christus sit aliud et aliud, vel quod sit duo, quia
 suppositum humanae naturae in Christo quod est persona filii Dei, non
 ponit in numerum cum natura divina, quae praedicatur de patre et filio.


&

第四異論については、以下のように言われるべきである。
「キリストは、父である何かである」と言われるとき、この「何か」は、神の本
 性の意味に理解され、それはまた、抽象的に、父と子に述語される。しかし、
 「キリストは、父でない何かである」と言われるときは、この「何か」は、抽
 象的に表示される限りでの人間本性それ自体として理解されるのではなく、具
 体的に表示される限りでのそれとして理解される。また、区別された個体とし
 てではなく、不分明な個体としてである。すなわち、個体化する諸固有性では
 なく、本性のもとにあるものとして。ゆえに、キリストが、「一つは〜、もう一
 つは〜」であることや、二であることは帰結しない。なぜなら、神の子のペル
 ソナであるキリストの中にある人間本性の個体は、父と子に述語される神の本
 性と、同じレベルで数えられることはないからである。

\\


{\scshape Ad quintum dicendum} quod in mysterio
 divinae Trinitatis natura divina praedicatur in abstracto de tribus
 personis, et ideo simpliciter potest dici quod tres personae sint
 unum. Sed in mysterio incarnationis non praedicantur ambae naturae in
 abstracto de Christo, et ideo non potest simpliciter dici quod Christus
 sit duo.


&

第五異論については、以下のように言われるべきである。
神の三位一体の秘密において、神の本性は、抽象的に、三つのペルソナについて
 述語される。それゆえ、端的に、三つのペルソナが一つであると言われうる。
しかし、受肉の秘密においては、両方の本性が、抽象的に、キリストについて述
 語されない。それゆえ、キリストが二であると端的に語られることはできない。


\\


{\scshape Ad sextum dicendum} quod duo dicitur quasi
 habens dualitatem, non quidem in aliquo alio, sed in ipso de quo duo
 praedicantur. Fit autem praedicatio de supposito, quod importatur per
 hoc nomen Christus. Quamvis igitur Christus habeat dualitatem
 naturarum, quia tamen non habet dualitatem suppositorum, non potest
 dici esse duo.


&

第六異論に対しては、以下のように言われるべきである。
二は、何か別のものにおいて二性をもつものとして語られるのではなく、それに
 ついて二が述語付けられるものにおいて、二性をもつものとして語られる。
ところで、述語付けは、「キリスト」というこの名によって意味される個体につ
 いてなされる。ゆえに、キリストが本性の二性をもっているとはいえ、個体の
 二性をもっていないので、二であると言われることはできない。


\\


{\scshape Ad septimum dicendum} quod alterum importat
 diversitatem accidentis, et ideo diversitas accidentis sufficit ad hoc
 quod aliquid simpliciter dicatur alterum. Sed aliud importat
 diversitatem substantiae. Substantia autem dicitur non solum natura,
 sed etiam suppositum, ut dicitur in {\itshape V Metaphys}. Et ideo diversitas
 naturae non sufficit ad hoc quod aliquid simpliciter dicatur aliud,
 nisi adsit diversitas secundum suppositum. Sed diversitas naturae facit
 aliud secundum quid, scilicet secundum naturam, si non adsit diversitas
 suppositi.


&

第七異論については、以下のように言われるべきである。
「他のように」は附帯性の異なりを意味するので、何かが端的に「他のように」
 と言われるためには、附帯性の異なりで十分である。これに対して「他のもの」
 は、実体の異なりを意味する。ところが、『形而上学』第五巻で言われるよう
 に、本性だけでなく個体もまた実体と言われる。ゆえに、本性の異なりは、個
 体における異なりがそこにないかぎり、何
 かが端的に「他のもの」であると言われるために十分でない。
他方、個体の異なりがそこにない場合には、本性の異なりは、ある意味における、
 つまり本性における「他のもの」を生む。



\end{longtable}

\newpage


\rhead{a.~2}
\begin{center}
 {\Large {\bf ARTICULUS SECUNDUS}}\\
 {\large UTRUM IN CHRISTO SIT TANTUM UNUM ESSE}\\
 {\footnotesize III {\itshape Sent.}, d.4, q.1, a.2, qu$^a$1; d.6, q.2, a.2;
 d.14, a.1, qu$^a$1; d.18, a.1, ad 3; {\itshape De Unione Verbi}, a.1,
 ad 10; a.4; {\itshape Quodl}.~IX, q.2, a.2; {\itshape Compend.~Theol.},
 cap.212.}\\
 {\Large 第二項\\キリストの中にただ一つの存在があるか}
\end{center}

\begin{longtable}{p{21em}p{21em}}

{\Huge A}{\scshape d secundum sic proceditur}. Videtur quod
 in Christo non sit tantum unum esse, sed duo. Dicit enim Damascenus, in
 III libro, quod ea quae consequuntur naturam in Christo
 duplicantur. Sed esse consequitur naturam, esse enim est a forma. Ergo
 in Christo sunt duo esse.


&

第二項の問題へ、議論は以下のように進められる。キリストの中に、ただ一つの
 存在があるのではなく、二つの存在があると思われる。理由は以下の通り。ダ
 マスケヌスは第三巻で、「本性にしたがうものどもは、キリストにおいて二つ
 にされる」と述べている。ところで、存在は本性にしたがう。なぜなら、存在
 は、形相によってあるからである。ゆえに、キリストの中に二つの存在がある。


\\



{\scshape 2 Praeterea}, esse filii Dei est ipsa divina
 natura, et est aeternum. Esse autem hominis Christi non est divina
 natura, sed est esse temporale. Ergo in Christo non est tantum unum
 esse.


&

さらに、神の子の存在は、神の本性それ自体であり、永遠である。ところで、キ
 リストという人間の存在は、神の本性ではなく、時間的である。ゆえに、キリ
 ストにおいて、ただ一つの存在があるのではない。


\\



{\scshape 3 Praeterea}, in Trinitate, quamvis sint
 tres personae, est tamen unum esse, propter unitatem naturae. Sed in
 Christo sunt duae naturae, quamvis sit una persona. Ergo in Christo non
 est unum esse tantum, sed duo.


&

さらに、三位一体において、三つのペルソナがあるが、本性の一性のため、た
だ一つの存在がある。ところで、キリストは一つのペルソナだが、その中に二
つの本性がある。ゆえに、キリストにおいて、ただ一つの存在があるのではな
く、二つの存在がある。


\\



{\scshape 4 Praeterea}, in Christo anima dat aliquod esse corpori, cum
sit forma eius. Sed non dat sibi esse divinum, cum sit increatum. Ergo
in Christo est aliud esse praeter esse divinum. Et sic in Christo non
est tantum unum esse.


&

さらに、キリストの中で、魂は身体の形相なので、魂は身体に何かを与える。
しかし、自らに神の存在を与えるのではない。神の存在は創造されざるものだ
からである。ゆえに、キリストの中に、神の存在とは異なる別の存在がある。
この意味で、キリストの中にただ一つの存在があるのではない。


\\



{\scshape Sed contra}, unumquodque, secundum quod dicitur ens, dicitur
unum, quia unum et ens convertuntur. Si ergo in Christo duo essent
esse, et non tantum unum, Christus esset duo, et non unum.


&

しかし反対に、各々のものは、在るものと言われる限りにおいて、一である。
なぜなら、一と在るものは置換されるからである。ゆえに、キリストの中に、
ただ一つではなく二つの存在があったならば、キリストは一ではなく二であっ
たであろう。


\\



{\scshape Respondeo dicendum} quod, quia in Christo sunt duae naturae
et una hypostasis, necesse est quod ea quae ad naturam pertinent in
Christo sint duo, quae autem pertinent ad hypostasim in Christo sint
unum tantum.

&

解答する。以下のように言われるべきである。キリストの中には二つの本性と
一つのヒュポスタシスがあるので、キリストの中で本性に属するものは二であ
り、キリストの中でヒュポスタシスに属するものはただ一つであることが必然
である。


\\

Esse autem pertinet ad hypostasim et ad naturam, ad hypostasim quidem
sicut ad id quod habet esse; ad naturam autem sicut ad id quo aliquid
habet esse; natura enim significatur per modum formae, quae dicitur
ens ex eo quod ea aliquid est, sicut albedine est aliquid album, et
humanitate est aliquis homo.

&

ところで、存在は、ヒュポスタシスと本性に属するが、ヒュポスタシスには、
存在をもつものに対するように、また、本性には、それによって何かが存在を
持つところのそれに対するように関係する。なぜなら、本性は、形相のあり方
によって表示されるが、形相は、それによって何かがあるということに基づい
て、在るものと言われるからである。たとえば、白性によって何かが白くあり、
人間性によって、何かが人間であるように。


\\

Est autem considerandum quod, si aliqua forma vel natura est quae non
pertineat ad esse personale hypostasis subsistentis, illud esse non
dicitur esse illius personae simpliciter, sed secundum quid, sicut
esse album est esse Socratis, non inquantum est Socrates, sed
inquantum est albus.


&

ところで、以下のことが考察されるべきである。すなわち、もしある形相や本
性が、自存するヒュポスタシスのペルソナ的存在に属さないならば、その存在
は、そのペルソナの存在と端的には言われず、ある意味において言われる。た
とえば、「白くある」ということがソクラテスの存在であるのは、それがソク
ラテスである限りにおいてではなく、それが白い者である限りにおいてである
ように。


\\

Et huiusmodi esse nihil prohibet multiplicari in una hypostasi vel
persona, aliud enim est esse quo Socrates est albus, et quo Socrates
est musicus.

&

そして、そのような存在が、一つのヒュポスタシスやペルソナにおいて、多数
化されることは差し支えない。たとえば、ソクラテスがそれによって白くある
ところの存在と、それによって教養的であるところの存在とは異なるように。


\\


Sed illud esse quod pertinet ad ipsam hypostasim vel personam secundum
se impossibile est in una hypostasi vel persona multiplicari, quia
impossibile est quod unius rei non sit unum esse.

&

しかし、ヒュポスタシスやペルソナ自体に、自体的に属する存在が、一つのヒュ
ポスタシスやペルソナにおいて多数化されることは不可能である。なぜなら、
一つの事物に一つの存在がないということは不可能だから。


\\

Si igitur humana natura adveniret filio Dei, non hypostatice vel
personaliter, sed accidentaliter, sicut quidam posuerunt, oporteret
ponere in Christo duo esse, unum quidem secundum quod est Deus; aliud
autem secundum quod est homo. Sicut in Socrate ponitur aliud esse
secundum quod est albus, aliud secundum quod est homo, quia esse album
non pertinet ad ipsum esse personale Socratis.


&

ゆえに、ある人々が論じたように、もし人間本性が神の息子に到来するという
ことが、ヒュポスタシスやペルソナとしてではなく、附帯的にであったならば、
キリストの中に二つの存在を措定しなければならなかったであろう。一つはそ
れが神であるかぎりで、もう一つは、それが人間であるかぎりで。ちょうど、
ソクラテスにおいて、それが白いものであるかぎりでと、人間であるかぎりと
では、別々の存在が措定されるように。というのも、白くあることは、ソクラ
テスのペルソナ的な存在自体には属さないからである。



\\


Esse autem capitatum, et esse corporeum, et esse animatum, totum
pertinet ad unam personam Socratis, et ideo ex omnibus his non fit
nisi unum esse in Socrate. Et si contingeret quod, post constitutionem
personae Socratis, advenirent Socrati manus vel pedes vel oculi, sicut
accidit in caeco nato, ex his non accresceret Socrati aliud esse, sed
solum relatio quaedam ad huiusmodi, quia scilicet diceretur esse non
solum secundum ea quae prius habebat, sed etiam secundum ea quae
postmodum sibi adveniunt.


&

これに対して、頭があること、物体的であること、魂があること、これらすべ
てはソクラテスの一つのペルソナに属するので、これらすべてからはソクラテ
スにおいて一つの存在しか生じない。そしてもし、ちょうど生まれつき目が見
えない人に生じるようなかたちで、ソクラテスのペルソナの構成の後に、ソク
ラテスの手や足や目が到来したならば、それらから、ソクラテスに他の存在が
増強されるのではなく、むしろ、そのようなものに対する何らかの関係だけが
加えられる。なぜなら、そのような場合には、以前からもっているものに即し
てだけでなく、後に彼に到来したものに即しても、「ある」と言われたであろ
うから。

\\


Sic igitur, cum humana natura coniungatur filio Dei hypostatice vel
personaliter, ut supra dictum est, et non accidentaliter, consequens
est quod secundum humanam naturam non adveniat sibi novum esse
personale, sed solum nova habitudo esse personalis praeexistentis ad
naturam humanam, ut scilicet persona illa iam dicatur subsistere, non
solum secundum naturam divinam, sed etiam humanam.


&

したがって、前に述べられたとおり、人間の本性は、附帯的にではなくヒュポ
 スタシスないしペルソナ的に神の子に結びつけられるので、人間本性に即し
 て、新たなペルソナ的存在が彼に到来することはなく、ただ、前もって存在
 するペルソナ的存在の、人間本性に対する新たな関係だけが到来する。すな
 わち、かのペルソナは、神の本性に即してだけでなく、その場合にはすでに
 人間本性に即しても、自存すると言われる。


\\



{\scshape Ad primum ergo dicendum} Quod esse consequitur naturam, non
sicut habentem esse, sed sicut qua aliquid est, personam autem, sive
hypostasim, consequitur sicut habentem esse. Et ideo magis retinet
unitatem secundum unitatem hypostasis, quam habeat dualitatem secundum
dualitatem naturae.


&

第一異論に対しては、それゆえ、以下のように言われるべきである。本性に存
在が伴うのは、存在をもつものに存在が伴うというかたちではなく、それによっ
て何かが存在するところのそれに、存在が伴うというかたちである。他方、ペ
ルソナやヒュポスタシスには、存在をもつものに伴うかたちで、存在が伴う。
ゆえに、本性の二性によって二性をもつよりも、ヒュポスタシスの一性によっ
て、より一性を強く保持する。

\\



{\scshape Ad secundum dicendum} quod illud esse aeternum filii Dei
quod est divina natura, fit esse hominis, inquantum humana natura
assumitur a filio Dei in unitate personae.


&

第二異論に対しては、以下のように言われるべきである。神の本性である神の
息子の永遠な存在は、人間本性が、ペルソナの一性において神の息子に受容さ
れるかぎりで、人間の存在となる。


\\



{\scshape Ad tertium dicendum} quod, sicut in prima parte dictum est,
quia persona divina est idem cum natura, in personis divinis non est
aliud esse personae praeter esse naturae, et ideo tres personae non
habent nisi unum esse. Haberent autem triplex esse, si in eis esset
aliud esse personae, et aliud esse naturae.


&

第三異論に対しては、以下のように言われるべきである。第一部で述べられた
ように、神のペルソナは本性と同一なので、神のペルソナにおいて、ペルソナ
の存在は本性の存在と別のものではない。ゆえに、三つのペルソナは一つの存
在しか持たない。しかしもし、神において、ペルソナの存在と本性の存在が別
のものであったならば、ペルソナは三つの存在を持ったであろう。

\\



{\scshape Ad quartum dicendum} quod anima in Christo dat esse corpori
inquantum facit ipsum actu animatum, quod est dare ei complementum
naturae et speciei. Sed si intelligatur corpus perfectum per animam
absque hypostasi habente utrumque, hoc totum compositum ex anima et
corpore, prout significatur nomine humanitatis, non significatur ut
quod est, sed ut quo aliquid est. Et ideo ipsum esse est personae
subsistentis, secundum quod habet habitudinem ad talem naturam, cuius
habitudinis causa est anima inquantum perficit humanam naturam
informando corpus.


&

第四異論に対しては、以下のように言われるべきである。魂は、キリストにお
いて、キリストを現実に魂あるものにする限りで、身体に存在を与える。これ
は、彼に、本性と種の完成を与えることである。しかしもし、魂と身体の両方
をもつヒュポスタシスを捨象して、「人間性」という名で表示されるものとし
ての、魂によって完成された身体が理解されるならば、魂と身体からのこの複
合体全体は、「在るもの」としてではなく、「それによって何かが在るところ
のそれ」として表示される。それゆえ、存在それ自体は、そのような本性に関
係する限りで、自存するペルソナに属する。そしてその関係の原因は、身体に
形相を与えることによって人間本性を完成する限りで、魂である。



\end{longtable}

\end{document}