\documentclass[10pt]{jsarticle} % use larger type; default would be 10pt
%\usepackage[utf8]{inputenc} % set input encoding (not needed with XeLaTeX)
%\usepackage[round,comma,authoryear]{natbib}
%\usepackage{nruby}
\usepackage{okumacro}
\usepackage{longtable}
%\usepqckage{tablefootnote}
\usepackage[polutonikogreek,english,japanese]{babel}
%\usepackage{amsmath}
\usepackage{latexsym}
\usepackage{color}

%----- header -------
\usepackage{fancyhdr}
\lhead{{\it Summa Theologiae} I, q.~104}
%--------------------

\bibliographystyle{jplain}

\title{{\bf PRIMA PARS}\\{\HUGE Summae Theologiae}\\Sancti Thomae
Aquinatis\\{\sffamily QUEAESTIO CENTESIMAQUARTA}\\DE EFFECTIBUS DIVINAE
GUBERNATIONIS IN SPECIALI}
\author{Japanese translation\\by Yoshinori {\sc Ueeda}}
\date{Last modified \today}


%%%% コピペ用
%\rhead{a.~}
%\begin{center}
% {\Large {\bf }}\\
% {\large }\\
% {\footnotesize }\\
% {\Large \\}
%\end{center}
%
%\begin{longtable}{p{21em}p{21em}}
%
%&
%
%
%\\
%\end{longtable}
%\newpage



\begin{document}
\maketitle
\pagestyle{fancy}

\begin{center}
{\Large 第104問\\神の統治の結果について特殊的に}\\

\end{center}

\begin{longtable}{p{21em}p{21em}}

Deinde considerandum est de effectibus divinae
 gubernationis in speciali. Et circa hoc quaeruntur quatuor. 

\begin{enumerate}
 \item utrum creaturae indigeant ut conserventur in esse a Deo.
 \item utrum conserventur a Deo immediate.
 \item utrum Deus possit aliquid redigere in nihilum.
 \item utrum aliquid in nihilum redigatur.
\end{enumerate}


&

次に、神の統治の諸結果について、特殊的に考察されるべきである。これを巡っ
 て、四つのことが問われる。

\begin{enumerate}
 \item 被造物は、神によって存在に保たれることを必要とするか。
 \item 神によって直接的に保たれるか。
 \item 神は何かを無へ帰すことができるか。
 \item 何かが無へ帰されるか。
\end{enumerate}

\end{longtable}
\newpage
\rhead{a.~1}
\begin{center}
 {\Large {\bf ARTICULUS PRIMUS}}\\
 {\large UTRUM CREATURAE INDIGEANT UT A DEO CONSERVENTUR}\\
 {\footnotesize III {\itshape Cont.~Gent.}, cap.~65; {\itshape De Pot.},
 q.~5, a.~1; {\itshape In Ioan.}, cap.~5, letct.~2; {\itshape Hebr.},
 cap.~1, lect.~2.}\\
 {\Large 第一項\\被造物は神によって保たれることを必要とするか}
\end{center}

\begin{longtable}{p{21em}p{21em}}


{\huge A}{\scshape d primum sic proceditur}. Videtur quod
creaturae non indigeant ut a Deo conserventur in esse. Quod enim non
potest non esse, non indiget ut conservetur in esse, sicut quod non
potest abscedere, non indiget ut conservetur ne abscedat. Sed quaedam
creaturae sunt quae secundum sui naturam non possunt non esse. Ergo non
omnes creaturae indigent ut a Deo conserventur in esse. 


&

第一項の問題へ、議論は以下のように進められる。被造物は、神によって存在の
 うちに保たれることを必要としないと思われる。理由は以下の通り。
存在しないことが不可能なものは、存在のうちに保たれることを必要としない。
 たとえばそれは、消えることが不可能なものは、消えないように保たれることを必要
 としないようにである。
ところが、ある被造物は、その本性上、存在しないことが不可能である。
ゆえに、すべての被造物が、神によって存在のうちに保たれることを必要とする
 わけではない。


\\



Probatio
mediae. Quod per se inest alicui, necesse est ei inesse, et oppositum
eius impossibile est ei inesse, sicut necessarium est binarium esse
parem, et impossibile est eum esse imparem. Esse autem per se
consequitur ad formam, quia unumquodque secundum hoc est ens actu, quod
habet formam. Quaedam autem creaturae sunt, quae sunt formae quaedam
subsistentes, sicut de Angelis dictum est; et sic per se inest eis
esse. Et eadem ratio est de illis quorum materia non est in potentia
nisi ad unam formam, sicut supra dictum est de corporibus
caelestibus. Huiusmodi ergo creaturae secundum suam naturam ex
necessitate sunt, et non possunt non esse, potentia enim ad non esse non
potest fundari neque in forma, quam per se sequitur esse; neque in
materia existente sub forma quam non potest amittere, cum non sit in
potentia ad aliam formam.


&

小前提\footnote{
\begin{enumerate}
 \item すべて、存在しないことが不可能なものは、神による保存を必要としな
       い。(Emp)
 \item あるものは、存在しないことが不可能である。(Ism)
 \item ゆえに、すべてのものが、神による保存を必要とするわけではない。(Osp)
\end{enumerate}
格式はFerio。
小前提(または小命題)とは、「あるもの(被造物)は、存在しないこ
 とが不可能である」を指す。}の証明。
何かに自体的に内在するものは、それに内在することが必然であり、それの反対
 がそれに内在することが不可能である。たとえば、2が偶数であることは必然で
 あり、それが奇数であることが不可能であるように。ところで、各々のものは、
 それが現実態における存在者であるかぎりで形相をもつから、存在は、自体的
 に形相に伴う。ところが、天使について言われたように\footnote{第1部50問2項、5項。}、ある被造物は、ある
 種の自存する形相であり、したがって、それらには自体的に存在が内在する。
また、前に天体について言われたように\footnote{第1部66問2項。}、ひとつの
 形相だけに対してのみ可能態にあるような質料をもつ物体についても同じことが言える。
それゆえ、このような被造物は、その本性にしたがって、必然的に存在し、存在
 しないことが不可能である。なぜなら、存在しないことへの可能態は、形相に
 基づくことはありえないし、というのも、形相には自体的に存在が伴うからだ
 が、また、他の形相にたいして可能態にあるわけではないので、失うことがあ
 りえない形相のもとにある質料に基づくこともできないからである。

\\


{\scshape 2 Praeterea}, Deus est potentior quolibet
creato agente. Sed aliquod creatum agens potest communicare suo effectui
ut conservetur in esse, etiam eius operatione cessante, sicut cessante
actione aedificatoris, remanet domus; et cessante actione ignis, remanet
aqua calefacta per aliquod tempus. Ergo multo magis Deus potest suae
creaturae conferre quod conservetur in esse, sua operatione cessante.

&

さらに、神は、どんな被造の作用者よりも力がある。ところが、ある被造の作用
 者は、自らが生み出した結果に、その働きが止んだあとでも、存在に保たれる
 ことを伝達することができる。たとえば、建築家の作用が止んでも、家が残り、
 陽の作用が止んでも、一定の時間、水が熱せられた状態に留まるように。ゆえ
 に、ましてや神は、自らの働きが止んでも、自らが生みだした被造物に、存在
 の内に保存されることを伝えることができる。

\\


{\scshape 3 Praeterea}, nullum violentum potest
contingere absque aliqua causa agente. Sed tendere ad non esse est
innaturale et violentum cuilibet creaturae, quia quaelibet creatura
naturaliter appetit esse. Ergo nulla creatura potest tendere in non
esse, nisi aliquo agente ad corruptionem. Sed quaedam sunt ad quorum
corruptionem nihil agere potest; sicut spirituales substantiae, et
corpora caelestia. Ergo huiusmodi creaturae non possunt tendere in non
esse, etiam Dei operatione cessante.

&

さらに、なんらかの作用因がなければ、どんな強制も発生しえない。ところが、
 どんな被造物も本性的に存在を欲求するので、存在しないことへ傾くことは、
どんな被造物にとっても、非本性的で強制的である。
ゆえに、どんな被造物も、消滅へ向かわせるなんらかの作用者によるのでないか
 ぎり、存在しないことへ傾くことはありえない。
ところが、霊的実体や天体のように、それらを消滅へ向かわせるように作用するこ
 とがけっしてできない被造物が存在する。ゆえに、そのような被造物は、神の
 働きが止んでも、存在しないことへ傾くことがありえない。

\\


{\scshape 4 Praeterea}, si Deus conservat res in esse,
hoc erit per aliquam actionem. Per quamlibet autem actionem agentis, si
sit efficax, aliquid fit in effectu. Oportet igitur quod per actionem
Dei conservantis aliquid fiat in creatura. Sed hoc non videtur. Non enim
per huiusmodi actionem fit ipsum esse creaturae, quia quod iam est, non
fit. Neque iterum aliquid aliud superadditum, quia vel non continue Deus
conservaret creaturam in esse, vel continue aliquid adderetur creaturae,
quod est inconveniens. Non igitur creaturae conservantur in esse a Deo.

&

さらに、もし神が事物を存在のうちに保つのなら、それはなんらかの作用による
 であろう。ところが、作用者のどんな作用によっても、もしそれが有効なもの
 であれば、結果のうちに何かが生じる。ゆえに、神の保存の作用によって、被
 造物の中に何かが生じなければならない。しかし、そうは見えない。なぜなら、
 すでに存在しているものが生じたりはしないので、そのような作用によって、
 被造物の存在それ自体が生じることはないし、また、他の何か付け加えられた
 ものが生じるわけではないからである。というのも、その場合には、神は連続
 的に被造物を存在のうちに保っているのではないか、あるいは、連続的に何か
 が被造物に付加されていることになるが、それは不適当である。ゆえに、被造
 物は神によって存在に保たれているのではない。

\\


{\scshape  Sed contra est} quod dicitur {\itshape Heb}.~{\scshape i}, {\itshape portans
omnia verbo virtutis suae}.

&

しかし反対に、『ヘブライ人への手紙』1章で「自らの力の言葉で万物を支える」
 \footnote{「万物をご自分の力ある言葉によって支えておられますが」(1:3)}
 と言われている。
\\


{\scshape Respondeo dicendum} quod necesse est dicere,
et secundum fidem et secundum rationem, quod creaturae conservantur in
esse a Deo. Ad cuius evidentiam, considerandum est quod aliquid
conservatur ab altero dupliciter. Uno modo, indirecte et per accidens,
sicut ille dicitur rem conservare, qui removet corrumpens; puta si
aliquis puerum custodiat ne cadat in ignem, dicitur eum conservare. Et
sic etiam Deus dicitur aliqua conservare, sed non omnia, quia quaedam
sunt quae non habent corrumpentia, quae necesse sit removere ad rei
conservationem.


&

解答する。以下のように言われるべきである。
信仰にしたがっても理性にしたがっても、被造物は神によって存在に保たれてい
 ると言うことが必要である。これを明らかにするためには、あるものが他のも
 のによって保たれるのに、二通りのかたちがあることが考えられるべきである。
一つには、間接的で附帯的なかたちであり、たとえば、ある人が、子供を火の中
 に落ちないように守るとき、彼は「[その子供を]保つ」と言われるように、[あ
 る事物を]消滅させるものを取り除く人は、[その事物を]保つと言われる場
 合である。この意味でも、神は、ある事物を保つが、しかし、すべての事物を
 保つわけではない。なぜなら、ある事物は、それを消滅させるものをもたない
 が、その事物を[この意味で]保つためには、それを消滅させるものを取り除
 くことが必要だからである。

\\


Alio modo dicitur aliquid rem aliquam conservare per se
et directe, inquantum scilicet illud quod conservatur, dependet a
conservante, ut sine eo esse non possit. Et hoc modo omnes creaturae
indigent divina conservatione. Dependet enim esse cuiuslibet creaturae a
Deo, ita quod nec ad momentum subsistere possent, sed in nihilum
redigerentur, nisi operatione divinae virtutis conservarentur in esse,
sicut Gregorius dicit.


&

別のかたちでは、あるものがある事物を、自体的に、そして直接的に保つと言わ
 れる。それは、保たれるものが、保つものに、それがなければ存在しえないと
 いうかたちで依存する場合である。そして、このかたちで、すべての被造物は
 神による保存を必要とする。なぜなら、どんな被造物の存在も神に依存し、グ
 レゴリウスが言うように、もし神の力の働きによって存在のうちに保たれない
 ならば、瞬時も自存することができず、無へ帰ったであろうからである。

\\

Et hoc sic perspici potest. Omnis enim effectus
dependet a sua causa, secundum quod est causa eius. Sed considerandum
est quod aliquod agens est causa sui effectus secundum fieri tantum, et
non directe secundum esse eius. Quod quidem contingit et in
artificialibus, et in rebus naturalibus. Aedificator enim est causa
domus quantum ad eius fieri, non autem directe quantum ad esse
eius.



&

そしてこのことは、次のように理解されうる。
あらゆる結果が自らの原因に依存するのは、その原因が、まさにその結果の原因
 であるという点においてでである。
ところで、ある作用者は、「成ること」にかんしてのみ自らの結果の原因であり、直
 接的に、その結果の「在ること」にかんしてではない、ということがよく考えられ
 るべきである。
このことは、実際、人工物においても自然物においても起こっている。
たとえば、建築家は、その「成ること」にかんして、家の原因だが、直接的に、
 その「在ること」にかんしては原因でない。

\\


Manifestum est enim quod esse domus consequitur formam eius, forma
autem domus est compositio et ordo, quae quidem forma consequitur
naturalem virtutem quarundam rerum. Sicut enim coquus coquit cibum
adhibendo aliquam virtutem naturalem activam, scilicet ignis; ita
aedificator facit domum adhibendo caementum, lapides et ligna, quae sunt
susceptiva et conservativa talis compositionis et ordinis. Unde esse
domus dependet ex naturis harum rerum, sicut fieri domus dependet ex
actione aedificatoris. Et simili ratione est considerandum in rebus
naturalibus. Quia si aliquod agens non est causa formae inquantum
huiusmodi, non erit per se causa esse quod consequitur ad talem formam,
sed erit causa effectus secundum fieri tantum.



&

じっさい、家の存在は家の形相に伴い、家の形相は複合や秩序であるが、この形
 相は、それら[家を構成する]諸事物の自然本性的な力に伴うことは明らかであ
 る。ちょうど、料理人が、なんらかの能動的な自然本性的力、すなわち火を用い
 ることによって料理を作るように、建築家は、切石や石ころや木材を使うことで
 家を建てるが、それらは、そのような複合や秩序を支え保つことができるもので
 ある。したがって、家の「在ること」は、ちょうど、家の「成ること」が建築家
 の作用に依存するように、これら諸事物の本性に依存する。このことは、自然的
 諸事物においても、同様に考えられるべきである。なぜなら、もし、ある作用者
 が、[結果が]そのようなものであるかぎりでの[結果の]形相の原因でないな
 らば、そのような形相に伴う「在ること」の自体的な原因でなく、たんに「成る
 こと」における結果の原因であるだろうから。


\\


Manifestum est autem
quod, si aliqua duo sunt eiusdem speciei, unum non potest esse per se
causa formae alterius, inquantum est talis forma, quia sic esset causa
formae propriae, cum sit eadem ratio utriusque. Sed potest esse causa
huiusmodi formae secundum quod est in materia, idest quod haec materia
acquirat hanc formam. Et hoc est esse causa secundum fieri; sicut cum
homo generat hominem, et ignis ignem. Et ideo quandocumque naturalis
effectus est natus impressionem agentis recipere secundum eandem
rationem secundum quam est in agente, tunc fieri effectus dependet ab
agente, non autem esse ipsius.

&

さて、二つのものが同一の種に属するとき、一方が他方の形相の、そのような形
 相であるかぎりでの自体的な原因でありえないことは明らかである。なぜなら、
 もしそうだとすると、その両者は同一の本質規定(ratio)をもつのだから、それ
 は自分自身の形相の原因であっただろうから[しかしそれはありえない。ゆえ
 に云々]。しかし、そのような形相の、それが質料の中にあるという点での原
 因、つまり、この質料がその形相を獲得するという点での原因ではありうる。
 そしてこれが、人間が人間を生み、火が火を生む場合のように、「成ること」
 における原因である、ということである。ゆえに、自然的な結果が、作用者の
 刻印を、それが作用者の中にあるのと同一の性格で受け取る本性をもつときに
 はいつも、結果の「成ること」は作用者に依存する。しかし、その「在ること」
 が作用者に依存するわけではない。


\\


Sed aliquando effectus non est natus
recipere impressionem agentis secundum eandem rationem secundum quam est
in agente, sicut patet in omnibus agentibus quae non agunt simile
secundum speciem; sicut caelestia corpora sunt causa generationis
inferiorum corporum dissimilium secundum speciem. Et tale agens potest
esse causa formae secundum rationem talis formae, et non solum secundum
quod acquiritur in hac materia, et ideo est causa non solum fiendi, sed
essendi. Sicut igitur fieri rei non potest remanere, cessante actione
agentis quod est causa effectus secundum fieri; ita nec esse rei potest
remanere, cessante actione agentis quod est causa effectus non solum
secundum fieri, sed etiam secundum esse.


&

しかし、たとえば、天体が、種において類似しない下
 位の諸物体の生成の原因であるように、種において類似しない作用を及ぼすす
 べての作用者において明らかなとおり、結果が作用者の刻印を、それが作用者
 の中にあるのと同一の性格で受け取るような本性をもたないとき、そのような
 作用者は、この質料の中に獲得される点においてだけでなく、そのような形相
 の性格における形相の原因でありうるのであり、それゆえ、「成ること」だけ
 でなく、「在ること」の原因でもある。ゆえに、ちょうど、「成ること」の点
 で結果の原因である作用者の作用が止むと、事物の「成ること」が留まること
 ができないように、「成ること」だけでなく「在ること」の点で結果の原因で
 ある作用者の作用が止むと、事物の「在ること」も、留まることができない。

\\



Et haec est ratio quare aqua
calefacta retinet calorem, cessante actione ignis; non autem remanet aer
illuminatus, nec ad momentum, cessante actione solis. Quia scilicet
materia aquae susceptiva est caloris ignis secundum eandem rationem qua
est in igne, unde si perfecte perducatur ad formam ignis, retinebit
calorem semper; si autem imperfecte participet aliquid de forma ignis
secundum quandam inchoationem, calor non semper remanebit, sed ad
tempus, propter debilem participationem principii caloris. 


&


そしてこれが、熱せられた水は火の作用が止んでも熱を保持するが、太陽の作用
 が止めば、一瞬たりとも照らされた空気が留まらない理由である。というのも、
 それはつまり、水の質料は、火の熱を、それが火の中にあるのと同じ性格で受
 け取りうるので、もし完全に火の形相まで導かれるならば、常に熱を保持する
 であろうし、もし、不完全に、なんらかの強制によって火の形相をもつ何かを
 分有するならば、熱は常には留まらず、熱の根源を分有する弱さのために、一
 定の時間だけ、熱が留まるであろう。

\\



Aer autem
nullo modo natus est recipere lumen secundum eandem rationem secundum
quam est in sole, ut scilicet recipiat formam solis, quae est principium
luminis, et ideo, quia non habet radicem in aere, statim cessat lumen,
cessante actione solis. Sic autem se habet omnis creatura ad Deum, sicut
aer ad solem illuminantem. Sicut enim sol est lucens per suam naturam,
aer autem fit luminosus participando lumen a sole, non tamen
participando naturam solis; ita solus Deus est ens per essentiam suam,
quia eius essentia est suum esse; omnis autem creatura est ens
participative, non quod sua essentia sit eius esse. 


&
これに対して、空気は、光の根源である太陽の形相を受け取るというかたちで、
 光を、それが太陽の中にあるのと同じ性格で受け取るような本性を、まったく
 もっていない。それゆえ、空気の中に光線がないので、太陽の作用が止めば、
 直ちに光は消える。さて、すべての被造物は、ちょうど空気が照らす太陽に関
 係するように、神に関係する。つまり、太陽が、自らの本性によって照らすも
 のであり、他方、空気は、太陽から、太陽の本性を分有することによってでは
 なく、光を分有することによって、明るくなるように、そのように、ただ神だ
 けが自らの本質によって有であり、というのも、神の本質は自らの存在だから
 であるが、これに対して、すべての被造物は分有によって在るものであり、自
 らの本質がその存在ではない。

\\


Et ideo, ut
Augustinus dicit IV {\itshape super Gen.~ad Litt}., {\itshape virtus Dei ab eis quae creata
sunt regendis si cessaret aliquando, simul et illorum cessaret species,
omnisque natura concideret}. Et in VIII eiusdem libri dicit quod, {\itshape sicut
aer praesente lumine fit lucidus, sic homo, Deo sibi praesente,
illuminatur, absente autem, continuo tenebratur}.

&

それゆえ、アウグスティヌスが『創世記逐語注解』4巻で言うように、「もし、
 あるときに、神の力が、創造された事物を支配することをしなくなったならば、
同時に、それらのすべての種は止まり、すべての本性は崩壊しただろう」。また、
 同書8巻でも、彼は次のように述べている。「ちょうど、光が現在することで、
 空気が明るくなるように、人間も、神が彼に現在することによって照らされ、
 不在であることによって、直ちに暗くなる」。


\\


{\scshape Ad primum ergo dicendum} quod esse per se
consequitur formam creaturae, supposito tamen influxu Dei, sicut lumen
sequitur diaphanum aeris, supposito influxu solis. Unde potentia ad non
esse in spiritualibus creaturis et corporibus caelestibus, magis est in
Deo, qui potest subtrahere suum influxum, quam in forma vel in materia
talium creaturarum.

&

第一異論に対しては、それゆえ、以下のように言われるべきである。
存在は、被造物の形相に伴うが、しかし神による注入を前提とする。ちょうど、
 光は空気の透明体に伴うが、太陽による注入を前提とするように。したがって、
 霊的被造物と天体における非存在への可能性は、そのような被造物の形相や質
 料のなかによりはむしろ、自らの注入をやめることができる神の中にある。


\\


{\scshape Ad secundum dicendum} quod Deus non potest
communicare alicui creaturae ut conservetur in esse, sua operatione
cessante; sicut non potest ei communicare quod non sit causa esse
illius. Intantum enim indiget creatura conservari a Deo, inquantum esse
effectus dependet a causa essendi. Unde non est simile de agente quod
non est causa essendi, sed fieri tantum.

&

第二異論に対しては、以下のように言われるべきである。神は、自らの働きをや
めて、なんらかの被造物に、存在に保たれることを伝達することはできない。そ
れは、[神が]それ[=被造物]の存在の原因でないことを、それ[=被造物]
に伝達できないのと同様である。被造物は、結果の存在が存在原因に依存するだ
け、それだけ、神によって保たれることを必要とする。したがって、「在ること」
の原因でなく、たんに「成ること」の原因である作用者についてと同様ではない。

\\


{\scshape Ad tertium dicendum} quod ratio illa procedit
de conservatione quae est per remotionem corrumpentis; qua non indigent
omnes creaturae, ut dictum est.

&

第三異論に対しては、以下のように言われるべきである。
この異論は、消滅させるものを排除することによってある保存について論じてい
 る。すでに述べられたとおり、すべての被造物が、このような保存を必要とす
 るわけではない。

\\


{\scshape Ad quartum dicendum} quod conservatio rerum a
Deo non est per aliquam novam actionem; sed per continuationem actionis
qua dat esse, quae quidem actio est sine motu et tempore. Sicut etiam
conservatio luminis in aere est per continuatum influxum a sole.

&

第四異論に対しては、以下のように言われるべきである。
神による諸事物の保存は、なんらかの新しい作用によるのではなく、存在を与え
 る作用の継続による。そして、この作用は運動や時間をもたない。ちょうど、
 空気の中の光の保存が、太陽からの連続的な流入によるのと同様である。


\end{longtable}
\newpage

\rhead{a.~2}
\begin{center}
 {\Large {\bf ARTICULUS SECUNDUS}}\\
 {\large UTRUM DEUS IMMEDIATE OMNEM CREATURAM CONSERVET}\\
 {\Large 第二項\\神はすべての被造物を直接に保存するか}
\end{center}

\begin{longtable}{p{21em}p{21em}}



{\huge A}{\scshape d secundum sic proceditur}. Videtur quod
Deus immediate omnem creaturam conservet. Eadem enim actione Deus est
conservator rerum, qua et creator, ut dictum est. Sed Deus immediate est
creator omnium. Ergo immediate est etiam conservator.

&

第二項の問題へ、議論は以下のように進められる。神は直接にすべての被造物を
保存すると思われる。理由は以下の通り。すでに述べられたとおり、神は、それ
によって創造者であるところの作用と同一の作用によって、諸事物の保存者で
ある。ところが、神は直接的に万物の創造者である。ゆえに、直接的に、保存者
 でもある。

\\


{\scshape 2 Praeterea}, unaquaeque res magis est
proxima sibi, quam rei alteri. Sed non potest communicari alicui
creaturae quod conservet seipsam. Ergo multo minus potest ei communicari
quod conservet aliam. Ergo Deus omnia conservat absque aliqua media
causa conservante.

&

さらに、各々の事物は、他の事物に最近接である以上に、自らに最近接である。
 ところが、自分自身を保存することが、なんらかの被造物へ伝達されることはで
 きない。ゆえに、他の事物を保存することは、さらにいっそう伝達されえない。
 ゆえに、神は、どんな保存の働きを行う中間原因もなく、万物を保存する。


\\


{\scshape 3 Praeterea}, effectus conservatur in esse ab
eo quod est causa eius non solum secundum fieri, sed etiam secundum
esse. Sed omnes causae creatae, ut videtur, non sunt causae suorum
effectuum nisi secundum fieri, non sunt enim causae nisi movendo, ut
supra habitum est. Ergo non sunt causae conservantes suos effectus in
esse.

&

さらに、「成ること」の点だけでなく、「在ること」の点でも結果の原因である
 ものによって、結果は存在に保たれる。ところが、見ればわかるとおり、すべ
 ての被造の諸原因は、「成ること」の点でしか、自らの結果の原因でない。と
 いうのも、前に論じられたとおり、それらは、動かすことによってのみ、原因
 だからである。ゆえに、[被造の諸原因は]自らの結果を存在に保つ原因では
 ない。

\\


{\scshape  Sed contra est} quod per idem conservatur
res, per quod habet esse. Sed Deus dat esse rebus mediantibus aliquibus
causis mediis. Ergo etiam res in esse conservat mediantibus aliquibus
causis.

&

しかし反対に、事物は、それによって存在をもつところのものと同一のものによっ
 て、保存される。ところが、神は、なんらかの中間原因を介して、事物に存在
 を与える。ゆえに、事物を存在に保つのも、なんらかの諸原因を介して行う。

\\


{\scshape Respondeo dicendum} quod, sicut dictum est,
dupliciter aliquid rem aliquam in esse conservat, uno modo, indirecte et
per accidens, per hoc quod removet vel impedit actionem corrumpentis;
alio modo, directe et per se, quia ab eo dependet esse alterius, sicut a
causa dependet esse effectus. Utroque autem modo aliqua res creata
invenitur esse alterius conservativa. 



&

解答する。以下のように言われるべきである。
すでに述べられたとおり、あるものが、ある事物を存在に保つのに二通りのかた
 ちがある。一つには、間接的、附帯的にであり、消滅させるものの作用を排除
 したり妨げたりすることによる。もう一つには、直接的、自体的にであり、ちょ
 うど結果の存在が原因に依存するように、他のものの存在がそれに依存する場
 合である。そしてどちらのかたちでも、ある被造の事物が、他のものの存在を
 保存しうることが見出される。

\\


Manifestum est enim quod etiam in
rebus corporalibus multa sunt quae impediunt actiones corrumpentium, et
per hoc dicuntur rerum conservativa; sicut sal impedit carnes a
putrefactione, et simile est in multis aliis. Invenitur etiam quod ab
aliqua creatura dependet aliquis effectus secundum suum esse. 
Cum enim
sunt multae causae ordinatae, necesse est quod effectus dependeat primo
quidem et principaliter a causa prima; secundario vero ab omnibus causis
mediis. Et ideo principaliter quidem prima causa est effectus
conservativa; secundario vero omnes mediae causae, et tanto magis quanto
causa fuerit altior et primae causae proximior. 

&
たとえば、塩が肉を腐敗から防いだり、他の多くの場合に類似したことがあるよ
 うに、物体的事物でも、消滅させるものの働きを妨げる多くのものがあり、その
 ことによって、その事物を保存しうると言われることが明らかである。また、あ
 る結果が、何かの被造物に、自らの存在の点で依存するのも見出される。
というのも、秩序づけられた多数の原因が存在するとき、結果が、第一にそして
 主要に、第一原因に依存するが、第二義的には、すべての中間的な原因に依存
 するのが必然である。ゆえに、たしかに、主要には、第一原因が結果を保存し
 うるが、しかし第二義的には、すべての中間的な原因がそうであり、また、原
 因がより高く、第一原因に近いほど、[結果を保存しうる]。

\\


Unde superioribus
causis, etiam in corporalibus rebus, attribuitur conservatio et
permanentia rerum, sicut philosophus dicit, in XII Metaphys., quod
primus motus, scilicet diurnus, est causa continuitatis generationis;
secundus autem motus, qui est per zodiacum, est causa diversitatis quae
est secundum generationem et corruptionem. Et similiter astrologi,
attribuunt Saturno, qui est supremus planetarum; res fixas et
permanentes. Sic igitur dicendum est quod Deus conservat res quasdam in
esse, mediantibus aliquibus causis.

&

したがって、物体的諸事物においてさえ、より上位の原因に、諸事物の保存と持
 続が帰せられる。それはちょうど、哲学者が『形而上学』12巻で「第一の運動、
 すなわち日周運動が、生成の連続の原因だが、黄道を通る第二の運動が、生
 成と消滅における多様性の原因だと述べるようにである。同様に、占星術師た
 ちは、惑星でもっとも上位にある土星に、固定し持続的な事物を帰している。
 したがって、神は、なんらかの中間的な原因を介して、ある諸事物を存在に保
 つと言われるべきである。


\\


{\scshape Ad primum ergo dicendum} quod Deus immediate
omnia creavit, sed in ipsa rerum creatione ordinem in rebus instituit,
ut quaedam ab aliis dependerent, per quas secundario conservarentur in
esse; praesupposita tamen principali conservatione, quae est ab ipso.

&

第一異論に対しては、それゆえ、以下のように言われるべきである。
神は直接的に万物を創造したが、その諸事物の創造において、諸事物の中に秩序
 を作り、ある事物が他の事物に依存し、それらの事物によって、第二義的に、
 存在に保たれるようにした。しかし、これは、根源的保存を前提とするのであ
 り、それは神による。

\\


{\scshape Ad secundum dicendum} quod, cum propria causa
sit conservativa effectus ab ea dependentis; sicut nulli effectui
praestari potest quod sit causa sui ipsius, potest tamen ei praestari
quod sit causa alterius; ita etiam nulli effectui praestari potest quod
sit sui ipsius conservativus, potest tamen ei praestari quod sit
conservativus alterius.

&


第二異論に対しては、以下のように言われるべきである。
固有の原因は、それに依存する結果を保存しうるので、ちょうど、どんな結果に
 も、自らの原因であることが与えられえないが、しかし他のものの原因であるこ
 とは与えられるように、どんな結果にも、自ら自身を保ちうることは与えられ
 えないが、しかし、他のものを保存しうることは与えられうる。

\\


{\scshape Ad tertium dicendum} quod nulla creatura
potest esse causa alterius, quantum ad hoc quod acquirat novam formam
vel dispositionem, nisi per modum alicuius mutationis, quia semper agit
praesupposito aliquo subiecto. Sed postquam formam vel dispositionem
induxit in effectu, absque alia immutatione effectus, huiusmodi formam
vel dispositionem conservat. Sicut in aere, prout illuminatur de novo,
intelligitur quaedam mutatio; sed conservatio luminis est absque aeris
immutatione, ex sola praesentia illuminantis.

&


第三異論に対しては、以下のように言われるべきである。
どんな被造物も、なんらかの基体を前提として作用するから、なんらかの変化の
 かたちによってでなければ、新たな形相や態勢を獲得するという点にかんして、
 他のものの原因ではありえない。しかし、形相や態勢を結果の中に導入したあ
 とになら、結果の他の変化なしに、そのような形相や態勢を保存することがで
 きる。たとえば、空気において、新たに照らされるかぎりでは、なんらかの変
 化が理解されるが、しかし、光の保存は、照らす太陽の現在に基づいて、空気
 の変化なしにある。


\end{longtable}
\newpage





\rhead{a.~3}
\begin{center}
 {\Large {\bf ARTICULUS TERTIUS}}\\
 {\large UTRUM DEUS POSSIT ALIQUID IN NIHILUM REDIGERE}\\
 {\footnotesize Supra, q.~9, a.~2; {\itshape De Verit.}, q.~1, a.~2, ad
 6; {\itshape De Pot.}, q.~5, a.~3; {\itshape Quodl.}~IV, q.~3, a.~1.}\\
 {\Large 第三項\\神は何かを無へ戻すことができるか}
\end{center}

\begin{longtable}{p{21em}p{21em}}

{\huge A}{\scshape d tertium sic proceditur}. Videtur quod Deus non
possit aliquid in nihilum redigere. Dicit enim Augustinus, in libro
{\itshape Octoginta Trium Quaest}., quod {\itshape Deus non est causa
tendendi in non esse}. Hoc autem contingeret, si aliquam creaturam
redigeret in nihilum. Ergo Deus non potest aliquid in nihilum redigere.


&


第三項の問題へ、議論は以下のように進められる。
神は何かを無へ戻すことができないと思われる。理由は以下の通り。
アウグスティヌスは、『八十三問題集』という書物の中で、神は、無へと傾くこ
 との原因ではないと述べている。しかし、もし[神が]何かを無へ戻すのであ
 れば、神は無へと傾くことの原因であっただろう。ゆえに、神が何かを無へ戻
 すことは不可能である。

\\


{\scshape 2 Praeterea}, Deus est causa rerum ut sint, per suam
bonitatem, quia, ut Augustinus dicit in libro {\itshape de
Doct. Christ}., {\itshape inquantum Deus bonus est, sumus}. Sed Deus non
potest non esse bonus. Ergo non potest facere ut res non sint. Quod
faceret, si eas in nihilum redigeret.

&
さらに、アウグスティヌスが『キリスト教の教義』という書物の中で「神が善で
 あるかぎりで、私たちは存在する」の述べているように、神は、自らの善性に
 よって、諸事物が存在する原因である。ところが、神は、善でないことがあり
 えない。ゆえに、事物が存在しないようにすることができない。もし諸事物を
 無へ戻したならば、それ[=諸事物を存在しないようにすること]ができたことになる。

\\


{\scshape 3 Praeterea}, si Deus in nihilum aliqua
redigeret, oporteret quod hoc fieret per aliquam actionem. Sed hoc non
potest esse, quia omnis actio terminatur ad aliquod ens; unde etiam
actio corrumpentis terminatur ad aliquid generatum, quia generatio unius
est corruptio alterius. Ergo Deus non potest aliquid in nihilum
redigere.

&


さらに、もし神が何かを無へ戻すならば、それはなんらかの作用によってなされ
 なければならない。ところが、作用はすべてなんらかの有に終わるから、これ
 はありえない。あるものの生成は別のものの消滅だから、消滅させる作用すら、
 なんらかの生成したものに終わるのである。ゆえに、神が何かを無へ戻すこと
 は不可能である。

\\


{\scshape  Sed contra est} quod dicitur {\it Ierem}.~{\scshape x},
{\itshape corripe me, domine, verumtamen in iudicio, et non in furore tuo, ne
forte ad nihilum redigas me}.

&

しかし反対に、『エレミヤ書』10章で「主よ、私を非難してください、しかし、
 あなたの怒りではなく裁きにおいて。あなたが私を無へ戻すことがないように。」
 \footnote{「主よ、わたしを懲らしめてください。しかし、正しい裁きによっ
 て。怒りによらず、わたしが無に帰することのないように。」(10:24)}
 と言われている。

\\


{\scshape Respondeo dicendum} quod quidam posuerunt quod
Deus res in esse produxit agendo de necessitate naturae. Quod si esset
verum, Deus non posset rem aliquam in nihilum redigere; sicut non potest
a sua natura mutari. Sed, sicut supra est habitum, haec positio est
falsa, et a fide Catholica penitus aliena, quae confitetur Deum res
libera voluntate produxisse in esse, secundum illud Psalmi, {\itshape omnia
quaecumque voluit dominus, fecit}. Hoc igitur quod Deus creaturae esse
communicat, ex Dei voluntate dependet. Nec aliter res in esse conservat,
nisi inquantum eis continue influit esse, ut dictum est. Sicut ergo
antequam res essent, potuit eis non communicare esse, et sic eas non
facere; ita postquam iam factae sunt, potest eis non influere esse, et
sic esse desisterent. Quod est eas in nihilum redigere.

&

解答する。以下のように言われるべきである。
ある人々は、神が諸事物を存在へ生み出したのは、本性の必然性から作用するこ
 とによってだと考えた。もしこれが本当なら、神は、どんな事物も無へ戻すこ
 とができなかったであろう。それはちょうど、自分の本性を変えることができ
 ないようなものである。しかし、前に論じられたとおり、この考えは誤りであ
 り、正統信仰からもまったく外れている。かの『詩編』「主は、欲したすべて
 のものを造った」\footnote{「主は何事をも御旨のままに行われる」(135:6)}にしたがって、その信仰は、神が諸事物を自由な意
 志によって存在へ生み出したことを認めるからである。
ゆえに、神が被造物に存在を伝えることは、神の意志に依存する。前に述べられ
 たとおり、諸事物を存在に保つのは、それらに継続的に存在を注ぎ入れること
 による。ゆえに、ちょうど、事物が存在する以前に、それらに存在を伝えず、
 そうすることで、それらを作らないことが可能であったように、すでに諸事物
 が作られたあとも、それらに存在を注ぎ込まないことが可能であり、そうなる
 と、諸事物は存在しなくなる。これは、諸事物を無へ戻すことである。

\\


{\scshape Ad primum ergo dicendum} quod non esse non
habet causam per se, quia nihil potest esse causa nisi inquantum est
ens; ens autem, per se loquendo, est causa essendi. Sic igitur Deus non
potest esse causa tendendi in non esse; sed hoc habet creatura ex
seipsa, inquantum est de nihilo. Sed per accidens Deus potest esse causa
quod res in nihilum redigantur, subtrahendo scilicet suam actionem a
rebus.

&


第一異論に対しては、それゆえ、以下のように言われるべきである。
非存在は、自体的に原因をもたない。なぜなら、何ものも、それが有である限り
 においてでなければ、原因でありえないが、有は、自体的に語られるならば、
 存在の原因だからである。ゆえに、神は、非存在へ傾くことの原因ではありえ
 ず、被造物は、それが無から生まれた限りにおいて、自ら自身に基づいて、そ
 のこと[=無へ帰すること]をもつ。しかし、附帯的に、神が、諸事物が無へ
 戻ることの原因でありうる。つまり、自らの作用を事物から取り下げることに
 よって。

\\


{\scshape Ad secundum dicendum} quod bonitas Dei est
causa rerum, non quasi ex necessitate naturae, quia divina bonitas non
dependet ex rebus creatis; sed per liberam voluntatem. Unde sicut potuit
sine praeiudicio bonitatis suae, res non producere in esse; ita absque
detrimento suae bonitatis, potest res in esse non conservare.

&

第二異論に対しては、以下のように言われるべきである。
神の善性は、諸事物の原因だが、本性の必然性にもとづいてではない。なぜなら、
 神の善性は、創造された諸事物に依存しないからである。そうではなく、それ
 は自由な意志による。したがって、神は自らの善性がいささかも損なわれるこ
 となく事物を存在へ生み出さないことができたように、自らの善性のいささか
 の損害もなく、事物を存在に保たないことができる。


\\


{\scshape Ad tertium dicendum} quod, si Deus rem
aliquam redigeret in nihilum, hoc non esset per aliquam actionem; sed
per hoc quod ab agendo cessaret.


&

第三異論に対しては、以下のように言われるべきである。
もし神がなんらかの事物を無へ戻すならば、これは、なんらかの作用によるので
 はなく、作用をやめることによる。

\end{longtable}
\newpage



\rhead{a.~4}
\begin{center}
 {\Large {\bf ARTICULUS QUARTUS}}\\
 {\large UTRUM ALIQUID IN NIHILM REDIGATUR}\\
 {\footnotesize Supra, q.~65, a.~1, ad 1; {\itshape De Pot.}, q.~5,
 a.~4; a~9, ad 1.}\\
 {\Large 第四項\\何かが無へ戻されるか}
\end{center}

\begin{longtable}{p{21em}p{21em}}


{\huge A}{\scshape d quartum sic proceditur}. Videtur quod
aliquid in nihilum redigatur. Finis enim respondet principio. Sed a
principio nihil erat nisi Deus. Ergo ad hunc finem res perducentur, ut
nihil sit nisi Deus. Et ita creaturae in nihilum redigentur.

&

第四項の問題へ、議論は以下のように進められる。
何かが無へ戻されると思われる。理由は以下の通り。
終わりは始まりに対応する。ところが、始め、神以外になにもなかった。
ゆえに、事柄は、神以外になにもないという目的へ向かって導かれている。
したがって、被造物は無へ戻されるであろう。

\\


{\scshape 2 Praeterea}, omnis creatura habet potentiam
finitam. Sed nulla potentia finita se extendit ad infinitum, unde in
VIII {\itshape Physic}. probatur quod potentia finita non potest movere tempore
infinito. Ergo nulla creatura potest durare in infinitum. Et ita
quandoque in nihilum redigetur.

&


さらに、被造物はすべて有限な能力をもつ。ところが、どんな有限な能力も、無
 限に自らを及ぼすことがない。このことから、『自然学』8巻で、有限な能力が
 無限な時間の間動くことがありえないことが証明されている。ゆえに、どんな
 被造物も、無限に持続することがありえない。したがって、いつかあるときに、
 無へと戻されるであろう。

\\


{\scshape 3 Praeterea}, forma et accidentia non habent
materiam partem sui. Sed quandoque desinunt esse. Ergo in nihilum
rediguntur.

&

さらに、形相と附帯性は、自らの部分としての質料をもたない。ところが、それ
 らは、存在をやめることがある。ゆえに、それらは無に戻る。

\\


{\scshape  Sed contra est} quod dicitur
{\itshape Eccle}.~{\scshape iii}. {\itshape Didici quod omnia opera quae fecit Deus, perseverant in
aeternum}.

&

しかし反対に、『コヘレトの言葉』3章で「神が為したすべての業は、永遠に続くこと
 を、わたしは学んだ」\footnote{「わたしは知った。すべて神の業は永遠に不
 変であり、付け加えることも除くことも許されない、と。」(3:14)}と言われている。

\\


{\scshape Respondeo dicendum} quod eorum quae a Deo
fiunt circa creaturam, quaedam proveniunt secundum naturalem cursum
rerum; quaedam vero miraculose operatur praeter ordinem naturalem
creaturis inditum, ut infra dicetur. Quae autem facturus est Deus
secundum ordinem naturalem rebus inditum, considerari possunt ex ipsis
rerum naturis, quae vero miraculose fiunt, ordinantur ad gratiae
manifestationem, secundum illud apostoli I {\itshape ad Cor}.~{\scshape xii}, {\itshape unicuique datur
manifestatio spiritus ad utilitatem}; et postmodum, inter cetera, subdit
de miraculorum operatione. 


&

解答する。以下のように言われるべきである。
被造物をめぐって神によって作られるものどもの中には、諸事物の自然の成り行
 きにしたがって生じるものもあれば、後で述べられることだが、被造物に与え
 られた自然の秩序を離れて、奇跡的に為されるものもある。さて、神が、諸事
 物に与えられた自然の秩序にしたがって神が為すであろうことは、諸事物の本
 性それ自体に基づいて考察されうるが、奇跡的に為されることは、『コリント
 の信徒への手紙一』12章でかの使徒が「各々には霊の顕現が与えられるの
 は、有益のため」と述べ、そして、そのあとに、他のものに混じって、奇跡的な働き
 について述べている\footnote{「一人一人に霊の働きが現れるのは、全体の益
 となるためです。ある人には霊によって智恵の言葉、(中略)ある人には奇跡
 を行う力、」(12:7-10)}ことによれば、恩恵を明らかにすることに秩序づけられて
 いる。

\\



Creaturarum autem naturae hoc demonstrant, ut
nulla earum in nihilum redigatur, quia vel sunt immateriales, et sic in
eis non est potentia ad non esse; vel sunt materiales, et sic saltem
remanent semper secundum materiam, quae incorruptibilis est, utpote
subiectum existens generationis et corruptionis. Redigere etiam aliquid
in nihilum, non pertinet ad gratiae manifestationem, cum magis per hoc
divina potentia et bonitas ostendatur, quod res in esse conservat. Unde
simpliciter dicendum est quod nihil omnino in nihilum redigetur.

&

ところで、被造物の本性は、それらのうちのどれも無に戻らないことを明示して
 いる。なぜなら、非質料的なものの場合、それらの中に非存在への可能態はな
 いし、質料的なものの場合、生成消滅の基体として不可滅的である質料におい
 て、常に留まるからである。さらに、何かが無へ戻ることは、恩恵を明示する
 ことにも属さない。なぜなら、諸事物を存在に保つことによっての方が、より、
 神の能力と善性が明示されるからである。したがって、まったく何も、無へ帰
 せられることはないだろうと言われるべきである。

\\


{\scshape Ad primum ergo dicendum} quod hoc quod res in
esse productae sunt, postquam non fuerunt, declarat potentiam
producentis. Sed quod in nihilum redigerentur, huiusmodi manifestationem
impediret, cum Dei potentia in hoc maxime ostendatur, quod res in esse
conservat, secundum illud apostoli {\itshape Heb}.~{\scshape i}, {\itshape portans omnia verbo virtutis
suae}.

&

第一異論に対しては、それゆえ、以下のように言われるべきである。
諸事物が、存在しなかったあとに、存在へと生み出されたということは、生み出
 すものの力を明示している。しかし、仮に無へ戻されるとするならば、それはそ
 のような明示を妨げたであろう。かの『ヘブライ人への手紙』1章「万物を、自
 らの力の言葉で支える方」の使徒によれば、神の知恵は、諸事物を存在に保つ
 ことにおいて明示されるのだから。


\\


{\scshape Ad secundum dicendum} quod potentia creaturae
ad essendum est receptiva tantum; sed potentia activa est ipsius Dei, a
quo est influxus essendi. Unde quod res in infinitum durent, sequitur
infinitatem divinae virtutis. Determinatur tamen quibusdam rebus virtus
ad manendum tempore determinato, inquantum impediri possunt ne
percipiant influxum essendi qui est ab eo, ex aliquo contrario agente,
cui finita virtus non potest resistere tempore infinito, sed solum
tempore determinato. Et ideo ea quae non habent contrarium, quamvis
habeant finitam virtutem, perseverant in aeternum.

&

第二異論に対しては、以下のように言われるべきである。
被造物の力は、存在することにたいして、たんに受容的だが、神自身の[存在す
 ることにたいする]力は能動的である。その神から、存在の流入があるのだか
 ら。したがって、諸事物が無限に持続することから、神の力の無限性が帰結す
 る。しかし、ある事物には、限定された時間だけ存続するように制限される。
 それは、ある反対する作用者によって、神からである存在の流入を分有しない
 ように妨げられうるかぎりにおいてである。有限な力は、そのような反対する
 作用者に、無限のあいだ抵抗することができず、たんに限られた時間抵抗でき
 るにすぎないからである。
ゆえに、反対のものをもたない事物は、有限な力しか持っていなくても、永遠に
 存続する。

\\


{\scshape Ad tertium dicendum} quod formae et
accidentia non sunt entia completa, cum non subsistant, sed quodlibet
eorum est aliquid entis, sic enim ens dicitur, quia eo aliquid est. Et
tamen eo modo quo sunt, non omnino in nihilum rediguntur; non quia
aliqua pars eorum remaneat, sed remanent in potentia materiae vel
subiecti.

&

第三異論に対しては、以下のように言われるべきである。
形相と附帯性は、自存しないので、完全な存在者でなく、それらのどれもが、存
 在者に属する何かである。この意味で、それによって何かが存在するので、
 それらは存在者と言われる。しかしまた、それらが存在するあらゆるかたちで、無へ帰
 るというわけではない。それは、それらのある部分が残存するからではなく、
 それらが質料や基体の可能態の中に残存するからである。


\end{longtable}
\newpage

\end{document}