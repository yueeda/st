\documentclass[10pt]{jsarticle}
\usepackage{okumacro}
\usepackage{longtable}
\usepackage[polutonikogreek,english,japanese]{babel}
\usepackage{latexsym}
\usepackage{color}
\usepackage{schemata}
\usepackage[T1]{fontenc}
\usepackage{lmodern}

%----- header -------
\usepackage{fancyhdr}
\pagestyle{fancy}
\lhead{{\it Summa Theologiae} I-II, q.61}
%--------------------

\bibliographystyle{jplain}

\title{{\bf PRIMA SECUNDAE}\\{\HUGE Summae Theologiae}\\Sancti Thomae
Aquinatis\\{\sffamily QUEAESTIO SEXAGESIMAPRIMA}\\DE VIRTUTIBUS CARDINALIBUS}
\author{Japanese translation\\by Yoshinori {\sc Ueeda}}
\date{Last modified \today}

%%%% コピペ用
%\rhead{a.~}
%\begin{center}
% {\Large {\bf }}\\
% {\large }\\
% {\footnotesize }\\
% {\Large \\}
%\end{center}
%
%\begin{longtable}{p{21em}p{21em}}
%
%&
%
%\\
%\end{longtable}
%\newpage

\begin{document}

\maketitle
\thispagestyle{empty}

\begin{center}
{\LARGE 『神学大全』第二部の一}\\
{\Large 第六十一問\\枢要徳について}
\end{center}

\begin{longtable}{p{21em}p{21em}}

Deinde considerandum est de virtutibus cardinalibus. Et circa hoc quaeruntur quinque.  
\begin{enumerate}
 \item utrum virtutes morales debeant dici cardinales, vel principales.
 \item de numero earum.
 \item quae sint.
 \item utrum differant ab invicem.
 \item utrum dividantur convenienter in virtutes politicas, et purgatorias, et purgati animi, et exemplares.
\end{enumerate}

&

 次に枢要徳について考察されるべきである。これを巡っては五つのことが問われる。  

\begin{enumerate}
 \item 道徳的徳は枢要的または主要的と言われるべきか。
 \item それらの数について。
 \item それらは何か。
 \item 相互に区別されるか。
 \item それらが国家社会的、浄化的、浄魂的、範型的徳へと適切に分割されるか。
\end{enumerate}
\end{longtable}



\newpage


\rhead{a.~1}
\begin{center}
{\Large {\bf ARTICULUS PRIMUS}}\\
{\large UTRUM VIRTUTES MORALES DEBEANT DICI CARDINALES, VEL PRINCIPALES}\\
{\footnotesize Infra, q.66, a.4; III {\itshape Sent.}, d.33, q.2, a.1, qu$^{a}$2; {\itshape De Virtut.}, q.1, a.12, a.24; q.5, a.1.}\\
{\Large 第一項\\道徳的徳は枢要的または主要的と言われるべきか}
\end{center}

\begin{longtable}{p{21em}p{21em}}

{\scshape Ad primum sic proceditur}. Videtur quod virtutes morales non
debeant dici cardinales, seu principales. Quae enim ex opposito
dividuntur, sunt simul natura, ut dicitur in {\itshape
Praedicamentis}, et sic unum non est altero principalius. Sed omnes
virtutes ex opposito dividunt genus virtutis. Ergo nullae earum debent
dici principales.


&

 第一項の問題へ、議論は以下のように進められる。道徳的徳は枢要的とか主
 要的とかと言われるべきではないと思われる。理由は以下の通り。『カテゴ
 リー論』で言われるように、対立するものに基づいて分割されるものは本性
 において同時である。したがって、一方が他方よりも主要的だということは
 ない。しかるにすべての徳は対立するものに基づいて徳の類を分割する。ゆ
 えにそれらのどれも主要的だと言われるべきでない。
 
\\



2.~{\scshape Praeterea}, finis principalior est his quae sunt ad
finem. Sed virtutes theologicae sunt circa finem, virtutes autem
morales circa ea quae sunt ad finem. Ergo virtutes morales non debent
dici principales, seu cardinales; sed magis theologicae.

&

さらに、目的は、目的のためにあるものどもよりも主要である。しかるに神学
的徳は目的を巡ってあるが、道徳的徳は目的のためにあるものどもを巡ってあ
る。ゆえに道徳的徳が主要的や枢要的と言われるべきでなく、むしろ神学的徳
がそう言われるべきである。
 
\\



3.~{\scshape Praeterea}, principalius est quod est per essentiam, quam
quod est per participationem. Sed virtutes intellectuales pertinent ad
rationale per essentiam, virtutes autem morales ad rationale per
participationem, ut supra dictum est. Ergo virtutes morales non sunt
principales, sed magis virtutes intellectuales.

&

さらに、本質によってあるものは分有によってあるものよりも主要である。し
かるに上述の如く、知性的徳は本質によって理性的なものに属するが、道徳的
徳は分有によって理性的なものに属する。ゆえに道徳的徳は主要なものではな
く、むしろ知性的徳がそうである。
 
\\



{\scshape Sed contra est} quod Ambrosius dicit, {\itshape super
Lucam}, exponens illud, {\itshape beati pauperes spiritu, scimus
virtutes esse quatuor cardinales, scilicet temperantiam, iustitiam,
prudentiam, fortitudinem}. Hae autem sunt virtutes morales. Ergo
virtutes morales sunt cardinales.

&

しかし反対に、アンブロシウスは『ルカによる福音書註解』でかの「霊におい
て貧しい者は幸福である」\footnote{「さて、イエスは目を上げ、弟子たちを
見て言われた。/「貧しい人々は、幸いである/神の国はあなたがたのもので
ある。」(6:20)} について解説して、「私たちは四つの徳が枢要的だと知って
いる。すなわち、節制、正義、思慮、勇気である」と言っている。しかるにこ
れらは道徳的徳である。ゆえに道徳的徳は枢要的である。
 
\\



 {\scshape Respondeo dicendum} quod, cum simpliciter de virtute
 loquimur, intelligimur loqui de virtute humana. Virtus autem humana,
 ut supra dictum est, secundum perfectam rationem virtutis dicitur,
 quae requirit rectitudinem appetitus, huiusmodi enim virtus non solum
 facit facultatem bene agendi, sed ipsum etiam usum boni operis
 causat.

&

 解答する。以下のように言われるべきである。私たちが端的に徳について語
 るとき、それが人間的な徳について語られていると理解する。しかるに人間
 的な徳は、上述の如く、徳の完全な性格に即して語られるが、その性格は欲
 求の正しさを要求する。実際、そのような徳は機能を善く働かせるだけでな
 く、善い業の使用自体の原因ともなる。

 
\\


 Sed secundum imperfectam rationem virtutis dicitur virtus quae non
 requirit rectitudinem appetitus, quia solum facit facultatem bene
 agendi, non autem causat boni operis usum. Constat autem quod
 perfectum est principalius imperfecto. Et ideo virtutes quae
 continent rectitudinem appetitus, dicuntur principales.


 

&

 しかし、不完全な徳の性格に即して、欲求の正しさを要求しない徳が語られ
 る。なぜなら、それはただ機能を善く働かせるだけであり、善い業の使用の
 原因とならないからである。しかるに、完全なものが不完全なものよりも主
 要であることは明らかである。ゆえに欲求の正しさを含む徳が主要的だと言
 われる。

 \\

 Huiusmodi
 autem sunt virtutes morales; et inter intellectuales, sola prudentia,
 quae etiam quodammodo moralis est, secundum materiam, ut ex
 supradictis patet, unde convenienter inter virtutes morales ponuntur
 illae quae dicuntur principales, seu cardinales.

 &

 しかるに、道徳的徳はそのようなものであり、また知性的徳の中では、上述
 のことから明らかなとおり、質料(対象)の点である意味で道徳的でもある
 思慮だけが、そのようなものである。したがって、主要的ないし枢要的と言
 われる徳が道徳的徳の中に置かれるのは適切である。

 
\\



{\scshape Ad primum ergo dicendum} quod, quando genus univocum
dividitur in suas species, tunc partes divisionis ex aequo se habent
secundum rationem generis; licet secundum naturam rei, una species sit
principalior et perfectior alia, sicut homo aliis animalibus. Sed
quando est divisio alicuius analogi, quod dicitur de pluribus secundum
prius et posterius; tunc nihil prohibet unum esse principalius altero,
etiam secundum communem rationem; sicut substantia principalius
dicitur ens quam accidens. Et talis est divisio virtutum in diversa
genera virtutum, eo quod bonum rationis non secundum eundem ordinem
invenitur in omnibus.


&

 第一異論に対しては、それゆえ、以下のように言われるべきである。一義的
 な類がその種へ分割されるとき、分割の部分は類の性格において等しく関係
 する。ただし事物の本性に即しては、ある種が他の種よりも主要であり完全
 である。たとえば人間が他の動物よりもそうであるように。しかし、あるア
 ナロギア的なものの分割であるとき、これは複数のものについてより先、よ
 り後に即して語られるが、その場合には、概念においてさえも、一つのもの
 が他のものよりも主要であることを妨げるものはない。たとえば実体は附帯
 性よりも主要であると言われるように。そしてさまざまな徳の類への徳の分
 割は、このようなものである。なぜなら理性の善は、すべてのものにおいて
 同じ秩序に即して見出されるわけではないからである。
 
\\



{\scshape Ad secundum dicendum} quod virtutes theologicae sunt supra
hominem, ut supra dictum est. Unde non proprie dicuntur virtutes
humanae, sed superhumanae, vel divinae.

&

 第二異論に対しては以下のように言われるべきである。上述の如く、神学的
 徳は人間を超えている。したがって、厳密には人間的な徳とは言われず、超
 人間的な、あるいは神的な徳と言われる。

 
\\



{\scshape Ad tertium dicendum}, quod aliae virtutes intellectuales a
prudentia, etsi sint principaliores quam morales quantum ad subiectum;
non tamen sunt principaliores quantum ad rationem virtutis, quae
respicit bonum, quod est obiectum appetitus.


&

 第三異論に対しては以下のように言われるべきである。思慮以外の知性的徳
 は、その主題の点で道徳的徳よりも主要であるにしても、欲求の対象である
 善に関係する徳の性格にかんして、より主要であるわけではない。
 

\end{longtable}
\newpage

\rhead{a.~2}
\begin{center}
{\Large {\bf ARTICULUS SECUNDUS}}\\
{\large UTRUM SINT QUATUOR VIRTUTES CARDINALIS}\\
{\footnotesize Infra, q.66, a.4; III {\itshape Sent.}, d.33, q.2, a.1, qu$^{a}$3; {\itshape De Virtut.}, q.1, a.12, ad 25; q.5, a.1; II {\itshape Ethic.}, lect.8.}\\
{\Large 第二項\\四つの枢要徳があるか}
\end{center}

\begin{longtable}{p{21em}p{21em}}
{\scshape Ad secundum sic proceditur}. Videtur quod non sint quatuor
virtutes cardinales. Prudentia enim est directiva aliarum virtutum
moralium, ut ex supradictis patet. Sed id quod est directivum aliorum,
principalius est. Ergo prudentia sola est virtus principalis.

&

 第二項の問題へ、議論は以下のように進められる。四つの枢要徳があるので
 はないと思われる。理由は以下の通り。上述のこと\footnote{Q.58, a.4.}か
 ら明らかなとおり、思慮は他の道徳的徳を導きうるものである。しかるに他
 のものを導きうるものは、より主要的である。ゆえに思慮だけが主要な徳で
 ある。

\\



2.~{\scshape Praeterea}, virtutes principales sunt aliquo modo
morales. Sed ad operationes morales ordinamur per rationem practicam,
et appetitum rectum, ut dicitur in VI {\itshape Ethic}. Ergo solae
duae virtutes cardinales sunt.


&

 さらに、主要な徳はある意味で道徳的である。しかるに道徳的な働きへ私た
 ちが秩序付けられるのは、『ニコマコス倫理学』第6巻で言われるように、実
 践的な理性と正しい欲求とによる。ゆえにただ二つの徳が枢要的である。


\\



3.~{\scshape Praeterea}, inter alias etiam virtutes una est
principalior altera. Sed ad hoc quod virtus dicatur principalis, non
requiritur quod sit principalis respectu omnium, sed respectu
quarundam. Ergo videtur quod sint multo plures principales virtutes.


&

さらに、他の諸徳の間でも、ある徳が他の徳よりも主要である。しかるに徳が
主要と言われるためにはすべての徳に関して主要である必要はなく、ある徳に
対してそうであればよい。ゆえにはるかに多くの主要な徳が存在すると思われ
る。

\\




{\scshape Sed contra est} quod Gregorius dicit, in II {\itshape
Moral}., {\itshape in quatuor virtutibus tota boni operis structura
consurgit}.

&

 しかし反対に、グレゴリウスは『道徳論』第2巻で「四つの徳の中に善い業の
 すべての構造が集められている」と言う。


\\




 {\scshape Respondeo dicendum quod} numerus aliquorum accipi potest
 aut secundum principia formalia aut secundum subiecta, et utroque
 modo inveniuntur quatuor cardinales virtutes. 


&

 解答する。以下のように言われるべきである。何かの数は形相的根源に即し
 て、または主題に即して理解されうるが、このどちらの仕方でも、四つの枢
 要徳が見出される。


\\


 Principium enim formale virtutis de qua nunc loquimur, est rationis
 bonum.  Quod quidem dupliciter potest considerari. Uno modo, secundum
 quod in ipsa consideratione rationis consistit. Et sic erit una
 virtus principalis, quae dicitur prudentia. 


&

理由は以下の通り。私たちが今語っている徳の形相的根源は理性の善である。
この善は二通りの仕方で考察されうる。一つには、それが理性の考察自体にお
いて成立する限りにおいてであり、その意味では、思慮と呼ばれる一つの主要
な徳が存在することになる。

 

\\


Alio modo, secundum quod circa aliquid ponitur rationis ordo. Et hoc
 vel circa operationes, et sic est {\itshape iustitia}, vel circa
 passiones, et sic necesse est esse duas virtutes. Ordinem enim
 rationis necesse est ponere circa passiones, considerata repugnantia
 ipsarum ad rationem.

 &

 もう一つには、あるものを巡って理性の秩序が定められる限りにおいてであ
 る。そしてこれは働きを巡ってあるか情念を巡ってあるかだが、前者の場合
 には「正義」がある。そして後者の場合には二つの徳があることが必然である。
 というのも、情念を巡って理性の秩序を定めるのは、情念の理性に対する反
 対を考慮に入れることが必要だからである。
 
 \\
 


 Quae quidem potest esse dupliciter. Uno modo secundum quod passio
 impellit ad aliquid contrarium rationi, et sic necesse est quod
 passio reprimatur, et ab hoc denominatur {\itshape temperantia}. Alio
 modo, secundum quod passio retrahit ab eo quod ratio dictat, sicut
 timor periculorum vel laborum, et sic necesse est quod homo firmetur
 in eo quod est rationis, ne recedat; et ab hoc denominatur {\itshape
 fortitudo}.

&

このような反対は二通りにありうる。一つは情念が理性に反対する何かへと駆
り立てる限りにおいてであり、この場合には情念が抑制されることが必要であ
り、このことから節制が名付けられる。もう一つの仕方では、たとえば危険や
労苦に対する怖れのように、情念が理性が命じることから退く限りにおいてで
あり、この場合には、人間は理性に属するものにおいて退かないようにと強化
される必要がある。そしてこのことによって「勇気」が名付けられる。
 

\\



 Similiter secundum subiecta, idem numerus invenitur. Quadruplex enim
 invenitur subiectum huius virtutis de qua nunc loquimur, scilicet
 rationale per essentiam, quod {\itshape prudentia} perficit; et
 rationale per participationem, quod dividitur in tria; idest in
 voluntatem, quae est subiectum {\itshape iustitiae}; et in
 concupiscibilem, quae est subiectum {\itshape temperantiae}; et in
 irascibilem, quae est subiectum {\itshape fortitudinis}.


&

 同様に、基体に即しても同じ数が見出される。すなわち、私たちが今語って
 いるこのような徳の基体は四通りに見出される。すなわち、本質によって理
 性的なものは思慮がこれを完成し、分有によって理性的なものは三つに分割
 され、意志は正義の基体であり、欲情は節制の基体、怒情は勇気の基体であ
 る。


\\




{\scshape Ad primum ergo dicendum} quod prudentia est simpliciter
principalior omnibus. Sed aliae ponuntur principales unaquaeque in suo
genere.


&

 第一異論に対しては、それゆえ、以下のように言われるべきである。思慮は
 端的にすべての徳よりも主要である。しかし他の諸徳もそれぞれその類にお
 いて主要であるとされる。

\\




{\scshape Ad secundum dicendum} quod rationale per participationem
dividitur in tria, ut dictum est.


&

 第二異論に対しては以下のように言われるべきである。分有によって理性的
 なものは、上述の如く、三つに分割される。


\\


{\scshape Ad tertium dicendum} quod omnes aliae virtutes, quarum una
est principalior alia, reducuntur ad praedictas quatuor, et quantum ad
subiectum, et quantum ad rationes formales.


&

 第三異論に対しては以下のように言われるべきである。そのうちの一つが他
 のものよりも主要であるようなすべての他の徳は、その基体においても形相
 的な性格においても、前述の四つに還元される。

\end{longtable}
\newpage
\section{cardinal virtues}\label{schema:cardinal_virtues}
\subsection{according to formal principles}
\schema
{\schemabox{bonum rationis}}
{
\schemabox{in ipsa ratione --- PRUDENTIA}
\schema{\schemabox{circa aliquid ponitur\\rationis ordo}}
{
\schemabox{circa operationes --- JUSTITIA}
\schema{\schemabox{circa passiones}}
{
\schemabox{passio reprimatur --- TEMPERANTIA}
\schemabox{homo firmetur --- FORTITUDO}
}
}
}

\vspace{2em}
\subsection{according to subjects}
\schema
{\schemabox{subiectum\\virtutis}}
{
\schemabox{rationale per essentiam --- PRUDENTIA}
\schema{\schemabox{rationale per\\ participationem}}
{
\schemabox{voluntas --- IUSTITIA}
\schemabox{concupiscibilis --- TEMPERANTIA}
\schemabox{irascibilis --- FORTITUDO}
}
}


\newpage

\rhead{a.~3}
\begin{center}
{\Large {\bf ARTICULUS TERTIUS}}\\
{\large UTRUM ALIAE VIRTUTES MAGIS DEBEANT DICI PRINCIPALES QUAM ISTAE}\\
{\footnotesize III {\itshape Sent.}, d.33, q.2, a.1, qu$^{a}$4; {\itshape De Virtut.}, q.1, a.12, ad 26; q.5, a.1; II {\itshape Ethic.}, lect.8.}\\
{\Large 第三項\\これらより他の徳の方が主要的と言われるべきか}
\end{center}

\begin{longtable}{p{21em}p{21em}}
{\scshape Ad tertium sic proceditur}. Videtur quod aliae virtutes
debeant dici magis principales quam istae. Id enim quod est maximum in
unoquoque genere, videtur esse principalius. Sed {\itshape
magnanimitas operatur magnum in omnibus virtutibus}, ut dicitur in IV
{\itshape Ethic}. Ergo magnanimitas maxime debet dici principalis
virtus.

 
&

第三項の問題へ、議論は以下のように進められる。他の徳がこれらよりも主要
的と言われるべきだと思われる。理由は以下の通り。各々の類において最大の
ものが、より主要的だと思われる。しかるに『ニコマコス倫理学』第4巻で言
われるように、「高邁はすべての徳の中でで最大のことを行う」。ゆえに高邁
が最大限に主要な徳と言われるべきである。
 
\\



2.~{\scshape Praeterea}, illud per quod aliae virtutes firmantur,
videtur esse maxime principalis virtus. Sed humilitas est huiusmodi,
dicit enim Gregorius quod {\itshape qui ceteras virtutes sine
humilitate congregat, quasi paleas in ventum portat}. Ergo humilitas
videtur esse maxime principalis.

 
&

 さらに、それによって他の徳が強化されるものが、最大限に主要な徳である
 と思われる。しかるに謙遜はそのようなものである。というのも、グレゴリ
 ウスが「謙遜なしに他の徳を集める人は風に向かって藁を運ぶようなものだ」
 と言っているからである。ゆえに謙遜が最大限に主要であると思われる。

 
\\



3.~{\scshape Praeterea}, illud videtur esse principale, quod est
perfectissimum. Sed hoc pertinet ad patientiam; secundum illud
{\itshape Iacobi} {\scshape i}, {\itshape patientia opus perfectum
habet}. Ergo patientia debet poni principalis.

 
&

 さらに、もっとも完全なものは主要なものだと思われる。しかるに、かの
 『ヤコブの手紙』第1章「忍耐は完全な業をもつ」\footnote{「あくまでも忍
 耐しなさい。そうすれば、何一つ欠けたところのない、完全で申し分のない
 人になります。」(1:4)}によれば、このことは忍耐に属する。ゆえに忍耐が
 主要なものとして認められるべきである。

 
\\



 {\scshape Sed contra est} quod Tullius, in sua {\itshape Rhetorica},
 ad has quatuor omnes alias reducit.

 
&

 しかし反対に、キケロは自分の『弁論術』の中で、これら四つへ他のすべて
 の徳を還元している。

 
\\



 {\scshape Respondeo dicendum} quod sicut supra dictum est, huiusmodi
 quatuor virtutes cardinales accipiuntur secundum quatuor formales
 rationes virtutis de qua loquimur. Quae quidem in aliquibus actibus
 vel passionibus principaliter inveniuntur. Sicut bonum consistens in
 consideratione rationis, principaliter invenitur in ipso rationis
 imperio; non autem in consilio, neque in iudicio, ut supra dictum
 est.
 
&

 解答する。以下のように言われるべきである。上述の如く、このような四つ
 の枢要徳は、私たちがそれについて語っている徳の四つの形相的な性格に即
 して理解されている。実際、それらはある作用や情念の中に主要に見出され
 る。たとえば、理性の考察において成立する善は、理性の命令自体の中に主
 要に見出されるが、上述の如く、思量や判断の中には見出されない。

 
\\



 Similiter autem bonum rationis prout ponitur in operationibus
 secundum rationem recti et debiti, principaliter invenitur in
 commutationibus vel distributionibus quae sunt ad alterum cum
 aequalitate.
 
&

 同様に、正しいものやしかるべきものという性格に即して働きの中に措定さ
 れるものとしての理性の善は、主要に、等しさを伴って他者に対してある交
 換や分配のなかに見出される。
 

 
\\

 Bonum autem refraenandi passiones principaliter
 invenitur in passionibus quas maxime difficile est reprimere,
 scilicet in delectationibus tactus.

&

 さらに情念を鎮める善は、抑制するのが最も困難な情念、すなわち触覚の快
 楽において主要に見出される。
 
\\

 Bonum autem firmitatis ad standum
 in bono rationis contra impetum passionum, praecipue invenitur in
 periculis mortis, contra quae difficillimum est stare.
 
&

また、情念の衝動に反して理性の善に踏みとどまる強さの善は、特に、踏みと
どまるのが最も困難である死の危険において見出される。

 
\\

 Sic igitur praedictas quatuor virtutes dupliciter considerare
 possumus. Uno modo, secundum communes rationes formales. Et secundum
 hoc, dicuntur principales, quasi generales ad omnes virtutes, utputa
 quod omnis virtus quae facit bonum in consideratione rationis,
 dicatur prudentia; et quod omnis virtus quae facit bonum debiti et
 recti in operationibus, dicatur iustitia; et omnis virtus quae
 cohibet passiones et deprimit, dicatur temperantia; et omnis virtus
 quae facit firmitatem animi contra quascumque passiones, dicatur
 fortitudo.
 
&

 ゆえに、このようにして、上述の四つの徳は二通りに考察されうる。一つに
 は共通の形相的性格に即してである。そしてこれに従えば、これらはすべて
 の徳に対して類的なものとして、主要と言われる。すなわち、理性の考察に
 おいて善を為すすべての徳は思慮と言われ、働きにおいてしかるべき事や正
 しいことの善を為すすべての徳は正義と言われる。そして情念を抑制し制御
 するすべての徳は節制と言われ、どんな情念であれそれに反して魂の強さを
 もたらすすべての徳が勇気と言われる。

 
\\



 Et sic multi loquuntur de istis virtutibus, tam sacri doctores quam
 etiam philosophi. Et sic aliae virtutes sub ipsis continentur unde
 cessant omnes obiectiones.

 
&


 このように、聖なる教師たちも哲学者たちも、多くの人々がこれらの徳につ
 いて語っている。その意味で他の徳はこれらのもとに含まれ、したがってす
 べての異論は止む。
 
\\


 Alio vero modo possunt accipi, secundum quod istae virtutes
 denominantur ab eo quod est praecipuum in unaquaque materia. Et sic
 sunt speciales virtutes, contra alias divisae. Dicuntur tamen
 principales respectu aliarum, propter principalitatem materiae, puta
 quod prudentia dicatur quae praeceptiva est; iustitia, quae est circa
 actiones debitas inter aequales; temperantia, quae reprimit
 concupiscentias delectationum tactus; fortitudo, quae firmat contra
 pericula mortis.
 
&

もう一つには、これらの徳が各々の質料(対象)において特別のものであるこ
とから名付けられていると理解することもできる。この意味では、これらは他
のさまざまな徳に対立して種的なものである。しかし、他の徳に比べて主要で
あると言われるのは、質料の主要性のためである。たとえば、命令する徳が思
慮と言われ、等しい人々の間のしかるべき行為を巡る徳が正義と言われ、触覚
の快楽への欲情を抑制する徳が節制と言われ、死の危険に対して強化する徳が
勇気と言われる。
 
 
\\

 Et sic etiam cessant obiectiones, quia aliae virtutes possunt habere
 aliquas alias principalitates, sed istae dicuntur principales ratione
 materiae, ut supra dictum est.
 
&

そしてこの意味でも異論は止む。なぜなら、他の諸徳は、何らかの他の主要性
をもつが、しかし前述の如く、これらは質料の性格において主要的と言われる
からである。
 
\\
\end{longtable}
\newpage
\rhead{a.~4}
\begin{center}
{\Large {\bf ARTICULUS QUARTUS}}\\
{\large UTRUM QUATUOR VIRTUTES CARDINALES DIFFERANT AB INVICEM}\\
{\footnotesize III {\itshape Sent.}, d.33, q.1, a.1, qu$^{a}$3; {\itshape De Virtut.}, q.1, a. 12, ad 23; qu.5, a.1, ad 1; II {\itshape Ethic.}, lect.8.}\\
{\Large 第四項\\四つの枢要徳は相互に異なるか}
\end{center}

\begin{longtable}{p{21em}p{21em}}
{\scshape Ad quartum sic proceditur}. Videtur quod quatuor praedictae
virtutes non sint diversae virtutes, et ab invicem distinctae. Dicit
enim Gregorius, in XXII {\itshape Moral}., {\itshape prudentia vera
non est, quae iusta, temperans et fortis non est; nec perfecta
temperantia, quae fortis, iusta et prudens non est; nec fortitudo
integra, quae prudens, temperans et iusta non est; nec vera iustitia,
quae prudens, fortis et temperans non est}.


&

第四項の問題へ、議論は以下のように進められる。上述の四つの徳は異なる徳
でなく相互に区別されないと思われる。理由は以下の通り。グレゴリウスは
『道徳論』第12巻で以下のように述べている。「正しく、節制され、勇気があ
るのでない徳は真の思慮でなく、勇気があり、正しく、思慮深いのでないよう
な徳は完全な節制でない。思慮があり節制され正しくない徳は完璧な勇気でな
く、思慮があり勇気があり節制されているのでない徳は真の正義でない」。
 
\\


 Hoc autem non contingeret,
si praedictae quatuor virtutes essent ab invicem distinctae, diversae
enim species eiusdem generis non denominant se invicem. Ergo
praedictae virtutes non sunt ab invicem distinctae.


&

しかしこのようなことは、もし前述の四つの徳が相互に区別され、同一の類に
属する異なる種であり、お互いを支配し合わないのであったならば起こりえな
かったであろう。ゆえに上述の徳は相互に区別されていない。
 
\\



2.~{\scshape Praeterea}, eorum quae ab invicem sunt distincta, quod
est unius, non attribuitur alteri. Sed illud quod est temperantiae,
attribuitur fortitudini, dicit enim Ambrosius, in I {\itshape libro de
Offic}., iure ea fortitudo vocatur, quando unusquisque seipsum vincit,
nullis illecebris emollitur atque inflectitur. De temperantia etiam
dicit quod {\itshape modum vel ordinem servat omnium quae vel agenda
vel dicenda arbitramur}. Ergo videtur quod huiusmodi virtutes non sunt
ab invicem distinctae.

&

 さらに、一方に属するものが他方に帰せられないということは、相互に区別
 されるものどもに属する。しかるに節制に属するものが、勇気に帰せられる。
 実際、アンブロシウスは『義務論』の第1巻で「各々の人が自分に打ち克ち、
 どんな魅力によっても弱められたり影響を受けたりしないときには、正しく
 それは勇気と呼ばれる」。さらに節制についても「それは私たちが為される
 べきあるいは言われるべきだと判断するすべての事柄の限度や秩序を守る」
 と言う。ゆえにこのような徳は相互に区別されていないと思われる。
 

 
\\



3.~{\scshape Praeterea}, philosophus dicit, in II {\itshape Ethic}.,
quod ad virtutem haec requiruntur, {\itshape primum quidem, si sciens;
deinde, si eligens, et eligens propter hoc; tertium autem, si firme et
immobiliter habeat et operetur}. Sed horum primum videtur ad
prudentiam pertinere, quae est recta ratio agibilium; secundum,
scilicet eligere, ad temperantiam, ut aliquis non ex passione, sed ex
electione agat, passionibus refraenatis; tertium, ut aliquis propter
debitum finem operetur, rectitudinem quandam continet, quae videtur ad
iustitiam pertinere; aliud, scilicet firmitas et immobilitas, pertinet
ad fortitudinem. Ergo quaelibet harum virtutum est generalis ad omnes
virtutes. Ergo non distinguuntur ad invicem.


&

 さらに、哲学者は『ニコマコス倫理学』第2巻で特には以下のことが必要だと
 言っている。「第一に、知っていること、次に選んでいて、そのために選ん
 でいること、さらに第三に堅く不動の仕方で持ちそして働いていること」。
 しかしこれらのうちの第一のものは為されうる事柄についての正しい理であ
 る思慮に属すると思われるし、第二のこと、すなわち選ぶことは節制に属す
 ると思われる。それは、情念を鎮めているときに、ある人は情念からではな
 く選択によって作用するからである。そして第三のこと、すなわち人がしか
 るべき目的のために働くことは、ある種の正しさを含んでいて、これは正義
 に属すると思われる。他方、強さと不動は勇気に属する。ゆえにこれらの徳
 のどれもすべての徳に対して類的なものである。ゆえにそれらは相互に区別
 されない。
 
 
\\



{\scshape Sed contra est} quod Augustinus dicit, in libro {\itshape de
moribus Eccles}., quod {\itshape quadripartita dicitur virtus, ex
ipsius amoris vario affectu}, et subiungit de praedictis quatuor
virtutibus. Praedictae ergo quatuor virtutes sunt ab invicem
distinctae.


&

しかし反対に、アウグスティヌスは『教会の道徳について』という書物の中で、
「徳は、その愛自身のさまざまな情動によって四つの部分に分かれたものと言
われる」と述べている。そしてそれを上述の四つの徳について結び付けている。
ゆえに前述の四つの徳は相互に区別されている。
 
\\



 Respondeo dicendum quod, sicut supra dictum est, praedictae quatuor
 virtutes dupliciter a diversis accipiuntur. Quidam enim accipiunt
 eas, prout significant quasdam generales conditiones humani animi,
 quae inveniuntur in omnibus virtutibus, ita scilicet quod prudentia
 nihil sit aliud quam quaedam rectitudo discretionis in quibuscumque
 actibus vel materiis; iustitia vero sit quaedam rectitudo animi, per
 quam homo operatur quod debet in quacumque materia; temperantia vero
 sit quaedam dispositio animi quae modum quibuscumque passionibus vel
 operationibus imponit, ne ultra debitum efferantur; fortitudo vero
 sit quaedam dispositio animae per quam firmetur in eo quod est
 secundum rationem, contra quoscumque impetus passionum vel
 operationum labores.


&

 解答する。以下のように言われるべきである。前に述べられたとおり、上述
 の四つの徳は異なる人々によって二通りの仕方で理解されている。すなわち
 ある人々はそれらを、人間の魂の類的な条件を表示するものとして、すべて
 の徳において見出されるものとして理解している。つまり思慮は、任意の作
 用や質料における判断の正しさに他ならず、正義は任意の質料において、そ
 れによって人がしかるべきことを行うところの魂の正しさであり、また節制
 は、しかるべき範囲を超えて行かないように、任意の情念や働きに限度を課
 する魂の態勢であり、そして勇気は、情念の衝動や働きの労苦に打ち克って
 理性に従うことにおいて固められる魂の態勢である。
 
 

 
\\



 Haec autem quatuor sic distincta, non important diversitatem habituum
 virtuosorum quantum ad iustitiam, temperantiam et
 fortitudinem. Cuilibet enim virtuti morali, ex hoc quod est habitus,
 convenit quaedam firmitas, ut a contrario non moveatur, quod dictum
 est ad fortitudinem pertinere. Ex hoc vero quod est virtus, habet
 quod ordinetur ad bonum, in quo importatur ratio recti vel debiti,
 quod dicebatur ad iustitiam pertinere. In hoc vero quod est virtus
 moralis rationem participans, habet quod modum rationis in omnibus
 servet, et ultra se non extendat, quod dicebatur pertinere ad
 temperantiam.


&

しかしこのように区別されたこれらの四つは、正義、節制、勇気に関して、有
徳な習慣の差異を意味しない。理由は以下の通り。どの道徳的な徳にとっても、
それが習慣であることから、反対のものによって動かされないというある種の
堅固さを持ち、これが勇気に属すると言われた。また、徳であるということか
ら、善へ秩序付けられている性格を持つが、この善において、正しいことやし
かるべきことというという性格が意味される、これが正義に属すると言われて
いた。また、理性を分有する道徳的徳であるということにおいて、それはすべ
てのものにおいて理性の限度を守り、自らを越えて出ていかないという性格を
持つが、これは節制に属すると言われていた。
 
\\


  Solum autem hoc quod est discretionem habere, quod attribuebatur
  prudentiae, videtur distingui ab aliis tribus, inquantum hoc est
  ipsius rationis per essentiam; alia vero tria important quandam
  participationem rationis, per modum applicationis cuiusdam ad
  passiones vel operationes. Sic igitur, secundum praedicta, prudentia
  quidem esset virtus distincta ab aliis tribus, sed aliae tres non
  essent virtutes distinctae ab invicem; manifestum est enim quod una
  et eadem virtus et est habitus, et est virtus, et est moralis.

&

 しかし思慮に帰せられていた区別をもつことだけは、これが本質によって理
 性自身に属する限りにおいて、他の三つから区別されるように思われる。こ
 れに対して他の三つの徳は、情念や働きへの何らかの適用という仕方で、理
 性のある種の分有を意味している。ゆえにこのようにして、上述のことによ
 れば、思慮は他の三つの徳から区別された徳であるが、他の三つは相互に区
 別された徳ではない、ということになったであろう。なぜなら、一つの同一
 の徳が、習慣であり、徳であり、道徳的なものであることが明らかだからで
 ある。

 
\\

 Alii vero, et melius, accipiunt has quatuor virtutes secundum quod
 determinantur ad materias speciales; unaquaeque quidem illarum ad
 unam materiam, in qua principaliter laudatur illa generalis conditio
 a qua nomen virtutis accipitur, ut supra dictum est. Et secundum hoc,
 manifestum est quod praedictae virtutes sunt diversi habitus,
 secundum diversitatem obiectorum distincti.

&

他方、他の人々は、そしてこちらの方が善い理解だが、これら四つ徳を、それ
らが種的な質料(対象)へ限定されている限りにおいて理解している。上述の
如く、実にこれらの各々が一つの質料へ限定され、そこにおいて、そこから徳
という名称が受け取られる類的な条件が主要に賞賛される
\footnote{\pageref{schema:cardinal_virtues}ページの図を参照。}。そして
この限りにおいて、上述の徳が対象の異なりに即して区別された異なる習慣で
あることが明らかである。
 
 
\\



{\scshape Ad primum ergo dicendum} quod Gregorius loquitur de
praedictis quatuor virtutibus secundum primam acceptionem. Vel potest
dici quod istae quatuor virtutes denominantur ab invicem per
 redundantiam quandam.

&

 第一異論に対しては、それゆえ、以下のように言われるべきである。グレゴ
 リウスは上述の四つの徳について、第一の理解に基づいて語っている。ある
 いは以下のように言えるかもしれない。すなわちこれら四つの徳は、相互に
 ある種の溢れ出しによって名付けられる、と。

 
\\

 Id enim quod est prudentiae, redundat in alias virtutes, inquantum a
prudentia diriguntur. Unaquaeque vero aliarum redundat in alias ea
ratione, quod qui potest quod est difficilius, potest et id quod minus
est difficile.


&

つまり以下のようなことである。思慮であるものは、他の諸徳が思慮によって
導かれる限りにおいて、他の諸徳へと溢れ出す。また他の諸徳の各々は、困難
なことをなしうるものは、それより困難でないものをなしうるという根拠に基
づいて、他の諸徳へと溢れ出す。
 
 
\\



 Unde qui potest refraenare
concupiscentias delectabilium secundum tactum, ne modum excedant, quod
est difficillimum; ex hoc ipso redditur habilior ut refraenet audaciam
in periculis mortis, ne ultra modum procedat, quod est longe facilius;
 et secundum hoc, fortitudo dicitur temperata.


&

 したがって、触覚において快適であるものへの欲情を限度を超えないように
 抑制することは最も困難だが、それができる徳は、死の危険において、限度
 を超えないように豪胆を抑制するという、はるかに容易なことを、より簡単
 にする。この限りで、勇気が節度あるものと言われる。
 
\\

 Temperantia etiam dicitur fortis, ex redundantia fortitudinis in
temperantiam, inquantum scilicet ille qui per fortitudinem habet
animum firmum contra pericula mortis, quod est difficillimum, est
habilior ut retineat animi firmitatem contra impetus delectationum;
quia, ut dicit Tullius in I {\itshape de Offici}., {\itshape non est
consentaneum ut qui metu non frangitur, cupiditate frangatur; nec qui
invictum se a labore praestiterit, vinci a voluptate}.


&

 また、節度も勇気あるものと言われるのであり、それは、勇気の節度への溢
 れ出しに基づく。すなわち、勇気によって死の危険をものともしない魂の強
 さを持つ人は、これは最も困難なことだが、快楽の衝動に打ち克つ魂の堅固
 さをより容易に保つ限りにおいてである。というのも、キケロが『義務論』
 第1巻で言うように、「恐怖に持ちこたえる人が欲情には負けるというのは理
 にかなうことではなく、労働には負けないことを示した人が、快楽には負け
 るのを示すというのもおかしなこと」だからである。

 
\\



Et per hoc etiam patet responsio ad secundum. Sic enim temperantia in
omnibus modum servat, et fortitudo contra illecebras voluptatum animum
servat inflexum, vel inquantum istae virtutes denominant quasdam
generales conditiones virtutum; vel per redundantiam praedictam.


&

 またこのことによって第二異論への解答も明らかである。つまり、このよう
 に節制はすべての事柄において限度を保持し、勇気は快楽の魅力に反して堅
 固な魂を保つ。これらの徳が徳のある類的な条件を名付ける限りにおいて、
 あるいは、上述の溢れ出しによって。

 
\\



{\scshape Ad tertium dicendum} quod illae quatuor generales virtutum
conditiones quas ponit philosophus, non sunt propriae praedictis
virtutibus. Sed possunt eis appropriari, secundum modum iam dictum.

&

 第三異論に対しては以下のように言われるべきである。哲学者が指定するか
 の徳の四つの類的な状態は、上述の徳に固有なものではなく、むしろそれら
 に、すでに述べられしかたで帰属されうる。


\end{longtable}
\newpage
\rhead{a.~5}
\begin{center}
{\Large {\bf ARTICULUS QUINTUS}}\\
{\large UTRUM VIRTUTES CARDINALES CONVENIENTER DIVIDANTUR IN VIRTUTES POLITICAS, PURGATORIAS, PURGATI ANIMI, ET EXEMPLARES}\\
{\footnotesize III {\itshape Sent.}, d.33, q.1, a.4, ad 2; d.34, q.1, a.1, arg.6; {\itshape De Virtut.}, q.26, a.8, ad 1.}\\
{\Large 第五項\\枢要徳は国家社会的、浄化的、浄魂的、範型的徳へと適切に分割されるか}
\end{center}

\begin{longtable}{p{21em}p{21em}}
{\scshape Ad quintum sic proceditur}. Videtur quod inconvenienter
huiusmodi quatuor virtutes dividantur in virtutes exemplares, purgati
animi, purgatorias, et politicas. Ut enim Macrobius dicit, in I
{\itshape super somnium Scipionis}, {\itshape virtutes exemplares sunt
quae in ipsa divina mente consistunt}. Sed philosophus, in X {\itshape
Ethic}., dicit quod {\itshape ridiculum est Deo iustitiam,
fortitudinem, temperantiam et prudentiam attribuere}. Ergo virtutes
huiusmodi non possunt esse exemplares.

&

 第五項の問題へ、議論は以下のように進められる。この四つの徳が範型的、
 浄魂的、浄化的、国家社会的な徳に分割されるのは適切でないと思われる。
 理由は以下の通り。マクロビウスが『スキピオの夢について』第1巻で述べて
 いるように「範型的徳とは神の精神自体において成立するものである」。し
 かるに哲学者は『ニコマコス倫理学』第10巻で「神に正義,勇気、節制、思
 慮を帰することは笑うべき事である」と述べている。ゆえにそのような徳は
 範型的ではありえない。
 
\\

2.~{\scshape Praeterea}, virtutes purgati animi dicuntur quae sunt
absque passionibus, dicit enim ibidem Macrobius quod {\itshape
temperantiae purgati animi est terrenas cupiditates non reprimere, sed
penitus oblivisci; fortitudinis autem passiones ignorare, non
vincere}. Dictum est autem supra quod huiusmodi virtutes sine
passionibus esse non possunt. Ergo huiusmodi virtutes purgati animi
esse non possunt.

&

 さらに、浄魂的徳と言われるのは情念がない徳である。というのも、マクロ
 ビウスが同じ箇所で「浄魂の節制には地上的な欲情を鎮めるのではなく、む
 しろまったく忘れることが属し、また情念に打ち克つのではなく情念を知ら
 ないことが勇気に属する」述べているからである。しかるにこのような徳が
 情念なしにあることは不可能であると上で述べられた。ゆえにこのような徳
 が浄魂的であることはできない。

 
\\



3.~{\scshape Praeterea}, virtutes purgatorias dicit esse eorum
{\itshape qui quadam humanorum fuga solis se inserunt divinis}. Sed
hoc videtur esse vitiosum, dicit enim Tullius, in I {\itshape de
Offic}., quod {\itshape qui despicere se dicunt ea quae plerique
mirantur imperia et magistratus, his non modo non laudi, verum etiam
vitio dandum puto}. Ergo non sunt aliquae virtutes purgatoriae.

&

 さらに、浄化的徳は「人間的な事からある種逃避して神的な事柄だけに自ら
 を参加させる」人々に属する。しかるにこれは悪徳だと思われる。というの
 もキケロが『義務論』第1巻で「多くの人が賞賛する公職や公権を自分は望ま
 ないと言う人々は、褒められないだけでなく、悪徳に数えられると私は思う」
 と述べているからである。ゆえに浄化的であるような徳は存在しない。

 
\\



4.~{\scshape Praeterea}, virtutes politicas esse dicit {\itshape
quibus boni viri reipublicae consulunt, urbesque tuentur}. Sed ad
bonum commune sola iustitia legalis ordinatur; ut philosophus dicit,
in V {\itshape Ethic}. Ergo aliae virtutes non debent dici politicae.

&

 さらに国家社会的な徳であると言うのは「それによって善い人々が国家の世
 話をし都市を防衛する」徳である。しかし哲学者が『ニコマコス倫理学』第5
 巻で言うように、共通善に秩序付けられるのはただ法の正義だけである。ゆ
 えに他の徳が国家社会的と言われるべきではない。

 
\\



{\scshape Sed contra est} quod Macrobius ibidem dicit, {\itshape
Plotinus, inter philosophiae professores cum Platone princeps, <<quatuor
sunt, inquit, quaternarum genera virtutum. Ex his primae politicae
vocantur; secundae, purgatoriae; tertiae autem, iam purgati animi;
quartae, exemplares.>>}


&

 しかし反対に、マクロビウスは同じ箇所で「哲学の教師の中でプラトンとと
 もに長であるプロティノスはこう述べる。「四つずつの徳の類が四つある。
 これらのうち第一のものは国家社会的な徳と呼ばれ、第二は浄化的、第三は
 すでに浄魂的、第四は範型的と呼ばれる」」。
 
\\



 {\scshape Respondeo dicendum} quod, sicut Augustinus dicit in libro
 {\itshape de moribus Eccles}., {\itshape oportet quod anima aliquid
 sequatur, ad hoc quod ei possit virtus innasci, et hoc Deus est, quem
 si sequimur, bene vivimus}. Oportet igitur quod exemplar humanae
 virtutis in Deo praeexistat, sicut et in eo praeexistunt omnium rerum
 rationes. Sic igitur virtus potest considerari vel prout est
 exemplariter in Deo, et sic dicuntur virtutes {\itshape exemplares}.


&

 解答する。以下のように言われるべきである。アウグスティヌスが『教会の
 道徳について』で述べているように「魂が、その中に徳が生み出されるため
 には、何かに従わなければならず、それは神であり、私たちが神に従うなら
 ば、私たちは善く生きる」。ゆえに人間的な徳の範型が神の中に先在してい
 なければならない。それはちょうど、神の中にすべての事物の理拠が先在し
 ているようにである。ゆえにこのようにして、徳は神の中に範型的に存在す
 るかぎりで考察されうるのであり、この意味で「範型的な」徳と言われる。

 
\\

 Ita scilicet quod ipsa divina mens in Deo dicatur prudentia;
 temperantia vero, conversio divinae intentionis ad seipsum, sicut in
 nobis temperantia dicitur per hoc quod concupiscibilis conformatur
 rationi; fortitudo autem Dei est eius immutabilitas; iustitia vero
 Dei est observatio legis aeternae in suis operibus, sicut Plotinus
 dixit.

&

 すなわち以下のようである。神の精神自体は神において思慮と言われる。他
 方、節制は神の意図の自分自身への向け変えである。これはちょうど私たち
 においても節制が、欲情的部分が理性に一致させられることによって言われ
 るのと同様である。さらに神の勇気は神の不変性であり、神の正義は、プロ
 ティノスが言ったように、自らの業において永遠法を守ることである。

 
\\



 Et quia homo secundum suam naturam est animal politicum, virtutes
 huiusmodi, prout in homine existunt secundum conditionem suae
 naturae, politicae vocantur, prout scilicet homo secundum has
 virtutes recte se habet in rebus humanis gerendis. Secundum quem
 modum hactenus de his virtutibus locuti sumus.

&

 そして人間は自らの本性に即してポリス的な動物なので、そのような徳は、
 人間の中に存在するものとして、自らの本性の状態に即して国家社会的な徳
 と呼ばれる。すなわち人間はこれらの徳に従って、人間が行うべき事柄にお
 いて正しくある。私たちはこの意味で、これらの徳についてここまで語って
 きた。

 
 
\\

 Sed quia ad hominem pertinet ut etiam ad divina se trahat quantum
 potest, ut etiam philosophus dicit, in X {\itshape Ethic}.; et hoc
 nobis in sacra Scriptura multipliciter commendatur, ut est illud
 Matth.~{\scshape v}, {\itshape estote perfecti, sicut et pater vester
 caelestis perfectus est}, necesse est ponere quasdam virtutes medias
 inter politicas, quae sunt virtutes humanae, et exemplares, quae sunt
 virtutes divinae.

&

 しかし人間には、『ニコマコス倫理学』第10巻で哲学者も述べているように、
 できる限り神的な事柄へと自らを導くことが属する。これはまた聖書でも多
 くのかたちで推奨されている。たとえばかの『マタイによる福音書』第5章
 「あなた方の天の父が完全であるように、完全でありなさい」のように。そ
 れゆえ、人間的な徳である国家社会的な徳と神の徳である範型的な徳のあい
 だに何らかの中間的な徳を置くことが必要である。

 
\\


 Quae quidem virtutes distinguuntur secundum diversitatem motus et
 termini. Ita scilicet quod quaedam sunt virtutes transeuntium et in
 divinam similitudinem tendentium, et hae vocantur virtutes {\itshape
 purgatoriae}. Ita scilicet quod prudentia omnia mundana divinorum
 contemplatione despiciat, omnemque animae cogitationem in divina sola
 dirigat; temperantia vero relinquat, inquantum natura patitur, quae
 corporis usus requirit; fortitudinis autem est ut anima non terreatur
 propter excessum a corpore, et accessum ad superna; iustitia vero est
 ut tota anima consentiat ad huius propositi viam.

&

 この徳は運動と終端の異なりに即して区別される。すなわち、ある徳は越え
 て行きつつある者、神の類似へ向かう者に属し、これが浄化的徳と呼ばれる。
 すなわち思慮はすべての地上的なものを神の観想によって見下し、魂のすべ
 ての思惟をただ神だけに向ける。節制は自然が耐えられるかぎり、身体の使
 用が必要とするものを捨てる。また勇気には、魂が身体から離れて上位のも
 のへ接近するために恐れないようにすることが属する。正義はこのように提
 示された道に魂全体で同意することである。

 
\\


 Quaedam vero sunt virtutes iam assequentium divinam similitudinem,
 quae vocantur virtutes iam {\itshape purgati animi}. Ita scilicet
 quod prudentia sola divina intueatur; temperantia terrenas
 cupiditates nesciat; fortitudo passiones ignoret; iustitia cum divina
 mente perpetuo foedere societur, eam scilicet imitando. Quas quidem
 virtutes dicimus esse beatorum, vel aliquorum in hac vita
 perfectissimorum.


&

 他方、ある徳はすでに神的な類似を獲得した人々に属し、それらは浄魂的と
 呼ばれる。すなわち、思慮はただ神的なことだけを直観し、節制は地上的な
 欲情を知らず、勇気は情念を無視し、正義は神の精神と、それを模倣するこ
 とによって、永遠の契りによって交わる。これらの徳を私たちは至福者、あ
 るいはこの世でもっとも完全な人々に属するものと言う。
 
\\


{\scshape Ad primum ergo dicendum} quod philosophus loquitur de his
virtutibus secundum quod sunt circa res humanas, puta iustitia circa
emptiones et venditiones, fortitudo circa timores, temperantia circa
concupiscentias. Sic enim ridiculum est eas Deo attribuere.

&

 第一異論に対しては、それゆえ、以下のように言われるべきである。哲学者
 はこれらの徳について、それらが人間的な事柄を巡る限りにおいて語ってい
 る。たとえば正義は売買を巡って、勇気は怖れを巡って、節制は欲情を巡っ
 て、というように。この意味では、これらを神に帰するのは笑うべき事であ
 る。

 
\\



{\scshape Ad secundum dicendum} quod virtutes humanae sunt circa
passiones, scilicet virtutes hominum in hoc mundo conversantium. Sed
virtutes eorum qui plenam beatitudinem assequuntur, sunt absque
passionibus. Unde Plotinus dicit quod {\itshape passiones politicae
virtutes molliunt}, idest ad medium reducunt; {\itshape secundae},
scilicet purgatoriae, {\itshape auferunt; tertiae}, quae sunt purgati
animi, {\itshape obliviscuntur; in quartis}, scilicet exemplaribus,
{\itshape nefas est nominari}. Quamvis dici possit quod loquitur hic
de passionibus secundum quod significant aliquos inordinatos motus.


&

 第二異論に対しては以下のように言われるべきである。人間的な徳は情念を
 巡ってある。すなわちそれらはこの世界の中で共に暮らす人間に属する徳で
 ある。しかし十分な至福を獲得している人々に属する徳は情念と関係しない。
 このことから、プロティノスは以下のように述べる。「情念を国家社会的な
 徳は穏やかにする」すなわち中庸にもたらし、「第二のもの」すなわち浄化
 的徳は「取り除く。そして第三のもの」すなわち浄魂的な徳は「忘却し、第
 四のものにおいて」すなわち範型的な徳においては「名付けられることが罪
 である」。ただし、ここで彼がある無秩序な運動を意味するかぎりで情念に
 ついて語っていると言われることは可能である。
 
\\

{\itshape Ad tertium dicendum} quod deserere res humanas ubi
necessitas imponitur, vitiosum est, alias est virtuosum. Unde parum
supra Tullius praemittit, {\itshape His forsitan concedendum est
rempublicam non capessentibus, qui excellenti ingenio doctrinae se
dederunt; et his qui aut valetudinis imbecillitate, aut aliqua
graviori causa impediti, a republica recesserunt; cum eius
administrandae potestatem aliis laudemque concederent}. Quod consonat
ei quod Augustinus dicit, XIX {\itshape de Civ.~Dei}, {\itshape Otium
sanctum quaerit caritas veritatis; negotium iustum suscipit necessitas
caritatis. Quam sarcinam si nullus imponit, percipiendae atque
intuendae vacandum est veritati, si autem imponitur, suscipienda est,
propter caritatis necessitatem}.


&

 第三異論に対しては以下のように言われるべきである。必要性が課せられて
 いる場合に人間的な事柄をしないのは悪徳だが、そうでない場合には有徳で
 ある。だからキケロは少し前に以下のように書いている。「学問の卓越した
 才能に身を捧げている人で、国家の事柄に携わらない人々には、このことが
 許されるべきである。また健康上の弱さによって、あるいは何か重大な原因
 で妨げられている人々にも、彼の任務上の権限と賞賛を他の人々に譲る場合、
 国家のことから退くことが許されるべきである。このことは、アウグスティ
 ヌスが『神の国』第19巻で述べていることに一致する。すなわち「真理への
 愛は聖なる暇を求める。愛の必要性は正しい交渉を引き受ける。だれもその
 重荷を課さないのならば、探究されるべき、直観されるべき真理に暇を充て
 るべきだが、もし課されるならば、愛の必要のために,それを引き受けるべ
 きである」。
 
\\



{\scshape Ad quartum dicendum} quod sola iustitia legalis directe
respicit bonum commune, sed per imperium omnes alias virtutes ad bonum
commune trahit, ut in V {\itshape Ethic}. dicit philosophus. Est enim
considerandum quod ad politicas virtutes, secundum quod hic dicuntur,
pertinet non solum bene operari ad commune, sed etiam bene operari ad
partes communis, scilicet ad domum, vel aliquam singularem personam.


&

 第四異論に対しては以下のように言われるべきである。『ニコマコス倫理学』
 第5巻で哲学者が言うように、法の正義だけが直接的に共通善に関係するが、
 命令によって、他のすべての徳は共通善へと導かれる。なぜなら、ここで語
 られるかぎりでの国家社会的な徳には、共通のことのために善く働くことだ
 けでなく、共通のものの部分、すなわち家庭や他の個々の人物のために善く
 働くことも属することが考えられるべきだからである。

\end{longtable}
\end{document}
