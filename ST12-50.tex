\documentclass[10pt]{jsarticle} % use larger type; default would be 10pt
\usepackage[utf8]{inputenc} % set input encoding (not needed with XeLaTeX)
\usepackage[T1]{fontenc}
%\usepackage[round,comma,authoryear]{natbib}
%\usepackage{nruby}
\usepackage{okumacro}
\usepackage{longtable}
%\usepqckage{tablefootnote}
\usepackage[polutonikogreek,english,japanese]{babel}
%\usepackage{amsmath}
\usepackage{latexsym}
\usepackage{color}
\usepackage{otf}
\usepackage{schemata}
%----- header -------
\usepackage{fancyhdr}
\pagestyle{fancy}
\lhead{{\it Summa Theologiae} I-II, q.50}
%--------------------


\bibliographystyle{jplain}


\title{{\bf Prima Secundae}\\{\HUGE Summae Theologiae}\\Sancti Thomae
Aquinatis\\{\sffamily QUAESTIO QUINQUAGESIMA}\\{\bf DE SUBIECTO HABITUUM}}
\author{Japanese translation\\by Yoshinori {\sc Ueeda}}
\date{Last modified \today}

%%%% コピペ用
%\rhead{a.~}
%\begin{center}
% {\Large {\bf }}\\
% {\large }\\
% {\footnotesize }\\
% {\Large \\}
%\end{center}
%
%\begin{longtable}{p{21em}p{21em}}
%
%&
%
%\\
%\end{longtable}
%\newpage

\begin{document}

\maketitle
\thispagestyle{empty}
\begin{center}
{\Large 『神学大全』第二部の一\\第五十問\\習慣の基体について}
\end{center}


\begin{longtable}{p{21em}p{21em}}

Deinde considerandum est de subiecto habituum. Et circa hoc quaeruntur
sex. 

\begin{enumerate}
 \item utrum in corpore sit aliquis habitus.
 \item utrum anima sit subiectum habitus secundum suam essentiam, vel secundum suam potentiam.
 \item utrum in potentiis sensitivae partis possit esse aliquis habitus.
 \item utrum in ipso intellectu sit aliquis habitus.
 \item utrum in voluntate sit aliquis habitus.
 \item utrum in substantiis separatis.
\end{enumerate}

&

次に習慣の基体について考察されるべきである。
そしてこれをめぐって六つのことが問われる。
\begin{enumerate}
 \item 身体の中に何らかの習慣があるか。
 \item 魂が習慣の基体であるのは自らの本質に即してかそれとも自らの能力に即してか。
 \item 感覚的部分の能力の中に何らかの習慣がありうるか。
 \item 知性自身の中に何らかの習慣があるか。
 \item 意志の中に何らかの習慣があるか。
 \item 離在実体の中に(何らかの習慣があるか)。
\end{enumerate}

\end{longtable}
\newpage
\rhead{a.~1}
\begin{center}
{\Large {\bf ARTICULUS PRIMUS}}\\
{\large UTRUM IN CORPORE SIT ALIQUIS HABITUS}\\
{\footnotesize III {\itshape Sent.}, d.23, q.1, a.1.}\\
{\Large 第一項\\身体の中に何らかの習慣があるか}
\end{center}

\begin{longtable}{p{21em}p{21em}}



{\scshape Ad primum sic proceditur}. Videtur quod in corpore non sit aliquis
habitus. Ut enim Commentator dicit, in III {\itshape de Anima}, {\itshape habitus est quo
quis agit cum voluerit}. Sed actiones corporales non subiacent
voluntati, cum sint naturales. Ergo in corpore non potest esse aliquis
habitus.


&

第一項の問題へ、議論は以下のように進められる。
身体の中にはどんな習慣もないと思われる。理由は以下の通り。
注解者は『デ・アニマ』第3巻で「習慣とは、それによって誰かが欲するときに働くところのものである」と述べている。
しかるに身体の作用は自然本性的なので意志に従属しない。
ゆえに身体の中に何らかの習慣が存在することは不可能である。

\\



2. {\scshape Praeterea}, omnes dispositiones corporales sunt facile mobiles. Sed habitus est qualitas difficile mobilis. Ergo nulla dispositio corporalis potest esse habitus.

&

さらに、全ての身体的な態勢は動きやすい。しかるに習慣は動きにくい性質である。
ゆえにどんな身体的な態勢も習慣でありえない。

\\



3. {\scshape Praeterea}, omnes dispositiones corporales subiacent alterationi. Sed alteratio non est nisi in tertia specie qualitatis, quae dividitur contra habitum. Ergo nullus habitus est in corpore.

&

さらに、全ての身体的な態勢は性質変化のもとにある。しかるに性質変化は性質の第三の種の中以外にはないが、それは習慣に対立して分かたれている。ゆえにどんな習慣も身体の中にはない。

\\



{\scshape Sed contra est} quod philosophus, in {\itshape
Praedicamentis}, sanitatem corporis, vel infirmitatem insanabilem,
habitum nominari dicit.

&

しかし反対に、哲学者は『カテゴリー論』の中で、身体の健康や、治癒できない病気は習慣と呼ばれる、と述べている。

\\



{\scshape Respondeo dicendum} quod, sicut supra dictum est, habitus est quaedam
dispositio alicuius subiecti existentis in potentia vel ad formam, vel
ad operationem. Secundum ergo quod habitus importat dispositionem ad
operationem, nullus habitus est principaliter in corpore sicut in
subiecto. 


&

解答する。以下のように言われるべきである。前に述べられたとおり、習慣と
は、可能態において存在するある基体の、形相または働きへのある種の態勢で
ある。ゆえに、習慣が働きへの態勢を意味する限りにおいて、原理的に、どん
な習慣も、基体の中にあるようにして身体の中にあることはない。

\\


Omnis enim operatio corporis est aut a naturali qualitate
corporis; aut est ab anima movente corpus. Quantum igitur ad illas
operationes quae sunt a natura, non disponitur corpus per aliquem
habitum, quia virtutes naturales sunt determinatae ad unum; dictum est
autem quod habitualis dispositio requiritur ubi subiectum est in
potentia ad multa. 



&

理由は以下の通り。身体の働きは全て、身体の本性的な性質によるか、あるい
は身体を動かす魂によるかのどちらかである。ゆえに、本性による働きにかん
して、身体が何らかの習慣を通して態勢付けられることはない。なぜなら本性
的な力は一つのものへ限定されているが、習慣的な態勢が必要とされるのは、
基体が多くのものに対して可能態にある場合だからである。


\\


Operationes vero quae sunt ab anima per corpus,
principaliter quidem sunt ipsius animae, secundario vero ipsius
corporis. Habitus autem proportionantur operationibus, unde ex
similibus actibus similes habitus causantur, ut dicitur in II
Ethic. 



&

他方、魂によって、身体を通してある働きは、主要にはその魂に属するが、第二義的にはその身体に属する。
しかるに習慣は働きに対して比例的に関係するから、『ニコマコス倫理学』第2巻で言われるとおり、
類似した習慣が類似した作用にもとづいて原因される。

\\


Et ideo dispositiones ad tales operationes principaliter sunt
in anima. In corpore vero possunt esse secundario, inquantum scilicet
corpus disponitur et habilitatur ad prompte deserviendum operationibus
animae.


&

ゆえに、そのような働きに対する態勢が、主要に魂の中にある。他方、身体の
中には、第二義的に、すなわち身体が迅速に魂の働きに仕えるために態勢付け
られ、習慣付けられる限りにおいて、第二義的にありうる。


\\


Si vero loquamur de dispositione subiecti ad formam, sic habitualis
dispositio potest esse in corpore, quod comparatur ad animam sicut
subiectum ad formam. Et hoc modo sanitas et pulchritudo, et huiusmodi,
habituales dispositiones dicuntur. Non tamen perfecte habent rationem
habituum, quia causae eorum ex sua natura de facili transmutabiles
sunt.


&

他方、もし基体の形相に対する態勢について私たちが語るならば、その場合、
身体は魂に対して、基体が形相に対するように関係するが、その身体の中に習慣的な態勢があ
りうる。そしてこのしかたで、健康や美しさやそのような習慣的な態勢が語ら
れる。しかしこれらは完全に習慣の性格を持つわけではない。なぜなら、それ
らの原因は自らの本性に基づいて容易に変化しうるからである。

\\


Alexander vero posuit nullo modo habitum vel dispositionem primae
speciei esse in corpore, ut Simplicius refert in Commento
{\itshape Praedicament}., sed dicebat primam speciem qualitatis pertinere tantum
ad animam. 


&

他方、シンプリキオスが『カテゴリー論注解』で言及しているように、アレク
サンドロスは、第一の種に属するどんな習慣も態勢も身体の中にはないと考え、
むしろ性質の第一の種はただ魂だけに属すると述べた。

\\


Et quod Aristoteles inducit in {\itshape Praedicamentis} de sanitate
et aegritudine, non inducit quasi haec pertineant ad primam speciem
qualitatis, sed per modum exempli, ut sit sensus quod sicut aegritudo
et sanitas possunt esse facile vel difficile mobiles, ita etiam
qualitates primae speciei, quae dicuntur habitus et dispositio. 


&

そして、アリストテレスが『カテゴリー論』の中で健康と病気を導入するのは、
これらが性質の第一の種に属するからではなく、例示するしかたで、すなわち、
ちょうど病気と健康が動きやすい、あるいは動きにくいように、習慣や態勢と
言われる第一の種に属する性質もまた云々、という意味である、と述べた。

\\


Sed patet hoc esse contra intentionem Aristotelis. Tum quia eodem modo
loquendi utitur exemplificando de sanitate et aegritudine, et de
virtute et de scientia. Tum quia in VII {\itshape Physic}. expresse ponit inter
habitus pulchritudinem et sanitatem.


&

しかしこれがアリストテレスの意図に反することは明らかである。それは一つ
には、健康と病気について例示する同じ語り方を、徳と知識についても用いて
いるからであり、またもう一つには、『自然学』第7巻ではっきりと、習慣の
中に美しさと健康を置いているからである。

\\



{\scshape Ad primum ergo dicendum} quod obiectio illa procedit de habitu secundum
quod est dispositio ad operationem, et de actibus corporis qui sunt a
natura, non autem de his qui sunt ab anima, quorum principium est
voluntas.


&

第一異論に対しては、それゆえ、以下のように言われるべきである。かの反論
は、働きへの態勢であるかぎりにおける習慣について、また本性による身体の
作用について述べられていて、その根源外資である、魂による作用については
述べられていない。


\\



{\scshape Ad secundum dicendum} quod dispositiones corporales non sunt
simpliciter difficile mobiles, propter mutabilitatem corporalium
causarum. Possunt tamen esse difficile mobiles per comparationem ad
tale subiectum, quia scilicet, tali subiecto durante, amoveri non
possunt, vel quia sunt difficile mobiles per comparationem ad alias
dispositiones. 



&

第二異論に対しては以下のように言われるべきである。身体的態勢は、身体的
原因の可動性のために、端的に動きにくいわけではない。しかし、そのような
基体への関係によって動きにくいことはありうる。なぜなら、そのような基体
が留まるときには取り去られることができないためであり、あるいは、他の態
勢との関係で動きにくいからである。

\\


Sed qualitates animae sunt simpliciter difficile
mobiles, propter immobilitatem subiecti. Et ideo non dicit quod
sanitas difficile mobilis simpliciter sit habitus, sed quod est ut
habitus, sicut in Graeco habetur. Qualitates autem animae dicuntur
simpliciter habitus.


&

しかし魂の性質は、基体の不動性のために、端的に動きにくい。ゆえに動きに
くい健康が端的に習慣であるとは言われず、ギリシア語で書かれているように、
習慣のようなものと言われている。これに対して魂の性質は端的に習慣と言わ
れている。

\\


{\scshape Ad tertium dicendum} quod dispositiones
corporales quae sunt in prima specie qualitatis, ut quidam posuerunt,
differunt a qualitatibus tertiae speciei in hoc, quod qualitates
tertiae speciei sunt ut in fieri et ut in motu, unde dicuntur
passiones vel passibiles qualitates. 
Quando autem iam pervenerint ad
perfectum, quasi ad speciem, tunc iam sunt in prima specie
qualitatis.

&

第三異論に対しては以下のように言われるべきである。
ある人々は以下のように考えた。すなわち、性質の第一の種の中にある身体的な態勢は、第三の種に属する性質から次の点で異なっている。第三の種に属する性質は、いわば生成や運動の中にあるので、それゆえ受動や受動的性質と言われている。
しかし、すでに種へと到達するようにして完成に到達したときには、すでに性質の第一の種のなかにある。
\\


Sed hoc improbat Simplicius, in commento praedicamentorum, quia
secundum hoc calefactio esset in tertia specie qualitatis, calor autem
in prima, Aristoteles autem ponit calorem in tertia.


&


しかし、シンプリキオスは『カテゴリー論注解』でこれを非難している。なぜ
なら、これによれば、熱することは性質の第三の種の中に、熱は第一の種の中
にあったであろうが、アリストテレスは熱を第三の種の中に入れているからで
ある。

\\

Unde
Porphyrius dicit, sicut idem Simplicius refert, quod passio vel
passibilis qualitas, et dispositio et habitus, differunt in corporibus
secundum intensionem et remissionem. 



&

それゆえ、同じシンプリキオスが言及していることだが、ポルピュリオスは、
受動ないし受動的性質と態勢と習慣が、身体(物体)において、増強と減衰に
おいて異なるという。

\\

Quando enim aliquid recipit
caliditatem secundum calefieri tantum, non autem ut calefacere possit;
tunc est passio, si sit cito transiens, vel passibilis qualitas, si
sit manens. 

&

つまり、あるものが熱を、ただ熱せられるだけ受け取って熱することができる
までは受け取らない場合、もしそれがすぐに移り変わるならば受動であり、も
し留まるならば受動的性質である。

\\


Quando autem iam ad hoc perducitur quod potest etiam
alterum calefacere, tunc est dispositio, si autem ulterius intantum
confirmetur quod sit difficile mobilis, tunc erit habitus, ut sic
dispositio sit quaedam intensio seu perfectio passionis vel passibilis
qualitatis, habitus autem dispositionis. 


&

これに対して、すでに他のものを熱することができるまでに到達しているなら、
その場合には態勢であり、もしさらに変化しにくいまでに固定されているなら
ば、その場合には習慣である。つまり態勢が、受動や受動的性質の増強ないし
完成であるように、習慣は態勢の増強ないし完成である。

\\

Sed hoc improbat Simplicius, quia talis intensio et remissio non
important diversitatem ex parte ipsius formae, sed ex diversa
participatione subiecti. Et ita non diversificarentur per hoc species
qualitatis.


&

しかしシンプリキオスはこれを非難している。なぜならそのような増強と減衰
は、形相の側からの差異を含意せず、むしろ基体のさまざまな分有に基づくの
で、したがって、このことによって性質の種が分けられることがないからであ
る。

\\


Et ideo aliter dicendum est quod, sicut supra dictum est,
commensuratio ipsarum qualitatum passibilium secundum convenientiam ad
naturam, habet rationem dispositionis, et ideo, facta alteratione
circa ipsas qualitates passibiles, quae sunt calidum et frigidum,
humidum et siccum, fit ex consequenti alteratio secundum aegritudinem
et sanitatem. Primo autem et per se non est alteratio secundum
huiusmodi habitus et dispositiones.


&

それゆえ、別のしかたで次のように言われるべきである。すでに述べられたと
おり、これらの受動的性質の共通性は、本性への一致にしたがって、態勢の性
格を持ち、それゆえ、熱と冷、湿と乾といったこれらの受動的性質をめぐって
変化が生じたとき、結果的に、病気と健康における変化が生じる。しかし、第
一に自体的に、そのような習慣と態勢において変化があるわけでは亡い。


\end{longtable}




\newpage



\rhead{a.~2}
\begin{center}
{\Large {\bf ARTICULUS SECUNDUS}}\\
{\large UTRUM ANIMA SIT SUBIECTUM HABITUS\\SECUNDUM SUAM ESSENTIAM, VEL SUAM POTENTIAM}\\
{\footnotesize II {\itshape Sent.}, d.26, a.3, ad 4, 5.}\\
{\Large 第二項\\魂が習慣の基体であるのは自らの本質においてか\\それとも自らの能力においてか}
\end{center}

\begin{longtable}{p{21em}p{21em}}


{\scshape Ad secundum sic proceditur}. Videtur quod habitus sint in
anima magis secundum essentiam quam secundum potentiam. Dispositiones
enim et habitus dicuntur in ordine ad naturam, ut dictum est. Sed
natura magis attenditur secundum essentiam animae quam secundum
potentias, quia anima secundum suam essentiam est natura corporis
talis, et forma eius. Ergo habitus sunt in anima secundum eius
essentiam et non secundum potentiam.


&

第二項の問題へ、議論は以下のように進められる。習慣は、能力よりもむしろ
本質において魂の中にあると思われる。理由は以下の通り。すでに述べられた
とおり、態勢と習慣は本性への秩序において語られる。しかし本性は魂の能力
においてよりも本質においてより見出される。なぜなら魂は自らの本質におい
て、これこれの本質の本性でアリ、それの形相だからである。ゆえに習慣は魂
の中に、その本質においてあるのであり、能力においてあるのではない。

\\



2. {\scshape Praeterea}, accidentis non est accidens. Habitus autem
est quoddam accidens. Sed potentiae animae sunt de genere accidentium,
ut in primo dictum est\footnote{Ad quintum dicendum quod, si accidens
accipiatur secundum quod dividitur contra substantiam, sic nihil
potest esse medium inter substantiam et accidens, quia dividuntur
secundum affirmationem et negationem, scilicet secundum esse in
subiecto et non esse in subiecto. Et hoc modo, cum potentia animae non
sit eius essentia, oportet quod sit accidens, et est in secunda specie
qualitatis. Si vero accipiatur accidens secundum quod ponitur unum
quinque universalium, sic aliquid est medium inter substantiam et
accidens. Quia ad substantiam pertinet quidquid est essentiale rei,
non autem quidquid est extra essentiam, potest sic dici accidens, sed
solum id quod non causatur ex principiis essentialibus
speciei. Proprium enim non est de essentia rei, sed ex principiis
essentialibus speciei causatur, unde medium est inter essentiam et
accidens sic dictum. Et hoc modo potentiae animae possunt dici mediae
inter substantiam et accidens, quasi proprietates animae
naturales. Quod autem Augustinus dicit, quod notitia et amor non sunt
in anima sicut accidentia in subiecto, intelligitur secundum modum
praedictum, prout comparantur ad animam, non sicut ad amantem et
cognoscentem; sed prout comparantur ad eam sicut ad amatam et
cognitam. Et hoc modo procedit sua probatio, quia si amor esset in
anima amata sicut in subiecto, sequeretur quod accidens transcenderet
suum subiectum; cum etiam alia sint amata per animam. ({\itshape ST} I, q.77, a.1, ad 5.)}. Ergo habitus
non est in anima ratione suae potentiae.

&

さらに、附帯性は附帯性に属さない。しかるに習慣は一種の附帯性である。し
かるに魂の能力は、第一部で述べられたとおり、附帯性の類に属する。ゆえに
習慣は自らの能力の性格において魂の中にあるのではない。

\\



3. {\scshape Praeterea}, subiectum est prius eo quod est in subiecto. Sed habitus,
cum pertineat ad primam speciem qualitatis, est prior quam potentia,
quae pertinet ad secundam speciem. Ergo habitus non est in potentia
animae sicut in subiecto.

&

さらに、基体は、基体においてあるものに先行する。しかるに習慣は性質の第
一の種に属するのだから、第二の種に属する能力に先行する。ゆえに習慣は魂
の能力の中に、基体の中にあるようにしてあるのではない。

\\



{\scshape Sed contra est} quod philosophus, in I {\itshape Ethic}., ponit diversos habitus
in diversis partibus animae.

&

しかし反対に、哲学者は『ニコマコス倫理学』第1巻で魂のさまざまな部分に
さまざまな習慣があるとしている。

\\



{\scshape Respondeo dicendum} quod, sicut supra dictum est, habitus importat
dispositionem quandam in ordine ad naturam, vel ad operationem. Si
ergo accipiatur habitus secundum quod habet ordinem ad naturam, sic
non potest esse in anima, si tamen de natura humana loquamur, quia
ipsa anima est forma completiva humanae naturae; unde secundum hoc,
magis potest esse aliquis habitus vel dispositio in corpore per
ordinem ad animam, quam in anima per ordinem ad corpus. 

&

解答する。以下のように言われるべきである。前に述べられたとおり、習慣は
本性または働きへの秩序におけるある種の態勢を意味する。ゆえにもし、習慣
が本性への秩序をもつかぎりにおいて理解されるならば、それは魂の中にはあ
りえない。しかしもし、人間の本性について語るならば、その魂は人間本性を
完成させうる形相なので、その限りにおいて、ある習慣や態勢は、身体への秩
序によって魂の中にあるよりも、魂への秩序によって身体の中にありうる。

\\

Sed si
loquamur de aliqua superiori natura, cuius homo potest esse particeps,
secundum illud II Petr. I, ut simus consortes naturae divinae, sic
nihil prohibet in anima secundum suam essentiam esse aliquem habitum,
scilicet gratiam, ut infra dicetur. 

&

しかしもし、『ペトロの手紙二』第二章「神の本性と同類になるために」
\footnote{「この栄光と力ある顕現によって、私たちには尊く大いなる約束が
与えられています。それは、あなたがたがこの約束によって、世の欲にまみれ
た腐敗を免れ、神の本性にあずかる者となるためです。」(1:4) }によれば、
人間がそれに与りうるようなより上位の何らかの本性について私たちが語るの
であれば、その場合には、魂の中に、自らの本質において何らかの習慣、すな
わち後に述べられるであろう恩恵があることを妨げるものは何もない。

\\

Si vero accipiatur habitus in
ordine ad operationem, sic maxime habitus inveniuntur in anima,
inquantum anima non determinatur ad unam operationem, sed se habet ad
multas, quod requiritur ad habitum, ut supra dictum est. Et quia anima
est principium operationum per suas potentias, ideo secundum hoc,
habitus sunt in anima secundum suas potentias.

&

他方で、働きへの秩序における習慣が理解されるならば、その場合には、習慣
は最大限に魂の中に見出される。それは魂が一つの働きへと限定されず、多く
の働きへ関係する限りにおいてである。前に述べられたとおり、習慣にはこの
ことが必要とされる。また魂は自らの能力を通して働きの根源であるから、そ
れゆえその限りにおいて、習慣は自らの能力において魂の中にある。

\\

\end{longtable}

\schema
{
    \schemabox{habitus est\\dispositio\\in ordine ad}
}
{
    \schema
    {
	\schemabox{naturam}
    }
    {
	\schemabox{humanam --- habitus sunt in corpore per ordine ad animam}
	\schemabox{superiorem --- habitus (gratia) est in anima sec. suam essentiam}
    }
    \schemabox{operationem --- habitus sunt in anima sec. suas potentias}
}


\begin{longtable}{p{21em}p{21em}}


{\scshape Ad primum ergo dicendum} quod essentia animae pertinet ad naturam
humanam, non sicut subiectum disponendum ad aliquid aliud, sed sicut
forma et natura ad quam aliquis disponitur.

&

第一異論に対しては、それゆえ、以下のように言われるべきである。魂の本質
は人間の本性に属するが、他の何かへ態勢付ける基体として属するのではない。
むしろそれは、誰かがそれへと態勢付けられるところの形相ないし本性として
属する。

\\



{\scshape Ad secundum dicendum} quod accidens per se non potest esse subiectum
accidentis. Sed quia etiam in ipsis accidentibus est ordo quidam,
subiectum secundum quod est sub uno accidente, intelligitur esse
subiectum alterius. Et sic dicitur unum accidens esse subiectum
alterius, ut superficies coloris. Et hoc modo potest potentia esse
subiectum habitus.

&

第二異論に対しては以下のように言われるべきである。附帯性は、それ自体では、
附帯性の基体ではありえない。しかし附帯性の中にもある種の秩序があるので、
ある附帯性のもとにある限りにおいて、ある基体が別のものの基体として理解
される。このようにして、ある附帯性が別のものの基体であると言われる。た
とえば、表面が色の基体である、と言われるように。そしてこのしかたで、能
力が習慣の基体でありうる。

\\



{\scshape Ad tertium dicendums} quod habitus praemittitur potentiae, secundum quod
importat dispositionem ad naturam, potentia autem semper importat
ordinem ad operationem, quae est posterior, cum natura sit operationis
principium. Sed habitus cuius potentia est subiectum, non importat
ordinem ad naturam, sed ad operationem. Unde est posterior
potentia. Vel potest dici quod habitus praeponitur potentiae sicut
completum incompleto, et actus potentiae. Actus enim naturaliter est
prior; quamvis potentia sit prior ordine generationis et temporis, ut
dicitur in VII et IX Metaphys.

&

第三異論に対しては以下のように言われるべきである。習慣が能力の前に置か
れるのは、それが本性への態勢を含意する限りにおいてである。しかし本性は
働きの根源なので、能力は常に、より後のものである働きへの秩序を含意する。
しかし能力が基体である習慣は、本性ではなく働きへの秩序を含意する。した
がってそれは能力よりも後である。あるいは別に以下のように言われうる。す
なわち、習慣は、完成されたものが未完成のものに先行し、現実態が可能態に
先行するような形で、能力の前に置かれる。なぜなら、『形而上学』第7巻と
第9巻で言われるように、可能態は生成と時間の秩序においては先行するが、
自然本性的には現実態が先行するからである。


\end{longtable}
\newpage





\rhead{a.~3}
\begin{center}
{\Large {\bf ARTICULUS TERTIUS}}\\
{\large UTRUM IN POTENTIIS SENSITIVAE PARTIS\\POSSIT ESSE ALIQUIS HABITUS}\\
{\footnotesize III {\itshape Sent.}, d.14, a.1, qu$^{a}$ 2; d.23, q.1, a.1; {\itshape De Virtut.}, q.1, a.1.}\\
{\Large 第三項\\感覚的部分の能力のなかに\\何らかの習慣はありうるか}
\end{center}

\begin{longtable}{p{21em}p{21em}}

{\scshape Ad tertium sic proceditur}. Videtur quod in potentiis
 sensitivae partis non possit esse aliquis habitus. Sicut enim
 potentia nutritiva pars est irrationalis, ita et sensitiva. Sed in
 potentiis nutritivae partis non ponitur aliquis habitus. Ergo nec in
 potentiis sensitivae partis aliquis habitus debet poni.

&

第三項の問題へ、議論は以下のように進められる。感覚的部分の能力の中には
どんな習慣もありえないと思われる。理由は以下の通り。栄養的部分が非理性
的な能力であるように、感覚的部分も非理性的な能力である。しかるに栄養的
部分の能力の中にはどんな習慣もない。ゆえに感覚的部分の能力の中にもどん
な習慣も置かれるべきでない。

\\



2. {\scshape Praeterea}, sensitivae partes sunt communes nobis et brutis. Sed in
 brutis non sunt aliqui habitus, quia non est in eis voluntas, quae in
 definitione habitus ponitur, ut supra dictum est. Ergo in potentiis
 sensitivis non sunt aliqui habitus.

&

さらに、感覚的部分は私たちと非理性的動物たちとに共通である。しかし非理
性的動物の中にはどんな習慣もない。なぜなら、前に述べられたとおり、それ
らの中には習慣の定義の中に置かれている意志がないからである。ゆえに感覚
的能力の中にはどんな習慣もない。

\\



3. {\scshape Praeterea}, habitus animae sunt scientiae et virtutes, et
sicut scientia refertur ad vim apprehensivam, ita virtus ad vim
appetitivam. Sed in potentiis sensitivis non sunt aliquae scientiae,
cum scientia sit universalium, quae vires sensitivae apprehendere non
possunt. Ergo etiam nec habitus virtutum in partibus sensitivis esse
possunt.

&

さらに、知と徳は魂の習慣である。そして知が把握力に県警するように徳は欲
求力に関係する。しかし、知は普遍的なものに関わり、感覚的な力は普遍的な
ものを把握できないので、感覚的能力の中に知はない。ゆえに徳の習慣も感覚
的な部分の中に存在しえない。

\\



{\scshape Sed contra est} quod philosophus dicit, in III {\itshape
 Ethic}., quod {\itshape aliquae virtutes}, scilicet temperantia et
 fortitudo, {\itshape sunt irrationabilium partium}.

&

しかし反対に、哲学者は『ニコマコス倫理学』第3巻で、「ある徳(すなわち
 節制と勇気)は非理性的な部分に属する」と言っている。

\\



{\scshape Respondeo dicendum} quod vires sensitivae dupliciter possunt
 considerari, uno modo, secundum quod operantur ex instinctu naturae;
 alio modo, secundum quod operantur ex imperio rationis. Secundum
 igitur quod operantur ex instinctu naturae, sic ordinantur ad unum,
 sicut et natura. 


&

解答する。以下のように言われるべきである。感覚的な力は二通りに考察され
うる。一つには、本性の本能に基づいて働く限りにおいてであり、もう一つに
は理性の命令に基づいて働く限りにおいてである。ゆえに、自然の本能に基づ
 いて働く限りにおいては、(感覚的な力は)本性と同じように、一つのもの
 へと秩序づけられている。

\\


Et ideo sicut in potentiis naturalibus non sunt
 aliqui habitus, ita etiam nec in potentiis sensitivis, secundum quod
 ex instinctu naturae operantur. Secundum vero quod operantur ex
 imperio rationis, sic ad diversa ordinari possunt. Et sic possunt in
 eis esse aliqui habitus, quibus bene aut male ad aliquid disponuntur.

&

そしてそれゆえ、本性的な能力においてどんな習慣もないように、本性の本能
 に基づいて働く限りにおいて、感覚的な能力の中にも習慣はない。
他方で、理性の命令に基づいて働く限りにおいてなら、それは様々なものへ秩
 序づけられうる。したがって、それらの中に、それによってよくまたは悪く何
 かへと態勢付けられるところの何らかの習慣がありうる。

\\



{\scshape Ad primum ergo dicendum} quod vires nutritivae partis non sunt natae
 obedire imperio rationis, et ideo non sunt in eis aliqui habitus. Sed
 vires sensitivae natae sunt obedire imperio rationis, et ideo in eis
 esse possunt aliqui habitus; nam secundum quod obediunt rationi,
 quodammodo rationales dicuntur, ut in I {\itshape Ethic}. dicitur.

&

第一異論に対しては、それゆえ、以下のように言われるべきである。栄養的部
分の力は理性の命令に従うような本性をもたないので、その中にはいかなる習
慣もない。しかし感覚的な力は理性の命令に従う本性をもつので、それらの中
に何らかの習慣があることが可能である。というのも『ニコマコス倫理学』第
1巻で言われているように、理性にしたがう限りにおいて、それらはある意味
で理性的なものと言われるからである。

\\



{\scshape Ad secundum dicendum} quod vires sensitivae in brutis
animalibus non operantur ex imperio rationis; sed si sibi relinquantur
bruta animalia, operantur ex instinctu naturae. Et sic in brutis
animalibus non sunt aliqui habitus ordinati ad operationes. Sunt tamen
in eis aliquae dispositiones in ordine ad naturam, ut sanitas et
pulchritudo.


&

第二異論に対しては以下のように言われるべきである。
非理性的な動物の感覚的な力は理性の命令に基づいて働くのではなく、もし
非理性的な動物が放置されたら、それらは本性の本能に基づいて働く。
したがって、非理性的な動物の中には、働きへと秩序づけられたどんな習慣も
 ない。しかし、それらの中に、健康や美しさのような、本性への秩序におけ
 るなんらかの態勢はある。

\\

Sed quia bruta animalia a ratione hominis per quandam consuetudinem
disponuntur ad aliquid operandum sic vel aliter, hoc modo in brutis
animalibus habitus quodammodo poni possunt, unde Augustinus dicit, in
libro {\itshape Octoginta Trium Quaest}., quod {\itshape videmus
immanissimas bestias a maximis voluptatibus absterreri dolorum metu,
quod cum in earum consuetudinem verterit, domitae et mansuetae
vocantur}.


&

しかし非理性的動物は、人間の理性から、何らかの習慣によって、ある働きを
 することへと態勢付けられることがあるので、この意味で、非理性的動物の
 中に、ある意味で、習慣が置かれることが可能である。このことから、アウ
 グスティヌスは『八十三問題集』の中で次のように述べている。「この上な
 く野生の獣が痛みへの恐怖によって最大の快楽から引き離されるのを私たち
 は目にするが、それがそれらの習いになったとき、飼い慣らされた、とか飼
 われた、と呼ばれる」。


\\

Deficit tamen ratio habitus quantum ad usum voluntatis, quia non
habent dominium utendi vel non utendi, quod videtur ad rationem
habitus pertinere. Et ideo, proprie loquendo, in eis habitus esse non
possunt.

&

しかし意志の使用にかんして、それは習慣の性格を欠く。なぜなら、習慣の性
 格に属すると思われる、使用するかしないかについての主導権をもたないか
 らである。ゆえに、厳密に言うならば、それらに習慣はありえない。

\\



{\scshape Ad tertium dicendum} quod appetitus sensitivus natus est moveri ab
 appetitu rationali, ut dicitur in III de anima, sed vires rationales
 apprehensivae natae sunt accipere a viribus sensitivis. 


&

第三異論に対しては以下のように言われるべきである。
『デ・アニマ』第3巻で言われるように、感覚的欲求は、理性的欲求によって
 動かされる本性を持つ。しかし理性的欲求の力は、感覚的な力から受け取る
 本性を持つ。

\\

Et ideo magis
 convenit quod habitus sint in viribus sensitivis appetitivis quam in
 viribus sensitivis apprehensivis, cum in viribus sensitivis
 appetitivis non sint habitus nisi secundum quod operantur ex imperio
 rationis. 


&

ゆえに、習慣は、感覚的把握の力よりも感覚的欲求の力の中にあることがより
適合的である。なぜなら、感覚的把握の力の中には、理性の命令に基づいて働
く限りにおいてでなければ習慣はないからである。

\\

Quamvis etiam in ipsis interioribus viribus sensitivis apprehensivis
possint poni aliqui habitus, secundum quos homo fit bene memorativus
vel cogitativus vel imaginativus, unde etiam philosophus dicit, in
cap.~{\itshape de Memoria}, quod {\itshape consuetudo multum operatur
ad bene memorandum}, quia etiam istae vires moventur ad operandum ex
imperio rationis.



&

さらに、人間がそれにしたがってよく記憶したり思考したり想像したりすると
 ころの何らかの習慣が、感覚的把握の内的な力の中にも置かれうる。そのこ
 とから、哲学者もまた『記憶について』のある章で、「習慣はよく記憶する
 ために大いに働く」と述べている。というのもこれらの力が理性の命令に基
 づいて働くことへと動かされるからである。

\\

Vires autem apprehensivae exteriores, ut visus et auditus et
huiusmodi, non sunt susceptivae aliquorum habituum, sed secundum
dispositionem suae naturae ordinantur ad suos actus determinatos;
sicut et membra corporis, in quibus non sunt habitus, sed magis in
viribus imperantibus motum ipsorum.

&

他方、視覚や聴覚のような外的な把握力は、どんな習慣も受け取り得ず、自ら
の本性の態勢に従って、自らの限定された作用へと秩序づけられる。ちょうど
そのなかに習慣がない四肢のように。習慣はむしろそれらの運動を命じる力の
中にある。


\end{longtable}
\newpage


\rhead{a.~4}
\begin{center}
{\Large {\bf ARTICULUS QUARTUS}}\\
{\large UTRUM IN IPSO INTELLECTU SIT ALIQUIS HABITUS}\\
{\footnotesize III {\itshape Sent.}, d.14, a.1, qu$^{a}$ 2; d.23, q.1,
 a.1; {\itshape De Verit.}, q.20, a.2; {\itshape De Virtut.}, q.1, a.1.}\\
{\Large 第四項\\知性自体の中に何らかの習慣はあるか}
\end{center}

\begin{longtable}{p{21em}p{21em}}

{\scshape Ad quartum sic proceditur}. Videtur quod in intellectu non sint aliqui
 habitus. Habitus enim operationibus conformantur, ut dictum est. Sed
 operationes hominis sunt communes animae et corpori, ut dicitur in I
 {\itshape de Anima}. Ergo et habitus. Sed intellectus non est actus corporis, ut
 dicitur in III {\itshape de Anima}. Ergo intellectus non est subiectum alicuius
 habitus.


&

第四項の問題へ、議論は以下のように進められる。知性の中にはどんな習慣も
ないと思われる。理由は以下の通り。すでに述べられたとおり、習慣は働きに
一致する。しかるに、『デ・アニマ』第1巻で言われるように、人間の働きは
魂と身体とに共通である。ゆえに習慣も共通である。しかるに『デ・アニマ』
第2巻で言われるように、知性は身体の作用ではない。ゆえに知性はどんな習慣の基体でもない。

\\




2. {\scshape Praeterea}, omne quod est in aliquo, est in eo per modum eius in quo
 est. Sed id quod est forma sine materia, est actus tantum, quod autem
 est compositum ex forma et materia, habet potentiam et actum
 simul. Ergo in eo quod est forma tantum, non potest esse aliquid quod
 sit simul in potentia et actu, sed solum in eo quod est compositum ex
 materia et forma. Sed intellectus est forma sine materia. Ergo
 habitus, qui habet potentiam simul cum actu, quasi medium inter
 utrumque existens, non potest esse in intellectu; sed solum in
 coniuncto, quod est compositum ex anima et corpore.

&

さらに、全て或るものの中に在るものは、その中に在るところのその或るものの在り方によっ
てその中に在る。しかるに質料をもたない形相であるものは、ただ現実態であ
る。これに対して質料と形相から複合されているものは、可能態と現実態を同
時に持つ。ゆえに形相だけであるものの中に在るものは、同時に可能態と現実
態にあるようなものではありえず、質料と形相から複合されたものの中だけに
ある。しかるに知性は質料をもたない形相である。ゆえに、現実態と同時に可
能態を持つ習慣は知性の中に在りえず、魂と身体から複合された結合体の中だ
けに在る。

\\




3. {\scshape Praeterea}, habitus est {\itshape dispositio secundum quam aliquis
bene vel male disponitur ad aliquid}, ut dicitur in V {\itshape Metaph}. Sed quod
aliquis bene vel male sit dispositus ad actum intellectus, provenit ex
aliqua corporis dispositione, unde etiam in II {\itshape de Anima} dicitur quod
{\itshape molles carne bene aptos mente videmus}. Ergo habitus cognoscitivi non
sunt in intellectu, qui est separatus; sed in aliqua potentia quae est
actus alicuius partis corporis.

&

さらに、『形而上学』第5巻で言われるように、習慣と態勢は、ある人が、そ
れにしたがってよりよくまたは悪く何かへと態勢付けられるところのものであ
る。しかるに、ある人がよくあるいは悪く知性の作用へ態勢付けられるという
ことは、身体のなんらかの態勢に基づいて到来する。このことから『デ・アニ
マ』第2巻でも「肉が柔らかい人々は精神も適合的であるように見える」と言
われている。ゆえに認識の習慣は(身体から)分離している知性の中にはなく、
身体の何らかの部分の作用であるような何らかの能力の中にある。

\\




{\scshape Sed contra est} quod philosophus, in VI {\itshape Ethic}., ponit scientiam et
 sapientiam et intellectum, qui est habitus principiorum, in ipsa
 intellectiva parte animae.

&

しかし反対に、哲学者は『ニコマコス倫理学』第6巻で、地と知恵と知性は、
諸原理の習慣であり、魂の知性的部分の中にあると述べている。

\\




{\scshape Respondeo dicendum} quod circa habitus cognoscitivos diversimode sunt
 aliqui opinati. Quidam enim, ponentes intellectum possibilem esse
 unum in omnibus hominibus, coacti sunt ponere quod habitus
 cognoscitivi non sunt in ipso intellectu, sed in viribus interioribus
 sensitivis. 

&

解答する。以下のように言われるべきである。人々は認識的習慣についてさま
ざまに論じてきた。可能知性がすべての人々において一つだと考える人々は、
認識的徳は知性の中にではなく、内的感覚の力においてあると考えるように強
いられた。

\\


Manifestum est enim quod homines in habitibus
 diversificantur, unde non possunt habitus cognoscitivi directe poni
 in eo quod, unum numero existens, est omnibus hominibus commune. 


&

というのも、人々が習慣においてさまざまであることは明らかなので、認識的
諸徳が、数において一つのものとして存在し、全ての人間に共通であるものの
中に直接に置かれることは不可能だからである。

\\

Unde si intellectus possibilis sit unus numero omnium hominum, habitus
scientiarum, secundum quos homines diversificantur, non poterunt esse
in intellectu possibili sicut in subiecto, sed erunt in viribus
interioribus sensitivis, quae sunt diversae in diversis.

&

このことから、もし可能知性が全ての人間に属し数的に一つであるならば、そ
れにしたがって人間がさまざまとなる知の習慣は、可能知性の中にそれを基体
として存在することはできず、異なる人においてさまざまである内的感覚の力
において存在することになろう。


\\

Sed ista positio, primo quidem, est contra intentionem
Aristotelis. Manifestum est enim quod vires sensitivae non sunt
rationales per essentiam, sed solum per participationem, ut dicitur in
I {\itshape Ethic}.

&

しかしこの考えは、第一にアリストテレスの意図に反する。理由は以下の通り。
『ニコマコス倫理学』第1巻で言われているように、感覚的能力はその本質に
よって理性的でなく、ただ分有によってのみ理性的である。


\\

Philosophus autem ponit intellectuales virtutes, quae sunt sapientia,
scientia et intellectus, in eo quod est rationale per essentiam. Unde
non sunt in viribus sensitivis, sed in ipso intellectu.

&

しかし哲学者は知恵、学知、直知といった知的な諸徳を、それが本質によって
理性的である点において、考えている。したがって、それらは感覚的な力では
なく知性においてある。

\\


Expresse etiam dicit, in III {\itshape de Anima}, quod intellectus possibilis,
{\itshape cum sic fiat singula}, idest cum reducatur in actum singulorum per
species intelligibiles, {\itshape tunc fit secundum actum eo modo quo sciens
dicitur esse in actu, quod quidem accidit cum aliquis possit operari
per seipsum}, scilicet considerando.

&

さらに『デ・アニマ』第3巻で、彼ははっきりと次のように述べている。すな
わち、可能知性は、「そのように個体になるので」、つまり可知的形象によっ
て個々のものに属する作用へと引き出されるので、「その場合には、知る者が
現実態にあると言われるのと同じしかたで現実態において生じるのであり、そ
れが生じるのは、ある者がそれ自身によって働くことができる(すなわち考察
することによって)ときである」。


\\

{\itshape Est quidem igitur et tunc potentia quodammodo; non tamen
similiter ut ante addiscere aut invenire}. Ipse ergo intellectus
possibilis est in quo est habitus scientiae quo potest considerare
etiam cum non considerat.

&

「ゆえに、そのときにもある意味で可能態にあるが、学んだり見出したりする
前と同じではないかたちで」。ゆえに、可能知性自身は、その中に知の習慣が
存在するところであり、それによって、考察していないときにでも考察するこ
とが可能である。

\\

Secundo etiam, haec positio est contra rei veritatem. Sicut enim eius
est potentia cuius est operatio, ita etiam eius est habitus cuius est
operatio. Intelligere autem et considerare est proprius actus
intellectus. Ergo et habitus quo consideratur, est proprie in ipso
intellectu.

&

さらに第二に、この考えは事柄の真理に反している。つまり、働きが属するも
のに能力も属するように、働きが属するものに習慣も属する。しかるに知性認
識することと考察することは知性の固有の作用である。ゆえに、それによって
考察されるところの習慣も、固有に、知性自身の中にある。

\\




{\scshape Ad primum ergo dicendum} quod quidam dixerunt, ut Simplicius refert in
 {\itshape Commento Praedicamentorum}, quod quia omnis operatio hominis est
 quodammodo coniuncti, ut philosophus dicit in I {\itshape de Anima}; ideo nullus
 habitus est animae tantum, sed coniuncti. Et per hoc sequitur quod
 nullus habitus sit in intellectu, cum intellectus sit separatus, ut
 ratio proposita procedebat. 

&

第一異論に対しては、それゆえ、以下のように言われるべきである。シンプリ
キオスが『カテゴリー論注解』で言及しているように、ある人々は、哲学者が
『デ・アニマ』第1巻で言うように、人間の全ての働きはある意味で結合体に
属するので、それゆえ、どんな習慣も魂だけに属することはなく、結合体に属
すると述べた。そしてこのことから、どんな習慣も知性の中にないことが帰結
する。なぜなら、前に提示された議論がそう論じていたように、知性は分離し
ているからである。


\\

Sed ista ratio non cogit. Habitus enim non est dispositio obiecti ad
potentiam, sed magis dispositio potentiae ad obiectum, unde habitus
oportet quod sit in ipsa potentia quae est principium actus, non autem
in eo quod comparatur ad potentiam sicut obiectum.

&

しかしこの論の結論はこうはならない。理由は以下の通り。習慣は、対象の能
力に対する態勢ではなく、むしろ能力の対象に対する態勢である。したがって
習慣は、対象として能力に関係するものの中にではなく、作用の根源である能
力自体の中にある必要がある。

\\


Ipsum autem intelligere non dicitur commune esse animae et corpori,
nisi ratione phantasmatis, ut dicitur in I {\itshape de Anima}. Patet autem quod
phantasma comparatur ad intellectum possibilem ut obiectum, ut dicitur
in III {\itshape de Anima}.

&

しかるに知性認識すること自体は、『デ・アニマ』第1巻で言われるように、
表象像を根拠としてでなければ、魂と身体に共通だとは言われない。ところが
表象像は、『デ・アニマ』第3巻で言われるように、可能知性に対して対象と
して関係する。

\\

Unde relinquitur quod habitus intellectivus sit principaliter ex parte
ipsius intellectus, non autem ex parte phantasmatis, quod est commune
animae et corpori.

&

したがって、知性的な習慣は主要に、知性自身の側からあり、魂と身体に共通
する表象像の側からではないことが帰結する。

\\

Et ideo dicendum est quod intellectus possibilis est subiectum
habitus, illi enim competit esse subiectum habitus, quod est in
potentia ad multa; et hoc maxime competit intellectui possibili. Unde
intellectus possibilis est subiectum habituum intellectualium.

&

ゆえに、以下のように言われるべきである。可能知性は習慣の基体である。と
いうのも、多くのものに対して可能態にあるものに、習慣の基体であることが
適合するが、これは可能知性に最大限に適合するからである。したがって、可
能知性は、知性的な諸習慣の基体である。

\\


{\scshape Ad secundum dicendum} quod, sicut potentia ad esse sensibile
convenit materiae corporali, ita potentia ad esse intelligibile
convenit intellectui possibili. Unde nihil prohibet in intellectu
possibili esse habitum, qui est medius inter puram potentiam et actum
perfectum.

&

第二異論に対しては以下のように言われるべきである。可感的存在への能力が
物体的質料に適合するように、可知的存在への能力は可能知性に適合する。し
たがって、純粋な可能態と完全な現実態の中間にある可能知性の中に習慣が在
ることは、何ら差し支えない。

\\




{\scshape Ad tertium dicendum} quod, quia vires apprehensivae interius
praeparant intellectui possibili proprium obiectum; ideo ex bona
dispositione harum virium, ad quam cooperatur bona dispositio
corporis, redditur homo habilis ad intelligendum. Et sic habitus
intellectivus secundario potest esse in istis viribus. Principaliter
autem est in intellectu possibili.

&

第三異論に対しては以下のように言われるべきである。内的な把握力は、可能
知性に固有の対象を準備する。ゆえに、それらの力のよい態勢に基づいて、人
間は知性認識するのに適したものとなるが、その態勢のために、身体のよい態
勢が協働する。このようにして知的な習慣が第二義的にこれらの力の中にも存
在しうる。しかし主要には、可能知性の中にある。

\end{longtable}
\newpage


\rhead{a.~5}
\begin{center}
{\Large {\bf ARTICULUS QUINTUS}}\\
{\large UTRUM IN VOLUNTATE SIT ALIQUIS HABITUS}\\
{\footnotesize II {\itshape Sent.}, d.27, a.1, ad 2; III, d.23, q.1, a.1; {\itshape De Verit.}, q.20, a.2; {\itshape De Virtut.}, q.1, a.1.}\\
{\Large 第五項\\意志の中に何らかの習慣があるか}
\end{center}

\begin{longtable}{p{21em}p{21em}}

{\scshape Ad quintum sic proceditur}. Videtur quod in voluntate non
sit aliquis habitus. Habitus enim qui in intellectu est, sunt species
intelligibiles, quibus intelligit actu. Sed voluntas non operatur per
aliquas species. Ergo voluntas non est subiectum alicuius habitus.

&

第五項の問題へ、議論は以下のように進められる。
意志の中にはどんな習慣もないと思われる。理由は以下の通り。
知性の中にある習慣は可知的形象であり、それによって現実に知性認識する。
しかるに意志は何らかの形象によって働くのではない。
ゆえに意志はどんな習慣の基体でもない。

\\



2. {\scshape Praeterea}, in intellectu agente non ponitur aliquis habitus, sicut in
 intellectu possibili, quia est potentia activa. Sed voluntas est
 maxime potentia activa, quia movet omnes potentias ad suos actus, ut
 supra dictum est. Ergo in ipsa non est aliquis habitus.

&

さらに、可能知性とは異なり、能動知性の中にはどんな習慣も置かれない。な
 ぜならそれは能動的な能力だからである。しかるに意志は最大限に能動的な
 能力である。なぜなら、前に述べられたとおり\footnote{I-IIae第9問第1項}、
 全ての能力を自らの働きへと動かすからである。ゆえに、その中にはどんな
 習慣もない。

\\



3. {\scshape Praeterea}, in potentiis naturalibus non est aliquis habitus, quia ex
 sua natura sunt ad aliquid determinatae. Sed voluntas ex sua natura
 ordinatur ad hoc quod tendat in bonum ordinatum ratione. Ergo in
 voluntate non est aliquis habitus.

&

さらに、自然的能力の中にはどんな習慣もない。なぜならそれらは自らの本性
 に基づいて何かへと限定されているからである。
しかるに意志は自らの本性に基づいて、理性によって秩序づけられた善へと向
 かうことへと秩序づけられている。
ゆえに意志の中にはどんな習慣もない。

\\



{\scshape Sed contra est} quod iustitia est habitus quidam. Sed
 iustitia est in voluntate, est enim iustitia {\itshape habitus
 secundum quem aliqui volunt et operantur iusta}, ut dicitur in V
 {\itshape Ethic}. Ergo voluntas est subiectum alicuius habitus.

&

しかし反対に、正義はある種の習慣である。しかるに正義は意志においてある。
 なぜなら、『ニコマコス倫理学』第5巻で言われるように正義は「それにした
 がってある人々が正しいことを意志し、働くところのもの」だからである。
 ゆえに意志はある習慣の基体である。

\\



{\scshape Respondeo dicendum} quod omnis potentia quae diversimode potest
 ordinari ad agendum, indiget habitu quo bene disponatur ad suum
 actum. Voluntas autem, cum sit potentia rationalis, diversimode
 potest ad agendum ordinari. 
Et ideo oportet in voluntate aliquem habitum ponere, quo bene
disponatur ad suum actum.

&

解答する。以下のように言われるべきである。
さまざまなしかたで作法することへと秩序づけられうる全ての能力は、それに
 よって自らの作用へとよくあるいは悪く秩序づけられるところの習慣を必要
 とする。
しかるに意志は、理性的能力なので、さまざまなしかたで作用することへと秩
 序づけられうる。
ゆえに、意志の中には、それによってよくあるいは悪く自らの作用へと秩序づ
 けられるところの何らかの習慣が置かれるべきである。

\\


 --- Ex ipsa etiam ratione habitus apparet
quod habet quendam principalem ordinem ad voluntatem, prout habitus
est {\itshape quo quis utitur cum voluerit}, ut supra dictum est.

&

--- さらに、習慣の性格自体からも、意志に対する何らかの主要な秩序
 を持つことが明らかである。それは、前に述べられたとおり、習慣が「人が
 意志したときに用いるもの」だということである。

\\



{\scshape Ad primum ergo dicendum} quod, sicut in intellectu est aliqua species
 quae est similitudo obiecti, ita oportet in voluntate, et in qualibet
 vi appetitiva, esse aliquid quo inclinetur in suum obiectum, cum
 nihil aliud sit actus appetitivae virtutis quam inclinatio quaedam,
 ut supra dictum est. 

&

第一異論に対しては、それゆえ、以下のように言われるべきである。
知性の中に対象の類似である何らかの形象があるように、意志の中にも、そし
 てどんな欲求的力の中にも、それによって自らの対象へと傾けられるところ
 の何かがなければならない。なぜなら、欲求的な力の作用とは、前に述べら
 れたとおり、ある種の傾向性に他ならないからである。

\\


Ad ea ergo ad quae sufficienter inclinatur per
naturam ipsius potentiae, non indiget aliqua qualitate
 inclinante. 


&

ゆえに、能力自体の本性によって十分にそれへと傾かされるようなものに対し
 ては、傾かせるどんな性質も必要としない。


\\

Sed quia necessarium est ad finem humanae vitae, quod vis appetitiva
inclinetur in aliquid determinatum, ad quod non inclinatur ex natura
potentiae, quae se habet ad multa et diversa; ideo necesse est quod in
voluntate, et in aliis viribus appetitivis, sint quaedam qualitates
inclinantes, quae dicuntur habitus.

&

しかし、人間の生の目的には、欲求的な力が、多くの多様なものに関係するそ
 の能力の本性に基づいては傾かされないような、限定された何かへ傾かされること
 が必要である。
ゆえに、意志の中には、そして他の欲求的な力の中には、何らかの傾かせる性
 質が存在することが必要であり、それが習慣と呼ばれる。



\\



{\scshape Ad secundum dicendum} quod intellectus agens est agens tantum, et nullo
 modo patiens. Sed voluntas, et quaelibet vis appetitiva, est movens
 motum, ut dicitur in III {\itshape de Anima}. Et ideo non est similis ratio de
 utroque, nam esse susceptivum habitus convenit ei quod est quodammodo
 in potentia.

&

第二異論に対しては以下のように言われるべきである。
能動知性はただ能動するだけであり、どんなしかたによっても受動しない。
しかし意志は、そしてどんな欲求的力も、『デ・アニマ』第3巻で言われるよ
 うに、動かされて動かすものである。ゆえに、両者に同じことが言えるわけ
 ではない。
すなわち、習慣の受動的な存在が、ある意味で可能態にある習慣に適合する。


\\



{\scshape Ad tertium dicendum} quod voluntas ex ipsa natura potentiae inclinatur
 in bonum rationis. Sed quia hoc bonum multipliciter diversificatur,
 necessarium est ut ad aliquod determinatum bonum rationis voluntas
 per aliquem habitum inclinetur, ad hoc quod sequatur promptior
 operatio.

&

第三異論に対しては以下のように言われるべきである。意志は能力の本性自体
に基づいて理性の善へ傾かされる。しかしこの善が多種多様であるために、よ
り迅速に働きが伴うように、意志がある限定された理性の善へと何らかの習慣
を通して傾かされることが必要である。

\end{longtable}
\newpage



\rhead{a.~6}
\begin{center}
{\Large {\bf ARTICULUS SEXTUS}}\\
{\large UTRUM IN ANGELIS SIT ALIQUIS HABITUS}\\
{\footnotesize III {\itshape Sent.}, d.14, a.1, qu$^{a}$ 2, ad 1.}\\
{\Large 第六項\\天使の中に何らかの習慣があるか}
\end{center}

\begin{longtable}{p{21em}p{21em}}

{\scshape Ad sextum sic proceditur}. Videtur quod in Angelis non sint
habitus. Dicit enim maximus, Commentator Dionysii, in {\scshape vii}
cap.~{\itshape de Cael.~Hier}., {\itshape non convenit arbitrari
virtutes intellectuales}, idest spirituales, {\itshape more
accidentium, quemadmodum et in nobis sunt, in divinis intellectibus},
scilicet Angelis, {\itshape esse, ut aliud in alio sit sicut in
subiecto, accidens enim omne illinc repulsum est}. Sed omnis habitus
est accidens. Ergo in Angelis non sunt habitus.

&

第六項の問題へ、議論は以下のように進められる。
天使の中に習慣はないと思われる。理由は以下の通り。
ディオニュシウスの注解者であるマクシムスは『天上階級論』第7章で以下の
 ように述べている。「知的な(すなわち霊的な)力が、附帯性のしかたで、
 ちょうど私たちにおいてそうであるように、神的な知性(すなわち天使たち)
 において存在し、他のものが基体としての他のものの中にあると考えること
 は適切でない。なぜなら附帯的なものは全てそこから排除されているからで
 ある」。
ゆえに天使たちの中に習慣はない。

\\



2. {\scshape Praeterea}, sicut Dionysius dicit, in {\scshape iv}
cap.~{\itshape de Cael.~Hier}., {\itshape sanctae caelestium essentiarum
dispositiones super omnia alia Dei bonitatem participant}. Sed semper
quod est per se, est prius et potius eo quod est per aliud. Ergo
Angelorum essentiae per seipsas perficiuntur ad conformitatem Dei. Non
ergo per aliquos habitus. --- Et haec videtur esse ratio maximi, qui
ibidem subdit, {\itshape si enim hoc esset, non utique maneret in semetipsa
harum essentia, nec deificari per se, quantum foret possibile,
valuisset}.

&

さらに、『天上階級論』第4章でディオニュシウスが言うように、「天の諸本質
 の聖なる態勢は、他のすべてのものに優って神の善性を分有している」。し
 かるに、自らによってあるものは常に、他のものによってあるものよりも、
 先であり強力である。
ゆえに天使たちの本質はそれ自体によって神との一致へと完成されている。ゆ
 えに、どんな習慣にもよらず、そうである。
--- そしてこれが、同じ箇所で以下のように述べているマクシムスの論であ
 ると思われる。「もしこうだったならば、これらのものの本質が自らに留ま
 ることはなかっただろうし、可能な限りで自らによって神化されることもで
 きなかったであろう」。

\\



3. {\scshape Praeterea}, habitus est dispositio quaedam, ut dicitur in
 V {\itshape Metaphys}. Sed dispositio, ut ibidem dicitur, est
 {\itshape ordo habentis partes}. Cum ergo Angeli sint simplices
 substantiae, videtur quod in eis non sint dispositiones et habitus.

&

さらに、『形而上学』第5巻で言われているように、習慣は一種の態勢である。
しかるに、同じ箇所で言われているように、態勢は「部分を持つものの秩序」
 である。ゆえに、天使は単純実体なので、その中に態勢や習慣はないと思わ
 れる。

\\



{\scshape Sed contra est} quod Dionysius dicit, {\scshape vii}
 cap.~{\itshape Cael.~Hier}., quod Angeli primae hierarchiae {\itshape
 nominantur calefacientes et throni et effusio sapientiae,
 manifestatio deiformis ipsorum habituum}.

&

しかし反対に、ディオニュシウスは『天上階級論』第7章で、第一の階級に属
 する天使たちは「燃やすもの(熾天使)、王位にあるもの(座天使)、知恵
 を注ぐもの(智天使)と名付けられ、神のような彼らの習慣を明示している」と述
 べている。

\\



{\scshape Respondeo dicendum} quod quidam posuerunt in Angelis non
 esse habitus; sed quaecumque dicuntur de eis, essentialiter
 dicuntur. Unde maximus, post praedicta verba quae induximus, dicit,
 {\itshape habitudines earum, atque virtutes quae in eis sunt,
 essentiales sunt, propter immaterialitatem}.

&

解答する。以下のように言われるべきである。
ある人々は天使の中に習慣はなく、何であれ天使たちについて語られるものは
 本質的に語られると考えた。このことからマクシムスは、私たちが引用した
 上述の言葉の後で以下のように述べている。
「彼らの習慣や彼らの中にある徳は、その非質料性のために本質的なものである」。


\\


Et hoc etiam Simplicius dicit, in {\itshape Commento
 Praedicamentorum}, {\itshape sapientia quae est in anima, habitus est, quae
 autem in intellectu, substantia. Omnia enim quae sunt divina, et per
 se sufficientia sunt, et in seipsis existentia}. 


&

そしてこのことは、シンプリキオスもまた『カテゴリー論注解』で述べること
 である。すなわち「魂の中にある知恵は習慣である。しかし知性の中にある
 それは実体である。なぜなら神的なものの中にある全てのものは、自体的に
 充足していて、それ自体において存在するからである」。


\\



Quae quidem positio
 partim habet veritatem, et partim continet falsitatem. Manifestum est
 enim ex praemissis quod subiectum habitus non est nisi ens in
 potentia. 


&

この考えは、実に、部分的に真理をもっているが、部分的に虚偽を含んでいる。
 理由は以下の通り。
先述のことから明らかなとおり、習慣の基体は、可能態においてある有である。

\\


Considerantes igitur praedicti commentatores quod Angeli
 sunt substantiae immateriales, et quod non est in illis potentia
 materiae; secundum hoc, ab eis habitum excluserunt, et omne
 accidens. 

&

ゆえに、上述の注解者たちは、天使たちが非質料的な実体であり、彼らの中に
 質料の可能態がないことを考察して、それにしたがって、天使たちから習慣
 と全ての附帯性を排除した。

\\


Sed quia, licet in Angelis non sit potentia materiae, est
 tamen in eis aliqua potentia (esse enim actum purum est proprium
 Dei); ideo inquantum invenitur in eis de potentia, intantum in eis
 possunt habitus inveniri. 

&

しかし、天使たちの中にたしかに質料の可能態はないが、何らかの可能態はあ
 る(純粋現実態であることは神に固有である)。ゆえに、彼らの中に可能態
 が見出される限り、その限りにおいて彼らの中に習慣が見出されうる。

\\


Sed quia potentia materiae et potentia
 intellectualis substantiae non est unius rationis, ideo per
 consequens nec habitus unius rationis est utrobique. 

&

しかし質料の可能態と知的実体の可能態は同一の性格を持たないので、結果的
 に、両者に同一の性格の習慣は属さない。

\\


Unde Simplicius dicit, in {\itshape Commento Praedicamentorum}, quod
 {\itshape habitus intellectualis substantiae non sunt similes his qui
 sunt hic habitibus; sed magis sunt similes simplicibus et
 immaterialibus speciebus quas continet in seipsa}.


&

このことからシンプリキオスは『カテゴリー論注解』の中で、以下のように述
 べている。「知的実体の習慣は、ここで習慣であるものに似ていない。むし
 ろ自らにおいて充足している単純なものや非質料的な種に似ている」。

\\


Circa huiusmodi tamen habitum aliter se habet intellectus angelicus,
et aliter intellectus humanus. Intellectus enim humanus, cum sit
infimus in ordine intellectuum, est in potentia respectu omnium
intelligibilium, sicut materia prima respectu omnium formarum
sensibilium, et ideo ad omnia intelligenda indiget aliquo habitu.


&

しかしこのような習慣をめぐって、天使の知性と人間の知性は異なった関係に
 ある。すなわち、人間の知性は、知性の秩序の最下位であるから、全ての可
 知的なものに対して可能態にある。それはちょうど第一質料が全ての可感的
 形相に対するのと同じ関係である。ゆえにあらゆるものを知性認識するために、
 何らかの習慣を必要とする。

\\


Sed intellectus angelicus non se habet sicut pura potentia in genere
intelligibilium, sed sicut actus quidam, non autem sicut actus purus
(hoc enim solius Dei est), sed cum permixtione alicuius potentiae, et
tanto minus habet de potentialitate, quanto est superior.

&

しかし天使の知性は可知的なものどもの類の中で純粋に可能態としてあるので
 はなく、ある種の現実態として、純粋現実態としてではないにしても(この
 ことは神だけに属する)、ある可能態との混合を伴ってある。そしてこの可
 能態性が少ないほど、より上位に位置する。

\\


Et ideo, ut in primo dictum est, inquantum est in potentia, indiget
perfici habitualiter per aliquas species intelligibiles ad operationem
propriam, sed inquantum est actu, per essentiam suam potest aliqua
intelligere, ad minus seipsum, et alia secundum modum suae
substantiae, ut dicitur in Lib. {\itshape de Causis}, et tanto perfectius, quanto
est perfectior.


&

それゆえ、第一部で言われたように、可能態にある程度に応じて、固有の働き
へと何らかの可知的形象によって習慣的に完成されることを必要とする。しか
し、『原因論』という書物で言われるように、現実態にある限りにおいて、自
らの本質を通して何かを知性認識しうる。少なくとも自己自身と、自己の実体
のあり方に即して他のものを認識しうる。そしてより完全であるほど完全に認
識する。

\\


Sed quia nullus Angelus pertingit ad perfectionem Dei, sed in
infinitum distat; propter hoc, ad attingendum ad ipsum Deum per
intellectum et voluntatem, indigent aliquibus habitibus, tanquam in
potentia existentes respectu illius puri actus.

&

しかし、どんな天使も神の完全性には到達せず、むしろ無限に神から隔たって
 いる。このため、知性と意志によって神に到達するためには、かの純粋な現
 実態にかんして可能態にあるものとして、何らかの習慣を必要とする。

\\


Unde Dionysius dicit habitus eorum esse {\itshape deiformes}, quibus
 scilicet Deo conformantur. Habitus autem qui sunt dispositiones ad
 esse naturale, non sunt in Angelis, cum sint immateriales.

&

このことからディオニュシウスは彼らの習慣を「神の形をした」と述べている。
すなわちそれによって神に一致するところのものである。しかし自然本性的な
 存在への態勢である習慣は、天使の中にない。それは天使たちが非質料的だからで
 ある。

\\



{\scshape Ad primum ergo dicendum} quod verbum Maximi intelligendum est de
 habitibus et accidentibus materialibus.


&

第一異論に対しては、それゆえ、以下のように言われるべきである。
マクシムスの言葉は習慣と質料的附帯性について理解されるべきである。


\\


{\scshape Ad secundum dicendum} quod quantum ad hoc quod convenit Angelis per
 suam essentiam, non indigent habitu. Sed quia non ita sunt per
 seipsos entes, quin participent sapientiam et bonitatem divinam; ideo
 inquantum indigent participare aliquid ab exteriori, intantum necesse
 est in eis ponere habitus.

&

第二異論に対しては以下のように言われるべきである。
自らの本質を通して天使に適合するところのものにかんしては、天使は習慣を
 必要としない。しかし、自らによって存在し、神の知恵や善性や分有しない
 ようなものではないので、外的なものから何かを分有する必要がある限りに
 おいて、その限りで天使たちの中に習慣を置くことが必要である。

\\



{\scshape Ad tertium dicendum} quod in Angelis non sunt partes essentiae, sed
 sunt partes secundum potentiam, inquantum intellectus eorum per
 plures species perficitur, et voluntas eorum se habet ad plura.

&

第三異論に対しては以下のように言われるべきである。
天使たちの中に本質の部分はないが、彼らの知性が複数の形象によって完成さ
 れ、彼らの意思が複数のものに関係する限りにおいて、能力に即した部分が
 存在する。


\end{longtable}


\end{document}