\documentclass[10pt]{jsarticle} % use larger type; default would be 10pt
%\usepackage[utf8]{inputenc} % set input encoding (not needed with XeLaTeX)
%\usepackage[round,comma,authoryear]{natbib}
%\usepackage{nruby}
\usepackage{okumacro}
\usepackage{longtable}
%\usepqckage{tablefootnote}
\usepackage[polutonikogreek,english,japanese]{babel}
%\usepackage{amsmath}
\usepackage{latexsym}
\usepackage{color}

%----- header -------
\usepackage{fancyhdr}
\lhead{{\it Summa Theologiae} I, q.~XXX}
%--------------------

\bibliographystyle{jplain}

\title{{\bf SECUNDA SECUNDAE}\\{\HUGE Summae Theologiae}\\Sancti Thomae
Aquinatis\\{\sffamily QUEAESTIO 16}\\XXXXX}
\author{Japanese translation\\by Yoshinori {\sc Ueeda}}
\date{Last modified \today}


%%%% コピペ用
%\rhead{a.~}
%\begin{center}
% {\Large {\bf }}\\
% {\large }\\
% {\footnotesize }\\
% {\Large \\}
%\end{center}
%
%\begin{longtable}{p{21em}p{21em}}
%
%&
%
%
%\\
%\end{longtable}
%\newpage



\begin{document}
\maketitle
\pagestyle{fancy}

\begin{center}
{\Large 第 問\\}
\end{center}

\begin{longtable}{p{21em}p{21em}}


&



\\

&


\end{longtable}

\newpage


\begin{longtable}{p{21em}p{21em}}
{\huge A}{\scshape d primum sic proceditur}. Videtur
quod in veteri lege dari debuerint praecepta credendi. Praeceptum enim
est de eo quod est debitum et necessarium. Sed maxime necessarium est
homini quod credat, secundum illud Heb. XI, sine fide impossibile est
placere Deo. Ergo maxime oportuit praecepta dari de fide.

\\

第一項の問題へ、議論は以下のように進められる。

&



2 Praeterea, novum testamentum
continetur in veteri sicut figuratum in figura, ut supra dictum est. Sed
in novo testamento ponuntur expressa mandata de fide, ut patet
Ioan. XIV, creditis in Deum, et in me credite. Ergo videtur quod in
veteri lege etiam debuerint aliqua praecepta dari de fide.


\\

&



3 Praeterea, eiusdem rationis est
praecipere actum virtutis et prohibere vitia opposita. Sed in veteri
lege ponuntur multa praecepta prohibentia infidelitatem, sicut Exod. XX,
non habebis deos alienos coram me; et iterum Deut. XIII mandatur quod
non audient verba prophetae aut somniatoris qui eos de fide Dei vellet
divertere. Ergo in veteri lege etiam debuerunt dari praecepta de fide.


\\

&



4 Praeterea, confessio est actus fidei,
ut supra dictum est. Sed de confessione et promulgatione fidei dantur
praecepta in veteri lege, mandatur enim Exod. XII quod filiis suis
interrogantibus rationem assignent paschalis observantiae; et Deut. XIII
mandatur quod ille qui disseminat doctrinam contra fidem occidatur. Ergo
lex vetus praecepta fidei debuit habere.


\\

&



5 Praeterea, omnes libri veteris
testamenti sub lege veteri continentur, unde dominus, Ioan. XV, dicit in
lege esse scriptum, odio habuerunt me gratis, quod tamen scribitur in
Psalmo. Sed Eccli. II dicitur, qui timetis dominum, credite illi. Ergo
in veteri lege fuerunt praecepta danda de fide.


\\

&



Sed contra est quod apostolus, ad
Rom. III, legem veterem nominat legem factorum, et dividit eam contra
legem fidei. Ergo in lege veteri non fuerunt praecepta danda de fide.


\\

&



Respondeo dicendum quod lex non
imponitur ab aliquo domino nisi suis subditis, et ideo praecepta legis
cuiuslibet praesupponunt subiectionem recipientis legem ad eum qui dat
legem. Prima autem subiectio hominis ad Deum est per fidem, secundum
illud Heb. XI, accedentem ad Deum oportet credere quia est. Et ideo
fides praesupponitur ad legis praecepta. Et propter hoc Exod. XX id quod
est fidei praemittitur ante legis praecepta, cum dicitur, ego sum
dominus Deus tuus, qui eduxi te de terra Aegypti. Et similiter Deut. VI
praemittitur, audi, Israel, dominus Deus tuus unus est, et postea statim
incipit agere de praeceptis. Sed quia in fide multa continentur ordinata
ad fidem qua credimus Deum esse, quod est primum et principale inter
omnia credibilia, ut dictum est; ideo, praesupposita fide de Deo, per
quam mens humana Deo subiiciatur, possunt dari praecepta de aliis
credendis, sicut Augustinus dicit, super Ioan., quod plurima sunt nobis
de fide mandata, exponens illud, hoc est praeceptum meum. Sed in veteri
lege non erant secreta fidei populo exponenda. Et ideo, supposita fide
unius Dei, nulla alia praecepta sunt in veteri lege data de credendis.


\\

&



Ad primum ergo dicendum quod fides est
necessaria tanquam principium spiritualis vitae. Et ideo praesupponitur
ad legis susceptionem.


\\

&



Ad secundum dicendum quod ibi etiam
dominus praesupponit aliquid de fide, scilicet fidem unius Dei, cum
dicit, creditis in Deum, et aliquid praecipit, scilicet fidem
incarnationis, per quam unus est Deus et homo; quae quidem fidei
explicatio pertinet ad fidem novi testamenti. Et ideo subdit, et in me
credite.


\\

&



Ad tertium dicendum quod praecepta
prohibitiva respiciunt peccata, quae corrumpunt virtutem. Virtus autem
corrumpitur ex particularibus defectibus, ut supra dictum est. Et ideo,
praesupposita fide unius Dei, in lege veteri fuerunt danda prohibitiva
praecepta, quibus homines prohiberentur ab his particularibus defectibus
per quos fides corrumpi posset.


\\

&



Ad quartum dicendum quod etiam
confessio vel doctrina fidei praesupponit subiectionem hominis ad Deum
per fidem. Et ideo magis potuerunt dari praecepta in veteri lege
pertinentia ad confessionem et doctrinam fidei quam pertinentia ad ipsam
fidem.


\\

&



Ad quintum dicendum quod in illa etiam
auctoritate praesupponitur fides per quam credimus Deum esse, unde
praemittit, qui timetis Deum, quod non posset esse sine fide. Quod autem
addit, credite illi, ad quaedam credibilia specialia referendum est, et
praecipue ad illa quae promittit Deus sibi obedientibus. Unde subdit, et
non evacuabitur merces vestra.


\\

&


\end{document}
