\documentclass[10pt]{jsarticle} % use larger type; default would be 10pt
%\usepackage[utf8]{inputenc} % set input encoding (not needed with XeLaTeX)
%\usepackage[round,comma,authoryear]{natbib}
%\usepackage{nruby}
\usepackage{okumacro}
\usepackage{longtable}
%\usepqckage{tablefootnote}
\usepackage[polutonikogreek,english,japanese]{babel}
%\usepackage{amsmath}
\usepackage{latexsym}
\usepackage{color}

%----- header -------
\usepackage{fancyhdr}
\pagestyle{fancy}
\lhead{{\it Summa Theologiae} II-IIae, q.~57}
%--------------------

\bibliographystyle{jplain}

\title{{\bf SECUNDA SECUNDAE}\\{\HUGE Summae Theologiae}\\Sancti Thomae
Aquinatis\\{\sffamily QUEAESTIO QUINQUAGESIMASEPTIMA}\\DE IURE}
\author{Japanese translation\\by Yoshinori {\sc Ueeda}}
\date{Last modified \today}


%%%% コピペ用
%\rhead{a.~}
%\begin{center}
% {\Large {\bf }}\\
% {\large }\\
% {\footnotesize }\\
% {\Large \\}
%\end{center}
%
%\begin{longtable}{p{21em}p{21em}}
%
%&
%
%\\
%\end{longtable}
%\newpage



\begin{document}
\maketitle

\begin{center}
{\Large 第五十七問\\法について}
\end{center}

\begin{longtable}{p{21em}p{21em}}

{\huge C}onsequenter post prudentiam considerandum est de iustitia. Circa quam
quadruplex consideratio occurrit, prima est de iustitia; secunda, de
partibus eius; tertia, de dono ad hoc pertinente; quarta, de praeceptis
ad iustitiam pertinentibus. Circa iustitiam vero consideranda sunt
quatuor, primo quidem, de iure; secundo, de ipsa iustitia; tertio, de
iniustitia; quarto, de iudicio. Circa primum quaeruntur quatuor. 

\begin{enumerate}
 \item  utrum ius sit obiectum iustitiae.
 \item utrum ius convenienter dividatur in ius naturale et positivum.
 \item utrum ius gentium sit ius naturale.
 \item utrum ius dominativum et paternum debeat specialiter distingui.

\end{enumerate}


&

続いて、思慮のあとに、正義が考察されるべきである。
この考察を巡って、四通りの考察が生じる。第一に、正義について、第二に、正義の部分について、第三に、それ?に属する賜物について、第四に、正義に属する原則について。
正義について、四つのことが考察されるべきである。第一に、法について、第二に、正義それ自体について、第三に、不正について、第四に、判断について。
第一にかんして、四つのことが問われる。
\begin{enumerate}
 \item 法は正義の対象か
 \item 法は、自然法と実定法に適切に分割されるか
 \item 万民法は自然法か
 \item 支配権と父権は種的に区別されるべきか
\end{enumerate}


\end{longtable}
\newpage

\rhead{a.~1}

\begin{center}
{\Large {\bfseries ARTICULUS PRIMUS}}\\
{\large UTRUM IUS SIT OBIECTUM IUSTITIAE}\\
{\Large 第一項\\法は正義の対象か}
\end{center}

\begin{longtable}{p{21em}p{21em}}

{\huge A}{\sc d primum sic proceditur}. 
Videtur quod ius non sit obiectum iustitiae. Dicit enim Celsus
 iurisconsultus quod {\itshape ius est ars boni et aequi}. Ars autem non est
 obiectum iustitiae, sed est per se virtus intellectualis. Ergo ius non
 est obiectum iustitiae.

&

第一に対しては次のように進められる。
法は正義の対象でないと思われる。法律家のケルススは「法は善と平等の技術で
 ある」と述べている。ところで、技術は、正義の対象でなく、それ自体によっ
 て知的徳である。ゆえに、法は正義の対象でない。


\\


{\scshape 2 Praeterea}, {\itshape lex}, sicut Isidorus dicit, in libro
 {\itshape Etymol}., {\itshape iuris est species}. Lex autem non est
 obiectum iustitiae, sed magis prudentiae, unde et philosophus
 legispositivam partem prudentiae ponit. Ergo ius non est obiectum
 iustitiae.


&

イシドルスが『語源』で述べているように、法律は法の種である。ところが、法
 律は正義の対象でなく、むしろ、思慮の対象である。したがって、哲学者は、
 法律を作る部分を思慮に属するものとしている。ゆえに、法は正義の対象でな
 い。

\\



{\scshape 3 Praeterea}, iustitia principaliter subiicit hominem Deo,
 dicit enim Augustinus, libro {\itshape de moribus Eccles}., quod
 {\itshape iustitia est amor Deo tantum serviens, et ob hoc bene
 imperans ceteris, quae homini subiecta sunt}. Sed ius non pertinet ad
 divina, sed solum ad humana, dicit enim Isidorus, in libro {\itshape
 Etymol}., quod {\itshape fas lex divina est, ius autem lex
 humana}. Ergo ius non est obiectum iustitiae.

&

さらに、正義は、主に、人間を神に服従させる。たとえばアウグスティヌスは、
 『教会の習俗moresについて』という書物で「正義は、神だけに仕える愛であり、
 このことから、人間に従属する他のものどもによく命令する」と述べる。とこ
 ろが、法は神にかんすることには属さず、人間にかんすることのみに属する。
 なぜなら、イシドルスは『語源』で「fasは神の法律、iusは人間の法律」と述
 べているからである。ゆえに、法は正義の対象でない。

\\



{\scshape Sed contra est} quod Isidorus dicit, in eodem, quod {\itshape
 ius dictum est quia est iustum}. Sed iustum est obiectum iustitiae,
 dicit enim philosophus, in V {\itshape Ethic}., quod {\itshape omnes
 talem habitum volunt dicere iustitiam a quo operativi iustorum
 sunt}. Ergo ius est obiectum iustitiae.


&
しかし反対に、イシドルスは同書で、「正しいiustumから、法iusと言われた」
 と述べている。ところで、正しいことは正義の対象である。なぜなら、哲学者
 が『ニコマコス倫理学』5巻で、「すべての人は、それによって正しいことをな
 しうる習態を、正義と呼びたいと意志する」と述べているからである。ゆえに、
 法は正義の対象である。

\\



{\scshape Respondeo dicendum} quod iustitiae proprium est inter alias
 virtutes ut ordinet hominem in his quae sunt ad alterum. Importat enim
 aequalitatem quandam, ut ipsum nomen demonstrat, dicuntur enim
 vulgariter ea quae adaequantur iustari. Aequalitas autem ad alterum
 est. 

&

答えて言わなければならない。他の諸徳の中で、正義に固有なのは、人間を、他
 のものに対してあるものどもにおいて秩序づけることである。
じっさい、その名自身が示しているとおり、それは一種の等しさを意味する。な
 ぜなら、等しくされることを、俗に、「iusにされる」(iustari)と言われるからで
 ある。そして等しさとは、他に対してのことである。

\\

Aliae autem virtutes perficiunt hominem solum in his quae ei
 conveniunt secundum seipsum. Sic igitur illud quod est rectum in
 operibus aliarum virtutum, ad quod tendit intentio virtutis quasi in
 proprium obiectum, non accipitur nisi per comparationem ad
 agentem. 

&

他方、他の諸徳は、人間を、その人自身に即してその人に適合する点でのみ、完成する。
したがって、他の諸徳の業においては、徳の意図が固有の目的として向かう「まっすぐであること」は、作用者への関係によってのみ理解される。


\\


Rectum vero quod est in opere iustitiae, etiam praeter
 comparationem ad agentem, constituitur per comparationem ad alium,
 illud enim in opere nostro dicitur esse iustum quod respondet secundum
 aliquam aequalitatem alteri, puta recompensatio mercedis debitae pro
 servitio impenso. 

&

これに対して、正義の業において「まっすぐであること」は、作用者への関係以外にも、
 他者への関係によっても構成される。たとえば、なされた労働に対してしかるべき賃金を支払うことなど、私たちの業のなかで正しいと言われるのは、何らかの他者への等しさにに即した対応だからである。



\\



Sic igitur iustum dicitur aliquid, quasi habens
 rectitudinem iustitiae, ad quod terminatur actio iustitiae, etiam non
 considerato qualiter ab agente fiat. Sed in aliis virtutibus non
 determinatur aliquid rectum nisi secundum quod aliqualiter fit ab
 agente. 



&


\\



Et propter hoc specialiter iustitiae prae aliis virtutibus
 determinatur secundum se obiectum, quod vocatur iustum. Et hoc quidem
 est ius. Unde manifestum est quod ius est obiectum iustitiae.


&


\\



[41389] IIª-IIae q. 57 a. 1 ad 1 Ad primum ergo dicendum quod consuetum est quod nomina a sui prima impositione detorqueantur ad alia significanda, sicut nomen medicinae impositum est primo ad significandum remedium quod praestatur infirmo ad sanandum, deinde tractum est ad significandum artem qua hoc fit. Ita etiam hoc nomen ius primo impositum est ad significandum ipsam rem iustam; postmodum autem derivatum est ad artem qua cognoscitur quid sit iustum; et ulterius ad significandum locum in quo ius redditur, sicut dicitur aliquis comparere in iure; et ulterius dicitur etiam ius quod redditur ab eo ad cuius officium pertinet iustitiam facere, licet etiam id quod decernit sit iniquum.

&


\\



[41390] IIª-IIae q. 57 a. 1 ad 2 Ad secundum dicendum quod sicut eorum quae per artem exterius fiunt quaedam ratio in mente artificis praeexistit, quae dicitur regula artis; ita etiam illius operis iusti quod ratio determinat quaedam ratio praeexistit in mente, quasi quaedam prudentiae regula. Et hoc si in scriptum redigatur, vocatur lex, est enim lex, secundum Isidorum, constitutio scripta. Et ideo lex non est ipsum ius, proprie loquendo, sed aliqualis ratio iuris.

&


\\



[41391] IIª-IIae q. 57 a. 1 ad 3 Ad tertium dicendum quod quia iustitia aequalitatem importat, Deo autem non possumus aequivalens recompensare, inde est quod iustum, secundum perfectam rationem, non possumus reddere Deo. Et propter hoc non dicitur proprie ius lex divina, sed fas, quia videlicet sufficit Deo ut impleamus quod possumus. Iustitia tamen ad hoc tendit ut homo, quantum potest, Deo recompenset, totaliter animam ei subiiciens.

&



\end{longtable}
\newpage
\end{document}

%\rhead{a.~}
%\begin{center}
% {\Large {\bf }}\\
% {\large }\\
% {\footnotesize }\\
% {\Large \\}
%\end{center}
%
%\begin{longtable}{p{21em}p{21em}}
%
%&
%
%\\
%\end{longtable}
%\newpage


Articulus 2

[41392] IIª-IIae q. 57 a. 2 arg. 1 Ad secundum sic proceditur. Videtur quod ius non convenienter dividatur in ius naturale et ius positivum. Illud enim quod est naturale est immutabile, et idem apud omnes. Non autem invenitur in rebus humanis aliquid tale, quia omnes regulae iuris humani in aliquibus casibus deficiunt, nec habent suam virtutem ubique. Ergo non est aliquod ius naturale.

[41393] IIª-IIae q. 57 a. 2 arg. 2 Praeterea, illud dicitur esse positivum quod ex voluntate humana procedit. Sed non ideo aliquid est iustum quia a voluntate humana procedit, alioquin voluntas hominis iniusta esse non posset. Ergo, cum iustum sit idem quod ius, videtur quod nullum sit ius positivum.

[41394] IIª-IIae q. 57 a. 2 arg. 3 Praeterea, ius divinum non est ius naturale, cum excedat naturam humanam. Similiter etiam non est ius positivum, quia non innititur auctoritati humanae, sed auctoritati divinae. Ergo inconvenienter dividitur ius per naturale et positivum.

[41395] IIª-IIae q. 57 a. 2 s. c. Sed contra est quod philosophus dicit, in V Ethic., quod politici iusti hoc quidem naturale est, hoc autem legale, idest lege positum.

[41396] IIª-IIae q. 57 a. 2 co. Respondeo dicendum quod, sicut dictum est, ius, sive iustum, est aliquod opus adaequatum alteri secundum aliquem aequalitatis modum. Dupliciter autem potest alicui homini aliquid esse adaequatum. Uno quidem modo, ex ipsa natura rei, puta cum aliquis tantum dat ut tantundem recipiat. Et hoc vocatur ius naturale. Alio modo aliquid est adaequatum vel commensuratum alteri ex condicto, sive ex communi placito, quando scilicet aliquis reputat se contentum si tantum accipiat. Quod quidem potest fieri dupliciter. Uno modo, per aliquod privatum condictum, sicut quod firmatur aliquo pacto inter privatas personas. Alio modo, ex condicto publico, puta cum totus populus consentit quod aliquid habeatur quasi adaequatum et commensuratum alteri; vel cum hoc ordinat princeps, qui curam populi habet et eius personam gerit. Et hoc dicitur ius positivum.

[41397] IIª-IIae q. 57 a. 2 ad 1 Ad primum ergo dicendum quod illud quod est naturale habenti naturam immutabilem, oportet quod sit semper et ubique tale. Natura autem hominis est mutabilis. Et ideo id quod naturale est homini potest aliquando deficere. Sicut naturalem aequalitatem habet ut deponenti depositum reddatur, et si ita esset quod natura humana semper esset recta, hoc esset semper servandum. Sed quia quandoque contingit quod voluntas hominis depravatur, est aliquis casus in quo depositum non est reddendum, ne homo perversam voluntatem habens male eo utatur, ut puta si furiosus vel hostis reipublicae arma deposita reposcat.

[41398] IIª-IIae q. 57 a. 2 ad 2 Ad secundum dicendum quod voluntas humana ex communi condicto potest aliquid facere iustum in his quae secundum se non habent aliquam repugnantiam ad naturalem iustitiam. Et in his habet locum ius positivum. Unde philosophus dicit, in V Ethic., quod legale iustum est quod ex principio quidem nihil differt sic vel aliter, quando autem ponitur, differt. Sed si aliquid de se repugnantiam habeat ad ius naturale, non potest voluntate humana fieri iustum, puta si statuatur quod liceat furari vel adulterium committere. Unde dicitur Isaiae X, vae qui condunt leges iniquas.

[41399] IIª-IIae q. 57 a. 2 ad 3 Ad tertium dicendum quod ius divinum dicitur quod divinitus promulgatur. Et hoc quidem partim est de his quae sunt naturaliter iusta, sed tamen eorum iustitia homines latet, partim autem est de his quae fiunt iusta institutione divina. Unde etiam ius divinum per haec duo distingui potest, sicut et ius humanum. Sunt enim in lege divina quaedam praecepta quia bona, et prohibita quia mala, quaedam vero bona quia praecepta, et mala quia prohibita.

Articulus 3

[41400] IIª-IIae q. 57 a. 3 arg. 1 Ad tertium sic proceditur. Videtur quod ius gentium sit idem cum iure naturali. Non enim omnes homines conveniunt nisi in eo quod est eis naturale. Sed in iure gentium omnes homines conveniunt, dicit enim iurisconsultus quod ius gentium est quo gentes humanae utuntur. Ergo ius gentium est ius naturale.

[41401] IIª-IIae q. 57 a. 3 arg. 2 Praeterea, servitus inter homines est naturalis, quidam enim sunt naturaliter servi, ut philosophus probat, in I Polit. Sed servitutes pertinent ad ius gentium, ut Isidorus dicit. Ergo ius gentium est ius naturale.

[41402] IIª-IIae q. 57 a. 3 arg. 3 Praeterea, ius, ut dictum est, dividitur per ius naturale et positivum. Sed ius gentium non est ius positivum, non enim omnes gentes unquam convenerunt ut ex communi condicto aliquid statuerent. Ergo ius gentium est ius naturale.

[41403] IIª-IIae q. 57 a. 3 s. c. Sed contra est quod Isidorus dicit, quod ius aut naturale est, aut civile, aut gentium. Et ita ius gentium distinguitur a iure naturali.

[41404] IIª-IIae q. 57 a. 3 co. Respondeo dicendum quod, sicut dictum est, ius sive iustum naturale est quod ex sui natura est adaequatum vel commensuratum alteri. Hoc autem potest contingere dupliciter. Uno modo, secundum absolutam sui considerationem, sicut masculus ex sui ratione habet commensurationem ad feminam ut ex ea generet, et parens ad filium ut eum nutriat. Alio modo aliquid est naturaliter alteri commensuratum non secundum absolutam sui rationem, sed secundum aliquid quod ex ipso consequitur, puta proprietas possessionum. Si enim consideretur iste ager absolute, non habet unde magis sit huius quam illius, sed si consideretur quantum ad opportunitatem colendi et ad pacificum usum agri, secundum hoc habet quandam commensurationem ad hoc quod sit unius et non alterius, ut patet per philosophum, in II Polit. Absolute autem apprehendere aliquid non solum convenit homini, sed etiam aliis animalibus. Et ideo ius quod dicitur naturale secundum primum modum, commune est nobis et aliis animalibus. A iure autem naturali sic dicto recedit ius gentium, ut iurisconsultus dicit, quia illud omnibus animalibus, hoc solum hominibus inter se commune est. Considerare autem aliquid comparando ad id quod ex ipso sequitur, est proprium rationis. Et ideo hoc quidem est naturale homini secundum rationem naturalem, quae hoc dictat. Et ideo dicit Gaius iurisconsultus, quod naturalis ratio inter omnes homines constituit, id apud omnes gentes custoditur, vocaturque ius gentium.

[41405] IIª-IIae q. 57 a. 3 ad 1 Et per hoc patet responsio ad primum.

[41406] IIª-IIae q. 57 a. 3 ad 2 Ad secundum dicendum quod hunc hominem esse servum, absolute considerando, magis quam alium, non habet rationem naturalem, sed solum secundum aliquam utilitatem consequentem, inquantum utile est huic quod regatur a sapientiori, et illi quod ab hoc iuvetur, ut dicitur in I Polit. Et ideo servitus pertinens ad ius gentium est naturalis secundo modo, sed non primo.

[41407] IIª-IIae q. 57 a. 3 ad 3 Ad tertium dicendum quod quia ea quae sunt iuris gentium naturalis ratio dictat, puta ex propinquo habentia aequitatem; inde est quod non indigent aliqua speciali institutione, sed ipsa naturalis ratio ea instituit, ut dictum est in auctoritate inducta.

Articulus 4

[41408] IIª-IIae q. 57 a. 4 arg. 1 Ad quartum sic proceditur. Videtur quod non debeat specialiter distingui ius paternum et dominativum. Ad iustitiam enim pertinet reddere unicuique quod suum est; ut dicit Ambrosius, in I de officiis. Sed ius est obiectum iustitiae, sicut dictum est. Ergo ius ad unumquemque aequaliter pertinet. Et sic non debet distingui specialiter ius patris et domini.

[41409] IIª-IIae q. 57 a. 4 arg. 2 Praeterea, ratio iusti est lex, ut dictum est. Sed lex respicit commune bonum civitatis et regni, ut supra habitum est, non autem respicit bonum privatum unius personae, aut etiam unius familiae. Non ergo debet esse aliquod speciale ius vel iustum dominativum vel paternum, cum dominus et pater pertineant ad domum, ut dicitur in I Polit.

[41410] IIª-IIae q. 57 a. 4 arg. 3 Praeterea, multae aliae sunt differentiae graduum in hominibus, ut puta quod quidam sunt milites, quidam sacerdotes, quidam principes. Ergo ad eos debet aliquod speciale iustum determinari.

[41411] IIª-IIae q. 57 a. 4 s. c. Sed contra est quod philosophus, in V Ethic., specialiter a iusto politico distinguit dominativum et paternum, et alia huiusmodi.

[41412] IIª-IIae q. 57 a. 4 co. Respondeo dicendum quod ius, sive iustum dicitur per commensurationem ad alterum. Alterum autem potest dici dupliciter. Uno modo, quod simpliciter est alterum, sicut quod est omnino distinctum, sicut apparet in duobus hominibus quorum unus non est sub altero, sed ambo sunt sub uno principe civitatis. Et inter tales, secundum philosophum, in V Ethic., est simpliciter iustum. Alio modo dicitur aliquid alterum non simpliciter, sed quasi aliquid eius existens. Et hoc modo in rebus humanis filius est aliquid patris, quia quodammodo est pars eius, ut dicitur in VIII Ethic.; et servus est aliquid domini, quia est instrumentum eius, ut dicitur in I Polit. Et ideo patris ad filium non est comparatio sicut ad simpliciter alterum, et propter hoc non est ibi simpliciter iustum, sed quoddam iustum, scilicet paternum. Et eadem ratione nec inter dominum et servum, sed est inter eos dominativum iustum. Uxor autem, quamvis sit aliquid viri, quia comparatur ad eam sicut ad proprium corpus, ut patet per apostolum, ad Ephes. V; tamen magis distinguitur a viro quam filius a patre vel servus a domino, assumitur enim in quandam socialem vitam matrimonii. Et ideo, ut philosophus dicit, inter virum et uxorem plus est de ratione iusti quam inter patrem et filium, vel dominum et servum. Quia tamen vir et uxor habent immediatam relationem ad domesticam communitatem, ut patet in I Polit.; ideo inter eos non est etiam simpliciter politicum iustum, sed magis iustum oeconomicum.

[41413] IIª-IIae q. 57 a. 4 ad 1 Ad primum ergo dicendum quod ad iustitiam pertinet reddere ius suum unicuique, supposita tamen diversitate unius ad alterum, si quis enim sibi det quod sibi debetur, non proprie vocatur hoc iustum. Et quia quod est filii est patris, et quod est servi est domini, ideo non est proprie iustitia patris ad filium, vel domini ad servum.

[41414] IIª-IIae q. 57 a. 4 ad 2 Ad secundum dicendum quod filius, inquantum filius, est aliquid patris; et similiter servus, inquantum servus, est aliquid domini. Uterque tamen prout consideratur ut quidam homo, est aliquid secundum se subsistens ab aliis distinctum. Et ideo inquantum uterque est homo, aliquo modo ad eos est iustitia. Et propter hoc etiam aliquae leges dantur de his quae sunt patris ad filium, vel domini ad servum. Sed inquantum uterque est aliquid alterius, secundum hoc deficit ibi perfecta ratio iusti vel iuris.

[41415] IIª-IIae q. 57 a. 4 ad 3 Ad tertium dicendum quod omnes aliae diversitates personarum quae sunt in civitate, habent immediatam relationem ad communitatem civitatis et ad principem ipsius. Et ideo ad eos est iustum secundum perfectam rationem iustitiae. Distinguitur tamen istud iustum secundum diversa officia. Unde et dicitur ius militare vel ius magistratuum aut sacerdotum, non propter defectum a simpliciter iusto, sicut dicitur ius paternum et dominativum, sed propter hoc quod unicuique conditioni personae secundum proprium officium aliquid proprium debetur.
