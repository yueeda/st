\documentclass[10pt]{jsarticle} % use larger type; default would be 10pt
%\usepackage[utf8]{inputenc} % set input encoding (not needed with XeLaTeX)
%\usepackage[round,comma,authoryear]{natbib}
%\usepackage{nruby}
\usepackage{okumacro}
\usepackage{longtable}
%\usepqckage{tablefootnote}
\usepackage[polutonikogreek,english,japanese]{babel}
%\usepackage{amsmath}
\usepackage{latexsym}
\usepackage{color}
%\usepackage{tikz}

%----- header -------
\usepackage{fancyhdr}
\pagestyle{fancy}
\lhead{{\it Summa Theologiae} I, q.~29}
%--------------------


\title{{\bf PRIMA PARS}\\{\HUGE Summae Theologiae}\\Sancti Thomae
Aquinatis\\{\sffamily QUEAESTIO VIGESIMANONA}\\DE PERSONIS DIVINIS}
\author{Japanese translation\\by Yoshinori {\sc Ueeda}}
\date{Last modified \today}

%%%% コピペ用
%\rhead{a.~}
%\begin{center}
% {\Large {\bf }}\\
% {\large }\\
% {\footnotesize }\\
% {\Large \\}
%\end{center}
%
%\begin{longtable}{p{21em}p{21em}}
%
%&
%
%\\
%\end{longtable}
%\newpage



\begin{document}

\maketitle

\begin{center}
{\Large 第二十九問\\神のペルソナについて}
\end{center}




\newpage
\begin{longtable}{p{21em}p{21em}}

{\Huge P}{\scshape raemissis} autem his quae de processionibus et relationibus
 praecognoscenda videbantur, necessarium est aggredi de personis. Et
 primo, secundum considerationem absolutam; et deinde secundum
 comparativam considerationem. Oportet autem absolute de personis, primo
 quidem in communi considerare; deinde de singulis personis. Ad communem
 autem considerationem personarum quatuor pertinere videntur, primo
 quidem, significatio huius nominis persona; secundo vero, numerus
 personarum; tertio, ea quae consequuntur numerum personarum, vel ei
 opponuntur, ut diversitas et solitudo, et huiusmodi; quarto vero, ea
 quae pertinent ad notitiam personarum. 

&

発出と関係について予め考察されるべきと思われたことが論じられたので、次に
ペルソナにかんする諸問題を扱う必要がある\footnote{STI, q.27, Introd.}。第一に非関係的な考察に即して、次に比較
 的な考察に即して論じられる\footnote{STI, q.39.}。ペルソナについて非関係的には、第一に共
 通的に、第二に個々のペルソナについて論じられる\footnote{STI, q.33.}。ペルソナの共通的な考
 察には四つのことが属すると思われる。第一に「ペルソナ」というこの名
 称の意味、第二にペルソナの数\footnote{STI, q.30.}、第三に差異や類似のような
 ペルソナの数に伴うものやそれに対置されるもの\footnote{STI, q.31.}、第四にペルソナ
 の指標に属するものである\footnote{STI, q.32.}。



\\


Circa primum quaeruntur
 quatuor. 


\begin{enumerate}
 \item de definitione personae.
 \item de comparatione personae ad essentiam, subsistentiam et hypostasim.
 \item utrum nomen personae competat in divinis.
 \item quid ibi significet.
\end{enumerate}


&



第一の問題を巡って四つのことが問われる。

\begin{enumerate}
 \item ペルソナの定義について。
 \item 本質、自存性、ヒュポスタシスと、ペルソナの比較について。
 \item ペルソナという名称は神に適合するか。
 \item 神においてペルソナは何を表示するか。
\end{enumerate}


\end{longtable}



\newpage




\rhead{a.~1}
\begin{center}
 {\Large {\bf ARTICULUS PRIMUS}}\\
 {\large DE DEFINITIONE PERSONAE}\\
 {\footnotesize Infra, a.3, ad 2, 4; III, q.2, a.2; I {\itshape Sent.},
 d.25, a.1; {\itshape De Pot.}, q.9, a.2; {\itshape De Unione Verbi}, a.1.}\\
 {\Large 第一項\\ペルソナの定義について}
\end{center}

\begin{longtable}{p{21em}p{21em}}


{\Huge A}{\scshape d primum sic proceditur}. Videtur quod incompetens sit definitio
personae quam Boetius assignat in libro {\itshape de duabus naturis}, quae talis
est, {\itshape persona est rationalis naturae individua substantia}. Nullum enim
singulare definitur. Sed persona significat quoddam singulare. Ergo
persona inconvenienter definitur.


&

第一項の問題へ議論は以下のように進められる。ボエティウスが『二つの本性
 について』という書物で与えた「ペルソナとは、理性的本性の個体的(individuus)実体である」
 という定義は不十分だと思われる。理由は以下の
 通り。いかなる個(singulare)も定義されない。しかしペルソナはある種の
 個である。ゆえにこのペルソナの定義は不適切である。

\\



2. {\scshape Praeterea}, substantia, prout ponitur in definitione personae, aut
sumitur pro substantia prima, aut pro substantia secunda. Si pro
substantia prima, superflue additur individua, quia substantia prima est
substantia individua. Si vero stat pro substantia secunda, falso
additur, et est oppositio in adiecto, nam secundae substantiae dicuntur
genera vel species. Ergo definitio est male assignata.


&

さらに実体はペルソナの定義の中に置かれるものとしては、第一実体と理解
 されるか第二実体と理解されるかのどちらかである。もし第一実体と理解さ
 れるならば「個体的」は不必要に加えられている。なぜなら第一実体とは
 個体的実体のことだから。しかし他方で第二実体と理解されるならば「個
 体的」が付加されるのは誤りであり、また修飾における対立である。すなわ
 ち第二実体と言われるのは類や種だからである。ゆえにこの定義はうまく
 定められていない。


\\



3. {\scshape Praeterea}, nomen intentionis non debet poni in definitione rei. Non enim
esset bona assignatio, si quis diceret, {\itshape homo est species animalis}, {\itshape homo}
enim est nomen rei, et {\itshape species} est nomen intentionis. Cum igitur {\itshape persona}
sit nomen rei (significat enim substantiam quandam rationalis naturae),
inconvenienter {\itshape individuum}, quod est nomen intentionis, in eius
definitione ponitur.


&

さらに概念の名を事物の定義の中に入れるべきではない。たとえば「人間は
 動物の種である」と誰かが言うとしたら、それはよい設定ではないだろう。な
 ぜなら「人間」は事物の名であり、「種」は概念の名であるから。
ゆえに「ペルソナ」は事物の名なので(というのも、理性的本性の何らかの実
 体を意味するから)、「個体」という概念の名をその中に入れるのは適切でな
 い。



\\



4. {\scshape Praeterea}, natura est {\itshape principium motus et quietis in eo in quo est per
se et non per accidens}, ut dicitur in II {\itshape Physic}. Sed persona est in
rebus immobilibus, sicut in Deo et in Angelis. Non ergo in definitione
personae debuit poni {\itshape natura}, sed magis {\itshape essentia}.


&

さらに本性は『自然学』第2巻で言われているように「自体的にそうあり、
 附帯的にでないものにおける、運動と静止の根源」である。しかしペルソナ
 は、神や天使のように不動の事物においてある。ゆえにペルソナの定義に
 「本性」を置くべきでなく、むしろ「本質」とすべきであった。


\\



5. {\scshape Praeterea}, anima separata est rationalis naturae individua
substantia. Non autem est persona. Inconvenienter ergo persona sic
definitur.


&

さらに分離した魂は理性的本性の個体的実体である。しかしそれはペルソ
 ナではない。ゆえにペルソナがそのように定義されるのは適切でない。



\\

(no {\scshape Sed ontra.})

&

(反対異論なし。\footnote{反対異論がないのはきわめて珍しい。})

\\


{\scshape Respondeo dicendum} quod, licet universale et particulare inveniantur in
omnibus generibus, tamen speciali quodam modo individuum invenitur in
genere substantiae. 


&

解答する。以下のように言われるべきである。
普遍と個別は全ての類の中に見出されるが、実体の類の中には個体が特別な
 しかたで見出される。


\\


Substantia enim individuatur per seipsam, sed
accidentia individuantur per subiectum, quod est substantia, dicitur
enim haec albedo, inquantum est in hoc subiecto. Unde etiam convenienter
individua substantiae habent aliquod speciale nomen prae aliis, dicuntur
enim hypostases, vel primae substantiae. 


&

実体はそれ自体によって個体化される。しかし附帯性は、基体つまり実体によって個体
 化される。たとえば「この白」は、それが「この基体」にある限りでそう言われる。
 したがって個体的実体が、他のものとは違う何らかの特殊な名をもつことが
 適切であり、それは「ヒュポスタシス」または「第一実体」と言われる。


\\


Sed adhuc quodam specialiori et
perfectiori modo invenitur particulare et individuum in substantiis
rationalibus, quae habent dominium sui actus, et non solum aguntur,
sicut alia, sed per se agunt, actiones autem in singularibus sunt. 



&

しかしさらに理性的実体において、より特殊で完全なしかたで個や個体が
 見出される。それらは自らの行為の主であり、他のもののように作用を受
 けるだけでなく、それ自体によって作用する。作用は個に属する。


\\


Et
ideo etiam inter ceteras substantias quoddam speciale nomen habent
singularia rationalis naturae. Et hoc nomen est persona. Et ideo in
praedicta definitione personae ponitur substantia individua, inquantum
significat singulare in genere substantiae, additur autem rationalis
naturae, inquantum significat singulare in rationalibus substantiis.


&

ゆえに他の実体の中で、理性的本性をもつ個は何らかの特殊な名をもって
 おり、それが「ペルソナ」である。
ゆえに上述の定義においてペルソナは、実体の類における個を表示するかぎ
 りで「個体的実体」と言われ、理性的実体における個を表示するかぎりで
 「理性的本性の」が加えられている。
 \footnote{particulare+substantia(=hypostasis)+rationalis natura = persona}


\\



{\scshape Ad primum ergo dicendum} quod, licet hoc singulare vel illud definiri non
possit, tamen id quod pertinet ad communem rationem singularitatis,
definiri potest, et sic philosophus definit substantiam primam. Et hoc
modo definit Boetius personam.


&

第一異論に対しては、それゆえ以下のように言われるべきである。
この個やあの個を定義することはできないが、個の共通の性格に属することは定
 義されうる。そしてそのように哲学者も第一実体を定義している。そしてボエ
 ティウスは、そのようにペルソナを定義している。


\\



{\scshape Ad secundum dicendum} quod, secundum quosdam, {\itshape substantia} in definitione
personae ponitur pro substantia prima, quae est hypostasis. Neque tamen
superflue additur {\itshape individua}. Quia nomine {\itshape hypostasis} vel {\itshape substantiae
primae}, excluditur ratio universalis et partis (non enim dicimus quod
homo communis sit hypostasis, neque etiam manus, cum sit pars), sed per
hoc quod additur {\itshape individuum}, excluditur a persona ratio assumptibilis;
humana enim natura in Christo non est persona, quia est assumpta a
digniori, scilicet a verbo Dei. 

&

第二異論に対しては、以下のように言われるべきである。
たしかに「実体」はペルソナの定義の中に第一実体、つまりヒュ
 ポスタシスとして置かれているが、「個体的」が置かれているのは重複
 ではないと論じる人々がいる。彼らの理由は以下の通りである。「ヒュポスタシス」や「第一実体」という
 名によって、普遍や部分という性格が排除される(私たちは、共通
 の人間をヒュポスタシスとは言わないし、手を、それは部分なので、ヒュポスタシス
 とも言わない)。しかし、「個体的」が付加されることによって、ペルソナか
 ら、「受け取られうる」という性格が排除される。なぜなら、キリストにおけ
 る人間本性はより偉大なもの、すなわち神の言葉によって受け取られるので、
 ペルソナではないからである。

\\

-- Sed melius dicendum est quod {\itshape substantia}
accipitur communiter, prout dividitur per primam et secundam, et per hoc
quod additur {\itshape individua}, trahitur ad standum pro substantia prima.

&

しかし、より正しくは次のように言われるべきである。
「実体」は、ここでは共通的に、つまり第一実体と第二実体とに分けられるもの
 として理解されていて、「個体的」が付加されることによって第一実体とし
 て理解されるべく限定される。



\\



{\scshape Ad tertium dicendum} quod, quia substantiales differentiae non sunt nobis
notae, vel etiam nominatae non sunt, oportet interdum uti differentiis
accidentalibus loco substantialium, puta si quis diceret, {\itshape ignis est
corpus simplex, calidum et siccum}, accidentia enim propria sunt effectus
formarum substantialium, et manifestant eas. Et similiter nomina
intentionum possunt accipi ad definiendum res, secundum quod accipiuntur
pro aliquibus nominibus rerum quae non sunt posita. Et sic hoc nomen
{\itshape individuum} ponitur in definitione personae, ad designandum modum
subsistendi qui competit substantiis particularibus.


&

第三異論に対しては、以下のように言われるべきである。
実体的差異は私たちに知られず名も付けられないので、実体的差異の代わり
 に附帯的な差異を用いなければならない。たとえば「火とは、熱く乾いて
 いる単純物体である」と言う場合のように。じっさい固有の附帯性は実体形
 相の結果であり、実体形相を明示するものである。同様に概念の名も、名が付け
 られていない事物の何らかの名の代わりとして理解される限りで事物を定義
 するために受け取られうる。その意味で「個体」というこの名が個別
 的実体に適合した自存様態を示すためにペルソナの定義の中に置かれている。


\\



{\scshape Ad quartum dicendum} quod, secundum philosophum, in V {\itshape Metaphys}., nomen
{\itshape naturae} primo impositum est ad significandam generationem viventium,
quae dicitur nativitas. 


&

第四異論に対しては、以下のように言われるべきである。
『形而上学』第5巻の哲学者によれば、「本性」は、第一に「出生」と呼ばれ
 る生きるものの生成を意味するために付けられた。




\\


Et quia huiusmodi generatio est a principio
intrinseco, extensum est hoc nomen ad significandum principium
intrinsecum cuiuscumque motus. Et sic definitur natura in II {\itshape Physic}.


&

そして、このような生成は、内在的根源からあるので、この名は全ての運動の
 内在的根源を表示するように拡張された。『自然学』第2巻ではこの意味で定義されている。


\\


 Et
quia huiusmodi principium est formale vel materiale, communiter tam
materia quam forma dicitur natura. 


&

そしてこのような根源は、形相的なものであるかあるいは質料的なもので
 あるので、形相も質料も共通に「本性」と言われる。

\\


Et quia per formam completur essentia
uniuscuiusque rei, communiter essentia uniuscuiusque rei, quam
significat eius definitio, vocatur natura. Et sic accipitur hic
natura. 


&

そして、各々の事物の本質は形相によって完成されるので、定義が表示する事物
 の本質が、共通に「本性」と呼ばれる。ここではこの意味で「本性」が理解さ
 れている。


\\


Unde Boetius in eodem libro dicit quod {\itshape natura est unumquodque
informans specifica differentia}, specifica enim differentia est quae
complet definitionem, et sumitur a propria forma rei. 


&

このことから、ボエティウスは同書で「本性とは、各々ものを形成する種を作る
 種差である」と述べる。種を作る種差は、定義を完成するものであ
 り、事物の固有の形相から採られるからである。



\\


Et ideo
convenientius fuit quod in definitione personae, quae est singulare
alicuius generis determinati, uteretur nomine {\itshape naturae}, quam {\itshape essentiae},
quae sumitur ab {\itshape esse}, quod est communissimum.


&

ゆえに、ある限定された類の何らかの個物であるペルソナの定義の中で、「本質」よ
 り「本性」という名を用いる方が好都合だった。本質は存在から採られ、存在
 はもっとも共通的なものだからである。


\\



{\scshape Ad quintum dicendum} quod anima est pars humanae speciei, et ideo, licet
sit separata, quia tamen retinet naturam unibilitatis, non potest dici
substantia individua quae est hypostasis vel substantia prima; sicut nec
manus, nec quaecumque alia partium hominis. Et sic non competit ei neque
definitio personae, neque nomen.


&

第五異論に対しては、以下のように言われるべきである。
魂は、人間の種の部分である。ゆえに、分離されていても、合一可能性という本
 性を保持しているので、ヒュポスタシス、あるいは第一実体であるところの、
 個体的実体と言われることはできない。これは、手や、その他なんであれ人間の部
 分も同様である。この意味で、分離した魂には、ペルソナの定義も名も適合しない。

\end{longtable}
\newpage




\rhead{a.~2}
\begin{center}
 {\Large {\bf ARTICULUS SECUNDUS}}\\
 {\large UTRUM PRESONA SIT INDEM QUOD \\ {\itshape HYPOSTASIS}, {\itshape
 SUBSISTENTIA} ET {\itshape ESSENTIA}}\\
 {\footnotesize I {\itshape Sent.}, d.23, a.1; {\itshape De Pot.}, q.9, a.1.}\\
 {\Large 第二項\\ペルソナは、ヒュポスタシス、自存体、本質と同一か}
\end{center}

\begin{longtable}{p{21em}p{21em}}

{\Huge A}{\scshape d secundum sic proceditur}. Videtur quod persona sit idem quod
{\itshape hypostasis}, {\itshape subsistentia} et {\itshape essentia}. Dicit enim Boetius, in libro {\itshape de
Duab. Natur}., quod Graeci {\itshape naturae rationalis individuam substantiam
hypostaseos nomine vocaverunt}. Sed hoc etiam, apud nos, significat nomen
{\itshape personae}. Ergo persona omnino idem est quod hypostasis.

&

第二項の問題へ議論は以下のように進められる。ペルソナは、ヒュポスタシス、
 自存体、本質と同一であると思われる。理由は以下の通り。
ボエティウスは『二つの本性について』という書物の中で、ギリシャ人たちは
 「理性的本性を持つ個的実体をヒュポスタシスという名で呼んだ」と言って
 いる。しかし私たちのもとでは、ペルソナという名もそれを意味する。ゆえ
 にペルソナは、ヒュポスタシスとまったく同じである。



\\




2 {\scshape Praeterea}, sicut in divinis dicimus tres personas, ita in divinis
dicimus tres subsistentias, quod non esset, nisi persona et subsistentia
idem significarent. Ergo idem significant persona et subsistentia.

&

さらに、私たちは、神において三つのペルソナを語るように、神において三つの自存体を語
 る。これは、もしペルソナと自存体が同じものを意味しなかったならば、なかっ
 たことである。ゆえに、ペルソナと自存体は、同一のものを意味する。

\\



3 {\scshape Praeterea}, Boetius dicit, in commento {\itshape Praedicamentorum}, quod {\itshape usia}, quod
est idem quod essentia, significat compositum ex materia et forma. Id
autem quod est compositum ex materia et forma, est individuum
substantiae, quod et hypostasis et persona dicitur. Ergo omnia praedicta
nomina idem significare videntur.

&

さらにボエティウスは『カテゴリー論』の注解で、「ウーシア」、これは本
 質と同じだが、は質料と形相から複合されたものを意味すると述べている。
 しかし質料と形相から複合されたものは個的実体であり、これはまたヒュ
 ポスタシスやペルソナと言われる。ゆえに前述のすべての名は同じものを
 意味すると思われる。


\\



1 {\scshape Sed contra est} quod Boetius dicit, in libro {\itshape de Duab. Natur}., quod {\itshape genera
et species subsistunt tantum; individua vero non modo subsistunt, verum
etiam substant}. Sed a {\itshape subsistendo} dicuntur subsistentiae, sicut a
{\itshape substando} substantiae vel hypostases. Cum igitur esse hypostases vel
personas non conveniat generibus vel speciebus, hypostases vel personae
non sunt idem quod subsistentiae.

&

しかし反対に、ボエティウスは『二つの本性について』という書物の中で、「類
 と種は自存するだけだが個体は自存するだけでなく自立する」。と言う。し
 かし自立することから実体やヒュポスタシスが語られるように、「自存する」
 ことから自存体が語られる。ゆえにヒュポスタシスやペルソナであるこ
 とが類や種に適合しないので、ヒュポスタシスやペルソナは自存体と同じで
 ない。


\\



2 {\scshape Praeterea}, Boetius dicit, in commento {\itshape Praedicamentorum}, quod {\itshape hypostasis}
dicitur materia, {\itshape usiosis} autem, idest {\itshape subsistentia}, dicitur forma. Sed
neque forma neque materia potest dici persona. Ergo persona differt a
praedictis.

&

さらに、ボエティウスは、『カテゴリー論』の注解で、質料は「ヒュポスタシス」
 と言われ、形相は「ウーシア」つまり「自存体」と言われる、
 と述べている。しかし、形相も質料もペルソナでない。ゆえに、ペルソナは、
 これらとは異なる。

\\



{\scshape Respondeo dicendum} quod, secundum philosophum, in V {\itshape Metaphys}.,
substantia dicitur dupliciter. Uno modo dicitur substantia {\itshape quidditas
rei}, quam significat definitio, secundum quod dicimus quod {\itshape definitio
significat substantiam rei}, quam quidem substantiam Graeci {\itshape usiam} vocant,
quod nos {\itshape essentiam} dicere possumus. 


&


解答する。以下のように言われるべきである。
『形而上学』第5巻の哲学者によれば、実体は二通りに語られる。一つは、私た
 ちが「定義は事物の実体を表示する」と言うのに即して、定義が表示する事物
 の何性が実体と言われる。ギリシャ人たちはこの実体を「ウーシア」と呼ぶが、
 私たちはそれを「本質」と言うことができる。

\\


Alio modo dicitur substantia
{\itshape subiectum vel suppositum quod subsistit in genere substantiae}. Et hoc
quidem, communiter accipiendo, nominari potest et nomine significante
intentionem, et sic dicitur {\itshape suppositum}. 

&

もう一つには「実体の類の中で自存する基体ないし個体」が実体と言われる。
 そしてこれは、共通的に理解されるならば、概念を表示する名によっても名付
 けられうるので、「個体(suppositum)」とも言われる。


\\


Nominatur etiam tribus nominibus
significantibus rem, quae quidem sunt {\itshape res naturae}, {\itshape subsistentia} et
{\itshape hypostasis}, secundum triplicem considerationem substantiae sic
dictae.  

&

さらに、この意味で言われる実体は、それを三つの角度から考察することによっ
 て、事物を意味表示する三つの名、「本性をもつ事物」「自存体」「ヒュポス
 タシス」によっても名付けられる。


\\


Secundum enim quod per se existit et non in alio, vocatur
{\itshape subsistentia}, illa enim subsistere dicimus, quae non in alio, sed in se
existunt.
Secundum vero quod supponitur alicui naturae communi, sic
dicitur {\itshape res naturae}; sicut hic homo est res naturae humanae. Secundum
vero quod supponitur accidentibus, dicitur {\itshape hypostasis} vel
{\itshape substantia}. 

&

すなわち、他のものにおいてではなく、それ自体によって存立する点で、「自存
 体」と呼ばれる。私たちは、他のものにおいてではなくそれ自体において存立
 するものを、自存すると言うからである。また、それが何らかの共通な本性のもとに
 あるかぎりで、「本性をもつ事物」と言われる。たとえば「この人間」は、人間
 の本性をもつ事物である。さらに、それが附帯性のもとにあるかぎりで、「ヒュ
 ポスタシス」や「実体」と言われる。

\\



Quod autem haec tria nomina significant communiter in toto
genere substantiarum, hoc nomen persona significat in genere rationalium
substantiarum.

&

しかしこれら三つの名は実体の類の中で共通に全体的に表示するのに対して、
 「ペルソナ」という名は理性的実体の類のなかで表示する。


\\


{\itshape Ad primum ergo dicendum} quod {\itshape hypostasis}, apud Graecos, ex propria
significatione nominis habet quod significet quodcumque individuum
substantiae, sed ex usu loquendi habet quod sumatur pro individuo
rationalis naturae, ratione suae excellentiae.

&

第一異論に対しては、それゆえ以下のように言われるべきである。
「ヒュポスタシス」はギリシャ人たちのもとで、その厳密な意味ではなんで
 あれ実体に属するすべての個体を意味するが、言葉の使用においてその卓越
 性のため理性的本性をもつ個体を意味するのに使われうる。


\\



{\scshape Ad secundum dicendum} quod, sicut nos dicimus in divinis pluraliter tres
personas et tres subsistentias, ita Graeci dicunt tres hypostases. Sed
quia nomen {\itshape substantiae}, quod secundum proprietatem significationis
respondet {\itshape hypostasi}, aequivocatur apud nos, cum quandoque significet
essentiam, quandoque hypostasim; ne possit esse erroris occasio,
maluerunt pro {\itshape hypostasi} transferre {\itshape subsistentiam}, quam {\itshape substantiam}.

&

第二異論に対しては、以下のように言われるべきである。
私たちが、神の中に三つのペルソナや三つの自存体を語るように、ギリシャ人た
 ちは三つのヒュポスタシスを語る。しかし、「実体」という名は、それ自体の
 意味としては「ヒュポスタシス」に対応するが、私たちのもとでは、あるとき
 には本質を意味し、あるときにはヒュポスタシスを意味するというように、両
 義的に使われるので、誤りが生じないように、「ヒュポスタシス」の訳語とし
 て、「実体」ではなく「自存体」を選好した。


\\



{\scshape Ad tertium dicendum} quod essentia proprie est id quod significatur per
definitionem. Definitio autem complectitur principia speciei, non autem
principia individualia. Unde in rebus compositis ex materia et forma,
essentia significat non solum formam, nec solum materiam, sed compositum
ex materia et forma communi, prout sunt principia speciei. 


&

第三異論に対しては、以下のように言われるべきである。
厳密に言えば、本質は、定義によって表示されるものである。
また定義は、種の根源を記述するが、個体の根源を記述しない。したがって、質
 料と形相から複合された事物において、本質は形相だけでなく、また質料だけ
 でもなく、種の根源としての共通的な質料と形相から複合されたものを意味す
 る。


\\


Sed
compositum ex {\itshape hac materia} et ex {\itshape hac forma}, habet rationem hypostasis et
personae, anima enim et caro et os sunt de ratione hominis, sed haec
anima et haec caro et hoc os sunt de ratione {\itshape huius hominis}. Et ideo
hypostasis et persona addunt supra rationem essentiae principia
individualia; neque sunt idem cum essentia in compositis ex materia et
forma, ut supra dictum est, cum de simplicitate divina ageretur.


&

しかし、「この質料」と「この形相」から複合されたものは、ヒュポスタシスや
 ペルソナの性格を持つ。なぜなら、魂と肉と骨は、人間の性格に属するが、こ
 の魂、この肉、この骨は、「この人間」の性格に属するからである。ゆえに、
 「ヒュポスタシス」や「ペルソナ」は、本質の性格に個体的根源を加えるの
 で、質料と形相から複合されたものにおいては(これらは)本質と同一でない。これは、
 前に、神の単純性について論じられたときに\footnote{STI, q.3, a.3.}述べられたことである。


\\




{\scshape Ad quartum dicendum} quod Boetius dicit genera et species subsistere,
inquantum individuis aliquibus competit subsistere, ex eo quod sunt sub
generibus et speciebus in praedicamento substantiae comprehensis, non
quod ipsae species vel genera subsistant, nisi secundum opinionem
Platonis, qui posuit species rerum separatim subsistere a
singularibus. {\itshape Substare} vero competit eisdem individuis in ordine ad
accidentia, quae sunt praeter rationem generum et specierum.

&


第四異論\footnote{第一反対異論のこと。}については、以下のように言われるべきである。
ボエティウスが、類や種が「自存する」と言うのは、個体の中のあるものに、そ
 れが実体のカテゴリーのもとに包含される類や種のもとにあるがゆえに、自
 存することが適合するかぎりにおいてであって、種や類似体が自存すると言っ
 ているのではない。事物の種が個物から離れて自存すると論じたプラトンの意
 見によらない限りは。
他方、「自立する」ことは、附帯性への秩序において、その同じ個体に適合する
 が、附帯性は類や種の性格の中には入らない。


\\



{\scshape Ad quintum dicendum} quod individuum compositum ex materia et forma,
habet quod substet accidenti, ex proprietate materiae. Unde et Boetius
dicit, in libro {\itshape de Trin}., {\itshape forma simplex subiectum esse non potest}. Sed
quod per se subsistat, habet ex proprietate suae formae, quae non
advenit rei subsistenti, sed dat esse actuale materiae, ut sic
individuum subsistere possit. Propter hoc ergo hypostasim attribuit
materiae, et {\itshape usiosim}, sive subsistentiam, formae, quia materia est
principium substandi, et forma est principium subsistendi.

&

第五異論\footnote{第二反対異論のこと。}に対しては、以下のように言われるべきである。
質料と形相から複合された個体は、質料の特質にもとづいて附帯性のもとにあ
 るという性格を持つ。このことからボエティウスも『三位一体論』という書
 物の中で、「単純形相は基体であることができない」と述べる。これに対し
 て、それ自体で自存するということは、これを形相の特質にもとづいて持つ。
 形相は、自存する事物に到来するのではなく、質料に現実的な存在を与えて、
 個体が自存できるようにする。ゆえにこのことから、彼はヒュポスタシスを質料
 に、「ウーシア」ないし自存体を形相に帰する。なぜなら質料は自立(substare)の根源
 であり、形相は自存の根源だからである。





\end{longtable}
\newpage





\rhead{a.~3}
\begin{center}
 {\Large {\bf ARTICULUS TERTIUS}}\\
 {\large UTRUM NOMEN {\itshape PERSONAE} SIT PONENDUM IN DIVINIS}\\
 {\footnotesize I {\itshape Sent.}, d.23, a.2; {\itshape De Pot.}, q.9, a.3.}\\
 {\Large 第三項\\ペルソナという名が神の中に置かれるべきか}
\end{center}

\begin{longtable}{p{21em}p{21em}}
{\Huge A}{\scshape d tertium sic proceditur}. Videtur quod nomen personae non sit ponendum
in divinis. Dicit enim Dionysius, in principio {\itshape de
Div. Nom}: {\itshape universaliter non est audendum aliquid dicere nec cogitare de
supersubstantiali occulta divinitate, praeter ea quae divinitus nobis ex
sanctis eloquiis sunt expressa}. Sed nomen {\itshape personae} non exprimitur nobis
in sacra Scriptura novi vel veteris testamenti. Ergo non est nomine
{\itshape personae} utendum in divinis.


&

第三項の問題へ、議論は以下のように進められる。
ペルソナという名は、神の中に置かれるべきでないと思われる。理由は以下の通
 り。
ディオニュシウスは、『神名論』の冒頭で以下のように述べている。「超実体的
 で隠された神性について、聖者たちが語ったことによって私たちに神的に示さ
 れたこと以外に、だれも何事かをあえて語ったり思考したりしようとしてはな
 らない」。しかし「ペルソナ」という名は、旧約聖書でも新約聖書でも、私た
 ちに示されていない。ゆえに、ペルソナという名を、神の中で用いるべきでな
 い。


\\



2. {\scshape Praeterea}, Boetius dicit, in libro {\itshape de
 Duab.~Natur.} : {\itshape nomen personae
videtur traductum ex his personis quae in comoediis tragoediisque
homines repraesentabant; persona enim dicta est a personando, quia
concavitate ipsa maior necesse est ut volvatur sonus. Graeci vero has
personas prosopa vocant, ab eo quod ponantur in facie, atque ante oculos
obtegant vultum}. Sed hoc non potest competere in divinis, nisi forte
secundum metaphoram. Ergo nomen {\itshape personae} non dicitur de Deo
 nisi metaphorice.


&

さらに、ボエティウスは『二つの本性について』という書物の中で、以下のよう
 に述べている。「ペルソナという名は、喜劇や悲劇で人物を表現していたペル
 ソナから伝えられたと思われる。じっさい、ペルソナは、「伝わり響く」こと
 からそう言われていて、それは、その穴によって必然的により大きな音が響き
 わたるためである。ギリシャ人たちは、これらのペルソナを、顔に付けて人々
 の目から表情を隠すので、プロソポンと呼んでいる。しかしこのことは神にお
 いて、比喩でなければ適合しない。ゆえに、「ペルソナ」という名は、比喩的
 にでなければ神について語られない。


\\



3 {\scshape Praeterea}, omnis persona est hypostasis. Sed nomen hypostasis non
videtur Deo competere, cum, secundum Boetium, significet id quod
subiicitur accidentibus, quae in Deo non sunt. Hieronymus etiam dicit
quod in hoc nomine hypostasis, venenum latet sub melle. Ergo hoc nomen
persona non est dicendum de Deo.


&

さらに、全てのペルソナはヒュポスタシスである。
しかし、「ヒュポスタシス」という名は、神に適合しないと思われる。なぜなら、
 ボエティウスによれば、それは附帯性の基体となるものを表示するが、神の中
 に附帯性はないからである。また、ヒエロニュムスも、「ヒュポスタシス」と
 いう名において、「蜜の中に毒が隠れている」と言っている。ゆえに、「ペル
 ソナ」というこの名は、神について語られるべきではない。

\\



4 {\scshape Praeterea}, a quocumque removetur definitio, et definitum. Sed definitio
personae supra posita non videtur Deo competere. Tum quia ratio importat
discursivam cognitionem, quae non competit Deo, ut supra ostensum est,
et sic Deus non potest dici {\itshape rationalis naturae}. Tum etiam quia Deus dici
non potest {\itshape individua substantia}, cum principium individuationis sit
materia, Deus autem immaterialis est; neque etiam accidentibus substat,
ut substantia dici possit. Nomen ergo {\itshape personae} Deo attribui non debet.


&

さらに、あらゆるものについて、定義が取り除かれると、定義されたものも取り
 除かれる。しかし、前に措定されたペルソナの定義は神に適合しないと思われ
 る。その理由は以下の通り。一つには、「理性」は推論的な認識を
 意味するが、前に示されたとおり\footnote{STI, q.14, a.7.}、これは神に適合しない。したがって、神は
 「理性的本性をもつ」と言われ得ないから。もう一つは、個体化の根源は質料
 だが、神は非質料的であり、神は「個体的実体」と言われ得ない。また、附帯
 性のもとにあることがないので、実体とも言われ得ない。ゆえに「ペルソナ」
 という名が神に帰せられるべきではない。


\\



{\scshape Sed contra est} quod dicitur in symbolo Athanasii, alia est persona
patris, alia filii, alia spiritus sancti.


&

しかし反対に、アタナシウス信教で「あるペルソナは父、別のペルソナは子、ま
 た別のペルソナは聖霊」と言われている。


\\



{\scshape Respondeo dicendum} quod persona significat id quod est
perfectissimum in tota natura, scilicet subsistens in rationali
natura. Unde, cum omne illud quod est perfectionis, Deo sit
attribuendum, eo quod eius essentia continet in se omnem perfectionem;
conveniens est ut hoc nomen persona de Deo dicatur. Non tamen eodem modo
quo dicitur de creaturis, sed excellentiori modo; sicut et alia nomina
quae, creaturis a nobis imposita, Deo attribuuntur; sicut supra ostensum
est, cum de divinis nominibus ageretur.


&

解答する。以下のように言われるべきである。
ペルソナは、全本性の中でもっとも完全なもの、すなわち理性的本性
 において自存するものを意味する。したがって神の本質はあらゆる完全性を
 その中に含むので、完全性に属するものは全て神に帰せられるべきだから、こ
 の「ペルソナ」という名が神について語られるのは理にかなっている。
しかし、それは被造物について語られるのと同じ意味ではなく、ちょうど被造物
 から私たちによって付けられる他の名と同じように、より優れたしかたで語ら
 れる。以前、神の名について論じられたところ\footnote{STI,
 q.13, a.3.}で明らかにされたように。


%Q. 自存する存在それ自体はペルソナという性格をもたないのか。
%
%自存する存在それ自体は、(拡張された意味で)実体であり、理性的(知性的)
% 本性を持ち、個であると思われる。そうすると、自存する存在それ自体もペル
% ソナだと思われる。
%
%しかし、ここではそういう論述内容になっていない。
%
%神がペルソナと言われることについては、三位一体論以前の自然神学的に考察さ
% れた神のペルソナ性が問題になると思われるが、ここではきわめて形式的な
% 論述に留まっている。
%
%自存する存在それ自体ではなく、言葉や愛、そしてそれを発するものがペルソナ
% だとされるのはどういうことなのか。




\\



{\scshape Ad primum ergo dicendum} quod, licet nomen personae in Scriptura veteris
vel novi testamenti non inveniatur dictum de Deo, tamen id quod nomen
significat, multipliciter in sacra Scriptura invenitur assertum de Deo;
scilicet quod est maxime per se ens, et perfectissime intelligens. 

&

第一異論に対しては、それゆえ、以下のように言われるべきである。
ペルソナという名が、旧約聖書にも新約聖書にも、神について語られたものとし
 て見出されないとしても、しかし、その名が表示するもの、すなわち、最大限
 に、自体的な存在者であり、もっとも完全に知性認識するもの、ということは、
 聖書の中で、神について主張されたものとして、多様に見出される。

\\


Si
autem oporteret de Deo dici solum illa, secundum vocem, quae sacra
Scriptura de Deo tradit, sequeretur quod nunquam in alia lingua posset
aliquis loqui de Deo, nisi in illa in qua primo tradita est Scriptura
veteris vel novi testamenti. 

&

もし、文字通り、聖書が神について伝えることだけが神について語られるべきで
 あるならば、旧約聖書と新約聖書が最初にその言語で伝えられた言語以外の言
 語では、何も神について語りえないことになってしまう。



\\


Ad inveniendum autem nova nomina, antiquam
fidem de Deo significantia, coegit necessitas disputandi cum
haereticis. Nec haec novitas vitanda est, cum non sit profana, utpote a
Scripturarum sensu non discordans, docet autem apostolus {\itshape profanas vocum
novitates} vitare, I {\itshape ad Tim}.~ult.


&

しかし、異端と論争する必要性が、神についての古代の信仰を表示する新しい名
 を見つけることを強いる。また、この新しさは、聖書の意味と一致しないよう
 なかたちでの冒涜ではないので、避けられるべきでない。使徒は『テモテへの
 手紙一』最終章で、「言葉の冒涜
 的な新しさを」\footnote{「テモテ、あなたにゆだねられているものを守り、俗悪な無駄話と、不当にも知識と呼ばれている反対論とを避けなさい。」(6:20)}避けることを教えている。



\\



{\scshape Ad secundum dicendum} quod, quamvis hoc nomen {\itshape persona} non conveniat Deo
quantum ad id a quo impositum est nomen, tamen quantum ad id ad quod
significandum imponitur, maxime Deo convenit. Quia enim in comoediis et
tragoediis repraesentabantur aliqui homines famosi, impositum est hoc
nomen persona ad significandum aliquos dignitatem habentes. 
Unde
consueverunt dici personae in Ecclesiis, quae habent aliquam
dignitatem.


&

第二異論に対しては、以下のように言われるべきである。
ペルソナというこの名が、語源にかんして神に適合しないとしても、意味しよう
 としているところにかんしては、最大限に神に適合する。なぜなら、喜劇や悲
 劇において、ある有名な人物が表現され、このペルソナという名は、威厳を持
 つ人々を表示するために付けられた。
このことから、教会では、何らかの威厳をもつものが、ペルソナと言われる習い
 となった。


\\


 Propter quod quidam definiunt personam, dicentes quod
persona est hypostasis proprietate distincta ad dignitatem
pertinente. 
Et quia magnae dignitatis est in rationali natura
subsistere, ideo omne individuum rationalis naturae dicitur persona, ut
dictum est. Sed dignitas divinae naturae excedit omnem dignitatem, et
secundum hoc maxime competit Deo nomen personae.


&

このため、ある人々はペルソナを定義して、ペルソナとは「威厳に属する固有性
 において傑出したヒュポスタシスである」と言う。
そして、理性的本性において自存することは大きな威厳に属するので、すでに述
 べられたとおり、全ての理
 性的本性をもつ個体がペルソナと言われる。しかし、神の本性の威厳はあらゆ
 る威厳にまさり、この点で、ペルソナという名は最大限に神に適合する。



\\



{\scshape Ad tertium dicendum} quod nomen hypostasis non competit Deo quantum ad id
a quo est impositum nomen, cum non substet accidentibus, competit autem
ei quantum ad id, quod est impositum ad significandum rem
subsistentem. Hieronymus autem dicit sub hoc nomine venenum latere, quia
antequam significatio huius nominis esset plene nota apud Latinos,
haeretici per hoc nomen simplices decipiebant, ut confiterentur plures
essentias, sicut confitentur plures hypostases; propter hoc quod nomen
{\itshape substantiae}, cui respondet in Graeco nomen {\itshape hypostasis}, communiter
accipitur apud nos pro essentia.


&

第三異論に対しては、以下のように言われるべきである。
ヒュポスタシスという名は、語源にかんして神に適合するのではない。神は附帯
 性のもとにあったりしないからである。むしろ、それが自存する事物を表示す
 るために付けられたことにかんして、神に適合する。しかしヒエロニュムスは、
 この名には毒が隠れていると言うが、それは、ラテン語を話す人々のもとでこ
 の名の意味が十分に知られる以前に、異端者たちがこの名によって、複数のヒュ
 ポスタシスがあるように複数の本質があることを認めるように、単純な人々
 を欺いたからである。それは、ギリシャ人たちの中でのヒュポスタシスという
 名に対応する「実体」という名は、私たちの間で、「本質」として共通に理解さ
 れているからである。


\\



{\scshape Ad quartum dicendum} quod Deus potest dici {\itshape rationalis naturae}, secundum
quod ratio non importat discursum, sed communiter intellectualem
naturam. {\itshape Individuum} autem Deo competere non potest quantum ad hoc quod
individuationis principium est materia, sed solum secundum quod importat
incommunicabilitatem. {\itshape Substantia} vero convenit Deo, secundum quod
significat existere per se. 


&


第四異論に対しては、以下のように言われるべきである。
神は、理性が推論を意味せず、共通に知性的本性を意味する限りで、理性的本性
 と言われうる。また、個体化の根源が質料であるかぎりでは、神に「個体」が
 適合することはできないが、非共有可能性を意味する限りでのみ、神に適合す
 る。さらに「実体」は、それがそれ自身によって存在することを意味する限り
 で、神に適合する。

\\

Quidam tamen dicunt quod definitio superius
a Boetio data, non est definitio personae secundum quod personas in Deo
dicimus. Propter quod Ricardus de sancto Victore, corrigere volens hanc
definitionem, dixit quod persona, secundum quod de Deo dicitur, est
{\itshape divinae naturae incommunicabilis existentia}.


&

しかし、ある人々は、前にボエティウスによって与えられた定義は、神において
 私たちが語る限りでのペルソナの定義でないと言う。そのため、サン・ヴィク
 トルのリカルドゥスは、この定義を修正しようとして、神について語られる限
 りでのペルソナは「共有されえない神の本性をもつ存在者」と言った。




\end{longtable}
\newpage





\rhead{a.~4}
\begin{center}
 {\Large {\bf ARTICULUS QUARTUS}}\\
 {\large UTRUM HOC NOMEN {\itshape PERSONA} SIGNIFICET RELATIONEM}\\
 {\footnotesize I {\itshape Sent.}, d.23, a.3; d.24, q.1, a.1; {\itshape
 De Pot.}, q.9, a.4.}\\
 {\Large 第四項\\ペルソナという名は関係を意味するか}
\end{center}

\begin{longtable}{p{21em}p{21em}}


{\Huge A}{\scshape d quartum sic proceditur}. Videtur quod hoc nomen persona non significet
relationem, sed substantiam, in divinis. Dicit enim Augustinus, in VII
{\itshape de Trin}., {\itshape cum dicimus personam patris, non aliud dicimus quam
substantiam patris; ad se quippe dicitur persona, non ad filium}.


&

第四項の問題へ、議論は以下のように進められる。
ペルソナという名は、神において、関係ではなく実体を意味すると思われる。理
 由は以下の通り。
アウグスティヌスは、『三位一体論』第7巻で以下のように述べている。「私た
 ちが父のペルソナと言うとき、父の実体以外のものを言わない。ペルソナは自
 分自身に対しても、子に対しても言われないからである」。


\\



2 {\scshape Praeterea}, {\itshape quid} quaerit de essentia. Sed, sicut dicit Augustinus in
eodem loco, cum dicitur, {\itshape tres sunt qui testimonium dant in caelo, pater,
verbum et spiritus sanctus}; et quaeritur, {\itshape quid tres?} respondetur, {\itshape tres
personae}. Ergo hoc nomen persona significat essentiam.


&

さらに、本質については、「何」が問われる。しかし、アウグスティヌスが同じ
 箇所で言うように、「天において証言するのは三つ、父、言葉、聖霊である」と言われ、
 そして「三とは何か」と問われると、「三つのペルソナである」と答えられる。
 ゆえに、ペルソナというこの名は、本質を意味する。



\\



3 {\scshape Praeterea}, secundum philosophum, IV {\itshape Metaphys}., id quod significatur per
nomen, est eius definitio. Sed definitio personae est {\itshape rationalis naturae
individua substantia}, ut dictum est. Ergo hoc nomen persona significat
substantiam.


&

さらに、『形而上学』第4巻の哲学者によれば、名によって表示されるのはその定義
 である。しかし、ペルソナの定義は、すでに述べられたとおり、「理性的本性
 をもつ個体的実体」である。ゆえに、このペルソナという名は実体を表示する。


\\



4 {\scshape Praeterea}, persona in hominibus et Angelis non significat relationem,
sed aliquid absolutum. Si igitur in Deo significaret relationem,
diceretur aequivoce de Deo et hominibus et Angelis.


&

さらに、ペルソナは、人間と天使において関係を表示せず、非関係的な何かを表
 示する。ゆえに、もし神において関係を表示したら、神と人間と天使とでは、
 異義的に語られることになる。

\\



{\scshape Sed contra est} quod dicit Boetius, in libro {\itshape de Trin}., quod omne nomen ad
personas pertinens, relationem significat. Sed nullum nomen magis
pertinet ad personas, quam hoc nomen {\itshape persona}. Ergo hoc nomen {\itshape persona}
relationem significat.


&

しかし反対に、ボエティウスは『三位一体論』という書物で、全てペルソナに属
 する名は関係を表示すると言っている。しかし、「ペルソナ」というこの名以
 上にペルソナに属する名はない。ゆえに、「ペルソナ」というこの名は関係
 を表示する。


\\



{\scshape Respondeo dicendum} quod circa significationem huius nominis persona in
divinis, difficultatem ingerit quod pluraliter de tribus praedicatur,
praeter naturam essentialium nominum; neque etiam ad aliquid dicitur,
sicut nomina quae relationem significant. 


&

解答する。以下のように言われるべきである。
この神における「ペルソナ」という名の意味を巡っては、本質的名称の本性を外
れて三つのものについて複数で述語されることと、関係を表示する名のように、関係
的に語られるのではないこととが、困難を生んでいる。


\\


Unde quibusdam visum est quod
hoc nomen persona simpliciter, ex virtute vocabuli, essentiam significet
in divinis, sicut hoc nomen {\itshape Deus}, et hoc nomen {\itshape sapiens}, sed propter
instantiam haereticorum, est accommodatum, ex ordinatione Concilii, ut
possit poni pro relativis; et praecipue in plurali, vel cum nomine
partitivo, ut cum dicimus {\itshape tres personas}, vel {\itshape alia est persona Patris,
alia Filii}. 
In singulari vero potest sumi pro absoluto, et pro
relativo.

&

このため、ある人々には、このペルソナという名が、言葉の意味から端的に、「神」
 や「知者」と同じように、神において本質を意味するように思われたが、それ
 では異端となるので、公会議の命令によって修正され、関係的なものを表示するた
 めに付けられうることになった。とくに、「三つのペルソナ」や「あるものは
 父のペルソナ、他のものは子のペルソナ」と言う場合のように、複数において、
 また、部分的な名と共に用いられるときには(関係的な名として使われる)。し
 かし単数では、非関係的にも関係的にも理解されうるとした。


\\


 Sed haec non videtur sufficiens ratio. Quia si hoc nomen
persona, ex vi suae significationis, non habet quod significet nisi
essentiam in divinis; ex hoc quod dictum est tres personas, non fuisset
haereticorum quietata calumnia, sed maioris calumniae data esset eis
occasio. 


&


しかし、これは十分な議論ではない。なぜなら、もしペルソナというこの名が、
 それ自身の意味から、神において本質しか表示できないならば、「三つのペル
 ソナ」と言われることから、異端者たちの虚偽の主張はおさまらず、よりその
 ような主張の機会を増やしたであろう。



\\


Et ideo alii dixerunt quod hoc nomen persona in divinis
significat simul essentiam et relationem. Quorum quidam dixerunt quod
significat essentiam in recto, et relationem in obliquo. Quia persona
dicitur quasi per se una, unitas autem pertinet ad essentiam. Quod autem
dicitur per se, implicat relationem oblique, intelligitur enim pater per
se esse, quasi relatione distinctus a filio. 


&

それゆえ、他の人々は、このペルソナという名は、神において、本質と関係を同
 時に表示すると言った。そして、さらにそのうちのある人々は、表の意味で本
 質を、裏の意味で関係を意味すると言った。というのも、「ペルソナ(persona)」は、
 「それ自体によって一(per se una)」というように言われるが、「一」は本質
 に属するからである。それに対して、「それ自体によって」は、裏の意味で関
 係を意味する。なぜなら、父がそれ自身によって存在することは、いわば、子
 から区別された関係によって、そうであると理解されるからである。


\\


Quidam vero dixerunt e
converso, quod significat relationem in recto, et essentiam in obliquo,
quia in definitione personae, natura ponitur in obliquo. Et isti
propinquius ad veritatem accesserunt. 

&

しかし、別のある人々は、逆に、表の意味で関係を意味し、裏の意味で本質を意
 味すると言った。なぜなら、ペルソナの定義の中で、「本性」は斜格に置かれ
 ているからである。そしてこの人々は、より真理に近づいた。


\\

Ad evidentiam igitur huius
quaestionis, considerandum est quod aliquid est de significatione minus
communis, quod tamen non est de significatione magis communis, rationale
enim includitur in significatione hominis, quod tamen non est de
significatione animalis. 
Unde aliud est quaerere de significatione
animalis, et aliud est quaerere de significatione animalis quod est
homo.

&

さて、この問題を明らかにするためには、狭い意味に含まれるものが、
広い意味に含まれない、ということが考察されるべきである。
 たとえば、「理性的」は、「人間」の意味に含まれるが、「動物」の意味に
 含まれない。
したがって、(広く)「動物」の意味について探究することと、人間という「動物」の意
 味にについて探究することとは異なる。

\\


 Similiter aliud est quaerere de significatione huius nominis
persona in communi, et aliud de significatione personae divinae. Persona
enim in communi significat substantiam individuam rationalis naturae, ut
dictum est. 
Individuum autem est quod est in se indistinctum, ab aliis
vero distinctum. Persona igitur, in quacumque natura, significat id quod
est distinctum in natura illa sicut in humana natura significat has
carnes et haec ossa et hanc animam, quae sunt principia individuantia
hominem; quae quidem, licet non sint de significatione personae, sunt
tamen de significatione personae humanae. 


&

同様に、ペルソナというこの名の意味を共通に探究することと、神のペルソナの
 意味について探究することとは異なる。すでに述べられたとおり、ペルソナは、
 共通的には、理性的本性をもつ個体的実体を意味する。
個体とは、それ自体において区別されず、他から区別されたもののことである。
 ゆえに、ペルソナは、どの本性においても、その本性において区別されたもの
 を意味する。ちょうど、人間本性において、人間を個体化する根源である、こ
 の肉と、この骨と、この魂を意味するように。これらは、ペルソナの意味の中
 に入らないが、しかし、人間のペルソナの意味の中には入る。

\\




&

\\


Distinctio autem in divinis
non fit nisi per relationes originis, ut dictum est supra. Relatio autem
in divinis non est sicut accidens inhaerens subiecto, sed est ipsa
divina essentia, unde est subsistens, sicut essentia divina
subsistit. Sicut ergo deitas est Deus, ita paternitas divina est Deus
pater, qui est persona divina. 


&

これに対し、神における区別は、前に述べられたとおり\footnote{STI q.28, a.3.}、起源の関係によってし
 か生じない。また、神における関係は、基体に内属する附帯性としてあるので
 はなく、神の本質それ自体である。したがって、それは、神の本質が自存する
 のと同様、自存するものである。ゆえに、神性が神であるように、神の父性は
 父である神であり、それは神のペルソナである。


\\


Persona igitur divina significat
relationem ut subsistentem. Et hoc est significare relationem per modum
substantiae quae est hypostasis subsistens in natura divina; licet
subsistens in natura divina non sit aliud quam natura divina. Et
secundum hoc, verum est quod hoc nomen persona significat relationem in
recto, et essentiam in obliquo, non tamen relationem inquantum est
relatio, sed inquantum significatur per modum hypostasis. 


&

ゆえに、神のペルソナは、自存するものとしての関係を意味する。そしてこれは、
 関係を、神の本性において自存するヒュポスタシスである実体のありかたで表
 示することである。ただし、神の本性において自存するものは、神の本性以外
 ではないのだが。そしてこの意味で、このペルソナという名が、表の意味で関
 係を意味し、裏の意味で本質を意味するというのは真である。しかし、関係で
 あるかぎりでの関係を意味するのではなく、ヒュポスタシスのありかたによっ
 て表示される限りにおいてである。


\\


Similiter
etiam significat essentiam in recto, et relationem in obliquo, inquantum
essentia idem est quod hypostasis; hypostasis autem significatur in
divinis ut relatione distincta; et sic relatio, per modum relationis
significata, cadit in ratione personae in obliquo. 


&


同様に、本質がヒュポスタシスと同じであるかぎりで、表の意味で本質を、裏の
 意味で関係を意味する。しかしヒュポスタシスは神において区別された関係と
 して表示される。その意味で、関係は、表示された関係のあり方によって、裏
 の意味で、ペルソナの性格の中に入る。


\\

Et secundum hoc etiam
dici potest, quod haec significatio huius nominis persona non erat
percepta ante haereticorum calumniam, unde non erat in usu hoc nomen
persona, nisi sicut unum aliorum absolutorum. 


&

そしてこのかぎりで、このペルソナと
 いう名のこの意味は、異端者たちの虚偽の主張の前には認識されておらず、し
 たがって、このペルソナという名の使用には、他の非関係的なものの一つ
 としての意味しかなかったとも言われうる。

\\


Sed postmodum accommodatum
est hoc nomen persona ad standum pro relativo, ex congruentia suae
significationis, ut scilicet hoc quod stat pro relativo, non solum
habeat ex usu, ut prima opinio dicebat, sed etiam ex significatione sua.


&

しかし、その後、このペルソナという名は、その意味の集合の中に、関係的なも
 のも収容するようになり、最初の意見が言っていたように、ただ使用に基
 づいてだけでなく、それ自身の意味に基づいても、関係的なものを意味するよ
 うになった。

\\



{\scshape Ad primum ergo dicendum} quod hoc nomen persona dicitur ad se, non ad
alterum, quia significat relationem, non per modum relationis, sed per
modum substantiae quae est hypostasis. Et secundum hoc Augustinus dicit
quod significat essentiam, prout in Deo essentia est idem cum hypostasi,
quia in Deo non differt {\itshape quod est} et {\itshape quo est}.


&

第一異論に対しては、それゆえ、以下のように言われるべきである。
ペルソナというこの名は、他のものではなく、自分自身に対して言われる。なぜ
 なら、関係を、関係のあり方によってではなく、ヒュポスタシスという実体の
 あり方によって表示するからである。このことから、アウグスティヌスは、そ
 れが、神において本質はヒュポスタシスと同一であるかぎりで、本質を意味す
 ると言っている。神において、「在るもの」と「それによって在ると
 ころのそれ」とは異ならないから。


\\



{\scshape Ad secundum dicendum} quod {\itshape quid} quandoque quaerit de natura quam
significat definitio; ut cum quaeritur, {\itshape Quid est homo?} et respondetur,
{\itshape Animal rationale mortale}. Quandoque vero quaerit suppositum; ut cum
quaeritur, {\itshape Quid natat in mari?} et respondetur, {\itshape Piscis}. Et sic
quaerentibus {\itshape Quid tres?} responsum est, {\itshape Tres personae}.


&

第二異論に対しては、以下のように言われるべきである。「何」は、「人間とは
 何か」と問われるときのように、あるときには定義が示す
 本性を問い、「理性的で死すべき動物である」と答えら
 れる。しかし、「海で泳ぐのは何か」と問われるときのように、個体を問うこ
 ともあり、その場合には「魚が」と答えられる。この意味で、「何が三か」と
 問う人々に、「三つのペルソナが」と答えられた。


\\



{\scshape Ad tertium dicendum} quod in intellectu substantiae individuae, idest
distinctae vel incommunicabilis, intelligitur in divinis relatio, ut
dictum est.


&

第三異論に対しては、以下のように言われるべきである。
すでに述べられたとおり、個体的実体、すなわち、区別され、共有不可能な実体
 という概念の中に、神においては、関係が理解される。


\\



{\scshape Ad quartum dicendum} quod diversa ratio minus communium non facit
aequivocationem in magis communi. Licet enim sit alia propria definitio
equi et asini, tamen univocantur in nomine animalis, quia communis
definitio animalis convenit utrique. Unde non sequitur quod, licet in
significatione personae divinae contineatur relatio, non autem in
significatione angelicae personae vel humanae, quod nomen personae
aequivoce dicatur. Licet nec etiam dicatur univoce, cum nihil univoce de
Deo dici possit et de creaturis, ut supra ostensum est.


&

第四異論に対しては、以下のように言われるべきである。
狭義の概念の異なりが、広義の概念のなかに両義性を生み出すことはない。たと
 えば、馬とロバはそれぞれ独自の定義を持つが、しかし、動物という名におい
 て一義的に語られる。なぜなら、その両者に、共通の動物の定義が当てはまる
 からである。したがって、神のペルソナの意味に関係が含まれ、天使と人間の
 ペルソナの意味にそれが含まれないことから、ペルソナという名が両義的に言
 われることは帰結しない。ただし、前に示されたとおり\footnote{STI, q.13,
 a.5.}、神と被造物とに一義的に言われるものはないので、一義的に語られるの
 でもないのだが。



\end{longtable}

\end{document}
