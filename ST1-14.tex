\documentclass[10pt]{jsarticle} % use larger type; default would be 10pt
%\usepackage[utf8]{inputenc} % set input encoding (not needed with XeLaTeX)
%\usepackage[round,comma,authoryear]{natbib}
%\usepackage{nruby}
\usepackage{okumacro}
\usepackage{longtable}
%\usepqckage{tablefootnote}
\usepackage[polutonikogreek,english,japanese]{babel}
%\usepackage{amsmath}
\usepackage{latexsym}
\usepackage{color}

%----- header -------
\usepackage{fancyhdr}
\lhead{{\it Summa Theologiae} I, q.14}
%--------------------

\bibliographystyle{jplain}

\title{{\bf PRIMA PARS}\\{\HUGE Summae Theologiae}\\Sancti Thomae
Aquinatis\\{\sffamily QUEAESTIO DECIMAQUARTA}\\DE SCIENTIA DEI}
\author{Japanese translation\\by Yoshinori {\sc Ueeda}}
\date{Last modified \today}


%%%% コピペ用
%\rhead{a.}
%\begin{center}
% {\Large {\bf }}\\
% {\large }\\
% {\footnotesize }\\
% {\Large \\}
%\end{center}
%
%\begin{longtable}{p{21em}p{21em}}
%
%&
%
%\\
%\end{longtable}
%\newpage



\begin{document}
\maketitle
\pagestyle{fancy}

\begin{center}
{\Large 第十四問\\神の知について}
\end{center}

\begin{longtable}{p{21em}p{21em}}

{\huge P}{\scshape ost} considerationem eorum quae ad divinam
substantiam pertinent, restat considerandum de his quae pertinent ad
operationem ipsius. Et quia operatio quaedam est quae manet in operante,
quaedam vero quae procedit in exteriorem effectum, primo agemus de
scientia et voluntate (nam intelligere in intelligente est, et velle in
volente); et postmodum de potentia Dei, quae consideratur ut principium
operationis divinae in effectum exteriorem procedentis. Quia vero
intelligere quoddam vivere est, post considerationem divinae scientiae,
considerandum erit de vita divina. Et quia scientia verorum est, erit
etiam considerandum de veritate et falsitate. Rursum, quia omne cognitum
in cognoscente est, rationes autem rerum secundum quod sunt in Deo
cognoscente, ideae vocantur, cum consideratione scientiae erit etiam
adiungenda consideratio de ideis.

&

神の実体に属する事柄を考察したあと、神の働きに属する事柄が考察されるべ
きこととして残されている。さて、働きには働くもののうちに留まるものもあ
れば、外の結果へと出て行くものもある。そこで、最初に、私たちは知と意志
\footnote{Q.19}について論じ(というのも、知性認識の働きは知性認識する
もののうちにあり、意志の働きは意志するもののうちにあるから)、そのあと
で、神の能力\footnote{Q.25}について論じよう。神の能力は、外の結果へと
出て行く神の働きの根源として考察される。また、知性認識することは、生き
ることの一種だから、神の知の考察のあとに、神の生命\footnote{Q.18}が考
察されるべきであろう。また、知とは真についての知であるから、真理
\footnote{Q.17}と虚偽\footnote{Q.18}についても考察されるべきであろう。
さらに、すべて認識されたものは認識するもののうちにあるが、諸事物の理念
は、認識者としての神のうちにある場合にはイデアと呼ばれるので、知につい
ての考察には、イデア\footnote{Q.15}についての考察も付け加えられるべきである。
\end{longtable}

\noindent
参考

\noindent
第14問 神の知について\\
第15問 イデアについて\\
第16問 真理について\\
第17問 虚偽について\\
第18問 神の生命について\\
第19問 神の意志について\\
第20問 神の愛について\\
第21問 神の正義と憐れみについて\\
第22問 神の摂理について\\
第23問 予定について\\
第24問 生命の書\\
第25問 神の能力について\\
第26問 神の至福について\\


\begin{longtable}{p{21em}p{21em}}
Circa scientiam vero quaeruntur sexdecim. 
\begin{enumerate}
 \item utrum in Deo sit scientia.
 \item utrum Deus intelligat seipsum.
 \item utrum comprehendat se.
 \item utrum suum intelligere sit sua substantia.
 \item utrum intelligat alia a se.
 \item utrum habeat de eis propriam cognitionem. 
 \item utrum scientia Dei sit discursiva.
 \item utrum scientia Dei sit causa rerum.
 \item utrum scientia Dei sit eorum quae non sunt.
 \item utrum sit malorum.
 \item utrum sit singularium.
 \item utrum sit infinitorum.
 \item utrum sit contingentium futurorum.
 \item utrum sit enuntiabilium.
 \item utrum scientia Dei sit variabilis.
 \item utrum Deus de rebus habeat speculativam scientiam vel practicam. 
\end{enumerate}
&
さて、知について、十六のことが問われる。
\begin{enumerate}
 \item 神の中に知があるか。
 \item 神は自己自身を知性認識するか。
 \item 神は自己自身を把握するか。
 \item 神の知性認識は神の実体か。
 \item 神は自己以外のものを知性認識するか。
 \item 神は自己以外のものについて固有の認識を持つか。
 \item 神の知は推論的か。
 \item 神の知は事物の原因か。
 \item 存在しないものについて神の知はあるか。
 \item 悪について神の知はあるか。
 \item 個物について神の知はあるか。
 \item 無限について神の知はあるか。
 \item 未来の非必然的な事柄について神の知はあるか。
 \item 命題的な事柄について神の知はあるか。
 \item 神の知は可変的か。
 \item 神は諸事物について観照的な知を持つか、それとも実践的な知を持つか。
\end{enumerate}

\\
\end{longtable}

\newpage
\rhead{a.1}
\begin{center}
{\Large {\bfseries ARTICULUS PRIMUS}}\\
{\large UTRUM IN DEO SIT SCIENTIA}\\
{\footnotesize I {\itshape Sent.}, d.35, a.1; I {\itshape SCG.}, c.44; {\itshape De Verit.}, q.2, a.1; {\itshape Comend.~Theol.}, c.28; XII {\itshape Metaphy.}, lect.8.}\\
{\Large 第一項\\神の中に知があるか}
\end{center}

\begin{longtable}{p{21em}p{21em}}

{\huge A}{\sc d primum sic proceditur}.  Videtur quod in Deo non sit
scientia. Scientia enim habitus est, qui Deo non competit, cum sit
medius inter potentiam et actum. Ergo scientia non est in Deo.

&

第一の問題へ議論は以下のように進められる。神の中に知はないと思われる。
理由は以下の通り。知は習態だが、習態は可能態と現実態の中間にあるので、
神に適しない。ゆえに、知は神の中にない。

\\

{\sc 2 Praeterea}, scientia, cum sit conclusionum, est quaedam
cognitio ab alio causata, scilicet ex cognitione principiorum. Sed
nihil causatum est in Deo. Ergo scientia non est in Deo.

&

さらに、知とは結論についての知だから、それは何か他のもの、すなわち原理
の認識が原因となって生じた認識である。しかし神の中に原因をもつようなも
のはない。ゆえに、神の中に知はない。

\\

{\scshape 3 Praeterea}, omnis scientia vel universalis vel
particularis est. Sed in Deo non est universale et particulare, ut ex
superioribus patet; ergo in Deo non est scientia.

&

さらに、すべて知は普遍的か個別的かである。ところが神の中には、上述のこ
とから明らかなとおり、普遍も個別もない。 ゆえに、神の中に知はない。

\\

{\scshape Sed contra est} quod apostolus dicit {\itshape
Rom}.~{\scshape xi}, {\itshape o altitudo divitiarum sapientiae et
scientiae Dei}.

&

しかし反対に、使徒は『ローマの信徒への手紙』11章で「おお、神の富と知恵
と知の深さよ」と述べている。\footnote{「ああ、神の富と知恵と知識のなん
と深いことか。」(11:33)}

\\

{\scshape Respondeo dicendum} quod in Deo perfectissime est
scientia. Ad cuius evidentiam, considerandum est quod cognoscentia a
non cognoscentibus in hoc distinguuntur, quia non cognoscentia nihil
habent nisi formam suam tantum; sed cognoscens natum est habere formam
etiam rei alterius, nam species cogniti est in cognoscente.

&

解答する。以下のように言われるべきである。神の中には、もっとも完全なか
たちで知がある。これを明らかにするために、以下のことが考察されるべきで
ある。認識するものは認識しないものから次の点で区別される。すなわち、認
識しないものは自分の形相しか持たないのに対して、認識するものは[自分の
形相に加えて]他のものの形相をももつように生まれついている。なぜなら、
認識するものの中には、認識されるものの形象が存在するからである。

\\

Unde manifestum est quod natura rei non cognoscentis est magis
coarctata et limitata, natura autem rerum cognoscentium habet maiorem
amplitudinem et extensionem. Propter quod dicit philosophus, III
{\itshape de anima}, quod {\itshape anima est quodammodo
omnia}. Coarctatio autem formae est per materiam. Unde et supra
diximus quod formae, secundum quod sunt magis immateriales, secundum
hoc magis accedunt ad quandam infinitatem.

&

このことから、認識しない事物の本性は、より限定され制限されているが、認
識する事物の本性は、より大きな豊かさと広がりをもつことが明らかである。
このために哲学者は『デ・アニマ』3巻で「魂はある意味ですべてのものであ
る」と言う。ところで形相の限定は質料による。したがって、私たちは上でも、
形相がより非質料的であればあるほど、それだけいっそうある種の無限へ近づ
くと言った。

\\

Patet igitur quod immaterialitas alicuius rei est ratio quod sit
cognoscitiva; et secundum modum immaterialitatis est modus
cognitionis. Unde in II {\itshape de anima} dicitur quod plantae non
cognoscunt, propter suam materialitatem. Sensus autem cognoscitivus
est, quia receptivus est specierum sine materia, et intellectus adhuc
magis cognoscitivus, quia magis separatus est a materia et immixtus,
ut dicitur in III {\itshape de anima}. Unde, cum Deus sit in summo
immaterialitatis, ut ex superioribus patet, sequitur quod ipse sit in
summo cognitionis.

&

ゆえに、ある事物の非質料性は、それが認識しうるものであることの根拠であ
ること、そして、認識のあり方は非質料性のあり方に従うということが明らか
である。このことから『デ・アニマ』2巻で、植物はその質料性のゆえに認識
しないと言われる。これに対して感覚は、質料なしに形象を受け取りうるので
認識しうるし、知性は、『デ・アニマ』3巻で言われるとおり、より質料から
離れ混合されていないので、さらにいっそう認識しうる。したがって、神は、
上述のことから明らかなとおり、非質料性の頂点にあるので、認識の頂点にあ
ることが帰結する。

\\

{\scshape Ad primum ergo dicendum} quod, quia perfectiones procedentes
a Deo in creaturas, altiori modo sunt in Deo, ut supra dictum est,
oportet quod, quandocumque aliquod nomen sumptum a quacumque
perfectione creaturae Deo attribuitur, secludatur ab eius
significatione omne illud quod pertinet ad imperfectum modum qui
competit creaturae. Unde scientia non est qualitas in Deo vel habitus,
sed substantia et actus purus.

&

第一異論に対しては、それゆえ、次のように言われるべきである。すでに述べ
られたように、神から被造物へ発出する完全性はより高度なかたちで神の中に
あるから、被造物のどの完全性から取られた名であっても、それが神に帰せら
れるときにはいつでも、その意味から被造物に適合する不完全なあり方に属す
るすべてのことが切り離されなければならない。したがって、知は神において
性質や習態ではなく、実体であり純粋現実態である。

\\

{\scshape Ad secundum dicendum} quod ea quae sunt divisim et
multipliciter in creaturis, in Deo sunt simpliciter et unite, ut supra
dictum est. Homo autem, secundum diversa cognita, habet diversas
cognitiones, nam secundum quod cognoscit principia, dicitur habere
{\itshape intelligentiam}; {\itshape scientiam} vero, secundum quod
cognoscit conclusiones; {\itshape sapientiam}, secundum quod cognoscit
causam altissimam; {\itshape consilium} vel {\itshape prudentiam},
secundum quod cognoscit agibilia. Sed haec omnia Deus una et simplici
cognitione cognoscit, ut infra patebit. Unde simplex Dei cognitio
omnibus istis nominibus nominari potest, ita tamen quod ab unoquoque
eorum, secundum quod in divinam praedicationem venit, secludatur
quidquid imperfectionis est, et retineatur quidquid perfectionis
est. Et secundum hoc dicitur {\itshape Iob} {\scshape xii}, {\itshape
apud ipsum est sapientia et fortitudo; ipse habet consilium et
intelligentiam}.

&

第二異論に対しては次のように言われるべきである。上述のように、被造物の
中に、さまざまで多様なかたちで存在するものは、神の中に単純で一なるかた
ちで存在する。ところで、人間は、認識されるさまざまなものに応じて、さま
ざまな認識をもつ。すなわち、根源[または原理]を認識する場合に、それは
「知性認識」[あるいは「知的直観」]をもつと言われ、結論を認識する場合
に「知」をもつと言われる。また、最高の原因を認識する場合、人間は「知恵」
をもつと言われ、なされうることを認識する場合に、「思量」あるいは「思慮」
をもつと言われる。しかし、以下で明らかになるとおり、神はこれらすべてを
一つで単純な認識によって認識する。したがって神の単純な認識がこれらすべ
ての名によって名付けられうるが、そのときこれらのどの名も、それが神の述
語となる場合、それらの名から不完全性に属するものがすべて除去され、完全
性に属するものがすべて保持されるというかたちをとる。『ヨブ記』12章で
「神のもとに知恵と強さがあり、神は思量と知性認識をもつ」\footnote{「神
と共に知恵と力はあり、神と共に思慮分別もある」(12:13)}と言われるのも、
これに従う。

\\



{\scshape Ad tertium dicendum} quod scientia est secundum modum
cognoscentis, scitum enim est in sciente secundum modum scientis. Et
ideo, cum modus divinae essentiae sit altior quam modus quo creaturae
sunt, scientia divina non habet modum creatae scientiae, ut scilicet
sit universalis vel particularis, vel in habitu vel in potentia, vel
secundum aliquem talem modum disposita.

&

第三異論に対しては、次のように言われるべきである。知は認識するもののあ
り方に従ってある。知られたものは、知るもののあり方に従って、知るものの
中にあるからである。ゆえに、神の本質のあり方は、被造物が存在するあり方
よりも高いので、神の知は、普遍的であるとか、個別的であるとか、習態にあ
るとか可能態にあるとか、何かそのような状態に即してあるという、被造の知
のあり方をしていない。



\end{longtable}
\newpage

\rhead{a.2}
\begin{center}
{\Large {\bf ARTICULUS SECUNDUS}}\\ {\large UTRUM DEUS INTELLIGIT
SE}\\ {\footnotesize I {\itshape SCG.}, c.47; {\itshape De Verit.},
q.2, a.2; {\itshape Compend.~Theol.}, c.30; XII {\itshape Metaph.},
lect.11; {\itshape De Causis}, lect.13.}\\ {\Large 第二項\\神は自己
を知性認識するか}
\end{center}

\begin{longtable}{p{21em}p{21em}}

{\huge A}{\scshape d secundum sic proceditur}. Videtur quod Deus non
intelligat se. Dicitur enim in libro {\itshape de causis}, quod
{\itshape omnis sciens qui scit suam essentiam, est rediens ad
essentiam suam reditione completa}. Sed Deus non exit extra essentiam
suam, nec aliquo modo movetur, et sic non competit sibi redire ad
essentiam suam. Ergo ipse non est sciens essentiam suam.


&

第二項の問題へ、議論は以下のように進められる。神は自己を知性認識しない
と思われる。理由は以下の通り。『原因論』という書で、「自己の本質を知る
ものはすべて、完全な帰還によって自らの本質へ帰る」と言われている。とこ
ろが、神は、自らの本質から外に出ることはなく、どんなかたちでも動かされ
ないので、自らの本質へ帰るということは、神に適合しない。ゆえに、神は自
己の本質を知るものではない。

\\


{\scshape 2 Praeterea}, intelligere est quoddam pati et moveri, ut
dicitur in III {\itshape de anima}, scientia etiam est assimilatio ad
rem scitam, et scitum etiam est perfectio scientis. Sed nihil movetur,
vel patitur, vel perficitur a seipso; {\itshape neque similitudo sibi
est}, ut Hilarius dicit. Ergo Deus non est sciens seipsum.


&

さらに、『デ・アニマ』3巻で言われるとおり、知性認識することは一種の受
動であり動かされることである。また、知は知られた事物への類似化であり、
さらに、知られたものは知るものの完成である。ところが、自己自身によって
動かされたり受動したり完成されたりするものはなにもない。ヒラリウスが言
うように「自分への類似はない」からである。ゆえに、神は自己自身を知るも
のではない。

\\


{\scshape 3 Praeterea}, praecipue Deo sumus similes secundum
intellectum, quia secundum mentem sumus ad imaginem Dei, ut dicit
Augustinus. Sed intellectus noster non intelligit se, nisi sicut
intelligit alia, ut dicitur in III {\itshape de anima}. Ergo nec Deus
intelligit se, nisi forte intelligendo alia.


&

さらに、アウグスティヌスが言うように、私たちは精神の点で、神の似姿に向
けてあるのだから、私たちは知性の点でとくに神に似ている。ところが『デ・
アニマ』3巻で言われるように、私たちの知性は他のものを認識するようにし
てのみ自己を認識する。ゆえに神もまた、他のものを知性認識することによっ
てでなければ自己を認識しない。

\\


{\scshape Sed contra est} quod dicitur I {\itshape ad Cor}.~{\scshape
ii}, {\itshape quae sunt Dei, nemo novit nisi spiritus Dei}.

&

しかし反対に、『コリントの信徒への手紙 一』「神にかんする事柄は、神の
霊以外にだれも知らない」\footnote{「神の霊以外に神のことを知る者はいま
せん」(2:11)}と言われている。

\\

{\scshape Respondeo dicendum} quod Deus se per seipsum intelligit. Ad
cuius evidentiam, sciendum est quod, licet in operationibus quae
transeunt in exteriorem effectum, obiectum operationis, quod
significatur ut terminus, sit aliquid extra operantem; tamen in
operationibus quae sunt in operante, obiectum quod significatur ut
terminus operationis, est in ipso operante; et secundum quod est in
eo, sic est operatio in actu.

&

解答する。以下のように言われるべきである。神は、自己によって自己を知性
認識する。これを明らかにするためには、以下のことが知られるべきである。
外の結果へ出て行く働きにおいて、終端として表示される働きの対象は、働く
ものの外にある何かだが、働くものの中にある働きでは、働きの終端として表
示される対象は、働くもの自身の中にある。そして、対象が働くもののうちに
あることにしたがって、働きは現実態にある。


\\


Unde dicitur in libro {\itshape de anima}, quod sensibile in actu est
sensus in actu, et intelligibile in actu est intellectus in actu. Ex
hoc enim aliquid in actu sentimus vel intelligimus, quod intellectus
noster vel sensus informatur in actu per speciem sensibilis vel
intelligibilis. Et secundum hoc tantum sensus vel intellectus aliud
est a sensibili vel intelligibili, quia utrumque est in potentia.

&

このことから、『デ・アニマ』という書物で、現実態にある可感的なものは現
実態にある感覚であり、現実態にある可知的なものは現実態にある知性である、
と言われている。つまり、私たちが何かを現実に感じたり知性認識したりする
のは、私たちの知性や感覚が、可感的なものや可知的なものの形象によって現
実に形成されることによる。そして、感覚や知性が可感的事物や可知的事物と
別のものであるのは、両者が可能態にある場合に限られる。

\\


Cum igitur Deus nihil potentialitatis habeat, sed sit actus purus,
oportet quod in eo intellectus et intellectum sint idem omnibus modis,
ita scilicet, ut neque careat specie intelligibili, sicut intellectus
noster cum intelligit in potentia; neque species intelligibilis sit
aliud a substantia intellectus divini, sicut accidit in intellectu
nostro, cum est actu intelligens; sed ipsa species intelligibilis est
ipse intellectus divinus. Et sic seipsum per seipsum intelligit.

&

ゆえに、神は可能態性に属するものを何ももたず、純粋現実態だから、神にお
いて、知性と知性認識されたものとはあらゆる点で同一でなければならない。
つまり、私たちの知性が可能態において知性認識する場合[=知性認識しうる
が現実には知性認識していない場合]には、可知的形象を欠いているが、神に
はそういうことがないし、また、私たちが現実態において知性認識する場合で
も、可知的形象と知性の実体とが別のものであるが、神の場合にはそういうこ
ともない。そうではなく、可知的形象自体が、神の知性そのものである。この
ようにして、神は自己を自己自身によって知性認識する。

\\


{\scshape Ad primum ergo dicendum} quod {\itshape redire ad essentiam
suam} nihil aliud est quam rem subsistere in seipsa. Forma enim,
inquantum perficit materiam dando ei esse, quodammodo supra ipsam
effunditur, inquantum vero in seipsa habet esse, in seipsam
redit. Virtutes igitur cognoscitivae quae non sunt subsistentes, sed
actus aliquorum organorum, non cognoscunt seipsas; sicut patet in
singulis sensibus. Sed virtutes cognoscitivae per se subsistentes,
cognoscunt seipsas. Et propter hoc dicitur in libro {\itshape de
causis}, quod {\itshape sciens essentiam suam, redit ad essentiam
suam}. Per se autem subsistere maxime convenit Deo. Unde secundum hunc
modum loquendi, ipse est maxime rediens ad essentiam suam, et
cognoscens seipsum.

&


第一異論に対しては、それゆえ、次のように言われるべきである。「自らの本
質へ帰る」とは、事物が自己自身において自存するということに他ならない。
つまり、形相は、質料に存在を与えることによって質料を完成するという点で、
ある意味、質料の上に注がれるが、他方で、自己のうちに存在をもつという点
で、自己自身へ帰る。ゆえに、自存せず、何らかの器官の現実態である認識力
は、個々の感覚において明らかなとおり、自己を認識しない。他方、自らによっ
て自存する認識力は、自己を認識する。このために『原因論』という書物では、
「自らの本質を知る者は自らの本質へ帰る」と言われる。ところで、自らによっ
て自存することは、最大限に神に適合する。ゆえに、このような語り口に従う
かぎり、神は最大限に自らの本質へ帰るものであり、自己を認識するものであ
る。

\\


{\scshape Ad secundum dicendum} quod moveri et pati sumuntur aequivoce
secundum quod intelligere dicitur esse quoddam moveri vel pati, ut
dicitur in III {\itshape de anima}. Non enim intelligere est motus qui
est actus imperfecti, qui est ab alio in aliud, sed actus perfecti,
existens in ipso agente.

&

第二異論に対しては、次のように言われるべきである。『デ・アニマ』3巻で
言われているように、知性認識が、一種の動かされることや受動することであ
ると語られる場合、その「動かされる」や「受動する」は、異義的に[通常と
は異なる意味で]用いられている。じっさい、知性認識は、不完全なものの現
実態としての運動ではない。そのような運動は、あるものから別のものへと運
動する。むしろ、知性認識は、完全なものの現実態としての運動であり、その
ような運動は、作用するものの中に存在する。


\\

Similiter etiam quod intellectus perficiatur ab intelligibili vel
assimiletur ei, hoc convenit intellectui qui quandoque est in
potentia, quia per hoc quod est in potentia, differt ab intelligibili,
et assimilatur ei per speciem intelligibilem, quae est similitudo rei
intellectae; et perficitur per ipsam, sicut potentia per actum. Sed
intellectus divinus, qui nullo modo est in potentia, non perficitur
per intelligibile, neque assimilatur ei, sed est sua perfectio et suum
intelligibile.


&


同様に、知性が可知的なものによって完成され、それへと類似化されるという
ことも、可能態にあるときもあるような知性に適合する。なぜなら、それが可
能態にあることによって、可知的なものと違うものとなり、知性認識された事
物の類似である可知的形象によって、それへと類似化されるからである。[そ
のような知性は]可能態が現実態によって完成されるように、この可知的形象
によって完成される。しかし、神の知性は、どのような点でも可能態にないの
だから、可知的なものによって完成されたり、それに類似化されたりというこ
とはなく、むしろ、自らの完成であり、自らが可知的なものである。


\\


Ad tertium dicendum quod esse naturale non est materiae primae, quae
est in potentia, nisi secundum quod est reducta in actum per
formam. Intellectus autem noster possibilis se habet in ordine
intelligibilium, sicut materia prima in ordine rerum naturalium, eo
quod est in potentia ad intelligibilia, sicut materia prima ad
naturalia. Unde intellectus noster possibilis non potest habere
intelligibilem operationem, nisi inquantum perficitur per speciem
intelligibilem alicuius.

&

第三異論に対しては、次のように言われるべきである。可能態にある第一質料
が自然的存在をもつのは、ただ、それが形相によって現実態へもたらされると
いうことによる。他方、私たちの可能知性は、可知的なものどもの秩序の中で、
自然的事物における第一質料と同じ位置にある。なぜなら、第一質料が自然的
なものに対して可能態にあるように、私たちの可能知性は可知的なものどもに
対して可能態にあるからである。このことから、私たちの可能知性は、何かに
ついての可知的形象を通して完成されないかぎり、可知的な働きを持てない。


\\

Et sic intelligit seipsum per speciem intelligibilem, sicut et alia,
manifestum est enim quod ex eo quod cognoscit intelligibile,
intelligit ipsum suum intelligere, et per actum cognoscit potentiam
intellectivam. Deus autem est sicut actus purus tam in ordine
existentium, quam in ordine intelligibilium, et ideo per seipsum,
seipsum intelligit.


&

だから私たちの知性は、自己を、他のものと同じように、可知的形象を通して
知性認識する。なぜなら、可知的なものを認識することに基づいて、自己が知
性認識することを知性認識し、その働きを通して、知性認識する能力を認識す
ることが明らかだからである。これに対して神は、存在するものどもの秩序に
おいて、可知的なものどもの秩序においてと同様に純粋現実態としてあるから、
自己によって自己を知性認識する。



\end{longtable}
\newpage

\rhead{a.3}
\begin{center}
{\Large {\bf ARTICULUS TERTIUS}}\\ {\large UTRUM DEUS COMPREHENDAT
SEIPSUM}\\ {\footnotesize I {\itshape Sent.}, d.43, q.1, a.1, ad 4;
III, d.14, a.2, qa 1; I {\itshape SCG.}, c.3; III, c.55; {\itshape
De Verit.}, q.2, a.2, ad 5; {\itshape Compend.~Theol.}, c.106.}\\
{\Large 第三項\\神は自己自身を把握するか}
\end{center}

\begin{longtable}{p{21em}p{21em}}
{\huge A}{\scshape d tertium sic proceditur}. Videtur quod Deus non
comprehendat seipsum. Dicit enim Augustinus, in libro {\itshape
octoginta trium quaest}., quod id {\itshape quod comprehendit se,
finitum est sibi}. Sed Deus est omnibus modis infinitus. Ergo non
comprehendit se.

&

第三項の問題へ議論は以下のように進められる。神は自己自身を把握しないと
思われる。理由は以下の通り。アウグスティヌスは『八十三問題集』で「自ら
を把握するものは、自らに限られる」と述べている。しかし、神はあらゆる点
で無限である。ゆえに、自らを把握しない。

\\


2 Si dicatur quod Deus infinitus est nobis, sed sibi finitus, contra,
verius est unumquodque secundum quod est apud Deum, quam secundum quod
est apud nos. Si igitur Deus sibi ipsi est finitus, nobis autem
infinitus, verius est Deum esse finitum, quam infinitum. Quod est
contra prius determinata. Non ergo Deus comprehendit seipsum.


&

もし、神は私たちにとって無限だが、神自身にとっては有限である、と言われ
るならば、それに反対して以下のように言う。あらゆることは、私たちのもと
でそうであるよりも、神のもとでそうであるというほうが真である。ゆえに、
もし神が、神自身にとって有限だが、私たちにとって無限であるとすれば、神
が無限であることよりも、神が有限であることの方が真である。これは、先に
明示されたことに反する。ゆえに、神は自己自身を把握しない。


\\


{\scshape Sed contra} est quod Augustinus dicit ibidem, {\itshape omne
quod intelligit se, comprehendit se}. Sed Deus intelligit se. Ergo
comprehendit se.

&

しかし反対に、アウグスティヌスは同書で「自己を知性認識するものはすべて、
自己を把握する」と述べている。ところが、神は自己を知性認識する。ゆえに、
自己を把握する。


\\


{\scshape Respondeo dicendum} quod Deus perfecte comprehendit
seipsum. Quod sic patet. Tunc enim dicitur aliquid comprehendi, quando
pervenitur ad finem cognitionis ipsius, et hoc est quando res
cognoscitur ita perfecte, sicut cognoscibilis est. Sicut propositio
demonstrabilis comprehenditur, quando scitur per demonstrationem, non
autem quando cognoscitur per aliquam rationem probabilem.

&

解答する。以下のように言われるべきである。神は完全に自己自身を把握する。
これは以下のようにして明らかである。あるものが「把握される」と言われる
のは、それの認識の最終地点にまで到達したときだが、これは、事物が、それ
が認識されうる限度一杯まで完全に認識されたときである。たとえば、論証可
能な命題が把握されるのは、論証によって知られるときであり、何らかの蓋然
的な理由によって認識されるときではない。
%\footnote{このsciturとcognosciturの使い分けがもし意図的に行われているな
% ら興味深い。論証によるのはscientiaであり、蓋然的な理由によるのは
% cognitioである。scientiaではないcognitioというものが考えられていること
% になる。}

\\



Manifestum est autem quod Deus ita perfecte cognoscit seipsum, sicut
perfecte cognoscibilis est. Est enim unumquodque cognoscibile secundum
modum sui actus, non enim cognoscitur aliquid secundum quod in
potentia est, sed secundum quod est in actu, ut dicitur in IX
{\itshape Metaphys}. Tanta est autem virtus Dei in cognoscendo, quanta
est actualitas eius in existendo, quia per hoc quod actu est, et ab
omni materia et potentia separatus, Deus cognoscitivus est, ut
ostensum est. Unde manifestum est quod tantum seipsum cognoscit,
quantum cognoscibilis est. Et propter hoc seipsum perfecte
comprehendit.

&

ところで、神が、自己を、それが完全に認識されうる限度一杯まで完全に認識
していることは明らかである。理由は以下の通り。各々のものは、それの現実
態のあり方に即して認識可能である。『形而上学』9巻で言われるとおり、あ
るものは、可能態にあることによってではなく、現実態にあることによって認
識されるからである。ところで、実在における神の現実性が大きいほど、それ
だけ、認識における神の力も大きい。なぜなら、すでに示されたとおり、現実
態においてあり、あらゆる質料と可能態から離れていることによって、神は認
識しうるものだからである。したがって、神は、それが認識可能である限度一
杯まで、自己自身を認識する。このため、自己を完全に把握する。

\\


{\scshape Ad primum ergo dicendum} quod {\itshape comprehendere}, si
proprie accipiatur, significat aliquid habens et includens alterum. Et
sic oportet quod omne comprehensum sit finitum, sicut omne
inclusum. Non sic autem comprehendi dicitur Deus a seipso, ut
intellectus suus sit aliud quam ipse, et capiat ipsum et includat. Sed
huiusmodi locutiones per negationem sunt exponendae. Sicut enim Deus
dicitur esse in seipso, quia a nullo exteriori continetur; ita dicitur
comprehendi a seipso, quia nihil est sui quod lateat ipsum. Dicit enim
Augustinus, in libro {\itshape de videndo Deum}, quod {\itshape totum
comprehenditur videndo, quod ita videtur, ut nihil eius lateat
videntem}.

&

第一異論に対しては、それゆえ、次のように言われるべきである。「把握する」
ということが厳密に理解されるならば、それは、何かが他のものをもち、かつ
それを含むということを意味する。この意味では、すべて把握されたものは、
含まれたすべてのものがそうであるように、有限である。しかし、神が自己自
身によって把握されると言われるのは、そのように、神の知性が神と異なるも
のであり、それが神を捉え、含む、というような意味ではない。そうではなく、
このような表現は、否定によって説明されるべきである。つまり、ちょうど、
どんな外的なものによっても含まれないために、神が自己自身において在ると
言われるように、神にとって、神を隠すものがなにもないために、自己自身に
よって把握されると言われる。たとえば、アウグスティヌスは、『神を見るこ
とについて』で「それに属するどんなものも、見る人から隠れていないように
見られるものは、それを見るとき、全体が把握される」と言っている。

\\


{\scshape Ad secundum dicendum} quod, cum dicitur {\itshape Deus
finitus est sibi}, intelligendum est secundum quandam similitudinem
proportionis; quia sic se habet in non excedendo intellectum suum,
sicut se habet aliquod finitum in non excedendo intellectum
finitum. Non autem sic dicitur Deus sibi finitus, quod ipse intelligat
se esse aliquid finitum.

&

第二異論に対しては、次のように言われるべきである。「神は自らによって限
られている」と言われるとき、ある種の比例の類似にしたがって理解されるべ
きである。つまり、何らかの有限なものが、有限な理解を超えないように、神
は、自らの理解を超えないという意味でである。しかし、神が、自己を、何か
有限なものとして理解しているという意味で、神が自らによって限られている
と言われるのではない。
%\footnote{
%secundum quandam similitudinem proportionisという表現に注意。{\itshape
% De Verit.}, q.2, a.21のproportionalitasを連想させる。ここでも、
%\[
% \frac{有限な理解を超えない}{有限なもの} = \frac{自らの理解を超えない}{神}
%\]
%という比例性の関係が考えられている。
%}



\end{longtable}
\newpage


\rhead{a.4}
\begin{center}
{\Large {\bf ARTICULUS QUARTUS}}\\ {\large UTRUM IPSUM INTELLIGERE DEI
SIT EIUS SUBSTANTIA}\\ {\footnotesize I {\itshape SCG.}, c.45;
{\itshape Compend.~Theol.}, c.31; XII {\itshape Metaphys.},
lect.11.}\\ {\Large 第四項\\神の知性認識の働きは神の実体か}
\end{center}

\begin{longtable}{p{21em}p{21em}}

{\huge A}{\scshape d quartum sic proceditur}. Videtur quod ipsum
intelligere Dei non sit eius substantia. Intelligere enim est quaedam
operatio. Operatio autem aliquid significat procedens ab
operante. Ergo ipsum intelligere Dei non est ipsa Dei substantia.

&

第四項の問題へ議論は以下のように進められる。神の知性認識の働きは、神の
実体でないと思われる。理由は以下の通り。知性認識の働きは、一種の働きで
ある。ところが、働きは、働くものから発出する何かを意味する。ゆえに、神
の知性認識の働きは、神の実体そのものではない。


\\


{\scshape 2 Praeterea}, cum aliquis intelligit se intelligere, hoc non
est intelligere aliquid magnum vel principale intellectum, sed
intelligere quoddam secundarium et accessorium. Si igitur Deus sit
ipsum intelligere, intelligere Deum erit sicut cum intelligimus
intelligere. Et sic non erit aliquid magnum intelligere Deum.

&

さらに、だれかが、自分が知性認識していることを知性認識するとき、これは、
何か重大なことや、より根源的な認識されたものを認識することではなくて、
ある種の二次的で付加的なことを認識することである。ゆえに、もし神が知性
認識そのものだとすると、神を知性認識することは、私たちが、知性認識する
ことを知性認識するときのようなものであろう。したがって、神を知性認識す
ることは、何ら重大なことではないことになろう。


\\


{\scshape 3 Praeterea}, omne intelligere est aliquid intelligere. Cum
ergo Deus intelligit se, si ipsemet non est aliud quam suum
intelligere, intelligit se intelligere, et intelligere se intelligere
se, et sic in infinitum. Non ergo ipsum intelligere Dei est eius
substantia.

&

さらに、知性認識することは、すべて、何かを知性認識することである。ゆえ
に、神が自らを知性認識するとき、もし、神が自らの知性認識の働きそのもの
であるならば、神は自己を認識することを認識し、自己を認識することを認識
することを認識し、というぐあいに無限に進む。ゆえに、神の知性認識の働き
それ自体が神の実体ではない。
%\footnote{
%「自己=自己を認識すること」
% だとすると、自己を認識する=自己を認識することを認識する=自己を認識す
% ることを認識することを認識する、、、と無限遡行が生じる(魂の中に無限
% の深みがあるとしばしば考えられるのは、こういうところに秘密があるかも知
% れない)。
%他方、「自己=認識」という仮定のもとでは、無限遡行は生じない。自己を
% 認識する=認識を認識する。認識には対象が必要という仮定を
% 入れても、「自己=xを認識すること」、自己を認識する=xを認識する
% ことを認識する、で止まる。
%}

\\


{\scshape Sed contra est} quod dicit Augustinus, Lib.~VII {\itshape de
Trin}., {\itshape Deo hoc est esse, quod sapientem esse}. Hoc autem
est sapientem esse, quod intelligere. Ergo Deo hoc est esse, quod
intelligere. Sed esse Dei est eius substantia, ut supra ostensum
est. Ergo intelligere Dei est eius substantia.

&

しかし反対に、アウグスティヌスは『三位一体論』7巻で「神にとって、
\kenten{ある}とは知者であるということである」と述べている。ところで、
知者であるということは、知性認識するということである。ゆえに、神にとっ
て、\kenten{ある}とは知性認識するということである。しかし、上で示され
たように、神の存在は神の実体である。ゆえに、神の知性認識の働きは、神の
実体である。


\\


{\scshape Respondeo dicendum} quod est necesse dicere quod intelligere
Dei est eius substantia. Nam si intelligere Dei sit aliud quam eius
substantia, oporteret, ut dicit philosophus in XII {\itshape
Metaphys}., quod aliquid aliud esset actus et perfectio substantiae
divinae, ad quod se haberet substantia divina sicut potentia ad actum
(quod est omnino impossibile), nam intelligere est perfectio et actus
intelligentis.

&

解答する。以下のように言われるべきである。神の知性認識の働きは、神の実
体だと言うことが必要である。もし、神の知性認識の働きが、神の実体と異な
るとすると、哲学者が『形而上学』12巻で言うように、神の実体とは違う何か
他のものが現実態であり完成であり、神の実体は、その何かに対して可能態が
現実態に関係するように関係することになったであろう(これはまったく不可
能である)。なぜなら、知性認識の働きは、知性認識する者の完成であり現実
態だからである。

\\



Hoc autem qualiter sit, considerandum est. Sicut enim supra dictum
est, intelligere non est actio progrediens ad aliquid extrinsecum, sed
manet in operante sicut actus et perfectio eius, prout esse est
perfectio existentis, sicut enim esse consequitur formam, ita
intelligere sequitur speciem intelligibilem.

&

しかし、これ[=知性認識が実体だということ]がどのようにしてそうなのか
が考察されるべきである。上で述べられたとおり、知性認識する働きは、何か
外のものへと出て行く作用ではなく、ちょうど、存在することが、存在するも
のの完成であるようなかたちで、働くもののうちに、それの現実態や完成とし
て留まる。じっさい、存在することは形相に伴うように、知性認識することは
可知的形象に伴う。


\\

In Deo autem non est forma quae sit aliud quam suum esse, ut supra
ostensum est.  Unde, cum ipsa sua essentia sit etiam species
intelligibilis, ut dictum est, ex necessitate sequitur quod ipsum eius
intelligere sit eius essentia et eius esse. Et sic patet ex omnibus
praemissis quod in Deo intellectus, et id quod intelligitur, et
species intelligibilis, et ipsum intelligere, sunt omnino unum et
idem. Unde patet quod per hoc quod Deus dicitur intelligens, nulla
multiplicitas ponitur in eius substantia.

&

ところが、上で示されたように、神において、その存在と別であるような形相
はない。したがって、すでに述べられたとおり、神の本質それ自身が、可知的
形象でもあるので、必然的に、神の知性認識の働き自体が、神の本質であり、
神の存在である。このようにして、すでに論じられたすべてのことから、神に
おいて、知性、知性認識されるもの、可知的形象、そして知性認識の働きそれ
自体が、あらゆる点で一つで同一であることが明らかである。したがって、神
が知性認識する者と言われることによって、神の実体のなかに、どんな多数性
も想定されていないことが明らかである。

\\


{\scshape Ad primum ergo dicendum} quod intelligere non est operatio
exiens ab ipso operante, sed manens in ipso.

&

第一異論に対しては、それゆえ、次のように言われるべきである。知性認識す
る働きは、働くものから出て行くのではなく、そのうちに留まる。


\\


{\scshape Ad secundum dicendum} quod, cum intelligitur illud
intelligere quod non est subsistens, non intelligitur aliquid magnum;
sicut cum intelligimus intelligere nostrum. Et ideo non est simile de
ipso intelligere divino, quod est subsistens.

&

第二異論に対しては、次のように言われるべきである。自存するのではない知
性認識の働きが知性認識されるとき、何か偉大なことが認識されているのでは
ない。ちょうど、私たちが、私たちの知性認識の働きを認識するときのように。
ゆえに、自存するものである神の知性認識についてと同様ではない。


\\


Et per hoc patet responsio ad tertium. Nam intelligere divinum, quod
est in seipso subsistens, est sui ipsius; et non alicuius alterius, ut
sic oporteat procedere in infinitum.

&

これによって、第三異論に対する解答は明らかである。すなわち、神の「知性
認識すること」は、そのこと自体において自存するが、それは神自身に属する
のであって、神以外の何か他のものに属するのではない。もしこの後者だった
ら、その場合には、無限に進まなければならなかっただろう。


\end{longtable}
\newpage



\rhead{a.5}
\begin{center}
{\Large {\bf ARTICULUS QUINTUS}}\\ {\large UTRUM DEUS COGNOSCIT ALIA A
SE}\\ {\footnotesize I {\itshape Sent.}, d.35, a.2; I {\itshape
SCG.}, c.48, 49; {\itshape De Verit.}, q.2, a.3; {\itshape
Compend.~Theol.}, c.30; XII {\itshape Metaphys.}, lect.11; {\itshape
De Causis}, lect.13.}\\ {\Large 第五項\\神は自己以外のものを認識する
か}
\end{center}

\begin{longtable}{p{21em}p{21em}}

{\huge A}{\scshape d quintum sic proceditur}. Videtur quod Deus non
cognoscat alia a se. Quaecumque enim sunt alia a Deo, sunt extra
ipsum. Sed Augustinus dicit, in libro {\itshape Octoginta trium Qu}.,
quod {\itshape neque quidquam Deus extra seipsum intuetur}. Ergo non
cognoscit alia a se.


&

第五項の問題へ議論は以下のように進められる。神は自己以外を認識しないと
思われる。理由は以下の通り。なんであれ神でないものは、神の外にある。と
ころが、アウグスティヌスは『八十三問題集』で「神は自己の外に何も見ない」
と言っている。ゆえに、神は自己以外のものを認識しない。

\\


{\scshape 2 Praeterea}, intellectum est perfectio intelligentis. Si
ergo Deus intelligat alia a se, aliquid aliud erit perfectio Dei, et
nobilius ipso. Quod est impossibile.

&

さらに、知性認識されたものは、知性認識するものの完成である。ゆえに、も
し神が自己以外のものを知性認識するならば、神以外の何かが神の完成であり、
神よりも高貴であることになる。これは不可能である。


\\


{\scshape 3 Praeterea}, ipsum intelligere speciem habet ab
intelligibili, sicut et omnis alius actus a suo obiecto, unde et ipsum
intelligere tanto est nobilius, quanto etiam nobilius est ipsum quod
intelligitur. Sed Deus est ipsum suum intelligere, ut ex dictis
patet. Si igitur Deus intelligit aliquid aliud a se, ipse Deus
specificatur per aliquid aliud a se, quod est impossibile. Non igitur
intelligit alia a se.


&

さらに、知性認識自体は、[対象である]可知的なものから形象(species:種、
かたち)をもつ。これは、他のすべての作用が、自らの対象から形象をもつの
と同様である。したがって、知性認識自体は、知性認識されるものが高貴であ
るほど、それだけいっそう高貴である。ところが、すでに述べられたことから
明らかなとおり、神は、知性認識それ自体である。ゆえに、もし神が自己以外
のものを知性認識するならば、神自身が、自己以外の何かによって、なんらか
の形象に限定されることになるが、これは不可能である。ゆえに、神は自己以
外のものを知性認識しない。


\\


{\scshape Sed contra est} quod dicitur {\itshape Hebr}.~{\scshape iv},
{\itshape omnia nuda et aperta sunt oculis eius}.


&

しかし反対に、『ヘブライ人への手紙』4章で「彼の目にとって、万物が裸で
露わである」と言われている。\footnote{「すべてのものが神の目には裸であ
り、さらけ出されているのです。」(4:13)}

\\


{\scshape Respondeo dicendum} quod necesse est Deum cognoscere alia a
se. Manifestum est enim quod seipsum perfecte intelligit, alioquin
suum esse non esset perfectum, cum suum esse sit suum intelligere.


&

解答する。以下のように言われるべきである。神が自己以外のものを認識する
ことは必然である。神の存在は、神の知性認識そのものだから、神は自らを完
全に知性認識することが明らかである。もしそうでなかったら、その存在は完
全ではなかったであろうから。

\\

Si autem perfecte aliquid cognoscitur, necesse est quod virtus eius
perfecte cognoscatur.  Virtus autem alicuius rei perfecte cognosci non
potest, nisi cognoscantur ea ad quae virtus se extendit. Unde, cum
virtus divina se extendat ad alia, eo quod ipsa est prima causa
effectiva omnium entium, ut ex supradictis patet; necesse est quod
Deus alia a se cognoscat.


&

ところで、なにかが完全に認識されるときには、その力もまた、完全に認識さ
れる。そして、その力がどこまで及ぶかが認識されなければ、ある事物の力が
完全に認識されることはない。したがって、上で述べられたことから明らかな
とおり、神の力は万有の第一作出因なのだから、神の力は神以外のものまで及
ぶ。ゆえに、神が自己以外のものを認識することは必然である。


\\

Et hoc etiam evidentius fit, si adiungatur quod ipsum esse causae
agentis primae, scilicet Dei, est eius intelligere. Unde quicumque
effectus praeexistunt in Deo sicut in causa prima, necesse est quod
sint in ipso eius intelligere; et quod omnia in eo sint secundum modum
intelligibilem, nam omne quod est in altero, est in eo secundum modum
eius in quo est.

&

そしてこのことは、もし、第一作用因、すなわち神の「存在すること」が、そ
の「知性認識すること」であることが考え合わされるならば、いっそう明らか
となる。つまり、第一原因としての神の中に先在するどんな結果も、必然的に、
神の知性認識そのものの中にあることになるし、神の知性認識の中にあるすべ
てのものは、可知的なかたちでそこにある。なぜなら、すべてXの中にあるも
のは、Xのあり方にしたがって、Xのなかにあるからである。

\\


Ad sciendum autem qualiter alia a se cognoscat, considerandum est quod
dupliciter aliquid cognoscitur, uno modo, in seipso; alio modo, in
altero. In seipso quidem cognoscitur aliquid, quando cognoscitur per
speciem propriam adaequatam ipsi cognoscibili, sicut cum oculus videt
hominem per speciem hominis.


&

ところで、神が自己以外のものをどのように認識するかが知られるには、ある
ものが認識されるのに二通りのかたちがあることが考察されるべきである。す
なわち、あるものは、一つには、それ自体において、もう一つには、他者にお
いて認識される。何かがそれ自体において認識されるのは、認識されうるもの
に対等した固有の形象を通して認識される場合であり、たとえば、目が、人間
の形象を通して人間を見るときがそれである。


\\


In alio autem videtur id quod videtur per speciem continentis, sicut
cum pars videtur in toto per speciem totius, vel cum homo videtur in
speculo per speciem speculi, vel quocumque alio modo contingat aliquid
in alio videri.



&

これに対して、それを含むものの形象を通して見られるものは、他者において
見られる。たとえば、部分が全体において、全体の形象を通して見られるとき
や、人間が、鏡において、鏡の形象を通して見られるとき、あるいは、その他
どのようなかたちであれ、何かが他者において見られる場合である。

\\

Sic igitur dicendum est quod Deus seipsum videt in seipso, quia
seipsum videt per essentiam suam. Alia autem a se videt non in ipsis,
sed in seipso, inquantum essentia sua continet similitudinem aliorum
ab ipso.

&

以上のことを踏まえて次のように言われるべきである。すなわち、神は自己自
身を、自己自身において見る。自己自身を、自己の本質を通して見るからであ
る。他方、自己以外のものを、それ自体においてではなく、自己の本質が、自
己以外のものの類似を含んでいるという点で、自己自身において見る。



\\


{\scshape Ad primum ergo dicendum} quod verbum Augustini dicentis quod
Deus nihil extra se intuetur, non est sic intelligendum, quasi nihil
quod sit extra se intueatur, sed quia id quod est extra seipsum, non
intuetur nisi in seipso, ut dictum est.


&

第一異論に対しては、それゆえ、次のように言わなければならない。神は自己
の外に何も見ないというアウグスティヌスの言葉は、自己の外にあるものを何
も見ないという意味ではなく、自己の外にあるものを、すでに述べられたよう
に、自己において以外のかたちでは見ないという意味で理解されるべきである。

\\

{\scshape Ad secundum dicendum} quod intellectum est perfectio
intelligentis non quidem secundum suam substantiam, sed secundum suam
speciem, secundum quam est in intellectu, ut forma et perfectio eius,
lapis enim non est in anima, sed species eius, ut dicitur in III
{\itshape de anima}. Ea vero quae sunt alia a Deo, intelliguntur a Deo
inquantum essentia Dei continet species eorum, ut dictum est. Unde non
sequitur quod aliquid aliud sit perfectio divini intellectus, quam
ipsa essentia Dei.


&

第二異論に対しては、次のように言われるべきである。『デ・アニマ』3巻で
言われるように、石ではなく、石の形象が魂の中にあるのだから、知性認識さ
れたものが知性認識するものの完成であるのは、知性認識されたものの実体に
即してではなく、その形象に即してである。この形象にしたがって、知性認識
されたものは、知性認識するものの形相ないし完成として、知性のうちにある。
他方、すでに述べられたとおり、神以外のものは、神によって、神の本質がそ
れらの形象を含むという点で、神によって知性認識される。したがって、神の
本質以外の何か他のものが、神の知性の完成であるということにはならない。


\\


{\scshape Ad tertium dicendum} quod ipsum intelligere non specificatur
per id quod in alio intelligitur, sed per principale intellectum, in
quo alia intelliguntur. Intantum enim ipsum intelligere specificatur
per obiectum suum, inquantum forma intelligibilis est principium
intellectualis operationis, nam omnis operatio specificatur per formam
quae est principium operationis, sicut calefactio per calorem. Unde
per illam formam intelligibilem specificatur intellectualis operatio,
quae facit intellectum in actu. Et haec est species principalis
intellecti, quae in Deo nihil est aliud quam essentia sua, in qua
omnes species rerum comprehenduntur. Unde non oportet quod ipsum
intelligere divinum, vel potius ipse Deus, specificetur per aliud quam
per essentiam divinam.


&

第三異論に対しては、次のように言われるべきである。知性認識の働きは、他
者において知性認識されているものによってではなく、そこにおいて他のもの
が認識されている主要な知性認識対象によって、種に限定される[≒特定の性
格を与えられる]。なぜなら、知性認識の働きがその対象によって種へ限定さ
れるのは、可知的な形相が知性的な働きの根源であるという点に依るからであ
る。じっさい、たとえば、熱する働きが熱によって種へ限定されるように、す
べての働きは、その働きの根源である形相によって種へと限定される。したがっ
て、知性を現実態にする可知的な形相によって、知性的な働きが種へと限定さ
れるが、そのような形相とは主要な知性認識対象の形象であり、神において、
それは、事物のすべての形象がそこに含まれている神の本質に他ならない。し
たがって、神の知性認識それ自体、いやむしろ神自身が、神の本質以外の他の
ものによって種へと限定されるのではない。


\end{longtable}
\newpage


\rhead{a.6}
\begin{center}
{\Large {\bf ARTICULUS SEXTUS}}\\ {\large UTRUM DEUS COGNOSCAT ALIA A
SE PROPRIA COGNITIONE}\\ {\footnotesize I {\itshape Sent.}, d.35,
a.3; I {\itshape SCG.}, c.50; {\itshape De Pot.}, q.6, a.1;
{\itshape De Verit.}, q.2, a.4; {\itshape De Causis}, lect.10.}\\
{\Large 第六項\\神は自己以外のものを固有の認識によって認識するか}
\end{center}

\begin{longtable}{p{21em}p{21em}}
{\huge A}{\scshape d sextum sic proceditur}. Videtur quod Deus non
cognoscat alia a se propria cognitione. Sic enim cognoscit alia a se,
ut dictum est, secundum quod alia ab ipso in eo sunt. Sed alia ab eo
sunt in ipso sicut in prima causa communi et universali. Ergo et alia
cognoscuntur a Deo, sicut in causa prima et universali. Hoc autem est
cognoscere in universali, et non secundum propriam cognitionem. Ergo
Deus cognoscit alia a se in universali, et non secundum propriam
cognitionem.

&

第六項の問題へ議論は以下のように進められる。神は自己以外のものを固有の
認識によって認識するのではないと思われる。理由は以下の通り。すでに述べ
られたとおり、神は自己以外のものを、自己以外のものが神においてあるよう
な、そのようなかたちで認識する。ところが、神以外のものは、共通で普遍的
な原因としての神の中にある。ゆえに、他のものは神によって、第一で普遍的
な原因の中にあるものとして認識される。ところが、これは、普遍において認
識することであり、固有の認識によって認識することではない。ゆえに、神は、
自己以外のものを、普遍において認識するのであり、固有の認識というかたち
で認識するのではない。

\\


{\scshape 2 Praeterea}, quantum distat essentia creaturae ab essentia
divina, tantum distat essentia divina ab essentia creaturae. Sed per
essentiam creaturae non potest cognosci essentia divina, ut supra
dictum est. Ergo nec per essentiam divinam potest cognosci essentia
creaturae. Et sic, cum Deus nihil cognoscat nisi per essentiam suam,
sequitur quod non cognoscat creaturam secundum eius essentiam, ut
cognoscat de ea quid est, quod est propriam cognitionem de re habere.


&

被造物の本質が神の本質から隔たっているだけ、神の本質は被造物の本質から
隔たる。ところが、上で述べられたとおり、被造物の本質によって、神の本質
は認識されえない。ゆえに、神の本質によっても被造物の本質は認識されえな
い。このようにして、神は自己の本質によってのみ認識を行うので、事物につ
いて固有の認識である、「何であるか」を事物について知るようなかたちでは、
自己の本質によって被造物を認識しないことが帰結する。

\\


{\scshape 3 Praeterea}, propria cognitio non habetur de re, nisi per
propriam eius rationem. Sed cum Deus cognoscat omnia per essentiam
suam, non videtur quod unumquodque per propriam rationem cognoscat,
idem enim non potest esse propria ratio multorum et diversorum. Non
ergo habet propriam cognitionem Deus de rebus, sed communem, nam
cognoscere res non secundum propriam rationem, est cognoscere res
solum in communi.

&

さらに、事物についての固有の認識は、その事物の固有の性格(ratio)によっ
てでなければ所有されない。ところが、神は万物を自己の本質によって認識す
るので、各々のものを固有の性格によって認識するようには思われない。なぜ
なら、同じものが、多くのさまざまなものの固有の性格ではありえないからで
ある。ゆえに、神は事物について、固有の認識を持たず、むしろ、共通の認識
を持つ。事物を固有の性格においてではないかたちで認識するのは、事物をた
だ共通的に認識することだからである。

\\


{\scshape Sed contra}, habere propriam cognitionem de rebus, est
cognoscere res non solum in communi, sed secundum quod sunt ab invicem
distinctae. Sic autem Deus cognoscit res. Unde dicitur {\itshape
Heb}.~{\scshape iv}, quod {\itshape pertingit usque ad divisionem
spiritus et animae, compagum quoque et medullarum; et discretor
cogitationum et intentionum cordis; et non est ulla creatura
invisibilis in conspectu eius}.


&

しかし反対に、事物について固有の認識をもつことは、事物を共通的にだけで
なく、相互に区別されたかたちで認識することである。ところが、神はこのよ
うなかたちで事物を認識する。それゆえ、『ヘブライ人への手紙』4章で、次
のように言われている。「霊と魂の区別、関節と髄の区別にまで達し、心の思
いと意図を区別する者であり、彼の目には、どんな被造物も見えないものはな
い」。\footnote{「精神と霊、関節と骨髄とを切り離すほどに刺し通して、心
の思いや考えを見分けることができるからです。さらに、神の御前では隠れた
被造物は一つもなく、」(4:12-13)}


\\


{\scshape Respondeo dicendum} quod circa hoc quidam erraverunt,
dicentes quod Deus alia a se non cognoscit nisi in communi, scilicet
inquantum sunt entia. Sicut enim ignis, si cognosceret seipsum ut est
principium caloris, cognosceret naturam caloris, et omnia alia
inquantum sunt calida; ita Deus, inquantum cognoscit se ut principium
essendi, cognoscit naturam entis, et omnia alia inquantum sunt entia.

&

解答する。以下のように言われるべきである。この問題をめぐって、ある人々
は、次のように言って誤った。すなわち、神は、自己以外のものを、共通的に、
つまりそれが有であるという点でしか認識しない。ちょうど、もしも火が、自
己自身を熱の根源として認識したとすれば、熱の本性と、熱いものであるとい
う点から他のすべてのものを、認識しただろうように、そのように神も、自己
を存在の根源として認識することで、有の本性と、有であるという点から他の
すべてのものを、認識する、と。

\\



Sed hoc non potest esse. Nam intelligere aliquid in communi, et non in
speciali, est imperfecte aliquid cognoscere. Unde intellectus noster,
dum de potentia in actum reducitur, pertingit prius ad cognitionem
universalem et confusam de rebus, quam ad propriam rerum cognitionem,
sicut de imperfecto ad perfectum procedens, ut patet in I {\itshape
Physic}.

&

しかしこれはありえない。なぜなら、何かを特殊的にではなく共通的に知性認
識することは、それを不完全に認識することである。このことから、『自然学』
1巻で明らかなとおり、私たちの知性は、可能態から現実態へ引き出されると
き、ちょうど不完全なものから完全なものへと進んでいくように、事物の固有
の認識に到達するより先に、事物についての普遍的で混乱した認識へ達する。

\\

Si igitur cognitio Dei de rebus aliis a se, esset in universali
tantum, et non in speciali, sequeretur quod eius intelligere non esset
omnibus modis perfectum, et per consequens nec eius esse, quod est
contra ea quae superius ostensa sunt. Oportet igitur dicere quod alia
a se cognoscat propria cognitione; non solum secundum quod communicant
in ratione entis, sed secundum quod unum ab alio distinguitur.

&

ゆえに、もし、自己以外の事物についての神の認識が、たんに普遍的なもので、
特殊的でなかったならば、神の知性認識は、そして結果的に神の存在は、あら
ゆる点で完全というわけではないことになっただろう。これは、先に示された
ことに反する。ゆえに、神は自己以外のものを、ただ有の性格において共通す
るという点だけでではなく、それぞれがひとつひとつ区別されるというかたち
で、固有の認識によって認識すると言わなければならない。

\\


Et ad huius evidentiam, considerandum est quod quidam, volentes
ostendere quod Deus per unum cognoscit multa, utuntur quibusdam
exemplis, ut puta quod, si centrum cognosceret seipsum, cognosceret
omnes lineas progredientes a centro; vel lux, si cognosceret seipsam,
cognosceret omnes colores. Sed haec exempla, licet quantum ad aliquid
similia sint, scilicet quantum ad universalem causalitatem; tamen
deficiunt quantum ad hoc, quod multitudo et diversitas non causantur
ab illo uno principio universali, quantum ad id quod principium
distinctionis est, sed solum quantum ad id in quo communicant.

&

これを明らかにするために、以下のことが考察されるべきである。ある人々は、
神が一によって多を認識することを示そうとして、ある種の例を用いた。たと
えば、もし中心が自己を認識したとすると、中心から出て行くすべての線を認
識しただろう、とか、もし光が自己を認識したならば、すべての色を認識した
だろう、などのように。しかし、たしかにこれらの例は、ある点で、すなわち、
普遍的原因性という点では[正解に]類似しているが、しかし別の点では欠陥
がある。つまり、多性や多様性が、一つの普遍的な根源が\kenten{原因となっ
て生み出される}というのは、[その根源が]区別の根源だというのではなく、
たんにその根源において共通しているという意味にすぎない。


\\


Non enim diversitas colorum causatur ex luce solum, sed ex diversa
dispositione diaphani recipientis, et similiter diversitas linearum ex
diverso situ. Et inde est quod huiusmodi diversitas et multitudo non
potest cognosci in suo principio secundum propriam cognitionem, sed
solum in communi.

&

たとえば、色の多様性は、光だけが原因ではなく、[光を]受け取る透明体の
さまざまな状態も原因となるし、また同様に、線の多様性は、[中心だけでな
く]多様な位置が原因となる。したがって、このような多様性と多性は、その
根源において、固有の認識というかたちでは認識されず、たんに共通的に認識
されるのみである。

\\

Sed in Deo non sic est. Supra enim ostensum est quod quidquid
perfectionis est in quacumque creatura, totum praeexistit et
continetur in Deo secundum modum excellentem. Non solum autem id in
quo creaturae communicant, scilicet ipsum esse, ad perfectionem
pertinet; sed etiam ea per quae creaturae ad invicem distinguuntur,
sicut vivere, et intelligere, et huiusmodi, quibus viventia a non
viventibus, et intelligentia a non intelligentibus distinguuntur.

&

しかし、神において、こうはならない。というのも、どんな被造物の中にある
どんな完全性も、その全体が、卓越したかたちで神の中に先在し包含されてい
るということが、上で示された。ところで、被造物が共通する点、つまり、
「存在すること」それ自体が完全性に属するだけでなく、被造物が相互に区別
されるもの、つまり、「生きること」によって、生きるものが生きていないも
のから区別され、「知性認識すること」によって、知性認識するものが知性認
識しないものから区別されるが、このような、「生きること」「知性認識する
こと」もまた、完全性に属する。

\\

Et omnis forma, per quam quaelibet res in propria specie constituitur,
perfectio quaedam est. Et sic omnia in Deo praeexistunt, non solum
quantum ad id quod commune est omnibus, sed etiam quantum ad ea
secundum quae res distinguuntur.

&

そして、各々の事物は、その形相によって固有の種に位置づけられるが、すべ
ての形相も、一種の完全性である。このような意味で、つまり、たんに万物に
共通する点においてだけでなく、事物が区別される根拠という点でも、万物は
神の中に先在する。


\\

Et sic, cum Deus in se omnes perfectiones contineat, comparatur Dei
essentia ad omnes rerum essentias, non sicut commune ad propria, ut
unitas ad numeros, vel centrum ad lineas; sed sicut perfectus actus ad
imperfectos, ut si dicerem, homo ad animal, vel senarius, qui est
numerus perfectus, ad numeros imperfectos sub ipso contentos.

&

このように、神は自己のうちにすべての完全性を含むので、神の本質は、諸事
物のすべての本質に対して、一が多に、あるいは、中心が線に関係するような
かたちで、共通なものが固有なものに対して関係するのではなく、むしろ、完
全な現実態が、不完全な現実態に対するように関係する。言ってみれば、それ
は、人間が動物に対する関係であり、完全数である6が、その中に含まれてい
る不完全な数に対するような関係である。

\\



Manifestum est autem quod per actum perfectum cognosci possunt actus
imperfecti, non solum in communi, sed etiam propria cognitione. Sicut
qui cognoscit hominem, cognoscit animal propria cognitione, et qui
cognoscit senarium, cognoscit trinarium propria cognitione.

&

ところで、完全な現実態によって、不完全な現実態が、共通的にだけでなく、
固有の認識によっても認識されうることは明らかである。たとえば、人間を認
識する人は、動物を固有の認識によって認識するし、6を認識する人は、固有
の認識によって3を認識する。

\\


Sic igitur, cum essentia Dei habeat in se quidquid perfectionis habet
essentia cuiuscumque rei alterius, et adhuc amplius, Deus in seipso
potest omnia propria cognitione cognoscere. Propria enim natura
uniuscuiusque consistit, secundum quod per aliquem modum divinam
perfectionem participat.

&

ゆえに、このようにして、神の本質は、自己のうちに、他のどんな事物の本質
がもつどんな完全性をも、またさらにそれ以上のものも持つので、神は、自己
において、万物を固有の認識によって認識することができる。じっさい、各々
のものの固有の本性は、何らかのかたちで、神の完全性を分有する点に成り立
つ。


\\


Non autem Deus perfecte seipsum cognosceret, nisi cognosceret
quomodocumque participabilis est ab aliis sua perfectio, nec etiam
ipsam naturam essendi perfecte sciret, nisi cognosceret omnes modos
essendi. Unde manifestum est quod Deus cognoscit omnes res propria
cognitione, secundum quod ab aliis distinguuntur.


&

さらにまた、もし神が、自己の完全性が他のものによって分有されうるあらゆ
るかたちを認識しなかったりすれば、神は自己を完全に認識しないことになる
し、あらゆる存在のあり方を認識しなかったりすれば、存在の本性をも完全に
認識しないことになるだろう。[しかしそれは不可能である。]したがって、
神がすべての事物を固有の認識によって、それらが他のものから区別されてい
る点で認識することは明らかである。


\\


{\scshape Ad primum ergo dicendum} quod sic cognoscere aliquid sicut
in cognoscente est, potest dupliciter intelligi. Uno modo, secundum
quod hoc adverbium sic importat modum cognitionis ex parte rei
cognitae. Et sic falsum est. Non enim semper cognoscens cognoscit
cognitum secundum illud esse quod habet in cognoscente, oculus enim
non cognoscit lapidem secundum esse quod habet in oculo; sed per
speciem lapidis quam habet in se, cognoscit lapidem secundum esse quod
habet extra oculum.

&

第一異論に対しては、それゆえ、次のように言われるべきである。「あるもの
を、それが認識者の中にある、ちょうどそのようなかたちで認識する」という
のは二通りに理解できる。一つには、この「そのようなかたちで」(sic)とい
う副詞が認識のあり方を認識される事物の側から意味する場合であり、その
場合には偽である。認識者は、常に、認識されるものを、それが認識者の中で
もつ存在にしたがって認識するとは限らないからである。たとえば、目は、石
を、石が目の中でもつ存在に即して認識するわけではない。むしろ目は、石が
自分自身の中にもつ石の形象を通して、石が目の外にもつ存在に即して石を認
識する。


\\



Et si aliquis cognoscens cognoscat cognitum secundum esse quod habet
in cognoscente, nihilominus cognoscit ipsum secundum esse quod habet
extra cognoscentem, sicut intellectus cognoscit lapidem secundum esse
intelligibile quod habet in intellectu, inquantum cognoscit se
intelligere; sed nihilominus cognoscit esse lapidis in propria natura.


&

もし、ある認識者が、認識されるものを、認識者の中でもつ存在に即して認識
するとしても、やはりその場合にも、認識者の外でもつ存在に即してそれを認
識する。たとえば、知性は、自らが[石を]知性認識していることを認識する
ことで、石を、石が知性の中でもつ可知的な存在に即して認識するが、しかし
やはり、[知性は、]固有の本性における石の存在を認識する。

\\



Si vero intelligatur secundum quod hoc adverbium sic importat modum ex
parte cognoscentis, verum est quod sic solum cognoscens cognoscit
cognitum, secundum quod est in cognoscente, quia quanto perfectius est
cognitum in cognoscente, tanto perfectior est modus cognitionis. Sic
igitur dicendum est quod Deus non solum cognoscit res esse in seipso;
sed per id quod in seipso continet res, cognoscit eas in propria
natura; et tanto perfectius, quanto perfectius est unumquodque in
ipso.

&

他方、「そのようなかたちで」というこの副詞が、認識者の側から(認識の)
あり方を意味している場合には、認識者が認識されるものを、それが認識者の
中にあることに即してのみ認識するというのは真である。なぜなら、認識され
るものが認識者の中に、より完全にあればあるほど、認識のあり方は完全だか
らである。ゆえに、このようにして、神は、諸事物がそれ自体においてあるこ
とを認識するだけでなく、自己自身の中に諸事物を含んでいることによって、
それらの事物を、固有の本性において認識すると言われるべきである。そして、
各々のものが神の中により完全にあればあるほど、より完全に[認識される]。

\\


{\scshape Ad secundum dicendum} quod essentia creaturae comparatur ad
essentiam Dei, ut actus imperfectus ad perfectum. Et ideo essentia
creaturae non sufficienter ducit in cognitionem essentiae divinae, sed
e converso.


&

第二異論に対しては以下のように言われるべきである。被造物の本質が神の本
質に対してもつ関係は、不完全な現実態が完全な現実態に対してもつ関係であ
る。ゆえに、被造物の本質は、神の本質の認識へと十分に導かないが、逆はそ
うではない。

\\


{\scshape Ad tertium dicendum} quod idem non potest accipi ut ratio
diversorum per modum adaequationis. Sed divina essentia est aliquid
excedens omnes creaturas. Unde potest accipi ut propria ratio
uniuscuiusque, secundum quod diversimode est participabilis vel
imitabilis a diversis creaturis.

&

第三異論に対しては、次のように言われるべきである。同一のものが、対等と
いうかたちで、異なるものどもの性格として理解されることは不可能である。
しかし、神の本質は、全ての被造物を越えている何かである。したがって、
[神の本質は、]さまざまな被造物によってさまざまなかたちで分有されうる、
あるいは模倣されうるという点で、各々のものの固有の性格として理解されう
る。

\\

\end{longtable}
\newpage



\rhead{a.7}
\begin{center}
{\Large {\bf ARTICULUS SEPTIMUS}}\\ {\large UTRUM SCIENTIA DEI SIT
DISCURSIVA}\\ {\footnotesize Infra, q.85, a.5; I {\itshape SCG.},
c.55, 57; {\itshape De Verit.}, q.2, a.1, ad 4, 5; a.3, ad 3;
a.13; {\itshape Compend.~Theol.}, c.29; in {\itshape Iob.}, c.12,
lect.2.}\\ {\Large 第七項\\神の知は推論的か}
\end{center}

\begin{longtable}{p{21em}p{21em}}


{\huge A}{\scshape d septimum sic proceditur}. Videtur quod scientia
Dei sit discursiva. Scientia enim Dei non est secundum scire in
habitu, sed secundum intelligere in actu. Sed secundum philosophum, in
II {\itshape Topic}., scire in habitu contingit multa simul, sed
intelligere actu unum tantum. Cum ergo Deus multa cognoscat, quia et
se et alia, ut ostensum est, videtur quod non simul omnia intelligat,
sed de uno in aliud discurrat.

&


第七項の問題へ議論は以下のように進められる。神の知は推論的であると思わ
れる。理由は以下の通り。神の知は、習態において知るというのではなく、現
実態において知性認識するというかたちをとる。ところが、『トピカ』2巻の
哲学者によれば、習態において、多くのものを同時に知ることはありうるが、
現実に知性認識するのはただ一つである。ゆえに、神は多くのものを認識する
のだから、というのも、すでに示されたとおり、自己と、自己以外のものを認
識するのだから、すべてのものを同時に知性認識するのではなく、一つのもの
からもう一つのものへと推論すると思われる。

\\


{\scshape 2 Praeterea}, cognoscere effectum per causam est scire
discurrentis. Sed Deus cognoscit alia per seipsum, sicut effectum per
causam. Ergo cognitio sua est discursiva.

&

さらに、原因によって結果を認識することは、推論するものに属する知り方で
ある。ところが、神は、原因を通して結果を認識するようなかたちで、自己を
通して他のものを認識する。ゆえに、神の認識は推論的である。

\\


{\scshape 3 Praeterea}, perfectius Deus scit unamquamque creaturam
quam nos sciamus. Sed nos in causis creatis cognoscimus earum
effectus, et sic de causis ad causata discurrimus. Ergo videtur
similiter esse in Deo.

&

さらに、神は、私たちが各々の被造物を知るよりも完全にそれらを知る。とこ
ろが、私たちは、創造された原因の中にそれらの結果を知り、この意味で、原
因からその結果へと推論する。ゆえに、神においても同様であると思われる。

\\


{\scshape Sed contra est} quod Augustinus dicit, in XV {\itshape de
Trin}., quod {\itshape Deus non particulatim vel singillatim omnia
videt, velut alternante conspectu hinc illuc, et inde huc; sed omnia
videt simul}.

&

しかし反対に、アウグスティヌスは、『三位一体論』15巻で「神は、ここから
あそこ、そしてそこへと視線を行き来させるようにして、個別的に、あるいは
個的に万物を見るのではなく、万物を同時に見る」と言っている。

\\


{\scshape Respondeo dicendum} quod in scientia divina nullus est
discursus. Quod sic patet. In scientia enim nostra duplex est
discursus. Unus secundum successionem tantum, sicut cum, postquam
intelligimus aliquid in actu, convertimus nos ad intelligendum aliud.
Alius discursus est secundum causalitatem, sicut cum per principia
pervenimus in cognitionem conclusionum.

&


解答する。以下のように言われるべきである。神の知の中に、推論的なものは
なにもない。これは次のようにして明らかである。私たちの知の中には、二通
りの推論がある。一つは、たんに継起によるもので、たとえば、私たちが何か
を現実に知性認識した後に、別のものを知性認識するために自分を向き変える
ときがそれである。もう一つの推論は、因果性によるものであり、たとえば、
私たちが、原理から結論の認識へ到達する場合がそれである。

\\

Primus autem discursus Deo convenire non potest. Multa enim, quae
successive intelligimus si unumquodque eorum in seipso consideretur,
omnia simul intelligimus si in aliquo uno ea intelligamus, puta si
partes intelligamus in toto, vel si diversas res videamus in
speculo. Deus autem omnia videt in uno, quod est ipse, ut habitum est.
Unde simul, et non successive omnia videt.

&

ところで、第一の意味での推論は、神に適合しない。たとえば、私たちが全体
の中に部分を知性認識し、鏡の中にさまざまな事物を見る場合のように、その
各々がそれ自体で考察されるならば継起的に知性認識するような多くのものは、
それらをなにか一つのものの中で知性認識する場合、私たちはそれらすべてを
同時に知性認識する。ところが、すでに論じられたとおり、神は万物を一つの
もの、すなわち自己自身の中に見る。したがって、神は、継起的でなく同時に、
万物を見る。


\\

Similiter etiam et secundus discursus Deo competere non potest. Primo
quidem, quia secundus discursus praesupponit primum, procedentes enim
a principiis ad conclusiones, non simul utrumque considerant.

&

同様に、第二の意味での推論も、神に適合しえない。その理由はまず、第二の
意味での推論は、第一の意味での推論を前提する。なぜなら、原理から結論へ
と進むものたちは、同時にその両方を考察することがないからである。

\\


Deinde, quia discursus talis est procedentis de noto ad ignotum.  Unde
manifestum est quod, quando cognoscitur primum, adhuc ignoratur
secundum. Et sic secundum non cognoscitur in primo, sed ex
primo. Terminus vero discursus est, quando secundum videtur in primo,
resolutis effectibus in causas, et tunc cessat discursus. Unde, cum
Deus effectus suos in seipso videat sicut in causa, eius cognitio non
est discursiva.

&

また、そのような推論は、知られたものから知られていないものへと進むもの
に属する。明らかに、第一のものが認識されるとき、第二のものは知られてい
ない。この意味で、第二のものは第一のもの\kenten{の中に}認識されるので
はなく、第一のもの\kenten{から}認識される。他方、推論の終わりは、結果
が原因へと還元され、第二のものが第一のものの中に見られるときである。そ
うなったとき、推論は行われない。したがって、神は自らの結果を、原因の中
に見るようにして自己の中に見るのだから、神の認識は推論的でない。

\\


{\scshape Ad primum ergo dicendum} quod, licet sit unum tantum
intelligere in seipso, tamen contingit multa intelligere in aliquo
uno, ut dictum est.

&

第一異論に対しては、それゆえ、次のように言われるべきである。知性認識の
働きは、それ自身において一つだけだが、すでに述べられたとおり、ある一つ
のものにおいて、多くのものを知性認識するということがありうる。



\\


{\scshape Ad secundum dicendum} quod Deus non cognoscit per causam
quasi prius cognitam, effectus incognitos, sed eos cognoscit in
causa. Unde eius cognitio est sine discursu, ut dictum est.

&


第二異論に対しては、次のように言われるべきである。神は、より先に知られ
たものとしての原因によって、知られていない結果を認識するのではない。そ
うではなく、結果を原因の中に認識する。したがって、神の認識は、すでに述
べられたとおり、推論なしにある。

\\


{\scshape Ad tertium dicendum} quod effectus causarum creatarum videt
quidem Deus in ipsis causis, multo melius quam nos, non tamen ita quod
cognitio effectuum causetur in ipso ex cognitione causarum creatarum,
sicut in nobis. Unde eius scientia non est discursiva.

&

第三異論に対しては、次のように言われるべきである。被造の原因の結果を、
神は、その原因そのものの中に見るが、私たちよりもはるかに優れて見るので
あり、私たちのように、被造の諸原因の認識が原因となり、その結果、結果の
認識が認識者の中に生じる、というかたちではない。したがって、神の知は推
論的でない。


\end{longtable}
\newpage




\rhead{a.8}
\begin{center}
{\Large {\bf ARTICULUS OCTAVUS}}\\ {\large UTRUM SCIENTIA DEI SIT
CAUSA RERUM}\\ {\footnotesize I {\itshape Sent.}, d.38, a.1;
{\itshape De Verit.}, q.2, a.14.}\\ {\Large 第八項\\神の知は事物の原
因か}
\end{center}

\begin{longtable}{p{21em}p{21em}}

{\huge A}{\scshape d octavum sic proceditur}. Videtur quod scientia
Dei non sit causa rerum. Dicit enim Origenes, super epistolam
{\itshape ad Rom}., {\itshape non propterea aliquid erit, quia id scit
Deus futurum; sed quia futurum est, ideo scitur a Deo antequam fiat}.

&

第八項の問題へ、議論は以下のように進められる。神の知は事物の原因でない
と思われる。理由は以下の通り。オリゲネスは『ロマ書注解』で「神が未来の
ことを知っているから、そのために未来に何かがあるのではなく、未来に何か
があるから、それが生じる前に神によって知られている」と述べている。


\\


{\scshape 2 Praeterea}, posita causa ponitur effectus. Sed scientia
Dei est aeterna. Si ergo scientia Dei est causa rerum creatarum,
videtur quod creaturae sint ab aeterno.


&

さらに、原因があれば、結果がある。しかし、神の知は永遠である。ゆえに、
もし神の知が創造される事物の原因であるならば、被造物は永遠から在ったと
思われる。

\\


{\scshape 3 Praeterea}, scibile est prius scientia, et mensura eius,
ut dicitur in X {\itshape Metaphys}. Sed id quod est posterius et
mensuratum, non potest esse causa. Ergo scientia Dei non est causa
rerum.


&

さらに、『形而上学』10巻で言われるように、知られうるものは、知よりも先
のものであり、知の尺度である。しかし、後のものであり、尺度によって測ら
れるものが、原因ではありえない。ゆえに、神の知は事物の原因でない。

\\


{\scshape Sed contra est} quod dicit Augustinus, XV {\itshape de
Trin}., {\itshape universas creaturas, et spirituales et corporales,
non quia sunt, ideo novit Deus; sed ideo sunt, quia novit}.


&

しかし反対に、アウグスティヌスは『三位一体論』15巻で「霊的なものも物体
的なものも、すべての被造物は、それらが存在するから神がそれを知っている
のではなく、神がそれを知っているからそれらが存在する」と述べている。


\\


{\scshape Respondeo dicendum} quod scientia Dei est causa rerum. Sic
enim scientia Dei se habet ad omnes res creatas, sicut scientia
artificis se habet ad artificiata. Scientia autem artificis est causa
artificiatorum, eo quod artifex operatur per suum intellectum, unde
oportet quod forma intellectus sit principium operationis, sicut calor
est principium calefactionis.


&

解答する。以下のように言われるべきである。神の知は事物の原因である。理
由は以下の通り。神の知は、創造されたすべての事物に対して、職人が製品に
対するように関係する。ところが、職人の知は、製品の原因である。なぜなら、
職人は、自分の知性によって働くので、ちょうど、熱が熱することの根源であ
るように、知性の形相が、働きの根源でなければならないからである。

\\



Sed considerandum est quod forma naturalis, inquantum est forma manens
in eo cui dat esse, non nominat principium actionis; sed secundum quod
habet inclinationem ad effectum. Et similiter forma intelligibilis non
nominat principium actionis secundum quod est tantum in intelligente,
nisi adiungatur ei inclinatio ad effectum, quae est per voluntatem.

&

しかし、次のことが考察されるべきである。自然的形相は、その形相が存在を
与えているもののうちに留まっているかぎりで、当の形相を「働きの根源」と
いう名で呼ぶことはなく、そう呼ばれるのは、それが、結果への傾向性をもつ
場合である。同様に、可知的形相も、それがたんに、知性認識するものの中に
あるだけのときに「働きの根源」という名で呼ばれることはなく、そう呼ばれ
るのは、それに、結果への傾向性が結びつけられる場合に限られる。そしてこ
れは、意志による。

\\

Cum enim forma intelligibilis ad opposita se habeat (cum sit eadem
scientia oppositorum), non produceret determinatum effectum, nisi
determinaretur ad unum per appetitum, ut dicitur in IX
Metaphys. Manifestum est autem quod Deus per intellectum suum causat
res, cum suum esse sit suum intelligere. Unde necesse est quod sua
scientia sit causa rerum, secundum quod habet voluntatem
coniunctam. Unde scientia Dei, secundum quod est causa rerum,
consuevit nominari {\itshape scientia approbationis}.

&

じっさい、『形而上学』9巻で言われるように、可知的形相は、対立するもの
に関係するので(というのも、対立するどもについては、同一の知があるから
だが)、欲求によって一方へ限定されなかったら、限定された結果を生み出さ
なかったであろう。ところで、神の存在は神の知性認識だから、神が、自らの
知性によって、原因として事物を生み出すことは明らかである。したがって、
神の知が、結びつけられた意志を持つ点で、事物の原因であることは必然であ
る。このことから、神の知は、それが事物の原因である点で、「是認の知」と
いう名で呼ばれる習わしとなっている。


\\


{\scshape Ad primum ergo dicendum} quod Origenes locutus est attendens
rationem scientiae, cui non competit ratio causalitatis, nisi adiuncta
voluntate, ut dictum est. --- Sed quod dicit ideo praescire Deum
aliqua, quia sunt futura, intelligendum est secundum causam
consequentiae, non secundum causam essendi. Sequitur enim, si aliqua
sunt futura, quod Deus ea praescierit, non tamen res futurae sunt
causa quod Deus sciat.


&

第一異論に対しては、それゆえ、次のように言われるべきである。オリゲネス
は、すでに述べられたとおり、意志が結びつけられないかぎり原因性という性
格が適合しないという知の性格に注目して、そう述べた。--- しかし、「未来
に何かがあるから、神がそれらを先立って知る」と彼が言うことについては、
存在の原因ではなく、帰結の原因という意味で理解されるべきである。つまり、
もし未来に何かがあるならば、神がそれらを知っていたということが帰結する
と彼は言っているのであって、事物が未来に存在するということが、神がそれ
を知っていることの原因だと言っているのではない。

\\


{\scshape Ad secundum dicendum} quod scientia Dei est causa rerum,
secundum quod res sunt in scientia. Non fuit autem in scientia Dei,
quod res essent ab aeterno. Unde, quamvis scientia Dei sit aeterna,
non sequitur tamen quod creaturae sint ab aeterno.

&


第二異論に対しては次のように言われるべきである。神の知が事物の原因であ
るのは、事物が知の中にあるからである。しかし、神の知の中に、「事物が永
遠から在る」ということは入ってなかった。したがって、神の知が永遠だから
といって、被造物が永遠から在ることは帰結しない。
%\footnote{
%神と世界の関係は、神の本性に基づく流出関係ではなく、神の意志に基づく関係である。
%cf.q.46, a.1, ad 9, ad10.
%}

\\


{\scshape Ad tertium dicendum} quod res naturales sunt mediae inter
scientiam Dei et scientiam nostram, nos enim scientiam accipimus a
rebus naturalibus, quarum Deus per suam scientiam causa est. Unde,
sicut scibilia naturalia sunt priora quam scientia nostra, et mensura
eius, ita scientia Dei est prior quam res naturales, et mensura
ipsarum. Sicut aliqua domus est media inter scientiam artificis qui
eam fecit, et scientiam illius qui eius cognitionem ex ipsa iam facta
capit.


&

第三異論に対しては次のように言われるべきである。自然的事物は、神の知と
私たちの知の間にある。私たちは、自然的事物から知を受け取り、神はその知
によって、自然的事物の原因である。したがって、ちょうど、可知的な自然的
なものが私たちの知より先であり、その尺度であるように、神の知は、自然的
事物より先であり、その尺度である。それはちょうど、ある家が、それを作っ
た職人の知と、すでに作られたその家からその家の認識を得る人の知との間に
あるのと同様である。


\end{longtable}

\newpage



\rhead{a.9}
\begin{center}
{\Large {\bf ARTICULUS NONUS}}\\ {\large UTRUM DEUS HABEAT SCIENTIAM
NON ENTIUM}\\ {\footnotesize I {\itshape Sent.}, d.38, a.4; III,
d.19, a.2, q$^a$ 2; I {\itshape SCG.}, c.66; {\itshape De Verit.},
q.2, a.8.}\\ {\Large 第九項\\神は存在しないものの知をもつか}
\end{center}

\begin{longtable}{p{21em}p{21em}}


{\huge A}{\scshape d nonum sic proceditur}. Videtur quod Deus non
habeat scientiam non entium. Scientia enim Dei non est nisi
verorum. Sed verum et ens convertuntur. Ergo scientia Dei non est non
entium.

&

第九項の問題へ議論は以下のように進められる。神は、存在しないものどもの
知をもたないと思われる。理由は以下の通り。神の知は、真についての知にほ
かならない。しかし、真と有は置換される。ゆえに、存在しないものについて
の神の知はない。

\\



{\scshape 2 Praeterea}, scientia requirit similitudinem inter scientem
et scitum. Sed ea quae non sunt, non possunt habere aliquam
similitudinem ad Deum, qui est ipsum esse. Ergo ea quae non sunt, non
possunt sciri a Deo.

&

さらに、知は、知るものと知られるものとの間の類似を必要とする。しかし、
存在しないものは、存在そのものである神に対して、何らの類似性ももつこと
ができない。ゆえに、存在しないものは、神によって知られえない。

\\



{\scshape 3 Praeterea}, scientia Dei est causa scitorum ab ipso. Sed
non est causa non entium, quia non ens non habet causam. Ergo Deus non
habet scientiam de non entibus.

&

さらに、神の知は、神によって知られたものの原因である。ところが、存在し
ないものは原因をもたないので、存在しないものの原因はない。ゆえに、神は、
存在しないものについての知をもたない。

\\



{\scshape Sed contra est} quod dicit apostolus {\itshape ad
Rom}.~{\scshape iv}, {\itshape qui vocat ea quae non sunt, tanquam ea
quae sunt}.

&

しかし反対に、使徒は『ローマの信徒への手紙』4章で、「存在しないものど
もを、あたかも存在するものであるかのように呼ぶ方」\footnote{「存在して
いないものを呼び出して存在させる神を」(4:17)}と述べている。


\\



{\scshape Respondeo dicendum} quod Deus scit omnia quaecumque sunt
quocumque modo. Nihil autem prohibet ea quae non sunt simpliciter,
aliquo modo esse. Simpliciter enim sunt, quae actu sunt. Ea vero quae
non sunt actu, sunt in potentia vel ipsius Dei, vel creaturae; sive in
potentia activa, sive in passiva, sive in potentia opinandi, vel
imaginandi, vel quocumque modo significandi.

&

解答する。以下のように言われるべきである。神は、どんなかたちでも存在す
るすべてのものを知る。ところが、端的には存在しないものが、何らかのかた
ちで存在するとしても、何ら差し支えはない。たとえば、現実態において存在
するものは、端的に存在するが、現実態において存在しないものは、神自身の
能力のなかに、あるいは、被造物の能力のなかに存在する。[被造物の能力と
しては]能動的能力、受動的能力、臆見の能力、想像する能力、あるいはまた、
どのような表現の仕方をしようと、そういった能力のなかに、存在する。

\\



Quaecumque igitur possunt per creaturam fieri vel cogitari vel dici,
et etiam quaecumque ipse facere potest, omnia cognoscit Deus, etiam si
actu non sint. Et pro tanto dici potest quod habet etiam non entium
scientiam.

&

ゆえに、なんであれ、被造物によって作られ、思惟され、語られうるもの、そ
してまた、なんであれ神自身が作りうるものを、神は、それらが現実に存在し
ていなくても、すべて認識する。この意味で、神は存在しないものの知ももつ
と言われうる。


\\


Sed horum quae actu non sunt, est attendenda quaedam diversitas.
Quaedam enim, licet non sint nunc in actu, tamen vel fuerunt vel
erunt, et omnia ista dicitur Deus scire scientia visionis. Quia, cum
intelligere Dei, quod est eius esse, aeternitate mensuretur, quae sine
successione existens totum tempus comprehendit, praesens intuitus Dei
fertur in totum tempus, et in omnia quae sunt in quocumque tempore,
sicut in subiecta sibi praesentialiter.

&

しかし、現実態において存在しないものどもの中にある、ある違いに注意され
るべきである。つまり、あるものどもは、今、現実態において存在しないが、
しかし、過去や未来に存在する。このようなすべてのものを、神は、
\kenten{直視}の知によって知ると言われる。なぜなら、神の知性認識は、神
の存在であり、永遠によって測られ、継起なしに存在して時間全体を包含する
ので、神の現在の直観が、全時間へ、そして、どんな時間にあるものであって
もそのすべてへ、現在というかたちで神に服するものとしてのそれらへ、もた
らされるからである。

\\

Quaedam vero sunt, quae sunt in potentia Dei vel creaturae, quae tamen
nec sunt nec erunt neque fuerunt. Et respectu horum non dicitur habere
scientiam visionis, sed simplicis intelligentiae. Quod ideo dicitur,
quia ea quae videntur apud nos, habent esse distinctum extra videntem.


&

他方、あるものどもは、神や被造物の能力のうちにあるが、しかし、現在、未
来、過去において、存在しない。これらにかんしては、\kenten{直視}の知を
もつとは言われず、\kenten{単純知解}の知をもつと言われる。こう言われる
のは、私たちのもとで見られるものは、見る人の外に、[見る人から]区別さ
れた存在をもつからである。


\\



{\scshape Ad primum ergo dicendum} quod, secundum quod sunt in
potentia, sic habent veritatem ea quae non sunt actu, verum est enim
ea esse in potentia. Et sic sciuntur a Deo.

&

第一異論に対しては、それゆえ、次のように言われるべきである。現実に存在
しないものどもは、可能態にある場合、可能態にあるというかたちで真理をも
つ。じっさい、それらが可能態において在るということが真なのだから。そし
て、そのようなかたちで、神に知られる。


\\



Ad secundum dicendum quod, cum Deus sit ipsum esse, intantum
unumquodque est, inquantum participat de Dei similitudine, sicut
unumquodque intantum est calidum, inquantum participat calorem. Sic et
ea quae sunt in potentia, etiam si non sunt in actu, cognoscuntur a
Deo.

&

第二異論に対しては次のように言われるべきである。神は存在そのものだから、
各々のものは、それが存在するだけ、それだけ神の類似を分有する。それはちょ
うど、各々ものが、それが熱いだけ、それだけ熱を分有するのと同じである。
この意味で、可能態にあるものも、それが現実態にない場合でさえ、神に認識
される。

\\



{\scshape Ad tertium dicendum} quod Dei scientia est causa rerum,
voluntate adiuncta. Unde non oportet quod quaecumque scit Deus, sint
vel fuerint vel futura sint, sed solum ea quae vult esse, vel
permittit esse. --- Et iterum, non est in scientia Dei ut illa sint,
sed quod esse possint.


&


第三異論に対しては、次のように言われるべきである。神の知は、意志が結び
つけられたときに、事物の原因である。したがって、神が知るものがすべて、
現在や過去や未来に存在しなければならないというわけではない。むしろ存在
するのは、神が存在することを欲し、あるいは、神が存在することを許すもの
だけである。--- また、神の知の中には、「それらが存在する」ということが
在るのではなく、「それらが存在しうる」ということが在る。

\end{longtable}
\newpage



\rhead{a.10}
\begin{center}
{\Large {\bf ARTICULUS DECIMUS}}\\ {\large UTRUM DEUS COGNOSCAT
MALA}\\ {\footnotesize I {\itshape Sent.}, d.36, q.1, a.2; I
{\itshape SCG.}, c.71; {\itshape De Verit.}, q.2, a.15; {\itshape
Quodl}.~XI, q.2.}\\ {\Large 第十項\\神は悪を認識するか}
\end{center}

\begin{longtable}{p{21em}p{21em}}



{\huge A}{\scshape d decimum sic proceditur}. Videtur quod Deus non
cognoscat mala. Dicit enim philosophus, in III {\itshape de Anima},
quod intellectus qui non est in potentia, non cognoscit
privationem. Sed malum est {\itshape privatio boni}, ut dicit
Augustinus. Igitur, cum intellectus Dei nunquam sit in potentia, sed
semper actu, ut ex dictis patet, videtur quod Deus non cognoscat mala.


&

第十項の問題へ議論は以下のように進められる。神は悪を認識しないと思われ
る。理由は以下の通り。哲学者は『デ・アニマ』3巻で、可能態にない知性は、
欠如を認識しないと述べている。しかし、アウグスティヌスが言うように、悪
は「善の欠如」である。ゆえに、すでに述べられたことから明らかなように、
神の知性はけっして可能態になく、常に現実態にあるので、神は悪を認識しな
いと思われる。

\\


{\scshape 2 Praeterea}, omnis scientia vel est causa sciti, vel
causatur ab eo. Sed scientia Dei non est causa mali, nec causatur a
malo. Ergo scientia Dei non est malorum.

&

さらに、知はすべて、知られたものの原因であるか、あるいは、知られたもの
が原因となって生じたかのどちらかである。しかし、神の知は、悪の原因でな
いし、悪が原因で生じたのでもない。ゆえに、神の知は悪にかかわらない。

\\


{\scshape 3 Praeterea}, omne quod cognoscitur, cognoscitur per suam
similitudinem, vel per suum oppositum. Quidquid autem cognoscit Deus,
cognoscit per suam essentiam, ut ex dictis patet. Divina autem
essentia neque est similitudo mali, neque ei malum opponitur, divinae
enim essentiae nihil est contrarium, ut dicit Augustinus, XII
{\itshape de Civ.~Dei}. Ergo Deus non cognoscit mala.

&

さらに、認識されるものはすべて、自らの類似を通して、また、自らに対立す
るものを通して認識される。さて、すでに述べられたことから明らかなように、
神は認識するすべてのものを、自らの本質を通して認識する。しかし、神の本
質は、悪の類似でも、悪に対立するものでもない。というのも、アウグスティ
ヌスが『神の国』12巻で言うように、神の本質に対立するものはなにもないか
らである。ゆえに、神は悪を認識しない。

\\


{\scshape 4 Praeterea}, quod cognoscitur non per seipsum, sed per
aliud, imperfecte cognoscitur. Sed malum non cognoscitur a Deo per
seipsum, quia sic oporteret quod malum esset in Deo; oportet enim
cognitum esse in cognoscente. Si ergo cognoscitur per aliud, scilicet
per bonum, imperfecte cognoscetur ab ipso, quod est impossibile, quia
nulla cognitio Dei est imperfecta. Ergo scientia Dei non est malorum.

&

さらに、それ自身によってではなく、他によって認識されるものは、不完全に
認識される。ところで、悪は、神によって、それ自身によって認識されるので
はない。なぜなら、認識されるものは、認識するものの中になければならない
から、もしそうだとすると、悪が神の中にあることになっただろうから。ゆえ
に、[悪は神によって]他によって、つまり善によって、不完全に認識される
ことになるが、これは不可能である。なぜなら、神のどんな認識も不完全でな
いからである。ゆえに、神の知は悪にかかわらない。


\\


{\scshape Sed contra est} quod dicitur {\itshape Proverb}.~{\scshape
xv}, {\itshape Infernus et perditio coram Deo}.

&

しかし反対に、『箴言』15章「地獄と破滅が神の前にある」\footnote{「陰府
も滅びの国も主の御前にある」(15:11)}と言われている。


\\


{\scshape Respondeo dicendum} quod quicumque perfecte cognoscit
aliquid, oportet quod cognoscat omnia quae possunt illi accidere. Sunt
autem quaedam bona, quibus accidere potest ut per mala
corrumpantur. Unde Deus non perfecte cognosceret bona, nisi etiam
cognosceret mala. Sic autem est cognoscibile unumquodque, secundum
quod est. Unde, cum hoc sit esse mali, quod est privatio boni, per hoc
ipsum quod Deus cognoscit bona, cognoscit etiam mala; sicut per lucem
cognoscuntur tenebrae. Unde dicit Dionysius, {\scshape vii}
cap.{\itshape de Div.~Nom}., quod Deus per {\itshape semetipsum
tenebrarum accipit visionem, non aliunde videns tenebras quam a
lumine}.

&

解答する。以下のように言われるべきである。何かを完全に認識するものはだ
れでも、それに附帯しうるすべてのことを認識するのでなければならない。と
ころで、「悪によって滅ぼされる」ということが附帯しうるような善が存在す
る。したがって、神は、もし悪を認識しなかったならば、善もまた完全には認
識しなかったであろう。さらに、各々のものが認識されうるのは、それが存在
するという点においてである。したがって、善の欠如であるということが、悪
にとって存在することなのだから、神は、善を認識すること自体によって、悪
もまた認識する。これはちょうど、光によって影が認識されるようなものなの
で、このことから、ディオニュシウスは『神名論』7章で、神は「光以外のど
こかから影を見るのではなく、自己自身によって、影の姿を見る」と言う。

\\


{\scshape Ad primum ergo dicendum} quod verbum philosophi est sic
intelligendum, quod intellectus qui non est in potentia, non cognoscit
privationem per privationem in ipso existentem. Et hoc congruit cum eo
quod supra dixerat, quod punctum et omne indivisibile per privationem
divisionis cognoscitur. Quod contingit ex hoc, quia formae simplices
et indivisibiles non sunt actu in intellectu nostro, sed in potentia
tantum, nam si essent actu in intellectu nostro, non per privationem
cognoscerentur. Et sic cognoscuntur simplicia a substantiis
separatis. Deus igitur non cognoscit malum per privationem in se
existentem, sed per bonum oppositum.

&

第一異論に対しては、それゆえ、次のように言われるべきである。哲学者の言
葉は、可能態にない知性は、自らのうちに存在する欠如によって、欠如を認識
しない、という意味に理解されるべきである。そしてこのことは、上で述べた
こと、すなわち、点など、すべて不可分なものは、分割の欠如によって認識さ
れる、ということと整合的である。これは、単純で不可分な形相が、私たちの
知性の中に、現実態において存在せず、ただ、可能態においてのみ存在するこ
とから生じる。というのも、もし、それらが現実に私たちの知性の中にあった
ならば、欠如によって認識されることはなかっただろうから。単純形相は、離
在実体によって、そのように[欠如によらずに]認識される。ゆえに、神は、
自らのうちに存在する欠如によってではなく、対置される善によって、悪を認
識する。

 \\
 \\


{\scshape Ad secundum dicendum} quod scientia Dei non est causa mali,
sed est causa boni, per quod cognoscitur malum.

&

第二異論に対しては次のように言われるべきである。神の知は、悪の原因では
ない。しかし、善の原因であり、その善によって、悪が認識される。

 \\
 \\


{\scshape Ad tertium dicendum} quod, licet malum non opponatur
essentiae divinae, quae non est corruptibilis per malum, opponitur
tamen effectibus Dei; quos per essentiam suam cognoscit, et eos
cognoscens, mala opposita cognoscit.

&

第三異論に対しては次のように言われるべきである。神の本質は悪によって滅
ぼされえないので、悪が神の本質に対立するわけではないが、しかし、悪は神
の結果には対立する。神は、それらの結果を自らの本質によって認識し、それ
らを認識することで、[結果に]対立する悪を認識する。

\\
\\

{\scshape Ad quartum dicendum} quod cognoscere aliquid per aliud
tantum, est imperfectae cognitionis, si illud sit cognoscibile per
se. Sed malum non est per se cognoscibile, quia de ratione mali est,
quod sit privatio boni. Et sic neque definiri, neque cognosci potest,
nisi per bonum.

&

第四異論に対しては次のように言われるべきである。何かを、それがそれ自体
によって認識されうるのに、ただ他のものによってのみ認識することは、不完
全な認識に属する。しかし、悪は、それ自体によって認識されうるものではな
い。なぜなら、悪の性格には、善の欠如ということが含まれるからである。そ
の意味で、悪は、善によらなければ定義されず、また、善によらなければ認識
もされない。


\end{longtable}
\newpage


\rhead{a.11}
\begin{center}
{\Large {\bf QUAESTIO UNDECIMUS}}\\ {\large UTRUM DEUS COGNOSCAT
SINGULARIA}\\ {\footnotesize I {\itshape Sent.}, d.36, q.1, a.1;
II, d.3, q.11, a.3; I {\itshape SCG.}, c.50, 63, 65;
Qu.Disp.{\itshape de Anima}, a.20; {\itshape De Verit.}, q.2,
a.5; {\itshape Compend.~Theol.}, c.132, 133; I {\itshape Periherm.},
lect.14.}\\ {\Large 第十一項\\神は個物を認識するか}
\end{center}

\begin{longtable}{p{21em}p{21em}}



{\huge A}{\scshape d undecimum sic proceditur}. Videtur quod Deus non
cognoscat singularia. Intellectus enim divinus immaterialior est quam
intellectus humanus. Sed intellectus humanus, propter suam
immaterialitatem, non cognoscit singularia, sed, sicut dicitur in II
{\itshape de Aanima}, {\itshape ratio est universalium, sensus vero
singularium}. Ergo Deus non cognoscit singularia.


&

第十一項の問題へ議論は以下のように進められる。神は個物を認識しないと思
われる。理由は以下の通り。神の知性は、人間の知性よりも非質料的である。
ところが、人間の知性は、その非質料性のために、個物を認識せず、かえって、
『デ・アニマ』2巻で言われるように「理性は普遍にかかわり、感覚は個物に
かかわる」。ゆえに、神は個物を認識しない。

\\
\\

{\scshape 2 Praeterea}, illae solae virtutes in nobis sunt singularium
cognoscitivae, quae recipiunt species non abstractas a materialibus
conditionibus. Sed res in Deo sunt maxime abstractae ab omni
materialitate. Ergo Deus non cognoscit singularia.


&

さらに、私たちの中にある個物を認識しうる能力は、質料的条件から抽象され
ていない形象を受け取る能力だけである。ところが、事物は、神の中で、あら
ゆる質料性から最大限に抽象されている。ゆえに、神は個物を認識しない。

\\
\\

{\scshape 3 Praeterea}, omnis cognitio est per aliquam
similitudinem. Sed similitudo singularium, inquantum sunt singularia,
non videtur esse in Deo, quia principium singularitatis est materia,
quae, cum sit ens in potentia tantum, omnino est dissimilis Deo, qui
est actus purus. Non ergo Deus potest cognoscere singularia.


&

さらに、すべての認識は何らかの類似による。ところが、個物の類似が、それ
が個物だという点で、神の中にあるとは思われない。なぜなら、個物性の根源
は質料だが、質料は、たんに可能態にある有なので、純粋現実態である神とは
まったく似ていないからである。ゆえに、神は個物を認識できない。

\\
\\

{\scshape Sed contra est} quod dicitur, {\itshape Proverb}.~{\scshape
xvi}, {\itshape omnes viae hominum patent oculis eius}.

&

しかし反対に、『箴言』16章で「人間のすべての道は、彼の目に明らかである」
\footnote{「人間の道は自分の目に清く見えるが、主はその精神を調べられる」
(16:2)}と言われている。

\\
\\

{\scshape Respondeo dicendum} quod Deus cognoscit singularia. Omnes
enim perfectiones in creaturis inventae, in Deo praeexistunt secundum
altiorem modum, ut ex dictis patet. Cognoscere autem singularia
pertinet ad perfectionem nostram. Unde necesse est quod Deus
singularia cognoscat. Nam et philosophus pro inconvenienti habet, quod
aliquid cognoscatur a nobis, quod non cognoscatur a Deo. Unde contra
Empedoclem arguit, in I {\itshape de Anima} et in III {\itshape
Metaphys}., quod accideret Deum esse insipientissimum, si discordiam
ignoraret. Sed perfectiones quae in inferioribus dividuntur, in Deo
simpliciter et unite existunt. Unde, licet nos per aliam potentiam
cognoscamus universalia et immaterialia, et per aliam singularia et
materialia; Deus tamen per suum simplicem intellectum utraque
cognoscit.


&

解答する。以下のように言われるべきである。神は個物を認識する。すでに述
べられたことから明らかなとおり、被造物の中に見出されるすべての完全性は、
神の中に、より高度なあり方で先在する。ところで、個物を認識することは、
私たちの完全性に属する。
%\footnote{
%cf. {\itshape ST}I, q.12, a.8, ad 4. ``
%Ad quartum dicendum quod naturale
%desiderium rationalis creaturae est ad sciendum
%omnia illa quae pertinent ad perfectionem
%intellectus; et haec sunt species et genera rerum,
%et rationes earum, quae in Deo videbit quilibet
%videns essentiam divinam. Cognoscere autem
%alia singularia, et cogitata et facta eorum, non
%est de perfectione intellectus creati, nec ad hoc
%eius naturale desiderium tendit, ...''\\
%「理性的被造物の自然本性的欲求は、知性の完全性に
%属するすべてのものを知ることへと向かう。その
%完全性とは、諸事物の種と類、そして、それらの理
%念である。これらを、神の本質を見る者は誰でも、
%神において見るであろう。しかし、他の個物や、そ
%れらの個物によって思惟されたことや為されたこ
%とは、被造知性の完全性に含まれず、こういうこと
%に、被造知性の自然本性的欲求は向かわない。」\\
%◎個物の認識は知性の完全性に含まれないと言う12問の論述との整合性の問題。
%「知性の完全性」と、「私たちの完全性」との違い。
%}
したがって、神が個物を認識することが必要である。それで、哲学者も、私た
ちによって認識されるものが、神によって認識されないのは、不都合なことだ
と考えている。ゆえに、『デ・アニマ』1巻や『形而上学』3巻で、エムペドク
レスに反対して、もし神が不和を知らなかったら、神はもっとも愚かだという
ことになっただろう、と論じている。さて、ところで、下位のものどもの中に
見出される諸々の完全性は、神の中に、単純で、一つのものとして存在する。
したがって、私たちは、普遍的で非質料的なものと個的で質料的なものを、別々
の能力によって認識するが、神は、自らの単純で一つの知性認識によって、
[それらを]認識する。

\\
\\


Sed qualiter hoc esse possit, quidam manifestare volentes, dixerunt
quod Deus cognoscit singularia per causas universales, nam nihil est
in aliquo singularium, quod non ex aliqua causa oriatur universali. Et
ponunt exemplum, sicut si aliquis astrologus cognosceret omnes motus
universales caeli, posset praenuntiare omnes eclipses futuras.

&

しかし、ある人々は、これがどのようにしてありうるかを示そうとして、神は
個物を普遍的な原因を通して認識すると言った。なぜなら、個物の中には、何
らかの普遍的な原因に起源をもたないものはないからである。そして彼らは、
もし、ある天文学者が天のすべての普遍的な運動を認識したとすると、彼は、
未来のすべての食を予知するであろう、という例を示している。

\\
\\
 
Sed istud non sufficit. Quia singularia ex causis universalibus
sortiuntur quasdam formas et virtutes, quae, quantumcumque ad invicem
coniungantur, non individuantur nisi per materiam individualem. Unde
qui cognosceret Socratem per hoc quod est albus vel Sophronisci
filius, vel quidquid aliud sic dicatur, non cognosceret ipsum
inquantum est hic homo. Unde secundum modum praedictum, Deus non
cognosceret singularia in sua singularitate.

&

しかし、これは不十分である。なぜなら、個物は、普遍的原因から、何らかの
形相や力へと種別されるが、これら[形相や力]は、どれほど相互に結びつけ
られようとも、個的質料によらなければ個体化されないからである。したがっ
て、「白いもの」や「ソプロニスコスの息子」や、その他そのように言われる
ものによってソクラテスを認識する人がいたとしたら、その人は、ソクラテス
を、「この人間」として認識していなかったであろう。したがって、神が個物
を上述のこの方法で認識したとすれば、神は個物をその個物性において認識し
なかったであろう。

\\
\\


Alii vero dixerunt quod Deus cognoscit singularia, applicando causas
universales ad particulares effectus. Sed hoc nihil est. Quia nullus
potest applicare aliquid ad alterum, nisi illud praecognoscat, unde
dicta applicatio non potest esse ratio cognoscendi particularia, sed
cognitionem singularium praesupponit.


&

他方、別の人々は、普遍的な原因を個別的な結果へ当てはめることによって、
神が個物を認識すると言った。しかしこれは違う。なぜなら、あるものを何か
に当てはめるといっても、その何かをあらかじめ認識していなければ、だれも
そのようなことはできないからである。したがって、言われるような「当ては
め」は、個的な認識の根拠ではなく、むしろ個的な認識をその前提にしている。

\\
\\

Et ideo aliter dicendum est, quod, cum Deus sit causa rerum per suam
scientiam, ut dictum est, intantum se extendit scientia Dei, inquantum
se extendit eius causalitas. Unde, cum virtus activa Dei se extendat
non solum ad formas, a quibus accipitur ratio universalis, sed etiam
usque ad materiam, ut infra ostendetur; necesse est quod scientia Dei
usque ad singularia se extendat, quae per materiam individuantur.

&

したがって、違うかたちで次のように言われるべきである。すでに言われたと
おり、神は自分の知によって事物の原因なのだから、神の知が及ぶところまで、
神の原因性も及ぶ。したがって、神の能動的な力は、普遍的な性格がそこから
受け取られる形相だけでなく、以下で示されるであろうように、質料にまで及
ぶのだから、神の知が、質料によって個体化される個物まで及ぶことが必要で
ある。

\\
\\
 
Cum enim sciat alia a se per essentiam suam, inquantum est similitudo
rerum velut principium activum earum, necesse est quod essentia sua
sit principium sufficiens cognoscendi omnia quae per ipsum fiunt, non
solum in universali, sed etiam in singulari. Et esset simile de
scientia artificis, si esset productiva totius rei, et non formae
tantum.


&

じっさい、神は自己以外のものを、自己の本質を通して、自己が諸事物の能動
的根源として諸事物の類似であるという点で知るのだから、神の本質が、神に
よって生じるすべてのものを、普遍的にだけでなく、個物としても認識する十
分な根源であることが必要である。もし、技術者の知が、たんに形相だけでな
く、事物全体を生み出しうるものであったとしたならば、同じことが、技術者
の知についても言えたであろう。


\\
\\
 

{\scshape Ad primum ergo dicendum} quod intellectus noster speciem
intelligibilem abstrahit a principiis individuantibus, unde species
intelligibilis nostri intellectus non potest esse similitudo
principiorum individualium. Et propter hoc, intellectus noster
singularia non cognoscit. Sed species intelligibilis divini
intellectus, quae est Dei essentia, non est immaterialis per
abstractionem, sed per seipsam, principium existens omnium
principiorum quae intrant rei compositionem, sive sint principia
speciei, sive principia individui. Unde per eam Deus cognoscit non
solum universalia, sed etiam singularia.


&

第一異論に対しては、それゆえ、次のように言われるべきである。私たちの知
性は、可知的形象を個体化の根源から抽象するので、私たちの知性の可知的形
象は個体の根源の類似でありえない。このため、私たちの知性は、個物を認識
できない。しかし神の知性の可知的形象、これは神の本質だが、は、抽象によっ
てではなくそれ自身によって非質料的であり、種の根源であれ、個体の根源で
あれ、事物の複合に入るすべての根源の根源である。したがって、それ[=神
の知性の可知的形象]によって、神は、普遍だけでなく、個物も認識する。


\\
\\

{\scshape Ad secundum dicendum} quod, quamvis species intellectus
divini secundum esse suum non habeat conditiones materiales, sicut
species receptae in imaginatione et sensu; tamen virtute se extendit
ad immaterialia et materialia, ut dictum est.


&

第二異論に対しては次のように言われるべきである。神の知性の形象が、自分
の存在において、想像力や感覚に受け取られた形象のような質料的な条件をも
たないとはいえ、しかし、すでに述べられたとおり、その力によって、非質料
的なものにも質料的なものにも及ぶ。


\\
\\

{\scshape Ad tertium dicendum} quod materia, licet recedat a Dei
similitudine secundum suam potentialitatem, tamen inquantum vel sic
esse habet, similitudinem quandam retinet divini esse.


&

第三異論に対しては、次のように言われるべきである。質料は、たしかに、そ
の可能態性において、神の類似から退いているが、それでもそのような存在を
もつという点で、神の存在のある種の類似を保持している。




\end{longtable}
\newpage


\rhead{a.12}
\begin{center}
{\Large {\bf ARTICULUS DUODECIMUS}}\\ {\large UTRUM DEUS POSSIT
COGNOSCERE INFINITA}\\ {\footnotesize I {\itshape Sent.}, d.39, q.1,
a.3; I {\itshape SCG.}, c.49; {\itshape De Verit.}, q.2, a.9;
q.20, a.4, ad 1; {\itshape Quod}.~III, q.2, a.1; {\itshape
Compend.~Theol.}, c.133.}\\ {\Large 第十二項\\神は無限を認識できるか}
\end{center}

\begin{longtable}{p{21em}p{21em}}

{\huge A}{\scshape d duodecimum sic proceditur}. Videtur quod Deus non
possit cognoscere infinita. Infinitum enim, secundum quod est
infinitum, est ignotum, quia infinitum est {\itshape cuius quantitatem
accipientibus semper est aliquid extra assumere}, ut dicitur in III
{\itshape Physic}. Augustinus etiam dicit, XII {\itshape de Civ.~Dei},
quod {\itshape quidquid scientia comprehenditur, scientis
comprehensione finitur}. Sed infinita non possunt finiri. Ergo non
possunt scientia Dei comprehendi.

&

第十二項の問題へ議論は以下のように進められる。神は無限を認識できないと
思われる。理由は以下の通り。無限は、無限であるという点で、知られない。
なぜなら、『自然学』3巻で言われるように、無限は、「その量を理解するも
のにとって、常に獲得の外に何かがある」からである。また、アウグスティヌ
スも、『神の国』12巻で「何であれ知によって把握されるものは、知るものの
把握によって限られる」と言う。ところが、無限なものは限られえない。ゆえ
に、[無限は]神の知によって、把握されない。


\\


2 Si dicatur quod ea quae in se sunt infinita, scientiae Dei finita
sunt, contra, ratio infiniti est quod sit impertransibile; et finiti
quod sit pertransibile, ut dicitur in III {\itshape Physic}. Sed
infinitum non potest transiri nec a finito, nec ab infinito, ut
probatur in VI {\itshape Physic}. Ergo infinitum non potest esse
finitum finito, neque etiam infinito. Et ita infinita non sunt finita
scientiae Dei, quae est infinita.

&

もし、それ自体において無限であるものも、神の知にとって有限であるといわ
れるならば、それに反対して次のように言う。『自然学』3巻で言われるよう
に、無限の性格は、「通過されえない」ということであり、有限の性格は、
「通過されうる」ということである。ところが、『自然学』6巻で証明される
ように、無限は、有限なものによっても無限なものによっても通過されえない。
ゆえに、無限は、有限なものによってだけでなく、無限なものによっても限ら
れない。したがって、無限は、無限な神の知にとっても有限でない。


\\


{\scshape 3 Praeterea}, scientia Dei est mensura scitorum. Sed contra
rationem infiniti est, quod sit mensuratum. Ergo infinita non possunt
sciri a Deo.


&

さらに、神の知は、知られたものの尺度[測るもの]である。ところが、「測
られる」ということは、無限の性格に反する。ゆえに、無限は神によって知ら
れえない。


\\


{\scshape Sed contra est} quod dicit Augustinus, XII {\itshape de
Civ.~Dei}, {\itshape quamvis infinitorum numerorum nullus sit numerus,
non est tamen incomprehensibilis ei, cuius scientiae non est numerus}.

&

しかし反対に、アウグスティヌスは『神の国』12巻で「無限の数にはどんな数
も属さないが、その知が数でないものにとって、把握されえないことはない」
と述べている。



\\


{\scshape Respondeo dicendum} quod, cum Deus sciat non solum ea quae
sunt actu, sed etiam ea quae sunt in potentia vel sua vel creaturae,
ut ostensum est; haec autem constat esse infinita; necesse est dicere
quod Deus sciat infinita.




&

解答する。以下のように言われるべきである。すでに示されたとおり、
\footnote{第九項。}神は現実態にあるものだけでなく、神や被造物の能力の
中にあるものも知る。それらが無限にあることは明らかである。ゆえに、神は
無限を知ると言うことが必要である。

\\

Et licet scientia visionis, quae est tantum eorum quae sunt vel erunt
vel fuerunt, non sit infinitorum, ut quidam dicunt, cum non ponamus
mundum ab aeterno fuisse, nec generationem et motum in aeternum
mansura, ut individua in infinitum multiplicentur, tamen, si
diligentius consideretur, necesse est dicere quod Deus etiam scientia
visionis sciat infinita.  Quia Deus scit etiam cogitationes et
affectiones cordium, quae in infinitum multiplicabuntur, creaturis
rationalibus permanentibus absque fine.

&

ところで、私たちは、世界が永遠から存在したことや、生成や運動が永遠に存
続することを否定するから、個体が無限に多数化されることを認めない。した
がって、ある人々が言うように、現在、過去、未来のいずれかに存在するもの
どもだけを対象とする\kenten{直視}の知は、無限を対象としない[という考
えが成り立つようにも見える\footnote{論旨から見て、ここの``licet''は、
内容上の譲歩ではなく、完全に内容を否定した上での、そのような意見を立て
る人々に対する語用論的な譲歩。}]。しかし、もし注意深く考察されていた
ならば、神が、直視の知によっても、無限を知ると言うことが必要である。な
ぜなら、神は心の思いや情念もまた知るが、理性的被造物は終わりなく存続す
るのだから、それらは無限に多数化されるからである。


\\


Hoc autem ideo est, quia cognitio cuiuslibet cognoscentis se extendit
secundum modum formae quae est principium cognitionis. Species enim
sensibilis, quae est in sensu, est similitudo solum unius individui,
unde per eam solum unum individuum cognosci potest.

&

さて、このこと[=神が無限を認識すること]の理由は以下の通りである。ど
んな認識者の認識も、認識の根源である形相のあり方にしたがって自らを広げ
る。たとえば、感覚の中にある可感的形象は、ただ一つの個体の類似であり、
したがって、それによって、ただ一つの個体が認識されうる。

\\


Species autem intelligibilis intellectus nostri est similitudo rei
quantum ad naturam speciei, quae est participabilis a particularibus
infinitis, unde intellectus noster per speciem intelligibilem hominis,
cognoscit quodammodo homines infinitos. Sed tamen non inquantum
distinguuntur ab invicem, sed secundum quod communicant in natura
speciei; propter hoc quod species intelligibilis intellectus nostri
non est similitudo hominum quantum ad principia individualia, sed
solum quantum ad principia speciei.

&

他方、私たちの知性がもつ可知的形象は、種の本性という点で事物の類似であ
るが、種の本性は、無限の個物によって分有されうる。したがって、私たちの
知性は、人間の可知的形象によって、ある意味で、無数の人間を認識する。と
はいえ、私たちの知性がもつ可知的形象は、個体の根源としてではなく、たん
に、種の根源として、人間の類似であるから、相互に区別されたものとしてで
はなく、種の本性において共通するかぎりで、[それら無数の人間を]認識す
るにすぎない。

\\

Essentia autem divina, per quam intellectus divinus intelligit, est
similitudo sufficiens omnium quae sunt vel esse possunt, non solum
quantum ad principia communia, sed etiam quantum ad principia propria
uniuscuiusque, ut ostensum est. Unde sequitur quod scientia Dei se
extendat ad infinita, etiam secundum quod sunt ab invicem distincta.


&

しかし、神は神の本質によって知性認識するが、その神の本質は、すでに示さ
れたとおり、共通の根源という点だけでなく、各々のものの固有の根源という
点でも、すべての存在するものや存在しうるものの十分な類似である。したがっ
て、神の知は、相互に区別されているものとしての無限のものへも及ぶことが
帰結する。


\\


{\scshape Ad primum ergo dicendum} quod infiniti ratio congruit
quantitati, secundum philosophum in I {\itshape Physic}. De ratione
autem quantitatis est ordo partium. Cognoscere ergo infinitum secundum
modum infiniti, est cognoscere partem post partem. Et sic nullo modo
contingit cognosci infinitum, quia quantacumque quantitas partium
accipiatur, semper remanet aliquid extra accipientem. Deus autem non
sic cognoscit infinitum vel infinita, quasi enumerando partem post
partem; cum cognoscat omnia simul, non successive, ut supra dictum
est. Unde nihil prohibet ipsum cognoscere infinita.


&

第一異論に対しては、それゆえ、次のように言われるべきである。『自然学』
第1巻の哲学者によれば、無限の性格は量に一致する。ところが、量の性格に
は、部分の秩序ということが含まれる。ゆえに、無限というかたちで無限を認
識することは、ある部分のあとに他の部分を認識するというかたちをとる。こ
う考えると、無限はけっして認識されることがない。なぜなら、部分の量がど
れだけ認識されても、常に何かが、理解する者の外に残るからである。しかし、
神は、無限ないし無限のものを、ある部分を別の部分のあとに数えるというか
たちで認識するのではない。なぜなら、上で示されたように、神は万物を、継
起的にではなく同時に認識するからである。したがって、神が無限を認識する
ことを妨げるものはなにもない。


\\

{\scshape Ad secundum dicendum} quod transitio importat quandam
successionem in partibus, et inde est quod infinitum transiri non
potest, neque a finito neque ab infinito. Sed ad rationem
comprehensionis sufficit adaequatio, quia id comprehendi dicitur,
cuius nihil est extra comprehendentem. Unde non est contra rationem
infiniti, quod comprehendatur ab infinito. Et sic, quod in se est
infinitum, potest dici finitum scientiae Dei, tanquam comprehensum,
non tamen tanquam pertransibile.

&

第二異論に対しては、次のように言われるべきである。通過は、部分における
一種の継起を意味する。だから、無限のものは、有限のものによっても無限の
ものによっても通過されえない。しかし、把握という性格のためには、対等と
いうことで十分である。なぜなら、\kenten{あるもの}が把握されると言われ
るのは、把握する者の外に、\kenten{それ}に属するものがなにもないときだ
から。したがって、無限によって把握されることは、無限の性格に反しない。
その意味で、それ自体において無限であるものが、神の知にとって、有限であ
ると言われうる。ただしその意味は、把握されたものとしてであり、通過され
うるものとしてではない。




\\


{\scshape Ad tertium dicendum} quod scientia Dei est mensura rerum,
non quantitativa, qua quidem mensura carent infinita; sed quia
mensurat essentiam et veritatem rei. Unumquodque enim intantum habet
de veritate suae naturae, inquantum imitatur Dei scientiam; sicut
artificiatum inquantum concordat arti. Dato autem quod essent aliqua
infinita actu secundum numerum, puta infiniti homines; vel secundum
quantitatem continuam, ut si esset aer infinitus, ut quidam antiqui
dixerunt, tamen manifestum est quod haberent esse determinatum et
finitum, quia esse eorum esset limitatum ad aliquas determinatas
naturas. Unde mensurabilia essent secundum scientiam Dei.


&

第三異論に対しては、次のように言われるべきである。神の知が事物の尺度で
あるのは、量の尺度という意味ではない。その意味であれば、たしかに、無限
は尺度を欠く[≒測られるものではない]。しかしそうではなく、事物の本質
と真理の尺度だという意味である。じっさい、各々のものは、自らの本性の真
理をもつという点で、神の知を模倣する。ちょうど、制作物が、技術に一致す
る点で[制作者の知を模倣するように]。さらにまた、かりに、無数の人間の
ように、数的に無限な何かが存在したとしても、あるいは、だれか古代の人が
言ったように、無限の空気が存在する場合のように、連続量という点で無限の
ものがあったとしても、これらの存在は、何らかの限定された本性に制限され
ているので、これらが限定され、有限な存在をもったであろうということは明
らかである。したがって、神の知から見れば、測られうるものだっただろう。



\end{longtable}
\newpage



\rhead{a.13}
\begin{center}
{\Large {\bf ARTICULUS DECIMUS TERTIUS}}\\ {\large UTRUM SCIENTIA DEI
SIT FUTURORUM CONTINGENTIUM}\\ {\footnotesize Infra q.86, a.4; I
{\itshape Sent.}, d.38, a.5; I {\itshape SCG.}, c.67; {\itshape De
Verit.}, q.2, a.12; {\itshape De Malo}, q.16, a.7; {\itshape
Quodl.}~XI, q.3; Opusc.II, {\itshape Contra Graecos, Armenos} etc.,
c.10; {\itshape Compend.~Theol.}, c.133; I {\itshape Periherm.},
lect.14.}\\ {\Large 第十三項\\神の知は未来の非必然的な事柄にかかわる
か}
\end{center}

\begin{longtable}{p{21em}p{21em}}


{\huge A}{\scshape d decimumtertium sic proceditur}. Videtur quod
scientia Dei non sit futurorum contingentium. A causa enim necessaria
procedit effectus necessarius. Sed scientia Dei est causa scitorum, ut
supra dictum est. Cum ergo ipsa sit necessaria, sequitur scita eius
esse necessaria. Non ergo scientia Dei est contingentium.

&

第十三項の問題へ議論は以下のように進められる。神の知は、未来の非必然的
なものを対象としないように思われる。理由は以下の通り。必然的な原因から
は、必然的な結果が生じる。ところが、上で言われたように、神の知は、知ら
れたものどもの原因である。ゆえに、それ[=神の知]が必然的なのだから、
それによって知られたものも必然的である。ゆえに、神の知は、非必然的なも
のを対象としない。

\\


2 {\scshape Praeterea}, omnis conditionalis cuius antecedens est
necessarium absolute, consequens est necessarium absolute. Sic enim se
habet antecedens ad consequens, sicut principia ad conclusionem, ex
principiis autem necessariis non sequitur conclusio nisi necessaria,
ut in I {\itshape Poster}.~probatur. Sed haec est quaedam
conditionalis vera, {\itshape si Deus scivit hoc futurum esse, hoc
erit}, quia scientia Dei non est nisi verorum. Huius autem
conditionalis antecedens est necessarium absolute, tum quia est
aeternum; tum quia significatur ut praeteritum. Ergo et consequens est
necessarium absolute. Igitur quidquid scitur a Deo, est
necessarium. Et sic scientia Dei non est contingentium.


&

さらに、前件が無条件に必然的であるような条件文の後件は、無条件に必然的
である\footnote{省略的で誤解を招く表現だが、正確な内容は、「ある真であ
る条件文があり、その前件が必然的に真であるならば、後件もまた必然的に真
である」ということ。$\alpha \supset \beta$と$\Box \alpha$から、$\Box
\beta$を導いてよい、という主張。しかしこれは一般には認められず、正しく
は、$\Box (\alpha \supset \beta)$と$\Box \alpha$からであれば、$\Box
\beta$を導いてもよい。$\Box (\alpha \supset \beta) \supset (\Box
\alpha \supset \Box \beta)$は、現代の様相論理でも$K$と呼ばれる有名な公
理。ちなみに、条件文が必然的に真でない場合、たとえば、$\alpha \supset
\beta$が偽である可能世界では、$\alpha$が真であっても、そこから$\beta$
が真であることは導かれない。それゆえ、$\alpha$が必然的に真であっても、
$\beta$が必然的に真とは限らない。トマスは第二異論解答で、この点を突
く。}。なぜなら、前件と後件は、原理と結論と同じように関係するが、『分
析論後書』1巻で証明されているように、必然的な原理からは、必然的な結論
しか生じないからである。ところで、「もし、神がこのことが将来あるだろう
ということを知っていたならば、それはあるだろう」は、一種の、真の条件文
である。なぜなら、神の知の対象は真である以外にないからである。また、こ
の条件文の前件は、無条件に必然的である。なぜなら、それは永遠のものだか
らであり、また、過去のものとして表示されているからである。ゆえに、後件
も、無条件に必然的である。ゆえに、神によって知られているものはなんであ
れ、必然的である。このようにして、神の知は、非必然的なものを対象としな
い。

\\


3 {\scshape Praeterea}, omne scitum a Deo necesse est esse, quia etiam
omne scitum a nobis necesse est esse, cum tamen scientia Dei certior
sit quam scientia nostra. Sed nullum contingens futurum necesse est
esse. Ergo nullum contingens futurum est scitum a Deo.

&

さらに、すべて神によって[存在するものとして\footnote{原文にはないが、
論旨の観点から補う。}]知られるものは、存在することが必然である。なぜ
なら、私たちによって[存在するものとして]知られるすべてのものもまた、
存在することが必然だが、神の知は、私たちの知よりも確実だからである。と
ころが、未来の非必然的なものは、どれも、存在することが必然的なものでは
ない。ゆえに、どんな未来の非必然的なものも、神によって知られていない。

\\


{\scshape Sed contra est} quod dicitur in {\itshape Psalmo} {\scshape
xxxii}, {\itshape qui finxit singillatim corda eorum, qui intelligit
omnia opera eorum}, scilicet hominum. Sed opera hominum sunt
contingentia, utpote libero arbitrio subiecta. Ergo Deus scit futura
contingentia.


&

しかし反対に、『詩編』32節で「彼ら(つまり人間)の心を一つずつ作り、彼
らのすべての業を理解する者」\footnote{「人の心をすべて造られた主は、彼
らの業をことごとく見分けられる」(33:15)}と言われている。しかし、人間の
業は、自由選択に従うものとして非必然的である。ゆえに、神は未来の非必然
的なことを知る。

\\


{\scshape Respondeo dicendum} quod, cum supra ostensum sit quod Deus
sciat omnia non solum quae actu sunt, sed etiam quae sunt in potentia
sua vel creaturae; horum autem quaedam sunt contingentia nobis futura;
sequitur quod Deus contingentia futura cognoscat.


&

解答する。以下のように言われるべきである。神は、たんに現実態において存
在するものだけでなく、自分の、あるいは被造物の能力の中にあるものも知る
ということが上で示されたので、そして、それらのうちのあるものは、私たち
にとって、未来の非必然的な事柄であるから、神が未来の非必然的な事柄を知
るということが帰結する。

\\


Ad cuius evidentiam, considerandum est quod contingens aliquod
dupliciter potest considerari. Uno modo, in seipso, secundum quod iam
actu est. Et sic non consideratur ut futurum, sed ut praesens, neque
ut ad utrumlibet contingens, sed ut determinatum ad unum. Et propter
hoc, sic infallibiliter subdi potest certae cognitioni, utpote sensui
visus, sicut cum video Socratem sedere.

&

これを明らかにするためには、ある非必然的なものが、二つの観点から考察さ
れるということが考えられるべきである。一つには、それがすでに現実に存在
するという観点から、それ自体において考察される。この意味では、それは未
来のこととしてではなく、現在のこととして、また、どちらにもなりうるもの
としてではなく、一つに定まったものとして、考察される。このために、視覚
のような特定の認識のもとに、不可謬的に、従属しうる。たとえば、私がソク
ラテスが座っているのを見るときのように\footnote{今、ソクラテスが座って
いることは、必然的な事柄ではないが、視覚はそれを明らかに認識する。この
ように、現在の非必然的な事柄の認識は、trivialに可能。問題は、
\kenten{未来}の非必然的な事柄の認識。}。

\\


Alio modo potest considerari contingens, ut est in sua causa. Et sic
consideratur ut futurum, et ut contingens nondum determinatum ad unum,
quia causa contingens se habet ad opposita. Et sic contingens non
subditur per certitudinem alicui cognitioni. Unde quicumque cognoscit
effectum contingentem in causa sua tantum, non habet de eo nisi
coniecturalem cognitionem.


&

もう一つには、その原因の中にあるものとして、非必然的なものが考えられう
る。この場合、それは未来のこととして、まだ一つに定まっていない非必然的
なこととして考えられる。非必然的な原因は、反対のものに関係するからであ
る。そしてこの場合、非必然的なものは、確実性を伴って、ある認識に従属す
ることがない。したがって、非必然的な結果を、その原因の中だけに認識する
者はだれでも、それについて、推測的な認識しか持たない。\footnote{因果関
係は、必ずしも必然的なものとは限らない。}

\\


Deus autem cognoscit omnia contingentia, non solum prout sunt in suis
causis, sed etiam prout unumquodque eorum est actu in seipso.  Et
licet contingentia fiant in actu successive, non tamen Deus successive
cognoscit contingentia, prout sunt in suo esse, sicut nos, sed
simul. Quia sua cognitio mensuratur aeternitate, sicut etiam suum
esse, aeternitas autem, tota simul existens, ambit totum tempus, ut
supra dictum est. Unde omnia quae sunt in tempore, sunt Deo ab aeterno
praesentia, non solum ea ratione qua habet rationes rerum apud se
praesentes, ut quidam dicunt, sed quia eius intuitus fertur ab aeterno
super omnia, prout sunt in sua praesentialitate.

&

しかし、神は、全ての非必然的なことを、その原因においてだけでなく、それ
らの各々が現実に自らの中に存在するものとしても認識する。また、非必然的
なものは、継起的に、現実に生じるが、神はそれら非必然的なものどもを、ちょ
うど私たちがするように、それら自身の存在においてそうであるように、継起
的に認識するのではなく、同時に認識する。なぜなら、神の認識は、神の存在
と同様、永遠によって測られているが、永遠は、上で述べられたとおり、全体
が同時に存在し、すべての時間を巡る。\footnote{cf.~{\itshape ST} I,
q.10, a.1, c.} したがって、すべて時間のうちに存在するものは、神にとっ
て、永遠から現在であり、それは、ある人々が言うように、諸事物の理念が神
のもとに現在するから、というだけでなく、神の直観が、永遠から、自らの現
在性において存在するものとしての万物へと向けられているからである。
\footnote{神は未来の非必然的な事柄を\kenten{予知}するのではない。それ
らを、神の現在において、目の当たりに見る。}


\\


Unde manifestum est quod contingentia et infallibiliter a Deo
cognoscuntur, inquantum subduntur divino conspectui secundum suam
praesentialitatem, et tamen sunt futura contingentia, suis causis
comparata.


&

したがって、非必然的なものは、神の視野の中に、神の現在性において従属す
るというかぎりで、神によって、不可謬的にも認識されることが明らかである。
しかし、それらは、自らの原因との関係においては、未来の非必然的な事柄で
ある。

\\


{\scshape Ad primum ergo dicendum} quod, licet causa suprema sit
necessaria, tamen effectus potest esse contingens, propter causam
proximam contingentem, sicut germinatio plantae est contingens propter
causam proximam contingentem, licet motus solis, qui est causa prima,
sit necessarius. Et similiter scita a Deo sunt contingentia propter
causas proximas, licet scientia Dei, quae est causa prima, sit
necessaria.

&

第一異論に対しては、それゆえ、次のように言われるべきである。たとえば、
第一原因である太陽の運動は必然的なのに、最近接原因が非必然的であるため、
植物の発芽が非必然的となるように、最高の原因が必然的であっても、最近接
原因が非必然的であるために、結果が非必然的となることはありうる
\footnote{$\Box \alpha$と$\alpha \supset \beta$とから導かれるのは
$\beta$であって$\Box \beta$ではない。}。同様に、神によって知られたもの
が非必然的であるのは、最近接原因のためである。この場合でも、第一原因で
ある神の知は必然的である。


\\


{\scshape Ad secundum dicendum} quod quidam dicunt quod hoc
antecedens, {\itshape Deus scivit hoc contingens futurum}, non est
necessarium, sed contingens, quia, licet sit praeteritum, tamen
importat respectum ad futurum. -- Sed hoc non tollit ei necessitatem,
quia id quod habuit respectum ad futurum, necesse est habuisse, licet
etiam futurum non sequatur quandoque.

&

第二異論に対しては、次のように言われるべきである。ある人々は、「神は未
来の非必然的な事柄を知っていた」という前件が、たしかに過去のことだけれ
ども、それは未来への関係を意味しているので、必然的ではなく、非必然的な
ものだと言う。\footnote{$\Box \alpha$を否定することで、$\Box \beta$が
導かれるのを避けようとした人々の意見(その1)。従属節の内容が未来の事
柄を意味しているとき、過去の事実を表す主節から必然性を除去する、という
主張。}しかし、この論拠は、この前件から必然性を除去しない。なぜなら、
未来への関係をもっていたものが、その関係をもっていた、ということは必然
的だから。かりに、その未来のことが続いて起こらないことがあるとしても。
\footnote{「太郎は『明日は雨が降るだろう』と思った」。という文で、仮に
その予想が外れたとしても、太郎がそう思ったという事実は変わらない。}


\\


Alii vero dicunt hoc antecedens esse contingens, quia est compositum
ex necessario et contingenti; sicut istud dictum est contingens,
{\itshape Socratem esse hominem album}. -- Sed hoc etiam nihil
est. Quia cum dicitur, {\itshape Deus scivit esse futurum hoc
contingens}, {\itshape contingens} non ponitur ibi nisi ut materia
verbi, et non sicut principalis pars propositionis, unde contingentia
eius vel necessitas nihil refert ad hoc quod propositio sit necessaria
vel contingens, vera vel falsa. Ita enim potest esse verum me dixisse
hominem esse asinum, sicut me dixisse Socratem currere, vel Deum esse,
et eadem ratio est de necessario et contingenti.

&

また、別の人々は、ちょうど「ソクラテスは白い人間である」と言われたこと
が非必然的であるように、この前件は、必然的なものと非必然的なものとから
複合されているので非必然的だと言う\footnote{$\Box \alpha$を否定するこ
とで、$\Box \beta$が導かれるのを避けようとした人々の意見(その2)。従
属節の内容が非必然的な内容を含むとき、主節の必然性が除去される、という
主張。}。しかし、これもダメである。なぜなら「神は、これが未来の非必然
的なものであることを知っていた」と言われるとき、この「非必然的なもの」
は、動詞[「知っていた」]の内容として置かれているのであり、命題の主要
部分として置かれているのではない。したがって、その[動詞の内容となって
いる]非必然性や必然性は、命題が必然的か非必然的かとか、あるいは真か偽
かということには関係しない。たとえば、「私が、「人間はロバである」と言っ
た」が真でありうるように。それは、「私が「ソクラテスが走っている」と言っ
た」や「私が「神が存在する」と言った」が真でありうるのと同様である。必
然的であるとか非必然的であるということについても、同様に論じられる。

\\


Unde dicendum est quod hoc antecedens est necessarium absolute. Nec
tamen sequitur, ut quidam dicunt, quod consequens sit necessarium
absolute, quia antecedens est causa remota consequentis, quod propter
causam proximam contingens est. -- Sed hoc nihil est. Esset enim
conditionalis falsa, cuius antecedens esset causa remota necessaria,
et consequens effectus contingens, ut puta si dicerem, {\itshape si
sol movetur, herba germinabit}.

&

したがって、この前件は無条件に必然的だと言われるべきである\footnote{こ
こで明示されるように、トマスは$\Box \alpha$を認める。}。しかし、ある人々
は次のように言う。前件は後件の遠因であり、後件はその近接原因のために非
必然的なものなので、このことから後件が無件的に必然的であることは帰結し
ない、と\footnote{$\Box \alpha$であったとしても$\Box(\alpha \supset
\beta)$でなくたんに$\alpha \supset \beta$であるならば、$\Box \beta$は
帰結しない。}。------ しかしこれは間違いである。なぜなら、もし仮に私が
「もし太陽が動くなら、牧草が芽吹くだろう」と言ったとしたならば、その前
件が必然的な遠因であり、またその後件が非必然的な結果であるそのような条
件文は偽だっただろうからである\footnote{{\itshape De Veritate} q.2,
a.12, ad7では類似した条件文がはっきりと偽と言われているので、この反事
実的条件文の前件は、``si dicerem''と解釈する。 Cf. ``But it is to no
purpuse. For the conditional would be false were its antecedent the
remote necessary cause, and the consequent a contingent effect; as,
for example, if I said, `if the sun moves, the grass will grow.'
(Blackfriar);「しかしこの説も無意味である。なぜならば、この説に従うな
らば、「もしも太陽が動くならば、草が生えるであろう」という場合のように、
その先件は遠い必然的原因であり、帰結は非必然的結果であるような条件文は
偽となるであろうからである」(山田); ``But this amounts to nothing. For
the proposition would be false if its antecedent were a remote
necessary cause and its consequent were a contingent effect—as, for
instance, if I were to say, ‘If the sun moves, the plant will
germinate.’''(Freddoso)}。

\\


Et ideo aliter dicendum est, quod quando in antecedente ponitur
aliquid pertinens ad actum animae, consequens est accipiendum non
secundum quod in se est, sed secundum quod est in anima, aliud enim
est esse rei in seipsa, et esse rei in anima. Ut puta, si dicam,
{\itshape si anima intelligit aliquid, illud est immateriale},
intelligendum est quod illud est immateriale secundum quod est in
intellectu, non secundum quod est in seipso. Et similiter si dicam,
{\itshape si Deus scivit aliquid, illud erit}, consequens
intelligendum est prout subest divinae scientiae, scilicet prout est
in sua praesentialitate. Et sic necessarium est, sicut et antecedens,
{\itshape quia omne quod est, dum est, necesse est esse}, ut dicitur
in I {\itshape Periherm}.


&

それゆえ、別のかたちで以下のように言われるべきである\footnote{神の知と、
事実の間に成り立つ関係は、この世界の中の因果関係のように、非必然性を含
むものではありえない。つまりその関係は、$\alpha \supset \beta$ではなく
$\Box(\alpha \supset \beta)$である。それゆえ、$\Box \alpha$から$\Box
\beta$が帰結する。このことと、神の知が非必然的なものの知であることをど
う両立させるかが、このパラグラフの論点。トマスの方針は、「必然的」とい
う様相に多義性を認めること。}。前件の中に、魂の作用に属する何かが置か
れるとき、後件は、それ自体においてあるかぎりでではなく、魂の中にあるか
ぎりで理解されるべきである\footnote{条件文において前件に思惟の対象が含
まれているならば、後件も思惟との関係において理解されるべきである、とい
う興味深い主張。}。なぜなら、それ自体における事物の存在と、魂における
事物の存在とは異なるからである。たとえば「もし魂がなにかを知性認識する
ならば、それは非質料的である」と私が言う場合、「それは非質料的である」
は、それがそれ自体で存在するという点ではなく、知性の中に存在するいう点
で、理解されるべきである。同様に、「もし神がなにかを知っていたならば、
将来それが在るだろう」と私が言うとき、後件は、神の知の中にあるものとし
て、つまり、神の現在性のうちにあるものとして、理解されるべきである。こ
の意味では、前件と同様に、[後件もまた]必然的である。「なぜなら、存在
するものはすべて、存在する間、存在することが必然であるから」。これは
『命題論』1巻で言われている。


\\


{\scshape Ad tertium dicendum} quod ea quae temporaliter in actum
reducuntur, a nobis successive cognoscuntur in tempore, sed a Deo in
aeternitate, quae est supra tempus. Unde nobis, quia cognoscimus
futura contingentia inquantum talia sunt, certa esse non possunt, sed
soli Deo, cuius intelligere est in aeternitate supra tempus. Sicut
ille qui vadit per viam, non videt illos qui post eum veniunt, sed
ille qui ab aliqua altitudine totam viam intuetur, simul videt omnes
transeuntes per viam.


&

第三異論に対しては、次のように言われるべきである。時間的に現実態へもた
らされるものどもは、私たちによって、時間の中で、継起的に認識されるが、
神によっては、時間の上位である永遠において認識される。したがって、私た
ちは未来の非必然的な事柄を、それらがそのようなものであるかぎりで認識す
るから、それらは私たちにとって確実なものではありえない。神の知性認識は、
時間の上位である永遠においてあるから、[それらが確実なものであるのは]
ただ神にとってのみである。たとえば、道を行く人は、彼の後から来る人々を
見ることがないが、しかし、どこか高いところから道全体を見る人は、道を通っ
ていくすべての人を同時に見るようなのものである。


\\

Et ideo illud quod scitur a nobis, oportet esse necessarium etiam
secundum quod in se est, quia ea quae in se sunt contingentia futura,
a nobis sciri non possunt. Sed ea quae sunt scita a Deo, oportet esse
necessaria secundum modum quo subsunt divinae scientiae, ut dictum
est, non autem absolute, secundum quod in propriis causis
considerantur.


&

ゆえに、私たちによって知られるものは、それ自体においてもまた、必然的な
ものでなければならない。なぜなら、それ自体において未来の非必然的なもの
は、私たちによって知られえないからである。しかし、神によって知られるも
のどもは、すでに述べられたとおり、神の知に従属するあり方に即しては必然
的でなければならないが、しかし、それぞれ固有の原因において考えられるか
ぎり、無条件に必然的なものでなくてもかまわない。\footnote{第二異論解答
で示された方針。}

\\

Unde et haec propositio, omne scitum a Deo necessarium est esse,
consuevit distingui. Quia potest esse de re, vel de dicto. Si
intelligatur de re, est divisa et falsa, et est sensus, omnis res quam
Deus scit, est necessaria. Vel potest intelligi de dicto, et sic est
composita et vera; et est sensus, hoc dictum, scitum a Deo esse, est
necessarium.


&

このことから、「神によって知られるすべてのものは、存在することが必然で
ある」という命題は、次のように区別されるのが慣例である。すなわち、それ
は、「事実にかんして」(de re)か、あるいは「表現にかんして」(de dicto)
でありうる。事実にかんして理解される場合、それは分割され\footnote {中
世の用語として、{\itshape de re}, {\itshape de dicto}の代わりに、
{\itshape in sensu diviso}, {\itshape in sensu composito}が用いられて
いた。cf.~'' While the de dicto/de re terminology was used, it was not
all that common. Medieval logicians preferred to use what they took to
be Aristotle's terminology, talking about modal sentences in the
composite sense (in sensu composito) and divided sense (in sensu
diviso). The structure of a composite modal sentence can be
represented as follows:
\begin{quote}
   (quantity/subject/copula, [quality]/predicate)mode
\end{quote}
A composite modal sentence corresponds to a de dicto modal
sentence. The word ‘composite’ is used because the mode is said to
qualify the composition of the subject and the predicate. The
structure of a divided modal sentence can be represented as follows:
\begin{quote}
quantity/subject/copula, mode, [quality]/predicate
\end{quote}
Here, the mode is thought to qualify the copula and thus to divide the
sentence into two parts (hence the name, ‘divided modal
sentence’). This type of modal sentence was characterized as de re
because what is modified is how things (res) are related to each
other, rather than the truth of what is said by the sentence (dictum)
(see Lagerlund 2000: 35–39, and the entry on medieval theories of
modality for further details).'' (``Medieval Theories of the
Syllogism',' in {\itshape Stanford Encyclopedia of Philosophy},
http://plato.stanford.edu/entries/medieval-syllogism/) }、偽である。そ
の意味は、「神が知るすべての事物は、必然的なものである」となる。あるい
は、この命題は表現にかんして理解され、その場合、複合され、真である。そ
の意味は、「『神によって知られたものが存在する』というこの表現は、必然
的である」となる。\footnote{様相述語論理では、様相演算子のスコープの中
に自由変項が出現しないとき``de dicto''、出現するとき、``de re''と言わ
れる。$D = ${神によって知られているもの}とすると、$\forall x \Box
Exist(x)$(「すべて神によって知られているものは必然的に存在する」)は
偽だが、$\Box \forall x Exist(x)$(「必然的に、すべて神に知られている
ものは存在する」)は真。}


\\

Sed obstant quidam, dicentes quod ista distinctio habet locum in
formis separabilibus a subiecto; ut si dicam, album possibile est esse
nigrum. Quae quidem de dicto est falsa, et de re est vera, res enim
quae est alba, potest esse nigra; sed hoc dictum, album esse nigrum,
nunquam potest esse verum. In formis autem inseparabilibus a subiecto,
non habet locum praedicta distinctio; ut si dicam, corvum nigrum
possibile est esse album, quia in utroque sensu est falsa. Esse autem
scitum a Deo, est inseparabile a re, quia quod est scitum a Deo, non
potest esse non scitum.


&

しかし、ある人々は、これに反対して次のように言う。たとえば「白は黒であ
りうる」と言う場合のように、この区別は、基体から分離可能な形相について
は成り立つ。たしかにこの命題は、表現にかんして偽だが、事実にかんしては
真である。なぜなら、白いものは黒いものでありうるが、「白いものは黒いも
のである」というこの表現は、けっして真でありえないからである。しかし、
基体から分離不可能な形相においては、このような区別は成り立たない。たと
えば「黒いカラスが白くありうる」は、どちらの意味でも偽である。
\footnote{カラスは必然的に黒いと仮定する。}ところが、神によって知られ
ているということは、事物から分離不可能である。なぜなら、神によって知ら
れているものは、[神に]知られていないことがありえないからである。



\\


Haec autem instantia locum haberet, si hoc quod dico scitum,
importaret aliquam dispositionem subiecto inhaerentem. Sed cum
importet actum scientis, ipsi rei scitae, licet semper sciatur, potest
aliquid attribui secundum se, quod non attribuitur ei inquantum stat
sub actu sciendi, sicut esse materiale attribuitur lapidi secundum se,
quod non attribuitur ei secundum quod est intelligibile.


&

しかし、この例は、もしこの、私が「知られている」と言うものが基体に内属
する何らかの状態[=本質的属性]を意味していたならば、成立したであろう。
しかし、それ[=「知られている」ということ]は知るものの働きを意味する
のだから、知られている事物は、たしかに[神によって]常に知られているけ
れども、[そのことは、基体に内属する状態ではない。ただし、]
\footnote{この括弧内の言葉が省略されていると解釈する。}知る働きのもと
にあるかぎりでそれに帰せられない何かが、それ自体において、その事物に帰
せられることはありうる。たとえば、質料的であるということは、石に、それ
自体において帰せられるのであって、その石が可知的なものであるかぎりで帰
せられるのではない。\footnote{この箇所は、「神に知られている」という関
係が、被造物は常に神に知られているにもかかわらず、被造物に必然的に内在
するのではないという主張をしているように見える。ここでも、必然性の多義
性が考えられている。}


\\
\end{longtable}
\newpage



\rhead{a.14}
\begin{center}
{\Large {\bf ARTICULUS DECIMUSQUARTUS}}\\ {\large UTRUM DEUS COGNOSCAT
ENUNTIABILIA}\\ {\footnotesize I {\itshape Sent.}, d.38, a.3; d.41,
a.5; I {\itshape SCG.}, c.58, 59; {\itshape De Verit.}, q.2,
a.7.}\\ {\Large 第十四項\\神は命題的なことがらを認識するか}
\end{center}

\begin{longtable}{p{21em}p{21em}}



{\huge A}{\scshape d decimumquartum sic proceditur}. Videtur quod Deus
non cognoscat enuntiabilia. Cognoscere enim enuntiabilia convenit
intellectui nostro, secundum quod componit et dividit. Sed in
intellectu divino nulla est compositio. Ergo Deus non cognoscit
enuntiabilia.


&

第十四項の問題へ議論は以下のように進められる。神は命題的なことがらを認
識しないと思われる。理由は以下の通り。命題的なことがらを認識することは、
複合分割するという点で、私たちの知性に適合する。ところが、神の知性の中
には、どんな複合もない。ゆえに、神は命題的なことがらを認識しない。

\\


{\scshape 2. Praeterea}, omnis cognitio fit per aliquam
similitudinem. Sed in Deo nulla est similitudo enuntiabilium, cum sit
omnino simplex. Ergo Deus non cognoscit enuntiabilia.


&

さらに、すべて認識は何らかの類似を通して生じる。ところが神はあらゆる点
で単純だから、神の中には、命題的なことがらのどんな類似もない。ゆえに、
神は命題的なことがらを認識しない。

\\


{\scshape Sed contra est} quod dicitur in Psalmo {\scshape xciii},
{\itshape dominus scit cogitationes hominum}. Sed enuntiabilia
continentur in cogitationibus hominum. Ergo Deus cognoscit
enuntiabilia.


&

しかし反対に、『詩編』93章で「主は、人間たちの思惟を知る」
\footnote{「主は知っておられる、人間の計らいを。それがいかに空しいかを」
(94:11)}と言われている。ところが、人間の思惟の中には命題的なことがらが
含まれる。ゆえに、神は命題的なことがらを認識する。

\\


{\scshape Respondeo dicendum} quod, cum formare enuntiabilia sit in
potestate intellectus nostri; Deus autem scit quidquid est in potentia
sua vel creaturae, ut supra dictum est; necesse est quod Deus sciat
omnia enuntiabilia quae formari possunt. Sed, sicut scit materialia
immaterialiter, et composita simpliciter, ita scit enuntiabilia non
per modum enuntiabilium, quasi scilicet in intellectu eius sit
compositio vel divisio enuntiabilium; sed unumquodque cognoscit per
simplicem intelligentiam, intelligendo essentiam uniuscuiusque.

&

解答する。以下のように言われるべきである。命題的なことがらを形成するこ
とは、私たちの知性の力の中にあるが、上で言われたように、神は、自分ので
あれ被造物のであれ、その能力の中にあるものをすべて知るのだから、神が、
形成されうるすべての命題的なことがらを知っていることは必然である。しか
し、質料的なものを非質料的なかたちで知り、複合されたものを単純なかたち
で知るように、命題的なことがらを、命題的なことがらのあり方によって、あ
たかも、神の知性の中に、命題的なことがらの複合や分割があるというしかた
で知るのではなく、それぞれを、それぞれの本質を知性認識することによって、
単純な知性認識によって認識する。

\\


Sicut si nos in hoc ipso quod intelligimus quid est homo,
intelligeremus omnia quae de homine praedicari possunt. Quod quidem in
intellectu nostro non contingit, qui de uno in aliud discurrit,
propter hoc quod species intelligibilis sic repraesentat unum, quod
non repraesentat aliud. Unde, intelligendo quid est homo, non ex hoc
ipso alia quae ei insunt, intelligimus; sed divisim, secundum quandam
successionem.



&

それはちょうど、私たちが、人間とはなにかを認識することにおいて、人間に
述語されうるすべてのことがらを知性認識するとしたら[そうであるようなし
かたによってである]。しかし、このようなことは、私たちの知性においては
生じない。なぜなら、私たちの知性は、可知的形象が、他のものを表現しない
というかたちで一つのものを表現するために、一つのものから他のものへと推
論するからである。したがって、私たちは、人間とはなにかを認識するときに、
そのこと自体に基づいて、人間に内在する他のことがらを知るのではなく、分
割的に、一種の継起というかたちで、知性認識する。

\\


Et propter hoc, ea quae seorsum intelligimus, oportet nos in unum
redigere per modum compositionis vel divisionis, enuntiationem
formando. Sed species intellectus divini, scilicet eius essentia,
sufficit ad demonstrandum omnia. Unde, intelligendo essentiam suam,
cognoscit essentias omnium, et quaecumque eis accidere possunt.


&

そしてこのために、私たちは、別々に知性認識することがらを、複合や分割と
いうかたちで命題を形成することによって、一つのものへともどさなければな
らない。ところが、神の可知的形象、つまり神の本質は、万物を明示するのに
十分である。したがって、自分の本質を知性認識することによって、万物の本
質と、なんであれそれに起こりうる事柄を、認識する。

\\


{\scshape Ad primum ergo dicendum} quod ratio illa procederet, si Deus
cognosceret enuntiabilia per modum enuntiabilium.


&

第一異論に対しては、それゆえ、次のように言われるべきである。この論は、
もし、神が、命題的なことがらを命題的なことがらのありかたによって認識す
るのだったら、成り立ったであろう。

\\


{\scshape Ad secundum dicendum} quod compositio enuntiabilis
significat aliquod esse rei, et sic Deus per suum esse, quod est eius
essentia, est similitudo omnium eorum quae per enuntiabilia
significantur.

&


第二異論に対しては、次のように言われるべきである。命題的なことがらの複
合は、事物の何らかの存在を表示する。この意味で、神は、その本質である自
分の存在によって、命題的なことがらによって表示されるすべてのものの類似
である。

\end{longtable}
\newpage



\rhead{a.15}
\begin{center}
{\Large {\bf ARTICULUS DECIMUSQUINTUS}}\\ {\large UTRUM SCIENTIA DEI
SIT VARIABILIS}\\ {\footnotesize I {\itshape Sent.}, d.38, a.2;
d.39, q.1, a.1, 2; d.41, a.5; {\itshape De Verit.}, q.2, a.5,
ad 11; a.13.}\\ {\Large 第十五項\\神の知は可変的か}
\end{center}

\begin{longtable}{p{21em}p{21em}}


{\huge A}{\scshape d decimumquintum sic proceditur}. Videtur quod
scientia Dei sit variabilis. Scientia enim relative dicitur ad
scibile. Sed ea quae important relationem ad creaturam, dicuntur de
Deo ex tempore, et variantur secundum variationem creaturarum. Ergo
scientia Dei est variabilis, secundum variationem creaturarum.


&

第十五項の問題へ議論は以下のように進められる。神の知は可変的だと思われ
る。理由は以下の通り。知は知られうるものに対して関係的に語られる。とこ
ろが、被造物への関係を意味することがらは、神について時間的に語られ、被
造物の変化にしたがって変化する\footnote{第1部第13問第7項参照。}。ゆえ
に、神の知は、被造物の変化にしたがって、可変的である。


\\


{\scshape 2.~Praeterea}, quidquid potest Deus facere, potest
scire. Sed Deus potest plura facere quam faciat. Ergo potest plura
scire quam sciat. Et sic scientia sua potest variari secundum
augmentum et diminutionem.

&

さらに、神は、なんであれ作ることができるものを知ることができる。ところ
が、神は、実際に作るよりも多くのものを作ることができる。ゆえに、神は、
知っているものよりも多くのものを知ることができる。このようにして、神の
知は、増加と減少に応じて変化しうる。


\\


{\scshape 3.~Praeterea}, Deus scivit Christum nasciturum. Nunc autem
nescit Christum nasciturum, quia Christus nasciturus non est. Ergo non
quidquid Deus scivit, scit. Et ita scientia Dei videtur esse
variabilis.

&

さらに、神は、キリストが生まれるであろうことを知っていた。ところが、神
は今、キリストが生まれるであろうことを知っていない。なぜなら、キリスト
が生まれるであろうということは[今、真では]ないからである。ゆえに、な
んであれ神が知っていたことすべてを、今、神が知っているわけではない。こ
のようにして、神の知は可変的だと思われる。

\\


{\scshape Sed contra est} quod dicitur Iac.~{\scshape i}, quod apud
Deum {\itshape non est transmutatio, neque vicissitudinis obumbratio}.

&

しかし反対に、『ヤコブの手紙』1章で、神のもとには「変化も、転変のかげ
りもない」\footnote{「御父には、移り変わりも、天体の動きにつれて生ずる
陰もありません」(1:17)}と言われている。

\\


{\scshape Respondeo dicendum} quod, cum scientia Dei sit eius
substantia, ut ex dictis patet; sicut substantia eius est omnino
immutabilis, ut supra ostensum est, ita oportet scientiam eius omnino
invariabilem esse.

&

解答する。以下のように言われるべきである。すでに述べられたことから明ら
かなとおり、神の知は神の実体である。また、上で示されたように、神の実体
はあらゆる点で不変であるように、神の知も、あらゆる点で、不可変的でなけ
ればならない。

\\


{\scshape Ad primum ergo dicendum} quod {\itshape dominus} et
{\itshape creator}, et huiusmodi, important relationes ad creaturas
secundum quod in seipsis sunt. Sed scientia Dei importat relationem ad
creaturas secundum quod sunt in Deo, quia secundum hoc est unumquodque
intellectum in actu quod est in intelligente. Res autem creatae sunt
in Deo invariabiliter, in seipsis autem variabiliter.

&

第一異論に対しては、それゆえ、次のように言われるべきである。「主」や
「創造者」などは、被造物への関係を意味するが、その関係は、被造物自身で
あるかぎりでの被造物への関係である。他方、神の知も、被造物への関係を意
味するが、その関係は神のうちにあるかぎりでの被造物への関係である。とい
うのも、各々のものが現実に知性認識されたものであるのは、それが知性認識
するもののうちにあるかぎりでだから。ところで、創造された事物は、神のう
ちに、不可変的なかたちで存在する。ただし、被造物自身においては、可変的
なかたちで存在するのだが。

\\


-- Vel aliter dicendum est, quod {\itshape dominus} et {\itshape
creator}, et huiusmodi, important relationes quae consequuntur actus
qui intelliguntur terminari ad ipsas creaturas secundum quod in
seipsis sunt, et ideo huiusmodi relationes varie de Deo dicuntur,
secundum variationem creaturarum. Sed scientia et amor, et huiusmodi,
important relationes quae consequuntur actus qui intelliguntur in Deo
esse, et ideo invariabiliter praedicantur de Deo.

&

あるいは、次のようにも言われるべきである。「主」「創造者」などは、ある
働きに伴う関係を意味するが、その働きは、被造物としての被造物に終局する
と理解される。それゆえ、そのような関係は、被造物の変化に応じて、神につ
いてさまざまに語られる。しかし、「知」や「愛」などは、神の中にあると理
解される働きに伴う関係を意味する。ゆえに、神について、不可変的に述語さ
れる。



\\


{\scshape Ad secundum dicendum} quod Deus scit etiam ea quae potest
facere et non facit. Unde ex hoc quod potest plura facere quam facit,
non sequitur quod possit plura scire quam sciat, nisi hoc referatur ad
scientiam visionis, secundum quam dicitur scire ea quae sunt in actu
secundum aliquod tempus.


&

第二異論に対しては次のように言われるべきである。神は、「作ることができ
るけれども作らないもの」も知る。したがって、実際に作るよりも多くのもの
を作ることができる、ということから、実際に知っているよりも多くのものを
知ることができる、ということは帰結しない。ただし、ある特定の時間に現実
に存在するものを知るという意味での「直視の知」について言われているので
あればこのかぎりではない。

\\


Ex hoc tamen quod scit quod aliqua possunt esse quae non sunt, vel non
esse quae sunt, non sequitur quod scientia sua sit variabilis, sed
quod cognoscat rerum variabilitatem. Si tamen aliquid esset quod prius
Deus nescivisset et postea sciret, esset eius scientia variabilis. Sed
hoc esse non potest, quia quidquid est vel potest esse secundum
aliquod tempus, Deus in aeterno suo scit.


&

しかし、「存在しないものが存在しうる」と知ることや、「存在するものが存
在しないことがありうる」と知ることから、神の知が可変的であることは帰結
せず、むしろ、神が事物の可変性を認識することが帰結する。しかし、もし、
あるものが存在し、それを以前には神が知らず、後に知った、ということがあ
れば、その神の知は可変的だっただろう。しかし、こういうことはありえない。
なぜなら、何らかの時間に存在し、あるいは存在しうるものは、なんであれ、
神がそれを自らの永遠において知るからである。

\\


Et ideo ex hoc ipso quod ponitur aliquid esse secundum quodcumque
tempus, oportet poni quod ab aeterno sit scitum a Deo. Et ideo non
debet concedi quod Deus possit plura scire quam sciat, quia haec
propositio implicat quod ante nesciverit et postea sciat.

&

ゆえに、あるものが、どんな時間であれある時に存在するとされること自体か
ら、それが永遠から神によって知られているとされなければならない。ゆえに、
「神は、現に知っているよりも多くのことを知りうる」ということが譲歩して
認められるべきではない。なぜなら、この命題は、前に知らなかったが後で知
る、ということを意味しているからである。

\\


{\scshape Ad tertium dicendum} quod antiqui nominales dixerunt idem
esse enuntiabile, {\itshape Christum nasci}, et {\itshape esse
nasciturum}, et {\itshape esse natum}, quia eadem res significatur per
haec tria, scilicet nativitas Christi. Et secundum hoc sequitur quod
Deus quidquid scivit, sciat, quia modo scit Christum natum, quod
significat idem ei quod est Christum esse nasciturum.


&

第三異論に対しては、次のように言われるべきである。昔の唯名論者たちは、
「キリストが生まれる」「キリストが生まれるであろう」「キリストが生まれ
た」が、同一の命題だと言った。なぜなら、これら三つによって、同一の事態、
すなわち、キリストの誕生が表示されているから。この考えによれば、神は、
知っていたすべてのことを知っている、ということが帰結する。なぜなら、今、
キリストが生まれたと知っているが、神にとって、これは、「キリストが生ま
れるだろう」と同じことを意味しているからである。

\\


-- Sed haec opinio falsa est. Tum quia diversitas partium orationis
diversitatem enuntiabilium causat. Tum etiam quia sequeretur quod
propositio quae semel est vera, esset semper vera, quod est contra
philosophum, qui dicit quod haec oratio, {\itshape Socrates sedet},
vera est eo sedente, et eadem falsa est, eo surgente.


&

しかし、この意見は誤りである。一つにはそれは、文の部分の違いは、命題の
違いを結果として生み出すからであり、もう一つには、[この意見によれば]
いったん真であった命題は、常に真であることになるからである。しかしこれ
は、哲学者に反する。彼は、「ソクラテスが座っている」という文は、彼が座っ
ているときには真だが、彼が立ち上がるときには、その同じ文が偽となる、と
言っている。

\\


-- Et ideo concedendum est quod haec non est vera, {\itshape quidquid
Deus scivit, scit}, si ad enuntiabilia referatur. Sed ex hoc non
sequitur quod scientia Dei sit variabilis. Sicut enim absque
variatione divinae scientiae est, quod sciat unam et eandem rem
quandoque esse et quandoque non esse; ita absque variatione divinae
scientiae est, quod scit aliquod enuntiabile quandoque esse verum, et
quandoque esse falsum.


&

ゆえに、「なんであれ神が知っていたことを、神は知っている」は、命題的な
ことがらにかんする限り、偽であることを認めるべきである。しかしこのこと
から、神の知が可変的だということは帰結しない。ちょうど、一つの同一の事
物が、ある時には存在し、別の時には存在しないことを知るということが、神
の知が変化することなしにあるように、ある命題が、ある時には真であり別の
時には偽であることを知るということが、神の知が変化することなしにある。

\\


Esset autem ex hoc scientia Dei variabilis, si enuntiabilia
cognosceret per modum enuntiabilium, componendo et dividendo, sicut
accidit in intellectu nostro. Unde cognitio nostra variatur, vel
secundum veritatem et falsitatem, puta si, mutata re, eandem opinionem
de re illa retineamus, vel secundum diversas opiniones, ut si primo
opinemur aliquem sedere, et postea opinemur eum non sedere. Quorum
neutrum potest esse in Deo.

&

しかし、ちょうど、私たちの知性の中で起こっているように、命題的なことが
らを、命題的なあり方によって、複合分割することで知るのであったならば、
このことから、神の知は可変的だということになっただろう。私たちの認識は、
事物が変化するのに、その事物について同じ意見を保持する場合のように、真
理と虚偽という点で変化するし、また、最初、ある人が座っていると思ってい
るのに、後で、彼が座っていないと思う、というように、意見が異なるという
かたちでも、変化する。しかしこのどちらも、神の中ではありえない。


\end{longtable}
\newpage




\rhead{a.16}
\begin{center}
{\Large {\bf ARTICULUS DECIMUS SEXTUS}}\\ {\large UTRUM DEUS DE REBUS
HABEAT SCIENTIAM SPECULATIVAM}\\ {\footnotesize {\itshape De Verit.},
q.3, a.3.}\\ {\Large 第十六項\\神は事物について観照的知ををもつか}
\end{center}

\begin{longtable}{p{21em}p{21em}}


{\huge A}{\scshape d decimumsextum sic proceditur}. Videtur quod Deus
de rebus non habeat scientiam speculativam. Scientia enim Dei est
causa rerum, ut supra ostensum est. Sed scientia speculativa non est
causa rerum scitarum. Ergo scientia Dei non est speculativa.

&

第十六項の問題へ議論は以下のように進められる。神は事物について観照的知
をもたないと思われる。理由は以下の通り。上で示されたように、神の知は事
物の原因である。ところが、観照的知は知られた事物の原因ではない。ゆえに、
神の知は観照的でない。


\\


{\scshape 2 Praeterea}. Scientia speculativa est per abstractionem a
rebus, quod divinae scientiae non competit. Ergo scientia Dei non est
speculativa.

&

さらに、観照的知は、事物からの抽象によるが、これは神の知に適合しない。
ゆえに、神の知は観照的でない。

\\


{\scshape Sed contra}, omne quod est nobilius, Deo est
attribuendum. Sed scientia speculativa est nobilior quam practica, ut
patet per philosophum, in principio {\itshape Metaphys}. Ergo Deus
habet de rebus scientiam speculativam.

&

しかし反対に、すべて、より高貴なものは、神に帰せられるべきである。とこ
ろで、『形而上学』1巻の哲学者によって明らかなとおり、観照的知は、実践
的知よりも高貴である。ゆえに、神は事物について観照的な知をもつ。


\\


{\scshape Respondeo dicendum quod} aliqua scientia est speculativa
tantum, aliqua practica tantum, aliqua vero secundum aliquid
speculativa et secundum aliquid practica. Ad cuius evidentiam,
sciendum est quod aliqua scientia potest dici speculativa
tripliciter. Primo, ex parte rerum scitarum, quae non sunt operabiles
a sciente, sicut est scientia hominis de rebus naturalibus vel
divinis.


&

解答する。以下のように言われるべきである。たんに観照的である知もあれば、
たんに実践的な知もあるが、ある観点から見ると観照的だが、別の観点から見
ると実践的であるような知もある。これを明らかにするためには、ある知が観
照的だと言われるのに、三通りの意味があることが知られるべきである。一つ
には、知られる事物の側からであり、たとえば、自然的事物や神的事物にかん
する人間の知がそうであるように、その事物が、知るものによる働きの対象と
なりえない場合である。

\\


Secundo, quantum ad modum sciendi, ut puta si aedificator consideret
domum definiendo et dividendo et considerando universalia praedicata
ipsius. Hoc siquidem est operabilia modo speculativo considerare, et
non secundum quod operabilia sunt, operabile enim est aliquid per
applicationem formae ad materiam, non per resolutionem compositi in
principia universalia formalia.


&

第二には、知り方にかんしてであり、たとえば、建築家が、家を、定義し、分
析し、家の普遍的な述語を考察することによって考えるような場合である。こ
れは、働きの対象となりうるものを、働きの対象としてではなく、観照的なし
かたで考察することである。じっさい、あるものが働きの対象となりうるのは、
形相を質料に当てはめることによってであり、複合体を形相的で普遍的な根源
へと分解することによるのではない。

\\


Tertio, quantum ad finem, nam intellectus practicus differt fine a
speculativo, sicut dicitur in III de anima. Intellectus enim practicus
ordinatur ad finem operationis, finis autem intellectus speculativi
est consideratio veritatis. Unde, si quis aedificator consideret
qualiter posset fieri aliqua domus, non ordinans ad finem operationis,
sed ad cognoscendum tantum, erit, quantum ad finem, speculativa
consideratio, tamen de re operabili.


&

第三に、目的にかんしてであり、『デ・アニマ』3巻で述べられるように、実
践的知性は、観照的知性から、目的において異なる。つまり、実践的知性は、
働きの目的へと秩序づけられているが、観照的知性の目的は、真理の考察であ
る。したがって、ある建築家が、ある家がどのように作られうるかを、働きの
目的へと秩序づけることなく、たんに認識するために考察するならば、それは、
働きの対象となりうる事物にかんする考察ではあるが、目的にかんして、観照
的な考察であるだろう。


\\


Scientia igitur quae est speculativa ratione ipsius rei scitae, est
speculativa tantum. Quae vero speculativa est vel secundum modum vel
secundum finem, est secundum quid speculativa et secundum quid
practica. Cum vero ordinatur ad finem operationis, est simpliciter
practica. Secundum hoc ergo, dicendum est quod Deus de seipso habet
scientiam speculativam tantum, ipse enim operabilis non est. De
omnibus vero aliis habet scientiam et speculativam et practicam.



&

ゆえに、知られる事物自体の性格によって観照的であるような知は、たんに観
照的である。他方、知り方や目的の点で観照的である知は、ある点で観照的だ
が、別の点では実践的である。また、働きの目的へ秩序づけられているときは、
端的に実践的である。それゆえ、この意味で、神は自分自身について、たんに
観照的な知をもつと言われるべきである。なぜなら、神自身は、神の働きの対
象でありえないから。他方、他のすべてのものについて、神は、観照的知と実
践的知をもつ。


\\


Speculativam quidem, quantum ad modum, quidquid enim in rebus nos
speculative cognoscimus definiendo et dividendo, hoc totum Deus multo
perfectius novit. Sed de his quae potest quidem facere, sed secundum
nullum tempus facit, non habet practicam scientiam, secundum quod
practica scientia dicitur a fine. Sic autem habet practicam scientiam
de his quae secundum aliquod tempus facit.


&

知り方にかんして、神が観照的な知をもつのは、私たちが諸事物において観照
的に、定義し、分析することによって知ることのすべてを、神ははるかに完全
に知る。しかし、作ることができるけれどもどの時間にもそれを作らないもの
について、神は、実践的な知をもたないが、これは、実践的知が目的から語ら
れることによる。この意味で、何らかの時間に作るものどもについて、神は実
践的な知をもつ。

\\


Mala vero, licet ab eo non sint operabilia, tamen sub cognitione
practica ipsius cadunt, sicut et bona, inquantum permittit vel impedit
vel ordinat ea, sicut et aegritudines cadunt sub practica scientia
medici, inquantum per artem suam curat eas.

&

他方、悪は、神による働きの対象ではないが、しかし、善と同様、それを許し、
妨げ、秩序づけるという点で、神の実践的認識の中に含まれる。それはちょう
ど、病気が、医者の実践的知のもとに含まれるのと同様である。医者は、自分
の技術によって、病気を治癒するからである。

\\


{\scshape Ad primum ergo dicendum} quod scientia Dei est causa, non
quidem sui ipsius, sed aliorum, quorundam quidem actu, scilicet eorum
quae secundum aliquod tempus fiunt; quorundam vero virtute, scilicet
eorum quae potest facere, et tamen nunquam fiunt.

&

第一異論に対しては、それゆえ、次のように言われるべきである。神の知は、
自分自身の原因ではなく、自分以外のものの原因である。それらのあるものに
ついて、すなわち、ある時間に生じるものどもについては現実に、また、べつ
のものについて、すなわち、作ることはできるが、けっして生じないものにつ
いては、能力的に、原因である。


\\


{\scshape Ad secundum dicendum} quod scientiam esse acceptam a rebus
scitis, non per se convenit scientiae speculativae, sed per accidens,
inquantum est humana.

&

第二異論に対しては次のように言われるべきである。「知られる事物から知が
受け取られる」ということは、観照的知に自体的に適合するのではなく、それ
が人間の知であるという点で附帯的に適合するにすぎない。

\\


\hspace{1em}Ad id vero quod in contrarium obiicitur, dicendum quod de
operabilibus perfecta scientia non habetur, nisi sciantur inquantum
operabilia sunt. Et ideo, cum scientia Dei sit omnibus modis perfecta,
oportet quod sciat ea quae sunt a se operabilia, inquantum huiusmodi,
et non solum secundum quod sunt speculabilia. Sed tamen non receditur
a nobilitate speculativae scientiae, quia omnia alia a se videt in
seipso, seipsum autem speculative cognoscit; et sic in speculativa sui
ipsius scientia, habet cognitionem et speculativam et practicam omnium
aliorum.

&

反対異論については次のように言われるべきである。働きの対象となりうるも
のについては、それが、働きの対象となりうるという点で知られないかぎり、
完全な知は所有されない。ゆえに、神の知はあらゆる点で完全なのだから、自
分による働きの対象となりうるものについては、それが観照の対象になりうる
という点だけでなく、働きの対象となりうるという点でも、神はそれを知るの
でなければならない。しかし、だからといって、[神のそのような実践的な知
が]観照的知の高貴さから後退するわけではない。なぜなら、神は自己以外の
すべてのものを自分自身において見るが、神は自分自身を観照的に認識する。
それゆえ、自分自身について観照的知において、自分以外のすべてのものにつ
いての観照的認識と実践的認識を持つからである。




\end{longtable}
\newpage


\end{document}

APPENDIX Loc. Parall. of a.13

Utrum scientia Dei sit contingentium

[2797] Super Sent., lib. 1 d. 38 q. 1 a. 5 arg. 1 Ad quintum sic
proceditur. Videtur quod scientia Dei non sit contingentium. Scientia
enim Dei est causa omnium scitorum bonorum. Sed omnis causa necessaria
inducit necessarios effectus: posita enim causa, necessario ponitur
effectus, nisi causa sit deficiens in minori parte, sicut causa
naturalis. Cum ergo scientia Dei non sit deficiens, videtur quod id
cujus est, sit necessarium, et ita non sit contingentium.

[2798] Super Sent., lib. 1 d. 38 q. 1 a. 5 arg. 2 Praeterea, scientia
non est nisi verorum. Sed in futuris contingentibus non est alterum
determinate verum, ut philosophus probat. Ergo videtur quod non possit
esse eorum scientia divina.

[2799] Super Sent., lib. 1 d. 38 q. 1 a. 5 arg. 3 Praeterea, si sit
contingentium, ponatur quod Deus sciat Socratem currere; inde sic. Aut
est possibile Socratem non currere, aut impossibile. Si enim est
impossibile ipsum non currere, ergo ab aequipollenti necesse est ipsum
currere, et sic haberetur propositum, quod illud cujus est scientia
Dei, sit necessarium. Si autem possibile est ipsum non currere,
ponatur ergo. Possibile enim est, secundum philosophum, quo posito,
non sequitur inconveniens. Sed positum erat quod Deus sciret Socratem
currere. Ergo scit esse quod non est. Omnis autem talis scientia falsa
est. Ergo scientia Dei erit falsa, quod est impossibile. Relinquitur
ergo quod Socratem non currere, non fuit possibile sed necessarium.

[2800] Super Sent., lib. 1 d. 38 q. 1 a. 5 arg. 4 Si dicatur, quod est
necessarium necessitate consequentiae et non necessitate consequentis;
sive necessitate conditionata, hac scilicet conditione si est
praescitum, et non necessitate absoluta: contra, in omni vera
conditionali si antecedens est necessarium absolute, et consequens est
necessarium absolute: quia ad antecedens semper sequitur consequens,
et ad necessarium nunquam sequitur falsum, quamvis e converso, ut
probatur in 1 Poster. Sed hujus conditionalis, Socrates currit si est
praescitum a Deo, antecedens est necessarium absolute: tum quia omne
praeteritum est necessarium, tum quia omne aeternum est
necessarium. Ergo et Socratem currere est necessarium absolute: ergo
videtur quod scientia non est nisi de necessariis.

[2801] Super Sent., lib. 1 d. 38 q. 1 a. 5 arg. 5 Praeterea, secundum
philosophum in Lib. priorum, ex majori de necessitate, et minori de
inesse, sequitur conclusio de necessitate. Inde sic. Omne scitum a Deo
necesse est esse verum. Sed hoc est scitum a Deo. Ergo necesse est
esse verum; et sic idem quod prius.

[2802] Super Sent., lib. 1 d. 38 q. 1 a. 5 arg. 6 Si dicas, quod ista
est duplex: scitum a Deo necesse est esse verum: quia potest esse de
dicto; et sic est composita et vera: vel potest esse de re; et sic est
divisa et falsa: hoc enim dictum est necessarium, scilicet scitum a
Deo est verum; sed hanc rem quae ponitur sciri a Deo non necesse est
fieri, quia contingenter fit: contra, ista distinctio tenet tantum in
formis separabilibus, in quibus subjectum esse potest, non autem in
formis inseparabilibus a subjecto. Unde istam nullus distinguit:
cygnum album possibile est esse nigrum: quia albedo a cygno non
separatur nisi secundum rationem et intellectum; unde quod non potest
simul esse cum albo, non potest simul esse cum cygno. Sed in hoc quod
dicitur scitum, importatur quiddam semper concomitans subjectum: quia
quod semel est scitum a Deo, non potest non esse scitum ab eo. Ergo
videtur quod distinctio illa nihil ad propositum valeat.

[2803] Super Sent., lib. 1 d. 38 q. 1 a. 5 arg. 7 Si dicas quod
contingens quando futurum est, potest, non esse; sed non ex quo
ponitur praesens vel praeteritum; et sic Deus, cognitionem ejus habet:
contra, omnis scientia quae est de aliquo praesente, de quo non erat
quando futurum erat, recipit additionem secundum temporum
successionem. Sed scientia Dei nihil accipit additionis a rebus, nec
est in ea aliqua successio secundum tempora. Ergo si futura
contingentia non cognoscit, ut futura sunt, nullo modo est
contingentium, ut sunt praesentia vel praeterita.

[2804] Super Sent., lib. 1 d. 38 q. 1 a. 5 s. c. 1 Sed contra, ea quae
subsunt libero arbitrio sunt maxime contingentia. Sed horum
cognitionem Deus habet: alias non redderet unicuique secundum opera
sua. Ergo sua scientia est contingentium.

[2805] Super Sent., lib. 1 d. 38 q. 1 a. 5 s. c. 2 Praeterea, supra
habitum est quod scientia Dei est omnium. Sed non omnia ex necessitate
contingunt, ut ad ipsum sensum patet, et a philosophis probatum est,
et in fide suppositum est. Ergo scientia Dei non tantum est
necessariorum, sed etiam contingentium.

[2806] Super Sent., lib. 1 d. 38 q. 1 a. 5 co. Respondeo dicendum,
quod propter hujusmodi difficultates, quidam philosophi negaverunt,
Deum de particularibus contingentibus cognitionem habere, cogitantes
intellectum divinum ad modum intellectus nostri; et ideo
erraverunt. Quod autem intellectus divinus non impediatur a cognitione
particularium ratione particulationis quae est ex materia, sicut
intellectus noster impeditur, in 36 distin., quaest. 1, art. 1,
ostensum est; et ideo nunc restat inquirere utrum impediatur ratione
contingentiae: contingentia enim videtur duplici ratione effugere
divinam cognitionem. Primo propter ordinem causae ad causatum. Quia
causae necessariae et immutabilis videtur esse effectus necessarius;
unde cum scientia Dei sit causa rerum, et sit immutabilis, videtur
quod non possit esse contingentium. Secundo propter ordinem scientiae
ad scitum; quia cum scientia sit certa cognitio, ex ipsa ratione
certitudinis etiam exclusa causalitate, requirit certitudinem et
determinationem in scito, quam contingentia excludit, ut patet in
scientia nostra, quae non est causa rerum, et in scientia Dei respectu
malorum. Sed neutrum horum removet scientiam contingentium a Deo. Et
de primo quidem satis manifeste potest accipi. Quandoque enim sunt
causae multae ordinatae, effectus ultimus non sequitur causam primam
in necessitate et contingentia, sed causam proximam; quia virtus
causae primae recipitur in causa secunda secundum modum causae
secundae. Effectus enim ille non procedit a causa prima nisi secundum
quod virtus causae primae recipitur in secunda causa: ut patet in
floritione arboris cujus causa remota est motus solis, proxima autem
virtus generativa plantae. Floritio autem potest impediri per
impedimentum virtutis generativae, quamvis motus solis invariabilis
sit. Similiter etiam scientia Dei est invariabilis causa omnium; sed
effectus producuntur ab ipso per operationes secundarum causarum; et
ideo mediantibus causis secundis necessariis, producit effectus
necessarios, ut motum solis et hujusmodi; sed mediantibus causis
secundis contingentibus producit effectus contingentes. Sed adhuc
manet dubitatio major de secunda: quia causa prima necessaria potest
simul esse cum defectu causae secundae, sicut motus solis cum
sterilitate arboris; sed scientia Dei non potest simul stare cum
defectu causae secundae. Non enim potest esse quod Deus sciat simul
hunc cursurum, et iste deficiat a cursu; et hoc est propter
certitudinem scientiae et non propter causalitatem ejus. Oportet enim
invenire ad hoc quod sit certa scientia, aliquam certitudinem in
scito. Sciendum est igitur, quod antequam res sit non habet esse nisi
in causis suis. Sed causae quaedam sunt ex quibus necessario sequitur
effectus, quae impediri non possunt, et in istis causis habet causatum
esse certum et determinatum, adeo quod potest ibi demonstrative sciri,
sicut est ortus solis, et eclypsis, et hujusmodi. Quaedam autem sunt
causae ex quibus consequuntur effectus ut in majori parte, sed tamen
deficiunt in minori parte; unde in istis causis effectus futuri non
habent certitudinem absolutam, sed quamdam, inquantum sunt magis
determinatae causae ad unum quam ad aliud; et ideo per istas causas
potest accipi scientia conjecturalis de futuris, quae tanto magis erit
certa, quanto causae sunt magis determinatae ad unum; sicut est
cognitio medici de sanitate et morte futura, et judicium astrologi de
ventis et pluviis futuris. Sed quaedam causae sunt quae se habent ad
utrumque: et in istis causis effectus de futuro nullam habent
certitudinem vel determinationem; et ideo contingentia ad utrumlibet
in causis suis nullo modo cognosci possunt. Sed quando jam efficiuntur
in rerum natura, tunc habent in seipsis esse determinatum; et ideo
quando sunt in actu, certitudinaliter cognoscuntur, ut patet in eo qui
videt Socratem currere, quia Socratem currere dum currit, necessarium
est; et certam cognitionem habere potest. Dico igitur, quod
intellectus divinus intuetur ab aeterno unumquodque contingentium non
solum prout est in causis suis, sed prout est in esse suo
determinato. Cum enim re existente ipsam rem videat prout in esse suo
determinato est, aliter cognosceret rem postquam est quam antequam
fiat; et sic ex eventibus rerum aliquid ejus accresceret
cognitioni. Patet etiam quod Deus ab aeterno non solum vidit ordinem
sui ad rem, ex cujus potestate res erat futura, sed ipsum esse rei
intuebatur. Quod qualiter sit, evidenter docet Boetius in fine de
Consol. Omnis enim cognitio est secundum modum cognoscentis, ut dictum
est. Cum igitur Deus sit aeternus, oportet quod cognitio ejus modum
aeternitatis habeat, qui est esse totum simul sine successione. Unde
sicut quamvis tempus sit successivum, tamen aeternitas ejus est
praesens omnibus temporibus una et eadem et indivisibilis ut nunc
stans; ita et cognitio sua intuetur omnia temporalia, quamvis sibi
succedentia, ut praesentia sibi, nec aliquid eorum est futurum
respectu ipsius, sed unum respectu alterius. Unde secundum Boetium
melius dicitur providentia quam praevidentia: quia non quasi futurum,
sed omnia ut praesentia uno intuitu procul videt, quasi ab
aeternitatis specula. Sed tamen potest dici praescientia, inquantum
cognoscit id quod futurum est nobis, non sibi. Quod ut melius pateat,
exemplis ostendatur. Sint quinque homines qui successive in quinque
horis quinque contingentia facta videant. Possum ergo dicere, quod
isti quinque vident haec contingentia succedentia praesentialiter. Si
autem poneretur quod isti quinque actus cognoscentium essent actus
unus, posset dici quod una cognitio esset praesentialiter de omnibus
illis cognitis successivis. Cum ergo Deus uno aeterno intuitu, non
successivo, omnia tempora videat, omnia contingentia in temporibus
diversis ab aeterno praesentialiter videt non tantum ut habentia esse
in cognitione sua. Non enim Deus ab aeterno cognovit in rebus tantum
se cognoscere ea, quod est esse in cognitione sua; sed etiam ab
aeterno vidit uno intuitu et videbit singula tempora, et rem talem
esse in hoc tempore, et in hoc deficere. Nec tantum videt hanc rem
respectu praecedentis temporis esse futuram, et respectu futuri
praeteritam: sed videt istud tempus in quo est praesens, et rem esse
praesentem in hoc tempore, quod tamen in intellectu nostro non potest
accidere, cujus actus est successivus secundum diversa tempora: et ita
patet quod nihil prohibet contingentium ad utrumlibet certam scientiam
Deum habere, cum intuitus ejus ad rem contingentem referatur secundum
hoc quod praesentialiter in actu est quando jam ejus esse determinatum
est, et certitudinaliter cognosci potest.

[2807] Super Sent., lib. 1 d. 38 q. 1 a. 5 ad 1 Ad primum ergo
dicendum, quod a causa prima non trahit effectus necessitatem, sed
solum a causa proxima, ut dictum est, in corp. art.; et ideo ratio non
procedit.

[2808] Super Sent., lib. 1 d. 38 q. 1 a. 5 ad 2 Ad secundum dicendum,
quod futurum contingens non est determinate verum antequam fiat, quia
non habet causam determinatam; et ideo ejus certa cognitio haberi non
potest ab intellectu nostro, cujus cognitio est in tempore determinato
et successive. Sed dum est in actu, determinate verum est; et ideo a
cognitione quae est praesens illi actui, potest certitudinaliter
cognosci; sicut patet etiam de visu corporali: et quia cognitio divina
aeternitate mensuratur, quae eadem manens omni tempori praesens est,
unumquodque contingentium videt prout est in suo actu.

[2809] Super Sent., lib. 1 d. 38 q. 1 a. 5 ad 3 Ad tertium dicendum,
quod actus divinae cognitionis transit supra contingens, etiam si
futurum sit nunc, sicut transit visus noster supra ipsum dum est; et
quia esse quod est, quando est, necesse est; quod tamen absolute non
est necessarium; ideo dicitur, quod in se consideratum est contingens,
sed relatum ad Dei cognitionem est necessarium; quia ad ipsam non
refertur nisi secundum quod est in esse actuali; et ideo simile est
sicut si ego videam Socratem praesentialiter currere; quod quidem in
se est contingens, sed relatum ad visum meum, est necessarium. Unde
bona est distinctio, quod est necessarium necessitate consequentiae et
non consequentis, vel necessitate conditionata, non absoluta.

[2810] Super Sent., lib. 1 d. 38 q. 1 a. 5 ad 4 Ad quartum dicendum
quod ad hoc argumentum multipliciter respondetur. Quidam enim dicunt,
quod hoc antecedens, scilicet hoc esse praescitum a Deo, non est
necessarium. Et si objiciatur, quod est dictum de praeterito, ergo est
necessarium; respondent, quod hoc habet instantiam in praeteritis quae
dicunt respectum ad futurum; unde cum dicitur hoc fuisse futurum,
quamvis sit dictum de praeterito; tamen quia dependet a futuro, non
est necessarium; quia quod fuit quandoque futurum, potest non esse
futurum; quia futurus quis incedere non incedet, ut dicitur in 2 de
Generat. Sed ista instantia nulla est; quia quamvis quod fuit futurum,
possit non esse futurum impeditis causis quae erant determinatae ad
effectum ut in majori parte, non tamen potest non fuisse futurum;
semper enim verum erit dicere: hoc quandoque fuit futurum. Similiter
non est ad propositum; quia cum dicitur praescitum, non importatur
tantum ordo ad futurum, sed etiam actus quidam, qui significatur ut
praeteritus. Et ideo alii dicunt, sicut videtur Magister dicere in
littera, quod hoc antecedens non est necessarium; quia praescitum,
quamvis secundum vocem consignificet tempus praeteritum, tamen
significat actum divinum, cui non accidit praeteritum: et ideo sicut
Deus potest non praescire, ita potest non praescisse. Sed istud etiam
non solvit; quia quamvis actus divinus non habeat necessitatem
coactionis, habet tamen necessitatem immobilitatis, loquendo de
actibus intrinsecis, ut velle, intelligere, et hujusmodi; unde non est
contingens non esse, si ponatur esse. Et quia consequens non potest
poni non esse, quin etiam antecedens ponatur non esse, consequens non
poterit poni non esse. Sed hoc certum est quod antecedens potest poni
esse; verum enim est determinate Deum aliquid futurum nunc scire; et
ita sequitur quod consequens non possit poni non esse, etiam absolute
sumptum; et multo minus quod possit contingere non esse. Et ideo alii
dicunt, quod istud antecedens est contingens, quia designantur ibi
duo, scilicet actus divinus qui immutabilis est, et ordo ad futurum,
qui mutabilis est mutabilitate rei; et ideo totum judicandum est
contingens propter alterum tantum. Istum enim esse hominem album,
contingens est, quamvis esse hominem sit necessarium. Sed istud etiam
non videtur dubitationem solvere. Cum enim dicitur Deus praescivisse
aliquid, ordo ille ad futurum designatur ibi ut objectum super quod
transit actus. Est enim sensus: praescivit, idest scivit hoc esse
futurum. Quando autem aliquod dictum ponitur ut materia alicujus
actus, ut dictum, oportet quod materialiter sumatur, et non secundum
quod ad significationem rei refertur; ut cum dicitur: scio istum
currere: ea autem quae sic sumuntur, nullam differentiam contingentiae
vel necessitatis in propositione faciunt; tum quia veritas et
necessitas propositionis ex principali verbo pendet, in quo
intelligitur compositio: tum etiam quia dictum hoc modo positum non
sumitur ut verum et falsum, vel ut necessarium vel contingens, sed ut
dictum quoddam tantum. Unde aequalis necessitas vel contingentia est
harum duarum propositionum: dico Socratem currere, et dico solem
moveri; etiam posito quod ipsum dicere sit necessarium; et ita posito
etiam quod ordo ille importatus ad futurum sit mutabilis, nihil
impeditur de necessitate antecedentis. Unde alii dicunt, hoc
antecedens esse necessarium; nec tamen consequens est necessarium;
quia illa maxima intelligitur tantum in illis conditionibus in quibus
antecedens est causa proxima consequentis. Sed hoc etiam non plene
solvit, quia regula illa non probatur a philosopho, ratione
causalitatis, sed ratione consequentiae, secundum quam ex necessario
sequitur necessarium, sive sit causa, sive sit effectus. Et ideo
aliter dicendum est, quod antecedens est necessarium absolute, tum ex
immobilitate actus tum etiam ex ordine ad scitum; quia ista res non
ponitur subjacere scientiae divinae nisi dum est in actu, secundum
quod determinationem et certitudinem habet. Ipsum enim necesse est
esse dum est; et ideo similis necessitas est inserenda in consequente,
ut scilicet accipiatur ipsum quod est Socratem currere, secundum quod
est in actu; et sic terminationem et necessitatem habet. Unde patet
quod si sumatur Socratem currere secundum hoc quod ex antecedente
sequitur, necessitatem habet: non enim sequitur ex antecedente nisi
secundum quod substat divinae scientiae, cui subjicitur prout
consideratur praesentialiter in suo esse actuali; unde etiam sic
sumendum est consequens, quomodo patet quod consequens necessarium
est: necesse enim est Socratem currere dum currit.

[2811] Super Sent., lib. 1 d. 38 q. 1 a. 5 ad 5 Ad quintum dicendum,
quod ista: omne scitum a Deo necesse est esse, est duplex, eo quod
potest esse de dicto, vel de re: et si sit de dicto vera est, et si
sit de re, falsa est: et similiter conclusio duplex est. Et hujus
distinctionis ratio est, quia potest istud sumi secundum conditionem
qua subjacet divinae scientiae; et hoc est secundum quod habet esse
determinatum in actu, et sic necessitatem habet, vel potest ista res
sumi sine aliqua conditione; et sic non est necessaria: quia potest
sic considerari ut est in causis suis antequam sit in actu, et ibi non
habet necessitatem, nec ibi est scita a Deo futura esse; non enim scit
Deus effectum contingentem esse determinatum in causa sua: quia esset
falsa scientia, cum in causa sua determinatum non sit.

[2812] Super Sent., lib. 1 d. 38 q. 1 a. 5 ad 6 Ad sextum dicendum,
quod quamvis iste respectus ad rem sit inseparabilis secundum quod
attingit eam; non tamen attingit eam nisi prout est in esse actuali
praesentialiter considerata; et ideo potest fieri distinctio secundum
quod res illa consideratur ut cadens sub respectu illo vel ut non
cadens. Verbi gratia, cursus Socratis subjacet certitudini divinae
scientiae, prout est in actu; et hoc non habuit semper, quia quandoque
erat in potentia tantum, et secundum quod sic tantum erat, non erat
subjicibilis certitudini divinae scientiae: si enim Deus vidisset
ipsam causam, ut Socratem, et non vidisset immediate effectum in esse
suo sicut nos futura cognoscimus, nunquam potuisset istud scire; et
ideo patet quod distinctio illa, scilicet quod possit esse de re, vel
de dicto, bona est.

[2813] Super Sent., lib. 1 d. 38 q. 1 a. 5 ad 7 Ad septimum dicendum,
quod Deus non tantum cognoscit ea quae sunt nobis praesentia, sed quae
sunt nobis praeterita et futura, supra quae tamen omnia intuitus
divinus cadit, secundum quod suis temporibus praesentia sunt. Unde non
sequitur quod aliquam rem Deus quandoque sciat quam aliquando
nescivit.




a.1, arg 3への脚注
\footnote{神の中に普遍も個別も
ないということについては、すでにI, q.13, a.9, ad 2で言われていた。こ
れについての詳しい説明は、I {\itshape Sent.}, d.19, q.4, a.2,
c. (Utrum in divinis sit totum universale)にある。(In {\itshape de
Trin}.~のDeckerの注による。cf.~p.120.)
\\
``Respondeo dicendum, quod in divinis non potest esse universale et
particulare. Et hujus ratio potest quadruplex assignari: primo, quia,
secundum Avicennam, ubicumque est genus et species, oportet esse
quidditatem differentem a suo esse, ut prius, dist.8, quaest.1,
art.1, dictum est; et hoc in divinis non competit; secundo, quia
essentia universalis non est eadem numero in suis inferioribus, sed
secundum rationem tantum; essentia autem divina est eadem numero in
pluribus personis; tertio, quia universale exigit pluralitatem in his
quae sub ipso continentur vel in actu vel in potentia: in actu sicut
est in genere, quod semper habet plures species; in potentia sicut in
aliquibus speciebus, quarum forma, quantum est de se, possibilis est
inveniri in multis, cum omnis forma sit de se communicabilis; sed quod
inveniatur tantum in uno, est ex parte materiae debitae illi speciei,
quae tota adunatur in uno individuo, ut patet in sole, qui constat ex
tota sua materia; et ista pluralitas est secundum numerum, qui numerus
simpliciter est fundatus in substantiali distinctione: tres autem
personae non numerantur tali numero, ut dictum est, art. antec., et
ideo essentia non habet rationem universalis; quarto, quia particulare
semper se habet ex additione ad universale. In divinis autem, propter
summam simplicitatem, non est possibilis additio, et ideo nec
universale nec particulare.''
 
\\

 普遍とは、類や種のようなもので、(1)存在と異なる何性を持つ、(2)普遍的
なものの本質は、下位のものどもにおいて、概念的にのみ同一であり数におい
て同一でない、(3)普遍的なものは、少なくとも可能的に、下位のものにおい
て複数化されうる、という三つの必要条件があるが、神はいずれも満たさない。
また、(4)個別は、普遍への何らかの付加によって得られるが、あらゆる点で
 単純である神にはこれも適合しない。
 
 \\
 
また、I {\itshape Sent.}, d.35, q.1, a.5, c.も参照。``Respondeo
dicendum, quod nihil dictorum divinae scientiae convenit, nisi hoc
solum quod est semper in actu esse: cujus ratio est quia conditiones
scientiae praecipue attenduntur secundum rationem medii, et similiter
cujuslibet cognitionis. Id autem quo Deus cognoscit quasi medio, est
essentia sua, quae non potest dici universale, quia omne universale
additionem recipit alicujus per quod determinatur; et ita est in
potentia, et imperfectum in esse; similiter non potest dici
particularis, quia particularis principium materia est, vel aliquid
loco materiae se habens, quod Deo non convenit. Similiter etiam ab
essentia ipsius omnis potentia passiva vel materialis remota est, cum
sit actus purus; unde nec etiam ratio habitus sibi competit, quia
habitus non est ultima perfectio, sed magis operatio quae perficit
habitum. Et ideo scientia sua neque universalis neque particularis
neque in potentia neque in habitu dici potest, sed tantum in actu.''
}