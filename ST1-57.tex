\documentclass[10pt]{jsarticle} % use larger type; default would be 10pt
%\usepackage[utf8]{inputenc} % set input encoding (not needed with XeLaTeX)
%\usepackage[round,comma,authoryear]{natbib}
%\usepackage{nruby}
\usepackage{okumacro}
\usepackage{longtable}
%\usepqckage{tablefootnote}
\usepackage[polutonikogreek,english,japanese]{babel}
%\usepackage{amsmath}
\usepackage{latexsym}
\usepackage{color}

%----- header -------
\usepackage{fancyhdr}
\lhead{{\it Summa Theologiae} I, q.~57}
%--------------------

\bibliographystyle{jplain}

\title{{\bf PRIMA PARS}\\{\HUGE Summae Theologiae}\\Sancti Thomae
Aquinatis\\{\sffamily QUEAESTIO QUINQUAGESIMASEPTIMA}\\DE ANGELORUM COGNITIONE
 RESPECTU RERUM MATERIALIUM}
\author{Japanese translation\\by Yoshinori {\sc Ueeda}}
\date{Last modified \today}


%%%% コピペ用
%\rhead{a.~}
%\begin{center}
% {\Large {\bf }}\\
% {\large }\\
% {\footnotesize }\\
% {\Large \\}
%\end{center}
%
%\begin{longtable}{p{21em}p{21em}}
%
%&
%
%
%\\
%\end{longtable}
%\newpage



\begin{document}
\maketitle
\pagestyle{fancy}

\begin{center}
{\Large 第五十七問\\天使の認識について:質料的事物にかんして}
\end{center}

\begin{longtable}{p{21em}p{21em}}

Deinde quaeritur de his materialibus quae ab Angelis cognoscuntur. Et
 circa hoc quaeruntur quinque.

\begin{enumerate}
 \item utrum Angeli cognoscant naturas rerum materialium.
 \item utrum cognoscant singularia.
 \item utrum cognoscant futura.
 \item utrum cognoscant cogitationes cordium.
 \item utrum cognoscant omnia mysteria gratiae.
\end{enumerate}


 &

 次に、天使たちによって認識される質料的事物について問われる。これについ
 て、五つのことが問われる。
 \begin{enumerate}
  \item 天使たちは、質料的事物の本性を認識するか。
  \item 個物を認識するか。
  \item 未来の事柄を認識するか。
  \item 心の思いを認識するか。
  \item 恩恵のすべて秘密を認識するか。
 \end{enumerate}

\end{longtable}
\newpage

\rhead{a.~1}
\begin{center}
 {\Large {\bf ARTICULUS PRIMUS}}\\
 {\large UTRUM ANGELI COGNOSCAT RES MATERIALES}\\
 {\footnotesize II {\itshape SCG.}, capt.~99; {\itshape De Verit.},
 q.~8, a.~8; q.~10, a.~4.}\\
 {\Large 第一項\\天使たちは質料的事物を認識するか}
\end{center}

\begin{longtable}{p{21em}p{21em}}


{\huge A}{\scshape d primum sic proceditur}. Videtur quod Angeli non
cognoscant res materiales. Intellectum enim est perfectio
intelligentis. Res autem materiales non possunt esse perfectiones
Angelorum, cum sint infra ipsos. Ergo Angeli non cognoscunt res
materiales.

&

第一項の問題へ、議論は以下のように進められる。天使たちは質料的事物を認
識しないと思われる。理由は以下の通り。知性認識されたものは、知性認識す
るものの完全性である。ところで、質料的事物が天使たちの完全性ではありえ
ない。なぜなら、天使たちの下位にあるからである。ゆえに、天使たちは質料
的事物を認識しない。

\\


{\scshape 2 Praeterea}, visio intellectualis est eorum quae sunt in
anima per sui essentiam, ut dicitur in Glossa, II {\itshape ad
Cor}.~{\scshape xii}. Sed res materiales non possunt esse in anima
hominis, vel in mente Angeli, per suas essentias. Ergo non possunt
intellectuali visione cognosci, sed solum imaginaria, qua
apprehenduntur similitudines corporum; et sensibili, quae est de ipsis
corporibus. In Angelis autem non est visio imaginaria et sensibilis,
sed solum intellectualis. Ergo Angeli materialia cognoscere non
possunt.


&


さらに、『コリントの信徒への手紙2』の『注解』で述べられているとおり、
知性的直視は、自らの本質によって魂の中に在るものどもについてある。とこ
ろが、質料的事物は、人間の魂の中にも、天使の精神の中にも、その本質によっ
て存在しえない。ゆえに、それらは知性的な直視によって認識されえず、ただ、
それによって物体の類似が捉えられるところの、表象的直視によって、そして、
物体そのものについてであるところの、感覚的直視によってのみ認識されうる。
ところが、天使の中には、表象的直視や感覚的直視はなく、ただ知性的直視だ
けがある。ゆえに、天使たちは質料的なものを認識できない。

\\


{\scshape 3 Praeterea}, res materiales non sunt intelligibiles in
actu, sed sunt cognoscibiles apprehensione sensus et imaginationis;
quae non est in Angelis. Ergo Angeli materialia non cognoscunt.


&

さらに、質料的事物は、現実態において可知的なものではなく、感覚と表象力
の把握によって認識されうるものであるが、これらは天使の中にはない。ゆえ
に、天使たちは質料的なものを認識しない。

\\


{\scshape Sed contra}, quidquid potest inferior virtus, potest virtus
superior. Sed intellectus hominis, qui est ordine naturae infra
intellectum Angeli, potest cognoscere res materiales. Ergo multo
fortius intellectus Angeli.


&

しかし反対に、下位の力がなしうる事柄は、何であれ、上位の力がなしうる。
ところで、人間の知性は、本性の秩序において天使の知性の下位であるのに、
質料的事物を認識しうる。ゆえに、天使の知性は、より強力に、それを認識し
うる。


\\


{\scshape Respondeo dicendum} quod talis est ordo in rebus, quod
superiora in entibus sunt perfectiora inferioribus, et quod in
inferioribus continetur deficienter et partialiter et multipliciter,
in superioribus continetur eminenter et per quandam totalitatem et
simplicitatem. Et ideo in Deo, sicut in summo rerum vertice, omnia
supersubstantialiter praeexistunt secundum ipsum suum simplex esse, ut
Dionysius dicit, in libro {\itshape de Div.~Nom}.


&

解答する。以下のように言われるべきである。諸事物における秩序は以下のよ
うになっている。すなわち、存在するものどもの中で上位のものは、下位のも
のよりも完全であり、下位のものの中に、欠陥を伴って、部分的に、多数化さ
れて含まれるものは、上位のものの中に、より優れて、ある種の全体性と単純
性によって存在する。ゆえに、ディオニュシウスが『神名論』で言うように、
諸事物の中で最高の頂上である神の中には、万物が、神自らの単純な存在自体
において、超実体的に先在する。

\\


Angeli autem inter ceteras creaturas sunt Deo propinquiores et
similiores, unde et plura participant ex bonitate divina, et
perfectius, ut Dionysius dicit, {\scshape iv} cap.~{\itshape
Cael.~Hier}. Sic igitur omnia materialia in ipsis Angelis
praeexistunt, simplicius quidem et immaterialius quam in ipsis rebus;
multiplicius autem et imperfectius quam in Deo.

&

他方、天使たちは、他の被造物の中では、より神に近く、神に似ている。した
がって、ディオニュシウスが『天上階級論』4章で述べるように、より多く、
より完全に、神の善性を分有している。それゆえ、質料的なものどもは、天使
たち自身の中に、それらの事物そのものにおいてよりも単純で非質料的なかた
ちで先在するが、神において存在するのに比べると、多数化されて、不完全な
かたちである。


\\


Omne autem quod est in aliquo; est in eo per modum eius in quo
est. Angeli autem secundum suam naturam sunt intellectuales. Et ideo,
sicut Deus per suam essentiam materialia cognoscit, ita Angeli ea
cognoscunt per hoc quod sunt in eis per suas intelligibiles species.



&

ところで、すべて何かの中に存在するものは、その何かのあり方によって存在
する。ところが、天使は、自らの本性において、知性的である。ゆえに、ちょ
うど神が、自分の本質を通して質料的なものを認識するように、天使たちは、
それらが自らの可知的な形象によって天使たちの中に存在することを通して、
それらを認識する。


\\

{\scshape Ad primum ergo dicendum} quod intellectum est perfectio
intelligentis secundum speciem intelligibilem quam habet in
intellectu. Et sic species intelligibiles quae sunt in intellectu
Angeli, sunt perfectiones et actus intellectus angelici.


&

第一異論に対しては、それゆえ、以下のように言われるべきである。知性認識
されたものが知性認識するものの完全性であるのは、知性の中に有する可知的
形象にしたがってである。この意味では、天使たちの知性の中にある可知的形
象は、天使の知性の完全性であり、現実態である。


\\


{\scshape Ad secundum dicendum} quod sensus non apprehendit essentias
rerum, sed exteriora accidentia tantum. Similiter neque imaginatio,
sed apprehendit solas similitudines corporum. Intellectus autem solus
apprehendit essentias rerum. Unde in III {\itshape de Anima} dicitur
quod obiectum intellectus est {\itshape quod quid est}, circa quod non
errat, sicut neque sensus circa proprium sensibile. Sic ergo essentiae
rerum materialium sunt in intellectu hominis vel Angeli, ut
intellectum est in intelligente, et non secundum esse suum
reale. Quaedam vero sunt quae sunt in intellectu vel in anima secundum
utrumque esse. Et utrorumque est visio intellectualis.


&

第二異論に対しては、以下のように言われるべきである。感覚は諸事物の本質
を捉えず、ただ外的な附帯性だけを捉える。表象力も同様であり、ただ物体の
類似だけを捉える。他方、知性だけは、諸事物の本質を捉える。このことから、
『デ・アニマ』第三巻で、知性の対象は「何であるか」であると言われるので
あり、ちょうど感覚が固有に可感的なものについて誤らないように、知性は
「何であるか」について誤らない。ゆえに、このようにして、質料的事物の本
質は、人間または天使の知性の中に、知性認識されたものが知性認識するもの
の中にあるかたちで存在するのであり、自らの実在的な存在に即して存在する
のではない。他方、知性や魂の中に存在する、あるものどもは、その両方の存
在\footnote{一つは、esse reale。もう一つは、「知性認識されたものが知性
認識するものの中にある」存在。ST I, 56, 2, c.ではesse naturaleとesse
intelligibileの区別として言及されていた。}に即して存在する
\footnote{quae節はQuaedamにかかり、animaまで。「知性の中に存在する
quaedamは、その両方の存在において在る」。非質料的事物は、それが現実世
界に存在するときと、認識されたときとでは、ただesseにおいてのみ異なる。
他方、質料的事物の本質は、知性の中に存在するときには個的質料が捨象され
るので、同一のessentiaが二つのesseにおいて存在するとは言えない。}。そ
して、その両者について、知性的な直視がある。


\\


{\scshape Ad tertium dicendum} quod, si Angelus acciperet cognitionem
rerum materialium ab ipsis rebus materialibus, oporteret quod faceret
eas intelligibiles actu, abstrahendo eas. Non autem accipit
cognitionem earum a rebus materialibus, sed per species actu
intelligibiles rerum sibi connaturales, rerum materialium notitiam
habet; sicut intellectus noster secundum species quas intelligibiles
facit abstrahendo.


&

第三異論に対しては、以下のように言われるべきである。もし天使が質料的事
物の認識を質料的事物自体から受け取るのであったならば、その認識を事物か
ら抽象することによって、それらを現実に可知的なものにすることが必要であっ
ただろう。しかし、天使はそれらの認識を質料的事物から受け取らず、自らに
生得的な、諸事物の現実的な可知的形象を通して、質料的諸事物についての知
をもつ。ちょうど、私たちの知性が、抽象することによって可知的なものにす
る形象にしたがってそうするように。




\end{longtable}
\newpage





\rhead{a.~2}
\begin{center}
{\Large {\bf ARTICULUS SECUNDUS}}\\ {\large UTRUM ANGELUS COGNOSCAT
SINGULARIA}\\ {\footnotesize Infra, q.~84, a.~4; II {\itshape Sent.},
d.~3, part.~2, q.~2, a.~3; II {\itshape SCG} cap.~100; {\itshape De
Verit.}, q.~8, a.~2; q.~10, a.~3; Qu.~{\itshape de Anima}, a.~20;
{\itshape Quodlib.}, VII, q.~1, a.~3; Opusc.~XV, {\itshape de
Angelis}, cap.~13, 15.}\\ {\Large 第二項\\天使は個物を認識するか}
\end{center}

\begin{longtable}{p{21em}p{21em}}


{\huge A}{\scshape d secundum sic proceditur}. Videtur quod Angelus
singularia non cognoscat. Dicit enim philosophus, in I {\itshape
Physic}., quod {\itshape sensus est singularium, ratio vero} (vel
intellectus) {\itshape universalium}. In Angelis autem non est vis
cognoscitiva nisi intellectiva, ut ex superioribus patet. Ergo
singularia non cognoscunt.


&

第二項の問題へ、議論は以下のように進められる。天使は個物を認識しないと
思われる。理由は以下の通り。哲学者は『自然学』第1巻で「感覚は個物にか
かわり、理性(あるいは知性)は普遍にかかわる」と述べている。ところで、
前に述べられたことから明らかなとおり、天使の中には知性的能力以外の認識
能力はない。ゆえに、天使は個物を認識しない。


\\


{\scshape 2 Praeterea}, omnis cognitio est per assimilationem aliquam
cognoscentis ad cognitum. Sed non videtur quod possit esse aliqua
assimilatio Angeli ad singulare inquantum est singulare, cum Angelus
sit immaterialis, ut supra dictum est, singularitatis vero principium
sit materia. Ergo Angelus non potest cognoscere singularia.


&

さらに、認識はすべて、認識するものの、認識されるものへのなんらかの類似
化によってある。ところで、前に述べられたことから明らかなとおり、天使が
非質料的であるのに対して、個物性の根源は質料であるから、個物であるかぎ
りにおける個物への、天使の類似化は決してありえない。ゆえに、天使は個物
を認識できない。

\\


{\scshape 3 Praeterea}, si Angelus scit singularia, aut per species
singulares, aut per species universales. Non per singulares, quia sic
oporteret quod haberet species infinitas. Neque per universales, quia
universale non est sufficiens principium cognoscendi singulare
inquantum est singulare, cum in universali singularia non cognoscantur
nisi in potentia. Ergo Angelus non cognoscit singularia.


&

さらに、もし天使が個物を知るならば、個的な形象によるか、あるいは、普遍
的な形象によるかのどちらかである。しかし、個的な形象によるのではない。
なぜなら、もしそうだとすると、無数の形象を持つ必要があっただろうから。
また、普遍的な形象によるのでもない。なぜなら、個物は、可能態においてで
なければ、普遍の中に認識されないから、普遍は、個物であるかぎりにおける
個物を認識する十分な根源でないからである。ゆえに、天使は個物を認識しな
い。

\\


{\scshape Sed contra}, nullus potest custodire quod non cognoscit. Sed
Angeli custodiunt homines singulares, secundum illud Psalmi {\scshape
xc}, {\itshape Angelis suis mandavit de te}, et cetera. Ergo Angeli
cognoscunt singularia.


&


しかし反対に、だれも、認識しないものを守ることはできない。ところが、
『詩編』90「自分の天使たちに、あなたについて命じた云々」\footnote{「主
はあなたのために、御使いに命じて、あなたの道のどこにおいても守らせてく
ださる。」(91, 11)}によれば、天使たちは、個々の人間たちを守る。ゆえに、
天使は個物を認識する。

\\


{\scshape Respondeo dicendum} quod quidam totaliter subtraxerunt
Angelis singularium cognitionem. Sed hoc primo quidem derogat
Catholicae fidei, quae ponit haec inferiora administrari per Angelos,
secundum illud {\itshape Heb.}~{\scshape i}, {\itshape Omnes sunt
administratorii spiritus}. Si autem singularium notitiam non haberent,
nullam providentiam habere possent de his quae in hoc mundo aguntur;
cum actus singularium sint. Et hoc est contra illud quod dicitur
{\itshape Eccle}.~{\scshape v}, {\itshape ne dicas coram Angelo, non
est providentia}. Secundo, etiam derogat philosophiae documentis,
secundum quae ponuntur Angeli motores caelestium orbium, et quod eos
moveant secundum intellectum et voluntatem.


&

解答する。以下のように言われるべきである。ある人々は、天使たちから個物
の認識を全面的に取り除いた。しかし、これは第一に、かの『ヘブライ人への
手紙』1「すべては奉仕する霊である」\footnote{「天使たちは皆、奉仕する
霊であって、救いを受け継ぐことになっている人々に仕えるために、遣わされ
たのではなかったですか。」(1:14)}によれば、これら下位のものどもが天使
たちによって奉仕されると主張するカトリック信仰を損なう。ところで、もし
個物にかんする知をもたなかったならば、働きは個物にかかわるから、この世
でなされる事柄について、どんな摂理も持ち得なかったであろう。そしてこれ
は、『コヘレトの言葉』5「摂理はない、と天使の前で言うな」
\footnote{「使者に「あれは間違いでした」などと言うな。」(5:5)}と言われ
ていることに反する。第二に、それは哲学の教えをも損なう。すなわちそれに
よれば、天使たちは天の軌道を動かす者であり、知性と意志とによって、それ
らを動かすとされている。

\\

Et ideo alii dixerunt quod Angelus habet quidem cognitionem
singularium, sed in causis universalibus, ad quas reducuntur
particulares omnes effectus, sicut si astrologus iudicet de aliqua
eclipsi futura, per dispositiones caelestium motuum. Sed haec positio
praedicta inconvenientia non evadit, quia sic cognoscere singulare in
causis universalibus, non est cognoscere ipsum ut est singulare, hoc
est ut est hic et nunc. Astrologus enim cognoscens eclipsim futuram
per computationem caelestium motuum, scit eam in universali; et non
prout est hic et nunc, nisi per sensum accipiat. Administratio autem
et providentia et motus sunt singularium, prout sunt hic et nunc.


&

それゆえ、他の人々は以下のように述べた。天使たちは、たしかに個物の認識
をもっているが、しかし、普遍的な原因においてそれをもっている。ちょうど、
もし、天文学者が天の諸運動の配置を通して、将来のなんらかの食について判
断するならばそうであろうように、すべての個別的な結果は、その普遍的な諸
原因に還元される、と。しかし、この主張は、前述の不都合を回避しない。と
いうのも、このように個物を普遍的な原因において認識することは、個物を個
物として認識すること、つまり、ここと今にあるものとして認識することでは
ないからである。じっさい、天の諸運動の計算を通して、将来の食を認識する
天文学者は、それを普遍において知っているのであり、感覚をとおして捉える
のでない限り、ここと今にあるものとして知っているのではない。ところが、
助けや予知や運動は、ここと今にあるものとしての、個物にかかわる。

\\

Et ideo aliter dicendum est quod, sicut homo cognoscit diversis
viribus cognitivis omnia rerum genera, intellectu quidem universalia
et immaterialia, sensu autem singularia et corporalia; ita Angelus per
unam intellectivam virtutem utraque cognoscit. Hoc enim rerum ordo
habet, quod quanto aliquid est superius, tanto habeat virtutem magis
unitam et ad plura se extendentem, sicut in ipso homine patet quod
sensus communis, qui est superior quam sensus proprius, licet sit
unica potentia, omnia cognoscit quae quinque sensibus exterioribus
cognoscuntur, et quaedam alia quae nullus sensus exterior cognoscit,
scilicet differentiam albi et dulcis. Et simile etiam est in aliis
considerare. Unde cum Angelus naturae ordine sit supra hominem,
inconveniens est dicere quod homo quacumque sua potentia cognoscat
aliquid, quod Angelus per unam vim suam cognoscitivam, scilicet
intellectum, non cognoscat. Unde Aristoteles pro inconvenienti habet
ut litem\footnote{l\={\i}s, l\={\i}tis, f. I.~a strife, despute,
quarrel.}, quam nos scimus, Deus ignoret; ut patet in I {\itshape de
Anima}, et in III {\itshape Metaphys}.


&

それゆえ、別のかたちで以下のように言われるべきである。ちょうど人間が、
知性によって普遍的なものと非質料的なものを、感覚によって個的なものと物
体的なものを、というように、さまざまな認識の力によって、諸事物のすべて
の類を認識するように、天使は、一つの知性的な力によって、この両方を認識
する。理由は以下の通り。諸事物の秩序は、あるものが上位であればあるほど、
それだけいっそう、より一つであり多くのものへと及ぶ力をもつ、という具合
になっている。ちょうど、人間自身の中でも、固有感覚よりも上位である共通
感覚が、一つの能力でありながら、五つの外部感覚によって認識されるすべて
を認識し、白と甘さの違いなど、どの感覚も認識しないその他のこともまた認
識するように。そして、これに類したことが、他のものにおいても考えられる。
したがって、天使は本性の秩序において人間より上位だから、人間が、自分の
どの能力によっても認識することを、天使が、自分の一つの認識力、つまり、
知性によって認識しないと語るのは不都合である。このことから、アリストテ
レスが、『デ・アニマ』第1巻と『形而上学』第3巻で明らかなとおり、私たち
が知っている「争い」を神が知らないのは不都合だと考えるのもこのためであ
る。


\\

Modus autem quo intellectus Angeli singularia cognoscit, ex hoc
considerari potest quod, sicut a Deo effluunt res ut subsistant in
propriis naturis, ita etiam ut sint in cognitione angelica. Manifestum
est autem quod a Deo effluit in rebus non solum illud quod ad naturam
universalem pertinet, sed etiam ea quae sunt individuationis
principia, est enim causa totius substantiae rei, et quantum ad
materiam et quantum ad formam. Et secundum quod causat, sic et
cognoscit, quia scientia eius est causa rei, ut supra ostensum
est. Sicut igitur Deus per essentiam suam, per quam omnia causat, est
similitudo omnium, et per eam omnia cognoscit non solum quantum ad
naturas universales, sed etiam quantum ad singularitatem; ita Angeli
per species a Deo inditas, res cognoscunt non solum quantum ad naturam
universalem, sed etiam secundum earum singularitatem, inquantum sunt
quaedam repraesentationes multiplicatae illius unicae et simplicis
essentiae.


&

さて、天使の知性が個物を認識する仕方については、ちょうど神から諸事物が
固有の本性において存立するものとして流れ出るように、天使の認識の中に存
在するようにも流れ出る、ということに基づいて考察されうる。神から諸事物
の中に流れ出たのは、普遍的な本性に属するものばかりではなく、個体化の根
源であるようなものもまた流れ出たことは明らかである。なぜならば、神は質
料にかんしても形相にかんしても、事物の実体全体の原因だからである。また、
前に示されたとおり、神の知は事物の原因だから、原因するのと同じように、
認識しもする。ゆえに、神が、それによって万物の原因であるところの自分の
本質を通して、万物の類似であり、そして、それを通して、普遍的な本性にか
んしてだけでなく、個物性にかんしても、万物を認識するように、天使たちは、
神から受け取った形象を通して、普遍的な本性にかんしてだけでなく、それら
の個物性に即しても、かの単一で単純な本性の、多数化されたある種の表現で
あるかぎりにおいて、諸事物を認識する。

\\


{\scshape Ad primum ergo dicendum} quod philosophus loquitur de
intellectu nostro, qui non intelligit res nisi abstrahendo; et per
ipsam abstractionem a materialibus conditionibus, id quod abstrahitur,
fit universale. Hic autem modus intelligendi non convenit Angelis, ut
supra dictum est, et ideo non est eadem ratio.


&

第一異論に対しては、それゆえ、以下のように言われるべきである。哲学者は、
諸事物を抽象することによってしか認識しない私たちの知性について述べてい
る。そして、この質料的諸条件からの抽象によって、抽象されたものが普遍に
なる。しかし、前に述べられたことから明らかなとおり、このような認識方法
は天使に適合しない。ゆえに、議論は同じでない。


\\


{\scshape Ad secundum dicendum} quod secundum suam naturam Angeli non
assimilantur rebus materialibus sicut assimilatur aliquid alicui
secundum convenientiam in genere vel in specie, aut in accidente; sed
sicut superius habet similitudinem cum inferiori, ut sol cum igne. Et
per hunc etiam modum in Deo est similitudo omnium, et quantum ad
formam et quantum ad materiam, inquantum in ipso praeexistit ut in
causa quidquid in rebus invenitur. Et eadem ratione species
intellectus Angeli, quae sunt quaedam derivatae similitudines a divina
essentia, sunt similitudines rerum non solum quantum ad formam, sed
etiam quantum ad materiam.


&

第二異論に対しては、以下のように言われるべきである。天使たちは、自分の
本性において、質料的事物に、あるものが何かに、類や種、あるいは附帯性に
おいて類似化するようなかたちで類似化するのではなく、むしろ、太陽が火の
類似をもつように、上位のものが下位のものの類似をもつようなかたちで類似
する。そして、このようなかたちでも、神の中に、形相にかんしても質料にか
んしても、諸事物の中に見出されるものは何であれ、原因としての神の中に先
在する限りにおいて、万物の類似が存在する。同じ理由で、天使の知性の形象
も、神の本質から派生したある種の類似であるかぎりで、たんに形相だけでな
く、質料にかんしてもまた、諸事物の類似である。


\\


{\scshape Ad tertium dicendum} quod Angeli cognoscunt singularia per
formas universales, quae tamen sunt similitudines rerum et quantum ad
principia universalia, et quantum ad individuationis
principia. Quomodo autem per eandem speciem possint multa cognosci iam
supra dictum est.


&

第三異論に対しては、以下のように言われるべきである。天使たちは普遍的な
形相を通して個物を認識するが、その形相は、普遍的な根源にかんしても個体
化の根源にかんしても、諸事物の類似である。多くのものがどのようにして同
一の形象によって認識されうるかは、前に述べられた。


\

\end{longtable}
\newpage


\rhead{a.~3}
\begin{center}
{\Large {\bf ARTICULUS TERTIUS}}\\ {\large UTRUM ANGELI COGNOSCANT
FUTURA}\\ {\Large 第三項\\天使は未来のことがらを認識するか}
\end{center}

\begin{longtable}{p{21em}p{21em}}

{\huge A}{\scshape d tertium sic proceditur}. Videtur quod Angeli
cognoscant futura. Angeli enim potentiores sunt in cognoscendo quam
homines. Sed homines aliqui cognoscunt multa futura. Ergo multo
fortius Angeli.


&

第三項の問題へ、議論は以下のように進められる。天使たちは未来のことがら
を認識すると思われる。理由は以下の通り。天使たちは、人間たちよりも、認
識することにおいて強い力を持つ。ところが、ある人間たちは、未来の多くの
ことがらを認識する。ゆえに、ましてや天使たちは、より力強く認識する。


\\


{\scshape 2 Praeterea}, praesens et futurum sunt differentiae
temporis. Sed intellectus Angeli est supra tempus, {\itshape
parificatur} enim {\itshape intelligentia aeternitati}, idest aevo, ut
dicitur in libro {\itshape de Causis}. Ergo quantum ad intellectum
Angeli, non differunt praeteritum et futurum; sed indifferenter
cognoscit utrumque.


&

さらに、現在と未来は、時間の差異である。ところで、『原因論』という書物
で言われるように、「知性体は永遠に」、すなわち永劫に「等しいものとされ
る」のだから、天使の知性は時間を越えている。ゆえに、天使の知性にかんす
る限り、過去と未来はことならず、両者を差別なく認識する。


\\


{\scshape 3 Praeterea}, Angelus non cognoscit per species acceptas a
rebus, sed per species innatas universales. Sed species universales
aequaliter se habent ad praesens, praeteritum et futurum. Ergo videtur
quod Angeli indifferenter cognoscant praeterita et praesentia et
futura.


&

さらに、天使は、諸事物から受け取った形象によって認識せず、生得的な普遍
的形象によって認識する。ところで、普遍的形象は、現在、過去、未来に等し
く関係する。ゆえに、天使たちは、現在、過去、未来の事柄を、差別なく認識
すると思われる。


\\


{\scshape 4 Praeterea}, sicut aliquid dicitur distans secundum tempus,
ita secundum locum. Sed Angeli cognoscunt distantia secundum
locum. Ergo etiam cognoscunt distantia secundum tempus futurum.


&

さらに、さらに、ちょうど、あるものが時間において隔たると言われるように、
場所においても隔たると言われる。ところで、天使たちは、場所において隔たっ
たことがらを認識する。ゆえに、未来の時間において隔たったことがらもまた
認識する。

\\


{\scshape Sed contra}, id quod est proprium signum divinitatis, non
convenit Angelis. Sed cognoscere futura est proprium signum
divinitatis; secundum illud Isaiae {\scshape xli}, {\itshape
annuntiate quae ventura sunt in futurum, et sciemus quod dii estis
vos}. Ergo Angeli non cognoscunt futura.


&

しかし反対に、神性に固有のしるしは、天使たちに属さない。ところが、『イ
ザヤ書』41「未来に来る事柄を知らせよ。そうすれば、私たちは、あなたたち
が神々だと知るだろう」\footnote{「将来にかかわるしるしは何か、告げてみ
よ。お前たちが神であることを悟るとしよう」(41:23)}によれば、未来を認識
することは、神性に固有なしるしである。ゆえに、天使たちは、未来のことが
らを認識しない。


\\


{\scshape Respondeo dicendum} quod futurum dupliciter potest
cognosci. Uno modo, in causa sua. Et sic futura quae ex necessitate ex
causis suis proveniunt, per certam scientiam cognoscuntur, ut solem
oriri cras. Quae vero ex suis causis proveniunt ut in pluribus,
cognoscuntur non per certitudinem, sed per coniecturam; sicut medicus
praecognoscit sanitatem infirmi. Et iste modus cognoscendi futura
adest Angelis; et tanto magis quam nobis, quanto magis rerum causas et
universalius et perfectius cognoscunt; sicut medici qui acutius vident
causas, melius de futuro statu aegritudinis prognosticantur. Quae vero
proveniunt ex causis suis ut in paucioribus, penitus sunt ignota,
sicut casualia et fortuita.


&

解答する。以下のように言われるべきである。未来は二通りに認識されうる。
一つにはその原因においてである。このしかたで、自らの諸原因から必然的に
到来する未来の事柄は、たしかな知によって認識される。たとえば、明日太陽
が昇ることが。他方、自らの諸原因から、多くのものにおいて、というかたち
で到来する事柄は、確実性によってではなく、推測によって認識される。たと
えば、医者が、病気の人が健康になることを、事前に知るように。そしてこの
しかたで未来を認識することが、天使のもとにある。そして、天使たちが、諸
事物の諸原因を、より普遍的に、より完全に認識するだけ、それだけいっそう、
私たちよりもよく認識する。ちょうど、諸原因をより鋭敏に見る医者が、病気
の未来の状態について、よりよく予後を診断するように。また他方、諸原因か
ら、より少ないものにおいて、というかたちで到来する事柄については、まっ
たく無知である。ちょうど、偶然的な事柄や運次第の事柄のように。

\\

Alio modo cognoscuntur futura in seipsis. Et sic solius Dei est futura
cognoscere, non solum quae ex necessitate proveniunt, vel ut in
pluribus, sed etiam casualia et fortuita, quia Deus videt omnia in sua
aeternitate, quae, cum sit simplex, toti tempori adest, et ipsum
concludit. Et ideo unus Dei intuitus fertur in omnia quae aguntur per
totum tempus sicut in praesentia, et videt omnia ut in seipsis sunt;
sicut supra dictum est cum de Dei scientia ageretur. Angelicus autem
intellectus, et quilibet intellectus creatus, deficit ab aeternitate
divina. Unde non potest ab aliquo intellectu creato cognosci futurum,
ut est in suo esse.


&

また別の仕方では、未来はそれ自体において認識される。この仕方で未来を認
識することは、ただ神だけに属する。この未来の事柄は、ただ必然的に到来す
るものだけでなく、多くのものにおいて、というかたちで到来するものや、偶
然や運任せの事柄もまたそうである。なぜなら、神は、すべてを自らの永遠に
おいて見るが、その永遠は、単純なので、全時間に臨在し、包含するからであ
る。ゆえに、神の一つの直観が、全時間を通して為されるすべての事柄へ、ちょ
うど現在の事柄へであるかのように運ばれ、万物をそれ自体においてあるもの
として見る。これは、前に、神の知について論じられたときに語られたとおり
である。しかし、天使の知性は、そして、何であれ被造の知性は、神の永遠性
を欠いている。したがって、未来の事柄が、それ自身の存在においてあるもの
として、なんらかの被造の知性によって認識されることは不可能である。


\\


{\scshape Ad primum} ergo dicendum quod homines non cognoscunt futura
nisi in causis suis, vel Deo revelante. Et sic Angeli multo subtilius
futura cognoscunt.

&

第一異論に対しては、それゆえ、以下のように言われるべきである。人間は、
その原因において、あるいは神が啓示することによってしか未来を認識しない。
そして、このしかたで、天使たちははるかに精妙に未来を認識する。

\\


{\scshape Ad secundum dicendum} quod, licet intellectus Angeli sit
supra tempus quo mensurantur corporales motus, est tamen in intellectu
Angeli tempus secundum successionem intelligibilium conceptionum;
secundum quod dicit Augustinus, VIII {\itshape super Gen.~ad Litt.},
quod {\itshape Deus movet spiritualem creaturam per tempus}. Et ita,
cum sit successio in intellectu Angeli, non omnia quae aguntur per
totum tempus, sunt ei praesentia.

&

第二異論に対しては、以下のように言われるべきである。たしかに天使は、そ
れによって物体的な運動が測られる時間を越えているが、しかし、『創世記逐
語注解』8巻のアウグスティヌスが、「神は霊的被造物を時間によって動かす」
と言うことによれば、天使の知性の中には、可知的な概念の継起における時間
が存在する。このように、天使の知性の中には継起があるので、全時間を通し
て為されるすべてのことが、天使にとって現在であるわけではない。


\\


{\scshape Ad tertium dicendum} quod, licet species quae sunt in
intellectu Angeli, quantum est de se, aequaliter se habeant ad
praesentia, praeterita et futura; tamen praesentia, praeterita et
futura non aequaliter se habent ad rationes. Quia ea quae praesentia
sunt, habent naturam per quam assimilantur speciebus quae sunt in
mente Angeli, et sic per eas cognosci possunt. Sed quae futura sunt,
nondum habent naturam per quam illis assimilentur, unde per eas
cognosci non possunt.

&

第三異論に対しては、以下のように言われるべきである。たしかに、天使の知
性の中にある形象は、それ自体にかんする限り、現在、過去、未来に等しく関
係する。しかし、現在、過去、未来の事柄は、これらの理念に等しく関係する
わけではない。なぜなら、現在の事柄は、天使の精神の中にある形象に、それ
を通して類似化されるところの本性をもっていて、そのために、それらの形象
によって認識されうる。しかし、未来の事柄は、それを通してそれら(=形象)
に類似化されるところの本性をまだもっていないので、それら(=形象)によっ
て認識されえない。\footnote{しかし、この論法だと、神も未来の事柄を認識
できないことにならないか。神から見て、未来の事柄は、神の永遠の現在のも
とにあるから、この論法は当てはまらないのか。}

\\


{\scshape Ad quartum dicendum} quod distantia secundum locum sunt iam
in rerum natura, et participant aliquam speciem, cuius similitudo est
in Angelo, quod non est verum de futuris, ut dictum est. Et ideo non
est simile.

&


第四異論に対しては、以下のように言われるべきである。場所における隔たり
は、すでに実在し、なんらかの形象を分有している。そしてその形象の類似が、
天使の中にある。しかし、すでに語られたとおり、このことは未来の事柄につ
いては真でない。ゆえに同様ではない。

\end{longtable}
\newpage


\rhead{a.~4}
\begin{center}
{\Large {\bf ARTICULUS QUARTUS}}\\ {\large UTRUM ANGELI COGNOSCANT
COGITATIONES CORDIUM}\\ {\footnotesize {\itshape De Verit.}, q.~8,
a.~13; {\itshape De Malo}, q.~16, a.~8; Opusc.~X, a.~38; Opusc.~XI,
a.~36; {\itshape I Cor.}, cap.~2, lect.~2.}\\ {\Large 第四項\\天使たち
は心の思いを認識するか}
\end{center}

\begin{longtable}{p{21em}p{21em}}

{\huge A}{\scshape d quartum sic proceditur}. Videtur quod Angeli
cognoscant cogitationes cordium. Dicit enim Gregorius, in {\itshape
Moralibus}, super illud {\scshape Iob} {\scshape xxviii}, {\itshape
Non aequabitur ei aurum vel vitrum}, quod {\itshape tunc}, scilicet in
beatitudine resurgentium, {\itshape unus erit perspicabilis alteri
sicut ipse sibi, et cum uniuscuiusque intellectus attenditur, simul
conscientia penetratur}. Sed resurgentes erunt similes Angelis, sicut
habetur Matth.~cap.~{\scshape xxii}. Ergo unus Angelus potest videre
id quod est in conscientia alterius.


&

第四の問題へ、議論は以下のように進められる。天使たちは心の思いを認識す
ると思われる。理由は以下の通り。グレゴリウスは、『道徳論』の中で、かの
「ヨブ記」28節の「金もガラスもそれにはかなわないだろう」\footnote{「金
も宝玉も知恵に比べられず」(28:17)}について、「そのとき、」すなわち復活
した人々の至福において、「人は他の人によって、自分自身と同じように見ら
れ、各々の人の知性が注意されるとき、同時に意識が貫かれる」と述べている。
ところが、『マタイによる福音書』22章で言われるように\footnote{「復活の
ときには、めとることも嫁ぐこともなく、天使のようになるのだ」(22:30)}、
復活した人々は、天使たちに似る。ゆえに、一人の天使は他の天使の意識の中
にあるものを見ることができる。


\\



{\scshape 2 Praeterea}, sicut se habent figurae ad corpora, ita se
habent species intelligibiles ad intellectum. Sed viso corpore,
videtur eius figura. Ergo visa substantia intellectuali, videtur
species intelligibilis quae est in ipsa. Ergo, cum Angelus videat
alium Angelum, et etiam animam, videtur quod possit videre
cogitationem utriusque.


&

さらに、形と物体の関係は、可知的形象と知性の関係に等しい。ところで、物
体が見られると、その形もまた見られる。ゆえに、知性的実体が見られると、
その中にある可知的形象も見られる。ゆえに、天使は他の天使を見、さらに魂
も見るのだから、どちらの思いも見ることができると思われる。


\\



{\scshape 3 Praeterea}, ea quae sunt in intellectu nostro, sunt
similiora Angelo quam ea quae sunt in phantasia, cum haec sint
intellecta in actu, illa vero in potentia tantum. Sed ea quae sunt in
phantasia, possunt cognosci ab Angelo sicut corporalia, cum phantasia
sit virtus corporis. Ergo videtur quod Angelus possit cognoscere
cogitationes intellectus.


&

さらに、私たちの知性の中にあるものは、表象像の中にあるものよりも、天使
に似ている。なぜなら、前者は現実態において知性認識されたものだが、後者
は可能態においてしか認識されていないものだからである。ところで、表象は
身体の能力だから、表象の中にあるものは、物体的なものと同様、天使によっ
て認識されうる。ゆえに、天使は知性の思いを認識しうる。


\\



{\scshape Sed contra}, quod est proprium Dei, non convenit
Angelis. Sed cognoscere cogitationes cordium est proprium Dei,
secundum illud Ierem.~{\scshape xvii}, {\itshape pravum est cor
hominis et inscrutabile, quis cognoscet illud? Ego, dominus, scrutans
corda}. Ergo Angeli non cognoscunt secreta cordium.


&

しかし反対に、神に固有なものは、天使たちに適合しない。ところが、心の思
いを認識することは、『エレミア書』17章「人間の心は歪んでいて詳しく見る
ことができない。だれがそれを認識しようか。主である私が、はらわたを精査
するものである」\footnote{「人の心は何にもまして、とらえ難く病んでいる。
誰がそれを知りえようか。心を探り、そのはらわたを究めるのは、主なる私で
ある。」(17:9-10)}によれば、神に固有である。ゆえに、天使たちは心の秘密
を認識しない。


\\



{\scshape Respondeo dicendum} quod cogitatio cordis dupliciter potest
cognosci. Uno modo, in suo effectu. Et sic non solum ab Angelo, sed
etiam ab homine cognosci potest; et tanto subtilius, quanto effectus
huiusmodi fuerit magis occultus. Cognoscitur enim cogitatio interdum
non solum per actum exteriorem, sed etiam per immutationem vultus, et
etiam medici aliquas affectiones animi per pulsum cognoscere
possunt. Et multo magis Angeli, vel etiam Daemones, quanto subtilius
huiusmodi immutationes occultas corporales perpendunt. Unde Augustinus
dicit, in libro {\itshape de divinatione Daemonum}, quod {\itshape
aliquando hominum dispositiones, non solum voce prolatas, verum etiam
cogitatione conceptas, cum signa quaedam in corpore exprimuntur ex
animo, tota facilitate perdiscunt}, quamvis in libro {\itshape
Retract}.~hoc dicat non esse asserendum quomodo fiat.


&

解答する。以下のように言われるべきである。心の思いは、二通りに知られう
る。一つは、その結果においてである。このかたちでは、天使だけでなく人間
によっても、それは知られうるのであり、そのような結果がより隠されている
だけ、それだけ薄く認識される。たとえば、思惟は、時として、外的な働きを
通してだけでなく、表情の変化を通しても認識されうるし、さらに医者は、脈
によって、なんらかの心の状態を知ることができる。天使たち、あるいは悪霊
たちも、このような隠された身体の変化を、より精妙に酌量すればするだけ、
より精妙に認識する。したがって、アウグスティヌスは『悪霊たちの予言につ
いて』で「時として、人々の状態を、それが言葉によって発せられたものだけ
でなく、思いによって抱かれたものもまた、なんらかのしるしが心を通して身
体において表現されるときには、そのすべてを見通す」と述べる。ただし、
『再考録』の中で、それがどのようにして生じるかを示すことはできないと言っ
ている。



\\


Alio modo possunt cognosci cogitationes, prout sunt in intellectu; et
affectiones, prout sunt in voluntate. Et sic solus Deus cogitationes
cordium et affectiones voluntatum cognoscere potest. Cuius ratio est,
quia voluntas rationalis creaturae soli Deo subiacet; et ipse solus in
eam operari potest, qui est principale eius obiectum, ut ultimus
finis; et hoc magis infra patebit. Et ideo ea quae in voluntate sunt,
vel quae ex voluntate sola dependent, soli Deo sunt nota. Manifestum
est autem quod ex sola voluntate dependet quod aliquis actu aliqua
consideret, quia cum aliquis habet habitum scientiae, vel species
intelligibiles in eo existentes, utitur eis cum vult. Et ideo dicit
apostolus, I {\itshape Cor}.~{\scshape ii}, quod {\itshape quae sunt
hominis, nemo novit nisi spiritus hominis, qui in ipso est}.


&

もう一つのかたちで、知性の中にあるものとしての思惟と、意志の中にあるも
のとしての情動が、認識されうる。そして、このかたちでは、ただ神だけが、
心の思惟と意志の情動を認識しうる。その理由は以下の通りである。理性的被
造物の意志は、ただ神だけに従属する。そして、究極目的としてその意志の主
要な対象である神だけが、その意志に向けて働きうる。そしてこのことは、以
下でより明らかとなるであろう。ゆえに、意志の中にあるもの、あるいは、意
志だけに依存するものは、神によってのみ知られる。ところで、ある人が、現
実に何かを考えることは、ただ意志だけに依存する。なぜなら、ある人が、知
の習態を持つとき、あるいは、可知的形象がその人の中に存在するとき、それ
を意志するときにそれを用いるからである。ゆえに、使徒は『コリントの信徒
への手紙1』第2章で、「人間に属するものどもは、その人自身の中にある人間
の霊だけが、それを知る」と述べている。

\\



{\scshape Ad primum ergo dicendum} quod modo cogitatio unius hominis
non cognoscitur ab alio, propter duplex impedimentum, scilicet propter
grossitiem corporis, et propter voluntatem claudentem sua
secreta. Primum autem obstaculum tolletur in resurrectione, nec est in
Angelis. Sed secundum impedimentum manebit post resurrectionem, et est
modo in Angelis. Et tamen qualitatem mentis, quantum ad quantitatem
gratiae et gloriae, repraesentabit claritas corporis. Et sic unus
mentem alterius videre poterit.


&


第一異論に対しては、それゆえ、次のように言われるべきである。今、ある人
の思惟が他の人によって知られないのは、二つの妨げ、すなわち、体の分厚さ
と、自分の秘密を隠す意志のためである。最初の障害は、復活においてなくな
り、天使においても存在しない。しかし、第二の障害は、復活のあと、そして、
今、天使の中にも存在する。しかし、身体の明澄さが、恩恵と栄光の量に応じ
て、精神の性質を表現するであろう。そしてこのようなかたちで、人は、他の
精神を見ることができるであろう。

\\



{\scshape Ad secundum dicendum} quod, etsi unus Angelus, species
intelligibiles alterius videat, per hoc quod modus intelligibilium
specierum, secundum maiorem et minorem universalitatem, proportionatur
nobilitati substantiarum; non tamen sequitur quod unus cognoscat
quomodo alius illis intelligibilibus speciebus utitur actualiter
considerando.

&

第二異論に対しては、次のように言われるべきである。ある天使は、他の天使
がもつ可知的形象を見ることができるが、それは、可知的形象の限度が、普遍
性の大小に即して、諸実体の高貴さに比例付けられているためにである。しか
しこのことから、ある天使が、他の天使が現実的に認識するときにかの可知的
諸形相をどのように使っているかを認識することは帰結しない。


\\



{\scshape Ad tertium dicendum} quod appetitus brutalis non est dominus
sui actus, sed sequitur impressionem alterius causae corporalis vel
spiritualis. Quia igitur Angeli cognoscunt res corporales et
dispositiones earum, possunt per haec cognoscere quod est in appetitu
et in apprehensione phantastica brutorum animalium; et etiam hominum,
secundum quod in eis quandoque appetitus sensitivus procedit in actum,
sequens aliquam impressionem corporalem, sicut in brutis semper
est. Non tamen oportet quod Angeli cognoscant motum appetitus
sensitivi et apprehensionem phantasticam hominis, secundum quod
moventur a voluntate et ratione, quia etiam inferior pars animae
participat aliqualiter rationem, sicut obediens imperanti, ut dicitur
in I Ethic. Nec tamen sequitur quod, si Angelus cognoscit quod est in
appetitu sensitivo vel phantasia hominis, quod cognoscat id quod est
in cogitatione vel voluntate, quia intellectus vel voluntas non
subiacet appetitui sensitivo et phantasiae, sed potest eis diversimode
uti.

&

第三異論に対しては、以下のように言われるべきである。非理性的動物の欲求
は、自らの行為の主ではなく、他の物体的、あるいは霊的原因の印象づけに伴
う。ゆえに、天使たちは物体的事物とそれらの諸状態を認識するので、それら
を通して非理性的動物の欲求や表象的把握の中にあるものを認識する。そして
人間のそのようなものもまた、人々の中に、時として、ちょうど非理性的動物
においては常にそうであるように、なんらかの物体的印象に伴うかたちで感覚
的欲求が行為へと出ていくかぎりで、認識する。しかし、天使たちが、人間の
感覚的欲求や表象的把握を、それらが意志や理性によって動かされるかぎりで
認識することにはならない。なぜなら、ちょうど『ニコマコス倫理学』第1巻
で言われるように、命令に従うものとして、魂の下位の部分ですら、なんらか
のかたちで理性を分有しているからである。また、仮に天使が、人間の感覚的
欲求や表象の中にあるののを認識するとしても、思惟や意志の中にあるものを
認識することは帰結しない。なぜなら、知性や意志は、感覚的欲求や表象に従
わず、むしろ、それらをさまざまに使用することができるからである。

\\


\end{longtable}
\newpage




\rhead{a.~5}
\begin{center}
{\Large {\bf ARTICULUS QUINTUS}}\\ {\large UTRUM ANGELI COGNOSCANT
MYSTERIA GRATIAE}\\ {\footnotesize IV {\itshape Sent.}, d.~10, a.~4,
qu$^a$ 4; {\itshape Ephes.}, cap.~3, lect.~3.}\\ {\Large 第五項\\天使
は恩恵の奥義を認識するか}
\end{center}

\begin{longtable}{p{21em}p{21em}}
{\huge A}{\scshape d quintum sic proceditur}. Videtur quod Angeli
mysteria gratiae cognoscant. Quia inter omnia mysteria excellentius
est mysterium incarnationis Christi. Sed hoc Angeli cognoverunt a
principio, dicit enim Augustinus, V {\itshape super Gen.~ad Litt}.,
quod {\itshape sic fuit hoc mysterium absconditum a saeculis in Deo,
ut tamen innotesceret principibus et potestatibus in caelestibus}. Et
dicit apostolus, I {\itshape ad Tim}.~{\scshape iii}, quod {\itshape
apparuit Angelis illud magnum sacramentum pietatis}. Ergo Angeli
mysteria gratiae cognoscunt.

&

第五項の問題へ、議論は以下のように進められる。天使たちは恩恵の奥義を認
識すると思われる。というのも、すべての奥義の中で、もっとも卓越している
のはキリストの受肉という奥義である。ところが、天使たちはこれを、はじめ
から認識していた。なぜなら、アウグスティヌスが『創世記逐語注解』5巻で、
「この奥義は、神の中に世々に隠されていたが、しかし、諸天において、諸々
の長や権力者には知られたものとして在った」と述べているからである。また、
使徒は『テモテへの手紙一』三章で、「かの偉大な敬神の秘蹟は、天使たちに
明らかだった」\footnote{「信心の秘められた真理はたしかに偉大です。すな
わち、キリストは肉において現れ、霊において義とされ、天使たちに見ら
れ、...」(3:16)}と言う。ゆえに、天使たちは、恩恵の奥義を認識する。


\\


{\scshape 2 Praeterea}, rationes omnium mysteriorum gratiae in divina
sapientia continentur. Sed Angeli vident ipsam Dei sapientiam, quae
est eius essentia. Ergo Angeli mysteria gratiae cognoscunt.


&

さらに、すべての恩恵の奥義の理念は、神の知恵の中に含まれている。ところ
が、天使たちは神の知恵それ自体を見る。それは神の本質だからである。ゆえ
に、天使たちは、恩恵の奥義を認識する。

\\


{\scshape 3 Praeterea}, prophetae per Angelos instruuntur, ut patet
per Dionysium, {\scshape iv} cap.~{\itshape Angel.~Hier}. Sed
prophetae mysteria gratiae cognoverunt, dicitur enim Amos {\scshape
iii}, {\itshape non faciet dominus verbum, nisi revelaverit secretum
ad servos suos, prophetas}. Ergo Angeli mysteria gratiae cognoscunt.


&


さらに、『天使階級論』四章のディオニュシウスによって明らかなとおり、預
言者たちは、天使たちを通して教えられる。ところが、預言者たちは恩恵の奥
義を認識していた。というのも、『アモス書』に「主は、自分の僕である預言
者たちに秘密を明らかにせずに、言葉を行うことはないだろう」
\footnote{「まことに、主なる神はその定められたことを僕なる預言者に示さ
ずには何事もなされない。」(3:7)}と言われているからである。ゆえに、天使
たちは、恩恵の奥義を認識する。

\\


{\scshape Sed contra est} quod nullus discit illud quod cognoscit. Sed
Angeli, etiam supremi, quaerunt de divinis mysteriis gratiae, et ea
discunt, dicitur enim {\scshape vii} cap.~{\itshape Cael.~Hier}., quod
sacra Scriptura {\itshape inducit quasdam caelestes essentias ad ipsum
Iesum quaestionem facientes, et addiscentes scientiam divinae eius
operationis pro nobis, et Iesum eas sine medio docentem}; ut patet
Isaiae {\scshape lxiii}, ubi quaerentibus Angelis, {\itshape quis est
iste qui venit de Edom?} respondit Iesus, {\itshape ego, qui loquor
iustitiam}. Ergo Angeli non cognoscunt mysteria gratiae.


&

しかし反対に、だれも、知っていることを教えてもらいはしない。ところが、
天使たちは、その最高の者たちですら、神の恩恵の奥義について尋ね、それを
学んでいる。たとえば『天上階級論』第七章で、聖書が「イエスその人に質問
し、私たちのための彼の働きについての神の知を学ぶ一種の天の諸本質と、彼
らに直接に教えるイエスを語っている」と述べられている。そしてこのことは、
『イザヤ書』63章に明らかな通りである。すなわちそこでは、「エドムから来
たその人はだれか」\footnote{「エドムから来るのは誰か。...」(63:1)}と尋
ねる天使たちに、イエスが「私だ。義を語る者だ」と答えている。ゆえに、天
使たちは恩恵の奥義を認識する。


\\


{\scshape Respondeo dicendum} quod in Angelis est cognitio duplex. Una
quidem naturalis, secundum quam cognoscunt res tum per essentiam suam,
tum etiam per species innatas. Et hac cognitione mysteria gratiae
Angeli cognoscere non possunt. Haec enim mysteria ex pura Dei
voluntate dependent, si autem unus Angelus non potest cognoscere
cogitationes alterius ex voluntate eius dependentes, multo minus
potest cognoscere ea quae ex sola Dei voluntate dependent. Et sic
argumentatur apostolus, I Cor. II, quae sunt hominis, nemo novit nisi
spiritus hominis, qui in ipso est. Ita et quae sunt Dei, nemo novit
nisi spiritus Dei.


&

解答する。以下のように言われるべきである。天使たちの中には、二通りの認
識がある。一つは本性的な認識であり、これに従って、天使たちは、あるとき
には自分の本質を通して、またあるときには生得の形象を通して事物を認識す
る。そして、この認識によって、天使たちが恩恵の奥義を認識することは不可
能である。というのも、これらの奥義は、神の純粋な意志に依存するが、もし、
ある天使が、別の天使がもつ、その天使の意志に依存する思惟を認識すること
ができないとすれば、ましてや、ただ神の意志だけに依存することがらを認識
することはできないからである。このような意味で、使徒は『コリントの信徒
への手紙一』二章で「ある人に属する事柄は、その人の中にあるその人の霊で
なければ知ることがない」\footnote{「人の内にある霊以外に、いったいだれ
が、人のことを知るでしょうか。」(2:11)}と論じるのである。

\\




Est autem alia Angelorum cognitio, quae eos beatos facit, qua vident
verbum et res in verbo. Et hac quidem visione cognoscunt mysteria
gratiae, non quidem omnia, nec aequaliter omnes sed secundum quod Deus
voluerit eis revelare; secundum illud apostoli, I {\itshape
Cor}.~{\scshape ii}, {\itshape nobis autem revelavit Deus per spiritum
suum}. Ita tamen quod superiores Angeli, perspicacius divinam
sapientiam contemplantes, plura mysteria et altiora in ipsa Dei
visione cognoscunt, quae inferioribus manifestant, eos illuminando. Et
horum etiam mysteriorum quaedam a principio suae creationis
cognoverunt; quaedam vero postmodum, secundum quod eorum officiis
congruit, edocentur.


&

他方、もう一つの天使の認識があり、それは天使たちを至福にするものであり、
それによって、天使たちは言葉を見、言葉において事物を見る。そしてこの直
視によって、天使たちは恩恵の奥義を認識するが、しかし、そのすべてをでは
なく、また、すべての天使が等しく見るのでもなく、かえって、神が天使たち
に啓示することを意志した限りでである。このことは、かの使徒の『コリント
の信徒への手紙一』二章「神は、自分の霊を通して、私たちに啓示した」
\footnote{「わたしたちには、神が霊によってそのことを明らかに示してくだ
さいました。」(2:10)}による。しかし、それは、次のようにである。すなわ
ち、上位の天使たちは、神の知恵をより透徹して眺めるので、より多くの、そ
してより深い奥義を、見神それ自体において認識し、それを、より下位の天使
たちに、彼らを照明することによって明示する。さらにまた、これらの奥義の
うち、あるものを、天使たちは自らの創造のはじめから認識していたが、別の
あるものは、後になって、彼らの任務に適合する限りにおいて、教えられる。


\\


{\scshape Ad primum ergo dicendum} quod de mysterio incarnationis
Christi dupliciter contingit loqui. Uno modo, in generali, et sic
omnibus revelatum est a principio suae beatitudinis. Cuius ratio est,
quia hoc est quoddam generale principium, ad quod omnia eorum officia
ordinantur, omnes enim sunt administratorii spiritus, ut dicitur
{\itshape Heb.} {\scshape i}, {\itshape in ministerium missi propter
eos qui haereditatem capiunt salutis}; quod quidem fit per
incarnationis mysterium. Unde oportuit de hoc mysterio omnes a
principio communiter edoceri.


&

第一異論に対しては、それゆえ、以下のように言われるべきである。キリスト
の受肉という奥義は、二通りに語られることがある。一つには、一般的にであ
り、その意味では、すべての天使たちに、自らの至福のはじめから、啓示され
ている。その根拠は以下の通りである。それは、天使たちのすべての任務がそ
れへと秩序づけられるところの、ある種一般的な原理だからである。というの
も、彼らは、『ヘブライ人への手紙』一章「救済の遺産をつかむ人々のために
仕えるために遣わされる」\footnote{「天使たちは皆、奉仕する霊であって、
救いを受け継ぐことになっている人々に仕えるために、遣わされたのではなかっ
たですか。」(1:14)}と言われるように、仕える霊だからである。そしてこの
救済は、受肉の奥義によって起こる。したがって、この奥義については、すべ
ての天使が、はじめから、共通に教えられる必要があった。



\\

--Alio modo possumus loqui de mysterio incarnationis quantum ad
speciales conditiones. Et sic non omnes Angeli a principio de omnibus
sunt edocti, immo quidam, etiam superiores Angeli, postmodum
didicerunt, ut patet per auctoritatem Dionysii inductam.

&

もう一つの意味で、私たちは、受肉の奥義について、その特殊な条件にかんし
て語りうる。この意味では、すべての天使がはじめからすべてについて教えら
れていたわけではなく、むしろ、引用されたディオニュシウスの権威によって
明らかなとおり、ある天使たちは、それが上位の天使たちであっても、後になっ
て教えられた。

\\


{\scshape Ad secundum dicendum} quod, licet Angeli beati divinam
sapientiam contemplentur, non tamen eam comprehendunt. Et ideo non
oportet quod cognoscant quidquid in ea latet.


&

第二異論に対しては、以下のように言われるべきである。至福な 天使たちが
神の知恵を見るとしても、しかし、それを把握するわけではない。ゆえに、天
使たちが神の知の中に隠されているものどもを何であれ認識するということに
はならない。

\\


{\scshape Ad tertium dicendum} quod quidquid prophetae cognoverunt de
mysteriis gratiae per revelationem divinam, multo excellentius est
Angelis revelatum. Et licet prophetis ea quae Deus facturus erat circa
salutem humani generis, in generali revelaverit; quaedam tamen
specialia apostoli circa hoc cognoverunt, quae prophetae non
cognoverant; secundum illud {\itshape Ephes}.~{\scshape iii},
{\itshape potestis, legentes, intelligere prudentiam meam in mysterio
Christi, quod aliis generationibus non est agnitum, sicut nunc
revelatum est sanctis apostolis eius}. Inter ipsos etiam prophetas,
posteriores cognoverunt quod priores non cognoverant; secundum illud
Psalmi {\scshape cxviii}, {\itshape super senes intellexi}. Et
Gregorius dicit quod {\itshape per successiones temporum,
crevit\footnote{cresco, -\u{e}re, cr\={e}vi, cr\={e}tum} divinae
cognitionis augmentum}.


&

第三異論に対しては、以下のように言われるべきである。預言者たちが、神の
啓示を通して恩恵の奥義について知っていたことは何であれ、天使たちに、は
るかに卓越したしかたで啓示されていた。そして、人類の救済のために神が行
うであろう事柄が、一般的なかたちで預言者たちに啓示されていたけれども、
これについて、ある特殊的なことがらを、預言者たちは知らなかったが、使徒
たちは知っていた、ということはある。それは、『エフェソの信徒への手紙』
三章「あなたたちは、それを読むならば、キリストの奥義におけるわたしの判
断を知解できる。それは、他の時代には、彼の聖なる使徒たちに啓示されてい
るようには、知られていなかった」\footnote{「あなたがたは、それを読めば、
キリストによって実現されるこの計画を、私がどのように理解しているかが分
かると思います。この計画は、キリスト以前の時代には人の子らに知らされて
いませんでしたが、今や霊によって、キリストの聖なる使徒たちは預言者たち
に啓示されました。」(3:4-5)}による。さらに、預言者たちの間でも、後の時
代の預言者たちは、先の時代の預言者たちが知らなかったことを知っていた。
これは、『詩編』118「私は老人たちを越えて理解した」\footnote{「長老た
ちにまさる英知を得させてください。」(119:100)}による。また、グレゴリウ
スは、「時間の経過によって、神の認識の議論は増加した」と述べている。

\end{longtable}
\end{document}