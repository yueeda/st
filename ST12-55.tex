\documentclass[10pt]{jsarticle}
\usepackage[utf8]{inputenc}
\usepackage[T1]{fontenc}
\usepackage{okumacro}
\usepackage{longtable}
\usepackage[polutonikogreek,english,japanese]{babel}
\usepackage{amsmath}
\usepackage{latexsym}
\usepackage{color}
\usepackage{otf}
\usepackage{schemata}
%----- header -------
\usepackage{fancyhdr}
\pagestyle{fancy}
\lhead{{\it Summa Theologiae} I-II, q.55}
%--------------------

\bibliographystyle{jplain}

\title{{\bf Prima Secundae}\\{\HUGE Summae Theologiae}\\Sancti Thomae
Aquinatis\\{\sffamily QUAESTIO QUINQUAGESIMAQUINTA}\\{\bf DE VIRTUTIBUS, \\QUANTUM AD SUAS ESSENTIAS}}
\author{Japanese translation\\by Yoshinori {\sc Ueeda}}
\date{Last modified \today}

%%%% コピペ用
%\rhead{a.~}
%\begin{center}
% {\Large {\bf }}\\
% {\large }\\
% {\footnotesize }\\
% {\Large \\}
%\end{center}
%
%\begin{longtable}{p{21em}p{21em}}
%
%&
%
%\\
%\end{longtable}
%\newpage

\begin{document}

\maketitle
\thispagestyle{empty}
\begin{center}
{\Large 『神学大全』第二部の一\\第五十五問\\徳について、その本質に関して}
\end{center}


\begin{longtable}{p{21em}p{21em}}
Consequenter considerandum est de habitibus in speciali. Et quia
habitus, ut dictum est, distinguuntur per bonum et malum, primo
dicendum est de habitibus bonis, qui sunt virtutes et alia eis
adiuncta, scilicet dona, beatitudines et fructus; secundo, de
habitibus malis, scilicet de vitiis et peccatis. Circa virtutes autem
quinque consideranda sunt, primo, de essentia virtutis; secundo, de
subiecto eius; tertio, de divisione virtutum; quarto, de causa
virtutis; quinto, de quibusdam proprietatibus virtutis. Circa primum
quaeruntur quatuor.


\begin{enumerate}
 \item utrum virtus humana sit habitus.
 \item utrum sit habitus operativus.
 \item utrum sit habitus bonus.
 \item de definitione virtutis.
\end{enumerate}

&

続いて、習慣について特殊的に考察されるべきである。
そして習慣は、すでに述べられたとおり、善と悪によって区別されるので、第一に善い習慣について語られるべきである。これは解くや徳に結び付けられた他のものども、すなわち、賜物、至福、果実である。そして第二に悪い習慣について、すなわち悪徳と罪について語られるべきである。そして徳については五つのことが考察されるべきである。第一に、徳の本質について、第二に、徳の基体について、第三に徳の区別について、第四に、徳の原因について、第五に、徳のある種の固有性について。第一をめぐって四つのことが問われる。


\begin{enumerate}
 \item 人間の徳は習慣か。
 \item それは活動的な習慣か。
 \item それは善い習慣か。
 \item 徳の定義について。
\end{enumerate}
\end{longtable}

\newpage
\rhead{a.~1}
\begin{center}
{\Large {\bf ARTICULUS PRIMUS}}\\
{\large UTRUM VIRTUS HUMANA SIT HABITUS}\\
{\footnotesize II {\itshape Sent.}, d.27, a.1; III, d.23, q.1, a.3, qu$^{a}$1, 3; {\itshape De Virtut.}, q.1, a.1; II {\itshape Ethic.}, lect.5.}\\
{\Large 第一項\\人間の徳は習慣か}
\end{center}

\begin{longtable}{p{21em}p{21em}}

{\scshape Ad primum sic proceditur}. Videtur quod virtus humana non
sit habitus. Virtus enim est ultimum potentiae, ut dicitur in I
{\scshape de Caelo}. Sed ultimum uniuscuiusque reducitur ad genus
illud cuius est ultimum, sicut punctum ad genus lineae. Ergo virtus
reducitur ad genus potentiae, et non ad genus habitus.

&

第一項の問題へ、議論は以下のように進められる。人間の徳は習慣でないと思
われる。理由は以下の通り。『天体論』第1巻で言われるように、徳は能力の
究極である。しかるに各々のものの究極は、それの究極であるところのものの
類に還元される。たとえば点は線の類に還元される。ゆえに徳は能力の類に還
元されるのであり、習慣の類に還元されるのではない。


\\

2. {\scshape Praeterea}, Augustinus dicit, in II {\itshape de libero
Arbit}., quod {\itshape virtus est bonus usus liberi arbitrii}. Sed
usus liberi arbitrii est actus. Ergo virtus non est habitus, sed
actus.

&

さらに、アウグスティヌスは『自由意志』第2巻で、「徳は自由裁量の善い使
用である」と述べている。しかるに自由裁量の善い使用は作用である。ゆえに
徳は習慣でなく作用である。

\\



3. {\scshape Praeterea}, habitibus non meremur, sed actibus, alioquin
homo mereretur continue, etiam dormiendo. Sed virtutibus meremur. Ergo
virtutes non sunt habitus, sed actus.

&

さらに、私たちは習慣によってではなく作用によって褒められる。そうでない
ならば、人は常に、寝ているときも、褒められただろう。しかるに徳によって
私たちは褒められる。ゆえに徳は習慣でなく作用である。

\\



4. {\scshape Praeterea}, Augustinus dicit, in libro {\itshape de
moribus Eccles}., quod {\itshape virtus est ordo amoris}. Et in libro
{\itshape Octoginta trium Quaest}., dicit quod {\itshape ordinatio
quae virtus vocatur, est fruendis frui, et utendis uti}. Ordo autem,
seu ordinatio, nominat vel actum, vel relationem. Ergo virtus non est
habitus, sed actus vel relatio.

&

さらに、アウグスティヌスは『教会の道徳について』という書物の中で「徳は
愛の秩序である」と述べている。そして『八十三問題集』という書の中で、
「徳と呼ばれる秩序づけは享受すべき者を享受し、用いられるべきものを用い
ることである」と述べている。しかるに秩序や秩序付けは、作用あるいは関係
の名である。ゆえに徳は習慣でなく、作用ないし関係である。

\\

5. {\scshape Praeterea}, sicut inveniuntur virtutes humanae, ita et
virtutes naturales. Sed virtutes naturales non sunt habitus, sed
potentiae quaedam. Ergo etiam neque virtutes humanae.

&

さらに、人間的な徳が見出されるように、自然的な徳もまた見出される。しか
るに自然的な徳は習慣でなく、ある種の能力である。ゆえに人間の徳も習慣で
ない。

\\

Sed contra est quod philosophus, in libro {\itshape Praedicament}.,
scientiam et virtutem ponit esse habitus.

&

しかし反対に、哲学者は『カテゴリー論』の中で、知と徳が習慣であるとして
いる。

\\



{\scshape Respondeo dicendum} quod virtus nominat quandam potentiae
perfectionem. Uniuscuiusque autem perfectio praecipue consideratur in
ordine ad suum finem. Finis autem potentiae actus est. Unde potentia
dicitur esse perfecta, secundum quod determinatur ad suum actum.

&

解答する。以下のように言われるべきである。徳とは、能力のある種の完成を
名付ける名である。しかるに各々のものの完全性は、とりわけそれ自身の目的
への秩序において考察される。そして能力の目的は作用である。このことから、
能力は、自らの作用へ限定される限りにおいて完全であると言われる。

\\



Sunt autem quaedam potentiae quae secundum seipsas sunt determinatae
ad suos actus; sicut potentiae naturales activae. Et ideo huiusmodi
potentiae naturales secundum seipsas dicuntur virtutes.


&

しかるに、それ自身において自らの作用へと限定されているような能力が存在
する。たとえば、自然的な作用的能力のように。ゆえに、このような自然的能
力は、それ自身に即して徳と言われる。

\\

---Potentiae autem rationales, quae sunt propriae hominis, non sunt
   determinatae ad unum, sed se habent indeterminate ad multa,
   determinantur autem ad actus per habitus, sicut ex supradictis
   patet. Et ideo virtutes humanae habitus sunt.

&

これに対して、人間に固有であるところの理性的な能力は、一つのものへ限定
されず、多くのものに対して限定されていない状態にある。しかし、前に述べ
られたことから明らかなとおり、習慣によって、作用へと限定される。ゆえに
人間的な徳は習慣である。


\\


{\scshape Ad primum ergo dicendum} quod quandoque virtus dicitur id ad
quod est virtus, scilicet vel obiectum virtutis, vel actus eius, sicut
fides dicitur quandoque id quod creditur, quandoque vero ipsum
credere, quandoque autem ipse habitus quo creditur.


&

第一異論に対しては、それゆえ、以下のように言われるべきである。時として、
徳がそれへと向かうもの、すなわち徳の対象や徳の作用が徳と言われる。それ
はちょうど、時として、信仰される事柄や、信じること自体、さらには、それ
によって信じられるところの習慣自体が「信仰」と言われるようにである。

\\

Unde quando dicitur quod virtus est ultimum potentiae, sumitur virtus
pro obiecto virtutis. Id enim in quod ultimo potentia potest, est id
ad quod dicitur virtus rei, sicut si aliquis potest ferre centum
libras et non plus, virtus eius consideratur secundum centum libras,
non autem secundum sexaginta. Obiectio autem procedebat ac si
essentialiter virtus esset ultimum potentiae.

&

このことから、徳は能力の究極である、と言われるとき、徳は、徳の対象とし
て理解されている。実際、能力が究極的に、そこまでなしうるものは、それに
関して事物の徳が語られるところのものである。たとえば、もしある人が100
リブラ\footnote{リブラは重さの単位で約12オンス。100リブラは約34kgとな
る。}を持ち運ぶことができるならば、彼の徳は60リブラではなく100リブラに
即して考慮されるように。しかし異論は、徳が本質的に能力の究極であるかの
ように論じていた。


\\

{\scshape Ad secundum dicendum} quod bonus usus liberi arbitrii
dicitur esse virtus, secundum eandem rationem, quia scilicet est id ad
quod ordinatur virtus sicut ad proprium actum. Nihil est enim aliud
actus virtutis quam bonus usus liberi arbitrii.

&

第二異論に対しては以下のように言われるべきである。自由裁量のよいしよう
が徳と言われるのも同じ理由による。すなわち、それが、固有の作用としての
それへ、徳が秩序付けられているからである。じっさい、徳の作用は、自由裁
量の善い使用に他ならない。

\\

{\scshape Ad tertium dicendum} quod aliquo dicimur mereri
dupliciter. Uno modo, sicut ipso merito, eo modo quo dicimur currere
cursu, et hoc modo meremur actibus. Alio modo dicimur mereri aliquo
sicut principio merendi, sicut dicimur currere potentia motiva, et sic
dicimur mereri virtutibus et habitibus.

&

第三異論に対しては以下のように言われるべきである。私たちが何かによって
褒められるというのには二通りの意味がある。一つには、「走りによって走る」
と私たちが言われるように、その褒められるもの自体によってである場合と、
もう一つには、「走ることは動く能力である」と言われるように、褒められる
事柄の根源としての何かによって、褒められると言われる。この意味で、私た
ちは、徳や習慣によって褒められると言われる。

\\



{\scshape Ad quartum dicendum} quod virtus dicitur ordo vel ordinatio
amoris, sicut id ad quod est virtus, per virtutem enim ordinatur amor
in nobis.

&

第四異論に対しては以下のように言われるべきである。徳は、徳がそれに対し
てあるところのものとして、愛への秩序あるいは秩序づけと言われる。なぜな
ら、徳によって、愛が私たちの中で秩序付けられるからである。


\\



{\scshape Ad quintum dicendum} quod potentiae naturales sunt de se
determinatae ad unum, non autem potentiae rationales. Et ideo non est
simile, ut dictum est.

&

第五異論に対しては以下のように言われるべきである。自然的な能力は、それ
自体から一つのものへと限定されているが、理性的な能力はそうでない。ゆえ
に、先に述べられたとおり、同じではない。

\end{longtable}
\newpage




\rhead{a.~2}
\begin{center}
{\Large {\bf ARTICULUS SECUNDUS}}\\
{\large UTRUM VIRTUS HUMANA SIT HABITUS OPERATIVUS}\\
{\footnotesize III {\itshape Sent.}, d.23, q.1, a.3, qu$^{a}$ 1.}\\
{\Large 第二項\\人間の徳は活動的な習慣か}
\end{center}

\begin{longtable}{p{21em}p{21em}}

{\scshape Ad secundum sic proceditur}. Videtur quod non sit de ratione
virtutis humanae quod sit habitus operativus. Dicit enim Tullius, in
IV {\itshape de Tuscul. quaest}., quod sicut est sanitas et
pulchritudo corporis, ita est virtus animae. Sed sanitas et
pulchritudo non sunt habitus operativi. Ergo neque etiam virtus.

&

第二項の問題へ、議論は以下のように進められる。人間の徳の性格に、活動的
な習慣であることは属していないと思われる。理由は以下の通り。キケロは
『トゥスクルム荘対談集』第4巻の中で、身体の健康や美があるように、その
ように魂の徳があると述べている。しかるに健康と美は活動的な習慣ではない。
ゆえに徳もそうでない。


\\



2. {\scshape Praeterea}, in rebus naturalibus invenitur virtus non
solum ad agere, sed etiam ad esse, ut patet per philosophum, in I
{\itshape de Caelo}, quod quaedam habent virtutem ut sint semper,
quaedam vero non ad hoc quod sint semper, sed aliquo tempore
determinato. Sed sicut se habet virtus naturalis in rebus naturalibus,
ita se habet virtus humana in rationalibus. Ergo etiam virtus humana
non solum est ad agere, sed etiam ad esse.

&

さらに、『天体論』第1巻の哲学者によって明らかなとおり、自然的諸事物に
おいては、作用するための徳だけでなく、存在するための徳もある。すなわち、
あるものどもは常にそう在るような徳を持つが、別のものどもは常にそう在る
ことではなく、ある限られたときにそう在ることのための徳を持つ。しかし自
然的事物において自然的な徳があるように、理性的なものにおいて人間の徳が
ある。ゆえに人間の徳もまた、作用することだけではなく、存在することのた
めにもある。

\\



3. {\scshape Praeterea}, philosophus dicit, in VII {\itshape Physic}.,
quod virtus est {\itshape dispositio perfecti ad optimum}. Optimum
autem ad quod hominem oportet disponi per virtutem, est ipse Deus, ut
probat Augustinus in libro II {\itshape de Moribus Eccles}.; ad quem
disponitur anima per assimilationem ad ipsum. Ergo videtur quod virtus
dicatur qualitas quaedam animae in ordine ad Deum, tanquam
assimilativa ad ipsum, non autem in ordine ad operationem. Non igitur
est habitus operativus.

&

さらに、哲学者は『自然学』第7巻の中で、徳は、「最善のものに向かう完全
なものの態勢」であると述べている。しかるに人間が徳によってそれへ向けて
態勢付けられるべき最善のものとは、アウグスティヌスが『教会の道徳につい
て』第1巻で証明するように、神自身である。魂は、神への類似を通して、神
へと態勢付けられる。ゆえに、徳は、神への秩序における、神へと類似化しう
るものとしての魂の一種の性質であって、働きへの秩序においてそのようなも
のではないと思われる。ゆえに、それは活動的な習慣でない。

\\



{\scshape Sed contra est} quod philosophus dicit, in II {\itshape
Ethic}., quod {\itshape virtus uniuscuiusque rei est quae opus eius
bonum reddit}.

&

しかし反対に、哲学者は『ニコマコス倫理学』第2巻で「各々の事物の徳は、
彼の業を善となすものである」と述べている。

\\



{\scshape Respondeo dicendum} quod virtus, ex ipsa ratione nominis,
importat quandam perfectionem potentiae, ut supra dictum est. Unde,
cum duplex sit potentia, scilicet potentia ad esse et potentia ad
agere, utriusque potentiae perfectio virtus vocatur.

&

解答する。以下のように言われるべきである。先に述べられたとおり、徳は、
その名の概念自体に基づけば、能力のなんらかの完全性を意味する。このこと
から、存在への能力と作用への能力という二通りの能力があるが、どちらの能
力の完全性もまた徳と呼ばれる。

\\

Sed potentia ad esse se tenet ex parte materiae, quae est ens in
potentia, potentia autem ad agere se tenet ex parte formae, quae est
principium agendi, eo quod unumquodque agit inquantum est actu.


&

また、存在への能力は、可能態における有である質料の側からあるが、他
方、作用への能力は、作用の根源である形相の側からある。各々のものは現実
態においてある限りにおいて作用するからである。


\\

In constitutione autem hominis, corpus se habet sicut materia, anima
vero sicut forma. Et quantum quidem ad corpus, homo communicat cum
aliis animalibus; et similiter quantum ad vires quae sunt animae et
corpori communes; solae autem illae vires quae sunt propriae animae,
scilicet rationales, sunt hominis tantum.


&

ところで、人間の構成において、身体は質料として、魂は形相として関係する。
そして身体の点で人間は他の動物と共通する。魂と身体に共通する力に関して
も同様である。しかし魂に固有の力、すなわち理性的な力は、ただ人間だけに
属する。


\\


Et ideo virtus humana, de qua loquimur, non potest pertinere ad
corpus; sed pertinet tantum ad id quod est proprium animae. Unde
virtus humana non importat ordinem ad esse, sed magis ad agere. Et
ideo de ratione virtutis humanae est quod sit habitus operativus.


&

ゆえに、私たちが語っている人間の徳は、身体に属することがありえず、魂に
固有であるもののみに属する。したがって、人間の徳は存在への秩序を意味せ
ず、むしろ作用への秩序を意味する。ゆえに人間の徳の概念には、活動的な習
慣であることが属する。


\\



{\scshape Ad primum ergo dicendum} quod modus actionis sequitur
dispositionem agentis, unumquodque enim quale est, talia operatur. Et
ideo, cum virtus sit principium aliqualis operationis, oportet quod in
operante praeexistat secundum virtutem aliqua conformis
dispositio. 


&

第一異論に対しては、それゆえ、以下のように言われるべきである。作用のあ
り方は、作用者の体制に従う。なぜなら、各々のものは、それ存在する仕方で
働くからである。ゆえに徳は何らかの働きの根源なので、働くものの中には、
何らかの徳にしたがって、適合的な態勢が先在しなければならない。

\\



Facit autem virtus operationem ordinatam.  Et ideo ipsa virtus est
quaedam dispositio ordinata in anima, secundum scilicet quod potentiae
animae ordinantur aliqualiter ad invicem, et ad id quod est extra. Et
ideo virtus, inquantum est conveniens dispositio animae, assimilatur
sanitati et pulchritudini, quae sunt debitae dispositiones
corporis. Sed per hoc non excluditur quin virtus etiam sit operationis
principium.

&

しかるに、徳は働きを秩序付けられたものにする。ゆえに、徳それ自体は、魂
の中で秩序付けられたある種の態勢である。すなわち、魂の能力が何らかのか
たちで、相互に、そして外的なものに対して秩序付けられる。ゆえに徳は、魂
の適合的な態勢である限りにおいて、身体のしかるべき態勢であるところの健
康や美に似たものとされる。しかしこのことによって、その徳が働きの根源で
あることが排除されるわけではない。

\\



{\scshape Ad secundum dicendum} quod virtus quae est ad esse, non est
propria hominis, sed solum virtus quae est ad opera rationis, quae
sunt propria hominis.

&

第二異論に対しては以下のように言われるべきである。存在に対する徳は人間
に固有のものではない。人間に固有なのは、人間に固有である理性の業に対す
る徳だけである。

\\



{\scshape Ad tertium dicendum} quod, cum Dei substantia sit eius
actio, summa assimilatio hominis ad Deum est secundum aliquam
operationem. Unde, sicut supra dictum est, felicitas sive beatitudo,
per quam homo maxime Deo conformatur, quae est finis humanae vitae, in
operatione consistit.

&

第三異論に対しては以下のように言われるべきである。神の実体は神の作用で
あるので、人間の神に対する最高の類似化は何らかの働きに即してある。した
がって、先に述べられたとおり、人間がそれによって最大限に神に一致するも
のである幸福ないし至福は、人間の生の目的であり、働きにおいて成立する。


\end{longtable}
\newpage


\rhead{a.~3}
\begin{center}
{\Large {\bf ARTICULUS TERTIUS}}\\
{\large UTRUM VIRTUS HUMANA SIT HABITU BONUS}\\
{\footnotesize III {\itshape Sent.}, III, d.23, q.1, a.3, qu$^{a}$ 1; d.24, q.2, a.1; II {\itshape Ethic}., lect.6.}\\
{\Large 第三項\\人間の徳は善い習慣か}
\end{center}

\begin{longtable}{p{21em}p{21em}}

{\scshape Ad tertium sic proceditur}. Videtur quod non sit de ratione
virtutis quod sit habitus bonus. Peccatum enim in malo semper
sumitur. Sed etiam peccati est aliqua virtus; secundum illud I
{\itshape ad Cor}.~{\scshape xv}: {\itshape Virtus peccati lex}. Ergo
virtus non semper est habitus bonus.

&

第三項の問題へ、議論は以下のように進められる。徳\footnote{この項はとく
に、virtusが「ちから」を意味することを念頭に読むべきであり、「徳」いう
訳語ではそれが伝わらないが、やむをえないので引き続き「徳」と訳す。}の
概念の中にそれが善い習慣であることは含まれていないと思われる。理由は以
下の通り。罪は常に悪において理解される。しかるに『コリントの信徒への手
紙一』第15章「罪の徳は法である」\footnote{「死の棘は罪であり、罪の力は
律法です」(15:56)}によれば、罪にも何らかの徳がある。ゆえに徳は常に善
い習慣とは限らない。

\\



2. {\scshape Praeterea}, virtus potentiae respondet. Sed potentia non
solum se habet ad bonum, sed etiam ad malum; secundum illud {\itshape
Isaiae} {\scshape v}: {\itshape Vae, qui potentes estis ad bibendum
vinum, et viri fortes ad miscendam ebrietatem}. Ergo etiam virtus se
habet et ad bonum et ad malum.

&

さらに、徳は能力に対応する。しかるに能力は、かの『イザヤ書』第5章「ワ
インを飲む力の強いあなたたち、酩酊を混ぜるのに強い男たちに災いあれ」
\footnote{「災いあれ、ぶどう酒を飲むことでは勇者/麦の酒を混ぜて飲むこ
とにかけては豪快な者に。」(5:22)}によれば、善だけでなく悪に対しても関
係する。ゆえに徳もまた善と悪に関係する。


\\



3. {\scshape Praeterea}, secundum apostolum, II {\itshape ad
Cor}.~{\scshape xii}, {\itshape virtus in infirmitate perficitur}. Sed
infirmitas est quoddam malum. Ergo virtus non solum se habet ad bonum,
sed etiam ad malum.

&

さらに、使徒によれば、『コリントの信徒への手紙二』第11章で「徳は弱さに
おいて完成される」。しかるに、弱さは一種の悪である。ゆえに徳は善だけで
なく悪にも関係する。

\\



{\scshape Sed contra est} quod Augustinus dicit, in libro {\itshape de
Moribus Eccles}., {\itshape nemo autem dubitaverit quod virtus animam
facit optimam}. Et philosophus dicit, in II {\itshape Ethic}.,
{\itshape quod virtus est quae bonum facit habentem, et opus eius
bonum reddit}.

&

しかし反対に、アウグスティヌスは『教会の道徳について』という書物の中で
「徳が魂を最善にすることをだれも疑わなかった」と言う。また哲学者は『ニ
コマコス倫理学』第2巻で「徳は、それを持つ人を善くし、彼の業を善いもの
にする」と言っている。

\\



{\scshape Respondeo dicendum} quod, sicut supra dictum est, virtus
importat perfectionem potentiae, unde virtus cuiuslibet rei
determinatur ad ultimum in quod res potest, ut dicitur in I {\itshape
de Caelo}. Ultimum autem in quod unaquaeque potentia potest, oportet
quod sit bonum, nam omne malum defectum quendam importat; unde
Dionysius dicit, in {\scshape iv} cap.~{\itshape de Div.~Nom}., quod
omne malum est {\itshape infirmum}. Et propter hoc oportet quod virtus
cuiuslibet rei dicatur in ordine ad bonum. Unde virtus humana, quae
est habitus operativus, est bonus habitus, et boni operativus.

&

解答する。以下のように言われるべきである。前に述べられたとおり、徳は能
力の完全性を意味する。したがって、『天体論』第1巻で言われるように、ど
んな事物の徳も、その事物ができる事柄において究極のものへと限定されてい
る。しかるに、各々の能力ができる事柄において究極のものは、善でなければ
ならない。なぜなら、すべて悪は、何らかの欠陥を意味するからである。この
ことからディオニュシウスは『神名論』第4章で、すべての悪は「弱い」と述
べている。そしてこのために、どんな事物の徳も、善への秩序において語られ
なければならない。したがって、活動的な習慣である人間の徳は、善い習慣で
あり、善をなしうるものである。

\\



{\scshape Ad primum ergo dicendum} quod sicut perfectum, ita et bonum
dicitur metaphorice in malis, dicitur enim et perfectus fur sive
latro, et bonus fur sive latro; ut patet per philosophum, in V
{\itshape Metaphys}. Secundum hoc ergo, etiam virtus metaphorice in
malis dicitur. Et sic {\itshape virtus peccati} dicitur lex, inquantum
scilicet per legem occasionaliter est peccatum augmentatum, et quasi
ad maximum suum posse pervenit.

&

第一異論に対しては、それゆえ、以下のように言われるべきである。悪におい
て、善ということが、完全ということと同様に、比喩的に語られる。たとえば、
『形而上学』第5巻の哲学者によって明らかなとおり、完全なスリや強盗とか、
善いスリや強盗と言われるように。それゆえ、この限りで、徳も比喩的に悪に
おいて語られる。この意味で、「罪の徳・ちから」が、法によって時として罪
が増やされ、あたかも自らの最大限に到達しうるかぎりにおいて、法と言われ
る。


\\



{\scshape Ad secundum dicendum} quod malum ebrietatis et nimiae
potationis, consistit in defectu ordinis rationis. Contingit autem,
cum defectu rationis, esse aliquam potentiam inferiorem perfectam ad
id quod est sui generis, etiam cum repugnantia vel cum defectu
rationis. Perfectio autem talis potentiae, cum sit cum defectu
rationis, non posset dici virtus humana.

&

第二異論に対しては以下のように言われるべきである。酩酊の悪、過度の飲酒
の悪は理性の秩序の欠陥において成立する。しかるに、理性の欠陥があるとき
に、ある下位の能力が、理性に反してでも、あるいは理性の欠陥を伴って、自
らの類に属するものとの関係で完全であることがありうる。しかしこのような
能力の完成は、理性の欠陥と共にあるのだから、人間の徳と言われることはで
きない。

\\



{\scshape Ad tertium dicendum} quod tanto ratio perfectior esse
ostenditur, quanto infirmitates corporis et inferiorum partium magis
potest vincere seu tolerare. Et ideo virtus humana, quae rationi
attribuitur, {\itshape in infirmitate perfici} dicitur, non quidem
rationis, sed in infirmitate corporis et inferiorum partium.

&

第三異論に対しては以下のように言われるべきである。理性がより完全である
ことが示されるほど、身体や下位の諸部分の弱さに、より打ち克ち滅すること
ができる。ゆえに理性に帰せられる人間の徳は、「弱さにおいて完成される」
と言われる。これは理性の弱さという意味ではなく、身体と下位の諸部分の弱
さにおいてということである。


\end{longtable}
\newpage



\rhead{a.~4}
\begin{center}
{\Large {\bf ARTICULUS QUARTUS}}\\
{\large UTRUM VIRTUS CONVENIENTER DEFINIATUR}\\
{\footnotesize II {\itshape Sent.}, II, d.27, a.2; {\itshape De Virtut.}, q.1, a.2.}\\
{\Large 第四項\\徳は適切に定義されているか}
\end{center}

\begin{longtable}{p{21em}p{21em}}
{\scshape Ad quartum sic proceditur}. Videtur quod non sit conveniens
definitio virtutis quae solet assignari, scilicet, {\itshape virtus
est bona qualitas mentis, qua recte vivitur, qua nullus male utitur,
quam Deus in nobis sine nobis operatur}. Virtus enim est bonitas
hominis, ipsa enim est {\itshape quae bonum facit habentem}. Sed
bonitas non videtur esse bona, sicut nec albedo est alba. Igitur
inconvenienter dicitur quod virtus est {\itshape bona qualitas}.

&

第四項の問題へ議論は以下のように進められる。通常指定されている徳の定義、
すなわち、「徳とは精神の善い性質であり、それによって正しく生きられ、だ
れもそれを悪く用いず、神が私たちの中に私たちなしに為すものである」は適
切な徳の定義でないと思われる。理由は以下の通り。徳とは人間の善性であり、
それは、それを持つ人を善いものにする。しかし白性が白くないように、善性
は善いものでない。ゆえに徳が「善い性質」と言われるのは不適切である。



\\



2. {\scshape Praeterea}, nulla differentia est communior suo genere,
cum sit generis divisiva. Sed bonum est communius quam qualitas,
convertitur enim cum ente. Ergo {\itshape bonum} non debet poni in
definitione virtutis, ut differentia qualitatis.


&

さらに、どんな種差も類より共通でない。種差は類を分けるものだから。しか
るに善は性質よりも共通である。善は有と置換されるのだから。ゆえに善は、
性質の種差として徳の定義の中に置かれるべきでない。

\\



3. {\scshape Praeterea}, sicut Augustinus dicit, in XII {\itshape de
Trin}., :{\itshape Ubi primo occurrit aliquid quod non sit nobis
pecoribusque commune, illud ad mentem pertinet}. Sed quaedam virtutes
sunt etiam irrationabilium partium; ut philosophus dicit, in III
{\itshape Ethic}. Non ergo omnis virtus est bona qualitas {\itshape
mentis}.

&

さらに、アウグスティヌスが『三位一体論』第12巻で述べるように、「私たち
と獣に共通でない何かが最初に生じるところで、それは精神に属する」。しか
るに哲学者が『ニコマコス倫理学』第3巻で言うように、ある徳は非理性的な
部分にも属する。ゆえにすべての徳が「精神の」善い性質であるわけではない。

\\



4. {\scshape Praeterea}, rectitudo videtur ad iustitiam pertinere,
unde idem dicuntur recti, et iusti. Sed iustitia est species
virtutis. Inconvenienter ergo ponitur rectum in definitione virtutis,
cum dicitur, {\itshape qua recte vivitur}.


&

さらに、正しさは正義に属すると思われる。したがって、同じ人々が正しいと
も正義とも言われる。しかるに正義は徳の種である。ゆえに「それによって正
しく生きられ」と言われるときに、正しいということが徳の定義の中に置かれ
るのは不適切である。

\\



5. {\scshape Praeterea}, quicumque superbit de aliquo, male utitur
eo. Sed multi superbiunt de virtute, dicit enim Augustinus, in
{\itshape Regula}, quod {\itshape superbia etiam bonis operibus
insidiatur, ut pereant}. Falsum est ergo quod nemo virtute male
utatur.


&

さらに、だれであれ何かを誇る人は、それを悪く用いている。しかるに、多く
の人々が徳を誇っている。たとえばアウグスティヌスは『基準』で「慢心は善
い行いを待ち伏せして、それを殺そうとしている」と述べている。ゆえにだれ
も徳を悪く使わないというのは偽である。

\\



6. {\scshape Praeterea}, homo per virtutem iustificatur. Sed
Augustinus dicit, super illud Ioan., {\itshape Maiora horum faciet :
Qui creavit te sine te, non iustificabit te sine te}. Inconvenienter
ergo dicitur quod virtutem {\itshape Deus in nobis sine nobis
operatur}.


&

さらに、人間は徳を通して義化される。しかるにアウグスティヌスはかの『ヨ
ハネによる福音書』の「これらのより大きなことを彼は為すだろう」について、
「あなたなしにあなたを造った人は、あなたなしにあなたを義化したりしない
だろう」と述べている。ゆえに、徳を「神は私たちのうちに私たちなしに働く」
と言われるのは不適切である。

\\



{\scshape Sed contra est} auctoritas Augustini, ex cuius verbis
praedicta definitio colligitur, et praecipue in II {\scshape de Libero
Arbitrio}.


&

しかし反対に、上述の定義がそこから、とくに『自由意志』第2巻で集められ
たアウグスティヌスの権威がある。

\\



{\scshape Respondeo dicendum} quod ista definitio perfecte
complectitur totam rationem virtutis. Perfecta enim ratio
uniuscuiusque rei colligitur ex omnibus causis eius. Comprehendit
autem praedicta definitio omnes causas virtutis. Causa namque formalis
virtutis, sicut et cuiuslibet rei, accipitur ex eius genere et
differentia, cum dicitur {\itshape qualitas bona}, genus enim virtutis
{\itshape qualitas} est, differentia autem {\itshape bonum}. Esset
tamen convenientior definitio, si loco qualitatis {\itshape habitus}
poneretur, qui est genus propinquum.


&

解答する。以下のように言われるべきである。この定義は、徳の全性格を完全
に包含する。理由は以下の通り。各々の事物の完全な性格は、それのすべての
原因から集められる。しかるに、前述の定義は徳のすべての原因を含んでいる。
すなわち、徳の形相的原因は、どんな事物の形相的原因とも同じように、それ
の類と種差から受け取られる。「善い性質」と言われるとき、徳の類は「性質」
であり、種差は「善いもの」である。しかし、もし性質のところに「習慣」が
置かれていたら、より適切だっただろう。それは近接する類だからである。

\\




Virtus autem non habet materiam {\itshape ex qua}, sicut nec alia
accidentia, sed habet materiam {\itshape circa quam}; et materiam
{\itshape in qua}, scilicet subiectum. Materia autem {\itshape circa
quam} est obiectum virtutis; quod non potuit in praedicta definitione
poni, eo quod per obiectum determinatur virtus ad speciem; hic autem
assignatur definitio virtutis in communi. Unde ponitur subiectum loco
causae materialis, cum dicitur quod est bona qualitas {\itshape
mentis}.

&

他方、徳は他の附帯性と同様に、「それに基づいて」であるような質料をもた
ず、むしろ「それをめぐって」であるような質料と、「それにおいて」である
ような質料、すなわち基体をもつ。「それをめぐって」である質料は、徳の対
象であり、それは上述の定義の中に入れられなかった。なぜなら、対象によっ
て徳は種へ限定されるが、ここでは徳の定義が共通に指定されているからであ
る。このことから、「精神の」善い性質であると言われるとき、質料因の場所
に基体が置かれている。


\\


Finis autem virtutis, cum sit habitus operativus, est ipsa
operatio. Sed notandum quod habituum operativorum aliqui sunt semper
ad malum, sicut habitus vitiosi; aliqui vero quandoque ad bonum, et
quandoque ad malum, sicut opinio se habet ad verum et ad falsum;
virtus autem est habitus semper se habens ad bonum. Et ideo, ut
discernatur virtus ab his quae semper se habent ad malum, dicitur,
{\itshape qua recte vivitur}, ut autem discernatur ab his quae se
habent quandoque ad bonum, quandoque ad malum, dicitur, {\itshape qua
nullus male utitur}.


&

また、徳の目的は、活動的な習慣なので、働きそれ自体である。しかし、以下
のことに注意されるべきである。すなわち、活動的な習慣のうち、あるものは、
邪悪な習慣のように、常に悪に向いている。他方ある習慣は、ちょうど意見が
真と偽に関係するように、あるときには善に、またあるときには悪に向く。し
かし徳は常に善へ関係する習慣である。ゆえに、徳が、常に悪に向いているも
のと区別されるように、「それによって正しく生きられる」と言われ、あると
きには善に、またあるときに把握に向くものと区別されるために、「それをだ
れも悪く用いない」と言われる。

\\


Causa autem efficiens virtutis infusae, de qua definitio datur, Deus
est. Propter quod dicitur, {\itshape quam Deus in nobis sine nobis
operatur}. Quae quidem particula si auferatur, reliquum definitionis
erit commune omnibus virtutibus, et acquisitis et infusis.

&

さらに、それについて定義が為されている、注入された徳の作出因は神である。
このため「神が私たちの中で私たちなしに働く」と言われている。この部分が
取り除かれると、定義の残りの部分は、獲得された徳と注入された徳のすべて
に共通するであろう。

\\



{\scshape Ad primum ergo dicendum} quod id quod primo cadit in
intellectu, est ens, unde unicuique apprehenso a nobis attribuimus
quod sit ens; et per consequens quod sit unum et bonum, quae
convertuntur cum ente. Unde dicimus quod essentia est ens et una et
bona; et quod unitas est ens et una et bona; et similiter de bonitate.

&

第一異論に対しては、それゆえ、以下のように言われるべきである。知性に第
一に入ってくるのは有である。このことから、私たちによって捉えられるどん
なものにも、それが有であることを私たちは帰する。結果的に、それは一であ
り善であり、それらは有と置換される。このことから、私たちは本質が有であ
り一であり善であると言い、一性が有であり善であり一であると言う。それは
善性についても同様である。


\\

Non autem hoc habet locum in specialibus formis, sicut est albedo et
sanitas, non enim omne quod apprehendimus, sub ratione albi et sani
apprehendimus. 

&

しかしこのことは、白性や健康さのような特殊的な形相には当てはまらない。
なぜなら、私たちが捉えるすべてのものを、白や健康という性格のもとで捉え
るわけではないから。


\\

Sed tamen considerandum quod sicut accidentia et formae
non subsistentes dicuntur entia, non quia ipsa habeant esse, sed quia
eis aliquid est; ita etiam dicuntur bona vel una, non quidem aliqua
alia bonitate vel unitate, sed quia eis est aliquid bonum vel
unum. Sic igitur et virtus dicitur bona, quia ea aliquid est bonum.

&

しかし、以下のことが考察されるべきである。ちょうど附帯性や自存しない形
相が有と言われるのは、それら自身が存在を持つからではなく、むしろそれら
によって何かが在るからであるように、あるものどもは、別の善性や一性によっ
て善や一と言われるのではなく、それによって何かが善や一であるから、そう
言われる。ゆえに、このような仕方で、徳もまた善と言われる。何かが徳によっ
て善であるからである。


\\


{\itshape Ad secundum dicendum} quod bonum quod ponitur in definitione
virtutis, non est bonum commune, quod convertitur cum ente, et est in
plus quam qualitas, sed est bonum rationis, secundum quod Dionysius
dicit, in {\scshape iv} cap.~{\itshape de Div.~Nom}., quod {\itshape
bonum animae est secundum rationem esse}.

&

第二異論に対しては以下のように言われるべきである。徳の定義の中に置かれ
る善は、有と置換され、性質よりも大きい共通の善ではなく、理性の善である。
これはディオニュシウスが『神名論』第4章で「理性に即してあることは、魂
の善である」による。

\\



{\itshape Ad tertium dicendum} quod virtus non potest esse in
irrationali parte animae, nisi inquantum participat rationem, ut
dicitur in I {\itshape Ethic}. Et ideo ratio, sive mens, est proprium
subiectum virtutis humanae.

&

第三異論に対しては以下のように言われるべきである。『ニコマコス倫理学』
第1巻で述べられるように、徳は、理性を分有する限りにおいてでないかぎり、
魂の非理性的部分の中にはありえない。ゆえに理性ないし精神は、人間の徳の
固有の基体である。

\\



{\itshape Ad quartum dicendum} quod iustitiae est propria rectitudo
quae constituitur circa res exteriores quae in usum hominis veniunt,
quae sunt propria materia iustitiae, ut infra patebit. Sed rectitudo
quae importat ordinem ad finem debitum et ad legem divinam, quae est
regula voluntatis humanae, ut supra dictum est, communis est omni
virtuti.

&

第四異論に対しては以下のように言われるべきである。後で明らかになるよう
に、正義は、正義の固有の質料であるところの、人間の使用のもとに入ってく
る外的な事柄をめぐって構成される固有の正しさである。しかるに、しかるべ
き目的や神の法への秩序を含意する正しさは、先に述べられたとおり人間の意
志の尺度であり、すべての徳に共通する。

\\



{\scshape Ad quintum dicendum} quod virtute potest aliquis male uti
tanquam obiecto, puta cum male sentit de virtute, cum odit eam, vel
superbit de ea, non autem tanquam principio usus, ita scilicet quod
malus sit actus virtutis.


&

第五異論に対しては以下のように言われるべきである。ある人が、対象として
の徳を悪く用いることはありうる。たとえば、徳を憎んだり、徳について自慢
したりする場合のように、徳について悪く感覚する場合のように。しかし使用
の根源として悪く用いる、すなわち徳の作用が悪いということはありえない。

\\



{\scshape Ad sextum dicendum} quod virtus infusa causatur in nobis a
Deo sine nobis agentibus, non tamen sine nobis consentientibus. Et sic
est intelligendum quod dicitur, {\itshape quam Deus in nobis sine
nobis operatur}. Quae vero per nos aguntur, Deus in nobis causat non
sine nobis agentibus, ipse enim operatur in omni voluntate et natura.


&

第六異論に対しては以下のように言われるべきである。注入された徳は、私た
ちのうちに、神によって私たちが働くことなく原因されるが、私たちの同意な
しに原因されるわけではない。この意味で、「それを神は私たちのうちに私た
ちなしに働く」と言われることが、理解されるべきである。他方、私たちによっ
て為されることを、神は、私たちが働くことなしには原因しない。じっさい神
はすべての意志と自然の中で働く。


\end{longtable}
\end{document}