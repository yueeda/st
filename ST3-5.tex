\documentclass[10pt]{jsarticle} % use larger type; default would be 10pt
%\usepackage[utf8]{inputenc} % set input encoding (not needed with XeLaTeX)
\usepackage{okumacro}
\usepackage{longtable}
\usepackage[polutonikogreek,english,japanese]{babel}
%\usepackage{amsmath}
\usepackage{latexsym}
\usepackage{color}

%----- header -------
\usepackage{fancyhdr}
\lhead{{\it Summa Theologiae} III, q.~5}
%--------------------

\title{{\bf TERTIA PARS}\\{\HUGE Summae Theologiae}\\Sancti Thomae
Aquinatis\\{\sffamily QUEAESTIO QUINTA}\\DE ASSUMPTIONE PARTIUM HUMANAE NATURAE}
\author{Japanese translation\\by Yoshinori {\sc Ueeda}}
\date{Last modified \today}


%%%% コピペ用
%\rhead{a.~}
%\begin{center}
% {\Large {\bf }}\\
% {\large }\\
% {\footnotesize }\\
% {\Large \}
%\end{center}
%
%\begin{longtable}{p{21em}p{21em}}
%
%&
%
%
%\\\\
%\end{longtable}
%\newpage



\begin{document}
\maketitle
\pagestyle{fancy}

\begin{center}
{\Large 第五問\\人間本性の部分の受容について}
\end{center}

\begin{longtable}{p{21em}p{21em}}


{\Huge D}{\scshape einde} considerandum est de assumptione partium
 humanae naturae. 

Et circa hoc quaeruntur quatuor. 

\begin{enumerate}
 \item utrum filius Dei debuerit assumere verum corpus.
 \item utrum assumere debuerit corpus terrenum, scilicet carnem et
       sanguinem.
 \item utrum assumpserit animam.
 \item utrum assumere debuerit intellectum.
\end{enumerate}


&

次に、人間本性の部分の受容について考察されるべきである。

そして、これについて四つのことが問われる。

\begin{enumerate}
 \item 神の息子は真の身体を受容する必要があったか。
 \item 地上の物体、すなわち、肉と血を受容する必要があったか。
 \item 魂を受容したか。
 \item 知性を受容する必要があったか。
\end{enumerate}


\end{longtable}

\newpage




\rhead{a.~1}
\begin{center}
 {\Large {\bf ARTICULUS PRIMUS}}\\
 {\large UTRUM FILIUS DEI ASSUMPSERIT VERUM CORPUS}\\
 {\footnotesize III {\itshape Sent.}, d.2, q.1, a.3, qu$^a$1, 2; d.4,
 q.2, a.1; IV, d.3, a.3, qu$^a$2, ad 2; IV {\itshape SCG.}, cap.29, 30;
 \\{\itshape Compend.~Theol.}, cap.207; {\itshape Ad Rom.}, cap.8, lect.1;
 II {\itshape ad Cor.}, cap.5, lect.4.}\\
 {\Large 第一項\\神の息子は真の身体を受容したか}
\end{center}

\begin{longtable}{p{21em}p{21em}}

{\Huge A}{\scshape d primum sic proceditur}. Videtur quod filius Dei non assumpserit verum
corpus. Dicitur enim Philipp. II quod in similitudinem hominum factus
est. Sed quod est secundum veritatem, non dicitur esse secundum
similitudinem. Ergo filius Dei non assumpsit verum corpus.


&


\\


[47005] IIIª q. 5 a. 1 arg. 2
Praeterea, assumptio corporis in nullo derogavit dignitati divinitatis,
dicit enim Leo Papa, in sermone de nativitate, quod nec inferiorem
naturam consumpsit glorificatio, nec superiorem minuit assumptio. Sed
hoc ad dignitatem Dei pertinet quod sit omnino a corpore separatus. Ergo
videtur quod per assumptionem non fuerit Deus corpori unitus.


&


\\


[47006] IIIª q. 5 a. 1 arg. 3
Praeterea, signa debent respondere signatis. Sed apparitiones veteris
testamenti, quae fuerunt signa et figurae apparitionis Christi, non
fuerunt secundum corporis veritatem, sed secundum imaginariam visionem,
sicut patet Isaiae VI, vidi dominum sedentem, et cetera. Ergo videtur
quod etiam apparitio filii Dei in mundum non fuerit secundum corporis
veritatem, sed solum secundum imaginationem.


&


\\


[47007] IIIª q. 5 a. 1 s. c.
Sed contra est quod Augustinus dicit, in libro octogintatrium quaest.,
si phantasma fuit corpus Christi, fefellit Christus. Et si fefellit,
veritas non est. Est autem veritas Christus. Ergo non phantasma fuit
corpus eius. Et sic patet quod verum corpus assumpsit.


&


\\


[47008] IIIª q. 5 a. 1 co.
Respondeo dicendum quod, sicut dicitur in libro de ecclesiasticis
dogmatibus, natus est Dei filius non putative, quasi imaginatum corpus
habens, sed corpus verum. Et huius ratio triplex potest
assignari. 

&



\\

Quarum prima est ex ratione humanae naturae, ad quam pertinet
verum corpus habere. Supposito igitur ex praemissis quod conveniens
fuerit filium Dei assumere humanam naturam, consequens est quod verum
corpus assumpserit. 



&



\\

Secunda ratio sumi potest ex his quae in mysterio
incarnationis sunt acta. Si enim non fuit verum corpus eius sed
phantasticum, ergo nec veram mortem sustinuit; nec aliquid eorum quae de
eo Evangelistae narrant, secundum veritatem gessit, sed solum secundum
apparentiam quandam. Et sic etiam sequitur quod non fuit vera salus
hominis subsecuta, oportet enim effectum causae proportionari. 



&



\\

Tertia
ratio potest sumi ex ipsa dignitate personae assumentis, quae cum sit
veritas, non decuit ut in opere eius aliqua fictio esset. Unde et
dominus hunc errorem per seipsum excludere dignatus est, Luc. ult., cum
discipuli, conturbati et conterriti, putabant se spiritum videre, et non
verum corpus, et ideo se eis palpandum praebuit, dicens, palpate et
videte, quia spiritus carnem et ossa non habet, sicut me videtis habere.


&


\\



Ad primum ergo dicendum quod similitudo illa exprimit veritatem humanae
naturae in Christo, per modum quo omnes qui vere in humana natura
existunt, similes specie esse dicuntur. Non autem intelligitur
similitudo phantastica. Ad cuius evidentiam, apostolus subiungit quod
factus est obediens usque ad mortem, mortem autem crucis, quod fieri non
potuisset si fuisset sola similitudo phantastica.


&


\\




Ad secundum dicendum quod per hoc quod filius Dei verum corpus
assumpsit, in nullo est eius dignitas diminuta. Unde Augustinus dicit,
in libro de fide ad Petrum, exinanivit seipsum, formam servi accipiens,
ut fieret servus, sed formae Dei plenitudinem non amisit. Non enim
filius Dei sic assumpsit verum corpus ut forma corporis fieret, quod
repugnat divinae simplicitati et puritati, hoc enim esset assumere
corpus in unitate naturae, quod est impossibile, ut ex supra dictis
patet. Sed, salva distinctione naturae, assumpsit in unitate personae.


&


\\



Ad tertium dicendum quod figura respondere debet quantum ad
similitudinem, non quantum ad rei veritatem, si enim per omnia esset
similitudo, iam non esset signum, sed ipsa res, ut Damascenus dicit, in
III libro. Conveniens igitur fuit ut apparitiones veteris testamenti
essent secundum apparentiam tantum, quasi figurae, apparitio autem filii
Dei in mundo esset secundum corporis veritatem, quasi res figurata sive
signata per illas figuras. Unde apostolus, Coloss. II, quae sunt umbra
futurorum, corpus autem Christi.


&





\end{longtable}
\newpage



\end{document}

%\rhead{a.~}
%\begin{center}
% {\Large {\bf }}\
% {\large }\
% {\footnotesize }\
% {\Large \}
%\end{center}
%
%\begin{longtable}{p{21em}p{21em}}
%
%&
%
%
%\
%\end{longtable}
%\newpage


ARTICULUS 2
[47012] IIIª q. 5 a. 2 arg. 1
Ad secundum sic proceditur. Videtur quod Christus non habuerit corpus carnale, sive terrestre, sed caeleste. Dicit enim apostolus, I Cor. XV, primus homo de terra, terrenus, secundus homo de caelo, caelestis. Sed primus homo, scilicet Adam, fuit de terra quantum ad corpus, ut patet Gen. II. Ergo etiam secundus homo, scilicet Christus, fuit de caelo quantum ad corpus.

[47013] IIIª q. 5 a. 2 arg. 2
Praeterea, I Cor. XV dicitur, caro et sanguis regnum Dei non possidebunt. Sed regnum Dei principaliter est in Christo. Ergo in ipso non est caro et sanguis, sed magis corpus caeleste.

[47014] IIIª q. 5 a. 2 arg. 3
Praeterea, omne quod est optimum est Deo attribuendum. Sed inter omnia corpora corpus nobilissimum est caeleste. Ergo tale corpus debuit Christus assumere.

[47015] IIIª q. 5 a. 2 s. c.
Sed contra est quod dominus dicit Luc. ult., spiritus carnem et ossa non habet, sicut me videtis habere. Caro autem et ossa non sunt ex materia caelestis corporis, sed ex inferioribus elementis. Ergo corpus Christi non fuit corpus caeleste, sed carneum et terrenum.

[47016] IIIª q. 5 a. 2 co.
Respondeo dicendum quod eisdem rationibus apparet quare corpus Christi non debuit esse caeleste, quibus ostensum est quod non debuit esse phantasticum. Primo enim, sicut non salvaretur veritas humanae naturae in Christo si corpus eius esset phantasticum, ut posuit Manichaeus; ita etiam non salvaretur si poneretur caeleste, sicut posuit Valentinus. Cum enim forma hominis sit quaedam res naturalis, requirit determinatam materiam, scilicet carnes et ossa, quae in hominis definitione poni oportet, ut patet per philosophum, in VII Metaphys. Secundo, quia hoc etiam derogaret veritati eorum quae Christus in corpore gessit. Cum enim corpus caeleste sit impassibile et incorruptibile, ut probatur in I de caelo, si filius Dei corpus caeleste assumpsisset, non vere esuriisset nec sitiisset, nec etiam passionem et mortem sustinuisset. Tertio, etiam hoc derogat veritati divinae. Cum enim filius Dei se ostenderet hominibus quasi corpus carneum et terrenum habens, fuisset falsa demonstratio si corpus caeleste habuisset. Et ideo in libro de ecclesiasticis dogmatibus dicitur, natus est filius Dei carnem ex virginis corpore trahens, et non de caelo secum afferens.

[47017] IIIª q. 5 a. 2 ad 1
Ad primum ergo dicendum quod Christus dicitur dupliciter de caelo descendisse. Uno modo, ratione divinae naturae, non ita quod divina natura esse in caelo defecerit; sed quia in infimis novo modo esse coepit, scilicet secundum naturam assumptam; secundum illud Ioan. III, nemo ascendit in caelum nisi qui descendit de caelo, filius hominis, qui est in caelo. Alio modo, ratione corporis, non quia ipsum corpus Christi secundum suam substantiam de caelo descenderit; sed quia virtute caelesti, idest spiritus sancti, est eius corpus formatum. Unde Augustinus dicit, ad Orosium, exponens auctoritatem inductam, caelestem dico Christum, quia non ex humano conceptus est semine. Et hoc etiam modo Hilarius exponit, in libro de Trinitate.

[47018] IIIª q. 5 a. 2 ad 2
Ad secundum dicendum quod caro et sanguis non accipiuntur ibi pro substantia carnis et sanguinis, sed pro corruptione carnis et sanguinis. Quae quidem in Christo non fuit quantum ad culpam. Fuit tamen ad tempus quantum ad poenam, ut opus nostrae redemptionis expleret.

[47019] IIIª q. 5 a. 2 ad 3
Ad tertium dicendum quod hoc ipsum ad maximam Dei gloriam pertinet quod corpus infirmum et terrenum ad tantam sublimitatem provehit. Unde in synodo Ephesina legitur verbum sancti Theophili dicentis, qualiter artificum optimi non pretiosis tantum materiebus artem ostendentes in admiratione sunt, sed, vilissimum lutum et terram dissolutam plerumque assumentes, suae disciplinae demonstrantes virtutem, multo magis laudantur; ita omnium optimus artifex, Dei verbum, non aliquam pretiosam materiam corporis caelestis apprehendens ad nos venit, sed in luto magnitudinem suae artis ostendit.



%\rhead{a.~}
%\begin{center}
% {\Large {\bf }}\
% {\large }\
% {\footnotesize }\
% {\Large \}
%\end{center}
%
%\begin{longtable}{p{21em}p{21em}}
%
%&
%
%
%\
%\end{longtable}
%\newpage


ARTICULUS 3
[47020] IIIª q. 5 a. 3 arg. 1
Ad tertium sic proceditur. Videtur quod filius Dei animam non assumpserit. Ioannes enim, incarnationis mysterium tradens, dixit, verbum caro factum est, nulla facta de anima mentione. Non autem dicitur caro factum eo quod sit in carnem conversum, sed quia carnem assumpsit. Non ergo videtur assumpsisse animam.

[47021] IIIª q. 5 a. 3 arg. 2
Praeterea, anima necessaria est corpori ad hoc quod per eam vivificetur. Sed ad hoc non fuit necessaria corpori Christi, ut videtur, quia ipsum Dei verbum est, de quo in Psalmo, domine, apud te est fons vitae. Superfluum igitur fuisset animam adesse, verbo praesente. Deus autem et natura nihil frustra faciunt, ut etiam philosophus dicit, in I de caelo. Ergo videtur quod filius Dei animam non assumpsit.

[47022] IIIª q. 5 a. 3 arg. 3
Praeterea, ex unione animae ad corpus constituitur natura communis, quae est species humana. In domino autem Iesu Christo non est communem speciem accipere, ut Damascenus dicit, in III libro. Non igitur assumpsit animam.

[47023] IIIª q. 5 a. 3 s. c.
Sed contra est quod Augustinus dicit, in libro de agone Christiano non eos audiamus qui solum corpus humanum dicunt esse susceptum a verbo Dei; et sic audiunt quod dictum est, verbum caro factum est, ut negent illum hominem vel animam, vel aliquid hominis habuisse nisi carnem solam.

[47024] IIIª q. 5 a. 3 co.
Respondeo dicendum quod, sicut Augustinus dicit, in libro de haeresibus, opinio primo fuit Arii, et postea Apollinaris, quod filius Dei solam carnem assumpserit, absque anima, ponentes quod verbum fuerit carni loco animae. Ex quo sequebatur quod in Christo non fuerunt duae naturae, sed una tantum, ex anima enim et carne una natura humana constituitur. Sed haec positio stare non potest, propter tria. Primo quidem, quia repugnat auctoritati Scripturae, in qua dominus de sua anima facit mentionem, Matth. XXVI, tristis est anima mea usque ad mortem; et Ioan. X, potestatem habeo ponendi animam meam. Sed ad hoc respondebat Apollinaris quod in his verbis anima metaphorice sumitur, per quem modum in veteri testamento Dei anima commemoratur, Isaiae I, Calendas vestras et solemnitates odivit anima mea. Sed, sicut dicit Augustinus, in libro octogintatrium quaest., Evangelistae in evangelica narratione narrant quod miratus est Iesus, et iratus, et contristatus, et quod esuriit. Quae quidem ita demonstrant eum veram animam habuisse, sicut ex hoc quod comedit et dormivit et fatigatus est, demonstratur habuisse verum corpus humanum. Alioquin, si et haec ad metaphoram referantur, cum similia legantur in veteri testamento de Deo, peribit fides evangelicae narrationis. Aliud est enim quod prophetice nuntiatur in figuris, aliud quod secundum rerum proprietatem ab Evangelistis historice scribitur. Secundo, derogat praedictus error utilitati incarnationis, quae est liberatio hominis. Ut enim argumentatur Augustinus, in libro contra Felicianum, si, accepta carne, filius Dei animam omisit, aut, innoxiam sciens, medicinae indigentem non credidit; aut, a se alienam putans, redemptionis beneficio non donavit; aut, ex toto insanabilem iudicans, curare nequivit; aut ut vilem, et quae nullis usibus apta videretur, abiecit. Horum duo blasphemiam important in Deum. Quomodo enim dicetur omnipotens, si curare non potuit desperatam? Aut quomodo omnium Deus, si non ipse fecit animam nostram? Duobus vero aliis, in uno animae causa nescitur, in altero meritum non tenetur. Aut intelligere causam putandus est animae qui eam, ad accipiendum legem habitu insitae rationis instructam, a peccato voluntariae transgressionis nititur separare? Aut quomodo eius generositatem novit qui ignobilitatis vitio dicit despectam? Si originem attendas, pretiosior est animae substantia, si transgressionis culpam, propter intelligentiam peior est causa. Ego autem Christum et perfectam sapientiam scio, et piissimam esse non dubito, quorum primo, meliorem et prudentiae capacem non despexit; secundo, eam quae magis fuerat vulnerata, suscepit. Tertio vero, haec positio est contra ipsam incarnationis veritatem. Caro enim et ceterae partes hominis per animam speciem sortiuntur. Unde, recedente anima, non est os aut caro nisi aequivoce, ut patet per philosophum, II de anima et VII Metaphys.

[47025] IIIª q. 5 a. 3 ad 1
Ad primum ergo dicendum quod cum dicitur, verbum caro factum est, caro ponitur pro toto homine, ac si diceret, verbum homo factum est, sicut Isaiae XL dicitur, videbit omnis caro salutare Dei nostri. Ideo autem totus homo per carnem significatur, quia, ut dicitur in auctoritate inducta, quia per carnem filius Dei visibilis apparuit, unde subditur et vidimus gloriam eius. Vel ideo quia, ut Augustinus dicit, in libro octogintatrium quaest. in tota illa unitate susceptionis principale verbum est, extrema autem atque ultima caro. Volens itaque Evangelista commendare pro nobis dilectionem humilitatis Dei, verbum et carnem nominavit, omittens animam, quae est verbo inferior, carne praestantior. Rationabile etiam fuit ut nominaret carnem, quae, propter hoc quod magis distat a verbo, minus assumptibilis videbatur.

[47026] IIIª q. 5 a. 3 ad 2
Ad secundum dicendum quod verbum est fons vitae sicut prima causa vitae effectiva. Sed anima est principium vitae corpori tanquam forma ipsius. Forma autem est effectus agentis. Unde ex praesentia verbi magis concludi posset quod corpus esset animatum, sicut ex praesentia ignis concludi potest quod corpus cui ignis adhaeret, sit calidum.

[47027] IIIª q. 5 a. 3 ad 3
Ad tertium dicendum quod non est inconveniens, immo necessarium dicere quod in Christo fuit natura quae constituitur per animam corpori advenientem. Damascenus autem negat in domino Iesu Christo esse communem speciem quasi aliquid tertium resultans ex unione divinitatis et humanitatis.



%\rhead{a.~}
%\begin{center}
% {\Large {\bf }}\
% {\large }\
% {\footnotesize }\
% {\Large \}
%\end{center}
%
%\begin{longtable}{p{21em}p{21em}}
%
%&
%
%
%\
%\end{longtable}
%\newpage


ARTICULUS 4
[47028] IIIª q. 5 a. 4 arg. 1
Ad quartum sic proceditur. Videtur quod filius Dei non assumpsit mentem humanam, sive intellectum. Ubi enim est praesentia rei, non requiritur eius imago. Sed homo secundum mentem est ad imaginem Dei, ut dicit Augustinus, in libro de Trinit. Cum ergo in Christo fuerit praesentia ipsius divini verbi, non oportuit ibi esse mentem humanam.

[47029] IIIª q. 5 a. 4 arg. 2
Praeterea, maior lux offuscat minorem. Sed verbum Dei, quod est lux illuminans omnem hominem venientem in hunc mundum, ut dicitur Ioan. I, comparatur ad mentem sicut lux maior ad minorem, quia et ipsa mens quaedam lux est, quasi lucerna illuminata a prima luce, Proverb. X, lucerna domini spiraculum hominis. Ergo in Christo, qui est verbum Dei, non fuit necessarium esse mentem humanam.

[47030] IIIª q. 5 a. 4 arg. 3
Praeterea, assumptio humanae naturae a Dei verbo dicitur eius incarnatio. Sed intellectus, sive mens humana, neque est caro neque est actus carnis, quia nullius corporis actus est, ut probatur in III de anima. Ergo videtur quod filius Dei humanam mentem non assumpserit.

[47031] IIIª q. 5 a. 4 s. c.
Sed contra est quod Augustinus dicit, in libro de fide ad Petrum, firmissime tene, et nullatenus dubites, Christum, filium Dei, habentem nostri generis carnem et animam rationalem. Qui de carne sua dicit, palpate et videte, quia spiritus carnem et ossa non habet, sicut me videtis habere, Luc. ult. Animam quoque se ostendit habere, dicens, ego pono animam meam, et iterum sumo eam, Ioan. X. Intellectum quoque se ostendit habere, dicens, discite a me, quia mitis sum et humilis corde, Matth. XI. Et de ipso per prophetam dominus dicit, ecce intelliget puer meus, Isaiae LII.

[47032] IIIª q. 5 a. 4 co.
Respondeo dicendum quod, sicut Augustinus dicit, in libro de haeresibus, Apollinaristae de anima Christi a Catholica Ecclesia dissenserunt, dicentes, sicut Ariani, Christum carnem solam sine anima suscepisse. In qua quaestione testimoniis evangelicis victi, mentem defuisse animae Christi dixerunt, sed pro hac ipsum verbum in ea fuisse. Sed haec positio eisdem rationibus convincitur sicut et praedicta. Primo enim, hoc adversatur narrationi evangelicae, quae commemorat eum fuisse miratum, ut patet Matth. VIII. Admiratio autem absque ratione esse non potest, quia importat collationem effectus ad causam; dum scilicet aliquis videt effectum cuius causam ignorat, et quaerit, ut dicitur in principio Metaphys. Secundo, repugnat utilitati incarnationis, quae est iustificatio hominis a peccato. Anima enim humana non est capax peccati, nec gratiae iustificantis, nisi per mentem. Unde praecipue oportuit mentem humanam assumi. Unde Damascenus dicit, in III libro, quod Dei verbum assumpsit corpus et animam intellectualem et rationalem, et postea subdit, totus toti unitus est, ut toti mihi salutem gratificet idest, gratis faciat, quod enim inassumptibile est, incurabile est. Tertio, hoc repugnat veritati incarnationis. Cum enim corpus proportionetur animae sicut materia propriae formae, non est vera caro humana quae non est perfecta anima humana, scilicet rationali et ideo, si Christus animam sine mente habuisset, non habuisset veram carnem humanam, sed carnem bestialem, quia per solam mentem anima nostra differt ab anima bestiali. Unde dicit Augustinus, in libro octogintatrium quaest., quod secundum hunc errorem sequeretur quod filius Dei beluam quandam cum figura humani corporis suscepisset. Quod iterum repugnat veritati divinae, quae nullam patitur fictionis falsitatem.

[47033] IIIª q. 5 a. 4 ad 1
Ad primum ergo dicendum quod, ubi est ipsa res per sui praesentiam non requiritur eius imago ad hoc quod suppleat locum rei, sicut, ubi erat imperator, milites non venerabantur eius imaginem. Sed tamen requiritur cum praesentia rei imago ipsius ut perficiatur ex ipsa rei praesentia, sicut imago in cera perficitur per impressionem sigilli, et imago hominis resultat in speculo per eius praesentiam. Unde, ad perficiendam humanam mentem, necessarium fuit quod eam sibi verbum Dei univit.

[47034] IIIª q. 5 a. 4 ad 2
Ad secundum dicendum quod lux maior evacuat lucem minorem alterius corporis illuminantis, non tamen evacuat, sed perficit lucem corporis illuminati. Ad praesentiam enim solis stellarum lux obscuratur, sed aeris lumen perficitur. Intellectus autem seu mens hominis est quasi lux illuminata a luce divini verbi. Et ideo per lucem divini verbi non evacuatur mens hominis, sed magis perficitur.

[47035] IIIª q. 5 a. 4 ad 3
Ad tertium dicendum quod, licet potentia intellectiva non sit alicuius corporis actus, ipsa tamen essentia animae humanae, quae est forma corporis, requiritur quod sit nobilior, ad hoc quod habeat potentiam intelligendi. Et ideo necesse est ut corpus melius dispositum ei respondeat.

