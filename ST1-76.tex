\documentclass[paper=a4paper,fontsize=10pt,jafontsize=9pt,titlepage]{jlreq}
\usepackage{pxrubrica} %ルビ、傍点
\usepackage{longtable}
\usepackage[polutonikogreek,english,japanese]{babel}
\usepackage{latexsym}
\usepackage{color}
\usepackage{url}

%----- header -------
\usepackage{fancyhdr}
\pagestyle{fancy}
\lhead{{\itshape Summa Theologiae} I, q.76}
%--------------------


\bibliographystyle{jplain}


\title{{\bfseries Prima Pars}\\{\Huge Summae Theologiae}\\Sancti Thomae
Aquinatis\\{\sffamily Quaestio Septuagesimasexta}\\{\bfseries DE UNIONE ANIMAE AD CORPUS}}
\author{Japanese translation\\by Yoshinori {\scshape Ueeda}}
\date{Last modified \today}

%%%% コピペ用
%\rhead{a.~}
%\begin{center}
% {\Large {\bfseries }}\\
% {\large }\\
% {\footnotesize }\\
% {\Large \\}
%\end{center}
%
%\begin{longtable}{p{21em}p{21em}}
%
%&
%
% 
%
%\\
%\end{longtable}
%\newpage

\begin{document}

\maketitle

\begin{center}
 {\Large 第七十六問\\魂の身体への合一について}
\end{center}

\begin{longtable}{p{21em}p{21em}}
 Deinde considerandum est de unione animae ad corpus. Et circa hoc quaeruntur octo.

 \begin{enumerate}
  \item utrum intellectivum principium uniatur corpori ut forma.
  \item utrum intellectivum principium numero multiplicetur secundum multiplicationem corporum; vel sit unus intellectus omnium hominum.
  \item utrum in corpore cuius forma est principium intellectivum, sit aliqua alia anima.
  \item utrum sit in eo aliqua alia forma substantialis.
  \item quale debeat esse corpus cuius intellectivum principium est forma.
  \item utrum tali corpori uniatur mediante aliquo alio corpore.
  \item utrum mediante aliquo accidente.
  \item utrum anima sit tota in qualibet parte corporis.
 \end{enumerate}

 &

 次に、魂の身体への合一について考察されるべきである。そしてこれをめぐって八つのことが問われる。
 \begin{enumerate}
  \item 知性的根源は身体に形相として合一されるか。
  \item 知性的根源は、身体の複数性に応じて複数化されるか、それともすべての人間に一つの知性が属するか。
  \item その形相が知性的根源である身体の中に、何か別の魂があるか。
  \item その中に、他の実体的形相があるか。
  \item それの知性的根源が形相である身体はどのようなものでなければならないか。
  \item そのような身体に何か別の物体が媒介となって合一されるか。
  \item 何かの附帯性が媒介となって合一されるか。
  \item 魂は身体のどの部分においても全体としてあるか。
 \end{enumerate}

\end{longtable}
\newpage 
 
\rhead{a.~1}
\begin{center}
{\Large {\bfseries ARTICULUS PRIMUS}}\\
{\large UTRUM INTELLECTIVUM PRINCIPIUM UNIATUR CORPORI UT FORMA}\\
{\footnotesize II {\itshape SCG}, c.61, 62, 65, 68 sqq.; {\itshape De Spirit.~Creat.}, a.4; {\itshape Qu.~de Anima}, a.1, 2; {\itshape De Unit.~Intell.}; II {\itshape De Anima}, lect.4, III, lect.7.}\\
{\Large 第一項\\知性的根源は身体に形相として合一されているか}
\end{center}

\begin{longtable}{p{21em}p{21em}}
{\scshape Ad primum sic proceditur}. Videtur quod intellectivum
principium non uniatur corpori ut forma. Dicit enim philosophus, in
III {\itshape de Anima}, quod intellectus est separatus, et quod
nullius corporis est actus. Non ergo unitur corpori ut forma.
 
&

 第一項の問題へ、議論は以下のように進められる。知性的根源は、形相とし
 て身体に合一されているのではないと思われる。理由は以下の通り。哲学者
 は『デ・アニマ』第3巻で、知性は分離していて、どんな身体の現実態でもな
 いと言っている。ゆえに、身体に形相として合一されているのではない。
 
\\


2. {\scshape Praeterea}, omnis forma determinatur secundum naturam
materiae cuius est forma, alioquin non requireretur proportio inter
materiam et formam. Si ergo intellectus uniretur corpori ut forma, cum
omne corpus habeat determinatam naturam, sequeretur quod intellectus
haberet determinatam naturam. Et sic non esset omnium cognoscitivus,
ut ex superioribus patet, quod est contra rationem intellectus. Non
ergo intellectus unitur corpori ut forma.
 
&

 さらに、すべての形相は、それの形相であるとところの質料の本性にしたがっ
 て限定されている。そうでなければ、質料と形相の間の比は要求されなかっ
 たであろう。ゆえに、もし知性が身体に形相として合一されていたならば、
 すべての身体は限定された本性を持つので、知性が限定された本性を持つこ
 とになっただろう。そしてそのようにして、すべてのものを認識しうるもの
 ではなかっただろうが、これは、前述のことから明らかなとおり、知性の性
 格に反する。ゆえに、知性は身体に形相として合一されているのではない。
 
\\

3. {\scshape Praeterea}, quaecumque potentia receptiva est actus
alicuius corporis, recipit formam materialiter et individualiter, quia
receptum est in recipiente secundum modum recipientis. Sed forma rei
intellectae non recipitur in intellectu materialiter et
individualiter, sed magis immaterialiter et universaliter, alioquin
intellectus non esset cognoscitivus immaterialium et universalium, sed
singularium tantum, sicut et sensus. Intellectus ergo non unitur
corpori ut forma.
 
&

さらに、どんな受容能力も何らかの物体の現実態であり、形相を質料的、個別
的に受け取る。なぜなら、受け取られるものは、受け取るものの中に、受け取
るもののあり方に従ってあるからである。しかるに、知性認識された事物の形
相は、質料的、個別的には受け取られず、むしろ非質料的、普遍的に受け取ら
れる。さもなければ、知性は、非質料的で普遍的なものを認識できず、感覚の
ようにただ個物を認識しうるだけだっただろう。ゆえに、知性は、身体に、形
相として合一されているのではない。
 
\\

4. {\scshape Praeterea}, eiusdem est potentia et actio, idem enim est
quod potest agere, et quod agit. Sed actio intellectualis non est
alicuius corporis, ut ex superioribus patet. Ergo nec potentia
intellectiva est alicuius corporis potentia. Sed virtus sive potentia
non potest esse abstractior vel simplicior quam essentia a qua virtus
vel potentia derivatur. Ergo nec substantia intellectus est corporis
forma.
 
&

 さらに、可能態と現実態は同一のものに属する。なぜなら、作用することが
 可能であるものと、現実に作用するものとは同一のものだからである。しか
 るに知性的な作用は、前に述べられたことから明らかなとおり、何らかの身
 体に属さない。しかし、ちからないし能力が、そこからちからないし能力が
 出てくるところの本質よりも、切り離されている、あるいは単純であること
 はできない。ゆえに知性的実体も、身体の形相ではない。
 
\\

5. {\scshape Praeterea}, id quod per se habet esse, non unitur corpori
ut forma, quia forma est quo aliquid est; et sic ipsum esse formae non
est ipsius formae secundum se. Sed intellectivum principium habet
secundum se esse, et est subsistens, ut supra dictum est. Non ergo
unitur corpori ut forma.
 
&

さらに、自体的に存在を持つものは、身体に形相として合一されない。なぜな
ら、形相は、それによって何かが存在するところのものだからである。この意
味で、形相の存在自体は、それ自体で、形相それ自体には属さない。しかし、
知性的根源は、前に述べられたとおり、それ自体に即して存在を持ち、自存す
る。ゆえに、身体に形相として合一されるのではない。
 
\\





6. {\scshape Praeterea}, id quod inest alicui rei secundum se, semper
inest ei. Sed formae secundum se inest uniri materiae, non enim per
accidens aliquod, sed per essentiam suam est actus materiae; alioquin
ex materia et forma non fieret unum substantialiter, sed
accidentaliter. Forma ergo non potest esse sine propria materia. Sed
intellectivum principium, cum sit incorruptibile, ut supra ostensum
est, remanet corpori non unitum, corpore corrupto. Ergo intellectivum
principium non unitur corpori ut forma.

 &

 さらに、自体的に、何かの事物の中にあるものは、常にその中にある。しか
 るに、形相には、自体的に、質料に合一されるということが内在する。なぜ
 なら、それは附帯的にではなく、自らの本質によって、質料の現実態だから
 である。さもなければ、質料と形相から、一が生じるのが、実体的にではな
 く、附帯的にであっただろう。ゆえに、形相は、固有の質料なしには存在で
 きない。しかし知性的根源は、前に示されたとおり不滅なので、身体に合一
 されていなくても留まる。ゆえに知性的根源は、身体に形相として合一され
 ているのではない。
 
\\





 {\scshape Sed contra}, secundum philosophum, in VIII {\itshape
 Metaphys}., differentia sumitur a forma rei. Sed differentia
 constitutiva hominis est rationale; quod dicitur de homine ratione
 intellectivi principii. Intellectivum ergo principium est forma
 hominis.
 
&

 しかし反対に、『形而上学』第7巻の哲学者によれば、種差は事物の形相から
 取られる。しかるに、人間を構成する種差は「理性的なもの」である。これ
 は、知性的根源が根拠となって、人間について語られる。ゆえに、知性的根
 源は、人間の形相である。
 
\\





 {\scshape Respondeo dicendum} quod necesse est dicere quod
 intellectus, qui est intellectualis operationis principium, sit
 humani corporis forma. Illud enim quo primo aliquid operatur, est
 forma eius cui operatio attribuitur, sicut quo primo sanatur corpus,
 est sanitas, et quo primo scit anima, est scientia; unde sanitas est
 forma corporis, et scientia animae.

 
&

 解答する。以下のように言われるべきである。知性、すなわち知的な働きの
 根源が人間の身体の形相であると語ることは必然である。理由は以下の通り。
 あるものが、それによって第一に働くところのものは、働きがそれに帰せら
 れるところのものの形相である。たとえば、身体がそれによって第一に健や
 かになるところのものとは、健康であり、魂がそれによって第一に知るとこ
 ろのものとは知である。このことから、健康は身体の形相であり、知は魂の
 形相である。
 
\\


 Et huius ratio est, quia nihil agit nisi secundum quod est actu, unde
 quo aliquid est actu, eo agit. Manifestum est autem quod primum quo
 corpus vivit, est anima. Et cum vita manifestetur secundum diversas
 operationes in diversis gradibus viventium, id quo primo operamur
 unumquodque horum operum vitae, est anima, anima enim est primum quo
 nutrimur, et sentimus, et movemur secundum locum; et similiter quo
 primo intelligimus.

 
&

そしてこの理由は以下の通りである。何ものも、現実態にある限りにおいてで
なければ作用しない。したがって、あるものがそれによって現実態にあるとこ
ろのものによって、それは作用する。しかるに、身体がそれによって第一に生
きるところのものとは魂である。そして生命は、生物のさまざまな段階におい
てさまざまな働きにおいて明示されるので、これらの生命の業の各々を、私た
ちが、それによって第一になすところのものが魂である。すなわち、魂とは、
私たちがそれによって栄養を取り、感覚し、場所的に動くところのものであり、
また同様に、それによって第一に私たちが知性認識するところのものである。
 
\\



 Hoc ergo principium quo primo intelligimus, sive dicatur intellectus
 sive anima intellectiva, est forma corporis. Et haec est demonstratio
 Aristotelis in II {\itshape de Anima}.


&

 ゆえに、私たちがそれによって第一に知性認識するところのこの根源が、知
 性と言われ、知性的魂と言われるものであり、それが身体の形相である。そ
 してこれは、アリストテレスの『デ・アニマ』第2巻での証明である。
 
\\



 Si quis autem velit dicere animam intellectivam non esse corporis
 formam, oportet quod inveniat modum quo ista actio quae est
 intelligere, sit huius hominis actio, experitur enim unusquisque
 seipsum esse qui intelligit.

 &

しかし、もしだれかが、知性的魂が身体の形相でないと言いたいならば、その
人は、知性認識というこの作用が、どのようなしかたでこの人間の作用である
かの、そのしかたを見つけなければならない。なぜなら、一人一人の人は、自
分が知性認識する者であることを経験しているからである。

 \\

 Attribuitur autem aliqua actio alicui tripliciter, ut patet per
 philosophum, V {\itshape Physic}., dicitur enim movere aliquid aut
 agere vel secundum se totum, sicut medicus sanat; aut secundum
 partem, sicut homo videt per oculum; aut per accidens, sicut dicitur
 quod album aedificat, quia accidit aedificatori esse album.
 
&

 ところで、『自然学』第5巻の哲学者によって明らかなとおり、ある作用が何
 かに帰属させられるのに三通りのしかたがある。すなわち、あるものを動か
 したり、あるものに働きかけると言われるのは、医者が健康にする(治療す
 る)ように自ら全体に即してか、人間が目を通してみる場合のように部分に
 即してか、あるいは、白くあることが建築家に附帯しているから、白いもの
 が建築すると言われる場合のように、附帯的にかである。

 
\\


 Cum igitur dicimus Socratem aut Platonem intelligere, manifestum est
 quod non attribuitur ei per accidens, attribuitur enim ei inquantum
 est homo, quod essentialiter praedicatur de ipso. Aut ergo oportet
 dicere quod Socrates intelligit secundum se totum, sicut Plato
 posuit, dicens hominem esse animam intellectivam, aut oportet dicere
 quod intellectus sit aliqua pars Socratis.
 
&

 ゆえに、ソクラテスやプラトンが知性認識すると私たちが言うとき、それが
 附帯的に帰属させられていないことは明らかである。なぜなら、それは人間
 である限りにおいて、その人に帰属されていて、それは、その人について本
 質的に述語されるからである。ゆえに、ソクラテスは、その人全体に即して
 知性認識すると言うべきか、あるいは、知性はソクラテスの何らかの部分で
 あると言うべきかのどちらかである。前者は人間とは知性的な魂であると言
 うプラトンの考え方である。
 
\\


 Et primum quidem stare non potest, ut supra ostensum est, propter hoc
 quod ipse idem homo est qui percipit se et intelligere et sentire,
 sentire autem non est sine corpore, unde oportet corpus aliquam esse
 hominis partem. Relinquitur ergo quod intellectus quo Socrates
 intelligit, est aliqua pars Socratis ita quod intellectus aliquo modo
 corpori Socratis uniatur.
 
&

そして、前に示されたように、この前者は成り立たない。それは、自分を知覚
する人と、知性認識する人と、感覚する人は、同じ人だが、感覚することは身
体なしにはないので、身体は人間の何らかの部分でなければならないからであ
る。ゆえに、素蔵鉄がそれによって知性認識するところの知性は、ソクラテス
の何らかの部分であり、知性が、何らかのしかたでソクラテスの身体に合一さ
れている。
 
\\



Hanc autem unionem Commentator, in III {\itshape de Anima}, dicit esse
per speciem intelligibilem. Quae quidem habet duplex subiectum, unum
scilicet intellectum possibilem; et aliud ipsa phantasmata quae sunt
in organis corporeis. Et sic per speciem intelligibilem continuatur
intellectus possibilis corpori huius vel illius hominis.
 
&

ところで、注釈家は『デ・アニマ』第3巻で、この合一が、可知的形象によっ
てあると述べている。可知的形象は、たしかに、二通りの基体をもち、一つは
可能知性であり、もう一つは身体の器官の中にある表象像である。そしてこの
ようにして、可知的形象を通して可能知性は、この人間やあの人間の身体に接
続される、というわけである。
 
\\

 Sed ista continuatio vel unio non sufficit ad hoc quod actio
 intellectus sit actio Socratis. Et hoc patet per similitudinem in
 sensu, ex quo Aristoteles procedit ad considerandum ea quae sunt
 intellectus. Sic enim se habent phantasmata ad intellectum, ut
 dicitur in III {\itshape de Anima}, sicut colores ad visum.
 
&

 しかしこの接続ないし合一は、知性の作用がソクラテスの作用であることの
 ために十分ではない。このことは、感覚の類似において明らかである。アリ
 ストテレスは、この感覚から始めて、知性に属する事柄を考察することへと
 進む。つまり、『デ・アニマ』第3巻で言われるように、表象像は知性に対し
 て、ちょうど色が視覚に対するように関係する。
 
\\


 Sicut ergo species colorum sunt in visu, ita species phantasmatum
 sunt in intellectu possibili. Patet autem quod ex hoc quod colores
 sunt in pariete, quorum similitudines sunt in visu, actio visus non
 attribuitur parieti, non enim dicimus quod paries videat, sed magis
 quod videatur. Ex hoc ergo quod species phantasmatum sunt in
 intellectu possibili, non sequitur quod Socrates, in quo sunt
 phantasmata, intelligat; sed quod ipse, vel eius phantasmata
 intelligantur.
 
 &

 ゆえに、色の形象が視覚の中にあるのと同じように、表象像の形象が可能知
 性の中にある。しかるに、それの類似が視覚の中にある色が壁においてある
 ことから、私たちは壁が見るとは言わず、むしろ見られると言う。ゆえに、
 表象像の形象が可能知性の中にあることから、表象がその中にあるソクラテ
 スが知性認識することは帰結せず、むしろソクラテス自身が、あるいはソク
 ラテスの表象像が知性認識されることが帰結する。\footnote{物体の側にも
 認識能力の側にも共通する形象があることから、その物体とその認識能力の
 結合を説明することはできない。色の形象は、壁と視覚の中にあるが、その
 ことから、壁が見ることは帰結しない。同様に、ソクラテスの表象像の形象
 が、ある物体(ソクラテス)とある可能知性の中にあることから、ソクラテ
 スが知性認識することは帰結しない。}

\\

 Quidam autem dicere voluerunt quod intellectus unitur corpori ut
 motor; et sic ex intellectu et corpore fit unum, ut actio intellectus
 toti attribui possit. Sed hoc est multipliciter vanum.
 
&

 他方、ある人々は、知性が身体に動者として合一され、そのようにして、知
 性と身体から、知性の作用が全体に帰属させられうるという意味で、一つの
 ものが生じると言おうとした。しかしこれは多くのしかたで誤っている。
 
\\


 Primo quidem, quia intellectus non movet corpus nisi per appetitum,
 cuius motus praesupponit operationem intellectus. Non ergo quia
 movetur Socrates ab intellectu, ideo intelligit, sed potius e
 converso, quia intelligit, ideo ab intellectu movetur Socrates.
 
&

第一に、知性は欲求によってでなければ身体を動かさないが、欲求の運動は知
性の働きを前提とする。ゆえにソクラテスが知性によって動かされることから、
ソクラテスが知性認識することにはならず、むしろ逆に、ソクラテスが知性認
識するから、ソクラテスは知性によって動かされる。
 
\\


Secundo quia, cum Socrates sit quoddam individuum in natura cuius
essentia est una, composita ex materia et forma; si intellectus non
sit forma eius, sequitur quod sit praeter essentiam eius; et sic
intellectus comparabitur ad totum Socratem sicut motor ad
motum. Intelligere autem est actio quiescens in agente, non autem
transiens in alterum, sicut calefactio. Non ergo intelligere potest
attribui Socrati propter hoc quod est motus ab intellectu.
 
&

第二に、ソクラテスは、質料と形相とから複合された一つの本質を持つ本性にお
ける何らかの個体である。ゆえに、もし知性が彼の形相でなければ、それは彼
の本質の外にあることになる。そして、知性はソクラテス全体に対して、動者
が動かされるものに対するように関係することになるだろう。しかるに、知性
認識することは、作用者の中に安らぐ作用であって、燃焼のように、他のもの
へと越え出ていく作用ではない。ゆえに知性認識は、それが知性による運動で
あることのためにソクラテスに帰属させられることはできない。
 
\\


 Tertio, quia actio motoris nunquam attribuitur moto nisi sicut
 instrumento, sicut actio carpentarii serrae. Si igitur intelligere
 attribuitur Socrati quia est actio motoris eius, sequitur quod
 attribuatur ei sicut instrumento. Quod est contra philosophum, qui
 vult quod intelligere non sit per instrumentum corporeum.

 
&

第三に、動者の作用が動かされるものに帰せられるのは、道具の場合だけであ
る。たとえば、大工の作用が、ノコギリに着せられる場合のように。ゆえに、
知性認識することが、それが動者であるソクラテスの作用だからソクラテスに
帰せられるとすると、それは道具としてのソクラテスに帰せられることになる
だろう。これは、知性認識が身体的な道具によらずにあることを欲する哲学者
に反する。
 
\\


 Quarto quia, licet actio partis attribuatur toti, ut actio oculi
 homini; nunquam tamen attribuitur alii parti, nisi forte per
 accidens, non enim dicimus quod manus videat, propter hoc quod oculus
 videt. Si ergo ex intellectu et Socrate dicto modo fit unum, actio
 intellectus non potest attribui Socrati. 

 
&

第四に、眼の作用が人間に帰せられるように、部分の作用が全体に帰せられる
ことはあるが、しかし、附帯的にでないかぎり、他の部分に帰せられることは
ない。たとえば、眼が見るために、手が見ると私たちは言わないように。ゆえ
に、もし知性とソクラテスから、先に述べられたしかた\footnote{少し前に、
``Quidam autem dicere voluerunt quod intellectus unitur corpori ut
motor; et sic ex intellectu et corpore fit unum, ut actio intellectus
toti attribui possit. Sed hoc est multipliciter vanum.''と言われた箇所
を受けるが、とくに、両者が部分である場合を考えている。}で一つのものが
生じるならば、知性の作用はソクラテスに帰せられることはできない。
 
\\


Si vero Socrates est totum quod componitur ex unione intellectus ad
 reliqua quae sunt Socratis, et tamen intellectus non unitur aliis
 quae sunt Socratis nisi sicut motor; sequitur quod Socrates non sit
 unum simpliciter, et per consequens nec ens simpliciter; sic enim
 aliquid est ens, quomodo et unum.

 &

 しかし他方で、もしソクラテスが、知性の、ソクラテスに属する知性以外の
 残りのものどもへの合一に基づいて複合されている全体であり、しかし知性
 は、ソクラテスに属する他のものどもに、動者としてしか合一されていない
 とすると、ソクラテスは端的な一ではなく、したがって、端的な有でもない
 ことになる。というのも、あるものはそれが有であるしかたで、一でもある
 のだから。


 \\

 
 Relinquitur ergo solus modus quem Aristoteles ponit, quod hic homo
 intelligit, quia principium intellectivum est forma ipsius. Sic ergo
 ex ipsa operatione intellectus apparet quod intellectivum principium
 unitur corpori ut forma.
 
&

ゆえに、アリストテレスが考えたただ一つのしかた、つまり、この人間が知性
認識するのは、知性的根源が、その人の形相だからだということが残される。
ゆえに、この意味で、知性の働きそれ自体に基づいて、知性的根源が形相とし
て身体に合一されていることが明らかである。
 
\\


 Potest etiam idem manifestari ex ratione speciei humanae. Natura enim
 uniuscuiusque rei ex eius operatione ostenditur. Propria autem
 operatio hominis, inquantum est homo, est intelligere, per hanc enim
 omnia animalia transcendit.
 
&

さらに、同じことが、人間の種の性格に基づいても示されうる。つまり、各々
の事物の本性は、それの働きに基づいて明示される。しかるに人間であるかぎ
りにおける人間の固有の働きは知性認識することである。なぜなら、この働き
によって、人間は他のすべての動物を越えるから。

 
\\


 Unde et Aristoteles, in libro {\itshape Ethic}., in hac operatione,
 sicut in propria hominis, ultimam felicitatem constituit. Oportet
 ergo quod homo secundum illud speciem sortiatur, quod est huius
 operationis principium. Sortitur autem unumquodque speciem per
 propriam formam. Relinquitur ergo quod intellectivum principium sit
 propria hominis forma.
 
&

このことから、アリストテレスも、『ニコマコス倫理学』という書物の中で、
人間の固有な働きとしてのこの働きの中に、究極の幸福を打ち立てた。ゆえに、
人間は、その種の中に分類される限りにおいて、この働きの根源である。しか
るに各々の種は、固有の形相によって種別化される。ゆえに、知性的根源が人
間の固有の形相であることが残される。
 
\\


 Sed considerandum est quod, quanto forma est nobilior, tanto magis
 dominatur materiae corporali, et minus ei immergitur, et magis sua
 operatione vel virtute excedit eam. Unde videmus quod forma mixti
 corporis habet aliquam operationem quae non causatur ex qualitatibus
 elementaribus.
 
&

しかし、以下のことが考えられるべきである。すなわち、形相がより高貴であ
るほど、物体的質料を支配し、それに浸されることが少なく、その働きやちか
らにおいて、質料をより超える。したがって、身体に混合した形相は、元素の
性質に原因された働きをもたないのを私たちは見る。
 
\\


 Et quanto magis proceditur in nobilitate formarum, tanto magis
 invenitur virtus formae materiam elementarem excedere, sicut anima
 vegetabilis plus quam forma metalli, et anima sensibilis plus quam
 anima vegetabilis. Anima autem humana est ultima in nobilitate
 formarum.
 
&

そして、形相の高貴さにおいてより進むほど、形相のちからが元素的な質料を
越えるのが見出される。たとえば、植物的魂は、鉱物的形相よりも、感覚的魂
は植物的魂よりも越えている。しかるに人間の魂は、形相の高貴さにおいて最
終のものである。\footnote{アリストテレス的な文脈、つまりこの世界の中で、
ということ。天使と神は除外している。}
 
\\


 Unde intantum sua virtute excedit materiam corporalem, quod habet
 aliquam operationem et virtutem in qua nullo modo communicat materia
 corporalis. Et haec virtus dicitur intellectus.
 
&

 したがって、自らのちからにおいて物体的質料を越えるほど、物体的質料と
 どんなしかたにおいても共有しない何らかの働きやちからを持つが、このち
 からが知性と言われる。
 
\\


 Est autem attendendum quod, si quis poneret animam componi ex materia
 et forma, nullo modo posset dicere animam esse formam corporis. Cum
 enim forma sit actus, materia vero sit ens in potentia tantum; nullo
 modo id quod est ex materia et forma compositum, potest esse alterius
 forma secundum se totum.

 
&

さらに次のことが注意されるべきである。もしある人が、魂が質料と形相から
複合されていると考えたならば、決して、魂が身体の形相だとは言えなかった
であろう。なぜなら、形相は現実態であり、質料はたんに可能態における有で
あるから、質料と形相から複合されたものが、それ自体の全体に即して他のも
のの形相であることはできなかっただろうから。
 
 
\\


 Si autem secundum aliquid sui sit forma, id quod est forma dicimus
 animam, et id cuius est forma dicimus primum animatum, ut supra
 dictum est.
 
&

 さらにまた、もし自らのある点に即して形相であるならば、その形相である
 ものを、私たちは形相と呼ぶ。そして、前に述べたとおり、それの形相であ
 るところのそれを、第一に魂を持つものと呼ぶ。

 
\\

 {\scshape Ad primum ergo dicendum} quod, sicut philosophus dicit in
 II {\itshape Physic}., ultima formarum naturalium, ad quam terminatur
 consideratio philosophi naturalis, scilicet anima humana, est quidem
 separata, sed tamen in materia; quod ex hoc probat, quia {\itshape
 homo ex materia generat hominem, et sol}.

 &

 第一異論に対しては、それゆえ、以下のように言われるべきである。哲学者
 が『自然学』第2巻で述べるように、自然哲学がそれへと終局する自然的形相
 の究極、すなわち人間の魂は、たしかに分離したものではあるが、しかしや
 はり質料においてある。彼はこのことを、「人間も太陽も、質料から人間を
 生む」ということから証明する。

 \\


 Separata quidem est secundum virtutem intellectivam, quia virtus
 intellectiva non est virtus alicuius organi corporalis, sicut virtus
 visiva est actus oculi: intelligere enim est actus qui non potest
 exerceri per organum corporale, sicut exercetur visio.

 &

 人間の魂は、たしかに知性的なちからにおいては分離している。というのも、
 知性的なちからは、たとえば見るちからが眼の作用であるというかたちで、
 何らかの身体的器官のちからではないからである。実際、知性認識は、視覚
 が働かされるのとは違って、身体的器官によって遂行されることができない
 作用である。

 \\

 Sed in materia
 est inquantum ipsa anima cuius est haec virtus, est corporis forma,
 et terminus generationis humanae. Sic ergo philosophus dicit in III
 {\itshape de Anima} quod intellectus est {\itshape separatus}, quia non est
 virtus alicuius organi corporalis.
 
&

 しかし、人間の魂は、このちからがそれに属するところの魂が身体の形相で
 あり人間の生成の終局である限りにおいて、質料の中にある。ゆえに哲学者
 は『デ・アニマ』第3巻で、知性は、何らの身体的器官のちからでもないがゆ
 えに「分離している」と言う。
  
\\

 Et per hoc patet responsio ad secundum et tertium. Sufficit enim ad
 hoc quod homo possit intelligere omnia per intellectum, et ad hoc
 quod intellectus intelligat immaterialia et universalia, quod virtus
 intellectiva non est corporis actus.

 
&

 そしてこのことから、第二異論と第三異論への解答は明らかである。すなわ
 ち、人間が知性によってすべてを知性認識できることと、知性が非質料的な
 ものや普遍的なものを認識することのためには、知性的なちからが身体の現
 実体でないことで十分である。
 
\\


{\scshape Ad quartum dicendum} quod humana anima non est forma in
materia corporali immersa, vel ab ea totaliter comprehensa, propter
suam perfectionem. Et ideo nihil prohibet aliquam eius virtutem non
esse corporis actum; quamvis anima secundum suam essentiam sit
corporis forma.
 
&

 第四異論に対しては、以下のように言われるべきである。人間の魂は、その
 完全性のため、物体的質料に浸された、あるいは、全体的に包含されたもの
 ではない。ゆえに、それの何らかのちからが身体の現実態でないことを妨げ
 るものはない。ただし、魂は、その本質に即しては身体の形相なのだけれど
 も。

 
\\




{\scshape Ad quintum dicendum} quod anima illud esse in quo ipsa subsistit,
 communicat materiae corporali, ex qua et anima intellectiva fit unum,
 ita quod illud esse quod est totius compositi, est etiam ipsius
 animae. Quod non accidit in aliis formis, quae non sunt
 subsistentes. Et propter hoc anima humana remanet in suo esse,
 destructo corpore, non autem aliae formae.
 
&

 第五異論に対しては、以下のように言われるべきである。魂は、それがそこ
 において自存するところの存在を、物体的質料に伝達し、その質料と知性的
 魂とから一が生じる。したがって、複合体全体の存在は、魂自身の存在でも
 ある。このことは、自存するものでない他の形相においては生じない。この
 ため、人間の魂は身体が破壊されても自らの存在において留まるが、他の形
 相はそうでない。
 
\\


 {\scshape Ad sextum dicendum} quod secundum se convenit animae
 corpori uniri, sicut secundum se convenit corpori levi esse
 sursum. Et sicut corpus leve manet quidem leve cum a loco proprio
 fuerit separatum, cum aptitudine tamen et inclinatione ad proprium
 locum; ita anima humana manet in suo esse cum fuerit a corpore
 separata, habens aptitudinem et inclinationem naturalem ad corporis
 unionem.
 
&

 第六異論に対しては、以下のように言われるべきである。ちょうど軽い物体
 には上方にあることが自体的に適合するように、魂には身体に合一されるこ
 とが自体的に適合する。また、軽い物体が、固有の場所から離されたとして
 も、固有の場所への適合性や傾向性を持ちながら、軽いものに留まるように、
 人間の魂も、身体から分離したとしても、身体との合一への適合性と傾向性
 を持ちながら、自らの存在において留まる。

\end{longtable}
\newpage


\rhead{a.~2}
\begin{center}
{\Large {\bfseries ARTICULUS SECUNDUS}}\\
{\large UTRUM INTELLECTIVUM PRINCIPIUM MULTIPLICETUR SECUNDUM MULTIPLICATIONEM CORPORUM}\\
{\footnotesize I {\itshape Sent}., d.8, q.5, a.2, ad 6; II, d.17, q.2, a.1; II {\itshape SCG}, cap.73, 75; {\itshape De Spirit.~Creat}, a.9; Qu.~{\itshape de Anima}, a.3; {\itshape Compend.~Theol.}, cap.85; {\itshape De Unitat.~Intell.} per tot.}\\
{\Large 第二項\\知性的根源は身体の多数性に即して多数化されるか}
\end{center}

\begin{longtable}{p{21em}p{21em}}
{\scshape Ad secundum sic proceditur}. Videtur quod intellectivum
principium non multiplicetur secundum multiplicationem corporum, sed
sit unus intellectus in omnibus hominibus. Nulla enim substantia
immaterialis multiplicatur secundum numerum in una specie. Anima autem
humana est substantia immaterialis, non enim est composita ex materia
et forma, ut supra\footnote{q.75, a.5.} ostensum est. Non ergo sunt multae in una
specie. Sed omnes homines sunt unius speciei. Est ergo unus
intellectus omnium hominum.

&

 第二項の問題へ、議論は以下のように進められる。知性的根源は、身体の多
 数性に即して多数化されず、すべての人間に一つの知性があると思われる。
 理由は以下の通り。どんな非質料的な実体も、一つの種において、数的に多
 数化されない。しかるに人間の魂は、前に示されたとおり、質料と形相から
 複合されていないので、非質料的な実体である。ゆえに、一つの種において
 多数ではない。しかるにすべての人間は一つの種に属する。ゆえにすべての
 人間に属する一つの知性がある。
 
 
\\

2. {\scshape Praeterea}, remota causa, removetur effectus. Si ergo
secundum multiplicationem corporum multiplicarentur animae humanae,
consequens videretur quod, remotis corporibus, multitudo animarum non
remaneret, sed ex omnibus animabus remaneret aliquod unum solum. Quod
est haereticum, periret enim differentia praemiorum et poenarum.

&

さらに、原因が取り去られると、結果も取り去られる。ゆえに、もし身体の多
数性に即して人間の魂が多数化されるならば、身体が取り去られると、魂の多
数性は留まらず、すべての魂からなにか一つだけのものが留まるように思われ
たであろう。これは、賞罰の差異がなくなるので、異端である。
 
\\



3. {\scshape Praeterea}, si intellectus meus est alius ab intellectu
tuo, intellectus meus est quoddam individuum, et similiter intellectus
tuus, particularia enim sunt quae differunt numero et conveniunt in
una specie. Sed omne quod recipitur in aliquo, est in eo per modum
recipientis. Ergo species rerum in intellectu meo et tuo reciperentur
individualiter, quod est contra rationem intellectus, qui est
cognoscitivus universalium.

&

さらに、もし私の知性があなたの知性と別のものであるならば、私の知性はあ
る種の個体であり、あなたの知性もまたそうである。なぜなら、それらは数に
おいて異なり、一つの種において一致する個体だからである。しかるに、すべ
てあるものにおいて受け取られるものは、受け取るもののあり方によってその
ものの中にある。ゆえに、諸事物の形象は、私の知性とあなたの知性の中に、
個的なかたちで受け取られたであろうが、これは、普遍的なものを認識しうる
知性の性格に反する。
 
\\



4. {\scshape Praeterea}, intellectum est in intellectu
intelligente. Si ergo intellectus meus est alius ab intellectu tuo,
oportet quod aliud sit intellectum a me, et aliud sit intellectum a
te. Et ita erit individualiter numeratum, et intellectum in potentia
tantum, et oportebit abstrahere intentionem communem ab utroque, quia
a quibuslibet diversis contingit abstrahere aliquod commune
intelligibile. Quod est contra rationem intellectus, quia sic non
videretur distingui intellectus a virtute imaginativa. Videtur ergo
relinqui quod sit unus intellectus omnium hominum.

&

さらに、知性認識されたものは、知性認識する人の知性の中にある。ゆえに、
もし私の知性があなたの知性と別のものであるならば、私によって知性認識さ
れたものと、あなたによって知性認識されたものとは別のものでなければなら
ない。そうして、それは個的に数えられたものであり、可能態における知性認
識されたものであり、両者から共通の概念を抽象する必要があるだろう。なぜ
なら、どんな異なったものからも、何か共通の可知的なものを抽象することが
ありうるからである。これは、知性の性格に反する。なぜなら、もしこのよう
であるならば、知性は表象力と区別されないと思われるからである。ゆえに、
すべての人に一つの知性があるということが残される。
 
\\



5. {\scshape Praeterea}, cum discipulus accipit scientiam a magistro,
non potest dici quod scientia magistri generet scientiam in discipulo,
quia sic etiam scientia esset forma activa, sicut calor; quod patet
esse falsum. Videtur ergo quod eadem numero scientia quae est in
magistro, communicetur discipulo. Quod esse non potest, nisi sit unus
intellectus utriusque. Videtur ergo quod sit unus intellectus
discipuli et magistri; et per consequens omnium hominum.

&

さらに、生徒が教師から知を受け取るとき、教師の知が生徒の中に知を生み出
すとは言われ得ない。なぜなら、その場合には知もまた熱のような能動的形相
であることになるが、これは偽であることが明らかだからである。ゆえに、教
師の中にある数的に同一の知が生徒に伝達されると思われる。このことは、も
し、一人に一つずつ知性があるならありえなかったであろう。ゆえに、生徒と
教師の知性は一つであると思われる。そして結果的に、すべての人間の知性は
一つであると思われる。

 
\\



6. {\scshape Praeterea}, Augustinus dicit, in libro {\itshape de
Quantitate Animae}, {\itshape Si plures tantum animas humanas dixerim,
ipse me ridebo}. Sed maxime videtur anima esse una quantum ad
intellectum. Ergo est unus intellectus omnium hominum.

&

さらに、アウグスティヌスは『魂の大きさについて』という書物の中で、「も
し私が複数の人間の魂だけを語ったとしたならば、私は自分を笑うだろう」と
述べている。しかるに、魂は知性にかんして最大限に一つであると思われる。
ゆえに、すべての人に一つの魂があると思われる。
 
\\

{\scshape Sed contra} est quod philosophus dicit, in II {\itshape
Physic}., quod sicut se habent causae universales ad universalia, ita
se habent causae particulares ad particularia. Sed impossibile est
quod una anima secundum speciem, sit diversorum animalium secundum
speciem. Ergo impossibile est quod anima intellectiva una numero, sit
diversorum secundum numerum.

&

 しかし反対に、哲学者は『自然学』第2巻で、普遍的な諸原因が普遍的なもの
 に関係するように、個別的な原因は個別的なものに関係すると述べている。
 しかるに、種において一つである魂が、種において異なる魂に属することは
 不可能である。ゆえに、数的に一つの知性的魂が、数的に多数のものに属す
 ることは不可能である。

 
\\


{\scshape Respondeo dicendum} quod intellectum esse unum omnium
 hominum, omnino est impossibile. Et hoc quidem patet, si, secundum
 Platonis sententiam homo sit ipse intellectus. Sequeretur enim, si
 Socratis et Platonis est unus intellectus tantum, quod Socrates et
 Plato sint unus homo; et quod non distinguantur ab invicem nisi per
 hoc quod est extra essentiam utriusque. Et erit tunc distinctio
 Socratis et Platonis non alia quam hominis tunicati et cappati, quod
 est omnino absurdum.

&

 解答する。以下のように言われるべきである。知性がすべての人に一つであ
 ることは、あらゆる意味で不可能である。確かにこのことは、もしプラトン
 の意見に従って、人間が知性それ自体だとすれば明らかである。つまり、も
 しソクラテスとプラトンが、ただ一つの知性ならば、ソクラテスとプラトン
 は一人の人間であることになっただろうし、そして、双方の本質の外にある
 ものによってしか相互に区別されないことになっただろう。そしてその場合
 には、ソクラテスとプラトンの区別は、チュニックを着た人とコートを着た
 人の区別と同じであることになるが、これはまったく馬鹿げたことである。
 
\\



 Similiter etiam patet hoc esse impossibile, si, secundum sententiam
 Aristotelis, inellectus ponatur pars, seu potentia, animae quae est
 hominis forma. Impossibile est enim plurium numero diversorum esse
 unam formam, sicut impossibile est quod eorum sit unum esse, nam
 forma est essendi principium.

&

 同様にまた、アリストテレスの意見に従って、知性が、人間の形相である魂
 の部分、あるいは能力であると考えられる場合も、これが不可能であること
 は明らかである。すなわち、数において複数である様々なものに、一つの形
 相が属することは不可能である。それはちょうど、それらに一つの存在が属
 することが不可能であるのと同じである。なぜなら、形相は存在の根源であ
 るから。


 \\


 Similiter etiam patet hoc esse impossibile quocumque modo quis ponat
 unionem intellectus ad hunc et ad illum hominem. Manifestum est enim
 quod, si sit unum principale agens et duo instrumenta, dici poterit
 unum agens simpliciter, sed plures actiones, sicut si unus homo
 tangat diversa duabus manibus, erit unus tangens, sed duo tactus.

&

 さらに同様に、どのようなしかたであれ、人が知性のこの人とあの人への合
 一を考えることが不可能であることは明らかである。つまり、もし主要な作
 用者が一つで、道具が二つあるならば、端的に一つの作用者が語られること
 が可能だが、その場合には複数の作用がある。ちょうど、もし一人の人間が、
 二つの手で異なるものに触れるならば、触れる人は一人だが、触ることは二
 つであるように。
\\


 Si vero e converso instrumentum sit unum et principales agentes
 diversi, dicentur quidem plures agentes, sed una actio, sicut si
 multi uno fune trahant navem, erunt multi trahentes, sed unus
 tractus. Si vero agens principale sit unum et instrumentum unum,
 dicetur unum agens et una actio, sicut cum faber uno martello
 percutit, est unus percutiens et una percussio.

&

逆に、もし道具が一つで、主要な作用者が様々である場合には、複数の作用者
と言われるが、しかし作用は一つである。ちょうど、多くの人が一つの綱で船
を引く場合、引く人は多いが、引くことは一つであるように。他方また、主要
な作用者が一つで、道具も一つでる場合、一つの作用者と一つの作用と言われ
るだろう。ちょうど、大工が一つの金槌で打つとき、打つ人は一人であり、打
つことも一つであるように。
 
\\

 Manifestum est autem quod, qualitercumque intellectus seu uniatur seu
 copuletur huic vel illi homini, intellectus inter cetera quae ad
 hominem pertinent, principalitatem habet, obediunt enim vires
 sensitivae intellectui, et ei deserviunt. Si ergo poneretur quod
 essent plures intellectus et sensus unus duorum hominum, puta si duo
 homines haberent unum oculum; essent quidem plures videntes, sed una
 visio.

&

 ところで、どのようなしかたで知性がこの人やあの人に合一されたり結び付
 けられたりするとしても、知性は、その人間に属するものの中で、主要性を
 もっていることは明らかである。なぜなら、感覚的な諸力は知性に従い、知
 性に奉仕するからである。ゆえに、もし、たとえば二人の人間が一つの眼を
 持つようにして、二人の人間に複数の知性と一つの感覚が属したとしたら、
 複数の見る人がいるが、見ることは一つである。

 \\

 Si vero intellectus est unus, quantumcumque diversificentur alia
 quibus omnibus intellectus utitur quasi instrumentis, nullo modo
 Socrates et Plato poterunt dici nisi unus intelligens. Et si addamus
 quod ipsum intelligere, quod est actio intellectus, non fit per
 aliquod aliud organum, nisi per ipsum intellectum; sequetur ulterius
 quod sit et agens unum et actio una; idest quod omnes homines sint
 unus intelligens, et unum intelligere; dico autem respectu eiusdem
 intelligibilis.

&

またもし、知性が一つであり、知性が道具として用いるすべての他のものがど
れほど多様化されようとも、決してソクラテスとプラトンは、一人の知性認識
者としか言われえないであろう。そしてもし、知性の作用である知性認識作用
は、知性それ自体以外のどんな他の期間によっても生じないことを私たちが加
えるならば、さらに、一つの作用者でありまた一つの作用であることが帰結す
るだろう。これはすなわち、すべての人間が一人の知性認識者であり一つの知
性認識作用であるということである。ただし私は一つの可知的なものにかんし
て語っている。
 
\\



 Posset autem diversificari actio intellectualis mea et tua per
 diversitatem phantasmatum, quia scilicet aliud est phantasma lapidis
 in me et aliud in te, si ipsum phantasma, secundum quod est aliud in
 me et aliud in te, esset forma intellectus possibilis, quia idem
 agens secundum diversas formas producit diversas actiones, sicut
 secundum diversas formas rerum respectu eiusdem oculi sunt diversae
 visiones.

 &

 また、もし表象像それ自体が、私とあなたの中にあるかぎりで、可能知性の
 形相だとしたならば、私とあなたの知性的作用は、表象像の多様性によって
 多様化されたであろう。なぜなら、私とあなたの中にある石の表象像は別の
 ものだから。というのも、同一の作用者は、異なる形相に従って、異なる作
 用を生み出すからである。たとえば、諸事物の異なる形相に従って、同一の
 眼に関して、異なる見ることがあるように。

\\

 Sed ipsum phantasma non est forma intellectus possibilis, sed species
 intelligibilis quae a phantasmatibus abstrahitur. In uno autem
 intellectu a phantasmatibus diversis eiusdem speciei non abstrahitur
 nisi una species intelligibilis. Sicut in uno homine apparet, in quo
 possunt esse diversa phantasmata lapidis, et tamen ab omnibus eis
 abstrahitur una species intelligibilis lapidis, per quam intellectus
 unius hominis operatione una intelligit naturam lapidis, non obstante
 diversitate phantasmatum.

&

 しかし、表象像が可能知性の形相なのではなく、表象像から抽象された可知
 的形象がそのような形相である。しかるに、一つの知性の中で、一つの種に
 属するさまざまな表象像からは、一つの可知的形象しか抽象されない。たと
 えば、一人の人間の中には石のさまざまな表象像が存在しうるが、しかし、
 それらすべてから、一つの石の可知的形象が抽象され、それによって一人の
 人間の知性が、一つの働きによって、表象像の多様性がそれを邪魔すること
 なく、石の本性を知性認識することが明らかである。
 
\\


 Si ergo unus intellectus esset omnium hominum, diversitas
 phantasmatum quae sunt in hoc et in illo, non posset causare
 diversitatem intellectualis operationis huius et illius hominis, ut
 Commentator fingit in III {\itshape de Anima}. Relinquitur ergo quod
 omnino impossibile et inconveniens est ponere unum intellectum omnium
 hominum.

&

ゆえに、注釈家が『デ・アニマ』第3巻で考えているように、かりに、すべて
の人間に属する一つの知性があったならば、この知性やあの知性の中にある表
象像の多様性は、この人やあの人の知性的な働きの多様性を原因することがで
きなかったであろう。ゆえに、残されるのは、すべての人間に一つの知性があ
るとすることは、あらゆる意味で不可能で不適切だということである。
 
\\


{\scshape Ad primum ergo dicendum} quod, licet anima intellectiva non
habeat materiam ex qua sit, sicut nec Angelus, tamen est forma
materiae alicuius; quod Angelo non convenit. Et ideo secundum
divisionem materiae sunt multae animae unius speciei, multi autem
Angeli unius speciei omnino esse non possunt.

&

 第一異論に対しては、それゆえ、以下のように言われるべきである。知性的
 魂は、天使と同じように、それに基づいてあるような質料をもたないが、何
 らかの質料の形相であり、このことは天使と一致しない。ゆえに、質料の分
 割において、一つの種に属する多くの魂があるが、一つの種に属する多くの
 天使というのは、あらゆる意味で不可能である。
 
\\



{\scshape Ad secundum dicendum} quod unumquodque hoc modo habet
unitatem, quo habet esse, et per consequens idem est iudicium de
multiplicatione rei, et de esse ipsius. Manifestum est autem quod
anima intellectualis, secundum suum esse, unitur corpori ut forma; et
tamen, destructo corpore, remanet anima intellectualis in suo esse. Et
eadem ratione multitudo animarum est secundum multitudinem corporum;
et tamen, destructis corporibus, remanent animae in suo esse
multiplicatae.

&

 第二異論に対しては、以下のように言われるべきである。各々のものは、存
 在を持つしかたによって、一性をもつ。その結果、事物の多数性についての
 判断とその事物の存在についての判断は同一である。ところで、知性的魂が、
 自らの存在に即して、形相として身体に合一されていることは明らかである。
 しかし、身体が壊れても、知性的魂は自らの存在において留まる。同じ理由
 で、魂の多数性は、身体の多数性においてあるが、しかし、それらの身体が
 壊れても、諸々の魂は、多数化された自らの存在において留まる。

 
\\



 {\scshape Ad tertium dicendum} quod individuatio intelligentis, aut
 speciei per quam intelligit, non excludit intelligentiam
 universalium, alioquin, cum intellectus separati sint quaedam
 substantiae subsistentes, et per consequens particulares, non possent
 universalia intelligere.

&

第三異論に対しては、以下のように言われるべきである。知性認識者や、それ
によって知性認識するところの形象の個体化は、普遍の知解を排除しない。そ
うでないならば、諸々の離在知性は一種の自存実体であり、結果的に個別的な
ものなので、普遍を知性認識できなかったであろう。
 
\\


 Sed materialitas cognoscentis et speciei per quam cognoscitur,
 universalis cognitionem impedit. Sicut enim omnis actio est secundum
 modum formae qua agens agit, ut calefactio secundum modum caloris;
 ita cognitio est secundum modum speciei qua cognoscens cognoscit.

&

しかし、認識者やそれによって認識されるところの形象の質料性は、普遍の認
識を妨げる。というのも、ちょうどすべての作用が、作用者がそれによって作
用するところの形相のあり方に従うように、たとえば加熱作用は熱のあり方に
従うように、認識は、それによって認識者が認識するところの形象のあり方に
従うからである。
 
\\

 Manifestum est autem quod natura communis distinguitur et
 multiplicatur secundum principia individuantia, quae sunt ex parte
 materiae. Si ergo forma per quam fit cognitio, sit materialis, non
 abstracta a conditionibus materiae, erit similitudo naturae speciei
 aut generis, secundum quod est distincta et multiplicata per
 principia individuantia, et ita non poterit cognosci natura rei in
 sua communitate.

&

しかるに、共通本性は個体化する根源に従って区別され多数化されるが、その
ような根源は質料の側からあることは明らかである。ゆえに、それによって認
識が生じるところの形相が質料的であり、質料の条件から切り離されていない
ならば、その形相は、それに従って個体化する根源によって区別され多数化さ
れる種や類の本性の類似であり、したがって事物の本性はその共通性において
認識されることができなかったであろう。
 
\\


 Si vero species sit abstracta a conditionibus materiae individualis,
 erit similitudo naturae absque iis quae ipsam distinguunt et
 multiplicant, et ita cognoscetur universale. Nec refert, quantum ad
 hoc, utrum sit unus intellectus vel plures, quia si etiam esset unus
 tantum, oporteret ipsum esse aliquem quendam, et speciem per quam
 intelligit esse aliquam quandam.

&

他方、もし形象が個的な質料の条件から切り離されていれば、それは自らを区
別し多数化するものどもがない本性の類似であり、したがって、普遍が認識さ
れるだろう。このことに関しては、一つの知性か複数の知性かということは関
係ない。なぜなら、かりに知性がただ一つだったとしても、その知性は何か或
るものであり、それによって知性認識するところの形象も、何か或るものでな
ければならないのだから。
 
\\



 {\scshape Ad quartum dicendum} quod, sive intellectus sit unus sive
 plures, id quod intelligitur est unum. Id enim quod intelligitur non
 est in intellectu secundum se, sed secundum suam similitudinem, lapis
 enim non est in anima, sed species lapidis, ut dicitur in III
 {\itshape de Anima}.

&

 第四異論に対しては、以下のように言われるべきである。知性が一つであれ
 複数であれ、知性認識されるものは一つである。つまり、知性認識されるも
 のは知性の中に自らに即してではなく、自らの類似に即してある。たとえば
 石が魂の中にあるのではなく、石の形象が魂の中にある。これは『デ・アニ
 マ』第3巻で言われているとおりである。
  
\\


 Et tamen lapis est id quod intelligitur, non autem species lapidis,
 nisi per reflexionem intellectus supra seipsum, alioquin scientiae
 non essent de rebus, sed de speciebus intelligibilibus.

 &

 しかし、知性認識されるものは石であって、知性の自分自身への反省による
 のでない限り、石の形象ではない。さもなければ、知は事物についてではな
 く、可知的形象についてのものだっただろう。

 \\

 Contingit autem eidem rei diversa secundum diversas formas
 assimilari.  Et quia cognitio fit secundum assimilationem
 cognoscentis ad rem cognitam, sequitur quod idem a diversis
 cognoscentibus cognosci contingit, ut patet in sensu, nam plures
 vident eundem colorem, secundum diversas similitudines. Et similiter
 plures intellectus intelligunt unam rem intellectam.

&

ところで、さまざまなものが、さまざまな形相に即して、同一の事物に類似化
されることがある。そして認識は認識するものの認識される事物への類似化に
即して生じるので、同一のものが多様な認識者によって認識されるということ
が生じる。これは感覚において明らかであり、複数の人が同一の色を異なる類
似に即して見る。同様に、複数の知性が一つの知性認識された事物を知性認識
する。
 
\\


 Sed hoc tantum interest inter sensum et intellectum, secundum
 sententiam Aristotelis, quod res sentitur secundum illam
 dispositionem quam extra animam habet, in sua particularitate, natura
 autem rei quae intelligitur, est quidem extra animam, sed non habet
 illum modum essendi extra animam, secundum quem intelligitur.

&

しかし、アリストテレスの文章によれば、感覚と知性は次の点だけが異なる。
すなわち、事物は、それが魂の外に持つ状態に即して、その個別性において感
覚されるが、知性認識される事物の本性は、確かに魂の外にあるが、それに従っ
 て知性認識されるようなあり方を、魂の外では持っていない。
 
 \\
 
 Intelligitur enim natura communis seclusis principiis
 individuantibus; non autem hunc modum essendi habet extra animam. Sed
 secundum sententiam Platonis, res intellecta eo modo est extra animam
 quo intelligitur, posuit enim naturas rerum a materia separatas.

&

つまり、共通の本性が、個体化する諸根源から切り離されて知性認識されるが、
このあり方を、魂の外では持っていない。しかし、プラトンの表現によれば、
知性認識された事物は、魂の外でも、それによって知性認識されるのと同じあ
り方である。つまり彼は、諸事物の本性が質料から切り離されていると考えた。
 
\\

{\scshape Ad quintum dicendum} quod scientia alia est in discipulo, et
alia in magistro. Quomodo autem causetur, in sequentibus ostendetur.

&

第五異論に対しては、以下のように言われるべきである。生徒と教師の中にあ
る知は別々のものである。それがどのように原因されるかは、後続する箇所で
明らかにされるだろう。
 
\\

{\scshape Ad sextum dicendum} quod Augustinus intelligit animas non
esse plures tantum, quin uniantur in una ratione speciei.

&

第六異論に対しては、以下のように言われるべきである。アウグスティヌスが
考えているのは、諸々の魂が、一つの種の性格において合一されていないとい
う意味で、複数なのではない、ということである。

\end{longtable}
\newpage





\rhead{a.~3}
\begin{center}
{\Large {\bfseries ARTICULUS TERTIUS}}\\
{\large UTRUM PRAETER ANIMAM INTELLECTIVAM SINT IN HOMINE ALIAE ANIMAE PER ESSENTIAM DIFFERENTES}\\
{\footnotesize II {\itshape SCG}, cap.58; {\itshape De Pot.}, q.3, a.9, ad 9; {\itshape De Spirit.~Creat.}, a.3; Qu.~{\itshape de Anima}, a.2; {\itshape Quodl.}~XI, q.5; {\itshape Compend.~Theol.}, cap.90 sqq.}\\
{\Large 第三項\\知性的魂の他に本質によって異なる他の魂が人間の中にあるか}
\end{center}

\begin{longtable}{p{21em}p{21em}}
{\scshape Ad tertium sic proceditur}. Videtur quod praeter animam
intellectivam sint in homine aliae animae per essentiam differentes,
scilicet sensitiva et nutritiva. Corruptibile enim et incorruptibile
non sunt unius substantiae. Sed anima intellectiva est
incorruptibilis; aliae vero animae, scilicet sensitiva et nutritiva,
sunt corruptibiles, ut ex superioribus patet. Ergo in homine non
potest esse una essentia animae intellectivae et sensitivae et
nutritivae.

&

 第三項の問題へ、議論は以下のように進められる。知性的魂の他に、本質に
 よって異なる他の魂、すなわち、感覚的魂や栄養的魂が人間の中にあると思
 われる。理由は以下の通り。可滅的なものと不可滅的なものとは同一の実体
 に属さない。しかるに、前述のことから明らかなとおり、知性的魂は不可滅
 的であるが、その他の魂、すなわち感覚的魂と栄養的魂は可滅的である。ゆ
 えに人間の中に、知性的魂、感覚的魂、栄養的魂の一つの本質があることは
 できない。 
 
\\



2. Si dicatur quod anima sensitiva in homine est incorruptibilis,
contra, {\itshape corruptibile et incorruptibile differunt secundum
genus}, ut dicitur in X {\itshape Metaphys}. Sed anima sensitiva in
equo et leone et aliis brutis animalibus est corruptibilis. Si igitur
in homine sit incorruptibilis, non erit eiusdem generis anima
sensitiva in homine et bruto. Animal autem dicitur ex eo quod habet
animam sensitivam. Neque ergo animal erit unum genus commune hominis
et aliorum animalium. Quod est inconveniens.

&

もし、人間の中の感覚的魂は不可滅的だと言われるならば、それに反対して以
下のように述べる。『形而上学』第10巻で言われるように、「可滅的なものと
不可滅的なものは、類において異なる」。しかるに、馬やライオンや、その他
の非理性的動物の感覚的魂は可滅的である。ゆえに、もし、(感覚的魂が)人
間の中で不可滅的であったならば、人間と非理性的動物の中にある感覚的魂は
同じ類に属さないことになる。しかるに、動物は、感覚的魂を持つことからそ
う呼ばれる。ゆえに、人間と、その他の動物が属する一つの類はないことにな
る。これは不適当である。
 
\\



3. {\scshape Praeterea}, philosophus dicit, in libro {\itshape de
Generat. Animal}., quod embryo prius est animal quam homo. Sed hoc
esse non posset, si esset eadem essentia animae sensitivae et
intellectivae, est enim animal per animam sensitivam, homo vero per
animam intellectivam. Non ergo in homine est una essentia animae
sensitivae et intellectivae.

&

さらに、哲学者は『動物生成論』という書物の中で、胎児は、人間である前に
動物であると言っている。しかるに、このことは、もし感覚的魂と知性的魂の
本質が同じであったならば、ありえなかったであろう。というのも、感覚的魂
によって動物であり、知性的魂によって人間なのだから。ゆえに、人間の中で、
感覚的魂と知性的魂は一つでない。
 
\\



4. {\scshape Praeterea}, philosophus dicit, in VIII {\itshape
Metaphys}., quod genus sumitur a materia, differentia vero a
forma. Sed {\itshape rationale}, quod est differentia constitutiva
hominis, sumitur ab anima intellectiva; {\itshape animal} vero dicitur
ex hoc quod habet corpus animatum anima sensitiva. Anima ergo
intellectiva comparatur ad corpus animatum anima sensitiva, sicut
forma ad materiam. Non ergo anima intellectiva est eadem per essentiam
cum anima sensitiva in homine; sed praesupponit eam sicut materiale
suppositum.

&

さらに、哲学者は『形而上学』第8巻で、類は質料から、種差は形相から取ら
れると言う。しかるに、人間を構成する種差である「理性的な」は、知性的魂
から取られ、「動物」は、感覚的魂によって魂を与えられた身体をもつことか
ら言われる。ゆえに知性的魂は、感覚的魂によって魂を与えられた身体に対し
て、形相が質料に対するように関係する。ゆえに、人間において、知性的魂は、
感覚的魂と本質によって同じではなく、むしろ、質料的な個体としてそれを前
提とする。
 
\\



{\scshape Sed contra} est quod dicitur in libro {\itshape de
Eccles. Dogmat}., {\itshape Neque duas animas esse dicimus in homine
uno, sicut Iacobus et alii Syrorum scribunt, unam animalem, qua
animatur corpus, et immixta sit sanguini, et alteram spiritualem, quae
rationi ministret, sed dicimus unam et eandem esse animam in homine,
quae et corpus sua societate vivificat, et semetipsam sua ratione
disponit}.

&

 しかし反対に、『教会教義について』という書物で、次のように言われてい
 る。「ヤコブやシリア人のうちの他の人々が書いているように、私たちは一
 人の人間の中に二つの魂があり、それによって身体に魂が与えられ、血に混
 じっている一つの動物的な魂と、もう一つ別の、理性に仕える霊的な魂があ
 るとは言わず、むしろ私たちは、人間の中には一つの同一の魂があり、それ
 が自らの交わりによって身体を生かし、自ら自身を自らの理性によって整え
 る、と言う」。
 
\\



{\scshape Respondeo dicendum} quod Plato posuit diversas animas esse
in corpore uno, etiam secundum organa distinctas, quibus diversa opera
vitae attribuebat; dicens vim nutritivam esse in hepate,
concupiscibilem in corde, cognoscitivam in cerebro.

&

解答する。以下のように言われるべきである。プラトンは、一つの身体の中に、
区別された器官においてもまた、多様な魂があると考え、それらに生命の多様
な働きを帰属させた。曰く、栄養的な力は肝臓に、欲情的な力は心臓に、認識
的な力は脳にある、と。
 
\\

Quam quidem opinionem Aristoteles reprobat, in libro de anima, quantum
ad illas animae partes quae corporeis organis in suis operibus
utuntur, ex hoc quod in animalibus quae decisa vivunt, in qualibet
parte inveniuntur diversae operationes animae, sicut sensus et
appetitus. Hoc autem non esset, si diversa principia operationum
animae, tanquam per essentiam diversae, diversis partibus corporis
distributa essent.

&

 アリストテレスは、『デ・アニマ』という書物の中で、自らの働きにおいて
 身体的な器官を用いる魂の部分に関してこの意見を非難した。それは以下の
 理由による。すなわち、切り離されたも生きている動物において、どの部分
 においても、感覚や欲求のような魂のさまざまな働きが見出される。しかる
 に、もし、魂の働きのさまざまな根源が、本質的に異なるものとして、身体
 のさまざまな部分に分配されていたならば、こんなことはなかったであろう。
 
\\

 Sed de intellectiva sub dubio videtur relinquere utrum sit
separata ab aliis partibus animae solum ratione, an etiam loco.

&

しかし、知性的な魂については、魂の他の諸部分から、たんに概念的に分離し
ているのか、それとも場所的に分離しているのかについて、疑念が残るように
思われる。
 
\\

Opinio autem Platonis sustineri utique posset, si poneretur quod anima
unitur corpori, non ut forma, sed ut motor, ut posuit Plato. Nihil
enim inconveniens sequitur, si idem mobile a diversis motoribus
moveatur, praecipue secundum diversas partes. Sed si ponamus animam
corpori uniri sicut formam, omnino impossibile videtur plures animas
per essentiam differentes in uno corpore esse. Quod quidem triplici
ratione manifestari potest.

&

しかしプラトンの意見は、もしプラトンが考えたように、魂が形相としてでは
なく動者として身体に合一されていたとしたら、確かに保持され得たであろう。
というのも、同一の可動的なものが、多様な動者によって、とくに多様な部分
に即して動かされることがあっても、不都合は帰結しないからである。しかし、
もし私たちが、魂が形相として身体に合一されていると考えるのであれば、本
質によって異なる複数の魂が、一つの身体の中にあることは、あらゆる点で不
可能である。このことは、三つの理由で明らかにされうる。
 
\\

Primo quidem, quia animal non esset simpliciter unum, cuius essent
animae plures. Nihil enim est simpliciter unum nisi per formam unam,
per quam habet res esse, ab eodem enim habet res quod sit ens et quod
sit una; et ideo ea quae denominantur a diversis formis, non sunt unum
 simpliciter, sicut homo albus.

&

第一に、もし動物に複数の魂が属しているならば、その動物は端的に一ではな
かっただろう。なぜなら、事物は同じものから有であることと一であることを
もつので、なにものも、事物がそれを通して存在をもつところの一つの形相に
よらなければ、端的な一でないからである。ゆえに、さまざまな形相によって
名付けられるものは、たとえば「白い人間」のように、端的に一ではない。
 
\\

 Si igitur homo ab alia forma haberet
quod sit vivum, scilicet ab anima vegetabili; et ab alia forma quod
sit animal, scilicet ab anima sensibili; et ab alia quod sit homo,
scilicet ab anima rationali; sequeretur quod homo non esset unum
simpliciter, sicut et Aristoteles argumentatur contra Platonem, in
VIII {\itshape Metaphys}., quod si alia esset idea animalis, et alia bipedis, non
 esset unum simpliciter animal bipes.

&

ゆえに、もしも人間が、ある形相、すなわち植物的魂から生きることを持ち、
また別の形相、すなわち感覚的魂から動物であることを持ち、また別の形相、
すなわち理性的魂から人間であることを持つとすれば、アリストテレスが『形
而上学』第8巻でプラトンに反対して論じるように、人間は端的な一ではない
ことになっただろう。彼はそこで、もし動物のイデアと二本足のイデアが別の
イデアであるならば、二本足の動物は端的な一でなかっただろうと述べている。
 
\\

 Et propter hoc, in I {\itshape de Anima}, contra ponentes diversas
animas in corpore, inquirit quid contineat illas, idest quid faciat ex
eis unum. Et non potest dici quod uniantur per corporis unitatem, quia
magis anima continet corpus, et facit ipsum esse unum, quam e
converso.

&

そしてこのため、『デ・アニマ』第1巻で、身体の中にさまざまな魂があると
考える人たちに反対して、何がそれを包含するのか、すなわち、何が、それら
から一を作るのか、と問うている。また、身体の一性によって合一されると言
われることもできない。なぜなら、むしろ魂が身体を含み、身体を一にするの
であって、その逆ではないからである。
 
\\

 Secundo, hoc apparet impossibile ex modo praedicationis. Quae enim
 sumuntur a diversis formis, praedicantur ad invicem vel per accidens,
 si formae non sint ad invicem ordinatae, puta cum dicimus quod album
 est dulce, vel, si formae sint ordinatae ad invicem, erit praedicatio
 per se, in secundo modo dicendi per se, quia subiectum ponitur in
 definitione praedicati.

&

第二に、これは述語のしかたからも不可能である。理由は以下の通り。さまざ
まな形相から取られるものは、それらの形相が相互に秩序付けられていない場
合には、附帯的に相互に述語付けられる。たとえば、私たちが「白いものが甘
い」と言う場合のように。あるいは、それらの形相が相互に秩序付けられてい
るならば、第二の意味の「自体的に」\footnote{「自体的に」と訳すper seは、
逐語的には、「自分自身によって」「自分自身が原因となって」という意味。
アリストテレスは複数の箇所で、その意味を分類している。Cf. In I
Post.~Analytic., lect.10. ``Primus ergo modus dicendi per se est,
quando id, quod attribuitur alicui, pertinet ad formam eius.'' ...
``Secundus modus dicendi per se est, quando haec praepositio per
designat habitudinem causae materialis, prout scilicet id, cui aliquid
attribuitur, est propria materia et proprium subiectum ipsius.''
... ``Deinde cum dicit: amplius quod non etc., ponit alium modum eius,
quod est per se, prout per se significat aliquid solitarium, sicut
dicitur quod per se est aliquod particulare, quod est in genere
substantiae, quod non praedicatur de aliquo subiecto.'' ... ``Deinde
cum dicit: item alio modo etc., ponit quartum modum, secundum quod
haec praepositio per designat habitudinem causae efficientis vel
cuiuscunque alterius.''}で、自体的に述語付けられる。なぜなら、主語が述
語の定義の中に置かれるからである。
 
\\

 Sicut superficies praeambula est ad colorem, si ergo dicamus quod
 corpus superficiatum est coloratum, erit secundus modus
 praedicationis per se. Si ergo alia forma sit a qua aliquid dicitur
 animal, et a qua aliquid dicitur homo, sequeretur quod vel unum horum
 non possit praedicari de altero nisi per accidens, si istae duae
 formae ad invicem ordinem non habent; vel quod sit ibi praedicatio in
 secundo modo dicendi per se, si una animarum sit ad aliam praeambula.

&

たとえば、表面は色の前提であるように。ゆえに、もし私たちが、「表面があ
る物体には色がある」と言うならば、「自体的に」を述語付ける第二のしかた
のであるだろう\footnote{ややわかりにくい。表面と色は、色のない表面を考
えることはできるが(光のない空間に存在する場合など)、表面のない色を考
えることはできないという意味で、表面が色の必要条件というかたちの秩序を
もつと考えられるが、ここでは、量のカテゴリーが性質のカテゴリーに先行す
るという了解に基づいて、表面というある種の量(二次元)が、色という性質
に先行するという了解があると思われる。すなわち、色を理解するときには必
ず表面の理解が先行するという意味で、色の定義には表面が入る。このとき、
「表面がある物体は(自体的に)色がある」と言うならば、この「自体的に」
は、質料因を示す(それ自体が質料となって)ので、「自体的に」の第二の意
味である。}。ゆえに、もし、それによってあるものが動物と言われる形相が、
それによってあるものが人間と言われる形相と別の形相であるならば、これら
二つの形相が相互に秩序付けられていない場合には、これらの一方が他方につ
いて附帯的にしか述語付けられないし、また、もし、これらの魂が他の魂の前
提である場合には、そこには第二の意味の「自体的に」での述語づけがあるこ
とになるだろう。
 
\\

 Utrumque autem horum est manifeste falsum, quia animal per se de
 homine praedicatur, non per accidens; homo autem non ponitur in
 definitione animalis, sed e converso. Ergo oportet eandem formam esse
 per quam aliquid est animal, et per quam aliquid est homo, alioquin
 homo non vere esset id quod est animal, ut sic animal per se de
 homine praedicetur.

&

しかしこれらのどちらも明らかに偽である。理由は以下の通り。動物が自体的
に人間に述語付けられるのは附帯的にではない。しかるに人間は動物の定義の
中には置かれず、むしろ逆である\footnote{人間の定義が「理性的動物」だと
すると、人間の定義には「動物」が入っている。「自体的」が、第二の意味で
ある場合、それは質料因を指すので、「動物は自体的に人間である」と言われ
ることになる。しかしこれは主語と述語が逆である。}。ゆえに、あるものが
それによって動物であるところの形相と、それによって人間であるところの形
相は同一の形相である。さもなければ、人間は、真に、動物であるところのも
のではなく、動物が自体的に人間に述語付けされることもなかったであろう。
 
\\

Tertio, apparet hoc esse impossibile per hoc, quod una operatio
animae, cum fuerit intensa, impedit aliam. Quod nullo modo
contingeret, nisi principium actionum esset per essentiam unum.

&

第三に、このことが不可能であるのは、魂の一つの働きが、それが強い場合に、
他の働きを妨げることから明らかである。こういったことは、もしそれらの作
用の根源が一つの本質によるのでなかったならば、決して起こらなかったであ
ろう。
 
\\

 Sic ergo dicendum quod eadem numero est anima in homine sensitiva et
 intellectiva et nutritiva. Quomodo autem hoc contingat, de facili
 considerari potest, si quis differentias specierum et formarum
 attendat.

 &

それゆえ、かくして、人間の中の感覚的魂と知性的魂と栄養的魂は、数的に同
一であると言われるべきである。これがどのようにしてそうであるかは、もし
種と形相の差異に注目するならば、容易に考えられうる。
 
\\

 Inveniuntur enim rerum species et formae differre ab invicem secundum
 perfectius et minus perfectum, sicut in rerum ordine animata
 perfectiora sunt inanimatis, et animalia plantis, et homines
 animalibus brutis, et in singulis horum generum sunt gradus diversi.

&

すなわち、諸事物の種と形相は、より完全であるかそうでないかということに
応じて相互に見出される。たとえば、諸事物の秩序においては、魂があるもの
が、そうでないものよりも完全であり、動物は植物よりも、人間は非理性的動
物よりも完全である。そしてこれらの類の個々のものにおいて、さまざまな段
階がある。
 
\\

 Et ideo Aristoteles, in VIII {\itshape Metaphys}., assimilat species
 rerum numeris, qui differunt specie secundum additionem vel
 subtractionem unitatis. Et in II {\itshape de Anima}, comparat
 diversas animas speciebus figurarum, quarum una continet aliam; sicut
 pentagonum continet tetragonum, et excedit.

&

それゆえ、アリストテレスは、『形而上学』第8巻で、諸事物の種を数にたと
えている。数は、一を足したり引いたりすることで異なる種となるからである。
また、『デ・アニマ』第2巻では、異なる魂を形の種にたとえている。これら
の形は、たとえば五角形は四角形を含むように、一つの形が他の形を含むから
である。
 
\\

 Sic igitur anima intellectiva continet in sua virtute quidquid habet
 anima sensitiva brutorum, et nutritiva plantarum. Sicut ergo
 superficies quae habet figuram pentagonam, non per aliam figuram est
 tetragona, et per aliam pentagona; quia superflueret figura
 tetragona, ex quo in pentagona continetur; ita nec per aliam animam
 Socrates est homo, et per aliam animal, sed per unam et eandem.

&

それゆえ、このようにして、知性的魂は自らのちからの中に、獣たちの感覚的
魂や、植物の栄養的魂がもつものをすべて含んでいる。ゆえに、五角形の形を
持つ表面が、他の図形や他の五角形によらずに、四角形をもつように、という
のも、それによって五角形に含まれるところの四角形の図形が無駄になっただ
ろうからだが、ちょうどそのように、ソクラテスは、ある魂によって人間であ
り、別の魂によって動物なのではなく、一つで同一の魂によって、人間であり
動物である。
 
\\



 {\scshape Ad primum ergo dicendum} quod anima sensitiva non habet
incorruptibilitatem ex hoc quod est sensitiva, sed ex hoc quod est
intellectiva, ei incorruptibilitas debetur. Quando ergo anima est
sensitiva tantum, corruptibilis est, quando vero cum sensitivo
intellectivum habet, est incorruptibilis. Licet enim sensitivum
incorruptionem non det, tamen incorruptionem intellectivo auferre non
potest.


&

 第一異論に対しては、それゆえ、以下のように言われるべきである。感覚的
 魂は、感覚的であることから、不可滅性をもつのではなく、その魂が知性的
 であることからそれをもつのであり、知性的魂に不可滅性を負っている。ゆ
 えに、魂がたんに感覚的であるにすぎない場合には、それは可滅的であり、
 他方、感覚的であることとともに知性的であることを持つ場合には、不可滅
 的である。つまり、不可滅性を、感覚的であることには負わないが、しかし、
 知性的なものから不可滅性を取り去ることはできない。
 
\\



{\scshape Ad secundum dicendum} quod formae non collocantur in genere
vel in specie, sed composita. Homo autem corruptibilis est, sicut et
alia animalia. Unde differentia secundum corruptibile et
incorruptibile, quae est ex parte formarum, non facit hominem secundum
genus ab aliis animalibus differre.

 &

 第二異論に対しては、以下のように言われるべきである。類や種の中に置か
 れるのは形相ではなく複合体である。ところで、人間は他の動物と同じよう
 に可滅的である。したがって、可滅的、不可滅的という点での差異、これは
 形相の側からあるのだが、は、人間を、他の動物と類において違うものにし
 ない。
 
\\


{\scshape Ad tertium dicendum} quod prius embryo habet animam quae est
sensitiva tantum; qua abiecta, advenit perfectior anima, quae est
simul sensitiva et intellectiva; ut infra\footnote{Q.118, a.2, ad2.} plenius ostendetur.

&

 第三異論に対しては、以下のように言われるべきである。後でより十分に示
 されることだが、胎児はたんに感覚的である魂を、より先に持っていて、そ
 れを捨てた後に、感覚的であり同時に知性的である、より完全な魂が到来す
 る。
 
\\



{\scshape Ad quartum dicendum} quod non oportet secundum diversas
rationes vel intentiones logicas, quae consequuntur modum
intelligendi, diversitatem in rebus naturalibus accipere, quia ratio
 unum et idem secundum diversos modos apprehendere potest.

 &

 第四異論に対しては、以下のように言われるべきである。知性認識のあり方
に伴う様々な概念や論理概念に応じて、自然的諸事物における多様性が理解さ
れる必要はない。理性は、一つで同一のものを、様々な仕方で把握することが
できるからである。

 \\

Quia igitur, ut dictum est, anima intellectiva virtute continet id
quod sensitiva habet, et adhuc amplius; potest seorsum ratio
considerare quod pertinet ad virtutem sensitivae, quasi quoddam
imperfectum et materiale. Et quia hoc invenit commune homini et aliis
animalibus, ex hoc rationem generis format. Id vero in quo anima
intellectiva sensitiva excedit, accipit quasi formale et completivum,
et ex eo format differentiam hominis.


&

ゆえに、上述のように、知性的魂がちからにおいて感覚的魂が持つものを含み、
さらにそれ以上のものを含むから、理性は、感覚的魂のちからに属するものを、
いわば不完全で質料的なものとして、切り離して考察することができる。そし
て、これが人間と他の動物に共通することを見出すので、このことから類の概
念を形成する。他方で、知性的魂が感覚的なものを超える点において、形相的
で完全なものを理解し、それに基づいて、人間の種差を形成する。
 

\end{longtable}
\newpage
\rhead{a.~4}
\begin{center}
{\Large {\bfseries ARTICULUS QUARTUS}}\\
{\large UTRUM IN HOMINE SIT ALIA FORMA PRAETER ANIMAM INTELLECTIVAM}\\
 {\footnotesize IV {\itshape Sent.}, d.44, q.1, a.1, qu$^{a}$.1, ad 4; IV {\itshape SCG}, cap.81; {\itshape De Spirit.~Creat.}, a.3; \\Qu.~{\itshape de Anima}, a.9;{\itshape Quodl.}~I, q.4, a.1; XI, q.5; {\itshape Compend.~Theol.}, cap.90.}\\
{\Large 第四項\\人間の中に知性的魂以外の形相があるか}
\end{center}

\begin{longtable}{p{21em}p{21em}}
{\scshape Ad quartum sic proceditur}. Videtur quod in homine sit alia
forma praeter animam intellectivam. Dicit enim philosophus, in II
{\itshape de Anima}, quod anima est {\itshape actus corporis physici
potentia vitam habentis}. Comparatur igitur anima ad corpus, sicut
forma ad materiam. Sed corpus habet aliquam formam substantialem per
quam est corpus. Ergo ante animam praecedit in corpore aliqua forma
substantialis.

&

 第四項の問題へ、議論は以下のように進められる。人間の中に、知性的魂以
 外の他の形相があると思われる。理由は以下の通り。哲学者は『デ・アニマ』
 第2巻で、魂とは「生命を持つ自然的身体の現実態」であると言う。ゆえに、
 魂は身体に対して、形相が質料に対するように関係する。しかるに、身体は、
 それによって身体であるところの何らかの実体形相をもつ。ゆえに、魂の前
 に、身体においてなんらかの実体形相が先行する。
 
\\


2. {\scshape Praeterea}, homo et quodlibet animal est movens
seipsum. Omne autem movens seipsum dividitur in duas partes, quarum
una est movens, et alia est mota, ut probatur in VIII {\itshape
Physic}. Pars autem movens est anima. Ergo oportet quod alia pars sit
talis quae possit esse mota. Sed materia prima non potest moveri, ut
dicitur in V {\itshape Physic}., cum sit ens solum in potentia,
quinimmo omne quod movetur est corpus. Ergo oportet quod in homine et
in quolibet animali sit alia forma substantialis, per quam
constituatur corpus.

&

 さらに、人間もどんな動物も、自らを動かす。しかるに、『自然学』第8巻で
 証明されているように、自分を動かすものは、動かすものと動かされるもの
 との二つの部分に分割される。しかるに、動かす部分は魂である。ゆえに、
 動かされるものでありうるような別の部分がなければならない。しかるに、
 『自然学』第5巻で言われているように、第一質料は、たんに可能態だけにあ
 る有なので、動かされえない。にもかかわらず、すべて動かされるものは物
 体(身体)である。ゆえに、人間もどんな動物も、その中に、それによって
 身体が構成されるところの他の実体形相がなければならない。
 
\\


3. {\scshape Praeterea}, ordo in formis attenditur secundum
habitudinem ad materiam primam, prius enim et posterius dicitur per
comparationem ad aliquod principium. Si ergo non esset in homine alia
forma substantialis praeter animam rationalem, sed immediate materiae
primae inhaereret; sequeretur quod esset in ordine imperfectissimarum
formarum, quae immediate inhaerent materiae.

&

 さらに、諸形相の中に、第一質料への関係に即して秩序が見出される。とい
 うのも、より先、より後、ということは、最初の何かへの関係によって言わ
 れるからである。ゆえに、もし人間の中に理性的魂以外の実体形相がないな
 く、直接的に第一質料に内在していたら、(理性的魂は)質料に直接的に内
 在するもっとも不完全な形相の秩序においてあることが帰結したであろう。

 
\\


4. {\scshape Praeterea}, corpus humanum est corpus mixtum. Mixtio
autem non fit secundum materiam tantum, quia tunc esset corruptio
sola. Oportet ergo quod remaneant formae elementorum in corpore mixto,
quae sunt formae substantiales. Ergo in corpore humano sunt aliae
formae substantiales praeter animam intellectivam.

&

さらに、人間の身体は混合した物体である。しかるに混合は、たんに質料だけ
に即しては生じない。なぜなら、その場合には、たんに消滅だけがあっただろ
うから。ゆえに、実体形相であるところの諸元素の形相が、混合物体の中に留
まっていなければならない。ゆえに、人間の身体の中には、知性的魂以外に他
の実体形相がある。
 
\\


{\scshape Sed contra}, unius rei est unum esse substantiale. Sed forma
substantialis dat esse substantiale. Ergo unius rei est una tantum
forma substantialis. Anima autem est forma substantialis hominis. Ergo
impossibile est quod in homine sit aliqua alia forma substantialis
quam anima intellectiva.

&

 しかし反対に、一つの事物には一つの実体的存在が属する。しかるに実体的
 形相は、実体的存在を与える。ゆえに、一つの事物には、ただ一つの実体形
 相しか属さない。しかるに魂は人間の実体形相である。ゆえに人間の中に、
 知性的魂以外の実体形相があることは不可能である。
 
\\


{\scshape Respondeo dicendum} quod, si poneretur anima intellectiva
non uniri corpori ut forma, sed solum ut motor, ut Platonici
posuerunt; necesse esset dicere quod in homine esset alia forma
substantialis, per quam corpus ab anima mobile in suo esse
constitueretur. Sed si anima intellectiva unitur corpori ut forma
substantialis, sicut supra iam diximus, impossibile est quod aliqua
 alia forma substantialis praeter eam inveniatur in homine.

 &

 解答する。以下のように言われるべきである。もし、知性的魂が形相として
 ではなくたんに動者として身体に合一されていると考えたのであれば、たと
 えばプラトン派の人々がそうしたように、人間の中に、それによって身体が
 魂によって動かされうるところの、自らの存在において構成される他の実体
 形相があると言うことが必然だったであろう。しかし、もし知性的魂が、私
 たちがすでに述べたように、実体形相として身体に合一されているのであれ
 ば、人間の中にすでに見出されている魂以外に、別の実体形相があることは
 不可能である。
 
 
\\

Ad cuius evidentiam, considerandum est quod forma substantialis in hoc
a forma accidentali differt quia forma accidentalis non dat esse
simpliciter, sed esse tale, sicut calor facit suum subiectum non
simpliciter esse, sed esse calidum. Et ideo cum advenit forma
accidentalis, non dicitur aliquid fieri vel generari simpliciter, sed
fieri tale aut aliquo modo se habens, et similiter cum recedit forma
accidentalis, non dicitur aliquid corrumpi simpliciter, sed secundum
quid.

&

これを明らかにするために、以下のことが考察されるべきである。実体形相が
附帯形相と異なるのは、附帯形相が端的な存在(「がある」)を与えず、むし
ろ「これこれである」を与えるという点である。例えば、熱は自らの基体を端
的に存在させるのではなく、「熱いものである」ようにする。ゆえに、附帯形
相が到来するとき、何かが端的に生じるとか生成するとは言われず、「このよ
うになる」とか「何らかの状態にある」と言われるし、同様に、附帯形相がな
くなるときには、あるものが端的に消滅するとは言われず、ある意味で消滅す
ると言われる。
 
\\

Forma autem substantialis dat esse simpliciter, et ideo per eius
adventum dicitur aliquid simpliciter generari, et per eius recessum
simpliciter corrumpi. Et propter hoc antiqui naturales, qui posuerunt
materiam primam esse aliquod ens actu, puta ignem aut aerem aut
aliquid huiusmodi, dixerunt quod nihil generatur aut corrumpitur
simpliciter, sed omne fieri statuerunt alterari, ut dicitur in I
Physic.

&

これに対して実体形相は端的に(な)存在を与え、それゆえ、それの到来によっ
て、何かが端的に生成すると言われ、それがなくなることで端的に消滅すると
言われる。このため、古代の自然学者たちは、第一質料がある意味で現実態に
おける有だと考えたので、それはたとえば、火や空気や、その他そのようなも
ののことだが、何ものも端的に生成したり消滅したりせず、すべての生成は変
化だと断じた。これは『自然学』第1巻で言われているとおりである。
 
\\

Si igitur ita esset, quod praeter animam intellectivam praeexisteret
quaecumque alia forma substantialis in materia, per quam subiectum
animae esset ens actu; sequeretur quod anima non daret esse
simpliciter; et per consequens quod non esset forma substantialis; et
quod per adventum animae non esset generatio simpliciter, neque per
eius abscessum corruptio simpliciter, sed solum secundum quid. Quae
 sunt manifeste falsa.

 &

ゆえに、もしこのようであったならば、すなわち、知性的魂以外に何であれ他
の実体形相が質料において先在し、それによって魂の基体が現実態における有
であったならば、魂は端的に存在を与えるのではなく、結果的に、実体形相で
はなかったであろう。そして、魂の到来によって端的な生成があるのではなく、
またそれがなくなることによって端的な消滅があるのでもなく、むしろある意
味での消滅があることになっただろう。これは明らかに偽である。
 
\\

Unde dicendum est quod nulla alia forma substantialis est in homine,
nisi sola anima intellectiva; et quod ipsa, sicut virtute continet
animam sensitivam et nutritivam, ita virtute continet omnes inferiores
formas, et facit ipsa sola quidquid imperfectiores formae in aliis
faciunt. Et similiter est dicendum de anima sensitiva in brutis, et de
nutritiva in plantis, et universaliter de omnibus formis
perfectioribus respectu imperfectiorum.

&

 したがって、以下のように言われるべきである。ただひとつの知性的魂以外
 に、人間の中に実体形相はなく、その魂が、言わばちからにおいて、感覚的
 魂と栄養的魂を含み、また、ちからにおいて、すべての下位の形相を含み、
 何であれ不完全な諸形相が他のものどもの中で行うことを、その魂だけによっ
 て行う。同様のことは、非理性的な動物における感覚的魂や、植物における
 栄養的魂について、そして、より不完全なものに関する、より完全なすべて
 の諸形相について普遍的に言われるべきである。
 
\\


{\scshape Ad primum ergo dicendum} quod Aristoteles non dicit animam
esse {\itshape actum corporis} tantum, sed {\itshape actum corporis
physici organici potentia vitam habentis}, et quod talis potentia non
abiicit animam. Unde manifestum est quod in eo cuius anima dicitur
actus, etiam anima includitur; eo modo loquendi quo dicitur quod calor
est actus calidi, et lumen est actus lucidi; non quod seorsum sit
lucidum sine luce, sed quia est lucidum per lucem. Et similiter
dicitur quod anima est actus corporis etc., quia per animam et est
corpus, et est organicum, et est potentia vitam habens. Sed actus
primus dicitur in potentia respectu actus secundi, qui est
operatio. Talis enim potentia est {\itshape non abiiciens}, idest non
excludens, animam.

&

 第一異論に対しては、それゆえ、以下のように言われるべきである。アリス
 トテレスは、魂がただ「身体の現実態」だとは言っておらず、「可能態にお
 いて生命を持つ自然的身体器官の現実態」だと言い、そのような可能態は、
 魂を捨てていない。したがって、それの魂が現実態だと言われるものにおい
 て、さらに魂が含まれていることは明らかである。それは、熱が熱いものの
 現実態であり、光が輝くものの現実態であると言われる言い方によってであ
 る。つまり、光と切り離されて輝くものがあるのではなく、輝くものは光に
 よってある。同様に、魂が身体の現実態云々と言われるのは、魂によって身
 体があり、組織され、可能態において生命を持つからである。しかし、第一
 現実態は、働きである第二現実態への関係において可能態においてあると言
 われる。なぜなら、そのような可能態は、魂を「捨てない」、すなわち排除
 しないものだから。
 
 
\\


{\scshape Ad secundum dicendum} quod anima non movet corpus per esse
suum, secundum quod unitur corpori ut forma; sed per potentiam
motivam, cuius actus praesupponit iam corpus effectum in actu per
animam; ut sic anima secundum vim motivam sit pars movens, et corpus
animatum sit pars mota.

&

 第二異論に対しては、以下のように言われるべきである。魂は、それに即し
 て形相として身体に合一されているところの存在によって身体を動かすので
 はなく、動かす能力によって動かす。その能力の現実態は、身体が魂によっ
 てすでに現実態に作出されていることを前提とする。その結果、魂は、動か
 す力に即して動かす部分であり、魂を持つ身体は、動かされる部分である。
 
\\

{\scshape Ad tertium dicendum} quod in materia considerantur diversi
gradus perfectionis, sicut esse, vivere sentire et intelligere. Semper
autem secundum superveniens priori, perfectius est. Forma ergo quae
dat solum primum gradum perfectionis materiae, est imperfectissima,
sed forma quae dat primum et secundum, et tertium, et sic deinceps,
est perfectissima; et tamen materiae immediata.

&

 第三異論に対しては、以下のように言われるべきである。質料において、様々
 な完全性の段階が考察される。たとえば、存在すること、生きること、感覚
 すること、知性認識すること、のように。しかるに、常に、より先のものに
 到来する第二のものは、より完全である。ゆえに、質料に第一段階の完全性
 だけを与える形相はもっとも不完全である。しかし、第一段階、第二段階、
 第三段階、そしてそれ以降の完全性も与える形相はもっとも完全である。し
 かしそれは直接的に質料に属する。
 
\\


{\scshape Ad quartum dicendum} quod Avicenna posuit formas
substantiales elementorum integras remanere in mixto, mixtionem autem
fieri secundum quod contrariae qualitates elementorum reducuntur ad
 medium.


&

 第四異論に対しては、以下のように言われるべきである。アヴィセンナは、
 諸元素の実体形相が、混合物において十全なものに留まり、混合は諸元素の
 相反する性質が中間のものになることで生じると考えた。
 
\\


 Sed hoc est impossibile. Quia diversae formae elementorum non
possunt esse nisi in diversis partibus materiae; ad quarum
diversitatem oportet intelligi dimensiones, sine quibus materia
 divisibilis esse non potest.
 
&

 しかしこれは不可能である。なぜなら、諸元素の多様な形相は、質料の多様
 な部分においてでなければありえず、そのような多様性には、次元が理解さ
 れなければならないからである。次元なしに質料が分割されうることはあり
 えないのだから。

 
\\

Materia autem dimensioni subiecta non invenitur nisi in
corpore. Diversa autem corpora non possunt esse in eodem loco. Unde
sequitur quod elementa sint in mixto distincta secundum situm. Et ita
non erit vera mixtio, quae est secundum totum, sed mixtio ad sensum,
quae est secundum minima iuxta se posita.

&

しかるに、次元のもとにある質料は物体の中にしか見出されない。さらに、様々
な物体が同一の場所に在ることはできない。したがって、諸元素は混合物の中
に、位置(situs)において区別されたものとしてあることが帰結する。そうし
て、全体に即した真の混合はなく、最短の距離で隣同士であるという感覚に対す
る混合であることになるだろう。
 
\\


Averroes autem posuit, in III {\itshape de Caelo}, quod formae
elementorum, propter sui imperfectionem, sunt mediae inter formas
accidentales et substantiales; et ideo recipiunt magis et minus; et
ideo remittuntur in mixtione et ad medium reducuntur, et conflatur ex
eis una forma.
 
&

 他方で、アヴェロエスは、『天体論』第3巻で、諸元素の形相は、自らの不完
 全性のために、附帯形相と実体形相の中間にあり、それゆえ、より多く、よ
 り少なくということを受け入れ、ゆえに、それらは混合において緩和され、
 中間のものになり、それらから一つの形相が作られると考えた。

 
\\

Sed hoc est etiam magis impossibile. Nam esse substantiale
cuiuslibet rei in indivisibili consistit; et omnis additio et
subtractio variat speciem, sicut in numeris, ut dicitur in VIII
Metaphys. Unde impossibile est quod forma substantialis quaecumque
recipiat magis et minus. Nec minus est impossibile aliquid esse medium
inter substantiam et accidens.

&

 しかしこれもまたもっと不可能である。理由は以下の通り。どんな事物の実
 体的存在も個体において成立する。そして『形而上学』第8巻で言われるよう
 に、数におけるように、すべての加算と減算は種を違うものにする。したがっ
 て、どんな実体形相も、より多く、より少なくということを受け入れること
 はできない。また、あるものが実体と附帯性の間の中間のものであることは、
 同じく不可能である。
 
\\


Et ideo dicendum est, secundum philosophum in I {\itshape de
Generat}., quod formae elementorum manent in mixto non actu, sed
virtute. Manent enim qualitates propriae elementorum, licet remissae,
in quibus est virtus formarum elementarium. Et huiusmodi qualitas
mixtionis est propria dispositio ad formam substantialem corporis
mixti, puta formam lapidis, vel animae cuiuscumque.

 &

 ゆえに以下のように言われるべきである。『生成消滅論』第1巻の哲学者によ
 れば、諸元素の形相は、現実態においてではなく、ちからにおいて、混合物
 の中に留まっている。つまり、諸事物の固有の性質は、減衰したとはいえ、
 留まっていて、そこにおいて諸元素の形相のちからがある。そしてこのよう
 な混合物の性質は、混合物体の実体形相、たとえば石の形相や、何のであれ
 魂に対して、固有の態勢である。

\end{longtable}
\newpage
\rhead{a.~5}
\begin{center}
{\Large {\bfseries ARTICULUS QUINTUS}}\\
{\large UTRUM ANIMA INTELLECTIVA CONVENIENTER TALI CORPORI UNIATUR}\\
{\footnotesize II {\itshape Sent.}, d.1, q.11, a.5; {\itshape De Malo}, q.5, a.5; Qu.~{\itshape de Anima}, a.8.}\\
{\Large 第五項\\知性的魂は適切にこのような身体に合一されているか}
\end{center}

\begin{longtable}{p{21em}p{21em}}
 {\scshape Ad quintum sic proceditur}. Videtur quod anima intellectiva
 inconvenienter tali corpori uniatur. Materia enim debet esse
 proportionata formae. Sed anima intellectiva est forma
 incorruptibilis. Non ergo convenienter unitur corpori corruptibili.
 
&

第五項の問題へ、議論は以下のように進められる。知性的魂は、不適切に、こ
のような身体に合一されていると思われる。理由は以下の通り。質料は質料に
比例していなければならない。しかるに知性的魂は不滅的な形相である。ゆえ
に、可滅的な身体に、不適切に合一されている。
 
\\


 2. {\scshape Praeterea}, anima intellectiva est forma maxime
 immaterialis, cuius signum est, quod habet operationem in qua non
 communicat materia corporalis. Sed quanto corpus est subtilius, tanto
 minus habet de materia. Ergo anima deberet subtilissimo corpori
 uniri, puta igni; et non corpori mixto, et terrestri magis.
 
&

 さらに、知性的魂は、最大限に非質料的な形相である。そのしるしは、身体
 的な質料と共有しない働きを持つことである。しかるに、身体が精妙であれ
 ばあるほど、より少なく質料をもつ。ゆえに、魂は、混合的物体で、より地
 上的である身体ではなく、たとえば火のような、もっとも精妙な物体と合一
 されるべきであった。
\\




3. {\scshape Praeterea}, cum forma sit principium speciei, ab una
forma non proveniunt diversae species. Sed anima intellectiva est una
forma. Ergo non debet uniri corpori quod componitur ex partibus
dissimilium specierum.
 
&

 さらに、形相は種の根源であるから、一つの形相から多様な種は出てこない。
 しかるに知性的魂は一つの形相である。ゆえに、類似しない種の部分から複
 合された身体に合一されるべきではない。
\\




4. {\scshape Praeterea}, perfectioris formae debet esse perfectius
susceptibile. Sed anima intellectiva est perfectissima animarum. Cum
igitur aliorum animalium corpora habeant naturaliter insita tegumenta,
puta pilorum loco vestium, et unguium loco calceamentorum; habeant
etiam arma naturaliter sibi data, sicut ungues, dentes et cornua, ergo
videtur quod anima intellectiva non debuerit uniri corpori imperfecto
tanquam talibus auxiliis privato.
 
&

 さらに、より完全な形相を受容しうるものは、より完全でなければならない。
 しかるに知性的魂は魂の中でもっとも完全である。ゆえに、他の動物の身体
 は、衣服の代わりに毛皮、靴の代わりに蹄、というように、自然本性的に覆
 いが備わっていて、さらに、爪、歯、角のような自然本性的に与えられた武
 器をもっているのだから、知性的魂は、このような補助を欠いた不完全な身
 体に合一されるべきではなかった。
\\




 {\scshape Sed contra est} quod dicit philosophus, in II {\itshape de
 Anima}, quod anima est {\itshape actus corporis physici organici
 potentia vitam habentis}.
 
&

 しかし反対に、哲学者は『デ・アニマ』第2巻で、「魂は可能態において魂を
 持つ自然的な器官の身体の現実態である」と言っている。
 
\\




 {\scshape Respondeo dicendum} quod, cum forma non sit propter
 materiam, sed potius materia propter formam; ex forma oportet
 rationem accipere quare materia sit talis, et non e converso. Anima
 autem intellectiva, sicut supra habitum est, secundum naturae
 ordinem, infimum gradum in substantiis intellectualibus tenet;
 intantum quod non habet naturaliter sibi inditam notitiam veritatis,
 sicut Angeli, sed oportet quod eam colligat ex rebus divisibilibus
 per viam sensus, ut Dionysius dicit, VII cap.~{\itshape de Div.~Nom}.


&

 解答する。以下のように言われるべきである。形相が質料のためにあるので
 はなく、むしろ質料は形相のためにあるので、なぜ質料がそのようであるか
 の理由は、形相から受け取らなければならなず、その逆ではない。しかるに
 知性的魂は、前に述べられたとおり、自然の秩序において、知性的実体の中
 で最下位に位置する。それは、天使たちのように、本性的に自らに植え付け
 られた真理の知をもたず、感覚の道を通って、分割可能な諸事物からそれを
 集めなければならないからである。これはディオニュシウスが『神名論』第7
 章で言うとおりである。
 
\\


 Natura autem nulli deest in necessariis, unde oportuit quod anima
 intellectiva non solum haberet virtutem intelligendi, sed etiam
 virtutem sentiendi. Actio autem sensus non fit sine corporeo
 instrumento. Oportuit igitur animam intellectivam tali corpori uniri,
 quod possit esse conveniens organum sensus.

&

 ところで何ものにおいても、必要なものにおいて自然が欠けることはない。
 したがって、知性的魂は、知性認識のちからだけでなく、感覚のちからを持
 つことが必要だった。しかるに、感覚の作用は、道具となる身体なしにはな
 されない。ゆえに、知性的魂が、感覚の適合的な器官でありうるような身体
 に合一されることが必要であった。
 
\\


 Omnes autem alii sensus fundantur supra tactum. Ad organum autem
 tactus requiritur quod sit medium inter contraria, quae sunt calidum
 et frigidum, humidum et siccum, et similia, quorum est tactus
 apprehensivus, sic enim est in potentia ad contraria, et potest ea
 sentire. Unde quanto organum tactus fuerit magis reductum ad
 aequalitatem complexionis, tanto perceptibilior erit tactus.

&

さて、他のすべての感覚は触覚に基礎付けられる。また、触覚の器官には、触
覚が把握しうる、熱と冷、湿と乾、その他類似したものどものような、反対の
ものどもの中間にあり、そのようなかたちで反対のものどもへの可能態にあり、
それらを感覚することができる、ということが求められる。このことから、触
覚の器官が、より高いレベルで混合の等しさをもっているほど、その触覚は、
より感受性が高くなる。
 
\\


 Anima autem intellectiva habet completissime virtutem sensitivam,
 quia quod est inferioris praeexistit perfectius in superiori ut dicit
 Dionysius in libro {\itshape de Div.~Nom}. Unde oportuit corpus cui unitur anima
 intellectiva, esse corpus mixtum, inter omnia alia magis reductum ad
 aequalitatem complexionis.

&

 ところで、知性的魂は、もっとも完全な感覚のちからを持つ。なぜなら、ディ
 オニュシウスが『神名論』という書物で言うように、下位のものに属するも
 のは、上位のものの中に、より完全に先在するからである。したがって、知
 性的魂に合一される身体は、他のすべての身体に優って、より大きな混合の
 等しさを持つ、混合された物体でなければならなかった。
 
\\


 Et propter hoc homo inter omnia animalia melioris est tactus. Et
 inter ipsos homines, qui sunt melioris tactus, sunt melioris
 intellectus. Cuius signum est, quod {\itshape molles carne bene aptos
 mente videmus}, ut dicitur in II de anima.

&

そしてこのために、人間は、他のすべての動物の中で、よい触覚を持つ。また、人間自身の中でも、より善い触覚をもつ人が、よりよい知性を持つ。そのしるしは、『デ・アニマ』第2巻で言われていることだが、「肉において柔らかい人たちが、精神においても、よく適していると私たちは思う」ということがある。
 
\\



 {\scshape Ad primum ergo dicendum} quod hanc obiectionem aliquis
 forte vellet evadere per hoc, quod diceret corpus hominis ante
 peccatum incorruptibile fuisse. Sed haec responsio non videtur
 sufficiens, quia corpus hominis ante peccatum immortale fuit non per
 naturam, sed per gratiae divinae donum; alioquin immortalitas eius
 per peccatum sublata non esset, sicut nec immortalitas Daemonis.

&

 第一異論に対しては、それゆえ、以下のように言われるべきである。ある人
 は、この反論を、次のように言うことで避けたいと考える。すなわち、人間
 の身体は、罪の前は不滅であった、と。しかし、この答えが十分とは思われ
 ない。なぜなら、罪以前の人間の身体は、自然本性によってではなく、神か
 ら与えられた恩恵によって不死だったからである。さもなければ、人間の不
 死性が、罪によって除かれることはなかっただろう。ちょうど悪魔の不死性
 が除かれなかったように。
 
\\


 Et ideo aliter dicendum est, quod in materia duplex conditio
 invenitur, una quae eligitur ad hoc quod sit conveniens formae; alia
 quae ex necessitate consequitur prioris dispositionis. Sicut artifex
 ad formam serrae eligit materiam ferream, aptam ad secandum dura; sed
 quod dentes serrae hebetari possint et rubiginem contrahere, sequitur
 ex necessitate materiae.

&

 それゆえ、別のしかたで、以下のように言われるべきである。すなわち、質
 料の中には二通りの条件が見出され、一つは、形相に適合するために選ばれ、
 もう一つは、より先の条件の必然性から帰結する。たとえば、職人が、ノコ
 ギリの質料として鉄を選ぶのは、硬いものを切るのに適しているからだが、
 しかし、ノコギリの歯が鈍くなったり、サビを集めたりすることは、質料の
 必然性から帰結する。
 
\\


 Sic igitur et animae intellectivae debetur corpus quod sit aequalis
 complexionis, ex hoc autem de necessitate materiae sequitur quod sit
 corruptibile. Si quis vero dicat quod Deus potuit hanc necessitatem
 vitare, dicendum est quod in constitutione rerum naturalium non
 consideratur quid Deus facere possit, sed quid naturae rerum
 conveniat, ut Augustinus dicit, II super Gen. ad Litt. Providit tamen
 Deus adhibendo remedium contra mortem per gratiae donum.

&

ゆえに、このように、知性的魂も、混合の等しさをもつ身体が必要だが、しか
し、このことから、質料に、それが可滅的であることが帰結する。もし、だれ
かが、神はこの必然性を避けることができたと言うならば、アウグスティヌス
が『創世記逐語注解』第2巻で言うように、自然的諸事物の構成においては、
神が何を為しうるかは考えられず、むしろ何が諸事物の本性に適合するかが考
えられる。神はしかし、無償の恩恵によって、死に対する救済をもたらすこと
で、摂理する。
 
\\



 {\scshape Ad secundum dicendum} quod animae intellectivae non debetur
 corpus propter ipsam intellectualem operationem secundum se; sed
 propter sensitivam virtutem, quae requirit organum aequaliter
 complexionatum.

 &

 第二異論に対しては、以下のように言われるべきである。知性的魂は、知性
 的な働きそれ自体において、その働きのために身体を必要とするのではなく、
 感覚的なちからのためにである。そして、このちからは、等しく混合した器
 官を必要とする。

\\

 
 Et ideo oportuit animam intellectivam tali corpori uniri, et non
 simplici elemento, vel corpori mixto in quo excederet ignis secundum
 quantitatem, quia non posset esse aequalitas complexionis, propter
 excedentem ignis activam virtutem. Habet autem hoc corpus aequaliter
 complexionatum quandam dignitatem, per hoc quod est remotum a
 contrariis; in quo quodammodo assimilatur corpori caelesti.
 
 &

 それゆえ、知性的魂は、このような身体に合一される必要があったのであり、
 単純な元素や、量において火が優勢な混合物体にではなかった。というのも、
 もしそのような混合物体の場合、火の優越する能動的ちからのために、混合
 の等しさがありえなかっただろうから。しかし等しく混合されたこの身体は、
 相反するものどもから離れていることによって、一種の威厳をもっている。
 この点において、それはある意味で、天体に似ている。

\\

{\scshape Ad tertium dicendum} quod partes animalis, ut oculus, manus,
caro et os, et huiusmodi, non sunt in specie, sed totum, et ideo non
potest dici, proprie loquendo, quod sint diversarum specierum, sed
quod sint diversarum dispositionum.
 
&

 第三異論に対しては、以下のように言われるべきである。眼、手、肉、骨の
 ような動物の諸部分は、種の中にはなく、全体である。ゆえに、厳密に言う
 ならば、それらは様々な種に属しているのではなく、むしろ、様々な態勢に
 属する。
 
\\

Et hoc competit animae intellectivae, quae quamvis sit una secundum
essentiam, tamen propter sui perfectionem est multiplex in virtute; et
ideo, ad diversas operationes, indiget diversis dispositionibus in
partibus corporis cui unitur. Et propter hoc videmus quod maior est
diversitas partium in animalibus perfectis quam in imperfectis, et in
his quam in plantis.
 
&

 そしてこのことが、知性的魂に適合する。つまりそれは、本質に即して一つ
 のものだが、しかし、自らの完全性のために、ちからにおいて多様であり、
 それゆえ、様々な働きのために、それに自らが合一されているところの身体
 の諸部分において、様々な態勢を必要とする。このため、私たちは、諸々の
 完全な動物において、不完全な動物よりも複雑な部分を見るし、また、不完
 全な動物においては、植物においてよりも、より複雑な部分を見る。
 
\\




 {\scshape Ad quartum dicendum} quod anima intellectiva, quia est
 universalium comprehensiva, habet virtutem ad infinita. Et ideo non
 potuerunt sibi determinari a natura vel determinatae existimationes
 naturales, vel etiam determinata auxilia vel defensionum vel
 tegumentorum; sicut aliis animalibus, quorum animae habent
 apprehensionem et virtutem ad aliqua particularia determinata.

&

第四異論に対しては、以下のように言われるべきである。知性的魂は、普遍を
把握できるので、無限のものどもに向かうちからをもつ。ゆえに、他の動物の
ように、自然によって、限定された自然本性的判断や、防御や覆いのような限
定された補助が、人間に限定されることはできなかった。(これに対して)他
の動物の魂は、何らかの個別的で限定されたものに向かう把握やちからをもつ。
 
\\


 Sed loco horum omnium, homo habet naturaliter rationem, et manus,
 quae sunt organa organorum, quia per eas homo potest sibi praeparare
 instrumenta infinitorum modorum, et ad infinitos effectus.

&

しかしこれらの代わりに、人間は自然本性的に理性を持ち、そして、器官たち
の器官である手を持つ。というのも、人間は手によって、無限のあり方の道具
を、無限の結果に向けて準備するからである。

\end{longtable}
\newpage

\rhead{a.~6}
\begin{center}
{\Large {\bfseries ARTICULUS SEXTUS}}\\
{\large UTRUM ANIMA INTELLECTIVA UNIATUR CORPORI MEDIANTIBUS DISPOSITIONIBUS ACCIDENTALIBUS}\\
{\footnotesize II {\itshape SCG}, cap.71; {\itshape De Spirit.~Creat.}, a.3; {\itshape Qu.~de Anima}, a.9; II {\itshape de Anima}, lect.1; VIII {\itshape Metaphys.}, lect.5.}\\
{\Large 第六項\\知性的魂は附帯的態勢を媒介として身体に合一されているか}
\end{center}

\begin{longtable}{p{21em}p{21em}}
{\scshape Ad sextum sic proceditur}. Videtur quod anima intellectiva
uniatur corpori mediantibus aliquibus dispositionibus
accidentalibus. Omnis enim forma est in materia sibi propria et
disposita. Sed dispositiones ad formam sunt accidentia quaedam. Ergo
oportet praeintelligi accidentia aliqua in materia ante formam
substantialem, et ita ante animam, cum anima sit quaedam substantialis
forma.

&

第六項の問題へ、議論は以下のように進められる。知性的魂は附帯的態勢を媒
介として身体に合一されていると思われる。理由は以下の通り。形相はすべて、
自らに固有であり態勢付けられた質料においてある。しかるに、形相への態勢
はある種の附帯性である。ゆえに、実体形相の前に、質料において、何らかの
附帯性が理解される必要がある。そのようにして、魂の前にもそうである。な
ぜなら、魂は、ある種の実体形相だから。
 
\\



2.{\scshape Praeterea}, diversae formae unius speciei requirunt
diversas materiae partes. Partes autem materiae diversae non possunt
intelligi nisi secundum divisionem dimensivarum quantitatum. Ergo
oportet intelligere dimensiones in materia ante formas substantiales,
quae sunt multae unius speciei.

&

 さらに、一つの種に属するさまざまな形相には、質料のさまざまな部分が必
 要である。しかるに質料のさまざまな部分は、量のさまざまな次元にしたがっ
 てでなければ理解されえない。ゆに、一つの種に属する多くのものであると
 ころの実体形相の前には、質料における諸次元が理解される必要がある。
 
\\



3.{\scshape Praeterea}, spirituale applicatur corporali per contactum
virtutis. Virtus autem animae est eius potentia. Ergo videtur quod
anima unitur corpori mediante potentia, quae est quoddam accidens.

&

 さらに、霊的なものは物体的なものに、ちからの接触によって結び付けられ
 る。しかるに魂のちからは、それの能力である。ゆえに、魂は、身体に、能
 力を媒介にして合一されると思われる。そしてその能力は、一種の附帯性で
 ある。
 
\\



{\scshape Sed contra est} quod accidens est posterius substantia
{\itshape et tempore et ratione}, ut dicitur in VII {\itshape
Metaphys}. Non ergo forma accidentalis aliqua potest intelligi in
materia ante animam, quae est forma substantialis.

&

 しかし反対に、附帯性は、『形而上学』第7巻で言われるように、「時間にお
 いても性格においても」実体より後である。ゆえに、何らかの附帯的形相が、
 実体形相である魂の前に、質料において理解されることは不可能である。
 
\\



 {\scshape Respondeo dicendum} quod, si anima uniretur corpori solum
 ut motor, nihil prohiberet, immo magis necessarium esset esse aliquas
 dispositiones medias inter animam et corpus, potentiam scilicet ex
 parte animae, per quam moveret corpus; et aliquam habilitatem ex
 parte corporis, per quam corpus esset ab anima mobile.

&

 解答する。以下のように言われるべきである。もし魂が身体に、たんに動者
 として合一されていたら、魂と身体との間に、何らかの中間の態勢があるこ
 とを妨げるものは何もなかったし、むしろそれが必要だっただろう。すなわ
 ちそれは、魂の側からは、それを通して身体を動かすところの能力であり、
 身体の側からは、それを通して身体が魂によって動かされうるところの何ら
 かの適合性(habilitas)である。
 
\\

 Sed si anima intellectiva unitur corpori ut forma substantialis,
 sicut iam supra dictum est, impossibile est quod aliqua dispositio
 accidentalis cadat media inter corpus et animam, vel inter quamcumque
 formam substantialem et materiam suam.
 
&

しかし、もし、すでに前に述べたとおり、知性的魂が身体に実体形相として合
一されているとすれば、何らかの附帯的な態勢が、身体と魂の間に、あるいは、
何であれ実体的形相とそれの質料との間に、媒介としてはいることは不可能で
ある。
 
\\

 Et huius ratio est quia, cum materia sit in potentia ad omnes actus
 ordine quodam, oportet quod id quod est primum simpliciter in
 actibus, primo in materia intelligatur. 

&

そしてこの理由は以下の通りである。すなわち、質料は、ある種の順番によっ
て、すべての現実態に対して可能態にあるのだから、現実態において端的に第
一のものが、質料において第一に理解される。
 
\\

Primum autem inter omnes actus est esse.  Impossibile est ergo
 intelligere materiam prius esse calidam vel quantam, quam esse in
 actu. Esse autem in actu habet per formam substantialem, quae facit
 esse simpliciter, ut iam dictum est. Unde impossibile est quod
 quaecumque dispositiones accidentales praeexistant in materia ante
 formam substantialem; et per consequens neque ante animam.

&

しかるに、すべての現実態の中で第一のものは存在である。ゆえに、質料を、
それが現実態において存在している(esse in actu)ということよりも先に、熱
いものであるとか、量のあるものであると理解することは不可能である。しか
るに、(質料は)現実態における存在を、実体形相によってもつ。実体形相は、
すでに述べられたとおり、端的に存在させるところのものである。したがって、
どんな附帯的な態勢であれ、実体形相の前に質料において先に存在することは
不可能であり、それは、魂の前も同様である。
 
\\



{\scshape Ad primum ergo dicendum} quod, sicut ex praedictis patet,
forma perfectior virtute continet quidquid est inferiorum formarum. Et
ideo una et eadem existens, perficit materiam secundum diversos
 perfectionis gradus.


&

 第一異論に対しては、それゆえ、以下のように言われるべきである。前に述
 べたことから明らかなとおり、ちからにおいてより完全な形相は、下位の諸
 形相に属するものはなんであれ含んでいる。ゆえに、存在する一つの同一の
 形相が、さまざまな完全性の段階に即して質料を完成させる。
 
\\

 Una enim et eadem forma est per essentiam, per
quam homo est ens actu, et per quam est corpus, et per quam est vivum,
et per quam est animal, et per quam est homo. Manifestum est autem
 quod unumquodque genus consequuntur propria accidentia.

 
&

 たとえば、それによって人間が現実態における存在者であり、物体であり、
 生きるものであり、動物であり、人間であるところの形相は、本質によって
 一つで同一の形相である。しかるに、各々の類には固有の附帯性が伴うこと
 が明らかである。
 
\\

 Sicut ergo materia praeintelligitur perfecta secundum esse ante
intellectum corporeitatis, et sic de aliis; ita praeintelliguntur
accidentia quae sunt propria entis, ante corporeitatem. Et sic
praeintelliguntur dispositiones in materia ante formam, non quantum ad
omnem eius effectum, sed quantum ad posteriorem.

&

ゆえに、ちょうど質料が、物体性の理解の前に、存在において完全なものとし
て理解され、他のものについても同様であるように、物体性の前に、有に属す
る固有なものが、あらかじめ理解される。このようにして、形相の前に、質料
における態勢が理解されるが、それは、それのすべての結果に関してではなく、
より後の結果にかんしてである。
 
\\

{\scshape Ad secundum dicendum} quod dimensiones quantitativae sunt
accidentia consequentia corporeitatem, quae toti materiae
convenit. Unde materia iam intellecta sub corporeitate et
dimensionibus, potest intelligi ut distincta in diversas partes, ut
sic accipiat diversas formas secundum ulteriores perfectionis
gradus. Quamvis enim eadem forma sit secundum essentiam quae diversos
perfectionis gradus materiae attribuit, ut dictum est; tamen secundum
considerationem rationis differt.

&

 第二異論に対しては、以下のように言われるべきである。量的次元は物体性
 に伴う附帯性であり、質料全体に適合する。したがって、すでに物体性と次
 元の元に理解されている質料は、さまざまな部分へと分割されたものとして
 理解されうるのであり、そのようにして、完全性のさらなる段階に即して、
 さまざまな形相を受け取る。つまり、質料にさまざまな完全性の段階を帰属
 させるのは、すでに述べられたとおり、本質において同一の形相だが、しか
 し、理性の考察に即しては、異なる。
 
\\

{\scshape Ad tertium dicendum} quod substantia spiritualis quae unitur
corpori solum ut motor, unitur ei per potentiam vel virtutem. Sed
anima intellectiva corpori unitur ut forma per suum esse. Administrat
tamen ipsum et movet per suam potentiam et virtutem.

&

 第三異論に対しては、以下のように言われるべきである。他だ動者としての
 み物体に合一される霊的実体は、能力ないしちからを通して、それに合一さ
 れる。しかし、知性的魂は、自らの存在によって、形相として身体に合一さ
 れる。ただし、身体を統率し動かすのは、自らの能力とちからを通してであ
 るが。

\end{longtable}
\newpage



\rhead{a.~7}
\begin{center}
{\Large {\bfseries ARTICULUS SEPTIMUS}}\\
{\large UTRUM ANIMA UNIATUR CORPORI ANIMALIS MEDIANTE ALIUQO CORPORE}\\
 {\footnotesize II {\itshape Sent.}, d.1, q.2, a.4, ad 3; II {\itshape SCG}, cap.71; {\itshape De Spirit.~Creat.}, a.3; Qu.~{\itshape de Anima}, a.9; II {\itshape de Anima}, lect.1; VII {\itshape Metaphys.}, lect.5.}\\
{\Large 第七項\\魂は動物の身体に何らかの物体を媒介して合一されるか}
\end{center}

\begin{longtable}{p{21em}p{21em}}
{\scshape Ad septimum sic proceditur}. Videtur quod anima uniatur
corpori animalis mediante aliquo corpore. Dicit enim Augustinus, VII
{\itshape super Gen.~ad Litt}., quod {\itshape anima per lucem, idest
ignem, et aerem, quae sunt similiora spiritui, corpus
administrat}. Ignis autem et aer sunt corpora. Ergo anima unitur
corpori humano mediante aliquo corpore.

&

 第七の問題へ、議論は以下のように進められる。魂は動物の身体に何らかの
 物体を媒介として合一されると思われる。理由は以下の通り。アウグスティ
 ヌスは『創世記逐語注解』第7巻で「魂は光によって、すなわち、霊により似
 ている火と空気によって、身体を治める」と言っている。しかるに火と空気
 は物体である。ゆえに、魂は人間の身体に、何らかの物体を媒介として合一
 される。


\\



2. {\scshape Praeterea}, id quo subtracto solvitur unio aliquorum
unitorum, videtur esse medium inter ea. Sed deficiente spiritu, anima
a corpore separatur. Ergo spiritus, qui est quoddam corpus subtile,
medium est in unione corporis et animae.

&

さらに、何らかの合一体の結合が、それが取り去られると解体するようなもの
は、それらの間の媒介であると思われる。しかるに、霊がなくなると、魂は身
体から切り離される。ゆえに、何らかの微細な物体である霊が、身体と魂の結
合の媒介である。

\\



3. {\scshape Praeterea}, ea quae sunt multum distantia, non uniuntur
nisi per medium. Sed anima intellectiva distat a corpore et quia est
incorporea, et quia est incorruptibilis. Ergo videtur quod uniatur ei
mediante aliquo quod sit corpus incorruptibile. Et hoc videtur esse
aliqua lux caelestis, quae conciliat elementa et redigit in unum.

&

さらに、大きく距たっているものどもは、媒介がなければ合一されない。しか
るに、知性的魂は、非物体的であることと、不滅であることから、身体から距
たっている。ゆえに、それに、何らかの不滅の物体が媒介となって合一されて
いると思われる。そしてそれは、なんらかの天の光であり、それが諸元素を調
和させ、一つのものにしている。

\\



{\scshape Sed contra est} quod philosophus dicit, in II de anima, quod
{\itshape non oportet quaerere si unum est anima et corpus, sicut
neque ceram et figuram}. Sed figura unitur cerae nullo corpore
mediante. Ergo et anima corpori.

&

しかし反対に、哲学者は『デ・アニマ』第2巻で「蝋と形(が一つかどうか)
を問わなくていいのと同様に、魂と身体が一つかどうかを問う必要はない」と
言っている。しかるに、形は蝋に、いかなる物体も媒介とせずに合一されてい
る。ゆえに魂もまた身体に(いかなる物体も媒介とせずに合一されている)。

\\



 {\scshape Respondeo dicendum} quod si anima, secundum Platonicos,
 corpori uniretur solum ut motor, conveniens esset dicere quod inter
 animam hominis, vel cuiuscumque animalis, et corpus aliqua alia
 corpora media intervenirent, convenit enim motori aliquid distans per
 media magis propinqua movere.

&

 解答する。以下のように言われるべきである。もし魂が、プラトン派の人々
 に従って、ただ動者としてのみ身体に合一されていたとしたならば、人間の、
 あるいはどんな動物の、魂と身体の間に、媒介となる別の物体が介在すると
 言うことが適切だっただろう。なぜなら、動者には、距たっている何かを、
 より近い媒介によって動かすことが適しているからである。

\\



 Si vero anima unitur corpori ut forma, sicut iam dictum est,
 impossibile est quod uniatur ei aliquo corpore mediante. Cuius ratio
 est, quia sic dicitur aliquid unum, quomodo et ens. Forma autem per
 seipsam facit rem esse in actu, cum per essentiam suam sit actus; nec
 dat esse per aliquod medium.

&

他方で、すでに述べられたとおり、魂が身体に形相として合一されているとす
れば、何らかの物体が媒介となってそれに合一されることは不可能である。そ
の理由は以下の通りである。何かは、それが有であるしかたに従って、一であ
ると言われる。しかるに、形相は自らによって、事物を現実態において存在さ
せる。というのも、(形相は)自らの本質によって現実態であり、何らかの媒
介を通して存在を与えるのではないからである。

\\

 Unde unitas rei compositae ex materia et forma est per ipsam formam,
 quae secundum seipsam unitur materiae ut actus eius. Nec est aliquid
 aliud uniens nisi agens, quod facit materiam esse in actu, ut dicitur
 in VIII {\itshape Metaphys}. Unde patet esse falsas opiniones eorum
 qui posuerunt aliqua corpora esse media inter animam et corpus
 hominis. Quorum quidam Platonici dixerunt quod anima intellectiva
 habet corpus incorruptibile sibi naturaliter unitum, a quo nunquam
 separatur, et eo mediante unitur corpori hominis corruptibili.

&

 従って、質料と形相から複合された事物の一性は、形相それ自体によるので
 あり、その形相は、自らに即して、質料の現実態として質料に合一されてい
 る。また、『形而上学』第8巻で言われているように、質料を現実態において
 存在させる作用者以外に、何か別のものが合一させることはない。このこと
 から、魂と人間の身体の間に何らかの物体が媒介していると考えた人々の意
 見が誤っていることが明らかである。そのような人々のうち、あるプラトン
 派の人々は、知性的魂が、自らに本性的に合一された不滅の物体を持ってい
 て、それから決して切り離されず、それを媒介として、可滅的な人間の身体
 に合一されていると語った。

\\


 --- Quidam vero dixerunt quod unitur corpori mediante spiritu
corporeo. Alii vero dixerunt quod unitur corpori mediante luce, quam
dicunt esse corpus, et de natura quintae essentiae, ita quod anima
vegetabilis unitur corpori mediante luce caeli siderei; anima vero
sensibilis, mediante luce caeli crystallini; anima vero
intellectualis, mediante luce caeli Empyrei. Quod fictitium et
derisibile apparet, tum quia lux non est corpus; tum quia quinta
essentia non venit materialiter in compositionem corporis mixti, cum
sit inalterabilis, sed virtualiter tantum; tum etiam quia anima
immediate corpori unitur ut forma materiae.

&

他方、ある人々は、(魂が)身体に物体的な霊を媒介として合一されると言っ
た。また別の人々は、彼らが物体だと言い、第五本質の本性だと言う光を媒介
として、身体に合一されると言った。それは次のようにしてである。すなわち、
植物的魂は恒星天の光が媒介となって身体に合一され、感覚的魂は水晶天の光
が媒介となり、知性的魂は最高天の光が媒介となって(身体に合一される)。この
ことが、作り話で笑うべきことであることは明らかである。その理由の一つに
は、光は物体でないし、また、第五本質は不変なので、質料的に混合物体の複
合には入らず、ただ能力的にのみそうなる。また、魂は身体に、形相が質料に
合一されるようにして、媒介なしに身体に合一されるからである。
 


\\



{\scshape Ad primum ergo dicendum} quod Augustinus loquitur de anima
inquantum movet corpus, unde utitur verbo administrationis. Et verum
est quod partes grossiores corporis per subtiliores movet. Et primum
instrumentum virtutis motivae est spiritus, ut dicit philosophus in
libro {\itshape de Causa Motus Animalium}.

&

 第一異論に対しては、それゆえ、以下のように言われるべきである。アウグ
 スティヌスは、身体を動かすかぎりにおける魂について語っている。このこ
 とから、統治という言葉を使っている。そして、(魂が)身体の大きな部分
 を、より微細な部分を通して動かすことは真である。そして哲学者が『動物
 の運動の原因について』という書物で言うように、動かしうるちからの第一
 の道具は、霊である。

\\



{\scshape Ad secundum dicendum} quod, subtracto spiritu, deficit unio
animae ad corpus, non quia sit medium; sed quia tollitur dispositio
per quam corpus est dispositum ad talem unionem. Est tamen spiritus
medium in movendo, sicut primum instrumentum motus.

&

第二異論については、以下のように言われるべきである。霊がなくなると魂と
身体の合一が損なわれるのは、霊が媒介だからではなく、身体がそのような合
一に向けて態勢付けられている態勢がなくなるからである。しかし、霊は、運
動の第一の道具として、動かすことにおける媒介である。

\\

{\scshape Ad tertium dicendum} quod anima distat quidem a corpore
plurimum, si utriusque conditiones seorsum considerentur, unde si
utrumque ipsorum separatim esse haberet, oporteret quod multa media
intervenirent. Sed inquantum anima est forma corporis, non habet esse
seorsum ab esse corporis; sed per suum esse corpori unitur
immediate. Sic enim et quaelibet forma, si consideretur ut actus,
habet magnam distantiam a materia, quae est ens in potentia tantum.

&

 第三異論については、以下のように言われるべきである。もし、どちらの状
 態も別々に考察されるならば、魂はたしかに身体から大いに距たっている。
 それゆえ、もし、これらのどちらもが別々に存在を持っていたならば、多く
 の媒介が介在したであろう。しかし、魂が身体の形相である限りにおいて、
 魂は身体の存在と別の存在を持つわけではない。むしろ、自らの存在によっ
 て身体に直接的に合一されている。実際、どんな形相も、現実態として考察
 されるならば、可能態にのみある有である質料から、大きな隔たりを持って
 いる。
\end{longtable}
\newpage
 
\rhead{a.~8}
\begin{center}
{\Large {\bfseries ARTICULUS OCTAVUS}}\\
{\large UTRUM ANIMA SIT TOTA IN QUALIBET PARTE CORPORIS}\\
{\footnotesize I {\itshape Sent.}, d.8, q.5, a.3; II {\itshape SCG}, cap.72; {\itshape De Spirit.~Creat.}, a.4; {\itshape Qu.~de Anima}, a.10.}\\
{\Large 第八項\\魂は身体のどの部分にも全体としてあるか}
\end{center}

\begin{longtable}{p{21em}p{21em}}
 {\scshape Ad octavum sic proceditur}. Videtur quod anima non sit tota
 in qualibet parte corporis. Dicit enim philosophus, in libro
 {\itshape de Causa Motus Animalium}, {\itshape non opus est in
 unaquaque corporis parte esse animam; sed in quodam principio
 corporis existente, alia vivere; eo quod simul nata sunt facere
 proprium motum per naturam}.

 &

 第八項の問題へ、議論は以下のように進められる。魂は、身体のどの部分に
 置いても全体としてあるわけではないと思われる。理由は以下の通り。哲学
 者は『動物の運動の原因について』という書物の中で以下のように言ってい
 る。「身体のどの部分においても魂がある必要はなく、魂が、身体のある根
 源に存在することで、他の部分が生きる。なぜなら、同時に自然本性によっ
 て固有の運動をするように生まれているからである」。
 
\\



2. {\scshape Praeterea}, anima est in corpore cuius est actus. Sed est
actus corporis organici. Ergo non est nisi in corpore organico. Sed
non quaelibet pars corporis hominis est corpus organicum. Ergo anima
non est in qualibet parte corporis tota.

 &

さらに、魂は、身体の現実態として身体においてある。しかるに、それは、器
官がある身体の現実態である。ゆえに、器官がある身体においてしか存在しな
い。しかるに、人間の身体のどの部分も器官がある身体であるわけではない。
ゆえに、身体のどの部分にも魂があるわけではない。
 
\\



3. {\scshape Praeterea}, in II {\itshape de Anima} dicitur quod sicut
se habet pars animae ad partem corporis, ut visus ad pupillam, ita
anima tota ad totum corpus animalis. Si igitur tota anima est in
qualibet parte corporis, sequitur quod quaelibet pars corporis sit
animal.

 &

 さらに、『デ・アニマ』第2巻で、視覚が眼に対するように、魂は身体の部分
 に関係するが、ちょうどそのように、魂全体は、動物の身体全体に関係する、
 と言われている。ゆえに、もし、魂全体が身体のどの部分にもあるならば、
 身体のどの部分も動物であることになるだろう。

\\



4. {\scshape Praeterea}, omnes potentiae animae in ipsa essentia
animae fundantur. Si igitur anima tota est in qualibet parte corporis,
sequitur quod omnes potentiae animae sint in qualibet corporis parte,
et ita visus erit in aure, et auditus in oculo. Quod est inconveniens.

 &

 さらに、魂のすべての能力は、自らの本質において、魂に基づいている。ゆ
 えに、魂全体が身体のどの部分にも存在するならば、魂のすべての能力が、
 身体のどの部分にも存在し、視覚が耳に、聴覚が眼にあることになる。しか
 しこれは不適当である。
\\



5. {\scshape Praeterea}, si in qualibet parte corporis esset tota
anima, quaelibet pars corporis immediate dependeret ab anima. Non ergo
una pars dependeret ab alia, nec una pars esset principalior quam
alia, quod est manifeste falsum. Non ergo anima est in qualibet parte
corporis tota.

 &

 さらに、もし、身体のどの部分にも魂が全体として存在するならば、身体の
 どの部分も直接的に魂に依存しただろう。ゆえに、一つの部分が他の部分に
 依存したり、ある部分多田の部分よりも根源的であるということはなかった
 だろう。これは明らかに偽である。ゆえに、魂が、身体のどの部分において
 も全体としてあるわけではない。
 
\\



 {\scshape Sed contra est} quod Augustinus dicit, in VI {\itshape de
 Trin}., quod {\itshape anima in quocumque corpore et in toto est tota, et in
 qualibet eius parte tota est}.

 &

 しかし反対に、アウグスティヌスは『三位一体論』第6巻で「魂は、どの身体
 においても、全体の中に全体としてあり、身体のどの部分においても全体と
 してある」と言う。

\\



 {\scshape Respondeo dicendum} quod, sicut in aliis iam dictum est, si
 anima uniretur corpori solum ut motor, posset dici quod non esset in
 qualibet parte corporis, sed in una tantum, per quam alias
 moveret. Sed quia anima unitur corpori ut forma, necesse est quod sit
 in toto, et in qualibet parte corporis. 

&

 解答する。以下のように言われるべきである。他の箇所ですでに述べられた
 ように、もし魂が身体に、たんに動者として合一されていたならば、身体の
 どの部分にもあるのではなく、むしろただ一つの部分にあり、その部分を通
 してたの部分を動かす、と言われえただろう。しかし、魂は身体に形相とし
 て合一されているので、身体の全体の中に、そしてどの部分においてもある
 ことが必要である。
 
 
\\


 Non enim est forma corporis accidentalis, sed substantialis.
 Substantialis autem forma non solum est perfectio totius, sed
 cuiuslibet partis. Cum enim totum consistat ex partibus, forma totius
 quae non dat esse singulis partibus corporis, est forma quae est
 compositio et ordo, sicut forma domus, et talis forma est
 accidentalis.

&

 理由は以下の通り。(魂は)身体の附帯形相ではなく、実体形相である。し
 かるに、実体形相は、全体の完全性であるだけでなく、どの部分の完全性で
 もある。というのも、全体は部分から構成されるので、身体の個々の部分に
 存在を与えるのでない全体の形相は、複合や秩序の形相である。たとえば、
 家の形相がそうであるように。そしてそのような形相は附帯的である。
 
\\


 Anima vero est forma substantialis,
 unde oportet quod sit forma et actus non solum totius, sed cuiuslibet
 partis. Et ideo, recedente anima, sicut non dicitur animal et homo
 nisi aequivoce, quemadmodum et animal pictum vel lapideum; ita est de
 manu et oculo, aut carne et osse, ut philosophus dicit.

&

これに対して、魂は実体形相なので、全体だけでなく、どの部分の形相でもあ
り現実態でもなければならない。ゆえに、魂がなくなると、ちょうど、描かれ
た動物や石で作られた動物と言う場合のように、異義的にでなければ動物や人
間と言われないように、そのように、手や眼、肉や骨についても同様である。
これは哲学者が言っているとおりである。
 
\\


 Cuius signum est, quod nulla pars corporis habet proprium opus, anima
 recedente, cum tamen omne quod retinet speciem, retineat operationem
 speciei. Actus autem est in eo cuius est actus. Unde oportet animam
 esse in toto corpore, et in qualibet eius parte.

&

 この印は、身体のどの部分も、魂がなくなると、固有の働きを持たないが、
 種を保持するものはすべて、その種の働きを保持するということにある。し
 かるに、作用は、作用がそれに属するものの中にある。したがって、魂は身
 体全体の中にあり、しかもそのどの部分においてもある。
 
\\

 Et quod tota sit in qualibet parte eius, hinc considerari potest,
 quia, cum totum sit quod dividitur in partes, secundum triplicem
 divisionem est triplex totalitas. Est enim quoddam totum quod
 dividitur in partes quantitativas, sicut tota linea vel totum
 corpus. Est etiam quoddam totum quod dividitur in partes rationis et
 essentiae; sicut definitum in partes definitionis, et compositum
 resolvitur in materiam et formam. Tertium autem totum est potentiale,
 quod dividitur in partes virtutis.

 &

 また、(魂の)全体が身体のどの部分においてもあるということは、以下の
 ことから考察されうる。すなわち、全体は部分に分割されるものなので、三
 通りの分割に即して三通りの全体がある。つまり、線分全体や立体全体のよ
 うに、量的な部分へと分割される全体がある。さらに、被定義項が定義の部
 分へと分割され、複合体が質料と形相に分割されるように、概念や本質の部
 分へと分割される全体がある。第三の全体は、ちからの部分へと分割される、
 能力的なものである。

\\

 Primus autem totalitatis modus non convenit formis, nisi forte per
 accidens; et illis solis formis, quae habent indifferentem
 habitudinem ad totum quantitativum et partes eius. Sicut albedo,
 quantum est de sui ratione, aequaliter se habet ut sit in tota
 superficie et in qualibet superficiei parte; et ideo, divisa
 superficie, dividitur albedo per accidens.

&

ところで、全体性の第一のあり方は、附帯的にでなければ、形相に適合しない。
 それも、量的な全体とその部分に、違いのない関係をもつ形相だけである。
 たとえば、白性は、自らの性格にかんする限り、表面全体と、表面のどの部
 分においても、等しく関係するように。それゆえ、表面が分割されると、白
 性が、附帯的に分割される。
  
\\

 Sed forma quae requirit diversitatem in partibus, sicut est anima, et
 praecipue animalium perfectorum, non aequaliter se habet ad totum et
 partes, unde non dividitur per accidens per divisionem quantitatis.

&

しかし、魂のように、部分における多様性を必要とする形相、それもとくに完
全な動物の魂は、全体と部分に対して等しく関係しない。それゆえ、量の分割
によって、附帯的に分割されない。
 
 
\\

 Sic ergo totalitas
 quantitativa non potest attribui animae nec per se nec per
 accidens. Sed totalitas secunda, quae attenditur secundum rationis et
 essentiae perfectionem, proprie et per se convenit formis. Similiter
 autem et totalitas virtutis, quia forma est operationis principium.

&

ゆえに、このようにして、量的な全体は、自体的にも附帯的にも魂に帰属され
得ない。しかし、第二の全体性、すなわち、概念と本質の完全性に即して見出
される全体は、固有にかつ自体的に、形相に適合する。同様に、ちからの完全
性もまたそうである。なぜなら、形相は働きの根源であるから。
 
\\

 Si ergo quaereretur de albedine, utrum esset tota in tota superficie
 et in qualibet eius parte, distinguere oporteret. Quia si fiat mentio
 de totalitate quantitativa, quam habet albedo per accidens, non tota
 esset in qualibet parte superficiei. Et similiter dicendum est de
 totalitate virtutis magis enim potest movere visum albedo quae est in
 tota superficie, quam albedo quae est in aliqua eius particula. Sed
 si fiat mentio de totalitate speciei et essentiae, tota albedo est in
 qualibet superficiei parte.

&

ゆえに、もし、白性について、その全体が表面全体とそのどの部分にもあるか
と問われたならば、(その答えは)区別しなければならなかっただろう。なぜ
なら、もし、白性が附帯的にもつ量的な全体について言っているのであれば、
(白性の)全体は表面のどの部分においてもあるわけではなかっただろう。ま
た、ちからの全体性についても同様に言われるべきである。というのも、表面
全体にある白性の方が、表面のある小さい部分にある白性よりも、より視覚を
動かすことができるからである。しかし、もし、種と本質の全体性について言
われているのであれば、白性全体は、表面のどの部分においてもある。
 
\\

 Sed quia anima totalitatem quantitativam non habet, nec per se nec
 per accidens, ut dictum est; sufficit dicere quod anima tota est in
 qualibet parte corporis secundum totalitatem perfectionis et
 essentiae; non autem secundum totalitatem virtutis. Quia non secundum
 quamlibet suam potentiam est in qualibet parte corporis; sed secundum
 visum in oculo, secundum auditum in aure, et sic de aliis.

 &

 しかし、上述のように、魂は、自体的にも附帯的にも量的な全体性をもたな
 いので、魂全体が、完全性と本質の全体性に即して、身体のどの部分におい
 てもあるが、ちからの全体性についてはそうでない、と言うことで十分であ
 る。というのも、自らのどの能力に即しても、身体のどの部分においてもあ
 る、というわけではなく、むしろ、視覚に即しては眼に、聴覚に即しては耳
 に、その他についても同様だからである。

\\

 Tamen attendendum est quod, quia anima requirit diversitatem in
 partibus, non eodem modo comparatur ad totum et ad partes, sed ad
 totum quidem primo et per se, sicut ad proprium et proportionatum
 perfectibile; ad partes autem per posterius, secundum quod habent
 ordinem ad totum.

 &

 しかし、以下のことが注意されるべきである。すなわち、魂は部分における
 多様性を必要とするので、同じしかたで全体と部分に関係するのではなく、
 固有でそれに比例して完成させられうるものとしての全体に対して第一に自
 体的に関係し、部分に対しては、それが全体に対して秩序をもつかぎりにお
 いて、より後なるしかたで関係する。

\\



 {\scshape Ad primum ergo} dicendum quod philosophus loquitur de
 potentia motiva animae.

 &

 第一異論に対しては、それゆえ、以下のように言われるべきである。
 哲学者は、魂の、動かしうる能力について語っている。
 
\\



 {\scshape Ad secundum dicendum} quod anima est actus corporis
 organici, sicut primi et proportionati perfectibilis.

 &

 第二異論に対しては、以下のように言われるべきである。魂は、器官がある
 身体の現実態だが、それは、第一の、比例して完成させられうるものの現実
 態である。

\\



 {\scshape Ad tertium dicendum} quod animal est quod componitur ex
 anima et corpore toto, quod est primum et proportionatum eius
 perfectibile. Sic autem anima non est in parte. Unde non oportet quod
 pars animalis sit animal.

 &

 第三異論に対しては、以下のように言われるべきである。動物は、魂と身体
 全体とから複合されたものだが、身体全体は、第一の、比例して完成させら
 れうるものである。しかるに、そのようなかたちで魂は部分の中にあるわけ
 ではない。このことから、動物の部分が動物である必要はない。

\\



 {\scshape Ad quartum dicendum} quod potentiarum animae quaedam sunt
 in ea secundum quod excedit totam corporis capacitatem, scilicet
 intellectus et voluntas, unde huiusmodi potentiae in nulla parte
 corporis esse dicuntur. Aliae vero potentiae sunt communes animae et
 corpori, unde talium potentiarum non oportet quod quaelibet sit in
 quocumque est anima; sed solum in illa parte corporis quae est
 proportionata ad talis potentiae operationem.

 &

 第四異論に対しては、以下のように言われるべきである。魂の諸能力の中で、
 ある能力は、身体の受容力全体を超えるかぎりで、魂の中にある。例えばそ
 れは知性と意志であり、従ってこのような能力は、どの身体器官の中にある
 とも言われない。一方、他の能力は、魂と身体とに共通し、したがって、そ
 のような能力には、どんな魂の能力も身体のどの場所にも在ると言うことは
 必要ない。むしろ、そのような能力の働きに比例する身体の部分においてだ
 け在る。

\\



 {\scshape Ad} quintum dicendum quod una pars corporis dicitur esse
 principalior quam alia, propter potentias diversas quarum sunt organa
 partes corporis. Quae enim est principalioris potentiae organum, est
 principalior pars corporis, vel quae etiam eidem potentiae
 principalius deservit.

 &

 第五異論に対しては、以下のように言われるべきである。身体のある部分が、
 他の部分よりも根源的だと言われるのは、身体の器官となる諸部分が、それ
 に属する多様な能力のためである。じっさい、器官のより根源的な能力に属
 するもの、あるいは、同じ能力に属するものでもより根源的に働くものは、
 身体のより根源的な部分である。

\\

 
\end{longtable}
\end{document}



