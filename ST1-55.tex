\documentclass[10pt]{jsarticle} % use larger type; default would be 10pt
%\usepackage[utf8]{inputenc} % set input encoding (not needed with XeLaTeX)
%\usepackage[round,comma,authoryear]{natbib}
%\usepackage{nruby}
\usepackage{okumacro}
\usepackage{longtable}
%\usepqckage{tablefootnote}
\usepackage[polutonikogreek,english,japanese]{babel}
%\usepackage{amsmath}
\usepackage{latexsym}
\usepackage{color}

%----- header -------
\usepackage{fancyhdr}
\lhead{{\it Summa Theologiae} I, q.~55}
%--------------------

\bibliographystyle{jplain}

\title{{\bf PRIMA PARS}\\{\HUGE Summae Theologiae}\\Sancti Thomae
Aquinatis\\{\sffamily QUEAESTIO QUINQUAGESIMAQUINTA}\\DE MEDIO COGNITIONIS ANGELICAE}
\author{Japanese translation\\by Yoshinori {\sc Ueeda}}
\date{Last modified \today}


%%%% コピペ用
%\rhead{a.~}
%\begin{center}
% {\Large {\bf }}\\
% {\large }\\
% {\footnotesize }\\
% {\Large \\}
%\end{center}
%
%\begin{longtable}{p{21em}p{21em}}
%
%&
%
%
%\\
%\end{longtable}
%\newpage



\begin{document}
\maketitle
\pagestyle{fancy}

\begin{center}
{\Large 第五十五問\\天使の認識の媒体について}
\end{center}

\begin{longtable}{p{21em}p{21em}}

Consequenter quaeritur de medio cognitionis angelicae. Et circa hoc
 quaeruntur tria. 
\begin{enumerate}
 \item utrum utrum Angeli cognoscant omnia per suam substantiam, vel per
       aliquas species.
 \item si per species, utrum per species connaturales, vel per species a
       rebus acceptas.
 \item utrum Angeli superiores cognoscant per species magis universales,
       quam inferiores.
\end{enumerate}

&

続いて、天使の認識の媒体について問われる。これをめぐって、3つのことが問
 われる。

\begin{enumerate}
 \item 天使たちは、万物を自分の実体によって認識するか、それとも、なんら
       かの形象によって認識するか。
 \item もし、形象によってだとすると、生得的な形象によってか、それとも、
       事物から受け取られた形象によってか。
 \item 上位の天使たちは、下位の天使たちよりも普遍的な形象によって認識す
       るか。
\end{enumerate}

\end{longtable}
\newpage


\rhead{a.~1}
\begin{center}
 {\Large {\bf ARTICULUS PRIMUS}}\\
 {\large UTRUM ANGELI COGNOSCANT OMNIA PER SUAM SUBSTANTIAM}\\
 {\footnotesize Infra, q.~84, a.~2; q.~87, a.~1; I$^a$II$^{ae}$, q.~50,
 a.~6; q.~51, a.~1, ad 2; II {\itshape Sent.}, d.~3, part.~2, q.~2,
 a.~1; III, d.~14, a.~1, qu$^a$ 2; II {\itshape SCG.}, cap.~98;
 {\itshape De Verit.}, q.~8, a.~8.}\\
 {\Large 第一項\\天使たちは万物を自らの実体によって認識するか}
\end{center}

\begin{longtable}{p{21em}p{21em}}


{\huge A}{\scshape d primum sic proceditur}. Videtur quod Angeli
cognoscant omnia per suam substantiam. Dicit enim Dionysius, {\scshape
 vii} cap.~{\itshape de Div.~Nom.}, quod Angeli sciunt ea quae {\itshape sunt in terra, secundum propriam
naturam mentis}. Sed natura Angeli est eius essentia. Ergo Angelus per
suam essentiam res cognoscit.


&

第一の問題へ、議論は以下のように進められる。
天使たちは、万物を、自らの実体によって認識すると思われる。理由は以下の通
 り。
ディオニュシウスは『神名論』7章で、天使たちは「大地の中にあるものどもを、
 自らに固有の精神の本性にしたがって」知る、と述べている。ところで、天使
 の本性は、その本質である。ゆえに、天使は自らの本質によって、諸事物を認
 識する。

\\


{\scshape 2 Praeterea}, secundum philosophum, in XII {\itshape Metaphys}., et in
III {\itshape de Anima}, {\itshape in his quae sunt sine materia, idem est intellectus et
quod intelligitur}. Id autem quod intelligitur est idem intelligenti
ratione eius quo intelligitur. Ergo in his quae sunt sine materia, sicut
sunt Angeli, id quo intelligitur est ipsa substantia intelligentis.

&

さらに、『形而上学』12巻と、『デ・アニマ』3巻の哲学者によれば、質料をも
 たないものどもにおいては、知性と知性認識されるものとは同一である。とこ
 ろで、知性認識されるものは、それによって認識されるところのもの〔=可知
 的形象〕の性格の点で、知性認識するものと同一である。ゆえに、天使のよう
 に質料をもたないものどもにおいて、それによって知性認識されるところのも
 のは、知性認識するものの実体それ自体である。


\\


{\scshape 3 Praeterea}, omne quod est in altero, est in eo per modum
eius in quo est. Sed Angelus habet naturam intellectualem. Ergo quidquid
est in ipso, est in eo per modum intelligibilem. Sed omnia sunt in eo,
quia inferiora in entibus sunt in superioribus essentialiter, superiora
vero sunt in inferioribus participative; et ideo dicit Dionysius, {\scshape iv}
cap.~{\itshape de Div.~Nom}., quod Deus {\itshape tota in totis congregat}, idest omnia in
omnibus. Ergo Angelus omnia in sua substantia cognoscit.

&

さらに、すべて他のものXの中にあるものは、その中にあるところのもの(X)のあ
 り方によってある。ところで、天使は、知性的本性をもつ。ゆえに、何であれ
 天使の中にあるものは、可知的なあり方で、その中にある。ところで、万物は
 天使の中にある。なぜなら、存在するものどもの中で、より下位のものどもは、
 より上位のものどもの中に、本質的にあり、他方、より上位のものどもは、よ
 り下位のものどもの中に、分有されるかたちであるからである。それゆえ、ディ
 オニュシウスもまた、『神名論』4章で、神は全体を全体において、すなわち、
 万物を万物において集める、と述べている。ゆえに、天使は万物を自分の実体
 において認識する。

\\


{\scshape  Sed contra est} quod Dionysius dicit, in eodem capite, quod
Angeli {\itshape illuminantur rationibus rerum}. Ergo cognoscunt per rationes
rerum, et non per propriam substantiam.

&
しかし反対に、ディオニュシウスは同じ章で、天使たちは諸事物の理拠によって
 証明される、と述べている。ゆえに、彼らは、諸事物の理拠によって認識する
 のであり、固有の実体によってではない。

\\


{\scshape Respondeo dicendum} quod illud quo intellectus intelligit,
comparatur ad intellectum intelligentem ut forma eius, quia forma est
quo agens agit. Oportet autem, ad hoc quod potentia perfecte compleatur
per formam, quod omnia contineantur sub forma, ad quae potentia se
extendit. Et inde est quod in rebus corruptibilibus forma non perfecte
complet potentiam materiae, quia potentia materiae ad plura se extendit
quam sit continentia formae huius vel illius. 


&

解答する。以下のように言われるべきである。
知性がそれによって知性認識するところのものは、知性認識する知性にたいして、
 その形相として関係する。なぜなら、形相とは、それによって作用者が作用す
 るところのものだからである。ところで、能力が形相によって完全に満たされ
 るためには、その能力が及ぶところのすべてのものが、その形相のもとに含まれ
 ていなければならない。したがって、可滅的諸事物において、形相は、質料の
 能力を完全には満たしていない。というのも、質料の能力は、この形相やあの
 形相に含まれるものを越えて複数のものへと及ぶからである。

\\

Potentia autem
intellectiva Angeli se extendit ad intelligendum omnia, quia obiectum
intellectus est ens vel verum commune. Ipsa autem essentia Angeli non
comprehendit in se omnia, cum sit essentia determinata ad genus et ad
speciem. Hoc autem proprium est essentiae divinae, quae infinita est, ut
in se simpliciter omnia comprehendat perfecte. Et ideo solus Deus
cognoscit omnia per suam essentiam. Angelus autem per suam essentiam non
potest omnia cognoscere; sed oportet intellectum eius aliquibus
speciebus perfici ad res cognoscendas.

&

さて、天使の知性能力は、万物を知性認識することまで及ぶ。というのも、知性
 の対象は、共通の有ないし真だからである。ところで、天使の本質それ自体は、
 自らの中に万物を含んでいない。というのも、それは類や種に限定された本質
 だからである。これに対して、その中に端的に万物を完全に含むことは、無限
 である神の本質に固有である。ゆえに、神だけが、万物を自分の本質によって
 認識する。他方、天使は、自分の本質によって万物を認識することができず、
 むしろ、天使の知性は、諸事物を認識するために、なんらかの形象によって完
 成されなければならない。

\\


{\scshape Ad primum ergo dicendum} quod, cum dicitur Angelum secundum
suam naturam res cognoscere, ly {\itshape secundum} non determinat medium
cognitionis, quod est similitudo cogniti; sed virtutem cognoscitivam,
quae convenit Angelo secundum suam naturam.

&

第一異論に対しては、それゆえ、以下のように言われるべきである。
天使が自らの本性によって諸事物を認識すると言われるとき、この「よって」は、
 認識の媒体を限定するのではなく、認識能力を限定する。前者は認識されるも
 のの類似だが、後者は、自らの本性によって天使に適合する。

\\


{\scshape Ad secundum dicendum} quod, sicut sensus in actu est sensibile
in actu, ut dicitur in III {\itshape de Anima}, non ita quod ipsa vis sensitiva sit
ipsa similitudo sensibilis quae est in sensu, sed quia ex utroque fit
unum sicut ex actu et potentia; ita et intellectus in actu dicitur esse
intellectum in actu, non quod substantia intellectus sit ipsa similitudo
per quam intelligit, sed quia illa similitudo est forma eius. Idem est
autem quod dicitur, {\itshape in his quae sunt sine materia, idem est intellectus
et quod intelligitur}, ac si diceretur quod {\itshape intellectus in actu est
intellectum in actu}, ex hoc enim aliquid est intellectum in actu quod
est immateriale.

&

第二異論に対しては、次のように言われるべきである。
『デ・アニマ』3巻で言われるように、現実態における感覚は、現実態における
 可感的なものだが、それは、感覚しうる力それ自体が、感覚の中にある、可感
 的なものの類似だからではなく、現実態と可能態とからであるようにして、両
 者から一つのものが生じるためである。同じように、現実態における知性が、
 現実態における「知性認識されたもの」であるというのは、知性の実体が、そ
 れによって知性認識するところの類似そのものだからではなく、その類似が、
 知性の形相だからである。ところで、「質料のないものどもにおいて、知性と、
 知性認識されるものとは同一である」と言われることは、「現実態における知
 性は現実態における「知性認識されたもの」である」と言われたとしても同じ
 ことである。つまり、あるものは、それが現実に非質料的であることに基づい
 て、現実に、知性認識されたものとなる。

\\


{\scshape Ad tertium dicendum} quod ea quae sunt infra Angelum, et ea
quae sunt supra ipsum, sunt quodammodo in substantia eius, non quidem
perfecte, neque secundum propriam rationem, cum Angeli essentia, finita
existens, secundum propriam rationem ab aliis distinguatur; sed secundum
quandam rationem communem. In essentia autem Dei sunt omnia perfecte et
secundum propriam rationem, sicut in prima et universali virtute
operativa, a qua procedit quidquid est in quacumque re vel proprium vel
commune. Et ideo Deus per essentiam suam habet propriam cognitionem de
rebus omnibus, non autem Angelus, sed solam communem.


&

第三異論に対しては、次のように言われるべきである。天使の下位にあるものども
や、天使の上位にあるものどもは、あるかたちで天使の実体の中にあるが、しか
 し、完全にあるのでも、固有の性格に即してあるのでもない。なぜなら、天使
 の本質は、有限なものとして存在するので、それ固有の性格に即して他のもの
 から区別されるからである。しかし、なんらかの共通の性格に即しては〔区別
 されない〕。これに対して、神の本質の中には、万物が、完全に、固有の性格
 に即して存在する。それは、そこから、固有であれ共通であれ、どんな事物の
 中にであれ存在するあらゆるものが生み出されるところの、第一の普遍的な作
 用的力の中に存在するという意味である。ゆえに、神は、自分の本質を通して、
 すべての事物についての固有の認識を持つが、天使はそうでなく、ただ共通的
 なものの認識を持つだけである。

\end{longtable}
\newpage


\rhead{a.~2}
\begin{center}
 {\Large {\bf ARTICULUS SECUNDUS}}\\
 {\large UTRUM ANGELI INTELLIGANT PER SPECIES A REBUS ACCEPTAS}\\
 {\footnotesize II {\itshape Sent.}, d.~3, part.~2, q.~2, a.~1, ad 2; II
 {\itshape SCG}, cap.~96; {\itshape De Verit.}, q.~8, a.~9.}\\
 {\Large 第二項\\天使は事物から受け取った形象によって知性認識するか}
\end{center}

\begin{longtable}{p{21em}p{21em}}

{\huge A}{\scshape d secundum sic proceditur}. Videtur quod Angeli
intelligant per species a rebus acceptas. Omne enim quod intelligitur,
per aliquam sui similitudinem in intelligente intelligitur. Similitudo
autem alicuius in altero existens, aut est ibi per modum exemplaris, ita
quod illa similitudo sit causa rei, aut est ibi per modum imaginis, ita
quod sit causata a re. Oportet igitur quod omnis scientia intelligentis
vel sit causa rei intellectae, vel causata a re. Sed scientia Angeli non
est causa rerum existentium in natura, sed sola divina scientia. Ergo
oportet quod species per quas intelligit intellectus angelicus, sint a
rebus acceptae.

&

第二項の問題へ、議論は以下のように進められる。
 天使は、事物から受け取った形象によって知性認識するのではないと思われる。
 理由は以下の通り。
 全て、知性認識されるものは、知性認識するものの中にある、自らについての
 何らかの類似を通して知性認識される。
 ところで、他のものの中に存在するあるものの類似は、範型というかたちでそ
 こにあるか、あるいは、像というかたちでそこにあるかのどちらかである。前
 者の場合、その類似が事物の原因であり、後者の場合、それは事物によって原
 因されたものである。
 それゆえ、知性認識するものがもつ全ての知は、知性認識された事物の原因で
 あるか、あるいは、事物によって原因されたものであるかのどちらかでなけれ
 ばならない。ところが、天使の知は、自然の中に存在する諸事物の原因でなく、
 そのようなものはただ神の知だけである。ゆえに、天使の知性がそれによって
 知性認識するところの形象は、諸事物から受け取られたものでなければならな
 い。

 \\


{\scshape 2 Praeterea}, lumen angelicum est fortius quam lumen
intellectus agentis in anima. Sed lumen intellectus agentis abstrahit
species intelligibiles a phantasmatibus. Ergo lumen intellectus angelici
potest abstrahere species etiam ab ipsis rebus sensibilibus. Et ita
nihil prohibet dicere quod Angelus intelligat per species a rebus
acceptas.

&

 さらに天使の光は、魂の中の能動知性の光よりも強い。ところが、能動知性
 の光は、表象像から可知的形象を抽象する。ゆえに、天使の知性の光は、可感
 的事物それ自体からも、形象を抽象できる。こうして、天使が諸事物から受け
 取った形象によって知性認識すると語ることを妨げるものは何もない。

\\


{\scshape 3 Praeterea}, species quae sunt in intellectu, indifferenter
se habent ad praesens et distans, nisi quatenus a rebus sensibilibus
accipiuntur. Si ergo Angelus non intelligit per species a rebus
acceptas, eius cognitio indifferenter se haberet ad propinqua et
distantia, et ita frustra secundum locum moveretur.

&

さらに、知性の中にある形象は、可感的事物から受け取られたという観点を除い
 て、近くのものと遠くのものに無差別に関係する。ゆえに、もし天使が諸事物
 から受け取られた形象によって知性認識するのでないならば、その認識は、近
 いものと離れたものとに無差別に関係することになり、かくして、その場所的
 運動は無駄になるだろう。
 

\\


{\scshape  Sed contra est} quod Dionysius dicit, VII cap. {\itshape de Div. Nom.},
quod Angeli {\itshape non congregant divinam cognitionem a rebus divisibilibus,
aut a sensibilibus}.

&

 しかし反対に、ディオニュシウスは、『神名論』7章で「天使は、分割可能な、
 あるいは可感的な諸事物から、神についての認識を集めるのではない」と述べ
 ている。
 

\\


{\scshape Respondeo dicendum} quod species per quas Angeli intelligunt,
non sunt a rebus acceptae, sed eis connaturales. Sic enim oportet
intelligere distinctionem et ordinem spiritualium substantiarum, sicut
est distinctio et ordo corporalium. Suprema autem corpora habent
potentiam in sui natura totaliter perfectam per formam, in corporibus
autem inferioribus potentia materiae non totaliter perficitur per
formam, sed accipit nunc unam, nunc aliam formam, ab aliquo
agente. Similiter et inferiores substantiae intellectivae, scilicet
animae humanae, habent potentiam intellectivam non completam
naturaliter; sed completur in eis successive, per hoc quod accipiunt
species intelligibiles a rebus. Potentia vero intellectiva in
substantiis spiritualibus superioribus, idest in Angelis, naturaliter
completa est per species intelligibiles, inquantum habent species
intelligibiles connaturales ad omnia intelligenda quae naturaliter
cognoscere possunt. 


&

解答する。以下のように言われるべきである。天使がそれによって知性認識する
ところの形象は、事物から受け取られたものではなく、天使たちにとって生得的
 なものである。理由は以下のとおり。
霊的実体の区別と秩序は、物体的実体の区別と秩序と同様に理解しなければなら
 ない。ところで、最上位の物体は、自らの本性において全体的に形相によって
 完成された可能態・能力を持っている。これに対して、下位の諸物体において、
 質料の能力・可能態は、形相によって全体的に完成されておらず、なんらかの
 作用者によって、今はこの形相を、また別の時には別の形相を受け取る。
同様に、下位の知性的実体、すなわち人間の魂もまた、自然本性的に完
 成された知性認識能力をもたず、諸事物から可知的形象を受け取ることを通し
 て、継起的に、それらにおいて完成される。他方、上位の霊的実体、すなわち
 天使たちにおける知性認識能力は、彼らが自然本性的に認識しうるすべてのこ
 とがらを知性認識するために、生得的な可知的形象をもつかぎりで、自然本性
 的に、可知的形象によって完成されている。





\\



Et hoc etiam ex ipso modo essendi huiusmodi
substantiarum apparet. Substantiae enim spirituales inferiores, scilicet
animae, habent esse affine corpori, inquantum sunt corporum formae, et
ideo ex ipso modo essendi competit eis ut a corporibus, et per corpora
suam perfectionem intelligibilem consequantur, alioquin frustra
corporibus unirentur. Substantiae vero superiores, idest Angeli, sunt a
corporibus totaliter absolutae, immaterialiter et in esse intelligibili
subsistentes, et ideo suam perfectionem intelligibilem consequuntur per
intelligibilem effluxum, quo a Deo species rerum cognitarum acceperunt
simul cum intellectuali natura. -- Unde Augustinus dicit, II {\itshape
 super Gen.~ad Litt.}, quod {\itshape cetera, quae infra Angelos sunt, ita creantur, ut prius
fiant in cognitione rationalis creaturae, ac deinde in genere suo.}

&

そして、このことは、このような実体の存在の仕方自体からも明らかである。
つまり、下位の霊的実体、すなわち魂は、それが身体の形相である限りにおいて、
 物体に結びついた存在を持ち、それゆえ、物体から、物体(身体)を通して、
 自らの可知的な完成を獲得することが、その存在の仕方に適合している。そう
 でなければ、魂が身体と一つになっていることは無駄なことだったであろう。
 これに対し、上位の実体、すなわち天使たちは、物体から全体的に切り離され、
 非質料的に、可知的な存在において自存している。ゆえに、自らの可知的な完
 成を、可知的な流出によって獲得する。天使たちは、その流出によって、認識
 された諸事物の形象を、知性的本性と同時に、神から受け取った。
このことから、アウグスティヌスは『創世記逐語注解』2巻で「天使たちの下位
 にある他のものどもは、初めに理性的被造物の認識の中に、次いで、それぞれ
 の類において作られる、というかたちで創造される」と述べている。


\\


{\scshape Ad primum ergo dicendum} quod in mente Angeli sunt
similitudines creaturarum, non quidem ab ipsis creaturis acceptae, sed a
Deo, qui est creaturarum causa, et in quo primo similitudines rerum
existunt. Unde Augustinus dicit, in eodem libro, quod {\itshape sicut ratio qua
creatura conditur, prius est in verbo Dei quam ipsa creatura quae
conditur, sic et eiusdem rationis cognitio prius fit in creatura
intellectuali, ac deinde est ipsa conditio creaturae}.

&

第一異論に対しては、それゆえ、以下のように言われるべきである。
天使の精神の中に、被造物の類似が存在するが、それは被造物自体からではなく、
 神から受け取られたものである。神は被造物の原因であり、その中には、第一
 に、諸事物の類似が存在する。したがって、アウグスティヌスは同書で以下の
 ように言っている。「それ
 によって被造物が作られるところの理拠は、作られる被造物自身よりも先に神
 の言葉の中に存在する。ちょうどそのように、同じ理拠の認識もまた、先に知
 性的被造物の中に作られ、次いで、被造物の制作自体がある。」

\\


{\scshape Ad secundum dicendum} quod de extremo ad extremum non
pervenitur nisi per medium. Esse autem formae in imaginatione, quod est
quidem sine materia, non tamen sine materialibus conditionibus, medium
est inter esse formae quae est in materia, et esse formae quae est in
intellectu per abstractionem a materia et a conditionibus
materialibus. Unde quantumcumque sit potens intellectus angelicus, non
posset formas materiales reducere ad esse intelligibile, nisi prius
reduceret eas ad esse formarum imaginatarum. Quod est impossibile, cum
careat imaginatione, ut dictum est. Dato etiam quod posset abstrahere
species intelligibiles a rebus materialibus, non tamen abstraheret, quia
non indigeret eis, cum habeat species intelligibiles connaturales.

&

第二異論に対しては、以下のように言われるべきである。
中間を通らないと、端から端へ行くことはない。
ところで、表象像の中にある形相の存在は、確かに質料なしにあるが、質料的諸
 条件がないわけではないので、質料の中にある形相の存在と、質料と質料的諸
 条件から抽象されて知性の中に存在する形相の存在の中間にある。したがって、
 天使の知性がどれだけ強くても、質料的形相を、まず、表象された形相の存在
 にしないかぎり、それらを可知的な存在にすることはできなかったであろう。
 しかし、すでに述べられたとおり、天使たちは表象を欠くので、このこと[質
 料的形相の存在を表象された形相の存在にすること]は不可能である。また、
 かりに、天使たちが質料的諸事物から可知的形象を抽象することができたとし
 ても、彼らはそういう抽象をしないであろう。なぜなら、彼らは生得的な可知
 的形象をもっているので、それを必要としないからである。


\\


{\scshape Ad tertium dicendum} quod cognitio Angeli indifferenter se
habet ad distans et propinquum secundum locum. Non tamen propter hoc
motus eius localis est frustra, non enim movetur localiter ad
cognitionem accipiendam, sed ad operandum aliquid in loco.

&


第三異論に対しては、以下のように言われるべきである。
天使の認識は、場所的に遠くのものと近くのものとに、無差別に関係する。しか
 し、このために、その場所的運動が無駄にはならない。なぜなら、天使は認識
 を得るためにではなく、場所において何かを行うために、場所的に動くからで
 ある。


\end{longtable}
\newpage


\rhead{a.~3}
\begin{center}
 {\Large {\bf ARTICULUS TERTIUS}}\\
 {\large UTRUM SUPERIORES ANGELI INTELLIGANT PER SPECIES MAGIS UNIVERSALES QUAM INFERIORES}\\
 {\footnotesize Infra, qu.~89, a.~1; II {\itshape Sent.}, d.~3, parte 2, q.~2, a.~2; II {\itshape SCG} cap.~98; {\itshape De Verit.}, q.~8, a.~10; Qu.~{\itshape de Anima}, a.~7, ad 5; a.~18; {\itshape De Causis}, lect.~10.}\\
 {\Large 第三項\\上位の天使は下位の天使より普遍的な形象によって知性認識するか}
\end{center}

\begin{longtable}{p{21em}p{21em}}

{\huge A}{\scshape d tertium sic proceditur}. Videtur quod superiores
 Angeli non intelligant per species magis universales quam
 inferiores. Universale enim esse videtur quod a particularibus
 abstrahitur. Sed Angeli non intelligunt per species a rebus
 abstractas. Ergo non potest dici quod species intellectus angelici sint
 magis vel minus universales.


&


第三項の問題へ、議論は以下のように進められる。
上位の天使は下位の天使より普遍的な形象によって知性認識するのではないと思
 われる。
理由は以下の通り。
普遍的なものは、個別的なものから抽象されていると思われる。
ところで、天使たちは、諸事物から抽象された形象によって知性認識するのでは
 ない。
ゆえに、天使の知性の形象が、より普遍的であるとかより普遍的でないとか言わ
 れることはできない。

\\


{\scshape 2 Praeterea}, quod cognoscitur in speciali,
perfectius cognoscitur quam quod cognoscitur in universali, quia
cognoscere aliquid in universali est quodammodo medium inter potentiam
et actum. Si ergo Angeli superiores cognoscunt per formas magis
universales quam inferiores, sequitur quod Angeli superiores habeant
scientiam magis imperfectam quam inferiores. Quod est inconveniens.


&


さらに、特殊的に認識されるものは、普遍的に認識されるよりも、完全に認識さ
 れる。なぜなら、何かを普遍的において認識することは、可能態と現実態の間
 にあるからである。ゆえに、もし上位の諸天使が、下位の諸天使よりも普遍的
 な形相によって認識するならば、上位の天使たちは下位の天使たちよりも不完
 全な知をもつことになる。これは不適当である。

\\


{\scshape 3 Praeterea}, idem non potest esse propria
ratio multorum. Sed si Angelus superior cognoscat per unam formam
universalem diversa, quae inferior Angelus cognoscit per plures formas
speciales, sequitur quod Angelus superior utitur una forma universali ad
cognoscendum diversa. Ergo non poterit habere propriam cognitionem de
utroque. Quod videtur inconveniens.


&


さらに、同一のものが、多くのものの固有の性格ではありえない。ところが、も
 し上位の天使が一つの普遍的な形相によってさまざまなものを認識し、それを
 下位の天使は、特殊的な複数の形相によって認識するならば、上位の天使は、
 一つの普遍的な形相を、さまざまなものを認識するために使っていることにな
 る。ゆえに、どちらについても固有の認識を持つことができないだろう。これ
 は不適当であるようにおもわれる。

\\


{\scshape  Sed contra est} quod dicit Dionysius, {\scshape xii}
cap.~{\itshape Angel.~Hier}., quod superiores Angeli participant scientiam magis in
universali quam inferiores.-- Et in libro {\itshape de Causis} dicitur quod Angeli
superiores habent formas magis universales.


&

しかし反対に、ディオニュシウスは『天使階級論』12章で、上位の天使たちは下
 位の天使たちよりも普遍的に知を分有する、と述べている。また、『原因論』
 という書物の中で、上位の天使たちは、より普遍的な形相をもつと言われてい
 る。


\\


{\scshape Respondeo dicendum} quod ex hoc sunt in rebus
aliqua superiora, quod sunt uni primo, quod est Deus, propinquiora et
similiora. In Deo autem tota plenitudo intellectualis cognitionis
continetur in uno, scilicet in essentia divina, per quam Deus omnia
cognoscit. Quae quidem intelligibilis plenitudo in intellectibus creatis
inferiori modo et minus simpliciter invenitur. Unde oportet quod ea quae
Deus cognoscit per unum, inferiores intellectus cognoscant per multa, et
tanto amplius per plura, quanto amplius intellectus inferior fuerit. 


&


解答する。以下のように言われるべきである。
一つの第一のもの、すなわち神により近く、より類似することに基づいて、諸事
 物において、或るものはより上位のものである。ところで、神の中で、知性的
 認識の全充満が、一つのもの、すなわち神の本質の中に含まれ、神はそれによっ
 て万物を認識する。じっさい、この知的充満は、被造の諸知性の中に、より下
 位のあり方で、より単純でないかたちで見出される。
このことから、神が一つのものによって認識することがらを、下位の諸知性は多
 くのものによって認識するのでなければならない。そして、それがより多くの
 ものによってであるほど、その知性はより下位のものである。

\\



Sic
igitur quanto Angelus fuerit superior, tanto per pauciores species
universitatem intelligibilium apprehendere poterit. Et ideo oportet quod
eius formae sint universaliores, quasi ad plura se extendentes
unaquaeque earum. Et de hoc exemplum aliqualiter in nobis perspici
potest. Sunt enim quidam, qui veritatem intelligibilem capere non
possunt, nisi eis particulatim per singula explicetur, et hoc quidem ex
debilitate intellectus eorum contingit. Alii vero, qui sunt fortioris
intellectus, ex paucis multa capere possunt.

&

ゆえに、このようにして、より上位の天使ほど、より少ない形象によって、可知的なものども
 の全体を捉えることができる。ゆえに、そのような天使の形相は、その各々が、
 いわば複数のものへ及ぶものであるという意味で、普遍的でなければならない。
そしてこのことについては、私たちの中にも、なんらかのかたちで、似た事例が
 感じられうる。たとえば、個別的に一つずつ説明されないと可知的な真理をつ
 かむことができない人がいるが、それは、そういう人たちの知性の弱さによっ
 て生じる。他方、別の人々は、より強い知性を持っているので、少しのことに
 よって、多くのものをつかむことができる。


\\


{\scshape Ad primum ergo dicendum} quod accidit
universali ut a singularibus abstrahatur, inquantum intellectus illud
cognoscens a rebus cognitionem accipit. Si vero sit aliquis intellectus
a rebus cognitionem non accipiens, universale ab eo cognitum non erit
abstractum a rebus, sed quodammodo ante res praeexistens, vel secundum
ordinem causae, sicut universales rerum rationes sunt in verbo Dei; vel
saltem ordine naturae, sicut universales rerum rationes sunt in
intellectu angelico.

&

 第一異論に対しては、それゆえ、以下のように言われるべきである。
 普遍的なものに、それが個物から抽象されるということが生じるのは、それを
認識する知性が、諸事物から認識を受け取る限りにおいてである。もし、他方で、
 諸事物から認識を受け取るのでないような知性があるならば、その知性によっ
 て認識された普遍は、諸事物から抽象されたものでなく、なんらかのかたちで、
 事物の前に先在しているものであるだろう。それは、諸事物の普遍的な性格が
 神の言葉のなかにあるというかたちで、原因の秩序によるか、あるいは、諸事
 物の普遍的な性格が、天使の知性の中にあるというかたちで、むしろ本性の秩
 序によるか、いずれかである。

\\



{\scshape Ad secundum dicendum} quod cognoscere aliquid
in universali, dicitur dupliciter. Uno modo, ex parte rei cognitae, ut
scilicet cognoscatur solum universalis natura rei. Et sic cognoscere
aliquid in universali est imperfectius, imperfecte enim cognosceret
hominem, qui cognosceret de eo solum quod est animal. Alio modo, ex
parte medii cognoscendi. Et sic perfectius est cognoscere aliquid in
universali, perfectior enim est intellectus qui per unum universale
medium potest singula propria cognoscere, quam qui non potest.

&

 第二異論に対しては、以下のように言われるべきである。
 何かを普遍的に認識する、ということは、二通りの意味で語られる。
 一つには、認識された事物の側からであり、その場合、事物の普遍的な本性だ
 けが認識される。この意味で、事物を普遍的に認識することは、より不完全で
 ある。たとえば、ある人が、人間について、それが動物であることだけを認識するならば、
 その人は人間を不完全に認識しているだろう。
 別の意味では、認識の媒体の側からである。この意味では、何かを普遍的に認
 識することは、より完全である。なぜなら、一つの普遍的な媒体で、個々の固
 有のものを認識する知性は、それができない知性よりも、完全だからである。

\\



{\scshape Ad tertium dicendum} quod idem non potest esse
plurium propria ratio adaequata. Sed si sit excellens, potest idem
accipi ut propria ratio et similitudo diversorum. Sicut in homine est
universalis prudentia quantum ad omnes actus virtutum; et potest accipi
ut propria ratio et similitudo particularis prudentiae quae est in leone
ad actus magnanimitatis, et eius quae est in vulpe ad actus cautelae, et
sic de aliis. Similiter essentia divina accipitur, propter sui
excellentiam, ut propria ratio singulorum, quia est in ea unde sibi
singula similentur secundum proprias rationes. Et eodem modo dicendum
est de ratione universali quae est in mente Angeli, quod per eam,
propter eius excellentiam, multa cognosci possunt propria cognitione.

&

 第三異論に対しては、以下のように言われるべきである。
 同じものが、複数のものの、対等した固有性格ではありえない。しかし、卓越
 するものであれば、同じものが様々なものの固有の性格や類似として受け取ら
 れうる。たとえば、人間の中に、諸徳の全ての行為に関する普遍的な思慮があ
 るが、これは、ライオンの中にある高邁な行為や、キツネの中にある注意深い
 行為、その他そのような行為にかんする個別的な思慮の、固有の性格や類似と
 して受け取られうる。同様に、神の本質は、その卓越性のために、個々の事柄
 の固有の性格として受け取られうる。なぜなら、その中に、個々の事物が、そ
 の固有の性格に即して、それに類似化する根拠をもつからである。
 そして、天使の精神の中にある普遍的な性格についても、同様に、その卓越性
 のゆえに、それによって、多くのものが、固有の認識によって認識されうる、
 と言われるべきである。


\end{longtable}

\newpage

\end{document}
