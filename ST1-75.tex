\documentclass[paper=a4paper,fontsize=10pt,jafontsize=9pt,titlepage]{jlreq}
\usepackage{pxrubrica} %ルビ、傍点
\usepackage{longtable}
\usepackage[polutonikogreek,english,japanese]{babel}
\usepackage{latexsym}
\usepackage{color}
\usepackage{url}

%----- header -------
\usepackage{fancyhdr}
\pagestyle{fancy}
\lhead{{\itshape Summa Theologiae} I, q.75}
%--------------------


\bibliographystyle{jplain}

\title{{\bfseries Prima Pars}\\{\Huge Summae Theologiae}\\Sancti Thomae
Aquinatis\\{\sffamily Quaestio Septuagesimaquinta}\\{\bfseries De Homine,
Qui ex Spirituali et Corporali Substantia Componitur.\\ Et Primo,
Quantum ad Essentiam Animae}}
\author{Japanese translation\\by Yoshinori {\scshape Ueeda}}
\date{Last modified \today}

%%%% コピペ用
%\rhead{a.~}
%\begin{center}
% {\Large {\bf }}\\
% {\large }\\
% {\footnotesize }\\
% {\Large \\}
%\end{center}
%
%\begin{longtable}{p{21em}p{21em}}
%
%&
%
%\\
%\end{longtable}
%\newpage

\begin{document}

\maketitle

\begin{center}
{\Large 第七十五問\\
霊的実体と物体的実体から複合されている人間について\\そして第一に、魂の
 本質について}
\end{center}


\begin{longtable}{p{21em}p{21em}}

Post considerationem creaturae spiritualis et corporalis,
considerandum est de homine, qui ex spirituali et corporali substantia
componitur. Et primo, de natura ipsius hominis; secundo, de eius
productione. Naturam autem hominis considerare pertinet ad theologum
ex parte animae, non autem ex parte corporis, nisi secundum
habitudinem quam habet corpus ad animam. Et ideo prima consideratio
circa animam versabitur. 

&

霊的被造物と物体的被造物の考察の後、霊的実体と物体的実体から複合されて
いる人間について考察されるべきである。そして第一に人間自身の本性につい
て、第二にその産出について(考察されるべきである)。さて、人間の本性を
考察することが神学に属するのは魂の側からであり、身体が魂に対してもつ関
係においてでないかぎり、身体の側からではない。
ゆえに、第一の考察は魂をめぐって行われる。

\\

Et quia, secundum Dionysium, XI
cap.~Angel.~Hier., tria inveniuntur in substantiis spiritualibus,
scilicet essentia, virtus et operatio; primo considerabimus ea quae
pertinent ad essentiam animae; secundo, ea quae pertinent ad virtutem
sive potentias eius; tertio, ea quae pertinent ad operationem
eius. 





& 

そして、ディオニュシウスの『天使階級論』11章によれば、霊的実体において
 は三つのもの、すなわち本質、力、働きが見出されるので、第一に私たちは
 魂の本質に属する事柄を考察し、第二に魂の力ないし能力に属する事柄を考
 察し、第三に魂の働きに属する事柄を考察する。


\\


Circa primum duplex occurrit consideratio, quarum prima est de
ipsa anima secundum se; secunda, de unione eius ad corpus. Circa
primum quaeruntur septem. 


&

第一をめぐって二通りの考察が生じ、その第一はそれ自体における魂そのもの
 についてであり、第二は魂の身体の合一についてである。第一について七つ
 のことが問われる。

\\

\begin{enumerate}
 \item utrum anima sit corpus.
 \item utrum anima humana sit aliquid subsistens.
 \item utrum animae brutorum sint subsistentes.
 \item utrum anima sit homo; vel magis homo sit aliquid compositum ex anima et corpore.
 \item utrum sit composita ex materia et forma.
 \item utrum anima humana sit incorruptibilis.
 \item utrum anima sit eiusdem speciei cum Angelo.
\end{enumerate}

&

\begin{enumerate}
 \item 魂は物体か。
 \item 人間の魂は自存する何かか。
 \item 非理性的動物\footnote{buruta animaliaは人間以外の動物を指す。人
       間が「理性的動物」と定義されることから「非理性的動物」と訳され
       る。したがってこの「非」は「理性的」ではなく「理性的動物」にかかることに
       注意。}の魂は自存するか。
 \item 魂が人間なのか、それとも人間は魂と身体から複合された何かか。
 \item 魂は質料と形相化羅複合されているか。
 \item 人間の魂は不滅か。
 \item 魂は天使と同じ種に属するか。
\end{enumerate}

\end{longtable}

\newpage
\rhead{a.~1}
\begin{center}
{\Large {\bfseries ARTICULUS PRIMUS}}\\
{\large UTRUM ANIMA SIT CORPUS}\\
{\footnotesize II {\itshape SCG.}, cap.45; II {\itshape de Anima}, lect.1}\\
{\Large 第一項\\魂は物体か}
\end{center}

\begin{longtable}{p{21em}p{21em}}



{\scshape Ad primum sic proceditur}. Videtur quod anima sit
 corpus\footnote{「物体」と訳したcorpusには「身体」という意味もある。
ここで「魂がcorpusを動かす」と言われる時、具体的には魂が身体を動
かすことが想定されている。}. Anima enim
est motor corporis. Non autem est movens non motum. Tum quia videtur
quod nihil possit movere nisi moveatur, quia nihil dat alteri quod non
habet, sicut quod non est calidum non calefacit. Tum quia, si aliquid
est movens non motum, causat motum sempiternum et eodem modo se
habentem, ut probatur in VIII {\itshape Physic}., quod non apparet in motu
animalis, qui est ab anima. Ergo anima est movens motum. Sed omne
movens motum est corpus. Ergo anima est corpus.

&

第一項の問題へ議論は以下のように進められる。魂は物体であると思われる。
理由は以下の通り。魂は物体を動かすものである。しかるに、動かされないも
のは動かすものではない。その理由は、一つには、例えば熱せられていないも
のは熱くないように、何ものももっていないものを他のものに与えることはな
いので、何ものも動かされることなしに動かすことはできないと思われるから
である。また別の理由としては、もし何かが動かされていないが動かすとすれ
ば、それは『自然学』第七巻で証明されているように、永続的で一様の運動の
原因となるが、これは魂に由来する動物の運動には見られない。ゆえに魂は動
かされて動かすものである。しかるに全て動かされて動かすものは物体である。
ゆえに魂は物体である。


\\



2.{\scshape Praeterea}, omnis cognitio fit per aliquam
similitudinem. Non potest autem esse similitudo corporis ad rem
incorpoream. Si igitur anima non esset corpus, non posset cognoscere
res corporeas.



&

さらに、認識はすべてなんらかの類似によって生じる。しかるに、物体が非物
体的事物に類似することはありえない。ゆえにもし魂が物体でなかったなら
ば、物体的事物を認識することはできなかったであろう。


\\

3. {\scshape Praeterea}, moventis ad motum oportet esse aliquem
contactum. Contactus autem non est nisi corporum. Cum igitur anima
moveat corpus, videtur quod anima sit corpus.


&

さらに、動かすものには、動かされるものへの何らかの接触がなければならな
い。しかるに接触は物体同士にしかない。ゆえに魂は物体(身体)を動かすのだから
魂は物体である。

\\


{\scshape Sed contra est} quod Augustinus dicit, VI de Trin., quod anima {\itshape simplex
 dicitur respectu corporis, quia mole non diffunditur per spatium
 loci}.


&

しかし反対に、アウグスティヌスは『三位一体論』第六巻で、魂は「その嵩が
場所の空間をわたって広がっていないので、物体に比して単純だと言われる」
と述べている。

\\



{\scshape Respondeo dicendum} quod ad inquirendum de natura animae, oportet
praesupponere quod anima dicitur esse primum principium vitae in his
quae apud nos vivunt: {\itshape animata} enim viventia dicimus, res vero
{\itshape inanimatas} vita carentes. Vita autem maxime manifestatur duplici
opere, scilicet cognitionis et motus. Horum autem principium antiqui
philosophi, imaginationem transcendere non valentes, aliquod corpus
ponebant; sola corpora res esse dicentes, et quod non est corpus,
nihil esse. Et secundum hoc, animam aliquod corpus esse
dicebant. 


&

解答する。以下のように言われるべきである。魂の本性を探究するためには、
魂が、私たちのもとで生きているものどもにおける生命の第一根源であると言
われることが前提とされる必要がある。じっさい、私たちは生きるものどもを
「魂あるもの」と呼び、生命を欠いているものどもを「魂を欠く」事物と呼ぶ。
ところで、生命は二つの働き、すなわち認識と運動という働きにおいて最大限
に明示される。しかしこれらの根源を、古代の哲学者たちは想像力を越えるこ
とができず、何らかの物体だと想定した。彼らは物体だけが事物であると言い、
物体でないものは何ものでもないと言った。そしてこの考えにしたがって、彼
らは魂が何らかの物体だと言った。

\\



Huius autem opinionis falsitas licet multipliciter ostendi
possit tamen uno utemur, quo et communius et certius patet animam
corpus non esse. Manifestum est enim quod non quodcumque vitalis
operationis principium est anima, sic enim oculus esset anima, cum sit
quoddam principium visionis; et idem esset dicendum de aliis animae
instrumentis. Sed primum principium vitae dicimus esse animam. Quamvis
autem aliquod corpus possit esse quoddam principium vitae, sicut cor
est principium vitae in animali; tamen non potest esse primum
principium vitae aliquod corpus. 


&

さて、この意見の誤りは多くのしかたで示されうるが、私たちは、魂が物体で
ないことがより共通的にまたたしかに明らかであるような一つのしかたを用い
よう。それは以下の通りである。生命の働きの根源であればなんでも魂である
わけではない。もしそうだったならば、目は見ることの何らかの根源だから、
目は魂だっただろう。そして同じことが他の魂の道具(的な器官)について言
われるべきだっただろう。そうではなく、私たちは生命の第一の根源が魂であ
ると言う。しかるに、心臓が動物における生命の根源であるように、ある物体
が生命の何らかの根源であることは可能だが、生命の第一の根源が何らかの物
体であることはありえない。



\\



Manifestum est enim quod esse
principium vitae, vel vivens, non convenit corpori ex hoc quod est
corpus, alioquin omne corpus esset vivens, aut principium
vitae. Convenit igitur alicui corpori quod sit vivens, vel etiam
principium vitae, per hoc quod est tale corpus. Quod autem est actu
tale, habet hoc ab aliquo principio quod dicitur actus eius. Anima
igitur, quae est primum principium vitae, non est corpus, sed corporis
actus, sicut calor, qui est principium calefactionis, non est corpus,
sed quidam corporis actus.


&

というのも、生命の根源であるということ、あるいは生きているということが、
物体であることに基づいて物体に適合することがないのは明らかである。さも
なければ全ての物体が生きているか生命の根源だったであろう。ゆえにある物
体に生きていることや生命の根源であることが適合するのは、これこれの物体
である限りにおいてである。しかるに、現実態においてこれこれであることは、
そのものの現実態と言われる何らかの根源からもつ。ゆえに魂は、生命の第一
根源だから、物体ではなく、物体の現実態である\footnote{魂は物体か、とい
う問いに対して予想される答えは、魂は物体ではなく霊的なもの、非物体的な
ものである、というものだが、このトマスの結論は、そうではなく、魂は物体
ではなく物体の現実態・作用(actus)だ、というものである。このactusの意味
 は正確に理解される必要があり、次の、熱は物体でなく物体のactusである、
 という説明が手がかりとなるる。「物体」とは縦・横・高さの三つの次元が
 指定されうるものであり、デカルト空間に位置を占める。熱はそのような意
 味での物体ではない。}。たとえば熱は熱することの根源であるが、物体では
なく、物体の現実態であるように。



\\



{\scshape Ad primum ergo dicendum} quod, cum omne quod movetur ab alio moveatur,
quod non potest in infinitum procedere, necesse est dicere quod non
omne movens movetur. Cum enim moveri sit exire de potentia in actum,
movens dat id quod habet mobili, inquantum facit ipsum esse in
actu. Sed sicut ostenditur in VIII Physic., est quoddam movens penitus
immobile, quod nec per se nec per accidens movetur, et tale movens
potest movere motum semper uniformem. 


&

第一異論に対しては、それゆえ、以下のように言われるべきである。全て動か
されるものは他のものによって動かされるが、これを無限に進めることはでき
ないので、全ての動かすものが動かされるわけではない。というのも、動かさ
れることは可能態から現実態へ出ていくことだから、動かすものは、自分がもっ
ているものを、それを現実態においてあるものにする限りにおいて、動かされ
うるものに与える。しかし、『自然学』第八巻で示されているとおり、まった
く不動であるような動かすものが存在し、それは自体的にも附帯的にも動かさ
れない。そしてそのような動者が、常に一様な運動を動かすことができる。


\\




Est autem aliud movens, quod non
movetur per se, sed movetur per accidens, et propter hoc non movet
motum semper uniformem. Et tale movens est anima. Est autem aliud
movens, quod per se movetur, scilicet corpus. Et quia antiqui
naturales nihil esse credebant nisi corpora, posuerunt quod omne
movens movetur, et quod anima per se movetur, et est corpus.


&

他方で他の動者が存在し、それは自体的には動かされないが附帯的に動かされ、
そのために常に一様な運動を動かさない。そのような動者は魂である。さら
にまた別の動者があり、それは自体的に動かされるものでありすなわちそれ
は物体である。そして古代の自然学者たちは物体以外のものが存在すると信
じなかったので、全て動かすものは動かされると考え、そして魂が自体的に
動かされ、物体であると考えた。


\\



{\scshape Ad secundum dicendum} quod non est necessarium quod similitudo rei
cognitae sit actu in natura cognoscentis, sed si aliquid sit quod
prius est cognoscens in potentia et postea in actu, oportet quod
similitudo cogniti non sit actu in natura cognoscentis, sed in
potentia tantum; sicut color non est actu in pupilla, sed in potentia
tantum. 

&

第二異論に対しては以下のように言われるべきである。認識される事物の類似
が現実態において認識するものの本性の中に存在する必要はないが、もし或る
ものが先立って可能態において認識するものでありその後現実態において認識
するものであるならば、認識されるものの類似は現実態において認識するもの
の本性の中にあってはならず、むしろただ可能態においてのみなければならな
い。ちょうど色が現実態において瞳にあることはなく可能態においてあるよう
に。

\\

Unde non oportet quod in natura animae sit similitudo rerum
corporearum in actu; sed quod sit in potentia ad huiusmodi
similitudines. Sed quia antiqui naturales nesciebant distinguere inter
actum et potentiam, ponebant animam esse corpus, ad hoc quod
cognosceret corpus; et ad hoc quod cognosceret omnia corpora, quod
esset composita ex principiis omnium corporum.


&

したがって、魂の本性において物体的事物の類似は現実態にある必要はなく、
 むしろそのような類似に対して可能態にあることが必要である。
しかし古代の自然学者たちは現実態と可能態を区別することを知らなかったの
 で、物体を認識するために魂が物体であると考えた。そして全ての物体を認
 識するために、全ての物体の根源から複合されていると考えた。

\\



{\scshape Ad tertium dicendum} quod est duplex contactus, quantitatis et
 virtutis. Primo modo, corpus non tangitur nisi a corpore. Secundo
 modo, corpus potest tangi a re incorporea quae movet corpus.


&

第三異論に対しては以下のように言われるべきである。
接触には二種類あり、それは量の接触と力の接触である。第一のしかたでは、
 物体は物体によってのみ触れられるが、第二のしかたでは、物体は、物体を
 動かす非物体的な事物によって触れられる。



\\




\end{longtable}
\newpage

\rhead{a.~2}
\begin{center}
{\Large {\bfseries ARTICULUS SECUNDUS}}\\
{\large UTRUM ANIMA HUMANA SIT ALIQUID SUBSISTENS}\\
{\footnotesize {\itshape De Pot.}, q.3, a.9, 11; {\itshape De
 Spirit.~Creat.}, a.2; Qu.~{\itshape de Anima}, a.1, 14; III {\itshape
 de Anima}, lect.7.}\\
{\Large 第二項\\人間の魂は自存する何かか}
\end{center}

\begin{longtable}{p{21em}p{21em}}



{\scshape Ad secundum sic proceditur}. Videtur quod anima humana non sit aliquid
subsistens. Quod enim est subsistens, dicitur hoc aliquid. Anima autem
non est hoc aliquid, sed compositum ex anima et corpore. Ergo anima
non est aliquid subsistens.


&

第二項の問題へ議論は以下のように進められる。
人間の魂は自存するものではないと思われる。理由は以下の通り。
自存するものは「このあるもの」と言われる。
しかるに魂は「このあるもの」ではなく、魂と身体から複合されたものが「こ
 のあるもの」である。ゆえに魂は自存するものではない。

\\



2. {\scshape Praeterea}, omne quod est subsistens, potest dici operari. Sed anima
non dicitur operari, quia, ut dicitur in I {\itshape de Anima}, {\itshape dicere animam
sentire aut intelligere, simile est ac si dicat eam aliquis texere vel
aedificare}. Ergo anima non est aliquid subsistens.


&

さらに、全て自存するものは、働くと言われうる。しかるに魂が働くとは言わ
 れ得ない。なぜなら『デ・アニマ』第一巻で言われるように「魂が知るとか
 魂が知性認識すると言われるのは、魂が織るとか魂が制作すると誰かが言う
 ようなもの」だからである。ゆえに魂は自存する何かではない。

\\



3. {\scshape Praeterea}, si anima esset aliquid subsistens, esset aliqua eius
operatio sine corpore. Sed nulla est eius operatio sine corpore, nec
etiam intelligere, quia non contingit intelligere sine phantasmate,
phantasma autem non est sine corpore. Ergo anima humana non est
aliquid subsistens.


&

さらに、もし魂が何か自存するものだったならば、身体を伴わない魂の何らか
の働きがあっただろう。しかし身体を伴わない魂の働きはなく、知性認識する
ことでさえ身体を伴う。なぜなら、知性認識は表象像なしに生じないが、表象
像は身体なしにないからである。ゆえに人間の魂は何か自存するものではない。

\\



{\scshape Sed contra est} quod Augustinus dicit, X {\itshape de Trin} : {\itshape Quisquis videt mentis
naturam et esse substantiam, et non esse corpoream, videt eos qui
opinantur eam esse corpoream, ob hoc errare, quod adiungunt ei ea sine
quibus nullam possunt cogitare naturam, scilicet corporum
phantasias}. Natura ergo mentis humanae non solum est incorporea, sed
etiam substantia, scilicet aliquid subsistens.


&

しかし反対に、アウグスティヌスは『三位一体論』第10巻で次のように言って
いる。「だれであれ精神の本性が実体であって物体的でないと理解する人は、それ
 が物体的であると言う意見をもっている人々が、それなしにはどんな本性を
 考えることもできないもの、すなわち物体の表象のために誤っているという
 ことを理解する」。ゆえに人間精神の本性は非物体的であるだけでなく実体、
 すなわち何か自存するものである。


\\



{\scshape Respondeo dicendum} quod necesse est dicere id quod est principium
intellectualis operationis, quod dicimus animam hominis, esse quoddam
principium incorporeum et subsistens. Manifestum est enim quod homo
per intellectum cognoscere potest naturas omnium corporum. 


&

解答する。以下のように言われるべきである。
知性的な働きの根源であるもの、私たちはそれを人間の魂と言うのだが、そえ
 は非物体的で自存する根源である。理由は以下の通り。
人間は知性を通して全ての物体の本性を認識することができる。

\\



Quod autem
potest cognoscere aliqua, oportet ut nihil eorum habeat in sua natura,
quia illud quod inesset ei naturaliter impediret cognitionem aliorum;
sicut videmus quod lingua infirmi quae infecta est cholerico et amaro
humore, non potest percipere aliquid dulce, sed omnia videntur ei
amara. 

&

ところで、あるものどもを認識することができるものは、自分の本性の中に、
それらのどれももっていてはならない。なぜなら、本性的にそれに内在したと
するならば、それは他のものどもの認識を妨げただろうから。たとえば胆汁質
 で苦い液体に浸された病弱な人の舌が、何も甘いものを知覚できず、彼にとっ
 て全てが苦く思われるように。


\\



Si igitur principium intellectuale haberet in se naturam
alicuius corporis, non posset omnia corpora cognoscere. Omne autem
corpus habet aliquam naturam determinatam. Impossibile est igitur quod
principium intellectuale sit corpus. 

&

ゆえに、もし知性的根源が自らの中に名kらかの物体の本性をもっていたなら
 ば、全ての物体を認識することはできなかったであろう。しかるに、全て物
 体は何らかの限定された本性をもっている。ゆえに知性的な根源が物体であ
 ることは不可能である。


\\


Et similiter impossibile est quod
intelligat per organum corporeum, quia etiam natura determinata illius
organi corporei prohiberet cognitionem omnium corporum; sicut si
aliquis determinatus color sit non solum in pupilla, sed etiam in vase
vitreo, liquor infusus eiusdem coloris videtur. 


&

同様に、物体的な器官を通して知性認識することも不可能である。なぜなら、
その器官の限定された本性がやはり全ての物体の認識を妨げたであろうから。
ちょうど、ある限定された色が瞳にあるときだけでなく、ガラスの花瓶の中に
ある場合にも、そこに注がれた液体は同じ色に見えるように。


\\

Ipsum igitur intellectuale principium, quod dicitur mens vel
intellectus, habet operationem per se, cui non communicat
corpus. Nihil autem potest per se operari, nisi quod per se
subsistit. Non enim est operari nisi entis in actu, unde eo modo
aliquid operatur, quo est. Propter quod non dicimus quod calor
calefacit, sed calidum. Relinquitur igitur animam humanam, quae
dicitur intellectus vel mens, esse aliquid incorporeum et subsistens.


&

ゆえに、精神や知性と言われる知性的な根源は、それ自身によって働きを持ち、
それには物体(身体)は参加しない。しかるに、それ自身によって自存するも
のでなければ、それ自身によって働くことができない。なぜなら、働くことは
現実態における有にしか属さないので、あるものは、それが存在するようなし
かたで働くからである。このため、私たちは熱が熱すると言わず、熱いものが
熱すると言う。ゆえに、知性や精神と言われる人間の魂は、非物体的で自存す
る何かであることが帰結する。



\\


{\scshape Ad primum ergo dicendum} quod hoc aliquid potest accipi
dupliciter, uno modo, pro quocumque subsistente, alio modo, pro
subsistente completo in natura alicuius speciei. Primo modo, excludit
inhaerentiam accidentis et formae materialis, secundo modo, excludit
etiam imperfectionem partis. Unde manus posset dici hoc aliquid primo
modo, sed non secundo modo. Sic igitur, cum anima humana sit pars
speciei humanae, potest dici hoc aliquid primo modo, quasi subsistens,
sed non secundo modo, sic enim compositum ex anima et corpore dicitur
hoc aliquid.


&

第一異論に対しては、それゆえ以下のように言われるべきである。「このある
もの」は二通りに理解可能である。一つは、どんなものであれ自存するものと
して理解できる。もう一つには、何らかの種の本性において完成された自存す
るものとして理解できる。第一のしかたでは附帯性と質料的形相の内在を排除
し、第二のしかたでは部分の不完全性も排除する。したがって、手は第一のし
かたでは「このあるもの」と言われうるが第二のしかたではそう言われえない。
ゆえに、人間の魂は人間の種の部分なので、第一の意味で、いわば自存するも
のとして「このあるもの」と言われうるが、第二のしかたではそう言われえな
い。この第二の意味では魂と身体からの複合体が「このあるもの」と言われる。


\\



{\scshape Ad secundum dicendum} quod verba illa Aristoteles dicit non
secundum propriam sententiam, sed secundum opinionem illorum qui
dicebant quod intelligere est moveri; ut patet ex iis quae ibi
praemittit. Vel dicendum quod per se agere convenit per se
existenti. Sed per se existens quandoque potest dici aliquid si non
sit inhaerens ut accidens vel ut forma materialis, etiam si sit pars.


&

第二異論に対しては以下のように言われるべきである。アリストテレスはその
言葉を自分自身の考えに即して語っているのではなく、知性認識することは動
くことであると言った人々の意見に即して語っている。これはその箇所に先行
するところから明らかである。あるいは以下のように言うべきである。自体的
に働くことは自体的に存在するものに適合する。しかるに、附帯性や質料的形
相のように内在するものでないならば、かりにそれが部分であるとしても、そ
れ自体で存在するものと言われうることがある。


\\

Sed proprie et per se subsistens dicitur quod neque est praedicto modo
inhaerens, neque est pars. Secundum quem modum oculus aut manus non
posset dici per se subsistens; et per consequens nec per se
operans. Unde et operationes partium attribuuntur toti per
partes. Dicimus enim quod homo videt per oculum, et palpat per manum,
aliter quam calidum calefacit per calorem, quia calor nullo modo
calefacit, proprie loquendo. Potest igitur dici quod anima intelligit,
sicut oculus videt, sed magis proprie dicitur quod homo intelligat per
animam.


&

しかし、上述のしかたで内在するものや部分であるものは、厳密に自体的に自
存するものとは言われない。その意味では、目や手は自体的に自存するとは言
 われえない。その結果、それらは自体的に働くものではない。
このことから、部分の働きは、部分を通して全体に帰せられる。たとえば私た
 ちは人間が目を通して見、手を通して触れると言う。しかしこれは熱いもの
 が熱するのとは異なるしかたによる。なぜなら厳密に言えば熱は決して熱し
 ないからである。ゆえに、目が見るというように魂が知性認識すると言わ
 れるのは可能だが、より厳密には人間が魂を通して知性認識すると言われる。


\\



{\scshape Ad tertium dicendum} quod corpus requiritur ad actionem
intellectus, non sicut organum quo talis actio exerceatur, sed ratione
obiecti, phantasma enim comparatur ad intellectum sicut color ad
visum. Sic autem indigere corpore non removet intellectum esse
subsistentem, alioquin animal non esset aliquid subsistens, cum
indigeat exterioribus sensibilibus ad sentiendum.


&

第三異論に対しては以下のように言われるべきである。身体が知性の働きに必
要とされるのは、それによってその働きが遂行される器官としてではなく、対
象の性格による。つまり表象像が知性に対して、色が視覚に対するように関係
するからである。しかし、このように身体を必要とすることは、知性が自存す
るものであることを排除しない。さもなければ動物は、感覚するために外界の
可感的なものを必要とするので、自存するものでないことになっていただろう。

\\


\end{longtable}
\newpage







\rhead{a.~3}
\begin{center}
{\Large {\bfseries ARTICULUS TERTIUS}}\\
{\large UTRUM ANIMAE BRUTORUM ANIMARLIUM SINT SUBSISTENTES}\\
{\footnotesize II {\itshape SCG}, cap.82.}\\
{\Large 第三項\\非理性的動物の魂は自存するか}
\end{center}

\begin{longtable}{p{21em}p{21em}}

{\scshape Ad tertium sic proceditur}. Videtur quod animae brutorum animalium sint
subsistentes. Homo enim convenit in genere cum aliis animalibus. Sed
anima hominis est aliquid subsistens, ut ostensum est. Ergo et animae
aliorum animalium sunt subsistentes.

&

第三項の問題へ議論は以下のように進められる。
非理性的動物の魂は自存すると思われる。理由は以下の通り。
人間は、他の動物と類が一致する。しかるに人間の魂は、すでに示されたとお
 り自存する何かである。ゆえに他の動物の魂も自存する。

\\



2. {\scshape Praeterea}, similiter se habet sensitivum ad sensibilia, sicut
intellectivum et intelligibilia. Sed intellectus intelligit
intelligibilia sine corpore. Ergo et sensus apprehendit sensibilia
sine corpore. Animae autem brutorum animalium sunt sensitivae. Ergo
sunt subsistentes, pari ratione qua et anima hominis, quae est
intellectiva.

&


さらに、感覚しうるものが感覚されうるもの(可感的なもの)に対する関係は、知性認識しうる
 ものが知性性認識されうるもの(可知的なもの)に対する関係に類似する。しかるに知性は身体
 なしに可知的なものを知性認識する。ゆえに感覚も身体なしに可感的なもの
 をとらえる。しかるに非理性的動物の魂は感覚しうるものである。ゆえにそ
 れは、知性認識しうるものである人間の魂と同じ理由で、自存する。


\\



3. {\scshape Praeterea}, brutorum animalium anima movet corpus. Corpus autem non
movet, sed movetur. Anima ergo bruti animalis habet aliquam
operationem sine corpore.

&

さらに、非理性的動物の魂は物体(身体)を動かす。
しかるに物体は動かさず、動かされる。
ゆえに非理性的動物の魂は、身体をもたない何らかの働きをもつ。

\\



{\scshape Sed contra est quod} dicitur in libro de Eccl. Dogmat., solum hominem
credimus habere animam substantivam; animalium vero animae non sunt
substantivae.

&

しかし反対に、『教会教義論』\footnote{アウグスティヌスの著作。第16, 17章}で「私たちは人間だけ
 が実体的な魂を持ち、動物たちの魂は実体的でないことを信じる」と言われ
 ている。


\\



{\scshape Respondeo dicendum} quod antiqui philosophi nullam distinctionem
ponebant inter sensum et intellectum, et utrumque corporeo principio
attribuebant, ut dictum est. Plato autem distinxit inter intellectum
et sensum; utrumque tamen attribuit principio incorporeo, ponens quod,
sicut intelligere, ita et sentire convenit animae secundum seipsam. Et
ex hoc sequebatur quod etiam animae brutorum animalium sint
subsistentes. 


&

解答する。以下のように言われるべきである。すでに述べられたとおり、古代
の哲学者たちは感覚と知性の区別をせず、このどちらをも物体的な根源に帰し
ていた。また、プラトンは知性と感覚の区別をしたが、知性認識することと同
様に感覚することも、それ自体における魂に適合すると考えて、両者を非物体
的な根源に帰した。そしてこのことから非理性的動物の魂も自存することが帰
結した。

\\



Sed Aristoteles posuit quod solum intelligere, inter
opera animae, sine organo corporeo exercetur. Sentire vero, et
consequentes operationes animae sensitivae, manifeste accidunt cum
aliqua corporis immutatione; sicut in videndo immutatur pupilla per
speciem coloris; et idem apparet in aliis. 


&

しかしアリストテレスは、魂の働きの中でただ知性認識だけが身体なしに遂行
 されると考えた。
感覚することやそれに伴う感覚的魂の働きは明らかに何らかの身体を伴って生
 じる。たとえば見ることにおいて瞳が色の形象によって変化を受けるし、他
 のものにおいても同じことが見られる。

\\


Et sic manifestum est quod
anima sensitiva non habet aliquam operationem propriam per seipsam,
sed omnis operatio sensitivae animae est coniuncti. Ex quo relinquitur
quod, cum animae brutorum animalium per se non operentur, non sint
subsistentes, similiter enim unumquodque habet esse et operationem.

&

こうして、感覚的魂は、それ自身による固有の働きを何ももたず、感覚的魂の
 全ての働きが結合体\footnote{coniunctumは身体(質料)と魂(形相)が結びついたもの
 を指す。より一般的にはcompositumとも言う。}に属することが明らかである。
以上のことから、非理性的動物の魂は、それ自身によって働かず、自存しない
 ことが帰結する。各々のものは、働きをもつのと同様のしかたで存在をもつ
 からである。

\\



{\scshape Ad primum ergo dicendum} quod homo, etsi conveniat in genere cum aliis
animalibus, specie tamen differt, differentia autem speciei attenditur
secundum differentiam formae. Nec oportet quod omnis differentia
formae faciat generis diversitatem.

&

第一異論に対しては、それゆえ以下のように言われるべきである。
人間は他の動物と類において一致するが種において異なる。
しかるに種差は形相の違いに即して見出される。
また、全ての形相の差異が類の異なりを生むとは限らない。

\\



{\scshape Ad secundum dicendum} quod sensitivum quodammodo se habet ad sensibilia
sicut intellectivum ad intelligibilia, inquantum scilicet utrumque est
in potentia ad sua obiecta. Sed quodammodo dissimiliter se habent,
inquantum sensitivum patitur a sensibili cum corporis immutatione,
unde excellentia sensibilium corrumpit sensum. Quod in intellectu non
contingit, nam intellectus intelligens maxima intelligibilium, magis
potest postmodum intelligere minora. Si vero in intelligendo fatigetur
corpus, hoc est per accidens, in quantum intellectus indiget
operatione virium sensitivarum, per quas ei phantasmata praeparantur.

&

第二異論に対しては以下のように言われるべきである。感覚しうるものは可感
的なものに対して、ある意味で、知性認識しうるものが可知的なものに関係す
るようなしかたで関係する。それは、どちらも、自らの対象に対して可能態に
ある点においてである。しかし、ある意味で異なったしかたで関係する。それ
は以下の点においてである。すなわち、感覚しうるものは可感的なものによっ
て身体の変化を伴って受動するので、したがって可感的なものが限度を超えると感覚
を破壊するが、このことは知性においては起こらない。知性は可知的なものの
 うちで最大限のものを認識すると、その後でより小さいものをよりよく知性認識
 しうるからである。もし知性認識において身体が疲労するとすると、それは
 附帯的に、知性がそれによって表象像が知性に準備されるところの感覚的な
 力の働きを必要とする限りにおいてである。

\\



{\scshape Ad tertium dicendum} quod vis motiva est duplex. Una quae imperat
motum, scilicet appetitiva. Et huius operatio in anima sensitiva non
est sine corpore; sed ira et gaudium et omnes huiusmodi passiones sunt
cum aliqua corporis immutatione. Alia vis motiva est exequens motum,
per quam membra redduntur habilia ad obediendum appetitui, cuius actus
non est movere, sed moveri. Unde patet quod movere non est actus
animae sensitivae sine corpore.

&

第三異論に対しては以下のように言われるべきである。動かす力には二通りあ
る。一つは欲求能力のように運動を命令するものであり、このような働きは感
覚的魂の中に身体なしにはない。怒りや喜びやその他そのような情念は、何ら
かの身体の変化を伴ってある。もう一つの動かす力は運動を実行するものであ
り、この力によって四肢は欲求に従う準備ができた状態になる。そしてこの力
の作用は動かすことではなく動かされることである。したがって、動かすこと
は、感覚的魂の身体を伴わない作用ではない。


\end{longtable}
\newpage


\rhead{a.~4}
\begin{center}
{\Large {\bfseries ARTICULUS QUARTUS}}\\
{\large UTRUM ANIMA SIT HOMO}\\
{\footnotesize III {\itshape Sent.}, d.5, q.3, a.2; d.22, q.1, a.1; II
 {\itshape SCG}, cap.57; Opusc.XVI, {\itshape de Unit.Intell.};
 {\itshape De Ent.~et Ess.}, cap.2; VII {\itshape Metaphys.}, lect.9.}\\
{\Large 第四項\\魂は人間か}
\end{center}

\begin{longtable}{p{21em}p{21em}}
{\scshape Ad quartum sic proceditur}. Videtur quod anima sit homo. Dicitur enim
II {\itshape ad Cor}.~{\scshape iv} : {\itshape licet is qui foris est noster homo corrumpatur, tamen
is qui intus est, renovatur de die in diem}. Sed id quod est intus in
homine, est anima. Ergo anima est homo interior.


&

第四項の問題へ議論は以下のように進められる。魂は人間だと思われる。理由
は以下の通り。『コリントの信徒への手紙二』第4章で「外側の私たちの人間
である人が滅びるとしても、内側にいる人は日々に新たにされる」
\footnote{「私たちの外なる人が朽ちるとしても、私たちの内なる人は日々新
たにされていきます。」(4:16)}と言われている。しかし人間において内側の
人とは魂である。ゆえに魂は内なる人である。

\\




2. {\scshape Praeterea}, anima humana est substantia quaedam. Non autem est
 substantia universalis. Ergo est substantia particularis. Ergo est
 hypostasis vel persona. Sed non nisi humana. Ergo anima est homo, nam
 persona humana est homo.


&

さらに、人間の魂はある種の実体である。
しかし普遍的な実体ではない。
ゆえに個的な実体である。
ゆえにそれはヒュポスタシスないしペルソナである。
しかしそれは人間のヒュポスタシスないしペルソナ以外ではない。
ゆえに魂は人間である。なぜなら、人間のペルソナは人間だから。

\\




{\scshape Sed contra est} quod Augustinus, XIX {\itshape de Civ.~Dei}, commendat Varronem,
 qui {\itshape hominem nec animam solam, nec solum corpus, sed animam simul et
 corpus esse arbitrabatur}.


&

しかし反対にアウグスティヌスは『神の国』第19巻で、ワッロを推奨しているが
 彼は、「人間を魂だけでも身体だけでもなく魂であると同時に身体であると
 判断している」。

\\




{\scshape Respondeo dicendum} quod {\itshape animam esse hominem}
dupliciter potest intelligi. Uno modo, quod {\itshape homo} sit anima,
sed {\itshape hic homo} non sit anima, sed compositum ex anima et
corpore, puta Socrates. Quod ideo dico, quia quidam posuerunt solam
formam esse de ratione speciei, materiam vero esse partem individui,
et non speciei.

&

解答する。以下のように言われるべきである。「魂は人間である」は二通りに
理解されうる。一つには、「人間」は魂だが、「この人間」は魂でなくむしろ
魂と身体からの複合体、たとえばソクラテスである、という意味である。私が
こう言うのは、ある人々が、形相だけが種の性格に属し、質料は個体の部分で
あっても種の部分ではないと考えたからである。

\\



Quod quidem non potest esse verum. Nam ad naturam speciei pertinet id
quod significat definitio. Definitio autem in rebus naturalibus non
significat formam tantum, sed formam et materiam. Unde materia est
pars speciei in rebus naturalibus, non quidem materia signata, quae
est principium individuationis; sed materia communis.

&

たしかにこれは真でありえない。なぜなら、種の本性には定義が表示するもの
が属するが、自然的事物において定義は形相だけでなく形相と質料を表示する
からである。したがって自然的事物において質料は種の部分である。ただしこ
の質料は、個体化の根源である指定された質料ではなく、共通の質料である。


\\


Sicut enim de ratione {\itshape huius hominis} est quod sit ex hac
anima et his carnibus et his ossibus; ita de ratione {\itshape
hominis} est quod sit ex anima et carnibus et ossibus. Oportet enim de
substantia speciei esse quidquid est communiter de substantia omnium
individuorum sub specie contentorum.


&

たとえば、「この人間」の性格に、この魂とこれらの肉とこれらの骨からだと
いうことが属するように、「人間」の性格には、魂と肉と骨からだということ
が属する。つまり、種の実体には、その種のもとに含まれるすべての個体の実
体に共通に属するものが含まれなければならない。

\\



Alio vero modo potest intelligi sic, quod etiam {\itshape haec anima}
sit {\itshape hic homo}. Et hoc quidem sustineri posset, si poneretur
quod animae sensitivae operatio esset eius propria sine corpore, quia
omnes operationes quae attribuuntur homini, convenirent soli animae;
illud autem est unaquaeque res, quod operatur operationes illius rei.


&

しかし別のしかたで以下のように理解されることもありうる。すなわち、「こ
の魂」もまた「この人間」である。そしてこれは、もし感覚的魂の働きが身体
なしにそれに固有だと考えられたとすればありえただろう。なぜなら、その場
合には人間に帰属する全ての働きが魂だけに適合しただろうが、ある事物の働
きを行うものが各々の事物だからである。


\\


Unde illud est homo, quod operatur operationes hominis. Ostensum est
autem quod sentire non est operatio animae tantum. Cum igitur sentire
sit quaedam operatio hominis, licet non propria, manifestum est quod
homo non est anima tantum, sed est aliquid compositum ex anima et
corpore. Plato vero, ponens sentire esse proprium animae, ponere
potuit quod homo esset anima utens corpore.


&

したがって、人間の働きをなすものが人間である。しかるに、感覚することは
魂だけの働きでないことが示された。ゆえに感覚することは人間の固有ではな
いもののある種の働きなのだから、人間が魂だけでなく魂と身体とからの何ら
かの複合であることが明らかである。しかしプラトンは、感覚することは魂に
固有のことだと考えるので、人間が、身体を用いる魂であると考えることがで
きた。

\\



{\scshape Ad primum ergo dicendum} quod, secundum philosophum in IX
Ethic., illud potissime videtur esse unumquodque, quod est principale
in ipso, sicut quod facit rector civitatis, dicitur civitas facere. Et
hoc modo aliquando quod est principale in homine, dicitur homo,
aliquando quidem pars intellectiva, secundum rei veritatem, quae
dicitur homo interior; aliquando vero pars sensitiva cum corpore,
secundum aestimationem quorundam, qui solum circa sensibilia
detinentur. Et hic dicitur homo exterior.


&

第一異論に対しては、それゆえ以下のように言われるべきである。『ニコマコ
ス倫理学』第9巻における哲学者によれば、各々のものの中でもっとも強くあ
ると思えるものが、そのものの中で主要なものである。たとえば、国家の指導
者が行うことは、その国家が行う言われる。この意味で、人間の中で主要なも
のが人間と言われることがある。実際、あるときには知性的部分が、事柄の真
理に即して、内なる人間と言われる。またあるときには、身体を伴う感覚的部
分が、ただ可感的なものにかかわる人々の判断によって、人間と言われる。こ
れが「外なる人間」と言われる。

\\




{\scshape Ad secundum dicendum} quod non quaelibet substantia
particularis est hypostasis vel persona, sed quae habet completam
naturam speciei. Unde manus vel pes non potest dici hypostasis vel
persona. Et similiter nec anima, cum sit pars speciei humanae.


&

第二異論に対しては以下のように言われるべきである。どの個的実体もヒュポ
スタシスやペルソナであるわけではなく、種の完全な本性をもつものだけがそ
う言われる。したがって、手や足はヒュポスタシスやペルソナと言われえない。
同様に、魂も、人間の種の部分なのでそうは言われえない。




\end{longtable}
\newpage





\rhead{a.~5}
\begin{center}
{\Large {\bfseries ARTICULUS QUINTUS}}\\
{\large UTRUM ANIMA SIT COMPOSITA EX MATERIA ET FORMA}\\
{\footnotesize I {\itshape Sent.}, d.8, q.5, a.2; II, d.17, q.1, a.2; II {\itshape SCG}, cap.50; {\itshape Quodl}.~III, q.8; IX, q.4, a.1;\\ {\itshape De Spirit.~Creat.}, a.1; a.9, ad 9; Qu.~{\itshape de Anima}, a.6; Opusc.~XV, {\itshape de Angelis}, cap.7.}\\
{\Large 第五項\\魂は質料と形相から複合されているか}
\end{center}

\begin{longtable}{p{21em}p{21em}}

{\scshape Ad quintum sic proceditur}. Videtur quod anima sit composita
ex materia et forma. Potentia enim contra actum dividitur. Sed omnia
quaecumque sunt in actu, participant primum actum, qui Deus est; per
cuius participationem omnia sunt et bona et entia et viventia, ut
patet per doctrinam Dionysii in libro {\itshape de Div.~Nom}. Ergo quaecumque
sunt in potentia, participant primam potentiam. Sed prima potentia est
materia prima. Cum ergo anima humana sit quodammodo in potentia, quod
apparet ex hoc quod homo quandoque est intelligens in potentia;
videtur quod anima humana participet materiam primam tanquam partem
sui.


&

第五項の問題へ議論は以下のように進められる。魂は質料と形相から複合され
ていると思われる。理由は以下の通り。可能態は現実態に対して分けられる。
しかるにすべてなんであれ現実態にあるものは、神である第一の現実態を分有
し、その分有を通して全てのものは善であり有であり生きるものである。これ
はディオニュシウス『神名論』における教説によって明らかなとおりである。
ゆえになんであれ可能態にあるものは、第一の可能態を分有する。しかるに第
一の可能態は第一質料である。ゆえに、人間の魂はある意味で可能態にあるの
だから、というのも人間は時として可能態において知性認識するものなので、
人間の魂は自分の部分として第一質料を分有すると思われる。

\\



2. {\scshape Praeterea}, in quocumque inveniuntur proprietates
materiae, ibi invenitur materia. Sed in anima inveniuntur proprietates
materiae, quae sunt subiici et transmutari, subiicitur enim scientiae
et virtuti, et mutatur de ignorantia ad scientiam, et de vitio ad
virtutem. Ergo in anima est materia.


&

さらに、質料の固有性が見出されるところにはどこにでも、質料が見出される。
しかるに、魂の中には質料の固有性が見出される。それは、元にあることと変
 化することである。つまり知と徳の元にあり、無知から知へ、悪徳から有徳
 へと変化する。ゆえに魂の中に質料がある。

\\



3. {\scshape Praeterea}, illa quae non habent materiam, non habent
causam sui esse, ut dicitur in VIII {\itshape Metaphys}. Sed anima habet causam
sui esse, quia creatur a Deo. Ergo anima habet materiam.


&

さらに、『形而上学』第8巻で言われるように、質料をもたないものは自らの
 存在の原因をもたない。しかるに魂は自らの存在の原因をもつ。ゆえに魂は
 質料をもつ。


\\



4. {\scshape Praeterea}, quod non habet materiam, sed est forma
tantum, est actus purus et infinitus. Hoc autem solius Dei est. Ergo
anima habet materiam.


&

さらに、質料をもたずただ形相だけであるものは、純粋現実態であり無限であ
 る。しかしこれは神にのみ属する。ゆえに魂は質料をもつ。

\\



{\scshape Sed contra est} quod Augustinus probat, in VII {\itshape
super Gen.~ad Litt}., quod anima non est facta nec ex materia
corporali, nec ex materia spirituali.


&

しかし反対に、アウグスティヌスは『創世記逐語注解』第7巻で、魂は物体的
 質料からも霊的質料からも作られていないことを証明している。

\\



{\scshape Respondeo dicendum} quod anima non habet materiam. Et hoc
potest considerari dupliciter. Primo quidem, ex ratione animae in
communi. Est enim de ratione animae, quod sit forma alicuius
corporis. Aut igitur est forma secundum se totam; aut secundum aliquam
partem sui. 


&

解答する。以下のように言われるべきである。
魂は質料をもたない。
このことは二通りに考察されうる。
第一に、魂の性格に基づいて共通的に考察される。
すなわち、魂の性格には、何らかの身体の形相であるということが属する。
ゆえにそれは、それ自身の全体において形相であるか、あるいは自らの
何らかの部分において形相であるかのいずれかである。


\\

Si secundum se totam, impossibile est quod pars eius sit
materia, si dicatur materia aliquod ens in potentia tantum, quia
forma, inquantum forma, est actus; id autem quod est in potentia
tantum, non potest esse pars actus, cum potentia repugnet actui,
utpote contra actum divisa. Si autem sit forma secundum aliquam partem
sui, illam partem, dicemus esse animam, et illam materiam cuius primo
est actus, dicemus esse primum animatum. 



&

もし、それ自身の全体において形相であるならば、それの部分が質料であるこ
とは、可能態においてあるなんらかの有が質料と言われるのであれば、不可能
 である。というのも、形相は形相である限りにおいて現実態だからである。
 しかしたんに可能態にあるものは、現実態の部分でありえない。なぜなら可
 能態は、現実態に対して分割されたものとして、現実態と相容れないからで
 ある。
また、自らのある部分に即して形相であるならば、私たちはその部分を魂と言
 うだろう。そして、第一にそれの現実態であるところのその質料を、私たち
 は「第一に魂をもつもの」と言うだろう。


\\


Secundo, specialiter ex
ratione humanae animae, inquantum est intellectiva. Manifestum est
enim quod omne quod recipitur in aliquo, recipitur in eo per modum
recipientis. Sic autem cognoscitur unumquodque, sicut forma eius est
in cognoscente. Anima autem intellectiva cognoscit rem aliquam in sua
natura absolute, puta lapidem inquantum est lapis absolute. 


&

第二に、特に人間の魂の性格、すなわちそれが知性的である限りにおける性格
に基づいて考察されうる。
全て受け取られるもの(b)は、受け取る者(a)のあり方によってそのもの(a)の中にある。
しかるに、各々のものは、そのものの形相が認識するものの中にあるかぎりで
認識される。しかるに知性的な魂は事物を自らの本性において無条件的に認
識する。たとえば石を石であるかぎりにおいて無条件的に認識する。



\\


Est igitur
forma lapidis absolute, secundum propriam rationem formalem, in anima
intellectiva. Anima igitur intellectiva est forma absoluta, non autem
aliquid compositum ex materia et forma. Si enim anima intellectiva
esset composita ex materia et forma, formae rerum reciperentur in ea
ut individuales, et sic non cognosceret nisi singulare, sicut accidit
in potentiis sensitivis, quae recipiunt formas rerum in organo
corporali, materia enim est principium individuationis
formarum. Relinquitur ergo quod anima intellectiva, et omnis
intellectualis substantia cognoscens formas absolute, caret
compositione formae et materiae.


&

ゆえに、石の形相は無条件的に、固有の形相的な性格において、知性的魂の中
に存在する。ゆえに知性的魂は無条件的な形相であり、質料と形相から複合
された何かではない。
もし知性的魂が質料と形相から複合されていたならば、事物の形相は魂の中に
 個的なものとして存在したであろうし、その結果、ちょうど感覚的な能力に
 おいて生じるように、知性は個物しか認識しなかったであろう。感覚的な能
 力は、事物の形相を身体的器官において受け取るが、質料は形相の個体化の
 根源なのである。ゆえに、形相を無条件的に認識する知性的魂や全ての知的
 実体が質料と形相の複合を欠くことが帰結する。

\\



{\scshape Ad primum ergo dicendum} quod primus actus est universale
principium omnium actuum, quia est infinitum, virtualiter in se omnia
praehabens, ut dicit Dionysius. Unde participatur a rebus, non sicut
pars, sed secundum diffusionem processionis ipsius. 



&

第一異論に対しては、それゆえ以下のように言われるべきである。
ディオニュシウスが言うように、第一の現実態が全ての現実態の普遍的な根源であるのは、それが無限であり能
 力的に自らの中に万物をあらかじめ持っているからである。したがって、そ
 れは諸事物に部分としてではなく、それ自体の発出の拡散に従って分有され
 る。

\\

Potentia autem, cum sit receptiva actus, oportet quod actui
proportionetur.  Actus vero recepti, qui procedunt a primo actu
infinito et sunt quaedam participationes eius, sunt diversi. Unde non
potest esse potentia una quae recipiat omnes actus, sicut est unus
actus influens omnes actus participatos, alioquin potentia receptiva
adaequaret potentiam activam primi actus.

&

他方で可能態は、現実態を受け取りうるものなので、現実態に比例的に
関係する。他方、現実態は、第一の無限の現実態から発出し、それのある種の
分有なので、さまざまに異なっている。したがって、全ての分有された現実
態を注ぎ込む一つの現実態が存在するというかたちで、全ての現実態を受け取
る一つの可能態が存在する、ということはありえない。さもなければ、受容す
る能力が、第一の現実態の作用的な能力に対等したであろう。


\\


Est autem alia potentia receptiva in anima intellectiva,
a potentia receptiva materiae primae, ut patet ex diversitate
receptorum, nam materia prima recipit formas individuales, intellectus
autem recipit formas absolutas. Unde talis potentia in anima
intellectiva existens, non ostendit quod anima sit composita ex
materia et forma.


&

さらに知性的魂の中には、第一質料の受容能力とは別の受容能力がある。この
ことは、受け取られるものの多様性から明らかである。すなわち第一質料は個
 的な諸形相を受け取るが、知性は無条件的な形相を受け取る。したがって、
 そのような能力が知性的魂の中に存在することが、魂が質料と形相から複合さ
 れていることを示していない。


\\



{\scshape Ad secundum dicendum} quod subiici et transmutari convenit
materiae secundum quod est in potentia. Sicut ergo est alia potentia
intellectus, et alia potentia materiae primae, ita est alia ratio
subiiciendi et transmutandi. Secundum hoc enim intellectus subiicitur
scientiae, et transmutatur de ignorantia ad scientiam, secundum quod
est in potentia ad species intelligibiles.


&

第二異論に対しては以下のように言われるべきである。元にあることや変化す
ることが質料に適合するのは、質料が可能態にあることに即してである。ゆえ
に、知性と第一質料の可能態が異なるように、元にあることや変化することの
性格も異なる。つまり知性が知の元にあり、無知から知へ変化するのは、知性
が可知的形象に対して可能態にある限りにおいてである。

\\



{\scshape Ad tertium dicendum} quod forma est causa essendi materiae,
et agens: unde agens, inquantum reducit materiam in actum formae
transmutando, est ei causa essendi. Si quid autem est forma
subsistens, non habet esse per aliquod formale principium, nec habet
causam transmutantem de potentia in actum. Unde post verba praemissa,
philosophus concludit quod in his quae sunt composita ex materia et
forma, {\itshape nulla est alia causa nisi movens ex potestate ad
actum, quaecumque vero non habent materiam, omnia simpliciter sunt
quod vere entia aliquid}.


&

第三異論に対しては以下のように言われるべきである。形相因は質料の存在原
因であるが、能動因もまた質料の存在原因である。したがって能動因は形相の
現実態へ変化させることによって質料を導くかぎりにおいて、質料にとって存
在の原因である。しかし、自存する形相があるならば、それはどんな形相的根
源によっても存在をもたず、可能態から現実態へ変化させる原因ももたない。
したがって、上述の言葉の後に、哲学者は、質料と形相から複合されたものに
おいては「可能態から現実態へと動かす原因以外に他の原因はないが、他方、質料
をもたないものはなんであれ、すべて端的に存在し、真に何かとして存在する」
と結論している。
\footnote{難読箇所。}

\\



{\scshape Ad quartum dicendum} quod omne participatum comparatur ad
participans ut actus eius. Quaecumque autem forma creata per se
subsistens ponatur, oportet quod participet esse, quia etiam ipsa
vita, vel quidquid sic diceretur, participat ipsum esse, ut dicit
Dionysius, V cap. {\itshape de Div.~Nom}. Esse autem participatum finitur ad
capacitatem participantis. Unde solus Deus, qui est ipsum suum esse,
est actus purus et infinitus. In substantiis vero intellectualibus est
compositio ex actu et potentia; non quidem ex materia et forma, sed ex
forma et esse participato. Unde a quibusdam dicuntur componi ex quo
est et quod est, ipsum enim esse est quo aliquid est.


&

第四異論に対しては以下のように言われるべきである。
全て分有されたものは分有するものに対して、それの現実態として関係する。
しかるに自存するとされる創造された形相はなんであれ、存在を分有しなけれ
 ばならない。なぜなら、『神名論』第5章で言われるように、生命それ自体や、
 なんであれそのように言われるものでさえ、存在自体を分有するからである。
しかるに分有された存在は分有するものの受容性に限られる。
したがって、存在それ自体である神だけが純粋現実態であり無限である。
他方、知的諸実体においては、現実態と可能態の複合がある。
それは質料と形相からではなく、形相と分有された存在からの複合である。
したがってある人々から、それは「それによってあるところのもの」と「ある
もの」から複合されていると言われている。存在とは、何かがそれによってあるところのものだからである。

\end{longtable}
\newpage




\rhead{a.~6}
\begin{center}
{\Large {\bfseries ARTICULUS SEXTUS}}\\
{\large UTRUM ANIMA HUMANA SIT CORRUPTIBILIS}\\
{\footnotesize II {\itshape Sent.}, d.19, q.1; IV, d.1, q.1, a.1;
 II {\itshape SCG}, cap.79 sqq.; {\itshape Quodl.}~X, q.3, a.2;
 Qu.~{\itshape de Anima}, a.14; {\itshape Compend.~Theol.}, cap.84.}\\
{\Large 第六項\\人間の魂は不滅か}
\end{center}

\begin{longtable}{p{21em}p{21em}}

{\scshape Ad sextum sic proceditur}. Videtur quod anima humana sit
corruptibilis. Quorum enim est simile principium et similis processus,
videtur esse similis finis. Sed simile est principium generationis
hominum et iumentorum, quia de terra facta sunt. Similis est etiam
vitae processus in utrisque, quia {\itshape similiter spirant omnia, et nihil
habet homo iumento amplius}, ut dicitur {\itshape Eccle}.~{\scshape iii}. Ergo, ut ibidem
concluditur, unus est interitus hominis et iumentorum, et aequa
utriusque conditio. Sed anima brutorum animalium est
corruptibilis. Ergo et anima humana est corruptibilis.


&

第六項の問題へ議論は以下のように進められる。人間の魂は不滅でないと思わ
れる。理由は以下の通り。似た始まりと似た過程をもつものは、似た終わりを
もつと思われる。しかるに人間と家畜の生成の根源は似ている。なぜなら土か
ら作られるからである。さらに両者における生の過程も似ている。なぜなら同
じように全てのものは呼吸をし、また、『伝導の書』第3章で言われているよ
うに、「人間は家畜より多くを持っているわけではない」\footnote{「人の子
らの運命と動物の運命は同じであり、これが死ねば、あれも死ぬ。両者にある
のは同じ息である。人が動物にまさるところはない。すべては空である。」
(3:19)}からである。ゆえに、同じ箇所で結論されているように、人間と家畜
の死は一つであり、どちらにも同じ状態が属する。しかるに非理性的動物の魂
は可滅的である。ゆえに人間の魂も可滅的である。


\\



2. {\scshape Praeterea}, omne quod est ex nihilo, vertibile est in nihilum, quia
finis debet respondere principio. Sed sicut dicitur {\itshape Sap}.~{\scshape ii}, {\itshape ex nihilo
nati sumus}, quod verum est non solum quantum ad corpus, sed etiam
quantum ad animam. Ergo ut ibidem concluditur, {\itshape post hoc erimus tanquam
non fuerimus}, etiam secundum animam.


&

さらに、全て無からであるものは無へと帰りうる。なぜなら終わりは始まりに
対応するべきだから。しかるに『知恵の書』第2章で言われるように「私たち
は無から生まれた」\footnote{「我々は偶然に生まれ/後には存在しなかった
かのようになる。/我々の鼻の息は煙にすぎず/思考は我々の心臓の鼓動から
出る/火花にすぎない。」(2:2)}。これは、身体にかんしてだけでなく魂にか
んしても真である。ゆえに同じところで結論されているように、この世の後、
私たちは魂に関してもあたかもいなかったかのようになる。


\\



3. {\scshape Praeterea}, nulla res est sine propria operatione. Sed propria operatio
animae, quae est intelligere cum phantasmate, non potest esse sine
corpore, nihil enim sine phantasmate intelligit anima; phantasma autem
non est sine corpore, ut dicitur in libro {\itshape de Anima}. Ergo anima non
potest remanere, destructo corpore.


&

さらに、どんな事物も固有の働きをもたないものはない。しかるに魂の固有の
働きは表象像を伴って知性認識することだが、それは身体なしにはありえない。
なぜなら、魂は表象像なしには何も知性認識しないが、表象像は、『デ・アニ
マ』で言われるように、身体なしにはないからである。ゆえに魂は身体が滅び
ると留まり得ない。


\\



{\scshape Sed contra est} quod Dionysius dicit, {\scshape iv} cap.~{\itshape de Div.~Nom}., quod {\itshape animae
humanae habent ex bonitate divina quod sint {\itshape intellectuales} et quod
habeant {\itshape substantialem vitam inconsumptibilem}}.


&

しかし反対に、ディオニュシウスは『神名論』第4章で、「人間の魂は神の善性
 から知性的であることと、燃え尽くされ得ない実体的生命をもつ」と言って
 いる。

\\



{\scshape Respondeo dicendum} quod necesse est dicere animam humanam, quam
dicimus intellectivum principium, esse incorruptibilem. Dupliciter
enim aliquid corrumpitur, uno modo, per se; alio modo, per
accidens. Impossibile est autem aliquid subsistens generari aut
corrumpi per accidens, idest aliquo generato vel corrupto. Sic enim
competit alicui generari et corrumpi, sicut et esse, quod per
generationem acquiritur et per corruptionem amittitur. 


&

解答する。以下のように言われるべきである。人間の魂、私たちはそれを知性
的根源と言うのだが、それは不滅だと言う必要がある。理由は以下の通り。
ものは二通りのしかたで消滅する。一つには自体的にであり、もう一つには附
 帯的にである。
ところで、自存する何かが附帯的に生成消滅する、すなわち、何かが生成したり消滅したり
 することによって生成消滅することは不可能である。というのも、生成消滅
 があるものに属するのは、それに存在が属するのと同じしかたによるからで
 ある。ものは生成によって存在を獲得し、消滅によって存在を失う。


\\

Unde quod per
se habet esse, non potest generari vel corrumpi nisi per se, quae vero
non subsistunt, ut accidentia et formae materiales, dicuntur fieri et
corrumpi per generationem et corruptionem compositorum. Ostensum est
autem supra quod animae brutorum non sunt per se subsistentes, sed
sola anima humana. Unde animae brutorum corrumpuntur, corruptis
corporibus, anima autem humana non posset corrumpi, nisi per se
corrumperetur. 

&

したがって、自体的に存在をもつものは自体的にでなければ生成または消滅し
 えないが、附帯性や質料的形相のような自存しないものどもは、複合体の生
 成消滅によって、生成するとか消滅するとかと言われる。ところで、非理性
 的動物の魂はそれ自体によって自存せず、ただ人間の魂だけが自存すること
 が前に明らかにされた。したがって、非理性的動物の魂は身体が滅びると消
 滅するが、人間の魂は、自体的に消滅しないかぎり、消滅することがありえ
 ない。


\\

Quod quidem omnino est impossibile non solum de ipsa,
sed de quolibet subsistente quod est forma tantum. Manifestum est enim
quod id quod secundum se convenit alicui, est inseparabile ab
ipso. Esse autem per se convenit formae, quae est actus. Unde materia
secundum hoc acquirit esse in actu, quod acquirit formam, secundum hoc
autem accidit in ea corruptio, quod separatur forma ab ea. Impossibile
est autem quod forma separetur a seipsa. Unde impossibile est quod
forma subsistens desinat esse. 

&

しかし自体的に消滅するということは、人間の魂についてだけでなく、形相だ
 けであるどんな自存体についてもありえない。なぜなら、自体的に何かに適
 合するものはそのものから分離しえないからである。したがって、質料は形
 相を獲得する限りにおいて、現実態において存在を獲得し、質料から形相が
 切り離される限りにおいて、質料において消滅が附帯する。しかし、形相が
 形相自身から切り離されることはありえない。したがって、自存する形相が
 存在を失うことはありえない。

\\

Dato etiam quod anima esset ex materia
et forma composita, ut quidam dicunt, adhuc oporteret ponere eam
incorruptibilem. Non enim invenitur corruptio nisi ubi invenitur
contrarietas, generationes enim et corruptiones ex contrariis et in
contraria sunt; unde corpora caelestia, quia non habent materiam
contrarietati subiectam, incorruptibilia sunt. In anima autem
intellectiva non potest esse aliqua contrarietas. Recipit enim
secundum modum sui esse, ea vero quae in ipsa recipiuntur, sunt absque
contrarietate; quia etiam rationes contrariorum in intellectu non sunt
contrariae, sed est una scientia contrariorum. Impossibile est ergo
quod anima intellectiva sit corruptibilis. 


&

さらに、ある人々が実際にそう述べているように、かりに魂が質料と形相から
複合されていたとしても、やはり魂が不滅だとしなければならない。理由は以
下の通り。
生成と消滅は、反対性が見出されるところにしか見出されない。というのも、
 生成と消滅は反対のものからの生成であり反対のものへの消滅だから。この
 ことから天体は、反対性のもとにある質料をもたないので不滅である。とこ
 ろで知性的魂の中に何らかの反対性は存在しえない。というのも、ものは自らの存
 在のあり方にしたがって受け取るが、知性的魂の中に受け取られるものは反
 対性なしにある。というのも反対のものどもについての概念でさえ、知性に
 おいて反対でなく、反対のものどもについての一つの知だからである。
ゆえに、知性的魂が消滅しうることは不可能である。

\\

Potest etiam huius rei
accipi signum ex hoc, quod unumquodque naturaliter suo modo esse
desiderat. Desiderium autem in rebus cognoscentibus sequitur
cognitionem. Sensus autem non cognoscit esse nisi sub hic et nunc, sed
intellectus apprehendit esse absolute, et secundum omne tempus. Unde
omne habens intellectum naturaliter desiderat esse semper. Naturale
autem desiderium non potest esse inane. Omnis igitur intellectualis
substantia est incorruptibilis.


&

さらにこのこと(不滅であること)のしるしは、各々のものは自然本性的に自
 らのしかたで存在を欲求することからも理解されうる。
さて、認識する諸事物における欲求は認識に伴う。
ところで、感覚は、ここと今のもとでなければ存在を認識しないが、知性は存
 在を無条件的にあらゆる時間に即してとらえる。したがって全て知性を持つ
 ものは自然本性的に常に存在することを欲求する。しかるに、自然本性的な
 欲求は無駄にはありえない。ゆえに全ての知性的実体は不滅である。


\\



{\scshape Ad primum ergo dicendum} quod Salomon inducit rationem illam ex persona
insipientium, ut exprimitur Sap.~{\scshape ii}. Quod ergo dicitur quod homo et
alia animalia habent simile generationis principium, verum est quantum
ad corpus, similiter enim de terra facta sunt omnia animalia. Non
autem quantum ad animam, nam anima brutorum producitur ex virtute
aliqua corporea, anima vero humana a Deo. 



&

第一異論に対しては、それゆえ以下のように言われるべきである。
『知恵の書』第2章で説明されているように、ソロモンはかの考えを愚かな人
 から引き出している。
それゆえ、人間と他の動物が類似した生成の根源を持つと言われているのは、
 身体にかんしては真である。すなわち、全ての動物は同様に土から作られて
 いる。しかし魂に関しては真でない。なぜなら非理性的動物の魂は何らかの
 身体の力に基づいて生み出されるが、人間の魂は神によって生み出されるか
 らである。

\\


Et ad hoc significandum
dicitur {\itshape Gen}.~{\scshape i}, quantum ad alia animalia, {\itshape producat terra animam
viventem}, quantum vero ad hominem, dicitur quod {\itshape inspiravit in faciem
eius spiraculum vitae}. Et ideo concluditur {\itshape Eccle}.~ult., {\itshape revertatur
pulvis in terram suam, unde erat, et spiritus redeat ad Deum qui dedit
illum}. 


&

そして、このことを意味するために『創世記』では他の動物たちに関して、
 「土によって生きる魂を作る」と言われるが、人間に関しては「彼の顔へ生
 命の息を吹き込んだ」と言われる。それゆえ『伝統の書』の最後で「塵はそ
 こから来た自らの土へ帰れ、そして霊はそれを与えた神へ帰れ」と結論され
 ている。



\\


Similiter processus vitae est similis quantum ad corpus; ad
quod pertinet quod dicitur in {\itshape Eccle}., {\itshape similiter spirant omnia}; et
{\itshape Sap}.~{\scshape ii}, {\itshape fumus et flatus est in naribus nostris} et cetera. Sed non est
similis processus quantum ad animam, quia homo intelligit, non autem
animalia bruta. Unde falsum est quod dicitur, nihil habet homo iumento
amplius. Et ideo similis est interitus quantum ad corpus, sed non
quantum ad animam.


&

同様に生命の発出は身体に関してにている。『伝導の書』で「万物は同様に息
 をする」とか『知恵の書』第2章「私たちの鼻には煙と風がある云々」と言わ
 れているのはこの点に属する。しかし魂に関しては同様の発出はない。なぜ
 なら人間は知性認識するが、非理性的動物はしないからである。したがって、
 人間が家畜以上のものを何も持たないというのは偽である。ゆえに、身体に
 関して滅びは似ているが、魂に関しては似ていない。


\\



{\scshape Ad secundum dicendum} quod, sicut posse creari dicitur aliquid non per
potentiam passivam, sed solum per potentiam activam creantis, qui ex
nihilo potest aliquid producere; ita cum dicitur aliquid vertibile in
nihil, non importatur in creatura potentia ad non esse, sed in
creatore potentia ad hoc quod esse non influat. Dicitur autem aliquid
corruptibile per hoc, quod inest ei potentia ad non esse.


&

第二異論に対しては以下のように言われるべきである。
創造されうるということが、受動的能力によってではなく、無から何かを生み
 出しうる創造者の能動的能力によってのみ語られうるように、何かが無に帰されうるということも、
被造物の中にある無への能力が意味されるのではなく、創造者の中にある、存
 在を流し込まないことに対する能力が意味される
これに対して、何かが消滅しうるのは、それに非存在への可能性が内在するこ
 とによる。

\\



{\scshape Ad tertium dicendum} quod intelligere cum phantasmate est
propria operatio animae secundum quod corpori est unita. Separata
autem a corpore habebit alium modum intelligendi, similem aliis
substantiis a corpore separatis, ut infra melius patebit.


&

第三異論に対しては以下のように言われるべきである。表象像を伴って知性認
識することが魂の働きに固有であるのは、それが身体に結び付けられている限
りにおいてである。しかし、身体から離れた魂は、これとは違った、身体から
離れた他の実体に似た知性認識のしかたをもつであろうが、これは後によりよ
く明らかになる。





\\

\end{longtable}
\newpage

\rhead{a.~7}
\begin{center}
{\Large {\bfseries ARTICULUS SEPTIMUS}}\\
{\large UTRUM ANIMA ET ANGELUS SINT UNIUS SPECIEI}\\
{\footnotesize II {\itshape Sent.}, d.3, q.1, a.6; II {\itshape SCG.},
 cap.94; Qu.~{\itshape de Anima} a.7.}\\
{\Large 第七項\\魂は天使と同一の種に属するか}
\end{center}

\begin{longtable}{p{21em}p{21em}}

{\scshape Ad septimum sic proceditur}. Videtur quod anima et Angelus
 sint unius speciei. Unumquodque enim ordinatur ad proprium finem per
 naturam suae speciei, per quam habet inclinationem ad finem. Sed idem
 est finis animae et Angeli, scilicet beatitudo aeterna. Ergo sunt
 unius speciei.

&

 第七項の問題へ、議論は以下のように進められる。魂と天使は一つの種に属
 すると思われる。理由は以下の通り。各々のものは、自らの種の本性によっ
 て、固有の目的へと秩序付けられ、そしてその種の本性によって、目的への
 傾向性を持つ。しかるに、魂と天使の目的は同じもの、すなわち、永遠の至
 福である。ゆえに、それらは一つの種に属する。
 
\\



2. {\scshape Praeterea}, ultima differentia specifica est nobilissima,
quia complet rationem speciei. Sed nihil est nobilius in Angelo et
anima quam intellectuale esse. Ergo conveniunt anima et Angelus in
ultima differentia specifica. Ergo sunt unius speciei.

&

さらに、最終の種差は、種の性格を完成するから、もっとも高貴である。しか
るに天使と魂において、知性的であることよりも高貴なものはない。ゆえに、
魂と天使は、最終の種差において一致する。ゆえに、それらは一つの種に属す
る。
 
\\



3. {\scshape Praeterea}, anima ab Angelo differre non videtur nisi per
hoc, quod est corpori unita. Corpus autem, cum sit extra essentiam
animae, non videtur ad eius speciem pertinere. Ergo anima et Angelus
sunt unius speciei.

&

 さらに、魂は、身体と一つになっている以外のことによって、天使と異なる
 ようには思われない。しかるに身体は、魂の本質の外にあるので、その種に
 は属さないと思われる。ゆえに、魂と天使は一つの種に属する。
 
\\



{\scshape Sed contra}, quorum sunt diversae operationes naturales,
ipsa differunt specie. Sed animae et Angeli sunt diversae operationes
naturales, quia ut dicit Dionysius, VII cap.~{\itshape de Div.~Nom}.,
mentes angelicae simplices et beatos intellectus habent, non de
visibilibus congregantes divinam cognitionem; cuius contrarium
postmodum de anima dicit. Anima igitur et Angelus non sunt unius
speciei.

&

 しかし反対に、異なる本性的な働きが属するものは、種において異なる。し
 かるに魂と天使には、異なる本性的働きが属する。というのも、ディオニュ
 シウスが『神名論』第7章で「天使的精神は、単純で至福な知性を持ち、可視
 的なものどもから神の認識を集めない」と言っていて、その反対のことが、
 この後で魂について言われているからである。ゆえに魂と天使は一つの種に
 属さない。
 
\\



{\scshape Respondeo dicendum} quod Origenes posuit omnes animas
 humanas et Angelos esse unius speciei. Et hoc ideo, quia posuit
 diversitatem gradus in huiusmodi substantiis inventam, accidentalem,
 utpote ex libero arbitrio provenientem, ut supra dictum est.

&

解答する。以下のように言われるべきである。オリゲネスは、すべての人間の
魂と天使は一つの種に属すると主張した。それは、前に述べたように、彼が、
自由裁量に基づいて到来し、このような実体の中に見出される、段階の差異を
考えたからである。

 
\\


Quod non potest esse, quia in substantiis incorporeis non potest esse
diversitas secundum numerum absque diversitate secundum speciem, et
absque naturali inaequalitate. Quia si non sint compositae ex materia
et forma, sed sint formae subsistentes, manifestum est quod necesse
erit in eis esse diversitatem in specie.

&

非物体的実体において、種における差異、そして本性的な不等性なしに、数に
おける差異はありえないので、このようなことはありえない。なぜなら、もし、
それらが質料と形相から複合されてなく、自存する形相であるならば、それら
において、種における差異があることが必然だからである。
 
\\


Non enim potest intelligi quod aliqua forma separata sit nisi una
unius speciei, sicut si esset albedo separata, non posset esse nisi
una tantum; haec enim albedo non differt ab illa nisi per hoc, quod
est huius vel illius.

&

実際、ある離在する形相は、一つの種にのみ属するとしか考えられない。たと
えば、もし離在する白があったならば、それは一つだけでしかありえなかった
だろう。なぜなら、この白はあの白から、これの白、あれの白、ということに
よってでなければ異ならないからである。
 
\\

Diversitas autem secundum speciem semper habet diversitatem naturalem
concomitantem, sicut in speciebus colorum unus est perfectior altero,
et similiter in aliis. Et hoc ideo, quia differentiae dividentes genus
sunt contrariae; contraria autem se habent secundum perfectum et
imperfectum, quia principium contrarietatis est privatio et habitus ut
dicitur in X {\itshape Metaphys}.

&

しかるに、種における差異は、常に、それに伴う本性の差異を持つ。例えば、
色の種において、一つの色は他の色よりも完全であり、他のものにおいても同
様であるように。そしてこれは、類を分ける種差が、対立するものだからであ
る。しかるに対立するものは、完全なものと不完全なものというかたちで関係
する。なぜなら、対立の根源は、『形而上学』第10巻で言われているように、
欠如と所有だからである。
 
\\


Idem etiam sequeretur, si huiusmodi substantiae essent compositae ex
materia et forma. Si enim materia huius distinguitur a materia illius,
necesse est quod vel forma sit principium distinctionis materiae, ut
scilicet materiae sint diversae propter habitudinem ad diversas
formas, et tunc sequitur adhuc diversitas secundum speciem et
inaequalitas naturalis.

&

もし、このような実体が質料と形相から複合されていたら、同じことが帰結し
ただろう。つまり、もし、これの質料とあれの質料が区別されるならば、形相
が質料の区別の根源であるか、質料が形相の区別の根源であるかのどちらかで
あることが必然である。前者の場合、それらの質料が、異なる形相への関係の
ために異なることになり、その場合、種と本性的な不等性における差異が帰結
するだろう。
 
\\


Vel materia erit principium distinctionis
 formarum; nec poterit dici materia haec alia ab illa, nisi secundum
 divisionem quantitativam, quae non habet locum in substantiis
 incorporeis, cuiusmodi sunt Angelus et anima. Unde non potest esse
 quod Angelus et anima sint unius speciei. Quomodo autem sint plures
 animae unius speciei infra ostendetur.

&

後者の場合、この質料があの質料と別であるということが、量における区別に
よらなければ言われえないであろう。しかし、量は、天使と魂がそのようであ
る非物体的な実体の中にはない。したがって、天使と魂が一つの種に属するこ
とはできない。しかし、複数の魂が一つの種に属するかということは、以下で
明らかにされる。
 
\\




{\scshape Ad primum ergo dicendum} quod ratio illa procedit de fine
 proximo et naturali. Beatitudo autem aeterna est finis ultimus et
 supernaturalis.

&

 第一異論に対しては、それゆえ、以下のように言われるべきである。この論
 は、最近接で自然的な目的について進んでいる。しかるに永遠の至福は、究
 極の、超自然的な目的である。
 
\\




{\scshape Ad secundum dicendum} quod differentia specifica ultima est
 nobilissima, inquantum est maxime determinata, per modum quo actus
 est nobilior potentia. Sic autem intellectuale non est nobilissimum,
 quia est indeterminatum et commune ad multos intellectualitatis
 gradus, sicut sensibile ad multos gradus in esse sensibili. Unde
 sicut non omnia sensibilia sunt unius speciei, ita nec omnia
 intellectualia.

&

第二異論に対しては、以下のように言われるべきである。最終の種差は、現実
態が可能態よりも高貴であるという意味で、最大に限定されている限りにおい
て、もっとも高貴である。しかるに、知性的なものは、このような意味でもっ
とも高貴なのではない。なぜなら、知性的なものは、ちょうど可感的なものが、
可感的存在における多くの段階に対してあるように、多くの可知性の段階に対
して非限定で共通だからである。したがって、すべての可感的なものが一つの
種に属するわけではないように、すべての可知的なものも同様である。
 
\\




{\scshape Ad tertium dicendum} quod corpus non est de essentia animae,
 sed anima ex natura suae essentiae habet quod sit corpori
 unibilis. Unde nec proprie anima est in specie; sed compositum. Et
 hoc ipsum quod anima quodammodo indiget corpore ad suam operationem,
 ostendit quod anima tenet inferiorem gradum intellectualitatis quam
 Angelus, qui corpori non unitur.

&

 第三異論に対しては、以下のように言われるべきである。身体は魂の本質に
 属さないが、魂は、自らの本質の本性に基づいて、身体に合一されうるとい
 うことをもつ。したがって、厳密には、魂が種の中にあるわけではなく、む
 しろ、複合体が種の中にある。そして、魂が、ある意味で、自らの働きのた
 めに身体を必要とすることそれ自体が、魂が、身体に合一されていない天使
 よりも、知性性の下位の段階にあることを示している。




\end{longtable}

\end{document}

