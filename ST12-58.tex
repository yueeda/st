\documentclass[10pt]{jsarticle}
\usepackage{okumacro}
\usepackage{longtable}
\usepackage[polutonikogreek,english,japanese]{babel}
\usepackage{latexsym}
\usepackage{color}

%----- header -------
\usepackage{fancyhdr}
\pagestyle{fancy}
\lhead{{\it Summa Theologiae} I-II, q.58}
%--------------------

\bibliographystyle{jplain}

\title{{\bf PRIMA SECUNDAE}\\{\HUGE Summae Theologiae}\\Sancti Thomae
Aquinatis\\{\sffamily QUEAESTIO QUINQUAGESIMAOCTAVA}\\DE DISTINCTIONE
VIRTUTUM MORALIUM AB INTELLECTUALIBUS}
\author{Japanese translation\\by Yoshinori {\sc Ueeda}}
\date{Last modified \today}

%%%% コピペ用
%\rhead{a.~}
%\begin{center}
% {\Large {\bf }}\\
% {\large }\\
% {\footnotesize }\\
% {\Large \\}
%\end{center}
%
%\begin{longtable}{p{21em}p{21em}}
%
%&
%
%\\
%\end{longtable}
%\newpage

\begin{document}
\maketitle

\begin{center}
{\LARGE 『神学大全』第二部の一}\\
{\Large 第五十八問\\道徳的徳の知的徳からの区別について}
\end{center}


\begin{longtable}{p{21em}p{21em}}
Deinde considerandum est de virtutibus moralibus. Et primo, de
distinctione earum a virtutibus intellectualibus; secundo, de
distinctione earum ab invicem, secundum propriam materiam; tertio, de
distinctione principalium, vel cardinalium, ab aliis. Circa primum
quaeruntur quinque.

\begin{enumerate}
 \item utrum omnis virtus sit virtus moralis.
 \item utrum virtus moralis distinguatur ab intellectuali.
 \item utrum sufficienter dividatur virtus per intellectualem et moralem.
 \item utrum moralis virtus possit esse sine intellectuali.
 \item utrum e converso, intellectualis virtus possit esse sine
 morali.
\end{enumerate}

&

次に道徳的徳な徳について考察されるべきである。第一に、それらの知的徳か
らの区別について。第二に、その固有の質料に即して、それら相互の区別につ
いて。第三に、他の徳からの、主要なあるいは枢要的な徳の区別について。第
一をめぐって五つのことが問われる。

\begin{enumerate}
 \item すべての徳は道徳的な徳か。
 \item 道徳的な徳は知的な徳から区別されるか。
 \item 知的徳と道徳的徳によって、徳は十分に分けられるか。
 \item 道徳的な徳は知的な徳なしにありうるか。
 \item 逆に、知的な徳は道徳的な徳なしにありうるか。
\end{enumerate}
\end{longtable}
\newpage



\rhead{a.~1}
\begin{center}
{\Large {\bf ARTICULUS PRIMUS}}\\
{\large UTRUM OMNIS VIRTUS SIT MORALIS}\\
{\footnotesize III {\itshape Sent.}, d.23, q.1, a.4, qu$^{a}$2; I {\itshape Ethic.}, lect.20; II, lect. 1.}\\
{\Large 第一項\\すべての徳は道徳的か}
\end{center}

\begin{longtable}{p{21em}p{21em}}
{\scshape Ad primum sic proceditur}. Videtur quod omnis virtus sit
moralis. Virtus enim moralis dicitur a {\itshape more}, idest
consuetudine. Sed omnium virtutum actus consuescere possumus. Ergo
omnis virtus est moralis.

&

第一項の問題へ、議論は以下のように進められる。すべての徳が道徳的だと思
われる。理由は以下の通り。。道徳的(moralis))徳は、習俗(mor)すなわち慣
習から言われる。しかるに私たちはすべての徳の作用を慣習化することができ
る。ゆえにすべての徳は道徳的である。

\\

2. {\scshape Praeterea}, philosophus dicit, in II {\itshape Ethic}.,
quod {\itshape virtus moralis est habitus electivus in medietate rationis
consistens}. Sed omnis virtus videtur esse habitus electivus, quia
actus cuiuslibet virtutis possumus ex electione facere. Omnis etiam
virtus aliqualiter in medio rationis consistit, ut infra patebit. Ergo
omnis virtus est moralis.

&

さらに、哲学者は『ニコマコス倫理学』第2巻で「道徳的徳は理性の媒介にお
いて成立する選択的な習慣である」と述べている。しかるにすべての徳は、選
択的な習慣であると思われる。なぜなら、どのような徳の作用も、わたしたち
はそれを選択に基づいて為すことができるからである。さらにすべての徳は、
後で明らかになるように、何らかのかたちで理性の媒介において成立する。ゆ
えにすべての徳は道徳的である。

\\



3. {\scshape Praeterea}, Tullius dicit, in sua {\itshape Rhetorica},
quod {\itshape virtus est habitus in modum naturae, rationi
consentaneus}. Sed cum omnis virtus humana ordinetur ad bonum hominis,
oportet quod sit consentanea rationi, cum bonum hominis sit secundum
rationem esse, ut Dionysius dicit. Ergo omnis virtus est moralis.

&

さらにキケロは彼の『弁論術』で「徳とは理性に同意する本性のような習慣で
ある」と述べている。しかるにすべての人間的徳は人間の善に秩序付けられて
いるので、それらは理性に同意していなければならない。ディオニュシウスが
言うように、人間の善は理性に即してあることだからである。ゆえにすべての
徳は道徳的である。

\\



{\scshape Sed contra est} quod philosophus dicit, in I {\itshape
Ethic}.: {\itshape Dicentes de moribus, non dicimus quoniam sapiens
vel intelligens; sed quoniam mitis vel sobrius}. Sic igitur sapientia
et intellectus non sunt morales. Quae tamen sunt virtutes, sicut supra
dictum est. Non ergo omnis virtus est moralis.

&

しかし反対に、哲学者は『ニコマコス倫理学』第1巻で「moresについて語ると
き、私たちは知恵があるとか理解力があるとは言わず、温和であるとか節度が
あると言う」と述べている。ゆえに智恵や直知は道徳的なものではない。しか
し、先に述べられたとおり、これらは徳である。ゆえにすべての徳が道徳的で
はない。

\\


{\scshape Respondeo dicendum} quod ad huius evidentiam, considerare
oportet quid sit mos, sic enim scire poterimus quid sit moralis
virtus. Mos autem duo significat.

&

解答する。以下のように言われるべきである。これを明らかにするためには、
mosとは何かを考察しなければならない。そうすれば私たちは、何が道徳的な
(moralis)徳かを知ることができるだろう。さて、mosは二つのことを意味する。

\\



Quandoque enim significat consuetudinem, sicut dicitur {\itshape
Act}.~{\scshape xv}, {\itshape Nisi circumcidamini secundum morem
Moysi, non poteritis salvi fieri}.

&

あるときには、それは慣習を意味する。たとえば『使徒言行録』第15節で「も
しあなたたちがモーゼの慣習に従って割礼を受けなければ救われる人々になる
ことはできないだろう」\footnote{「ある人々がユダヤから下って来て、「モー
セの慣習に従って割礼を受けなければ、あなたがたは救われない」と兄弟たち
に教えていた。」(15:1)}と言われるように。


\\



Quandoque vero significat inclinationem quandam naturalem, vel quasi
naturalem, ad aliquid agendum, unde etiam et brutorum animalium
dicuntur aliqui mores; unde dicitur II {\itshape Machab}.~{\scshape
xi}, quod {\itshape leonum more irruentes in hostes, prostraverunt
eos}. Et sic accipitur mos in Psalmo {\scshape lxvii}, ubi dicitur,
{\itshape Qui habitare facit unius moris in domo}.

&

他方であるときには、為されるべき何かに対する、ある種の自然本性的な、あ
るいは自然本性的であるかのような傾向性を意味する。このことから非理性的
動物にも何らかのmosが属すると言われる。それゆえ『マカバイ記二』第2章
「ライオンのように敵に突進して彼らを打ちのめした」\footnote{「そして獅
子のように敵と激突し、歩兵一万一千と騎兵千六百を打ちのめし、全員に敗走
を余儀なくさせた。」(11:11)}と言われている。また、『詩編』67ではmosが
この意味で理解されていて、そこでは「一人の仕方で家に住まわせる方」
\footnote{「神は孤独な人を家に住ませ/捕らわれ人を幸福へと導き出され
る。/背く者は焼けつく地に住まなければならない。」(68:7)}と言われてい
る。

\\

Et hae quidem duae
significationes in nullo distinguuntur, apud Latinos, quantum ad
vocem. In Graeco autem distinguuntur, nam {\itshape ethos}, quod apud
nos morem significat, quandoque habet primam longam, et scribitur per
\textgreek{h}, Graecam litteram; quandoque habet primam correptam, et
scribitur per \textgreek{e}.

&

そしてこの二つの意味が、発音においてはラテンの人々の間では全く区別され
ない。しかしギリシアの人々の間では区別される。すなわち、私たちのもとで
mosを意味する「エトス」が、長い第一音節を持つときがあり、「エートス」
と書かれ、別の時には短い第一音節を持ち、「エトス」と書かれる。

\\

Dicitur autem virtus moralis a more, secundum quod mos significat
quandam inclinationem naturalem, vel quasi naturalem, ad aliquid
agendum.  Et huic significationi moris propinqua est alia
significatio, qua significat consuetudinem, nam consuetudo quodammodo
vertitur in naturam, et facit inclinationem similem
naturali. 

&

さて、道徳的徳がmosに由来すると言われるのは、mosが、為されるべき事柄に
対するある種の自然本性的、あるいは自然本性的に見える傾向性を意味する限
りにおいてである。それによって習俗を意味するもう一つのmosの意味は、mos
のこの意味に近い。すなわち習俗とは、あるしかたで本性に転換し、自然本性
的な傾向性に類似した傾向性を作るからである。

\\

Manifestum est autem quod inclinatio ad actum proprie
convenit appetitivae virtuti, cuius est movere omnes potentias ad
agendum, ut ex supradictis patet. Et ideo non omnis virtus dicitur
moralis, sed solum illa quae est in vi appetitiva.

&

しかし、先に述べられたことから明らかなとおり、すべての能力を作用するこ
とへと動かすことが属する欲求的な力に、厳密にはこの作用への傾向性が適合
する。ゆえにすべての徳が道徳的とは言われず、ただ欲求的力の中にあるもの
だけがそう言われる。

\\



{\scshape Ad primum ergo dicendum} quod obiectio illa procedit de
more, secundum quod significat consuetudinem.

&

第一異論に対しては、それゆえ、以下のように言われるべきである。
彼の異論は、慣習を意味するmosについて論じている。

\\



{\scshape Ad secundum dicendum} quod omnis actus virtutis potest ex
electione agi, sed electionem rectam agit sola virtus quae est in
appetitiva parte animae, dictum est enim supra quod eligere est actus
appetitivae partis. Unde habitus electivus, qui scilicet est
electionis principium, est solum ille qui perficit vim appetitivam,
quamvis etiam aliorum habituum actus sub electione cadere possint.

&

第二異論に対しては以下のように言われるべきである。すべての徳の作用は、
選択に基づいて為されうるが、魂の欲求的部分の中にある徳だけが、正しい選
択を行うことができる。なぜなら、選ぶことは欲求的部分の作用であると言わ
れたからである。したがって、他の習慣の作用も選択のもとに入ってきうると
しても、選択の根源である選択する習慣は、欲求的力を完成するものだけであ
る。

\\


{\scshape Ad tertium dicendum} quod natura est principium motus, sicut
dicitur in II {\itshape Physic}. Movere autem ad agendum proprium est
appetitivae partis. Et ideo assimilari naturae in consentiendo
rationi, est proprium virtutum quae sunt in vi appetitiva.

&

第三異論に対しては以下のように言われるべきである。『自然学』第2巻で言
われるように、自然本性は運動の根源である。しかるに、固有の為されるべき
事柄へと動くことは欲求的部分に属する。ゆえに、理性に同意することで本性
に類似することは、欲求敵地からの中にある徳に固有のことである。

\end{longtable}
\newpage




\rhead{a.~2}
\begin{center}
{\Large {\bf ARTICULUS SECUNDUS}}\\
{\large UTRUM VIRTUS MORALIS DISTINGUATUR AB INTELLECTUALI}\\
{\footnotesize III {\itshape Sent.}, d.23, q.1, a.4, qu$^{a}$1, a.12; I {\itshape Ethic.}, lect.20.}\\
{\Large 第二項\\道徳的徳は知性的徳から区別されるか}
\end{center}

\begin{longtable}{p{21em}p{21em}}
{\scshape Ad secundum sic proceditur}. Videtur quod virtus moralis ab
intellectuali non distinguatur. Dicit enim Augustinus, in libro
{\itshape de Civ.~Dei}, quod {\itshape virtus est ars recte
vivendi}. Sed ars est virtus intellectualis. Ergo virtus moralis ab
intellectuali non differt.

&

第二項の問題へ、議論は以下のように進められる。道徳的徳は知性的徳から区
別されないと思われる。理由は以下の通り。アウグスティヌスは『神の国』と
いう書物で「徳とは善く生きるための技術である」と述べている。しかるに技
術は知性的徳である。ゆえに道徳的徳は知性的徳から区別されない。

\\


2. {\scshape Praeterea}, plerique in definitione virtutum moralium ponunt
scientiam, sicut quidam definiunt quod perseverantia est {\itshape scientia vel
habitus eorum quibus est immanendum vel non immanendum}; et sanctitas
est scientia faciens fideles et servantes quae ad Deum iusta. Scientia
autem est virtus intellectualis. Ergo virtus moralis non debet
distingui ab intellectuali.

&

 さらに、道徳的徳の定義の中にしばしば知(学知)が置かれる。たとえば、
 ある人々は忍耐を「留まるべきことと留まるべきでないことがらにおける知
 ないし習慣である」と定義する。また気高さを「神に対して正しい、人々を
 信心深くし仕えるものにする知」と定義する。しかるに知は知性的な徳であ
 る。ゆえに道徳的徳は知性的徳から区別される必要がない。

 
\\



3. {\scshape Praeterea}, Augustinus dicit, in I {\itshape Soliloq}.,
quod {\itshape virtus est recta et perfecta ratio}. Sed hoc pertinet
ad virtutem intellectualem, ut patet in VI {\itshape Ethic}. Ergo
virtus moralis non est distincta ab intellectuali.

&

 さらに、アウグスティヌスは『ソリロクィア』第1巻で「徳とは正しく完全な
 理である」と述べている。しかるにこれは『ニコマコス倫理学』第6巻で明ら
 かなとおり、知性的な徳に属する。ゆえに道徳的徳は知性的徳から区別され
 ない。
 
\\



4. {\scshape Praeterea}, nihil distinguitur ab eo quod in eius
definitione ponitur. Sed virtus intellectualis ponitur in definitione
virtutis moralis, dicit enim philosophus, in II {\itshape Ethic}.,
quod {\itshape virtus moralis est habitus electivus existens in
medietate determinata ratione, prout sapiens determinabit}. Huiusmodi
autem recta ratio determinans medium virtutis moralis, pertinet ad
virtutem intellectualem, ut dicitur in VI {\itshape Ethic}. Ergo
virtus moralis non distinguitur ab intellectuali.

&

 さらに、なにものも、その定義の中に置かれるものから区別されない。しか
 るに知性的徳は道徳的徳の定義の中に置かれる。たとえば哲学者は『ニコマ
 コス倫理学』第2巻で「道徳的徳は、ちょうど智恵が決定するであろうように、
 理によって決定された中項において存在する選択的な習慣」と述べている。
 しかるにこのような道徳的徳の中項を決定する正しい理は、『ニコマコス倫
 理学』第6巻で言われるように知性的徳に属する。ゆえに道徳的徳は知性的徳
 から区別されない。
 
\\


{\scshape Sed contra est} quod dicitur in I {\itshape Ethic}.:
{\itshape Determinatur virtus secundum differentiam hanc, dicimus enim
harum has quidem intellectuales, has vero morales}.

&

 しかし反対に、『ニコマコス倫理学』第1巻で「徳はこの際にしたがって限定
 される。すなわち、私たちは、これらのあるものを知性的と言い、あるもの
 を道徳的と言う」と述べられている。


 \\



 {\scshape Respondeo dicendum} quod omnium humanorum operum principium
 primum ratio est, et quaecumque alia principia humanorum operum
 inveniantur, quodammodo rationi obediunt; diversimode tamen. Nam
 quaedam rationi obediunt omnino ad nutum, absque omni contradictione,
 sicut corporis membra, si fuerint in sua natura consistentia; statim
 enim ad imperium rationis, manus aut pes movetur ad opus.


&

 解答する。以下のように言われるべきである。すべての人間の業の第一の根
 源は理性であり、人間の業の他の根源は何であれ、何らかの仕方で理性に従
 う。しかしその従い方はさまざまである。すなわちあるものは、全く逆らう
 ことなく、完全に意のままに、理性に従う。たとえば身体の四肢は、もしそ
 れが自らの本性と整合的であれば、ただちに理性の命令に従う。ちょうど、
 手や足が業へと動かされるように。

 
\\

 Unde philosophus dicit, in I {\itshape Polit}., quod {\itshape anima
 regit corpus despotico principatu}, idest sicut dominus servum, qui
 ius contradicendi non habet. Posuerunt igitur quidam quod omnia
 principia activa quae sunt in homine, hoc modo se habent ad
 rationem. Quod quidem si verum esset, sufficeret quod ratio esset
 perfecta, ad bene agendum.


&

 このことから哲学者は『政治学』第1巻で「魂は身体を絶対的な主権で支配す
 る」すなわち、主人が逆らう権利のない奴隷を支配するように支配すると述
 べている。ゆえにある人々は、人間の中にあるすべての能動的な根源が、こ
 のようなかたちで理性に関係すると考えた。このことがもし真であったなら
 ば、善く作用するためには、理性が完全であることが十分であっただろう。

 
\\

 Unde, cum virtus sit habitus quo perficimur ad bene agendum,
 sequeretur quod in sola ratione esset, et sic nulla virtus esset nisi
 intellectualis. Et haec fuit opinio Socratis, qui dixit {\itshape
 omnes virtutes esse prudentias}, ut dicitur in VI {\itshape
 Ethic}. Unde ponebat quod homo, scientia in eo existente, peccare non
 poterat; sed quicumque peccabat, peccabat propter ignorantiam.

&

 したがって、徳はそれによって私たちが善く作用するために完成されるとこ
 ろのものなので、徳はただ理性の中だけにあることになっただろう。そして
 知性的徳以外のどんな徳もなかったでであろう。そしてこれがソクラテスの
 意見であった。彼は、『ニコマコス倫理学』第6巻で言われているように、す
 べての徳が思慮であると言った。このことから彼は、人間は、その中に知が
 あるときには、罪を犯すことができないのであって、だれであれ罪を犯す人
 は、無知のために罪を犯すと言った。

 
\\

 Hoc autem procedit ex suppositione falsi. Pars enim appetitiva obedit
 rationi non omnino ad nutum, sed cum aliqua contradictione, unde
 philosophus dicit, in I {\itshape Polit}., quod ratio imperat appetitivae
 principatu politico, quo scilicet aliquis praeest liberis, qui habent
 ius in aliquo contradicendi.


&

しかしこれは、偽の前提に基づいて論を進めている。というのも、欲求的部分
は理性に全く意のままに従うのではなく、何らかの反対を伴って従う。このこ
とから哲学者は『政治学』第1巻で「理性は欲求的部分に、政治的な主権で命
じる」すなわち、ある人が、ある程度言い返す権利を持つ自由人たちに命じる
ような仕方で命じると述べている。
 
\\


 Unde Augustinus dicit, {\itshape super Psalm}., quod {\itshape
 interdum praecedit intellectus, et sequitur tardus aut nullus
 affectus}, intantum quod quandoque passionibus vel habitibus
 appetitivae partis hoc agitur, ut usus rationis in particulari
 impediatur.

&

 このことからアウグスティヌスは『詩編注解』で「時として知性は先を行き、
 感情はのろのろと付いていく、あるいはまったく付いていかない」と述べて
 いる。つまり欲求的部分の情念や習慣によって、個別的なことにおける理性
 の使用が妨げられるということが起こる限りにおいて。
 
\\


 Et secundum hoc, aliqualiter verum est quod Socrates dixit, quod
 scientia praesente, non peccatur, si tamen hoc extendatur usque ad
 usum rationis in particulari eligibili.

&

 そしてこの点で、ある意味においてソクラテスが言ったこと、すなわち「知
 が現在しているときは、罪を犯すことはない」というのは真である。しかし
 それは、(知の現在が)個別的な選択可能な事柄における理性の使用にまで
 拡張される場合に限られる。
 
\\



 Sic igitur ad hoc quod homo bene agat, requiritur quod non solum
 ratio sit bene disposita per habitum virtutis intellectualis; sed
 etiam quod vis appetitiva sit bene disposita per habitum virtutis
 moralis.

&

 それゆえ以上のことから、人間が善く作用するためには、ただ理性が知性的
 な徳の習慣によって善く態勢付けられるだけではなく、欲求的な力が道徳的
 な徳の習慣によって善く態勢付けられることが必要である
 
\\



 Sicut igitur appetitus distinguitur a ratione, ita virtus moralis
 distinguitur ab intellectuali. Unde sicut appetitus est principium
 humani actus secundum quod participat aliqualiter rationem, ita
 habitus moralis habet rationem virtutis humanae, inquantum rationi
 conformatur.

&

 ゆえに、ちょうど欲求が理性から区別されるように、道徳的な徳が知性的な
 徳から区別される。したがって、欲求が何らかの仕方で理性を分有する限り
 において人間的行為の根源であるように、道徳的習慣は理性に合致する限り
 において人間的な徳の性格を持つ。

 
\\


{\scshape Ad primum ergo dicendum} quod Augustinus communiter accipit
artem, pro qualibet recta ratione. Et sic sub arte includitur etiam
prudentia, quae ita est recta ratio agibilium, sicut ars est recta
ratio factibilium. Et secundum hoc, quod dicit quod virtus est ars
recte vivendi, essentialiter convenit prudentiae, participative autem
aliis virtutibus, prout secundum prudentiam diriguntur.

&

 第一異論に対しては、それゆえ、以下のように言われるべきである。アウグ
 スティヌスは技術を共通的に、なんであれ正しい理として理解している。そ
 の意味で、技術のもとに思慮もまた含まれる。思慮は、ちょうど技術が作ら
 れるべき事柄の正しい理であるように、為されるべき事柄の正しい理である。
 そしてこの限りで、徳は正しく生きる技術であると言われることは、本質的
 に思慮に属し、思慮に従って命じられる限りにおいて、分有というかたちで
 他の諸徳にも属する。
 
\\



{\scshape Ad secundum dicendum} quod tales definitiones, a
quibuscumque inveniantur datae, processerunt ex opinione Socratica, et
sunt exponendae eo modo quo de arte praedictum est.

&

 第二異論に対しては以下のように言われるべきである。そのような定義は、
 だれによって与えられたものであれ、ソクラテスの意見に基づいて出てきた
 のであって、それらは技術についてすでに述べられたかたちで説明されるべ
 きである。
 
\\


Et similiter dicendum est ad tertium.

&

 第三異論に対しても同様に言われるべきである。
 
\\



{\scshape Ad quartum dicendum} quod recta ratio, quae est secundum
prudentiam, ponitur in definitione virtutis moralis, non tanquam pars
essentiae eius, sed sicut quiddam participatum in omnibus virtutibus
moralibus, inquantum prudentia dirigit omnes virtutes morales.

&

 第四異論に対しては、以下のように言われるべきである。思慮に従ってある
 正しい理は、道徳的な徳の定義のなかに置かれるが、それはその本質の部分
 として置かれているのではなく、思慮がすべての道徳的な徳を指導するかぎ
 りにおいて、すべての道徳的な徳の中にある種分有されたものとして置かれ
 ている。

\end{longtable}
\newpage


\rhead{a.~3}
\begin{center}
{\Large {\bf ARTICULUS TERTIUS}}\\
{\large UTRUM SUFFICIENTER VIRTUS DIVIDATUR PER MORALEM ET INTELLECTUALEM}\\
{\Large 第三項\\道徳的徳と知的徳によって徳は十分に分けられているか}
\end{center}

\begin{longtable}{p{21em}p{21em}}
{\scshape Ad tertium sic proceditur}. Videtur quod virtus humana non
sufficienter dividatur per virtutem moralem et
intellectualem. Prudentia enim videtur esse aliquid medium inter
virtutem moralem et intellectualem, connumeratur enim virtutibus
intellectualibus in VI {\itshape Ethic}.; et etiam ab omnibus
communiter connumeratur inter quatuor virtutes cardinales, quae sunt
morales, ut infra patebit. Non ergo sufficienter dividitur virtus per
intellectualem et moralem, sicut per immediata.

&

第三項の問題へ議論は以下のように進められる。人間的な徳は道徳的徳と知性
的徳によって十分に分けられていないと思われる。理由は以下の通り。思慮は
道徳的徳と知的徳の中間の何かであると思われる。実際、『ニコマコス倫理学』
第6巻で、それは知的徳とともに数えられているが、この先で\footnote{Q.61,
a.1.}明らかになるように、道徳的徳である四つの枢要徳の中に、すべての人々
によって共通に数えられている。ゆえに徳は、知的徳と道徳的徳によって、直
接的な部分によって分けられるかたちで十分に分けられない。

\\



2. {\scshape Praeterea}, continentia et perseverantia, et etiam
patientia, non computantur inter virtutes intellectuales. Nec etiam
sunt virtutes morales, quia non tenent medium in passionibus, sed
abundant in eis passiones. Non ergo sufficienter dividitur virtus per
intellectuales et morales.

&

さらに、自制、堅忍、さらには忍耐は、知的な徳には数えられない。さらに、
それらは道徳的な徳でもない。なぜなら、それらは情念の中で中庸を保持せず、
むしろそれらの中に情念があふれているからである。ゆえに徳は知的徳と道徳
的徳によって十分に分けられない。

\\



3. {\scshape Praeterea}, fides, spes et caritas quaedam virtutes
sunt. Non tamen sunt virtutes intellectuales, hae enim solum sunt
quinque, scilicet scientia, sapientia, intellectus, prudentia et ars,
ut dictum est. Nec etiam sunt virtutes morales, quia non sunt circa
passiones, circa quas maxime est moralis virtus. Ergo virtus non
sufficienter dividitur per intellectuales et morales.

&

さらに、信仰、希望、愛はある種の徳である。しかしそれらは知的徳ではない。
なぜならすでに述べられたとおり、知的徳とは学知、智恵、直知、思慮、技術
の五つだけだからである。また道徳的徳でもない。なぜなら、それらは道徳的
徳が最大限にそれを巡ってあるところの情念にかんするものではないからであ
る。ゆえに徳は知的徳と道徳的徳によって十分に分けられない。

\\



{\scshape Sed contra est} quod philosophus dicit, in II {\itshape
Ethic}., {\itshape duplicem esse virtutem, hanc quidem intellectualem,
illam autem moralem}.

&

しかし反対に、哲学者は『ニコマコス倫理学』第2巻で、「徳に二とおりあり、
一つは知的、もう一つは道徳的である」と述べている。

\\



{\scshape Respondeo dicendum} quod virtus humana est quidam habitus
perficiens hominem ad bene operandum. Principium autem humanorum
actuum in homine non est nisi duplex, scilicet intellectus sive ratio,
et appetitus, haec enim sunt duo moventia in homine, ut dicitur in III
{\itshape de Anima}. 

&

解答する。以下のように言われるべきである。人間的な徳は、善く働くために
人間を完成するある種の習慣である。しかるに人間における人間的な行為の根
源は二とおり、すなわち、知性ないし理性と欲求しかない。これは『デ・アニ
マ』第3巻で述べられているとおりである。

\\

Unde omnis virtus humana oportet quod sit
perfectiva alicuius istorum principiorum. Si quidem igitur sit
perfectiva intellectus speculativi vel practici ad bonum hominis
actum, erit virtus intellectualis, si autem sit perfectiva appetitivae
partis, erit virtus moralis. Unde relinquitur quod omnis virtus humana
vel est intellectualis vel moralis.

&

したがって、すべての人間的な徳は、これらの諸根源のうちのどれかを完成し
うるものでなければならない。ゆえにもし、人間の善い行為に向けて観照的あ
るいは実践的知性を完成しうるならば、それは知的徳であろうし、もし欲求的
部分を完成しうるものであれば、それは道徳的徳であろう。したがって、人間
的なすべての徳は知的であるか道徳的であるかのどちらかである。

\\


{\scshape Ad primum ergo dicendum} quod prudentia, secundum essentiam
suam, est intellectualis virtus. Sed secundum materiam, convenit cum
virtutibus moralibus, est enim recta ratio agibilium, ut supra dictum
est. Et secundum hoc, virtutibus moralibus connumeratur.

&

第一異論に対しては、それゆえ、以下のように言われるべきである。思慮は、
それ自身の本質に即しては知的な徳である。しかしその質料(対象)に即して
は、道徳的な徳と一致する。すなわちそれは、前に述べられたとおり、為され
るべき事柄についての正しい理だからである。この限りで、それは道徳的徳と
ともに数えられる。

\\


{\scshape Ad secundum dicendum} quod continentia et perseverantia non
sunt perfectiones appetitivae virtutis sensitivae. Quod ex hoc patet,
quod in continente et perseverante superabundant inordinatae
passiones, quod non esset, si appetitus sensitivus esset perfectus
aliquo habitu conformante ipsum rationi.

&

第二異論に対しては以下のように言われるべきである。自制と堅忍は感覚的欲
求の徳の完成ではない。このことは、自制や堅忍をする人の中に、無秩序な情
念があふれていることから明らかである。もし感覚的欲求が、それを理性に一
致させる何らかの習慣によって完成されていたら、このことはなかったであろ
う。


\\

Est autem continentia, seu perseverantia, perfectio rationalis partis,
quae se tenet contra passiones ne deducatur. Deficit tamen a ratione
virtutis, quia virtus intellectiva quae facit rationem se bene habere
circa moralia, praesupponit appetitum rectum finis, ut recte se habeat
circa principia, idest fines, ex quibus ratiocinatur; quod continenti
et perseveranti deest.

&

他方、自制ないし堅忍は、引きずられないように情念に反対することを保って
いる理性的部分の完成である。しかし徳の性格は欠いている。なぜなら、道徳
的な事柄を巡って理性を善い状態に保つ知的徳は、それに基づいて推論を行う
ところの原理すなわち目的を巡って善い状態にあるように、目的への正しい欲
求を前提するが、それを、自制や堅忍は欠くからである。

\\


Neque etiam potest esse perfecta operatio quae a duabus potentiis
procedit, nisi utraque potentia perficiatur per debitum habitum, sicut
non sequitur perfecta actio alicuius agentis per instrumentum, si
instrumentum non sit bene dispositum, quantumcumque principale agens
sit perfectum.

&

さらに、二つの能力から出てくる働きは、その両方の能力がしかるべき習慣に
よって完成されていないかぎり、完全な働きではありえない。それはちょうど、
主要な作用者がどれほど完全であっても、道具が善く態勢付けられていないか
ぎり、道具によるある作用者の作用が完全であることは帰結しないようにであ
る。

\\


Unde si appetitus sensitivus, quem movet rationalis pars, non sit
perfectus; quantumcumque rationalis pars sit perfecta, actio
consequens non erit perfecta. Unde nec principium actionis erit
virtus. Et propter hoc, continentia a delectationibus, et
perseverantia a tristitiis, non sunt virtutes, sed aliquid minus
virtute, ut philosophus dicit, in VII {\itshape Ethic}.

&

したがって、もし理性的部分が動かす感覚的欲求が完全でなければ、理性的部
分がどれほど完全であっても、帰結する作用は完全ではないだろう。したがっ
て、作用の根源も徳ではないことになる。このため、哲学者が『ニコマコス倫
理学』第7巻で述べるように、快楽における自制と悲しみからの堅忍は、徳で
はなく、徳よりも小さい何かである。


\\



3 {\scshape Ad tertium} dicendum quod fides, spes et caritas sunt
supra virtutes humanas, sunt enim virtutes hominis prout est factus
particeps divinae gratiae.

&

第三異論に対しては以下のように言われるべきである。信仰、希望、愛は人間
的な徳よりも上位である。なぜなら、それらは神の恩恵を分かち持つものにさ
れるかぎりでの人間の徳だからである。

\end{longtable}
\newpage



\rhead{a.~4}
\begin{center}
{\Large {\bf ARTICULUS QUARTUS}}\\
{\large UTRUM MORALIS VIRTUS POSSIT ESSE SINE INTELLECTUALI}\\
{\footnotesize Infra, q.65, a.1; {\itshape Virtut.}, q.5, a.2; {\itshape Quodlib.}~XII, q.15, a.1; VI {\itshape Ethic.}, lect.10, 11.}\\
{\Large 第四項\\道徳的徳は知性的徳なしにありうるか}
\end{center}

\begin{longtable}{p{21em}p{21em}}
{\scshape Ad quartum sic proceditur}. Videtur quod virtus moralis
possit esse sine intellectuali. Virtus enim moralis, ut dicit Tullius,
est {\itshape habitus in modum naturae, rationi consentaneus}. Sed
natura etsi consentiat alicui superiori rationi moventi, non tamen
oportet quod illa ratio naturae coniungatur in eodem, sicut patet in
rebus naturalibus cognitione carentibus. Ergo potest esse in homine
virtus moralis in modum naturae, inclinans ad consentiendum rationi,
quamvis illius hominis ratio non sit perfecta per virtutem
intellectualem.

&

第四項の問題へ議論は以下のように進められる。道徳的徳は知性的徳なしにあ
りうると思われる。理由は以下の通り。道徳的徳は、キケロが言うように「理
性に一致した、本性のようなあり方をする習慣」である。しかるに、認識を欠
く自然的事物において明らかなとおり、本性は上位の何らかの動かす理性に一
致するとしても、その理性が同一のものにおいて本性に結び付けられていなく
てもかまわない。ゆえに、知性的徳によって完成された理性がその人間に属し
ていなくても、人間の中に、理性に一致する方へ傾く本性のようなあり方をす
る道徳的徳がありうる。

\\



2. {\scshape Praeterea}, per virtutem intellectualem homo consequitur
rationis usum perfectum. Sed quandoque contingit quod aliqui in quibus
non multum viget usus rationis, sunt virtuosi et Deo accepti. Ergo
videtur quod virtus moralis possit esse sine virtute intellectuali.

&

さらに、知性的徳によって人間は理性の完全な使用を獲得する。しかるにある
人々がある事柄において、それほど華々しく理性の使用を行っていないのに、
有徳であり神に受け入れられることがある。ゆえに道徳的徳は知性的徳なしに
ありうると思われる。

\\



3. {\scshape Praeterea}, virtus moralis facit inclinationem ad bene
operandum. Sed quidam habent naturalem inclinationem ad bene
operandum, etiam absque rationis iudicio. Ergo virtutes morales
possunt esse sine intellectuali.

&

さらに、道徳的徳は善く働くことへの傾向性を作る。しかるにある人々は、理
性の判断がなくても、善く働くことへの自然本性的な傾向性を持っている。ゆ
えに道徳的徳は知性的な徳なしにありうる。

\\



{\scshape Sed contra est} quod Gregorius dicit, in XXII {\itshape
Moral}., quod {\itshape ceterae virtutes, nisi ea quae appetunt,
prudenter agant, virtutes esse nequaquam possunt}. Sed prudentia est
virtus intellectualis, ut supra dictum est. Ergo virtutes morales non
possunt esse sine intellectualibus.

&

しかし反対に、グレゴリウスは『道徳論』第22巻で「ある徳は、もし欲求する
ものを思慮深く行わないと、徳であることがけっしてできない」と述べている。
しかるに前に述べられたとおり、思慮は知性的な徳である。ゆえに道徳的徳は
知性的徳なしにはありえない。

\\



{\scshape Respondeo dicendum} quod virtus moralis potest quidem esse
sine quibusdam intellectualibus virtutibus, sicut sine sapientia,
scientia et arte, non autem potest esse sine intellectu et
prudentia. Sine prudentia quidem esse non potest moralis virtus, quia
moralis virtus est habitus electivus, idest faciens bonam electionem.

&

解答する。以下のように言われるべきである。道徳的徳は、智恵や学知や技術
といったある知的徳がなくてもありうるが、しかし直知と思慮なしにはありえ
ない。そして思慮なしに道徳的徳はありえない。なぜなら道徳的徳は選択しう
る習慣、すなわち善い選択を行う習慣だからである。

\\



Ad hoc autem quod electio sit bona, duo requiruntur. Primo, ut sit
debita intentio finis, et hoc fit per virtutem moralem, quae vim
appetitivam inclinat ad bonum conveniens rationi, quod est finis
debitus.

&

ところで、選択が善いためには二つのことが必要とされる。一つには、目的へ
のしかるべき意図であり、これは道徳的な徳によって生じる。なぜなら、それ
は欲求的な力を理性に一致する善、すなわちしかるべき目的へと傾かせるから
である。

\\

Secundo, ut homo recte accipiat ea quae sunt ad finem, et hoc non
potest esse nisi per rationem recte consiliantem, iudicantem et
praecipientem; quod pertinet ad prudentiam et ad virtutes sibi
annexas, ut supra dictum est. Unde virtus moralis sine prudentia esse
non potest.

&

もう一つには、人間が正しく目的のためにあるものを理解することであり、こ
れは正しく思量し判断し命令する理性によらなければありえない。そしてこの
ことは、上述の如く、思慮とそれに結びついた諸徳に属する。したがって道徳
的徳は思慮なしにありえない。

\\

Et per consequens nec sine intellectu. Per intellectum enim
cognoscuntur principia naturaliter nota, tam in speculativis quam in
operativis. Unde sicut recta ratio in speculativis, inquantum procedit
ex principiis naturaliter cognitis, praesupponit intellectum
principiorum; ita etiam prudentia, quae est recta ratio agibilium.

&

そしてその結果、直知なしにもありえない。というのも、直知によって観照的
な事柄においても活動的な事柄においても自然本性的に知られる諸原理が認識
されるからである。したがって、ちょうど正しい理性が、観照的な事柄におい
て、自然本性的に知られた諸原理から進めるかぎりにおいて、諸原理の知解を
前提するように、なされるべき事柄についての正しい理である思慮もまた、そ
うである。

\\



{\scshape Ad primum ergo dicendum} quod inclinatio naturae in rebus
carentibus ratione, est absque electione, et ideo talis inclinatio non
requirit ex necessitate rationem. Sed inclinatio virtutis moralis est
cum electione, et ideo ad suam perfectionem indiget quod sit ratio
perfecta per virtutem intellectualem.

&

第一異論に対しては、それゆえ、以下のように言われるべきである。理性を欠
く事物における本性の傾向性は選択を欠くので、そのような傾向性が必然的に
理性を必要とすることはない。しかし道徳的徳の傾向性は選択を伴うので、ゆ
えに自らの完成のために知性的徳によって完成された理性を必要とする。

\\



{\scshape Ad secundum dicendum} quod in virtuoso non oportet quod
vigeat usus rationis quantum ad omnia, sed solum quantum ad ea quae
sunt agenda secundum virtutem. Et sic usus rationis viget in omnibus
virtuosis. Unde etiam qui videntur simplices, eo quod carent mundana
astutia, possunt esse prudentes; secundum illud {\itshape
Matth}.~{\scshape x}, {\itshape Estote prudentes sicut serpentes, et
simplices sicut columbae}.

&

第二異論に対しては以下のように言われるべきである。有徳な人においては、
すべてのものにかんして理性を大いに用いる必要はなく、その必要があるのは
徳にしたがって行われるべき事柄にかんしてのみである。この意味で、理性の
使用はすべての有徳な人において大いに行われる。したがって、世間的な如
才なさを欠いているために朴訥に見える人であっても、思慮深いことがありう
る。これは『マタイによる福音書』第10章「ヘビのように思慮深く、ハトのよ
うに朴訥でありなさい」\footnote{「私があなたがたを遣わすのは、狼の中に
羊を送り込むようなものである。だから、あなたがたは蛇のように賢く、鳩の
ように無垢でありなさい。」(10:16)}による。

\\



{\scshape Ad tertium dicendum} quod naturalis inclinatio ad bonum
virtutis, est quaedam inchoatio virtutis, non autem est virtus
perfecta. Huiusmodi enim inclinatio, quanto est fortior, tanto potest
esse periculosior, nisi recta ratio adiungatur, per quam fiat recta
electio eorum quae conveniunt ad debitum finem, sicut equus currens,
si sit caecus, tanto fortius impingit et laeditur, quanto fortius
currit. 

&

第三異論に対しては以下のように言われるべきである。徳の善へ向かう自然本
性的な傾向性は、ある種の徳の始まりではあるが、完全な徳ではない。なぜな
ら、そのような傾向性は、強ければ強いほど、正しい理性によって補助されな
いかぎり、危険なものでありうるからである。その理性によってしかるべき目
的に一致するものどもについての正しい選択が行われる。たとえば走る馬は、
目が見えない場合、よりパワフルに走っているほど、より激しくぶつかりひど
い怪我をするように。

\\



Et ideo, etsi virtus moralis non sit ratio recta, ut Socrates dicebat;
non tamen solum est {\itshape secundum rationem rectam}, inquantum
inclinat ad id quod est secundum rationem rectam, ut Platonici
posuerunt; sed etiam oportet quod sit {\itshape cum ratione recta}, ut
Aristoteles dicit, in VI {\itshape Ethic}.

&

ゆえに、道徳的徳はソクラテスが言ったようには正しい理性でないが、しかし、
プラトン派の人々が考えたように、正しい理性にしたがってあるところのもの
へと傾く限りにおいて「正しい理性に即して」であるだけでなく、アリストテ
レスが『ニコマコス倫理学』第6巻で言うように、「正しい理性とともに」で
なければならない。


\end{longtable}
\newpage


\rhead{a.~5}
\begin{center}
{\Large {\bf ARTICULUS QUINTUS}}\\
{\large UTRUM INTELLECTUALIS VIRTUS POSSIT ESSE SINE MORALI}\\
{\footnotesize Infra, q.65, a.1; {\itshape De Virtut.}, q.5, a.2; {\itshape Quodl.}~XII, q.15, a.1; VI {\itshape Ethic.}, lect.10.}\\
{\Large 第五項\\知性的徳は道徳的徳なしにありうるか}
\end{center}

\begin{longtable}{p{21em}p{21em}}
{\scshape Ad quintum sic proceditur}. Videtur quod virtus
intellectualis possit esse sine virtute morali. Perfectio enim prioris
non dependet a perfectione posterioris. Sed ratio est prior appetitu
sensitivo, et movens ipsum. Ergo virtus intellectualis quae est
perfectio rationis, non dependet a virtute morali, quae est perfectio
appetitivae partis. Potest ergo esse sine ea.


&

第五項の問題へ議論は以下のように進められる。知性的徳は道徳的徳なしにあり
うると思われる理由は以下の通り。より先の完全性はより後の完全性に依存し
ない。しかるに理性は感覚的欲求より先であり、それを動かすものである。ゆ
えに理性の完成である知性的徳は、欲求的部分の完成である道徳的徳に依存し
ない。ゆえにそれなしにありうる。

\\



2. {\scshape Praeterea}, moralia sunt materia prudentiae, sicut
factibilia sunt materia artis. Sed ars potest esse sine propria
materia, sicut faber sine ferro. Ergo et prudentia potest esse sine
virtutibus moralibus, quae tamen inter omnes intellectuales virtutes,
maxime moralibus coniuncta videtur.

&

さらに、道徳的な事柄は思慮の質料(対象)である。ちょうど、作られうるも
のが技術の質料であるように。しかるに技術は固有の質料なしにありうる。た
とえば鍛冶屋は鉄なしにありうる。ゆえに思慮も道徳的な徳なしにありうる。
しかし思慮はすべての知性的な徳の中で最も道徳的なものに結びついているよ
うに見える。

\\



3. {\scshape Praeterea}, {\itshape prudentia est virtus bene
consiliativa}, ut dicitur in VI {\itshape Ethic}. Sed multi bene
consiliantur, quibus tamen virtutes morales desunt. Ergo prudentia
potest esse sine virtute morali.

&

さらに、思慮は『ニコマコス倫理学』第6巻で言われているように、善く思量
しうる徳である。しかるに多くの人々は、道徳的徳がないのに善く思量する。
ゆえに思慮は道徳的な徳なしにありうる。

\\



{\scshape Sed contra}, velle malum facere opponitur directe virtuti
morali; non autem opponitur alicui quod sine virtute morali esse
potest. Opponitur autem prudentiae quod {\itshape volens peccet}, ut
dicitur in VI {\itshape Ethic}. Non ergo prudentia potest esse sine
virtute morali.

&

しかし反対に、悪をなすことを意志することは、直接的に道徳的な徳に対立す
るが、道徳的徳なしにありうるものには対立しない。しかるに思慮には、『ニ
コマコス倫理学』第6巻で言われるように、「みずから意志して失敗する」こ
とが対立する。ゆえに思慮は道徳的な徳なしにありえない。

\\



{\scshape Respondeo dicendum quod} aliae virtutes intellectuales sine
virtute morali esse possunt, sed prudentia sine virtute morali esse
non potest. Cuius ratio est, quia prudentia est recta ratio agibilium;
non autem solum in universali, sed etiam in particulari, in quibus
sunt actiones.

&

解答する。以下のように言われるべきである。思慮以外の知性的徳は道徳的徳
なしにありうるが、思慮は道徳的徳なしにありえない。その理由は、思慮は、
普遍においてだけでなく、そこに行為がある個別的なものにおいても、なされ
るべき事柄についての正しい理だからである。


 \\


Recta autem ratio praeexigit principia ex quibus ratio
procedit. Oportet autem rationem circa particularia procedere non
solum ex principiis universalibus, sed etiam ex principiis
particularibus. Circa principia quidem universalia agibilium, homo
recte se habet per naturalem intellectum principiorum, per quem homo
cognoscit quod nullum malum est agendum; vel etiam per aliquam
scientiam practicam.

&

しかるに正しい理は、それに基づいて理が進む原理を必要とする。しかし、個
別的なことについての理は、普遍的な諸原理だけでなく、個別的な諸原理に基
づいても進められる。為されるべき事柄についての普遍的な原理をめぐって、
人間は、諸原理についての自然本性的な知性によって正しく関係する。この知
性によって、あるいは何らかの実践的な学知によっても、人間はどんな悪もな
すべきでないことを認識する。

\\


Sed hoc non sufficit ad recte ratiocinandum circa
particularia. Contingit enim quandoque quod huiusmodi universale
principium cognitum per intellectum vel scientiam, corrumpitur in
particulari per aliquam passionem, sicut concupiscenti, quando
concupiscentia vincit, videtur hoc esse bonum quod concupiscit, licet
sit contra universale iudicium rationis.

&

しかしこのことは、個別的なことを巡って正しく推論するために十分でない。
なぜなら、そのような知性や学知によって認識された普遍的な根源は、何らか
の情念によって個別的なものにおいて時として消滅するからである。たとえば、
欲情する人にとって、欲情が勝つとき、理性の普遍的な判断に反するにもかか
わらず、欲情するものが善であるように思われるように。

\\

Et ideo, sicut homo disponitur ad recte se habendum circa principia
universalia, per intellectum naturalem vel per habitum scientiae; ita
ad hoc quod recte se habeat circa principia particularia agibilium,
quae sunt fines, oportet quod perficiatur per aliquos habitus secundum
quos fiat quodammodo homini connaturale recte iudicare de fine. Et hoc
fit per virtutem moralem, virtuosus enim recte iudicat de fine
virtutis, quia {\itshape qualis unusquisque est, talis finis videtur
ei}, ut dicitur in III {\itshape Ethic}. Et ideo ad rectam rationem
agibilium, quae est prudentia, requiritur quod homo habeat virtutem
moralem.


&

ゆえに、ちょうど人間が、自然本性的な知性や学知の習慣によって、普遍的な
諸根源をめぐって正しい状態に態勢付けられるように、なされうる事柄につい
ての個別的な諸原理すなわち目的をめぐって、正しい状態に態勢付けられるた
めに、それによって目的について正しく判断することが人間にある意味でとも
に本性的になるような何らかの習慣によって完成されなければならない。そし
てこれが、道徳的な徳によってなされる。じっさい、『ニコマコス倫理学』第
3巻で言われるように、「その人がどのような人であるかに応じて、目的がそ
の人にそのように見える」ので、有徳な人は正しく徳の目的について判断する
からである。ゆえに、なされるべき事柄についての正しい理、これが思慮であ
るが、には人間が道徳的徳を持つことが必要とされる。

\\



{\scshape Ad primum ergo dicendum} quod ratio, secundum quod est
apprehensiva finis, praecedit appetitum finis, sed appetitus finis
praecedit rationem ratiocinantem ad eligendum ea quae sunt ad finem,
quod pertinet ad prudentiam. Sicut etiam in speculativis, intellectus
principiorum est principium rationis syllogizantis.


&

 第一異論に対しては、それゆえ以下のように言われるべきである。理性は目
 的をとらえうる限りにおいて、目的への欲求に先行するが、目的への欲求は
 目的のためにあるものどもを選択するための推論する理性に先行する。この
 選択は思慮に属する。これはちょうど、観照的な事柄においても同様であり、
 諸原理についての直知が三段論法で推論する理性の根源である。


\\



{\scshape Ad secundum dicendum} quod principia artificialium non
diiudicantur a nobis bene vel male secundum dispositionem appetitus
nostri, sicut fines, qui sunt moralium principia, sed solum per
considerationem rationis. Et ideo ars non requirit virtutem
perficientem appetitum, sicut requirit prudentia.

&

 第二異論に対しては以下のように言われるべきである。技術的な事柄につい
 ての根源は、道徳的な事柄の根源である目的がのように、私たちによって、
 私たちの欲求の態勢に応じて善くあるいは悪く判断されることはなく、理性
 の考察によってのみ判断される。ゆえに技術は、思慮が必要とするようには、
 欲求を完成させる徳を必要としない。

 
\\



{\scshape Ad tertium dicendum} quod prudentia non solum est bene
consiliativa, sed etiam bene iudicativa et bene praeceptiva. Quod esse
non potest, nisi removeatur impedimentum passionum corrumpentium
iudicium et praeceptum prudentiae; et hoc per virtutem moralem.

&

第三異論に対しては以下のように言われるべきである。思慮は善く思量しうる
ものであるだけでなく、善く判断し善く命令しうるものでもある。このことは、
思慮の判断と命令をだめにする情念の妨げが取り除かれないかぎりありえない。
\end{longtable}
\end{document}