\documentclass[10pt]{jsarticle} % use larger type; default would be 10pt
%\usepackage[utf8]{inputenc} % set input encoding (not needed with XeLaTeX)
%\usepackage[round,comma,authoryear]{natbib}
%\usepackage{nruby}
\usepackage{okumacro}
\usepackage{longtable}
%\usepqckage{tablefootnote}
\usepackage[polutonikogreek,english,japanese]{babel}
%\usepackage{amsmath}
\usepackage{latexsym}
\usepackage{color}
\usepackage{url}
%----- header -------
\usepackage{fancyhdr}
\lhead{{\it Summa Theologiae} I-IIae, q.~22}
%--------------------

\bibliographystyle{jplain}

\title{{\bf PRIMA SECUNDAE}\\{\HUGE Summae Theologiae}\\Sancti Thomae
Aquinatis}
\author{Japanese translation\\by Yoshinori {\sc Ueeda}}
\date{Last modified \today}


%%%% コピペ用
%\rhead{a.~}
%\begin{center}
% {\Large {\bf }}\\
% {\large }\\
% {\footnotesize }\\
% {\Large \\}
%\end{center}
%
%\begin{longtable}{p{21em}p{21em}}
%
%&
%
%
%\\
%\end{longtable}
%\newpage







\begin{document}
\maketitle
\pagestyle{fancy}

\rhead{Prologos}

\begin{center}
 {\Large {\bf QUAESTIO VESTIMASECUNDA}}\\
 {\large DE SUBIECTO PASSIONUM ANIMAE}\\

 {\Large 第二十二問\\魂の情念の基体について}
\end{center}

\begin{longtable}{p{21em}p{21em}}
Post hoc considerandum est de passionibus animae, et primo, in generali;
 secundo, in speciali. In generali autem, quatuor occurrunt circa eas
 consideranda, primo quidem, de subiecto earum; secundo, de differentia
 earum; tertio, de comparatione earum ad invicem; quarto, de malitia et
 bonitate ipsarum. Circa primum quaeruntur tria. 


\begin{enumerate}
 \item utrum aliqua passio sit in anima.
 \item utrum magis in parte appetitiva quam in apprehensiva.
 \item utrum magis sit in appetitu sensitivo quam intellectivo, qui dicitur voluntas.
\end{enumerate}
 

 
&

このあとに、魂の諸情念について考察されるべきである。第一に一般的に、第二
 に特殊的に。一般的には四つのことがそれらについて考察されるべく生じる。
 第一にそれらの基体について、第二にそれらの差異について、第三にそれら相
 互の関係について、第四にそれらの悪さと善さについて。第一の事柄について
 三つのことが問われる。


 \begin{enumerate}
  \item 魂の中に何らかの情念があるか。
  \item (魂の)把握的部分よりも欲求的部分の中にあるか。
  \item 意志と言われる知性的欲求よりも感覚的欲求の中にあるか。
 \end{enumerate}

\end{longtable}

\newpage
\rhead{a.1}
\begin{center}
 {\Large {\bf ARTICULUS PRIMUS}}\\
 {\large UTRUM ALIQUA PASSIO SIT IN ANIMA}\\
 {\footnotesize III {\itshape Sent.}, d.15, q.2, a.1, qu${^a}$ 2; {\itshape
 De Verit.}, q.26, a.1, 2.}\\
 {\Large 第一項\\魂の中に何らかの情念があるか}
\end{center}

\begin{longtable}{p{21em}p{21em}}


{\scshape Ad primum sic proceditur}. Videtur quod nulla passio sit in
 anima. Pati enim est proprium materiae. Sed anima non est composita ex
 materia et forma, ut in primo habitum est. Ergo nulla passio est in
 anima.

&

第一の問題へ議論は以下のように進められる。
魂の中にどんな情念(受動)\footnote{passio(情念)がpatior(受動する)に由
 来する点に注意。}もないと思われる。理由は以下の通り。
受動するということは質料に固有である。しかるに第一部で論じられたとおり、魂は質料と形相から複合され
 ていない。ゆえにどんな情念も魂の中にない。

\\


2. {\scshape Praeterea}, passio est motus, ut dicitur in III {\itshape Physic}. Sed anima non
 movetur, ut probatur in I {\itshape de Anima}. Ergo passio non est in
 anima.

&

さらに、『自然学』第3巻で述べられているように情念は運動である。しかるに
 『デ・アニマ』第1巻で証明されているように魂は動かされない。ゆえに魂の中
 に情念はない。


\\



3. {\scshape Praeterea}, passio est via in corruptionem, nam {\itshape omnis passio, magis
 facta, abiicit a substantia}, ut dicitur in libro {\itshape Topicorum}. Sed anima
 est incorruptibilis. Ergo nulla passio est in anima.

&

さらに情念は消滅への道である。なぜなら『トピカ』という書物で述べられてい
 るように「すべての受動は多く為されると基体を破壊する」からである。
しかし魂は不可滅的なものである。ゆえにどんな受動も魂の中にない。


\\



{\scshape Sed contra est} quod Apostolus dicit, {\itshape ad Rom}.~{\scshape vii}, {\itshape Cum essemus in carne,
 passiones peccatorum, quae per legem erant, operabantur in membris
 nostris}. Peccata autem sunt proprie in anima. Ergo et passiones, quae
 dicuntur {\itshape peccatorum}, sunt in anima.

&

しかし反対に使徒は「ローマの信徒への手紙」第7章で「私たちは肉の中にあっ
 たので、法によってあった罪の諸情念が私たちの四肢の中で働いていた」\footnote{「私たちが肉にあったときは、律法による罪の欲情が五体の内に働き、死に至る実を結んでいました。」(7:5)}と言
 う。しかるに、罪は固有の意味で魂の中にある。ゆえに「罪の」と言われる情
 念も魂の中にある。


\\



{\scshape Respondeo dicendum} quod pati dicitur tripliciter. Uno modo, communiter,
 secundum quod omne recipere est pati, etiam si nihil abiiciatur a re,
 sicut si dicatur aerem pati, quando illuminatur. Hoc autem magis
 proprie est perfici, quam pati. 

&

解答する。以下のように言われるべきである。「受ける」は三つの意味で言われる。
一つは共通的に、すべて「受け取る」は「受ける」であることにしたがって、何
 もその事物から取り除かれなくても、「受ける」と言われる。たとえば「空気
 が照らされるときに空気が〔光を〕受ける」と言われるように。



\\
--- Alio modo dicitur pati proprie, quando
 aliquid recipitur cum alterius abiectione. Sed hoc contingit
 dupliciter. Quandoque enim abiicitur id quod non est conveniens rei,
 sicut cum corpus animalis sanatur, dicitur pati, quia recipit
 sanitatem, aegritudine abiecta. 

&

もう一つの意味では厳密な意味で「受ける」が語られるが、それは別のなにかが
捨てられることを伴ってなにかが受け取られるときである。しかるにこのことは
二通りのしかたで生じる。ある場合には、事物に適合しないものが捨てられる。
 たとえば病気が捨てられて健康を受け取るので、動物の身体が癒されるときに
「受ける」と言われる場合がそれである。


\\

--- Alio modo, quando e converso contingit,
 sicut aegrotare dicitur pati, quia recipitur infirmitas, sanitate
 abiecta. Et hic est propriissimus modus passionis. Nam pati dicitur ex
 eo quod aliquid trahitur ad agentem, quod autem recedit ab eo quod est
 sibi conveniens, maxime videtur ad aliud trahi. Et similiter in I {\itshape de
 Generat}. dicitur quod, quando ex ignobiliori generatur nobilius, est
 generatio simpliciter, et corruptio secundum quid, e converso autem
 quando ex nobiliori ignobilius generatur. 


&


もう一つのしかたは、それが逆に生じるとき、たとえば健康が捨てられて弱さ
を受け取るがゆえに、病気になることが「受ける」と言われるときである。
そしてこれが、もっとも厳密な意味での受動のあり方である。なぜなら、「受け
る」はなにかが作用者へと引かれることに基づいて語られるが\footnote{英訳
 ``For a thing is said to be passive from its being drawn to the
 agent:'' (\url{https://www.newadvent.org/summa/} Literally translated by Fathers of the English Dominican Province
Online Edition Copyright {\copyright} 2017 by Kevin Knight.)}、自らに適合して
 いるものから退くものが、最大限に他のものへ引かれているように見える
 からである。同様に『生成消滅論』第1巻で「高貴でないものから高貴なものが
 生成するとき、それは端的に生成でありある意味で消滅だが、より高貴なも
 のから高貴でないものが生じるときはその逆である、と述べられている。


\\

Et his tribus modis contingit
 esse in anima passionem. Nam secundum receptionem tantum dicitur quod
 {\itshape sentire et intelligere est quoddam pati}. Passio autem cum abiectione
 non est nisi secundum transmutationem corporalem, unde passio proprie
 dicta non potest competere animae nisi per accidens, inquantum scilicet
 compositum patitur. Sed et in hoc est diversitas, nam quando huiusmodi
 transmutatio fit in deterius, magis proprie habet rationem passionis,
 quam quando fit in melius. Unde tristitia magis proprie est passio quam
 laetitia.

&

そしてこれら三つのしかたで魂の中に情念がありうる。すなわち、ただ受動にし
 たがってのみでは、感覚することや知性認識することがある種の「受ける」だ
 と言われる。しかし棄却を伴う受動は物体的変容においてでなければないので、
 受動は厳密な意味では魂に適合しえず、複合体が受動する限りにおいてただ附
 帯的にのみ適合する。しかし、このことにおいても違いがあり、このような変
 容が悪い方へ生じるとき、よい方へ生じるときよりもより厳密な意味で受動の
 性格を持つ。したがって悲しみは喜びよりもより固有の意味で受動である。

\\



{\scshape Ad primum igitur dicendum} quod pati, secundum quod est cum abiectione et
 transmutatione, proprium est materiae, unde non invenitur nisi in
 compositis ex materia et forma. Sed pati prout importat receptionem
 solam, non est necessarium quod sit materiae, sed potest esse
 cuiuscumque existentis in potentia. Anima autem, etsi non sit composita
 ex materia et forma, habet tamen aliquid potentialitatis, secundum quam
 convenit sibi recipere et pati, secundum quod intelligere pati est, ut
 dicitur in III {\itshape de Anima}.

&

第一異論に対しては、それゆえ、以下のように言われるべきである。
「受ける」は棄却や変容を伴う限りにおいて質料に固有である。したがって、質
 料と形相から複合されているものにおいてでなければ見出されない。しかした
 だ受容だけを意味する「受ける」は質料に属する必要がなく、可能態にあるも
 のなら何にでも属しうる。他方で魂は、質料と形相から複合されていないが、
 何らかの可能態性をもっていて、それに従って受け取ることや受けることが適
 合するのであり、その限りで『デ・アニマ』第3巻で述べられるように、知性認識することが受けることである。

\\



{\scshape Ad secundum dicendum} quod pati et moveri, etsi non conveniat animae per
 se, convenit tamen ei per accidens, ut in I {\itshape de Anima} dicitur.

&

第二異論に対しては、次のように言われるべきである。受けることと動くことは、
 それ自体によって魂に適合するのではないが、『デ・アニマ』第1巻で述べられ
 ているように、附帯的にそれに適合する。


\\



{\scshape Ad tertium dicendum} quod ratio illa procedit de passione quae est cum
 transmutatione ad deterius. Et huiusmodi passio animae convenire non
 potest nisi per accidens, per se autem convenit composito, quod est
 corruptibile.

&

第三異論に対しては、以下のように言われるべきである。この論は、悪い方への
 変容を伴う受動について論じている。そのような受動は附帯的にしか魂に適合
 せず、可滅的である複合体に、自体的に適合する。


\end{longtable}
\newpage
\rhead{a.~2}
\begin{center}
 {\Large {\bf ARTICULUS SECUNDUS}}\\
 {\large UTRUM PASSIO MAGIS SIT IN PARTE APPETITIVA QUAM IN APPREHENSIVA}\\
 {\footnotesize III {\itshape Sent.}, d.15, q.2, a.1, qu$^a$ 2;
 {\itshape De Verit.}, q.26, a.3; {\itshape De Div.~Nom.}, cap.2,
 lect.4; II {\itshape Ethic.}, lect.5.}\\
 {\Large 第二項\\情念は把握的部分よりも欲求的部分の中にあるか}
\end{center}

\begin{longtable}{p{21em}p{21em}}

{\scshape Ad secundum sic proceditur}. Videtur quod passio magis sit in parte
 animae apprehensiva quam in parte appetitiva. Quod enim est primum in
 quolibet genere videtur esse maximum eorum quae sunt in genere illo, et
 causa aliorum, ut dicitur in II {\itshape Metaphys}. Sed passio prius invenitur in
 parte apprehensiva quam in parte appetitiva, non enim patitur pars
 appetitiva, nisi passione praecedente in parte apprehensiva. Ergo
 passio est magis in parte apprehensiva quam in parte appetitiva.


&

第二項の問題へ議論は以下のように進められる。
情念は魂の欲求的部分より把握的部分にあると思われる。理由は以下の通り。
『形而上学』第2巻で述べられているとおり、どんな類においてもその中で第一
 のものは、その類の中にあるものどものうちで最大であり、また他のものども
 の原因である。しかるに情念(受動)は欲求的部分よりも把握的部分の中で先
 に見出される。なぜなら、欲求的部分は、把握的部分における受動が先行する
 ことによってのみ受動するからである。ゆえに、情念(受動)は欲求的部分よ
 りも把握的部分の中にある。

\\



2. {\scshape Praeterea}, quod est magis activum, videtur esse minus passivum, actio
 enim passioni opponitur. Sed pars appetitiva est magis activa quam pars
 apprehensiva. Ergo videtur quod in parte apprehensiva magis sit passio.


&

さらに、能動と受動は対立するので、より多く能動的なものはより少なく受動的だと
 思われる。しかるに欲求的部分は把握的部分よりもより能動的である。ゆえに
 把握的部分の中にむしろ情念(受動)は存在する。

\\


3. {\scshape Praeterea}, sicut appetitus sensitivus est virtus in organo
 corporali, ita et vis apprehensiva sensitiva. Sed passio animae fit,
 proprie loquendo, secundum transmutationem corporalem. Ergo non magis
 est passio in parte appetitiva sensitiva quam in apprehensiva
 sensitiva.



&


さらに、ちょうど感覚的欲求が身体的器官の力(virtus)であるように、感覚的
 把握力もまたそうである。しかるに、厳密に言うならば、魂の受動は身体的な
 変容にしたがって生じる。ゆえに、感覚的把握の部分よりも感覚的欲求の部分
 の中により受動があるわけではない。

\\


{\scshape Sed contra est quod} Augustinus dicit, in IX {\itshape de
 Civ.~Dei}, quod {\itshape motus
 animi, quos Graeci pathe, nostri autem quidam, sicut Cicero,
 perturbationes, quidam affectiones vel affectus, quidam vero, sicut in
 Graeco habetur, expressius passiones vocant}. Ex quo patet quod
 passiones animae sunt idem quod affectiones. Sed affectiones manifeste
 pertinent ad partem appetitivam, et non ad apprehensivam. Ergo et
 passiones magis sunt in appetitiva quam in apprehensiva.


&

しかし反対に、アウグスティヌスは『神の国』第9巻で次のように述べている。
 「ギリシア人たちがパテーと呼ぶ魂の運動は、私たちのもとでは、キケロのよ
 うにある人々は擾乱(perturbatio)、別の人々は感情(affectioやaffectus)、またある人々は、
 ギリシア人に理解されているように、よりはっきりと情念(passio)と呼んでいる」。
 このことから魂の情念は感情と同じであることが明らかである。
しかるに、感情は明らかに欲求的部分に属し、把握的部分には属さない。ゆ
 えに情念も把握的部分よりも欲求的部分にある。

\\



{\scshape Respondeo dicendum} quod, sicut iam dictum est, in nomine passionis
 importatur quod patiens trahatur ad id quod est agentis. Magis autem
 trahitur anima ad rem per vim appetitivam quam per vim
 apprehensivam. Nam per vim appetitivam anima habet ordinem ad ipsas
 res, prout in seipsis sunt, unde philosophus dicit, in VI {\itshape Metaphys}.,
 quod {\itshape bonum et malum, quae sunt obiecta appetitivae potentiae, sunt in
 ipsis rebus}. 


&


解答する。以下のように言われるべきである。
すでに述べられたとおり、情念(受動)という名称においては、受けるものが、
 作用するものに属する事柄へと引かれていくことが意味されている。しか
 るに、魂は把握の力よりも欲求の力によって、より事物へと引かれていく。な
 ぜなら、魂は欲求の力によって、それ自体において存在するかぎりでの事物そ
 のものへ秩序を有する。このことから哲学者は『形而上学』第6巻で「欲求能力の対象である
 善と悪は事物自体の中にある」と述べる。

\\

Vis autem apprehensiva non trahitur ad rem, secundum quod
 in seipsa est; sed cognoscit eam secundum intentionem rei, quam in se
 habet vel recipit secundum proprium modum. Unde et ibidem dicitur quod
 {\itshape verum et falsum}, quae ad cognitionem pertinent, {\itshape non sunt in rebus, sed
 in mente}. Unde patet quod ratio passionis magis invenitur in parte
 appetitiva quam in parte apprehensiva.


&

これに対して把握の力は、それ自体において存在するかぎりでの事物へと引かれ
 るのではなく、事物を事物の概念(intentio)に即して認識する。その概念は、
 〔魂に〕固有のしかたに即して、自らの中に所持しあるいは受け取る。
したがって、同じ箇所で認識に属する「真と偽は事物ではなく精神の中にある」
 と言われている。
ゆえに、情念(受動)という性格は把握的部分より欲求的部分においてより見出
 されることが明らかである。

\\


{\scshape Ad primum ergo dicendum} quod e contrario se habet in his quae pertinent
 ad perfectionem, et in his quae pertinent ad defectum. Nam in his quae
 ad perfectionem pertinent, attenditur intensio per accessum ad unum
 primum principium, cui quanto est aliquid propinquius, tanto est magis
 intensum, sicut intensio lucidi attenditur per accessum ad aliquid
 summe lucidum, cui quanto aliquid magis appropinquat, tanto est magis
 lucidum. 


&

第一異論に対しては、それゆえ、以下のように言われるべきである。
完全性に属しているものどもと不完全性に属しているものどもとでは、その中に
 逆の関係がある。
すなわち、完全性に属しているものどもの中では、それに近いほどより強く何か
 であるような一つの第一の根源へ接近する
ことによって、強化が見出される。たとえば近づけば近づくほど明るくなるような
 何か最高に明るいものへの接近によって、光の強化が見出されるように。


\\

Sed in his quae ad defectum pertinent, attenditur intensio non
 per accessum ad aliquod summum, sed per recessum a perfecto, quia in
 hoc ratio privationis et defectus consistit. Et ideo quanto minus
 recedit a primo, tanto est minus intensum, et propter hoc, in principio
 semper invenitur parvus defectus, qui postea procedendo magis
 multiplicatur. 

&

しかし不完全性に属するものどもの中では、ある最高のものへの接近によってで
 はなく、完全なものからの離脱によって強化が見出される。なぜならこのこと
 〔完全なものからの離脱〕において欠如や不完全の性格が見出されるからである。ゆえに、第一のものか
 ら離脱していないほど、それだけ強度が低く、そのため根源においては常に少
 ない不完全さが見出され、その後、そこから出ていくことによってより多数化され
 る。

\\

Passio autem ad defectum pertinet, quia est alicuius
 secundum quod est in potentia. Unde in his quae appropinquant primo
 perfecto, scilicet Deo, invenitur parum de ratione potentiae et
 passionis, in aliis autem consequenter, plus. Et sic etiam in priori vi
 animae, scilicet apprehensiva, invenitur minus de ratione passionis.

&

ところで情念(受動)は、可能態においてある限りでの何かに属するから、不完
 全なものに属する。したがって、第一根源すなわち神に接近するものどもにお
 いては可能態や受動という性格がより少なく見出され、そうでないものどもに
 おいては、結果的に、より多く見出される。このようにして、魂のより先の力、
 すなわち欲求的な力において、より少なく受動の性格が見出される。



\\



{\scshape Ad secundum dicendum} quod vis appetitiva dicitur esse magis activa, quia
 est magis principium exterioris actus. Et hoc habet ex hoc ipso ex quo
 habet quod sit magis passiva, scilicet ex hoc quod habet ordinem ad rem
 ut est in seipsa, per actionem enim exteriorem pervenimus ad
 consequendas res.


&

第二異論に対しては、以下のように言われるべきである。
欲求の力がより能動的だと言われるのは、それがより外的作用の根源だからであ
 る。そしてこのことは、それがより受動的であるということの根拠、
すなわち、それが、それ自体においてあるものとしての事物へと秩序をもつとい
 うことに由来する。なぜなら、私たちは外的な作用を通して事物を獲得することに
 到達するのだから。


\\



{\scshape Ad tertium dicendum} quod, sicut in primo dictum est, dupliciter organum
 animae potest transmutari. Uno modo, transmutatione spirituali,
 secundum quod recipit intentionem rei. Et hoc per se invenitur in actu
 apprehensivae virtutis sensitivae, sicut oculus immutatur a visibili,
 non ita quod coloretur, sed ita quod recipiat intentionem coloris. 


&

第三異論に対しては、以下のように言われるべきである。第一部で述べられたよ
 うに、魂の器官は二通りのしかたで変容しうる。一つには霊的な変容によって
 であり、それによって事物のインテンチオを受け取る。そしてこれは感覚的な
 把握の力の作用において自体的に現実態において見出される。たとえば目が見
 られうるものによって、色づくようにしてではなく色のイン
 テンチオを受け取るようにして変容を受ける場合のように。


\\

Est
 autem alia naturalis transmutatio organi, prout organum transmutatur
 quantum ad suam naturalem dispositionem, puta quod calefit aut
 infrigidatur, vel alio simili modo transmutatur. Et huiusmodi
 transmutatio per accidens se habet ad actum apprehensivae virtutis
 sensitivae, puta cum oculus fatigatur ex forti intuitu, vel dissolvitur
 ex vehementia visibilis. 


&

これとは違う別の自然的な器官の変容がある。それは器官が自らの自然的な状態
 にかんして変容される場合であり、熱せられるとか冷やされるとか、その他類
 似のしかたによって変容を受ける場合である。そしてこのような変容は感覚的
 把握の力の作用に附帯的に関係する。たとえば、目が強く見ることで疲労した
 り、見られうるものの強さによって破壊されたりしたときのように。


\\



Sed ad actum appetitus sensitivi per se
 ordinatur huiusmodi transmutatio, unde in definitione motuum
 appetitivae partis, materialiter ponitur aliqua naturalis transmutatio
 organi; sicut dicitur quod {\itshape ira est accensio sanguinis circa cor}. Unde
 patet quod ratio passionis magis invenitur in actu sensitivae virtutis
 appetitivae, quam in actu sensitivae virtutis apprehensivae, licet
 utraque sit actus organi corporalis.


&


しかし、感覚的欲求の作用に対して、このような変容は自体的に秩序付けられる。
このことから、欲求的部分の運動の定義の中に、器官の何らかの自然的変容が質
 料的に措定される。たとえば「怒りは心臓の周りの血の発火である」というよ
 うに。したがって、感覚的把握の力の作用も感覚的欲求の力の作用もどちらも
 身体的器官の作用だが、後者よりも前者において受動(情念)の性格が見出されること
 が明らかである。

\end{longtable}
\newpage


\rhead{a.~3}
\begin{center}
 {\Large {\bf ARTICULUS TERTIUS}}\\
 {\large UTRUM PASSIO SIT MAGIS IN APPETITU SENSITIVO QUAM
 INTELLECTIVO, QUI DICITUR VOLUNTAS}\\
 {\footnotesize Part.~I, q.20, a.1, ad1; III {\itshape Sent.}, d.15,
 q.2, a.1, qu$^a$ 2; IV, d.49, q.3, a.1, qu$^a$ 2, ad 1; {\itshape De
 Verit.}, q.26, a.3; II {\itshape Ethic.}, lect.~5.}\\
 {\Large 第三項\\情念は意志と言われる知性的欲求よりも感覚的欲求の中にあ
 るか}
\end{center}

\begin{longtable}{p{21em}p{21em}}

{\scshape Ad tertium sic proceditur}. Videtur quod passio non magis sit in appetitu
sensitivo quam in appetitu intellectivo. Dicit enim Dionysius, {\scshape ii}
cap.~{\itshape de Div.~Nom}., quod Hierotheus {\itshape ex quadam est doctus diviniore
inspiratione, non solum discens, sed etiam patiens divina}. Sed passio
divinorum non potest pertinere ad appetitum sensitivum, cuius obiectum
est bonum sensibile. Ergo passio est in appetitu intellectivo, sicut et
in sensitivo.

&

第三項の問題へ議論は以下のように進められる。情念は知性的欲求よりも感覚的
 欲求の中にあるのではないと思われる。理由は以下の通り。
ディオニュシウスは『神名論』第2章でヒエロテウスが「神的なことを学ぶだけ
 でなく受けもすることで、神からのある種の霊感に教えられる」と述べている。
 しかし神的な事柄を受けることは、その対象が可感的な善である感覚的欲求に
 属すことはできない。ゆえに情念(受動)は感覚的欲求と同じように知性的欲
 求においてもある。



\\



2. {\scshape Praeterea}, quanto activum est potentius, tanto passio est fortior. Sed
obiectum appetitus intellectivi, quod est bonum universale, est
potentius activum quam obiectum appetitus sensitivi, quod est bonum
particulare. Ergo ratio passionis magis invenitur in appetitu
intellectivo quam in appetitu sensitivo.

&


さらに、能動が力強いほど受動はより強い。しかるに知性的欲求の対象である普
 遍的善は、感覚的欲求の対象である個別的善よりも力強い能動である。ゆえに、
 感覚的欲求よりも知性的欲求の中により受動の性格が見出される。

\\



3. {\scshape Praeterea}, gaudium et amor passiones quaedam esse dicuntur. Sed haec
inveniuntur in appetitu intellectivo, et non solum in sensitivo,
alioquin non attribuerentur in Scripturis Deo et Angelis. Ergo passiones
non magis sunt in appetitu sensitivo quam in intellectivo.

&

さらに、喜びと愛は一種の情念だと言われる。しかるにこれらは感覚的だけでな
 く知性的な欲求の中にも見出される。そうでなければ、聖書において神と天使に帰せら
 れることはなかったであろう。ゆえに情念は、知性的欲求より感覚的欲求にお
 いてあるわけではない。

\\



{\scshape Sed contra est} quod dicit Damascenus, in II libro, describens animales
passiones, {\itshape passio est motus appetitivae virtutis sensibilis in
imaginatione boni vel mali. Et aliter, passio est motus irrationalis
animae per suspicionem boni vel mali}.

&

しかし反対に、ダマスケヌスは本\footnote{『正統信仰論』のこと。}の第2巻で
 動物的な情念を記述して以下のように述べている。「情念は、善と悪の表象に
 おける可感的欲求の力の運動である。あるいは別の言い方をすると、情念とは、
 善と悪を受け取ることによる非理性的魂の運動である」。


\\



{\scshape Respondeo dicendum} quod, sicut iam dictum est, passio proprie invenitur
ubi est transmutatio corporalis. Quae quidem invenitur in actibus
appetitus sensitivi; et non solum spiritualis, sicut est in
apprehensione sensitiva, sed etiam naturalis. In actu autem appetitus
intellectivi non requiritur aliqua transmutatio corporalis, quia
huiusmodi appetitus non est virtus alicuius organi. Unde patet quod
ratio passionis magis proprie invenitur in actu appetitus sensitivi quam
intellectivi; ut etiam patet per definitiones Damasceni inductas.

&

解答する。以下のように言われるべきである。
すでに述べられたとおり、情念は厳密には身体的変容があるところに見出される。
そしてこの変容は、、感覚的把握におけるように霊的作用だけでなく、自然的作用においても感覚的欲求の
 作用において見出される。
これに対して知性的把握の作用においては、何らの身体的変容も必要とされない。
 なぜならそのような欲求は、どんな器官の力でもないからである。したがって、
 情念という性格は知性的欲求よりも感覚的欲求の作用において、より厳密な意
 味で見出される。これはダマスケヌスが導いた諸定義によっても明らかである。


\\



{\scshape Ad primum ergo dicendum} quod passio divinorum ibi dicitur affectio ad
divina, et coniunctio ad ipsa per amorem, quod tamen fit sine
transmutatione corporali.

&

第一異論に対しては、それゆえ、以下のように言われるべきである。
ここで神的な事柄の受動と言われているのは、神的な事柄への愛徳と愛によるそ
 れらとの結びつきであり、これは身体の変容なしに生じる。


\\



{\scshape Ad secundum dicendum} quod magnitudo passionis non solum dependet ex
virtute agentis, sed etiam ex passibilitate patientis, quia quae sunt
bene passibilia, multum patiuntur etiam a parvis activis. Licet ergo
obiectum appetitus intellectivi sit magis activum quam obiectum
appetitus sensitivi, tamen appetitus sensitivus est magis passivus.

&

第二異論に対しては、次のように言われるべきである。
受動の大きさは能動者の力だけでなく受動者の受動性にも依存する。なぜなら、
 欲受動しうるものは、より小さな農道からでも多くを受けるからである。ゆえ
 に、知性的欲求の対象が感覚的欲求の対象よりもより能動的であるとしても、
 感覚的欲求はより多く受動しうる。

\\



{\scshape Ad tertium dicendum} quod amor et gaudium et alia huiusmodi, cum
attribuuntur Deo vel Angelis, aut hominibus secundum appetitum
intellectivum, significant simplicem actum voluntatis cum similitudine
effectus, absque passione. Unde dicit Augustinus, IX {\itshape de Civ.~Dei}, {\itshape sancti
Angeli et sine ira puniunt et sine miseriae compassione subveniunt. Et
tamen, istarum nomina passionum, consuetudine locutionis humanae, etiam
in eos usurpantur, propter quandam operum similitudinem, non propter
affectionum infirmitatem}.

&


第三異論に対しては以下のように言われるべきである。
愛と喜び、その他このようなものが、神や天使に、あるいは知性的欲求に即して
 人間に帰せられるとき、意志の単純な作用を意味する。それは結果への類似を
 伴うが、情念を伴ってはいない。このことからアウグスティヌスは『神の国』第9
 巻で次のように述べている。「聖なる天使たちは怒りなしに罰し、憐れみの共
 感なしに助ける。しかしこれら情念の名前は人間の語りの習慣によって、感情の低さのためにでなくある種
 の働きの類似のために、それらへと拡張されて用いられる」。


\end{longtable}


\end{document}
