\documentclass[10pt]{jsarticle} % use larger type; default would be 10pt
%\usepackage[utf8]{inputenc} % set input encoding (not needed with XeLaTeX)
%\usepackage[round,comma,authoryear]{natbib}
%\usepackage{nruby}
\usepackage{okumacro}
\usepackage{longtable}
%\usepqckage{tablefootnote}
\usepackage[polutonikogreek,english,japanese]{babel}
%\usepackage{amsmath}
\usepackage{latexsym}
\usepackage{color}

%----- header -------
\usepackage{fancyhdr}
\lhead{{\it Summa Theologiae} I, q.~51}
%--------------------

\bibliographystyle{jplain}

\title{{\bf PRIMA PARS}\\{\HUGE Summae Theologiae}\\Sancti Thomae
Aquinatis\\{\sffamily QUEAESTIO QUINQUAGESIMAPRIMA}\\DE COMPARATIONE
ANGELORUM AD CORPORA}
\author{Japanese translation\\by Yoshinori {\sc Ueeda}}
\date{Last modified \today}


%%%% コピペ用
%\rhead{a.~}
%\begin{center}
% {\Large {\bf }}\\
% {\large }\\
% {\footnotesize }\\
% {\Large \\}
%\end{center}
%
%\begin{longtable}{p{21em}p{21em}}
%
%&
%
%
%\\
%\end{longtable}
%\newpage



\begin{document}
\maketitle
\pagestyle{fancy}

\begin{center}
{\Large 第五十一問\\天使の物体への関係について}
\end{center}

\begin{longtable}{p{21em}p{21em}}

{\huge D}{\scshape einde} quaeritur de Angelis per comparationem ad
corporalia. Et primo, de comparatione Angelorum ad corpora; secundo, de
comparatione Angelorum ad loca corporalia; tertio, de comparatione
Angelorum ad motum localem. Circa primum quaeruntur tria.


\begin{enumerate}
 \item utrum Angeli habeant corpora naturaliter sibi unita. 
 \item utrum assumant corpora. 
 \item utrum in corporibus assumptis exerceant opera vitae.
\end{enumerate}

&

次に、天使について、物体への関係を通して探究される。第一に、天使の、物体
 (身体)への関係について、第二に、天使の、物体的場所への関係について、
 第三に、天使の、場所的運動への関係について。第一にかんして、三つのこと
 が問われる。

\begin{enumerate}
 \item 天使は、本性的に自らに合一された身体をもつか。
 \item 天使は、身体を取るか。
 \item 天使は、取った身体の中で、生命の働きを行うか。
\end{enumerate}

\end{longtable}
\newpage


\rhead{a.~1}
\begin{center}
 {\Large {\bf ARTICULUS PRIMUS}}\\
 {\large UTRUM ANGELI HABEANT CORPORA NATURALITER SIBI UNITA}\\
 {\footnotesize II {\itshape Sent.}, d.~8, a.~1; II {\itshape SCG.},
 c.~91; {\itshape De Pot.}, q.~6, a.~8; {\itshape De Malo}, q.~16, a.~1;
 {\itshape De Spirit.~Creat.}, a.~5; Opusc.~XV, {\itshape De Anngelis}, c.~18.}\\
 {\Large 第一項\\天使は本性的に自らに合一された身体をもつか}
\end{center}

\begin{longtable}{p{21em}p{21em}}


{\huge A}{\scshape d primum sic proceditur}. Videtur quod
Angeli habeant corpora naturaliter sibi unita. Dicit enim Origenes, in
libro {\itshape Peri Archon} : {\itshape solius Dei, idest patris et filii et spiritus sancti,
naturae illud proprium est, ut sine materiali substantia et absque ulla
corporeae adiectionis societate, intelligatur existere}. Bernardus etiam
dicit, in VI homilia {\itshape super Cant}., {\itshape demus Deo soli, sicut immortalitatem
sic incorporeitatem, cuius natura sola neque propter se neque propter
alium, s\={o}l\={a}t[=c]io indiget instrumenti corporei. Liquet autem omnem spiritum
creatum corporeo indigere solatio}. -- Augustinus etiam dicit, {\itshape
 super Gen.~ad Litt}., {\itshape Daemones \={a}\u{e}rea dicuntur animalia, quia corporum aereorum natura
vigent}. Eadem autem est natura Daemonis et Angeli. Ergo Angeli habent
corpora naturaliter sibi unita.


&
第一の問題へは次のように議論が進められる。
天使は、本性的に自らに合一された身体をもつと思われる。理由は以下のと
 おり。オリゲネスは『諸原理について』という書物で次のように述べている。
 「質料的実体なしに存在する、あるいはまた、どんな物体的な結びつきによる共同体もなしに存在すると考えられることは、ただ神、すなわち、父、子、聖霊の本性だけ
 に固有なことである」。また、ベルナルドゥスは『雅歌教話』6巻で「不死性、
 またそえゆえに非物体性を、その本性だけが、自らのためにも他のもののため
 にも、物体的道具の慰めを必要としないただ神だけに与えよう。これに対して、すべて
 の被造の霊は、明らかに物体的慰めを必要とする」と述べる。また、アウグス
 ティヌスは『創世記逐語注解』で「悪魔たちは、空気的動物と言われるが、そ
 れは、彼らが空気的身体の本性を必要とするからである」と述べている。悪魔
 と天使は同一の本性をもつ。ゆえに、天使は、本性的に自らに合一された
 身体をもつ。

\\


{\scshape 2 Praeterea}, Gregorius, in Homilia
Epiphaniae, nominat Angelum {\itshape rationale animal}. Omne autem animal
componitur ex corpore et anima. Ergo Angeli habent corpora naturaliter
sibi unita.


&

さらに、グレゴリウスは『公現講解』で天使を「理性的動物」と呼んでいる。と
 ころで、すべての動物は、身体と魂とから複合されている。ゆえに、天使たち
 は、本性的に自らに合一された身体をもつ。


\\


{\scshape 3 Praeterea}, perfectior est vita in Angelis
quam in animabus. Sed anima non solum vivit, sed etiam vivificat
corpus. Ergo Angeli vivificant corpora naturaliter sibi unita.


&

さらに、動物よりも天使において、生命はより完全である。ところで、魂は、生
 きるだけでなく、身体を生かす。ゆえに、天使は、本性的に自らに結びつけら
 れた身体を生かす。

\\


 {\scshape Sed contra est} quod dicit Dionysius, {\scshape iv} cap.~{\itshape de
Div.~Nom}., quod Angeli {\itshape sicut incorporales intelliguntur}.


&

しかし反対に、ディオニュシウスは『神名論』4章で、天使たちは「非物体的な
 ものとして理解される」と述べている。


\\


{\scshape Respondeo dicendum} quod Angeli non habent
corpora sibi naturaliter unita. Quod enim accidit alicui naturae, non
invenitur universaliter in natura illa, sicut habere alas, quia non est
de ratione animalis, non convenit omni animali. 


&

解答する。以下のように言われるべきである。
天使たちは、本性的に自らに合一された身体をもたない。
ある本性に附帯することは、その本性の中に普遍的に見出されることはない。た
 とえば、翼をもつことは、動物の性格に含まれないので、すべての動物に属す
 わけではない。

\\



Cum autem intelligere
non sit actus corporis nec alicuius virtutis corporeae, ut infra
patebit, habere corpus unitum non est de ratione substantiae
intellectualis inquantum huiusmodi, sed accidit alicui substantiae
intellectuali propter aliquid aliud; sicut humanae animae competit uniri
corpori, quia est imperfecta et in potentia existens in genere
intellectualium substantiarum, non habens in sui natura plenitudinem
scientiae, sed acquirens eam per sensus corporeos a sensibilibus rebus,
ut infra dicetur. 


&

ところで、あとで明らかにされるように、知性認識することは、身体の働きでな
 いし、なんらかの身体的能力の働きでもないので、合一された身体をもつこ
 とは、知性的実体であるかぎりでの知性的実体の性格に属さず、何か別のことの
 ために、なんらかの知性的実体に附帯する。たとえば、人間の魂には、身体に合
 一されることが適合するが、それは、あとで語られるように、人間の魂が知性的
 諸実体の類の中で、不完全でありまた可能態にあり、自らの本性の中に知の充溢
 をもたず、可感的事物から身体的感覚を通して知を獲得するからである。


\\



In quocumque autem genere invenitur aliquid
imperfectum, oportet praeexistere aliquid perfectum in genere illo. Sunt
igitur aliquae substantiae perfectae intellectuales in natura
intellectuali, non indigentes acquirere scientiam a sensibilibus
rebus. Non igitur omnes substantiae intellectuales sunt unitae
corporibus; sed aliquae sunt a corporibus separatae. Et has dicimus
Angelos.


&

ところで、不完全な何かが見出されるようなどの類の中にも、その類の中で完全
 な何かが先在しなければならない。ゆえに、知的本性のなかに、可感的事物か
 ら知を獲得する必要がない、完全な知的実体が存在する。ゆえに、すべての知
 的実体が身体に合一されているわけではなく、ある実体が、身体から分離して
 いる。私たちはそれらを天使と呼ぶ。

\\


{\scshape Ad primum ergo dicendum} quod, sicut supra dictum est,
quorundam opinio fuit quod omne ens esset corpus. Et ex hac
existimatione derivatum videtur, quod aliqui existimaverunt nullas
substantias incorporeas esse nisi corporibus unitas; adeo quod quidam
etiam posuerunt Deum esse animam mundi, ut Augustinus narrat in VII
{\itshape de Civ.~Dei}. Sed quia hoc fidei Catholicae repugnat, quae
ponit Deum super omnia exaltatum, secundum illud {\itshape Psalmi}
{\scshape viii}, {\itshape elevata est magnificentia tua super caelos},
Origenes, hoc de Deo dicere recusans, de aliis secutus est aliorum
opinionem; sicut et in multis aliis deceptus fuit, sequens antiquorum
philosophorum opiniones. Verbum autem Bernardi potest exponi, quod
spiritus creati indigeant corporali instrumento, non naturaliter unito,
sed ad aliquid assumpto, ut infra dicetur. -- Augustinus autem loquitur non
asserendo, sed opinione Platonicorum utens, qui ponebant esse quaedam
animalia aerea, quae Daemones nominabant.


&

第一異論に対しては、それゆえ、次のように言われるべきである。前に述べられ
たとおり、すべての有が物体であるという、ある人々の意見があった。ある人々
が、身体に合一されなければどんな非物体的実体もない、と考え、さらには、ア
ウグスティヌスが『神の国』7巻で述べるように、ある人々は、神が世界の魂であ
るとすら述べたのは、この考えから出てきているように思われる。しかし、これ
は、かの『詩編』8「あなたの偉大さは天の上に上げられている」にしたがって、
神を万物を越えてまさっていると考える正統信仰に反するので、オリゲネスは、
これを神について語ることを拒み、[神以外の]他のものどもについて、他の人々
の意見に従った。それは、古代の哲学者たちの意見に従って、他の多くのことが
らにおいて誤ったようにである。また、ベルナルドゥスの言葉は、後で述べられ
るように、被造の霊が、本性的に合一されたのではなく、何かのために取られた
ものとしての、物体的な道具を必要とする、と説明されるうる。また、アウグス
 ティヌスは、自分の意見として主張しているのではなく、彼らが悪霊と呼んで
 いた、なんらかの空気的な動物がいると考えたプラトン派の人々の意見を用い
 て語っている。


\\


{\scshape Ad secundum dicendum} quod Gregorius nominat
Angelum rationale animal metaphorice, propter similitudinem rationis.


&
第二異論に対しては、次のように言われるべきである。
グレゴリウスは、理性の類似性のために、比喩的に、天使を理性的動物と呼んで
 いる。

\\


{\scshape Ad tertium dicendum} quod vivificare effective simpliciter
perfectionis est. Unde et Deo convenit secundum illud I {\itshape
Reg}.~{\scshape ii} : {\itshape Dominus mortificat et vivificat}. Sed
vivificare formaliter est substantiae quae est pars alicuius naturae, et
non habentis in se integram naturam speciei. Unde substantia
intellectualis quae non est unita corpori, est perfectior quam ea quae
est corpori unita.


&

第三異論に対しては、次のように言われるべきである。
作出的に生かすことは、端的に完全性に属する。ゆえに、『サムエル記上』2節
 「主は殺し、生かす」\footnote{「主は命を絶ち、また命を与え」(2:6)}によ
 れば、それは神にも適合する。しかし、形相的に生かすことは、何らかの本性
 の部分であり、自らの中に主の完全な本性をもたない実体に属する。したがっ
 て、身体に合一されない知性的実体は、身体に合一されたそのような実体より
 も完全である。


\end{longtable}
\newpage

\rhead{a.~2}
\begin{center}
 {\Large {\bf ARTICULUS SECUNDUS}}\\
 {\large UTRUM ANGELI ASSUMANT CORPORA}\\
 {\footnotesize II {\itshape Sent.}, d.~8, a.~2; {\itshape de Pot.},
 q.~6, a.~7.}\\
 {\Large 第二項\\天使は身体を取るか}
\end{center}

\begin{longtable}{p{21em}p{21em}}

{\huge A}{\scshape d secundum sic proceditur}. Videtur quod
Angeli non assumant corpora. In opere enim Angeli nihil est superfluum;
sicut neque in opere naturae. Sed superfluum esset
 quod\footnote{{\itshape si} PFGab} Angeli corpora
assumerent, Angelus enim non indiget corpore, cum eius virtus omnem
virtutem corporis excedat. Ergo Angelus non assumit corpus.

&
第二の問題へ向けて、議論は以下のように進められる。
天使は身体を取らないと思われる。理由は以下の通り。
自然の働きの中に無駄がないように、天使の働きの中に無駄なものはない。
ところが、もし天使が身体を取ることは[or取ったならば、それは]、無駄なことであっただろう。
 なぜなら、天使の能力は、物体[=身体]のあらゆる能力を超えているので、
 天使は身体を必要としないからである。ゆえに、天使は身体を取らない。

\\


{\scshape 2 Praeterea}, omnis assumptio ad aliquam unionem terminatur,
quia {\itshape assumere} dicitur quasi {\itshape ad se sumere}. Sed
corpus non unitur Angelo ut formae, sicut dictum est. Ex eo autem quod
unitur sibi ut motori, non dicitur assumi, alioquin sequeretur quod
omnia corpora mota ab Angelis, essent ab eis assumpta. Ergo Angeli non
assumunt corpora.

&

さらに、「取るassumere」とは「自らに向けて取るad se sumere」ことを言うの
 で、すべて何かを取ることは、なんらかの合一に終局する。ところが、すでに
 述べられたとおり、身体は、形相としての天使に合一されない。また、動者と
 しての自己に合一されることからは、「取る」とは言われない。さもなければ、
 天使たちによって動かされるすべての物体が、天使たちによって取られていること
 になっただろう。ゆえに、天使たちは身体を取らない。

\\


{\scshape 3 Praeterea}, Angeli non assumunt corpora de
terra vel aqua, quia non subito disparerent; neque iterum de igne, quia
comburerent ea quae contingerent; neque iterum ex aere, quia aer
infigurabilis est et incolorabilis. Ergo Angeli corpora non assumunt.

&
さらに、天使たちは、土や水からなる身体を取らない。なぜなら、速やかに消え
 去ることがないからである。また、火からなる身体を取らない。なぜなら、
 [もしそうなら]それらが接触するものどもを燃やしたであろうから。
また、空気からなる身体を取らない。なぜなら、空気は形をとらず、色も付けら
 れないからである。

\\


Sed contra est quod Augustinus dicit, XVI {\itshape de
Civ.~Dei}, quod Angeli in assumptis corporibus Abrahae apparuerunt.

&
しかし反対に、アウグスティヌスは『神の国』16巻で、天使は、取られた身体
 の中で、アブラハムに現れた、と述べている。

\\


{\scshape Respondeo dicendum} quod quidam dixerunt
Angelos nunquam corpora assumere, sed omnia quae in Scripturis divinis
leguntur de apparitionibus Angelorum, contigisse in visione proph\={e}t\={\i}ae,
hoc est secundum imaginationem. Sed hoc repugnat intentioni
Scripturae. Illud enim quod imaginaria visione videtur, est in sola
imaginatione videntis, unde non videtur indifferenter ab
omnibus. Scriptura autem divina sic introducit interdum Angelos
apparentes, ut communiter ab omnibus viderentur; sicut Angeli apparentes
Abrahae, visi sunt ab eo et tota familia eius, et a Loth, et a civibus
Sodomorum. Similiter Angelus qui apparuit Tobiae, ab omnibus
videbatur. Ex quo manifestum fit huiusmodi contigisse secundum corpoream
visionem, qua videtur id quod positum est extra videntem, unde ab
omnibus videri potest. Tali autem visione non videtur nisi corpus. Cum
igitur Angeli neque corpora sint, neque habeant corpora naturaliter sibi
unita, ut ex dictis patet, relinquitur quod interdum corpora assumant.

&
解答する。以下のように言われるべきである。
ある人々は、天使はけっして身体を取らず、聖書の中で、天使の出
 現について読まれるすべてのことがらは、預言の映像という形で、つまり想像
 力によって生じた、と言った。しかし、これは聖書の意図に反する。理由は以
 下の通り。創造の映像によって見られるものは、見る人の想像の中だけにある
 ので、すべての人に違いなく見られることがない。ところが、聖書はときに、天使たち
 の出現を、万人に共通に見られることがらとして書き入れている。たとえば、
 アブラハムに現れた天使たちは、彼と彼の全家族に見られたし、ロトや、ソド
 ムの市民たちによっても見られた。
同様に、トビアに現れた天使も、すべての人々によって見られた。このことから、
 このようなことは、身体的な視覚において生じたことが明らかとなる。この視
 覚によって、見る者の外にあるものが見られ、したがって、すべての人によっ
 て見られうる。ところが、物体以外に、このような視覚によって見られる
ものはない。ゆえに、天使たちは物体でなく、また、既述のことから明らかな
 とおり、本性的に自らに結びつけられた身体をもたないので、時として身体を
 取ることが帰結として残される。

\\


{\scshape Ad primum ergo dicendum} quod Angeli non
indigent corpore assumpto propter seipsos, sed propter nos; ut
familiariter cum hominibus conversando, demonstrent intelligibilem
societatem quam homines expectant cum eis habendam in futura vita. Hoc
etiam quod Angeli corpora assumpserunt in veteri testamento, fuit
quoddam figurale indicium quod verbum Dei assumpturum esset corpus
humanum, omnes enim apparitiones veteris testamenti ad illam
apparitionem ordinatae fuerunt, qua filius Dei apparuit in carne.

&

第一異論に対しては、それゆえ、以下のように言われるべきである。天使たちは、
それら自身のために、取られた身体を必要とするのではなく、私たちのために必
要とする。つまりそれは、人間たちと親しく会話することで、人間たちが、将来
の生において、彼ら[=天使たち]と共にもつであろうと期待する、可知的な交
わりを明示するためにである。また、旧約聖書の中で天使が身体を取ったことは、
 神の言葉が人間の肉体を取るであろうことの、比喩的なしるしである。という
 のも、旧約聖書のすべての出現は、神の子が肉を取った、かの出現に秩序づけ
 られていたからである。

\\


{\scshape Ad secundum dicendum} quod corpus assumptum
unitur Angelo, non quidem ut formae, neque solum ut motori; sed sicut
motori repraesentato per corpus mobile assumptum. Sicut enim in sacra
Scriptura proprietates rerum intelligibilium sub similitudinibus rerum
sensibilium describuntur, ita corpora sensibilia divina virtute sic
formantur ab Angelis, ut congruant ad repraesentandum Angeli
intelligibiles proprietates. Et hoc est Angelum assumere corpus.

&

第二異論に対しては、次のように言われるべきである。
天使によって取られた身体は、形相としての天使にでも、たんに動者としての天
 使にでもなく、取られた可動的身体によって、表現された動者としての天使に
 合一されている。ちょうど、聖書で、可知的事物の固有性が、可感的事物の類
 似のもとに記述されるように、可感的身体が、神の力によって、天使の可知的
 固有性を表現するために集まるというかたちで、天使によって形成される。こ
 れが、天使が身体を取るということである。


\\


{\scshape Ad tertium dicendum} quod, licet aer, in sua
raritate manens, non retineat figuram neque colorem; quando tamen
condensatur, et figurari et colorari potest, sicut patet in nubibus. Et
sic Angeli assumunt corpora ex aere, condensando ipsum virtute divina,
quantum necesse est ad corporis assumendi formationem.

&

第三異論に対しては、次のように言われるべきである。
空気は、その希薄さにとどまるかぎり、形も色も留めない。しかし、雲において
 明らかなとおり、凝縮されれば、形も色もとりうる。同じように、天使も、取
 られるべき身体の形成のために必要であるかぎりで、神の力によって空気を凝
 縮することで、空気から身体を取る。


\end{longtable}
\newpage



\rhead{a.~3}
\begin{center}
 {\Large {\bf ARTICULUS TERTIUS}}\\
 {\large UTRUM ANGELI IN CORPORIBUS ASSUMPTIS OPERA VITAE EXERCEANT}\\
 {\footnotesize II {\itshape Sent.}, d.~8, a.~4; {\itshape de Pot.},
 q.~6, a.~8.}\\
 {\Large 第三項\\天使たちは、取った身体の中で生命の働きをなすか}
\end{center}

\begin{longtable}{p{21em}p{21em}}


{\huge A}{\scshape d tertium sic proceditur}. Videtur quod
Angeli in corporibus assumptis opera vitae exerceant. Angelos enim
veritatis non decet aliqua fictio. Esset autem fictio, si corpus ab eis
assumptum, quod vivum videtur et opera vitae habens, non haberet
huiusmodi. Ergo Angeli in assumpto corpore opera vitae exercent.

&

第三の問題に向けて、議論は以下のように進められる。天使たちは、とった身体
 の中で、生命の働きをなすと思われる。理由は以下の通り。いかなる虚構も天
 使の真実にふさわしくない。ところが、もし、生きているように見え、生命の
 働きをもっているように見える身体が、天使たちに取られていながら、それが、
 そのようなもの[=生命の働き]をもたないとすれば、それは虚構である。ゆ
 えに、天使たちは、取った身体の中で、生命の働きをなしている。

\\


{\scshape 2 Praeterea}, in operibus Angeli non sunt
aliqua frustra. Frustra autem in corpore assumpto per Angelum
formarentur oculi et nares et alia sensuum instrumenta, nisi per ea
Angelus sentiret. Ergo Angelus sentit per corpus assumptum. Quod est
propriissimum opus vitae.

&
さらに、天使の働きの中に無駄なものはなにもない。
ところが、天使によって取られた身体の中に、眼や鼻や、その他感覚器官が形成
 されたならば、天使がそれらを通して感覚しないかぎり、無駄だっただろう。
 ゆえに、天使は取られた身体を通して感覚する。これは、この上なく固有な生
 命の働きである。

\\


{\scshape 3 Praeterea}, moveri motu processivo est unum
de operibus vitae, ut patet in II {\itshape de Anima}. Manifeste autem Angeli
apparent in assumptis corporibus moveri. Dicitur enim {\itshape Gen}.~{\scshape xviii}, quod
{\itshape Abraham simul gradiebatur, deducens} Angelos qui ei apparuerant. Et
Angelus Tobiae quaerenti : {\itshape Nosti viam quae ducit in civitatem Medorum?}
Respondit : {\itshape Novi, et omnia itinera eius frequenter ambulavi}. Ergo Angeli
in corporibus assumptis frequenter exercent opera vitae.

&
さらに、『デ・アニマ』2巻で明らかなとおり、進行運動によって動くことは、
 生命の働きの一つである。ところが、明らかに、天使たちは身体を取って
 動いているように見える。たとえば『創世記』18章で「アブラハムは」彼に現
 れた天使を「導きながら一緒に歩いた」\footnote{「アブラハムも、彼らを見
 送るために一緒に行った。」(18:16)}と言われている。また、「メドロスへ行
 く道を知っているか」と尋ねるトビアに、天使は「知っている。私はそのすべ
 ての道をしばしば歩いたことがある」と答えている。ゆえに、天使たちは、取っ
 た身体の中で、しばしば、生命の働きをなしている。

\\


{\scshape 4 Praeterea}, locutio est opus viventis, fit
enim per vocem, quae est sonus ab ore animalis prolatus ut dicitur in II
{\itshape de Anima}. Manifestum est autem ex multis locis Scripturae, Angelos in
assumptis corporibus locutos fuisse. Ergo in corporibus assumptis
exercent opera vitae.
&
さらに、語ることは、生きるものの働きである。なぜなら、『デ・アニマ』2巻
 で言われるように、それは、動物の口から発せられた音である音声によって生
 じるからである。ところが、聖書の多くの箇所から、取った身体の中で、天使
 たちが語ったことが明らかである。ゆえに、取った身体の中で、天使たちは生
 命の働きを行う。

\\



{\scshape 5 Praeterea}, comedere est proprium opus
animalis, unde dominus post resurrectionem, in argumentum resumptae
vitae, cum discipulis manducavit, ut habetur Lucae ultimo. Sed Angeli in
assumptis corporibus apparentes comederunt, et Abraham eis cibos
obtulit, quos tamen prius adoraverat, ut habetur {\itshape Gen}.~{\scshape xviii}. Ergo Angeli
in assumptis corporibus exercent opera vitae.
&
さらに、食べることは、動物に固有の働きである。このことから、『ルカによる
 福音書』最終章で、復活の後、主は弟子たちと共に、生命が復活したことを示
 すために、食事をしている。ところで、『創世記』18章で書かれているように、
 天使たちは、身体を取って現れ、食事をした。アブラハムも、前もって崇拝した彼
 らに食べ物を運んだ。ゆえに、天使たちは、取った身体の中で、生命の働きを
 行う。

\\


{\scshape 6 Praeterea}, generare hominem est actus
vitae. Sed hoc competit Angelis in assumptis corporibus dicitur enim
{\itshape Gen}.~{\scshape vi} : {\itshape Postquam ingressi sunt filii Dei ad filias hominum, illaeque
genuerunt, isti sunt potentes a saeculo viri famosi}. Ergo Angeli
exercent opera vitae in corporibus assumptis.

&

さらに、人間を生むことは生命の働きである。しかし、これは、取った身体の中
 にいる天使たちに適合する。なぜなら、『創世記』6章に「神の息子たちが人間
 の娘たちに入った後、彼らもまた生まれた。この人々は、世々に有名な強い男
 である」\footnote{「当時もその後も、地上にはネフィリムがいた。これは、
 神の子らが人の娘たちに入って産ませた者であり、大昔の名高い英雄たちで
 あった」(6:4)}ゆえに、天使たちは、取った身体の中で、生命の働きを行う。
\\



{\scshape Sed contra}, corpora assumpta ab Angelis non
vivunt, ut supra dictum est. Ergo nec opera vitae per ea exerceri
possunt.
&

しかし反対に、前に述べられたとおり、天使によって取られた身体は生きていな
 い。ゆえに、生命の働きが、それによって行われることは不可能である
\\



{\scshape Respondeo dicendum} quod quaedam opera viventium habent
aliquid commune cum aliis operibus, ut locutio, quae est opus viventis,
convenit cum aliis sonis inanimatorum, inquantum est sonus; et
progressio cum aliis motibus, inquantum est motus. Quantum ergo ad id
quod est commune utrisque operibus, possunt opera vitae fieri ab Angelis
per corpora assumpta. Non autem quantum ad id quod est proprium
viventium, quia secundum philosophum, in libro {\itshape de Somn.~et
Vig.}, {\itshape cuius est potentia, eius est actio}; unde nihil potest
habere opus vitae, quod non habet vitam, quae est potentiale principium
talis actionis. 

&

解答する。以下のように言われるべきである。
生命の働きの中には、他の働きと共通のものをもつものがある。
たとえば、語ることは、生きるものの働きだが、生きていないものどもの他の音
 と、それが音であるかぎりで一致する。また、進むことも、運動であるかぎり
 で、他の運動と一致する。ゆえに、どちらの働きにも共通である点にかんして、
 生命の働きが、取った身体を通して天使によってなされることはできる。しか
 し、生きているものに固有のことにかんしてはそうではない。なぜなら、『睡
 眠と覚醒について』という書物の中の哲学者によれば、「能力は、働きが属す
 るものに属する」からである。したがって、そのような[=生命の]働きの能
 力的な根源である生命をもたないものが、生命の働きを持つことはできない。


\\



{\scshape Ad primum ergo dicendum} quod, sicut non est
contra veritatem quod in Scriptura intelligibilia sub figuris
sensibilibus describuntur, quia hoc non dicitur ad adstruendum quod
intelligibilia sint sensibilia, sed per figuras sensibilium proprietates
intelligibilium secundum similitudinem quandam dantur intelligi; ita non
repugnat veritati sanctorum Angelorum quod corpora ab eis assumpta
videntur homines viventes, licet non sint. Non enim assumuntur nisi ut
per proprietates hominis et operum hominis, spirituales proprietates
Angelorum et eorum spiritualia opera designentur. Quod non ita congrue
fieret, si veros homines assumerent, quia proprietates eorum ducerent in
ipsos homines, non in Angelos.

&

第一異論に対しては、以下のように言われるべきである。聖書の中で、可知的な
事柄が可感的な形態のもとに描かれても、それは真理に反しない。なぜなら、こ
れは、可知的な事柄が可感的なことがらであるとするために語られているのでは
なく、可感的なことがらの形態を通して、可知的な事柄の固有性が、ある種の類
似によって与えられていると理解されることが語られているからである。ちょう
どそのように、天使たちによって取られた身体が、生きてなく、しかし人間に生
きているように見えたとしても、それは聖なる天使たちの真理に反しない。なぜ
なら、人間の、そして人間の働きの固有性によって、天使たちの霊的な固有性と、
それらの霊的な業が示されるためにでなければ、[身体が]取られることはない
からである。しかしもし、天使が本当の人間を取っていたとしたら、このように
うまくはいかなかったであろう。なぜなら、それらの固有性は、天使ではなく、
その人間自体に導いたであろうから。


\\



{\scshape Ad secundum dicendum} quod sentire est
totaliter opus vitae, unde nullo modo est dicendum quod Angeli per
organa assumptorum corporum sentiant. Nec tamen superflue sunt
formata. Non enim ad hoc sunt formata, ut per ea sentiatur, sed ad hoc
ut per huiusmodi organa virtutes spirituales Angelorum designentur;
sicut per oculum designatur virtus cognitiva Angeli, et per alia membra
aliae eius virtutes, ut Dionysius docet, ult.~cap.~{\itshape Cael.~Hier}.

&

第二異論に対しては、以下のように言われるべきである。
感覚することは、全面的に、生命の働きである。したがって、天使たちが、取っ
 た身体の器官を通して感覚するとはけっして言われるべきでない。しかし、
 [天使が取った身体の感覚器官が]無駄に作られたのでもない。なぜなら、
 それらは、それらによって感覚されるために作られたのではなく、そのような
 器官を通して、天使の霊的な力が示されるために作られたのだからである。たとえば、
ディオニュシウスが『天上階級論』最終章で教えるように、眼によって、天使の認識する力が、また、他の四肢によって、天使の他の諸能力が示されるために。


\\



{\scshape Ad tertium dicendum} quod motus qui est a
motore coniuncto, est proprium opus vitae. Sic autem non moventur
corpora assumpta ab eis, quia Angeli non sunt eorum formae. Moventur
tamen Angeli per accidens, motis huiusmodi corporibus, cum sint in eis
sicut motores in mobilibus, et ita sunt hic quod non alibi, quod de Deo
dici non potest. Unde licet Deus non moveatur, motis his in quibus est,
quia ubique est; Angeli tamen moventur per accidens ad motum corporum
assumptorum. Non autem ad motum corporum caelestium, etiamsi sint in eis
sicut motores in mobilibus, quia corpora caelestia non recedunt de loco
secundum totum; nec determinatur spiritui moventi orbem locus secundum
aliquam determinatam partem substantiae orbis, quae nunc est in oriente,
nunc in occidente; sed secundum determinatum situm, quia semper est in
oriente virtus movens, ut dicitur in VIII {\itshape Physic}.
&

第三異論に対しては、以下のように言われるべきである。結びつけられた動者に
よる運動は、生命の固有の働きである。しかし、取られた身体は、天使たちによっ
て、このようには動かされない。なぜなら天使たちは、それら[=身体]の形相
でないからである。しかし天使たちは、そのような身体が動かされることによっ
て、附帯的に動かされる。なぜなら、天使たちは、ちょうど、動者が可動的なも
のの中にあるように、それら[=身体]の中にあり、そのようにして、他のとこ
ろではない\kenten{ここ} にいる。これは、神については言われえない。このこ
とから、神は至る所に存在するため、神がいるものが動かされても動かされない
が、しかし天使は、取られた身体の運動に応じて、附帯的に動かされる。他方、
天使は、動者が可動的なものの中にあるようにして天体の中にあるが、しかし、
天体の運動に応じては動かされない。なぜなら、天体は、全体として場所から退
くことがないからであり、また、天球の実体の何らかの限定された部分に即して、
たとえば、今は西にあり、今は東にあるというかたちで、場所が、天球を動かす
霊にとって限定されることもないからである。むしろそうではなく、『自然学』
 8巻によれば、動かす力は常に東にあるからである。


\\



{\scshape Ad quartum dicendum} quod Angeli proprie non
loquuntur per corpora assumpta, sed est aliquid simile locutioni,
inquantum formant sonos in aere similes vocibus humanis.

&
第四異論に対しては、次のように言われるべきである。
天使たちは、固有には、取られた身体を通して語らないが、人間の声ににた何か
 の音を空中に形成する限りで、語りに似た何かがある。

\\


{\scshape Ad quintum dicendum} quod nec etiam comedere,
proprie loquendo, Angelis convenit, quia comestio importat sumptionem
cibi convertibilis in substantiam comedentis. Et quamvis in corpus
Christi post resurrectionem cibus non converteretur, sed resolveretur in
praeiacentem materiam, tamen Christus habebat corpus talis naturae in
quod posset cibus converti, unde fuit vera comestio. Sed cibus assumptus
ab Angelis neque convertebatur in corpus assumptum, neque corpus illud
talis erat naturae in quod posset alimentum converti, unde non fuit vera
comestio, sed figurativa spiritualis comestionis. Et hoc est quod
Angelus dixit, {\itshape Tob}.~{\scshape xii} : {\itshape Cum essem vobiscum, videbar quidem manducare et
bibere, sed ego potu invisibili et cibo utor}. Abraham autem obtulit eis
cibos, existimans eos homines esse; in quibus tamen Deum venerabatur,
{\itshape sicut solet} Deus {\itshape esse in prophetis}, ut Augustinus dicit, XVI {\itshape de
Civ.~Dei}.


&

第五異論に対しては、以下のように言われるべきである。食事することもまた、厳密
 に言えば、天使たちに適合しない。なぜなら、「食事」は、「食事する者の実体へと
 転化しうる食物の摂取」を意味するからである。そして、たしかに、食物が復
 活後のキリストの身体へ転化することなく、もとの材料へ分解されることも可
 能だったが、しかし、キリストは、食物がそれへと転化しうるような本性の身体を持っ
 ていたのであり、したがって、そこには真の食事があった。これに対して、天
 使たちによって取られた食物は、取られた身体へ転化しなかったし、かの身体
 が、栄養がそれへと転化されうるような本性のものでもなかった。したがって、そこに
 真の食事はなく、霊的な食事のたとえとなる食事があった。そしてこれが、
 『トビト記』12節で、天使が「私があなたたちと共にいるとき、食べたり飲ん
 だりしているように私が見えたが、しかし私は見えない飲み物と食べ物を用いている」
 と言っていることである。また、アブラハムが彼らに食べ物を運んだのは、彼
 らが人間だと考えたからである。しかし、それらの食事において、神が崇拝さ
 れていた。アウグスティヌスが『神の国』16巻で「ちょうど、神が預言者たちの中
 にいるように」と言うように。

\\



{\scshape Ad sextum dicendum} quod, sicut Augustinus
dicit, XV {\itshape de Civ.~Dei}, {\itshape multi se expertos, vel ab expertis audisse
confirmant, Silvanos et Faunos, quos vulgus incubos vocat, improbos
saepe extitisse mulieribus et earum expetisse atque peregisse
concubitum. Unde hoc negare impudentiae videtur. Sed Angeli Dei sancti
nullo modo sic labi ante diluvium potuerunt. Unde per filios Dei
intelliguntur filii Seth, qui boni erant, filias autem hominum nominat
Scriptura eas quae natae erant de stirpe Cain. Neque mirandum est quod
de eis gigantes nasci potuerunt, neque enim omnes gigantes fuerunt, sed
multo plures ante diluvium quam post}. ---Si tamen ex coitu Daemonum aliqui
interdum nascuntur, hoc non est per semen ab eis decisum, aut a
corporibus assumptis, sed per semen alicuius hominis ad hoc acceptum,
utpote quod idem Daemon qui est succubus ad virum, fiat incubus ad
mulierem; sicut et aliarum rerum semina assumunt ad aliquarum rerum
generationem, ut Augustinus dicit, III {\itshape de Trin}. ; ut sic ille qui
nascitur non sit filius Daemonis, sed illius hominis cuius est semen
acceptum.

&

第六異論に対しては、次のように言われるべきである。アウグスティヌスは『神
の国』15巻で以下のように述べている。「多くの人々が、自分で経験して、ある
いは、経験した人から聞いて、シルワーヌスやファウヌスたち、民衆は彼らを夢
魔と呼んでいるが、彼らが邪悪にもしばしば女性たちに現れて、彼女らと同衾す
ることを願いそれを成し遂げたと信じている。したがって、このことを否定する
のも気が引ける。しかし、神の聖なる天使たちは、洪水前に、けっして、そのよ
うに堕落することはあり得なかった。したがって、「神の息子たち」ということ
で、善人であるセトの息子たちが理解されるが、しかし、聖書は、カインという
幹から生まれた女性たちを「人間の娘たち」と名付けている。彼女らから巨人た
ちが産まれえたことも、驚くべきではない。なぜなら、すべての巨人たちがいた
のではなく、むしろ、洪水の前に、洪水後よりも多くの巨人がいたからであ
る」。---しかし、悪魔との交わりから時として人が生まれることがあるのは、そ
 の悪魔から、あるいは、悪魔が取った身体から放出された精子によるのではな
 く、このために採取されたある人間の精子による。つまり、男性に対してサキュ
 バスである同一悪魔が、女性にはインキュバスとなる。
 アウグスティヌスが『三位一体論』3巻で述べるように、なんらかの事物の産出
 のために、他の事物の種を採取するように。したがって、産み出されるのは悪
 魔の息子ではなく、その精子が採取された人間の息子である。



\end{longtable}


\end{document}