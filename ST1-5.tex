\documentclass[10pt]{jsarticle}
\usepackage{okumacro}
\usepackage{longtable}
\usepackage[polutonikogreek,english,japanese]{babel}
\usepackage{latexsym}
\usepackage{color}
%----- header -------
\usepackage{fancyhdr}
\pagestyle{fancy}
\lhead{{\it Summa Theologiae} I, q.~5}
%--------------------

\bibliographystyle{jplain}
\title{{\bf Prima Pars}\\{\HUGE Summae Theologiae}\\Sancti Thomae
Aquinatis\\Quaestio Quinta\\{\bf De Bono in Communi}}
\author{Japanese translation\\by Yoshinori {\sc Ueeda}}
\date{Last modified \today}


%%%% コピペ用
%\rhead{a.~}
%\begin{center}
% {\Large {\bf }}\\
% {\large }\\
% {\footnotesize }\\
% {\Large \\}
%\end{center}
%
%\begin{longtable}{p{21em}p{21em}}
%
%&
%
%\\
%\end{longtable}
%\newpage


\begin{document}

\maketitle
\begin{center}
 {\Large 第五問\\善一般について}
\end{center}
\begin{longtable}{p{21em}p{21em}}
{\scshape Deinde} quaeritur de bono, et primo de bono in communi;
 secundo de bonitate Dei. Circa primum quaeruntur sex. 
\begin{enumerate}
 \item utrum
 bonum et ens sint idem secundum rem.
 \item supposito quod differant
 ratione tantum, quid sit prius secundum rationem, utrum bonum vel
 ens.
 \item supposito quod ens sit prius, utrum omne ens sit
 bonum.
 \item ad quam causam ratio boni reducatur. 
 \item utrum ratio
 boni consistat in modo, specie et ordine.
 \item quomodo dividatur
 bonum in honestum, utile et delectabile.
\end{enumerate}

&

次に、善について問われる。第一に、善一般について。第二に、神の善性につ
いて。第一については、六つのことが問われる。

\begin{enumerate}
 \item 善と有は、事物において同一か。
 \item 善か有が、たんに概念においてのみ異なるとして、どちらが概念において先行するか。       
 \item 有が先行するとして、有はすべて善か。
 \item 善の性格はどの原因に還元されるか。
 \item 善の性格は、限度、形象、秩序において成立するか。
 \item 善は、どのようにして有徳、有用、快適に分割されるか。
\end{enumerate}

\end{longtable}

\newpage
\rhead{a.~1}

\begin{center}
 {\Large {\bf ARTICULUS PRIMUS}}\\
 {\large UTRUM BONUM DIFFERAT SECUNDUM REM AB ENTE}\\
 {\footnotesize I {\itshape Sent.}, d.~8, q.~1, a.~3; d.~19, q.~5, a.~1,
 ad 3; {\itshape De Verit.}, q.~1, a.~1; q.~21, a.~1; {\itshape De
 Pot.}, q.~9, a.~7, ad 6.}\\
 {\Large 第一項\\善は有と事物において異なるか}
\end{center}

\begin{longtable}{p{21em}p{21em}}

{\huge A}{\scshape d primum sic proceditur}. Videtur quod bonum
differat secundum rem ab ente. Dicit enim Boetius, in libro {\itshape
de Hebdom}., {\itshape intueor in rebus aliud esse quod sunt bona, et
aliud esse quod sunt}. Ergo bonum et ens differunt secundum rem.

&

第一に対しては、次のように進められる。善は、事物において、有と異なると
思われる。なぜなら、ボエティウスは、『デ・ヘブドマディブス』で「事物に
おいて、あるものが善であるところのものであり、別のものが、存在するとこ
ろのものであることを、私は見る」と述べている。ゆえに、善と有は、事物に
おいて異なる。

\\


2.~{\scshape Praeterea}, nihil informatur seipso. Sed bonum dicitur
per informationem entis, ut habetur in commento libri {\itshape de
Causis}. Ergo bonum differt secundum rem ab ente.

&

さらに、自分自身によって形相を受け取るものはなにもない。しかるに、『原因
 論』の注解で言われるように、有が形相を受け取ることによって、善と言われ
 る。ゆえに、善は、事物において、有と異なる。

\\

3.~{\scshape Praeterea}, bonum suscipit magis et minus. Esse autem non
 suscipit magis et minus. Ergo bonum differt secundum rem ab ente.

&

さらに、善は、より多く、より少なく、ということを受け入れる。これに対して、
 存在することは、より多く、より少なく、ということを受け入れない。ゆえに、
 善は、事物において、有と異なる。

\\

{\scshape Sed contra est} quod Augustinus dicit, in libro {\itshape de
 Doctrina Christiana}, quod {\itshape inquantum sumus, boni sumus}.

&

しかし反対に、アウグスティヌスは、『キリスト教の教え』という書物で、「私
 たちは存在している限りにおいて善い」と述べている。

\\

{\scshape Respondeo dicendum quod} bonum et ens sunt idem secundum rem,
 sed differunt secundum rationem tantum. Quod sic patet. Ratio enim boni
 in hoc consistit, quod aliquid sit appetibile, unde philosophus, in I
 {\itshape Ethic}., dicit quod bonum est {\itshape quod omnia appetunt}. Manifestum est autem
 quod unumquodque est appetibile secundum quod est perfectum, nam omnia
 appetunt suam perfectionem. Intantum est autem perfectum unumquodque,
 inquantum est actu, unde manifestum est quod intantum est aliquid
 bonum, inquantum est ens, esse enim est actualitas omnis rei, ut ex
 superioribus patet. Unde manifestum est quod bonum et ens sunt idem
 secundum rem, sed bonum dicit rationem appetibilis, quam non dicit ens.
&

解答する。以下のように言われるべきである。善と有は、事物において同一で
 あり、ただ概念(観点)だ
 けにおいて異なる。これは次のようにして明らかである。善の概念は、何かが
 欲求されうるものであるという点に成立する。このことから、哲学者は『ニコ
 マコス倫理学』第1巻で「善とは万物が欲求するものである」と述べる。しかる
 に、各々のものが欲求されうるものであるのは、それが完全であるかぎりにお
 いてである。なぜなら、万物は、自らの完成(完全になること)を欲求するの
 だから。しかるに、各々のものは、現実態においてあるかぎりにおいて、完全
 である。したがって、あるものは、それが有である限りにおいて、善であるこ
 とが明白である。なぜなら、上述のことから明らかなとおり、存在とは、すべ
 ての事物の現実性だからである。したがって、善と有は、事物において同一で
 あるが、善は、有が語らない、欲求されうるもの、という概念を語る、という
 ことが明らかである。

\\


{\scshape Ad primum ergo dicendum} quod, licet bonum et ens sint idem
 secundum rem, quia tamen differunt secundum rationem, non eodem modo
 dicitur aliquid ens simpliciter, et bonum simpliciter.  Nam cum ens
 dicat aliquid proprie esse in actu; actus autem proprie ordinem habeat
 ad potentiam; secundum hoc simpliciter aliquid dicitur ens, secundum
 quod primo discernitur ab eo quod est in potentia tantum.  Hoc autem
 est esse substantiale rei uniuscuiusque; unde per suum esse
 substantiale dicitur unumquodque ens simpliciter.  Per actus autem
 superadditos, dicitur aliquid esse {\itshape secundum quid}, sicut esse
 album significat esse secundum quid, non enim esse album aufert esse in
 potentia simpliciter, cum adveniat rei iam praeexistenti in actu.
Sed bonum dicit rationem perfecti, quod est appetibile, et per
 consequens dicit rationem ultimi. 
Unde id quod est ultimo perfectum, dicitur bonum simpliciter.


&

第一に対しては、それゆえ、次のように言われるべきである。善と有が事物にお
 いて同一であるにしても、それらは概念において異なるのだから、あるものが端
 的に有と言われるしかたと、端的に善と言われるしかたとは同じでない。すなわ
 ち、有とは、固有の意味で、何かが現実態において存在することを言い、また、
 現実態は、固有の意味で、可能態に対して秩序を持つのだから、あるものが端的
 に有と言われるのは、それが可能態だけにあるものから最初に区別されるかぎり
 においてである。しかるに、これは、各々の事物の実体的存在である。したがっ
 て、自らの実体的存在によって、各々のものは、端的に有と言われる。これに対
 して、付け加えられた現実態によっては、ある意味において、存在すると言われ
 る。たとえば、白いものであることは、ある意味において存在することを意味す
 る。実際、白いものであることは、端的に、可能態においてあることを除去しな
 い。なぜなら、すでに現実態において前もって存在しているもの[基体]に、到
 来するのだから。しかし、善は、「完全なもの」という性格をもつ。これは、
 「欲求されうるもの」であり、結果的に、「最後のもの」という性格をもつ。した
 がって、最終的に完全なものが、端的に、善と言われる。


\\


Quod autem non habet ultimam perfectionem quam debet habere, quamvis
 habeat aliquam perfectionem inquantum est actu, non tamen dicitur
 perfectum simpliciter, nec bonum simpliciter, sed secundum quid.  Sic
 ergo secundum primum esse, quod est substantiale, dicitur aliquid ens
 simpliciter et bonum secundum quid, idest inquantum est ens, secundum
 vero ultimum actum dicitur aliquid ens secundum quid, et bonum
 simpliciter.  Sic ergo quod dicit Boetius, quod {\itshape in rebus
 aliud est quod sunt bona, et aliud quod sunt}, referendum est ad esse
 bonum et ad esse {\itshape simpliciter}, quia secundum primum actum est
 aliquid ens simpliciter; et secundum ultimum, bonum simpliciter. Et
 tamen secundum primum actum est quodammodo bonum, et secundum ultimum
 actum est quodammodo ens.

&

これに対して、もつべき最終的な完全性をもたないものは、現実態において有である
 かぎりで何らかの完全性をもつとしても、端的に完全なものとは言われず、ま
 た端的に善とも言われない。そのようなものは、いずれも、ある意味において完
 全であり善いものである。ゆえに、実体的であるところの第一の存在において、
 あるものは、端的に有であり、ある意味において、つまり「存在するもの」であ
 るかぎりにおいて善と言われる。他方、最終的な現実態において、あるものは、
 ある意味において有であり、端的に善と言われる。この意味で、ボエティウスが、
 「事物において、善であるところのものと有であるところのものは異なる」と言
 うことは、端的な意味で、「善いものであること」と「存在すること」とへ関係
 付けられるべきである。というのも、第一の現実態において、あるものは、端的
 に有であり、最終的なもの[=現実態]において、端的に善なのだから。しかし、第一
 の現実態においても、ある意味で善であり、最終的な現実態においても、ある意
 味で、有である。

\\



{\scshape Ad secundum dicendum} quod bonum dicitur per informationem,
 prout accipitur bonum simpliciter, secundum ultimum actum.


&

第二に対しては、次のように言わなければならない。形相を受け取ることによっ
 て善と言われるのは、最終的な現実態において、端的に善として理解される限
 りでである。

\\

Et similiter dicendum ad tertium, quod bonum dicitur secundum magis et
 minus, secundum actum supervenientem; puta secundum scientiam vel
 virtutem.

&

第三に対しても同様に言われるべきである。善が、より多く、より少なく言われ
 るのは、学知や徳など、さらに加わってくる現実態に即してである。


 

\end{longtable}

\newpage
\rhead{a.~2}

\begin{center}
 {\Large {\bf ARTICULUS SECUNDUS}}\\
 {\large UTRUM BONUM SECUNDUM RATIONEM SIT PRIUS QUAM ENS}\\
 {\footnotesize I {\itshape Sent.}, d.~8, q.~1, a.~3; III {\itshape
 SCG.}, c.~20; {\itshape De Verit.}, q.~21, a.~2, ad 5; a.~3.}\\
 {\Large 第二項\\善は概念において有より先か}
\end{center}

\begin{longtable}{p{21em}p{21em}}


{\huge A}{\scshape d secundum sic proceditur}. Videtur quod bonum
 secundum rationem sit prius quam ens. Ordo enim nominum est secundum
 ordinem rerum significatarum per nomina. Sed Dionysius, inter alia
 nomina Dei, prius ponit bonum quam ens, ut patet, in III cap.~{\itshape de
 Div.~Nom.} Ergo bonum secundum rationem est prius quam ens.

&
第二に対しては、次のように進められる。善は、有よりも、概念において先だと
 思われる。なぜなら、名称の順番は、その名称によって表示された事物の順番
 に従う。しかるに、ディオニュシウスは、『神名論』第3巻で明らかなように、
 神の他の名称の中で、善を有より先に置いている。ゆえに、善は有よりも、概
 念において先である。

\\



2.~{\scshape Praeterea}, illud est prius secundum rationem, quod ad
 plura se extendit. Sed bonum ad plura se extendit quam ens, quia, ut
 dicit Dionysius, V cap.~{\itshape de Div.~Nom.}, {\itshape bonum se
 extendit ad existentia et non existentia, ens vero ad existentia
 tantum}. Ergo bonum est prius secundum rationem quam ens.

&

さらに、概念において先であるものとは、より大きな外延をもつものである。し
 かるに、善は、有よりも、大きな外延をもつ。これは、ディオニュシウスが
 『神名論』第五巻で「善は存在するものと存在しないものに及ぶのにたいして、
 有は存在するものだけに及ぶ」と述べているとおりである。ゆえに、善は有よ
 りも、概念において先である。

\\


3.~{\scshape Praeterea}, quod est universalius, est prius secundum
 rationem. Sed bonum videtur universalius esse quam ens, quia bonum
 habet rationem appetibilis; quibusdam autem appetibile est ipsum non
 esse; dicitur enim, Matth.~{\scshape xxvi}, de Iuda, {\itshape bonum
 erat ei, si natus non fuisset} et cetera. Ergo bonum est prius quam
 ens, secundum rationem.

&

さらに、より普遍的なものが、概念において先である。しかるに、善は、有より
 も、普遍的であると思われる。なぜなら、善は、「欲求されうるもの」という
 概念をもつが、ある人々にとっては、存在しないものもまた、欲求されうるか
 らである。たとえば、『マタイによる福音書』26章で、ユダについて、「彼はも
 し生まれてこなかったならば、それが彼にとってよかったであろうに、云々」
 \footnote{「生まれなかった方が、その者のためにはよかった。」(26:24)}
 と言われている。ゆえに、善は有よりも、概念において先である。


\\

4.~{\scshape Praeterea}, non solum esse est appetibile, sed et vita et
 sapientia, et multa huiusmodi, et sic videtur quod esse sit quoddam
 particulare appetibile, et bonum, universale. Bonum ergo simpliciter
 est prius secundum rationem quam ens.

&

さらに、たんに存在することだけが「欲求されうるもの」ではなくて、生命や、
 知恵や、その他多くのそのようなものもまた「欲求されうるもの」である。そ
 の意味で、存在することは、ある特定の「欲求されうるもの」である。そして、
 善は、普遍的な「欲求されうるもの」である。ゆえに、善は、有よりも、概念
 において、端的に先である。

\\

{\scshape Sed contra est} quod dicitur in libro {\itshape de Causis},
 quod {\itshape prima rerum creatarum est esse}.

&

しかし反対に、『原因論』という書物の中で、「創造された事物の中で、第一の
 事物は、存在である」と言われている。

\\


{\scshape Respondeo dicendum} quod ens secundum rationem est prius quam
 bonum. Ratio enim significata per nomen, est id quod concipit
 intellectus de re, et significat illud per vocem, illud ergo est prius
 secundum rationem, quod prius cadit in conceptione intellectus. Primo
 autem in conceptione intellectus cadit ens, quia secundum hoc
 unumquodque cognoscibile est, inquantum est actu, ut dicitur in IX
 {\itshape Metaphys}. Unde ens est proprium obiectum intellectus, et sic
 est primum intelligibile, sicut sonus est primum audibile. Ita ergo
 secundum rationem prius est ens quam bonum.

&

答えて言わなければならない。有は、善よりも、概念(ratio)において先である。名称に
 よって表示される概念は、知性が事物についてとらえるものであり、知性はそれ
 を、音声によって表示する。ゆえに、概念において先であるものとは、知性の懐
 念(conceptio)の中に、より先に入ってくるもののことである。しかるに、第一
 に知性の懐念に入ってくるものとは、有である。なぜなら、『形而上学』第9巻
 で言われているように、各々のものが認識されうるものであるのは、それが現実
 態において存在する限りにおいてだからである。したがって、有は、知性の固有
 の対象であり、その意味で、第一の可知的なものである。それはちょうど、音が、
 第一に聞かれうるものであるのと同様である。ゆえに、このように、有は、善よ
 りも概念において先である。

\\


{\scshape Ad primum ergo dicendum} quod Dionysius determinat de divinis
 nominibus secundum quod important circa Deum habitudinem causae,
 nominamus enim Deum, ut ipse dicit, ex creaturis, sicut causam ex
 effectibus. Bonum autem, cum habeat rationem appetibilis, importat
 habitudinem causae finalis, cuius causalitas prima est, quia agens non
 agit nisi propter finem, et ab agente materia movetur ad formam, unde
 dicitur quod finis est causa causarum. Et sic, in causando, bonum est
 prius quam ens, sicut finis quam forma, et hac ratione, inter nomina
 significantia causalitatem divinam, prius ponitur bonum quam ens. Et
 iterum quia, secundum Platonicos, qui, materiam a privatione non
 distinguentes, dicebant materiam esse non ens, ad plura se extendit
 participatio boni quam participatio entis. Nam materia prima participat
 bonum, cum appetat ipsum (nihil autem appetit nisi simile sibi), non
 autem participat ens, cum ponatur non ens. Et ideo dicit Dionysius quod
 bonum extenditur ad non existentia.

&

第一に対しては、それゆえ、次のように言われるべきである。ディオニュシウス
 は、神にかんして、被造物への関係を含意する限りで、神の名前について述べ
 ている。なぜなら、彼が言うとおり、私たちは神を、被造物に基づいて名付け
 るからである。それはちょうど、結果に基づいて原因を名付けるのと同じであ
 る。ところで、善は、欲求されうるものという概念をもつので、目的因の関係
 を含意する。そして、目的因の原因性は第一である。なぜなら、働くものは目
 的のためにでなければ働かないからであり、働くものによって、質料は形相へ
 と動かされるからである。したがって、目的は、原因の中の原因と言われてい
 る。この意味で、原因として働くことにおいて、善は有よりも先である。それ
 はちょうど、目的が形相よりも先であるのと同じである。そしてこの理由によっ
 て、神の原因性を表示する名称の中で、善は有よりも先に置かれる。またさら
 に、プラトン派の人々は、質料を欠如から区別せず、質料は有ではない、と言っ
 たのだが、彼らによれば、善の分有は、有の分有よりも、広い範囲に及ぶ。な
 ぜなら、第一質料は、(自らに似たものでなければなにも欲しないのだが)善
 を欲し、それを分有するが、質料は「存在しないもの」とされるので、有は分
 有しない。このことから、ディオニュシウスは、「善は非存在よりも広くに及
 ぶ」と述べるのである。

\\

Unde patet solutio ad secundum. -- Vel dicendum quod bonum extenditur ad
 existentia et non existentia, non secundum praedicationem, sed secundum
 causalitatem, ut per non existentia intelligamus, non ea simpliciter
 quae penitus non sunt, sed ea quae sunt in potentia et non in actu,
 quia bonum habet rationem finis, in quo non solum quiescunt quae sunt
 in actu, sed ad ipsum etiam ea moventur quae in actu non sunt sed in
 potentia tantum. Ens autem non importat habitudinem causae nisi
 formalis tantum, vel inhaerentis vel exemplaris, cuius causalitas non
 se extendit nisi ad ea quae sunt in actu.

&

以上のことから、第二異論に対する解答は明らかである。あるいは、次のように
 言われるべきである。善は、存在するものと存在しないものへ及ぶが、それは
 述語付けによってではなく、原因性によってである。\footnote{cf.~I
 {\itshape Sent.}, d.~8, q.~1, a.~3, ad 2. ``Ad secundum dicendum, quod
 bonum est communius non secundum ambitum praedicationis, quia sic
 convertitur cum ente, sed secundum rationem causalitatis; causalitas
 enim efficiens exemplaris extenditur tantum ad ea quae participant
 formam actu suae causae exemplaris; et ideo causalitas entis, secundum
 quod est divinum nomen, extenditur tantum ad entia, et vitae ad
 viventia; sed causalitas finis extenditur etiam ad ea quae nondum
 participant formam, quia etiam imperfecta desiderant et tendunt in
 finem nondum participantia rationem finis, quia sunt in via ad
 eum. Vocat enim Dionysius non ens materiam propter privationem
 adjunctam; unde etiam dicit 4 cap.~{\itshape de Div.~Nom}.~quod ipsum non ens
 desiderat bonum.''}ちょうど、存在しないも
 のによって、私たちは、まったく端的に存在しないものを理解するのではなく、
 可能態にあって現実態にないものを理解するように。善は目的の性格をもつが、
 (目的というものは、)現実態にあるものが、そこ(目的)において安らぐだ
 けでなく、可能態だけにあって現実態にないものが、そこ(目的)へと動かさ
 れる。しかるに、有が原因という関係を含意するとしたら、それは、内在的
 (形相)であれ範型的(形相)であれ、形相的な原因でしかない。そして、そ
 のような原因性をもつものの範囲は、現実態において存在するものだけにとど
 まる。

\\



{\scshape Ad tertium dicendum} quod non esse secundum se non est
 appetibile, sed per accidens, inquantum scilicet ablatio alicuius mali
 est appetibilis, quod malum quidem aufertur per non esse. Ablatio vero
 mali non est appetibilis, nisi inquantum per malum privatur quodam
 esse. Illud igitur quod per se est appetibile, est esse, non esse vero
 per accidens tantum, inquantum scilicet quoddam esse appetitur, quo
 homo non sustinet privari. Et sic etiam per accidens non esse dicitur
 bonum.

&

第三に対しては、次のように言われるべきである。非存在は、それ自体において
 欲求されるのではなく、付帯的に、つまり、非存在によって除去されるような
 何らかの悪の除去が欲求されうる限りで、欲求される。他方、悪の除去が欲求
 されうるのは、その悪によって、何らかの存在が欠けているときに限られる。
 ゆえに、それ自体によって欲求されうるのは存在であり、非存在は、それを欠
 くことを人間が耐えられないような、ある存在を欲求するかぎりにおいて、た
 んに付帯的にのみ欲求される。この意味で、非存在は、付帯的に善と言われる。

\\

{\scshape Ad quartum dicendum} quod vita et scientia, et alia huiusmodi,
 sic appetuntur ut sunt in actu, unde in omnibus appetitur quoddam
 esse. Et sic nihil est appetibile nisi ens, et per consequens nihil est
 bonum nisi ens.

&

第四に対しては、次のように言われるべきである。生命、知恵、その他そのよ
うなものは、それらが現実態においてあるかぎにりにおい欲求される。したがって、
 あらゆるものにおいて、何らかの存在が欲求されている。この意味で、有で
 ないならば、なにも欲求されえない。したがって、何ものも、有でないなら
 ば、善ではない。




\end{longtable}
\newpage
\rhead{a.~3}
 

\begin{center}
 {\Large {\bf ARTICULUS TERTIUS}}\\
 {\large UTRUM OMNE ENS SIT BONUM}\\
 {\footnotesize I {\itshape Sent.}, d.~8, q.~1, a.~3; II {\itshape
 SCG.}, c.~41; III, c.~20; {\itshape De Verit.}, q.~21, a.~2; in Boet.~
 {\itshape de Hebd.}, lect.~2.}\\
 {\Large 第三項\\すべての有は善か}
\end{center}

\begin{longtable}{p{21em}p{21em}}



{\huge A}{\scshape d tertium sic proceditur}. Videtur quod non omne ens
 sit bonum. Bonum enim addit supra ens, ut ex dictis patet. Ea vero quae
 addunt aliquid supra ens, contrahunt ipsum, sicut substantia,
 quantitas, qualitas, et alia huiusmodi. Ergo bonum contrahit ens. Non
 igitur omne ens est bonum.

&

第三に対しては、次のように進められる。すべての有が善とは限らないと思われ
 る。なぜなら、すでに述べられたことから明らかなとおり、善は有に、何かを
 加える。他方で、有に何かを加えるものは、それ(有)を制限する。たとえば、
 実体、量、性質、その他そのようなものがそうである。ゆえに、善は有を制限
 する。ゆえに、有がすべて善とは限らない。


\\

2.~{\scshape Praeterea}, nullum malum est bonum, Isaiae {\scshape v},
 {\itshape vae qui dicitis malum bonum, et bonum malum}. Sed aliquod ens
 dicitur malum. Ergo non omne ens est bonum.

&

さらに、いかなる「悪いもの」も善ではない。『イザヤ書』5章「ああ、悪を善
 と言い、善を悪と言うあなた方よ」。\footnote{「災いだ、悪を善と言い、善
 を悪という者は。」(5:20)}しかるに、ある有は「悪いもの」と言われ
 る。ゆえに、有がすべて善とは限らない。

\\

3.~{\scshape Praeterea}, bonum habet rationem appetibilis. Sed materia
 prima non habet rationem appetibilis, sed appetentis tantum. Ergo
 materia prima non habet rationem boni. Non igitur omne ens est bonum.

&

さらに、善は、欲求されうるという性格をもつ。しかるに、第一質料は、欲求さ
 れうるという性格をもたず、欲求するという性格をもつだけである。ゆえに、
 第一質料は善という性格をもたない。ゆえに、すべての有が善とは限らない。

\\

4.~{\scshape Praeterea}, philosophus dicit, in III {\itshape Metaphys}.,
 quod in mathematicis non est bonum. Sed mathematica sunt quaedam entia,
 alioquin de eis non esset scientia. Ergo non omne ens est bonum.

&

さらに、哲学者は『形而上学』第3巻で、数学的なものの中に善はない、と述べ
 ている。しかるに、数学的なものは、一種の有である。もしそうでないならば、
 それらについての学問はなかったであろう。ゆえに、すべての有が善とはかぎ
 らない。

\\

{\scshape Sed contra}, omne ens quod non est Deus, est Dei creatura. Sed
 {\itshape omnis creatura Dei est bona}, ut dicitur I {\itshape ad
 Tim}., {\scshape iv} cap., Deus vero est maxime bonus. Ergo omne ens
 est bonum.

&

しかし反対に、神以外のすべての有は、神の被造物である。しかるに、『テモテ
 への手紙1』4章にあるように、「神の被造物はすべて善である」。
 \footnote{「というのは、神がお造りになったものはすべて良いものであ
 り、」(4:4)}他方、神は
 最大限に善である。ゆえに、すべて有は善である。

\\

{\scshape Respondeo dicendum} quod omne ens, inquantum est ens, est
 bonum. Omne enim ens, inquantum est ens, est in actu, et quodammodo
 perfectum, quia omnis actus perfectio quaedam est. Perfectum vero habet
 rationem appetibilis et boni, ut ex dictis patet. Unde sequitur omne
 ens, inquantum huiusmodi, bonum esse.

&

解答する。以下のように言われるべきである。すべての有は、有であるかぎり、善である。なぜ
 なら、すべての有は、有であるかぎり、現実態にあり、なんらかの意味で完全な
 ものだからである。なぜなら、すべての現実態は何らかの完全性だから。しかる
 に、すでに述べられたことから明らかなとおり、完全なものは、欲求されうるも
 の、すなわち善の性格をもつ。したがって、すべての有はそのようなもの[有]
 であるかぎり善である、ということが帰結する。

\\


{\scshape Ad primum ergo dicendum} quod substantia, quantitas et qualitas, et
 ea quae sub eis continentur, contrahunt ens applicando ens ad aliquam
 quidditatem seu naturam. Sic autem non addit aliquid bonum super ens,
 sed rationem tantum appetibilis et perfectionis, quod convenit ipsi
 esse in quacumque natura sit. Unde bonum non contrahit ens.

&

第一に対しては、それゆえ、次のように言われるべきである。実体、量、性質、
 そしてそれらの下に含まれるものどもは、有を、ある何らかの何性や本性に当
 てはめることによって、有を制限する。しかし、善は、このようなかたちで、
 何かを「在るもの」に加えるわけではなく、むしろ、「欲求されうる」や「完
 全性」という概念だけを加える。こういったことは、どんな本性においても、
 存在それ自体に適合する。したがって、善は有を制限しない。

\\


{\scshape Ad secundum dicendum} quod nullum ens dicitur malum inquantum est
 ens, sed inquantum caret quodam esse, sicut homo dicitur malus
 inquantum caret esse virtutis, et oculus dicitur malus inquantum caret
 acumine visus.

&

第二に対しては、次のように言われるべきである。どんな有も、それが有である
 限りにおいて、「悪いもの」と言われることはなく、むしろ、何らかの存在を
 欠く限りにおいてである。たとえば、人間は、徳の存在を欠く限りにおいて、
 「悪い」と言われ、目は、視覚の鋭さを欠く限りにおいて「悪い」と言われる。

\\



{\scshape Ad tertium dicendum} quod materia prima, sicut non est ens nisi in
 potentia, ita nec bonum nisi in potentia. Licet, secundum Platonicos,
 dici possit quod materia prima est non ens, propter privationem
 adiunctam. Sed tamen participat aliquid de bono, scilicet ipsum ordinem
 vel aptitudinem ad bonum. Et ideo non convenit sibi quod sit
 appetibile, sed quod appetat.

&

第三に対しては、次のように言われるべきである。第一質料は、可能態において
 でなければ有でないように、可能態においてでなければ善でない。ただし、そ
 う言いたければ、プラトン派の人々にしたがって、第一質料は、結びつけられ
 た欠如のために[=必ず欠如の状態にあるために]、有でない、と言われるこ
 とは可能である。しかし、それでも、それは善の何かを分有するのであり、す
 なわちそれは、善への秩序ないし適合性それ自体である。ゆえに、それ(=第
 一質料)には、「欲求されうる」ではなく、「欲求する」[という性格]が適
 合する。


\\

{\scshape Ad quartum dicendum} quod mathematica non subsistunt separata
 secundum esse, quia si subsisterent, esset in eis bonum, scilicet ipsum
 esse ipsorum. Sunt autem mathematica separata secundum rationem tantum,
 prout abstrahuntur a motu et a materia, et sic abstrahuntur a ratione
 finis, qui habet rationem moventis. Non est autem inconveniens quod in
 aliquo ente secundum rationem non sit bonum vel ratio boni, cum ratio
 entis sit prior quam ratio boni, sicut supra dictum est.

&

第四に対しては、次のように言われるべきである。数学的なものは、存在におい
 て分離し自存する、ということがない。なぜなら、もしそれらが自存したら、
 それらにおいて善があっただろう。すなわち、それらの存在そのものが[、善
 であっただろう]。しかし、数学的なものは、運動と質料から抽象されるとい
 うかたちで、概念という点だけで、分離されている。その意味で、それらは、
 動かすものという性格を持つ目的から抽象されている。しかし、概念的に有の
 なかに、善や善の概念が存在しなくても、不都合ではない。上で述べられたよ
 うに、有の概念は、善の概念に先行するからである。

\end{longtable}

\newpage
\rhead{a.~4}

\begin{center}
 {\Large {\bf ARTICULUS QUARTUS}}\\
 {\large UTRUM BONUM HABEAT RATIONEM CAUSAE FINALIS}\\
 {\footnotesize Supra, a.~2, ad 1; I {\itshape Sent.}, d.~34, q.~2,
 a.~1, ad 4; I {\itshape SCG.}, c.~40; {\itshape De Verit.}, q.~21,
 a.~1; {\itshape de Div.~Nom.}, c.~1, lect.~3; II {\itshape Physic.}, lect.~5.}\\
 {\Large 第四項\\善は目的因の性格をもつか}
\end{center}

\begin{longtable}{p{21em}p{21em}}







{\huge A}{\scshape d quartum sic proceditur}. Videtur quod bonum non
 habeat rationem causae finalis, sed magis aliarum. Ut enim dicit
 Dionysius, {\scshape iv} cap.~{\itshape de Div.~Nom}., bonum laudatur
 ut pulchrum{\itshape }. Sed pulchrum importat rationem causae
 formalis. Ergo bonum habet rationem causae formalis.

&


第四に関しては次のように進められる。善は、目的因の性格を持たず、むしろ別
 の[原因の]性格を持つ。なぜなら、ディオニュシウスは『神名論』第4章で、
 「善いものは、美しいものとして褒められる」と言う。しかるに、美しいもの
 は、形相因の性格をもつ。ゆえに、善は形相因の性格を持つ。


\\


2.~{\scshape Praeterea}, bonum est diffusivum sui esse, ut ex verbis
 Dionysii accipitur, quibus dicit quod {\itshape bonum est ex quo omnia
 subsistunt et sunt}. Sed esse diffusivum importat rationem causae
 efficientis. Ergo bonum habet rationem causae efficientis.

&


さらに、ディオニュシウスの「善いものは、それから万物が自存し存在するとこ
 ろのものである」という言葉から理解されるように、善は、自らの存在を広め
 うるものである。しかるに、「存在を広めうるもの」というのは、作出因の性
 格を持つ。ゆえに、善は作出因の性格を持つ。



\\


3.~{\scshape Praeterea}, dicit Augustinus in I {\itshape de
 Doctr.~Christ}., quod {\itshape quia Deus bonus est, nos sumus}. Sed ex
 Deo sumus sicut ex causa efficiente. Ergo bonum importat rationem
 causae efficientis.

&


さらに、アウグスティヌスは、『キリスト教の教え』第1巻で、「神が善いので、
 私たちは存在する」と述べる。しかるに、神から、私たちは存在するが、それ
 は、作出因から、という意味である。ゆえに、善は、作出因の性格をもってい
 る。

\\


Sed contra est quod philosophus dicit, in II {\itshape Physic}., quod
 {\itshape illud cuius causa est, est sicut finis et bonum
 aliorum}. Bonum ergo habet rationem causae finalis.

&


しかし反対に、哲学者は『自然学』第2巻で、「AがBのおかげで存在するとき、
 BはAにとって目的であり善である」と述べている。ゆえに、善は、
 目的因の性格を持つ。

\\


Respondeo dicendum quod, cum bonum sit quod omnia appetunt, hoc autem
 habet rationem finis; manifestum est quod bonum rationem finis
 importat. 
Sed tamen ratio boni praesupponit rationem causae efficientis, et
 rationem causae formalis. Videmus enim quod id quod est primum in
 causando, ultimum est in causato, ignis enim primo calefacit quam
 formam ignis inducat, cum tamen calor in igne consequatur formam
 substantialem. 

&


解答する。以下のように言われるべきである。善は、万物がそれを欲求するところのそれ[=欲
 求の対象]であり、そしてこれ[=欲求の対象]は目的の性格を持つから、善
 が目的の性格を含むことは明らかである。
しかし、善という性格は、作出因の性格と、形相因の性格とを前提する。なぜな
 ら、私たちは、原因として働くという観点で最初のものが、原因から生み出さ
 れるという観点では最後であることを見るからである。たとえば、火は、火の
 形相(=形相因)を導き出すよりも先に、最初に、熱する(=作出因)が、火
 における熱は、実体形相に伴ってついてくるものなのである。


\\


In causando autem, primum invenitur bonum et finis, qui movet
 efficientem; secundo, actio efficientis, movens ad formam; tertio
 advenit forma. 
Unde e converso esse oportet in causato, quod primum sit ipsa forma, per
 quam est ens; secundo consideratur in ea virtus effectiva, secundum
 quod est perfectum in esse (quia unumquodque tunc perfectum est, quando
 potest sibi simile facere, ut dicit philosophus in IV Meteor.); tertio
 consequitur ratio boni, per quam in ente perfectio fundatur.

&


さて、原因として働くという点で、第一のものとして見いだされるのは、善と目
 的であり、これが作出者を動かす。第二に、形相へ向かって動く作出者の働き
 があり、第三に、形相が到来する。
したがって、原因から生み出されるものにおいては、この逆でなければならず、
 第一は、それによって有であるところの形相であり、第二に、その形相におい
 て、作出しうる力が考察され、その力によって、存在において完成されたもの
 である(なぜなら、哲学者が『気象学』第4巻で述べるように、各々のものは、
 自らに似たものを生み出すことができるときに、完全だからである)。そして
 第三に、善の性格が伴う。この性格によって、完全性は、有において基礎をも
 つ。

\\



{\scshape Ad primum ergo dicendum} quod pulchrum et bonum in subiecto quidem
 sunt idem, quia super eandem rem fundantur, scilicet super formam, et
 propter hoc, bonum laudatur ut pulchrum. Sed ratione differunt. Nam
 bonum proprie respicit appetitum, est enim bonum quod omnia
 appetunt. Et ideo habet rationem finis, nam appetitus est quasi quidam
 motus ad rem. Pulchrum autem respicit vim cognoscitivam, pulchra enim
 dicuntur quae visa placent. Unde pulchrum in debita proportione
 consistit, quia sensus delectatur in rebus debite proportionatis, sicut
 in sibi similibus; nam et sensus ratio quaedam est, et omnis virtus
 cognoscitiva. Et quia cognitio fit per assimilationem, similitudo autem
 respicit formam, pulchrum proprie pertinet ad rationem causae formalis.

&


第一に対しては、それゆえ、次のように言われるべきである。「美しいもの」と
 善は、たしかに、基体において同一である。なぜなら、どちらも同一の事物、
 つまり、形相に基礎をもつからであり、このために、善は「美しいもの」と賛
 美される。しかし、これらの概念において異なる。すなわち、善は、その固有
 の意味において、欲求に関係する。善とは、万物がそれを欲求するところのそ
 れのことだからである。ゆえに、善は目的の性格を持つ。欲求は、いわば、事
 物への一種の運動だからである。これに対して、「美しいもの」は、認識する
 力に関係する。実際、見られたときに心地よいものが、「美しいもの」と言わ
 れるのである。このことから、「美しいもの」は、あるべき比率において成立
 する。というのも、感覚は、たとえば自らに似たもののように、ちょうどよい
 比率にある事物において喜ぶからである。これは、感覚もまた、一種の理性で
 あり、すべての認識能力もまたそうであるからである。そして、認識は、[対
 象への]類似化によって生じ、類似は形相に関係するから、「美しいもの」は、
 その固有の意味において、形相因の性格に関係する。

\\


{\scshape Ad secundum dicendum} quod bonum dicitur diffusivum sui esse, eo
 modo quo finis dicitur movere.

&


第二に対しては、次のように言われるべきである。善が、自らの存在を広めうる
 と言われるのは、目的が[何かを]動かすと言われるような、そういう意味に
 おいてである。

\\


{\scshape Ad tertium dicendum} quod quilibet habens voluntatem, dicitur
 bonus inquantum habet bonam voluntatem, quia per voluntatem utimur
 omnibus quae in nobis sunt. Unde non dicitur bonus homo, qui habet
 bonum intellectum, sed qui habet bonam voluntatem. Voluntas autem
 respicit finem ut obiectum proprium, et sic, quod dicitur, {\itshape
 quia Deus est bonus, sumus}, refertur ad causam finalem.

&


第三に対しては、次のように言われるべきである。誰であれ意志を持つ人は、善
 い意志を持つ限りにおいて、善い人と言われる。なぜなら、意志によって、私
 たちは私たちの中にあるすべてのものを用いるからである。したがって、善い
 知性を持つ人ではなく、善い意志を持つ人が、善い人と言われる。しかるに、
 意志は、固有の対象として目的に関わる。このように、「神が善いので、私た
 ちは存在する」と言われるのは、目的因に関わる。



\end{longtable}
\newpage
\rhead{a.~5}
 
\begin{center}
 {\Large {\bf ARTICULUS QUINTUS}}\\
 {\large UTRUM RATIO BONI CONSISTAT IN MODO SPECIE ET ORDINE}\\
 {\footnotesize I-II, q.~85, a.~4; {\itshape De Verit.}, q.~21, a.~6.}\\
 {\Large 第五項\\善の性格は、限度、形象、秩序において成立するか}
\end{center}

\begin{longtable}{p{21em}p{21em}}


{\huge A}{\scshape d quintum sic proceditur}. Videtur quod ratio boni
 non consistat in modo, specie et ordine. Bonum enim et ens ratione
 differunt, ut supra dictum est. Sed modus, species et ordo pertinere ad
 rationem entis videntur, quia, sicut dicitur {\itshape Sap}.~{\scshape
 xi}, {\itshape omnia in numero, pondere et mensura disposuisti}, ad
 quae tria reducuntur species, modus et ordo, quia, ut dicit Augustinus,
 IV {\itshape super Gen.~ad litteram}, {\itshape mensura omni rei modum
 praefigit, et numerus omni rei speciem praebet, et pondus omnem rem ad
 quietem et stabilitatem trahit}. Ergo ratio boni non consistit in modo,
 specie et ordine.

&


第五に関しては次のように進められる。善の性格は、限度、形象、秩序において
 成立するのではないと思われる。なぜなら、上で述べられたように、善と有は
 概念において異なる。しかるに、限度、形象、秩序は、有の性格に属するよう
 に思われる。なぜなら、ちょうど『知恵の書』11章で「あなたは万物を数、重さ、
 尺度のなかに配置した」と言われるように、これら三つに、形象、限度、秩序
 が還元される。というのも、アウグスティヌスが『創世記逐語注解』4巻で言う
 ように、「尺度がすべての事物に限度を指定し、数がすべての事物に形象を与
 え、重さがすべての事物を静止と安定へ導く」からである。ゆえに、善の概念
 は、限度、形象、秩序において成立するのではないと思われる。

\\


2.~{\scshape Praeterea}, ipse modus, species et ordo bona quaedam sunt. Si ergo
 ratio boni consistit in modo, specie et ordine, oportet etiam quod
 modus habeat modum, speciem et ordinem, et similiter species et
 ordo. Ergo procederetur in infinitum.

&


さらに、この限度、形象、秩序自体、ある種の善である。ゆえに、もしも善の概
 念が、限度、形象、秩序において成立するならば、限度が限度、形象、秩序を
 もち、さらに形象も秩序も同様であるはずである。ゆえに、無限に遡ったであ
 ろう。[しかしこれは不合理。ゆえに、云々。]


\\

3.~{\scshape Praeterea}, malum est privatio modi et speciei et ordinis. Sed
 malum non tollit totaliter bonum. Ergo ratio boni non consistit in
 modo, specie et ordine.

&

さらに、「悪いもの」は、限度、形象、秩序の欠如である。しかるに、「悪いも
 の」は、完全に善を除去するわけではない。ゆえに、善の概念は、限度、形象、
 秩序において成立するのではない。

\\


4.~{\scshape Praeterea}, illud in quo consistit ratio boni, non potest dici
 malum. Sed dicitur malus modus, mala species, malus ordo. Ergo ratio
 boni non consistit in modo, specie et ordine.

&

さらに、善の概念がそこにおいて成立するところのそれは、「悪いもの」とは言
 われえない。しかるに、悪い限度、悪い形象、悪い秩序と言われる。ゆえに、
 善の性格は、限度、形象、秩序において成立するのではない。

\\


5.~{\scshape Praeterea}, modus, species et ordo ex pondere, numero et
 mensura causantur, ut ex auctoritate Augustini inducta patet. Non autem
 omnia bona habent pondus, numerum et mensuram, dicit enim Ambrosius, in
 {\itshape Hexaemeron}, quod {\itshape lucis natura est, ut non in
 numero, non in pondere, non in mensura creata sit}. Non ergo ratio boni
 consistit in modo, specie et ordine.

&


さらに、すでに引用されたアウグスティヌスの権威によって明らかであるとおり、
 限度、形象、秩序は、重さ、数、尺度が原因となって生じる。しかるに、必ず
 しもすべて善が、重さ、数、尺度をもつとは限らない。なぜなら、アンブロシ
 ウスは『ヘクサエメロン』で「光の本性は、数においても重さにおいても尺度
 においても創造されなかった」と述べるからである。ゆえに、善の性格は、尺
 度、形象、秩序において成立するのではない。



\\

{\scshape Sed contra est} quod dicit Augustinus, in libro {\itshape de
 Natura Boni}, {\itshape Haec tria, modus, species et ordo, tanquam
 generalia bona sunt in rebus a Deo factis, et ita, haec tria ubi magna
 sunt, magna bona sunt; ubi parva, parva bona sunt; ubi nulla, nullum
 bonum est}. Quod non esset, nisi ratio boni in eis consisteret. Ergo
 ratio boni consistit in modo, specie et ordine.

&

しかし反対に、アウグスティヌスは、『善の本性について』という書物の中で、
 「これら3つ、すなわち限度、形象、秩序は、いわば神によって作られた事物に
 おける一般的な善であり、したがって、これらの三つが大きいところには、大
 きな善があり、少ないところには少ない善が、無いところには、善がない」と
 述べている。もし善の概念がそれらにおいて成立しなかったならば、こういう
 ことはなかったであろう。ゆえに、善の概念は、限度、形象、秩序において成
 立する。

\\


{\scshape Respondeo dicendum} quod unumquodque dicitur bonum, inquantum est
 perfectum, sic enim est appetibile, ut supra dictum est. Perfectum
 autem dicitur, cui nihil deest secundum modum suae perfectionis. 

&

解答する。以下のように言われるべきである。各々のものは、それが完全なものであるかぎりに
 おいて善いものと言われる。なぜなら、上で述べられたように、[完全なもの
 であるかぎりにおいて]欲求されうるものだから。ところで、完全なものと言
 われるのは、それ自身の完全性の限度に応じて、それに欠けるところがなにも
 ないものである。

\\

Cum autem unumquodque sit id quod est, per suam formam; forma autem
 praesupponit quaedam, et quaedam ad ipsam ex necessitate consequuntur;
 ad hoc quod aliquid sit perfectum et bonum, necesse est quod formam
 habeat, et ea quae praeexiguntur ad eam, et ea quae consequuntur ad
 ipsam.

&


さらに、各々のものは、自らの形相によって、その在るところのものであるし、
 また、形相は、何かを前提とし、また、何かが必然的に、その形相に伴うので、
 なにかあるものが完全であり善いものであるためには、それが形相と、形相に
 前もって必要とされるものと、形相に伴うものを持つことが必要である。

\\

Praeexigitur autem ad formam determinatio sive commensuratio
 principiorum, seu materialium, seu efficientium ipsam, et hoc
 significatur per {\itshape modum}, unde dicitur quod {\itshape mensura modum praefigit}.

&


さて、形相に前もって必要とされるのは、諸根源の限定ないし一致である。この
 根源は質料的なものでも作出的なものでもありうる。これが、限度によって意
 味される。このことから、形象は限度を指定する、と言われる。

\\

Ipsa autem forma significatur per {\itshape speciem}, quia per formam unumquodque
 in specie constituitur. Et propter hoc dicitur quod {\itshape numerus speciem
 praebet}, quia definitiones significantes speciem sunt sicut numeri,
 secundum philosophum in VIII {\itshape Metaphys}.; sicut enim unitas addita vel
 subtracta variat speciem numeri, ita in definitionibus differentia
 apposita vel subtracta. 

&


また、この形相は形象によっても表示される。なぜなら、形相によって、各々の
 ものは種の中に立てられるからである。このため、数は形象[=種]を与える、
 と言われる。なぜなら、『形而上学』8巻の哲学者によれば、種を表示する定義
 は、いわば数のようなものだからである。ちょうど、1が加えられたり減じられ
 たりすることで、数の種が異なるように、定義においては種差が加えられたり
 引かれたりする。

\\

Ad formam autem consequitur inclinatio ad finem, aut ad actionem, aut ad
 aliquid huiusmodi, quia unumquodque, inquantum est actu, agit, et
 tendit in id quod sibi convenit secundum suam formam. Et hoc pertinet
 ad {\itshape pondus} et {\itshape ordinem}. Unde ratio boni, secundum quod consistit in
 perfectione, consistit etiam in modo, specie et ordine.

&

さらに、形相には、目的や働きや、なにかそのようなものへの傾向性が伴う。な
 ぜなら、各々のものは、現実態にあるかぎりで、働き、そして、自らの形相に
 従って自らに適合するものに向かって傾くからである。そしてこのことは、重
 さと秩序に属する。したがって、善の概念は、それが完全性において成立する
 かぎり、限度、形象、秩序においてもまた成立する。


\\

{\scshape Ad primum ergo dicendum} quod ista tria non consequuntur ens, nisi
 inquantum est perfectum, et secundum hoc est bonum.

&

第一に対しては、それゆえ、次のように言われるべきである。これら三つは、完
 全なものであるかぎりにおける有にしか伴わない。そしてこのことにしたがっ
 て、[存在するものは]善なのである。

\\

{\scshape Ad secundum dicendum} quod modus, species et ordo eo modo dicuntur
 bona, sicut et entia, non quia ipsa sint quasi subsistentia, sed quia
 eis alia sunt et entia et bona. Unde non oportet quod ipsa habeant
 aliqua alia, quibus sint bona. Non enim sic dicuntur bona, quasi
 formaliter aliis sint bona; sed quia ipsis formaliter aliqua sunt bona;
 sicut albedo non dicitur ens quia ipsa aliquo sit, sed quia ipsa
 aliquid est secundum quid, scilicet album.

&

第二に対しては、次のように言われるべきである。限度、形象、秩序は、有と同
 じように善と言われるが、自存するものとしてではなく、それらによって、他
 のものが有であり善であるところのものとしてである。したがって、これらが、
 それによって善であるところの他の何かをもたなければならない、ということ
 はない。なぜなら、これらは、他のものによって形相的に[=形相的な他のも
 のによって]善いものであるのではなく、それらによって形相的に[=形相的
 なそれらによって]、他のものが善いものだからである。ちょうど、白さは、
 それが何かによって存在するから有と言われるのではなく、それによって、何
 かがある意味において存在する、すなわち、白いものであるから[存在するも
 のと言われるように]。

\\

{\scshape Ad tertium dicendum} quod quodlibet esse est secundum formam
 aliquam, unde secundum quodlibet esse rei, consequuntur ipsam modus,
 species et ordo, sicut homo habet speciem, modum et ordinem, inquantum
 est homo; et similiter inquantum est albus, habet similiter modum,
 speciem et ordinem; et inquantum est virtuosus, et inquantum est
 sciens, et secundum omnia quae de ipso dicuntur. Malum autem privat
 quodam esse, sicut caecitas privat esse visus, unde non tollit omnem
 modum, speciem et ordinem; sed solum modum, speciem et ordinem quae
 consequuntur esse visus.

&


第三に対しては、次のように言われるべきである。どんな存在も何らかの形相に
 したがってある。このことから、事物のどんな存在にも、限度、形象、秩序が
 伴う。ちょうど人間が、人間であるかぎりにおいて、限度、形象、秩序をもつ
 ように。同様に、白いかぎりにおいて、同様に限度、形象、秩序を持ち、有徳
 であるかぎりにおいても、知者である限りにおいても、人間について言われる
 すべてのことにおいて、そうである。ところが、「悪いもの」は、ちょうど盲
 目が視覚を欠くように、何らかの存在を欠く。したがって、必ずしもすべての
 限度、形象、秩序を除去するわけではなく、ただ、視覚の存在に伴う限度、形
 象、秩序を除去するのである。

\\

{\scshape Ad quartum dicendum} quod, sicut dicit Augustinus in libro
 {\itshape de Natura Boni}, {\itshape omnis modus, inquantum modus,
 bonus est} (et sic potest dici de specie et ordine), {\itshape sed
 malus modus, vel mala species, vel malus ordo, aut ideo dicuntur quia
 minora sunt quam esse debuerunt; aut quia non his rebus accommodantur,
 quibus accommodanda sunt; ut ideo dicantur mala, quia sunt aliena et
 incongrua}.

&

第四に対しては、次のように言われるべきである。アウグスティヌスが『善の本
 性について』という書物で言うように、「すべての限度は、限度であるかぎり
 において、善いものである(このことは形象と秩序についても言われうる)が、
 悪い限度、悪い形象、悪い秩序と言われるのは、そうあるべきよりも少なくそ
 うだからか、あるいは、ちょうど、異様で相応しくないから悪いと言われるよ
 うに、それに適合すべき事物に適合していないからかのどちらかである。


\\

{\scshape Ad quintum dicendum} quod natura lucis dicitur esse sine numero et
 pondere et mensura, non simpliciter, sed per comparationem ad
 corporalia, quia virtus lucis ad omnia corporalia se extendit,
 inquantum est qualitas activa primi corporis alterantis, scilicet
 caeli.

&


第五に対しては、次のように言われるべきである。光の本性が数、重さ、尺度を
 持たないと言われるのは、端的にではなく、物体的なものとの比較によってで
 ある。なぜなら、光の力は、変化を与える第一物体、すなわち天の能動的な性
 質として、すべての物体に及ぶからである。



\end{longtable}

\newpage
\rhead{a.~6}

\begin{center}
 {\Large {\bf ARTICULUS SEXTUS}}\\
 {\large UTRUM CONVENIENTER DIVIDATUR BONUM PER HONESTUM, UTILE ET DELECTABILE}\\
 {\footnotesize II-IIae, q.~145, a.~3; II {\itshape Sent.}, d.~21, q.~1,
 a.~3; I {\itshape Ethic.}, lect.~5.}\\
 {\Large 第六項\\善は、有徳、有用、快適によって適切に分割されるか}
\end{center}

\begin{longtable}{p{21em}p{21em}}


{\huge A}{\scshape d sextum sic proceditur}. Videtur quod non
 convenienter dividatur bonum per honestum, utile et delectabile. Bonum
 enim, sicut dicit philosophus in I {\itshape Ethic}., dividitur per decem
 praedicamenta. Honestum autem, utile et delectabile inveniri possunt in
 uno praedicamento. Ergo non convenienter per haec dividitur bonum.

&


第6に関しては次のように進められる。善は有徳、有用、快適によって適切に分
 かたれるわけではないと思われる。なぜなら、善は、哲学者が『ニコマコス倫
 理学』1巻で述べるように、10のカテゴリーによって分かたれる。しかるに、有
 徳、有用、快適は、一つのカテゴリーの中に[=性質のカテゴリーの中に]見
 出される。ゆえに、これらによって善が適切に分かたれるわけではない。

\\


2.~{\scshape Praeterea}, omnis divisio fit per opposita. Sed haec tria non
 videntur esse opposita, nam honesta sunt delectabilia, nullumque
 inhonestum est utile (quod tamen oportet, si divisio fieret per
 opposita, ut opponerentur honestum et utile), ut etiam dicit Tullius,
 in libro {\itshape de Officiis}. Ergo praedicta divisio non est conveniens.

&


さらに、分割はすべて対立するものによって生じる。しかるに、キケロが『義務
 論』の中で述べるように、これら三つは、対立しているように見えない。なぜ
 なら、有徳は快適であるし、有徳でないものは有用でもないからである(もし、
 分割が対立するものによって生じ、有徳と有用が対立していたとしたら、有徳
 でないものは有用であっただろう)。ゆえに、前述の分割は、適切でない。

\\


3.~{\scshape Praeterea}, ubi unum propter alterum, ibi unum tantum est. Sed
 utile non est bonum nisi propter delectabile vel honestum. Ergo non
 debet utile dividi contra delectabile et honestum.

&

さらに、ひとつのものが他のもののためにある場合、そこには、ただひとつのも
 のだけがある。しかるに、有用は、快適や有徳のためにでなければ善ではない。
 ゆえに、有用は、快適と有徳に反して分かたれてはならない。


\\

{\scshape Sed contra est} quod Ambrosius, in libro {\itshape de
 Officiis}, utitur ista divisione boni.

&


しかし反対に、アンブロシウスは、『義務論』の中で、善のこの分割を用いている。

\\

{\scshape Respondeo dicendum} quod haec divisio proprie videtur esse boni
 humani. Si tamen altius et communius rationem boni consideremus,
 invenitur haec divisio proprie competere bono, secundum quod bonum
 est. Nam bonum est aliquid, inquantum est appetibile, et terminus motus
 appetitus. Cuius quidem motus terminatio considerari potest ex
 consideratione motus corporis naturalis. Terminatur autem motus
 corporis naturalis, simpliciter quidem ad ultimum; secundum quid autem
 etiam ad medium, per quod itur ad ultimum quod terminat motum, et
 dicitur aliquis terminus motus, inquantum aliquam partem motus
 terminat. Id autem quod est ultimus terminus motus, potest accipi
 dupliciter, vel ipsa res in quam tenditur, utpote locus vel forma; vel
 quies in re illa. Sic ergo in motu appetitus, id quod est appetibile
 terminans motum appetitus secundum quid, ut medium per quod tenditur in
 aliud, vocatur {\itshape utile}. Id autem quod appetitur ut ultimum, terminans
 totaliter motum appetitus, sicut quaedam res in quam per se appetitus
 tendit, vocatur {\itshape honestum}, quia honestum dicitur quod per se
 desideratur. Id autem quod terminat motum appetitus ut quies in re
 desiderata, est {\itshape delectatio}.

&


解答する。以下のように言われるべきである。この分割は、固有の意味では人間の善に関すると
 思われる。しかし、もし、私たちが、より高い、より共通な善の概念を考察する
 ならば、この分割が、固有の意味で、善であるかぎりにおける善に適合すること
 が見出される。すなわち、善とは、欲求されうるものであるかぎりにおける、何
 かであり、欲求の運動の終極である。この運動の終極は、自然的な物体の運動の
 考察から考えられうる。ところで、自然的物体の運動は、端的に、最終的な場所
 にいたって終わるが、ある意味において、そこを通って最終的な場所へといたる
 中間においてもまた終わるのであり、運動のある部分が終わるかぎりにおいて、
 何らかのものが運動の終極と言われる。ところで、運動の最終的な終極は、二通
 りに理解されうる。一つは、そこへと終極する物自体であり、たとえば、場所や
 形相である。あるいは、その事物における休息としても理解されうる。それゆえ、
 以上のような意味で、欲求の運動において、欲求の運動をある意味において終わ
 らせる「欲求されうるもの」が、そこを通って他のものへ向かう中間として、
 「有用」と呼ばれる。これに対して、最終的なものとして欲求されるものは、欲
 求が自体的にそこへと向かう事物として、欲求の運動を全体的に終わらせるもの
 であり、これが「有徳」と呼ばれる。なぜなら、有徳と言われるものは、それ自
 体によって欲求されるものだから。さらに、欲求された事物における休息として
 運動を終わらせるものは、「快適」である。

\\


{\scshape Ad primum ergo dicendum} quod bonum, inquantum est idem subiecto cum
 ente, dividitur per decem praedicamenta, sed secundum propriam
 rationem, competit sibi ista divisio.

&


第一に対しては、それゆえ、次のように言われるべきである。善は、有と基体において同一であるかぎり、10のカテゴリーによって分かたれる。しかし、その固有の概念においては、この分割が適合するのである。

\\

{\scshape Ad secundum dicendum} quod haec divisio non est per oppositas res,
 sed per oppositas rationes. Dicuntur tamen illa proprie delectabilia,
 quae nullam habent aliam rationem appetibilitatis nisi delectationem,
 cum aliquando sint et noxia et inhonesta. Utilia vero dicuntur, quae
 non habent in se unde desiderentur; sed desiderantur solum ut sunt
 ducentia in alterum, sicut sumptio medicinae amarae. Honesta vero
 dicuntur, quae in seipsis habent unde desiderentur.

&


第二に対しては、次のように言われるべきである。この分割は、対立する事物によるのではなく、対立する概念による。しかし、固有の意味で快適と言われるのは、時には有害で不徳であるにしても、快適以外のどんな他の「欲求されうるもの」の概念ももたないものである。他方、有用と言われるのは、自らの中に望まれる理由をもたず、ただ他のものへ導くものとして望まれるものである。たとえば、苦い薬を飲むことがこれに当たる。さらに、有徳と言われるのは、自らの中に、望まれる理由を持つものである。

\\

{\scshape Ad tertium dicendum} quod bonum non dividitur in ista tria sicut
 univocum aequaliter de his praedicatum, sed sicut analogum, quod
 praedicatur secundum prius et posterius. Per prius enim praedicatur de
 honesto; et secundario de delectabili; tertio de utili.

&


第三に対しては、次のように言われるべきである。善は、これらについて等しく
 述語される一義的なものとしてこの三つに分かたれるのではなく、アナロギア
 的なもの、つまり、より先・よりあとにしたがって述語されるものとして分か
 たれる。じっさい、より先に、有徳に述語付けられ、第二に、快適に、第三に、
 有用に述語付けられる。



\end{longtable}
\end{document}