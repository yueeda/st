\documentclass[10pt]{jsarticle} % use larger type; default would be 10pt
%\usepackage[utf8]{inputenc} % set input encoding (not needed with XeLaTeX)
%\usepackage[round,comma,authoryear]{natbib}
%\usepackage{nruby}
\usepackage{okumacro}
\usepackage{longtable}
%\usepqckage{tablefootnote}
\usepackage[polutonikogreek,english,japanese]{babel}
%\usepackage{amsmath}
\usepackage{latexsym}
\usepackage{color}
%----- header -------
\usepackage{fancyhdr}
\pagestyle{fancy}
\lhead{{\it Summa Theologiae} I, q.~6}
%--------------------

\bibliographystyle{jplain}
\title{{\bf Prima Pars}\\{\HUGE Summae Theologiae}\\Sancti Thomae
Aquinatis\\Quaestio Sexta\\{\bf De Bonitate Dei}}
\author{Japanese translation\\by Yoshinori {\sc Ueeda}}
\date{Last modified \today}




%%%% コピペ用
%\rhead{a.~}
%\begin{center}
% {\Large {\bf }}\\
% {\large }\\
% {\footnotesize }\\
% {\Large \\}
%\end{center}
%
%\begin{longtable}{p{21em}p{21em}}
%
%&
%
%\\
%\end{longtable}
%\newpage



\begin{document}

\maketitle

\begin{center}
 {\Large 第六問\\神の善性について}
\end{center}

\begin{longtable}{p{21em}p{21em}}

Deinde quaeritur de bonitate Dei. Et circa hoc quaeruntur quatuor. 
\begin{enumerate}
 \item utrum esse bonum conveniat Deo.
 \item utrum Deus sit summum bonum.
 \item utrum ipse solus sit bonus per suam essentiam.
 \item utrum omnia sint bona bonitate divina.
\end{enumerate}
&

次に、神の善性について問われる。そしてこれにかんして、4つのことが問われ
 る。
\begin{enumerate}
 \item 善であることは神に適合するか。
 \item 神は最高善か。
 \item 神だけが、自己の本質によって善か。
 \item 万物は神の善性によって善か。
\end{enumerate}
\end{longtable}

\newpage
\rhead{a.~1}

\begin{center}
 {\Large {\bf ARTICULUS PRIMUS}}\\
 {\large UTRUM ESSE BONUM DEO CONVENIAT}\\
 {\footnotesize I {\itshape SCG.}, c.~37; {\itshape XII Metaphysic.}, l.~7.}\\
 {\Large 第一項\\善であることは神に適合するか}
\end{center}

\begin{longtable}{p{21em}p{21em}}

{\huge A}{\scshape d primum sic proceditur}. Videtur quod esse bonum non
 conveniat Deo. Ratio enim boni consistit in modo, specie et
 ordine. Haec autem non videntur Deo convenire, cum Deus immensus sit,
 et ad aliquid non ordinetur. Ergo esse bonum non convenit Deo.

&

第一に対しては、次のように進められる。善であることは、神に適合しないと思
 われる。なぜなら、善の概念は、限度、形象、秩序において成立する。しかる
 に、これらが神に適合するとは思われない。なぜなら、神は限度がなく、どん
 なものへも秩序付けられないからである。ゆえに、善であることは神に適合し
 ない。


\\

{\scshape Praeterea}, bonum est quod omnia appetunt. Sed Deum non omnia
 appetunt, quia non omnia cognoscunt ipsum, nihil autem appetitur nisi
 notum. Ergo esse bonum non convenit Deo.

&

さらに、善とは、万物が欲求するところのものである。しかるに、万物が神を欲
 求するわけではない。なぜなら、万物が神を認識するわけではないが、知られ
 たものでなければ、なにも欲求されないからである。ゆえに、善であることは、
 神に適合しない。

\\

{\scshape Sed contra est} quod dicitur {\itshape Thren}. {\scshape iii},
 {\itshape bonus est dominus sperantibus in eum, animae quaerenti
 illum}.

&

しかし反対に、『哀歌』三章で「主へ求める人々にとって、主を求める魂に
 とって、主は善い」\footnote{「主に望みをおき尋ね求める魂に主は幸いをお
 与えになる。」(3:25)}と言われている。


\\

{\scshape Respondeo dicendum} quod bonum esse praecipue Deo
 convenit. Bonum enim aliquid est, secundum quod est
 appetibile. Unumquodque autem appetit suam perfectionem. Perfectio
 autem et forma effectus est quaedam similitudo agentis, cum omne agens
 agat sibi simile. Unde ipsum agens est appetibile, et habet rationem
 boni, hoc enim est quod de ipso appetitur, ut eius similitudo
 participetur. Cum ergo Deus sit prima causa effectiva omnium,
 manifestum est quod sibi competit ratio boni et appetibilis. Unde
 Dionysius, in libro {\itshape de Div.~Nom.}, attribuit bonum Deo sicut primae
 causae efficienti, dicens quod bonus dicitur Deus, {\itshape sicut ex quo omnia
 subsistunt}.

&

答えて言わなければならない。善であることは、とくに、神に適合する。理由は
 以下のとおり。何かあるものは、それが欲求されうる限りにおいて、善である。
 しかるに、各々のものは、自らの完成を欲求する。ところで、結果の完成と形
 相は、働くもののなんらかの類似である。なぜなら、すべて働くものは、自ら
 に似たものを生み出すからである。したがって、働くもの自身は、欲求されう
 るものであり、善の性格をもつ。つまり、それ(働くもの)について欲求され
 るのは、「それの類似が分有される」ということである。そういうわけだから、
 神は万物の第一作出因なので、善や欲求されうるという性格が、神に適合する
 ことは明らかである。このことから、ディオニュシウスは『神名論』で、「そ
 こから万物が存在するものとして、神は善と言われる」と述べて、第一作出因
 としての神に、善を帰属させている。


\\



Ad primum ergo dicendum quod habere modum, speciem et ordinem, pertinet
 ad rationem boni causati. Sed bonum in Deo est sicut in causa, unde ad
 eum pertinet imponere aliis modum, speciem et ordinem. Unde ista tria
 sunt in Deo sicut in causa.

&

第一に対しては、それゆえ、次のように言われるべきである。限度、形象、秩序
 をもつことは、なんらかの原因によって生み出された善の性格にあてはまる。
 しかし、善は、神の中に、原因の中にあるようなかたちである。このことから、
 神には、他のものどもに、限度、形象、秩序を与えることが属する。したがっ
 て、これら三つは、原因のなかにあるようなかたちで、神の中にある。


\\

Ad secundum dicendum quod omnia, appetendo proprias perfectiones,
 appetunt ipsum Deum, inquantum perfectiones omnium rerum sunt quaedam
 similitudines divini esse, ut ex dictis patet. Et sic eorum quae Deum
 appetunt, quaedam cognoscunt ipsum secundum seipsum, quod est proprium
 creaturae rationalis. Quaedam vero cognoscunt aliquas participationes
 suae bonitatis, quod etiam extenditur usque ad cognitionem
 sensibilem. Quaedam vero appetitum naturalem habent absque cognitione,
 utpote inclinata ad suos fines ab alio superiori cognoscente.

&


第二に対しては、次のように言われるべきである。すでに言われたように、すべ
 ての事物の完全性が、神の存在の一種の類似であるかぎりにおいて、万物は、
 自分に固有の完全性を欲求することによって、神そのものを欲求している。こ
 のように、神を欲求するものどもの中で、あるものどもは、神を神自身におい
 て認識する。これは、理性的被造物に固有のことである。また、あるものども
 は、神の善性のなんらかの分有を認識するが、これは、感覚的な認識までも広
 がる。さらにあるものどもは、認識なしに、自然本性的な欲求をもつ。たとえ
 ば、他の上位の認識者によって、自らの目的へと傾かされるものどものように。


\end{longtable}

\newpage
\rhead{a.~2}

\begin{center}
 {\Large {\bf ARTICULUS SECUNDUS}}\\
 {\large UTRUM DEUS SIT SUMMUM BONUM}\\
 {\footnotesize II {\itshape Sent.}, d.~1, q.~2, a.~2, ad 4; I {\itshape
 SCG.}, c.~16.}\\
 {\Large 第二項\\神は最高善か}
\end{center}

\begin{longtable}{p{21em}p{21em}}

{\huge A}{\scshape d secundum sic proceditur}. Videtur quod Deus non sit summum
 bonum. Summum enim bonum addit aliquid supra bonum, alioquin omni bono
 conveniret. Sed omne quod se habet ex additione ad aliquid, est
 compositum. Ergo summum bonum est compositum. Sed Deus est summe
 simplex, ut supra ostensum est. Ergo Deus non est summum bonum.

&


第二に対しては、次のように進められる。神は最高善ではないと思われる。理由
 は以下のとおり。最高善は、善に何かを加える。そうでなければ、すべての善
 が最高善だということになったであろう。しかるに、すべて、何かに加えるこ
 とによってあるものは、複合されたものである。ゆえに、最高善は、複合され
 たものである。しかし、神は、上で明らかにされたとおり、最高度に単純であ
 る。ゆえに、神は最高善ではない。


\\

{\scshape Praeterea}, {\itshape bonum est quod omnia appetunt}, ut dicit
 philosophus. Sed nihil aliud est quod omnia appetunt, nisi solus Deus,
 qui est finis omnium. Ergo nihil aliud est bonum nisi Deus. Quod etiam
 videtur per id quod dicitur {\itshape Matth}. {\scshape xix}, nemo
 bonus nisi solus Deus. Sed summum dicitur in comparatione aliorum;
 sicut summum calidum in comparatione ad omnia calida. Ergo Deus non
 potest dici summum bonum.

&


さらに、哲学者が言うように、善とは、万物が欲求するところのものである。し
 かるに、万物が欲求するところのものとは、神以外にはない。神は、万物の目的
 なのだから。ゆえに、神以外に善はない。これは、『マタイによる福音書』19章
 で「ただ神一人を除いて善いものはいない」\footnote{「イエスは言われた。
 「なぜ、善いことについて、わたしに尋ねるのか。善い方はおひとりである。も
 し命を得たいのなら、掟を守りなさい。」」(19:17)}と述べられていることによっ
 てわかる。しかし、最高ということは、他のものとの比較で言われる。たとえば、
 最高に熱いものというのは、すべての熱いものとの比較の中で言われる。ゆえに、
 神が最高善と言われることはできない。


\\

{\scshape Praeterea}, summum comparationem importat. Sed quae non sunt unius
 generis, non sunt comparabilia; sicut dulcedo inconvenienter dicitur
 maior vel minor quam linea. Cum igitur Deus non sit in eodem genere cum
 aliis bonis, ut ex superioribus patet, videtur quod Deus non possit
 dici summum bonum respectu eorum.

&

さらに、最高ということは比較を含意する。しかるに、一つの類に属さないものは比較されえない。たとえば、甘さが直線より大きいとか小さいとか言われるのは不適切である。ゆえに、上で述べられたことから明らかな通り、神は他の善と同じ類に属さないのだから、神は、それらの善との関係で最高善と言われることはできない。

\\

{\scshape Sed contra est} quod dicit Augustinus, I {\itshape de Trin}.,
 quod Trinitas divinarum personarum {\itshape est summum bonum, quod
 purgatissimis mentibus cernitur}.

&

しかし反対に、アウグスティヌスは、『三位一体論』第一巻で、「神のペルソナ
 の三位一体は最高善であり、もっとも清らかな精神によって見られる」、と述
 べている。


\\

{\scshape Respondeo dicendum} quod Deus est summum bonum simpliciter, et
 non solum in aliquo genere vel ordine rerum. Sic enim bonum Deo
 attribuitur, ut dictum est, inquantum omnes perfectiones desideratae
 effluunt ab eo, sicut a prima causa. Non autem effluunt ab eo sicut ab
 agente univoco, ut ex superioribus patet, sed sicut ab agente quod non
 convenit cum suis effectibus, neque in ratione speciei, nec in ratione
 generis. Similitudo autem effectus in causa quidem univoca invenitur
 uniformiter, in causa autem aequivoca invenitur excellentius, sicut
 calor excellentiori modo est in sole quam in igne. Sic ergo oportet
 quod cum bonum sit in Deo sicut in prima causa omnium non univoca, quod
 sit in eo excellentissimo modo. Et propter hoc dicitur summum bonum.

&

答えて言わなければならない。神は、事物のなんらかの類や秩序において最高善
 であるだけでなく、端的に、最高善である。すでに述べられたように、善が神
 に帰せられるのは、欲求されるすべての完全性が、第一原因としての神から流
 れ出るかぎりにおいてである。ところで、上述のことから明らかなように、神
 から流れ出るのは、一義的作用者から流れ出るようにではなく、種の性格にお
 いても類の性格においても、それの結果と一致しない作用者から流れ出るよう
 にである。しかるに、一義的原因の中に、結果の類似は一様に見出されるが、
 異義的原因の中に、(結果の類似は)より優れたかたちで見出される。ちょう
 ど、熱が、火においてよりも太陽において、より優れたかたちで存在するよう
 に。ゆえに、このように、善は神の中に、万物の一義的でない第一原因の中に
 あるようなかたちであるのだから、神の中に、(善は)もっとも優れたかたち
 で存在しなければならない。そしてこのために、(神は)最高善と言われる。


\\


{\scshape Ad primum ergo dicendum} quod summum bonum addit super bonum, non rem
 aliquam absolutam, sed relationem tantum. Relatio autem qua aliquid de
 Deo dicitur relative ad creaturas, non est realiter in Deo, sed in
 creatura; in Deo vero secundum rationem; sicut scibile relative dicitur
 ad scientiam, non quia ad ipsam referatur, sed quia scientia refertur
 ad ipsum. Et sic non oportet quod in summo bono sit aliqua compositio,
 sed solum quod alia deficiant ab ipso.

&


第一に対しては、それゆえ、次のように言われるべきである。最高善は、善に、
 無条件的(非関係的)ななんらかの事物を加えるのではなく、たんに関係を加
 えるだけである。ところで、被造物にたいして、神について何かが関係的に言
 われるとき、その関係は、神においては実在的でなく、(実在的なのは)被造
 物においてであり、これに対して、神においては、概念においてのみある。そ
 れはちょうど、「知られうる」ということが、知に対して関係的に言われるの
 は、それが知へと関係するからではなく、知がそれ(知の対象)へと関係する
 からであるようなものである。このように、最高善の中に、なんらかの複合が
 なければならないというわけではなく、むしろ、他の善が、最高善より劣って
 いるということでなければならないだけである。

\\

{\scshape Ad secundum dicendum} quod, cum dicitur {\itshape bonum est quod omnia appetunt},
 non sic intelligitur quasi unumquodque bonum ab omnibus appetatur, sed
 quia quidquid appetitur, rationem boni habet. Quod autem dicitur, nemo
 bonus nisi solus Deus, intelligitur de bono per essentiam, ut post
 dicetur.

&

第二に対しては、次のように言われるべきである。善とは万物が欲求するところ
 のものである、と言われるとき、各々の善が、万物によって欲求される、と理
 解されるのではなく、欲求されるものはなんであれ、善の性格を持つ、という
 意味に理解される。これに対して、「一人神以外に善いものはいない」と言わ
 れることは、あとで言われるように、\footnote{次項。}本質による善について理解されている。


\\

{\scshape Ad tertium dicendum} quod ea quae non sunt in eodem genere, si quidem
 sint in diversis generibus contenta, nullo modo comparabilia sunt. De
 Deo autem negatur esse in eodem genere cum aliis bonis, non quod ipse
 sit in quodam alio genere; sed quia ipse est extra genus, et principium
 omnis generis. Et sic comparatur ad alia per excessum. Et huiusmodi
 comparationem importat summum bonum.

&

第三に対しては、次のように言われるべきである。同じ類の中にないものは、異
 なる類に含まれているのであれば、どのようなかたちによっても比較されえな
 い。しかし、神については、他の善と同じ類の中にあるということは否定され
 るが、神は、他のなんらかの類にあるのではなく、むしろ神は類の外にあり、
 すべての類の根源である。かくして、神は、他の善に対して、超越的に(=限
 度をはみ出ているかたちで)関係する。このような関係を、最高善は意味して
 いる。

 
\end{longtable}
\newpage
\rhead{a.~3}

\begin{center}
 {\Large {\bf ARTICULUS TERTIUS}}\\
 {\large UTRUM ESSE BONUM PER ESSENTIAM SIT PROPRIUM DEI}\\
 {\footnotesize I {\itshape SCG.}, c.~38; III, c.~20; {\itshape De
 Verit.}, q.~21, a.~1, ad 1; a.~5; {\itshape Compend.~Theol.}, c.~109;
 {\itshape de Div.~Nom.}, c.~4, lect.~1; in Boet.~{\itshape de
 Hebdomad.}, lect.~3, 4.}\\
 {\Large 第三項\\善であることは、本質によって神に固有か }
\end{center}

\begin{longtable}{p{21em}p{21em}}

{\huge A}{\scshape d tertium sic proceditur}. Videtur quod esse bonum
per essentiam non sit proprium Dei. Sicut enim unum convertitur cum
ente, ita et bonum, ut supra habitum est. Sed omne ens est unum per suam
essentiam, ut patet per philosophum in IV {\itshape Metaphys}. Ergo omne
ens est bonum per suam essentiam.  &

第三に対しては次のように進められる。本質によって善であることは、神に固有
ではないと思われる。理由は以下のとおり。上で述べられたとおり、「一つのも
の」と「存在するもの」が置換されるように、「善いもの」も「存在するもの」
と置換される。しかるに、『形而上学』第4巻の哲学者によって明らかなとおり、
すべて存在するものは、本質によって一つのものである。ゆえに、すべて存在す
るものは、本質によって善いものである。

\\


{\scshape Praeterea}, si bonum est quod omnia appetunt, cum ipsum esse sit
desideratum ab omnibus, ipsum esse cuiuslibet rei est eius bonum. Sed
quaelibet res est ens per suam essentiam. Ergo quaelibet res est bona
per suam essentiam.

&

さらに、もし、善は万物が欲求するところのものであるとすれば、存在そのもの
 は万物によって欲求されるのだから、どんな事物であれその事物の存在そのも
 のが、そのものの善であろう。しかるに、どんな事物も、その事物の本質によっ
 て、存在するものである。ゆえに、どんな事物も、その事物の本質によって、
 善いものである。


\\

{\scshape Praeterea}, omnis res per suam bonitatem est bona. Si igitur
aliqua res est quae non sit bona per suam essentiam, oportebit quod eius
bonitas non sit sua essentia. Illa ergo bonitas, cum sit ens quoddam,
oportet quod sit bona, et si quidem alia bonitate, iterum de illa
bonitate quaeretur. Aut ergo erit procedere in infinitum, aut venire ad
aliquam bonitatem quae non erit bona per aliam bonitatem. Eadem ergo
ratione standum est in primo. Res igitur quaelibet est bona per suam
essentiam.

&

さらに、すべての事物は、その事物の善性によって善である。ゆえに、もし、自
 らの本質によって善でないような事物があれば、それの善性は、その事物の本
 質ではないことになろう。ゆえに、その善性は、なんらかの存在するものなの
 だから、善でなければならず、また、もし、他の善性によって善なのであれば、
 さらにその善性について問われることになろう。ゆえに、それは無限に進行す
 るか、他の善性によって善なのではない善性に到達するかのいずれかである。
 ゆえに、同じ理由によって、最初のところにとどまるべきである。ゆえに、ど
 んな事物も、その事物の本質によって善である。

\\

{\scshape Sed contra est} quod dicit Boetius, in libro {\itshape de
Hebdomad}., quod alia omnia a Deo sunt bona per participationem. Non
igitur per essentiam.

&

しかし反対に、ボエティウスは、『デ・ヘブドマディブス』という書物で、神以外の全てのものは、分有によって善である、と述べている。ゆえに、本質によって善なのではない。
\\

{\scshape Respondeo dicendum} quod solus Deus est bonus per suam
essentiam. Unumquodque enim dicitur bonum, secundum quod est
perfectum. Perfectio autem alicuius rei triplex est. Prima quidem,
secundum quod in suo esse constituitur. Secunda vero, prout ei aliqua
accidentia superadduntur, ad suam perfectam operationem
necessaria. Tertia vero perfectio alicuius est per hoc, quod aliquid
aliud attingit sicut finem. Utpote prima perfectio ignis consistit in
esse, quod habet per suam formam substantialem, secunda vero eius
perfectio consistit in caliditate, levitate et siccitate, et huiusmodi,
tertia vero perfectio eius est secundum quod in loco suo quiescit. 

&

答えて言わなければならない。ただ神だけが自らの本質によって善である。理由
 は以下のとおり。各々のものは、完全であるかぎりにおいて善と言われる。と
 ころで、何かの事物の完全性は、三通りにある。第一の完全性は、それが、自
 らの存在において立てられることにおいてである。第二の完全性は、自らの完
 全な働きのために必要ななんらかの付帯性が、それに加えられる限りでである。
 また、何かの第三の完全性は、目的としての何か他のものへ到達することによ
 る。たとえば、火の第一完全性は、存在において成立し、(火は)それを自ら
 の実体形相によってもつ。また、火の第二完全性は、熱さ、軽さ、乾燥、その
 他において成立する。さらに火の第三完全性は、自らの場所に安らうかぎりに
 おいてある。


\\

Haec
autem triplex perfectio nulli creato competit secundum suam essentiam,
sed soli Deo, cuius solius essentia est suum esse; et cui non adveniunt
aliqua accidentia; sed quae de aliis dicuntur accidentaliter, sibi
conveniunt essentialiter, ut esse potentem, sapientem, et huiusmodi,
sicut ex dictis patet. Ipse etiam ad nihil aliud ordinatur sicut ad
finem, sed ipse est ultimus finis omnium rerum. Unde manifestum est quod
solus Deus habet omnimodam perfectionem secundum suam essentiam. Et ideo
ipse solus est bonus per suam essentiam.

&


これら三とおりの完全性は、どんな被造物においても本質にした
 がって属することはなく、ただ神にのみ(本質によって属する)。ただ神の本
 質だけが、自らの存在であり、神にはどんな付帯性も到来せず、たとえば、す
 でに述べられたことから明らかなとおり、力がある、知恵があるなど、他のも
 のについて付帯的に言われることが、神には本質的に到来する。さらに、神は、
 他のどんなものにも、目的へ向かうように秩序付けられていない。むしろ、神
 がすべての事物の究極目的である。したがって、神だけが、自らの本質にした
 がって、あらゆる意味における完全性をもつことが明らかである。ゆえに、神
 だけが、自らの本質によって善である。


\\

{\scshape Ad primum ergo dicendum} quod unum non importat rationem perfectionis,
sed indivisionis tantum, quae unicuique rei competit secundum suam
essentiam. Simplicium autem essentiae sunt indivisae et actu et
potentia, compositorum vero essentiae sunt indivisae secundum actum
tantum. Et ideo oportet quod quaelibet res sit una per suam essentiam,
non autem bona, ut ostensum est.

&

第一に対しては、それゆえ、次のように言われるべきである。「一つのもの」は、
 完全性という概念を含意せず、ただ不可分という概念を含意するだけである。
 この不可分ということは、各々の事物に、それの本質にしたがって適合する。
 ただし、単純なものどもの本質は、現実態においても可能態においても不可分
 だが、複合体の本質は、現実態においてのみ不可分である。ゆえに、どんな事
 物も、その本質によって一つでなければならないが、すでに示されたとおり、
 (その本質によって)善でなければならないというわけではない。




\\

{\scshape Ad secundum dicendum} quod, licet unumquodque sit bonum inquantum habet
esse, tamen essentia rei creatae non est ipsum esse, et ideo non
sequitur quod res creata sit bona per suam essentiam.

&

第二に対しては、次のように言われるべきである。各々のものが存在をもつかぎ
 りにおいて善であるにしても、しかし、創造された事物の本質は、存在そのも
 のではないので、創造された事物が、その本質によって善だということは帰結
 しない。


\\

{\scshape Ad tertium dicendum} quod bonitas rei creatae non est ipsa
eius essentia, sed aliquid superadditum; vel ipsum esse eius, vel aliqua
perfectio superaddita, vel ordo ad finem. Ipsa tamen bonitas sic
superaddita dicitur bona sicut et ens, hac autem ratione dicitur ens,
quia ea est aliquid, non quia ipsa aliquo alio sit. Unde hac ratione
dicitur bona, quia ea est aliquid bonum, non quia ipsa habeat aliquam
aliam bonitatem, qua sit bona.

&

第三に対しては、次のように言われるべきである。創造された事物の善性は、そ
 れの本質そのものではなく、加えられた何かである。それは、そのものの存在
 それ自体であるか、あるいは、加えられた完全性であるか、あるいは、目的へ
 の秩序である。しかし、そのように加えられた善性は、「存在するもの」と言
 われるのと同じように、「善いもの」と言われる。しかるに、それが「存在す
 るもの」と言われるのは、それによって何かが存在するからであり、何か他の
 ものによってそれが存在するからではない。そのように、何かが「善いもの」
 と言われるのは、それによって何かが善いからであって、それが、何か他の善
 性を持ち、その、他の善性によって、それが善いから、というのではない。

\end{longtable}
\newpage
\rhead{a.~4}

\begin{center}
 {\Large {\bf ARTICULUS QUARTUS}}\\
 {\large UTRUM OMNIA SINT BONA BONITATE DIVINA}\\
 {\footnotesize I {\itshape Sent.}, d.~19, q.~5, a.~2, ad 3; {\itshape
 SCG.}, c.~40; {\itshape De Verit.}, q.~21, a.~4}\\
 {\Large 第四項\\すべてのものは神の善性によって善か }
\end{center}

\begin{longtable}{p{21em}p{21em}}

{\huge A}{\scshape d quartum sic proceditur}. Videtur quod omnia sint
bona bonitate divina. Dicit enim Augustinus, VII {\itshape de Trin}.,
{\itshape bonum hoc et bonum illud, tolle hoc et tolle illud, et vide
ipsum bonum, si potes, ita Deum videbis, non alio bono bonum, sed bonum
omnis boni}. Sed unumquodque est bonum suo bono. Ergo unumquodque est
bonum ipso bono quod est Deus.

&

第四に対しては次のように進められる。万物は神の善性によって善であると思わ
 れる。アウグスティヌスは『三位一体論』第7巻で、「この善、あの善、「こ
 の」を捨て、「あの」を捨てよ。そしてあなたに可能なら、善そのものを見よ。
 そうすれば、あなたは神を見るだろう。他の善によって善であるものではなく、
 すべての善の善であるものを」と述べている。しかるに、各々のものは、自ら
 の善によって善である。ゆえに、各々のものは、神である善そのものによって、
 善である。

\\

{\scshape Praeterea}, sicut dicit Boetius, in libro {\itshape de
Hebdomad}., omnia dicuntur bona inquantum ordinantur ad Deum, et hoc
ratione bonitatis divinae. Ergo omnia sunt bona bonitate divina.

&

さらに、ボエティウスが『デ・ヘブドマディブス』で述べているように、万物は、
 神へ秩序付けられるかぎりにおいて善と言われるが、これは、神の善性のため
 にである。ゆえに、万物は、神の善性によって善である。

\\

{\scshape Sed contra est} quod omnia sunt bona inquantum sunt. Sed non
dicuntur omnia entia per esse divinum, sed per esse proprium. Ergo non
omnia sunt bona bonitate divina, sed bonitate propria.

&

しかし反対に、万物は、存在するかぎりにおいて善である。しかし、万物は、神
 の存在によってではなく、固有の存在によって、存在するものと言われる。ゆ
 えに、万物は神の善性によって善なのではなく、固有の善性によって善である。


\\

{\scshape Respondeo dicendum} quod nihil prohibet in his quae relationem
 important, aliquid ab extrinseco denominari; sicut aliquid denominatur
 locatum a loco, et mensuratum a mensura.  Circa vero ea quae absolute
 dicuntur, diversa fuit opinio.

&

答えて言わなければならない。関係を含意するものどもにおいて、何かが外的な
 ものによって命名されることがあってもかまわない。位置にあるものが位置に
 よって、計られるものが尺度によって名付けられる場合のように。これに対し
 て、絶対的に(非関係的に)言われるものどもにかんしては、さまざまな意見
 があった。

\\

 Plato enim posuit omnium rerum species separatas; et quod ab eis
 individua denominantur, quasi species separatas participando; ut puta
 quod Socrates dicitur homo secundum ideam hominis separatam. Et sicut
 ponebat ideam hominis et equi separatam, quam vocabat {\itshape per se
 hominem} et {\itshape per se equum}, ita ponebat ideam entis et ideam
 unius separatam, quam dicebat {\itshape per se ens} et {\itshape per se
 unum}, et eius participatione unumquodque dicitur ens vel unum. Hoc
 autem quod est per se bonum et per se unum, ponebat esse summum Deum ,
 a quo omnia dicuntur bona per modum participationis.\footnote{この文は、
 Leonina版では次のようになっている。``Hoc autem quod est per se
 {\bfseries ens} et per se unum, ponebat esse summum {\bfseries
 bonum}. {\bfseries Et quia bonum convertitur cum ente, sicut et unum,
 ipsum per se bonum dicebat esse Deum}, q quo omnia dicuntur bona per
 modum participationis.''\\「さらに彼は、有自体や、一自体であるものが善だとした。そして一と同様に、善と有は置換されるので、善そのものが神であると
 言った。この神によって、すべてのものは、分有というかたちで善と
 言われる。」}

&

たとえばプラトンは、すべての事物の形象が分離したものだとしたが、それらに
 よって、あたかも分離した形象を分有することによってであるかのようにして、
 個々のものが名付けられるとした。たとえば、ソクラテスは、人間の分離したイ
 デアにしたがって人間と言われるのである。彼はまた、ちょうど人間や馬の分離
 したイデアを措定して、それらを「人間それ自体」「馬それ自体」と呼んだよう
 に、有や一の分離したイデアも措定して、「有それ自体」「一それ自体」と呼ん
 だ。そしてそれらを分有することによって、各々のものは、有や一と言われるの
 である。さらに彼は、善それ自体や、一それ自体であるものが、最高神だとし、
 その神によって、すべてのものは、分有というかたちで、善と言われるとした。

\\

 -- Et quamvis haec opinio irrationabilis videatur
 quantum ad hoc, quod ponebat species rerum naturalium separatas per se
 subsistentes, ut Aristoteles multipliciter probat; tamen hoc absolute
 verum est, quod est aliquod unum per essentiam suam bonum,\footnote{quod
 以下、Leonina版では次の通り。``quod aliquid est primum, quod per suam
 essentiam est ens et bonum.''\\「何かが第一のものであり、それは、自らの本
 質によって有であり善である。」} quod
 dicimus Deum, ut ex superioribus patet. Huic etiam sententiae concordat
 Aristoteles. A primo igitur per suam essentiam ente et bono,
 unumquodque potest dici bonum et ens, inquantum participat ipsum per
 modum cuiusdam assimilationis, licet remote et deficienter, ut ex
 superioribus patet. Sic ergo unumquodque dicitur bonum bonitate divina,
 sicut primo principio exemplari, effectivo et finali totius
 bonitatis. Nihilominus tamen unumquodque dicitur bonum similitudine
 divinae bonitatis sibi inhaerente, quae est formaliter sua bonitas
 denominans ipsum. Et sic est bonitas una omnium; et etiam multae
 bonitates.  &


この意見は、アリストテレスが多くのしかたで証明しているように、自然的事物
 の形象が分離して自存するとした点で不合理なものに見えるが、しかし、な
 にか一つの自らの本質によって善であるものが存在す
 るとした点は、絶対に真である。上述のことから明らかなように、私たちは
 これを神と言う。アリストテレスも、この文には同意する。
ゆえに、自らの本質によって存在し善である第一のものによって、各々のものは、
 善いものや存在するものと言われうるが、それは、一種の類似化というかたち
 によって、その第一のものを分有するかぎりにおいてである。ただし、上述の
 ことから明らかなとおり、遠く隔たり不完全にではあるが。それゆえ、このよ
 うに、各々のものは、神の善性によって善と言われるが、それは、全善性の範
 型的かつ作出因的かつ目的因的な第一原理によって、そう言われる。しかしそ
 れでもやはり、各々のものは、自分の中にある神の善性への類似によって、善
 と言われる。それは、形相的に自らの善性であり、それを(「善」と)名付け
 るものである。このような意味で、万物が所有する一つの善性が存在し、さら
 に、多くの善性がある。

\\


Et per hoc patet responsio ad obiecta.
&
これによって、異論に対する解答は明らかである。


\end{longtable}

\end{document}