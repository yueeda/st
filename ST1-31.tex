\documentclass[10pt]{jsarticle} % use larger type; default would be 10pt
%\usepackage[utf8]{inputenc} % set input encoding (not needed with XeLaTeX)
%\usepackage[round,comma,authoryear]{natbib}
%\usepackage{nruby}
\usepackage{okumacro}
\usepackage{longtable}
%\usepqckage{tablefootnote}
\usepackage[polutonikogreek,english,japanese]{babel}
%\usepackage{amsmath}
\usepackage{latexsym}
\usepackage{color}
%\usepackage{tikz}

%----- header -------
\usepackage{fancyhdr}
\pagestyle{fancy}
\lhead{{\it Summa Theologiae} I, q.~31}
%--------------------


\title{{\bf PRIMA PARS}\\{\HUGE Summae Theologiae}\\Sancti Thomae
Aquinatis\\{\sffamily QUEAESTIO TRIGESTIMAPRIMA}\\DE HIS QUAE AD
UNITATEM VEL PLURALITATEM PERTINENT IN DIVINIS}
\author{Japanese translation\\by Yoshinori {\sc Ueeda}}
\date{Last modified \today}

%%%% コピペ用
%\rhead{a.~}
%\begin{center}
% {\Large {\bf }}\\
% {\large }\\
% {\footnotesize }\\
% {\Large \\}
%\end{center}
%
%\begin{longtable}{p{21em}p{21em}}
%
%&
%
%\\
%\end{longtable}
%\newpage



\begin{document}

\maketitle

\begin{center}
{\Large 第三十一問\\神において一性や複数性に属する事柄について}
\end{center}


\begin{longtable}{p{21em}p{21em}}

{\Huge P}{\scshape ost} haec considerandum est de his quae ad unitatem
 vel pluralitatem pertinent in divinis. Et circa hoc quaeruntur quatuor.
quatuor. 

\begin{enumerate}
 \item de ipso nomine {\itshape trinitatis}.
 \item utrum possit dici, {\itshape Filius est alius a Patre}.
 \item utrum dictio exclusiva, quae videtur alietatem excludere,
       possit adiungi nomini essentiali in divinis.
 \item utrum possit adiungi termino personali.
\end{enumerate}


&



これらの後に、神において一性や複製に属する事柄が考察されるべきである。そ
 してこれをめぐって四つのことが問われる。

\begin{enumerate}
 \item 「三性」というこの名称について。
 \item 「息子は父と別人である」と言われうるか。
 \item 異他性を排除すると思われる排除的な言い回しは神において本質的名
       称に結び付けられうるか。
 \item それはペルソナ的な語に結び付けられうるか。
\end{enumerate}

\end{longtable}




\newpage

\rhead{a.~1}
\begin{center}
{\Large {\bf ARTICULUS PRIMUS}}\\
{\large UTRUM SIT TRINITAS IN DIVINIS}\\
{\footnotesize I {\itshape Sent.}, d.24, q.2, a.2.}\\
{\Large 第一項\\神において三性\footnote{trinitasはキリスト教神学で「三位一体」と訳
 されるが、「三・三つずつ」(trini)の抽象名詞で「三性」の意(本項第
 五異論とその訳者注を参照)。とくに本問題では「三
 位一体」と訳すと意味が取れない箇所が多いので辞書的に正しい「三性」と
 訳す。(既出)}があるか}
\end{center}

\begin{longtable}{p{21em}p{21em}}

{\scshape Ad primum sic proceditur}. Videtur quod non sit trinitas in
divinis. Omne enim nomen in divinis vel significat substantiam, vel
relationem. Sed hoc nomen trinitas non significat substantiam,
praedicaretur enim de singulis personis. Neque significat relationem,
quia non dicitur secundum nomen {\itshape ad aliud}. Ergo nomine {\itshape trinitatis} non
est utendum in divinis.


&

第一項の問題へ議論は以下のように進められる。
神において三性は存在しないと思われる。
理由は以下の通り。
神においてすべての名称は実体か関係を表示する。
しかるに「三性」というこの名称は実体を表示しない。なぜなら、もし実
 体を表示したら各々のペルソナに述語付けられただろうから\footnote{三性
が実体を表すなら、「父は三性である」「息子は三性である」
 「聖霊は三性である」と言えただろうが、実際にはそうは言えない。ゆ
 えに実体を表す名称ではない。}。また関係を表
 示するのでもない。なぜなら、名称において「他のものに対して」言われて
 いないからである。ゆえに「三性」というこの名称を神において用いる
 べきではない。


\\



2. {\scshape Praeterea}, hoc nomen {\itshape trinitas} videtur esse nomen collectivum, cum
significet multitudinem. Tale autem nomen non convenit in divinis, cum
unitas importata per nomen collectivum sit minima unitas, in divinis
autem est maxima unitas. Ergo hoc nomen trinitas non convenit in
divinis.


&

さらに「三性」というこの名称は多を表示するので集合名詞であるように
 思われる。しかるにそのような名称は神において適合しない。なぜなら集合
 名詞によって表示される一性は最少の一性だが、神においては最大の一性が
 あるからである。ゆえにこの「三性」という名称は神において適合しな
 い。

\\



3. {\itshape Praeterea}, omne trinum est triplex. Sed in Deo non est {\itshape triplicitas},
cum triplicitas sit species inaequalitatis. Ergo nec {\itshape trinitas}.


&

さらに、すべて三であるものは三重である。しかるに神において三重はない。
 なぜなら三重は不等性の種だからである。ゆえに三性もない。

\\



4. {\scshape Praeterea}, quidquid est in Deo, est in unitate essentiae divinae, quia
Deus est sua essentia. Si igitur trinitas est in Deo, erit in unitate
essentiae divinae. Et sic in Deo erunt tres essentiales unitates, quod
est haereticum.


&

さらに、神においてあるものは何であれ神の本質の一性においてある。なぜな
 ら神は自らの本質だから。ゆえに三性が神の中にあればそれは神の本質の一
 性の中にあることになる。こうして神の中に三つの本質的な一性があること
 になるが、これは異端である。


\\



5. {\scshape Praeterea}, in omnibus quae dicuntur de Deo, concretum praedicatur de
abstracto, deitas enim est Deus, et paternitas est pater. Sed trinitas
non potest dici {\itshape trina}, quia sic essent novem res in divinis, quod est
erroneum. Ergo nomine {\itshape trinitatis} non est utendum in divinis.


&

さらに、神について語られるすべてのものにおいて、具体的なものが抽象的な
 ものについて述語される。たとえば神性は神であり父性は父である。しかる
 に三性は「三つずつ」であるとは言われえない。なぜなら、そうなると神の中に九
 つのものがあっただろうがこれは誤りである。ゆえに三性という名称を神に
 おいて用いるべきではない。\footnote{この異論はtrinitasがtriniの抽象名
 詞だということを前提としている。よくtrinitasはtres +
 unitasだという解説を目にするがそれならtres unitatesとなろう。また、本項第一異論解答に見られるようにトマ
 スは語源的にはtrium + unitas(三つのものに属する一性)だという説を紹
 介しているが、それがトマスの時代の共通了解ではないことがこの異論から
 わかる。現代の辞書でもtrinitasはtriniの抽象名詞と説明されている。}

\\



{\scshape Sed contra est} quod Athanasius dicit, quod {\itshape unitas in trinitate, et
trinitas in unitate veneranda sit}.


&

しかし反対にアウグスティヌスは「三性において一性が、一性において三性が
 崇められるべきである」と述べている。

\\



{\scshape Respondeo dicendum} quod nomen {\itshape trinitatis} in divinis significat
determinatum numerum personarum. Sicut igitur ponitur pluralitas
personarum in divinis, ita utendum est nomine trinitatis, quia hoc
idem quod significat pluralitas indeterminate, significat hoc nomen
trinitas determinate.


&

解答する。以下のように言われるべきである。
「三性」という名称は神においてペルソナの限定された数を表示する。ゆえに
 神においてペルソナの複数性が措定されるように「三性」という名称が用い
 られるべきである。なぜなら「複数性」という名称が非限定的に表示する同
 じものを「三性」というこの名称が限定的に表示するからである。

\\



{\scshape Ad primum ergo dicendum} quod hoc nomen {\itshape trinitas}, secundum etymologiam
vocabuli, videtur significare unam essentiam trium personarum,
secundum quod dicitur trinitas quasi {\itshape trium unitas}. Sed secundum
proprietatem vocabuli, significat magis numerum personarum unius
essentiae. Et propter hoc non possumus dicere quod pater sit trinitas,
quia non est tres personae. Non autem significat ipsas relationes
personarum, sed magis numerum personarum ad invicem relatarum. Et inde
est quod, secundum nomen, ad aliud non refertur.


&

第一異論に対してはそれゆえ以下のように言われるべきである。
「三性」というこの名称は、言葉の語源によれば、三性(trinitas)が「三つの
 ものの一性(trium unitas)」のように言われることにおいて、三つのペルソナの一つの本質
 を表示しているように思われる。しかし言葉の固有性にしたがえば、むしろ
 それは一つの本質に属するペルソナの数を表示している。このため私たちは
 父は三性であると言うことはできない。なぜなら(父は)三つのペルソナでないから
 である。さらにペルソナの関係自体を表示するのでもない。むしろ相互に関
 係するペルソナの数を表示する。それゆえ、名称としては、他の何かへと関
 係づけられない。


\\



{\scshape Ad secundum dicendum} quod nomen collectivum duo importat, scilicet
pluralitatem suppositorum, et unitatem quandam, scilicet ordinis
alicuius, {\itshape populus} enim est multitudo hominum sub aliquo ordine
comprehensorum. Quantum ergo ad primum, hoc nomen {\itshape trinitas} convenit
cum nominibus collectivis, sed quantum ad secundum differt, quia in
divina trinitate non solum est unitas ordinis, sed cum hoc est etiam
unitas essentiae.


&


第二異論に対しては以下のように言われるべきである。
集合名詞は二つのことを意味している。すなわち個体の複数性と一種の一性、
 つまり何らかの秩序の一性を意味している。たとえば「人々」は何らかの秩
 序のもとに含まれた人々の多を表示する。それゆえ、第一のことに関して、
 この「三性」という名称は集合名詞に一致するが、第二の点にかんしては異
 なる。なぜなら神の三性においては秩序の一性だけでなくそれとともに本質
 の一性があるからである。


\\



{\scshape Ad tertium dicendum} quod {\itshape trinitas} absolute dicitur, significat enim
numerum ternarium personarum. Sed triplicitas significat proportionem
inaequalitatis, est enim species proportionis inaequalis, sicut patet
per Boetium in {\itshape Arithmetica}. Et ideo non est in Deo triplicitas, sed
trinitas.


&

第三異論に対しては以下のように言われるべきである。
「三性」は非関係的に語られ、実際、三つのペルソナの数を表示する。
しかし「三重性」は不等性の比を表示し、ボエティウスの『算術学』で明らか
 なとおり不等な比の種である。ゆえに神において三重性はないが三性はある。
 \footnote{ボエティウスの『算術学』に基づく議論で詳細は不明。}

\\



{\scshape Ad quartum dicendum} quod in trinitate divina intelligitur et numerus,
et personae numeratae. Cum ergo dicimus {\itshape trinitatem in unitate}, non
ponimus numerum in unitate essentiae, quasi sit ter una, sed personas
numeratas ponimus in unitate naturae, sicut supposita alicuius naturae
dicuntur esse in natura illa. E converso autem dicimus unitatem in
trinitate, sicut natura dicitur esse in suis suppositis.


&

第四異論に対しては以下のように言われるべきである。
神の三性においては数と数えられたペルソナとが理解される。ゆえに「一性に
 おいて三性が」と私たちが言うときには、三つ目の一、というように本質の一
 性において数を措定しているのではなく、ある本性を持つ個体がその本性に
 おいてあると言われるように、一つの本性において数えられたペルソナを措
 定している。また逆に、本性がその個体においてあると言われるようなしか
 たで一性が三性においてあると私たちは言う。


\\



{\scshape Ad quintum dicendum} quod, cum dicitur, {\itshape trinitas est trina}, ratione
numeri importati significatur multiplicatio eiusdem numeri in seipsum,
cum hoc quod dico trinum, importet distinctionem in suppositis illius
de quo dicitur. Et ideo non potest dici quod trinitas sit trina, quia
sequeretur, si trinitas esset trina, quod tria essent supposita
trinitatis; sicut cum dicitur, {\itshape Deus est trinus}, sequitur quod sunt
tria supposita deitatis.


&


第五異論に対しては以下のように言われるべきである。「三性が三つずつであ
 る」だと言われるとき、意味されている数の概念によって、同じ数のそれ自身への乗
 算が意味される。私が「三つずつ」と言うこの言葉は、それについて語られてい
 るものの個体における区別を含意するからである。ゆえに三性が三つずつと
 は言われえない。なぜなら、もし三性が三つずつだったとすれば三性の個体
 が三つだったであろうから。ちょうど「神は三つずつである」と言われると
 き、神性の三つの個体が帰結するように。



\end{longtable}
\newpage


\rhead{a.~2}
\begin{center}
{\Large {\bf ARTICULUS SECUNDUS}}\\
{\large UTRUM FILIUS SIT ALIUS A PATRE}\\
{\footnotesize I {\itshape Sent.}, d.9, q.1, a.1; d.19, q.1, a.1, ad
 2; d.24, q.2, a.1; {\itshape De Pot.}, q.9, a.8.}\\
{\Large 第二項\\息子は父と別人\footnote{aliusは男性単数形なので通常は
 人を表す。この場合はもちろん神のペルソナのことなので、「人」はペルソ
 ナのことだと理解する必要がある}か}
\end{center}

\begin{longtable}{p{21em}p{21em}}

{\scshape Ad secundum sic proceditur}. Videtur quod filius non sit alius a
patre. {\itshape Alius} enim est relativum diversitatis substantiae. Si igitur
filius est alius a patre, videtur quod sit a patre {\itshape diversus}. Quod est
contra Augustinum, VII {\itshape de Trin}., ubi dicit quod, cum dicimus tres
personas, {\itshape non diversitatem intelligere volumus}.


&

第二項の問題へ議論は以下のように進められる。
息子は父と別人でないと思われる。理由は以下の通り。
「別人」は実体の異なりに関係する。ゆえにもし息子が父と別人で
 あるならば、父と「異なる人(diversus)」であることになる。これはアウグスティヌスが
 『三位一体論』第7巻で、私たちが三つのペルソナと言うとき「異なり(diversitas)を理解
 したいわけではない」と言うことに反する。


\\

2. {\scshape Praeterea}, quicumque sunt alii ab invicem, aliquo modo ab invicem
differunt. Si igitur filius est alius a patre, sequitur quod sit
{\itshape differens} a patre. Quod est contra Ambrosium, in I {\itshape de Fide}, ubi ait,
{\itshape pater et filius deitate unum sunt, nec est ibi substantiae
differentia, neque ulla diversitas}.

&

さらに、お互いに別人である人はだれでも、何らかのかたちで相互に違っている(differre)。
 ゆえにもし息子が父と別人であるならば、父と違っている。これは『信仰につ
 いて』第1巻のアンブロシウスに反する。彼はそこで次のように言っている。
 「父と息子が神性において一であり、そこに実体の違い(differentia)もどんな異なりもな
 い」。

\\



3. {\itshape Praeterea}, ab alio {\itshape alienum} dicitur. Sed filius non est alienus a
patre, dicit enim Hilarius, in VII de Trin., quod in divinis personis
nihil est diversum, nihil alienum, nihil separabile. Ergo filius non
est alius a patre.


&

さらに、(お互いに別であるものは何でも)他のものと「別様(alienum)」である。しか
 るに息子は父と別様ではない。なぜならヒラリウスは『三位一体論』第7巻で
 「神のペルソナにおいて異なったり別様であったり分離可能であるものはなに
 もない」と述べているからである。ゆえに息子は父と別ではない。

\\



4. {\scshape Praeterea}, {\itshape alius} et {\itshape aliud} idem significant, sed sola generis
consignificatione differunt. Si ergo filius est alius a patre, videtur
sequi quod filius sit aliud a patre.


&

さらに、別人(alius)と別物(aliud)は同じことを意味するがその意味に含む類の点で相違する。
 ゆえにもし息子が父と別人であるならば、息子は父と別物であることになる。

\\



{\scshape Sed contra est} quod Augustinus dicit, in libro {\itshape de
 Fide ad Petrum : Una
est enim essentia patris et filii et spiritus sancti, in qua non est
aliud pater, aliud filius, aliud spiritus sanctus; quamvis
personaliter sit alius pater, alius filius, alius spiritus sanctus}.


&

しかし反対にアウグスティヌスは『ペトルスへ信仰について』という書物で次
 のように述べている。「じっさい父と息子と聖霊の本質は一つであり、そこ
 において父と息子と聖霊は別人ではない。ただしペルソナ的に父と息子と
 聖霊は別人であるが」。

\\



{\scshape Respondeo dicendum} quod, quia ex verbis inordinate prolatis incurritur
haeresis, ut Hieronymus dicit, ideo cum de Trinitate loquimur, cum
cautela et modestia est agendum, quia, ut Augustinus dicit, in I {\itshape de
Trin., nec periculosius alicubi erratur, nec laboriosius aliquid
quaeritur, nec fructuosius aliquid invenitur}. 

&

解答する。以下のように言われるべきである。
ヒエロニュムスが言うように、秩序立てずに発せられた言葉から異端は生じる
ので、三性について私たちが語るときには注意して慎み深く論じられるべきで
 ある。というのもアウグスティヌスが『三位一体論』第1巻で言うように「こ
 れ以上誤る危険があり、これ以上探究に骨が折れ、これ以上豊かな発見があ
 るところはない」からである。

\\



Oportet autem in his
quae de trinitate loquimur, duos errores oppositos cavere, temperate
inter utrumque procedentes, scilicet errorem Arii, qui posuit cum
trinitate personarum trinitatem substantiarum; et errorem Sabellii,
qui posuit cum unitate essentiae unitatem personae. 

&

ところで、私たちが三性について語る事柄において、二つの対立する誤りを両
 者の間を適切に進みながら避けなければならない。それはすなわちアリウス
 の誤りとサベリウスの誤りである。前者はペルソナの三性とともに実体の三
 性を主張し、後者は本質の一性とともにペルソナの一性を主張した。

\\


Ad evitandum
igitur errorem Arii, vitare debemus in divinis nomen {\itshape diversitatis} et
{\itshape differentiae}, ne tollatur unitas essentiae, possumus autem uti nomine
{\itshape distinctionis}, propter oppositionem relativam. Unde sicubi in aliqua
Scriptura authentica diversitas vel differentia personarum invenitur,
sumitur diversitas vel differentia pro distinctione. 

&

それゆえ、アリウスの誤謬を避けるために、本質の一性を損なわないように、神において「異なり」(diversitas)
や「違い」(differentia)という名称を避けなければならないが、しかし他方
 で、私たちは区別(distinctio)という名称を関係的な対立のために用いるこ
 とができる。このことから何らかの権威ある書物のどこかにペルソナの異なりや違いと
 いうことが見出されるならば、それは異なりや違いが区別として解釈される。

\\


Ne autem tollatur
simplicitas divinae essentiae, vitandum est nomen {\itshape separationis} et
{\itshape divisionis}, quae est totius in partes. Ne autem tollatur aequalitas,
vitandum est nomen {\itshape disparitatis}. Ne vero tollatur similitudo, vitandum
est nomen alieni et {\itshape discrepantis}, dicit enim Ambrosius, in libro {\itshape de
Fide}, quod in patre et filio {\itshape non est discrepans, sed una divinitas}, et
secundum Hilarium, ut dictum est, in divinis {\itshape nihil est alienum, nihil
separabile}. 


&

他方、神の本質の単純性が損なわれないように、「分離」(separatio)や、全
 体の部分への「分割」(divisio)という名称が避けられるべきである。さらに、
 等しさが否定されないように「不等性」(disparitas)という名称が避けられ
 るべきである。他方、類似が否定されないように「別様」(alienus)や「不協」
 (discrepans)という名称もまた避けられるべきである。たとえばアンブロシ
 ウスは『信仰について』という書物で、父と子において「不協はなく一つの
 神性が」あると述べている。またヒラリウスによれば、すでに述べられたと
 おり、神において「別様であったり分離可能なものはなにもない」。

\\



Ad vitandum vero errorem Sabellii, vitare debemus
singularitatem, ne tollatur communicabilitas essentiae divinae, unde
Hilarius dicit, VII {\itshape de Trin. : Patrem et filium singularem Deum
praedicare, sacrilegum est}. Debemus etiam vitare nomen {\itshape unici}, ne
tollatur numerus personarum, unde Hilarius in eodem libro dicit quod a
Deo excluditur {\itshape singularis atque unici intelligentia}. 


&

これに対して、サベリウスの誤謬を避けるために、私たちは、神の本質の共有
 可能性が否定されないように「単一性」(singularitas)を避けなければなら
 ない。このことからヒラリウスは『三位一体論』第7巻で「父と息子が単一の
 神である」と公言することは冒涜であると述べている。さらに私たちはペル
 ソナの数が否定されないように「唯一のもの」(unicus)という名称も避ける
 べきである。それゆえヒラリウスは同書で神から「単一で唯一の知性活動」
 が排除されると述べる。

\\


Dicimus tamen
{\itshape unicum filium}, quia non sunt plures filii in divinis. Neque tamen
dicimus {\itshape unicum Deum}, quia pluribus deitas est communis vitamus etiam
nomen {\itshape confusi}, ne tollatur ordo naturae a personis, unde Ambrosius
dicit, I {\itshape de Fide} : {\itshape Neque confusum est quod unum est, neque multiplex
esse potest quod indifferens est}. Vitandum est etiam nomen {\itshape solitarii},
ne tollatur consortium trium personarum, dicit enim Hilarius, in IV {\itshape de
Trin}. : {\itshape Nobis neque solitarius, neque diversus Deus est
confitendus}. 

&

しかし私たちは「たった一人の息子」と言うが、それは神において複数の息子
 はいないからである。また「たった一人の神」と言うがそれは複数のものに
 神性が共通するからである。さらにペルソナによって本性の秩序が否定され
 ないように「混乱したもの」(confusi)という名称を私たちは避ける。このこ
 とからアンブロシウスは『信仰について』第1巻で「一つであるものは混乱し
 たものでなく、差別なくあるものは多様でもありえない」と述べる。さらに
 三つのペルソナの交わりが否定されないように「孤独な」という名称を避け
 る。じっさいヒラリウスは『三位一体論』第4巻で「私たちは、孤独でも異な
 ることもない神を告白すべきである」と述べている。

\\



Hoc autem nomen alius, masculine sumptum, non importat
nisi distinctionem suppositi. Unde convenienter dicere possumus quod
filius est alius a patre, quia scilicet est aliud suppositum divinae
naturae, sicut est alia persona, et alia hypostasis.


&

他方、男性名詞として取られた「別人」というこの名称は個体の区別のみを意
 味する。したがって、「息子は父と別人である」と適切に言うことができる。
 なぜなら、別のペルソナ、別のヒュポスタシスであるように、神の本性の別
 の個体だからである。

\\



{\scshape Ad primum ergo dicendum} quod {\itshape alius}, quia est sicut quoddam particulare
nomen, tenet se ex parte suppositi, unde ad eius rationem sufficit
distinctio substantiae quae est hypostasis vel persona. Sed {\itshape diversitas}
requirit distinctionem substantiae quae est essentia. Et ideo non
possumus dicere quod filius sit diversus a patre, licet sit alius.


&

第一異論に対してはそれゆえ次のように言われるべきである。
「他人」は一種の個別的な名称として個体の側からある。したがってヒュポス
 タシスあるいはペrソナである実体の区別がその概念にとって十分である。
 しかし「異なり」は本質である実体の区別が必要である。ゆえに私たちは息
 子が父の「他人」だと言えるが、父と「異なっている」と言うことはできない。

\\



{\scshape Ad secundum dicendum} quod {\itshape differentia} importat distinctionem
formae. Est autem tantum una forma in divinis, ut patet per id quod
dicitur {\itshape Philip}.~{\scshape ii} : {\itshape Qui cum in forma Dei esset}. Et ideo nomen
{\itshape differentis} non proprie competit in divinis, ut patet per auctoritatem
inductam. Utitur tamen Damascenus nomine {\itshape differentiae} in divinis
personis, secundum quod proprietas relativa significatur per modum
formae, unde dicit quod non differunt ab invicem hypostases secundum
substantiam, sed secundum determinatas proprietates. Sed {\itshape differentia}
sumitur pro {\itshape distinctione}, ut dictum est.


&

第二異論に対しては以下のように言われるべきである。
「違い」は形相の区別を意味する。しかるに、『フィリピの信徒への手紙』第
 2章「彼は神の形においてあったので」\footnote{「キリストは/神の形であ
 りながら/神と等しくあることに固執しようとは思わず 」(2:6)}によれば神
 においてはただ一つの形相がある。ゆえに「違い」という名称は神において
 厳密には適合しない。これは引用された権威によって明らかなとおりである。
しかしダマスケヌスは「違い」を神のペルソナにおいて用いているが、それは
 関係的な固有性が形相の形で表示されるからである。したがって、ヒュポス
 タシスは実体において相互に違わず、限定された固有性において違うと述べ
 ている。しかしすでに述べられたとおり、「違い」は「区別」の意味で理解
 されている。


\\



{\scshape Ad tertium dicendum} quod {\itshape alienum} est quod est extraneum et
dissimile. Sed hoc non importatur cum dicitur {\itshape alius}. Et ideo dicimus
filium alium a patre, licet non dicamus alienum.


&

第三異論に対しては以下のように言われるべきである。
「別様のもの」は外的で類似していないもののことである。しかしこれは「他
 人」と言われるときには含意されていない。ゆえに私たちは息子は父と別人であると
 言うが別様であるとは言わない。


\\



{\scshape Ad quartum dicendum} quod neutrum genus est informe, masculinum autem
est formatum et distinctum, et similiter femininum. Et ideo
convenienter per neutrum genus significatur essentia communis, per
masculinum autem et femininum, aliquod suppositum determinatum in
communi natura. Unde etiam in rebus humanis, si quaeratur, {\itshape quis est
iste?} Respondetur, {\itshape Socrates}, quod nomen est suppositi, si autem
quaeratur, {\itshape quid est iste?} Respondetur, {\itshape animal rationale et mortale}. Et
ideo, quia in divinis distinctio est secundum personas, non autem
secundum essentiam, dicimus quod pater est {\itshape alius} a filio, sed non
{\itshape aliud}, et e converso dicimus quod sunt {\itshape unum}, sed non {\itshape unus}.


&

第四異論に対しては以下のように言われるべきである。
中性の類は形相をもたないが男性は形相をもち区別される。女性もまたそうであ
 る。ゆえに適切に中性によって共通の本質が表示され、これに対して男性と
 女性によって共通の本性において限定された何らかの個体を表示する。した
 がって人間的な事物においても「この人は誰か」と問われると「ソクラテス」
 と答えられるが、この名称は個体に属する。しかし「この人は何か」と問わ
 れると「理性的で死すべき動物」と答えられる。ゆえに、神において区別は
 ペルソナにおいてあり、本質においてはないので、父は息子と「別人」であ
 ると言うが「別物」とは言わない。逆に私たちは(父と息子は)「一つ」で
 あると言うが「一人」であるとは言わない。



\end{longtable}
\newpage




\rhead{a.~3}
\begin{center}
{\Large {\bf ARTICULUS TERTIUS}}\\
{\large UTRUM DICTIO EXCLUSIVA {\itshape SOLUS} SIT APPENDA TERMINO
 ESSENTIALI IN DIVINIS}\\
{\footnotesize I {\itshape Sent.}, d.19, q.1, a.1.}\\
{\Large 第三項\\神において「独り」という排他的な語が本質的な用語に付加
 されるべきか}
\end{center}

\begin{longtable}{p{21em}p{21em}}

{\scshape Ad tertium sic proceditur}. Videtur quod dictio exclusiva {\itshape solus} non sit
 addenda termino essentiali in divinis. Quia secundum philosophum, in
 II {\itshape Elench}., solus est {\itshape qui cum alio non est}. Sed Deus est cum Angelis
 et sanctis animabus. Ergo non possumus dicere {\itshape Deum solum}.

&

第三項の問題へ議論は以下のように進められる。
「独り」という排他的な語が神において本質的な用語に付加されるべきでない
 と思われる。理由は以下の通り。
『詭弁論駁論』第2巻のアリストテレスによれば「独りの人」とは「他人といない
 人」のことである。しかし神は天使や聖なる魂たちとともにいる。ゆえに私
 たちは「独りの神」と言うことはできない。


\\




2. {\scshape Praeterea}, quidquid adiungitur termino essentiali in divinis, potest
 praedicari de qualibet persona per se, et de omnibus simul, quia enim
 convenienter dicitur {\itshape sapiens Deus}, possumus dicere, {\itshape pater est sapiens
 Deus}, et {\itshape trinitas est sapiens Deus}. Sed Augustinus, in VI {\itshape de Trin}.,
 dicit, {\itshape consideranda est illa sententia, qua dicitur non esse patrem
 verum Deum solum}. Ergo non potest dici solus Deus.

&

さらに、神において本質的な用語に結び付けられるものはなんであれ、どのペ
 ルソナについても自体的に述語付けられるし、すべてのペルソナについて同
 時に述語付けられる。たとえば「知恵ある神」と適切に言われるので、私た
 ちは「父は知恵ある神である」と言うことができ、また「三性は知恵ある神
 である」と言える。しかしアウグスティヌスは『三位一体論』第6巻で「父独
 りが真の神なのではないと言われるかの文が考察されるべきである」と述べ
 ている。ゆえに「独りの神」と言われることはできない。


\\




3. {\scshape Praeterea}, si haec dictio solus adiungitur termino essentiali, aut hoc
 erit respectu praedicati personalis, aut respectu praedicati
 essentialis. Sed non respectu praedicati personalis, quia haec est
 falsa, {\itshape solus Deus est pater}, cum etiam homo sit pater. Neque etiam
 respectu praedicati essentialis. Quia si haec esset vera, {\itshape solus Deus
 creat}, videtur sequi quod haec esset vera, {\itshape solus pater creat}, quia
 quidquid dicitur de Deo, potest dici de patre. Haec autem est falsa,
 quia etiam filius est creator. Non ergo haec dictio {\itshape solus} potest in
 divinis adiungi termino essentiali.

&


さらに、もしこの「独り」という語が本質的な用語に結び付けられるならば、
 ペルソナ的な述語にかんしてか本質的な述語にかんしてかのどちらかである。
 しかしペルソナ的な述語にかんしてではない。なぜなら「独り神のみが父であ
 る」は偽だからである。人間も父なのだから。また本質的な述語にかんして
 でもない。なぜなら、もし「独り神のみが創造する」が真であったならば
 「独り父のみが創造する」が真であっただろう。なぜなら神について語ら
 れることは父についても語られうるからである。しかしこれは偽である。な
 ぜなら息子もまた創造者だから。ゆえに「独り」というこの語は神において
 本質的な用語に結び付けられえない。



\\




{\scshape Sed contra est} quod dicitur I {\itshape ad Tim}.~{\scshape
 i} : {\itshape Regi saeculorum immortali,
 invisibili, soli Deo}.

&

しかし反対に、『テモテへの手紙一』第1章で「不死であり目に見えず独りの
 神である世々の王に」\footnote{「永遠の王、不滅で目に見えない唯一の神
 に、誉れと栄光が世々限りなくありますように、アーメン」(1:17)。}と言わ
 れている。

\\




{\scshape Respondeo dicendum} quod haec dictio {\itshape solus} potest accipi ut
 categorematica vel syncategorematica. Dicitur autem dictio
 {\itshape categorematica}, quae absolute ponit rem significatam circa aliquod
 suppositum; ut {\itshape albus} circa hominem, cum dicitur {\itshape homo albus}. 


&

解答する。以下のように言われるべきである。
「独り」というこの語は自立語としてあるいは共義語として理解できる。とこ
 ろである語が自立語と言われるのは、その語が非関係的に表示された事物をある個体
 にかんして措定する場合である。たとえば「白い人間」と言われるとき、
 「白い」が人間にかんして措定されている。

\\


Si ergo
 sic accipiatur haec dictio {\itshape solus}, nullo modo potest adiungi alicui
 termino in divinis, quia poneret solitudinem circa terminum cui
 adiungeretur, et sic sequeretur Deum esse solitarium; quod est contra
 praedicta. 

&

ゆえに、「独り」がこの意味で理解されるならば、それは神におけるどんな用
 語にも結び付けられない。なぜなら、それに結び付けられた用語にかんして
 孤独さを措定することになり、神が孤独だということになるがこれは上述の
 ことに反するからである。

\\

Dictio vero {\itshape syncategorematica} dicitur, quae importat
 ordinem praedicati ad subiectum, sicut haec dictio {\itshape omnis}, vel
 {\itshape nullus}. 
Et similiter haec dictio {\itshape solus}, quia excludit omne aliud
 suppositum a consortio praedicati. Sicut, cum dicitur, {\itshape solus Socrates
 scribit}, non datur intelligi quod Socrates sit solitarius; sed quod
 nullus sit ei consors in scribendo, quamvis cum eo multis
 existentibus. 


&

他方、ある語が共義語と言われるのは、それが述語の主語への秩序を含意する
 場合である。たとえば「すべての」や「どの〜もない」のように。
同様に、この「独り」という語も(共義語である)、なぜなら述語の交わりか
 らすべての個体を排除するからである。たとえば「独りソクラテスだけが書
 く」と言われるとき、ソクラテスが孤独だということが理解されることは許
 されず、彼と共に多くの人がいるけれども、書くことにおいて彼と一緒であ
 る人はいないと理解されるからである。

\\


Et per hunc modum nihil prohibet hanc dictionem solus
 adiungere alicui essentiali termino in divinis, inquantum excluduntur
 omnia alia a Deo a consortio praedicati, ut si dicamus, {\itshape solus Deus
 est aeternus}, quia nihil praeter Deum est aeternum.

&

この意味であれば、「独り」というこの語が、述語の交わりから神以外のすべ
 てのものが排除される限りにおいて、神における何らかの本質的な用語
 に結び付けられることを妨げるものはない。たとえば「独り神のみが永遠で
 ある」と私たちが言うように。これは、神以外の何も永遠でないからである。


\\




{\scshape Ad primum ergo dicendum} quod, licet Angeli et animae sanctae semper
 sint cum Deo, tamen, si non esset pluralitas personarum in divinis,
 sequeretur, quod Deus esset solus vel solitarius. Non enim tollitur
 solitudo per associationem alicuius quod est extraneae naturae,
 dicitur enim aliquis solus esse in horto, quamvis sint ibi multae
 plantae et animalia. Et similiter diceretur Deus esse solus vel
 solitarius, Angelis et hominibus cum eo existentibus, si non essent
 in divinis personae plures. Consociatio igitur Angelorum et animarum
 non excludit solitudinem absolutam a divinis, et multo minus
 solitudinem respectivam, per comparationem ad aliquod praedicatum.

&

第一異論に対してはそれゆえ以下のように言われるべきである。
天使と聖なる魂が常に神と共にいるとしても、もし神の中にペルソナの複数性
 がなかったならば、神が独りであることや孤独であることが帰結したであろ
 う。なぜなら、何か外的な本性との交わりによって孤独は除去されないから
 である。たとえばそこに多くの植物や生物がいるとしても人は庭で孤独であ
 ると言われる。同様に、もし神の中に複数のペルソナがいなかったならば、
 天使や人間が神と共にいたとしても神は独りまたは孤独だと言われただろう。
 ゆえに天使や魂との交わりは神における非関係的なな孤独を排除せず、ましてや、
 前述の何かへの関係による関係的な孤独を排除しない。


\\




{\scshape Ad secundum dicendum} quod haec dictio {\itshape solus}, proprie loquendo, non
 ponitur ex parte praedicati, quod sumitur formaliter, respicit enim
 suppositum, inquantum excludit aliud suppositum ab eo cui
 adiungitur. Sed hoc adverbium {\itshape tantum}, cum sit exclusivum, potest poni
 ex parte subiecti, et ex parte praedicati, possumus enim dicere,
 {\itshape tantum Socrates currit}, idest {\itshape nullus alius}; et, {\itshape Socrates currit
 tantum}, idest {\itshape nihil aliud facit}. Unde non proprie dici potest, {\itshape pater
 est solus Deus}, vel, {\itshape trinitas est solus Deus}, nisi forte ex parte
 praedicati intelligatur aliqua implicatio, ut dicatur, {\itshape trinitas est
 Deus qui est solus Deus}. Et secundum hoc etiam posset esse vera ista,
 {\itshape pater est Deus qui est solus Deus}, si relativum referret praedicatum,
 et non suppositum. Augustinus autem, cum dicit patrem non esse solum
 Deum, sed trinitatem esse solum Deum, loquitur expositive, ac si
 diceret, cum dicitur, {\itshape regi saeculorum, invisibili, soli Deo}, non est
 exponendum de persona patris, sed de sola trinitate.

&

第二異論に対しては以下のように言われるべきである。
厳密に言うならば「独り」というこの語は、形相的に取られる述語の側から述べ
られることはない。なぜならそれは他の個体を、「独り」が修飾する個体か
ら排除する限りにおいて、個体に関係するからである。これに対して「だけ」
 (tantum)というこの副詞は排除語なので主語の側にも述語の側にもかかりう
 る。たとえば私たちは「ソクラテスだけが走る」つまり「他のだれも(走ら
 ない)」と言い、「ソクラテスは走るだけである」つまり「他のことはしな
 い」と言うことができる。したがって、「父は神独りである」「三性は神独
 りである」とは厳密には言えない。ただし、述語の側から何らかの含意が理
 解されるのでないかぎり。たとえば「三性は独りの神であるところの神であ
 る」のように。この意味であれば、この「父は一人の神であるところの神で
 ある」もまた真でありえただろう。もし関係的なものが主語ではなく述語に
 言及しているならば。さらにアウグスティヌスが、父が独りの神でなく、三
 性が独りの神であると言うとき、説明して、それは以下のような意味だと述
 べる。すなわち、「見られえず独りの神である世々の王に」と言われる場合
 のようである。これは父のペルソナではなく、独りの三性について説明されるべき
 である。

\\




{\scshape Ad tertium dicendum} quod utroque modo potest haec dictio {\itshape solus} adiungi
 termino essentiali. Haec enim propositio, {\itshape solus Deus est pater}, est
 duplex. Quia ly {\itshape pater} potest praedicare personam patris, et sic est
 vera, non enim homo est illa persona. Vel potest praedicare
 relationem tantum, et sic est falsa, quia relatio paternitatis etiam
 in aliis invenitur, licet non univoce. Similiter haec est vera, solus
 {\itshape Deus creat}. Nec tamen sequitur, {\itshape ergo solus pater}, quia, ut sophistae
 dicunt, dictio exclusiva immobilitat terminum cui adiungitur, ut non
 possit fieri sub eo descensus pro aliquo suppositorum; non enim
 sequitur, {\itshape solus homo est animal rationale mortale, ergo solus
 Socrates}.

&

第三異論に対しては以下のように言われるべきである。この「独り」という語
 はどちらのしかたでも本質的な用語に結び付けられうる。理由は以下のとお
 り。この「神独りが父である」という命題は二通りである。というのもこの
 「父」は父のペルソナを示しうるがその場合には真である。なぜなら人間は
 かのペルソナでないから。あるいはただ関係だけを示しうるが、その場合に
 は偽である。父という関係は、一義的でないにしても他のものにおいても見
 出されるからである。同様にこの「神独りが創造する」も真である。しかし
 「父独りが(創造する)」ことは帰結しない。なぜならソフィストたちが言
 うように、排除語は、それから何らかの個体への下降が生じないように、そ
 れが修飾する用語を固定するからである。たとえば「独り人間が理性的で死
 すべき動物である、ゆえに独りソクラテスが云々」とはならないからである。





\end{longtable}
\newpage



\rhead{a.~4}
\begin{center}
{\Large {\bf ARTICULUS QUARTUS}}\\
{\large UTRUM DICTIO EXCLUSIVA POSSIT ADIUNGI TERMINO PERSONALI}\\
{\footnotesize I {\itshape Sent.}, d.21, q.1, a.2; in {\itshape
 Matth.}, cap.11.}\\
{\Large 第四項\\排除語はペルソナ的用語を修飾しうるか}
\end{center}

\begin{longtable}{p{21em}p{21em}}

{\scshape Ad quartum sic proceditur}. Videtur quod dictio exclusiva possit
adiungi termino personali, etiam si praedicatum sit commune. Dicit
enim dominus, ad patrem loquens, Ioan. XVII, ut cognoscant te, solum
Deum verum. Ergo solus pater est Deus verus.

&

第四項の問題へ議論は以下のように進められる。
排除語は、たとえ述語が共通であるとしても、ペルソナ的な用語を修飾できな
 いと思われる。理由は以下の通り。
主は父に語りかけて「彼らが独り真の神であるあなたを知るように」(『ヨハ
 ネによる福音書』第17章)\footnote{「永遠の命とは、唯一のまことの神で
 あられるあなたと、あなたのお遣わしになったイエス・キリストを知ること
 です。」(17:3)}と言っている。ゆえに独り父のみが真の神である。

\\



2. {\scshape Praeterea}, Matth. XI dicitur, {\itshape nemo novit filium nisi pater}; quod idem
significat ac si diceretur, {\itshape solus pater novit filium}. Sed nosse filium
est commune. Ergo idem quod prius.

&

さらに、『マタイによる福音書』11章で「父以外に息子を知る者はいない」\footnote{「すべてのことは、父から私に任せられています。父のほかに子を知る者はなく、子と、子が示そうと思う者のほかに、父を知る者はいません。」(11:27) }と
 言われているが、これは「独り父のみが息子を知る」と言われた場合と同じ
 ことを意味する。しかるに息子を知ることは共通のことである。ゆえに前の
 異論と同じ。

\\



3. {\scshape Praeterea}, dictio exclusiva non excludit illud quod est de intellectu
termini cui adiungitur, unde non excludit partem, neque universale,
non enim sequitur, {\itshape solus Socrates est albus, ergo manus eius non est
alba}; vel, {\itshape ergo homo non est albus}. Sed una persona est in intellectu
alterius, sicut pater in intellectu filii, et e converso. Non ergo per
hoc quod dicitur, {\itshape solus pater est Deus}, excluditur filius vel spiritus
sanctus. Et sic videtur haec locutio esse vera.

&

さらに、排除語は、それが修飾する語の理解に含まれるものを排除しない。し
 たがって、部分も不変も排除しない。たとえば「独りソクラテスが白い、ゆ
 えに彼の手は白くない」とか「ゆえに人間は白くない」とはならないように。
 しかるにあるペルソナは別のペルソナの理解に含まれる。たとえば父は息子
 の理解に含まれ、逆もまた然りである。ゆえに「独り父のみが神である」か
 ら息子や聖霊は排除されない。かくしてこの文は真であると思われる。


\\



4. {\scshape Praeterea}, ab Ecclesia cantatur, {\itshape tu solus altissimus, Iesu Christe}.

&

さらに、「イエス・キリストよ、あなた独りが最高だ」と教会によって歌われ
 ている。

\\


{\scshape Sed contra}, haec locutio, {\itshape solus pater est Deus}, habet duas
expositivas, scilicet, {\itshape pater est Deus}, et, {\itshape nullus alius a patre est
Deus}. Sed haec secunda est falsa, quia filius alius est a patre, qui
est Deus. Ergo et haec est falsa, {\itshape solus pater est Deus}. Et sic de
similibus.

&

しかし反対に、「独り父だけが神である」という表現には二つの説明があり、
 一つは「父は神である」もう一つは「父以外のだれも神でない」である。し
 かしこの後者は偽である。なぜなら息子は父とは別人だが神だからである。
 ゆえに「独り父だけが神である」は偽であり、これに類似した表現について
 も同様である。


\\



{\scshape Respondeo dicendum} quod, cum dicimus, {\itshape solus pater est Deus}, haec
propositio potest habere multiplicem intellectum. Si enim {\itshape solus} ponat
solitudinem circa patrem, sic est falsa, secundum quod sumitur
categorematice. Secundum vero quod sumitur syncategorematice, sic
iterum potest intelligi multipliciter. 


&

解答する。以下のように言われるべきである。
「独り父だけが神である」と私たちが言うとき、この命題は複数の内容を持ち
 うる。すなわち、もし「独り」が父にかんする孤独を言うのであれば、自立
 語として解される限りにおいてそれは偽である。他方、共義語として理解さ
 れる限りにおいても、さらに複数の意味に理解されうる。

\\

Quia si excludat a forma
subiecti, sic est vera, ut sit sensus, {\itshape solus pater est Deus}, idest,
{\itshape ille cum quo nullus alius est pater, est Deus}. Et hoc modo exponit
Augustinus, in VI {\itshape de Trin}., cum dicit, {\itshape solum patrem dicimus, non quia
separatur a filio vel spiritu sancto; sed hoc dicentes, significamus
quod illi simul cum eo non sunt pater}. 

&

もし主語の形相によって排除されるならばそれは真であり、その意味は「独り
 父だけが神である」すなわち「それ以外のだれも父でない者が神である」と
 なる。アウグスティヌスが『三位一体論』第4巻で「独りの父と私たちが言う
 のは、息子や聖霊から切り離されているからではなく、これを言うことで、息子と聖霊が父と共
 に同時に父であることはないということを示すようにである」。

\\

Sed hic sensus non habetur ex
consueto modo loquendi, nisi intellecta aliqua implicatione, ut si
dicatur, ille qui solus dicitur pater, est Deus.  


&

しかし、たとえば「独り彼だけが父と言われるその者が神である」
 というような、何らかの含意が理解されていなければ、通常の語り方はこの
 意味をもたない。

\\

Secundum vero
proprium sensum, excludit a consortio praedicati.
Et sic haec
propositio est falsa, si excludit {\itshape alium} masculine, est autem vera, si
excludit {\itshape aliud} neutraliter tantum, quia filius est alius a patre, non
tamen aliud; similiter et spiritus sanctus. 


&

他方、厳密な意味ではそれは述語の交わりから排除する。この意味では、「別
 人を」と男性で排除するならこの命題は偽だが、中性で「別物を」と排除す
 るなら真である。なぜなら息子は父と別人だが、別物ではないからである。
 これは聖霊についても同様である。

\\

Sed quia haec dictio solus
respicit proprie subiectum, ut dictum est, magis se habet ad
excludendum {\itshape alium} quam {\itshape aliud}. Unde non est extendenda talis locutio;
sed pie exponenda, sicubi inveniatur in authentica Scriptura.

&

しかし「独り」というこの語は、前に述べられたとおり、厳密には主語に関係
 するので、「別物」よりは「別人」を排除するようにしてある。したがって
 このような言い方が広げられるべきでないが、権威ある書物の中に見出され
 る場合には、敬虔に説明されるべきである。


\\



{\scshape Ad primum ergo dicendum} quod, cum dicimus, {\itshape te solum Deum verum}, non
intelligitur de persona patris, sed de tota trinitate, ut Augustinus
exponit. Vel, si intelligatur de persona patris, non excluduntur aliae
personae, propter essentiae unitatem, prout ly {\itshape solus} excludit tantum
{\itshape aliud}, ut dictum est.

&

第一異論に対してはそれゆえ以下のように言われるべきである。
私たちが「独りあなたが真の神」と言うとき、アウグスティヌスが説明してい
 るように、父のペルソナではなく三性全体が理解されている。あるいは、も
 し父のペルソナが理解されているとしても、すでに述べられたとおり、この「独り」が「別物」だけを排
 除するものとして、本質の一性のために他のペルソナは排除されない。

\\



Et similiter dicendum est {\scshape ad secundum}. Cum enim aliquid essentiale
dicitur de patre, non excluditur filius vel spiritus sanctus, propter
essentiae unitatem. Tamen sciendum est quod in auctoritate praedicta,
haec dictio nemo non idem est quod nullus homo, quod videtur
significare vocabulum (non enim posset excipi persona patris), sed
sumitur, secundum usum loquendi, distributive pro quacumque rationali
natura.

&

第二異論に対しても同様に言われるべきである。何か本質的なことが父につい
 て言われるとき、本質の一性のために息子や聖霊が排除されることはない。
 しかし、上述の権威においては、「だれも〜ない」というこの語が、その語
 彙が表示すると思われる「どんな人も〜ない」と同じではなく(父のペルソ
 ナは排除されえないのだから)、むしろ、慣例によりどんな理性的な
 本性にも分配的に理解されている。

\\



{\scshape Ad tertium dicendum} quod dictio exclusiva non excludit illa quae sunt
de intellectu termini cui adiungitur, si non differunt secundum
suppositum, ut pars et universale. Sed filius differt supposito a
patre, et ideo non est similis ratio.

&

第三異論に対しては以下のように言われるべきである。
排除語は、それが修飾する語の理解に属する者を排除しないが、それは、部分
 や普遍のように個体において異ならない場合である。しかし息子は個体にお
 いて父と違うので、同じ推論はできない。

\\



{\scshape Ad quartum dicendum} quod non dicimus absolute quod solus filius sit
altissimus, sed quod solus sit altissimus {\itshape cum spiritu sancto, in
gloria Dei patris}.

&

第四異論に対しては以下のように言われるべきである。
私たちは無関係的に、独り息子だけが最高だと言うのではなく、「父なる神の
 栄光において聖霊と共に」独り最高であると言う。

\end{longtable}
\newpage

\end{document}