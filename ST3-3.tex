\documentclass[10pt]{jsarticle} % use larger type; default would be 10pt
%\usepackage[utf8]{inputenc} % set input encoding (not needed with XeLaTeX)
%\usepackage[round,comma,authoryear]{natbib}
%\usepackage{nruby}
\usepackage{okumacro}
\usepackage{longtable}
%\usepqckage{tablefootnote}
\usepackage[polutonikogreek,english,japanese]{babel}
%\usepackage{amsmath}
\usepackage{latexsym}
\usepackage{color}

%----- header -------
\usepackage{fancyhdr}
\lhead{{\it Summa Theologiae} III, q.~3}
%--------------------

\bibliographystyle{jplain}

\title{{\bf TERTIA PARS}\\{\HUGE Summae Theologiae}\\Sancti Thomae
Aquinatis\\{\sffamily QUEAESTIO TERTIA}\\DE UNIONE EX PARTE PERSONAE
ASSUMENTIS}
\author{Japanese translation\\by Yoshinori {\sc Ueeda}}
\date{Last modified \today}


%%%% コピペ用
%\rhead{a.~}
%\begin{center}
% {\Large {\bf }}\\
% {\large }\\
% {\footnotesize }\\
% {\Large \\}
%\end{center}
%
%\begin{longtable}{p{21em}p{21em}}
%
%&
%
%
%\\
%\end{longtable}
%\newpage



\begin{document}
\maketitle
\pagestyle{fancy}



\begin{center}
{\Large 第三問\\合一について:受容するペルソナの側から}
\end{center}



\begin{longtable}{p{21em}p{21em}}




{\Huge D}{\scshape einde} considerandum est de unione ex parte personae assumentis. Et
circa hoc quaeruntur octo. 


\begin{enumerate}
 \item utrum assumere conveniat personae divinae.
 \item utrum conveniat naturae divinae. 
 \item utrum natura possit assumere, abstracta personalitate.
 \item utrum una persona possit assumere sine alia.
 \item utrum quaelibet persona possit assumere.
 \item utrum plures personae possint assumere unam naturam numero. 
 \item utrum una persona possit assumere duas naturas numero.
 \item utrum magis fuerit conveniens de persona filii quod assumpsit humanam naturam, quam de alia persona divina.
\end{enumerate}


&

次に、受容するペルソナの側から合一について考察されるべきである。これにつ
 いては八つのことが問われる。

\begin{enumerate}
 \item 受容することは神のペルソナに適合するか。
 \item 神の本性に適合するか。
 \item ペルソナ性から切り離された本性が受容することができるか。
 \item 一つのペルソナが、他のペルソナなしに、受容することができるか。
 \item どのペルソナも受容することができるか。
 \item 複数のペルソナが、数的に一つの本性を受容できるか。
 \item 一つのペルソナが、数的に二つの本性を受容できるか。
 \item 人間本性を受容したことは、神の他のペルソナよりも、息子のペルソナ
       について適合していたか。
\end{enumerate}

\end{longtable}
\newpage

\rhead{a.~1}
\begin{center}
 {\Large {\bf ARTICULUS PRIMUS}}\\
 {\large UTRUM PERSONAE DIVINAE CONVENIAT ASSUMERE NATURAM CREATAM}\\
 {\footnotesize III {\itshape Sent.}, d.5, q.2, a.1.}\\
 {\Large 第一項\\被造の本性を受容することは神のペルソナに適合するか}
\end{center}

\begin{longtable}{p{21em}p{21em}}




{\scshape Ad primum sic proceditur}. Videtur quod personae divinae non conveniat
assumere naturam creatam. Persona enim divina significat aliquid maxime
perfectum. Perfectum autem est cui non potest fieri additio. Cum igitur
assumere sit quasi {\itshape ad se sumere}, ita quod assumptum addatur assumenti,
videtur quod personae divinae non conveniat assumere naturam creatam.


&


第一項の問題へ、議論は以下のように進められる。神のペルソナに、被造の本性
 を受容することは適合していないと思われる。理由は以下の通り。
神のペルソナは、最大限に完全な何かを表示する。しかし、完全なものとは、そ
 れに付加がなされえないものである。ゆえに、受容することは、いわば、自ら
 に何かを採ることであり、受容されたものが受容するものに付加されるので、
 神のペルソナに、被造の本性を受容することは適合しないと思われる。

\\



2. {\scshape Praeterea}, illud ad quod aliquid assumitur, communicatur quodammodo ei
quod in ipsum assumitur, sicut dignitas communicatur ei qui in
dignitatem assumitur. Sed de ratione personae est quod sit
incommunicabilis, ut in prima parte dictum est. Ergo personae divinae
non convenit assumere, quod est ad se sumere.


&

さらに、AがBへ受容されるとき、BはAに、なんらかのかたちで伝達される。
たとえば、人が権威の中に受容されるとき、権威は人へ伝達される。
しかし、ペルソナの性格には、第一部で述べられたとおり\footnote{STI, q.29,
 a.3, ad 4, etc.}、不可伝達性というこ
 とが属する。
ゆえに、神のペルソナに、自らへと採ることである「受容する」ということは適
 合しない。

\\



3. {\scshape Praeterea}, persona constituitur per naturam. Sed inconveniens est quod
constitutum assumat constituens, quia effectus non agit in suam
causam. Ergo personae non convenit assumere naturam.


&


さらに、ペルソナは、本性によって構成される。しかし、構成されたものが構成
 するものを受容するのは不適切である。なぜなら、結果が自分の原因へ作用す
 ることはないからである。ゆえに、ペルソナには、本性を受容することが適し
 ていない。

\\



{\scshape Sed contra est} quod Augustinus dicit, in libro {\itshape de Fide ad Petrum},
{\scshape formam, idest naturam servi in suam accepit Deus ille, scilicet
unigenitus, personam}. Sed Deus unigenitus est persona. Ergo personae
competit accipere naturam, quod est assumere.



&

しかし反対に、アウグスティヌスは、『ペテロ宛信仰について』という書物で、
 「僕(しもべ)の形相、すなわち本性を、かの神、すなわち独り子が、自らのペルソナへ
 と受け取った」と述べている。しかし、独り子である神はペルソナである。ゆ
 えに、ペルソナには、本性を受け取ること、すなわち受容することが適合する。

\\


{\scshape Respondeo dicendum} quod in verbo {\itshape assumptionis} duo importantur, videlicet
principium actus, et terminus, dicitur enim assumere quasi {\itshape ad se aliquid
sumere}. Huius autem assumptionis persona est et principium et
terminus. Principium quidem, quia personae proprie competit agere,
huiusmodi autem sumptio carnis per actionem divinam facta est. Similiter
etiam persona est huius sumptionis terminus, quia, sicut supra dictum
est, unio facta est in persona, non in natura. Et sic patet quod
propriissime competit personae assumere naturam.


&

解答する。以下のように言われるべきである。
「受容」という言葉には二つのことが意味されている。すなわち、作用の始まりと、
 作用の終わりである。なぜなら、受容するということは、「自らに何かを受け取
 る」というように言われるからである。
ところで、ペルソナは、このような受容の始まりであり終わりである。
始まりだというのは、作用することが、厳密には、ペルソナに適合するからであ
 る。しかし、このような肉の受容は、神の作用によってなされた。
同様に、ペルソナは、そのような受容の終わりである。というのも、上述のごと
 く\footnote{q.2, a.1, a.2.}合一は本性でなくペルソナにおいてなされたから
 である。この意味で、本性を受容することは、もっとも固有の意味で、ペルソ
 ナに適合することが明らかである。


\\



{\scshape Ad primum ergo dicendum} quod, cum persona divina sit infinita, non
potest ei fieri additio. Unde Cyrillus dicit, in Epistola Synodali
Ephesini Concilii, {\itshape non secundum coappositionem coniunctionis
intelligimus modum}. Sicut etiam in unione hominis ad Deum quae est per
gratiam adoptionis, non additur aliquid Deo, sed id quod divinum est
apponitur homini. Unde non Deus, sed homo perficitur.


&

第一異論に対しては、それゆえ、以下のように言われるべきである。
神のペルソナは無限なので、それに付加がなされることはできない。このことか
 ら、キュリッルスは、エフェソス公会議の書簡で「私たちは、結合の付与のし
 かたで理解しない」と述べている。また、受容の恩恵による、人間の神への合
 一において、神に何かが付加されることはなく、むしろ、神的なものが人間に
 付与される。したがって、神ではなく、人間が完成される。


\\



{\scshape Ad secundum dicendum} quod persona dicitur incommunicabilis inquantum non
potest de pluribus suppositis praedicari. Nihil tamen prohibet plura de
persona praedicari. Unde non est contra rationem personae sic
communicari ut subsistat in pluribus naturis. Quia etiam in personam
creatam possunt plures naturae concurrere accidentaliter, sicut in
persona unius hominis invenitur quantitas et qualitas. Hoc autem est
proprium divinae personae, propter eius infinitatem, ut fiat in ea
concursus naturarum, non quidem accidentaliter, sed secundum
subsistentiam.


&

第二異論に対しては、以下のように言われるべきである。
ペルソナが不可伝達的と言われるのは、複数の個体について述語となりえないか
 らである。しかし、複数のことが、ペルソナについて述語されるのはかまわな
 い。したがって、複数の本性において自存するかたちで伝達されることは、ペルソナの性格に反し
 ない。さらに、一つの被造のペルソナへ、複数の本性が、附帯的に集まること
 も可能である。たとえば、一人の人間のペルソナにおいて、複数の量や性質が見出され
 るように。しかし、附帯的にでなく、自存体に即して、諸本性が集まることが
 そこでなされるのは、その無限性のために、神のペルソナに固有のことである。

\\



{\scshape Ad tertium dicendum} quod, sicut supra dictum est, natura humana non
constituit personam divinam simpliciter, sed constituit eam secundum
quod denominatur a tali natura. Non enim ex natura humana habet filius
Dei quod sit simpliciter, cum fuerit ab aeterno, sed solum quod sit
homo. Sed secundum naturam divinam constituitur persona divina
simpliciter. Unde persona divina non dicitur assumere divinam naturam,
sed humanam.


&

第三異論に対しては、以下のように言われるべきである。
すでに述べられたとおり\footnote{q.2, a.6, ad 2.}、人間本性は神のペルソナ
 を端的には構成せず、そのような本性によって名付けられる限りで、それを構成
 するに過ぎない。なぜなら、神の息子は永遠から存在するので、人間本性に基
 づいては、端的に存在することを持つのではなく、ただ人間であるということ
 だけを持つからである。しかし、
 神の本性にしたがって、神のペルソナは端的に構成される。したがって、神の
 ペルソナが神の本性を受容するとは言われず、人間の本性を受容すると言わ
 れる。


\end{longtable}
\newpage






\rhead{a.~2}
\begin{center}
 {\Large {\bf ARTICULUS SECUNDUS}}\\
 {\large UTRUM NATURAE DIVINAE CONVENIAT ASSUMERE}\\
 {\footnotesize III {\itshape Sent.}, d.5, q.2, a.2.}\\
 {\Large 第二項\\受容することは神の本性に適合するか}
\end{center}

\begin{longtable}{p{21em}p{21em}}

{\Huge A}{\scshape d secundum sic proceditur}. Videtur quod naturae divinae non conveniat
 assumere. Quia, sicut dictum est assumere dicitur quasi {\itshape ad se
 sumere}. Sed natura divina non sumpsit ad se humanam naturam, quia non
 est facta unio in natura, sed in persona, sicut supra dictum est. Ergo
 naturae divinae non competit assumere naturam humanam.

&

第二項の問題へ、議論は以下のように進められる。
神の本性には、受容することが適合しないと思われる。
理由は以下の通り。すでに述べられたように、受容することは、いわば「自分へ採る」というように語られ
 る。しかし、神の本性は、自らに人間本性を採るわけではない。なぜなら、前
 に述べられたとおり、合
 一は本性においてでなく、ペルソナにおいてなされたからである。ゆえに、神
 の本性に、人間本性を受容することは適合しない。

\\



2. {\scshape Praeterea}, natura divina communis est tribus personis. Si igitur naturae
 convenit assumere, sequitur quod conveniat tribus personis et ita pater
 assumpsit humanam naturam, sicut et filius. Quod est erroneum.

&


さらに、神の本性は、三つのペルソナに共通である。ゆえに、もし、受容するこ
 とが本性に適合するなら、三つのペルソナに適合することになり、したがって、
 息子と同じく、父も人間本性を採ったことになる。これは誤りである。

\\



3. {\scshape Praeterea}, assumere est agere. Agere autem convenit personae, non
 naturae, quae magis significatur ut principium quo agens agit. Ergo
 assumere non convenit naturae.


&


さらに、受容することは作用することである。しかし、作用することは、本性で
 はなくペルソナに適合する。本性は、むしろ、それによって作用するものが作
 用するところの根源である。ゆえに、受容することは本性に適合しない。

\\




{\scshape Sed contra est} quod Augustinus dicit, in libro {\itshape de fide ad Petrum}, {\itshape illa
 natura quae semper genita manet ex patre}, idest, quae est per
 generationem aeternam accepta a patre, {\itshape naturam nostram sine peccato
 suscepit}.


&

しかし反対に、アウグスティヌスは『ペテロ宛信仰について』という書物で、
 「常に父から生まれたものにとどまる本性が」つまり、父から、永遠の生成に
 よって受け取られた本性が、「私たちの本性を罪がないものに保っている」
 と述べている。

\\




{\scshape Respondeo dicendum} quod, sicut dictum est, in verbo assumptionis duo
 significantur, scilicet principium actionis, et terminus eius. Esse
 autem assumptionis principium convenit naturae divinae secundum
 seipsam, quia eius virtute assumptio facta est. 




&

解答する。以下のように言われるべきである。
すでに述べられたとおり、受容という言葉において、二つのことが意味されてい
 る。すなわち、作用の始まりと終わりである。ところで、受容の始まりである
 ということは、神の本性に、それ自体において適合する。なぜなら、神の力に
 よって、受容がなされたからである。


\\


Sed esse terminum
 assumptionis non convenit naturae divinae secundum seipsam, sed ratione
 personae in qua consideratur. Et ideo primo quidem et propriissime
 persona dicitur assumere, secundario autem potest dici quod etiam
 natura assumit naturam ad sui personam. 


&


他方、受容の終わりであるということは、それ自身において神の本性に適合する
 のではなく、むしろ、そこにおいて考察されている、ペルソナの性格において
 である。ゆえに、第一に、そしてもっとも固有の意味で、ペルソナが受容する
 と言われるが、第二の意味で、本性もまた、自らのペルソナへ、本性を受容す
 ると言われうる。

\\


Et secundum etiam hunc modum
 dicitur natura incarnata, non quasi sit in carnem conversa; sed quia
 naturam carnis assumpsit. Unde dicit Damascenus, {\itshape dicimus naturam Dei
 incarnatam esse, secundum beatos Athanasium et Cyrillum}.


&

そして、この意味で、肉へ転換するという意味ではなく、肉
 の本性を受容したので、受肉した本性と言われる。このことから、ダマスケヌ
 スは、「至福なるアタナシウスとキュリッルスにしたがって、私たちは、神の
 本性が受肉したと言う」と述べている。



\\




{\scshape Ad primum ergo dicendum} quod ly {\slshape se} est reciprocum, et refert idem
 suppositum. Natura autem divina non differt supposito a persona
 Verbi. Et ideo, inquantum natura divina sumit naturam humanam ad
 personam Verbi, dicitur eam ad se sumere. Sed quamvis Pater assumat
 naturam humanam ad personam Verbi, non tamen propter hoc sumit eam ad
 se, quia non est idem suppositum Patris et Verbi. Et ideo non potest
 dici proprie quod Pater assumat naturam humanam.


&

第一異論に対しては、それゆえ、以下のように言われるべきである。
この「自分へ」の「自分」は再帰的である。そして、個体と同じものを意味する。
 しかし、神の本性は、個体において、言葉のペルソナと異ならない。ゆえに、
 神の本性が人間本性を言葉のペルソナへ採る限りで、それを自分へ採ると言わ
 れる。しかし、父が人間本性を言葉のペルソナへ採るとはいえ、このために、
 それを自分へ採るとは言われない。なぜなら、父と言葉の個体は同じでないか
 らである。ゆえに、父が人間本性を受容するとは、厳密には言われえない。

\\




{\scshape Ad secundum dicendum} quod id quod convenit divinae naturae secundum se,
 convenit tribus personis, sicut bonitas, sapientia et huiusmodi. Sed
 assumere convenit ei ratione personae verbi, sicut dictum est. Et ideo
 soli illi personae convenit.


&

第二異論に対しては、以下のように言われるべきである。
神の本性に自体的に適合するものは、三つのペルソナに適合する。善、知恵、そ
 のようなものがそうである。しかし、すでに述べられたとおり\footnote{主文。}、受容することは、言葉のペルソナが根拠
 となって、神の本性に適合する。ゆえに、そのペルソナだけに適合する。

\\




{\scshape Ad tertium dicendum} quod, sicut in Deo idem est quod est et quo est, ita
 etiam in eo idem est quod agit et quo agit, quia unumquodque agit
 inquantum est ens. Unde natura divina et est id quo Deus agit, et est
 ipse Deus agens.


&

第三異論に対しては、以下のように言われるべきである。
神において、「あるもの」と「それによってあるところもの」が同一であるよう
 に、「作用するもの」と「それによって作用するところのもの」は同一である。
 なぜなら、各々のものは、存在するものである限りにおいて作用するからであ
 る。したがって、神の本性は、それによって神が作用するところのものであり、
 かつ、作用する神それ自体である。

\end{longtable}
\newpage



\rhead{a.~3}
\begin{center}
 {\Large {\bf ARTICULUS TERTIUS}}\\
 {\large UTRUM, ABSTRACTA PERSONALITATE PER INTELLECTUM,\\ NATURA POSSIT ASSUMERE}\\
 {\footnotesize III {\itshape Sent.}, d.5, q.2, a.3.}\\
 {\Large 第三項\\本性は、知性によってペルソナ性から抽象されても、\\受容する
 ことができるか}
\end{center}

\begin{longtable}{p{21em}p{21em}}




{\Huge A}{\scshape d tertium sic proceditur}. Videtur quod, abstracta personalitate per
intellectum, natura non possit assumere. Dictum est enim quod naturae
convenit assumere ratione personae. Sed quod convenit alicui ratione
alicuius, remoto eo, non potest ei convenire, sicut corpus, quod est
visibile ratione coloris, sine colore videri non potest. Ergo, abstracta
personalitate per intellectum, natura assumere non potest.


&

第三項の問題へ、議論は以下のように進められる。
知性によってペルソナ性から抽象されると、本性は受容することができないと思
 われる。理由は以下の通り。
本性に受容することが適合するのは、ペルソナが根拠となってであると述べられ
 た。しかし、Aが根拠となってBが何かに適合するとき、Aが取り除かれると、B
 はそれに適合しえない。たとえば、物体は、色が根拠となって、可視的なもの
 であるので、色がなければ、見られることができない。ゆえに、ペルソナ性が
 知性によって抽象されると、本性は受容することができない。

\\



2. {\scshape Praeterea}, assumptio importat terminum unionis, ut dictum est. Sed unio
non potest fieri in natura, sed solum in persona. Ergo, abstracta
personalitate, natura divina non potest assumere.


&


さらに、すでに述べられたとおり、受容は、合一の終わりを意味する。しかし、
 合一は、本性においては生じえず、ただペルソナにおいて生じうる。ゆえに、
 ペルソナ性が抽象されると、神の本性は受容することができない。

\\



3. {\scshape Praeterea}, in prima parte dictum est quod in divinis, abstracta
personalitate, nihil manet. Sed assumens est aliquid. Ergo, abstracta
personalitate, non potest divina natura assumere.


&

さらに、第一部で言われたとおり\footnote{q.40, a.3.}、神において、ペルソ
 ナ性が抽象されると、何も残らない。しかし、受容することは何かである。ゆ
 えに、ペルソナ性が抽象されると、神の本性は受容できない。

\\



{\scshape Sed contra est} quod in divinis personalitas dicitur proprietas
personalis, quae est triplex, scilicet paternitas, processio et
filiatio, ut in prima parte dictum est. Sed, remotis his per
intellectum, adhuc remanet Dei omnipotentia, per quam est facta
incarnatio, sicut Angelus dixit, Luc.~{\scshape i}, {\itshape non erit impossibile apud Deum
omne verbum}. Ergo videtur quod, etiam remota personalitate, natura
divina possit assumere.


&

しかし反対に、第一部で言われたとおり\footnote{q.30, a.2.}、神において、
 ペルソナ性と言われるのは、ペルソナの固有性であり、それは父性、発出、子
 性の三通りである。しかし、これらが知性によって取り除かれても、神の全能
 は残り、それによって、受肉がなされた。ちょうど天使が『ルカによる福音書』
 第1章で「神の元では全ての言葉が不可能でない」\footnote{「神にできないこ
 とは何一つない」(1:37)}と言うように。ゆえに、ペルソナ性が取り除かれても、
 神の本性は受容できると思われる。

\\



{\scshape Respondeo dicendum} quod intellectus dupliciter se habet ad divina. Uno
modo, ut cognoscat Deum sicuti est. Et sic impossibile est quod
circumscribatur per intellectum aliquid a Deo quod aliud remaneat, quia
totum quod est in Deo est unum, salva distinctione personarum; quarum
tamen una tollitur, sublata alia, quia distinguuntur solum relationibus,
quas oportet esse simul. 


&

解答する。以下のように言われるべきである。
知性は、二通りのしかたで神の事柄に関係する。一つは、神をありのままに認
 識するというしかたである。そしてこの意味では、知性によって何かが神から
 取り除かれ、別のものが残るということは不可能である。なぜなら、神の中に
 あるものは、ペルソナの区別を除いて全て一つだからである。そしてペルソナの区別
 も、一つがなくなると、他のものもなくなる。なぜなら、同時に存在しなけれ
 ばならない諸関係によってのみ、区別されているからである。


\\


Alio modo se habet intellectus ad divina, non
quidem quasi cognoscens Deum ut est, sed per modum suum, scilicet
multipliciter et divisim id quod in Deo est unum. 


&

知性が神の事柄に関係するもう一つのしかたは、神をあるがままにではなく、知
 性のあり方によって、つまり、神において一つのものを、多様に分割して認識
 するしかたである。


\\

Et per hunc modum
potest intellectus noster intelligere bonitatem et sapientiam divinam,
et alia huiusmodi, quae dicuntur essentialia attributa, non intellecta
paternitate vel filiatione, quae dicuntur personalitates. Et secundum
hoc, abstracta personalitate per intellectum, possumus adhuc intelligere
naturam assumentem.


&

そしてこのしかたによれば、私たちの知性は、本質的属性と言われる神の善性、知恵、その他そのよう
 なものを、ペルソナ性と言われる父性や息子性を認識せずに、認識することが
 できる。そしてこの限りで、ペルソナ性が知性によって抽象されても、私たち
 は、本性が
 受容することを知性認識することができる。

\\



{\scshape Ad primum ergo dicendum} quod, quia in divinis idem est quo est et quod
est, quidquid eorum quae attribuuntur Deo in abstracto secundum se
consideretur, aliis circumscriptis, erit aliquid subsistens, et per
consequens persona, cum sit in natura intellectuali. 


&

第一異論に対しては、それゆえ、以下のように言われるべきである。
神において、それによって在るところのもの(quo est)と、在るもの(quod est)とは同一であるから、
 神に帰属する各々のものは、抽象的に、それ自体において考察され、他のもの
 が取り除かれるならば、自存するなにかである。したがって、ペルソナもまた、
 知性的本性において存在するのだから、自存する。

\\


Sicut igitur nunc,
positis proprietatibus personalibus in Deo, dicimus tres personas, ita,
exclusis per intellectum proprietatibus personalibus, remanebit in
consideratione nostra natura divina ut subsistens, et ut persona. Et per
hunc modum potest intelligi quod assumat naturam humanam ratione suae
subsistentiae vel personalitatis.


&

ゆえに、今私たちは、ペルソナ的固有性を神の中に措定して、三つのペルソナと
 言うが、同じように、知性によって、ペルソナ的固有性が除去されても、自存する
 もの、つまりペルソナとしての神の本性が私たちの考察の
 中に留まるであろう。このようにして、その自存性ないしペルソナ性のゆえに、
 人間本性を受容することが知解されうる。



\\



{\scshape Ad secundum dicendum} quod, etiam circumscriptis per intellectum
personalitatibus trium personarum, remanebit in intellectu una
personalitas Dei, ut Iudaei intelligunt, ad quam poterit terminari
assumptio, sicut nunc dicimus eam terminari ad personam verbi.


&

第二異論に対しては、以下のように言われるべきである。
現にユダヤの人々が考えるように、知性によって、三つのペルソナのペルソナ性が除
 去されても、知性の中に、神の一つのペルソナ性が残るであろう。そしてそ
 れに、受容が終極することが可能であっただろう。ちょうど、今、それが言葉
 のペルソナに終極する、と私たちが言うように。


\\



{\scshape Ad tertium dicendum} quod, abstracta personalitate per intellectum,
dicitur nihil remanere per modum resolutionis, quasi aliud sit quod
subiicitur relationi, et aliud ipsa relatio, quia quidquid consideratur
in Deo, consideratur ut suppositum subsistens. Potest tamen aliquid
eorum quae dicuntur de Deo intelligi sine alio, non per modum
resolutionis, sed per modum iam dictum.


&

第三異論に対しては、以下のように言われるべきである。
知性によってペルソナ性が抽象されると、還元のしかたでは、何も残らない。そ
 れは、関係のもとにあることと、関係自体とが別のものであるような意味であ
 る。なぜなら、神において考察されるものはどれも、自存する個体として考察
 されるからである。しかし、神について言われる
 もののうちの何かが、還元のしかたではなく、すでに述べられたしかたで、他
 のものなしに知解されうる。



\end{longtable}
\newpage





\rhead{a.~4}
\begin{center}
 {\Large {\bf ARTICULUS QUARTUS}}\\
 {\large UTRUM UNA PERSONA POSSIT ASSUMERE\\NATURAM CREATAM, ALIA NON ASSUMENTE}\\
 {\footnotesize III {\itshape Sent.}, d.1, q.2, a.1; IV {\itshape SCG}, cap.39.}\\
 {\Large 第四項\\他のペルソナが受容しないのに、\\一つのペルソナが受容しう
 るか}
\end{center}

\begin{longtable}{p{21em}p{21em}}

{\Huge A}{\scshape d quartum sic proceditur}. Videtur quod una persona non possit assumere
naturam creatam, alia non assumente. {\itshape Indivisa} enim {\itshape sunt opera
Trinitatis}, ut dicit Augustinus, in {\itshape Enchirid}., sicut enim trium
personarum est una essentia, ita una operatio. Sed assumere est operatio
quaedam. Ergo non potest convenire uni personae divinae quin conveniat
alii.


&

第四項の問題へ、議論は以下のように進められる。
他のペルソナが受容しないならば、一つのペルソナが被造の本性を受容すること
 はできないと思われる。理由は以下の通り。
アウグスティヌスが『エンキリディオン』で言うように、「個体は三位一体の業
 である」。すなわち、ちょうど三つのペルソナに一つの本質があるように、一つの働きが
 ある。しかし、受容することは、ある種の働きである。ゆえに、他のペルソナ
 に適合しないのに、神の一つのペルソナに適合することはありえない。

\\



2. {\scshape Praeterea}, sicut dicimus personam filii incarnatam, ita et naturam, {\itshape tota}
enim {\itshape divina natura in una suarum hypostasum incarnata est}, ut dicit
Damascenus, in III libro. Sed natura communis est tribus personis. Ergo
et assumptio.


&

さらに、子のペルソナが受肉したと私たちが言うように、本性もまた受肉し
 た。なぜなら、ダマスケヌスが第三巻で言うように「神の本性全体が、自らの
 ヒュポスタシスの一つにおいて受肉した」からである。しかし、本性は、三つ
 のペルソナに共通である。ゆえに、受容も共通である。


\\



3. {\scshape Praeterea}, sicut humana natura in Christo assumpta est a Deo, ita etiam
et homines per gratiam assumuntur ab ipso, secundum illud {\itshape Rom}.~{\scshape xiv}, {\itshape Deus
illum assumpsit}. Sed haec assumptio communiter pertinet ad omnes
personas. Ergo et prima.


&

さらに、人間本性が、
 神によって、キリストにおいて、受容されたように、『ローマの信徒への手
 紙』第14章「神は彼を受容した」\footnote{「食べる人は、食べない人を軽蔑
 してはならないし、また、食べない人は、食べる人を裁いてはなりません。神
 はこのような人をも受け入れられたからです。」(14:3)}によれば、人間たちは、恩恵によっ
 て、神によって受容された。しかし、この後者の受容は、全てのペルソナに共通して
 属する。ゆえに前者の受容もまた全てのペルソナに共通する。

\\



{\scshape Sed contra est} quod Dionysius, {\scshape ii} cap.~{\itshape de Div.~Nom}., incarnationis
mysterium dicit pertinere ad {\itshape discretam theologiam}, secundum quam
scilicet aliquid distinctum dicitur de divinis personis.


&


しかし反対に、ディオニュシウスは、『神名論』第2章で、受肉の秘蹟が「個別
 的な神学」に属すると言っている。それによれば、ある区別された事柄が、神
 の諸ペルソナについて言われる。

\\



{\scshape Respondeo dicendum} quod, sicut dictum est, assumptio duo importat,
scilicet actum assumentis, et terminum assumptionis. Actus autem
assumentis procedit ex divina virtute, quae communis est tribus
personis, sed terminus assumptionis est persona, sicut dictum est. Et
ideo id quod est actionis in assumptione, commune est tribus personis,
sed id quod pertinet ad rationem termini, convenit ita uni personae quod
non alii. Tres enim personae fecerunt ut humana natura uniretur uni
personae filii.


&

解答する。以下のように言われるべきである。
すでに述べられたとおり\footnote{a.1, a.2.}、受容は二つのことを意味する。受容する作用と、受容
 の終極である。さて、受容の作用は、神の力から出ていくが、この力は、三つ
 のペルソナに共通である。しかし、受容の終極は、前に述べられたとおり\footnote{a.2}、ひ
 とつのペルソナである。ゆえに、受容において作用に属するものは、三つのペ
 ルソナに共通だが、終極に属するものは、他のものにではないかたちで、一つ
 のペルソナに属する。つまり、三つのペルソナが、人間本性が一つの息子のペ
 ルソナに合一されるようにしたのである。



\\



{\scshape Ad primum ergo dicendum} quod ratio illa procedit ex parte
operationis. Et sequeretur conclusio si solam illam operationem
importaret absque termino, qui est persona.


&


第一異論に対しては、それゆえ、以下のように言われるべきである。
この論は、働きの側から進められている。したがって、もし、終極つまりペルソ
 ナを含まず、その働きだけを意味していたならば、その結論が得
 られたであろう。


\\



{\scshape Ad secundum dicendum} quod natura dicitur incarnata, sicut et assumens,
ratione personae ad quam terminata est unio, sicut dictum est, non autem
prout est communis tribus personis. Dicitur autem {\itshape tota natura divina
incarnata}, non quia sit incarnata in omnibus personis, sed quia nihil
deest de perfectione divinae naturae personae incarnatae.


&

第二異論に対しては、以下のように言われるべきである。
すでに述べられたとおり、本性が、受容すると言われるのと同じように受肉した
 と言われるのは、合一がそれに終極するペルソナが根拠となってのことであり、
 三つのペルソナに共通のこととしてではない。しかし、「神の全本性が受肉した」
 と言われるのは、全てのペルソナにおいて受肉したからではなく、受肉したペ
 ルソナの神の本性の完全性に、欠けるところがないからである。


\\



{\scshape Ad tertium dicendum} quod assumptio quae fit per gratiam adoptionis,
terminatur ad quandam participationem divinae naturae secundum
assimilationem ad bonitatem illius, secundum illud II Pet.~{\scshape i}, {\itshape Ut divinae
consortes naturae}, et cetera. Et ideo huiusmodi assumptio communis est
tribus personis et ex parte principii et ex parte termini. Sed assumptio
quae est per gratiam unionis, est communis ex parte principii, non autem
ex parte termini, ut dictum est.


&

第三異論に対しては、以下のように言われるべきである。
『ペトロの手紙二』「神と同類の本性になるた
 めに」\footnote{「この栄光と力ある業とによって、わたしたちは尊くすばら
 しい約束を与えられています。それは、あなたがたがこれらによって、情欲に
 染まったこの世の退廃を免れ、神の本性にあずからせていただくようになるた
 めです。」(1:4)}によれば、養子の恵みによってなされる受容は、神の善性への類似化にしたがって、神の本
 性の何らかの分有へ終極する。それゆえ、このような受容は、始まりの側から
 も終わりの側からも、三つのペルソナに共通する。しかし、合一の恩恵による
 受容は、始まりの側からは共通だが、終わりの側からは共通でない。すでに述
 べられたとおりである。





\end{longtable}
\newpage






\rhead{a.~5}
\begin{center}
 {\Large {\bf ARTICULUS QUINTUS}}\\
 {\large UTRUM ALIA PERSONA DIVINA POTUERIT\\HUMANAM NATURAM ASSUMERE,
 PRAETER PERSONAM FILII}\\
 {\footnotesize III {\itshape Sent.}, d.1, q.2, a.3.}\\
 {\Large 第五項\\子のペルソナ以外の神のペルソナが人間本性を受容できたか}
\end{center}

\begin{longtable}{p{21em}p{21em}}

{\Huge A}{\scshape d quintum sic proceditur}. Videtur quod nulla alia persona divina potuit
humanam naturam assumere, praeter personam filii. Per huiusmodi enim
assumptionem factum est quod Deus sit filius hominis. Sed inconveniens
esset quod esse filium conveniret patri vel spiritui sancto, hoc enim
vergeret in confusionem divinarum personarum. Ergo pater et spiritus
sanctus carnem assumere non possent.


&

第五項の問題へ、議論は以下のように進められる。
子のペルソナ以外のどの神のペルソナも、人間本性を受容することはできなかっ
 たと思われる。理由は以下の通り。
このような受容によって、神が人間の息子であるという事実が作られた。しかし、
 息子であることが、父や聖霊に適合したら、不都合が生じたであろう。なぜな
 ら、神のペルソナを混同することになっただろうから。ゆえに、父と聖霊は、
 肉を受容することができなかった。


\\



2. {\scshape Praeterea}, per incarnationem divinam homines sunt assecuti adoptionem
filiorum, secundum illud {\itshape Rom}.~{\scshape viii}, {\itshape non accepistis spiritum servitutis
iterum in timore, sed spiritum adoptionis filiorum}. Sed filiatio
adoptiva est participata similitudo filiationis naturalis, quae non
convenit nec patri nec spiritui sancto, unde dicitur {\itshape Rom}.~{\scshape viii}, {\itshape quos
praescivit et praedestinavit conformes fieri imaginis filii sui}. Ergo
videtur quod nulla alia persona potuit incarnari praeter personam filii.


&

さらに、『ローマの信徒への手紙』第8章「あなたたちは、再び恐怖において奴
 隷とする霊を受けたのではなく、息子たちを養子とする霊を受けた」
 \footnote{「あなたがたは、人を奴隷として再び恐れに陥れる霊ではなく、神
 の子とする霊を受けたのです。この霊によってわたしたちは、「アッバ、父よ」
 と呼ぶのです。」(8:15)}によれば、神の受肉によって、人間たちは、息子たち
 を養子とすることを獲得した。しかし、養子によって息子となることは、
 本性的に息子となることの、分有された類似であるが、この後者は、父にも聖霊
 にも適合しない。このことから、『ローマの信徒への手紙』第8章で、「自分の
 息子の像に一致するようになる人々を予知し、予定した」\footnote{「神は前
 もって知っておられた者たちを、御子の姿に似たものにしようとあらかじめ定
 められました。それは、御子が多くの兄弟の中で長子となられるためです。」
 (8:29)}と言われている。ゆえに、息子のペルソナ以外のどのペルソナも、受肉
 することができなかったと思われる。


\\



3. {\scshape Praeterea}, filius dicitur missus, et genitus nativitate temporali,
secundum quod incarnatus est. Sed patri non convenit mitti, qui est
innascibilis, ut in prima parte habitum est. Ergo saltem persona patris
non potuit incarnari.


&

さらに、息子は、受肉したものである限りにおいて、遣わされたもの、時間的な出生によって生まれたものと言われ
 る。しかし、父には、遣わされることは適合しない。第一部で言われたとおり、
 父は生み出されえないものだからである。ゆえに、父のペルソナが受肉す
 ることは不可能だった。

\\



{\scshape Sed contra}, quidquid potest filius, potest pater, alioquin, non esset
eadem potentia trium. Sed filius potuit incarnari. Ergo similiter pater
et spiritus sanctus.


&

しかし反対に、息子にできることは何でも、父にもできる。さもなければ、三つ
 の能力は同じでなかっただろう。しかし、息子は受肉することができた。ゆえ
 に、同様に、父も聖霊も、受肉することができた。

\\



{\scshape Respondeo dicendum} quod, sicut dictum est, assumptio duo importat,
scilicet ipsum actum assumentis, et terminum assumptionis. Principium
autem actus est virtus divina, terminus autem est persona. 


&

解答する。以下のように言われるべきである。
すでに述べられたとおり、受容は二つのことを含む。すなわち、受容の作用自体
 と、受容の終極である。そして、作用の始まりは神の力であり、終わりはペル
 ソナである。


\\

Virtus autem
divina communiter et indifferenter se habet ad omnes personas. Eadem
etiam est communis ratio personalitatis in tribus personis, licet
proprietates personales sint differentes. 


&


神の力は、全てのペルソナに共通に、差別なく関係する。また、ペルソナの固有
 性は異なっているが、三つのペルソナにおいて、ペルソナ性という性格は共通
 する。

\\

Quandocumque autem virtus
aliqua indifferenter se habet ad plura, potest ad quodlibet eorum suam
actionem terminare, sicut patet in potentiis rationalibus, quae se
habent ad opposita, quorum utrumque agere possunt. 


&

ところで、なんらかの力が、複数のものに差別なく関係するときはいつも、その
 作用は、どれにも終極しうる。たとえば、理性的能力は、反対するものに関係
 するので、そのどちらにも作用できることから明らかである。

\\

Sic ergo divina
virtus potuit naturam humanam unire vel personae patris vel spiritus
sancti, sicut univit eam personae filii. Et ideo dicendum est quod pater
vel spiritus sanctus potuit carnem assumere, sicut et filius.


&

このようにして、それゆえ、神の力は、人間本性を息子のペルソナに合一させたように、父のペルソナや聖霊のペルソナに
 合一させることができた。ゆえに、父や聖霊は、息子と同様、肉を受容するこ
 とができたと言われるべきである。

\\



{\scshape Ad primum ergo dicendum} quod filiatio temporalis, qua Christus dicitur
filius hominis, non constituit personam ipsius, sicut filiatio aeterna,
sed est quiddam consequens nativitatem temporalem. Unde, si per hunc
modum nomen filiationis ad patrem vel spiritum sanctum transferretur,
nulla sequeretur confusio divinarum personarum.


&

第一異論に対しては、それゆえ、以下のように言われるべきである。
永遠的な息子性はそのペルソナを構成するが、キリストが人間の息子と言われる
 かぎりで言われる時間的な息子性は、ペルソナを構成しない。それは、時間的
 な出生に伴うなにかである。したがって、この意味での息子性が父や聖霊に移
 動されたとしても、神のペルソナの混乱はまったく生じない。


\\



{\scshape Ad secundum dicendum} quod filiatio adoptiva est quaedam participata
similitudo filiationis naturalis; sed fit in nobis appropriate a patre,
qui est principium naturalis filiationis; et per donum spiritus sancti,
qui est amor patris et filii; secundum illud {\itshape Galat}.~{\scshape iv}, {\itshape misit Deus
spiritum filii sui in corda nostra, clamantem, abba, pater}. 


&

第二異論に対しては、以下のように言われるべきである。
養子によって息子となることは、
 本性的に息子となることの、分有された類似である。しかし、私たちにおいて
 固有的に生じるのは、自然的に息子となることの根源である父によって、また、
 『ガラテヤの信徒への手紙』第4
 章「神は、「アッバ、父よ」と叫ぶ自分の息子の霊を私たちの心へと送った
 \footnote{「あなたがたが子であることは、神が、「アッバ、父よ」と叫ぶ御
 子の霊を、わたしたちの心に送ってくださった事実から分かります。」(4:6)}
 によれば、父と子の愛である聖霊の賜物によってである。

\\

Et ideo
sicut, filio incarnato, adoptivam filiationem accipimus ad similitudinem
naturalis filiationis eius; ita, patre incarnato, adoptivam filiationem
reciperemus ab eo tanquam a principio naturalis filiationis; et a
spiritu sancto, tanquam a nexu communi patris et filii.


&

ゆえに、受肉した息子によって、本性的に神の息子となることの類似に向けて、
 養子によって息子となることを私たちが受けたように、受肉した父からは、
 本性的に息子となることの根源からであるようにして、養子によって息子とな
 ることを私たちは受けただろうし、聖霊からは、父と子の共通の結びつきから
 のようにして、それを受けたであろう。


\\



{\scshape Ad tertium dicendum} quod patri convenit esse innascibilem secundum
nativitatem aeternam, quod non excluderet nativitas temporalis. Mitti
autem dicitur filius secundum incarnationem, eo quod est ab illo, sine
quo incarnatio non sufficeret ad rationem missionis.


&

第三異論に対しては、以下のように言われるべきである。
父に、生み出されえないことが適合するのは、永遠の出生に即してであるが、それ
 は、時間的な出生を排除しない。また、息子が遣わされると言われるのは、受
 肉に即してであるが、それは、彼のものからだからである。もしそれがなけれ
 ば、受肉は、派遣という性格に十分でなかっただろう。



\end{longtable}
\newpage



\rhead{a.~6}
\begin{center}
 {\Large {\bf ARTICULUS SEXTUS}}\\
 {\large UTRUM DUAE PERSONAE DIVINAE POSSINT ASSUMERE\\UNAM ET EADEM
 NUMERO NATURAM}\\
 {\footnotesize III {\itshape Sent.}, d.1, q.2, a.4.}\\
 {\Large 第六項\\神の二つのペルソナが数的に同一の本性を受容できるか}
\end{center}

\begin{longtable}{p{21em}p{21em}}




{\Huge A}{\scshape d sextum sic proceditur}. Videtur quod duae personae divinae non possunt
assumere unam et eandem numero naturam. Hoc enim supposito aut essent
unus homo, vel plures. Sed non plures, sicut enim una natura divina in
pluribus personis non patitur esse plures deos, ita una humana natura in
pluribus personis non patitur esse plures homines. Similiter etiam non
 possent esse unus homo, quia unus homo est {\itshape iste homo}, qui demonstrat
unam personam; et sic tolleretur distinctio trium personarum divinarum,
quod est inconveniens. Non ergo duae aut tres personae possunt accipere
unam naturam humanam.


&


第六項の問題へ、議論は以下のように進められる。
二つの神のペルソナが数において同一の本性を受容することはできないと思われ
 る。理由は以下の通り。
もし受容することができたとすると、その結果生まれたのは、一人の人間か、ある
 いは複数の人間かのどちらかだっただろう。しかし複数ではない。なぜなら、
 複数のペルソナに一つの神の本性があっても、複数の神々がいることに
 ならないように、複数のペルソナに一つの人間本性があっても、複数の人間が
 いることにならないからである。同様にまた、一人の人間でもありえなかっただろう。な
 ぜなら、一人の人間は「この人間」であり、それは、一つのペルソナを指示す
 る。そうすると、神の三つのペルソナの区別がなくなり、不適切である。ゆえ
 に、二つないし三つのペルソナが、一つの人間本性を受け取ることはできない。

\\



2. {\scshape Praeterea}, assumptio terminatur ad unitatem personae, ut dictum est. Sed
non est una persona patris et filii et spiritus sancti. Ergo non possunt
tres personae assumere unam naturam humanam.


&

さらに、すでに述べられたとおり、受容は、ペルソナの一性に終極する。
しかし、父と子と聖霊に属する一つのペルソナはない。
ゆえに、三つのペルソナが一つの人間本性を受容することはできない。

\\



3. {\scshape Praeterea}, Damascenus dicit, in III libro, et Augustinus, in I {\itshape
 de Trin}.,
quod ex incarnatione filii Dei consequitur quod quidquid dicitur de
filio Dei, dicitur de filio hominis, et e converso. Si ergo tres
personae assumerent unam naturam humanam, sequitur quod quidquid dicitur
de qualibet trium personarum, diceretur de illo homine, et e converso ea
quae dicerentur de illo homine, possent dici de qualibet trium
personarum. Sic ergo id quod est proprium patris, scilicet generare
filium ab aeterno, diceretur de illo homine, et per consequens diceretur
de filio Dei, quod est inconveniens. Non ergo est possibile quod tres
personae divinae assumant unam naturam humanam.


&

さらに、ダマスケヌスは第三巻で、また、アウグスティヌスは『三位一体論』第
 1巻で述べているように、神の息子の受肉から、何であれ神の息子について言わ
 れることは人間の息子について言われ、逆もまたそうであるということが帰結
 する。ゆえに、もし、三つのペルソナが一つの人間本性を受容したならば、三
 つのペルソナにどれについて言われたことも、かの人間についても言われ、逆
 に、かの人間について言われたことが、三つのペルソナのどれについても言わ
 れえたであろう。かくして、父に固有のこと、すなわち、永遠から息子を生むこ
 とが、かの人間について言われ、結果的に、神の息子について言われたであろ
 うが、これは不都合である。ゆえに、三つの神のペルソナが一つの人間本性を
 受容することは不可能である。


\\



{\scshape Sed contra}, persona incarnata subsistit in duabus naturis, divina
scilicet et humana. Sed tres personae possunt subsistere in una natura
divina. Ergo etiam possunt subsistere in una natura humana, ita scilicet
quod sit una natura humana a tribus personis assumpta.


&

反対に、受肉したペルソナは、神と人間の二つの本性において自存する。しかし、
 三つのペルソナが、一つの神の本性において自存することが可能である。ゆえ
 に、一つの人間本性において自存することも可能である。こうして、一つの人
 間本性が三つのペルソナによって受容された。

\\



{\scshape Respondeo} dicendum quod, sicut supra dictum est, ex unione animae et
corporis in Christo non fit neque nova persona neque hypostasis, sed fit
una natura assumpta in personam vel hypostasim divinam. 

&

解答する。以下のように言われるべきである。
前に述べられたとおり、魂と身体から、キリストにおいて、新しいペルソナやヒュ
 ポスタシスは生じず、一つの本性が、神のペルソナないしヒュポスタシスに受
 容されるということが生じる。

\\


Quod quidem non
fit per potentiam naturae humanae, sed per potentiam personae
divinae. 


&

このことは、人間の本性の能力によってではなく、神のペルソナの能力による。

\\


Est autem talis divinarum personarum conditio quod una earum
non excludit aliam a communione eiusdem naturae, sed solum a communione
eiusdem personae. 

&

しかし、神の諸々のペルソナの条件は次のようである。すなわち、そのうちの一
 つのペルソナが、他のペルソナを、同一の本性の共有から排除することはない。
しかし、同一のペルソナの共有からは、他のペルソナを排除する。


\\


Quia igitur in mysterio incarnationis {\itshape tota ratio facti
est potentia facientis}, ut Augustinus dicit, in epistola {\itshape ad Volusianum};
magis est circa hoc iudicandum secundum conditionem personae assumentis
quam secundum conditionem naturae humanae assumptae. 


&


ゆえに、受肉の秘蹟において、アウグスティヌスが『ウォルシアヌス宛』書簡で
 述べるように、「なされたものの全根拠はなすものの能力である」ので、これ
 については、受容された人間本性の条件に即してよりは、受容するペルソナ
 の条件に即して判断されるべきである。

\\


Sic igitur non est
impossibile divinis personis ut duae vel tres assumant unam naturam
humanam. 




&

そうすると、二つないし三つのものとしての神のペルソナにとって、一つの人間
 本性を受容することは不可能でない。

\\


Esset tamen impossibile ut assumerent unam hypostasim vel unam
personam humanam, sicut Anselmus dicit, in libro {\itshape De Conceptu Virginali},
quod {\itshape plures personae non possunt assumere unum eundemque hominem}.


&

しかし、アンセルムスが『処女懐胎について』で「複数のペルソナが同一
 の人間を受容することはできない」と言うように、一つの人間のヒュポスタシ
 スないしペルソナを受容することは不可能だっただろう。


\\



{\scshape Ad primum ergo dicendum} quod, hac positione facta, quod scilicet tres
personae assumerent unam humanam naturam, verum esset dicere quod tres
personae essent unus homo, propter unam humanam naturam, sicut nunc
verum est dicere quod sunt unus Deus, propter unam divinam naturam. Nec
ly {\itshape unus} importat unitatem personae, sed unitatem in natura humana. Non
enim posset argui ex hoc quod tres personae sunt unus homo, quod essent
unus simpliciter, nihil enim prohibet dicere quod homines qui sunt
plures simpliciter, sint unus quantum ad aliquid, puta unus populus;
sicut Augustinus dicit, VI {\itshape de Trin}., {\itshape diversum est natura spiritus
hominis et spiritus Dei, sed inhaerendo fit unus spiritus, secundum
illud I Cor. VI, qui adhaeret Deo, unus spiritus est}.


&


第一異論に対しては、それゆえ、以下のように言われるべきである。
この仮定が事実だったならば、すなわち、三つのペルソナが一つの人間本性を受容したと
 すると、三つのペルソナが一つの人間であると言うことは、一つの人間本性の
 ために、真であっただろう。ちょうどいま、神の一つの本性のために、一つの
 神と言うのが真であるように。しかしこの「一つ」は、ペルソナの一性を意味
 せず、人間本性における一性を意味する。じっさい、三つのペルソナが一人の
 人間であるということから、端的に一人であると論じることはできないであろ
 う。なぜなら、端的に複数である人間たちが、ある点にかんして、たとえば一
 つの国民であるという点で、一つであることは差し支えないからである。
ちょうど、アウグスティヌスが『三位一体論』第6巻で「人間の霊と神の霊は本
 性上異なるが、内在することによって、一つの霊となる」と言うように。これ
 は、『コリントの信徒への手紙一』第6章「神に固着する人は、一つの霊である」\footnote{「しかし、主に結び付く者は主と一つの霊となるのです。」(6:17)} による。



\\



{\scshape Ad secundum dicendum} quod, illa positione facta, humana natura esset
assumpta in unitate non unius personae, sed in unitate singularum
personarum, ita scilicet quod, sicut divina natura habet naturalem
unitatem cum singulis personis, ita natura humana haberet unitatem cum
singulis per assumptionem.


&

第二異論に対しては、以下のように言われるべきである。
この仮定が事実だったならば、人間本性は、一つのペルソナの一性においてではなく、個々
 のペルソナの一性において受容されたであろう。すなわち、ちょうど、神の本
 性が、個々のペルソナと共に、本性的な一性を持つように、人間本性は、受容
 によって、個々のものと共に一性を持ったであろう。

\\



{\scshape Ad tertium dicendum} quod circa mysterium incarnationis fuit communicatio
proprietatum pertinentium ad naturam, quia quaecumque conveniunt
naturae, possunt praedicari de persona subsistente in natura illa,
cuiuscumque naturae nomine significetur. 

&

第三異論に対しては、以下のように言われるべきである。
受肉の秘蹟を巡っては、本性に属する固有性の伝達があった。なぜなら、本性に
 適合するものは何であれ、どの本性の名で意味されようとも、その本性において自存するペルソナについて述語さ
 れうるからである。



\\

Praedicta ergo positione facta,
de persona patris poterunt praedicari et ea quae sunt humanae naturae,
et ea quae sunt divinae, et similiter de persona filii et spiritus
sancti. 


&

ゆえに、前述の仮定が事実ならば、父のペルソナについて、人間本性に属するこ
 とと、神の本性に属することが述語されえただろうし、息子と聖霊のペルソナ
 についても同様だっただろう。

\\


Non autem illud quod conveniret personae patris ratione propriae
personae, posset attribui personae filii aut spiritus sancti, propter
distinctionem personarum, quae remaneret. 


&

しかし、ペルソナ固有の性格において父のペルソナに適合するものは、残存する
 ペルソナの区別のために、息子と聖霊のペルソナに帰せられることはできなかっ
 ただろう。

\\


Posset ergo dici quod, sicut
pater est ingenitus, ita homo esset ingenitus, secundum quod ly {\itshape homo}
supponeret pro persona patris. Si quis autem ulterius procederet, {\itshape Homo
est ingenitus, Filius est homo, ergo Filius est ingenitus}, esset
fallacia figurae dictionis vel accidentis. 


&

ゆえに、「人間」が父のペルソナを示すかぎり、父が不生であるように、人間も
 不生であると言われえた。しかし、もし、さらに、「人間は不生である。息子は
 人間である。ゆえに、息子は不生である」と言う人がいたら、それは、語形によ
 る誤謬、ないし附帯性の誤謬であっただろう。

\\


Sicut et nunc dicimus Deum
esse ingenitum, quia pater est ingenitus, nec tamen possumus concludere
quod filius sit ingenitus, quamvis sit Deus.


&


それはちょうど、今私たちが、父が不生であるために「神は不生である」と言
 うが、息子が神だからといって、息子が不生であると結論できないのと同じで
 ある。


\end{longtable}
\newpage


\rhead{a.~7}
\begin{center}
 {\Large {\bf ARTICULUS SEPTIMUS}}\\
 {\large UTRUM UNA PERSONA DIVINA POSSIT ASSUMERE\\ DUAS NATURAS HUMANAS}\\
 {\footnotesize III {\itshape Sent.}, d.1, q.2, a.5.}\\
 {\Large 第七項\\神の一つのペルソナが二つの人間本性を受容することができ
 るか}
\end{center}

\begin{longtable}{p{21em}p{21em}}


{\Huge A}{\scshape d septimum sic proceditur}. Videtur quod una persona divina non possit
assumere duas naturas humanas. Natura enim assumpta in mysterio
incarnationis non habet aliud suppositum praeter suppositum personae
divinae, ut ex supra dictis patet. Si ergo ponatur esse una persona
divina assumens duas humanas naturas, esset unum suppositum duarum
naturarum eiusdem speciei. Quod videtur implicare contradictionem, non
enim natura unius speciei multiplicatur nisi secundum distinctionem
suppositorum.


&

第七項の問題へ、議論は以下のように進められる。
神の一つのペルソナが、二つの人間本性を受容することはできないと思われる。
 理由は以下の通り。
 受肉の秘蹟において受容された本性は、前に述べられたことから明らかなとお
 り\footnote{Q.2, a.3, a.6.}、神のペルソナという個体以外の個体を持たない。
 ゆえに、二つの人間本性を受容する一つの神のペルソナがあると想定されるな
 らば、同一の種に属する二つの本性をもつ一つの個体があることになるだろう。
 これは、矛盾を含意しているように思われる。なぜなら、一つの種の本性は、
 個体の区別に即してのみ多数化されるからである。


\\



2. {\scshape Praeterea}, hac suppositione facta, non posset dici quod persona divina
incarnata esset unus homo, quia non haberet unam naturam
humanam. Similiter etiam non posset dici quod essent plures homines,
quia plures homines sunt supposito distincti, et ibi esset unum tantum
suppositum. Ergo praedicta positio esset omnino impossibilis.


&

さらに、この想定が事実だったとすると、受肉した神のペルソナは一人の人間で
 あると言われえなかっただろう。なぜなら、一つの人間本性を持っていないか
 らである。同様に、複数の人間であるとも言われえなかっただろう。なぜなら、
 複数の人間は、個体において区別されるが、この場合には、一つの個体しかな
 いからである。ゆえに、上述の想定は、まったく不可能だろう。

\\



3. {\scshape Praeterea}, in incarnationis mysterio tota divina natura est unita toti
naturae assumptae, idest cuilibet parti eius, est enim Christus
{\itshape perfectus Deus et perfectus homo, totus Deus et totus homo}, ut
Damascenus dicit, in III libro. Sed duae humanae naturae non possent
totaliter sibi invicem uniri, quia oporteret quod anima unius esset
unita corpori alterius, et quod etiam duo corpora essent simul, quod
etiam confusionem induceret naturarum. Non ergo est possibile quod
persona divina duas humanas naturas assumeret.


&


さらに、受肉の秘蹟において、神の全本性が、受容された全本性に、すなわち、
 そのどの部分にも合一した。これは、ダマスケヌスが第三巻で、キリストは
 「完全な神でありかつ完全な人間、まったき神でありかつまったき人間」と述
 べていることによる。しかし、二つの人間本性は、全体的に、相互に合一され
 えない。なぜなら、一方の魂が、他方の身体に合一し、さらに、二つの身体が
 同時にあることになるが、これは、さらなる本性の混乱をもたらすであろう。
 ゆえに、神のペルソナが二つの人間本性を受容することは不可能である。

\\



{\scshape Sed contra est} quod quidquid potest Pater, potest Filius. Sed Pater post
incarnationem Filii, potest assumere naturam humanam aliam numero ab ea
quam Filius assumpsit, in nullo enim per incarnationem Filii est
diminuta potentia Patris vel Filii. Ergo videtur quod Filius, post
incarnationem, possit aliam humanam naturam assumere, praeter eam quam
assumpsit.


&

しかし反対に、父ができることは何でも、息子もできる。しかし、父は、息子の
 受肉の後、息子が受容した人間本性とは数的に異なる人間本性を受容すること
 ができる。息子の受肉によって、父の能力であれ息子の能力であれ、どこにお
 いても能力が減ることはないからである。ゆえに、息子は、受肉の後、受肉し
 た人間本性とは別の人間本性を受容することができる。

\\



{\scshape Respondeo dicendum} quod id quod potest in unum et non in amplius, habet
potentiam limitatam ad unum. Potentia autem divinae personae est
infinita, nec potest limitari ad aliquid creatum. Unde non est dicendum
quod persona divina ita assumpserit unam naturam humanam quod non
potuerit assumere aliam. 


&


解答する。以下のように言われるべきである。
一つのことができるが、それ以上のことができないものは、一つのものへ限られ
 た能力をもつ。しかし、神のペルソナの能力は無限であり、何らかの被造物へ
 限られることはありえない。したがって、神のペルソナが、他の本性は受容で
 きないというかたちで、一つの人間本性を受容したと言われるべきではない。

\\


Videretur enim ex hoc sequi quod personalitas
divinae naturae esset ita comprehensa per unam humanam naturam quod ad
eius personalitatem alia assumi non possit. Quod est impossibile, non
enim increatum a creato comprehendi potest. 


&

というのも、もしそうならば、そのことから、神の本性のペルソナ性が、一つの
 人間本性によって、そのペルソナ性へ、他の本性が受容されえないかた
 ちで、包含されていることになると思われたであろう。これは不可能である。なぜなら、
 創造されないものが、創造されたものに含まれることはできないからである。



\\

Patet ergo quod, sive
consideremus personam divinam secundum virtutem, quae est principium
unionis; sive secundum suam personalitatem, quae est terminus unionis,
oportet dicere quod persona divina, praeter naturam humanam quam
assumpsit possit aliam numero naturam humanam assumere.


&


ゆえに、神のペルソナを、合一の始原である力の点で考察しても、あるいは、合
 一の終極であるそのペルソナ性の点で考察しても、神のペルソナが、受容した
 人間本性意外の、数的に異なる他の人間本性を授与することが可能だと言わな
 ければならない。

\\



{\scshape Ad primum ergo dicendum} quod natura creata perficitur in sua ratione per
formam, quae multiplicatur secundum divisionem materiae. Et ideo, si
compositio formae et materiae constituat novum suppositum, consequens
est quod natura multiplicetur secundum multiplicationem
suppositorum. Sed in mysterio incarnationis unio formae et materiae,
idest animae et corporis, non constituit novum suppositum, ut supra
dictum est. Et ideo posset esse multitudo secundum numerum ex parte
naturae, propter divisionem materiae, absque distinctione suppositorum.


&


第一異論に対しては、それゆえ、以下のように言われるべきである。
被造の本性は、その性格において、形相によって完成される。そしてその形相は、
 質料の分割において多数化される。ゆえに、もし、形相と質料の複合が新しい
 個体を構成するならば、結果的に、本性が、個体の多数性に従って多数化され
 ることになる。しかし、受肉の秘蹟において、形相と質料、すなわち魂と身体
 の合一は、前に述べられたとおり\footnote{前項。Q.2, a.5, ad 1.}、新しい
 個体を構成しない。それゆえ、個体を区別することなく、質料の分割のために、
 本性の側から数的な多数化が可能であった。


\\



{\scshape Ad secundum dicendum quod} posset videri quod, praedicta positione facta,
consequeretur quod essent duo homines, propter duas naturas, absque hoc
quod essent ibi duo supposita, sicut e converso tres personae dicerentur
unus homo, propter unam naturam humanam assumptam, ut supra dictum
est. 

&

第二異論に対しては、以下のように言われるべきである。
前述の想定が実現すれば、そこに二つの個体があるわけではないのに、二つの本
 性のために、二人の人間がいることになると思われるかも知れない。ちょうど逆に、前に述べられたと
 おり\footnote{前項、ad 1.}、三つのペルソナが、
 受容した一つの人間本性のために、一人の人間と言われるであろうように。


\\

Sed hoc non videtur esse verum. Quia nominibus est utendum secundum
quod sunt ad significandum imposita. Quod quidem est ex consideratione
eorum quae apud nos sunt. Et ideo oportet, circa modum significandi et
consignificandi, considerare ea quae apud nos sunt. 


&

しかしこれが真だとは思われない。なぜなら、名称は、それを表示するために付
 けられたところにしたがって用いられるべきだからである。これは、私たちの
 もとにある事柄の考察に基づいてある。ゆえに、表示と共表示のしかたにつ
 いて、私たちのもとにある事柄を考察しなければならない。


\\



In quibus nunquam
nomen ab aliqua forma impositum pluraliter dicitur nisi propter
pluralitatem suppositorum, homo enim qui est duobus vestimentis indutus,
non dicitur duo vestiti, sed {\itshape unus vestitus duobus vestimentis}; et qui
habet duas qualitates, dicitur singulariter {\itshape aliqualis secundum duas
qualitates}. 


&

私たちのもとにある事柄において、ある形相から付けられた名称は、個体の複数
 性のためにでなければ、複数で語られない。たとえば、二着の服を着た人は、
 二人の服を着た人とは言われず、「一人の、二着の服を着た人」と言われる。
 また、二つの性質を持つものは、単数で、「二つの性質において、ある(一つの)性質のも
 の」と言われる。

\\


Natura autem assumpta quantum ad aliquid se habet per modum
indumenti, licet non sit similitudo quantum ad omnia, ut supra dictum
est. 


&

ところで、受容された本性は、ある点において、衣服のようなあり方にある。す
 でに述べられたとおり\footnote{Q.2, a.6, ad 1.}、あらゆる点で類似しているわけではないけれども。

\\


Et ideo, si persona divina assumeret duas naturas humanas, propter
unitatem suppositi diceretur {\itshape unus homo habens duas naturas
humanas}. 



&

ゆえに、もし神のペルソナが二つの人間本性を受容したとすれば、個体の一性の
 ために、「二つの人間本性を持つ一人の人間」と言われたであろう。

\\


Contingit autem quod plures homines dicuntur unus populus,
propter hoc quod conveniunt in aliquo uno, non autem propter unitatem
suppositi. Et similiter, si duae personae divinae assumerent unam numero
humanam naturam, dicerentur unus homo, ut supra dictum est, non propter
unitatem suppositi, sed inquantum conveniunt in aliquo uno.


&

しかし、複数の人間が一つの国民と言われることはありえる。それは、何らかの
 一つのものにおいて一致するためにであって、個体の一性のためではない。同
 様に、もし二つの神のペルソナが、数的に一つの人間本性を受容したならば、
 すでに述べられたとおり\footnote{前項、ad 1.}、個体の一性のためにではなく、何らかの一つのもの
 における一致のために、一人の人と言われたであろう。




\\



{\scshape Ad tertium dicendum} quod divina et humana natura non eodem ordine se
habent ad unam divinam personam, sed per prius comparatur ad ipsam
divina natura, utpote quae est unum cum ea ab aeterno; sed natura humana
comparatur ad personam divinam per posterius, utpote assumpta ex tempore
a divina persona, non quidem ad hoc quod natura sit ipsa persona, sed
quod persona in natura subsistat; Filius enim Dei est sua deitas, sed non
est sua humanitas. 

&

第三異論に対しては、以下のように言われるべきである。
神と人間の本性は、同一の秩序において、一つの神のペルソナに関係せず、より
 先に、神の本性が、永遠からそれと一つであるものとして、関係する。しかし、
 人間本性は、神のペルソナへ、より後に、時間的に、神のペルソナによって受
 容されたものとして関係する。本性がペルソナそれ自体だというのではなく、
 ペルソナが、本性において自存するためにである。じっさい、神の息子は、自ら
 の神性であるが、自らの人間性ではない。


\\

Et ideo ad hoc quod natura humana assumatur a divina
persona, relinquitur quod divina natura unione personali uniatur toti
naturae assumptae, idest secundum omnes partes eius. Sed duarum
naturarum assumptarum esset uniformis habitudo ad personam divinam, nec
una assumeret aliam.
Unde non oporteret quod una earum totaliter alteri
uniretur, idest, omnes partes unius omnibus partibus alterius.


&


ゆえに、人間本性が神のペルソナによって受容されるために、神の本性が、ペル
 ソナ的合一によって、受容された本性全体に、すなわちそれのあらゆる部分に
 おいて合一されることが残される。しかし、受容された二つの本性には、神の
 ペルソナへの一様の関係があり、一方が他方を受容することはない。したがっ
 て、それらの一方が、全体的に他のものに合一される、すなわち、一方の全て
 の部分が、他方の全ての部分に合一される必要はない。





\end{longtable}
\newpage



\rhead{a.~8}
\begin{center}
 {\Large {\bf ARTICULUS OCTAVUS}}\\
 {\large UTRUM FUERIT MAGIS CONENIENS FILIUM DEI INCARNARI\\QUAM PATREM
 VEL SPRITUM SANCTUM}\\
 {\footnotesize III {\itshape Sent.}, d.1, q.2, a.2; IV {\itshape SCG}, cap.42.}\\
 {\Large 第八項\\父や聖霊よりも息子が受肉することがより適切だったか}
\end{center}

\begin{longtable}{p{21em}p{21em}}

{\Huge A}{\scshape d octavum sic proceditur}. Videtur quod non fuerit magis conveniens
filium Dei incarnari quam patrem vel spiritum sanctum. Per mysterium
enim incarnationis homines ad veram Dei cognitionem sunt perducti,
secundum illud Ioan.~{\scshape xviii}, {\itshape in hoc natus sum, et ad hoc veni in mundum,
ut testimonium perhiberem veritati}.


 &

第八項の問題へ、議論は以下のように進められる。
父や聖霊よりも、神の息子が受肉する方が、より適切だったわけではないと思わ
 れる。理由は以下の通り。
『ヨハネによる福音書』第18章「真理に証言をするために、私はここで生
 まれ、この世界に来た」\footnote{「そこでピラトが、「それでは、やはり王なのか」と言うと、イエスはお答えになった。「わたしが王だとは、あなたが言っていることです。わたしは真理について証しをするために生まれ、そのためにこの世に来た。真理に属する人は皆、わたしの声を聞く。」(18:37)}によれば、
受肉の秘蹟によって、人間は、神についての真の認識へ導かれる。


 \\

Sed ex hoc quod persona filii Dei
est incarnata, multi impediti fuerunt a vera Dei cognitione, ea quae
dicuntur de filio secundum humanam naturam referentes ad ipsam filii
personam, sicut Arius, qui posuit inaequalitatem personarum propter hoc
quod dicitur Ioan.~{\scshape xiv},{\itshape  Pater maior me est}, qui quidem error non
provenisset si persona Patris incarnata fuisset; nullus enim
existimasset Patrem Filio minorem. Magis ergo videtur conveniens fuisse
quod persona Patris incarnaretur quam persona Filii.

&

しかし、神の息子が受肉したことによって、人間本性に即して息子について言わ
 れることを、息子のペルソナ自体へ関係させることで、多くの人々が神の真理から妨げられ
 た。たとえばアリウスは、『ヨハネによる福音書』第14章「父は私より大きい」\footnote{「『わたしは去って行くが、また、あなたがたのところへ戻って来る』と言ったのをあなたがたは聞いた。わたしを愛しているなら、わたしが父のもとに行くのを喜んでくれるはずだ。父はわたしよりも偉大な方だからである。」(14:28)}
 と言われることのために、ペルソナ間の不等性を論じた。実際、この誤謬は、
 もし、父のペルソナが受肉していたら起きなかったであろう。なぜなら、父が
 子より小さいと考える人はいないからである。ゆえに、息子のペルソナよりも
 父のペルソナが受肉した方が適切であったと思われる。

\\



2. {\scshape Praeterea}, incarnationis effectus videtur esse recreatio quaedam humanae
naturae, secundum illud {\itshape Galat}.~ult., {\itshape in Christo Iesu neque circumcisio
aliquid valet neque praeputium, sed nova creatura}. Sed potentia creandi
appropriatur Patri. Ergo magis decuisset patrem incarnari quam filium.

&

さらに、『ガラテヤの信徒への手紙』最終章の、かの「イエス・キリストにおい
 て、割礼や包皮ではなく、新たな被造物が重要である」\footnote{「割礼の有
 無は問題ではなく、大切なのは、新しく創造されることです。」(6:15)}によれ
 ば、受肉の結果は、人間本性のある種の再創造だと思われる。しかし、創造の
 能力は、父に固有である。ゆえに、息子よりも父が受肉する方がふさわしかっ
 た。

\\



3. {\scshape Praeterea}, incarnatio ordinatur ad remissionem peccatorum, secundum
illud Matth.~{\scshape i}, {\itshape vocabis nomen eius Iesum, ipse enim salvum faciet
populum suum a peccatis eorum}. Remissio autem peccatorum attribuitur
spiritui sancto, secundum illud Ioan.~{\scshape xx}, {\itshape accipite spiritum sanctum,
quorum remiseritis peccata, remittentur eisi}. Ergo magis congruebat
personam spiritus sancti incarnari quam personam filii.

&


さらに、『マタイによる福音書』第1章「あなたは彼の名をイエスと呼びなさい。
 なぜなら彼は、自分の民を彼らの罪から救うであろう」\footnote{「マリアは
 男の子を産む。その子をイエスと名付けなさい。この子は自分の民を罪から救
 うからである。」(1:21)}によれば、受肉は罪の救いへと秩序付けられている。
 しかし、『ヨハネによる福音書』第20章「聖霊を受けよ。あなたたちが
 赦せば、彼らに赦される」\footnote{「そう言ってから、彼らに息を吹きかけ
 て言われた。「聖霊を受けなさい。だれの罪でも、あなたがたが赦せば、その
 罪は赦される。だれの罪でも、あなたがたが赦さなければ、赦されないまま残
 る。」」(20:22-23)}によれば、罪の許しは聖霊に帰せられる。ゆえに、息子の
 ペルソナではなく、聖霊のペルソナが受肉する方が適切であった。

\\



{\scshape Sed contra est} quod Damascenus dicit, in III libro, {\itshape in mysterio
incarnationis manifestata est sapientia et virtus Dei, sapientia quidem,
quia invenit difficillimi solutionem pretii valde decentissimam; virtus
autem, quia victum fecit rursus victorem}. Sed virtus et sapientia
appropriantur Filio, secundum illud I Cor.~{\scshape i}, {\itshape Christum Dei virtutem et
Dei sapientiam}. Ergo conveniens fuit personam Filii incarnari.

&

しかし反対に、ダマスケヌスは第三巻で次のように述べている。「受肉の秘蹟に
 おいて、神の知恵と力が明示される。知恵というのは、もっとも困難な対価の
 この上なく素晴らしい解放を見出したからであり、また、力というのは、敗者
 を再び勝者にしたからである」。しかし、力と知恵は、『コリントの信徒への
 手紙一』第1章「キリストは神の力であり神の知恵である」\footnote{「ユダヤ人であろうがギリシア人であろうが、召された者には、神の力、神の知恵であるキリストを宣べ伝えているのです。」(1:24)}によれば、キリストに固
 有である。ゆえに、息子のペルソナが受肉することが適切であった。

\\



{\scshape Respondeo dicendum} quod convenientissimum fuit personam Filii
incarnari. Primo quidem, ex parte unionis. Convenienter enim ea quae
sunt similia, uniuntur. Ipsius autem personae Filii, qui est verbum Dei,
attenditur, uno quidem modo, communis convenientia ad totam
creaturam. 

&


解答する。以下のように言われるべきである。
息子のペルソナが受肉したことはこの上なく適切であった。理由は以下の通り。
第一に、合一の側からである。
実際、似たものが合一されることが適している。ところで、息子のペルソナには、
 神の言葉だが、一つには、全被造物への共通の一致が見出される。


\\

Quia verbum artificis, idest conceptus eius, est similitudo
exemplaris eorum quae ab artifice fiunt. Unde verbum Dei, quod est
aeternus conceptus eius, est similitudo exemplaris totius creaturae. Et
ideo, sicut per participationem huius similitudinis creaturae sunt in
propriis speciebus institutae, sed mobiliter; ita per unionem verbi ad
creaturam non participativam sed personalem, conveniens fuit reparari
creaturam in ordine ad aeternam et immobilem perfectionem, nam et
artifex per formam artis conceptam qua artificiatum condidit, ipsum, si
collapsum fuerit, restaurat. 

&


なぜなら、技術者の言葉、すなわち彼の概念は、技術者に作られるものどもの範
 型である。ゆえに、神の言葉、その永遠の概念は、全被造物の範型的類似であ
 る。ゆえに、この類似を分有することによって、被造物はそれ自身の種におい
 て、、可動てきにだが、設置されるように、分有的ではなくペルソナ的な、言葉
 の被造物への合一によって、被造物が永遠で不動の完全性へ秩序付けられるこ
 とが準備されることが、適切だった。すなわち、技術者もまた、それによって
 技術作品を作るところの、捉えられた技術の形相によって、作品が壊れても、
 それをもう一度作りなおすように。
 ら


\\

Alio modo, habet convenientiam specialiter
cum humana natura, ex eo quod verbum est conceptus aeternae sapientiae a
qua omnis sapientia hominum derivatur. Et ideo homo per hoc in sapientia
proficit, quae est propria eius perfectio prout est rationalis, quod
participat verbum Dei, sicut discipulus instruitur per hoc quod recipit
verbum magistri. Unde et {\itshape Eccli}.~{\scshape i} dicitur, {\itshape fons sapientiae verbum Dei in
excelsis}. 
Et ideo, ad consummatam hominis perfectionem, conveniens fuit
ut ipsum verbum Dei humanae naturae personaliter uniretur. 

&

もう一つには、人間本性との適合性をとくに持つからである。つまりそれは、言
 葉が、人間の全ての知恵がそこに由来する、永遠の知恵の概念であることに基
 づく。
ゆえに、人間は、神の言葉を分有することによって、理性的であるかぎりでの人
 間固有の完成である知恵において、前進する。それはちょうど、生徒が、教師
 の言葉を受け取ることによって教育されるのと同様である。このことから、
 『集会の書』第1章で「知恵の泉は、高いところにいる神の言葉」
 \footnote{「知恵の泉は、いと高き所にいます神の言葉、/知恵の歩みは、永
 遠の掟。」(1:5) }と言われている。
それゆえ、人間の最高の完全性のために、神の言葉自体が人間本性にペルソナ的
 に合一されることが適切であった。


\\



Secundo
potest accipi ratio huius congruentiae ex fine unionis, qui est impletio
praedestinationis, eorum scilicet qui praeordinati sunt ad hereditatem
caelestem, quae non debetur nisi filiis, secundum illud {\itshape
 Rom}.~{\scshape viii}, {\itshape Filii et heredes}. 


&

第二に、この適切さの根拠を、合一の目的から理解できる。その目的とは、天の遺産へ予め秩序付け
 られた人たちの予定の成就であり、その遺産を受け取る権利は、『ローマの信徒への手紙』
 第8章「息子たち、相続人たち」\footnote{「もし子供であれば、相続人でもありま
 す。神の相続人、しかもキリストと共同の相続人です。キリストと共に苦しむ
 なら、共にその栄光をも受けるからです。」(8:17)}によれば、まさに息子たち
 にある。



\\

Et ideo congruum fuit ut per eum qui est filius naturalis,
homines participarent similitudinem huius filiationis secundum
adoptionem, sicut apostolus ibidem dicit, {\itshape quos praescivit et
praedestinavit conformes fieri imagini filii eius}. 

&


ゆえに、自然的な息子である人によって、人々が、養子とすることにおけるこの
 息子性の類似を分有することが、適切であった。ちょうど、使徒が、同じ箇所
 で「予め知っていた人々を、彼の息子の像に一致するように予定した」
 \footnote{「神は前もって知っておられた者たちを、御子の姿に似たものにし
 ようとあらかじめ定められました。それは、御子が多くの兄弟の中で長子とな
 られるためです。」(8:29)}と言うように。



\\


Tertio potest accipi
ratio huius congruentiae ex peccato primi parentis, cui per
incarnationem remedium adhibetur. 
Peccavit enim primus homo appetendo
scientiam, ut patet ex verbis serpentis promittentis homini {\itshape scientiam
boni et mali}. Unde conveniens fuit ut per verbum verae sapientiae homo
reduceretur in Deum, qui per inordinatum appetitum scientiae recesserat
a Deo.

&

第三に、この適切さの根拠は、受肉によって救済策が与えられる、第一の両親の
 罪から理解できる。第一の人間は、知恵を欲求することによって罪を犯した。
 これは、人間に「善と悪の知識」\footnote{「それを食べると、目が開け、神
 のように善悪を知るものとなることを神はご存じなのだ。」『創世記』(3:5)}を与えるヘビの言葉から明らかである。した
 がって、真の知恵の言葉によって、秩序を外れた知恵への欲求によって神から
 離れた人間が、神へ戻るが適切であった。


\\



{\scshape Ad primum ergo dicendum} quod nihil est quo humana malitia non posset
abuti, quando etiam ipsa Dei bonitate abutitur, secundum illud {\itshape Rom}.~{\scshape ii},
{\itshape an divitias bonitatis eius contemnis?} Unde et, si persona Patris fuisset
incarnata, potuisset ex hoc homo alicuius erroris occasionem assumere,
quasi Filius sufficere non potuisset ad humanam naturam reparandam.

&

第一異論に対しては、それゆえ、以下のように言われるべきである。
人間の悪意が悪用できないものはない。
『ローマの信徒への手紙』第2章「あなたは神の善性の豊かさを軽んじるのか」
 \footnote{「あるいは、神の憐れみがあなたを悔い改めに導くことも知らない
 で、その豊かな慈愛と寛容と忍耐とを軽んじるのですか。」(2:4)}によれば、
 神の善性自体さえも悪用するのだから。したがって、仮に父のペルソナが受肉
 したとしても、人間は、息子が人間本性の回復に十分でなかったかのように、何らかの
 誤りの機会を利用したであろう。

\\



{\scshape Ad secundum dicendum} quod prima rerum creatio facta est a potentia Dei
Patris per Verbum. Unde et recreatio per Verbum fieri debuit a potentia
Dei patris, ut recreatio creationi responderet, secundum illud II
{\itshape Cor}.{\scshape v}, {\itshape Deus erat in Christo mundum reconcilians sibi}.

&

第二異論に対しては、以下のように言われるべきである。
諸事物の第一の創造は、父である神の能力によって、言葉をとおしてなされた。
 したがって、『コリントの信徒への手紙二』第5章\footnote{「つまり、神はキ
 リストによって世を御自分と和解させ、人々の罪の責任を問うことなく、和解
 の言葉をわたしたちにゆだねられたのです。」(5:19)} によれば、再創造が創
 造に対応するように、言葉による再創造も、父である神の能力によってなされ
 る必要があった。

\\



{\scshape Ad tertium dicendum} quod spiritus sancti proprium est quod sit donum
Patris et Filii. Remissio autem peccatorum fit per Spiritum Sanctum
tanquam per donum Dei. Et ideo convenientius fuit ad iustificationem
hominum quod incarnaretur Filius, cuius Spiritus Sanctus est donum.

&

第三異論に対しては、以下のように言われるべきである。
聖書には、父と子の贈り物であることが固有である。しかし、罪の許しは、神の
 贈り物をとおしてであるかのようにして、聖霊によってなされる。ゆえに、
息子が受肉することが、人間の正当化には、より適切であった。聖霊は、息子の
 贈り物だからである。



\end{longtable}
\newpage



\end{document}


