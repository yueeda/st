\documentclass[10pt]{jsarticle} % use larger type; default would be 10pt
%\usepackage[utf8]{inputenc} % set input encoding (not needed with XeLaTeX)
%\usepackage[round,comma,authoryear]{natbib}
%\usepackage{nruby}
\usepackage{okumacro}
\usepackage{longtable}
%\usepqckage{tablefootnote}
\usepackage[polutonikogreek,english,japanese]{babel}
\usepackage{amsmath}
\usepackage{latexsym}
\usepackage{color}
%----- header -------
\usepackage{fancyhdr}
\pagestyle{fancy}
\lhead{{\it Summa Theologiae} I, q.~2}
%--------------------

\bibliographystyle{jplain}

\title{{\bf Prima Pars}\\{\HUGE Summae Theologiae}\\Sancti Thomae
Aquinatis\\Quaestio Secunda\\{\bf De Deo, An Deus Sit}}
\author{Japanese translation\\by Yoshinori {\sc Ueeda}}
\date{Last modified \today}


%%%% コピペ用
%\rhead{a.~}
%\begin{center}
% {\Large {\bf }}\\
% {\large }\\
% {\footnotesize }\\
% {\Large \\}
%\end{center}
%
%\begin{longtable}{p{21em}p{21em}}
%
%&
%
%\\
%\end{longtable}
%\newpage



\begin{document}
\maketitle
\begin{center}
 {\Large 第二問\\神について、神は存在するか}
\end{center}

\begin{longtable}{p{21em}p{21em}}
Quia igitur principalis intentio huius sacrae
doctrinae est Dei cognitionem tradere, et non solum secundum quod in se
est, sed etiam secundum quod est principium rerum et finis earum, et
specialiter rationalis creaturae, ut ex dictis est manifestum; ad huius
doctrinae expositionem intendentes, primo tractabimus de Deo; secundo,
de motu rationalis creaturae in Deum; tertio, de Christo, qui, secundum
quod homo, via est nobis tendendi in Deum. 

&

この聖なる教えの主要な目的は、神についての認識を伝えることであり、それも、
 たんに神それ自身においてだけでなく、神が諸事物の根源であり、また、諸事
 物の、とくに、すでに述べたことから明らかなように、理性的被造物の目的であ
 る限りにおいても、神の認識を伝えることを目的とするので、この教えを説明
 しようとする私たちは、第一に、神について、第二に、理性的被造物の神への
 運動について\footnote{第2部}、第三に、キリストについて\footnote{第3部}論じる。キリストは、彼が人間であ
 る限りにおいては、神へと向かう人間にとっての道だからである。

\\

Consideratio autem de Deo
tripartita erit. Primo namque considerabimus ea quae ad essentiam
divinam pertinent; secundo, ea quae pertinent ad distinctionem
personarum; tertio, ea quae pertinent ad processum creaturarum ab
ipso. 

&

神についての考察は三つの部分を持つ。第一に、私たちは神の本質に関する事柄
 を考察し、第二に、ペルソナの区別に関する事柄を\footnote{第27問以降}、そして第三に、被造物の
 神自身からの発出に関する事柄を\footnote{第44問以降。}考察する。

\\

Circa essentiam vero divinam, primo considerandum est an Deus sit;
secundo, quomodo sit, vel potius quomodo non sit; tertio considerandum
erit de his quae ad operationem ipsius pertinent, scilicet de scientia
et de voluntate et potentia. Circa primum quaeruntur tria. 

&

神の本質については、第一に、神が存在するかどうかを、第二に、神がどのよう
 に存在するかを、いや、むしろどのように存在していないかを\footnote{第3問
 以降。}考察すべきであ
 る。第三には、神の働きに属する事柄、すなわち、知、意志、能力について
 \footnote{第14問以降。}考
 察すべきである。
第一にかんして、三つのことが問われる。

\\

\begin{enumerate}
 \item utrum Deum esse sit per se notum. 
 \item utrum sit demonstrabile. 
 \item an Deus sit.
\end{enumerate}

&

\begin{enumerate}
 \item 神が存在することは、自明か。
 \item 神が存在することは、証明できるか。 
 \item 神は存在するか。
\end{enumerate}


\\
\end{longtable}



\newpage
\rhead{a.~1}

\begin{center}
 {\Large {\bf ARTICULUS PRIMUS}}\\
 {\large UTRUM  DEUM ESSE SIT PER SE NOTUM}\\
 {\Large 第一項\\神が存在することは自明か}\\
 {\footnotesize I {\it Sent.}, d.~3, q.~1, a.~2; I {\it SCG.}, c.~10, 11;
 III, c.~38; {\it De Verit.}, q.~10, a.~12; {\it De Pot.}, q.~7, a.~2,
 ad 11; in Psalm.~8; in Boet.~{\it de Trin.}, q.~1, a.~3, ad 6.}

\end{center}

\begin{longtable}{p{21em}p{21em}}

{\huge A}{\sc d primum sic proceditur}. Videtur quod Deum
esse sit per se notum. Illa enim nobis dicuntur per se nota, quorum
cognitio nobis naturaliter inest, sicut patet de primis principiis. Sed,
sicut dicit Damascenus in principio libri sui, {\it omnibus cognitio
existendi Deum naturaliter est inserta}. Ergo Deum esse est per se notum.


&

第一項の問題へ、議論は以下のように進められる。神が存在することは、自明で
 あると思われる。
理由は以下の通り。
自明なものとは、ちょうど論証の第一原理がそうであるように、それの認識が、
 私たちに自然本性的に内在するもののことである。ところが、ダマスケヌスが
 彼の著作の冒頭で述べているように、神が存在するという認識はすべての人に
 自然本性的に植え付けられている。ゆえに、神が存在することは自明である。


\\


2.~{\sc Praeterea}, illa dicuntur esse per se nota,
quae statim, cognitis terminis, cognoscuntur, quod philosophus attribuit
primis demonstrationis principiis, in I {\it Poster}.: scito enim quid est
totum et quid pars, statim scitur quod omne totum maius est sua
parte. 

&

さらに、用いられている言葉の意味がわかると直ちにわかるようなものが、自明
 なものと言われる。哲学者は『分析論後書』で、論証の第一原理がそのような
 性質だとしている。たとえば、「全体」「部分」という言葉の意味がわかれば、
 「全体はすべてその部分よりも大きい」ということが知られる。

\\

Sed intellecto quid significet hoc nomen Deus, statim habetur
quod Deus est. Significatur enim hoc nomine id quo maius significari non
potest, maius autem est quod est in re et intellectu, quam quod est in
intellectu tantum, unde cum, intellecto hoc nomine Deus, statim sit in
intellectu, sequitur etiam quod sit in re. Ergo Deum esse est per se
notum.


&
ところが、
 「神」というこの名が何を意味しているかが理解されるならば、神が存在する
 ことは直ちに了解される。理由は以下の通り。その名によって、「それ以上大きいもの
 が意味され得ないもの」が意味される。ところが、実在と理解
 において存在するものは、たんに理解において存在するものよりも大きい。従っ
 て、「神」というこの名が理解されるならば、直ちにそれは理解において存在
 するのだから、それが実在においても存在することが帰結する。ゆえに、神が
 存在することは自明である。

\\


3.~{\sc Praeterea}, veritatem esse est per se notum,
quia qui negat veritatem esse, concedit veritatem esse, si enim veritas
non est, verum est veritatem non esse. Si autem est aliquid verum,
oportet quod veritas sit. Deus autem est ipsa veritas, Ioann.~{\sc xiv}:
 {\it Ego sum via, veritas et vita}. Ergo Deum esse est per se notum.


&

さらに、真理が存在することは自明である。なぜなら、真理が存在することを否
 定する者でも、真理が存在することを認めているのである。なぜなら、もし真
 理が存在しないならば、真理が存在しないことが真だからである。しかるに、もしも何か
 が真であれば、真理が存在しなければならない。ところが『ヨハネによる福音
 書』第14章「私は道であり真理であり生命である」\footnote{「イエスは言われた。「私は道であり、真理であり、命である。私を通らなければ、誰も父のもとに行くことができない。 」(14:06)}によれば、神は真理
 そのものである。ゆえに、神が存在することは自明である。

\\


{\sc Sed contra}, nullus potest cogitare oppositum eius quod est per se
notum ut patet per philosophum, in IV {\it Metaphys}.~et I {\it
Poster}., circa prima demonstrationis principia. Cogitari autem potest
oppositum eius quod est Deum esse, secundum illud Psalmi {\sc lii}: {\it
Dixit insipiens in corde suo, non est Deus}. Ergo Deum esse non est per
se notum.


&
しかし反対に、『形而上学』第4巻および『分析論後書』第1巻の哲学者によって
 明らかなように、だれも自明なものの反対を考えることはできない。ところが、
 「神が存在する」の反対を考えることはできる。『詩編』52章に「愚かな者
 は心の中で神はいないと言った」\footnote{「神を知らぬ者は心に言う、「神
 などいない」と。」(53:2)}とあるからである。ゆえに、神が存在するこ
 とは、自明でない。

\\


{\sc Respondeo dicendum} quod contingit aliquid esse
per se notum dupliciter, uno modo, secundum se et non quoad nos; alio
modo, secundum se et quoad nos. Ex hoc enim aliqua propositio est per se
nota, quod praedicatum includitur in ratione subiecti, ut {\it homo est
animal}, nam {\it animal} est de ratione hominis. Si igitur notum sit omnibus
de praedicato et de subiecto quid sit, propositio illa erit omnibus per
se nota, sicut patet in primis demonstrationum principiis, quorum
termini sunt quaedam communia quae nullus ignorat, ut ens et non ens,
totum et pars, et similia. 


&

解答する。以下のように言われるべきである。あるものが自明であるのに、二通りのかたちがあ
 る。一つは、それ自体においてはそうだが私たちにとってはそうでない、とい
 うかたちである。もう一つは、それ自体においても私たちにとってもそうだ、
 というかたちである。そもそも、ある命題が自明であるのは、主語の概念の中
 に述語が含まれる事による。たとえば、「人間は動物である」という命題の場
 合、「動物」が「人間」の概念の中に含まれているので、自明である。ゆえに、
 もしも、この述語と主語について、それらが何であるかが万人に知られている
 のなら、この命題はすべての人にとって自明となるだろう。ちょうどそれは、
 論証の第一原理においても明らかである。これらの原理で使われる用語は、
 「在る」「無い」「全体」「部分」などのように、誰も知らない者がいないよ
 うな、何か共通のものである。

\\

Si autem apud aliquos notum non sit de
praedicato et subiecto quid sit, propositio quidem quantum in se est,
erit per se nota, non tamen apud illos qui praedicatum et subiectum
propositionis ignorant. Et ideo contingit, ut dicit Boetius in libro
 {\it de Hebdomadibus}, quod quaedam sunt communes animi conceptiones et per se
notae, apud sapientes tantum, ut {\it incorporalia in loco non esse}. 


&

これに対して、述語と主語について、それが何
 であるかが、ある人々に知られていないならば、その命題は、それ自体におい
 て、自明であるが、その命題の主語と述語を知らない人にとっては、自明では
 ないだろう。このことから、ボエティウスが『デ・ヘブドマディブス』で言っ
 ているように、ある事柄は、ただ知恵ある者たちにとってだけ、魂の共通懐念
 であり自明であるようなものが存在する。たとえば、「非物体的なものは場所
 には存在しない」のような命題がそれである。

\\


Dico
ergo quod haec propositio, {\it Deus est}, quantum in se est, per se nota est,
quia praedicatum est idem cum subiecto; Deus enim est suum esse, ut
infra patebit. Sed quia nos non scimus de Deo quid est, non est nobis
per se nota, sed indiget demonstrari per ea quae sunt magis nota quoad
nos, et minus nota quoad naturam, scilicet per effectus.


&
ゆえに、私は次のように言う。「神が存在する」というこの命題は、それ自体に
 おいて、自明である。なぜなら、述語が主語と同一だからである。というのも、
 後に明らかになるごとく、神は自らの存在なのだから。しかし、私たちは、神
 が何であるかを知ることがないのだから、私たちにとっては自明でなく、むし
 ろ、私たちにとってよりよく知られている神の結果を通して、証明される必要
 があるのである。ただし、この神の結果は、自然本性においては、より知られるこ
 とが少ないものなのである。


\\

{\sc Ad primum ergo dicendum} quod cognoscere Deum
esse in aliquo communi, sub quadam confusione, est nobis naturaliter
insertum, inquantum scilicet Deus est hominis beatitudo, homo enim
naturaliter desiderat beatitudinem, et quod naturaliter desideratur ab
homine, naturaliter cognoscitur ab eodem. Sed hoc non est simpliciter
cognoscere Deum esse; sicut cognoscere venientem, non est cognoscere
Petrum, quamvis sit Petrus veniens, multi enim perfectum hominis bonum,
quod est beatitudo, existimant divitias; quidam vero voluptates; quidam
autem aliquid aliud.


&
第一に対しては、それゆえ、次のように言わなければならない。神が存在するこ
 とを、ある共通点のもとで、一種の混乱を含みつつ認識することは、私たちに
 自然本性的に与えられている。つまり、神は人間の至福であり、人間は自然本
 性的に至福を欲する。ところが、人間が自然本性的に欲する対象は、人間に自
 然本性的に認識されている。しかしこれは、神が存在することを端的に認識す
 ることではない。ちょうど、だれかがやって来るという認識は、ペトロを認識
 することではないようなものである。たとえ、やって来るのがペトロだとして
 も。じっさい、多くの人々は、人間の完全な善である至福が富であると考えて
 いるし、別の人は快楽を、またある人は別の何かをそうだと考えている。

\\

{\sc Ad secundum dicendum} quod forte ille qui audit
hoc nomen {\it Deus}, non intelligit significari aliquid quo maius cogitari
non possit, cum quidam crediderint Deum esse corpus. Dato etiam quod
quilibet intelligat hoc nomine {\it Deus} significari hoc quod dicitur,
scilicet illud quo maius cogitari non potest; non tamen propter hoc
sequitur quod intelligat id quod significatur per nomen, esse in rerum
natura; sed in apprehensione intellectus tantum. Nec potest argui quod
sit in re, nisi daretur quod sit in re aliquid quo maius cogitari non
potest, quod non est datum a ponentibus Deum non esse.


 &

 第二に対しては、次のように言わなければならない。ある人々は神が物体だと
 信じたのだから、おそらく、この「神」という名を聞く者は、「それ以上大き
 いものが考えられえない何か」が表示されているとは理解しないだろう。また、
 仮に、この「神」という名前によって、だれもが、そういうこと、つまり「そ
 れ以上大きいものが考えられられないもの」が表示されていると理解するのだ
 としても、このことから、その名によって表示されているものが、事物の本性
 の中に[=実在の世界に]存在すると理解することは帰結せず、むしろ帰結するの
 は、知性の理解の中に存在するということだけである。
そして、それ以
上大きいものが考えられないものが事物の中に[=実在的に]存在すると認めら
 れないかぎり、[神が]事物の中に存在する[=実在する]、と論じられえな
 い。そして、このことが、神が存在しないと考える人々によって、認められて
 いない。\footnote{「「それ以上大きなものが考え
られえないもの」が実在することが認められないかぎり、それが実在することは
論証されえない。まさにこのことを「神なし」と主張する人々は認めないのであ
る。」(山田訳)}



\\


{\sc Ad tertium dicendum} quod veritatem esse in
communi, est per se notum, sed primam veritatem esse, hoc non est per se
notum quoad nos.


&
第三に対しては、言わなければならない。真理が共通に存在することは自明であ
 る。しかし、第一真理が存在すること、これは、私たちにとって自明ではない。


\end{longtable}
\newpage
\subsubsection*{考察 Aquinas on the Ontological Argument}
第二異論解答は、トマスによる存在論的論証の評価である。一見すると、その評
価は同語反復のようにも見え、わかりにくい。

複数の解釈が可能だが、一つの解釈を示しておく。

トマスは、ここでの論証(いわゆる存在論的論証)を、以下のように理解してい
る。

\begin{enumerate}
 \item 「神」とは「それ以上大きいものが考えられないもの」である。
 \item 「それ以上大きいものが考えられないもの」は実在する。
 \item ゆえに、神は実在する。
\end{enumerate}

上の論証の2のステップについては、深く立ち入っていない。したがってこれは、
デカルトの以下の論証と同じタイプに属する。

\begin{enumerate}
 \item 「神」はあらゆる完全性をもつものである。
 \item 実在は完全性の一つである。
 \item ゆえに、神は実在する。
\end{enumerate}

単純化すれば、このタイプの論証は、以下の形式を取る。

\begin{enumerate}
 \item なんだかんだ。
 \item 「神」は、その概念に「実在」を含む。
 \item ゆえに、神は実在する。
\end{enumerate}
\noindent
これは、概念の話である。一階述語論理の表現では、述語についての議論である。
「xは神である」という述語について、その内容を定義している。しかしこのこ
とから、議論領域中の何かがxであること($\exists xG(x)$)は帰結しない。

\begin{enumerate}
 \item $G(x)$とはなんだかんだ。
 \item ゆえに$G(x)$は$\exists y(x=y)$を含意する。
\end{enumerate}



この議論は、神の存在証明ではなく、「神である」という述語ををそのように定
義する、という話である。これに対するトマスの批判は、「それ以
上大きいものが考えられないものが事物の中に[=実在的に]存在することが認
められないかぎり、[神が]事物の中に存在する[=実在する]、と論じられえな
 い」である。

 これは、述語の定義をしても、その述語に当てはまる個体が議論領域の中にあ
 るとされないかぎり、神が存在するとは言えない、と言っているように解釈で
 きる。

 「神とは存在するもののことである」 $G(x) \overset{\mathrm{def}}{=}\exists y(x=y)$と「神は存在する」$\exists xG(x)$は、まったく異なる主張である。トマスはそれを指摘している。
 



\newpage
\rhead{a.~2}
\begin{center}
 {\Large {\bf ARTICULUS SECUNDUS}}\\
 {\large UTRUM DEUM ESSE SIT DEMONSTRABILE}\\
 {\footnotesize {\it ST} I, q.~3, a.~5; III {\it Sent.}, d.~24, q.~1,
 a.~2, qu.~2; I {\it SCG} c.~12; {\it De Pot.}, q.~8, a.~3; in
 Boet.~{\it de Trin.}, q.~1, a.~2}\\
 {\Large 第二項\\神が存在することは証明されうるか}
\end{center}

\begin{longtable}{p{21em}p{21em}}


{\huge A}{\sc d secundum sic proceditur}. Videtur quod
 Deum esse non sit demonstrabile. Deum enim esse est articulus
 fidei. Sed ea quae sunt fidei, non sunt demonstrabilia, quia
 demonstratio facit scire, fides autem de non apparentibus est, ut patet
 per apostolum, {\it ad Hebr}.~{\sc xi}. Ergo Deum esse non est demonstrabile.


&

第二項の問題へ、議論は以下のように進められる。神が存在することを論証することはで
 きないと思われる。理由は以下の通り。神が存在することは、信仰箇条である。ところが、信仰に
 属することは論証できない。なぜなら、あるものを論証すれば、私たちはそれ
 を知るようになるのだが、『ヘブライ人への手紙』第11章\footnote{「信仰
 とは、望んでいる事柄を確信し、見えない事実を確認することです。」(11:1)}の使徒によって明らかである
 ように、信仰は、見えない事柄についてあるからである。ゆえに、神が存在す
 ることを論証することはできない。

\\



2.~{\sc Praeterea}, medium demonstrationis est {\it quod quid est}. Sed
 de Deo non possumus scire {\it quid est}, sed solum quid non est, ut
 dicit Damascenus. Ergo non possumus demonstrare Deum esse.

&
さらに、論証の媒介は「何であるか」である。ところが、ダマスケヌスが言うよ
 うに、神について、私たちはそれの何であるかを知ることができず、何でない
 かを知りうるだけである。ゆえに、私たちは神が存在することを論証すること
 ができない。

\\


3.~{\sc Praeterea}, si demonstraretur Deum esse, hoc
 non esset nisi ex effectibus eius. Sed effectus eius non sunt
 proportionati ei, cum ipse sit infinitus, et effectus finiti; finiti
 autem ad infinitum non est proportio. Cum ergo causa non possit
 demonstrari per effectum sibi non proportionatum, videtur quod Deum
 esse non possit demonstrari.

&


さらに、もし、神が存在することが論証されるならば、それは、神の結果による
 論証であろう。しかし、神は無限だが結果は有限であり、有限なものは無限な
 ものに比例しないから、神の結果は神に比例しない。比例していない結果によっ
 て原因が論証されることはないので、それゆえ、神が存在することは論証でき
 ないと思われる。

\\



{\sc Sed contra est} quod apostolus dicit, {\it ad Rom}.~I: {\it
 invisibilia Dei per ea quae facta sunt, intellecta, conspiciuntur}. Sed
 hoc non esset, nisi per ea quae facta sunt, posset demonstrari Deum
 esse, primum enim quod oportet intelligi de aliquo, est an sit.

&
しかし反対に、使徒は『ローマの信徒への手紙』第1章で「神の見えないところ
 は、作られたものを通して理解され、明らかに認められる」\footnote{「世界
 が造られたときから、目に見えない神の性質、つまり神の永遠の力と神性は被
 造物に現れており、これを通して神を知ることができます」(1:20)}と言っている。こ
 ういうことは、もしも、作られたものによって神の存在が論証できなかったと
 したら、なかっただろう。なぜなら、何かについて第一に理解されるべきは、
 それが存在するかどうかなのだから。


\\



{\sc Respondeo dicendum} quod duplex est demonstratio. Una quae est per
 causam, et dicitur {\it propter quid}: et haec est per priora
 simpliciter. Alia est per effectum, et dicitur demonstratio {\it quia}:
 et haec est per ea quae sunt priora quoad nos, cum enim effectus
 aliquis nobis est manifestior quam sua causa, per effectum procedimus
 ad cognitionem causae. Ex quolibet autem effectu potest demonstrari
 propriam causam eius esse (si tamen eius effectus sint magis noti quoad
 nos), quia, cum effectus dependeant a causa, posito effectu necesse est
 causam praeexistere. Unde Deum esse, secundum quod non est per se notum
 quoad nos, demonstrabile est per effectus nobis notos.

&
答えて言わなければならない。論証には二種類ある。一つは、原因による論証で
 あり、「何のために」論証と言われる。これは、端的に、先行するものによる
 論証である。もう一つは、結果による論証であり、「である」論証と言われる
 \footnote
{アリストテレス三段論法において、中項が結論の原因である場合を「原因によ
 る論証」と言い、中項が結論の結果である場合を「結果による論証」と言う。
たとえば、「すべて近くにあるものは瞬かない。すべての惑星は近くにある。ゆ
 えに、すべての惑星は瞬かない。」という論証において、中項「近くにある」
 は、結論の「惑星が瞬かない」ことの原因であるから、これは原因による論証である。他
 方、「すべての瞬かないものは近くにある。すべての惑星は瞬かない。ゆえに、
 すべての惑星は近くにある。」において、中項「瞬かない」は、結論の「惑星
 が近くにある」ことの原因ではなく結果である。したがって、これは結果によ
 る論証と呼ばれる。
}。
 これは、私たちから見て先行するものによる論証である。というのも、結果は、
 それの原因よりも、私たちに明らかであり、私たちは、結果を通して、原因の
 認識へと進んでいくからである。ところで、どんな結果からも、それに固有の
 原因が存在することが論証できる(結果の方が私たちによく知られている場合)。
 結果は原因に依存するので、結果があるときには、その原因が前もって存在す
 る必要があるからである。したがって、私たちにとって自明ではないような意
 味での神の存在は、私たちに知られている結果によって論証が可能である。

\\



{\sc Ad primum ergo dicendum} quod Deum esse, et
 alia huiusmodi quae per rationem naturalem nota possunt esse de Deo, ut
 dicitur {\it Rom}.~I non sunt articuli fidei, sed praeambula ad articulos,
 sic enim fides praesupponit cognitionem naturalem, sicut gratia
 naturam, et ut perfectio perfectibile. Nihil tamen prohibet illud quod
 secundum se demonstrabile est et scibile, ab aliquo accipi ut
 credibile, qui demonstrationem non capit.

& それ故、第一については次のように言わなければならない。神が存在すること
や、その他、神について自然理性によって知られうるものは、『ローマの信徒へ
の手紙』第1章に言われているように \footnote{「不義によって真理の働きを妨
げる人間のあらゆる不信心と不義に対して、髪は天から怒りを現されます。なぜ
なら、神について知りうる事柄は、彼らにも明らかだからです」(1:18-19)}、信
仰箇条ではなく、その前提である。恩寵が自然を前提し、完成するものが完成さ
れるものを前提するように、信仰は自然理性を前提するのである。もっとも、そ
れ自体においては論証できるもので知ることができるものであっても、その論証
を理解できない人が、それを信仰の対象として受け入れることがあっても、それ
は一向に差し支えない。

\\



{\sc Ad secundum dicendum} quod cum demonstratur
 causa per effectum, necesse est uti effectu loco definitionis causae,
 ad probandum causam esse, et hoc maxime contingit in Deo. Quia ad
 probandum aliquid esse, necesse est accipere pro medio {\it quid significet
 nomen} non autem {\it quod quid est}, quia quaestio {\it quid est}, sequitur ad
 quaestionem {\it an est}. Nomina autem Dei imponuntur ab effectibus, ut
 postea ostendetur, unde, demonstrando Deum esse per effectum, accipere
 possumus pro medio quid significet hoc nomen {\it Deus}.

&

第二に対しては次のように言わなければならない。原因が結果によって論証され
 るとき、原因が存在することを論証するために、原因の定義の場所に、結果を用
 いる必要がある。これは、神の場合に最もよく当てはまる。あるものが存在する
 ことを証明するためには、それの何であるか[=本質]ではなく、その名称が何
 を意味しているか[=言葉の意味]を媒介[または中項]として理解する必要が
 ある。なぜなら、「何であるか」という問いは、「存在するか」という問いの後
 に来るのだから。後に明らかになるように、神という名称は、結果から付けられ
 ている。したがって、神が存在することを結果によって論証するときに、この神
 という名称が何を意味しているかを、媒介として受け取ることができる。

\\



{\sc Ad tertium dicendum} quod per effectus non
 proportionatos causae, non potest perfecta cognitio de causa haberi,
 sed tamen ex quocumque effectu potest manifeste nobis demonstrari
 causam esse, ut dictum est. Et sic ex effectibus Dei potest demonstrari
 Deum esse, licet per eos non perfecte possimus eum cognoscere secundum
 suam essentiam.

&


第三に対しては次のように言わなければならない。原因に比例していない結果を
 通して、その原因についての完全な認識を得ることはできない。しかし、すで
 に述べたように、どんな結果からでも、原因が存在することは、明らかに私た
 ちに論証されうる。したがって、神の結果から、神をその本質において完全に
 認識することはできないにしても、神が存在することは、論証できるのである。



\end{longtable}

\newpage
\rhead{a.~3}
\begin{center}
 {\Large {\bf ARTICULUS TERTIUS}}\\
 {\large UTRUM DEUS SIT}\\
 {\footnotesize I {\it Sent.}, d.~3, div.~prim.~part.~textus; I {\it
 SCG.}, c.~13, 15, 16, 41, 44; II, c.~15; III, c.~44; {\it De Verit.},
 q.~5, a.~2; {\it De Pot.}, q.~3, a.~5; {\it Compend.~Theol.}, c.~3; VII
 {\it Physic.}, l.~2; VIII, l.~9 sqq.; XII {\it Metaph.}, l.~5 sqq.}\\
 {\Large 第三項\\神は存在するか}
\end{center}

\begin{longtable}{p{21em}p{21em}}




{\Huge A}{\sc d tertium sic proceditur}. Videtur quod Deus
non sit. Quia si unum contrariorum fuerit infinitum, totaliter
destruetur aliud. Sed hoc intelligitur in hoc nomine Deus, scilicet quod
sit quoddam bonum infinitum. Si ergo Deus esset, nullum malum
inveniretur. Invenitur autem malum in mundo. Ergo Deus non est.


&
第三については、次のように進められる。神は存在しないと思われる。
相反するものの一方が無限であったら、もう一方は完全に破壊されるであろう。
 ところが、「神」というこの名において、ある種、無限の善というものが理解
 される。ゆえに、もしも神が存在するのなら、悪はなかったであろう。ところ
 が、この世界には悪がある。ゆえに、神は存在しない。


\\


2.~{\sc  Praeterea}, quod potest compleri per pauciora
principia, non fit per plura. Sed videtur quod omnia quae apparent in
mundo, possunt compleri per alia principia, supposito quod Deus non sit,
quia ea quae sunt naturalia, reducuntur in principium quod est natura;
ea vero quae sunt a proposito, reducuntur in principium quod est ratio
humana vel voluntas. Nulla igitur necessitas est ponere Deum esse.


&
さらに、より少ない原理によって完成されうるものは、それよりも多い原理によっ
 て生じることはない。ところが、世界の中に現れているすべてのものは、神が
 存在しないと仮定しても、他の原理によって完成されうるように思われる。な
 ぜなら、自然的なものは、自然という原理に還元される(自然という原理によっ
 て生じる)し、意図によるものは、人間の理性や意志という原理に還元
 されるからである。ゆえに、神の存在を措定する必要性は何もない。



\\


{\sc Sed contra est} quod dicitur {\it Exodi} {\sc iii}, ex
persona Dei, {\it Ego sum qui sum}.


&

しかし反対に、『出エジプト記』3章で、神自身の口から「我は在りて在る
 者(私は存在する者だ)」\footnote{「神はモーセに、「わたしはある。わた
 しはあるという者だ」と言われ、また、「イスラエルの人々にこう言うがよ
 い。『わたしはある』という方が私をあなたたちに遣わされたのだと。」」(3:14)}と言われている。


\\


{\sc Respondeo dicendum} quod Deum esse quinque viis
probari potest. Prima autem et manifestior via est, quae sumitur ex
parte motus. Certum est enim, et sensu constat, aliqua moveri in hoc
mundo. Omne autem quod movetur, ab alio movetur. Nihil enim movetur,
nisi secundum quod est in potentia ad illud ad quod movetur, movet autem
aliquid secundum quod est actu. Movere enim nihil aliud est quam educere
aliquid de potentia in actum, de potentia autem non potest aliquid
reduci in actum, nisi per aliquod ens in actu, sicut calidum in actu, ut
ignis, facit lignum, quod est calidum in potentia, esse actu calidum, et
per hoc movet et alterat ipsum. Non autem est possibile ut idem sit
simul in actu et potentia secundum idem, sed solum secundum diversa,
quod enim est calidum in actu, non potest simul esse calidum in
potentia, sed est simul frigidum in potentia. Impossibile est ergo quod,
secundum idem et eodem modo, aliquid sit movens et motum, vel quod
moveat seipsum. Omne ergo quod movetur, oportet ab alio moveri. 

& 

答えて言わなければならない。神が存在することは五つの道によって証明可能
である。第一の、そして比較的明らかな道は、運動の側面から取られる。すなわ
ち、この世界において、何かが動かされていることは確実であり、感覚にも明ら
かである。ところで、動かされるものはすべて、他者によって動かされる。なぜ
なら、なにかが或るものへ動かされるのは、それがその或るものにたいして可能
態にある限りにおいてに他ならないからである。他方で、動かすのは、現実態に
ある限りにおいてである。つまり、動かすとは、何かを、可能態から現実態へ引
き出すことに他ならない。ところが、何か現実態において存在するものによらな
いかぎり、あるものが可能態から現実態へ引き出されるということはありえない。
たとえば、現実態において熱いもの、たとえば「火」は、可能態において熱いも
の、たとえば「木」を、現実に熱いものにし、このことによって、木を動かし変
化させる。ところで、一つのものが、同時に、同じ観点から、現実態でありかつ
可能態であることはできない。できるとすれば、それは異なる観点からである。
たとえば、現実態において熱いものは、同時に可能態において熱いものではあり
えないが、同時に可能態において冷たいものではありうる。ゆえに、一つのもの
が同一の観点から、動かすものでありかつ動かされるものであることはありえな
い。つまり、自己自身を動かすことはありえない。ゆえに、動かされるものはす
べて、他者によって動かされる。

\\

Si ergo
id a quo movetur, moveatur, oportet et ipsum ab alio moveri et illud ab
alio. Hic autem non est procedere in infinitum, quia sic non esset
aliquod primum movens; et per consequens nec aliquod aliud movens, quia
moventia secunda non movent nisi per hoc quod sunt mota a primo movente,
sicut baculus non movet nisi per hoc quod est motus a manu. Ergo necesse
est devenire ad aliquod primum movens, quod a nullo movetur, et hoc
omnes intelligunt Deum. 


&




ゆえに、もしもAがBを動かすとき、Aが動かされ
ているならば、Aもまた、別のものCによって動かされ、そのCも、別のものによっ
て動かされることになる。しかし、これが無限に進むことはない。なぜなら、も
し無限に進んだりしたら、どんな第一動者も存在しないことになり、その結果、
その他のどんな動者も存在しないことになるからである。なぜなら、二番目に動
かすものは、一番目に動かすものに動かされて(三番目のものを)動かすしかな
いからである。ちょうど、手が杖を動かすことによってでない限り、杖がものを
動かすことはないように。ゆえに、何ものによっても動かされない何らかの第一の
動者へ到達することが必要であり、それをすべての人たちは神と理解する。



\\

Secunda via est ex ratione causae
efficientis. Invenimus enim in istis sensibilibus esse ordinem causarum
efficientium, nec tamen invenitur, nec est possibile, quod aliquid sit
causa efficiens sui ipsius; quia sic esset prius seipso, quod est
impossibile. Non autem est possibile quod in causis efficientibus
procedatur in infinitum. Quia in omnibus causis efficientibus ordinatis,
primum est causa medii, et medium est causa ultimi, sive media sint
plura sive unum tantum, remota autem causa, removetur effectus, ergo, si
non fuerit primum in causis efficientibus, non erit ultimum nec
medium. Sed si procedatur in infinitum in causis efficientibus, non erit
prima causa efficiens, et sic non erit nec effectus ultimus, nec causae
efficientes mediae, quod patet esse falsum. Ergo est necesse ponere
aliquam causam efficientem primam, quam omnes Deum nominant. 



&

第二の道は、作出因の観点からのものである。
あなたの周囲にあって感覚されているもののなかには作出因の系列が存在すると
 いうことに、わたしたちは気付く。ところで、あるものが、自分自身の作出因
 であるところを見かけることはないし、またそんなことはありえない。なぜな
 ら、もしそのようなことがあれば、自分が自分よりも先に存在することになる
 のであって、それは不可能だからである。
しかし、作出因の系列を無限にさかのぼることはできない。
なぜなら、すべての作出因を並べると、第一のものは中間のものの原因であり、
 中間のものは最後のものの原因である。このことは、中間のものが複数あろう
 と一つだけだろうと変わらない。しかし、原因が無くなれば、その結果も無く
 なる。ゆえに、もしも作出因の中の第一のものが無ければ、最後のものも中間
 のものも無いであろう。しかし、もしも作出因の中を無限にさかのぼり、第一
 の作出因が無いとしたら、最後の作出因も、中間の作出因も存在しないことに
 なるが、これは明らかに偽である。
ゆえに、何らかのものを第一の作出因としなければならない。それをすべての人
 は神と名付けている。



\\

Tertia via
est sumpta ex possibili et necessario, quae talis est. Invenimus enim in
rebus quaedam quae sunt possibilia esse et non esse, cum quaedam
inveniantur generari et corrumpi, et per consequens possibilia esse et
non esse. Impossibile est autem omnia quae sunt, talia
 esse\footnote{Leoninaは、``omnia quae sunt talia, semper esse''となって
 いる。}, quia quod
possibile est non esse, quandoque non est. Si igitur omnia sunt
possibilia non esse, aliquando nihil fuit in rebus. Sed si hoc est
verum, etiam nunc nihil esset, quia quod non est, non incipit esse nisi
per aliquid quod est; si igitur nihil fuit ens, impossibile fuit quod
aliquid inciperet esse, et sic modo nihil esset, quod patet esse
falsum. Non ergo omnia entia sunt possibilia, sed oportet aliquid esse
necessarium in rebus. Omne autem necessarium vel habet causam suae
necessitatis aliunde, vel non habet. Non est autem possibile quod
procedatur in infinitum in necessariis quae habent causam suae
necessitatis, sicut nec in causis efficientibus, ut probatum est. Ergo
necesse est ponere aliquid quod sit per se necessarium, non habens
causam necessitatis aliunde, sed quod est causa necessitatis aliis, quod
omnes dicunt Deum. 


& 第三の道は、可能なものと必然的なものという観点からであるが、それは次の
とおり。わたしたちは事物の中に、存在することも存在しないことも可能である
ようなものを見出す。というのも、あるものは、生成・消滅する、つまり結果的
に、存在することも存在しないことも可能であることが見出されるからである。
ところで、存在するすべてのものが、そのようなもの(=存在することも存在し
ないことも可能なもの)であることは不可能である\footnote{レオニナ版の読み
を取れば「そのようなものすべてが、常に存在することは不可能である」とな
る。}。なぜなら、存在しないことが可能なものは、現に存在しないことがあるか
らである。それゆえ、すべてのものが存在しないことが可能なものならば、現実
に何も存在しなかった時点が過去にある。しかし、もしこれ(「現実に何も存在
しなかった過去の時点がある」こと)が真なら、今も、何も存在しなかったであ
ろう。なぜなら、存在しないものは、何か他の存在しているものによらなければ、
存在し始めることはないからである。ゆえに、もしも、存在しているものが何も
ないならば、何かが存在し始めることは不可能だし、その結果、今も何も存在し
なかったであろう。これは明らかに偽である。ゆえに、すべての存在者が可能的
なものであるわけではなく、事物において何かが必然的なものでなければならな
い。ところで、すべて、必然的なものは、自らの必然性を、他のどこかから受け
取るか、それともそうでないかのどちらかである。しかし、自らの必然性の原因
を持つ必然的なものどもの中を、無限に進むことは不可能である。これは、すで
に証明されたように、作出因の中と同じである。ゆえに、それ自身によって必然
的である何かを措定する必要がある。それは、他のどこかから必然性の原因を受
け取るのではなく、それ自身が、他の必然的なものにとっての原因である。すべ
ての人はこれを神という。



\\


Quarta via sumitur ex gradibus qui in rebus
inveniuntur. Invenitur enim in rebus aliquid magis et minus bonum, et
verum, et nobile, et sic de aliis huiusmodi. Sed {\it magis} et {\it minus} dicuntur
de diversis secundum quod appropinquant diversimode ad aliquid quod
maxime est, sicut magis calidum est, quod magis appropinquat maxime
calido. Est igitur aliquid quod est verissimum, et optimum, et
nobilissimum, et per consequens maxime ens, nam quae sunt maxime vera,
sunt maxime entia, ut dicitur II {\it Metaphys}. Quod autem dicitur maxime
tale in aliquo genere, est causa omnium quae sunt illius generis, sicut
ignis, qui est maxime calidus, est causa omnium calidorum, ut in eodem
libro dicitur. Ergo est aliquid quod omnibus entibus est causa esse, et
bonitatis, et cuiuslibet perfectionis, et hoc dicimus Deum. 


&
第四の道は、物事の中に見つけられる階級の観点からのものである。
物事の中には、善、真、高貴、などということについて、より多く・より少なく
 ということが見いだされる。
しかし、さまざまなものについて、より多く・より少なく、ということは、何か
 最大限にそうであるものにどれだけ近づいているかに応じていうのである。た
 とえば、より熱いものとは、もっとも熱いものにより近づいているものである。
ゆえに、何かが、もっとも真であるもの、もっとも善いもの、もっとも高貴であ
 るものであり、それは結局、最大限に存在するものである。なぜなら、『形而
 上学』第2巻で述べられているように、最大限に真であるものは最大限に存在するも
 のだからである。
ところで、ある類の中で最大限にこれこれであると言われるものは、その類に含まれる
 すべてのものの原因である。たとえば、火は最大限に熱いものなので、「熱い
 もの」という類に含まれるすべてのものの原因である。これも『形而上学』で
 言われている。
ゆえに、すべての存在するものにとって、その存在、善性、その他何であれ完全
 性の原因である何かが存在する。わたしたちは、これを神という。


\\


Quinta via
sumitur ex gubernatione rerum. Videmus enim quod aliqua quae cognitione
carent, scilicet corpora naturalia, operantur propter finem, quod
apparet ex hoc quod semper aut frequentius eodem modo operantur, ut
consequantur id quod est optimum; unde patet quod non a casu, sed ex
intentione perveniunt ad finem. Ea autem quae non habent cognitionem,
non tendunt in finem nisi directa ab aliquo cognoscente et intelligente,
sicut sagitta a sagittante. Ergo est aliquid intelligens, a quo omnes
res naturales ordinantur ad finem, et hoc dicimus Deum.


&




第五の道は、諸事物の統治という観点からである。
わたしたちは、認識を欠くもの、つまり自然物が、目的のために働くのを見る。
 このことは、最善のことがらが生じるように、常に、頻繁に、同じ仕方で働く
 ことからも明らかである。したがって、それらは偶然に動いているのではなく、
 意図的に、目的へ到達しているのである。ところで、認識をもたないものは、
 ちょうど矢が射手によって[的へ向けて射られるように]、だれか認識する者
 や知性認識する者によって向けられるのでない限り、目的へとは向かわない。
ゆえに、何らかの知性を持つものが存在し、その者によって、すべての自然物は、
 目的へと向かっている。これをわたしたちは神という。


\\


{\sc Ad primum ergo dicendum} quod, sicut dicit Augustinus in {\it
Enchiridio}: {\it Deus, cum sit summe bonus, nullo modo sineret aliquid
mali esse in operibus suis, nisi esset adeo omnipotens et bonus, ut bene
faceret etiam de malo}. Hoc ergo ad infinitam Dei bonitatem pertinet, ut
esse permittat mala, et ex eis eliciat bona.


&

第一に対しては、それゆえ、次のように言わなければならない。アウグスティヌ
 スが『提要』で述べているように、「神は、最高度に善であり、悪からさえも善
 を作るほどに全能で善でなかったならば、自分の業の中にいかなる悪も決して
 存在させなかったであろう」。ゆえに、悪の存在を赦し、悪から膳を引き出すこ
 とは、神の無限の善性に属するのである。


\\


{\sc Ad secundum dicendum} quod, cum natura propter
determinatum finem operetur ex directione alicuius superioris agentis,
necesse est ea quae a natura fiunt, etiam in Deum reducere, sicut in
primam causam. Similiter etiam quae ex proposito fiunt, oportet reducere
in aliquam altiorem causam, quae non sit ratio et voluntas humana, quia
haec mutabilia sunt et defectibilia; oportet autem omnia mobilia et
deficere possibilia reduci in aliquod primum principium immobile et per
se necessarium, sicut ostensum est.


&
第二に対しては、次のように言わなければならない。自然が決まった目的のため
 に働くのは、何か上位の作用者の導きによるのだから、自然から生じるものは、
 やはり第一原因としての神に導かれているのである。同様に、意図的になされ
 ることも、より高い何らかの原因に導かれるのだが、それは人間の理性や意志
 ではない。なぜなら、人間の理性や意志は、変わりやすく欠陥を持ちやすいの
 だが、すべて変化し欠陥を持ちうるものは、不動の、そしてそれ自体によって
 必然的である何らかの第一の根源に導かれる必要があるからである。これはす
 でに述べられたことである。





\end{longtable}


\end{document}