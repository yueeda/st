\documentclass[10pt]{jsarticle}
\usepackage{okumacro}
\usepackage{amsmath}
\usepackage{longtable}
\usepackage[polutonikogreek,english,japanese]{babel}
\usepackage{latexsym}
\usepackage{color}


%----- header -------
\usepackage{fancyhdr}
\pagestyle{fancy}
\lhead{{\it Summa Theologiae} I, q.~2}
%--------------------

\bibliographystyle{jplain}

\title{{\bf Prima Pars}\\{\HUGE Summae Theologiae}\\Sancti Thomae
Aquinatis\\Quaestio Secunda\\{\bf De Deo, An Deus Sit}}
\author{Japanese translation\\by Yoshinori {\sc Ueeda}}
\date{Last modified \today}


%%%% コピペ用
%\rhead{a.~}
%\begin{center}
% {\Large {\bf }}\\
% {\large }\\
% {\footnotesize }\\
% {\Large \\}
%\end{center}
%
%\begin{longtable}{p{21em}p{21em}}
%
%&
%
%\\
%\end{longtable}
%\newpage



\begin{document}
\maketitle
\begin{center}
 {\Large 第二問\\神について、神は存在するか}
\end{center}

\begin{longtable}{p{21em}p{21em}}
Quia igitur principalis intentio huius sacrae doctrinae est Dei
cognitionem tradere, et non solum secundum quod in se est, sed etiam
secundum quod est principium rerum et finis earum, et specialiter
rationalis creaturae, ut ex dictis est manifestum; ad huius doctrinae
expositionem intendentes, primo tractabimus de Deo; secundo, de motu
rationalis creaturae in Deum; tertio, de Christo, qui, secundum quod
homo, via est nobis tendendi in Deum.

&

この聖なる教えの主要な目的は、神についての認識を伝えることであり、それ
も、たんに神それ自身においてだけでなく、神が諸事物の根源であり、また、
諸事物の、とくに、すでに述べたことから明らかなように、理性的被造物の目
的である限りにおいても、神の認識を伝えることを目的とするので、この教え
を説明しようとする私たちは、第一に、神について、第二に、理性的被造物の
神への運動について\footnote{第2部}、第三に、キリストについて
\footnote{第3部}論じる。キリストは、彼が人間である限りにおいては、神へ
と向かう人間にとっての道だからである。

\\

Consideratio autem de Deo tripartita erit. Primo namque considerabimus
ea quae ad essentiam divinam pertinent; secundo, ea quae pertinent ad
distinctionem personarum; tertio, ea quae pertinent ad processum
creaturarum ab ipso.

&

神についての考察は三つの部分を持つ。第一に、私たちは神の本質に関する事
柄を考察し、第二に、ペルソナの区別に関する事柄を\footnote{第27問以降}、
そして第三に、被造物の神自身からの発出に関する事柄を\footnote{第44問以
降。}考察する。

\\

Circa essentiam vero divinam, primo considerandum est an Deus sit;
secundo, quomodo sit, vel potius quomodo non sit; tertio considerandum
erit de his quae ad operationem ipsius pertinent, scilicet de scientia
et de voluntate et potentia. Circa primum quaeruntur tria.

&

神の本質については、第一に、神が存在するかどうかを、第二に、神がどのよ
うに存在するかを、いや、むしろどのように存在していないかを\footnote{第
3問以降。}考察すべきである。第三には、神の働きに属する事柄、すなわち、
知、意志、能力について\footnote{第14問以降。}考察すべきである。第一に
かんして、三つのことが問われる。

\\

\begin{enumerate}
 \item utrum Deum esse sit per se notum. 
 \item utrum sit demonstrabile. 
 \item an Deus sit.
\end{enumerate}

&

\begin{enumerate}
 \item 神が存在することは、自明か。
 \item 神が存在することは、証明できるか。 
 \item 神は存在するか。
\end{enumerate}

\end{longtable}

\newpage
\rhead{a.~1}

\begin{center}
 {\Large {\bf ARTICULUS PRIMUS}}\\
 {\large UTRUM  DEUM ESSE SIT PER SE NOTUM}\\
 {\Large 第一項\\神が存在することは自明か}\\
 {\footnotesize I {\it Sent.}, d.~3, q.~1, a.~2; I {\it SCG.}, c.~10, 11;
 III, c.~38; {\it De Verit.}, q.~10, a.~12; {\it De Pot.}, q.~7, a.~2,
 ad 11; in Psalm.~8; in Boet.~{\it de Trin.}, q.~1, a.~3, ad 6.}

\end{center}

\begin{longtable}{p{21em}p{21em}}

{\huge A}{\sc d primum sic proceditur}. Videtur quod Deum esse sit per
se notum. Illa enim nobis dicuntur per se nota, quorum cognitio nobis
naturaliter inest, sicut patet de primis principiis. Sed, sicut dicit
Damascenus in principio libri sui, {\it omnibus cognitio existendi
Deum naturaliter est inserta}. Ergo Deum esse est per se notum.


&

第一項の問題へ、議論は以下のように進められる。神が存在することは、自明
であると思われる。理由は以下の通り。自明なものとは、ちょうど論証の第一
原理がそうであるように、それの認識が、私たちに自然本性的に内在するもの
のことである。ところが、ダマスケヌスが彼の著作の冒頭で述べているように、
神が存在するという認識はすべての人に自然本性的に植え付けられている。ゆ
えに、神が存在することは自明である。


\\


2.~{\sc Praeterea}, illa dicuntur esse per se nota,
quae statim, cognitis terminis, cognoscuntur, quod philosophus attribuit
primis demonstrationis principiis, in I {\it Poster}.: scito enim quid est
totum et quid pars, statim scitur quod omne totum maius est sua
parte. 

&

さらに、用いられている言葉の意味がわかると直ちにわかるようなものが、自
明なものと言われる。哲学者は『分析論後書』で、論証の第一原理がそのよう
な性質だとしている。たとえば、「全体」「部分」という言葉の意味がわかれ
ば、「全体はすべてその部分よりも大きい」ということが知られる。

\\

Sed intellecto quid significet hoc nomen Deus, statim habetur quod
Deus est. Significatur enim hoc nomine id quo maius significari non
potest, maius autem est quod est in re et intellectu, quam quod est in
intellectu tantum, unde cum, intellecto hoc nomine Deus, statim sit in
intellectu, sequitur etiam quod sit in re. Ergo Deum esse est per se
notum.


&

ところが、「神」というこの名が何を意味しているかが理解されるならば、神
が存在することは直ちに了解される。理由は以下の通り。その名によって、
「それ以上大きいものが意味され得ないもの」が意味される。ところが、実在
と理解において存在するものは、たんに理解において存在するものよりも大き
い。従って、「神」というこの名が理解されるならば、直ちにそれは理解にお
いて存在するのだから、それが実在においても存在することが帰結する。ゆえ
に、神が存在することは自明である。

\\


3.~{\sc Praeterea}, veritatem esse est per se notum, quia qui negat
veritatem esse, concedit veritatem esse, si enim veritas non est,
verum est veritatem non esse. Si autem est aliquid verum, oportet quod
veritas sit. Deus autem est ipsa veritas, Ioann.~{\sc xiv}: {\it Ego
sum via, veritas et vita}. Ergo Deum esse est per se notum.


&

さらに、真理が存在することは自明である。なぜなら、真理が存在することを
否定する者でも、真理が存在することを認めているのである。なぜなら、もし
真理が存在しないならば、真理が存在しないことが真だからである。しかるに、
もしも何かが真であれば、真理が存在しなければならない。ところが『ヨハネ
による福音書』第14章「私は道であり真理であり生命である」\footnote{「イ
エスは言われた。「私は道であり、真理であり、命である。私を通らなければ、
誰も父のもとに行くことができない。 」(14:06)}によれば、神は真理そのも
のである。ゆえに、神が存在することは自明である。

\\


{\sc Sed contra}, nullus potest cogitare oppositum eius quod est per
se notum ut patet per philosophum, in IV {\it Metaphys}.~et I {\it
Poster}., circa prima demonstrationis principia. Cogitari autem potest
oppositum eius quod est Deum esse, secundum illud Psalmi {\sc lii}:
{\it Dixit insipiens in corde suo, non est Deus}. Ergo Deum esse non
est per se notum.


&

しかし反対に、『形而上学』第4巻および『分析論後書』第1巻の哲学者によっ
て明らかなように、だれも自明なものの反対を考えることはできない。ところ
が、「神が存在する」の反対を考えることはできる。『詩編』52章に「愚かな
者は心の中で神はいないと言った」\footnote{「神を知らぬ者は心に言う、
「神などいない」と。」(53:2)}とあるからである。ゆえに、神が存在するこ
とは、自明でない。

\\


{\sc Respondeo dicendum} quod contingit aliquid esse per se notum
dupliciter, uno modo, secundum se et non quoad nos; alio modo,
secundum se et quoad nos. Ex hoc enim aliqua propositio est per se
nota, quod praedicatum includitur in ratione subiecti, ut {\it homo
est animal}, nam {\it animal} est de ratione hominis. Si igitur notum
sit omnibus de praedicato et de subiecto quid sit, propositio illa
erit omnibus per se nota, sicut patet in primis demonstrationum
principiis, quorum termini sunt quaedam communia quae nullus ignorat,
ut ens et non ens, totum et pars, et similia.


&

解答する。以下のように言われるべきである。あるものが自明であるのに、二
通りのかたちがある。一つは、それ自体においてはそうだが私たちにとっては
そうでない、というかたちである。もう一つは、それ自体においても私たちに
とってもそうだ、というかたちである。そもそも、ある命題が自明であるのは、
主語の概念の中に述語が含まれる事による。たとえば、「人間は動物である」
という命題の場合、「動物」が「人間」の概念の中に含まれているので、自明
である。ゆえに、もしも、この述語と主語について、それらが何であるかが万
人に知られているのなら、この命題はすべての人にとって自明となるだろう。
ちょうどそれは、論証の第一原理においても明らかである。これらの原理で使
われる用語は、「在る」「無い」「全体」「部分」などのように、誰も知らな
い者がいないような、何か共通のものである。

\\

Si autem apud aliquos notum non sit de praedicato et subiecto quid
sit, propositio quidem quantum in se est, erit per se nota, non tamen
apud illos qui praedicatum et subiectum propositionis ignorant. Et
ideo contingit, ut dicit Boetius in libro {\it de Hebdomadibus}, quod
quaedam sunt communes animi conceptiones et per se notae, apud
sapientes tantum, ut {\it incorporalia in loco non esse}.


&

これに対して、述語と主語について、それが何であるかが、ある人々に知られ
ていないならば、その命題は、それ自体において、自明であるが、その命題の
主語と述語を知らない人にとっては、自明ではないだろう。このことから、ボ
エティウスが『デ・ヘブドマディブス』で言っているように、ある事柄は、た
だ知恵ある者たちにとってだけ、魂の共通懐念であり自明であるようなものが
存在する。たとえば、「非物体的なものは場所には存在しない」のような命題
がそれである。

\\


Dico ergo quod haec propositio, {\it Deus est}, quantum in se est, per
se nota est, quia praedicatum est idem cum subiecto; Deus enim est
suum esse, ut infra patebit. Sed quia nos non scimus de Deo quid est,
non est nobis per se nota, sed indiget demonstrari per ea quae sunt
magis nota quoad nos, et minus nota quoad naturam, scilicet per
effectus.


&

ゆえに、私は次のように言う。「神が存在する」というこの命題は、それ自体
において、自明である。なぜなら、述語が主語と同一だからである。というの
も、後に明らかになるごとく、神は自らの存在なのだから。しかし、私たちは、
神が何であるかを知ることがないのだから、私たちにとっては自明でなく、む
しろ、私たちにとってよりよく知られている神の結果を通して、証明される必
要があるのである。ただし、この神の結果は、自然本性においては、より知ら
れることが少ないものなのである。


\\

{\sc Ad primum ergo dicendum} quod cognoscere Deum esse in aliquo
communi, sub quadam confusione, est nobis naturaliter insertum,
inquantum scilicet Deus est hominis beatitudo, homo enim naturaliter
desiderat beatitudinem, et quod naturaliter desideratur ab homine,
naturaliter cognoscitur ab eodem. Sed hoc non est simpliciter
cognoscere Deum esse; sicut cognoscere venientem, non est cognoscere
Petrum, quamvis sit Petrus veniens, multi enim perfectum hominis
bonum, quod est beatitudo, existimant divitias; quidam vero
voluptates; quidam autem aliquid aliud.


&

第一に対しては、それゆえ、次のように言わなければならない。神が存在する
ことを、ある共通点のもとで、一種の混乱を含みつつ認識することは、私たち
に自然本性的に与えられている。つまり、神は人間の至福であり、人間は自然
本性的に至福を欲する。ところが、人間が自然本性的に欲する対象は、人間に
自然本性的に認識されている。しかしこれは、神が存在することを端的に認識
することではない。ちょうど、だれかがやって来るという認識は、ペトロを認
識することではないようなものである。たとえ、やって来るのがペトロだとし
ても。じっさい、多くの人々は、人間の完全な善である至福が富であると考え
ているし、別の人は快楽を、またある人は別の何かをそうだと考えている。

\\

{\sc Ad secundum dicendum} quod forte ille qui audit hoc nomen {\it
Deus}, non intelligit significari aliquid quo maius cogitari non
possit, cum quidam crediderint Deum esse corpus. Dato etiam quod
quilibet intelligat hoc nomine {\it Deus} significari hoc quod
dicitur, scilicet illud quo maius cogitari non potest; non tamen
propter hoc sequitur quod intelligat id quod significatur per nomen,
esse in rerum natura; sed in apprehensione intellectus tantum. Nec
potest argui quod sit in re, nisi daretur quod sit in re aliquid quo
maius cogitari non potest, quod non est datum a ponentibus Deum non
esse.


 &

第二に対しては、次のように言わなければならない。ある人々は神が物体だと
信じたのだから、おそらく、この「神」という名を聞く者は、「それ以上大き
いものが考えられえない何か」が表示されているとは理解しないだろう。また、
仮に、この「神」という名前によって、だれもが、そういうこと、つまり「そ
れ以上大きいものが考えられられないもの」が表示されていると理解するのだ
としても、このことから、その名によって表示されているものが、事物の本性
の中に[=実在の世界に]存在すると理解することは帰結せず、むしろ帰結す
るのは、知性の理解の中に存在するということだけである。そして、それ以上
大きいものが考えられないものが事物の中に[=実在的に]存在すると認めら
れないかぎり、[神が]事物の中に存在する[=実在する]、と論じられえな
い。そして、このことが、神が存在しないと考える人々によって、認められて
いない。\footnote{「「それ以上大きなものが考えられえないもの」が実在す
ることが認められないかぎり、それが実在することは論証されえない。まさに
このことを「神なし」と主張する人々は認めないのである。」(山田訳)}



\\


{\sc Ad tertium dicendum} quod veritatem esse in communi, est per se
notum, sed primam veritatem esse, hoc non est per se notum quoad nos.


&

第三に対しては、言わなければならない。真理が共通に存在することは自明で
ある。しかし、第一真理が存在すること、これは、私たちにとって自明ではな
い。


\end{longtable}
\newpage
\subsubsection*{考察 Aquinas on the Ontological Argument}
第二異論解答は、トマスによる存在論的論証の評価である。一見すると、その評
価は同語反復のようにも見え、わかりにくい。

複数の解釈が可能だが、一つの解釈を示しておく。

トマスは、ここでの論証(いわゆる存在論的論証)を、以下のように理解してい
る。

\begin{enumerate}
 \item 「神」とは「それ以上大きいものが考えられないもの」である。
 \item 「それ以上大きいものが考えられないもの」は実在する。
 \item ゆえに、神は実在する。
\end{enumerate}

上の論証の2のステップについては、深く立ち入っていない。したがってこれは、
デカルトの以下の論証と同じタイプに属する。

\begin{enumerate}
 \item 「神」はあらゆる完全性をもつものである。
 \item 実在は完全性の一つである。
 \item ゆえに、神は実在する。
\end{enumerate}

単純化すれば、このタイプの論証は、以下の形式を取る。

\begin{enumerate}
 \item なんだかんだ。
 \item 「神」は、その概念に「実在」を含む。
 \item ゆえに、神は実在する。
\end{enumerate}
\noindent
これは、概念の話である。一階述語論理の表現では、述語についての議論である。
「xは神である」という述語について、その内容を定義している。しかしこのこ
とから、議論領域中の何かがxであること($\exists xG(x)$)は帰結しない。

\begin{enumerate}
 \item $G(x)$とはなんだかんだ。
 \item ゆえに$G(x)$は$\exists y(x=y)$を含意する。
\end{enumerate}



この議論は、神の存在証明ではなく、「神である」という述語ををそのように
定義する、という話である。これに対するトマスの批判は、「それ以上大きい
ものが考えられないものが事物の中に[=実在的に]存在することが認められ
ないかぎり、[神が]事物の中に存在する[=実在する]、と論じられえない」
である。

これは、述語の定義をしても、その述語に当てはまる個体が議論領域の中にあ
るとされないかぎり、神が存在するとは言えない、と言っているように解釈で
きる。

「神とは存在するもののことである」
$G(x) \overset{\mathrm{def}}{=} \exists y(x=y)$
と
「神は存在する」
$\exists xG(x)$
は、まったく異なる主張である。トマスはそれを指摘している。
 

\newpage
\rhead{a.~2}
\begin{center}
 {\Large {\bf ARTICULUS SECUNDUS}}\\
 {\large UTRUM DEUM ESSE SIT DEMONSTRABILE}\\
 {\footnotesize {\it ST} I, q.~3, a.~5; III {\it Sent.}, d.~24, q.~1,
 a.~2, qu.~2; I {\it SCG} c.~12; {\it De Pot.}, q.~8, a.~3; in
 Boet.~{\it de Trin.}, q.~1, a.~2}\\
 {\Large 第二項\\神が存在することは証明されうるか}
\end{center}

\begin{longtable}{p{21em}p{21em}}


{\huge A}{\sc d secundum sic proceditur}. Videtur quod Deum esse non
sit demonstrabile. Deum enim esse est articulus fidei. Sed ea quae
sunt fidei, non sunt demonstrabilia, quia demonstratio facit scire,
fides autem de non apparentibus est, ut patet per apostolum, {\it ad
Hebr}.~{\sc xi}. Ergo Deum esse non est demonstrabile.


&

第二項の問題へ、議論は以下のように進められる。神が存在することを論証す
ることはできないと思われる。理由は以下の通り。神が存在することは、信仰
箇条である。ところが、信仰に属することは論証できない。なぜなら、あるも
のを論証すれば、私たちはそれを知るようになるのだが、『ヘブライ人への手
紙』第11章\footnote{「信仰とは、望んでいる事柄を確信し、見えない事実を
確認することです。」(11:1)}の使徒によって明らかであるように、信仰は、
見えない事柄についてあるからである。ゆえに、神が存在することを論証する
ことはできない。

\\



2.~{\sc Praeterea}, medium demonstrationis est {\it quod quid
 est}. Sed de Deo non possumus scire {\it quid est}, sed solum quid
 non est, ut dicit Damascenus. Ergo non possumus demonstrare Deum
 esse.

&

さらに、論証の媒介は「何であるか」である。ところが、ダマスケヌスが言う
ように、神について、私たちはそれの何であるかを知ることができず、何でな
いかを知りうるだけである。ゆえに、私たちは神が存在することを論証するこ
とができない。

\\


3.~{\sc Praeterea}, si demonstraretur Deum esse, hoc non esset nisi ex
effectibus eius. Sed effectus eius non sunt proportionati ei, cum ipse
sit infinitus, et effectus finiti; finiti autem ad infinitum non est
proportio. Cum ergo causa non possit demonstrari per effectum sibi non
proportionatum, videtur quod Deum esse non possit demonstrari.

&


さらに、もし、神が存在することが論証されるならば、それは、神の結果によ
る論証であろう。しかし、神は無限だが結果は有限であり、有限なものは無限
なものに比例しないから、神の結果は神に比例しない。比例していない結果に
よって原因が論証されることはないので、それゆえ、神が存在することは論証
できないと思われる。

\\



{\sc Sed contra est} quod apostolus dicit, {\it ad Rom}.~I: {\it
invisibilia Dei per ea quae facta sunt, intellecta,
conspiciuntur}. Sed hoc non esset, nisi per ea quae facta sunt, posset
demonstrari Deum esse, primum enim quod oportet intelligi de aliquo,
est an sit.

&

しかし反対に、使徒は『ローマの信徒への手紙』第1章で「神の見えないとこ
ろは、作られたものを通して理解され、明らかに認められる」\footnote{「世
界が造られたときから、目に見えない神の性質、つまり神の永遠の力と神性は
被造物に現れており、これを通して神を知ることができます」(1:20)}と言っ
ている。こういうことは、もしも、作られたものによって神の存在が論証でき
なかったとしたら、なかっただろう。なぜなら、何かについて第一に理解され
るべきは、それが存在するかどうかなのだから。


\\



{\sc Respondeo dicendum} quod duplex est demonstratio. Una quae est
per causam, et dicitur {\it propter quid}: et haec est per priora
simpliciter. Alia est per effectum, et dicitur demonstratio {\it
quia}: et haec est per ea quae sunt priora quoad nos, cum enim
effectus aliquis nobis est manifestior quam sua causa, per effectum
procedimus ad cognitionem causae. Ex quolibet autem effectu potest
demonstrari propriam causam eius esse (si tamen eius effectus sint
magis noti quoad nos), quia, cum effectus dependeant a causa, posito
effectu necesse est causam praeexistere. Unde Deum esse, secundum quod
non est per se notum quoad nos, demonstrabile est per effectus nobis
notos.

&

答えて言わなければならない。論証には二種類ある。一つは、原因による論証
であり、「何のために」論証と言われる。これは、端的に、先行するものによ
る論証である。もう一つは、結果による論証であり、「である」論証と言われ
る\footnote {アリストテレス三段論法において、中項が結論の原因である場
合を「原因による論証」と言い、中項が結論の結果である場合を「結果による
論証」と言う。たとえば、「すべて近くにあるものは瞬かない。すべての惑星
は近くにある。ゆえに、すべての惑星は瞬かない。」という論証において、中
項「近くにある」は、結論の「惑星が瞬かない」ことの原因であるから、これ
は原因による論証である。他方、「すべての瞬かないものは近くにある。すべ
ての惑星は瞬かない。ゆえに、すべての惑星は近くにある。」において、中項
「瞬かない」は、結論の「惑星が近くにある」ことの原因ではなく結果である。
したがって、これは結果による論証と呼ばれる。}。これは、私たちから見て
先行するものによる論証である。というのも、結果は、それの原因よりも、私
たちに明らかであり、私たちは、結果を通して、原因の認識へと進んでいくか
らである。ところで、どんな結果からも、それに固有の原因が存在することが
論証できる(結果の方が私たちによく知られている場合)。結果は原因に依存
するので、結果があるときには、その原因が前もって存在する必要があるから
である。したがって、私たちにとって自明ではないような意味での神の存在は、
私たちに知られている結果によって論証が可能である。

\\



{\sc Ad primum ergo dicendum} quod Deum esse, et alia huiusmodi quae
per rationem naturalem nota possunt esse de Deo, ut dicitur {\it
Rom}.~I non sunt articuli fidei, sed praeambula ad articulos, sic enim
fides praesupponit cognitionem naturalem, sicut gratia naturam, et ut
perfectio perfectibile. Nihil tamen prohibet illud quod secundum se
demonstrabile est et scibile, ab aliquo accipi ut credibile, qui
demonstrationem non capit.

& 


それ故、第一異論に対しては次のように言わなければならない。神が存在する
ことや、その他、神について自然理性によって知られうるものは、『ローマの
信徒への手紙』第1章に言われているように \footnote{「不義によって真理の
働きを妨げる人間のあらゆる不信心と不義に対して、髪は天から怒りを現され
ます。なぜなら、神について知りうる事柄は、彼らにも明らかだからです」
(1:18-19)}、信仰箇条ではなく、その前提である。恩寵が自然を前提し、完成
するものが完成されるものを前提するように、信仰は自然理性を前提するので
ある。もっとも、それ自体においては論証できるもので知ることができるもの
であっても、その論証を理解できない人が、それを信仰の対象として受け入れ
ることがあっても、それは一向に差し支えない。

\\



{\sc Ad secundum dicendum} quod cum demonstratur causa per effectum,
necesse est uti effectu loco definitionis causae, ad probandum causam
esse, et hoc maxime contingit in Deo. Quia ad probandum aliquid esse,
necesse est accipere pro medio {\it quid significet nomen} non autem
{\it quod quid est}, quia quaestio {\it quid est}, sequitur ad
quaestionem {\it an est}. Nomina autem Dei imponuntur ab effectibus,
ut postea ostendetur, unde, demonstrando Deum esse per effectum,
accipere possumus pro medio quid significet hoc nomen {\it Deus}.

&

第二異論に対しては次のように言わなければならない。原因が結果によって論
証されるとき、原因が存在することを論証するために、原因の定義の場所に、
結果を用いる必要がある。これは、神の場合に最もよく当てはまる。あるもの
が存在することを証明するためには、それの何であるか[=本質]ではなく、
その名称が何を意味しているか[=言葉の意味]を媒介[または中項]として
理解する必要がある。なぜなら、「何であるか」という問いは、「存在するか」
という問いの後に来るのだから。後に明らかになるように、神という名称は、
結果から付けられている。したがって、神が存在することを結果によって論証
するときに、この神という名称が何を意味しているかを、媒介として受け取る
ことができる。

\\



{\sc Ad tertium dicendum} quod per effectus non proportionatos causae,
non potest perfecta cognitio de causa haberi, sed tamen ex quocumque
effectu potest manifeste nobis demonstrari causam esse, ut dictum
est. Et sic ex effectibus Dei potest demonstrari Deum esse, licet per
eos non perfecte possimus eum cognoscere secundum suam essentiam.

&


第三異論に対しては次のように言わなければならない。原因に比例していない結果を
通して、その原因についての完全な認識を得ることはできない。しかし、すで
 に述べたように、どんな結果からでも、原因が存在することは、明らかに私た
 ちに論証されうる。したがって、神の結果から、神をその本質において完全に
 認識することはできないにしても、神が存在することは、論証できるのである。



\end{longtable}

\newpage
\rhead{a.~3}
\begin{center}
 {\Large {\bf ARTICULUS TERTIUS}}\\
 {\large UTRUM DEUS SIT}\\
 {\footnotesize I {\it Sent.}, d.~3, div.~prim.~part.~textus; I {\it
 SCG.}, c.~13, 15, 16, 41, 44; II, c.~15; III, c.~44; {\it De Verit.},
 q.~5, a.~2; {\it De Pot.}, q.~3, a.~5; {\it Compend.~Theol.}, c.~3; VII
 {\it Physic.}, l.~2; VIII, l.~9 sqq.; XII {\it Metaph.}, l.~5 sqq.}\\
 {\Large 第三項\\神は存在するか}
\end{center}

\begin{longtable}{p{21em}p{21em}}




{\Huge A}{\sc d tertium sic proceditur}. Videtur quod Deus
non sit. Quia si unum contrariorum fuerit infinitum, totaliter
destruetur aliud. Sed hoc intelligitur in hoc nomine Deus, scilicet quod
sit quoddam bonum infinitum. Si ergo Deus esset, nullum malum
inveniretur. Invenitur autem malum in mundo. Ergo Deus non est.


&

第三項の問題へ、議論は次のように進められる。神は存在しないと思われる。
相反するものの一方が無限であったら、もう一方は完全に破壊されるであろう。
ところが、「神」というこの名において、ある種、無限の善というものが理解
される。ゆえに、もしも神が存在するのなら、悪はなかったであろう。ところ
が、この世界には悪がある。ゆえに、神は存在しない。


\\


2.~{\sc  Praeterea}, quod potest compleri per pauciora
principia, non fit per plura. Sed videtur quod omnia quae apparent in
mundo, possunt compleri per alia principia, supposito quod Deus non sit,
quia ea quae sunt naturalia, reducuntur in principium quod est natura;
ea vero quae sunt a proposito, reducuntur in principium quod est ratio
humana vel voluntas. Nulla igitur necessitas est ponere Deum esse.


&

さらに、より少ない原理によって完成されうるものは、それよりも多い原理に
よって生じることはない。ところが、世界の中に現れているすべてのものは、
神が存在しないと仮定しても、他の原理によって完成されうるように思われる。
なぜなら、自然的なものは、自然という原理に還元される(自然という原理に
よって生じる)し、意図によるものは、人間の理性や意志という原理に還元さ
れるからである。ゆえに、神の存在を措定する必要性は何もない。



\\


{\sc Sed contra est} quod dicitur {\it Exodi} {\sc iii}, ex
persona Dei, {\it Ego sum qui sum}.


&

しかし反対に、『出エジプト記』3章で、神自身の口から「我は在りて在る者
(私は存在する者だ)」\footnote{「神はモーセに、「わたしはある。わたし
はあるという者だ」と言われ、また、「イスラエルの人々にこう言うがよい。
『わたしはある』という方が私をあなたたちに遣わされたのだと。」」
(3:14)}と言われている。


\\


{\sc Respondeo dicendum} quod Deum esse quinque viis probari
potest. Prima autem et manifestior via est, quae sumitur ex parte
motus. Certum est enim, et sensu constat, aliqua moveri in hoc
mundo. Omne autem quod movetur, ab alio movetur. Nihil enim movetur,
nisi secundum quod est in potentia ad illud ad quod movetur, movet
autem aliquid secundum quod est actu. Movere enim nihil aliud est quam
educere aliquid de potentia in actum, de potentia autem non potest
aliquid reduci in actum, nisi per aliquod ens in actu, sicut calidum
in actu, ut ignis, facit lignum, quod est calidum in potentia, esse
actu calidum, et per hoc movet et alterat ipsum. Non autem est
possibile ut idem sit simul in actu et potentia secundum idem, sed
solum secundum diversa, quod enim est calidum in actu, non potest
simul esse calidum in potentia, sed est simul frigidum in
potentia. Impossibile est ergo quod, secundum idem et eodem modo,
aliquid sit movens et motum, vel quod moveat seipsum. Omne ergo quod
movetur, oportet ab alio moveri.

& 

解答する。以下のように言われるべきである。神が存在することは五つの道に
よって証明可能である。第一の、そして比較的明らかな道は、運動の側面から
取られる。すなわち、この世界において、何かが動かされていることは確実で
あり、感覚にも明らかである。ところで、動かされるものはすべて、他者によっ
て動かされる。なぜなら、なにかが或るものへ動かされるのは、それがその或
るものにたいして可能態にある限りにおいてに他ならないからである。他方で、
動かすのは、現実態にある限りにおいてである。つまり、動かすとは、何かを、
可能態から現実態へ引き出すことに他ならない。ところが、何か現実態におい
て存在するものによらないかぎり、あるものが可能態から現実態へ引き出され
るということはありえない。たとえば、現実態において熱いもの、たとえば
「火」は、可能態において熱いもの、たとえば「木」を、現実に熱いものにし、
このことによって、木を動かし変化させる。ところで、一つのものが、同時に、
同じ観点から、現実態でありかつ可能態であることはできない。できるとすれ
ば、それは異なる観点からである。たとえば、現実態において熱いものは、同
時に可能態において熱いものではありえないが、同時に可能態において冷たい
ものではありうる。ゆえに、一つのものが同一の観点から、動かすものであり
かつ動かされるものであることはありえない。つまり、自己自身を動かすこと
はありえない。ゆえに、動かされるものはすべて、他者によって動かされる。

\\

Si ergo id a quo movetur, moveatur, oportet et ipsum ab alio moveri et
illud ab alio. Hic autem non est procedere in infinitum, quia sic non
esset aliquod primum movens; et per consequens nec aliquod aliud
movens, quia moventia secunda non movent nisi per hoc quod sunt mota a
primo movente, sicut baculus non movet nisi per hoc quod est motus a
manu. Ergo necesse est devenire ad aliquod primum movens, quod a nullo
movetur, et hoc omnes intelligunt Deum.


&

ゆえに、もしもAがBを動かすとき、Aが動かされているならば、Aもまた、別の
ものCによって動かされ、そのCも、別のものによって動かされることになる。
しかし、これが無限に進むことはない。なぜなら、もし無限に進んだりしたら、
どんな第一動者も存在しないことになり、その結果、その他のどんな動者も存
在しないことになるからである。なぜなら、二番目に動かすものは、一番目に
動かすものに動かされて(三番目のものを)動かすしかないからである。ちょ
うど、手が杖を動かすことによってでない限り、杖がものを動かすことはない
ように。ゆえに、何ものによっても動かされない何らかの第一の動者へ到達す
ることが必要であり、それをすべての人たちは神と理解する。



\\

Secunda via est ex ratione causae efficientis. Invenimus enim in istis
sensibilibus esse ordinem causarum efficientium, nec tamen invenitur,
nec est possibile, quod aliquid sit causa efficiens sui ipsius; quia
sic esset prius seipso, quod est impossibile. Non autem est possibile
quod in causis efficientibus procedatur in infinitum. Quia in omnibus
causis efficientibus ordinatis, primum est causa medii, et medium est
causa ultimi, sive media sint plura sive unum tantum, remota autem
causa, removetur effectus, ergo, si non fuerit primum in causis
efficientibus, non erit ultimum nec medium. Sed si procedatur in
infinitum in causis efficientibus, non erit prima causa efficiens, et
sic non erit nec effectus ultimus, nec causae efficientes mediae, quod
patet esse falsum. Ergo est necesse ponere aliquam causam efficientem
primam, quam omnes Deum nominant.



&

第二の道は、作出因の観点からのものである。あなたの周囲にあって感覚され
ているもののなかには作出因の系列が存在するということに、わたしたちは気
付く。ところで、あるものが、自分自身の作出因であるところを見かけること
はないし、またそんなことはありえない。なぜなら、もしそのようなことがあ
れば、自分が自分よりも先に存在することになるのであって、それは不可能だ
からである。しかし、作出因の系列を無限にさかのぼることはできない。なぜ
なら、すべての作出因を並べると、第一のものは中間のものの原因であり、中
間のものは最後のものの原因である。このことは、中間のものが複数あろうと
一つだけだろうと変わらない。しかし、原因が無くなれば、その結果も無くな
る。ゆえに、もしも作出因の中の第一のものが無ければ、最後のものも中間の
ものも無いであろう。しかし、もしも作出因の中を無限にさかのぼり、第一の
作出因が無いとしたら、最後の作出因も、中間の作出因も存在しないことにな
るが、これは明らかに偽である。ゆえに、何らかのものを第一の作出因としな
ければならない。それをすべての人は神と名付けている。



\\

Tertia via est sumpta ex possibili et necessario, quae talis
est. Invenimus enim in rebus quaedam quae sunt possibilia esse et non
esse, cum quaedam inveniantur generari et corrumpi, et per consequens
possibilia esse et non esse. Impossibile est autem omnia quae sunt,
talia esse\footnote{Leoninaは、``omnia quae sunt talia, semper esse''
となっている。}, quia quod possibile est non esse, quandoque non
est. Si igitur omnia sunt possibilia non esse, aliquando nihil fuit in
rebus. Sed si hoc est verum, etiam nunc nihil esset, quia quod non
est, non incipit esse nisi per aliquid quod est; si igitur nihil fuit
ens, impossibile fuit quod aliquid inciperet esse, et sic modo nihil
esset, quod patet esse falsum. Non ergo omnia entia sunt possibilia,
sed oportet aliquid esse necessarium in rebus. Omne autem necessarium
vel habet causam suae necessitatis aliunde, vel non habet. Non est
autem possibile quod procedatur in infinitum in necessariis quae
habent causam suae necessitatis, sicut nec in causis efficientibus, ut
probatum est. Ergo necesse est ponere aliquid quod sit per se
necessarium, non habens causam necessitatis aliunde, sed quod est
causa necessitatis aliis, quod omnes dicunt Deum.


& 


第三の道は、可能なものと必然的なものという観点からであるが、それは次の
とおり。わたしたちは事物の中に、存在することも存在しないことも可能であ
るようなものを見出す。というのも、あるものは、生成・消滅する、つまり結
果的に、存在することも存在しないことも可能であることが見出されるからで
ある。ところで、存在するすべてのものが、そのようなもの(=存在すること
も存在しないことも可能なもの)であることは不可能である\footnote{レオニ
ナ版の読みを取れば「そのようなものすべてが、常に存在することは不可能で
ある」となる。}。なぜなら、存在しないことが可能なものは、現に存在しな
いことがあるからである。それゆえ、すべてのものが存在しないことが可能な
ものならば、現実に何も存在しなかった時点が過去にある。しかし、もしこれ
(「現実に何も存在しなかった過去の時点がある」こと)が真なら、今も、何
も存在しなかったであろう。なぜなら、存在しないものは、何か他の存在して
いるものによらなければ、存在し始めることはないからである。ゆえに、もし
も、存在しているものが何もないならば、何かが存在し始めることは不可能だ
し、その結果、今も何も存在しなかったであろう。これは明らかに偽である。
ゆえに、すべての存在者が可能的なものであるわけではなく、事物において何
かが必然的なものでなければならない。ところで、すべて、必然的なものは、
自らの必然性を、他のどこかから受け取るか、それともそうでないかのどちら
かである。しかし、自らの必然性の原因を持つ必然的なものどもの中を、無限
に進むことは不可能である。これは、すでに証明されたように、作出因の中と
同じである。ゆえに、それ自身によって必然的である何かを措定する必要があ
る。それは、他のどこかから必然性の原因を受け取るのではなく、それ自身が、
他の必然的なものにとっての原因である。すべての人はこれを神という。



\\


Quarta via sumitur ex gradibus qui in rebus inveniuntur. Invenitur
enim in rebus aliquid magis et minus bonum, et verum, et nobile, et
sic de aliis huiusmodi. Sed {\it magis} et {\it minus} dicuntur de
diversis secundum quod appropinquant diversimode ad aliquid quod
maxime est, sicut magis calidum est, quod magis appropinquat maxime
calido. Est igitur aliquid quod est verissimum, et optimum, et
nobilissimum, et per consequens maxime ens, nam quae sunt maxime vera,
sunt maxime entia, ut dicitur II {\it Metaphys}. Quod autem dicitur
maxime tale in aliquo genere, est causa omnium quae sunt illius
generis, sicut ignis, qui est maxime calidus, est causa omnium
calidorum, ut in eodem libro dicitur. Ergo est aliquid quod omnibus
entibus est causa esse, et bonitatis, et cuiuslibet perfectionis, et
hoc dicimus Deum.


&


第四の道は、物事の中に見つけられる階級の観点からのものである。物事の中
には、善、真、高貴、などということについて、より多く・より少なくという
ことが見いだされる。しかし、さまざまなものについて、より多く・より少な
く、ということは、何か最大限にそうであるものにどれだけ近づいているかに
応じていうのである。たとえば、より熱いものとは、もっとも熱いものにより
近づいているものである。ゆえに、何かが、もっとも真であるもの、もっとも
善いもの、もっとも高貴であるものであり、それは結局、最大限に存在するも
のである。なぜなら、『形而上学』第2巻で述べられているように、最大限に
真であるものは最大限に存在するものだからである。ところで、ある類の中で
最大限にこれこれであると言われるものは、その類に含まれるすべてのものの
原因である。たとえば、火は最大限に熱いものなので、「熱いもの」という類
に含まれるすべてのものの原因である。これも『形而上学』で言われている。
ゆえに、すべての存在するものにとって、その存在、善性、その他何であれ完
全性の原因である何かが存在する。わたしたちは、これを神という。


\\


Quinta via sumitur ex gubernatione rerum. Videmus enim quod aliqua
quae cognitione carent, scilicet corpora naturalia, operantur propter
finem, quod apparet ex hoc quod semper aut frequentius eodem modo
operantur, ut consequantur id quod est optimum; unde patet quod non a
casu, sed ex intentione perveniunt ad finem. Ea autem quae non habent
cognitionem, non tendunt in finem nisi directa ab aliquo cognoscente
et intelligente, sicut sagitta a sagittante. Ergo est aliquid
intelligens, a quo omnes res naturales ordinantur ad finem, et hoc
dicimus Deum.


&




第五の道は、諸事物の統治という観点からである。わたしたちは、認識を欠く
もの、つまり自然物が、目的のために働くのを見る。このことは、最善のこと
がらが生じるように、常に、頻繁に、同じ仕方で働くことからも明らかである。
したがって、それらは偶然に動いているのではなく、意図的に、目的へ到達し
ているのである。ところで、認識をもたないものは、ちょうど矢が射手によっ
て[的へ向けて射られるように]、だれか認識する者や知性認識する者によっ
て向けられるのでない限り、目的へとは向かわない。ゆえに、何らかの知性を
持つものが存在し、その者によって、すべての自然物は、目的へと向かってい
る。これをわたしたちは神という。


\\


{\sc Ad primum ergo dicendum} quod, sicut dicit Augustinus in {\it
Enchiridio}: {\it Deus, cum sit summe bonus, nullo modo sineret
aliquid mali esse in operibus suis, nisi esset adeo omnipotens et
bonus, ut bene faceret etiam de malo}. Hoc ergo ad infinitam Dei
bonitatem pertinet, ut esse permittat mala, et ex eis eliciat bona.


&

第一異論に対しては、それゆえ、次のように言わなければならない。アウグス
ティヌスが『提要』で述べているように、「神は、最高度に善であり、悪から
さえも善を作るほどに全能で善でなかったならば、自分の業の中にいかなる悪
も決して存在させなかったであろう」。ゆえに、悪の存在を赦し、悪から膳を
引き出すことは、神の無限の善性に属するのである。


\\


{\sc Ad secundum dicendum} quod, cum natura propter determinatum finem
operetur ex directione alicuius superioris agentis, necesse est ea
quae a natura fiunt, etiam in Deum reducere, sicut in primam
causam. Similiter etiam quae ex proposito fiunt, oportet reducere in
aliquam altiorem causam, quae non sit ratio et voluntas humana, quia
haec mutabilia sunt et defectibilia; oportet autem omnia mobilia et
deficere possibilia reduci in aliquod primum principium immobile et
per se necessarium, sicut ostensum est.


&

第二異論に対しては、次のように言わなければならない。自然が決まった目的
のために働くのは、何か上位の作用者の導きによるのだから、自然から生じる
ものは、やはり第一原因としての神に導かれているのである。同様に、意図的
になされることも、より高い何らかの原因に導かれるのだが、それは人間の理
性や意志ではない。なぜなら、人間の理性や意志は、変わりやすく欠陥を持ち
やすいのだが、すべて変化し欠陥を持ちうるものは、不動の、そしてそれ自体
によって必然的である何らかの第一の根源に導かれる必要があるからである。
これはすでに述べられたことである。

\end{longtable}
\end{document}