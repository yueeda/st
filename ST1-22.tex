\documentclass[10pt]{jsarticle} % use larger type; default would be 10pt
%\usepackage[utf8]{inputenc} % set input encoding (not needed with XeLaTeX)
%\usepackage[round,comma,authoryear]{natbib}
%\usepackage{nruby}
\usepackage{okumacro}
\usepackage{longtable}
%\usepqckage{tablefootnote}
\usepackage[polutonikogreek,english,japanese]{babel}
%\usepackage{amsmath}
\usepackage{latexsym}
\usepackage{color}

%----- header -------
\usepackage{fancyhdr}
\lhead{{\it Summa Theologiae} I, q.~22}
%--------------------

\bibliographystyle{jplain}

\title{{\bf PRIMA PARS}\\{\HUGE Summae Theologiae}\\Sancti Thomae
Aquinatis\\{\sffamily QUEAESTIO VIGESIMASECUNDA}\\DE PROVIDENTIA DEI}
\author{Japanese translation\\by Yoshinori {\sc Ueeda}}
\date{Last modified \today}


%%%% コピペ用
%\rhead{a.~}
%\begin{center}
% {\Large {\bf }}\\
% {\large }\\
% {\footnotesize }\\
% {\Large \\}
%\end{center}
%
%\begin{longtable}{p{21em}p{21em}}
%
%&
%
%
%\\
%\end{longtable}
%\newpage



\begin{document}
\maketitle
\pagestyle{fancy}

\begin{center}
{\Large 第二十二問\\神の摂理について}
\end{center}


\begin{longtable}{p{21em}p{21em}}
Consideratis autem his quae ad voluntatem absolute pertinent,
 procedendum est ad ea quae respiciunt simul intellectum et
 voluntatem. Huiusmodi autem est providentia quidem respectu omnium;
 praedestinatio vero et reprobatio, et quae ad haec consequuntur,
 respectu hominum specialiter, in ordine ad aeternam salutem. Nam et
 post morales virtutes, in scientia morali, consideratur de prudentia,
 ad quam providentia pertinere videtur. 


&

さて、無条件的に意志に属する事柄が考察されたので、知性と意志とに同時に
 関係する事柄へと進まなければならない。ところで、「摂理」は、すべての
 事柄にかかわるが、「予定」「劫罰」、そして、これらに伴うものは、永遠
 の救済への秩序において、とくに人間にかかわる。じっさい、倫理的学知に
 おいても、倫理的諸徳の後に、摂理がそれに属すると思われる思慮が考察さ
 れる。\footnote{倫理的徳のあとで、知性的徳が考察される、『ニコマコス
 倫理学』での順序を念頭に置いているか。}

\\

Circa providentiam autem Dei
 quaeruntur quatuor. 

\begin{enumerate}
 \item utrum Deo conveniat providentia.
 \item utrum omnia divinae providentiae subsint.
 \item utrum divina providentia immediate sit de omnibus.
 \item utrum providentia divina imponat necessitatem rebus provisis.
\end{enumerate}

&

さて、神の摂理について、四つのことが問われる。

\begin{enumerate}
 \item 摂理は神に適合するか。
 \item 万物が神の摂理に服するか。
 \item 神の摂理は、万物について、直接的であるか。
 \item 神の摂理は、摂理される諸事物に必然性を与えるか。
\end{enumerate}


\end{longtable}
\newpage


\rhead{a.~1}
\begin{center}
 {\Large {\bf ARTICULUS PRIMUS}}\\
 {\large UTRUM PROVIDENTIA DEO CONVENIAT}\\
 {\footnotesize I {\itshape Sent.}, d.34, q.2, a.1; {\itshape De
 Verit.}, q.5, a.1, 2.}\\
 {\Large 第一項\\摂理は神に適合するか}
\end{center}

\begin{longtable}{p{21em}p{21em}}




{\Huge A}{\scshape d primum sic proceditur}. Videtur quod
providentia Deo non conveniat. Providentia enim, secundum Tullium, est
pars prudentiae. Prudentia autem, cum sit bene consiliativa, secundum
philosophum in VI {\itshape Ethic}., Deo competere non potest, qui nullum dubium
habet, unde eum consiliari oporteat. Ergo providentia Deo non competit.


&

第一項の問題へ、議論は以下のように進められる。摂理は神に適合しないと思
われる。理由は以下の通り。摂理は、キケロによれば、思慮の部分である。と
ころで、思慮は、『ニコマコス倫理学』第6巻の哲学者によれば、よく思量し
うるもだから、神に適合しえない。なぜなら、疑いを持つから思量しなければ
ならないのだが、神はいかなる疑いも持たないからである。ゆえに、摂理は神
に適合しない。

\\


{\scshape 2 Praeterea}, quidquid est in Deo, est
aeternum. Sed providentia non est aliquid aeternum, est enim {\itshape circa
existentia}, quae non sunt aeterna, secundum Damascenum. Ergo providentia
non est in Deo.


&

さらに、何であれ神の中にあるものは永遠である。ところが、摂理は、何か永
遠なものではない。なぜなら、ダマスケヌスによれば、それは「存在するもの
どもにかかわる」が、それらは永遠でないからである。ゆえに、摂理は神の中
にない。


\\


{\scshape 3 Praeterea}, nullum compositum est in
Deo. Sed providentia videtur esse aliquid compositum, quia includit in
se voluntatem et intellectum. Ergo providentia non est in Deo.


&

さらに、神の中にはいかなる複合体もない。ところが、摂理は、何か複合され
たものであると思われる。なぜなら、その中に、意志と知性を含むからである。
ゆえに、摂理は神の中にない。


\\


{\scshape Sed contra est} quod dicitur Sap.~{\scshape xiv}, {\itshape tu
autem, pater, gubernas omnia providentia}.


&

しかし反対に、『知恵の書』14章で「父よ、あなたは、万物を摂理によって導く
 」\footnote{「しかし父よ、船を導くのはあなたの摂理。」(14:3)}と言われている。


\\


{\scshape Respondeo dicendum} quod necesse est ponere
providentiam in Deo. Omne enim bonum quod est in rebus, a Deo creatum
est, ut supra ostensum est. In rebus autem invenitur bonum, non solum
quantum ad substantiam rerum, sed etiam quantum ad ordinem earum in
finem, et praecipue in finem ultimum, qui est bonitas divina, ut supra
habitum est. Hoc igitur bonum ordinis in rebus creatis existens, a Deo
creatum est. Cum autem Deus sit causa rerum per suum intellectum, et sic
cuiuslibet sui effectus oportet rationem in ipso praeexistere, ut ex
superioribus patet; necesse est quod ratio ordinis rerum in finem in
mente divina praeexistat. 

&

解答する。以下のように言われるべきである。神の中に摂理を置くことは必然
である。理由は以下の通り。前に示されたとおり\footnote{第6問第4項「万物
は神の善性によって善であるか」}、諸事物の中にあるあらゆる善は、神によっ
て創造された。ところで、諸事物の中には、諸事物の実体にかんしてだけでな
く、それらの目的への秩序、とくに究極目的、これは、前に述べられたとおり
\footnote{第21問第4項主文。 ``Et cum non sit procedere in infinitum,
oportet devenire ad aliquid quod ex sola bonitate divinae voluntatis
dependeat, quae est ultimus finis.''}、神の善性であるが、への秩序にか
んしてもまた、善が見出される。ゆえに、創造された諸事物の中にある、秩序
のこの善は、神によって創造された。ところで、神は、自分の知性を通して、
諸事物の原因だから、その意味で、前に述べられたことから明らかなとおり
\footnote{第15問第4項「複数のイデアが存在するか」、第19問第4項「神の意
志は事物の原因か」。}、自分のどんな結果の理拠(ratio)もまた、神自身の中
にあらかじめ存在しなければならない。



\\


Ratio autem ordinandorum in finem, proprie
providentia est. Est enim principalis pars prudentiae, ad quam aliae
duae partes ordinantur, scilicet memoria praeteritorum, et intelligentia
praesentium; prout ex praeteritis memoratis, et praesentibus
intellectis, coniectamus de futuris providendis. 

&

ところで、目的へと秩序づけられるべき事についての理解・理拠(ratio)が、
固有の意味で、摂理・予知(providentia)\footnote{providentiaの語源である
providereは、「前を見る」「予見する」の意。キリスト教の文脈で「摂理」
と訳されるが、「予知」の意味合いが濃厚であることが、この主文の説明でも
わかる。}である。じっさい、それは思慮の主要な部分であり、他の二つの部
分、すなわち、過去の事柄の記憶と、現在の事柄の知解とが、それに秩序づけ
られる。つまり、記憶された過去の事柄と、知解された現在の事柄に基づいて、
予知されるべき将来の事柄について、推測する。

\\

Prudentiae autem
proprium est, secundum philosophum in VI {\itshape Ethic}., ordinare alia in finem;
sive respectu sui ipsius, sicut dicitur homo prudens, qui bene ordinat
actus suos ad finem vitae suae; sive respectu aliorum sibi subiectorum
in familia vel civitate vel regno, secundum quem modum dicitur
Matt.~{\scshape xxiv}, {\itshape fidelis servus et prudens, quem constituit dominus super
familiam suam}. Secundum quem modum prudentia vel providentia Deo
convenire potest, nam in ipso Deo nihil est in finem ordinabile, cum
ipse sit finis ultimus. Ipsa igitur ratio ordinis rerum in finem,
 providentia in Deo nominatur.
 
&

ところで、『ニコマコス倫理学』第6巻の哲学者によれば、思慮に固有なこと
は、他のものを目的へ秩序づけることであるが、それは、自分の行為を自分の
人生の目的へとよく秩序づける人が、思慮がある人と言われるように、自分自
身にかんしてでも、あるいは、家族、国家、王国において、自分に従属する他
の人々にかんしてでもよい。この後者の意味で、『マタイによる福音書』24章
「信用があり思慮がある奴隷、主人は彼を、自分の家族の上に置いた」
\footnote{「主人がその家の使用人たちの上に立てて、時間通りに彼らに食事
を与えさせることにした忠実で賢い僕は、いったいだれであろうか」(24:45)}
と言われている。そして、この後者の意味で、思慮や摂理が、神に適合しうる。
なぜなら、神自身が究極目的だから、神自身の中には、目的へと秩序づけられ
うるものはないからである。ゆえに、諸事物の目的への秩序の理解・理拠が、
神において、摂理と名付けられる。


\\

Unde Boetius, IV {\itshape de Consol}., dicit quod
{\itshape providentia est ipsa divina ratio in summo omnium principe constituta,
quae cuncta disponit}. Dispositio autem potest dici tam ratio ordinis
rerum in finem, quam ratio ordinis partium in toto.

&

このことから、ボエティウスは『哲学の慰め』4巻で「節理とは、万物の中で
最高の第一人者の中に作られた、万物を配置する神の理解・理拠そのものであ
る」と述べる。配置は、諸部分の全体への秩序の理解という意味でも、諸事物
の目的への秩序の理解という意味でも、言われうる。

\\


{\scshape Ad primum ergo dicendum} quod, secundum
philosophum in VI {\itshape Ethic}., prudentia proprie est praeceptiva\footnote{praecipereは、あらかじめ(prae)とらえる、考える(capere)の意だが、転じて、指導する、命じる、を意味する。} eorum, de
quibus eubulia recte consiliatur, et synesis recte iudicat. Unde, licet
consiliari non competat Deo, secundum quod consilium est inquisitio de
rebus dubiis; tamen praecipere de ordinandis in finem, quorum rectam
rationem habet, competit Deo, secundum illud Psalmi, {\itshape praeceptum posuit,
et non praeteribit}. Et secundum hoc competit Deo ratio prudentiae et
providentiae. 


&


第一異論に対しては、それゆえ、以下のように言われるべきである。『ニコマ
コス倫理学』第6巻の哲学者によれば、思慮は、厳密に言えば、「すぐれた熟
慮」(eubulia)が正しく思量し、「理解力」(synesis)\footnote{eubulia,
synesisの訳語については朴訳(『ニコマコス倫理学』西洋古典叢書、京都大
学学術出版会)にしたがう。}が正しく判断する事柄を、あらかじめ考えうる
ものである。したがって、思量が、疑わしい事柄についての探究であるかぎり
で、思量することが神に適合しないとしても、しかし、目的へ秩序づけられる
べき事柄、それらについての正しい理拠(ratio)を神はもつのだが、を、あら
かじめ考えることは、神に適合する。これは、かの『詩編』「彼はあらかじめ
考えられたものを置いた。そして彼はそれを無視しないだろう」
\footnote{「主はそれらを世々限りなく立て、越ええない掟を与えられた」
(148:6)}による。そしてこの限りで、神に、「思慮」「摂理」という性格が適
合する。


\\


-- Quamvis etiam dici possit, quod ipsa ratio rerum agendarum
consilium in Deo dicitur; non propter inquisitionem, sed propter
certitudinem cognitionis, ad quam consiliantes inquirendo
perveniunt. Unde dicitur {\itshape Ephes}.~{\scshape i}, {\itshape qui operatur omnia secundum consilium
voluntatis suae}.

&

ただし、さらに以下のように言われうる。なされるべき事柄の理拠が、神にお
いて思量と言われるのは、神がそれを探究するからではなく、認識の確実性の
ためにである。思量する人々は、探究することによって、この確実性に到達す
る。このことから、『エフェソの信徒への手紙』第1章で、「すべてのことを、
自分の意志の思量にしたがって行う方」\footnote{「キリストにおいてわたし
たちは、御心のままにすべてのことを行われる方の御計画によって前もって定
められ、約束されたものの相続者とされました」(1:11)}と言われる。



\\


{\scshape Ad secundum dicendum} quod ad curam duo
pertinent, scilicet ratio ordinis, quae dicitur providentia et
dispositio; et executio ordinis, quae dicitur gubernatio. Quorum primum
est aeternum, secundum temporale.


&

第二異論に対しては、以下のように言われるべきである。配慮には二つのもの
が属する。一つは秩序の理拠であり、これが摂理や配置と呼ばれる。もう一つ
は、秩序を遂行することであり、これが統治と言われる。このうち、前者は永
遠だが、後者は時間的である。


\\


{\scshape Ad tertium dicendum} quod providentia est in
intellectu, sed praesupponit voluntatem finis, nullus enim praecipit de
agendis propter finem, nisi velit finem. Unde et prudentia praesupponit
virtutes morales, per quas appetitus se habet ad bonum, ut dicitur in VI
{\itshape Ethic}. Et tamen si providentia ex aequali respiceret voluntatem et
intellectum divinum, hoc esset absque detrimento divinae simplicitatis;
cum voluntas et intellectus in Deo sint idem, ut supra dictum est.


&


第三異論に対しては、以下のように言われるべきである。摂理は知性の中にあ
るが、目的への意志を前提とする。なぜなら、誰も、目的を意志しないならば、
目的のために、なされるべき事柄について、あらかじめ考えることはないから
である。したがって、思慮も、道徳的な徳を前提とする。『ニコマコス倫理学』
第6巻によれば、欲求は、それらの道徳的な徳を通して善に関係する。しかし、
かりにもし摂理が、神の意志と知性に等しく関係したとしても、それは神の単
純性を損なわずにそうだったであろう。というのも、前に述べられたとおり
\footnote{第19問第1項主文。``Et sicut suum intelligere est suum esse,
ita suum velle.'' および同第4項第2異論解答。``quia essentia Dei est
eius intelligere et velle, ex hoc ipso quod per essentiam suam agit,
sequitur quod agat per modum intellectus et voluntatis.''}、意志と知性
は、神において、同一だからである。


\end{longtable}
\newpage

\rhead{a.~2}
\begin{center}
 {\Large {\bf ARTICULUS SECUNDUS}}\\
 {\large UTRUM OMNIA SINT SUBIECTA DIVINA PROVIDENTIAE}\\
 {\footnotesize Infra, q.103, a.5; I {\itshape Sent.}, d.39, q.2, a.2;
 III {\itshape SCG}, cap.1, 64, 75, 94; {\itshape De Verit.}, q.5, a.2,
 sqq.; {\itshape Compend.~Theol.}, cap.123, 132; Opusc.~XV, {\itshape De
 Angelis}, cap.13, 14, 15; {\itshape De Div.~Nom.}, cap.3, lect.1.}\\
 {\Large 第二項\\万物が神の摂理のもとにあるか}
\end{center}

\begin{longtable}{p{21em}p{21em}}

{\Huge A}{\scshape d secundum sic proceditur}. Videtur quod
non omnia sint subiecta divinae providentiae. Nullum enim provisum est
fortuitum. Si ergo omnia sunt provisa a Deo, nihil erit fortuitum, et
sic perit casus et fortuna. Quod est contra communem opinionem.

&


第二項の問題へ、議論は以下のように進められる。万物が神の摂理のもとにあ
るわけではないと思われる。理由は以下の通り。偶然的なものは、なにも、予
見・摂理されない。ゆえに、もしも万物が神に摂理されていたら、偶然的なも
のはなくなり、かくして偶々や運はなくなるだろう。これは、共通の意見に反
している。

\\


{\scshape 2 Praeterea}, omnis sapiens provisor excludit
defectum et malum, quantum potest, ab his quorum curam gerit. Videmus
autem multa mala in rebus esse. Aut igitur Deus non potest ea impedire,
et sic non est omnipotens, aut non de omnibus curam habet.

&


さらに、摂理するすべての知者は、配慮するものから欠陥や悪をできるだけ取
り除く。ところが、わたしたちは諸事物の中に多くの悪があるのを見る。ゆえ
に、神は、それらを妨げることができず、したがって全能でないか、あるいは、
万物について配慮していないかのどちらかである。

\\


{\scshape 3 Praeterea}, quae ex necessitate eveniunt,
providentiam seu prudentiam non requirunt, unde, secundum philosophum in
VI {\itshape Ethic}., prudentia est recta ratio contingentium, de quibus est
consilium et electio. Cum igitur multa in rebus ex necessitate eveniant,
non omnia providentiae subduntur.

&


さらに、必然的に生じるものは、摂理や思慮を必要としない。したがって、
『ニコマコス倫理学』第6巻の哲学者によれば、思慮とは偶然的なことがらに
ついての正しい理性であり、思量と選択がそれにかかわる。ゆえに、事物にお
いて多くのことが必然的に生じているので、すべてが摂理のもとにあるわけで
はない。

\\


{\scshape 4 Praeterea}, quicumque dimittitur sibi, non
subest providentiae alicuius gubernantis. Sed homines sibi ipsis
dimittuntur a Deo, secundum illud {\itshape Eccli}.~{\scshape xv}: {\itshape Deus ab initio constituit
hominem}, et reliquit eum in manu consilii sui; et specialiter mali,
secundum illud, {\itshape dimisit illos secundum desideria cordis eorum}. Non
igitur omnia divinae providentiae subsunt.

&

さらに、誰であれその人自身に委ねられている者は、支配するだれかの摂理の
もとにはない。ところが、かの『シラ書』15章の「神は、初めから、人間を造
り、彼を自分の思量の手に残した」\footnote{「主が初めに人間を造られたと
き、自分で判断する力をお与えになった」(15:14)}によれば、人間は、神によっ
て、自分自身に委ねられている。そしてとくに、「彼らを彼らの心の願望に従っ
て委ねた」\footnote{「私は頑なな心の彼らを突き放し、思いのままに歩かせ
た」『詩編』(81:13)}によれば、悪人たちはそうである。ゆえに、万物が神の
摂理のもとにあるわけではない。

\\


{\scshape 5 Praeterea}, apostolus, I {\itshape Cor}.~{\scshape ix}, dicit quod
{\itshape non est Deo cura de bobus}, et eadem ratione, de aliis creaturis
irrationalibus. Non igitur omnia subsunt divinae providentiae.

&

さらに、使徒は『コリントの信徒への手紙一』9章で「神には牛についての配
慮はない」\footnote{「神が心にかけておられるのは、牛のことですか」
(9:9)}と言っている。同じ理由で、他の非理性的被造物についての配慮もない。
ゆえに、万物が神の摂理のもとにあるわけではない。


\\


{\scshape Sed contra} est quod dicitur {\itshape Sap}.~{\scshape viii}, de
divina sapientia, quod {\itshape attingit a fine usque ad finem fortiter, et
disponit omnia suaviter}.

&

しかし反対に、『知恵の書』8章で、神の知恵について「力強く端から端まで
達し、万物を心地よく配置する」\footnote{「知恵は地の果てから果てまでそ
の力を及ぼし、慈しみ深くすべてをつかさどる」(8:1)}と言われている。

\\


{\scshape Respondeo dicendum} quod quidam totaliter
providentiam negaverunt, sicut Democritus et Epicurei, ponentes mundum
factum esse casu. Quidam vero posuerunt incorruptibilia tantum
providentiae subiacere; corruptibilia vero, non secundum individua, sed
secundum species; sic enim incorruptibilia sunt. Ex quorum persona
dicitur {\itshape Iob} {\scshape xxii}, {\itshape nubes latibulum eius, et circa cardines caeli
perambulat, neque nostra considerat}. A corruptibilium autem generalitate
excepit Rabbi Moyses homines, propter splendorem intellectus, quem
participant, in aliis autem individuis corruptibilibus, aliorum
opinionem est secutus. 



&

解答する。以下のように言われるべきである。デモクリトスやエピクロスのよ
うな、ある人々は、世界が偶然に造られたと言って、摂理を全面的に否定した。
他方、ある人々は、不滅なものどもだけが摂理のもとにあると主張した。また、
可滅的なものどもは、個に即してではなく、種に即して摂理のもとにあり、そ
のかぎりで不滅であると論じた。この人々を代表して、『ヨブ記』22章の「雲
は彼の隠れ家。天の極をあちこち歩いて、私たちのことを考えはしない」
\footnote{「雲に遮られて見ることもできず、天の丸天井を行き来されるだけ
だ」(22:14)}が語られている。しかしマイモニデスは、可滅的なもの一般から、
人間を、分有している知性の輝きのために、除外したが、その他の可滅的な個
体については、他の人々の意見に従った。


\\



Sed necesse est dicere omnia divinae providentiae
subiacere, non in universali tantum, sed etiam in singulari. Quod sic
patet. Cum enim omne agens agat propter finem, tantum se extendit
ordinatio effectuum in finem, quantum se extendit causalitas primi
agentis. 
Ex hoc enim contingit in operibus alicuius agentis aliquid
provenire non ad finem ordinatum, quia effectus ille consequitur ex
aliqua alia causa, praeter intentionem agentis.

&

しかし、万物が、普遍においてだけでなく、個物においても、神の摂理のもと
にあると言うことが必要である。これは以下のようにして明らかである。すべ
て作用するものは、目的のために作用するので、諸結果を目的へ秩序付ける働
きは、第一作用者の原因性が達するところまで及ぶ。実際、ある作用者の働き
において、秩序付けられた目的に向かってではないかたちで、何かが生じると
いうことは、その結果が、作用者の意図を外れて、何か他の原因から出てくる
ことによって起こる。




\\


 Causalitas autem Dei,
qui est primum agens, se extendit usque ad omnia entia, non solum
quantum ad principia speciei, sed etiam quantum ad individualia
principia, non solum incorruptibilium, sed etiam corruptibilium.



&

ところが、第一作用者である神の原因性は、種の根源だけでなく、個の根源に
かんしても、また、不滅的なものの根源だけでなく、可滅的なものの根源にか
んしても、すべての存在するものにまで及ぶ。


\\

 Unde
necesse est omnia quae habent quocumque modo esse, ordinata esse a Deo
in finem, secundum illud apostoli, {\itshape ad Rom}.~{\scshape xiii}: {\itshape quae a Deo sunt,
ordinata sunt}. Cum ergo nihil aliud sit Dei providentia quam ratio
ordinis rerum in finem, ut dictum est, necesse est omnia, inquantum
participant esse, intantum subdi divinae providentiae. 

&

したがって、使徒の、かの『ローマの信徒への手紙』13章「神によってあるも
のは、秩序付けられている」\footnote{「神に由来しない権威はなく、今ある
権威はすべて神によって立てられたものだからです」(13:1)}によれば、どん
なかたちであれ存在を持つものはすべて、神によって目的へ秩序付けられてい
るのでなければならない。ゆえに、すでに述べられたとおり、神の摂理とは、
諸事物の、目的への秩序の理念に他ならないから、万物は、存在を分有するか
ぎり、神の摂理のもとにあることが必然である。


\\


Similiter etiam
supra ostensum est quod Deus omnia cognoscit, et universalia et
particularia. Et cum cognitio eius comparetur ad res sicut cognitio
artis ad artificiata, ut supra dictum est, necesse est quod omnia
supponantur suo ordini, sicut omnia artificiata subduntur ordini artis.

&

同様に、神は普遍的なものも個別的なものも、すべてのものを認識することが、
前に示された。そして、前に述べられたとおり、神の認識は、事物に対して、
ちょうど、技術者が技術作品に対するように関係するので、ちょうど、すべて
の技術作品が技術のもとに服するように、万物は神の秩序に服するのが必然で
ある。

\\


{\scshape Ad primum ergo dicendum} quod aliter est de
causa universali, et de causa particulari. Ordinem enim causae
particularis aliquid potest exire, non autem ordinem causae
universalis. 

&

第一異論に対しては、それゆえ、以下のように言われるべきである。普遍的原
因と個別的原因とはありかたが異なる。すなわち、何かあるものが、個別的原
因の秩序を離れることはありうるが、普遍的原因の秩序を離れることはありえ
ない。

\\


Non enim subducitur aliquid ab ordine causae particularis,
nisi per aliquam aliam causam particularem impedientem, sicut lignum
impeditur a combustione per actionem aquae. Unde, cum omnes causae
particulares concludantur sub universali causa, impossibile est aliquem
effectum ordinem causae universalis effugere. 


&

あるものは、何か他の、阻害する個別的原因によってでなければ、個別的原因
の秩序から外されることはない。たとえば、木材が、水の作用によって、燃焼
から妨げられるように。したがって、すべての個別的原因は、普遍的原因に含
まれるから、何らかの結果が、普遍的原因の秩序から逃れることは不可能であ
る。


\\


Inquantum igitur aliquis
effectus ordinem alicuius causae particularis effugit, dicitur esse
casuale vel fortuitum, respectu causae particularis, sed respectu causae
universalis, a cuius ordine subtrahi non potest, dicitur esse
provisum. 


&

ゆえに、ある結果が、何らかの個別的原因の秩序から逃れる限りにおいて、そ
の個別的原因との関係で、偶然的とか、偶運的と言われるが、しかし、普遍的
原因との関係では、それの秩序から引き離されることはありえないので、予見・
摂理されている、と言われる。


\\


Sicut et concursus duorum servorum, licet sit casualis quantum
ad eos, est tamen provisus a domino, qui eos scienter sic ad unum locum
mittit, ut unus de alio nesciat.



&

たとえば、二人の奴隷が出会うことも、彼らにかんする限りでは偶然でも、主
人が彼らを、わざと、それぞれが他方のことを知らないように、一つの場所に
行かせるとしたら、その主人によっては、予見・摂理されているように。


\\




{\scshape Ad secundum dicendum} quod aliter de eo est
qui habet curam alicuius particularis, et de provisore universali. Quia
provisor particularis excludit defectum ab eo quod eius curae subditur,
quantum potest, sed provisor universalis permittit aliquem defectum in
aliquo particulari accidere, ne impediatur bonum totius. 

&

第二異論に対しては、以下のように言われるべきである。
ある個別的なものへの配慮を持つ者と、普遍的に摂理を行う者とは、あり方が異なる。
個別的な摂理を行う者は、可能な限り、配慮するものから欠陥を取り除くが、普
 遍的な摂理を行う者は、全体の善が妨げられないように、何らかの欠陥が、あ
 る個別のものに生じることを認める。



\\


Unde
corruptiones et defectus in rebus naturalibus, dicuntur esse contra
naturam particularem; sed tamen sunt de intentione naturae universalis,
inquantum defectus unius cedit in bonum alterius, vel etiam totius
universi; nam corruptio unius est generatio alterius, per quam species
conservatur. 

&

したがって、自然的諸事物における消滅や欠陥は、個別的な自然本性に反する
と言われるが、しかし、あるものの欠陥が、他のものの善、あるいはさらに全
宇宙の善になるかぎりでは、普遍的な自然の意図に属する。たとえば、あるも
のの消滅は、別のものの生成であり、それによって種が保存されるというよう
に。


\\


Cum igitur Deus sit universalis provisor totius entis, ad
ipsius providentiam pertinet ut permittat quosdam defectus esse in
aliquibus particularibus rebus, ne impediatur bonum universi
perfectum. 

&

ゆえに、神は、存在するもの全体の、普遍的な摂理を行う者なので、神の摂理
には、宇宙の完全な善が妨げられないように、何らかの個別的なものの中に、
なんらかの欠陥があることを認める、ということが属する。


\\

Si enim omnia mala impedirentur, multa bona deessent
universo, non enim esset vita leonis, si non esset occisio animalium;
nec esset patientia martyrum, si non esset persecutio tyrannorum. 

&

じっさい、もしすべての悪が妨げられたなら、多くの善が、宇宙からなくなっ
ていたであろう。たとえば、動物を殺すことがなかったならば、ライオンの命
はなかったであろうし、暴君の迫害がなかったならば、殉教者の忍耐もなかっ
たであろう。


\\

Unde
dicit Augustinus in {\itshape Enchirid}.~: 
{\itshape Deus omnipotens nullo modo sineret malum
aliquod esse in operibus suis, nisi usque adeo esset omnipotens et
bonus, ut bene faceret etiam de malo}.


&

このことから、アウグスティヌスは『エンキリディオン』で以下のように述べ
ている。「全能の神は、悪でさえ良く用いるほどに全能で善でなかったならば、
自分の作品の中に何らかの悪があることを許さなかったであろう」。

\\

-- Ex his autem duabus rationibus
quas nunc solvimus, videntur moti fuisse, qui divinae providentiae
subtraxerunt corruptibilia, in quibus inveniuntur casualia et mala.

&

しかし、神の摂理を、偶然的な事柄と悪が見出される、可滅的なものどもから
取り除いた人々は、私たちがいま解答した二つの議論に動かされたと思われる。


\\



{\scshape Ad tertium dicendum} quod homo non est
institutor naturae, sed utitur in operibus artis et virtutis, ad suum
usum, rebus naturalibus. Unde providentia humana non se extendit ad
necessaria, quae ex natura proveniunt. Ad quae tamen se extendit
providentia Dei, qui est auctor naturae. Et ex hac ratione videntur moti
fuisse, qui cursum rerum naturalium subtraxerunt divinae providentiae,
attribuentes ipsum necessitati materiae; ut Democritus, et alii
naturales antiqui.

&


第三異論に対しては、以下のように言われるべきである。人間は、自然の設立
者ではなく、自分の使用のために、技術や徳の働きにおいて、自然的諸物を使
用する。したがって、人間の摂理は、必然的なものまでは及ばない。それらは
自然に基づいて生じるからである。これに対して、自然の設立者である神の摂
理は、必然的なものにまで及ぶ。デモクリトスや、その他の古代の自然学者た
ちのような、自然的諸事物の振る舞いを、神の摂理から引き離し、それを質料
の必然性に帰した人々は、この根拠に動かされたように思われる。

\\


{\scshape Ad quartum dicendum} quod in hoc quod dicitur
Deum hominem sibi reliquisse, non excluditur homo a divina providentia,
sed ostenditur quod non praefigitur ei virtus operativa determinata ad
unum, sicut rebus naturalibus; quae aguntur tantum, quasi ab altero
directae in finem, non autem seipsa agunt, quasi se dirigentia in finem,
ut creaturae rationales per liberum arbitrium, quo consiliantur et
eligunt. Unde signanter dicit, {\itshape in manu consilii sui}. 

&

第四異論に対しては、以下のように言われるべきである。神が人間を自らに委
ねた、と言われることにおいて、人間が神の摂理から排除されているわけでは
なく、むしろ、自然的諸事物のように、一つへ限定された働きの力が、人間に
あつらえられているわけではないことが、示されている。それら自然的諸事物
は、他のものによって目的へ向けられたかのようにして作用するだけであり、
ちょうど理性的被造物が自由選択(これによって思量し、選択する)を通して
行うように、自らを目的へ向けるようにして自分自身では作用しない。だから、
「自らの思量の手に」と言うのには意味がある。

\\


Sed quia ipse actus
liberi arbitrii reducitur in Deum sicut in causam, necesse est ut ea
quae ex libero arbitrio fiunt, divinae providentiae subdantur,
providentia enim hominis continetur sub providentia Dei, sicut causa
particularis sub causa universali. 

&

しかし、自由選択の作用は、原因としての神へ還元されるから、自由選択から
生じる事柄が、神の摂理にもとにあることは必然である。というのも、人間の
摂理は、個別的な原因が普遍的な原因に含まれるようにして、神の摂理に含ま
れるからである。



\\



Hominum autem iustorum quodam
excellentiori modo Deus habet providentiam quam impiorum, inquantum non
permittit contra eos evenire aliquid, quod finaliter impediat salutem
eorum, nam {\itshape diligentibus Deum omnia cooperantur in bonum}, ut dicitur
{\itshape Rom}.~{\scshape viii}. 

&

しかし、神は、正しい人々にたいして、不敬虔な人々にたいするよりも、ある
種より優れたしかたで摂理をもつ。それは、彼らの救いを最終的に妨げるもの
が、彼らに対して生じることを許さない限りにおいてである。というのも、
『ローマの信徒への手紙』8章で「神を愛する人々にとって、すべては善に向
けて協力する」\footnote{「神を愛する者たち、つまり、御計画に従って召さ
れた者たちには、万事が益となるように共に働くということを、わたしたちは
知っています。」(8:28)}と言われるからである。

\\


Sed ex hoc ipso quod impios non retrahit a malo culpae,
dicitur eos dimittere. 
Non tamen ita, quod totaliter ab eius providentia
excludantur, alioquin in nihilum deciderent, nisi per eius providentiam
conservarentur. --Et ex hac ratione videtur motus fuisse Tullius, qui res
humanas, de quibus consiliamur, divinae providentiae subtraxit.


&

これに対して、[神が]不敬虔な人々を、罪の悪から引き離さすことをしない
ことから、彼らを[自らに]委ねると言われるが、しかし、彼らが神の摂理か
ら全面的に排除されるというかたちではない。さもなければ、つまり、神の摂
理によって保存されていなければ、彼らは無に帰したであろうから。さて、キ
ケロは上述の理由に動かされて、私たちが思量する人間的な事柄を、神の摂理
から取り除いたのだと思われる。


\\



{\scshape Ad quintum dicendum} quod, quia creatura
rationalis habet per liberum arbitrium dominium sui actus, ut dictum
est, speciali quodam modo subditur divinae providentiae; ut scilicet ei
imputetur aliquid ad culpam vel ad meritum, et reddatur ei aliquid ut
poena vel praemium. Et quantum ad hoc curam Dei apostolus a bobus
removet. Non tamen ita quod individua irrationalium creaturarum ad Dei
providentiam non pertineant, ut Rabbi Moyses existimavit.

&

第五異論に対しては、以下のように言われるべきである。すでに述べられたよ
うに、理性的被造物は、選択の自由を通して、自分の行為の所有権をもつので、
ある種特殊なしかたで神の摂理に服する。すなわち、何かが人間に、罪または
功徳として帰せられ、また、何かが人間に、罰または報償として与えられる。
そしてこのことにかんして、使徒は神の配慮を牛から除外する。しかし、マイ
モニデスが考えたように、非理性的被造物の諸々の個体が、神の摂理に属さな
い、というわけではない。



\end{longtable}
\newpage



\rhead{a.~3}
\begin{center}
 {\Large {\bf ARTICULUS TERTIUS}}\\
 {\large UTRUM DEUS IMMEDIATE OMNIBUS PROVIDEAT}\\
 {\footnotesize Infra, qu.103, art.6; III {\itshape SCG}, cap.76, 77,
 83, 94; {\itshape Compend.~Theol.}, cap.103, 131; Opusc.~XV, {\itshape
 De Angelis}, cap.14.}\\
 {\Large 第三項\\神は直接的に万物を摂理するか}
\end{center}

\begin{longtable}{p{21em}p{21em}}

{\Huge A}{\scshape d tertium sic proceditur}. Videtur quod
Deus non immediate omnibus provideat. Quidquid enim est dignitatis, Deo
est attribuendum. Sed ad dignitatem alicuius regis pertinet, quod habeat
ministros, quibus mediantibus subditis provideat. Ergo multo magis Deus
non immediate omnibus providet.

&

第三の問題へ、議論は以下のように進められる。神は、万物を直接的に摂理す
るわけではないと思われる。理由は以下の通り。威厳に属するものは何であれ、
神に帰せられるべきである。ところで、従者をもち、その従者を介して、家来
たちを摂理することは、王の威厳に属する。ゆえに、より一層、神は、直接的
に万物を摂理しない。
 
\\


2 {\scshape Praeterea}, ad providentiam pertinet res in
finem ordinare. Finis autem cuiuslibet rei est eius perfectio et
bonum. Ad quamlibet autem causam pertinet effectum suum perducere ad
bonum. Quaelibet igitur causa agens est causa effectus providentiae. Si
igitur Deus omnibus immediate providet, subtrahuntur omnes causae
secundae.

&

さらに、摂理には、事物を目的へ秩序付けることが属する。ところで、どんな
事物であれ、その事物の目的は、それの完成、すなわちそのものの善である。
また、どんな原因にも、自らの結果を善へ導くことが属する。ゆえに、どんな
作用原因も、摂理の結果の原因である。ゆえに、神が万物を直接的に摂理した
ならば、あらゆる第二原因は取り除かれるであろう。

\\


3 {\scshape Praeterea}, Augustinus dicit, in {\itshape Enchirid}.,
quod {\itshape melius est quaedam nescire quam scire}, ut vilia, et idem dicit
philosophus, in XII {\itshape Metaphys}. Sed omne quod est melius, Deo est
attribuendum. Ergo Deus non habet immediate providentiam quorundam
vilium et malorum.

&

さらに、アウグスティヌスは『エンキリディオン』で、たとえば悪徳のように、
「あるものを知らないことは、知っていることよりも善い」と言うし、哲学者
も『形而上学』第12巻で同じことを述べている。ところで、すべて、より善い
ものは、神に帰せられるべきである。ゆえに、神は、ある悪徳や、悪について
は、直接的な摂理を持たない。

\\


{\scshape Sed contra est quod} dicitur {\itshape Iob} {\scshape xxxiv}, {\itshape quem
constituit alium super terram? Aut quem posuit super orbem quem
fabricatus est?} Super quo dicit Gregorius, {\itshape mundum per seipsum regit,
quem per seipsum condidit}.

&

しかし反対に、『ヨブ記』34章で「大地の上に、他の誰を据えたか。あるいは、
彼が作った世界の上に、誰を置いたか」\footnote{「誰が神に全地をゆだね、
全世界を負わせたというのか」(34:13)}と言われているが、グレゴリウスはこ
れを注解して「自分自身によって造った世界を、自分自身によって支配する」
と述べている。
 
\\


{\scshape Respondeo dicendum} quod ad providentiam duo
pertinent, scilicet ratio ordinis rerum provisarum in finem; et executio
huius ordinis, quae gubernatio dicitur. Quantum igitur ad primum horum,
Deus immediate omnibus providet. Quia in suo intellectu habet rationem
omnium, etiam minimorum, et quascumque causas aliquibus effectibus
praefecit, dedit eis virtutem ad illos effectus producendos. Unde
oportet quod ordinem illorum effectuum in sua ratione
praehabuerit. 


&

解答する。以下のように言われるべきである。摂理には、二つのものが属する。
すなわち、目的へと摂理される諸事物の秩序の理念と、この秩序の実行である。
後者は「統治」と言われる。それゆえ、このうちの最初のものにかんして、神
は直接に万物を摂理する。なぜなら、神は自分の知性の中に、たとえそれが最
少のものであっても、それを含むすべてのものの理念をもっているし、また、
どんな結果を生み出すどんな原因であっても、神はそれらの原因に、それらの
結果を生み出す力を与えたからである。したがって、それら諸結果の秩序を、
神は自分の理念の中に、あらかじめ持っていたのでなければならない。


\\


Quantum autem ad secundum, sunt aliqua media divinae
providentiae. Quia inferiora gubernat per superiora; non propter
defectum suae virtutis, sed propter abundantiam suae bonitatis, ut
dignitatem causalitatis etiam creaturis communicet. 



&

これに対して、二番目のことにかんしては、神の摂理の何らかの媒介が存在す
る。なぜなら、神は下位のものを、上位のものを通して統治するからである。
ただしそれは、神の力に欠陥があるからではなく、原因性という威厳を被造物
にも伝えるという、神の善性の充溢のためにである。



\\

Et secundum hoc
excluditur opinio Platonis, quam narrat Gregorius Nyssenus, triplicem
providentiam ponentis. Quarum prima est summi Dei, qui primo et
principaliter providet rebus spiritualibus; et consequenter toti mundo,
quantum ad genera, species et causas universales. Secunda vero
providentia est, qua providetur singularibus generabilium et
corruptibilium, et hanc attribuit diis qui circumeunt caelos, idest
substantiis separatis, quae movent corpora caelestia
circulariter. Tertia vero providentia est rerum humanarum, quam
attribuebat Daemonibus, quos Platonici ponebant medios inter nos et
deos, ut narrat Augustinus IX {\itshape de Civ.~Dei}.

&

そして、このことに従うと、ニュッサのグレゴリウスが語るところの、プラト
ンの意見が排除される。彼は三通りの摂理を論じたという。一つには、最高神
の摂理であり、最高神は、第一に、また主要に、霊的諸事物を摂理するが、結
果的に、類、種、そして普遍的な原因にかんして、世界全体を摂理する。また、
第二の摂理は、生成消滅するものどものうち、個々のものが摂理されるもので
あり、これを、天を巡る神々、すなわち、天体を円状に動かす離在諸実体に帰
した。さらに、第三の摂理は、人間的諸事物にかんするものであり、これをダ
イモーンたちに帰していた。『神の国』でアウグスティヌスが語るとおり、プ
ラトン派の人々は、私たちと神々の間に、このような中間のものを置いていた
のである。


\\


{\scshape Ad primum ergo dicendum} quod habere ministros
executores suae providentiae, pertinet ad dignitatem regis, sed quod non
habeat rationem eorum quae per eos agenda sunt, est ex defectu
ipsius. Omnis enim scientia operativa tanto perfectior est, quanto magis
particularia considerat, in quibus est actus.

&

第一異論に対しては、それゆえ、以下のように言われるべきである。自分の摂
理を実行する従者たちをもつことは、王の威厳に属するが、彼らによってなさ
れる事柄の理念をもたないことは、王の欠陥に由来する。なぜなら、働きにか
んする知はすべて、より個別的なことを考察すればするだけ、より完全だから
である。現実態(働き)は、個別的な事柄においてあるからである。


\\


{\scshape Ad secundum dicendum} quod per hoc quod Deus
habet immediate providentiam de rebus omnibus, non excluduntur causae
secundae, quae sunt executrices huius ordinis, ut ex supra dictis patet.

&

第二異論に対しては、以下のように言われるべきである。神が、すべての事物
について、直接的に摂理を持つことによって、諸々の第二原因が取り除かれる
わけではない。上述のことから明らかなとおり、それらはこの秩序を実行する
ものだからである。


\\


Ad tertium dicendum quod nobis melius est non
cognoscere mala et vilia, inquantum per ea impedimur a consideratione
meliorum, quia non possumus simul multa intelligere, et inquantum
cogitatio malorum pervertit interdum voluntatem in malum. Sed hoc non
habet locum in Deo, qui simul omnia uno intuitu videt, et cuius voluntas
ad malum flecti non potest.

&


第三異論に対しては、以下のように言われるべきである。悪や悪徳を知らない
ことが、私たちにとってより善いのは、私たちは同時に多くのことを知性認識
できないので、それらによって、より善い事柄の考察が妨げられるという点、
そして、悪の認識が、時として、意志を悪へと曲げる点においてである。しか
し、このことは神の中では生じない。神は、万物を一つの直観によって見るし、
神の意志が悪へと曲げられることは不可能だからである。



\end{longtable}
\newpage

\rhead{a.~4}
\begin{center}
 {\Large {\bf ARTICULUS QUARTUS}}\\
 {\large UTRUM PROVIDENTIA REBUS PROVISIS NECESSITATEM IMPONAT}\\
 {\footnotesize I {\itshape Sent}., d.39, q.2, a.2; III {\itshape SCG},
 cap.72, 94; {\itshape De Malo}, q.14, a.7, ad 15; Opusc.~II, {\itshape
 Contra Graecos, Armenos} etc., cap.10; {\itshape Compend.~Theol.},
 cap.139, 140; Opusc.~XV, {\itshape De Angelis}, cap.15; I {\itshape
 Periherm.}, lect.14; VI {\itshape Metaphys.}, lect.3}\\
 {\Large 第四項\\摂理は摂理された事物に必然性を与えるか}
\end{center}

\begin{longtable}{p{21em}p{21em}}


{\Huge A}{\scshape d quartum sic proceditur}. Videtur quod
divina providentia necessitatem rebus provisis imponat. Omnis enim
effectus qui habet aliquam causam per se, quae iam est vel fuit, ad quam
de necessitate sequitur, provenit ex necessitate, ut philosophus probat
in VI {\itshape Metaphys}. Sed providentia Dei, cum sit aeterna, praeexistit; et ad
eam sequitur effectus de necessitate; non enim potest divina providentia
frustrari. Ergo providentia divina necessitatem rebus provisis imponit.

&

第四項の問題へ、議論は以下のように進められる。神の摂理は、摂理された事
物に必然性を与えると思われる。理由は以下の通り。哲学者が『形而上学』第
6巻で証明しているように、すべて、何らかの自体的な原因(今あるものであ
れ、過去にあったものであれ)をもち、その原因に必然的に伴う結果は、必然
的に生じる。ところで、神の摂理は、永遠なので、(結果よりも)先に存在す
る。そして、結果は必然的にそれに伴う。なぜなら、神の摂理が無駄になるこ
とはないからである。ゆえに、神の摂理は、摂理された諸事物に必然性を与え
る。


\\


2 {\scshape Praeterea}, unusquisque provisor stabilit
opus suum quantum potest, ne deficiat. Sed Deus est summe potens. Ergo
necessitatis firmitatem rebus a se provisis tribuit.

&

さらに、各々の摂理者は、誤ることがないように、自分の働きをできる限り確
かなものにする。ところが、神は、最高度に力がある。ゆえに、自らによって
摂理された諸事物に、必然性という確かさを与える。


\\


3 {\scshape Praeterea}, Boetius dicit, IV {\itshape de Consol}.,
quod fatum, {\itshape ab immobilis providentiae proficiscens exordiis, actus
fortunasque hominum indissolubili causarum connexione
constringit}. Videtur ergo quod providentia necessitatem rebus provisis
imponat.

&

さらに、ボエティウスは、『哲学の慰め』第4巻で、運命は「不動の摂理の始
原から出てきて、人間の行為と運とを、諸原因のほどけない連結によって結び
つける」と述べている。ゆえに、摂理は摂理された諸事物に必然性を与えると
思われる。

\\


{\scshape Sed contra est} quod dicit Dionysius, IV
cap.~{\itshape de Div. Nom}., quod {\itshape corrumpere naturam non est providentiae}. Hoc
autem habet quarundam rerum natura, quod sint contingentia. Non igitur
divina providentia necessitatem rebus imponit, contingentiam excludens.

&


しかし反対に、ディオニュシウスは『神名論』第4章で、「本性を破壊するこ
とは、摂理に属さない」と言う。しかし、偶然的であるということを、ある事
物の本性は持っている。ゆえに、神の摂理は、偶然性を排除するというかたち
で、事物に必然性を与えることはない。


\\


{\scshape Respondeo dicendum} quod providentia divina
quibusdam rebus necessitatem imponit, non autem omnibus, ut quidam
crediderunt. Ad providentiam enim pertinet ordinare res in finem. Post
bonitatem autem divinam, quae est finis a rebus separatus, principale
bonum in ipsis rebus existens, est perfectio universi, quae quidem non
esset, si non omnes gradus essendi invenirentur in rebus. 


&

解答する。以下のように言われるべきである。神の摂理は、ある事物に必然性
を与えるが、ある人々が信じていたように、すべてに与えるわけではない。理
由は以下の通り。摂理には、諸事物を目的へ秩序付けることが属する。ところ
で、神の善性は、諸事物から分離した目的であるだが、その神の善性のあと、
諸事物自身の中にある主要な善は、宇宙の完全性である。じっさい、もし、存
在のすべての段階が、諸事物の中に見出されなかったならば、この完全性はな
かったであろう。


\\

Unde ad
divinam providentiam pertinet omnes gradus entium producere. Et ideo
quibusdam effectibus praeparavit causas necessarias, ut necessario
evenirent; quibusdam vero causas contingentes, ut evenirent
contingenter, secundum conditionem proximarum causarum.

&

したがって、存在するものがもつあらゆる段階を生み出すということが、神の
摂理には属している。ゆえに、神は、ある諸結果には、必然的な原因を準備し、
ある諸結果には、偶然的な原因を準備した。前者は、必然的に生じ、後者は最
近接原因の状態に応じて偶然的に生じる。


\\


{\scshape Ad primum ergo dicendum} quod effectus divinae
providentiae non solum est aliquid evenire quocumque modo; sed aliquid
evenire vel contingenter vel necessario. Et ideo evenit infallibiliter
et necessario, quod divina providentia disponit evenire infallibiliter
et necessario, et evenit contingenter, quod divinae providentiae ratio
habet ut contingenter eveniat.

&

第一異論に対しては、それゆえ、以下のように言われるべきである。神の摂理
の結果は、どんなかたちであれ何かが生じることだけでなく、何かが、偶然的
に、あるいは必然的に生じる、ということが属している。ゆえに、神の摂理が、
間違いなく、必然的に生じるように配慮するものは、間違いなく、必然的に生
じるが、神の摂理の理念の中に、偶然的に生じることが含まれるものは、偶然
的に生じる。




\\


{\scshape Ad secundum dicendum} quod in hoc est
immobilis et certus divinae providentiae ordo, quod ea quae ab ipso
providentur, cuncta eveniunt eo modo quo ipse providet, sive necessario
sive contingenter.

&

第二異論に対しては、以下のように言われるべきである。神の摂理の秩序が不
動で確実であるのは、神によって摂理されるものがすべて、必然的であれ偶然
的であれ、神が摂理するとおりのしかたで生じる、というこの点にある。



\\


{\scshape Ad tertium dicendum} quod indissolubilitas
illa et immutabilitas quam Boetius tangit, pertinet ad certitudinem
providentiae, quae non deficit a suo effectu, neque a modo eveniendi
quem providit, non autem pertinet ad necessitatem effectuum. Et
considerandum est quod necessarium et contingens proprie consequuntur
ens, inquantum huiusmodi. Unde modus contingentiae et necessitatis cadit
sub provisione Dei, qui est universalis provisor totius entis, non autem
sub provisione aliquorum particularium provisorum.

&

第三異論に対しては、以下のように言われるべきである。ボエティウスが言及
している、かの非分解性と不可動性は、失敗することなく自らの結果を生み出
し、必ず摂理したあり方で生じさせる、摂理の確実性に属するのであり、諸結
果が有する必然性に属するのではない。また、必然と偶然は、存在するもので
あるかぎりでの存在するものに、固有の意味で伴うということが、考察される
べきである。このために、偶然性、必然性というあり方は、存在するもの全体
の、普遍的な摂理者である神の摂理のもとに入り、ある個別的な摂理者の摂理
のもとには入らない。

\end{longtable}
\newpage
\end{document}
