\documentclass[10pt]{jsarticle} % use larger type; default would be 10pt
%\usepackage[utf8]{inputenc} % set input encoding (not needed with XeLaTeX)
%\usepackage[round,comma,authoryear]{natbib}
%\usepackage{nruby}
\usepackage{okumacro}
\usepackage{longtable}
%\usepqckage{tablefootnote}
\usepackage[polutonikogreek,english,japanese]{babel}
%\usepackage{amsmath}
\usepackage{latexsym}
\usepackage{color}
\usepackage[at]{easylist}

%----- header -------
\usepackage{fancyhdr}
\lhead{{\it Summa Theologiae} I, q.~50}
%--------------------

\bibliographystyle{jplain}

\title{{\bf PRIMA PARS}\\{\HUGE Summae Theologiae}\\Sancti Thomae
Aquinatis\\{\sffamily QUEAESTIO QUINQUAGESIMA}\\DE SUBSTANTIA ANGELORUM ABSOLUTE}
\author{Japanese translation\\by Yoshinori {\sc Ueeda}}
\date{Last modified \today}


%%%% コピペ用
%\rhead{a.~}
%\begin{center}
% {\Large {\bf }}\\
% {\large }\\
% {\footnotesize }\\
% {\Large \\}
%\end{center}
%
%\begin{longtable}{p{21em}p{21em}}
%
%&
%
%
%\\
%\end{longtable}
%\newpage



\begin{document}
\maketitle
\pagestyle{fancy}

\begin{center}
{\Large 第五十問\\天使の実体について、無条件的に}
\end{center}

\begin{longtable}{p{21em}p{21em}}
Post haec considerandum est de distinctione corporalis et spiritualis
creaturae. Et primo, de creatura pure spirituali, quae in Scriptura
sacra Angelus\footnote{angelusは、ギリシア語\lgrfont{>'aggelos}(使者)
に由来する。} nominatur; secundo, de creatura pure corporali; tertio,
de creatura composita ex corporali et spirituali, quae est homo.


&

これらの後に、物体的被造物と霊的被造物の区別について考察されるべきであ
 る。第一に、聖書の中で聖なる使いと名付けられている、純粋に霊的な被造
 物について、第二に、純粋に物体的な被造物について、第三に、物体的なも
 のと霊的なものとから複合された被造物、すなわち人間について、考察され
 なければならない。

\\

Circa vero Angelos, considerandum est primo de his quae pertinent ad
eorum substantiam; secundo, de his quae pertinent ad eorum
intellectum; tertio, de his quae pertinent ad eorum voluntatem;
 quarto, de his quae pertinent ad eorum creationem.
 

&

 さて、天使について、第一に、それらの実体に関係することがらが、第二に、
 それらの知性にかかわることが、第三に、それらの意志にかんすることが、
 第四に、それらの創造に属することが、考察されるべきである。

\\

 De substantia autem eorum considerandum est et absolute, et per
comparationem ad corporalia. Circa substantiam vero eorum absolute,
quinque quaeruntur.

\begin{enumerate}
 \item utrum sit aliqua creatura omnino spiritualis, et penitus
       incorporea.
 \item supposito quod Angelus sit talis, quaeritur utrum Angelus sit
       compositus ex materia et forma.
 \item quaeritur de multitudine eorum.
 \item de differentia ipsorum ab invicem.
 \item de immortalitate, seu incorruptibilitate ipsorum.
\end{enumerate}


&

 また、それらの実体については、無条件的に、そして、物体的なものとの関
 係を通して、考察されるべきである。そして無条件的には、それらの実体を
 めぐって、五つのことが問われる。

\begin{enumerate}
 \item あらゆる点で霊的であり、まったく非物体な、なんらかの被造物が存
 在するか。
 \item 天使がそのようなものであるとして、天使は質料と形相から複合されて
       いるかが問われる。
 \item (その仮定のもとで)天使の多さについて問われる。
 \item 天使たち相互の差異について。
 \item それらの不死、あるいは不滅について。

\end{enumerate}

\end{longtable}

\newpage


\rhead{a.~1}
\begin{center}
 {\Large {\bf ARTICULUS PRIMUS}}\\
 {\large UTRUM ANGELUS SIT OMNINO INCORPOREUS}\\
 {\footnotesize II {\itshape SCG.}, c.~46, 49; Opusc.~XV, {\itshape de
 Angelis}, c.~18.}\\
 {\Large 第一項\\天使はあらゆる点で非物体的か}
\end{center}

\begin{longtable}{p{21em}p{21em}}


{\huge A}{\scshape d primum sic proceditur}. Videtur quod Angelus non
sit omnino incorporeus. Illud enim quod est incorporeum solum quoad nos,
et non quoad Deum, non est incorporeum simpliciter. Sed Damascenus
dicit, in libro II, quod {\itshape Angelus incorporeus et immaterialis
dicitur quantum ad nos, sed comparatus ad Deum, corporeus et materialis
invenitur}. Non ergo est incorporeus simpliciter.

&

 第一の問題へ、以下のように議論が進められる。天使は、あらゆる点で非物
 体的であるわけではないと思われる。なぜなら、私たちにとってのみ非物体
 的で、神にとって非物体的でないものは、端的に非物体的なものではない。
 ところが、ダマスケヌスは第二巻\footnote{『正統信仰論』2巻3章}で「天使
 は、私たちに対してだけ、非物体的とか非質料的と言われるのであり、神と
 比較すれば、物体的であり質料的であることがわかる」と述べている。ゆえ
 に、端的に非物体的ではない。

\\


{\scshape 2 Praeterea}, nihil movetur nisi corpus, ut probatur in VI
{\itshape Physic}. Sed Damascenus dicit ibidem quod Angelus est
{\itshape substantia intellectualis semper mobilis}. Angelus ergo est
substantia corporea.


&

 さらに、『自然学』6巻で証明されているように、物体以外にはなにも動かさ
 れない。ところが、ダマスケヌスが同じ箇所で、天使は「常に動いている知
 的実体」であると述べている。ゆえに、天使は物体的実体である。

\\


{\scshape 3 Praeterea}, Ambrosius dicit, in libro {\itshape de Spiritu
Sancto}: {\itshape Omnis creatura certis suae naturae circumscripta
est limitibus}. Circumscribi autem proprium est corporum. Ergo omnis
creatura est corporea. Angeli autem sunt Dei creaturae, ut patet in
Psalmo {\scshape cxlviii}, {\itshape Laudate Dominum, omnes Angeli
eius}; et postea subditur: {\itshape Quoniam ipse dixit, et facta
sunt, ipse mandavit, et creata sunt}. Ergo Angeli sunt corporei.


&

 さらに、アンブロシウスは『聖霊について』という書物で「全ての被造物は、
 自らの本性の一定の限界によって限られている」と述べている。ところで、
 限られるということは、物体に固有なことである。ゆえに、全ての被造物は
 物体的である。しかるに、『詩編』148「神のすべての天使たちよ、主を賛美
 せよ」、\footnote{「御使いらよ、こぞって、主を賛美せよ」(148:2)}そし
 てこの後に、「彼が語ったので、それらは作られ、彼が命じ、それらは創造された
 のだから」\footnote{「主は命じられ、すべてのものは創造された」
 (148:5)}と言われていることから、天使は、神の被造物である。ゆえに、天
 使は物体的なものである。\footnote{
推論の構造に注意。\\
1. すべての被造物は限られている。(アンブロシウス)\\
2. すべての限られているものは物体的である。(仮定)\\
2.1. 「限られているということは物体に固有である」は、こう解釈される必要がある。 \\
3. ゆえに、すべての被造物は物体的である。(1, 2, MP)\\
4. すべての天使は被造物である。(詩編)\\
5. ゆえに、すべての天使は物体的である。(3, 4, MP)
}

 \\

 {\scshape Sed contra est} quod dicitur in Psalmo {\scshape ciii},
{\itshape Qui facit Angelos suos spiritus\footnote{spiritusは第四変化
名詞。複数対格。}}.


&

しかし反対に、『詩編』103「彼は、自らの使いたちを霊にする」
 \footnote{「さまざまな風を伝令とし、燃える火を御もとに仕えさせられる」
 (104:4)}と言われている。
\\


{\scshape Respondeo dicendum} quod necesse est ponere aliquas
creaturas incorporeas. Id enim quod praecipue in rebus creatis Deus
intendit est bonum quod consistit in assimilatione ad Deum. Perfecta
autem assimilatio effectus ad causam attenditur, quando effectus
imitatur causam secundum illud per quod causa producit effectum; sicut
calidum facit calidum.

&

解答する。以下のように言われるべきである。なんらかの非物体的な被造物が
あると考えることが必然的である。以下に理由を示す。創造された事物におい
て、神がとくに意図しているのは善であり、それは神への類似において成立す
る。ところで、結果の原因に対する完全な類似が見出されるのは、結果が、原
因が結果をそれによって生み出すところのものに即して、原因を模倣するとき
である。たとえば、熱いものが熱いものを作るように\footnote{原因である熱
いものと、結果である熱いものとの間には、完全な類似がある。なぜなら、結
果である熱いものは、原因が結果をそれによって生み出すところのもの(つま
り熱さ)に即して、原因である熱いものを模倣しているからである。}。

\\

Deus autem creaturam producit per intellectum et voluntatem, ut supra
ostensum est. Unde ad perfectionem universi requiritur quod sint
aliquae creaturae intellectuales. Intelligere autem non potest esse
actus corporis, nec alicuius virtutis corporeae, quia omne corpus
determinatur ad hic et nunc. Unde necesse est ponere, ad hoc quod
universum sit perfectum, quod sit aliqua incorporea creatura.

&

 ところで、神は、以前に示されたとおり、知性と意志を通して被造物を生み
 出す。したがって、宇宙の完全性のために、何らかの知的な被造物が存在す
 ることが求められる。ところで、知性認識の働きは、物体や、何らかの物体
 的能力の現実態ではありえない。なぜなら、すべて物体は、ここや今に限定
 されているからである。したがって、宇宙が完全なものであるためには、な
 んらかの非物体的な被造物が存在することを措定することが必然的である。

\\

Antiqui autem, ignorantes vim intelligendi, et non distinguentes inter
sensum et intellectum, nihil esse existimaverunt in mundo, nisi quod
sensu et imaginatione apprehendi potest. Et quia sub imaginatione non
cadit nisi corpus, existimaverunt quod nullum ens esset nisi corpus;
ut philosophus dicit in IV {\itshape Physic}. Et ex his processit
Sadducaeorum error, dicentium {\itshape non esse spiritum}. Sed hoc
ipsum quod intellectus est altior sensu, rationabiliter ostendit esse
aliquas res incorporeas, a solo intellectu comprehensibiles.

&

 しかし、古代の人々は、知性認識の力を知らず、また、感覚と知性を区別し
 なかったので、世界の中に、感覚と想像力でとらえうるものしか存在しない
 と判断した。そして、想像力の中には、物体しか入ってこないので、どんな
 有も物体以外ではないと判断した。これは哲学者が『自然学』4巻で述べると
 おりである。また、このことから、霊は存在しないと言うサドカイ派の人々
 の誤り\footnote{「サドカイ派は復活も天使も霊もないと言い、ファリサイ
 派はこのいずれをも認めているからである」『使徒言行録』(23:8)}が出てく
 る。しかし、知性が感覚よりも高次のものであるということそれ自体が、知
 性だけによってとらえられうる、なんらかの非物体的な事物が存在すること
 を、合理的に示している。

\\


Ad primum ergo dicendum quod substantiae incorporeae medium sunt inter
Deum et creaturas corporeas. Medium autem comparatum ad unum extremum,
videtur alterum extremum; sicut tepidum comparatum calido, videtur
frigidum. Et hac ratione dicitur quod Angeli, Deo comparati, sunt
materiales et corporei: non quod in eis sit aliquid de natura
corporea.

&

第一異論に対しては、それゆえ、次のように言われるべきである。非物体的実
体は、神と物体的被造物の中間にある。ところで、中間は、一方の端との関係
で見れば、もう一方の端に見える。たとえば、熱いものとの関係で見れば、中
間の温度のものは、冷たく感じられる。この理由で、天使は、神と比べると、
質料的で物体的だと言われているのであって、天使の中に、物体的本性に属す
る何かがあるから、そう言われているのではない。\footnote{ ``non quod in
eis ...''の部分は、``dicitur quod''ではなく``hac ratione''と対照させる。
ダマスケヌスの発言の理由が問題になっているので。}
 
\\

Ad secundum dicendum quod motus ibi accipitur prout intelligere et
velle motus quidam dicuntur. Dicitur ergo Angelus substantia semper
mobilis, quia semper est actu intelligens, non quandoque actu et
quandoque potentia, sicut nos. Unde patet quod ratio procedit ex
aequivoco.

&

第二異論に対しては、次のように言われるべきである。ここで、「運動」は、
知性認識や意志の働きが運動と言われるような意味で、理解されている。それ
ゆえ、天使が常に動いている実体であるのは、私たちのように、ある時には現
実態にあり、またあるときには可能態にあるようなかたちではなく、常に、現
実態において知性認識するものだからである。したがって、この推論は、同名
異議語に基づいて進んでいることが明らかである。

\\

Ad tertium dicendum quod circumscribi terminis localibus est proprium
corporum, sed circumscribi terminis essentialibus est commune cuilibet
creaturae, tam corporali quam spirituali. Unde dicit Ambrosius, in
libro {\itshape de Spir.~Sanct.}, quod licet quaedam locis
corporalibus non contineantur, circumscriptione tamen substantiae non
carent.

&

第三異論に対しては、次のように言われるべきである。場所の限界による制限
は物体に固有だが、本質的限界による制限は、物体的なものにも霊的なものに
も、あらゆる被造物に共通する。このことから、アンブロシウスは『聖霊につ
いて』という書物の中で、あるものどもは、物体的場所に含まれないけれども、
実体の制限を欠くことはないと言っている。\footnote{先の分析で、推論の2
が誤り。すなわち、「すべての限られているものは物体的である」は偽であり、
霊的なものも、被造物であれば、本質的限界という点で限られている。}

\end{longtable}
\newpage

\rhead{a.~2}
\begin{center}
 {\Large {\bf ARTICULUS SECUNDUS}}\\
 {\large UTRUM ANGELUS SIT COMPOSITUS EX MATERIA ET FORMA}\\
 {\footnotesize I {\itshape Sent.}, d.~8, q.~5, a.~2; II, d.~3, q.~1,
 a.~1; II {\itshape SCG}, c.~50, 51; {\itshape De Spirit.~Creat.}, a.~1;
 {\itshape Quodl.}~III, q.~8; IX, q.~4, a.~1; {\itshape
 Compend.~Theol.}, c.~74; Opusc.~XV, {\itshape de Angelis}, c.~5 seqq.;
 c.~18; {\itshape De Ent.~et Ess.}, c.~5.}\\
 {\Large 第二項\\天使は質料と形相から複合されているか}
\end{center}

\begin{longtable}{p{21em}p{21em}}

{\huge A}{\scshape d secundum sic proceditur}. Videtur quod Angelus sit
compositus ex materia et forma. Omne enim quod continetur sub aliquo
genere, est compositum ex genere et differentia, quae, adveniens generi,
constituit speciem. Sed genus sumitur ex materia, differentia vero ex
forma, ut patet in VIII {\itshape Metaphys}. Ergo omne quod est in
genere, est compositum ex materia et forma. Sed Angelus est in genere
substantiae. Ergo est compositus ex materia et forma.


&

 第二項の問題へ、議論は以下のように進められる。天使は質料と形相から複
 合されていると思われる。理由は以下の通り。すべてなんらかの類の中に含
 まれるものは、類と種差から複合されている。この種差は、類に到来するこ
 とで、種を構成する。ところが、『形而上学』第8巻で明らかなように、類は
 質料から、種差は形相から取られる。ゆえに、すべての類の中にあるものは、
 質料と形相から複合されている。ところが、天使は、「実体」という類の中
 にある。ゆえに、それは質料と形相から複合されている。

\\

{\scshape 2 Praeterea}, in quocumque inveniuntur proprietates
materiae, ibi invenitur materia. Proprietates autem materiae sunt
recipere et substare; unde dicit Boetius, in libro {\itshape de
Trin.}, quod {\itshape forma simplex subiectum esse non potest}. Haec
autem inveniuntur in Angelo. Ergo Angelus est compositus ex materia et
forma.

&

 さらに、そのなかに質料の固有性が見出されるものはなんでも、そこに質料
 が見出される。ところで、受け取ることやもとにあることは、質料の固有性
 である。このことから、ボエティウスは『三位一体論』で「単純形相は基体
 でありえない」と述べる。しかし、これらのことが天使の中に見出される。
 ゆえに、天使は、質料と形相から複合されている。

\\

{\scshape 3 Praeterea}, forma est actus. Quod ergo est forma tantum,
est actus purus. Sed Angelus non est actus purus, hoc enim solius Dei
est. Ergo non est forma tantum, sed habet formam in materia.

&

 さらに、形相は現実態である。ゆえに、形相だけであるものは、純粋現実態
 である。ところが、純粋現実態であることは神のみに属するから、天使は純
 粋現実態でない。ゆえに、天使は形相だけであるのではなく、質料の中に形
 相をもつ。

\\

 {\scshape 4 Praeterea}, forma proprie limitatur et finitur per
materiam. Forma ergo quae non est in materia, est forma infinita. Sed
forma Angeli non est infinita, quia omnis creatura finita est. Ergo
forma Angeli est in materia.

&

 さらに、形相は、質料によって、固有の意味で制限され限定される。ゆえに、
 質料の中にあるのではない形相は、限定されていない形相である。ところが、
 天使の形相は、無限ではない。なぜなら、すべての被造物は有限だからであ
 る。ゆえに、天使の形相は質料の中にある。

\\

{\scshape Sed contra est} quod Dionysius dicit, {\scshape iv}
cap.~{\itshape de Div.~Nom}., quod primae creaturae {\itshape sicut
incorporales et immateriales intelliguntur}.

&

 しかし反対に、ディオニュシウスは『神名論』4章で、第一の被造物たちは
 「非物体的で非質料的なものと理解される」と述べている。

\\

 {\scshape Respondeo dicendum} quod quidam ponunt Angelos esse
 compositos ex materia et forma. Et hanc opinionem astruere nititur
 Avicebron in libro {\itshape Fontis Vitae}. Supponit enim quod
 quaecumque distinguuntur secundum intellectum, sint etiam in rebus
 distincta. In substantia autem incorporea intellectus apprehendit
 aliquid per quod distinguitur a substantia corporea, et aliquid per
 quod cum ea convenit. Unde ex hoc vult concludere quod illud per quod
 differt substantia incorporea a corporea, sit ei quasi forma, et
 illud quod subiicitur huic formae distinguenti quasi commune, sit
 materia eius. Et propter hoc ponit quod eadem est materia universalis
 spiritualium et corporalium, ut intelligatur quod forma incorporeae
 substantiae sic sit impressa in materia spiritualium, sicut forma
 quantitatis est impressa in materia corporalium.

&

 解答する。以下のように言われるべきである。ある人々は、天使が質料と形
 相から複合されていると論じる。そしてアヴィケブロンは、『生命の泉』と
 いう書で、この意見を打ち立てようとしている。彼は、なんであれ知性によっ
 て区別されるものは、事物においても区別されると想定する。そして、知性
 は、非物体的実体のなかに、それによって、それが物体的実体から区別され
 る何かと、それによって物体的実体と一致する何かをとらえる。そうして、
 このことに基づいて、非物体的実体が物体的実体と、それによって区別され
 るところのものを、それにとっていわば形相のようなものであり、そして、
 この区別する形相のもとに、いわば共通のものとしてあるものを、それの質
 料だと結論しようとする。そしてこのために、彼は、霊的なものと物体的な
 ものに属する普遍的な同一の質料が存在すると主張する。その結果、ちょう
 ど、量の形相が物体的なものの質料において刻印されるように、非物体的実
 体の形相が、霊的なものの質料において刻印されると考えられる。

\\

Sed primo aspectu apparet esse impossibile unam esse materiam
spiritualium et corporalium. Non enim est possibile quod forma
spiritualis et corporalis recipiatur in una parte materiae, quia sic una
et eadem res numero esset corporalis et spiritualis. Unde relinquitur
quod alia pars materiae sit quae recipit formam corporalem, et alia quae
recipit formam spiritualem. Materiam autem dividi in partes non
contingit nisi secundum quod intelligitur sub quantitate, qua remota,
remanet substantia indivisibilis, ut dicitur in I {\itshape Physic}. Sic
igitur relinquitur quod materia spiritualium sit subiecta quantitati,
quod est impossibile. Impossibile est ergo quod una sit materia
corporalium et spiritualium.


&

しかし、霊的なものと物体的なものとに、一つの質料が属することが不可能で
 あることは、一見して明らかである。まず、霊的形相と物体的形相が、質料
 の一つの部分に受け取られることは不可能である。なぜなら、もしそういう
 ことが起これば、数的に一つの同一の事物が、物体的でありかつ霊的なもの
 となっただろうから。したがって、質料のある部分が物体的形相を受け取り、
 それとは別のなんらかの部分が霊的形相を受け取るということになるが、質
 料が複数の部分に分割されるということは、質料が量という観点のもとに理
 解される場合しかない。『自然学』第1巻で言われるように、量が取り除かれ
 ると、実体は不可分のままに残るからである。ゆえに、霊的なものの質料が、
 量にとっての基体であることになるが、これは不可能である。ゆえに、物体
 的なものと霊的なものの質料が一つであることは不可能である。


\\



Sed adhuc ulterius impossibile est quod substantia intellectualis
habeat qualemcumque materiam. Operatio enim cuiuslibet rei est
secundum modum substantiae eius. Intelligere autem est operatio
penitus immaterialis. Quod ex eius obiecto apparet, a quo actus
quilibet recipit speciem et rationem, sic enim unumquodque
intelligitur, inquantum a materia abstrahitur; quia formae in materia
sunt individuales formae, quas intellectus non apprehendit secundum
quod huiusmodi. Unde relinquitur quod omnis substantia intellectualis
est omnino immaterialis.


&

 しかしまた、知的実体が、どのようなものであれ質料をもつことは、それに
 も増して不可能である。その理由はというと、どんな事物の働きであれ、そ
 の働きはその実体のあり方に従う。ところが、知性認識の働きは、まったく
 非質料的である。このことは、その対象から明らかである。というのも、ど
 んな作用も、対象から種と性格を受け取るからである。じっさい、それぞれ
 のものは、質料から抽象されるかぎりで、知性認識される。というのも、質
 料の中にある形相は、個的な形相だが、これらの形相を、知性は、そのよう
 な個的なものであるかぎりで捉えるのではないからである。したがって、す
 べての知的実体は、あらゆる点で非質料的であることが、結論として残され
 る。

\\



Non est autem necessarium quod ea quae distinguuntur secundum
intellectum, sint distincta in rebus, quia intellectus non apprehendit
res secundum modum rerum, sed secundum modum suum. Unde res
materiales, quae sunt infra intellectum nostrum, simpliciori modo sunt
in intellectu nostro, quam sint in seipsis. Substantiae autem
angelicae sunt supra intellectum nostrum. Unde intellectus noster non
potest attingere ad apprehendendum eas secundum quod sunt in seipsis;
sed per modum suum, secundum quod apprehendit res compositas. Et sic
etiam apprehendit Deum, ut supra dictum est.


&

ところで、知性において区別されるものが、事物においても区別されなけ
 ればならない、ということはない。なぜなら、知性は、事物のあり方にしたがっ
 て事物を捉えるのではなく、自らのあり方にしたがって捉えるからである。
したがって、私たちの知性の下位に位置する質料的事物は、私たちの知性の中に
 あるときに、それ自体においてあるよりも、より単純なあり方をしている。こ
 れに対して、天使の実体は、私たちの知性よりも上位に位置する。したがって、
 私たちの知性は、天使の実体を、それ自体のありかたに即して捉えるまでに到
 達できず、私たちの知性のあり方によって捉える。つまり、複合された事物を
 捉えるようなあり方で、それを捉える。前に述べられたとおり、私たちの知性
 は、神をも、そのようなかたちで捉える。


\\


{\scshape Ad primum ergo dicendum} quod differentia est
quae constituit speciem. Unumquodque autem constituitur in specie,
secundum quod determinatur ad aliquem specialem gradum in entibus, quia
species rerum sunt sicut numeri, qui differunt per additionem et
subtractionem unitatis, ut dicitur in VIII {\itshape Metaphys}. In rebus autem
materialibus aliud est quod determinat ad specialem gradum, scilicet
forma, et aliud quod determinatur, scilicet materia, unde ab alio
sumitur genus, et ab alio differentia. Sed in rebus immaterialibus non
est aliud determinans et determinatum, sed unaquaeque earum secundum
seipsam tenet determinatum gradum in entibus. Et ideo genus et
differentia in eis non accipitur secundum aliud et aliud, sed secundum
unum et idem. Quod tamen differt secundum considerationem nostram,
inquantum enim intellectus noster considerat illam rem ut indeterminate,
accipitur in eis ratio generis; inquantum vero considerat ut
determinate, accipitur ratio differentiae.


&

 第一異論に対しては、それゆえ、次のように言われるべきである。種差とは
 種を構成するものである。ところで、各々のものは、存在するものどもの中
 のある特定の階級に限定されることによって、種の中に構成される。という
 のも、『形而上学』8巻で言われるように、事物の種は、一を足したり引いた
 りすることによって異なる数[=自然数]のようなものだからである。とこ
 ろで、質料的事物では、特定の階級へと限定するもの、すなわち形相と、限
 定されるもの、すなわち質料とは別のものである。したがって、類と種差は
 別々のものから取られる。しかし、非質料的事物では、限定するものと限定
 されるものとは別々のものでなく、それらの各々は、自分自身によって、存
 在するものどもの中での限定された階級にある。ゆえに、それらにおいて、
 類と種差は、別々のものというかたちではなく、むしろ、一つで同一のもの
 というかたちで、理解される。しかし、私たちの考察においては異なるので
 あり、それは、私たちがその事物を非限定的に考察する場合には、それらに
 おける類の性格が理解され、限定的に考察する場合には、種差の性格が理解
 されるということによる。



\\


{\scshape Ad secundum dicendum} quod ratio illa ponitur
in libro {\itshape Fontis Vitae}. Et esset necessaria, si idem esset modus quo
recipit intellectus, et quo recipit materia. Sed hoc patet esse
falsum. Materia enim recipit formam, ut secundum ipsam constituatur in
esse alicuius speciei, vel aeris, vel ignis, vel cuiuscumque
alterius. Sic autem intellectus non recipit formam, alioquin
verificaretur opinio Empedoclis, qui posuit quod terram terra
cognoscimus, et ignem igne. Sed forma intelligibilis est in intellectu
secundum ipsam rationem formae, sic enim cognoscitur ab intellectu. Unde
talis receptio non est receptio materiae, sed est receptio substantiae
immaterialis.


&


第二異論に対しては、以下のように言われるべきである。この論は『生命の泉』
という書の中で主張されているが、もし、知性の受け取り方と質料の受け取り方
とが同じだったら、この論は必然的だっただろう。しかし、これ[=前件]が偽
であることは明らかである。なぜなら、質料が形相を受け取ると、その形相によっ
て、空気や、火や、あるいは他の何であれ、何らかの種の存在において構成され
るが、知性はそのようなかたちで形相を受け取らないからである。もしそういう
受け取り方をするのであれば、私たちは、土を土によって、火を火によって認識
すると主張したエムペドクレスの意見が正しいことになる。しかし、可知的な形
相は、形相の性格自体に即して知性の中にあり、知性によってそのように認識さ
れる。したがって、このような受け取り方は、質料の受け取り方ではなく、むし
ろ、非質料的実体の受け取り方である。



\\


{\scshape Ad tertium dicendum} quod, licet in Angelo non sit compositio
formae et materiae, est tamen in eo actus et potentia. Quod quidem
manifestum potest esse ex consideratione rerum materialium, in quibus
invenitur duplex compositio. Prima quidem formae et materiae, ex quibus
constituitur natura aliqua. Natura autem sic composita non est suum
esse, sed esse est actus eius. Unde ipsa natura comparatur ad suum esse
sicut potentia ad actum. Subtracta ergo materia, et posito quod ipsa
forma subsistat non in materia, adhuc remanet comparatio formae ad ipsum
esse ut potentiae ad actum. Et talis compositio intelligenda est in
Angelis. Et hoc est quod a quibusdam dicitur, quod Angelus est
compositus ex {\itshape quo est} et {\itshape quod est}, vel ex
{\itshape esse} et {\itshape quod est}, ut Boetius dicit, nam {\itshape
quod est} est ipsa forma subsistens; ipsum autem esse est {\itshape quo}
substantia est, sicut cursus est quo currens currit. Sed in Deo non est
aliud esse et quod est, ut supra ostensum est. Unde solus Deus est actus
purus.


&

第三異論に対しては、次のように言われるべきである。
天使の中には、たしかに形相と質料の複合がない。しかし、その中に現実態と可
 能態がある。このことは、質料的事物の考察から明らかとなりうる。すなわち、
 質料的事物の中には、二通りの複合が見出される。第一の複合は、形相と質料
 からの複合であり、これらから、何らかの本性が構成される。しかし、このように複合
 された本性は、自らの存在ではなく、存在は、それ[=本性]の現実態である。
したがって、この本性自体は、自らの存在に対して、可能態が現実態に関係する
 ように関係する。ゆえに、質料が取り除かれ、その形相が質料の中に存在して
 いないとされる場合でも、形相の、自らの存在への関係が、可能態の現実態へ
 の関係として留まる。このような複合が、天使の中に理解されるべきである。
そしてこれが、ある人々によって、天使は「それによって存在するところのもの」と
 「存在するもの」とから、あるいは、ボエティウスが言うように、「存在」と
 「存在するもの」とから複合されていると言われる事態である。「存在するも
 の」とは自存する形相であり、他方、存在とは、「それによって」実体が存在
 するところのものだからである。ちょうど、「走り」は、「それによって」走
 る人が走るところのものであるように。
しかし、前に示されたとおり、神の中では、存在と存在するものは別ではない。
 したがって、神だけが、純粋現実態である。



\\


{\scshape Ad quartum dicendum} quod omnis creatura est
finita simpliciter, inquantum esse eius non est absolutum subsistens,
sed limitatur ad naturam aliquam cui advenit. Sed nihil prohibet aliquam
creaturam esse secundum quid infinitam. Creaturae autem materiales
habent infinitatem ex parte materiae, sed finitatem ex parte formae,
quae limitatur per materiam in qua recipitur. Substantiae autem
immateriales creatae sunt finitae secundum suum esse, sed infinitae
secundum quod eorum formae non sunt receptae in alio. Sicut si diceremus
albedinem separatam existentem esse infinitam quantum ad rationem
albedinis, quia non contrahitur ad aliquod subiectum; esse tamen eius
esset finitum, quia determinatur ad aliquam naturam specialem. Et
propter hoc dicitur in libro {\itshape de Causis}, quod intelligentia est {\itshape finita
superius}, inquantum scilicet recipit esse a suo superiori; sed est
{\itshape infinita inferius}, inquantum non recipitur in aliqua materia.

&

第四異論に対しては、次のように言われるべきである。
被造物はすべて、その存在が無条件に自存するものでなく、存在がそれに到来す
 るなんらかの本性に制限されている点で、端的に有限である。
しかし、ある被造物が、なんらかの点で無限であることは問題ない。
質料的被造物は、質料の側から無限性を持つが、形相の側からは有限性を持つ。
 その形相は、受け取る質料によって制限されるからである。
被造の非質料的実体は、その存在において有限だが、それらの形相が他のものに
 受け取られていないという点では、無限である。ちょうど、かりにもし、私たちが、
 分離して存在する白性は、なんらかの基体に制限されていないので、白の性格
 にかんして無限だと言ったとしても、その存在は、なんらかの種的な本性に限
 定されているので、有限だったであろう。このために、『原因論』という書物
 で、知性体は、自分の上位にあるものから存在を受け取るので「上に対して有
 限」であり、なんらの質料にも受け取られていない点で、「下に対して無限」
 だと言われている。

\end{longtable}
\newpage



\rhead{a.~3}
\begin{center}
 {\Large {\bf ARTICULUS TERTIUS}}\\
 {\large UTRUM ANGELI SINT IN ALIQUO MAGNO NUMERO}\\
 {\footnotesize Infra, q.~112, a.~4, ad 2; II {\itshape Sent.}, d.~3,
 q.~1, a.~3; II {\itshape SCG.}, c.~92; {\itshape De Pot.}, q.~6, a.~6;
 Opusc.~XV, {\itshape de Angelis}, c.~2.}\\
 {\Large 第三項\\天使たちは多数存在するか}
\end{center}

\begin{longtable}{p{21em}p{21em}}

{\huge A}{\scshape d tertium sic proceditur}. Videtur quod
Angeli non sint in aliquo magno numero. Numerus enim species quantitatis
est, et sequitur divisionem continui. Hoc autem non potest esse in
Angelis cum sint incorporei, ut supra ostensum est. Ergo Angeli non
possunt esse in aliquo magno numero.


&

第三項の問題へ、議論は以下のように進められる。
天使は多数存在しないと思われる。なぜなら、数は、量の種であり、連続するも
 のの分割の結果として生じる。しかし、前に示されたとおり、天使は非物体的
 なものなので、このことは天使においては生じえない。ゆえに、天使は多数存
 在しえない。

\\


{\scshape 2 Praeterea}, quanto aliquid est magis
propinquum uni, tanto minus est multiplicatum, ut in numeris
apparet. Natura autem angelica inter alias naturas creatas est Deo
propinquior. Cum ergo Deus sit maxime unus, videtur quod in natura
angelica inveniatur minimum de multitudine.


&
さらに、数において明らかなように、あるものは、一に近いほど多から離れる。
 ところが、天使の本性は、他の被造の本性の中では、より神に近い。ゆえに、
 神は最大限に一なのだから、天使の本性の中には、最小の多が存在すると思わ
 れる。
\\


{\scshape 3 Praeterea}, proprius effectus separatarum
substantiarum videtur esse motus corporum caelestium. Sed motus corporum
caelestium sunt secundum aliquem determinatum numerum paucum, qui a
nobis comprehendi potest. Ergo Angeli non sunt in maiori multitudine,
quam motus corporum caelestium.


&


さらに、離在実体の固有の結果は、天体の運動だと思われる。ところが、天体の
 運動は、私たちに把握されうるような、ある限定された小さい数にしたがって
 ある。ゆえに、天使は、天体の運動より多く存在するわけではない。

\\


{\scshape 4 Praeterea}, Dionysius dicit, {\scshape iv} cap.~{\itshape de
Div.~Nom}., quod {\itshape propter radios divinae bonitatis subsistunt
intelligibiles et intellectuales omnes substantiae}. Sed radius non
multiplicatur nisi secundum diversitatem recipientium. Non autem potest
dici quod materia sit receptiva intelligibilis radii, cum substantiae
intellectuales sint immateriales, ut supra ostensum est. Ergo videtur
quod multiplicatio substantiarum intellectualium non possit esse nisi
secundum exigentiam primorum corporum, scilicet caelestium, ut ad ea
quodammodo processus praedictorum radiorum terminetur. Et sic idem quod
prius.


&

さらに、ディオニュシウスは、『神名論』4章で、「神の善性の光線のために、可
 知的で知性的なすべての実体が自存する」と述べている。しかし、光線は、受け
 取るものの多様性によってしか多数化されない。ところが、前に示されたとおり、
 知性的実体は非質料的なので、質料が、知性的な光線を受け取りうると言うこと
 はできない。ゆえに、知性的実体の多数化は、第一物体、すなわち天体がそれを
 必要とするためでしかありえないと思われる。つまりその結果、すでに述べられ
 た光線の発出が、あるかたちで、それらの天体へと終局する。このようにして
 前と同じ結論となる。


\\


{\scshape Sed contra est} quod dicitur Dan.~{\scshape vii}, {\itshape Millia
millium ministrabant ei, et decies millies centena millia assistebant
ei}.

&
しかし反対に、『ダニエル書』7章で「千の千倍が彼に仕え、$10 \times 1,000 \times
 100$倍の千が彼を助けていた\footnote{「幾千人が御前に仕え、幾万人が御前
 に立った」(7:10)}。

\\


{\scshape Respondeo dicendum} quod circa numerum
substantiarum separatarum, diversi diversis viis processerunt. Plato
enim posuit substantias separatas esse species rerum sensibilium, utpote
si poneremus ipsam naturam humanam esse separatam. Et secundum hoc
oportebat dicere quod substantiae separatae sint secundum numerum
specierum sensibilium. Sed hanc positionem improbat Aristoteles, ex eo
quod materia est de ratione speciei horum sensibilium. Unde substantiae
separatae non possunt esse species exemplares horum sensibilium, sed
habent quasdam naturas altiores naturis rerum sensibilium. 



&

解答する。以下のように言われるべきである。離在実体の数については、さまざ
 まな人がさまざまなかたちで論じた。たとえばプラトンは、離在実体が可感的
 事物の種であると主張した。たとえば、私たちが、人間本性それ自体が離在す
 ると主張した場合のように。この意見に従うならば、離在実体は、可感的事物
 の種の数に応じてあると言わなければならなかったであろう。しかし、アリス
 トテレスは、これら可感的事物の種の性格には質料が含まれることから、この主
 張を批判する。このことから、離在実体は、これら可感的事物の範型的な種では
 ありえず、むしろ、可感的事物の本性よりもある種上位の本性を持つ。


\\

Posuit tamen
Aristoteles quod illae naturae perfectiores habent ordinem ad sensibilia
ista, secundum rationem moventis et finis. Et ideo secundum numerum
primorum motuum, conatus est adinvenire numerum substantiarum
separatarum. Sed quia hoc videtur repugnare documentis sacrae
Scripturae, Rabbi Moyses, Iudaeus, volens utrumque concordare, posuit
quod Angeli, secundum quod dicuntur substantiae immateriales,
multiplicantur secundum numerum motuum vel corporum caelestium, secundum
Aristotelem. Sed posuit quod Angeli in Scriptura dicuntur etiam homines
divina annuntiantes; et iterum virtutes rerum naturalium, quae Dei
omnipotentiam manifestant. --- Sed hoc est alienum a consuetudine
Scripturae, quod virtutes rerum irrationabilium Angeli nominentur. 

&

しかし、アリストテレスは、これらのより完全な本性が、動者や目的という性格
 にしたがって、これら可感的事物への秩序を持つと主張した。それゆえ、第一
 の運動の数に応じて、離在実体の数を見出すように強いられた。しかしこれは
 聖書に書かれていることに反するように思われるので、ユダヤ人のマイモニデ
 スは、両者を和解させようとして、天使は、非質料的実体と言われる限りでは、
 アリストテレスにしたがって、運動の数、あるいは、天体の数にしたがって多
 数化されると主張した。しかし、彼はまた、天使は聖書の中で、神的なことが
 らや、さらに、神の全能を明示する自然的諸事物の力も、人間に伝えると言わ
 れていると主張した。---しかし、非理性的な事物の力を天使と名付けるのは、
 聖書の習いに反することではあるが。



\\




Unde
dicendum est quod etiam Angeli secundum quod sunt immateriales
substantiae, in quadam multitudine maxima sunt, omnem materialem
multitudinem excedentes. Et hoc est quod dicit Dionysius, {\scshape xiv}
cap.~{\itshape Caelest.~Hierarch}., {\itshape multi sunt beati exercitus supernarum mentium,
infirmam et constrictam excedentes nostrorum materialium numerorum
commensurationem}. Et huius ratio est quia, cum perfectio universi sit
illud quod praecipue Deus intendit in creatione rerum, quanto aliqua
sunt magis perfecta tanto in maiori excessu sunt creata a Deo. Sicut
autem in corporibus attenditur excessus secundum magnitudinem, ita in
rebus incorporeis potest attendi excessus secundum multitudinem. Videmus
autem quod corpora incorruptibilia, quae sunt perfectiora inter corpora,
excedunt quasi incomparabiliter secundum magnitudinem corpora
corruptibilia, nam tota sphaera activorum et passivorum est aliquid
modicum respectu corporum caelestium. Unde rationabile est quod
substantiae immateriales excedant secundum multitudinem substantias
materiales, quasi incomparabiliter.


&

したがって、以下のように言われるべきである。
天使もまた、非質料的実体であるかぎり、あらゆる質料的な多数性を越えて、あ
 る種の最大の多数性においてある。ディオニュシウスが『天上階級論』14章で
 「天の精神からなる至福な軍隊は、私たちの質料的数との弱く制限された共約
 を越えて多数である」と述べるのはこのことである。
その理由は、宇宙の完全性は、神が諸事物の創造に際してとくに意図しているこ
 とだから、何かがより完全であるほど、それだけいっそう、より大きな限度で、
 神に創造される。しかし、ちょうど物体的事物において、大きさにおける過剰
 が見出されるように、非物体的事物においては、多さにおける過剰が見出され
 る。さらに私たちは、物体の中でより完全なものである不可滅的な物体が、大
 きさにおいて、比較を絶するほど可滅的物体を凌駕するのを見る。というのも、
 能動受動的なものどもの球体は、天体に比べると、ごくささやかなものだから。
 したがって、非質料的実体は、多数性において、比較を絶するほど質料的実体
 を凌駕するのが合理的である。


\\


{\scshape Ad primum ergo dicendum} quod in Angelis non
est numerus qui est quantitas discreta, causatus ex divisione continui,
sed causatus ex distinctione formarum, prout multitudo est de
transcendentibus, ut supra dictum est.


&

第一異論に対しては、それゆえ、次のように言われるべきである。
天使の中には、連続するものの分割によって引き起こされる、分割された量であ
 る数はないが、しかし、形相の区別によって引き起こされる数がある。これは、
 前に言われたとおり、\footnote{``Est autem duplex divisio. Una
 materialis, quae fit secundum divisionem continui, et hanc consequitur
 numerus qui est species quantitatis. Unde talis numerus non est nisi in
 rebus materialibus habentibus quantitatem. Alia est divisio formalis,
 quae fit per oppositas vel diversas formas, et hanc divisionem sequitur
 multitudo quae non est in aliquo genere, sed est de transcendentibus,
 secundum quod ens dividitur per unum et multa. Et talem multitudinem
 solam contingit esse in rebus immaterialibus.'' ({\itshape ST} I,
 q.~30, a.~3, c.)}多さが超越的なものに属することによる。

\\


{\scshape Ad secundum dicendum} quod ex hoc quod natura
angelica est Deo propinqua, oportet quod habeat minimum de multitudine
in sui compositione, non autem ita quod in paucis salvetur.


&

第二異論に対しては、次のように言われるべきである。
天使の本性が神に近いということから、自分自身の中の複合において、最小の多
 数性をもたなければならないが、しかし、少数のものにおいて救われるという
 かたちではない。

\\


Ad tertium dicendum quod ratio illa est
Aristotelis in XII {\itshape Metaphys}. Et ex necessitate concluderet, si
substantiae separatae essent propter substantias corporales, sic enim
frustra essent immateriales substantiae, nisi ex eis aliquis motus in
rebus corporalibus appareret. Non est autem hoc verum, quod substantiae
immateriales sint propter corporales, quia finis nobilior est his quae
sunt ad finem. Unde etiam Aristoteles dicit ibidem quod haec ratio non
est necessaria, sed probabilis. Coactus autem fuit hac ratione uti, quia
ad cognoscendum intelligibilia non possumus pervenire nisi per
sensibilia.


&

第三異論に対しては、次のように言われるべきである。
この論は、『形而上学』12巻のアリストテレスのものである。
そして、もしも、離在実体が物体的実体のために存在していて、もしそれらから、
 なんらかの運動が物体的事物の中に現れなかったら、非質料的実体は無駄に存
 在することになる、ということだったとすると、必然的にそう結論したであろう。
しかし、質料的実体が物体のために存在するということは真でない。なぜなら、
 目的は、目的に向かってあるものどもよりも高貴だからである。したがって、
 アリストテレスは同じ箇所で、この論は必然的でなく蓋然的だと述べている。
 しかし彼がこの論を用いることを強いられたのは、可感的なものを通さなけれ
 ば、可知的なものの認識へ到達することができないからである。


\\


Ad quartum dicendum quod ratio illa procedit
secundum opinionem eorum qui causam distinctionis rerum ponebant esse
materiam. Hoc autem improbatum est. Unde multiplicatio Angelorum neque
secundum materiam, neque secundum corpora est accipienda, sed secundum
divinam sapientiam, diversos ordines immaterialium substantiarum
excogitantem.


&

第四異論に対しては、次のように言われるべきである。
その議論は、事物の区別の原因が質料だと主張した人々の意見に従って進んでい
 る。しかし、これは批判される。したがって、天使たちの多数化は、質料にし
 たがっても物体にしたがっても理解されるべきでなく、むしろ、非質料的実体
 のさまざまな秩序を考え出す神の知恵にしたがって、理解されるべきである。


\end{longtable}


\newpage

\rhead{a.~4}
\begin{center}
 {\Large {\bf ARTICULUS QUARTUS}}\\
 {\large UTRUM ANGELI DIFFERANT SPECIE}\\
 {\footnotesize II {\itshape Sent}., d.~3, q.~1, a.~4; d.~32, q.~2,
 a.~3; IV, d.~12, q.~1, a.~1, qu$^a$ 3, ad 3; II {\itshape SCG}., c.~93;
 {\itshape De Spirit.~Creat}., a.~8; Qu.~{\itshape de Anima}, a.~3;
 {\itshape De Ente et Ess.}, c.~5.}\\
 {\Large 第四項\\天使たちは種において異なるか}
\end{center}

\begin{longtable}{p{21em}p{21em}}
{\huge A}{\scshape d quartum sic proceditur}. Videtur quod
Angeli non differant specie. Cum enim differentia sit nobilior genere,
quaecumque conveniunt secundum id quod est nobilissimum in eis,
conveniunt in ultima differentia constitutiva; et ita sunt eadem
secundum speciem. Sed omnes Angeli conveniunt in eo quod est
nobilissimum in eis, scilicet in intellectualitate. Ergo omnes Angeli
sunt unius speciei.


&

第四の問いへ、以下のように議論が進められる。天使たちは、種において異なる
 のではないと思われる。種差は類よりも高貴なので、その中でもっとも高貴な
 ものにしたがって一致するものは何であれ、最終的な構成的種差において一致
 し、そのようにして、種において同一である。ところが、すべての天使は、そ
 れらの中でもっとも高貴なもの、すなわち知性性という点で一致する。ゆえに、
 すべての天使は一つの種に属する。


\\


{\scshape 2 Praeterea}, {\itshape magis} et {\itshape minus} non diversificant
speciem. Sed Angeli non videntur differre ad invicem nisi secundum magis
et minus; prout scilicet unus alio est simplicior, et perspicacioris
intellectus. Ergo Angeli non differunt specie.


&
さらに、「より多い、より少ない」ということで、種が異なることはない。とこ
 ろが、天使は、「より多い、より少ない」ということでのみ、相互に異なると
 思われる。つまり、一つの天使が、他の天使よりも、より単純な、そしてより
 透徹した知性であるという点でである。ゆえに、天使は種において異なるので
 はない。

\\

{\scshape 3 Praeterea}, anima et Angelus ex opposito
dividuntur. Sed omnes animae sunt unius speciei. Ergo et Angeli.

&

さらに、魂と天使は、対立するものとして分かたれる。ところが、すべての魂
 は一つの種に属する。ゆえに、天使もまた[一つの種に属する]。

\\


4 Praeterea, quanto aliquid est perfectius in
natura, tanto magis debet multiplicari. Hoc autem non esset, si in una
specie esset unum tantum individuum. Ergo multi Angeli sunt unius
speciei.


&


さらに、あるものは本性においてより完全であるほど、より多数化されるべきで
 ある。しかし、もし、一つの種に一つの個体しかなかったならば、こういうこ
 とはなかったであろう。ゆえに、一つの種の中に多数の天使がいる。

 \\

 {\scshape Sed contra est} quod in his quae sunt unius speciei, non
est invenire prius et posterius, ut dicitur in III {\itshape
Metaphys}. Sed in Angelis, etiam unius ordinis, sunt primi et medii et
ultimi, ut dicit Dionysius, {\scshape x} cap.~{\itshape
Ang.~Hier}. Ergo Angeli non sunt eiusdem speciei.


&

 しかし反対に、『形而上学』3巻で言われるように、一つの種に属するものの中
 には、「より先、より後」ということは見出されない。ところが、『天上階級
 論』10章のディオニュシウスが言うように、天使の中には、一つの階級に属す
 るものですら、先頭、中、最後、の区別がある。ゆえに、天使たちは、同一の種に
 属しているのではない。


\\



 {\scshape Respondeo dicendum} quod quidam dixerunt omnes substantias
 spirituales esse unius speciei, etiam animas. Alii vero quod omnes
 Angeli sunt unius speciei, sed non animae. Quidam vero quod omnes
 Angeli unius hierarchiae, aut etiam unius ordinis.



&

 解答する。以下のように言われるべきである。ある人々は、すべての霊的実
 体が、魂も含め、一つの種に属すると言った。またある人々は、すべての天
 使は一つの種に属するが、魂はその種に属さないと言った。さらにまたある
 人々は、すべての天使は、一つの階級、あるいは、一つの秩序に属すると言っ
 た。


\\

Sed hoc est
impossibile. 
Ea enim quae conveniunt specie et differunt numero,
conveniunt in forma, et distinguuntur materialiter. Si ergo Angeli non
sunt compositi ex materia et forma, ut dictum est supra, sequitur quod
impossibile sit esse duos Angelos unius speciei. 


&

しかしこれは不可能である。
種において一致し、数的に異なるものどもは、形相において一致し、質料的に区
 別される。ゆえに、前に述べられたとおり、もし天使が質料と形相から複合さ
 れていないならば、二つの天使が一つの種に属することは不可能である。


\\


Sicut etiam impossibile
esset dicere quod essent plures albedines separatae, aut plures
humanitates; cum albedines non sint plures nisi secundum quod sunt in
pluribus substantiis. 






&

これはちょうど、複数の離在する白性や、複数の人間性が存在すると言ったとすれ
 ば、それが不可能だったであろうようなものである。白性は、複数の実体の中
 にあることに従ってのみ、複数だからである。


\\

Si tamen Angeli haberent materiam, nec sic possent
esse plures Angeli unius speciei. 
Sic enim oporteret quod principium
distinctionis unius ab alio esset materia, non quidem secundum
divisionem quantitatis, cum sint incorporei, sed secundum diversitatem
potentiarum. Quae quidem diversitas materiae causat diversitatem non
solum speciei, sed generis.


&

しかし、かりに天使が質料をも
 つとしても、その場合でも、複数の天使が一つの種に属することは不可能だっ
 たであろう。
この場合には、一つの天使の他の天使からの区別の根源は質料であっただろうが、
 天使は非物体的なものだから、量の区別によるのではなく、能力の区別による
 のでなければならなかっただろう。しかしこの質料の区別は、種だけでなく、類の区
 別をももたらす。


\\


{\scshape Ad primum ergo dicendum} quod differentia est
nobilior genere, sicut determinatum indeterminato et proprium communi;
non autem sicut alia et alia natura. Alioquin oporteret quod omnia
animalia irrationalia essent unius speciei; vel quod esset in eis aliqua
alia perfectior forma quam anima sensibilis. Differunt ergo specie
animalia irrationalia secundum diversos gradus determinatos naturae
sensitivae. Et similiter omnes Angeli differunt specie secundum diversos
gradus naturae intellectivae.


&

第一異論に対しては、それゆえ、次のように言われるべきである。
種差が類より高貴なのは、限定されたものが限定されていないものよりも、固有
 のものが共通のものよりも高貴だという意味であり、ある本性が別の本性より
 高貴だという意味ではない。もしそうでなければ、すべての非理性的動物が一
 つの種に属したであろう。あるいは、それらの中に、感覚的魂以上に完全な他
 の形相が存在したことであろう。ゆえに、非理性的動物は、感覚的本性のさま
 ざまな限定された段階に応じて、種において異なる。同様に、すべての天使は、
 知性的本性 のさまざまな段階に応じて、種において異なる。



\\


{\scshape Ad secundum dicendum} quod magis et minus,
secundum quod causantur ex intensione et remissione unius formae, non
diversificant speciem. Sed secundum quod causantur ex formis diversorum
graduum, sic diversificant speciem, sicut si dicamus quod ignis est
perfectior aere. Et hoc modo Angeli diversificantur secundum magis et
minus.


&


第二異論に対しては、次のように言われるべきである。
「より多く、より少なく」は、一つの形相の強弱から生じる場合には、種の異な
 りを生み出さない。そうではなく、異なる段階に属する形相に基づいて、それ
 が生じる場合、異なる種を生み出す。たとえば、火は空気よりも完全である私
 たちが言う場合のように。このしかたで、天使たちは、「より多く、より少な
 く」という点で多様化されている。

\\


{\scshape Ad tertium dicendum} quod bonum speciei
praeponderat bono individui. Unde multo melius est quod multiplicentur
species in Angelis, quam quod multiplicentur individua in una specie.

&

第三異論に対しては、次のように言われるべきである。
種の善は、個の善よりも重要である。したがって、一つの種の中で個が多数化さ
 れることよりも、天使の中で種が多数化されることの方が、より善い。


\\


{\scshape Ad quartum dicendum} quod multiplicatio
secundum numerum, cum in infinitum protendi possit, non intenditur ab
agente, sed sola multiplicatio secundum speciem, ut supra dictum
est. Unde perfectio naturae angelicae requirit multiplicationem
specierum, non autem multiplicationem individuorum in una specie.


&

第四異論に対しては、次のように言われるべきである。
前に述べられたとおり、数的な多数化は、無限になされうるので、作用者によっ
 て意図されることはなく、意図されるのは、種的な多数化のみである。従って、
 天使的本性の完成は、種の多数化を必要とするのであって、一つの種の中での
 個の多数化を必要とするのではない。




\end{longtable}

\newpage

\rhead{a.~5}
\begin{center}
 {\Large {\bf ARTICULUS QUINTUS}}\\
 {\large UTRUM ANGELI SINT INCORRUPTIBILES}\\
 {\footnotesize Supra, q.~9, a.~2; II {\itshape Sent.}, d.~7, q.~1,
 a.~1; II {\itshape SCG}., c.~55; {\itshape De Pot}., q.~5, a.~3;
 {\itshape Compend.~Theol}., c.~74.}\\
 {\Large 第五項\\天使は不滅か}
\end{center}

\begin{longtable}{p{21em}p{21em}}

{\huge A}{\scshape d quintum sic proceditur}. Videtur quod
Angeli non sint incorruptibiles. Dicit enim Damascenus de Angelo, quod
est {\itshape substantia intellectualis, gratia et non natura immortalitatem
suscipiens}.


&

第五項の問題へは、以下のように議論が進められる。天使たちは不滅でないと思
 われる。理由は以下の通り。ダマスケヌスは、天使について、「知的実体は、
 本性によってではなく恩恵によって、不死生を授けられている」と述べている。

\\


{\scshape 2 Praeterea}, Plato dicit, in {\itshape Timaeo}, {\itshape O
dii deorum, quorum opifex\footnote{ \u{o}p\u{\i}fex , \u{\i}cis,
comm. opus-facio, one who does a work.  I. Lit., a worker, maker,
framer, fabricator (class.; cf.: faber, artifex. operarius).  } idem
paterque ego, opera siquidem vos mea, dissolubilia natura, me tamen ita
volente indissolubilia}. Hos autem deos non aliud quam Angelos
intelligere potest. Ergo Angeli natura sua sunt corruptibiles.

&

さらに、プラトンは『ティマイオス』で次のように述べている。「おお、神々の
 神々よ。わたしは汝らの制作者であり同時に父でもある。汝らはわたしの作品
 なので、滅びる本性だが、わたしがそう欲するから、滅ばない」。ところで、
 この「神々」を、天使以外のものと理解することはできない。ゆえに、天使た
 ちは、自らの本性としては、滅びうる。

\\


{\scshape 3 Praeterea}, secundum Gregorium, {\itshape omnia in
nihilum deciderent, nisi ea manus omnipotentis conservaret}. Sed quod in
nihilum redigi potest, est corruptibile. Ergo, cum Angeli sint a Deo
facti, videtur quod sint corruptibiles secundum suam naturam.


&
さらに、グレゴリウスによれば、「もし全能者の手が万物を保たなかったら、それ
 らは無に帰するだろう」。ところが、無に帰しうるものは、可滅的である。ゆ
 えに、天使たちは神によって作られたのだから、その本性の点では可滅的で
 あると思われる

\\


{\scshape Sed contra est} quod Dionysius dicit, {\scshape iv}
cap.~{\itshape de Div.~Nom.}, quod intellectuales substantiae {\itshape
vitam habent indeficientem, ab universa corruptione, morte et materia et
generatione mundae existentes}.


&
しかし反対に、ディオニュシウスは『神名論』4章で、知性的諸実体は「欠陥の
 ない生命を持ち、すべての消滅、死、質料、生成の汚れがない」と述べ
 ている。

\\


{\scshape Respondeo dicendum} quod necesse est dicere
Angelos secundum suam naturam esse incorruptibiles. Cuius ratio est,
quia nihil corrumpitur nisi per hoc, quod forma eius a materia
separatur, unde, cum Angelus sit ipsa forma subsistens, ut ex dictis
patet, impossibile est quod eius substantia sit corruptibilis. 


&

解答する。以下のように言われるべきである。天使は、自らの本性の点で不滅で
 あると言うことが必然である。理由は以下の通り。なにものも、その形相が質
 料から分離することによらない限り、消滅しない。したがって、既述のことか
 ら明らかなとおり、天使は、自存する形相そのものだから、その実体が可滅的
 であることは不可能である。


\\

Quod enim
convenit alicui secundum se, nunquam ab eo separari potest, ab eo autem
cui convenit per aliud, potest separari, separato eo secundum quod ei
conveniebat. R\u{o}tund\u{\i}tas enim a circulo separari non potest, quia
convenit ei secundum seipsum, sed a\={e}n\u{e}us circulus potest amittere
rotunditatem per hoc, quod circularis figura separatur ab aere. 

&

じっさい、それ自体としてOに属するものAが、Oから分離されることは決してあり
 えないが、Aが他のものMを通してOに属するときには、AがMから分離されることによっ
 て、そのOから分離されうる。AはMにしたがってOに属するのだから。
たとえば、丸さAが円Oから切り離されることはありえない。なぜなら、丸さはそれ自体の
 点で円Oに属するから。しかし、銅の円Oは、円の形状が銅Mから切り離され
 ることによって、丸さを失うことがありうる。

\\

Esse
autem secundum se competit formae, unumquodque enim est ens actu
secundum quod habet formam. Materia vero est ens actu per
formam. Compositum igitur ex materia et forma desinit esse actu per hoc,
quod forma separatur a materia. Sed si ipsa forma subsistat in suo esse,
sicut est in Angelis, ut dictum est, non potest amittere esse. Ipsa
igitur immaterialitas Angeli est ratio quare Angelus est incorruptibilis
secundum suam naturam. 



&

さて、存在は、それ自体で形相に適合する。なぜなら、各々のものは、形相をも
 つという点で、現実態における存在者だから。他方、質料は、形相によって、
 現実態における存在者である。ゆえに、質料と形相から複合されたものは、形
 相が質料から分離されることによって、現実態における存在者であることをや
 める。しかし、もし形相自体が自らの存在において自存するならば、記述の通
 り天使はそのようなものだが、存在を捨てることが不可能である。ゆえに、天
 使の非質料性自体が、天使が自らの本性の点で不可滅的であることの根拠であ
 る。

\\





Et huius incorruptibilitatis signum accipi potest
ex eius intellectuali operatione, quia enim unumquodque operatur
secundum quod est actu, operatio rei indicat modum esse ipsius. Species
autem et ratio operationis ex obiecto comprehenditur. Obiectum autem
intelligibile, cum sit supra tempus, est sempiternum. Unde omnis
substantia intellectualis est incorruptibilis secundum suam naturam.


&


そして、この不可滅性のしるしは、天使の知性的働きから理解しうる。なぜなら、
 各々のものは、現実態にあるかぎりで働くので、事物の働きは、その事物の存
 在のあり方を示すからである。ところで、働きの種、ないし性格は、対象に基
 づいて把握される。ところで、可知的対象は、時間を超えているので、常にあ
 るものである。したがって、すべての知的実体は、その本性の点で、不可滅的
 である。

\\


{\scshape Ad primum ergo dicendum} quod Damascenus accipit
immortalitatem perfectam, quae includit omnimodam immutabilitatem, quia
{\itshape omnis mutatio est quaedam mors}, ut Augustinus
dicit. Perfectam autem immutabilitatem Angeli non nisi per gratiam
assequuntur, ut infra patebit.

&

第一異論に対しては、それゆえ、次のように言われるべきである。
ダマスケヌスは、完全な不死性のことを考えているのであり、それは、あらゆる
 点での不変性を含んでいる。というのも、アウグスティヌスが述べるように、
 「あらゆる変化は、一種の死である」からである。ところで、後に\footnote{
Q.~63, a.~2 ``Utrum in angelis possit esse tantum peccatum superbiae et
 invidiae'', a.~8 ``Utrum peccatum primi angeli fuerit aliis causa peccandi
}明らかに
 なるように、天使の完全な不死性は、恩恵によらなければ獲得されない。

\\


{\scshape Ad secundum dicendum} quod Plato per deos
intelligit corpora caelestia, quae existimabat esse ex elementis
composita, et ideo secundum suam naturam dissolubilia, sed voluntate
divina semper conservantur in esse.


&

第二異論に対しては、次のように言われるべきである。
プラトンは、「神々」ということで、天体のことを考えている。天体は、諸元素
 から複合されていると彼は考えていたので、その本性上、解体しうるものだが、
 しかし、神の意志によって、常に存在に保たれている。

\\


{\scshape Ad tertium dicendum} quod, sicut supra dictum
est, quoddam necessarium est quod habet causam suae necessitatis. Unde
non repugnat necessario nec incorruptibili, quod esse eius dependeat ab
alio sicut a causa. Per hoc ergo quod dicitur quod omnia deciderent in
nihilum nisi continerentur a Deo, et etiam Angeli, non datur intelligi
quod in Angelis sit aliquod corruptionis principium, sed quod esse
Angeli dependeat a Deo sicut a causa. Non autem dicitur aliquid esse
corruptibile, per hoc quod Deus possit illud in non esse redigere,
subtrahendo suam conservationem, sed per hoc quod in seipso aliquod
principium corruptionis habet, vel contrarietatem vel saltem potentiam
materiae.


&
第三異論に対しては、次のように言われるべきである。
以前に述べたとおり、必然的なものには、自らの必然性の原因を持つものがある。
 したがって、あるものが、原因としての他のものに依存することは、必然的で
 あることや不可滅であることと矛盾しない。ゆえに、万物は、天使さえも、神
 に含まれなかったら無に帰すると言われることによって、天使の中に、何か消
 滅の根源があるということが理解されるのではなく、むしろ、天使の存在は、
 原因としての神に依存すると言うことが理解されるべく与えられる。
ところで、あるものが可滅的だと言われるのは、神が、自らの保存の働きを引き
 下げることによって、それを無に帰することができるためではなく、むしろ、
 それ自身の中に、対立や、あるいは、質料の可能態など、なんらかの消滅の根
 源をもつからである。

\end{longtable}



\end{document}