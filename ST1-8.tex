\documentclass[10pt]{jsarticle} % use larger type; default would be 10pt
%\usepackage[utf8]{inputenc} % set input encoding (not needed with XeLaTeX)
%\usepackage[round,comma,authoryear]{natbib}
%\usepackage{nruby}
\usepackage{okumacro}
\usepackage{longtable}
%\usepqckage{tablefootnote}
\usepackage[polutonikogreek,english,japanese]{babel}
%\usepackage{amsmath}
\usepackage{latexsym}
\usepackage{color}
%----- header -------
\usepackage{fancyhdr}
\pagestyle{fancy}
\lhead{{\it Summa Theologiae} I, q.~8}
%--------------------

\bibliographystyle{jplain}
\title{{\bf Prima Pars}\\{\HUGE Summae Theologiae}\\Sancti Thomae
Aquinatis\\Quaestio Octava\\{\bf De Existentia Dei In Rebus}}
\author{Japanese translation\\by Yoshinori {\sc Ueeda}}
\date{Last modified \today}

%%%% コピペ用
%\rhead{a.~}
%\begin{center}
% {\Large {\bf }}\\
% {\large }\\
% {\footnotesize }\\
% {\Large \\}
%\end{center}
%
%\begin{longtable}{p{21em}p{21em}}
%
%&
%
%\\
%\end{longtable}
%\newpage



\begin{document}
\maketitle
\begin{center}
{\Large 第八問\\神の諸事物における存在について}
\end{center}



\begin{longtable}{p{21em}p{21em}}

Quia vero infinito convenire videtur quod ubique et in omnibus sit,
 considerandum est utrum hoc Deo conveniat. Et circa hoc quaeruntur
 quatuor. 
\begin{enumerate}
 \item utrum Deus sit in omnibus rebus.
 \item utrum Deus sit ubique.
 \item utrum Deus sit ubique per essentiam et potentiam et praesentiam.
 \item utrum esse ubique sit proprium Dei.
\end{enumerate}

&

無限なものには、いたるところに、万物のなかに存在することが適合すると思わ
 れるので、このことが、神に適合するかどうかが考察されるべきである。これ
 を巡って四つのことが問われる。
\begin{enumerate}
 \item 神はすべての事物の中に存在するか。
 \item 神は遍在するか。
 \item 神は、本質、能力、現在によって、遍在するか。
 \item 遍在することは、神に固有か。
\end{enumerate}
\end{longtable}

\newpage
\rhead{a.~1}
\begin{center}
 {\Large {\bf ARTICULUS PRIMUS}}\\
 {\large UTRUM DEUS SIT IN OMNIBUS REBUS}\\
 {\footnotesize I {\itshape Sent.}, d.~37, q.~1, a.~1; III {\itshape
 SCG.}, c.~68.}\\
 {\Large 第一項\\神はすべての事物の中に存在するか}
\end{center}

\begin{longtable}{p{21em}p{21em}}

{\huge A}{\scshape d primum sic proceditur}. Videtur quod Deus non sit in omnibus
 rebus. Quod enim est supra omnia, non est in omnibus rebus. Sed Deus
 est supra omnia, secundum illud Psalmi, {\itshape excelsus super omnes gentes
 dominus}, et cetera. Ergo Deus non est in omnibus rebus.
&

第一にかんしては、次のように進められる。神は、すべての事物の中に存在する
 のではないと思われる。すべてのものの上にあるものは、すべての事物の中に
 あるのではない。しかるに、かの『詩編』「主はすべての種族より高く」
 \footnote{「主はすべての国を超えて高くいまし/その栄光は天を超える。」(113:4)}云々
 によれば、神は万物の上にある。ゆえに、神はすべての事物の中に存在するわ
 けではない。

\\

2.~{\scshape Praeterea}, quod est in aliquo, continetur ab eo. Sed Deus non
 continetur a rebus, sed magis continet res. Ergo Deus non est in rebus,
 sed magis res sunt in eo. Unde Augustinus, in libro {\itshape Octoginta trium
 Quaest}., dicit quod {\itshape in ipso potius sunt omnia, quam ipse alicubi}.

&

さらに、何かの中にあるものは、それによって含まれる。しかるに、神は、事物
 によって含まれず、むしろ事物を含む。ゆえに、神は事物の中になく、むしろ
 事物が神の中にある。このことからアウグスティヌスは、『八十三問題集』の
 中で、「神がどこかに存在するというよりは、神の中に万物が存在する」と述
 べている。


\\


3.~{\scshape Praeterea}, quanto aliquod agens est virtuosius, tanto ad magis
 distans eius actio procedit. Sed Deus est virtuosissimum agens. Ergo
 eius actio pertingere potest ad ea etiam quae ab ipso distant, nec
 oportet quod sit in omnibus.

&

 さらに、ある作用者が、力強ければ強いほど、
 彼の働きは、より遠くのものまで進んでいく。しかるに、神は、もっとも力強
 い作用者である。ゆえに、彼の働きは、彼から離れているものどもにまで到達
 することができるのであり、万物の中に存在する必要はない。

\\

4.~{\scshape Praeterea}, Daemones res aliquae sunt. Nec tamen Deus est in
 Daemonibus, non enim est {\itshape conventio lucis ad tenebras}, ut dicitur II {\itshape ad Cor}.~{\scshape vi}. Ergo Deus non est in omnibus rebus.
&

さらに、悪魔たちは、なんらかの事物である。しかし、神は悪魔たちの中にはな
 い。なぜなら、『コリント人への第二の手紙』6章で言われるように、「光が闇
 に交わることはない」\footnote{「光と闇とに何のつながりがありますか」
 (6:14)}からである。ゆえに、神が万物の中に存在するわけではない。

\\


{\scshape Sed contra}, ubicumque operatur aliquid, ibi est. Sed Deus
 operatur in omnibus, secundum illud {\itshape Isaiae} {\scshape xxvi},
 {\itshape omnia opera nostra operatus es in nobis, Domine}. Ergo Deus
 est in omnibus rebus.  &

しかし反対に、どこであれ何かが働くところには、その何かが存在する。しかる
 に、『イザヤ書』26章「主よ、あなたは私たちのすべての働きを私たちの中で働
 いた」\footnote{「主よ、平和をわたしたちにお授けください。私たちのすべて
 の業を成し遂げてくださるのはあなたです。」(26:12)}によれば、神は万物にお
 いて働く。ゆえに、神はすべての事物の中に存在する。

\\


{\scshape Respondeo dicendum} quod Deus est in omnibus rebus, non quidem sicut pars
 essentiae, vel sicut accidens, sed sicut agens adest ei in quod
 agit. Oportet enim omne agens coniungi ei in quod immediate agit, et
 sua virtute illud contingere, unde in VII {\itshape Physic}. probatur quod motum
 et movens oportet esse simul. 

&
答えて言わなければならない。神は、すべての事物の中に存在するが、それは、
 本質の部分や付帯性としてではなく、作用者が、それへと働くところ
 のもの[=作用の対象]に臨在するというかたちである。じっさい、すべて
 の作用者は、それに向かって直接的に作用するところのものに結び付けられ、
 自らの力によってそれ[=対象]に触れるのでなければならない。このことか
 ら、『自然学』第7巻で、動かされるものと動かすものは、同時に存在しなけれ
 ばならないことが証明されている。

\\

Cum autem Deus sit ipsum esse per suam essentiam, oportet quod
 esse creatum sit proprius effectus eius; sicut ignire est proprius
 effectus ignis. Hunc autem effectum causat Deus in rebus, non solum
 quando primo esse incipiunt, sed quandiu in esse conservantur; sicut
 lumen causatur in aere a sole quandiu aer illuminatus manet. Quandiu
 igitur res habet esse, tandiu oportet quod Deus adsit ei, secundum
 modum quo esse habet. 

&
しかるに、神は自らの本質によって存在それ自身なので、創造された存在
 は神の固有の結果でなければならない。ちょうど、燃やすことが、火の固有の結果である
 ように。神はこの結果[=存在]を、諸事物が最初に存在し始めるときだけ
 でなく、存在において保存されているあいだ、原因している(生み出してい
 る)。ちょうど、空
 気に光が満ちているあいだ、光が太陽によって原因されている(生み出され
 ている)ように。それ
 ゆえ、事物が存在をもっているあいだ、それが存在をもつあり方に応じて、神
 がその事物に臨在していなければならない。

\\

Esse autem est illud quod est magis intimum cuilibet, et quod
 profundius omnibus inest, cum sit formale respectu omnium quae in re
 sunt, ut ex supra dictis patet. Unde oportet quod Deus sit in omnibus
 rebus, et intime.

&


また存在は、上述のことから明らかなとおり、事物の中にある全てのものとの
 関係において形相的なものなので、どんな事物にとっても最奥にあり、どん
 なものよりも深く内在している。したがって、神はすべての事物の中に存在
 し、かつ、もっとも奥に存在するのでなければならない。

\\

{\scshape Ad primum ergo dicendum} quod Deus est supra omnia per excellentiam
 suae naturae, et tamen est in omnibus rebus, ut causans omnium esse, ut
 supra dictum est.

&

第一に対しては、それゆえ、次のように言われるべきである。神は、自らの本性
 の卓越性によって万物の上にあるが、しかし、上述のように、万物の存在を生
 み出すものとして、すべての事物の中に存在する。

\\

{\scshape Ad secundum dicendum} quod, licet corporalia dicantur esse in aliquo
 sicut in continente, tamen spiritualia continent ea in quibus sunt,
 sicut anima continet corpus. Unde et Deus est in rebus sicut continens
 res. Tamen, per quandam similitudinem corporalium, dicuntur omnia esse
 in Deo, inquantum continentur ab ipso.

&

第二に対しては、次のように言われるべきである。物体的なものの場合、「XがY
 の中にある」と言われるとき、YがXを含むが、霊的なもの(=非物体的なもの)
 の場合、XがYを含む 。たとえば、[魂は身体の中にあるが、]魂は身体を含む
 のである。したがって、神も、諸事物を含むものとして、諸事物の中にある。
 しかし、ある種の物体的なものどもの類似によって、万物が神に含まれるかぎ
 りにおいて、万物が神の中にあるとも言われる。

\\


{\scshape Ad tertium dicendum} quod nullius agentis, quantumcumque virtuosi,
 actio procedit ad aliquid distans, nisi inquantum in illud per media
 agit. Hoc autem ad maximam virtutem Dei pertinet, quod immediate in
 omnibus agit. Unde nihil est distans ab eo, quasi in se illud Deum non
 habeat. Dicuntur tamen res distare a Deo per dissimilitudinem naturae
 vel gratiae, sicut et ipse est super omnia per excellentiam suae
 naturae.

&

第三に対しては、次のように言われるべきである。どんな作用者でも、それがど
 れほど力強くても、媒介を通して働くのでなければ、隔たったものへ、その働
 きが進んでいくことはない。これに対して、直接的に(=媒介なしに)万物に
 おいて働くことは、この上ない神の能力に属している。このことから、自らの
 中にその神を持たないほどまでに、神から離れているものは何もない。しかし、
 神は自らの本性の卓越性によって万物の上にあると言われる場合のように、事
 物は、本性や恩恵の不類似によって、神から隔たると言われる。

\\


{\scshape Ad quartum dicendum} quod in Daemonibus intelligitur et natura, quae
 est a Deo, et deformitas culpae, quae non est ab ipso. Et ideo non est
 absolute concedendum quod Deus sit in Daemonibus, sed cum hac
 additione, {\itshape inquantum sunt res quaedam}. In rebus autem quae nominant
 naturam non deformatam, absolute dicendum est Deum esse.

&

第四に対しては、次のように言われるべきである。悪魔においては、神に由来する自然本性と、
 神に由来しない罪の醜さが理解される。ゆえに、
 神が悪魔の中にいるということは、無条件に認められるべきではなく、ただ、
 悪魔がある種の事物であるかぎりにおいて認められるべきである。これに対
 して、醜くされていない本性を名付ける諸事物の中には(=「事物」という名
 称が、醜くされていない本性(=悪魔以外のもの)を指す場合には)、無条件
 に神がいると言われるべきである。


\end{longtable}
\newpage
\rhead{a.~2}
 

\begin{center}
 {\Large {\bf ARTICULUS SECUNDUS}}\\
 {\large UTRUM DEUS SIT UBIQUE}\\
 {\footnotesize Infra, q.~16, a.~7, ad 2; q.~52, a.~2; I {\itshape
 Sent.}, d.~37, q.~2, a.~1; III {\itshape SCG.}, c.~68; {\itshape
 Quodl.}~XI, a.~1.}\\
 {\Large 第二項\\神は遍在するか}
\end{center}

\begin{longtable}{p{21em}p{21em}}
{\huge A}{\scshape d secundum sic proceditur}. Videtur quod Deus non sit
 ubique. Esse enim ubique significat esse in omni loco. Sed esse in omni
 loco non convenit Deo, cui non convenit esse in loco, nam incorporalia,
 ut dicit Boetius, in libro {\itshape de Hebdomad}., non sunt in loco. Ergo Deus
 non est ubique.

&

第二にかんしては次のように進められる。神は遍在しないと思われる。「遍在す
 る」は、「すべての場所に存在する」を意味する。しかるに、場所に存在する
 ということが適合しない神に、すべての場所に存在することは適合しない。な
 ぜなら、ボエティウスが『デ・ヘブドマディブス』という書物で述べるように、
 非物体的なものどもは、場所に存在しないからである。

\\

2.~{\scshape Praeterea}, sicut se habet tempus ad successiva, ita se habet
 locus ad permanentia. Sed unum indivisibile actionis vel motus, non
 potest esse in diversis temporibus. Ergo nec unum indivisibile in
 genere rerum permanentium, potest esse in omnibus locis. Esse autem
 divinum non est successivum, sed permanens. Ergo Deus non est in
 pluribus locis. Et ita non est ubique.

&

さらに、時間が継起するものにたいする関係は、場所が恒存するものにたいする
 関係に等しい。しかるに、働きであれ運動であれ、ひとつの不可分のものが、異
 なる時間に存在することはできない。ゆえに、恒存する諸事物の類における、ひ
 とつの不可分のものも、すべての場所にあることはできない。しかるに、神の存
 在は、継起するものではなく恒存するものである。ゆえに、神は複数の場所に存
 在することはできない。したがって、遍在しない。\footnote{ややわかりにくい
 議論なので内容を取ると以下の通り。「時間と図形のあいだには対応関係がある。
 不可分の時間(=瞬間)が、さまざまな時刻に対応することがないように、不可
 分の図形(=点)が、さまざまな場所に存在することもない。神は、不可分の存
 在者だから、特定の点のように、さまざまな場所に存在することはできない。」}


\\



3.~{\scshape Praeterea}, quod est totum alicubi, nihil eius est extra locum
 illum. Sed Deus, si est in aliquo loco, totus est ibi, non enim habet
 partes. Ergo nihil eius est extra locum illum. Ergo Deus non est
 ubique.

&

さらに、全体がある場所に存在するものは、それに属するどれも(=どの部分
 も)、その場所の外にはない。しかるに、神は、もしある場所に存在するなら
 ば、全体がそこにある。なぜなら、神は部分を持たないからである。ゆえに、
 神はその場所の外に、神に属するどんなものも存在しない。ゆえに、神は遍在
 しない。

\\

{\scshape Sed contra est} quod dicitur {\itshape Ierem}.~{\scshape xxiii}, {\itshape caelum et terram ego impleo}.
&
しかし反対に、『エレミヤ書』23章で、「私は天と地を満たす」\footnote{「天をも地をも、わたしは満たしているではないかと主は言われる。」(23:24)}と言われている。

\\

{\scshape Respondeo dicendum} quod, cum locus sit res quaedam, esse aliquid in loco
 potest intelligi dupliciter, vel per modum aliarum rerum, idest sicut
 dicitur aliquid esse in aliis rebus quocumque modo, sicut accidentia
 loci sunt in loco; vel per modum proprium loci, sicut locata sunt in
 loco. 

&

答えて言わなければならない。場所は一種の事物だから、あるものが場所に存在
 するということは、二通りに理解される。ひとつには、[場所以外の]他の諸
 事物のしかた、つまり、あるものが、他の事物において、何らかのかたちで存
 在すると言われるしかたによってである。たとえば、場所の付帯性が場所にお
 いてある、という場合のように。または、場所に固有なしかたによって理解さ
 れ、たとえば、場所に置かれたものが場所にある、というしかたによってであ
 る。

\\

Utroque autem modo, secundum aliquid, Deus est in omni loco,
 quod est esse ubique. Primo quidem, sicut est in omnibus rebus, ut dans
 eis esse et virtutem et operationem, sic enim est in omni loco, ut dans
 ei esse et virtutem locativam. Item, locata sunt in loco inquantum
 replent locum, et Deus omnem locum replet. Non sicut corpus, corpus
 enim dicitur replere locum, inquantum non compatitur secum aliud
 corpus; sed per hoc quod Deus est in aliquo loco, non excluditur quin
 alia sint ibi, imo per hoc replet omnia loca, quod dat esse omnibus
 locatis, quae replent omnia loca.
&

さて、このどちらのしかたによっても、ある意味で、神はすべての場所にある。
 つまり、遍在する。第一に、神が、万物に存在と能力と働きを与えるものとし
 て、万物の中に存在するように、すべての場所に、存在と場所としての力を与
 えるものとして、すべての場所の中に存在する。さらに、場所にあるものは、
 場所を満たすかぎりにおいて、その場所に存在するが、神はすべての場所を満
 たす。しかしそれは物体のようにではない。物体は、他の物体を、自分と一緒
 に持たないというかたちで場所を満たすが、神がある場所に存在することによっ
 て、他のものがそこに存在することが排除されることはなく、むしろ、すべて
 の場所を満たしている、すべての「場所にあるもの」に存在を与えることによっ
 て、すべての場所を満たす。

\\

{\scshape Ad primum ergo dicendum} quod incorporalia non sunt in loco per
 contactum quantitatis dimensivae, sicut corpora, sed per contactum
 virtutis.

&

第一に対しては、それゆえ、次のように言われるべきである。非物体的なものは、
 物体のように、次元量の接触によって場所に存在することはないが、力の接触
 によって場所に存在する。

\\

{\scshape Ad secundum dicendum} quod indivisibile est duplex. Unum quod est
 terminus continui, ut punctus in permanentibus, et momentum in
 successivis. Et huiusmodi indivisibile, in permanentibus, quia habet
 determinatum situm, non potest esse in pluribus partibus loci, vel in
 pluribus locis, et similiter indivisibile actionis vel motus, quia
 habet determinatum ordinem in motu vel actione, non potest esse in
 pluribus partibus temporis. 

&
第二に対しては、次のように言われるべきである。不可分のものには二通りある。
 ひとつは、連続体の端であり、たとえば、恒存するものにおける点や、継起的
 なものにおける瞬間がそれである。このような不可分のものは、恒存するもの
 において、限定された場所をもつので、ひとつの場所の複数の部分に、あるい
 は複数の場所に存在することができない。同様に、働きや運動に属する不可分
 のものも、運動や働きにおける限定された順序をもつので、時間の複数の部分
 において存在することができない。

\\

Aliud autem indivisibile est, quod est extra totum genus
 continui, et hoc modo substantiae incorporeae, ut Deus, Angelus et
 anima, dicuntur esse indivisibiles. Tale igitur indivisibile non
 applicatur ad continuum sicut aliquid eius, sed inquantum contingit
 illud sua virtute. Unde secundum quod virtus sua se potest extendere ad
 unum vel multa, ad parvum vel magnum, secundum hoc est in uno vel
 pluribus locis, et in loco parvo vel magno.

&

もうひとつの不可分のものとは、連続体のすべての類の外にあるものであり、
 このしかたで、神、天使、そして[知性的]魂のような非物体的な実体が、不
 可分のものと言われる。ゆえに、このような不可分のものは、連続体の何かと
 して連続体に結びつくのではなく、自らの力によってそれに触れるかぎりにお
 いて、それに結びつく。したがって、自らの力が、一つのものや多くのものへ、
 小さいものや大きいものへ及ぶかぎりにおいて、一つのものや多くの場所に、
 小さい場所や大きい場所に、存在する。

\\

{\scshape Ad tertium dicendum} quod totum dicitur respectu partium. Est autem
 duplex pars, scilicet pars essentiae, ut forma et materia dicuntur
 partes compositi, et genus et differentia partes speciei; et etiam pars
 quantitatis, in quam scilicet dividitur aliqua quantitas. Quod ergo est
 totum in aliquo loco totalitate quantitatis, non potest esse extra
 locum illum, quia quantitas locati commensuratur quantitati loci, unde
 non est totalitas quantitatis, si non sit totalitas loci. 

&

第三に対しては、次のように言われるべきである。全体は、部分との関係におい
 て言われる。しかるに、部分には二通りある。一つには、形相と質料が複合体
 の部分と言われたり、類と種差が種の部分と言われるように、本質の部分であ
 り、もうひとつは、量の部分であり、それへとなんらかの量が分割される。ゆ
 えに、量の全体性によって、なんらかの場所に全体があるものは、その場所の
 外にはありえない。なぜなら、場所に置かれたものの量は、場所の量と同じ尺
 度で測られるので、もし場所の全体性がないならば、量の全体性もないからで
 ある。

\\

Sed totalitas essentiae non commensuratur totalitati loci. Unde
 non oportet quod illud quod est totum totalitate essentiae in aliquo,
 nullo modo sit extra illud. Sicut apparet etiam in formis
 accidentalibus, quae secundum accidens quantitatem habent, albedo enim
 est tota in qualibet parte superficiei, si accipiatur totalitas
 essentiae, quia secundum perfectam rationem suae speciei invenitur in
 qualibet parte superficiei, si autem accipiatur totalitas secundum
 quantitatem, quam habet per accidens, sic non est tota in qualibet
 parte superficiei. In substantiis autem incorporeis non est totalitas,
 nec per se nec per accidens, nisi secundum perfectam rationem
 essentiae. Et ideo, sicut anima est tota in qualibet parte corporis,
 ita Deus totus est in omnibus et singulis.

&

これに対して、本質の全体性は、場所の全体性と同じ尺度で測られない。したがっ
 て、本質の全体性によってあるものにおいて全体が存在するものは、それ以外
 のところに決して存在しない、という必然性はない。たとえば、付帯的に量を
 もつ付帯形相においてもこのことは明らかである。すなわち、白は、本質の全
 体性が理解されるならば、全体が、表面のどの部分にも存在する。なぜなら、
 表面のどの部分においても自らの種の完全な性格において見出されるからであ
 る。これに対して、量における全体性が理解されるならば、白はそれを付帯的
 にもつのだが、その場合には、表面のどの部分においても全体としては存在し
 ない。ところで、非物体的な実体においては、自体的にも付帯的にも、本質の
 完全な性格における以外に全体性はない。ゆえに、ちょうど魂が、身体のどの
 部分においても全体としてあるように、神は、万物において、個々のものにお
 いて、全体として存在する。


\end{longtable}
\newpage
\rhead{a.~3} 

\begin{center}
 {\Large {\bf ARTICULUS TERTIUS}}\\
 {\large UTRUM DEUS SIT UBIQUE PER ESSENTIAM, PRAESENTIAM ET POTENTIAM}\\
 {\footnotesize I {\itshape Sent.}, d.~37, q.~1, a.~2; et in expos.~lit.}\\
 {\Large 第三項\\神は本質、現在、能力によって遍在するか}
\end{center}

\begin{longtable}{p{21em}p{21em}}



{\huge A}{\scshape d tertium sic proceditur}. Videtur quod male assignentur modi
 existendi Deum in rebus, cum dicitur quod Deus est in omnibus rebus per
 essentiam, potentiam et praesentiam. Id enim per essentiam est in
 aliquo, quod essentialiter est in eo. Deus autem non est essentialiter
 in rebus, non enim est de essentia alicuius rei. Ergo non debet dici
 quod Deus sit in rebus per essentiam, praesentiam et potentiam.

&

第三に関しては、次のように進められる。神はすべての事物の中に、本質、能力、
 現在によって存在すると言われるとき、神が事物において存在するあり方が、
 悪く指定されていると思われる。あるものの中に本質によって存在するものは、
 そのものの中に本質的に存在する。しかるに、神は、本質的に事物の中に存在
 するのではない。なぜなら、神はどんな事物の本質にも属さないからである。
 ゆえに、神は本質、現在、そして能力によって事物の中に存在すると言われる
 べきでない。

\\


2.~{\scshape Praeterea}, hoc est esse praesentem alicui rei, scilicet non
 deesse illi. Sed hoc est Deum esse per essentiam in rebus, scilicet non
 deesse alicui rei. Ergo idem est esse Deum in omnibus per essentiam et
 praesentiam. Superfluum ergo fuit dicere quod Deus sit in rebus per
 essentiam, praesentiam et potentiam.

&

さらに、ある事物に現在するとは、その事物にとって不在でないということであ
 る。しかるに、神が、事物の中に本質によって存在するとは、ある事物にとっ
 て不在でないということである。ゆえに、神が万物の中に本質によって存在す
 ることと現在によって存在することとは同一である。ゆえに、神が事物の中に、
 本質、現在、能力によって存在するというのは冗長である。

\\

3.~{\scshape Praeterea}, sicut Deus est principium omnium rerum per suam
 potentiam, ita per scientiam et voluntatem. Sed non dicitur Deus esse
 in rebus per scientiam et voluntatem. Ergo nec per potentiam.

&

さらに、神が、その能力によって、すべての事物の根源であるように、知によっ
 ても意志によっても、根源である。しかるに、神が、知と意志によって、事物
 の中に存在するとは言われない。ゆえに、能力によっても、そのようには言わ
 れない。

\\

4.~{\scshape Praeterea}, sicut gratia est quaedam perfectio superaddita
 substantiae rei, ita multae sunt aliae perfectiones superadditae. Si
 ergo Deus dicitur esse speciali modo in quibusdam per gratiam, videtur
 quod secundum quamlibet perfectionem debeat accipi specialis modus
 essendi Deum in rebus.

&

さらに、ちょうど、恩恵が、事物の実体に加えられた一種の完全性であるように、
 加えられた他の完全性は数多い。ゆえに、もし神が、恩恵によって、あるもの
 どもの中に、特別なしかたで存在すると言われるとすれば、どんな完全性に従っ
 ても、神が事物の中に存在する特殊な存在のしかたが理解されるべきである。

\\

{\scshape Sed contra est} quod Gregorius dicit, super {\itshape
 Cant.~Cantic.}, quod {\itshape Deus communi modo est in omnibus rebus
 praesentia, potentia et substantia, tamen familiari modo dicitur esse
 in aliquibus per gratiam}.

&

しかし反対に、グレゴリウス が『雅歌注解』で「神は、共通的なかたちでは、
 すべての事物の中に現前、能力、実体によって存在するが、しかし、より親密
 には、ある人々の中に恩恵によって存在すると言われる」と述べている。

\\

Respondeo dicendum quod Deus dicitur esse in re aliqua dupliciter. Uno
 modo, per modum causae agentis, et sic est in omnibus rebus creatis ab
 ipso. Alio modo, sicut obiectum operationis est in operante, quod
 proprium est in operationibus animae, secundum quod cognitum est in
 cognoscente, et desideratum in desiderante. Hoc igitur secundo modo,
 Deus specialiter est in rationali creatura, quae cognoscit et diligit
 illum actu vel habitu. Et quia hoc habet rationalis creatura per
 gratiam, ut infra patebit, dicitur esse hoc modo in sanctis per
 gratiam. 

&

答えて言わなければならない。神が事物の中に存在するということは、二通りの
意味で言われる。ひとつには、作出因のかたちによってであり、この意味では、
神は、神によって創造されたすべての事物のなかに存在する。もう一つには、
働きの対象が働くもののうちにある、というかたちによってであり、このことは、
認識されるものが認識するもののうちにあり、欲求されるものが欲求するものの
うちにあるかぎりにおいて、魂の働きに固有である。ゆえに、この二番目のかた
ちで、神は特別に、現実的にであれ習態的にであれ神を認識し愛する理性的被造
物のなかに存在する。そして、以下に明らかにされるように、理性的被造物はこ
のことを恩恵によってもつので、神はこのかたちで、聖なるものどもの中に、恩
恵によって存在すると言われる。

\\

In rebus vero aliis ab ipso creatis quomodo sit, considerandum est ex
 his quae in rebus humanis esse dicuntur. Rex enim dicitur esse in toto
 regno suo per suam potentiam, licet non sit ubique praesens.  Per
 praesentiam vero suam, dicitur aliquid esse in omnibus quae in
 prospectu ipsius sunt; sicut omnia quae sunt in aliqua domo, dicuntur
 esse praesentia alicui, qui tamen non est secundum substantiam suam in
 qualibet parte domus.  Secundum vero substantiam vel essentiam, dicitur
 aliquid esse in loco in quo eius substantia habetur.


&

他方、神によって創造された他の事物の中に、神がどのようなかたちで存在するかということは、人間的な事柄のなかに存在すると言われるものどもに基づいて、考察されなければならない。たとえば、王は、あらゆるところに現在しているわけではないが、能力によって、彼の王国全体の中に存在すると言われる。
他方、あるものは、その人の視界の中に存在するすべてのもののなかに、自らの現在によって存在すると言われる。たとえば、ある家の中に存在するすべてのものが、ある人に現在していると言われる。しかし、彼は、自らの実体において、家のあらゆる部分に存在するわけではない。
また、実体ないし本質にしたがって、あるものが、それの実体が保持される場所の中に存在すると言われる。

\\

Fuerunt ergo aliqui, scilicet Manichaei, qui dixerunt divinae potestati
 subiecta spiritualia esse et incorporalia, visibilia vero et corporalia
 subiecta esse dicebant potestati principii contrarii. Contra hos ergo
 oportet dicere quod Deus sit in omnibus per potentiam suam. 
Fuerunt vero alii, qui licet crederent omnia esse subiecta divinae potentiae, tamen providentiam divinam usque ad haec inferiora corpora non extendebant, ex quorum persona dicitur Iob XXII, circa cardines caeli perambulat, nec nostra considerat. Et contra hos oportuit dicere quod sit in omnibus per suam praesentiam. 
Fuerunt vero alii, qui licet dicerent omnia ad Dei providentiam
 pertinere, tamen posuerunt omnia non immediate esse a Deo creata, sed
 quod immediate creavit primas creaturas, et illae creaverunt alias. Et
 contra hos oportet dicere quod sit in omnibus per essentiam. 

&

それゆえ、マニ教徒のように、霊的なものや非物体的なものは神の権能に従属し、
可視的なものや物体的なものは、[神とは]反対の原理に従属すると言った人た
ちがいた。それゆえ、この人たちに反対して、神は万物の中に、神の能力によっ
て存在すると言うべきである。他方、万物が神の権能に従属することを信じて
はいるが、神の摂理は、これらの下位の物体にまでは広がらないと[言った]人
たちがいた。このような人たちの人格に基づいて[=このような人たちを代表し
て]、『ヨブ記』22章で、「[神は]天のいと高き所を歩み、私たちのことを顧
みない」と言われている。そして、このような人たちに反対して、神は万物の中
に、自らの現在によって存在すると言わなければならなかった。また、万物が
神の摂理に属すると言いながら、しかし、万物は、直接的に神によって創造され
たのではなく、直接的には第一の被造物を創造し、彼らが他の被造物を創造した
と考えた人たちもいた。そしてこのような人たちに反対して、神は、本質によっ
て、万物の中に存在すると言わなければならない。

\\

Sic ergo est in omnibus per potentiam, inquantum omnia eius potestati
 subduntur. Est per praesentiam in omnibus, inquantum omnia nuda sunt et
 aperta oculis eius. Est in omnibus per essentiam, inquantum adest
 omnibus ut causa essendi, sicut dictum est.

&

それゆえ、このようにして、神は万物の中に、万物が神の権能に従属するかぎり
 において、能力によって存在し、万物が神の目から見て、裸であり露わである
 かぎりにおいて、現在によって存在し、すでに述べられたように、存在の原因
 として万物に臨在するかぎりにおいて、万物の中に本質によって存在する。

\\

{\scshape Ad primum ergo dicendum} quod Deus dicitur esse in omnibus per
 essentiam, non quidem rerum, quasi sit de essentia earum, sed per
 essentiam suam, quia substantia sua adest omnibus ut causa essendi,
 sicut dictum est.

&

第一に対しては、それゆえ、次のように言われるべきである。神は万物の中に本
 質によって存在するが、その本質は、神が事物の本質に属するというような意
 味で、事物の本質ではない。そうではなく、神は神の本質によって、事物の中
 に存在する。なぜなら、すでに述べられたように、神の実体が、存在の原因と
 して、万物に臨在するからである。

\\

{\scshape Ad secundum dicendum} quod aliquid potest dici praesens alicui,
 inquantum subiacet eius conspectui, quod tamen distat ab eo secundum
 suam substantiam, ut dictum est. Et ideo oportuit duos modos poni,
 scilicet per essentiam, et praesentiam.

&

第二に対しては、次のように言われるべきである。すでに述べられたように、
 「XがYに現在する」は、XがYから、その実体において隔たっていても、Yの視野
 のもとにあるかぎりにおいて、言われうる。ゆえに、本質によって、現在によっ
 て、という二つのあり方が示される必要があった。

\\


{\scshape Ad tertium dicendum} quod de ratione scientiae et voluntatis est,
 quod scitum sit in sciente, et volitum in volente, unde secundum
 scientiam et voluntatem, magis res sunt in Deo, quam Deus in rebus. Sed
 de ratione potentiae est, quod sit principium agendi in aliud, unde
 secundum potentiam agens comparatur et applicatur rei exteriori. Et sic
 per potentiam potest dici agens esse in altero.

&

第三に対しては、次のように言われるべきである。知と意志の性格には、知られ
 たものが知るもの中に、意志されたものが意志するものの中に在る、というこ
 とが属するので、知と意志に従っては、神が事物の中にあるというよりは、む
 しろ事物が神の中にある。これに対して、能力の性格の中には、それが、他の
 ものに向かって働きかける根源であることが含まれるので、能力にしたがって
 は、作用者が、外の事物に関係付けられ当てはめられる。このようにして、作
 用者は、能力によって、他のものの中にあると言われうる。

\\


{\scshape Ad quartum dicendum} quod nulla alia perfectio superaddita
 substantiae, facit Deum esse in aliquo sicut obiectum cognitum et
 amatum, nisi gratia, et ideo sola gratia facit singularem modum essendi
 Deum in rebus. Est autem alius singularis modus essendi Deum in homine
 per unionem, de quo modo suo loco agetur.

&

第四に対しては、次のように言われるべきである。恩恵以外のどんな完全性が実
 体に加えられても、神が、認識され愛される対象として、そのもののうちに存
 在することにはならない。ゆえに、恩恵だけが、神が事物の中に存在する、特
 別なあり方を作る。ただし、神が人間の中に、合一によって存在する特別なあ
 り方がもう一つあるが、それについては、それについての場所で 論じられる。

\end{longtable}
\newpage
\rhead{a.~4}

\begin{center}
 {\Large {\bf ARTICULUS QUARTUS}}\\
 {\large UTRUM ESSE UBIQUE SIT PROPRIUM DEI}\\
 {\footnotesize Infra q.~52, a.~2; 112, a.~1; I {\itshape Sent.}, d.~37,
 q.~2, a.~2; q.~3, a.~2; IV {\itshape SCG.}, c.~17; {\itshape
 Quodl.}~XI, a.~1; {\itshape De Div.~Nom.}, c.~3, lect.~1.}\\
 {\Large 第四項\\遍在は神に固有か}
\end{center}

\begin{longtable}{p{21em}p{21em}}

{\huge A}{\scshape d quartum sic proceditur}. Videtur quod esse ubique non sit
 proprium Dei. Universale enim, secundum philosophum, est ubique et
 semper, materia etiam prima, cum sit in omnibus corporibus, est
 ubique. Neutrum autem horum est Deus, ut ex praemissis patet. Ergo esse
 ubique non est proprium Dei.

&

第一にかんしては、次のように進められる。遍在は神に固有ではないと思われる。
 哲学者によれば、普遍はいたるところに常に存在し、また第一質料もすべ
 ての物体の中に存在するのだから遍在する。しかし前述のことから明らか
 なとおり、このどちらも神ではない。ゆえに遍在することは神に固有でない。

\\


2.~{\scshape Praeterea}, numerus est in numeratis. Sed totum universum est
 constitutum in numero, ut patet Sap. XI. Ergo aliquis numerus est, qui
 est in toto universo, et ita ubique.

&

さらに、数は数えられるものの中にある。しかるに『知恵の書』11章に明らか
 なように、宇宙全体は数において整えられた。ゆえに全宇宙の中にあるなん
 らかの数があるのであり、それゆえそれは遍在する。

\\


3.~{\scshape Praeterea}, totum universum est quoddam totum corpus
 perfectum, ut dicitur in I {\itshape Caeli et Mundi}. Sed totum
 universum est ubique, quia extra ipsum nullus locus est. Non ergo solus
 Deus est ubique.

&

さらに『天体論』第1巻で言われるように、全宇宙は一種の完全な物体の全体
 である。しかるに宇宙全体は遍在する。なぜなら、その外には場所がないの
 だから。ゆえに神だけが遍在するわけではない。

\\


4.~{\scshape Praeterea}, si aliquod corpus esset infinitum, nullus locus esset
 extra ipsum. Ergo esset ubique. Et sic, esse ubique non videtur
 proprium Dei.

&

さらに、もしある物体が無限だったならばどんな場所もその外になかったであ
 ろう。ゆえに、〔その物体は〕遍在しただろう。この意味で、遍在することは神に固有ではな
 いと思われる。

\\


5.~{\scshape Praeterea}, anima, ut dicit Augustinus, in VI {\itshape de
 Trin}., est {\itshape tota in toto corpore, et tota in qualibet eius
 parte}. Si ergo non esset in mundo nisi unum solum animal, anima eius
 esset ubique. Et sic, esse ubique non est proprium Dei.

&

さらに、アウグスティヌスが『三位一体論』第6巻で言うように、魂は身体全体
 に全体として存在し、身体のどの部分にも全体として存在する。ゆえに、もし
 も世界に一つの動物しかいなかったならば、それの魂は遍在したであろう
 \footnote{世界にただ一匹の動物しか存在しない
 とき、つまり、それ以外にどんな物体的なものも存在せず、世界全体がその動
 物の身体であったならば、「場所」は
 その動物の身体だけである。そしてその動物の魂が、身体のどの部分にも存在す
 るならば、結果的に、魂はその世界のすべての場所を占めることになる。}。こ
 のように、遍在は神に固有ではない。

\\


6.~{\scshape Praeterea}, ut Augustinus dicit in epistola ad Volusianum,
 {\itshape anima ubi videt, ibi sentit; et ubi sentit, ibi vivit; et ubi
 vivit, ibi est}. Sed anima videt quasi ubique, quia successive videt
 etiam totum caelum. Ergo anima est ubique.

&

さらに、アウグスティヌスがウォルシアヌスへの書簡で述べるように、魂は見る
 ところで感じ、感じるところで生き、生きるところに存在する。しかるに魂は、
 いたるところに存在するかのように見る。なぜなら、天の全体でさえ、継起的
 に見るからである。ゆえに魂は遍在する。

\\

{\scshape Sed contra est} quod Ambrosius dicit, in libro {\itshape de
 Spiritu Sancto}, {\itshape Quis audeat creaturam dicere spiritum
 sanctum, qui in omnibus et ubique et semper est; quod utique
 divinitatis est proprium?}

&

しかし反対に、アンブロシウスは『聖なる霊について』という書物の中で、次の
 ように述べている。「万物の中にいたるところに常に存在する聖霊が被造
 物であるとあえてだれが言うだろうか。このことは実に神性に固有であ
 る」。

\\

{\scshape Respondeo dicendum} quod esse ubique primo et per se, est proprium
 Dei. Dico autem esse ubique {\itshape primo}, quod secundum se totum est
 ubique. Si quid enim esset ubique, secundum diversas partes in diversis
 locis existens, non esset primo ubique, quia quod convenit alicui
 ratione partis suae, non convenit ei primo; sicut si homo est albus
 dente, albedo non convenit primo homini, sed denti. 


&

答えて言わなければならない。第一義的で自体的な遍在は神に固有である。第
 一義的な遍在と私が言うのは、自己全体において遍在することである。つま
 り、もし何かが自己の様々な部分においてさまざまな場所に存在すること
 で遍在するならば、それは第一義的な遍在ではない。なぜなら、XがYの部分
 を根拠としてYに属するとしても、Xは第一義的にYに属するのではないからであ
 る。たとえば、もし人間が歯において白いならば、白さは第一義的に人
 間に属するのではなく歯に属するように。

\\


Esse autem ubique {\itshape per se} dico id cui non convenit esse ubique
 per accidens, propter aliquam suppositionem factam, quia sic granum
 milii  esset ubique, supposito quod nullum aliud corpus esset. Per se
 igitur convenit esse ubique alicui, quando tale est quod, qualibet
 positione facta, sequitur illud esse ubique. Et hoc proprie convenit
 Deo. Quia quotcumque loca ponantur, etiam si ponerentur infinita
 praeter ista quae sunt, oporteret in omnibus esse Deum, quia nihil
 potest esse nisi per ipsum. Sic igitur esse ubique primo et per se
 convenit Deo, et est proprium eius, quia quotcumque loca ponantur,
 oportet quod in quolibet sit Deus, non secundum partem, sed secundum
 seipsum.

&

また、自体的に遍在すると私が言うのは、遍在することが、付帯的にではなく、
 つまり、なんらかの作られた前提のもとにではなく属するという意味である。
 なぜなら、もしそれでいいなら(=作られた前提のもとでもいいならば)、一
 粒の穀物の他にどんな物体も存在しないと仮定したならば、その一粒が遍在
 しただろうから 。ゆえにあるものに自体的に遍在が適合するのは、どのよう
 な前提が置かれてもそれが遍在することが帰結するような場合である。そし
 てこのことは神に固有である。なぜなら、現にある場所以外に無限の場所
 が設定されたとしても、どんなにたくさんの場所が設定されても、そのすべて
 の場所に神が存在しなければならないからである。なぜなら、何ものも神に
 よらなければ存在することができないのだから。ゆえに、このようにして、第
 一義的に自体的に遍在は神に適合し、かつ、それは神に固有である。なぜ
 なら、どれだけたくさんの場所が設定されても、そのどの場所にも部分にお
 いてではなく神自身において神が存在しなければならないからである。

\\

{\scshape Ad primum ergo dicendum} quod universale et materia prima sunt
 quidem ubique, sed non secundum idem esse.

&

第一に対しては、それゆえ、次のように言わなければならない。普遍と第一質料
 は確かに遍在するが、同一の存在において遍在するのではない。

\\


{\scshape Ad secundum dicendum} quod numerus, cum sit accidens, non est per se
 sed per accidens, in loco. Nec est totus in quolibet numeratorum, sed
 secundum partem. Et sic non sequitur quod sit primo et per se ubique.

&

第二に対しては、次のように言われるべきである。数は付帯性なので、自体的
 にではなく付帯的に場所にある。また数えられるもののどこにも全体として
 在るのではなく、部分において在る。こうして第一義的に自体的に遍在す
 るわけではない。

\\


{\scshape Ad tertium dicendum} quod totum corpus universi est ubique, sed non
 primo, quia non totum est in quolibet loco, sed secundum suas
 partes. Nec iterum per se, quia si ponerentur aliqua alia loca, non
 esset in eis.

&

第三に対しては、次のように言われるべきである。宇宙の全物体は遍在するが、
 しかしどの場所にも全体があるわけではないので、第一義的に遍在するので
 はない。また自体的にでもない。なぜなら、もし別の場所が(宇宙の外に)
 設定されていたならば、そこには存在しなかっただろうから。

\\

{\scshape Ad quartum dicendum} quod, si esset corpus infinitum, esset ubique;
 sed secundum suas partes.

&

第四に対しては、次のように言われるべきである。もし無限の物体が存在したな
 らばそれは遍在しただろう。しかしそれは部分においてに過ぎない。

\\



{\scshape Ad quintum dicendum} quod, si esset unum solum animal, anima eius
 esset ubique primo quidem, sed per accidens.

&

第五に対しては、次のように言われるべきである。ただひとつ動物だけが存在し
 たならば、その魂はたしかに第一義的に遍在しただろう。しかし(自体的に
 ではなく)付帯的にである。

\\


{\scshape Ad sextum dicendum quod}, cum dicitur anima alicubi videre, potest
 intelligi dupliciter. Uno modo, secundum quod hoc adverbium alicubi
 determinat actum videndi ex parte obiecti. Et sic verum est quod, dum
 caelum videt, in caelo videt, et eadem ratione in caelo sentit. Non
 tamen sequitur quod in caelo vivat vel sit, quia vivere et esse non
 important actum transeuntem in exterius obiectum. 

&


第六に対しては、次のように言われるべきである。魂がどこかで見る、というの
 は二通りに理解される。一つは、「どこかで」というこの副詞が、対象の側か
 ら見る働きを限定するようにである。この意味では、天を見るとき、天におい
 て見るというのは真であり、同じ理由で、天において感覚するというのも真で
 ある。しかし、このことから、天において生きたり天において存在するという
 ことは帰結しない。なぜなら、生きることと存在することは、外の対象へ超え
 出ていく働きを意味しないからである。

\\


Alio modo potest intelligi secundum quod adverbium determinat actum
 videntis, secundum quod exit a vidente. Et sic verum est quod anima ubi
 sentit et videt, ibi est et vivit, secundum istum modum loquendi. Et
 ita non sequitur quod sit ubique.

&

もう一つの意味では、この副詞が、見る人から出ていく限りにおける見る人の働きを限定するように読める。この意味では、魂が感じたり見たりするところで存在し生きるということは、そういう言い方によるかぎり、真である。しかしこの意味では、〔見る人が〕遍在することは帰結しない。


\end{longtable}



\end{document}