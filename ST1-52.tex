\documentclass[10pt]{jsarticle} % use larger type; default would be 10pt
%\usepackage[utf8]{inputenc} % set input encoding (not needed with XeLaTeX)
%\usepackage[round,comma,authoryear]{natbib}
%\usepackage{nruby}
\usepackage{okumacro}
\usepackage{longtable}
%\usepqckage{tablefootnote}
\usepackage[polutonikogreek,english,japanese]{babel}
%\usepackage{amsmath}
\usepackage{latexsym}
\usepackage{color}

%----- header -------
\usepackage{fancyhdr}
\lhead{{\it Summa Theologiae} I, q.~52}
%--------------------

\bibliographystyle{jplain}

\title{{\bf PRIMA PARS}\\{\HUGE Summae Theologiae}\\Sancti Thomae
Aquinatis\\{\sffamily QUEAESTIO QUINQUAGESIMASECUNDA}\\DE COMPARATIONE ANGELORUM AD LOCA}
\author{Japanese translation\\by Yoshinori {\sc Ueeda}}
\date{Last modified \today}


%%%% コピペ用
%\rhead{a.~}
%\begin{center}
% {\Large {\bf }}\\
% {\large }\\
% {\footnotesize }\\
% {\Large \\}
%\end{center}
%
%\begin{longtable}{p{21em}p{21em}}
%
%&
%
%
%\\
%\end{longtable}
%\newpage



\begin{document}
\maketitle
\pagestyle{fancy}

\begin{center}
{\Large 第五十二問\\天使の場所への関係について}
\end{center}

\begin{longtable}{p{21em}p{21em}}

Deinde quaeritur de loco Angeli. Et circa hoc quaeruntur tria. 
\begin{enumerate}
 \item utrum Angelus sit in loco.
 \item utrum possit esse in pluribus locis simul.
 \item utrum plures Angeli possint esse in eodem loco.
\end{enumerate}

&

次に、天使の場所について問われる。これにかんして、三つのことが問われる。
\begin{enumerate}
 \item 天使は場所に存在するか。
 \item 天使は複数の場所に同時に存在しうるか。
 \item 複数の天使が同一の場所に存在しうるか。
\end{enumerate}


\end{longtable}

\newpage


\rhead{a.~1}
\begin{center}
 {\Large {\bf ARTICULUS PRIMUS}}\\
 {\large UTRUM ANGELUS SIT IN LOCO}\\
 {\footnotesize I {\itshape Sent.}, d.~37, q.~3, a.~1; II, d.~4, q.~1, a.~3; {\itshape De Pot.}, q.~3, a.~19, ad 2; {\itshape Quodl}.~I, q.~3, a.~1; Opusc.~XV, {\itshape de Angelis}, c.~18.}\\
 {\Large 第一項\\天使は場所に存在するか}
\end{center}

\begin{longtable}{p{21em}p{21em}}

{\huge A}{\scshape d primum sic proceditur}. Videtur quod Angelus non
 sit in loco. Dicit enim Boetius, in libro {\itshape de Hebd.}:
 {\itshape Communis animi conceptio apud sapientes est, incorporalia in
 loco non esse}. Et Aristoteles, in IV {\itshape Physic}., dicit quod
 {\itshape non omne quod est, est in loco, sed mobile corpus}. Sed
 Angelus non est corpus, ut supra ostensum est. Ergo Angelus non est in
 loco.

&

第一項の問題へは以下のように議論が進められる。天使は場所に存在しないと思
 われる。理由は以下の通り。ボエティウスは、『デ・ヘブドマディブス』とい
 う書物の中で、「非物体的なものが場所に存在しないことは、知恵ある人々の
 もとで、魂の共通の理解である」と述べている。アリストテレスもまた『自
 然学』4巻で「存在するものがすべて場所に存在するのではなく、可動的な物体
 だけが場所に存在する」と述べている。ところが、前に示されたとおり、天使
 は物体ではない。ゆえに、天使は場所に存在しない

\\


2 {\scshape Praeterea}, locus est quantitas positionem habens. Omne ergo
 quod est in loco, habet aliquem situm. Sed habere situm non potest
 convenire Angelo, cum substantia sua sit immunis a quantitate, cuius
 propria differentia est positionem habere. Ergo Angelus non est in
 loco.

&

さらに、場所とは、位置を持つ量である。ゆえに、すべて場所に存在するものは、
 なんらかの位置を持つ。ところが、位置を持つことは、天使に適合しえない。
 なぜなら、天使の実体は量を免れているからである。位置を持つことが、量の
 固有の差異なのだから。ゆえに、天使は場所に存在しない

\\


3 {\scshape Praeterea}, esse in loco est mensurari loco et contineri a
 loco, ut patet per philosophum in IV {\itshape Physic}. Sed Angelus non potest
 mensurari neque contineri a loco, quia continens est formalius
 contento, sicut aer aqua, ut dicitur in IV Physic. Ergo Angelus non est
 in loco.

&

さらに、『自然学』4巻の哲学者によって明らかなとおり、場所に存在すること
 は、場所に測られ、場所によって含まれることである。ところが、天使が場所
 に測られたり含まれたりすることは不可能である。なぜなら、『自然学』4巻で
 言われるとおり、空気が水にたいしてそうであるように、含むものは、含
 まれるものよりも形相的だから。ゆえに、天使は場所に存在しない。


\\


{\scshape Sed contra est} quod in collecta dicitur, {\itshape Angeli tui sancti,
 habitantes in ea, nos in pace custodiant}.

&

しかし反対に、『祈祷集』の中で「あなた方聖なる天使たちは、そこに住まい、私たち
 を平和の内に守る」と言われている。

\\


{\scshape Respondeo dicendum} quod Angelo convenit esse in loco,
 aequivoce tamen dicitur Angelus esse in loco, et corpus. Corpus enim
 est in loco per hoc, quod applicatur loco secundum contactum dimensivae
 quantitatis. Quae quidem in Angelis non est; sed est in eis quantitas
 virtualis. Per applicationem igitur virtutis angelicae ad aliquem locum
 qualitercumque, dicitur Angelus esse in loco corporeo. 

&

解答する。以下のように言われるべきである。
天使には、場所に存在することが適合する。しかし、天使と物体とが「場所に存
 在する」と言われるのは、同名異義的にである。
なぜなら、物体は、次元量の接触にしたがって、場所に当てはめられることによっ
 て、場所に存在するが、このようなことは、天使においてはない。天使におい
 ては、むしろ、力の量が存在する
それゆえ、天使の力を、なんらかのかたちで、ある場所に適用することによって、
 天使が、物体的場所に存在すると言われる。


\\

Et secundum hoc
 patet quod non oportet dicere quod Angelus commensuretur loco; vel quod
 habeat situm in continuo. Haec enim conveniunt corpori locato, prout
 est quantum quantitate dimensiva. Similiter etiam non oportet propter
 hoc, quod contineatur a loco. Nam substantia incorporea sua virtute
 contingens rem corpoream, continet ipsam, et non continetur ab ea,
 anima enim est in corpore ut continens, et non ut contenta. Et
 similiter Angelus dicitur esse in loco corporeo, non ut contentum, sed
 ut continens aliquo modo.

&

また、このことによって、次のことも明らかである。天使が場所によって測られ
 るとか、天使が連続体において位置を占めるとか言われる必要はない。なぜな
 ら、これらのことは、場所に置かれた物体に、それが次元的量によって量をも
 つものであるかぎりで適合することだからである。じっさい、非物体的実体は、
 みずからの力において、物体的事物を含むので、事物を含み、事物に含まれる
 ことがない。たとえば魂は、身体に含まれるものとしてではなく、身体を含む
 ものとして、身体の中にある。同様に、天使が物体的場所に存在すると言われ
 るのは、含まれたものとしてではなく、なんらかのかたちで、含むものとして
 である。

\\


Et per hoc patet responsio ad obiecta.


&

このことによって、反論にたいする解答は明らかである。


\end{longtable}
\newpage

\rhead{a.~2}
\begin{center}
 {\Large {\bf ARTICULUS SECUNDUS}}\\
 {\large UTRUM ANGELUS POSSIT ESSE IN PLURIBUS LOCIS SIMUL}\\
 {\footnotesize Supra, q.~8, a.~2, ad 2; infra, q.~103, a.~1; I
 {\itshape Sent.}, d.~37, q.~3, a.~2; IV, d.~10, q.~1, a.~3, qu$^a$ 2;
 Qu.~{\itshape de Anima}, a.~10, ad 18.}\\
 {\Large 第二項\\天使は複数の場所に同時に存在しうるか}
\end{center}

\begin{longtable}{p{21em}p{21em}}




{\huge A}{\scshape d secundum sic proceditur}. Videtur quod
Angelus possit esse in pluribus locis simul. Angelus enim non est
minoris virtutis quam anima. Sed anima est simul in pluribus locis, quia
est tota in qualibet parte corporis, ut Augustinus dicit. Ergo Angelus
potest esse in pluribus locis simul.

&


第二項の問題へは以下のように議論が進められる。
天使は複数の場所に同時に存在しうると思われる。理由は以下の通り。
天使の力が魂の力よりも小さいことはない。ところが、アウグスティヌスが言う
 ように、身体のどの場所にも、魂全体が存在するのだから、魂は複数の場所に
 同時に存在する。ゆえに、天使は複数の場所に同時に存在しうる。

\\


{\scshape 2 Praeterea}, Angelus est in corpore assumpto;
et cum assumat corpus continuum, videtur quod sit in qualibet eius
parte. Sed secundum partes eius considerantur diversa loca. Ergo Angelus
est simul in pluribus locis.

&
さらに、天使は、取られた身体の中に存在するが、天使が連続体である身体を取
っているとき、天使はその身体のどの部分にも存在していると思われる。ところが、そ
 の部分に応じて異なる場所が考えられる。ゆえに、天使は複数の場所に同時に
 存在する。

\\


{\scshape 3 Praeterea}, Damascenus dicit quod {\itshape ubi
angelus operatur, ibi est}. Sed aliquando operatur simul in pluribus
locis, ut patet de Angelo subvertente Sodomam, {\itshape Gen}.~{\scshape xix}. Ergo Angelus
potest esse in pluribus locis simul.

&

さらに、ダマスケヌスは「天使が働くところ、そこに天使は存在する」と言って
 いる。ところが、ソドムを滅ぼす天使について明らかなように、天使は時とし
 て複数の場所で同時に働く。ゆえに、天使は、複数の場所に同時に存在しうる。
\\


{\scshape Sed contra est} quod Damascenus dicit, quod
Angeli, {\itshape dum sunt in caelo, non sunt in terra}.

&
しかし反対に、ダマスケヌスは、「天使は、天にいるあいだ、地にはいない」と
 述べている。

\\


{\scshape Respondeo dicendum} quod Angelus est virtutis
et essentiae finitae. Divina autem virtus et essentia infinita est, et
est universalis causa omnium, et ideo sua virtute omnia contingit, et
non solum in pluribus locis est, sed ubique. Virtus autem Angeli, quia
finita est, non se extendit ad omnia, sed ad aliquid unum
determinatum. Oportet enim quidquid comparatur ad unam virtutem, ut unum
aliquid comparari ad ipsam. Sicut igitur universum ens comparatur ut
unum aliquid ad universalem Dei virtutem, ita et aliquod particulare ens
comparatur ut aliquid unum ad Angeli virtutem. Unde cum Angelus sit in
loco per applicationem virtutis suae ad locum, sequitur quod non sit
ubique, nec in pluribus locis, sed in uno loco tantum. 


&
解答する。以下のように言われるべきである。
天使は、有限な力と本質をもつ。これに対して、神の力と本質は無限であり、万
 物の普遍的な原因であり、それゆえ、みずからの力によって万物に触れ、複数
 の場所のみならず、いたるところに存在する。
対するに、天使の力有限であり、万物にみずからを及ぼすことがなく、ある限定
 された一つのものに及ぼす。実に、一つの力に関係づけられるものは何でも、
 なにか一つのものとして、その力に関係づけられるのでなければならない。ゆ
 えに、ちょうど、普遍的な有が、なにか一つのものとして、神の普遍的な力に
 関係づけられるように、ある個別的な有もまた、ある一つのものとして、天使
 の力に関係づけられる。したがって、天使は、みずからの力を場所に適用する
 ことによって、場所に存在するのだから、天使は遍在せず、複数の場所にも存
 在せず、一つの場所だけに存在することが帰結する。

\\


Circa hoc tamen
aliqui decepti sunt. Quidam enim, imaginationem transcendere non
valentes, cogitaverunt indivisibilitatem Angeli ad modum
indivisibilitatis puncti, et inde crediderunt quod Angelus non posset
esse nisi in loco punctali. Sed manifeste decepti sunt. Nam punctum est
indivisibile habens situm, sed Angelus est indivisibile extra genus
quantitatis et situs existens. Unde non est necesse quod determinetur ei
unus locus indivisibilis secundum situm; sed vel divisibilis vel
indivisibilis, vel maior vel minor, secundum quod voluntarie applicat
suam virtutem ad corpus maius vel minus. Et sic totum corpus cui per
suam virtutem applicatur, correspondet ei ut unus locus. 


&
しかし、この問題をめぐっては、ある人々が間違いを犯した。
すなわち、ある人々は、想像を超えることができずに、天使の不可分性を点の不
 可分性に即して考え、そのことから、天使は、点としての場所にしか存在しえ
 ないと信じた。しかし、彼らは明らかに誤っていた。なぜなら、点は、不可分
 だが位置を持つ。しかし、天使は、量や位置をもって存在する類の外にある、
 不可分のものである。したがって、天使に、位置において不可分な一つの場所
 が限定される必要はなく、むしろ、分割可能であれ不可分であれ、大きくても
 小さくても、天使が意志的にみずからの力を大きいあるいは小さい物体に適用
 することに応じて、[場所が限定されればよい]。このようにして、天使の力
 によって適用される物体全体が、天使に、一つの場所として対応する。

\\


Nec tamen oportet quod si aliquis Angelus movet caelum, quod sit
ubique. Primo quidem, quia non applicatur virtus eius nisi ad id quod
primo ab ipso movetur, una autem pars caeli est in qua primo est motus,
scilicet pars orientis, unde etiam philosophus, in VIII {\itshape
Physic}., {\itshape virtutem motoris caelorum attribuit parti
orientis}. Secundo, quia non ponitur a philosophis quod una substantia
separata moveat omnes orbes immediate. Unde non oportet quod sit ubique.


&

しかしまた、「もし、ある天使が天を動かすならば、天使はいたるところに存在す
 る」ということにもならない。なぜなら、天使の力は、天使によって第一に動
 かされるところにしか適用されないからである。しかるに、天には、そこにお
 いて第一に運動が存在する部分があり、それは、東の部分である。このことか
 ら、哲学者もまた、『自然学』8巻で「天の動者の力を、東の部分に帰する」と
 述べている。第二に、一つの離在実体が、すべての天球を直接に動かすという
 ことは、哲学者たちによって認められていない。したがって、天使が遍在する
 必要はない。



\\

Sic igitur patet quod
diversimode esse in loco convenit corpori, et Angelo, et Deo. Nam corpus
est in loco circumscriptive, quia commensuratur loco. Angelus autem non
circumscriptive, cum non commensuretur loco, sed definitive, quia ita
est in uno loco, quod non in alio. Deus autem neque circumscriptive
neque definitive, quia est ubique.

&

それゆえ、このようにして、物体と天使と神とには、異なるかたちで、「場所に
 存在する」ということが適合することが明らかである。すなわち、物体は、場
 所によって測られるので、制限的に場所に存在する。他方、天使は、場所に測
 られないから、制限的にではないが、限定的に存在する。すなわち、他の場所
 に存在しないというかたちで、或る場所に存在する。神は、制限的にでも限定
 的にでも、場所には存在しない。なぜなら、神は遍在するからである。

\\


Et per hoc patet de facili responsio ad obiecta, quia totum illud cui
immediate applicatur virtus Angeli, reputatur ut unus locus eius, licet
sit continuum.

&

これによって、異論にたいする解答は容易に明らかである。なぜなら、天使の力
 が直接に適用される場所全体が、それが連続体であったとしても、天使の一つ
 の場所と見なされるからである。


\end{longtable}


\newpage

\rhead{a.~3}
\begin{center}
 {\Large {\bf ARTICULUS TERTIUS}}\\
 {\large UTRUM PLURES ANGELI POSSINT SIMUL ESSE IN EODEM LOCO}\\
 {\footnotesize I {\itshape Sent.}, d.~37, q.~3, a.~3; {\itshape De
 Pot.}, q.~3, a.~7, ad 11; a.~19, ad 1; {\itshape Quodl}.~I, q.~3, a.~1,
 ad 2.}\\
 {\Large 第三項\\複数の天使が同時に同じ場所に存在しうるか}
\end{center}

\begin{longtable}{p{21em}p{21em}}

{\huge A}{\scshape d tertium sic proceditur}. Videtur quod
 plures Angeli possint simul esse in eodem loco. Plura enim corpora non
 possunt esse simul in eodem loco, quia replent locum. Sed Angeli non
 replent locum, quia solum corpus replet locum, ut non sit vacuum, ut
 patet per philosophum, in IV {\itshape Physic}. Ergo plures Angeli possunt esse in
 uno loco.

&

第三の問題へ、議論は以下のように進められる。複数の天使が同時に同じ場所に
 存在することができると思われる。理由は以下の通り。複数の物体が、同時に
 同じ場所に存在することはできない。それは、物体が場所を満たすからである。
 しかし、『自然学』4巻の哲学者によって明らかなとおり、物体だけが、場所を
 満たして、そこを空虚でなくするのだから、天使たちは、場所を満たさない。
 ゆえに、複数の天使が一つの場所に存在することができる。

\\



{\scshape 2 Praeterea}, plus differt Angelus et corpus
 quam duo Angeli. Sed Angelus et corpus sunt simul in eodem loco, quia
 nullus locus est qui non sit plenus sensibili corpore, ut probatur in
 IV {\itshape Physic}. Ergo multo magis duo Angeli possunt esse in eodem loco.

&
さらに、天使同士よりも天使と物体の方がより異なっている。ところが、『自然
 学』4巻で証明されているとおり、可感的物体によって満たされていない場所
 は存在しないのだから、天使と物体は同時に同じ場所に存在しうる。ゆえに、
 ましてや、二人の天使も同じ場所に存在しうる。

\\



{\scshape 3 Praeterea}, anima est in qualibet parte
 corporis, secundum Augustinum. Sed Daemones, licet non illabantur
 mentibus, illabuntur tamen interdum corporibus, et sic anima et Daemon
 sunt simul in eodem loco. Ergo, eadem ratione, quaecumque aliae
 spirituales substantiae.

&

さらに、アウグスティヌスによれば、魂は、身体のどの部分にも存在する。とこ
 ろが、悪魔は、精神の中に侵入することはないが、時として、身体に侵入する。
 このような場合、魂と悪霊が、同一の場所に存在する。ゆえに、同じ理由で、
 他のどんな霊的実体もまた[同一の場所に存在しうる]。

\\



{\scshape Sed contra}, duae animae non sunt in eodem
 corpore. Ergo, pari ratione, neque duo Angeli in eodem loco.

&

しかし反対に、二つの魂が一つの身体の中に存在することはない。ゆえに、同じ
 理由で、二人の天使が同一の場所に存在することはない。

\\



{\scshape Respondeo dicendum} quod duo Angeli non sunt
 simul in eodem loco. Et ratio huius est, quia impossibile est quod duae
 causae completae sint immediatae unius et eiusdem rei. Quod patet in
 omni genere causarum, una enim est forma proxima unius rei, et unum est
 proximum movens, licet possint esse plures motores remoti. Nec habet
 instantiam de pluribus trahentibus navem, quia nullus eorum est
 perfectus motor, cum virtus uniuscuiusque sit insufficiens ad movendum;
 sed omnes simul sunt in loco unius motoris, inquantum omnes virtutes
 eorum aggregantur ad unum motum faciendum. Unde cum Angelus dicatur
 esse in loco per hoc quod virtus eius immediate contingit locum per
 modum continentis perfecti, ut dictum est, non potest esse nisi unus
 Angelus in uno loco.

&
解答する。以下のように言われるべきである。
二人の天使が同時に同じ場所に存在することはない。そしてこの理由は、二つの
 完全な原因が、一つで同一の事物の、直接的な原因でありえない、ということ
 による。このことは、原因のすべての類において明らかである。たとえば、一
 つの事物の最近接の形相は一つであり、動因も、隔たった動因は複数ありうるとし
 ても、最近接の動因は一つである。複数の人が一艘の船を引く場合も、この
 反例にはならない。なぜなら、その場合には、引いている人たちの内のどの人
 も、完全な動者でないからである。なぜなら、一人一人の力は動かすために
 不十分だが、彼らのすべての力が、一つの運動を行うために集められれている
 という点で、すべての人が同時に、一つの動者の場所に存在するからである。
したがって、天使が、すでに述べられたように、天使の力が、完全な含むものの
 あり方で、場所に直接に触れることによって、天使は場所に存在すると言われ
 るのだから、一つの場所には一人の天使しか存在しえない。

\\



{\scshape Ad primum ergo dicendum} quod plures Angelos
 esse in uno loco non impeditur propter impletionem loci, sed propter
 aliam causam, ut dictum est.

&
第一異論に対しては、それゆえ、次のように言われるべきである。
複数の天使が一つの場所に存在することが妨げられるのは、場所を満たすことに
 よるのではなく、すでに述べられたとおり、別の原因のためにである。

\\



{\scshape Ad secundum dicendum} quod Angelus et corpus
 non eodem modo sunt in loco, unde ratio non sequitur.

&

第二異論に対しては、次のように言われるべきである。
天使と物体は、同じかたちで場所に存在するのではない。ゆえに、この推論のよ
 うにはならない。

\\



{\scshape Ad tertium dicendum} quod nec etiam Daemon et
anima comparantur ad corpus secundum eandem habitudinem causae; cum
anima sit forma, non autem Daemon. Unde ratio non sequitur.


&

第三異論に対しては、次のように言われるべきである。
悪魔と魂は、同一の原因関係によって身体に関係するのではない。魂は[身体の]
 形相だが、悪魔はそうではないからである。したがって、そうはならない。


\end{longtable}
\newpage

\end{document}