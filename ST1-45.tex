\documentclass[10pt]{jsarticle} % use larger type; default would be 10pt
%\usepackage[utf8]{inputenc} % set input encoding (not needed with XeLaTeX)
%\usepackage[round,comma,authoryear]{natbib}
%\usepackage{nruby}
\usepackage{okumacro}
\usepackage{longtable}
%\usepqckage{tablefootnote}
\usepackage[polutonikogreek,english,japanese]{babel}
%\usepackage{amsmath}
\usepackage{latexsym}
\usepackage{color}

%----- header -------
\usepackage{fancyhdr}
\pagestyle{fancy}
\lhead{{\it Summa Theologiae} I q.~45}
%--------------------


\bibliographystyle{jplain}


\title{{\bf PRIMA PARS}\\{\HUGE Summae Theologiae}\\Sancti Thomae
Aquinatis\\{\sffamily QUEAESTIO QUADRAGESIMAQUINTA}\\DE MODO
EMANATIONIS RERUM A PRIMO PRINCIPIO}
\author{Japanese translation\\by Yoshinori {\sc Ueeda}}
\date{Last modified \today}

%%%% コピペ用
%\begin{center}
% {\Large {\bf }}\\
% {\large }\\
% {\footnotesize }\\
% {\Large \\}
%\end{center}
%
%\begin{longtable}{p{21em}p{21em}}
%
%&
%
%\\
%\end{longtable}
%\newpage



\begin{document}

\maketitle

\begin{center}
{\Large 第四十五問\\第一根源からの諸事物の流出のしかたについて}
\end{center}

\begin{longtable}{p{21em}p{21em}}

Deinde quaeritur de modo emanationis rerum a primo principio, qui
 dicitur creatio. De qua quaeruntur octo.
\begin{enumerate}
 \item quid sit creatio.
 \item utrum Deus possit aliquid creare.
 \item utrum creatio sit aliquod ens in rerum natura.
 \item cui competit creari.
 \item utrum solius Dei sit creare.
 \item utrum commune sit toti Trinitati, aut proprium alicuius personae.
 \item utrum vestigium aliquod Trinitatis sit in rebus creatis.
 \item utrum opus creationis admisceatur in operibus naturae et voluntatis.
\end{enumerate}


&

次に、諸事物が第一根源から流出するしかたについて問われる。この流出は、創
 造といわれる。
創造について、八つのことが問われる。
\begin{enumerate}
 \item 創造とはなにか。
 \item 神は何かを創造することができるか。
 \item 創造は、諸事物の本性における何かか。
 \item 創造されることは何に適合するか。
 \item 創造することができるのは神だけか。
 \item (創造することは)三位一体全体に共通か、それとも、なんらかのペル
       ソナに固有か。
 \item 三位一体のなんらかの痕跡が、創造された諸事物においてあるか。
 \item 創造の業は、本性と意志の業の中に混合されているか。
\end{enumerate}


\end{longtable}

\newpage
\rhead{a.~1}
\begin{center}
{\Large {\bf ARTICULUS PRIMUS}}\\
 {\large UTRUM CREARE SIT EX NIHILO ALIQUID FACERE}\\
 {\footnotesize II {\itshape Sent.}, d.~1, q.~1, a.~2.}\\
 {\Large 第一項\\創造は無から何かを作ることか}
\end{center}

\begin{longtable}{p{21em}p{21em}}
{\huge A}{\scshape d primum sic proceditur}. Videtur quod creare non sit ex nihilo aliquid
 facere. Dicit enim Augustinus, {\it Contra Adversarium Legis et
 Prophetarum} : {\it facere est quod omnino non erat, creare vero est ex
 eo quod iam erat educendo aliquid constituere}.

&

第一に対しては次のように進められる。
創造することは、なにかを無から作ることではないと思われる。
なぜなら、アウグスティヌスは『法と預言に反対する者を駁す』で「作るとは、
 まったく無いものを作ることだが、創造するとは、すでに在ったものから、な
 にかを引き出すことによって構成することである」と述べている。

\\

2.~{\scshape Praeterea}, nobilitas actionis et motus ex
 terminis consideratur. Nobilior igitur est actio quae ex bono in bonum
 est, et ex ente in ens, quam quae est ex nihilo in aliquid. Sed creatio
 videtur esse nobilissima actio, et prima inter omnes actiones. Ergo non
 est ex nihilo in aliquid, sed magis ex ente in ens.
&

さらに、働きと運動の高貴さは両端から考察される。
ゆえに、善から出て善へ向かう働きや、有から出て有へ向かう働きの方が、無か
 ら出てなにかへ向かう働きよりも高貴である。
しかるに、創造は、もっとも高貴な働きであり、すべての働きの中で第一のもの
 だと思われる。ゆえに、無からなにかへではなく、有から有へである。

\\

3.~{\scshape Praeterea}, haec praepositio ex importat habitudinem alicuius causae, et
 maxime materialis; sicut cum dicimus quod statua fit ex aere.
Sed nihil
 non potest esse materia entis, nec aliquo modo causa eius. Ergo creare
 non est ex nihilo aliquid facere.

&

さらに、「から(ex)」というこの格助詞(前置詞)は、なんらかの原因の関係、
 とくに、質料因の関係を含意する。たとえば、私たちが、像が銅から作られる、
 と言うように。しかるに、なにものも、有の質料ではありえないし、なんらか
 のかたちで有の原因であることも不可能である。ゆえに、創造することは、な
 にかを無から作ることではない。

\\

{\scshape Sed contra est} quod super illud {\it Gen}. {\sc i} : {\it In principio
 creavit Deus caelum} etc., dicit Glossa quod {\it creare} est {\it aliquid ex nihilo facere}.


&

しかし反対に、『創世記』第1章「始めに神は天を創造し云々」\footnote
{「初めに、神は天地を創造された。」(1:1)
}について、『注釈』は
 「創造する」とは、「なにかを無から作ることである」と述べている。

\\

{\scshape Respondeo dicendum} quod, sicut supra dictum est, non solum oportet
 considerare emanationem alicuius entis particularis ab aliquo
 particulari agente, sed etiam emanationem totius entis a causa
 universali, quae est Deus, et hanc quidem emanationem designamus nomine
 creationis. 

&

答えて言わなければならない。
上で述べられたとおり、\footnote{{\itshape ST} I, q.~44, a.~2}ある個別的な有の、ある個別的な作用者からの流出だけ
 でなく、有全体の、普遍的な原因、すなわち神からの流出も、考察しなければ
 ならないが、この後者の流出を、私たちは創造という名で指し示す。

\\

Quod autem procedit secundum emanationem particularem, non
 praesupponitur emanationi, sicut, si generatur homo, non fuit prius
 homo, sed homo fit ex non homine, et album ex non albo. Unde, si
 consideretur emanatio totius entis universalis a primo principio,
 impossibile est quod aliquod ens praesupponatur huic emanationi. Idem
 autem est nihil quod nullum ens. Sicut igitur generatio hominis est ex
 non ente quod est non homo, ita creatio, quae est emanatio totius esse,
 est ex non ente quod est nihil.

&

ところで、個別的な流出において生まれるものは、その流出に前提されない。た
 とえば、もし人間が生み出されるならば、それに先だって人間は存在しなかっ
 たのであり、むしろ、人間が人間でないものから、白くないものから白いもの
 が生じる。
したがって、もし、第一根源からの、有全体の普遍的な流出が考察されるならば、
 なんらかの有が、この流出に前提されることは不可能である。ところで、「無」
 とは、「なにも有がない」と同じである。ゆえに、ちょうど人間の生成が、有
 でないもの、すなわち、人間でないものからであるように、全存在の流出であ
 る創造は、有でないもの、すなわち無からである。
 

\\

{\scshape Ad primum ergo dicendum} quod Augustinus
 aequivoce utitur nomine creationis, secundum quod creari dicuntur ea
 quae in melius reformantur, ut cum dicitur aliquis creari in
 episcopum. Sic autem non loquimur hic de creatione, sed sicut dictum
 est.

&

第一に対しては、それゆえ、次のように言われるべきである。
アウグスティヌスは、両義的にcreatioという名を使っている。それによれば、
 より良い方へ改良されるものが、「creareされる」と言われる。たとえば、「だ
 れかが、司教にcreareされる」と言われるときのように。しかし、私たちはこ
 こで、このような意味でcreatioについて語っていないのであって
それはすでに述べられたとおりである。


\\

{\scshape Ad secundum dicendum} quod mutationes
 accipiunt speciem et dignitatem non a termino a quo, sed a termino ad
 quem. Tanto ergo perfectior et prior est aliqua mutatio, quanto
 terminus ad quem illius mutationis est nobilior et prior; licet
 terminus a quo, qui opponitur termino ad quem, sit imperfectior. Sicut
 generatio simpliciter est nobilior et prior quam alteratio, propter hoc
 quod forma substantialis est nobilior quam forma accidentalis, tamen
 privatio substantialis formae, quae est terminus a quo in generatione,
 est imperfectior quam contrarium, quod est terminus a quo in
 alteratione. Et similiter creatio est perfectior et prior quam
 generatio et alteratio, quia terminus ad quem est tota substantia
 rei. Id autem quod intelligitur ut terminus a quo, est simpliciter non
 ens.

&

第二に対しては、次のように言われるべきである。
変化が種と威厳をもつのは、出発点からではなく、終着点からである。
ゆえに、任意の変化は、その変化の終着点が高貴であり先行するほど、完全で、
 先行する。これは、終着点に対置される出発点が、より不完全であるとしても
 そうである。たとえば、生成が、端的に、性質変化よりも高貴であり先行する
 のは、実体形相が附帯形相よりも高貴だからである。しかし、実体形相の欠如
 は生成における出発点だが、性質変化の出発点である「対立するもの」よりも
 不完全である。同様に、創造が、生成や性質変化よりも完全であり先行するの
 は、その終着点が、事物の全実体だからである。これに対して、その出発点と
 して理解されるのは、端的に、非有である。


\\

{\scshape Ad tertium dicendum} quod, cum dicitur aliquid
 ex nihilo fieri, haec praepositio ex non designat causam materialem,
 sed ordinem tantum; sicut cum dicitur, ex mane fit meridies, idest,
 post mane fit meridies. Sed intelligendum est quod haec praepositio ex
 potest includere negationem importatam in hoc quod dico nihil, vel
 includi ab ea. Si primo modo, tunc ordo remanet affirmatus, et
 ostenditur ordo eius, quod est ad non esse praecedens. Si vero negatio
 includat praepositionem, tunc ordo negatur, et est sensus, fit ex
 nihilo, idest non fit ex aliquo; sicut si dicatur, iste loquitur de
 nihilo, quia non loquitur de aliquo. Et utroque modo verificatur, cum
 dicitur ex nihilo aliquid fieri. Sed primo modo, haec praepositio ex
 importat ordinem, ut dictum est, secundo modo, importat habitudinem
 causae materialis, quae negatur.

&
第三に対しては、次のように言われるべきである。
なにかが無から生じると言われるとき、この「から」 は、質料因を示すのでは
 なく、たんなる秩序(順序)を示す。たとえば、「朝から昼になる」のように。すなわ
 ちこれは、朝のあとに昼になる、ということを示すだけである。
しかし(この話とは別に)、この「から」は、私が「無」と言うものにおいて含
 意されている否定を含みうるし、あるいはまた、否定に含まれうるということは、理解されるべきである。
前者の意味では、秩序が肯定されたままに留まり、非存在に先行するものの秩序
 が明示される。他方、否定が「から」を含む場合は、秩序は否定され、「無か
 ら生じる」とはつまり「なにからも生じない」という意味になる。たとえば、
何についても語っていないので、「その人は無について語っている」と言われる
 ように。「無からなにかが生じる」と言われるとき、この両方の意味で、それ
 は正しいとされる。しかし、第一の意味では、この「から」は、すでに述べられたと
 おり秩序を意味し、第二の意味では、質料因の関係が含意され、それが否定さ
 れる。


\end{longtable}




\newpage
\rhead{a.~2}
\begin{center}
 {\Large {\bf ARTICULUS SECUNDUS}}\\
 {\large UTRUM DEUS POSSIT ALIQUID CREARE}\\
 {\footnotesize II {\itshape Sent.}, d.~1, q.~1, a.~2; II {\itshape
 SCG.}, c.~16; {\itshape De Pot.}, q.~3, a.~1; {\itshape
 Compend.~Theol.}, c.~69; Opusc.~XXXVII, {\itshape de Quatuor Opposit.},
 c.~4, VIII {\itshape Physic.}, lect.~2.}\\
 {\Large 第二項\\神はなにかを創造することができるか}
\end{center}


\begin{longtable}{p{21em}p{21em}}


{\huge A}{\scshape d secundum sic proceditur}. Videtur quod Deus non possit aliquid
 creare. Quia secundum philosophum, I {\it Physic}., antiqui philosophi
 acceperunt ut communem conceptionem animi, {\it ex nihilo nihil fieri}. Sed
 potentia Dei non se extendit ad contraria primorum principiorum; utpote
 quod Deus faciat quod totum non sit maius sua parte, vel quod
 affirmatio et negatio sint simul vera. Ergo Deus non potest aliquid ex
 nihilo facere, vel creare.

&

第二に対しては次のように進められる。
神がなにかを創造することはできないと思われる。
なぜなら、『自然学』第1巻の哲学者によれば、古代の哲学者たちは、魂の共通
 概念として「無から何も生じない」ということを理解していた。
しかるに、神の能力は、第一諸原理の反対までは及ばない。たとえば、神が、全
 体をその部分よりも大きくなくするとか、肯定と否定が同時に真であるように
 するとかといったことである。ゆえに神は、無からなにかを作る、あるいは、
 創造することができない。

\\

2.~{\scshape Praeterea}, si creare est aliquid ex nihilo facere, ergo creari est
 aliquid fieri. Sed omne fieri est mutari. Ergo creatio est mutatio. Sed
 omnis mutatio est ex subiecto aliquo, ut patet per definitionem motus,
 nam motus est {\it actus existentis in potentia}. Ergo est impossibile
 aliquid a Deo ex nihilo fieri.

&
さらに、もし創造することが、無からなにかを作ることであるならば、創造され
 ることは、なにかとして生成することである。しかるに、すべての生成は、変
 化である。ゆえに、創造は変化である。しかるに、運動の定義、すなわち、「可
 能態において在るものの現実態」、によって明らかなとおり、すべて変化は、な
 んらかの基体にもとづいている。ゆえに、なにかが神によって無から生じるこ
 とは不可能である。


\\

3.~{\scshape Praeterea}, quod factum est, necesse est aliquando fieri. Sed non potest
 dici quod illud quod creatur, simul fiat et factum sit, quia in
 permanentibus, quod fit, non est, quod autem factum est, iam est; simul
 ergo aliquid esset et non esset. Ergo, si aliquid fit, fieri eius
 praecedit factum esse. Sed hoc non potest esse, nisi praeexistat
 subiectum in quo sustentetur ipsum fieri. Ergo impossibile est aliquid
 fieri ex nihilo.

&
さらに、作られたものは、いつかあるときに、作られていることが必要である。しかる
 に、創造されるものが、同時に、作られていて、かつ、作られたと言われることはでき
 ない。なぜなら、永続するものどもにおいて、作られているものは、存在せず、
他方、作られたものは、すでに存在するからである。ゆえに、なにかが、存在し、
 かつ存在しないということになったであろう。ゆえに、もしなにかが作られて
 いるならば、それが作られていることが、作られたことに先行する。しかし、
 このことは、作られていること自体が、そこにおいて保持されるところの、基
 体が先に存在していなければ不可能である。ゆえに、なにかが、無から作られ
 ることは不可能である。

\\

4.~{\scshape Praeterea}, infinitam distantiam non est pertransire. Sed infinita
 distantia est inter ens et nihil. Ergo non contingit ex nihilo aliquid
 fieri.

&
さらに、無限の隔たりを越えることはできない。しかるに、有と無のあいだには、
 無限の隔たりがある。ゆえに、無からなにかが生じるということは起こらない。

\\

{\scshape Sed contra est} quod dicitur {\itshape Gen}.~{\scshape i}, {\itshape in principio creavit Deus caelum et
 terram}.

&
しかし反対に、『創世記』1章に「はじめに、神は天と地を創造した」\footnote
{「初めに、神は天地を創造された。」 (1:1)
}と言われ
 ている。
\\

{\scshape Respondeo dicendum} quod non solum non est impossibile a Deo aliquid
 creari, sed necesse est ponere a Deo omnia creata esse, ut ex
 praemissis habetur. Quicumque enim facit aliquid ex aliquo, illud ex
 quo facit praesupponitur actioni eius, et non producitur per ipsam
 actionem, sicut artifex operatur ex rebus naturalibus, ut ex ligno et
 aere, quae per artis actionem non causantur, sed causantur per actionem
 naturae. Sed et ipsa natura causat res naturales quantum ad formam, sed
 praesupponit materiam. Si ergo Deus non ageret nisi ex aliquo
 praesupposito, sequeretur quod illud praesuppositum non esset causatum
 ab ipso. Ostensum est autem supra quod nihil potest esse in entibus
 quod non sit a Deo, qui est causa universalis totius esse. Unde necesse
 est dicere quod Deus ex nihilo res in esse producit.

&

答えて言わなければならない。神によってなにかが創造されることが不可能でな
 いだけでなく、すでに述べられたことから理解されるとおり、神によって万物
 が創造されたとすることが必要である。理由は以下の通りである。
だれであれ、或るものXを何かYから作る人は、そのYを、自分の働きに先立って
 前提する。つまりYは、その働き自体によって生み出されない。たとえば、職人
 が、木や銅といった自然的諸事物にもとづいて働くが、それらは技術の働きが
 原因となってではなく、自然の働きが原因となって生じる。
しかし、この自然もまた、形相にかんしては、自然的諸事物を生み出すが、しか
 し、質料を前提とする。ゆえに、もし、神が何かの前提に基づかなければ働か
 ないのであったならば、その前提されたものは、神が原因によって生じたので
 はないことになっただろう。しかし、存在全体の普遍的原因である神によって
 存在するのでないものは、有の中にありえないことが上で示された。
ゆえに、神は、無から事物を存在へ生み出す、と言うことが必要である。


\\

{\scshape Ad primum ergo dicendum} quod antiqui philosophi, sicut supra dictum est,
 non consideraverunt nisi emanationem effectuum particularium a causis
 particularibus, quas necesse est praesupponere aliquid in sua actione,
 et secundum hoc erat eorum communis opinio, ex nihilo nihil fieri. Sed
 tamen hoc locum non habet in prima emanatione ab universali rerum
 principio.

&

第1に対しては、それゆえ、次のように言われるべきである。
古代の哲学者たちは、上で述べられたように、個別的な原因からの個別的な結果
 の流出しか考察しなかった。この流出には、何かをその働きにおいて前提する
 ことが必要であり、その限りで、「無から何も生じない」という彼らの共通意
 見があった。しかし、このことは、諸事物の普遍的根源からの第一の流出にお
 いては、入り込む余地がない。

\\

{\scshape Ad secundum dicendum} quod creatio non est mutatio nisi secundum modum
 intelligendi tantum. Nam de ratione mutationis est, quod aliquid idem
 se habeat aliter nunc et prius, nam quandoque est idem ens actu, aliter
 se habens nunc et prius, sicut in motibus secundum quantitatem et
 qualitatem et ubi; quandoque vero est idem ens in potentia tantum,
 sicut in mutatione secundum substantiam, cuius subiectum est
 materia.

&
第二に対しては、次のように言われるべきである。
創造は変化でない。変化である(ように見える)のは、ただ理解のしかたに応じ
 てのことにすぎない。
すなわち、変化の概念には、何か同一のものが、今と前とではことなっているこ
 とが含まれている。じっさい、量、性質、位置における運動の
 場合のように、現実態にある同一の有が、今と前とでは異なるあり方をするこ
 ともあれば、実体における変化のように、この場合、質料が基体となるのだが、
 可能態においてのみある同一の有が(今と前とでは異なるあり方をすることも
 ある)。


\\

 Sed in creatione, per quam producitur tota substantia rerum,
 non potest accipi aliquid idem aliter se habens nunc et prius, nisi
 secundum intellectum tantum; sicut si intelligatur aliqua res prius non
 fuisse totaliter, et postea esse. Sed cum actio et passio conveniant in
 substantia motus, et differant solum secundum habitudines diversas, ut
 dicitur in III {\itshape Physic}., oportet quod, subtracto motu, non remaneant
 nisi diversae habitudines in creante et creato. 

&

しかし、創造においては、それによって事物の全実体が生み出されるのだが、何
 か同一のものが、今と前とで、異なるあり方をするということは、たんに知性
 認識に従う以外には理解されえない。たとえば、ある事物が、前には全体的に
 存在しなかったのが、その後に存在する、ということが理解されるように。
しかし、『自然学』第3巻で言われるように、能動と受動は、運動の実体におい
 て一致し、たんに異なる関係に応じて区別されるのだから、運動が取り除かれ
 ると、創造するものと創造されるものにおいて、異なる関係以外のものは残ら
 ないはずである。

\\

-- Sed quia modus significandi sequitur modum intelligendi, ut dictum
 est, creatio significatur per modum mutationis, et propter hoc dicitur
 quod creare est ex nihilo aliquid facere. Quamvis {\itshape facere} et
 {\itshape fieri} magis in hoc conveniant quam {\itshape mutare} et
 {\itshape mutari}, quia facere et fieri important habitudinem causae ad
 effectum et effectus ad causam, sed mutationem ex consequenti.

&

しかし、すでに述べられたように、表示のあり方は、知性認識のあり方に従うの
 で、創造は、変化のあり方によって表示され、このために、創造することは、
 無から何かを作ることだと言われる。しかし、作る、作られるは、この点にお
 いて、変化させる、させられる、よりも適している。なぜなら、作る、作られ
 るは、原因の結果に対する、結果の原因に対する関係を意味し、結果的にしか
 変化を意味しないからである。


\\

{\scshape Ad tertium dicendum} quod in his quae fiunt sine motu, simul est fieri et
 factum esse, sive talis factio sit terminus motus, sicut illuminatio
 (nam simul aliquid illuminatur et illuminatum est); sive non sit
 terminus motus, sicut simul formatur verbum in corde et formatum
 est. Et in his, quod fit, est, sed cum dicitur fieri, significatur ab
 alio esse, et prius non fuisse. Unde, cum creatio sit sine motu, simul
 aliquid creatur et creatum est.

&

第三に対しては、次のように言われるべきである。運動なしに生じるものどもに
 おいては、作られることと作られたこととは同時である。これは、その「作り」
 が運動の終極であろうと、そうでなかろうと同様である。前者の場合、たとえ
 ば、照明がその例であり(あるものが照明されることと、照明されたことと
 は同時である)、後者の場合は、たとえば、言葉が心の中で形成され、同時に、
 形成されたというように。これらのものにおいては、作られていることが、在
 ることである。しかし、作られていると言われるとき、他のものによって存在
 すること、そして、以前には存在しなかったことが意味される。したがって、
 創造は運動なしにあるので、何かが創造されていることと創造されたこととは
 同時である。

\\

{\scshape Ad quartum dicendum} quod obiectio illa procedit ex falsa imaginatione,
 ac si sit aliquod infinitum medium inter nihilum et ens, quod patet
 esse falsum. Procedit autem falsa haec imaginatio ex eo quod creatio
 significatur ut quaedam mutatio inter duos terminos existens.

&

第四に対しては、次のように言われるべきである。
かの反論は、無と有との間に無限の媒体が存在するかのような、誤った想像から
 出ているが、これが誤りであることは明らかである。また、この誤った想像は、
 創造が、存在する二つの端のあいだの一種の変化として表示されることに由来する。


\end{longtable}

\newpage
\rhead{a.~3}

\begin{center}
 {\Large {\bf ARTICULUS TERTIUS}}\\
 {\large UTRUM CREATIO SIT ALIQUID IN CREATURA}\\
 {\footnotesize I {\itshape Sent.}, d.~11, q.~1, a.~1, ad 1; II, d.~1,
 q.~1, a.~2, ad 4, 5; II {\itshape SCG.}, c.~18; {\itshape De Pot.},
 q.~3, a.~3.}\\
 {\Large 第三項\\創造は被造物の中の何かか}
\end{center}


\begin{longtable}{p{21em}p{21em}}

{\huge A}{\scshape d tertium sic proceditur}. Videtur quod creatio non sit aliquid in
 creatura. Sicut enim creatio passive accepta attribuitur creaturae, ita
 creatio active accepta attribuitur creatori. Sed creatio active accepta
 non est aliquid in creatore, quia sic sequeretur quod in Deo esset
 aliquid temporale. Ergo creatio passive accepta non est aliquid in
 creatura.

&

第三に対しては、次のように進められる。
創造は、被造物の内にある何かではないと思われる。
なぜなら、受動的に理解された創造が被造物に帰せられるように、能動的に理解
 された創造は、創造者に帰せられる。しかるに、能動的に理解された創造は、
 創造者の内にある何かではない。なぜなら、もしそうだとしたら、神の中に、
 なにか時間的なものがあっただろうから。ゆえに、受動的に理解された創造も、
 被造物の内にある何かではない。



\\

2.~{\scshape Praeterea}, nihil est medium inter creatorem et creaturam. Sed creatio
 significatur ut medium inter utrumque, non enim est creator, cum non
 sit aeterna; neque creatura, quia oporteret eadem ratione aliam ponere
 creationem qua ipsa crearetur, et sic in infinitum. Creatio ergo non
 est aliquid.

& さらに、創造者と被造物のあいだに中間的なものはなにもない。しかるに、創
造は、両者のあいだにある中間的なものとして表示される。なぜなら、それは永
遠のものでないから創造者ではなく、また、被造物でもないからである。被造物
でないというのは、もし被造物だとすると、[被造物が存在するには創造が必要
だという]同じ理由で、創造[という被造物]が創造されるための創造を措定し
なければならず、そのようにして無限に進むからである。ゆえに、創造は何かで
はない。

\\

3.~{\scshape Praeterea}, si creatio est aliquid praeter substantiam creatam, oportet
 quod sit accidens eius. Omne autem accidens est in subiecto. Ergo res
 creata esset subiectum creationis. Et sic idem esset subiectum
 creationis et terminus. Quod est impossibile, quia subiectum prius est
 accidente, et conservat accidens; terminus autem posterius est actione
 et passione cuius est terminus, et eo existente cessat actio et
 passio. Igitur ipsa creatio non est aliqua res.

&

さらに、もし創造が創造された実体以外の何かだとすると、それは実体の附帯
 性でなければならない。ところがすべて附帯性は基体においてある。ゆえに創
 造された事物が創造の基体だということになっただろう。したがっ
 て同じものが創造の基体と終端となったであろう。基体は附帯性に先行し、附
 帯性を保持する。また、終端は、それの終端であるところの能動・受動に後続し、
 終端が出現することで能動・受動が終わる。ゆえに、これは不可能である。ゆ
 えに、創造そのものは何らの事物でもない。



\\

{\scshape Sed contra}, maius est fieri aliquid secundum totam substantiam, quam
 secundum formam substantialem vel accidentalem. Sed generatio
 simpliciter vel secundum quid, qua fit aliquid secundum formam
 substantialem vel accidentalem, est aliquid in generato. Ergo multo
 magis creatio, qua fit aliquid secundum totam substantiam, est aliquid
 in creato.

&

しかし反対に、何かが全実体の点で生じることは、実体形相の点で、あるいは附
 帯形相の点で生じることよりも重大なことである。ところが、何かが実体形相
 や附帯形相の点で生じるのは生成によってであるが、この生成は、端的なもの
 であれある意味においてのものであれ、生成したも
 のにおける何かである。ゆえに創造によって何かがその全実体の点で生じる
 のだから、ましてや創造は創造されたものにおける何かである。

\\

{\scshape Respondeo dicendum} quod creatio ponit aliquid in creato secundum
 relationem tantum. Quia quod creatur, non fit per motum vel per
 mutationem. Quod enim fit per motum vel mutationem, fit ex aliquo
 praeexistenti, quod quidem contingit in productionibus particularibus
 aliquorum entium; non autem potest hoc contingere in productione totius
 esse a causa universali omnium entium, quae est Deus. Unde Deus,
 creando, producit res sine motu. Subtracto autem motu ab actione et
 passione, nihil remanet nisi relatio, ut dictum est. Unde relinquitur
 quod creatio in creatura non sit nisi relatio quaedam ad creatorem, ut
 ad principium sui esse; sicut in passione quae est cum motu, importatur
 relatio ad principium motus.

&

答えて言わなければならない。創造は、創造されたものにおいて、たんに関係の
点で、何かを措定する。というのも、創造されるものは、運動や変化を通して生
じるのではないからである。運動や変化によって生じるものは、先に存在する何
かから生じるが、これは、なんらかの有の個別的な産出において起こる。しかし、
すべての有の普遍的原因、つまり神による全存在の産出において、こういうこと
は起こらない。したがって神は、創造するとき、事物を運動なしに生み出す。と
ころが、能動・受動から運動が除去されると、すでに述べられたように、関係し
か残らない。したがって、被造物における創造は、自らの存在の根源としての創
造者への、一種の関係以外でない。これはちょうど、運動とともにある受動の中
に、運動の根源への関係が含まれているのと同様である。


\\

Ad primum ergo dicendum quod creatio active significata significat
 actionem divinam, quae est eius essentia cum relatione ad
 creaturam. Sed relatio in Deo ad creaturam non est realis, sed secundum
 rationem tantum. Relatio vero creaturae ad Deum est relatio realis, ut
 supra dictum est, cum de divinis nominibus ageretur.

&

第一に対しては、それゆえ、次のように言われるべきである。
能動的に表示された創造は、神の作用を表示し、それは、被造物への関係を伴っ
 た神の本質である。しかし、神における被造物への関係は実在的でなく、たん
 に概念的である。他方、上で、神の名について論じられたときに述べられたと
 おり、被造物の神への関係は実在的である。


\\

{\scshape Ad secundum dicendum} quod, quia creatio significatur ut
 mutatio, sicut dictum est; mutatio autem media quodammodo est inter
 movens et motum, ideo etiam creatio significatur ut media inter
 creatorem et creaturam. Tamen creatio passive accepta est in creatura,
 et est creatura. Neque tamen oportet quod alia creatione creetur, quia
 relationes, cum hoc ipsum quod sunt, ad aliquid dicantur, non
 referuntur per aliquas alias relationes, sed per seipsas; sicut etiam
 supra dictum est, cum de aequalitate personarum ageretur.

&

第二に対しては、次のように言われるべきである。
すでに述べられたように、創造は変化として表示される。
ところで、変化は、動かすものと動かされるものとのあいだにある中間的な何か
 である。ゆえに、創造もまた、創造者と被造物とのあいだにある中間的なもの
 として表示される。しかし、被造物において受動的に理解された創造は、それ
 もまた被造物である。しかし、それが別の創造によって創造される必要はない。
 なぜなら、上で、ペルソナの等しさが論じられたときに述べられたとおり、関
 係は、それが「ある」ということ自体が何かとの関係で語られるのだから、他
 の何らかの関係によって関係づけられることはなく、むしろ、それ自体によっ
 て関係づけられるからである。

\\

{\scshape Ad tertium dicendum} quod creationis, secundum quod significatur ut
 mutatio, creatura est terminus, sed secundum quod vere est relatio,
 creatura est eius subiectum, et prius ea in esse, sicut subiectum
 accidente. Sed habet quandam rationem prioritatis ex parte obiecti ad
 quod dicitur, quod est principium creaturae. Neque tamen oportet quod,
 quandiu creatura sit, dicatur creari, quia creatio importat habitudinem
 creaturae ad creatorem cum quadam novitate seu incoeptione.

&

第三に対しては、次のように言われるべきである。
創造が変化として表示される場合には、被造物が創造の終端である。しかし、創
 造が本当は関係であるという点で言えば、被造物がその基体であり、存在にお
 いて基体が附帯性よりも先であるのと同様、存在において被造物が創造よりも
 先である。しかし、創造は、それに対して語られるところの対象の側から、つ
 まり、被造物の根源であるという側面から、ある種の先行性の性格をもつ。し
 かし、被造物が存在するあいだ、ずっと創造されると言われる必要はない。な
 ぜなら、創造は、ある種の新しさや始まりを伴った、被造物の創造者への関係
 を意味するからである。


\\

\end{longtable}
\newpage

\rhead{a.~4}

\begin{center}
 {\Large {\bf ARTICULUS QUARTUS}}\\
 {\large UTRUM CREARI SIT PROPRIUM COMPOSITORUM ET SUBSISTENTIUM}\\
 {\footnotesize {\itshape De Pot.}, q.~3, a.~1, ad 12; a.~3, ad 2; a.~8,
 ad 3; {\itshape De Verit.}, q.~27, a.~3, ad 9; {\itshape Quodl.}~IX,
 q.~5, a.~1; Opusc.~XXXVII, {\itshape de Quaruor Opposit.}, c.~4.}\\
 {\Large 第四項\\創造されることは複合体や自存するものに固有のことか}
\end{center}

\begin{longtable}{p{21em}p{21em}}

{\huge A}{\scshape d quartum sic proceditur}. Videtur quod creari non
sit proprium compositorum et subsistentium. Dicitur enim in libro
{\itshape de Causis}, {\itshape prima rerum creatarum est esse}. Sed
esse rei creatae non est subsistens. Ergo creatio proprie non est
subsistentis et compositi.


&
第四に対しては、次のように進められる。
創造されることは、複合体や自存するものに固有ではないと思われる。
なぜなら、『原因論』という書物の中で、「創造された事物の中で第一のものは
 存在である」と言われている。ところが、被造物の存在は、自存するものでは
 ない。ゆえに、創造は、自存するものや複合体に固有に属するのではない。

\\


2.~{\scshape Praeterea}, quod creatur est ex
nihilo. Composita autem non sunt ex nihilo, sed ex suis
componentibus. Ergo compositis non convenit creari.


&

さらに、創造されるものは、無から創造される。ところが、複合体は、無からで
 はなく、それの構成物から生じる。ゆえに、複合体に創造されることは適合し
 ない。

\\


3.~{\scshape Praeterea}, illud proprie producitur per
primam emanationem, quod supponitur in secunda, sicut res naturalis per
generationem naturalem, quae supponitur in operatione artis. Sed illud
quod supponitur in generatione naturali, est materia. Ergo materia est
quae proprie creatur, et non compositum.


&

さらに、ちょうど、技術の働きにおいて前提されるものは、自然的生成によって
 固有に生み出される自然的事物であるように、第二の流出において前提されるも
 のは、第一の流出によって固有に生み出される。
しかし、自然的生成において前提されるものとは質料である。ゆえに、固有の意
 味で創造されるのは質料であり、複合体ではない。

\\


{\scshape Sed contra est} quod dicitur {\itshape Gen}.~{\scshape i},
{\itshape In principio creavit Deus caelum et terram}. Caelum autem et
terra sunt res compositae subsistentes. Ergo horum proprie est creatio.


&
しかし反対に、『創世記』1章で「初めに神は天と地を創造した」と言われてい
 る。ところで、天と地は、自存する複合された事物である。ゆえに、創造は固
 有にこれらに属する。

\\


{\scshape Respondeo dicendum} quod creari est quoddam fieri, ut dictum
est. Fieri autem ordinatur ad esse rei. Unde illis proprie convenit
fieri et creari, quibus convenit esse. Quod quidem convenit proprie
subsistentibus, sive sint simplicia, sicut substantiae separatae; sive
sint composita, sicut substantiae materiales. Illi enim proprie convenit
esse, quod habet esse; et hoc est subsistens in suo esse. Formae autem
et accidentia, et alia huiusmodi, non dicuntur entia quasi ipsa sint,
sed quia eis aliquid est; ut albedo ea ratione dicitur ens, quia ea
subiectum est album. Unde, secundum philosophum, accidens magis proprie
dicitur {\itshape entis} quam {\itshape ens}. Sicut igitur accidentia et
formae, et huiusmodi, quae non subsistunt, magis sunt coexistentia quam
entia; ita magis debent dici {\itshape concreata} quam creata. Proprie
vero creata sunt subsistentia.

&

答えて言わなければならない。
すでに述べられたとおり、創造されることは、ある種の生じることである。
ところで、生じることは、事物の存在へ秩序付けられる。
したがって、生じることや創造されることは、存在が適合するものに固有に適合
 する。
そして存在は、離在実体のように単純なものであれ、質料的実体のように複合さ
 れたものであれ、自存するものに固有に適合する。
なぜなら、存在は、存在を持つもの、すなわち、自らの存在において自存するも
 のに、固有に適合するからである。また、形相や附帯性や、その他このような
 ものは、それ自身が有と言われるのではなく、それらによって何かであると言
 われる。たとえば、白性は、それによって基体が白くあるために、有と言われ
 る。このことから、哲学者によれば、附帯性は、厳密には「有」ではなく「有
 に属するなにか」である。ゆえに、ちょうど附帯性や形相など自存しないもの
 が、有というよりは、共にあるもの(coexistens)であるように、それらは創造
 されたものと言うよりは、共に創造されたもの(concreatum)である。他方、固
 有の意味で創造されたのは、自存するものである。

\\


{\scshape Ad primum ergo dicendum} quod, cum dicitur, {\itshape prima
rerum creatarum est esse}, ly {\itshape esse} non importat subiectum
creatum; sed importat propriam rationem obiecti creationis. Nam ex eo
dicitur aliquid creatum, quod est ens, non ex eo quod est hoc ens, cum
creatio sit emanatio totius esse ab ente universali, ut dictum est. Et
est similis modus loquendi, sicut si diceretur quod {\itshape primum
visibile est color}, quamvis illud quod proprie videtur, sit {\itshape
coloratum}.

&

第一に対しては、それゆえ、次のように言われるべきである。「創造された事物
の中で第一のものは存在である」と言われるとき、この「存在」は、創造された
基体を意味するのではなく、創造の対象の固有の性格を意味する。つまり、或る
ものが創造されたと言われるのは、それが有だからであって、それが「この有」
だからではない。なぜなら、すでに言われたように、創造とは、普遍的存在者か
らの全存在の流出だからである。またこれは、固有の意味で見られるのは[色で
はなく]「色がついたもの」なのだが、「第一の見られうるものは色である」と
言われるとしたら、それに似た言い方である。


\\


{\scshape Ad secundum dicendum} quod creatio non dicit constitutionem
rei compositae ex principiis praeexistentibus, sed compositum sic
dicitur creari, quod simul cum omnibus suis principiis in esse
producitur.

&
第二に対しては、次のように言われるべきである。
創造は、先行して存在する根源から複合された事物の構成を言うのではない。複
 合体が創造されると言われるのは、それの全ての根源と同時に存在へと生み
 出されるという意味である。


\\


{\scshape Ad tertium dicendum} quod ratio illa non
probat quod sola materia creetur; sed quod materia non sit nisi ex
creatione. Nam creatio est productio totius esse, et non solum materiae.

&

この異論は、質料だけが創造されることを証明しているのではなく、創造によら
 なければ質料が存在しないことを証明している。創造は全存在の産出であり、
 質料だけの産出ではない。



\end{longtable}
\newpage

\rhead{a.~5}




\begin{center}
 {\Large {\bf ARTICULUS QUINTUS}}\\
 {\large UTRUM SOLIUS DEI SIT CREARE}\\
 {\footnotesize Infra, q.~65, a.~3; q.~90, a.~3; II {\itshape Sent.},
 d.~1, q.~1, a.~3; IV, d.~5, q.~1, a.~3, qa 3; II {\itshape SCG.},
 c.~20, 21; {\itshape De Verit.}, q.~5, a.~9; {\itshape De Pot.}, q.~3,
 a.~4; {\itshape Quodl.}~III, q.~3, a.~1; {\itshape Compend.~Theol.},
 c.~70; Opusc.~XV, {\itshape de Angelis}, c.~10; XXXVII, {\itshape de
 Quatuor Opposit.}, c.~4.}\\
 {\Large 第四項\\創造は神だけに属するか}
\end{center}

\begin{longtable}{p{21em}p{21em}}

{\huge A}{\scshape d quintum sic proceditur}. Videtur quod non solius
 Dei sit creare. Quia secundum philosophum, {\itshape perfectum} est
 quod potest sibi simile facere. Sed creaturae immateriales sunt
 perfectiores creaturis materialibus, quae faciunt sibi simile, ignis
 enim generat ignem, et homo generat hominem. Ergo substantia
 immaterialis potest facere substantiam sibi similem. Sed substantia
 immaterialis non potest fieri nisi per creationem, cum non habeat
 materiam ex qua fiat. Ergo aliqua creatura potest creare.

&

第四に対しては次のように進められる。
創造することは神だけに属するのではないと思われる。
なぜなら、哲学者によれば「完全なもの」とは、自らに似たものを作ることがで
 きるものである。ところが、非質料的な被造物は質料的な被造物よりも完全だ
 が、この質料的被造物は自らに似たものを作ることができる。たとえば、火は
 火を生み、人間は人間を生む。ゆえに、非質料的な実体は、自らに似た実体を
 作ることができる。ところが、非質料的な実体は、そこから生じるための質料
 をもたないので、創造によってしか生じえない。ゆえに、なんらかの被造物は
 創造することができる。

\\


2.~{\scshape Praeterea}, quanto maior est resistentia ex
 parte facti, tanto maior virtus requiritur in faciente. Sed plus
 resistit contrarium quam nihil. Ergo maioris virtutis est aliquid
 facere ex contrario, quod tamen creatura facit; quam aliquid facere ex
 nihilo. Multo magis igitur creatura hoc facere potest.

&
さらに、作られるものの側からの抵抗が大きいほど、それだけいっそう大きな力
 が作るものの中に必要とされる。ところが、反対物は、無より多く抵抗する。
 ゆえに、無から何かを作るよりも、反対物から何かを作る方が、より大きな力
 に属する。被造物は、反対物から何かを作るのだから、いっそう容易に無から何か
 を作ることができる。

\\


3.~{\scshape Praeterea}, virtus facientis consideratur
 secundum mensuram eius quod fit. Sed ens creatum est finitum, ut supra
 probatum est, cum de Dei infinitate ageretur. Ergo ad producendum per
 creationem aliquid creatum, non requiritur nisi virtus finita. Sed
 habere virtutem finitam non est contra rationem creaturae. Ergo non est
 impossibile creaturam creare.

&

さらに、作るものの力は、作られたものの尺度に即して考察される。
ところで、上で、神の無限性について論じられたときに証明されたように、創造
 された有は有限である。ゆえに、創造によってなんらかの被造物を生み出すた
 めに、有限な力しか必要とされない。
ところが、有限な力を持つことは、被造物の性格に反しない。ゆえに、被造物が
 創造することは不可能でない。

\\


{\scshape Sed contra est} quod Augustinus dicit, in III
 {\itshape de Trin}., quod neque boni neque mali Angeli possunt esse creatores
 alicuius rei. Multo minus igitur aliae creaturae.

&

しかし反対に、アウグスティヌスは『三位一体論』3巻で、善い天使も悪い天使
 も、何らかの事物の創造者でありえないと述べている。ましてや、他の被造物
 は創造者でありえない。

\\


{\scshape Respondeo dicendum} quod satis apparet in primo
 aspectu, secundum praemissa, quod creare non potest esse propria actio
 nisi solius Dei. Oportet enim universaliores effectus in universaliores
 et priores causas reducere. Inter omnes autem effectus,
 universalissimum est ipsum esse. Unde oportet quod sit proprius
 effectus primae et universalissimae causae, quae est Deus. Unde etiam
 dicitur libro {\itshape de Causis}, quod neque intelligentia vel anima nobilis dat
 esse, nisi inquantum operatur operatione divina. Producere autem esse
 absolute, non inquantum est hoc vel tale, pertinet ad rationem
 creationis. Unde manifestum est quod creatio est propria actio ipsius
 Dei. 


&

答えて言わなければならない。
すでに述べられたことによるならば、創造することが、唯一、神だけに固有な働
 きであることは、一見して容易に見て取れる。理由は以下の通り。
より普遍的な結果は、より普遍的でより先の原因に還元されなければならない。
ところで、全ての結果の中で、もっとも普遍的なものは存在自体である。
したがって、存在自体は、第一でもっとも普遍的な原因、つまり神の固有の結果
 でなければならない。このことから、『原因論』という書物でも、知性体や高
 貴な魂も、神の働きによって働くという点でなければ、存在を与えないと言わ
 れている。ところで、この存在やあの存在を生み出すということではなく、無
 条件的に存在を生み出すことは、創造の性格に属する。したがって、創造が神
 自身の固有の作用であることは明らかである。


\\


Contingit autem quod aliquid participet actionem propriam alicuius
 alterius, non virtute propria, sed instrumentaliter, inquantum agit in
 virtute alterius; sicut aer per virtutem ignis habet calefacere et
 ignire. Et secundum hoc, aliqui opinati sunt quod, licet creatio sit
 propria actio universalis causae, tamen aliqua inferiorum causarum
 inquantum agit in virtute primae causae, potest creare. 

&

しかし、たとえば、空気が、火の力によって、熱したり燃やしたりすることがで
 きるように、或るものが他のものの力において働くという意味で、或るものが、
 固有の力によってではなく、道具として、他の何かの固有の働きを分有すること
 はありうる。
そしてこの意味で、ある人々は、創造は普遍的原因の固有の作用だが、下位の諸
 原因のうちのある原因は、第一原因の力において働く点で、創造することがで
 きるという意見を持った。

\\


Et sic posuit
 Avicenna quod prima substantia separata, creata a Deo, creat aliam post
 se, et substantiam orbis, et animam eius; et quod substantia orbis
 creat materiam inferiorum corporum. Et secundum hunc etiam modum
 Magister dicit, in V dist.~IV {\itshape Sent}., quod Deus potest creaturae
 communicare potentiam creandi, ut creet per ministerium, non propria
 auctoritate. 


&

そしてアヴィセンナは、この意味で、神によって創造された第一の離在実体が、
 自分の後の他の実体と、天球の実体とその魂を創造し、天球の実体は、
 下位の物体の質料を創造すると考えた。そして、これにしたがって、教師[=ロ
 ンバルドゥス]もまた、『命題集』第4巻第5区分で、神は、固有の権能でもっ
 てではなく、使いとして創造するように、神は被造物に創造する能力を伝達す
 ることができると述べている。


\\




Sed hoc esse non potest. Quia causa secunda instrumentalis
 non participat actionem causae superioris, nisi inquantum per aliquid
 sibi proprium dispositive operatur ad effectum principalis agentis. Si
 igitur nihil ibi ageret secundum illud quod est sibi proprium, frustra
 adhiberetur ad agendum, nec oporteret esse determinata instrumenta
 determinatarum actionum. Sic enim videmus quod securis, scindendo
 lignum, quod habet ex proprietate suae formae, producit scamni formam,
 quae est effectus proprius principalis agentis. Illud autem quod est
 proprius effectus Dei creantis, est illud quod praesupponitur omnibus
 aliis, scilicet esse absolute. Unde non potest aliquid operari
 dispositive et instrumentaliter ad hunc effectum, cum creatio non sit
 ex aliquo praesupposito, quod possit disponi per actionem
 instrumentalis agentis. -- Sic igitur impossibile est quod alicui
 creaturae conveniat creare, neque virtute propria, neque
 instrumentaliter sive per ministerium. 



&

しかし、これはありえない。なぜなら、道具的な第二原因が上位の原因の働きを
分有するのは、主要な作用者の結果にたいして、第二原因が、自らに固有な何か
を通して、態勢的に[≒安定して]働くというかたちで分有するのに限られるか
らである。ゆえに、もし、そのような第二原因が自らに固有なものにかんして何
も作用しなかったならば、[その第二原因は]その作用のために無駄に使われて
 いることになっただろうし、限定された作用に属する限定された道具でなくて
 もよいことになっただろう。この意味で、たとえば私たちは、斧が、斧自らの
 形相の固有性に基づく木を切るということによって、長椅子
 の形相を生み出すが、この形相は、主要な作用者[=椅子を作る人]の固有の
 結果である、ということを目にする。ところで、創造する神の固有の結果であ
 るものは、他のすべてのことに前提されること、すなわち、無条件的に存在す
 ることである。したがって、何かがこの結果のために、態勢的に、道具として
 働くことは不可能である。なぜなら、創造は、道具的作用者の働きによって整
 えられうるような何らかの前提に基づくものではないからである。-- ゆえに、
 このような意味で、固有の力によっても、道具的にであっても、使い(ministerium)を通して
 も、なんらかの被造物に創造することが適合することは不可能である。


\\



Et hoc praecipue inconveniens
 est dici de aliquo corpore, quod creet, cum nullum corpus agat nisi
 tangendo vel movendo; et sic requirit in sua actione aliquid
 praeexistens, quod possit tangi et moveri; quod est contra rationem
 creationis.


&

そして、創造するということが、何らかの物体について語られるのは、とくに不
 適切である。なぜなら、どんな物体も、触れたり動かしたりしない限り作用し
 ないので、その作用において、触れられたり動かされたりする何かを前提とす
 るが、これは、創造の性格に反するからである。

\\



{\scshape Ad primum ergo dicendum} quod aliquod
 perfectum participans aliquam naturam, facit sibi simile, non quidem
 producendo absolute illam naturam, sed applicando eam ad aliquid. Non
 enim hic homo potest esse causa naturae humanae absolute, quia sic
 esset causa sui ipsius, sed est causa quod natura humana sit in hoc
 homine generato. Et sic praesupponit in sua actione determinatam
 materiam per quam est hic homo. Sed sicut hic homo participat humanam
 naturam, ita quodcumque ens creatum participat, ut ita dixerim, naturam
 essendi, quia solus Deus est suum esse, ut supra dictum est. 


&

第一に対しては、それゆえ、次のように言われるべきである。
何らかの本性を分有する完全なものが自己に似たものを作るというのは、その本
 性を無条件的に生み出すことによるのではなく、その本性を何かに適用するこ
 とによる。じっさい、この[個別的な]人間が、無条件的に、人間本性の原因
 であることは不可能である。なぜなら、もしそうなら、それは自己自身の原因
 であることになっただろうから。したがって、そうではなく、この人間は、人
 間本性が、生み出されたこの人間において在ることの原因である。このように、
 それは自らの作用において、限定された質料を前提し、この限定された質料に
 よって、この人間が存在する。しかし、この人間が人間本性を分有するように、
 どんな創造された有も、言うならば存在の本性を分有する。なぜなら、上で言
 われたように、神だけが、自らの存在だから。

\\

Nullum igitur ens creatum potest producere aliquod ens absolute, nisi
 inquantum esse causat {\itshape in hoc}, et sic oportet quod
 praeintelligatur id per quod aliquid est hoc, actioni qua facit sibi
 simile.  In substantia autem immateriali non potest praeintelligi
 aliquid per quod sit {\itshape haec}, quia est haec per suam formam, per quam
 habet esse, cum sint formae subsistentes. Igitur substantia
 immaterialis non potest producere aliam substantiam immaterialem sibi
 similem, quantum ad esse eius; sed quantum ad perfectionem aliquam
 superadditam; sicut si dicamus quod superior Angelus illuminat
 inferiorem, ut Dionysius dicit.

&

ゆえに、どんな創造された有も、
 無条件的になんらかの有を生み出すことはできず、ただ、「この[具体的・個別
 的な]もの」において存在の原因となるという点でのみ、なんらかの有を生み出
 すにすぎない。したがって、自分に似たものを作り出す作用よりも先に、或る
 ものがそれによって「これ」であるところのもの[=「これ」であるために必
 要な何か、つまり指定された質料]が、理解されなければならない。
他方、非質料的実体においては、それらが「これ」であるために必要な何かが、
 先だって理解されることはできない。なぜなら、それらは、自存する形相だか
 ら、自分の形相によって存在をもち、また、その形相によって「これ」である
 から。ゆえに、非質料的実体は、その存在にかんして、自分に似た非質料的実
 体を生み出すことができないが、付加された何らかの完全性にかんしては、そ
 れができる。たとえば、ディオニュシウスが言うように、上位の天使が下位の
 天使を照らす、と私たちが言う場合のように。

\\


Secundum quem modum etiam in caelestibus est paternitas, ut ex verbis
 apostoli patet, {\itshape Ephes}.~{\scshape iii}, {\itshape ex quo
 omnis paternitas in caelo et in terra nominatur}. Et ex hoc etiam
 evidenter apparet quod nullum ens creatum potest causare aliquid, nisi
 praesupposito aliquo. Quod repugnat rationi creationis.

&

この意味では、『エフェソの信徒への手紙』3章「このことから、全ての父性が、天
 と知において名付けられる」\footnote{「御父から、天と地にある全ての家族がその名を与えられています。」(3:15)}という使徒の言葉によって明らかなとおり、天に
 おいても父性がある。そしてこのことからも、どんな被造の有も、何かを前提
 としなければ、何も原因として生み出さないことが、はっきりと明らかである。
 そしてこのことは、創造の性格に反する。

\\



{\scshape Ad secundum dicendum} quod ex contrario fit aliquid per
 accidens, ut dicitur in I {\itshape Physic}., per se autem fit aliquid
 ex subiecto, quod est in potentia. Contrarium igitur resistit agenti,
 inquantum impedit potentiam ab actu in quem intendit reducere agens,
 sicut ignis intendit reducere materiam aquae in actum sibi similem, sed
 impeditur per formam et dispositiones contrarias, quibus quasi ligatur
 potentia ne reducatur in actum. Et quanto magis fuerit potentia ligata,
 tanto requiritur maior virtus in agente ad reducendam materiam in
 actum. Unde multo maior potentia requiritur in agente, si nulla
 potentia praeexistat. Sic ergo patet quod multo maioris virtutis est
 facere aliquid ex nihilo, quam ex contrario.

&

第二に対しては、次のように言われるべきである。
『自然学』1巻で言われるように、何かが反対のものから生ま
 れるのというのは附帯的にであり、自体的には、基体から何かが生まれる。そ
 して、基体は[生まれるものに対して]可能態にある。ゆえに、反対のもの
 が作用者に抵抗するのは、作用者が可能態[=基体]を現実態へ導くときに、
 その現実態に到達しないように、可能態の邪魔をするかたちになる。たとえば、
 火が、水の質料を、自分に似た現実態へ導こうとするが、しかし、反対の形相と
 状態によって妨げられ、可能態は、いわば、現実態へ向かわないように縛り付
 けられるようにである。
そして、可能態がひどく縛り付けられるほど、質料を現実態へもたらすために、
 作用者の中により大きな力が必要とされる。
したがって、もし何の可能態もあらかじめ存在しなければ、はるかに大きな能力が
 作用者の中に必要とされる。ゆえに、反対のものから何かを作ることよりも、無か
 ら何かを作ることの方が、はるかに大きな力に属することが明らかである。

\\


{\scshape Ad tertium dicendum} quod virtus facientis non
 solum consideratur ex substantia facti, sed etiam ex modo faciendi,
 maior enim calor non solum magis, sed etiam citius calefacit. Quamvis
 igitur creare aliquem effectum finitum non demonstret potentiam
 infinitam, tamen creare ipsum ex nihilo demonstrat potentiam
 infinitam. Quod ex praedictis patet. Si enim tanto maior virtus
 requiritur in agente, quanto potentia est magis remota ab actu, oportet
 quod virtus agentis ex nulla praesupposita potentia, quale agens est
 creans, sit infinita, quia nulla proportio est nullius potentiae ad
 aliquam potentiam, quam praesupponit virtus agentis naturalis, sicut et
 non entis ad ens. Et quia nulla creatura habet simpliciter potentiam
 infinitam, sicut neque esse infinitum, ut supra probatum est,
 relinquitur quod nulla creatura possit creare.


&
第三に対しては、次のように言われるべきである。
作るものの力は、作られたものの実体によって考察されるだけでなく、作り方か
 らも考察される。たとえば、より大きな熱は、たんにより多く熱するだけでなく、
 より早く熱する。それゆえ、ある有限な結果を創造することは、たしかに無限の能力を
 示さないが、しかし、それを無から創造することは、無限の能力を示す。
このことは、すでに述べられたことから明らかである。
じっさい、もしも、大きな力が作用において必要とされるほど、それだけ多くの可能態が
 現実態から除去されるならば、どんな可能態も前提せずに作用するものの力は
 無限でなければならず、創造するものはそのような作用者である。というのも、
 非有から有に対する比がないように、可能態がないことから、自然的作用者の
 力が前提する何らかの可能態への比は存在しないからである。また、上で証明
 されたとおり、どんな被
 造物も、無限な存在をもたないのと同様、端的に無限な能力は持たないから、
 どんな被造物も創造することができないと言うことが帰結する。


\end{longtable}
\newpage
\rhead{a.~6}
\begin{center}
 {\Large {\bf ARTICULUS SEXTUS}}\\
 {\large UTRUM CREARE SIT PROPRIUM ALICUIUS PERSONAE}\\
 {\footnotesize II {\itshape Sent.}, prol.; {\itshape De Pot.}, q.~9,
 a.~5, ad 20.}\\
 {\Large 第六項\\創造することはどれかのペルソナに固有か}
\end{center}

\begin{longtable}{p{21em}p{21em}}
{\huge A}{\scshape d sextum sic proceditur}. Videtur quod
creare sit proprium alicuius personae. Quod enim est prius, est causa
eius quod est post; et perfectum imperfecti. Sed processio divinae
personae est prior quam processio creaturae, et magis perfecta, quia
divina persona procedit in perfecta similitudine sui principii, creatura
vero in imperfecta. Ergo processiones divinarum personarum sunt causa
processionis rerum. Et sic creare est proprium personae.

&

第六に対しては、次のように進められる。
創造することは、どれかのペルソナに固有だと思われる。
なぜなら、より先のものは、より後のものの原因であり、完全なものは不完全な
 ものの原因である。ところが、神のペルソナの発出は被造物の発出より先であ
 り、より完全である。というのも、神のペルソナは、自らの根源の完全な類似
 において発出するが、被造物は不完全な類似において発出するからである。ゆ
 えに、神のペルソナの発出は、諸事物の発出の原因である。したがって、創造
 することはペルソナに固有である。

\\


{\scshape 2 Praeterea}, personae divinae non distinguuntur ab invicem
nisi per suas processiones et relationes. Quidquid igitur differenter
attribuitur divinis personis, hoc convenit eis secundum processiones et
relationes personarum. Sed causalitas creaturarum diversimode
attribuitur divinis personis, nam in symbolo fidei patri attribuitur
quod sit {\itshape creator omnium visibilium et invisibilium}; filio
autem attribuitur quod {\itshape per eum omnia facta sunt}; sed spiritui
sancto, quod sit {\itshape dominus et vivificator}. Causalitas ergo
creaturarum convenit personis secundum processiones et relationes.

&

さらに、神のペルソナが相互に区別されるのは、それらの発出と関係によってに
 他ならない。ゆえに、神のペルソナに異なるかたちで帰属されるものは全て、
 ペルソナの発出と関係に応じてそれらに適合する。
ところが、被造物の原因であるという性格は、神のペルソナにさまざまな形で帰
 属される。すなわち、信教では、父に「見えるものと見えないもの全ての創造
 者」が、子に「彼を通して万物が作られた」が、聖霊に「主であり生かす者」
 が帰属される。ゆえに、被造物の原因であるという性格は、発出と関係に応じ
 て、ペルソナに適合する。

\\

{\scshape 3 Praeterea}, si dicatur quod causalitas
creaturae attenditur secundum aliquod attributum essentiale quod
appropriatur alicui personae, hoc non videtur sufficiens. Quia quilibet
effectus divinus causatur a quolibet attributo essentiali, scilicet
potentia, bonitate et sapientia, et sic non magis pertinet ad unum quam
ad aliud. Non deberet ergo aliquis determinatus modus causalitatis
attribui uni personae magis quam alii, nisi distinguerentur in creando
secundum relationes et processiones.

& 


さらに、被造物の原因であるという性格は、あるペルソナに固有化される何らか
の本質的属性に応じて見出されると言われるならば、それは不十分だと思われる。
なぜなら、神のどんな結果も、能力、善性、知恵といった、本質的属性のどれに
よっても原因されるので、したがって、他ではないある一つの属性に帰属される
ことがないからである。ゆえに、もし、[原因性が]創造において関係と発出に
したがって区別されるのでなかったと仮定するのであれば、原因性の何らかの限
定されたありかたが、 他ではないある一つのペルソナに帰属される必要はないこ
とになっていただろう。

\\

{\scshape Sed contra} est quod dicit Dionysius, {\scshape ii}
cap.~{\itshape de Div.~Nom}., quod communia totius divinitatis sunt
{\itshape omnia causalia}.

&

しかし反対に、ディオニュシウスは、『神名論』2章で、神性全体に属する共通
 のものが、「万物の原因に関係する」と述べている。

\\

{\scshape Respondeo dicendum} quod creare est proprie
causare sive producere esse rerum. Cum autem omne agens agat sibi
simile, principium actionis considerari potest ex actionis effectu,
ignis enim est qui generat ignem. Et ideo creare convenit Deo secundum
suum esse, quod est eius essentia, quae est communis tribus
personis. Unde creare non est proprium alicui personae, sed commune toti
Trinitati. 


&

答えて言わなければならない。
固有の意味で、創造することとは、諸事物の存在の原因となること、あるいはそ
 れを生み出すことである。
ところで、作用者は全て自らに似たものを生み出すので、作用の根源は、作用の
 結果から考察されうる。たとえば火とは、火を生むものである。ゆえに、創造
 することは、神に、神の存在に即して適合するが、この存在は神の本質であり、
 神の本質は三つのペルソナに共通である。したがって、創造することは、どれ
 かのペルソナに固有ではなく、三位一体全体に共通である。

\\



Sed tamen divinae personae secundum rationem suae
processionis habent causalitatem respectu creationis rerum. Ut enim
supra ostensum est, cum de Dei scientia et voluntate ageretur, Deus est
causa rerum per suum intellectum et voluntatem, sicut artifex rerum
artificiatarum. Artifex autem per verbum in intellectu conceptum, et per
amorem suae voluntatis ad aliquid relatum, operatur. Unde et Deus pater
operatus est creaturam per suum verbum, quod est filius; et per suum
amorem, qui est spiritus sanctus. Et secundum hoc processiones
personarum sunt rationes productionis creaturarum, inquantum includunt
essentialia attributa, quae sunt scientia et voluntas.

&

しかし、神のペルソナは、自らの発出の性格に応じて、諸事物の創造にかんして、
 原因性をもっている。上で、神の知と意志について論じられたときに示されたよ
 うに、神は、自らの知性と意志によって諸事物の原因である。それはちょうど、技術
 者が、技術によって生み出された事物の原因であるのと同様である。
ところで、技術者は、知性の中に抱かれた言葉を通して、また、関係する何かへ
 の自らの意志に属する愛を通して、働く。
したがって、父である神も、自らの言葉を通して被造物に働きかけたが、この言
 葉は子である。また、自らの愛を通して働きかけたが、この愛は聖霊である。
この点で、ペルソナの発出は、知と意志という本質的属性を含む限りで、被造物
 の産出の根拠である。


\\

{\scshape Ad primum ergo dicendum} quod processiones
divinarum personarum sunt causa creationis sicut dictum est.

&
第一に対しては、それゆえ、次のように言われるべきである。
神のペルソナの発出は、すでに述べられたように、創造の原因である。

\\


{\scshape Ad secundum dicendum} quod, sicut natura
divina, licet sit communis tribus personis, ordine tamen quodam eis
convenit, inquantum filius accipit naturam divinam a patre, et spiritus
sanctus ab utroque; ita etiam et virtus creandi, licet sit communis
tribus personis, ordine tamen quodam eis convenit; nam filius habet eam
a patre, et spiritus sanctus ab utroque. Unde creatorem esse attribuitur
patri, ut ei qui non habet virtutem creandi ab alio. De filio autem
dicitur per quem omnia facta sunt, inquantum habet eandem virtutem, sed
ab alio, nam haec praepositio per solet denotare causam mediam, sive
principium de principio. Sed spiritui sancto, qui habet eandem virtutem
ab utroque, attribuitur quod dominando gubernet, et vivificet quae sunt
creata a patre per filium. 

&

第二に対しては、次のように言われるべきである。神の本性は、三つのペルソナ
に共通だが、神の本性を、子は父から、聖霊はこの両者から受け取るという点で、
ある秩序によってそれらに属する。同様に、創造の力も、三つのペルソナに共通
だが、子は父から、聖霊はこの両者からそれを受け取るという点で、ある秩序に
よってそれらに属する。
したがって、創造者であるということも、父には、他から創造の力をもらうので
 はないものとして帰属する。他方、子については、彼を通して万物が作られた
 と言われるが、それは、子は創造の同じ力を持つが、しかし、他によって持つ、
 という意味で言われる。なぜなら、この「通して」という前置詞は、媒介とな
 る原因、あるいは、根源からの根源を指示することが常だからである。
また、この同じ力を父と子の両者から持つ聖霊には、支配することで統治するこ
 と、あるいは、父によって子を通して創造されたものを生かすことが帰属され
 る。


\\



-- Potest etiam huius attributionis communis
ratio accipi ex appropriatione essentialium attributorum. Nam, sicut
supra dictum est, patri appropriatur potentia, quae maxime manifestatur
in creatione, et ideo attribuitur patri creatorem esse. Filio autem
appropriatur sapientia, per quam agens per intellectum operatur, et ideo
dicitur de filio, {\itshape per quem omnia facta sunt}. Spiritui sancto autem
appropriatur bonitas, ad quam pertinet gubernatio deducens res in
debitos fines, et vivificatio, nam vita in interiori quodam motu
consistit, primum autem movens est finis et bonitas.

&

さらに、この帰属の共通の性格は、本質的属性の固有化の側面からも理解されう
 る。つまり、上で述べられたとおり、父には能力が固有化されるが、それは創
 造において最大限に顕示され、それゆえ、創造者であることが父に帰属される。
他方、子には知恵が固有化されるが、知恵を通して、知性によって作用するもの
 は働く。ゆえに、子について、「彼を通して万物が作られた」と言われる。
また、聖霊には善性が固有化されるが、それには、事物をしかるべき目的へ導く
 統治と、生かすことが属する。生命は、下位のものにおいて、何らかの運動に
 よって成立するが、第一動者は、目的であり善性である。

\\


{\scshape Ad tertium dicendum} quod, licet quilibet
effectus Dei procedat ex quolibet attributorum, tamen reducitur
unusquisque effectus ad illud attributum, cum quo habet convenientiam
secundum propriam rationem, sicut ordinatio rerum ad sapientiam, et
iustificatio impii ad misericordiam et bonitatem se superabundanter
diffundentem. Creatio vero, quae est productio ipsius substantiae rei,
reducitur ad potentiam.


&

第三に対しては、次のように言われるべきである。
どんな神の結果も、属性のどれからも発出するが、しかし、各々の結果は、固有
 の性格に応じた適合性をそれと共に持つ属性へと還元される。
たとえば、諸事物の秩序付けは知恵へ、不敬虔者の義化は、自らをあふれるばか
 りに注ぎ入れる慈悲と善性へ、という具合に。
また、創造は、事物の実体そのものの産出なので、能力に還元される。

\end{longtable}
\newpage
\rhead{a.~7}

\begin{center}
 {\Large {\bf ARTICULUS SEPTIMUS}}\\
 {\large UTRUM IN CREATURIS SIT NECESSE INVENIRI VESTIGIUM TRINITATIS}\\
 {\footnotesize Infra, q.~93, a.~6; I {\itshape Sent.}, d.~3, q.~2,
 a.~2; IV {\itshape SCG.}, c.~24; {\itshape De Pot.}, q.~9, a.~9.}\\
 {\Large 第七項\\被造物の中に三位一体の痕跡が見出される必要があるか}
\end{center}

\begin{longtable}{p{21em}p{21em}}

{\huge A}{\scshape d septimum sic proceditur}. Videtur quod in
 creaturis non sit necesse inveniri vestigium Trinitatis. Per sua enim
 vestigia unumquodque investigari potest. Sed Trinitas personarum non
 potest investigari ex creaturis, ut supra habitum est. Ergo vestigia
 Trinitatis non sunt in creatura.


&

第7にかんしては次のように進められる。
被造物の中に三位一体の痕跡が見出される必要はないと思われる。
各々のものは、自己の痕跡を通して探究されうる。
しかし、ペルソナの三位一体は、上で述べられたとおり、被造物をもとにしては
 探究されえない。ゆえに、三位一体の痕跡は、被造物の中にはない。

\\



{\scshape 2 Praeterea}, quidquid in creatura est,
 creatum est. Si igitur vestigium Trinitatis invenitur in creatura
 secundum aliquas proprietates suas, et omne creatum habet vestigium
 Trinitatis, oportet in unaquaque illarum inveniri etiam vestigium
 Trinitatis, et sic in infinitum.

&

さに、なんであれ被造物の中にあるものは創造されたものである。
ゆえに、三位一体の痕跡が、その何らかの固有性にしたがって被造物の中に見出
 されるならば、全ての被造物が三位一体の痕跡を持つのだから、その固有性そ
 れぞれの中にも三位一体の痕跡が見出されなければならず、そうして無限に進
 むだろう。

\\


{\scshape 3 Praeterea}, effectus non repraesentat nisi
 suam causam. Sed causalitas creaturarum pertinet ad naturam communem,
 non autem ad relationes, quibus personae distinguuntur et
 numerantur. Ergo in creatura non invenitur vestigium Trinitatis, sed
 solum unitatis essentiae.


&

さらに、結果が表現するのは原因に他ならない。
しかし、被造物の原因であるという性格は、ペルソナがそれによって区別され数
 えられるところの関係にではなく、共通本性に属する。
ゆえに、被造物の中に三位一体の痕跡は見出されず、ただ本質の一性の痕跡が見
 出されるだけである。

\\



 Sed contra est quod Augustinus dicit, VI {\itshape de
 Trin}., quod {\itshape Trinitatis vestigium in creatura apparet}.


&
しかし反対に、アウグスティヌスは『三位一体論』6巻で、「三位一体の痕跡は
 被造物において明らかである」と述べている。

\\



{\scshape Respondeo dicendum} quod omnis effectus
 aliqualiter repraesentat suam causam, sed diversimode. Nam aliquis
 effectus repraesentat solam causalitatem causae, non autem formam eius,
 sicut fumus repraesentat ignem, et talis repraesentatio dicitur esse
 repraesentatio {\itshape vestigii}; vestigium enim demonstrat motum alicuius
 transeuntis, sed non qualis sit. Aliquis autem effectus repraesentat
 causam quantum ad similitudinem formae eius, sicut ignis generatus
 ignem generantem, et statua Mercurii Mercurium, et haec est
 repraesentatio {\itshape imaginis}. 

&

答えて言わなければならない。全ての結果は、何らかのかたちで自らの原因を表
現するが、それは異なるしかたによる。ある結果は、原因の原因性だけを表現し、
それの形相を表現しない。たとえば、煙が火を表現する場合がそれである。この
ような表現は「痕跡」による表現と言われる。痕跡[=足跡]は、行く人の運動
を示すが、それがどのような人かを示さないからである。またある結果は、原因
の形相の類似にかんして、原因を表現する。生み出された火が生み出す火を表現
する場合や、メルキュリウス[=ヘルメス]の像がメルキュリウスを表現する場
合がそれである。そしてこのような表現は、「似姿」の表現である。

\\


Processiones autem divinarum personarum
 attenduntur secundum actus intellectus et voluntatis, sicut supra
 dictum est, nam filius procedit ut verbum intellectus, spiritus sanctus
 ut amor voluntatis. In creaturis igitur rationalibus, in quibus est
 intellectus et voluntas, invenitur repraesentatio Trinitatis per modum
 imaginis, inquantum invenitur in eis verbum conceptum et amor
 procedens. 

&

ところで、上で述べられたとおり、神のペルソナの発出は、知性と意志の働きに
 即して見出される。すなわち、子は知性の言葉として、聖霊は意志の愛として
 発出する。ゆえに、知性と意志がある理性的被造物において、そのうちに抱か
 れた言葉と発出する愛が見出される点で、三位一体の表現が、似姿のかたちに
 よって見出される。

\\

Sed in creaturis omnibus invenitur repraesentatio Trinitatis
 per modum vestigii, inquantum in qualibet creatura inveniuntur aliqua
 quae necesse est reducere in divinas personas sicut in
 causam. Quaelibet enim creatura subsistit in suo esse, et habet formam
 per quam determinatur ad speciem, et habet ordinem ad aliquid
 aliud. Secundum igitur quod est quaedam substantia creata, repraesentat
 causam et principium, et sic demonstrat personam patris, qui est
 principium non de principio. Secundum autem quod habet quandam formam
 et speciem, repraesentat verbum; secundum quod forma artificiati est ex
 conceptione artificis. Secundum autem quod habet ordinem, repraesentat
 spiritum sanctum, inquantum est amor, quia ordo effectus ad aliquid
 alterum est ex voluntate creantis. 
&

しかし、どの被造物の中にも、原因としての神のペルソナへ還元される必要があ
 る何かが見出される点で、三位一体の表現が、全ての被造物において、痕跡のか
 たちによって見出される。つまり、どの被造物も、自らの存在において自存し、
 種へと限定される形相をもち、他の何かへの秩序をもつ。ゆえに、[全ての被造
 物は]創造された何らかの実体であるという点で、原因ないし根源を表現し、そ
 のようにして、根源であり根源からでない父のペルソナを明示する。さらに、何
 らかの形相ないし種をもつという点で、言葉を表現する。これは、制作物の形相
 が、制作者の懐念に基づいてある、という点に即してである。また、秩序をもつ
 という点では、それが愛であるという点で、聖霊を表現する。これは、結果がも
 つ他の何かへの秩序は、創造者の意志に基づくからである。

\\


Et ideo dicit Augustinus, in VI Lib.~{\itshape de Trin}., quod vestigium
 Trinitatis invenitur in unaquaque creatura, secundum quod {\itshape
 unum aliquid est}, et secundum quod {\itshape aliqua specie formatur},
 et secundum quod {\itshape quendam ordinem tenet}. Et ad haec etiam
 reducuntur illa tria, {\itshape numerus, pondus et mensura}, quae
 ponuntur {\itshape Sap}.~{\scshape xi}, nam {\itshape mensura} refertur
 ad substantiam rei limitatam suis principiis, {\itshape numerus} ad
 speciem, {\itshape pondus} ad ordinem.Et ad haec etiam
 reducuntur alia tria quae ponit Augustinus, {\itshape modus species et
 ordo}. 

&

以上のことから、アウグスティヌスは『三位一体論』6巻で、三位一体の痕跡は、
 各々の被造物の中に「なにか一つのものである」という点、「何かの種によっ
 て形成される」という点、「ある秩序をもつ」点で見出されると言う。
そして、これらに、あの『知書』9章にある三つのもの、「数、重さ、尺度」が
 還元される。つまり「尺度」は、自らの根源によって制限された事物の実体へ、
 「数」は種へ、「重さ」は秩序へと還元される。
また、アウグスティヌスが言う「限度、形象、秩序」という別の三つのものも、
 これらへと還元される。

\\



-- Et ea quae ponit in libro {\itshape Octoginta trium
 Quaest}.~{\itshape quod constat, quod discernitur, quod congruit},
 constat enim aliquid per suam substantiam, discernitur per formam,
 congruit per ordinem. -- Et in idem de facili reduci possunt quaecumque
 sic dicuntur.


&

また、『八十三問題』で「立つもの、区別されるもの、集まる
 もの」\footnote{43問題}と述べているものも、[この同じものへ還元される。]
 なぜなら、あるものは、自らの実体によって立ち、形相によって区別され、秩
 序によって集まるからである。そして、なんであれこのように語られるものど
 もは、同じことへ容易に還元されうる。

\\


{\scshape Ad primum ergo dicendum} quod repraesentatio
 vestigii attenditur secundum appropriata, per quem modum ex creaturis
 in Trinitatem divinarum personarum veniri potest, ut dictum est.


&

第一に対しては、それゆえ、次のように言われるべきである。
すでに述べられたとおり、痕跡の表現は、固有化されたものにしたがって見出され、
 そのようにして、被造物から神のペルソナの三位一体へ至りうる。

\\


{\scshape Ad secundum dicendum} quod creatura est res
 proprie subsistens, in qua est praedicta tria invenire. Neque oportet
 quod in quolibet eorum quae ei insunt, haec tria inveniantur, sed
 secundum ea vestigium rei subsistenti attribuitur.


&

第二に対しては、次のように言われるべきである。固有の意味で被造物であるの
は自存する事物(res subsistens)であり、そこにおいて、前述の三つが見出され
る。しかし、自存する事物に内在するもののどれにおいても、これら三つが見出
される必要はなく、痕跡は、それら[三つ]の点で、自存する事物に帰せられる。

\\


{\scshape Ad tertium dicendum} quod etiam processiones
 personarum sunt causa et ratio creationis aliquo modo, ut dictum est.


&

第三に対しては、次のように言われるべきである。
ペルソナの発出も、何らかの意味で、創造の原因であり根拠である。そしてこれ
 はすでに述べられたとおりである。


\end{longtable}
\newpage
\rhead{a.~8}


\begin{center}
 {\Large {\bf ARTICULUS OCTAVUS}}\\
 {\large UTRUM CREATIO ADMISCEATUR IN OPERIBUS NATURAE ET ARTIS}\\
 {\footnotesize II {\itshape Sent.}, d.~1, q.~1, a.~3, ad 5; a.~4, ad 4;
 {\itshape De Pot.}, q.~3, a.~8; VII {\itshape Metaphys.}, l.~7.}\\
 {\Large 第八項\\創造は、自然や技術の業に混合されるか}
\end{center}

\begin{longtable}{p{21em}p{21em}}


{\huge A}{\scshape d octavum sic proceditur}. Videtur quod
creatio admisceatur in operibus naturae et artis. In qualibet enim
operatione naturae et artis producitur aliqua forma. Sed non producitur
ex aliquo, cum non habeat materiam partem sui. Ergo producitur ex
nihilo. Et sic in qualibet operatione naturae et artis est creatio.


&

第八に対しては、次のように進められる。
創造は、自然や技術の業に混合されると思われる。なぜなら、
自然や技術のどんな働きにおいても、何らかの形相が産出される。ところが、形
 相はどんなものからも産出されない。なぜなら、形相は、自らの部分として、
 質料をもたないからである。ゆえに、形相は無から産出される。この意味で、
 どんな自然や技術の働きの中にも創造がある。

\\


{\scshape 2 Praeterea}, effectus non est potior sua
causa. Sed in rebus naturalibus non invenitur aliquid agens nisi forma
accidentalis, quae est forma activa vel passiva. Non ergo per
operationem naturae producitur forma substantialis. Relinquitur igitur
quod sit per creationem.


&
さらに、結果は、自らの原因ほど強くない。
ところが、自然的事物の中には、能動的形相と受動的形相という附帯形相以外の
 作用者は見出されない。ゆえに、自然の働きによって、実体形相は産出されな
 い。ゆえに、実体形相は創造によってあるということにならざるをえない。

\\


{\scshape 3 Praeterea}, natura facit sibi simile. Sed
quaedam inveniuntur generata in natura non ab aliquo sibi simili, sicut
patet in animalibus generatis per putrefactionem. Ergo eorum forma non
est a natura, sed a creatione. Et eadem ratio est de aliis.

&

さらに、自然は自らに似たものを生み出す。ところで、自然の中には、自らに似
たものからではないかたちで生成したものが見出される。たとえば、腐敗から生
成する動物のように。ゆえに、[腐敗から生じた]それらのものの形相は自然に
よるのではなく創造による。[腐敗以外の]他のものについても同様に論じられ
る。

\\


{\scshape 4 Praeterea}, quod non creatur, non est
creatura. Si igitur in his quae sunt a natura non adiungatur creatio,
sequitur quod ea quae sunt a natura, non sunt creaturae. Quod est
haereticum.

&

さらに、創造されないものは、被造物でない。ゆえに、もし、自然によって存在
 するものに創造が結びつけられないならば、自然によって存在するものは被造
 物でないことになる。これは異端である。

\\

{\scshape Sed contra est} quod Augustinus, {\itshape super
Gen. ad Lit.}, distinguit opus propagationis, quod est opus naturae, ab
opere creationis.

&
しかし反対に、アウグスティヌスは『創世記逐語注解』で、自然の業である繁殖
 の業を、創造の業から区別している。


\\

{\scshape Respondeo dicendum} quod haec dubitatio
inducitur propter formas. Quas quidam posuerunt non incipere per
actionem naturae, sed prius in materia extitisse, ponentes latitationem
formarum. Et hoc accidit eis ex ignorantia materiae, quia nesciebant
distinguere inter potentiam et actum, quia enim formae praeexistunt in
materia in potentia, posuerunt eas simpliciter praeexistere. 


&

答えて言わなければならない。
この疑念が入ってくるのは、形相のせいである。
ある人々は、形相が、自然の作用によって存在し始めるのではなく、形相の隠れ
 ということを考えて、それより先に質料の中に存在していたとした。
こうなったのは、彼らが質料について無知だったからである。というのも、彼ら
 は、可能態と現実態を区別することを知らず、形相が質料の中に可能態におい
 て先在することから、形相が端的に先在すると考えたのだから。


\\


Alii vero
posuerunt formas dari vel causari ab agente separato, per modum
creationis. Et secundum hoc cuilibet operationi naturae adiungitur
creatio. 
Sed hoc accidit eis ex ignorantia formae. Non enim
considerabant quod forma naturalis corporis non est subsistens, sed quo
aliquid est, et ideo, cum fieri et creari non conveniat proprie nisi rei
subsistenti, sicut supra dictum est, formarum non est fieri neque
creari, sed concreata esse. Quod autem proprie fit ab agente naturali,
est compositum, quod fit ex materia. Unde in operibus naturae non
admiscetur creatio, sed praesupponitur ad operationem naturae.


&

他方、別の人々は、形相が、創造というかたちで、離在する作用者によって与え
 られ原因されると考えた。これによれば、自然のどんな働きにも創造が結び
 つけられる。
しかし、彼らがこうなるのは、形相についての無知からである。
彼らは、自然的物体の形相が自存せず、むしろ、それによって何かが存在すると
 ころのものであるということを考察しなかった。それゆえ、上で述べられたとおり、
 生じることや創造されることは、固有の意味では、自存する事物にしか適合し
 ないので、形相に属するのは、生じることや創造されることではなく、共に創
 造されていること(concreata esse)である。
これに対して、固有の意味で、自然的作用者によって作られるものは、質料から生じる複合体である。したがって、自然の業において、創造が混合されることはなく、むしろ
 創造は、自然の働きに前提される。

\\


{\scshape Ad primum ergo dicendum} quod formae incipiunt
esse in actu, compositis factis, non quod ipsae fiant per se, sed per
accidens tantum.


&

第一に対しては、それゆえ、次のように言われるべきである。
形相は、複合体が作られるときに、現実態において存在し始める。しかしそれは、
 形相自体が、それ自体によって生成するのではない。[形相が生成するという
 のは]附帯的な意味においてにすぎない。

\\


{\scshape Ad secundum} dicendum quod qualitates activae
in natura agunt in virtute formarum substantialium. Et ideo agens
naturale non solum producit sibi simile secundum qualitatem, sed
secundum speciem.


&

第二に対しては、次のように言われるべきである。
自然における能動的性質は、実体形相の力において働く。
ゆえに、自然的作用者は、性質の点で自らに似たものを生み出すだけでなく、種
 の点でも似たものを生み出す。

\\


{\scshape Ad tertium dicendum} quod ad generationem
animalium imperfectorum sufficit agens universale, quod est virtus
caelestis, cui assimilantur non secundum speciem, sed secundum analogiam
quandam, neque oportet dicere quod eorum formae creantur ab agente
separato. Ad generationem vero animalium perfectorum non sufficit agens
universale, sed requiritur agens proprium, quod est generans univocum.


&

第三に対しては、次のように言われるべきである。
不完全な動物の生成のためには、普遍的な作用者で十分であり、そのような
 作用者とは天体の力である。それらの動物は、この天体の力に、種の点で類似
 するわけではなく、ある種のアナロギアによって類似する。また、これらの動
 物の形相が、離在する作用者によって創造されると言う必要もない。
他方、完全な動物の生成には、普遍的作用者だけでは十分でなく、固有の作用者
 が必要とされる。この作用者とは一義的な「生む者」である。

\\


{\scshape Ad quartum dicendum} quod operatio naturae non
est nisi ex praesuppositione principiorum creatorum, et sic ea quae per
naturam fiunt, creaturae dicuntur.

&

第四に対しては、次のように言われるべきである。
自然の働きは、創造された諸根源の前提に基づかなければならない。この意味で、
 自然によって生じるものも、被造物と言われる。


\end{longtable}

\end{document}