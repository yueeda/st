\documentclass[10pt]{jsarticle}
\usepackage{okumacro}
\usepackage{longtable}
\usepackage[polutonikogreek,english,japanese]{babel}
\usepackage{latexsym}
\usepackage{color}
\usepackage{schemata}
\usepackage[T1]{fontenc}
\usepackage{lmodern}

%----- header -------
\usepackage{fancyhdr}
\pagestyle{fancy}
\lhead{{\it Summa Theologiae} I-II, q.66}
%--------------------

\bibliographystyle{jplain}

\title{{\bf PRIMA SECUNDAE}\\{\HUGE Summae Theologiae}\\Sancti Thomae
Aquinatis\\{\sffamily QUEAESTIO SEXAGESIMASEXTA}\\DE AEQUALITATE VIRTUTUM}
\author{Japanese translation\\by Yoshinori {\sc Ueeda}}
\date{Last modified \today}

%%%% コピペ用
%\rhead{a.~}
%\begin{center}
% {\Large {\bf }}\\
% {\large }\\
% {\footnotesize }\\
% {\Large \\}
%\end{center}
%
%\begin{longtable}{p{21em}p{21em}}
%
%&
%
%
%\\
%
%\end{longtable}
%\newpage

\begin{document}

\maketitle
\thispagestyle{empty}

\begin{center}
{\LARGE 『神学大全』第二部の一}\\
{\Large 第六十六問\\徳の等しさについて}
\end{center}

\begin{longtable}{p{21em}p{21em}}

Deinde considerandum est de aequalitate virtutum. Et circa hoc
quaeruntur sex. 

\begin{enumerate}
 \item utrum virtus possit esse maior vel minor.
 \item utrum omnes virtutes simul in eodem existentes, sint aequales.
 \item de comparatione virtutum moralium ad intellectuales.
 \item de comparatione virtutum moralium ad invicem.
 \item de comparatione virtutum intellectualium ad invicem.
 \item de comparatione virtutum theologicarum ad invicem.
\end{enumerate}

&

次に徳の等しさについて考察されるべきである。これを巡って六つのことが問われる。

\begin{enumerate}
 \item 徳はより大きくあるいはより小さいことがありうるか。
 \item 同時に同じものの中に存在する全ての徳は等しいか。
 \item 道徳的の知性的徳への関係について。
 \item 道徳的徳相互の関係について。
 \item 知性的相互の関係について。
 \item 神学的徳相互の関係について。
\end{enumerate}
\end{longtable}

\newpage

\rhead{a.~1}
\begin{center}
{\Large {\bf ARTICULUS PRIMUS}}\\
{\large UTRUM VIRTUS POSSIT ESSE MAIOR VEL MINOR}\\
{\footnotesize III {\itshape Sent.}, d.36, a.3; {\itshape De Malo}, q.2, a.9, ad 8; {\itshape De Virtut.}, q.5, a.3.}\\
{\Large 第一項\\徳はより大きくあるいはより小さいことがありうるか}
\end{center}

\begin{longtable}{p{21em}p{21em}}

 {\scshape Ad primum sic proceditur}. Videtur quod virtus non possit
 esse maior vel minor. Dicitur enim in {\itshape Apoc}.~{\scshape xxi}, quod latera
 civitatis Ierusalem sunt aequalia. Per haec autem significantur
 virtutes, ut Glossa dicit ibidem. Ergo omnes virtutes sunt
 aequales. Non ergo potest esse virtus maior virtute.
 
&

第一項の問題へ議論は以下のように進められる。徳はより大きくあるいはより
小さいことがありえないと思われる。理由は以下の通り。『ヨハネの黙示録』
第21章で、エルサレムの都の辺は等しいと言われている\footnote{「この都は
四角形で、長さと幅が同じであった。天使が物差しで都を測ると、一万二千ス
タディオンあった。長さも幅も高さも同じである。」(21:16)}。そして『註解』
がこの箇所で言うように、このことによって徳が意味されている。ゆえに全て
の徳は等しい。ゆえに、ある徳が別の徳より大きいことはありえない。

\\



2.~{\scshape Praeterea}, omne illud cuius ratio consistit in maximo,
non potest esse maius vel minus. Sed ratio virtutis consistit in
maximo, est enim virtus {\itshape ultimum potentiae}, ut philosophus
dicit in I {\itshape de Caelo}; et Augustinus etiam dicit, in II
{\itshape de Lib.~Arb}., quod {\itshape virtutes sunt maxima bona,
quibus nullus potest male uti}. Ergo videtur quod virtus non possit
esse maior neque minor.

 
&

 さらに、全てその性格が最大のものにおいて成り立つものは、より大きい、
 小さいということがありえない。しかるに徳の性格は最大のものにおいて成
 り立つ。なぜなら徳とは哲学者が『天体論』第1巻で言うように「能力の極地」
 だからである。またアウグスティヌスも『自由意志』第2巻で「徳は最大の善
 であり、だれもそれを悪く使うことができない」と言う。ゆえに徳はより大
 きくもより小さくもありえないと思われる。

\\

 3.~{\scshape Praeterea}, quantitas effectus pensatur secundum
 virtutem agentis. Sed virtutes perfectae, quae sunt virtutes infusae,
 sunt a Deo, cuius virtus est uniformis et infinita. Ergo videtur quod
 virtus non possit esse maior virtute.
 
&

 さらに、結果の量は作用者の徳(ちから)に即して計られる。しかるに完全
 な徳、すなわち注入された徳は神によってあり、その神の力は一様で無限で
 ある。ゆえにある徳が他の徳より大きいことはありえないと思われる。
 

\\



 {\scshape Sed contra, ubicumque} potest esse augmentum et
 superabundantia, potest esse inaequalitas. Sed in virtutibus
 invenitur superabundantia et augmentum, dicitur enim Matth.~{\scshape
 v} : {\itshape Nisi abundaverit iustitia vestra plus quam Scribarum
 et Pharisaeorum, non intrabitis in regnum caelorum;} et {\itshape
 Proverb}.~{\scshape xv} dicitur: {\itshape in abundanti iustitia
 virtus maxima est}. Ergo videtur quod virtus possit esse maior vel
 minor.

 
&

 しかし反対に、増大や過剰があるところにはどこでも不等性がある。しかる
 に徳において過剰や増大がある。たとえば『マタイによる福音書』第5章で次
 のように言われている。「もしあなたたちの正義が律法学者やファイリサイ
 派の人々の正義よりも多くなければ、あなたたちは天の王国に入らないだろ
 う」\footnote{「言っておくが、あなたがたの義が律法学者やファリサイ派
 の人々の義にまさっていなければ、あなたがたは決して天の国に入ることが
 できない。」(5:20)}。また『箴言』第15章では、「豊かな人々において、正
 義は最大の徳である」\footnote{「正しき者の家には多くの富があり/悪し
 き者には収穫のときにも煩いがある。」(15:6)}と言われている。ゆえに、徳
 はより大きくまたより小さくありうる。
 

\\


 {\scshape Respondeo dicendum} quod cum quaeritur utrum virtus una
 possit esse maior alia, dupliciter intelligi potest quaestio. Uno
 modo, in virtutibus specie differentibus. Et sic manifestum est quod
 una virtus est alia maior.
 
&

 解答する。以下のように言われるべきである。ある徳が別の徳よりも大きい
 ことがありうるか、と問われるとき、この問いは二通りに理解されうる。一
 つには、種において異なる諸々の徳においてである。この場合には、ある徳
 が別の徳より大きいことは明らかである。

\\

 Semper enim est potior causa suo effectu, et
 in effectibus, tanto aliquid est potius, quanto est causae
 propinquius. Manifestum est autem ex dictis quod causa et radix
 humani boni est ratio.
 
&

 理由は以下の通りである。常に原因は結果より先であり、そして結果におい
 ては、ある結果が原因に近いほど、より力がある。しかるに上述のことから、
 人間的な善の原因と根が理性であることは明らかである。

\\

 Et ideo prudentia, quae perficit rationem,
 praefertur in bonitate aliis virtutibus moralibus, perficientibus vim
 appetitivam inquantum participat rationem. Et in his etiam tanto est
 una altera melior, quanto magis ad rationem accedit.
 
&

 ゆえに理性を完成させる思慮が、理性を分有する限りにおいて欲求的力を完
 成させる他の道徳的徳よりも、善性において優越する。そして道徳的徳の中
 においても、それが理性に近づくほど、或るものが他のよりも善い。

\\



Unde et iustitia, quae est in voluntate, praefertur aliis virtutibus
 moralibus, et fortitudo, quae est in irascibili, praefertur
 temperantiae, quae est in concupiscibili, quae minus participat
 rationem, ut patet in VII {\itshape Ethic}.
 
&

 したがって、意志の中にある正義は、他の道徳的徳よりも前に置かれるし、
 怒情的部分の中にある勇気は、欲情的部分の中にある節制より前である。そ
 れは『ニコマコス倫理学』第7巻で明らかなとおり、欲情的部分は理性を分有
 するところがより少ないからである。

\\

 Alio modo potest intelligi quaestio in virtute eiusdem speciei. Et
 sic, secundum ea quae dicta sunt supra, cum de intensionibus habituum
 ageretur, virtus potest dupliciter dici maior et minor, uno modo,
 secundum seipsam; alio modo, ex parte participantis subiecti.
 
&

 もう一つの意味では、この問いは同一の種に属する徳において理解されうる。
 この場合、前に習慣の強さについて論じられたとき\footnote{Q.52, a.1.}に
 述べられたことに従えば、徳は二通りに、より大きいやより小さいと言われ
 うる。一つにはそれ自体に即して、もう一つには、分有する基体の側からで
 ある。

\\

 Si igitur secundum seipsam consideretur, magnitudo vel parvitas eius
 attenditur secundum ea ad quae se extendit. Quicumque autem habet
 aliquam virtutem, puta temperantiam, habet ipsam quantum ad omnia ad
 quae se temperantia extendit.
 
&

 ゆえに、もしそれ自体に即して考察されるならば、その大きさや小ささは、
 自らが及んでいくものに即して見出される。しかるに、だれであれ何かの徳
 を持つ人は、例えば節制を持つ人は、それを、節制が及んでいく全てのもの
 に関して持つ。

\\

 Quod de scientia et arte non contingit, non enim quicumque
 est grammaticus, scit omnia quae ad grammaticam pertinent. Et
 secundum hoc bene dixerunt Stoici, ut Simplicius dicit in commento
 praedicamentorum, quod virtus non recipit magis et minus, sicut
 scientia vel ars; eo quod ratio virtutis consistit in maximo.
 
&

 このことは、学知や技術については起こらない。なぜなら、文法かである人
 はだれDも、文法学に属する全てのものを知っているわけではないからであ
 る。そしてこの限りで、ストア派の人々は上手に語ったのであり、たとえば
 シンプリキオスは『カテゴリー論註解』で、徳は、学知や技術のようには、
 より大きいやより小さいを受け取らない、なぜなら徳の性格は最大において
 成り立つから、と言っている。

\\

 Si vero
 consideretur virtus ex parte subiecti participantis, sic contingit
 virtutem esse maiorem vel minorem, sive secundum diversa tempora, in
 eodem; sive in diversis hominibus.
 
&

 しかし、徳が分有する基体の側から考察されるならば、その場合には、徳が
 より大きくあるいはより小さいということが生じる。それは同じ人の中で異
 なる時に即してか、あるいは異なる人の中においてである。

\\

 Quia ad attingendum medium virtutis, quod est secundum rationem
 rectam, unus est melius dispositus quam alius, vel propter maiorem
 assuetudinem, vel propter meliorem dispositionem naturae, vel propter
 perspicacius iudicium rationis, aut etiam propter maius gratiae
 donum, quod unicuique donatur {\itshape secundum mensuram donationis
 Christi}, ut dicitur {\itshape ad Ephes}.~{\scshape iv}.
 
&

その理由は、徳の中間、これは正しい理性に即しているが、に到達するために、
ある人が別の人よりも、より慣れているために、あるいは本性のよりよい態勢
のために、あるいは理性のより鋭い判断のために、あるいはより大きな恩恵の
賜物のために、よりよく態勢付けられているからである。この賜物は、『エフェ
ソの信徒への手紙』第4章で言われているように「キリストの賜物の尺度に従っ
て」\footnote{「しかし、私たち一人一人に、キリストの賜物の秤に従って、
恵みが与えられています。」(4:7)}、一人一人に与えられている。


\\


 Et in hoc deficiebant Stoici, aestimantes nullum esse
 virtuosum dicendum, nisi qui summe fuerit dispositus ad virtutem. Non
 enim exigitur ad rationem virtutis, quod attingat rectae rationis
 medium in indivisibili, sicut Stoici putabant, sed sufficit prope
 medium esse, ut in II {\itshape Ethic}.~dicitur.
 
&

 そしてここにおいてストア派の人々は誤ったのであり、それは彼らが、徳へ
 最高に態勢付けられていなければだれも有徳と言われるべきでないと考えた
 からだった。というのも、徳の性格には、ストア派の人々が考えたように、
 正しい理性の不可分の\footnote{点としての、という意味か。}中間に到達す
 ることは必要でなく、むしろ、『ニコマコス倫理学』第2巻で言われるように、
 中間に近いということで十分だからである。

\\


 Idem etiam indivisibile signum unus propinquius et promptius attingit
 quam alius, sicut etiam patet in sagittatoribus trahentibus ad certum
 signum.
 
&

 さらに、ある人が他の人よりも、同じ不可分の印に、より近くより素早く到
 達するということがある。たとえば射手が特定の印に向けて射るときに明ら
 かなように。

\\



 {\scshape Ad primum ergo dicendum} quod aequalitas illa non est
 secundum quantitatem absolutam, sed est secundum proportionem
 intelligenda, quia omnes virtutes proportionaliter crescunt in
 homine, ut infra dicetur.


&

 第一異論に対しては、それゆえ、以下のように言われるべきである。かの等
 しさは無条件的な量に即したものではなく、比に即して理解されるべきであ
 る。というのも、全ての徳は、後で語られるように、人の中で比例的に増加
 するからである。

\\

 {\scshape Ad secundum dicendum} quod illud ultimum quod pertinet ad
 virtutem, potest habere rationem magis vel minus boni secundum
 praedictos modos, cum non sit ultimum indivisibile, ut dictum est.
 
&

 第二異論に対しては以下のように言われるべきである。徳に属する究極は、
 すでに述べられたとおり、不可分な(点としての)究極ではないので、前述
 の仕方で、より大きい、より小さい善という性格を持ちうる。

\\

 {\scshape Ad tertium dicendum} quod Deus non operatur secundum
 necessitatem naturae, sed secundum ordinem suae sapientiae, secundum
 quam diversam mensuram virtutis hominibus largitur, secundum illud
 {\itshape ad Ephes}.~{\scshape iv}, {\itshape unicuique vestrum data
 est gratia secundum mensuram donationis Christi}.

 &

 第三異論に対しては以下のように言われるべきである。神は本性の必然性に
 よって働かず、自らの知恵の秩序に従って働き、その知恵に従って、徳のさ
 まざまな尺度を人間に恵む。それは『エフェソの信徒への手紙』第4章「あな
 た方の各々に、キリストの賜物の尺度に従って恩恵が与えられた」による。

 
\end{longtable}
\newpage


\rhead{a.~2}
\begin{center}
{\Large {\bf ARTICULUS SECUNDUS}}\\
{\large UTRUM OMNES VIRTUTES SIMUL IN EODEM EXISTENTES, SINT AEQUALES}\\
{\footnotesize II {\itshape Sent.}, d.42, q.2, a.5, ad 6; III, d.36, a.4; {\itshape De Malo}, q.2, a.9, ad 8; {\itshape De Virtut.}, q.5, a.3.}\\
{\Large 第二項\\同じ人の中に同時にある全ての徳は等しいか}
\end{center}

\begin{longtable}{p{21em}p{21em}}

{\scshape Ad secundum sic proceditur}. Videtur quod non omnes virtutes
in uno et eodem sint aequaliter intensae. Dicit enim apostolus, I
{\itshape ad Cor}.~{\scshape vii}, {\itshape unusquisque habet
proprium donum a Deo, alius quidem sic}, alius autem sic. Non esset
autem unum donum magis proprium alicui quam aliud, si omnes virtutes
dono Dei infusas quilibet aequaliter haberet. Ergo videtur quod non
omnes virtutes sint aequales in uno et eodem.


&

 第二項の問題へ、議論は以下のように進められる。一人の同じ人の中にある
 全ての徳は等しく強いわけではないと思われる。理由は以下の通り。使徒は
 『コリントの信徒への手紙一』第7章で「一人一人が神から固有の賜物をもらっ
 ている。ある人はこれを、別の人はあれを」\footnote{「私としては、皆が
 私のようであってほしい。しかし、人はそれぞれ神から賜物をいただいてい
 るので、人によって生き方が違います。」(7:7)}と言っている。しかしもし
 だれもが神の賜物によって注入された全ての徳を等しく持っていたとしたら、
 一つの賜物が、別の人よりも或る人に固有であることはなかっただろう。ゆ
 えに全ての徳が一人の同一の人において等しいことはないと思われる。

\\



2.~{\scshape Praeterea}, si omnes virtutes essent aeque intensae in
uno et eodem, sequeretur quod quicumque excederet aliquem in una
virtute, excederet ipsum in omnibus aliis virtutibus. Sed hoc patet
esse falsum, quia diversi sancti de diversis virtutibus praecipue
laudantur; sicut Abraham de fide, Moyses de mansuetudine, Iob de
patientia. Unde et de quolibet confessore cantatur in Ecclesia,
{\itshape non est inventus similis illi, qui conservaret legem
excelsi}; eo quod quilibet habuit praerogativam alicuius virtutis. Non
ergo omnes virtutes sunt aequales in uno et eodem.


&

 さらに、もし全ての徳が一人の同一の人の中で等しく強かったならば、一つ
 の徳においてだれかを超える人はだれでも、他のすべての徳においてその人
 を超えただろう。しかしこれは明らかに偽である。なぜなら、さまざまな聖
 人がさまざまな徳について特に賞賛されているからである。たとえばアブラ
 ハムは信仰について、モーゼは温和について、ヨブは忍耐について、という
 ように。このことから教会の中でどの告白者についても次のように歌われて
 いる。「教会の法を守った人に似ている人は見出されなかった」。これは、
 どの人もある徳に属する特権を持っていたからである。ゆえに全ての徳が、
 一人の同じ人の中で等しいわけではない。

\\



3.~{\scshape Praeterea}, quanto habitus est intensior, tanto homo
secundum ipsum delectabilius et promptius operatur. Sed experimento
patet quod unus homo delectabilius et promptius operatur actum unius
virtutis quam actum alterius. Non ergo omnes virtutes sunt aequales in
uno et eodem.

&

 さらに、習慣がより強いほど、人はそれに即して喜びを感じ、より速く働く。
 しかし、人がある徳の作用よりも別の徳の作用の方をより喜んで迅速に行う
 ことは、経験が示すとおりである。ゆえに全ての徳が、一人の同一の人にお
 いて等しいことはない。

\\



{\scshape Sed contra est} quod Augustinus dicit, in VI {\itshape de
Trin}., quod {\itshape quicumque sunt aequales in fortitudine,
aequales sunt in prudentia et temperantia}; et sic de aliis. Hoc autem
non esset, nisi omnes virtutes unius hominis essent aequales. Ergo
omnes virtutes unius hominis sunt aequales.


&

 しかし反対に、アウグスティヌスは『三位一体論』第6巻で「勇気において等
 しい人々はみな、思慮と節制においても等しい」と述べているし、それ以外
 の徳についても同様である。このことはもし、一人の人に属する全ての徳が
 等しくなかったならば、なかっただろう\footnote{Dubium: A(2,3,5,7),
 B(2,3,5,7), C(2,3,5,7) ...のように、それぞれが持つ徳の強さは等しくな
 くても、同じx, y, z, sの値を持つことがありうる。また、A(2,2,2,2),
 B(3,3,3,3), C(5,5,5,5)...のように、それぞれの人が持つ徳の強さが、それ
 ぞれの人の中で等しいとしても、このことから、これらの人がx, y, z, sに
 おいて等しいことは帰結しない。}。ゆえに一人の人が持つ全ての徳は等しい。

\\


 {\scshape Respondeo dicendum} quod quantitas virtutum, sicut ex
 dictis patet, potest attendi dupliciter. Uno modo, secundum rationem
 speciei. Et sic non est dubium quod una virtus unius hominis sit
 maior quam alia, sicut caritas fide et spe.

&

 解答する。以下のように言われるべきである。すでに述べられたことから明
 らかなとおり、徳の量は二通りに見出される。一つには種の性格に即してで
 あり、その場合には、一人の人に属する一つの徳が他の徳よりも大きいこと
 は疑いない。たとえば、愛徳は信仰や希望よりも大きい。

\\


 Alio modo potest attendi secundum participationem subiecti, prout
 scilicet intenditur vel remittitur in subiecto. Et secundum hoc,
 omnes virtutes unius hominis sunt aequales quadam aequalitate
 proportionis, inquantum aequaliter crescunt in homine, sicut digiti
 manus sunt inaequales secundum quantitatem, sed sunt aequales
 secundum proportionem, cum proportionaliter augeantur.

&

 もう一つには、基体の分有に即して、すなわち徳が基体において増強された
 り減弱されたりするかぎりにおいて見出される。そしてこの限りでは、一人
 の人の全ての徳は、その人の中で等しく増大するかぎりで、ある種の比の等
 しさにおいて等しい。ちょうど手の指は量において(e.g.それぞれの長さの
 点では)等しくないが、しかし、比例的に長くなるので比に即しては等しい
 ようにである。

\\


 Huiusmodi autem aequalitatis oportet eodem modo rationem accipere,
 sicut et connexionis, aequalitas enim est quaedam connexio virtutum
 secundum quantitatem. Dictum est autem supra quod ratio connexionis
 virtutum dupliciter assignari potest.

&

 ところで、このような等しさの根拠は、結びつきの根拠と同じしかたで理解
 しなければならない。なぜなら、その等しさは量に即した徳の結びつきの一
 種だからである。しかるに、徳の結びつきの根拠は二通りに指定されうるこ
 とが語られた。

\\


 Uno modo, secundum intellectum eorum qui intelligunt per has quatuor
 virtutes, quatuor conditiones generales virtutum, quarum una simul
 invenitur cum aliis in qualibet materia. Et sic virtus in qualibet
 materia non potest aequalis dici, nisi habeat omnes istas conditiones
 aequales.



 
&

 一つには、これら四つの徳によって、徳の四つの一般的条件を理解する人々
 の理解に即してである。つまりそれらのうちの一つの徳は、どんな質料にお
 いても、他の徳と一緒に見出される。この意味では、どんな質料においても、
 これら全ての条件を等しく持たないかぎり、徳は等しいと言われえない。

\\


Et hanc rationem aequalitatis virtutum assignat Augustinus, in VI
 {\itshape de Trin}., dicens, {\itshape si dixeris aequales esse istos
 fortitudine, sed illum praestare prudentia; sequitur quod huius
 fortitudo minus prudens sit. Ac per hoc, nec fortitudine aequales
 sunt, quando est illius fortitudo prudentior. Atque ita de ceteris
 virtutibus invenies, si omnes eadem consideratione percurras}.

 &

 そしてこの等しさの性格を、アウグスティヌスは『三位一体論』第6巻で次の
 ように述べて指定している。「もしあなたが、この人々が勇気において等し
 いが、彼は思慮において優ると言ったとしたら、この人の勇気はより少なく
 思慮深いことになる。またこのことによって、彼の勇気がより思慮があると
 きには、勇気においても等しくない。また他の諸徳についても、同じ考察に
 よって全ての徳に目を通すならば、同じことを見出すだろう」。

 \\
 

 Alio modo assignata est ratio connexionis virtutum secundum eos qui
 intelligunt huiusmodi virtutes habere materias determinatas. Et
 secundum hoc, ratio connexionis virtutum moralium accipitur ex parte
 prudentiae, et ex parte caritatis quantum ad virtutes infusas, non
 autem ex parte inclinationis, quae est ex parte subiecti, ut supra
 dictum est.




&

 もう一つの仕方では、徳の結びつきの根拠が、これらの徳が限定された質料
 をもつと理解する人々に即して指定される。そしてこれに従えば、道徳的徳
 の結びつきの根拠は思慮の側から、注入された徳に関しては愛徳の側から理
 解される。しかし、上述の如く、傾向性の側、すなわち基体の側からは理解
 されない。

\\

 Sic igitur et ratio aequalitatis virtutum potest accipi
 ex parte prudentiae, quantum ad id quod est formale in omnibus
 virtutibus moralibus, existente enim ratione aequaliter perfecta in
 uno et eodem, oportet quod proportionaliter secundum rationem rectam
 medium constituatur in qualibet materia virtutum.

&

 ゆえにこのように、徳の等しさの根拠も思慮の側から理解されうるが、それ
 は全ての道徳的徳における形相的なものに関してである。すなわち、一人の
 同一の人において、等しく完成された理性があることによって、徳のどの質
 料においても正しい比に即した中間が打ち立てられなければならない。

\\



 Quantum vero ad id quod est materiale in virtutibus moralibus,
 scilicet inclinationem ipsam ad actum virtutis; potest esse unus homo
 magis promptus ad actum unius virtutis quam ad actum alterius, vel ex
 natura, vel ex consuetudine, vel etiam ex gratiae dono.

&

 他方、道徳的徳における質料的なもの、すなわち徳の作用への傾向性それ自
 体に関しては、一人の人が、ある徳よりも別の徳の作用へと、より迅速であ
 りうるが、その違いは本性、慣れ、さらには恩恵の賜物による。

\\


{\scshape Ad primum ergo dicendum} quod verbum apostoli potest
intelligi de donis gratiae gratis datae, quae non sunt communia
omnibus, nec omnia aequalia in uno et eodem. Vel potest dici quod
refertur ad mensuram gratiae gratum facientis; secundum quam unus
abundat in omnibus virtutibus plus quam alius, propter maiorem
abundantiam prudentiae, vel etiam caritatis, in qua connectuntur omnes
virtutes infusae.

&

 第一異論に対しては、それゆえ、以下のように言われるべきである。使徒の
 言葉は、無償の恩恵\footnote{無償の恩恵(gratia gratis data)と成聖の恩
 恵(gratia gratum faciens)については以下の箇所を参照。\\{\scshape
 Respondeo dicendum} quod, sicut apostolus dicit, ad Rom.~{\scshape
 xiii}, {\itshape quae a Deo sunt, ordinata sunt}. In hoc autem ordo
 rerum consistit, quod quaedam per alia in Deum reducuntur; ut
 Dionysius dicit, in {\itshape Cael.~Hier}. Cum igitur gratia ad hoc
 ordinetur ut homo reducatur in Deum, ordine quodam hoc agitur, ut
 scilicet quidam per alios in Deum reducantur. Secundum hoc igitur
 duplex est gratia. Una quidem per quam ipse homo Deo coniungitur,
 quae vocatur {\bfseries gratia gratum faciens}. Alia vero per quam
 unus homo cooperatur alteri ad hoc quod ad Deum reducatur. Huiusmodi
 autem donum vocatur {\bfseries gratia gratis data}, quia supra
 facultatem naturae, et supra meritum personae, homini conceditur, sed
 quia non datur ad hoc ut homo ipse per eam iustificetur, sed potius
 ut ad iustificationem alterius cooperetur, ideo non vocatur gratum
 faciens. Et de hac dicit apostolus, I {\itshape ad Cor}.~{\scshape
 xii}, unicuique datur manifestatio spiritus ad utilitatem, scilicet
 aliorum. (STI-II, q.111, a.1, c.)}について理解されうる。その恩恵はす
 べての人に共通でないし、一人の同一の人においても等しくない。あるいは、
 これは成聖の恩恵の尺度に言及されていると言われうる。この恩恵に即して、
 ある人は、思慮やそこにおいて全ての注入された徳が結び付けられる愛徳の
 充溢のために、他の人よりも全ての徳において満たされる。

\\



{\scshape Ad secundum dicendum} quod unus sanctus laudatur praecipue
de una virtute, et alius de alia, propter excellentiorem
promptitudinem ad actum unius virtutis, quam ad actum alterius.

&

 第二異論に対しては以下のように言われるべきである。ある成人が特に一つ
 の徳について賞賛され、別の成人はまた別の徳で賞賛されるのは、他の徳の
 作用よりもその徳の作用へ向かう迅速さが卓越しているためにである。

\\


Et per hoc etiam patet responsio ad tertium.


&

そしてこれによって、第三異論への解答も明らかである。


\end{longtable}
\newpage





\rhead{a.~3}
\begin{center}
{\Large {\bf ARTICULUS TERTIUS}}\\
{\large UTRUM VIRTUTES MORALES PRAEEMINEANT INTELLECTUALIBUS}\\
 {\footnotesize II$^{a}$II$^{ae}$, q.23, a.6, ad 1; IV {\itshape Sent.}, d.33, q.3, a.3.}\\
{\Large 第三項\\道徳的徳は知性的徳よりも卓越しているか}
\end{center}

\begin{longtable}{p{21em}p{21em}}
{\scshape Ad tertium sic proceditur}. Videtur quod virtutes morales
praeemineant intellectualibus. Quod enim magis est necessarium, et
permanentius, est melius. Sed virtutes morales sunt {\itshape
permanentiores etiam disciplinis}, quae sunt virtutes intellectuales,
et sunt etiam magis necessariae ad vitam humanam. Ergo sunt
praeferendae virtutibus intellectualibus.


&

 第三項の問題へ、議論は以下のように進められる。道徳的徳は知性的徳より
 も卓越すると思われる。理由は以下の通り。より必要でより永続的なものは、
 より善いものである。しかるに道徳的徳は、「学ばれる事柄」すなわち知性
 的徳「よりも永続的」であり、また、人間の生活により必要である。ゆえに
 それは知性的徳を卓越する。

\\

2.~{\scshape Praeterea}, de ratione virtutis est quod bonum faciat
habentem. Sed secundum virtutes morales dicitur homo bonus, non autem
secundum virtutes intellectuales, nisi forte secundum solam
prudentiam. Ergo virtus moralis est melior quam intellectualis.

&

 さらに、徳の性格には「それを持つ人を善いものにする」ということが含ま
 れる。しかるに道徳的徳に即して人は善い人と言われる。これに対して、知
 性的特に即しては、ただ思慮を除いては、そうは言われない。ゆえに道徳的
 徳は知性的徳よりも善い。

\\


3. {\scshape Praeterea}, finis est nobilior his quae sunt ad
finem. Sed sicut dicitur in VI {\itshape Ethic}., {\itshape virtus
moralis facit rectam intentionem finis; prudentia autem facit rectam
electionem eorum quae sunt ad finem}. Ergo virtus moralis est nobilior
prudentia, quae est virtus intellectualis circa moralia.

&

 さらに、目的は、目的のためにあるものよりも善い。しかるに『ニコマコス
 倫理学』第6巻で言われているように、「道徳的徳が目的の正しい意図を作り、
 思慮が目的に対してあるものどもの正しい選択を作る」。ゆえに道徳的徳は、
 道徳的な事柄を巡る知性的徳である思慮よりも高貴である。
 

\\
{\scshape Sed contra}, virtus moralis est in rationali per
participationem; virtus autem intellectualis in rationali per
essentiam, sicut dicitur in I Ethic. Sed rationale per essentiam est
nobilius quam rationale per participationem. Ergo virtus
intellectualis est nobilior virtute morali.

&

しかし反対に、道徳的徳は分有によって理性的なものの中にあるが、知性的徳
は本質によって理性的なものの中にある。これは『ニコマコス倫理学』第1巻
で言われている。しかるに本質によって理性的なものは分有によって理性的な
ものよりも高貴である。ゆえに知性的徳は道徳的徳よりも高貴である。

\\


{\scshape Respondeo dicendum} quod aliquid potest dici maius vel
minus, dupliciter, uno modo, simpliciter; alio modo, secundum
quid. Nihil enim prohibet aliquid esse melius simpliciter, ut
philosophari quam ditari, quod tamen non est melius secundum quid,
idest necessitatem patienti.

&

 解答する。以下のように言われるべきである。あるものが大きいとか小さい
 とかは、二通りに言われうる。一つは端的に、もう一つはある点に即してで
 ある。たとえば「哲学をすることは金儲けよりも善い」のように、何かが端
 的により善いが、しかしある点に即しては、すなわち困窮している人にとっ
 ての必要に即してはそうでない、ということことを妨げるものはない。

\\

 Simpliciter autem consideratur unumquodque, quando consideratur
secundum propriam rationem suae speciei. Habet autem virtus speciem ex
obiecto, ut ex dictis patet. Unde, simpliciter loquendo, illa virtus
nobilior est quae habet nobilius obiectum.

&

ところで、各々のものは、それの種の固有の性格に即して考察されるときに、
端的に考察される。しかるに、すでに述べられたことから明らかなとおり、徳
は種を対象から持つ。したがって、端的に言うならば、対象がより高貴である
徳が、より高貴である。

\\

 Manifestum est autem quod obiectum rationis est nobilius quam
obiectum appetitus, ratio enim apprehendit aliquid in universali; sed
appetitus tendit in res, quae habent esse particulare. Unde,
simpliciter loquendo, virtutes intellectuales, quae perficiunt
rationem, sunt nobiliores quam morales, quae perficiunt appetitum.

&

 さて、理性の対象が欲求の対象より高貴であることは明らかである。理性は
 何かを普遍において捉えるが、欲求は個別的な存在を持つ事物へと向かうか
 らである。したがって、端的に言うと、理性を完成させる知性的徳は、欲求
 を完成させる道徳的徳よりも高貴である。

\\

 Sed si consideretur virtus in ordine ad actum, sic virtus moralis,
quae perficit appetitum, cuius est movere alias potentias ad actum, ut
supra dictum est, nobilior est. Et quia virtus dicitur ex eo quod est
principium alicuius actus, cum sit perfectio potentiae, sequitur etiam
quod ratio virtutis magis competat virtutibus moralibus quam
virtutibus intellectualibus, quamvis virtutes intellectuales sint
nobiliores habitus simpliciter.

&

 しかし、もし作用への秩序において徳が考察されるならば、上述の如く、他
 の能力を作用へと動かす欲求を完成させる道徳的徳がより高貴である。そし
 て徳は、能力の完成であるのだから、ある作用の根源であることから言われ
 るので、徳の性格が知性的徳よりも道徳的徳に適合することが帰結する。た
 だし、知性的徳の方が端的にはより高貴な習慣である。

\\

 {\scshape Ad primum ergo dicendum} quod virtutes morales sunt magis
 permanentes quam intellectuales, propter exercitium earum in his quae
 pertinent ad vitam communem. Sed manifestum est quod obiecta
 disciplinarum, quae sunt necessaria et semper eodem modo se habentia,
 sunt permanentiora quam obiecta virtutum moralium, quae sunt quaedam
 particularia agibilia.

&

 第一異論に対しては、それゆえ、以下のように言われるべきである。道徳的
 徳が知性的徳よりも永続的なのは、共通の生に属する事柄における、それら
 の徳の実践のためである。しかし、学習の対象は必然的で常に同じようにあ
 るので、ある種の個別的な為されうる事柄である道徳的徳の対象よりも永続
 的であるのは明らかである。
 

\\

 Quod autem virtutes morales sunt magis necessariae ad vitam humanam,
 non ostendit eas esse nobiliores simpliciter, sed quoad hoc. Quinimmo
 virtutes intellectuales speculativae, ex hoc ipso quod non ordinantur
 ad aliud sicut utile ordinatur ad finem, sunt digniores. Hoc enim
 contingit quia secundum eas quodammodo inchoatur in nobis beatitudo,
 quae consistit in cognitione veritatis, sicut supra dictum est.


&

 しかし、道徳的徳が人間の生活により必要であることは、それらが端的に高
 貴であることを示しているわけではなく、むしろある意味においてである。
 これに対して観照的な知性的徳は、有用なものが目的に秩序付けられるよう
 なかたちで、何か別のものへ秩序付けられているわけではないという点で、
 より価値がある。これは、これらの徳において、ある意味で、私たちの中で
 至福が始まっているからである。上述の如く、至福は真理の認識において成
 立するからである。

\\



{\scshape Ad secundum dicendum} quod secundum virtutes morales dicitur
homo bonus simpliciter, et non secundum intellectuales, ea ratione,
quia appetitus movet alias potentias ad suum actum, ut supra dictum
est. Unde per hoc etiam non probatur nisi quod virtus moralis sit
melior secundum quid.


&

 第二異論に対しては以下のように言われるべきである。人が道徳的に即して
 端的に善いと言われ、知性的徳に即してでないのは、上述の如く、欲求が他
 の能力を自らの行為へと動かすからである。したがって、このことから証明
 されるのは、道徳的徳がある意味においてより善いということだけである。

\\


{\scshape Ad tertium dicendum} quod prudentia non solum dirigit
virtutes morales in eligendo ea quae sunt ad finem, sed etiam in
praestituendo finem. Est autem finis uniuscuiusque virtutis moralis
attingere medium in propria materia, quod quidem medium determinatur
secundum rectam rationem prudentiae, ut dicitur in II et VI {\itshape
Ethic}.

&

 第三異論に対しては以下のように言われるべきである。思慮は、目的のため
 あるものを選択することにおいて道徳的徳を導くだけでなく、目的を設定す
 ることにおいても導く。しかるに各々の道徳的徳の目的は、固有の質料にお
 いて中間に到達することである。この中間は、『ニコマコス倫理学』第2巻と
 第6巻で言われているように、思慮の正しい理にそくして決定される。

\end{longtable}
\newpage




\rhead{a.~4}
\begin{center}
{\Large {\bf ARTICULUS QUARTUS}}\\
{\large UTRUM IUSTITIA SIT PRAECIPUA INTER VIRTUTES MORALES}\\
{\footnotesize Supra, a.1; II$^{a}$II$^{ae}$, q.58, a.12; q.123, a.12; q.141, a.8; IV {\itshape Sent.}, d.33, q.3, a.3; {\itshape De Virtut.}, q.5, a.3.}\\
{\Large 第四項\\正義は道徳的徳の中で主要であるか}
\end{center}

\begin{longtable}{p{21em}p{21em}}
{\scshape Ad quartum sic proceditur}. Videtur quod iustitia non sit
praecipua inter virtutes morales. Maius enim est dare alicui de
proprio, quam reddere alicui quod ei debetur. Sed primum pertinet ad
liberalitatem; secundum autem ad iustitiam. Ergo videtur quod
liberalitas sit maior virtus quam iustitia.


&

第四項の問題へ、議論は以下のように進められる。正義が道徳的徳の中で主要
であるわけではないと思われる。理由は以下の通り。だれかに自分のものから
与えることの方が、その人に対する債務を返却することよりもよい。しかるに
前者は気前の善さに属し、後者は正義に属する。ゆえに気前の善さは正義より
も大きい徳であると思われる。

\\



2.~{\scshape Praeterea}, illud videtur esse maximum in unoquoque, quod
est perfectissimum in ipso. Sed sicut dicitur Iac.~{\scshape i},
{\itshape patientia opus perfectum habet}. Ergo videtur quod patientia
sit maior quam iustitia.

&

さらに、各々のものの中でもっとも完全であるものが、もっとも大きいと思わ
れる。しかるに『ヤコブの手紙』第1章「忍耐が完全な業を持つ」
\footnote{「あくまでも忍耐しなさい。そうすれば、何一つ欠けたところのな
い、完全で申し分のない人になります。」(1:4)}と言われているように、忍耐
が正義よりも大きいと思われる。

\\



3.~{\scshape Praeterea}, {\itshape magnanimitas operatur magnum, in
omnibus virtutibus}, ut dicitur in IV {\itshape Ethic}. Ergo
magnificat etiam ipsam iustitiam. Est igitur maior quam iustitia.

&

しかるに、『ニコマコス倫理学』第4巻で言われているように、「高邁は全て
の徳の中で最大の働きをする」。ゆえに、それは正義も大きくする。ゆえにそ
れは正義よりも大きい。

\\


{\scshape Sed contra est} quod philosophus dicit, in V {\itshape
Ethic}., quod {\itshape iustitia est praeclarissima virtutum}.


&

しかし反対に、哲学者は『ニコマコス倫理学』第5巻で「正義は徳の中でもっ
とも際立っている」と言っている。

\\



{\scshape Respondeo dicendum} quod virtus aliqua secundum suam speciem
potest dici maior vel minor, vel simpliciter, vel secundum
quid. Simpliciter quidem virtus dicitur maior, secundum quod in ea
maius bonum rationis relucet, ut supra dictum est.


&

解答する。以下のように言われるべきである。ある徳は、自らの種に即して、
端的に、あるいはある意味において、より大きいやより小さいと言われうる。
端的には、上述の如く、その徳の中により大きな理性の善が輝いている限りに
おいて、徳はより大きいといわれる。

\\



Et secundum hoc, iustitia inter omnes
virtutes morales praecellit, tanquam propinquior rationi. Quod patet
et ex subiecto, et ex obiecto. Ex subiecto quidem, quia est in
voluntate sicut in subiecto, voluntas autem est appetitus rationalis,
ut ex dictis patet.


&

そしてこの限りで、正義はすべての道徳的徳の中で、より理性に近いものとし
て、際立っている。このことは、基体と対象から明らかである。基体からは、
それが基体として意志の中にあるが、意志は、上述の如く、理性的欲求である。

\\



Secundum autem obiectum sive materiam, quia est circa operationes,
quibus homo ordinatur non solum in seipso, sed etiam ad alterum. Unde
{\itshape iustitia est praeclarissima virtutum}, ut dicitur in V
{\itshape Ethic}. 


&

対象ないし質料の点では、人が自分自身の中だけでなく他人に対しても、それ
によって秩序付けられるところの働きを巡ってある。このことから、『ニコマ
コス倫理学』第5巻で言われるように、「正義は徳の中で最も優れている」。

\\


Inter alias autem virtutes morales, quae sunt circa
passiones, tanto in unaquaque magis relucet rationis bonum, quanto
circa maiora motus appetitivus subditur rationi.

&



これに対して、情念を巡ってある他の道徳的徳の中では、各々においてより理
性の善が輝いているほど、より大きなものを巡って欲求の運動が理性に従属す
る。


\\


Maximum autem in his quae ad hominem pertinent, est vita, a qua omnia
alia dependent. Et ideo fortitudo, quae appetitivum motum subdit
rationi in his quae ad mortem et vitam pertinent, primum locum tenet
inter virtutes morales quae sunt circa passiones, tamen ordinatur
infra iustitiam.

&

しかるに、人間に属することで最大のものは命であり、他のすべてはそれに依
存する。ゆえに、死と生に属する事柄において欲求の運動を理性に従属させる
勇気が、情念を巡る道徳的徳の中で第一位を占めるが、しかしそれも正義の下
に位置づけられる。

\\

Unde philosophus dicit, in I {\itshape Rhetoric}., quod {\itshape
necesse est maximas esse virtutes, quae sunt aliis honoratissimae,
siquidem est virtus potentia benefactiva. Propter hoc, fortes et
iustos maxime honorant, haec quidem enim in bello}, scilicet
fortitudo; {\itshape haec autem}, scilicet iustitia, {\itshape et in
bello et in pace utilis est}.

&

 このことから、哲学者は『弁論術』第1巻で以下のように言う。「他の人々に
 とってもっとも賞賛される徳が最大の徳であることは必然である。もしそれ
 が利益をもたらす能力であるとすれば。このため、強い人々や正しい人々を、
 彼らは最大限に賞賛する。一方は戦時において(すなわち勇気が)、もう一
 方は戦時と平時において(すなわち正義が)有用である」。

\\



Post fortitudinem autem ordinatur temperantia, quae subiicit rationi
appetitum circa ea quae immediate ordinantur ad vitam, vel in eodem
secundum numerum, vel in eodem secundum speciem, scilicet in cibis et
venereis. Et sic istae tres virtutes, simul cum prudentia, dicuntur
esse principales etiam dignitate.


&

 さて、勇気の次に節制が位置づけられる。これは、食物やヴィーナス的な事
 柄といった、個人においてあるいは種において命に直接的に秩序付けられる
 ものを巡って欲求を理性に従属させる。このようにして、この三つの徳は、
 思慮と並んで、偉大さにおいてもまた主要だと言われる。

\\



Secundum quid autem dicitur aliqua virtus esse maior, secundum quod
adminiculum vel ornamentum praebet principali virtuti. Sicut
substantia est simpliciter dignior accidente; aliquod tamen accidens
est secundum quid dignius substantia, inquantum perficit substantiam
in aliquo esse accidentali.


&

 さらに、ある意味において、ある徳は、主要な徳に奉仕や装飾を与える限り
 において、より大きいと言われる。ちょうど、実態は端的に附帯性よりも偉
 大だが、しかしある附帯性は、実体を何らかの附帯的存在において完成させ
 るかぎりで、ある意味で実体よりも偉大であるようにである。

\\



{\scshape Ad primum ergo dicendum} quod actus liberalitatis oportet
quod fundetur super actum iustitiae, {\itshape non} enim {\itshape
esset liberalis datio, si non de proprio daret}, ut in II {\itshape
Polit}.~dicitur. Unde liberalitas sine iustitia esse non posset, quae
secernit suum a non suo. Iustitia autem potest esse sine
liberalitate. Unde iustitia simpliciter est maior liberalitate,
tanquam communior, et fundamentum ipsius, liberalitas autem est
secundum quid maior, cum sit quidam ornatus iustitiae, et complementum
eius.


&

 第一異論に対しては、それゆえ、以下のように言われるべきである。気前の
 善さの行為は正義に基づいていなければならない。というのも「もし適切に
 与えなかったならば、気前のよい贈与ではなかっただろう」からである。し
 たがって、自分のものを自分のものでないものから区別する正義がなければ
 気前の善さはありえない。しかし正義は気前の善さがなくてもありうる。し
 たがって、正義は端的に気前の善さよりも、より共通のものとして、より大
 きく、気前の善さの基礎である。これに対して気前の善さは、正義の装飾や
 正義の完成であるという理由で、ある意味でより大きい。

\\



{\scshape Ad secundum dicendum} quod patientia dicitur habere opus
perfectum in tolerantia malorum, in quibus non solum excludit iniustam
vindictam, quam etiam excludit iustitia; neque solum odium quod facit
caritas; neque solum iram, quod facit mansuetudo; sed etiam excludit
tristitiam inordinatam, quae est radix omnium praedictorum.


&

 第二異論に対しては以下のように言われるべきである。忍耐が悪を耐えるこ
 とにおいて完全な業をもつと言われるのは、悪において正義もまた排除する
 不正な復讐を排除し、愛徳が排除する憎しみを排除し、さらに温和が排除す
 る怒りを排除するだけでなく、上述の全ての根源である、無秩序の悲しみを
 排除するからである。

\\



Et ideo in hoc est perfectior et maior, quod in hac materia extirpat
radicem. Non autem est simpliciter perfectior omnibus aliis
virtutibus. Quia fortitudo non solum sustinet molestias absque
perturbatione, quod est patientiae, sed etiam ingerit se eis, cum opus
fuerit. Unde quicumque est fortis, est patiens, sed non convertitur,
est enim patientia quaedam fortitudinis pars.


&

 ゆえに、忍耐はこの質料の中で根を引き抜くことにおいて、より完全でより
 大きい。しかし、端的に全ての徳よりも完全なのではない。なぜなら勇気は、
 忍耐に属する、困難を取り乱すことなく耐えるだけでなく、必要なときには
 それに攻撃するからである。したがって、勇気がある人はだれでも忍耐強い
 が、逆は成り立たない。なぜなら、忍耐は勇気の一部だからである。

 
\\



{\scshape Ad tertium dicendum} quod magnanimitas non potest esse nisi
aliis virtutibus praeexistentibus, ut dicitur in IV {\itshape
Ethic}. Unde comparatur ad alias sicut ornatus earum. Et sic secundum
quid est maior omnibus aliis, non tamen simpliciter.


&

第三異論に対しては以下のように言われるべきである。『ニコマコス倫理学』
第4巻で言われているように、高邁は前もって他の徳が存在していないならば
ありえない。したがって、それは他の徳に対して、それらの装飾として関係す
る。この意味で、高邁はある意味で他のすべての徳よりも大きいが、端的にで
はない。



\end{longtable}
\newpage



\rhead{a.~5}
\begin{center}
{\Large {\bf ARTICULUS QUINTUS}}\\
{\large UTRUM SAPIENTIA SIT MAXIMA INTER VIRTUTES INTELLECTUALES}
{\footnotesize Supra, q.62, a.2, ad 2; VI {\itshape Ethic.}, lect.6.}\\
{\Large 第五項\\知恵は知性的徳の中で最大か}
\end{center}

\begin{longtable}{p{21em}p{21em}}
{\scshape Ad quintum sic proceditur}. Videtur quod sapientia non sit
maxima inter virtutes intellectuales. Imperans enim maius est eo cui
imperatur. Sed prudentia videtur imperare sapientiae, dicitur enim I
{\itshape Ethic}., quod {\itshape quales disciplinarum debitum est
esse in civitatibus, et quales unumquemque addiscere, et usquequo,
haec praeordinat}, scilicet politica, quae ad prudentiam pertinet, ut
dicitur in VI {\itshape Ethic}. Cum igitur inter disciplinas etiam
sapientia contineatur, videtur quod prudentia sit maior quam
sapientia.


&

 第五項の問題へ、議論は以下のように進められる。知恵が知性的徳の中で最
 大ではないと思われる。理由は以下の通り。命令するものは命令されるもの
 よりも大きい。しかるに思慮が知恵に命令すると思われる。なぜなら『ニコ
 マコス倫理学』第1巻で「国家の中で学習がどのようであるべきか、各々の人
 はどのように学ぶべきか、そしてどこまで学ぶべきか、これらをあらかじめ
 命じる」と言われているが、これをするのは政治学であり、政治学は『ニコ
 マコス倫理学』第6巻で言われるように、思慮に属する。ゆえに、学習の中に
 知恵も含まれるのだから、思慮が知恵よりも大きいと思われる。

\\




2.~{\scshape Praeterea}, de ratione virtutis est quod ordinet hominem
ad felicitatem, est enim virtus {\itshape dispositio perfecti ad
optimum}, ut dicitur in VII {\itshape Physic}. Sed prudentia est recta
ratio agibilium, per quae homo ad felicitatem perducitur, sapientia
autem non considerat humanos actus, quibus ad beatitudinem
pervenitur. Ergo prudentia est maior virtus quam sapientia.


&

さらに、徳の性格には人間を幸福へ秩序付けることが含まれる。なぜなら徳と
は『自然学』第7巻で言われるように「最善へと完成されたものの態勢」だか
らである。しかるに思慮は、為されうる事柄の正しい理であり、それによって
インゲンは幸福へと導かれる。他方で知恵は、それによって至福へと到達する
ところの人間的な行為を考察しない。ゆえに思慮は知恵よりも大きな徳である。
 
\\

3.~{\scshape Praeterea}, quanto cognitio est perfectior, tanto videtur
esse maior. Sed perfectiorem cognitionem habere possumus de rebus
humanis, de quibus est scientia, quam de rebus divinis, de quibus est
sapientia, ut distinguit Augustinus in XII {\itshape de Trin}., quia
divina incomprehensibilia sunt, secundum illud {\itshape Iob}
{\scshape xxxvi}, {\itshape ecce Deus magnus, vincens scientiam
nostram}. Ergo scientia est maior virtus quam sapientia.


&

 さらに、認識が完全であるほど、それはより大きいと思われる。しかるに、
 アウグスティヌスが『三位一体論』第12巻で区別しているように、私たちは
 学知が関わる人間的な事柄について、知恵が関わる神の事柄についてよりも
 完全な認識をもつことができる。なぜなら、『ヨブ記』第36章「見よ、偉大
 な神は私たちの学知を打ち負かす」\footnote{「見よ、神は偉大で、私たち
 には知りえない。/その年の数も計り知ることはできない。」(36:26)}によ
 れば、神の事柄は把握されえないからである。ゆえに学知は知恵よりも大き
 い徳である。

\\



4.~{\scshape Praeterea}, cognitio principiorum est dignior quam
cognitio conclusionum. Sed sapientia concludit ex principiis
indemonstrabilibus, quorum est intellectus; sicut et aliae
scientiae. Ergo intellectus est maior virtus quam sapientia.


&

 さらに、原理についての認識は、結論についての認識よりも大きい。しかる
 に知恵は、他の学知と同様に、証明されえない諸原理から結論するが、その
 諸原理には直知がかかわる。ゆえに直知が知恵よりも大きい徳である。

\\




{\scshape Sed contra est} quod philosophus dicit, in VI {\itshape
Ethic}., quod sapientia est {\itshape sicut caput} inter virtutes
intellectuales.


&

 しかし反対に、哲学者は『ニコマコス倫理学』第6巻で、知恵が知性的徳の中
 で「いわば頭のようなもの」であると述べている。

\\




{\scshape Respondeo dicendum} quod, sicut dictum est, magnitudo
virtutis secundum suam speciem, consideratur ex obiecto. Obiectum
autem sapientiae praecellit inter obiecta omnium virtutum
intellectualium, considerat enim causam altissimam, quae Deus est, ut
 dicitur in principio {\itshape Metaphys}.


&

 解答する。以下のように言われるべきである。上述の如く、その種における
 徳の大きさは、対象から考察される。しかるに知恵の対象は、全ての知性的
 徳の対象の中で抜きんでている。なぜならそれは、『形而上学』冒頭で言わ
 れているように、神であるところの最高の原因を考察するからである。
 
\\



 Et quia per causam iudicatur
de effectu, et per causam superiorem de causis inferioribus; inde est
quod sapientia habet iudicium de omnibus aliis virtutibus
intellectualibus; et eius est ordinare omnes; et ipsa est quasi
architectonica respectu omnium.


&

 そして、原因を通して結果について判断され、上位の原因を通して下位の原
 因について判断されるので、知恵が他のすべての知性的徳について判断し、
 それら全ての徳を秩序付けるのもの知恵である。そして知恵は全ての徳につ
 いて棟梁のようなものである。

\\




{\scshape Ad primum ergo dicendum} quod, cum prudentia sit circa res
humanas, sapientia vero circa causam altissimam; impossibile est quod
prudentia sit maior virtus quam sapientia, {\itshape nisi}, ut dicitur
in VI {\itshape Ethic}., {\itshape maximum eorum quae sunt in mundo,
esset homo}. Unde dicendum est, sicut in eodem libro dicitur, quod
prudentia non imperat ipsi sapientiae, sed potius e converso, quia
spiritualis iudicat omnia, et {\itshape ipse a nemine iudicatur}, ut
 dicitur I {\itshape ad Cor}.~{\scshape ii}.


&

 第一異論に対しては、それゆえ、以下のように言われるべきである。思慮は
 人間的な事柄に関わり知恵は最高の原因に関わるので、思慮が知恵よりも大
 きい徳であることは不可能である。ただし、『ニコマコス倫理学』第6巻で言
 われるように、「人間が世界の中に在るものどもの中で最大だというのでな
 い限り」。したがって、同書で言われているように、思慮は知恵に命令せず、
 むしろその逆である。なぜなら、霊的なものが万物を導き、『コリントの信
 徒への手紙一』第2章で言われるように「霊自身はだれからも判断されない」
 \footnote{「霊の人は一切を判断しますが、その人自身は誰からも判断され
 たりしません。」(2:15) }からである。
 
 

\\


 Non enim prudentia habet se intromittere de altissimis, quae
considerat sapientia, sed imperat de his quae ordinantur ad
sapientiam, scilicet quomodo homines debeant ad sapientiam
pervenire. Unde in hoc est prudentia, seu politica, ministra
sapientiae, introducit enim ad eam, praeparans ei viam, sicut
ostiarius ad regem.


&

じっさい、思慮は、知恵が考察する最高のものどもについて関わらず、むしろ
知恵へ秩序付けられるものどもについて、すなわち、どのようにして人間が知
恵に到達すべきかについて命令する。したがって、この点において思慮や政治
学は知恵の従者であり、王に対する門番のように、知恵へと導き、知恵に道を
準備する。
 
\\


{\scshape Ad secundum dicendum} quod prudentia considerat ea quibus
pervenitur ad felicitatem, sed sapientia considerat ipsum obiectum
felicitatis, quod est altissimum intelligibile. Et si quidem esset
perfecta consideratio sapientiae respectu sui obiecti, esset perfecta
felicitas in actu sapientiae. Sed quia actus sapientiae in hac vita
est imperfectus respectu principalis obiecti, quod est Deus; ideo
actus sapientiae est quaedam inchoatio seu participatio futurae
felicitatis. Et sic propinquius se habet ad felicitatem quam
prudentia.

&

 第二異論に対しては以下のように言われるべきである。思慮は、それによっ
 て幸福へ到達するところのものを考察するが、知恵は可知的な最高のもので
 ある幸福の対象自体を考察する。そしてもし、自らの目的に関する知恵の最
 高の考察があったならば、知恵の作用の中に完全な幸福があったであろう。
 しかし、この生における知恵の作用は、主要な対象、すなわち神に関しては
 不完全なので、ゆえに知恵の作用は、将来の幸福のある種の始まりないし分
 有である。この意味で、思慮よりも幸福へより近く関係する。

\\


{\scshape Ad tertium dicendum} quod, sicut philosophus dicit, in I
{\itshape de anima}, {\itshape una notitia praefertur alteri aut ex eo
quod est nobiliorum, aut propter certitudinem}. Si igitur subiecta
sint aequalia in bonitate et nobilitate, illa quae est certior, erit
maior virtus. Sed illa quae est minus certa de altioribus et
maioribus, praefertur ei quae est magis certa de inferioribus rebus.

&

 第三異論に対しては以下のように言われるべきである。哲学者が『デ・アニ
 マ」第1巻で言うように、「ある知が他の知より選好されるのは、それが高貴
 な物についての知だからか、あるいは、その確実性のためかのどちらかであ
 る」。ゆえに、主題が善性と高貴さにおいて等しいならば、より確実な方が
 より大きな徳である。しかし、より高く大きなものについて、より少なく確
 実であるような徳は、下位の諸事物についてのより確実な徳よりも選好され
 る。
 

\\



 Unde philosophus dicit, in II {\itshape de Caelo}, quod magnum est de
rebus caelestibus aliquid posse cognoscere etiam debili et topica
ratione. Et in I de partibus Animal., dicit quod {\itshape amabile est
magis parvum aliquid cognoscere de rebus nobilioribus quam multa
cognoscere de rebus ignobilioribus}.


&

 それゆえ哲学者は『天体論』第2巻で、天の物体について何かを認識すること
 は、それが弱く局所的な理由によるのであっても偉大なことだと言う。また、
 『動物部分論』第1巻では以下のように述べている。「高貴な事物について何
 かを少し認識することは、高貴でない事物について多くを認識することより
 も愛されうることである」。

\\

 Sapientia igitur ad quam pertinet Dei cognitio, homini, maxime in
statu huius vitae, non potest perfecte advenire, ut sit quasi eius
possessio; sed {\itshape hoc solius Dei est}, ut dicitur in I
{\itshape Metaphys}. Sed tamen illa modica cognitio quae per
sapientiam de Deo haberi potest, omni alii cognitioni praefertur.

&

 ゆえに、神を認識することが属する知恵は、とくにこの生の状態においては、
 それを所有するようなかたちで人間に完全に到来することがない。むしろこ
 れは『形而上学』第1巻で言われるように、「ただ神にのみ属する」。しかし、
 知恵を通して神について持たれうるささやかなかの認識は、他のすべての認
 識よりも選好される。

\\




{\scshape Ad quartum dicendum} quod veritas et cognitio principiorum
indemonstrabilium dependet ex ratione terminorum, cognito enim quid
est totum et quid pars, statim cognoscitur quod omne totum est maius
sua parte. Cognoscere autem rationem entis et non entis, et totius et
partis, et aliorum quae consequuntur ad ens, ex quibus sicut ex
terminis constituuntur principia indemonstrabilia, pertinet ad
sapientiam, quia ens commune est proprius effectus causae altissimae,
scilicet Dei. Et ideo sapientia non solum utitur principiis
indemonstrabilibus, quorum est intellectus, concludendo ex eis, sicut
aliae scientiae; sed etiam iudicando de eis, et disputando contra
negantes. Unde sequitur quod sapientia sit maior virtus quam
intellectus.


&

 第四異論に対しては、以下のように言われるべきである。論証不可能な原理
 の真理と認識は、用語の概念に依存する。たとえば全体とは何か、部分とは
 何かが認識されると、ただちに、全て全体は自らの部分より大きいことが認
 識される。しかるに、有と非有の概念や全体と部分の概念、そして有に伴う
 他の事柄の概念は、ちょうど用語に基づくようにして、それらに基づいて論
 証不可能な諸原理が認識されるところのものだが、それらは知恵に属する。
 なぜなら共通の有は、最高原因すなわち神の固有の結果だからである。ゆえ
 に知恵は、直知に属する論証不可能な諸原理を、他の諸学とと同様にそれら
 から結論を導くだけでなく、それらについて判断し、それらを否定する人々
 と論争する場合にも用いる。このことから、知恵が知性よりも大きい徳であ
 ることが帰結する。




\end{longtable}
\newpage





\rhead{a.~6}
\begin{center}
{\Large {\bf ARTICULUS SEXTUS}}\\
{\large UTRUM CARITAS SIT MAXIMA INTER VIRTUTES THEOLOGICAS}\\
{\footnotesize II$^{a}$II$^{ae}$, q.23, a.6.}\\
{\Large 第六項\\愛徳は神学的徳の中で最大か}
\end{center}

\begin{longtable}{p{21em}p{21em}}
{\scshape Ad sextum sic proceditur}. Videtur quod caritas non sit
maxima inter virtutes theologicas. Cum enim fides sit in intellectu,
spes autem et caritas in vi appetitiva, ut supra dictum est; videtur
quod fides comparetur ad spem et caritatem, sicut virtus
intellectualis ad moralem. Sed virtus intellectualis est maior morali,
ut ex dictis patet. Ergo fides est maior spe et caritate.

&
 
 第六項の問題へ、議論は以下のように進められる。愛徳が神学的徳の中で最
 大なのではないと思われる。理由は以下の通り。先に述べられたとおり、信
 仰は知性においてあり、希望と愛徳は欲求的力においてあるので、信仰は希
 望と愛徳に対して、知性的徳が道徳的徳に対するように関係すると思われる。
 しかるにすでに述べられたとおり、知性的徳は道徳的徳よりも大きい。ゆえ
 に信仰が、希望と愛徳よりも大きい。

\\


2.~{\scshape Praeterea}, quod se habet ex additione ad aliud, videtur
esse maius eo. Sed spes, ut videtur, se habet ex additione ad
caritatem, praesupponit enim spes amorem, ut Augustinus dicit in
Enchirid.; addit autem quendam motum protensionis in rem amatam. Ergo
spes est maior caritate.

&

 さらに、何かへの付加によってあるものは、その何かよりも大きいと思われ
 る。しかるに希望は愛徳への付加によってあると思われる。なぜなら、アウ
 グスティヌスが『エンキリディオン』で言うように、希望は愛徳を前提とす
 るからである。また、希望は、愛された事物へのある種の到達する運動を付
 加する。ゆえに希望は愛徳より大きい。

\\

3.~{\scshape Praeterea}, causa est potior effectu. Sed fides et spes
sunt causa caritatis, dicitur enim Matth. I, in Glossa, quod {\itshape
fides generat spem, et spes caritatem}. Ergo fides et spes sunt
maiores caritate.

 &

 さらに、原因は結果より力がある。しかるに信仰と希望は愛徳の原因である。
 なぜなら『マタイによる福音書』第1章の『註解』で、「信仰が希望を生み、
 希望は愛徳を生む」と言われているからである。ゆえに信仰と希望は愛徳よ
 りも大きい。

 \\


{\scshape Sed contra est} quod apostolus dicit, I {\itshape ad
Cor}.~{\scshape xiii}, {\itshape nunc autem manent fides, spes,
caritas, tria haec; maior autem horum est caritas}.


 &

 しかし反対に、使徒は『コリントの信徒への手紙一』第13章で「今残ってい
 るのは信仰、希望、愛徳のこの三つである。これらの中でより大きいのは愛
 徳である」\footnote{「それゆえ、信仰と、希望と、愛、この三つは、いつ
 までも残ります。その中で最も大いなるものは、愛です。」(13:13)}と言っ
 ている。

 \\


 {\scshape Respondeo dicendum} quod, sicut supra dictum est, magnitudo
 virtutis secundum suam speciem, consideratur ex obiecto. Cum autem
 tres virtutes theologicae respiciant Deum sicut proprium obiectum,
 non potest una earum dici maior altera ex hoc quod sit circa maius
 obiectum; sed ex eo quod una se habet propinquius ad obiectum quam
 alia.


 &

 解答する。以下のように言われるべきである。上述の如く、その種に即した
 徳の大きさは、その対象から考察される。しかるに三つの神学的徳は固有の
 対象として神に関係するので、それがより大きな対象に関わることによって、
 これらの一つが他より大きいと言われることはできない。しかし、ある徳が
 他よりも対象により近いことから、そう言われることは可能である。
 

 \\


 Et hoc modo caritas est maior aliis. Nam aliae important in sui
 ratione quandam distantiam ab obiecto, est enim fides de non visis,
 spes autem de non habitis. Sed amor caritatis est de eo quod iam
 habetur, est enim amatum quodammodo in amante, et etiam amans per
 affectum trahitur ad unionem amati; propter quod dicitur I
 Ioan.~{\scshape iv}, {\itshape qui manet in caritate, in Deo manet,
 et Deus in eo}.

 &

 そしてこの意味では、愛徳が他よりも大きい。すんわち他の徳は自らの性格
 の中に対象からの何らかの距離を含んでいる。たとえば信仰は見えないもの
 についてだし、希望は所有されていないものについてあるというように。し
 かし、愛徳の愛はすでに所有されたものについてある。というのも、愛され
 たものはある意味で愛するものの中にあり、さらに愛するものは、愛情によっ
 て、愛されるものとの合一へと引かれるからである。このため、『ヨハネの
 手紙一』第4章で、「愛徳にとどまる者は神にとどまり、神が彼のうちに留ま
 る」\footnote{「私たちは、神が私たちに抱いておられる愛を知り、信じて
 います。神は愛です。愛の内にとどまる人は、神の内にとどまり、神もその
 人の内にとどまってくださいます。」(4:16)}と言われている。

 \\



 {\scshape Ad primum ergo dicendum} quod non hoc modo se habent fides
 et spes ad caritatem, sicut prudentia ad virtutem moralem. Et hoc
 propter duo.


 &

 第一異論に対しては、それゆえ、以下のように言われるべきである。信仰と
 希望は愛徳に対して、ちょうど思慮が道徳的徳に対するように関係する。そ
 してこれは以下の二つの点のためである。
 

 \\


 Primo quidem, quia virtutes theologicae habent obiectum
 quod est supra animam humanam, sed prudentia et virtutes morales sunt
 circa ea quae sunt infra hominem. In his autem quae sunt supra
 hominem, nobilior est dilectio quam cognitio. Perficitur enim
 cognitio, secundum quod cognita sunt in cognoscente, dilectio vero,
 secundum quod diligens trahitur ad rem dilectam. Id autem quod est
 supra hominem, nobilius est in seipso quam sit in homine, quia
 unumquodque est in altero per modum eius in quo est. E converso autem
 est in his quae sunt infra hominem.


 &

 第一に、神学的徳は人間の魂の上にある対象を持つが、思慮と道徳的徳は人
 間の下にあるものどもを巡ってある。しかるに、人間の上にあるものにおい
 ては、認識よりも愛の方が高貴である。なぜなら認識は、認識されたものが
 認識するものの中にあることにおいて完成されるが、愛は愛するものが愛さ
 れる事物へとひかれることにおいて完成される。しかるに人間より上のもの
 は、人間の中にあるよりもそれ自体である方が高貴である。なぜなら、各々
 のものは、他のものの中に、その他のもののあり方によってあるからである。
 そして人間の下にあるものにおいてはこの逆である。

 \\


 Secundo, quia prudentia moderatur motus appetitivos ad morales
 virtutes pertinentes, sed fides non moderatur motum appetitivum
 tendentem in Deum, qui pertinet ad virtutes theologicas; sed solum
 ostendit obiectum. Motus autem appetitivus in obiectum, excedit
 cognitionem humanam; secundum illud {\itshape ad Ephes}.~{\scshape
 iii}, {\itshape supereminentem scientiae caritatem Christi}.

 &

 第二に、思慮は道徳的徳に属する欲求的な運動を制限するが、信仰は神へ傾
 く神学的徳に属する欲求の運動を制限せず、ただ対象を明示するだけである。
 しかるに対象への欲求の運動は人間の認識を越える。これは『エフェソの信
 徒への手紙』第3章「知を優越するキリストの愛徳」\footnote{「人知をはる
 かに超えたキリストの愛を知ることができ、神の満ち溢れるものすべてに向
 かって満たされますように。」(3:19)}による。

 \\



 {\scshape Ad secundum dicendum} quod spes praesupponit amorem eius
 quod quis adipisci se sperat, qui est amor concupiscentiae, quo
 quidem amore magis se amat qui concupiscit bonum, quam aliquid
 aliud. Caritas autem importat amorem amicitiae, ad quam pervenitur
 spe, ut supra dictum est.


 &

 第二異論に対しては以下のように言われるべきである。希望が前提するのは、
 人が、自分がそれに到達することを希望するものへの愛であり、この愛は欲
 情の愛である。実にこの愛によって、善を欲情する人は他のものよりも自己
 を愛する。これに対して愛徳は友愛の愛を意味するのであり、上述の如く、
 希望によってそれへと到達する。

 \\



 {\scshape Ad tertium dicendum} quod causa perficiens est potior
 effectu, non autem causa disponens. Sic enim calor ignis esset potior
 quam anima, ad quam disponit materiam, quod patet esse falsum. Sic
 autem fides generat spem, et spes caritatem, secundum scilicet quod
 una disponit ad alteram.

 &

 第三異論に対しては以下のように言われるべきである。完成させる原因は結
 果よりも強いが、態勢付ける原因はそうでない。もしこの区別がないならば、
 火の熱は、質料をそれへと態勢付ける魂よりも強いことになっただろうが、
 これは偽であることが明らかである。同様に、信仰は希望を生み、希望は愛
 徳を生むが、それは一方が他方を態勢付ける限りにおいてである。

\end{longtable}
\end{document}