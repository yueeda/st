\documentclass[10pt]{jsarticle}
\usepackage{okumacro}
\usepackage{longtable}
\usepackage[polutonikogreek,english,japanese]{babel}
\usepackage{latexsym}
\usepackage{color}

%----- header -------
\usepackage{fancyhdr}
\pagestyle{fancy}
\lhead{{\it Summa Theologiae} I, q.~4}
%--------------------


\bibliographystyle{jplain}
\title{{\bf Prima Pars}\\{\HUGE Summae Theologiae}\\Sancti Thomae
Aquinatis\\Quaestio Quarta\\{\bf De Dei Perfectione}}
\author{Japanese translation\\by Yoshinori {\sc Ueeda}}
\date{Last modified \today}

%%%% コピペ用
%\rhead{a.~}
%\begin{center}
% {\Large {\bf }}\\
% {\large }\\
% {\footnotesize }\\
% {\Large \\}
%\end{center}
%
%\begin{longtable}{p{21em}p{21em}}
%
%&
%
%\\
%\end{longtable}
%\newpage


\begin{document}

\maketitle
\begin{center}
 {\Large 第四問\\神の完全性について}
\end{center}

\begin{longtable}{p{21em}p{21em}}
Post considerationem divinae simplicitatis, de perfectione ipsius Dei
 dicendum est. Et quia unumquodque, secundum quod perfectum est, sic
 dicitur bonum, primo agendum est de perfectione divina; secundo de eius
 bonitate. Circa primum quaeruntur tria. 
\begin{enumerate}
 \item utrum Deus sit perfectus.
 \item utrum Deus sit universaliter perfectus omnium in se
 perfectiones habens.
 \item utrum creaturae similes Deo dici possint.
\end{enumerate}

&

神の単純性の考察の後に、神自身の完全性について語られるべきである。各々
のものは完全である限りにおいて善であると言われるので、第一に神の完全性
について、第二に神の善性について論じられるべきである。第一をめぐっては
三つのことが問われる。
\begin{enumerate}
 \item 神は完全であるか。
 \item 神は、自らのうちに万物の完全性を有するものとして、普遍的に完全か。
 \item 被造物が神に似ていると言われうるか。
\end{enumerate}

\end{longtable}

\newpage
\rhead{a.~1}
\begin{center}
 {\Large {\bf ARTICULUS PRIMUS}}\\
 {\large UTRUM DEUS SIT PERFECTUS}\\
 {\footnotesize I {\itshape SCG.}, c.~28; {\itshape De Verit.}, q.~2,
 a.~3, ad 13; {\itshape Compend.~Theol.}, c.~20; {\itshape De
 Div.~Nom.}, c.~13, l.~1.}\\
 {\Large 第一項\\神は完全か}
\end{center}

\begin{longtable}{p{21em}p{21em}}

{\huge A}{\scshape d primum sic proceditur}. Videtur quod esse
 perfectum non conveniat Deo. Perfectum enim dicitur quasi totaliter
 factum. Sed Deo non convenit esse factum. Ergo nec esse perfectum.

&

第一の問題へ議論は以下のように進められる。完全であるということは、神に
ふさわしくないと思われる。理由は以下の通り。完全なものとは、全く作られ
たもの、という意味で言われる。しかし作られるということは神にふさわしく
ない。ゆえに、完全であるということも神にふさわしくない。

\\


2.~{\scshape Praeterea}, Deus est primum rerum principium. Sed
principia rerum videntur esse imperfecta, semen enim est principium
animalium et plantarum. Ergo Deus est imperfectus.

&

さらに、神は事物の第一根源である。しかるに事物の根源は不完全であるよう
に見える。たとえば、種子は動物と植物の根源である。ゆえに神は不完全であ
る。

\\


3.~{\scshape Praeterea}, ostensum est supra quod essentia Dei est
ipsum esse. Sed ipsum esse videtur esse imperfectissimum, cum sit
communissimum, et recipiens omnium additiones. Ergo Deus est
imperfectus.  

&

さらに、以前に、神の本質は神の存在であることが示された。しかるに存在そ
のものはもっとも不完全なものであるように思われる。なぜならそれはもっと
も共通で、すべてのものの付加を受け取るからである。ゆえに神は不完全であ
る。


\\

 {\scshape Sed contra est} quod dicitur {\itshape Matt}.~{\scshape v},
 {\itshape estote perfecti, sicut et pater vester caelestis perfectus
 est}.
&

しかし反対に、『マタイによる福音書』5章で「あなたたちの天の父が完全で
あるように、あなたたちも完全でありなさい」\footnote{「だから、あなたが
たの天の父が完全であられるように、あなたがたも完全な者となりなさい」
(5:48)}と言われている。

\\


{\scshape Respondeo dicendum quod}, sicut philosophus narrat in XII
{\itshape Metaphys}., quidam antiqui philosophi, scilicet Pythagorici
et Speusippus, non attribuerunt optimum et perfectissimum primo
principio. Cuius ratio est, quia philosophi antiqui consideraverunt
principium materiale tantum, primum autem principium materiale
imperfectissimum est. Cum enim materia, inquantum huiusmodi, sit in
potentia, oportet quod primum principium materiale sit maxime in
potentia; et ita maxime imperfectum.

&

解答する。以下のように言われるべきである。哲学者が『形而上学』第12巻で
語るように、古代のある哲学者たち、すなわち、ピュタゴラス派の人々やスペ
ウシッポスは、最善や最完全ということを、第一根源に帰属させなかった。そ
の理由は、古代の哲学者たちが、質料的な根源だけを探求したことにある。第
一の質料的な根源は、もっとも不完全なものなのである。なぜなら、質料は、
質料である限り、可能態にあるので、質料的な第一根源は、最大限に可能態に
あり、したがって、最大限に不完全でなければならない。


\\

Deus autem ponitur primum principium, non materiale, sed in genere
causae efficientis, et hoc oportet esse perfectissimum. Sicut enim
materia, inquantum huiusmodi, est in potentia; ita agens, inquantum
huiusmodi, est in actu. Unde primum principium activum oportet maxime
esse in actu, et per consequens maxime esse perfectum. Secundum hoc
enim dicitur aliquid esse perfectum, secundum quod est actu, nam
perfectum dicitur, cui nihil deest secundum modum suae perfectionis.

&

ところで、神は、質料的な第一根源ではなく、作出因の類の中で、第一根源だ
とされる。そして、そのようなものは、もっとも完全でなければならない。質
料が質料である限りにおいて可能態にあるように、能動者は能動者である限り
において現実態にある。したがって、能動的な第一根源は、最大限に現実態に
あり、そしてこのことから最大限に完全であることが帰結する。なぜなら、あ
るものが完全であると言われるのは、それが現実態にある限りにおいてなのだ
から。すなわち、完全と言われるのは、自らの完全性の度合いに応じて、何も
欠けることがないものなのである。

\\


{\scshape Ad primum ergo dicendum} quod, sicut dicit Gregorius,
balbutiendo ut possumus, excelsa Dei resonamus, quod enim factum non
est, perfectum proprie dici non potest. Sed quia in his quae fiunt,
tunc dicitur esse aliquid perfectum, cum de potentia educitur in
actum; transumitur hoc nomen perfectum ad significandum omne illud cui
non deest esse in actu, sive hoc habeat per modum factionis, sive non.
&第一異論に対しては、それ故、次のように言われるべきである。グレゴリウ
スが述べているように、「私たちは、できるかぎり、口ごもりながら、神の卓
越したところを語る。作られたのではないものが、完全であるとは、厳密には
言われ得ない」。しかし、作られたものどもにおいて、それが可能態から現実
態に引き出されるときに、完成されたと言われるので、「完全」というこの言
葉が、現実態においてあるということが欠けていないすべてのものを意味する
ように、作られるというかたちでそれを持つかどうかにかかわらず、転用され
るのである。


\\

{\scshape Ad secundum dicendum} quod principium materiale, quod apud
nos imperfectum invenitur, non potest esse simpliciter primum, sed
praeceditur ab alio perfecto. Nam semen, licet sit principium animalis
generati ex semine, tamen habet ante se animal vel plantam unde
deciditur. Oportet enim ante id quod est in potentia, esse aliquid
actu, cum ens in potentia non reducatur in actum, nisi per aliquod ens
in actu.  


&

第二異論に対しては次のように言われるべきである。質料的根源は、私たちの
もとで、不完全なものとして見出されるので、端的に第一のものではありえず、
他の完全なものがそれに先行する。たとえば、種子は、種子から生まれる動物
の根源だが、種以前に、種がそこから落ちてきた動物や植物がある。じっさい、
可能態にあるものの前に、なにか現実態においてあるものがある。なぜなら、
可能態において存在するものは、現実態においてある何かによらなければ、現
実態に引き出されないからである。


\\

{\scshape Ad tertium dicendum} quod ipsum esse est perfectissimum
omnium, comparatur enim ad omnia ut actus. Nihil enim habet
actualitatem, nisi inquantum est, unde ipsum esse est actualitas
omnium rerum, et etiam ipsarum formarum. Unde non comparatur ad alia
sicut recipiens ad receptum, sed magis sicut receptum ad
recipiens. Cum enim dico esse hominis, vel equi, vel cuiuscumque
alterius, ipsum esse consideratur ut formale et receptum, non autem ut
illud cui competit esse.

&

第三異論に対しては次のように言われるべきである。存在そのものは、すべて
のものの中でもっとも完全である。なぜなら、すべてのものに対して現実態と
して関係するからである。じっさい、何ものも、存在する限りにおいてでなけ
れば、現実性を持たない。このことから、存在それ自体は、すべての事物の現
実性であり、形相自体の現実態ですらある。したがって、存在は、他のものに
対して、受け取るものが受け取られるものに対するように関係するのではなく
て、受け取られるものが受け取るものに対するように関係する。私が人間の存
在、馬の存在、あるいは何であれその他のものの存在と言うとき、存在それ自
体は、形相的なものとして、受け取られたものとして考えられているのであり、
存在することがそれに当てはまるところのそれ[つまり本質]と考えられてい
るのではない。

\end{longtable}


\newpage
\rhead{a.~2}

\begin{center}
 {\Large {\bf ARTICULUS SECUNDUS}}\\
 {\large UTRUM IN DEO SINT PERFECTIONES OMNIUM RERUM}\\
 {\footnotesize I {\itshape Sent.}, d.~2, a.~2, 3; I {\itshape SCG.},
 c.~28, 31; II, c.~2; {\itshape De Verit.}, q.~2, a.~1; {\itshape
 Compend.~Theol.}, c.~21, 22; {\itshape de Div.~Nom.}, c.~5, l.~1, 2.}\\
 {\Large 第二項\\神の中にすべての事物の完全性があるか }
\end{center}

\begin{longtable}{p{21em}p{21em}}

 

{\huge A}{\scshape d secundum sic proceditur}. Videtur quod in Deo non
sint perfectiones omnium rerum. Deus enim simplex est, ut ostensum
est. Sed perfectiones rerum sunt multae et diversae. Ergo in Deo non
sunt omnes perfectiones rerum.  


&

第二項の問題へ、議論は以下のように進められる。神の中に万物の完全性があ
るわけがないと思われる。理由は以下の通り。神はすでに示されたように単純
である。しかるに、事物の完全性は数多くさまざまである。ゆえに、神におい
て万物の完全性があるわけではない。

\\

2.~{\scshape Praeterea}, opposita non possunt esse in eodem. Sed
perfectiones rerum sunt oppositae, unaquaeque enim species perficitur
per suam differentiam specificam; differentiae autem quibus dividitur
genus et constituuntur species, sunt oppositae. Cum ergo opposita non
possint simul esse in eodem, videtur quod non omnes rerum perfectiones
sint in Deo.  


&

さらに、対立するものは同一のものの中に存在しえない。しかるに諸事物の完
全性は対立する。なぜなら、各々の種はみずからの種差によって完成されるが、
類がそれによって分けられ、それによって種が構成されるところの種差は対立
するからである。ゆえに、対立するものは同一のものの中に同時にありえない
ので、万物の完全性が神の内にあるのではないと思われる。

\\

3.~{\scshape Praeterea}, vivens est perfectius quam ens, et sapiens
quam vivens, ergo et vivere est perfectius quam esse, et sapere quam
vivere. Sed essentia Dei est ipsum esse. Ergo non habet in se
perfectionem vitae et sapientiae, et alias huiusmodi perfectiones.

&

さらに、生きるものは存在するものよりも完全であり、知恵あるものは生きる
ものよりも完全である。ゆえに、生きることは存在することよりも完全であり、
知恵あることは生きることよりも完全である。しかるに神の本質は神の存在で
ある。ゆえに、自らの中に生命や知恵やそのほかの完全性をもつわけではない。

\\


{\scshape Sed contra est} quod dicit Dionysius, cap.~{\scshape v}
 {\itshape de Div.~Nom.}, quod {\itshape Deus in uno existentia omnia
 praehabet}.


&

しかし反対に、ディオニュシウスは『神名論』第5章で、神は一つのものにお
いて存在するすべてをあらかじめもつ、と述べている。

\\

{\scshape Respondeo dicendum} quod in Deo sunt perfectiones omnium
rerum. Unde et dicitur universaliter perfectus, quia non deest ei
aliqua nobilitas quae inveniatur in aliquo genere, ut dicit
Commentator in V {\itshape Metaphys}. Et hoc quidem ex duobus
considerari potest.


&

解答する。以下のように言われるべきである。解答する。以下のように言われ
るべきである。神において、万物の完全性が存在する。このことから、神はま
た普遍的に完全だと言われる。なぜなら、『形而上学』第5巻において注釈家
が述べるように、どんな類の中に見いだされるどんな高貴さ(≒完全性)も、
神に欠けていないからである。そしてこのことは、二つのことから考察されう
る。

\\

Primo quidem, per hoc quod quidquid perfectionis est in effectu,
oportet inveniri in causa effectiva, vel secundum eandem rationem, si
sit agens univocum, ut homo generat hominem; vel eminentiori modo, si
sit, agens aequivocum, sicut in sole est similitudo eorum quae
generantur per virtutem solis.  Manifestum est enim quod effectus
praeexistit virtute in causa agente, praeexistere autem in virtute
causae agentis, non est praeexistere imperfectiori modo, sed
perfectiori; licet praeexistere in potentia causae materialis, sit
praeexistere imperfectiori modo, eo quod materia, inquantum huiusmodi,
est imperfecta; agens vero, inquantum huiusmodi, est perfectum. Cum
ergo Deus sit prima causa effectiva rerum, oportet omnium rerum
perfectiones praeexistere in Deo secundum eminentiorem modum. Et hanc
rationem tangit Dionysius, cap.~{\scshape v} {\itshape de Div.~Nom}.,
dicens de Deo quod {\itshape non hoc quidem est, hoc autem non est,
sed omnia est, ut omnium causa}.

&

第一に、結果の中にある完全性に属するものはなんであれ、それを作り出す原
因の中に、同一の性格に従って見いだされるか、あるいは、よりすぐれたかた
ちで見いだされるかのどちらかだ、ということによってである。前者の場合は、
その原因が一義的作用者の場合であり、たとえば、人間が人間を生むようなと
きである。後者は、異義的作用者の場合であり、太陽の中に、太陽の力によっ
て生み出されたものの類似が存在するようなときである。結果が、能動的な原
因の中に、力において先在することは明らかである。ところで、能動的原因の
力のなかに先在することは、不完全なかたちで先在することではなく、むしろ、
より完全な形で先在することである。ただし、質料的原因の能力の中に先在す
ることは、より不完全なかたちで先在することである。なぜなら、質料は、質
料である限りにおいて、不完全なものだから。これに対して、作用者(能動者)
は、作用者である限りにおいて、完全なものである。ゆえに、神は、諸事物を
作出する第一原因だから、万物の完全性が、神の内に、より優れたかたちで先
在するはずである。ディオニュシウスが『神名論』第5章で、神について「こ
れであり、これでない、というのではなく、万物の原因として万物である」と
言うとき、彼はこの点に触れている。

\\


Secundo vero, ex hoc quod supra ostensum est, quod Deus est ipsum esse
per se subsistens, ex quo oportet quod totam perfectionem essendi in
se contineat. Manifestum est enim quod, si aliquod calidum non habeat
totam perfectionem calidi, hoc ideo est, quia calor non participatur
secundum perfectam rationem, sed si calor esset per se subsistens, non
posset ei aliquid deesse de virtute caloris. Unde, cum Deus sit ipsum
esse subsistens, nihil de perfectione essendi potest ei deesse.

&

第二に、上で示されたこと、すなわち、神が自らによって自存する存在そのも
のであるということから(万物の完全性が神の中にあることが考えられる)。
つまりこのことから、神は、存在の全完全性を自らの中に含むことになる。も
し、ある熱いものが、熱いものの全完全性をもたないならば、それは、その熱
いものに、熱が、完全な性格において分有されていないからである。しかし、
もし、熱が、それ自身によって自存したならば、その熱に、熱の力の何かが欠
けるということはありえなかっただろう。したがって、神は、自存する存在そ
のものなので、存在の完全性のうちの何も欠けていない。

\\


Omnium autem perfectiones pertinent ad perfectionem essendi, secundum
hoc enim aliqua perfecta sunt, quod aliquo modo esse habent. Unde
sequitur quod nullius rei perfectio Deo desit. Et hanc etiam rationem
tangit Dionysius, cap.~{\scshape v} {\itshape de Div.~Nom.}, dicens
quod Deus {\itshape non quodammodo est existens, sed simpliciter et
incircumscripte totum in seipso uniformiter esse praeaccipit, et
postea subdit quod ipse est esse subsistentibus}.

&

ところで、万物の完全性は、存在の完全性に属する。なぜなら、あるものが完
全であるのは、何らかのかたちで、それが存在を持つからなのだから。このこ
とから、どの事物の完全性も、神に欠けていないことが帰結する。ディオニュ
シウスが、『神名論』第5章で「神は、なんらかのかたちで存在するものでは
なく、端的に、限られることなく、自らの内に一様に全存在を予め保持する
」と言い、後に、「神は、自存するものたちにとっての存在である」と述べる
とき、彼は、この理由に触れている。

\\


{\scshape Ad primum ergo dicendum} quod, sicut sol, ut dicit
Dionysius, cap.~{\scshape v} {\itshape de Div.~Nom}., {\itshape
sensibilium substantias et qualitates multas et differentes, ipse unus
existens et uniformiter lucendo, in seipso uniformiter praeaccipit;
ita multo magis in causa omnium necesse est praeexistere omnia
secundum naturalem unionem}. Et sic, quae sunt diversa et opposita in
seipsis, in Deo praeexistunt ut unum, absque detrimento simplicitatis
ipsius.

&

第一異論に対しては、それゆえ、次のように言わなければならない。ディオニュ
シウスが、『神名論』第5章で述べるように、「太陽が、一つの存在するもの
でありながら、一様に照らすことによって、可感的なものがもつ多くの異なる
実体と性質を、自らの中に一様に予め持つように、更にも増して、万物の原因
の中には、必然的に、本性的な一性によって、万物が先在する」。このように、
それ自身においてはさまざまで対立するものが、神において、神の単純性を損
なうことなく、ひとつのものとして先在する。

\\



Et per hoc patet solutio ad secundum.

&

このことによって、第二異論に対する解答は明らかである。

\\

{\scshape Ad tertium dicendum} quod, sicut in eodem capite idem
Dionysius dicit, licet ipsum esse sit perfectius quam vita, et ipsa
vita quam ipsa sapientia, si considerentur secundum quod distinguuntur
ratione, tamen vivens est perfectius quam ens tantum, quia vivens
etiam est ens; et sapiens est ens et vivens. Licet igitur ens non
includat in se vivens et sapiens, quia non oportet quod illud quod
participat esse, participet ipsum secundum omnem modum essendi, tamen
ipsum esse Dei includit in se vitam et sapientiam; quia nulla de
perfectionibus essendi potest deesse ei quod est ipsum esse
subsistens.

&

第三異論に対しては、次のように言わなければならない。同じ章でディオニュ
シウスが述べるように、概念(ratio)によって区別されるかぎりで考察され
たならば、存在そのものは生命よりも完全であり、生命そのものは知恵そのも
のよりも完全だが、生きているものは存在しているだけのものよりも完全であ
る。なぜなら、生きているものは、存在しているものでもあるのだから。また、
知恵あるものは、存在しているものでありかつ生きているものである。ゆえに、
存在するものは、自らの内に生きているものと知恵あるものを含まないにして
も、というのは、存在を分有するものは、存在のすべてのあり方に従ってそれ
を分有するとは限らないからなのだが、しかし、神の存在そのものは、自らの
内に生命と知恵を含む。なぜなら、自存する存在そのものであるものから、存
在の完全性に属する何かが欠けることはありえないからである。


\end{longtable}

\newpage
\rhead{a.~3}

\begin{center}
 {\Large {\bf ARTICULUS TERTIUS}}\\
 {\large UTRUM ALIQUA CREATURA POSSIT ESSE SIMILIS DEO}\\
 {\footnotesize I {\itshape Sent.}, d.~48, q.~1, a.~1; II, d.~16, q.~1,
 a.~1, ad 3; I {\itshape SCG.}, c.~29; {\itshape De Verit.}, q.~2,
 a.~11; q.~3, a.~1, ad 9; q.~23, a.~7, ad 9 sqq.; {\itshape De Pot.},
 q.~7, a.~7; {\itshape De Div.~Nom.}, c.~9, lect.~3.}\\
 {\Large 第三項\\なんらかの被造物が神に似たものでありうるか }
\end{center}

\begin{longtable}{p{21em}p{21em}}
{\huge A}{\scshape d tertium sic proceditur}. Videtur quod nulla
creatura possit esse similis Deo. Dicitur enim in Psalmo, {\itshape
non est similis tui in diis, Domine}. Sed inter omnes creaturas,
excellentiores sunt quae dicuntur dii participative. Multo ergo minus
aliae creaturae possunt dici Deo similes.


&

第三項の問題へ、議論は以下のように進められる。どんな被造物も神に似てい
ることは不可能だと思われる。理由は以下の通り。『詩篇』に、「神々の中に、
あなたに似ているものはありません、主よ」\footnote{「主よ、あなたのよう
な神は神々のうちになく、あなたの御業に並ぶものはありません」(86:8)}と
言われているからである。ところで、すべての被造物の中で、より優れたもの
たちが、分有のかたちで、神々と言われている。ゆえに、ましてや、それ以外
の被造物が神に似ているとは言われえない。

\\

2.~{\scshape Praeterea}, similitudo est comparatio quaedam. Non est
 autem comparatio eorum quae sunt diversorum generum; ergo nec
 similitudo, non enim dicimus quod dulcedo sit similis albedini. Sed
 nulla creatura est eiusdem generis cum Deo, cum Deus non sit in
 genere, ut supra ostensum est. Ergo nulla creatura est similis Deo.

 &

さらに、類似とは、一種の比較である。しかるに、異なる類に属するものども
を比較することはできず、それゆえ、類似もありえない。たとえば、私たちは、
甘さが白さに似ているとは言わない。しかるに、上で示されたとおり、神は類
の中にないから、どんな被造物も神と同じ類に属さない。ゆえに、どんな被造
物も神に似ることはない。

\\

3.~{\scshape Praeterea}, similia dicuntur quae conveniunt in
 forma. Sed nihil convenit cum Deo in forma, nullius enim rei essentia
 est ipsum esse, nisi solius Dei. Ergo nulla creatura potest esse
 similis Deo.

&

さらに、似ているものとは、形相において一致するもののことである。しかる
に、神と形相において一致するものはない。なぜなら、ただ神ひとりを除いて、
事物の本質がそれの存在であるものはないからである。ゆえに、どんな被造物
も神に似ることはありえない。

\\

4.~{\scshape Praeterea}, in similibus est mutua similitudo, nam simile
est simili simile. Si igitur aliqua creatura est similis Deo, et Deus
erit similis alicui creaturae. Quod est contra id quod dicitur Isaiae
{\scshape xl}, {\itshape cui similem fecistis Deum}?

&

さらに、似ているものどもにおいては、相互的な類似がある。すなわち、似る
とは、似ているものに似るのである。ゆえに、もしある被造物が神に似ていた
ら、神も何らかの被造物に似ていることになろう。これは、『イザヤ書』40章
「あなた方は、神を何に似ているものとしたのか」\footnote{「お前たちは、
神を誰に似せ、どのような像に仕立てようというのか。」(40:18)}と言われて
いることに反する。

\\

{\scshape Sed contra est quod} dicitur {\itshape Gen}.~{\scshape i},
{\itshape Faciamus hominem ad imaginem et similitudinem nostram}; et I
Ioann.~{\scshape iii}, {\itshape cum apparuerit, similes ei erimus}.

&

しかし反対に、『創世記』1章「人間を、私たちの似姿と類似に向けて創ろう」
\footnote{「我々にかたどり、我々に似せて、人を造ろう。」(1:26)}、『ヨ
ハネの手紙1』3章「それが現れたとき、私たちは彼に似たものとなるであろう」
\footnote{「御子が現れるとき、御子に似たものになるということを知ってい
ます。」(3:2)}と言われている。

\\

{\scshape Respondeo dicendum} quod, cum similitudo attendatur secundum
 convenientiam vel communicationem in forma, multiplex est similitudo,
 secundum multos modos communicandi in forma. Quaedam enim dicuntur
 similia, quae communicant in eadem forma secundum eandem rationem, et
 secundum eundem modum, et haec non solum dicuntur similia, sed
 aequalia in sua similitudine; sicut duo aequaliter alba, dicuntur
 similia in albedine. Et haec est perfectissima similitudo.

&


解答する。以下のように言われるべきである。類似は、形相における一致ない
し共通性に応じて見出されるので、形相の共通性のさまざまなあり方に応じて、
さまざまな類似がある。あるものどもは、同じ形相において、同じ性格と、同
じあり方に従って共通するために、似ていると言われる。これらは、似ている
と言われるだけでなく、その類似において等しいとも言われる。たとえば、二
つの等しく白いものが、白さにおいて似ていると言われる場合がこれである。
そしてこれは、もっとも完全な類似である。

\\

Alio modo dicuntur similia, quae communicant in forma secundum eandem
rationem, et non secundum eundem modum, sed secundum magis et minus;
ut minus album dicitur simile magis albo. Et haec est similitudo
imperfecta.

&

二つ目に似ていると言われるのは、同じ性格で形相において共通するが、同じ
あり方ではなく、より多く、より少なくというかたちで共通する場合である。
たとえば、あまり白くないものが、より白いものに似ていると言われる場合で
ある。そしてこれは、不完全な類似である。

\\

Tertio modo dicuntur aliqua similia, quae communicant in eadem forma,
sed non secundum eandem rationem; ut patet in agentibus non
univocis. Cum enim omne agens agat sibi simile inquantum est agens,
agit autem unumquodque secundum suam formam, necesse est quod in
effectu sit similitudo formae agentis. Si ergo agens sit contentum in
eadem specie cum suo effectu, erit similitudo inter faciens et factum
in forma, secundum eandem rationem speciei; sicut homo generat
hominem. Si autem agens non sit contentum in eadem specie, erit
similitudo, sed non secundum eandem rationem speciei, sicut ea quae
generantur ex virtute solis, accedunt quidem ad aliquam similitudinem
solis, non tamen ut recipiant formam solis secundum similitudinem
speciei, sed secundum similitudinem generis.

&

第三に、あるものが似ていると言われるのは、同一の形相において共通するが、
同一の性格においてではない場合である。これは、一義的でない作用者の場合
に明らかである。すべて作用者は、作用者である限りにおいて、自らに似たも
のを結果として生み出すが、各々のものは自らの形相に従って作用するので、
必然的に、結果の中に、作用者の形相の類似がある。ゆえに、作用者が、自ら
の結果と同じ種に含まれる場合には、為すものと為されるものとの間の類似が、
同一の種の性格においてあることになる。たとえば、人間が人間を生む場合が
それである。他方、作用者が、[結果と]同一の種に含まれない場合には、類
似はあるが、しかし、同一の種の性格における類似ではない。たとえば、太陽
の力によって生み出されるものは、太陽の何らかの類似へ近づくけれども、そ
の結果、太陽の形相を種の類似に従って受け取るのではなく、類の類似に従っ
て受け取る。

\\

Si igitur sit aliquod agens, quod non in genere contineatur, effectus
eius adhuc magis accedent remote ad similitudinem formae agentis, non
tamen ita quod participent similitudinem formae agentis secundum
eandem rationem speciei aut generis, sed secundum aliqualem analogiam,
sicut ipsum esse est commune omnibus. Et hoc modo illa quae sunt a
Deo, assimilantur ei inquantum sunt entia, ut primo et universali
principio totius esse.

&

ゆえに、類の中に含まれない何らかの作用者が存在するならば、それの結果は、
作用者の形相の類似へと遠くから近づくが、作用者の形相の類似を、同一の種
や類の性格において分有することにはならず、むしろ、何らかのアナロギア
(類比)に従って分有する。たとえば、存在そのものは、すべてのものに共通
であるように。このようにして、神によって存在するものは、存在するもので
ある限りにおいて、存在全体の第一で普遍的な根源としての神に類似する。

\\

{\scshape Ad primum ergo dicendum} quod, sicut dicit Dionysius
cap.~{\scshape ix} {\itshape de Div.~Nom}., cum sacra Scriptura dicit
aliquid non esse simile Deo, {\itshape non est contrarium
assimilationi ad ipsum. Eadem enim sunt similia Deo, et dissimilia,
similia quidem secundum quod imitantur ipsum, prout contingit eum
imitari qui non perfecte imitabilis est dissimilia vero, secundum quod
deficiunt a sua causa}; non solum secundum intensionem et remissionem,
sicut minus album deficit a magis albo; sed quia non est convenientia
nec secundum speciem nec secundum genus.

&

第一異論に対しては、それゆえ、次のように言わなければならない。ディオニュ
シウスが『神名論』第9章で述べるように、「聖書が、何かが神に似ていない
と言うとき、神への類似に反するという意味ではない。実際、同じものが、神
に似ていて、かつ似ていないのである。似ているというのは、完全に模倣され
得ないものを、あらん限りに模倣する点においてであり、似ていないのは、自
らの原因に及ばないからである」。及ばないというのは、あまり白くないもの
がより白いものに及ばないように、強度の強弱によってだけでなく、種におい
ても類においても一致しないからである。


\\

{\scshape Ad secundum dicendum} quod Deus non se habet ad creaturas
sicut res diversorum generum, sed sicut id quod est extra omne genus,
et principium omnium generum.  

&

第二異論に対しては、次のように言わなければならない。神は、被造物に対し
て、異なる類に属するものとして関係するのではなく、すべての類の外にある
ものとして、すべての類の根源として、関係する。

\\

{\scshape Ad tertium dicendum} quod non dicitur esse similitudo
creaturae ad Deum propter communicantiam in forma secundum eandem
rationem generis et speciei, sed secundum analogiam tantum; prout
scilicet Deus est ens per essentiam, et alia per participationem.

&

第三異論に対しては、次のように言わなければならない。被造物の神への類似
があると言われるのは、同じ類や種の性格にしたがって、形相において共通す
るためではなく、たんにアナロギアにおいて共通するからである。すなわち、
神は本質による存在者、他のものは分有による存在者として。

\\

{\scshape Ad quartum dicendum} quod, licet aliquo modo concedatur quod
creatura sit similis Deo, nullo tamen modo concedendum est quod Deus
sit similis creaturae, quia, ut dicit Dionysius cap.~{\scshape ix}
{\itshape de Div.~Nom}., in his quae unius ordinis sunt, recipitur
mutua similitudo, non autem in causa et causato, dicimus enim quod
imago sit similis homini, et non e converso. Et similiter dici potest
aliquo modo quod creatura sit similis Deo, non tamen quod Deus sit
similis creaturae.  


&

第四異論に対しては、次のように言わなければならない。被造物が神に似てい
るということは、ある意味において、認められるけれども、神が被造物に似て
いるということは、けっして認められるべきではない。なぜなら、ディオニュ
シウスが『神名論』第9章で言うように、「一つの秩序に属するものどもにお
いては、相互に類似が受け取られるが、原因と原因によって生じたものの間に
は、そのような相互性はない」。実際、私たちは、像が人間に似ているとは言
うが、人間が像に似ているとは言わない。同様に、被造物が神に似るというこ
とはある意味で言うことが可能だが、神が被造物に似ているということは、ど
んな意味でも言うことができない。

\end{longtable}
\end{document}