\documentclass[10pt]{jsarticle} % use larger type; default would be 10pt
%\usepackage[utf8]{inputenc} % set input encoding (not needed with XeLaTeX)
%\usepackage[round,comma,authoryear]{natbib}
%\usepackage{nruby}
\usepackage{okumacro}
\usepackage{longtable}
%\usepqckage{tablefootnote}
\usepackage[polutonikogreek,english,japanese]{babel}
%\usepackage{amsmath}
\usepackage{latexsym}
\usepackage{color}
%----- header -------
\usepackage{fancyhdr}
\pagestyle{fancy}
\lhead{{\it Summa Theologiae} I, q.~46}
%--------------------

\bibliographystyle{jplain}

\title{{\bf PRIMA PARS}\\{\HUGE Summae Theologiae}\\Sancti Thomae
Aquinatis\\{\sffamily QUEAESTIO QUADRAGESIMASEXTA}\\DE PRINCIPIO
DURATIONIS RERUM CREATARUM}
\author{Japanese translation\\by Yoshinori {\sc Ueeda}}
\date{Last modified \today}


%%%% コピペ用
%\rhead{a.~}
%\begin{center}
% {\Large {\bf }}\\
% {\large }\\
% {\footnotesize }\\
% {\Large \\}
%\end{center}
%
%\begin{longtable}{p{21em}p{21em}}
%
%&
%
%\\
%\end{longtable}
%\newpage

\begin{document}

\maketitle

\begin{center}
{\Large 第四十六問\\創造された事物の持続の始原について}
\end{center}

\begin{longtable}{p{21em}p{21em}}

Consequenter considerandum est de principio durationis rerum
 creatarum. Et circa hoc quaeruntur tria. 

\begin{enumerate}
 \item utrum creaturae semper fuerint. 
 \item utrum eas incoepisse sit articulus fidei. 
 \item quomodo Deus dicatur in principio caelum et terram creasse.
\end{enumerate}
&

続いて、創造された事物の持続の始原について考察されるべきである。
これについて、三つのことが問われる。
\begin{enumerate}
 \item 被造物は常に存在したか。
 \item 被造物が存在し始めた、ということは信仰箇条か。
 \item 神が初めに天と地を作った、とはどのような意味で言われるか。
\end{enumerate}

\end{longtable}


\newpage
\rhead{a.~1}

\begin{center}
 {\Large {\bf ARTICULUS PRIMUS\\ 
}}
 {\large UTRUM UNIVERSITAS CREATURARUM SEMPER FUERIT}\\
{\footnotesize II {\itshape Sent.}, d.~1, q.~1, a.~5; II {\itshape
 SCG.}, c.~31, sqq.; {\itshape De Pot.}, q.~3, a.~17; {\itshape
 Quodl.}~III, q.~14, a.~2; {\itshape Compend.~Theol.}, c.~98; VIII
 {\itshape Physic.}, l.~2; I {\itshape de Cael.~et Mund.}, l~6, 29; XII
 {\itshape Metaphys.}, l.~5.}\\
{\Large 第一項\\
被造物の世界は常に存在したか}
\end{center}


\begin{longtable}{p{21em}p{21em}}
{\huge A}{\sc d primum sic proceditur. Videtur} quod universitas creaturarum, quae
{\itshape mundi} nomine nuncupatur,
%\footnote
%{nunc\u{u}po , \={a}vi, \={a}tum, 1 (nuncupassit for nuncupaverit;
%I.v. in the foll.), v. a. nomencapio, to call by name, to call, name
%}
non incoeperit, sed fuerit ab aeterno. Omne
enim quod incoepit esse, antequam fuerit, possibile fuit ipsum esse,
alioquin impossibile fuisset ipsum fieri. Si ergo mundus incoepit esse,
antequam inciperet, possibile fuit ipsum esse. Sed quod possibile est
esse, est materia, quae est in potentia ad esse, quod est per formam, et
ad non esse, quod est per privationem. Si ergo mundus incoepit esse,
ante mundum fuit materia. Sed non potest esse materia sine forma,
materia autem mundi cum forma, est mundus. Fuit ergo mundus antequam
esse inciperet, quod est impossibile.


& 第一に対しては、次のように進められる。「世界」という名で呼ばれている被
造物の宇宙は、始まったのではなく、永遠からあったと思われる。なぜなら、す
べて存在し始めたものは、存在する前に、それ自身であることが可能であった。
そうでなければ、それ自身になること自体が不可能だったであろう。ゆえに、も
し世界が存在し始めたのであれば、存在し始める前に、世界であることが可能で
あった。しかし、存在することが可能なものは質料であり、それは、存在と非存
在に対して可能態にある。存在に対しては形相によって、非存在に対しては欠如
によって。ゆえに、もし世界が存在し始めるとすれば、世界に前には質料があっ
た。しかし、質料は形相なしにありえないが、形相を伴う世界の質料とは世界で
ある。ゆえに、世界は、世界が存在し始める前に存在した。これは不可能である。


\end{longtable}

%\begin{enumerate}
% \item 世界が始まる前、世界になりうる物が先行しなければならない。
% \item 世界になりうる物とは世界の質料である。
% \item 質料は単独で存在できない。
% \item 世界の質料はなんらかの形相をもつ。
% \item 形相を持つ世界の質料は世界である。
% \item 世界が始まる前、世界が存在する。
% \item これは偽。
% \item ゆえに、仮定が誤り。(どの?)
%\end{enumerate}

\begin{longtable}{p{21em}p{21em}}

2.~{\sc Praeterea}, nihil quod habet virtutem ut sit
 semper, quandoque est et quandoque non est, quia ad quantum se extendit
 virtus alicuius rei, tandiu est. Sed omne incorruptibile habet virtutem
 ut sit semper, non enim virtutem habet ad determinatum durationis
 tempus. Nullum ergo incorruptibile quandoque est et quandoque non
 est. Sed omne quod incipit esse, quandoque est et quandoque non
 est. Nullum ergo incorruptibile incipit esse. Sed multa sunt in mundo
 incorruptibilia, ut corpora caelestia, et omnes substantiae
 intellectuales. Ergo mundus non incoepit esse.

&
さらに、常に存在する力を持つものが、あるときに存在し、あるときに存在しな
 いということはない。なぜなら、どんな事物の力も、それが存在するかぎり、
 自らの力を及ぼすからである。しかるに、すべて不可滅的なものは、常に存在
 する力を持つ。なぜなら、持続の限定された時間に対する力をもたないからで
 ある。ゆえに、不可滅的なものはどれも、存在したりしなかったりすることは
 ない。ゆえに、どんな不可滅的なものも、存在し始めることはない。
しかし、天体やすべての知的実体のように、世界の中にはたくさんの不可滅的な
 ものがある。ゆえに、世界が存在し始めたということはない。

\\


3.~{\sc Praeterea}, nullum ingenitum incoepit esse. Sed philosophus probat in I
 {\itshape Physic}., quod materia est ingenita; et in I {\itshape de caelo et mundo}, quod
 caelum est ingenitum. Non ergo universitas rerum incoepit esse.

&

さらに、生み出されたのでないものはどれも、存在し始めたのではない。しかる
 に、哲学者は『自然学』第一巻で、質料が生み出されないことを証明し、『天体
 論』第一巻で、天が生み出されないことを証明している。ゆえに、諸事物の宇宙
 が存在し始めたということはない。


\\

4.~{\sc Praeterea}, vacuum est ubi non est corpus, sed possibile est esse. Sed
 si mundus incoepit esse, ubi nunc est corpus mundi, prius non fuit
 aliquod corpus, et tamen poterat ibi esse, alioquin nunc ibi non
 esset. Ergo ante mundum fuit vacuum, quod est impossibile.

&

さらに、空虚とは、そこに物体が存在せず、しかし存在しうるところのことであ
 る。しかるに、もし世界が存在し始めたのであれば、今、世界という物体が存
 在するところに、その前はどんな物体も存在しなかったのであり、しかしまた、
 そこに物体が存在しえたのである。さもなければ、今そこに世界はなかったで
 あろう。ゆえに、世界の前には空虚があったことになる。これは不可能である。


\\



5.~{\sc Praeterea}, nihil de novo incipit moveri, nisi per hoc quod movens vel
 mobile aliter se habet nunc quam prius. Sed quod aliter se habet nunc
 quam prius, movetur. Ergo ante omnem motum de novo incipientem, fuit
 aliquis motus. Motus ergo semper fuit. Ergo et mobile, quia motus non
 est nisi in mobili.

&


さらに、新たに動き始めることは、動かすものや動かされるものが、今、以前とは違
 う在り方をすることによる。
しかるに、今、以前とは違う在り方をするものは、動いている。
ゆえに、新たに動き始めるすべての運動の前に、なんらかの運動があった。
ゆえに、運動は常にあった。ゆえに、動かされるものも常にあった。
なぜなら、運動は、動かされるものにおいてしか存在しないのだから。

\\


6.~{\sc Praeterea}, omne movens aut est naturale, aut est voluntarium. Sed
 neutrum incipit movere, nisi aliquo motu praeexistente. Natura enim
 semper eodem modo operatur. Unde, nisi praecedat aliqua immutatio vel
 in natura moventis vel in mobili, non incipiet a movente naturali esse
 motus, qui non fuit prius. Voluntas autem absque sui immutatione
 retardat facere quod proponit, sed hoc non est nisi per aliquam
 immutationem quam imaginatur, ad minus ex parte ipsius temporis. Sicut
 qui vult facere domum cras, et non hodie, expectat aliquid futurum
 cras, quod hodie non est; et ad minus expectat quod dies hodiernus
 transeat, et crastinus adveniat; quod sine mutatione non est, quia
 tempus est numerus motus. Relinquitur ergo quod ante omnem motum de
 novo incipientem, fuit alius motus. Et sic idem quod prius.

&

さらに、動者はすべて自然的なものか意志的なものである。しかし、どちらも、
先だって存在するなんらかの運動がなければ、動かし始めない。理由は以下の通
り。まず、自然は常に同様に働く。したがって、動かすものの本性において、あ
るいは、動かされるものにおいて、なんらかの変化が先行しなければ、自然的動
者から、それ以前に存在しなかった運動が存在し始めることはない。次に、意志
は、自らの変化なしに、計画していることをなす事を遅らせるが、しかしこれに
は、少なくとも時間自体の側から、想像されたなんらかの変化がなければならな
い。たとえば、「今日ではなく明日、家を造ろう」と意志している人は、今日は
ないなんらかの未来を、明日、期待している。そして少なくとも、今日という
日が過ぎていって、明日という日がやってくることを期待している。これは、変
化なしにはない。なぜなら、時間は運動の数だからである。ゆえに、新たに始ま
るすべての運動の前に、なんらかの運動があったことが帰結する。ゆえに、前項
と同じ。

\end{longtable}
%◎意志の方が少しわかりにくい。計画したことを遅らせる場合、意志はなにもし
%ていないように見える。しかし、行為Aを、T1からT2にまで遅らせる場合には、
%T1からT2への時間の流れを理解することが必要となる。時間の流れは変化を必要
%とし、そのような変化を想像力において捉えるためには、認識者の中に変化が必
%要となる。
%
%しかし、知的直観によって時間の関係を捉える場合はどうか。
%
%創造の場合、神が意図的に創造を遅らせたとして、遅らせるという行為の中に変
%化があるはずだ、と論じているのだろうか。
%


\begin{longtable}{p{21em}p{21em}}

7.~{\sc Praeterea}, quidquid est semper in principio et semper in fine, nec
 incipere nec desinere potest, quia quod incipit, non est in suo fine;
 quod autem desinit, non est in suo principio. Sed tempus semper est in
 suo principio et fine, quia nihil est temporis nisi nunc, quod est
 finis praeteriti, et principium futuri. Ergo tempus nec incipere nec
 desinere potest. Et per consequens nec motus, cuius numerus tempus est.

&
さらに、始まりにおいて常にあり、終わりにおいて常にあるものは、始まること
 も終わることもありえない。なぜなら、始まるものは、自らの終わりにはなく、
 終わるものは、自らの初めにはないからである。しかし、時間は常に自らの始
 めと終わりにある。なぜなら、時間に属するのは今以外になく、それは現在の終わり
 でありかつ未来の始まりだから。ゆえに、時間は、始まることも終わることもでき
 ない。ゆえに、時間とは運動の数なのだから、運動もまた云々。

\end{longtable}
%◎$000000000000012345678900000000000000000$という数列が、一秒に一個ずつモニ
%ターに出現するような装置を見ているとき、1が出現しているときに9は出現して
%いないし、9が出現しているときに1は出現していない。
%
%時間を構成しているものが「今」だとする。今は常に現在の終わりであるととも
%に未来の始まりである。
%1が出現したとき、それは未来の始まりとして今であり、9が見えたときは、
%現在の終わりとして今である。
%
%1から9までの数列が時間だとする。1が見えたとき、それは未来の始まりとして
%今であり、9が見えたとき、現在の終わりとして今である。
%
%これだけではなんの不思議もないが、「今が今である限り、それは同一のもので
%あり、同一のものは同一の性質(関係)を持つ」という妙な前提を入れると、未
%来の始まりである今、現在の終わりでもあることになり、1以前に時間があること
%になる。同様に、現在の終わりである今、未来の始まりでもあるので、9以後にも
%時間があることになる。
%

\begin{longtable}{p{21em}p{21em}}

8.~{\sc Praeterea}, Deus aut est prior mundo natura tantum, aut duratione. Si
 natura tantum, ergo, cum Deus sit ab aeterno, et mundus est ab
 aeterno. Si autem est prior duratione; prius autem et posterius in
 duratione constituunt tempus, ergo ante mundum fuit tempus; quod est
 impossibile.

&
さらに、神は世界より先であるが、それはたんに本性においてか、あるいは、持
 続においてもそうかのいずれかである。
もし、たんに本性においてであるならば、神は永遠から存在するので、世界も永
 遠から存在する。しかし、持続において先行するならば、持続におけるより先
 より後は時間を構成するので、世界の前に時間があったことになる。これは不
 可能である。

\\



9.~{\sc Praeterea}, posita causa sufficienti, ponitur effectus, causa enim ad
 quam non sequitur effectus, est causa imperfecta, indigens alio ad hoc
 quod effectus sequatur. Sed Deus est sufficiens causa mundi; et
 finalis, ratione suae bonitatis; et exemplaris, ratione suae
 sapientiae; et effectiva, ratione suae potentiae; ut ex superioribus
 patet. Cum ergo Deus sit ab aeterno, et mundus fuit ab aeterno.

&

さらに、十分な原因があるところには、結果もある。なぜなら、結果が伴わない
 原因は不十分な原因であり、結果が伴うために他のものを必要とするような原因
 だからである。しかし神は世界の十分な原因である。すでに述べたことから明ら
 かなとおり、その善性のために目的因であり、その知恵のために範型因であり、
 その能力のために作出因である。ゆえに、神が永遠からあるのだから、世界も
 永遠からあった。


\\



10.~{\sc Praeterea}, cuius actio est aeterna, et effectus
 aeternus. Sed actio Dei est eius substantia, quae est aeterna. Ergo et
 mundus est aeternus.

&
さらに、その働きが永遠であるものの結果も永遠である。しかるに、神の働きは
 神の実体であり、神の実体は永遠である。ゆえに、世界もまた永遠である。

\\



{\sc Sed contra est quod} dicitur Ioan.~{\sc xvii}, {\it clarifica me,
 pater, apud temetipsum, claritate quam habui priusquam mundus fieret};
 et Proverb.~{\sc viii}, {\it dominus possedit me in initio viarum
 suarum, antequam quidquam faceret a principio}.

&

しかし反対に、『ヨハネによる福音書』17節で、「父よ、私を照らしてください。
 あなた自身のもとで、世界が生じる前に私が持っていた明るさによって」
 \footnote {「父よ、今、御前でわたしに栄光を与えてください。世界が作られ
 る前に、わたしがみもとで持っていたあの栄光を」(17:5) イエスの言葉とし
 て。}とあり、また、『箴言』8では「最初になにかを作る前、主は、自らの道の
 始まりにおいて、私を持っていた」\footnote {「主は、その道の初めにわたし
 を作られた。いにしえの御技になお、先だって」(8:22) 「知恵」の言葉とし
 て。}とある。


\\



{\sc Respondeo dicendum} nihil praeter Deum ab aeterno fuisse. Et hoc
 quidem ponere non est impossibile. Ostensum est enim supra quod
 voluntas Dei est causa rerum. Sic ergo aliqua necesse est esse, sicut
 necesse est Deum velle illa, cum necessitas effectus ex necessitate
 causae dependeat, ut dicitur in V {\it Metaphys}. Ostensum est autem
 supra quod, absolute loquendo, non est necesse Deum velle aliquid nisi
 seipsum. Non est ergo necessarium Deum velle quod mundus fuerit
 semper. Sed eatenus mundus est, quatenus Deus vult illum esse, cum esse
 mundi ex voluntate Dei dependeat sicut ex sua causa. Non est igitur
 necessarium mundum semper esse. Unde nec demonstrative probari
 potest. Nec rationes quas ad hoc Aristoteles inducit, sunt
 demonstrativae simpliciter, sed secundum quid, scilicet ad
 contradicendum rationibus antiquorum, ponentium mundum incipere
 secundum quosdam modos in veritate impossibiles. Et hoc apparet ex
 tribus. Primo quidem, quia tam in VIII {\it Physic}. quam in I {\it de
 Caelo}, praemittit quasdam opiniones, ut Anaxagorae et Empedoclis et
 Platonis, contra quos rationes contradictorias inducit. Secundo, quia,
 ubicumque de hac materia loquitur, inducit testimonia antiquorum, quod
 non est demonstratoris, sed probabiliter persuadentis. Tertio, quia
 expresse dicit in I Lib. {\it Topic}., quod quaedam sunt problemata
 dialectica, de quibus rationes non habemus, ut {\it utrum mundus sit
 aeternus}.

&

答えて言わなければならない。神以外のものが永遠からあったことはない。しか
し、このことを措定することは不可能ではない。理由は以下の通り。上で、神の
意志は事物の原因であることが示された。ゆえに、『形而上学』第五巻で述べら
れるように、結果の必然性は原因の必然性に依存するから、何かが必然的である
のは、神がそれを意志することが必然であるのと同様である。しかるに上で、無
条件的に語るならば、神が自己以外のものを意志することは必然でない。ゆえに、
神が、世界が常に存在することを意志することは、必然でない。しかし、神が世
界が存在することを意志する間、世界は存在する。世界の存在は、自らの原因と
しての神の意志に依存するからである。ゆえに、世界が常に存在することは必然
でない。したがって、このこと(世界が常に存在すること)は、論証によって証
明されることもできない。したがって、アリストテレスをこのことへと導く推論
も、端的に論証的ではなく、ある意味においてのみ論証的である。すなわち、真
理において不可能であるようなかたちで世界が始まると措定する古代の哲学者た
ちの理由に反論するために、論証的であるに過ぎない。このことは、三つの点か
ら明らかである。第一に、『自然学』第八巻でも『天体論』第一巻でも、アナク
サゴラスやエムペドクレスやプラトンのような、ある意見を先に置いて、この人
たちに反対して、反論する理由を彼は導いている。第二に、彼がこの題材につい
て述べるときはいつも、古代の人々の証言を導入するが、こうすることは論証す
る人に属することではなく、蓋然的に説得する人に属することである。第三に、
『トピカ』第一巻で、彼ははっきりと次のように述べている。ある弁論的な問題
は、それについて私たちが理由を持っていないようなものであり、たとえば、世
界が永遠かどうか、という問題がそれである。


\\



{\sc Ad primum ergo dicendum} quod, antequam mundus esset, possibile fuit
 mundum esse, non quidem secundum potentiam passivam, quae est materia;
 sed secundum potentiam activam Dei. Et etiam secundum quod dicitur
 aliquid absolute possibile, non secundum aliquam potentiam sed ex sola
 habitudine terminorum, qui sibi non repugnant; secundum quod possibile
 opponitur {\it impossibili}, ut patet per philosophum, in V {\it Metaphys}.

&
第一にたいしては、それゆえ、次のように言われるべきである。
世界が存在する前、世界は存在しえた。しかしそれは受動的能力において、つま
 り質料においてではなく、神の能動的能力においてである。
また、何かが無条件的に可能的だと言われることにおいても、なんらかの能力に
 おいてでなはく、相互に矛盾しない用語の関係のみにもとづいて言われる。
 『形而上学』第五巻で哲学者によって明らかなとおり、この意味で可能的なも
 のは不可能的なものと対立する。

\\



{\sc Ad secundum dicendum} quod illud quod habet virtutem ut sit semper ex quo
 habet illam virtutem, non quandoque est et quandoque non est, sed
 antequam haberet illam virtutem, non fuit. Unde haec ratio, quae
 ponitur ab Aristotele in I {\it de Caelo}, non concludit simpliciter quod
 incorruptibilia non incoeperunt esse, sed quod non incoeperunt esse per
 modum naturalem, quo generabilia et corruptibilia incipiunt esse.

&

第二に対しては次のように言われるべきである。
常に存在する力を持つものは、あるときもないときもあるのではない、という力
 をそこから持つが、しかし、その力を持つ前には、存在しなかった。
したがって、アリストテレスによって『天体論』第一巻で措定されている理由は、
 端的に不可滅的なものが存在し始めたことがない、ということを結論するので
 はなく、生成消滅するものどもが存在し始めるようなかたちで、自然的なあり
 方によって、存在し始めることはないことを結論しているに過ぎない。



\\



{\sc Ad tertium dicendum} quod Aristoteles, in I {\it Physic}., probat materiam esse
 ingenitam, per hoc quod non habet subiectum de quo sit. In I autem {de
 Caelo et Mundo}, probat caelum ingenitum, quia non habet contrarium ex
 quo generetur. Unde patet quod per utrumque non concluditur nisi quod
 materia et caelum non incoeperunt per generationem, ut quidam ponebant,
 praecipue de caelo. Nos autem dicimus quod materia et coelum producta
 sunt in esse per creationem, ut ex dictis patet.

&

第三に対しては、次のように言われるべきである。
アリストテレスは『自然学』第一巻で、質料が生み出されたものでないことを、
 質料がそれから生じるような基体がないということから証明している。
また、『天体論』第一巻で、天が生み出されたものでないことを、そこから天が
 生み出されるような対立物がないことから証明している。
したがって、どちらによっても、質料と天が、生成によって始まるのではないこ
 とが結論されている。とくに天については、それが生成によって始まると考え
 ていた人々がいるが。
しかし、すでに述べられたことから明らかなとおり\footnote{STI, q.~45, a.~2}、私たちは、質料と天は、創
 造によって存在へと生み出されたという。


\\



{\sc Ad quartum dicendum} quod ad rationem vacui non sufficit in quo nihil
 est, sed requiritur quod sit spatium capax corporis, in quo non sit
 corpus, ut patet per Aristotelem, in IV {\it Physic}. Nos autem dicimus non
 fuisse locum aut spatium ante mundum.

&

第4に対しては、次のように言われるべきである。
『自然学』第四巻のアリストテレスによってあきらかなとおり、空虚の概念には、
 そこになにもないことだけでは十分でなく、そこに物体がない、物体を受け入
 れる空間があることが求められる。しかしわたしたちは、世界の前には、場所
 も空間もなかったと言う。

\\



Ad quintum dicendum quod primus motor semper eodem modo se habuit: primum
 autem mobile non semper eodem modo se habuit, quia incoepit esse, cum
 prius non fuisset. Sed hoc non fuit per mutationem, sed per creationem,
 quae non est mutatio, ut supra dictum est. Unde patet quod haec ratio,
 quam ponit Aristoteles in VIII {\it Physic}., procedit contra eos qui
 ponebant mobilia aeterna, sed motum non aeternum; ut patet ex
 opinionibus Anaxagorae et Empedoclis. Nos autem ponimus, ex quo mobilia
 incoeperunt, semper fuisse motum.

&
第5に対しては、次のように言われるべきである。第一動者は常に同様の状態に
 あったが、第一被動者は、常に同様の状態にあったわけではない。なぜなら、
 それは存在し始めたからである。それ以前にはなかったのだから。
しかし、これは、変化によるのではなく、創造によるのであった。創造は、すで
 に述べられたように、変化ではない。したがって、アリストテレスが『自然学』
 第八巻で示している理由は、動かされるものは永遠だが、運動は永遠でないと
 論じていた人々に反対して進んでいる。アナクサゴラスやエンペドクレスの意
 見から明らかなように。
私たちは、これに対して、そこから動かされうるものが存在し始めたところから、
 常に運動があったと措定する。


\\



{\sc Ad sextum dicendum} quod primum agens est agens voluntarium. Et quamvis
 habuit voluntatem aeternam producendi aliquem effectum, non tamen
 produxit aeternum effectum. Nec est necesse quod praesupponatur aliqua
 mutatio, nec etiam propter imaginationem temporis. Aliter enim est
 intelligendum de agente particulari, quod praesupponit aliquid, et
 causat alterum, et aliter de agente universali, quod producit
 totum. Sicut agens particulare producit formam, et praesupponit
 materiam, unde oportet quod formam inducat secundum proportionem ad
 debitam materiam. Unde rationabiliter in ipso consideratur quod inducit
 formam in talem materiam et non in aliam, ex differentia materiae ad
 materiam. Sed hoc non rationabiliter consideratur in Deo, qui simul
 producit formam et materiam, sed consideratur rationabiliter in eo,
 quod ipse producit materiam congruam formae et fini. Agens autem
 particulare praesupponit tempus, sicut et materiam. Unde rationabiliter
 consideratur in eo, quod agit in tempore posteriori et non in priori,
 secundum imaginationem successionis temporis post tempus. Sed in agente
 universali, quod producit rem et tempus, non est considerare quod agat
 nunc et non prius, secundum imaginationem temporis post tempus, quasi
 tempus praesupponatur eius actioni, sed considerandum est in eo, quod
 dedit effectui suo tempus quantum voluit, et secundum quod conveniens
 fuit ad suam potentiam demonstrandam. Manifestius enim mundus ducit in
 cognitionem divinae potentiae creantis, si mundus non semper fuit, quam
 si semper fuisset, omne enim quod non semper fuit, manifestum est
 habere causam; sed non ita manifestum est de eo quod semper fuit.

&
第6に対しては、次のように言われるべきである。
第一の作用者は、意志的な作用者である。
それが、なんらかの結果を生み出す永遠の意志を持っていたとしても、しかし、
 永遠の結果を作り出したのではない。
また、時間の想像のためにであっても、なんらの変化も必要とされない。
理由は以下の通り。
なにかを前提し、他のものの原因となる個別的作用者と、全体を生み出す普遍的
 作用者については、別様に理解されるべきである。
個別的作用者は、当然、形相を生みだすが、同じく当然に、質料を前提する。
したがって、しかるべき質料への比例に従って形相を導入しなければならない。
このことから、質料同士の相違に基づいて、あの質料ではなくこれこれの質料へ
 形相を導入することが、作用者において考察されるのは理にかなっている。
しかし、このことが、神において考察されるのはおかしい。なぜなら、神は形相と
 質料を同時に生み出すからである。しかし、神が、形相や目的に一致した質料
 を作っているかということが考察されるのは、理にかなっている。
さらに、個別的作用者は、質料と同様に、時間を前提する。それゆえ、より先の
 時間ではなくあとの時間に働くということが、時間の継起の想像によって考察
 されることは不思議ではない。
しかし、事物と時間を作り出す普遍的作用者において、以前ではなく今作用する、
 ということを、あたかも彼の作用に時間が前提されているかのように、時間の
 あとの時間を想像することによって、考えるのは馬鹿げている。
むしろ、そこで考えられるべきなのは、その作用者が、自らの結果に、それが意
 志するだけの、そして、自らの能力を明示するために適している限りにおいて、
 時間を与えたことである。
じっさい、もし世界が常になかったならば、それが常にあったと仮定した場合よ
 りも、世界は、創造する神の能力を認識することへ、より明らかに導く。なぜ
 なら、すべて常に存在したわけではないものは、原因を持つことが明らかだが、
 常に存在したものについては、それほど明らかではないからである。

\\



{\sc Ad septimum dicendum} quod, sicut dicitur in IV {\it Physic}., prius et
 posterius est in tempore, secundum quod prius et posterius est in
 motu. Unde principium et finis accipienda sunt in tempore, sicut et in
 motu. Supposita autem aeternitate motus, necesse est quod quodlibet
 momentum in motu acceptum sit principium et terminus motus, quod non
 oportet, si motus incipiat. Et eadem ratio est de nunc temporis. Et sic
 patet quod ratio illa instantis nunc, quod semper sit principium et
 finis temporis, praesupponit aeternitatem temporis et motus. Unde
 Aristoteles hanc rationem inducit, in VIII {\it Physic}., contra eos qui
 ponebant aeternitatem temporis, sed negabant aeternitatem motus.

&

第7に対しては、次のように言われるべきである。
『自然学』第六巻で言われているように、運動において前後があるのに従って、
 時間において前後がある。
ゆえに、時間における始まりと終わりは、運動におけるそれらと同様に理解され
 るべきである。
永遠の運動が前提される場合、運動のどの瞬間が受け取られても、それは運動の
 始まりでありかつ終わりであるが、運動が始まる場合は、そうであるとは限ら
 ない。
同じことが時間の今についても言える。
このようにして、常に時間の始まりであり終わりである、という瞬間の今の規定
 は、時間と運動の永遠性を前提としていることが明らかである。
したがって、アリストテレスは『自然学』第八巻で、時間の永遠性を措定しなが
 ら運動の永遠性を否定した人々に反対して、かの理由を持ち出している。

\\



{\sc Ad octavum dicendum} quod Deus est prior mundo duratione. Sed ly prius
 non designat prioritatem temporis, sed aeternitatis. Vel dicendum quod
 designat aeternitatem temporis imaginati, et non realiter
 existentis. Sicut, cum dicitur, supra caelum nihil est, ly supra
 designat locum imaginatum tantum, secundum quod possibile est imaginari
 dimensionibus caelestis corporis dimensiones alias superaddi.

&


第8に対しては、次のように言われるべきである。
神は持続において世界より先である。
しかし、この「先」は、時間の先行性ではなく、永遠の先行性を意味する。
あるいは次のように言われるべきである。
これは、想像された時間の永遠性を指示しているのであり、現実に存在する時間
 の永遠性を指示しているのではない。
ちょうど、「天の上になにもない」と言われるとき、この「上に」は、天体の広
 がりに他の広がりが付加されることが想像されうる限りにおいて、想像され
 た場所を指示するに過ぎないようにである。


\\



{\sc Ad nonum dicendum} quod, sicut effectus sequitur a causa agente
 naturaliter secundum modum suae formae, ita sequitur ab agente per
 voluntatem secundum formam ab eo praeconceptam et definitam, ut ex
 superioribus patet. Licet igitur Deus ab aeterno fuerit sufficiens
 causa mundi, non tamen oportet quod ponatur mundus ab eo productus,
 nisi secundum quod est in praedefinitione suae voluntatis; ut scilicet
 habeat esse post non esse, ut manifestius declaret suum auctorem.

&

第9に対しては、次のように言われるべきである。すでに述べられたことから明ら
かなとおり、ちょうど結果が、自らの形相のあり方に従って、自然的に、作用因
から出てくるように、意志による作用者から、彼が前もってとらえ限定した形相
に従って、結果が出てくる。ゆえに、神が永遠から、世界の十分な原因であった
ことは認めるが、しかし、世界は、あらかじめ定まった彼の意志のなかにあるか
ぎりにおいてのみ、彼によって作られる。すなわち、世界は、自らの作者をより明らかに
 示すために、存在しなかった後に存在を持つように。


\\


{\sc Ad decimum dicendum} quod, posita actione, sequitur effectus secundum
 exigentiam formae quae est principium actionis. In agentibus autem per
 voluntatem, quod conceptum est et praedefinitum, accipitur ut forma
 quae est principium actionis. Ex actione igitur Dei aeterna non
 sequitur effectus aeternus, sed qualem Deus voluit, ut scilicet haberet
 esse post non esse.

&

第10に対しては、次のように言われるべきである。
作用が措定されると、その作用の根源である形相の必要性に応じて、結果が伴う。
 しかし、意志による作用者においては、受け取られたものやあらかじめ定めら
 れたものが、作用の根源である形相として受け取られる。
ゆえに、神の永遠の作用から、永遠の結果は帰結せず、神が意志するように、つ
 まり、存在しなかった後に存在を持つように、結果がある。


\\


\end{longtable}
\newpage
\rhead{a.~2}
\begin{center}
 {\Large {\bfseries ARTICULUS SECUNDUS}}\\
{\large UTRUM MUNDUM INCOEPISSE SIT ARTICULUS FIDEI}\\
{\footnotesize II {\itshape Sent.}, d.~1, q.~1, a.~5; II {\itshape
 SCG.}, c.~38; {\itshape De Pot.}, q.~3, a.~14; {\itshape Quodl.}~XII,
 q.~6, a.~1; Opusc.}~XXVII, {\itshape de Aeternit.~Mundi.}\\
{\Large 第二項\\世界が始まったということは信仰箇条か}
\end{center}

\begin{longtable}{p{21em}p{21em}}

{\LARGE A}{\sc d secundum sic proceditur}. Videtur quod mundum
 incoepisse non sit articulus fidei, sed conclusio demonstrabilis. Omne
 enim factum habet principium suae durationis. Sed demonstrative probari
 potest quod Deus sit causa effectiva mundi, et hoc etiam probabiliores
 philosophi posuerunt. Ergo demonstrative probari potest quod mundus
 incoeperit.


&
第二に対しては次のように進められる。
世界が始まったということは信仰箇条でなく論証可能な結論であると思われる。
すべて作られたものは、自らの持続の始まりを持つ。
しかるに、神が世界の作出因であることは、論証的に証明されうる。そして、哲
 学者たちは、これを、より確からしいものとして措定した。ゆえに、世界が始
 まったことは、論証的に証明されうる。



\\

2.~{\sc Praeterea}, si necesse est dicere quod mundus factus est a Deo,
 aut ergo ex nihilo, aut ex aliquo. Sed non ex aliquo, quia sic materia
 mundi praecessisset mundum; contra quod procedunt rationes Aristotelis
 ponentis caelum ingenitum. Ergo oportet dicere quod mundus sit factus
 ex nihilo. Et sic habet esse post non esse. Ergo oportet quod esse
 incoeperit.


&

さらに、もし、世界が神によって作られたと言うことが必然であるならば、無か
 らか、何かからかのいずれかである。
しかるに、何かからではない。なぜなら、もしそうだとすると、世界の質料が、
 世界に先行しただろうから。天が不生であることを措定するアリストテレスの
 議論は、これに反対して進む。
ゆえに、世界は無から作られたと言うべきである。したがって、世界は、非存在
 の後に存在を持つ。ゆえに、世界は存在を持ち始めたのでなければならない。

\\


3.~{\sc Praeterea}, omne quod operatur per intellectum, a quodam
 principio operatur, ut patet in omnibus artificibus. Sed Deus est agens
 per intellectum. Ergo a quodam principio operatur. Mundus igitur, qui
 est eius effectus, non fuit semper.


&


さらに、すべての職人において明らかなとおり、知性によって働くすべてのこと
 は、なんらかの始原から働く。しかるに、神は知性によって働くものである。
 ゆえに、なんらかの始原から働く。ゆえに、神の結果である世界は、常に在っ
 たのではない。

\\


4.~{\sc Praeterea}, manifeste apparet artes aliquas, et habitationes
 regionum, ex determinatis temporibus incoepisse. Sed hoc non esset, si
 mundus semper fuisset. Mundum igitur non semper fuisse manifestum est.


&
さらに、ある技術や、地域の居住が、特定の時間から始まったことは明らかであ
 る。しかし、もし世界が常に在ったのだとすると、こういうことはなかったで
 あろう。
ゆえに、世界が常に在ったのではないことは明らかである。

\\


5.~{\sc Praeterea}, certum est nihil Deo aequari posse. Sed si mundus
 semper fuisset, aequipararetur Deo in duratione. Ergo certum est non
 semper mundum fuisse.

&
さらに、何ものも神に等しくはありえない。
しかし、もし世界が常に在ったとすると、持続において神と等しくなるであろう。
 ゆえに、世界が常に在ったのではないことは確かである。

\\


6.~{\sc Praeterea}, si mundus semper fuit, infiniti dies praecesserunt
 diem istum. Sed infinita non est pertransire. Ergo nunquam fuisset
 perventum ad hunc diem, quod est manifeste falsum.

&

さらに、もし世界が常に在ったとすると、この日を無限の日が過ぎたことになっ
 ただろう。しかるに、無限のものが通過することはできない。ゆえに、けっし
 てこの日まで到達されなかったであろう。しかしこれは明らかに偽である。

\\


7.~{\sc Praeterea}, si mundus fuit aeternus, et generatio fuit ab
 aeterno. Ergo unus homo genitus est ab alio in infinitum. Sed pater est
 causa efficiens filii, ut dicitur in II {\it Physic}. Ergo in causis
 efficientibus est procedere in infinitum, quod improbatur in II
 {\it Metaphys}.

&
さらに、もし世界が永遠であったならば、生成も永遠からあっただろう。
ゆえに、一人の人間が別の人間から生まれることが無限にあっただろう。
しかし、『自然学』二巻で言われるように、父は子の作出因である。
ゆえに、作出因において無限に遡ることができるが、これは『形而上学』二巻で
 否定されている。

\\


8.~{\sc Praeterea}, si mundus et generatio semper fuit, infiniti homines
 praecesserunt. Sed anima hominis est immortalis. Ergo infinitae animae
 humanae nunc essent actu, quod est impossibile. Ergo ex necessitate
 sciri potest quod mundus incoeperit; et non sola fide tenetur.

&
さらに、もし世界と生成が常に在ったならば、無数の人間が過ぎ去ったであろう。
 しかし、人間の魂は不死である。ゆえに、無数の人間の魂が今現実に存在する
 ことになるがこれは不可能である。ゆえに必然的に世界が始まったと知りうる
 のであり、信仰のみによって保持されるのではない。


\\


{\sc Sed contra}, fidei articuli demonstrative probari non possunt: quia
 fides de {\it non apparentibus} est, ut dicitur {\it ad Hebr}.~{\sc xi}. Sed Deum esse
 creatorem mundi, sic quod mundus incoeperit esse, est articulus fidei,
 dicimus enim: {\it Credo in unum Deum etc}. Et iterum, Gregorius dicit,
 in Homil.~I {\it in Ezech}., quod Moyses prophetizavit de praeterito, dicens
 {\it In principio creavit Deus caelum et terram}; in quo novitas mundi
 traditur. Ergo novitas mundi habetur tantum per revelationem. Et ideo
 non potest probari demonstrative.


&

しかし反対に、信仰箇条は、論証的に証明されえない。なぜなら、『ヘブライ人
 への手紙』11章に言われるように「信仰は見えないものにかかわる」
 \footnote{「信仰とは、望んでいる事柄を確信し、見えない事実を確認すること
 です」(11:1)}からである。しかるに、神が世界の創造者であること、その意味
 で世界が存在し始めたことは、信仰箇条である。実際、私たちは「私は一なる神
 を信じる、云々」と言う。さらに、グレゴリウスは『エゼキエル書』についての
 注解で、モーゼは「初めに神は天と地を作った」と述べて、過去についての預言
 を行ったと言うが、ここで、世界の新しさが伝えられている。ゆえに、世界の新
 しさは啓示によってのみ知られる。ゆえに、論証的に証明されえない。


\\


{\sc Respondeo dicendum quod} mundum non semper fuisse, sola fide
 tenetur, et demonstrative probari non potest, sicut et supra de
 mysterio Trinitatis dictum est. Et huius ratio est, quia novitas mundi
 non potest demonstrationem recipere ex parte ipsius
 mundi. Demonstrationis enim principium est quod quid est. Unumquodque
 autem, secundum rationem suae speciei, abstrahit ab hic et nunc,
 propter quod dicitur quod universalia sunt ubique et semper. Unde
 demonstrari non potest quod homo, aut caelum, aut lapis non semper
 fuit. Similiter etiam neque ex parte causae agentis, quae agit per
 voluntatem. Voluntas enim Dei ratione investigari non potest, nisi
 circa ea quae absolute necesse est Deum velle, talia autem non sunt
 quae circa creaturas vult, ut dictum est. 


&

答えて言わなければならない。世界が常に在ったのではないということは、信仰
によってのみ保持され、論証的に証明されえない。これは上で、三位一体の神秘
について言われたのと同様である。その理由は、世界の新しさは、世界自身の側
から証明を受け取ることができないからである。というのも、論証の根源は「何
であるか」である。しかるに、各々のものは、自らの種の性格において、「ここ・
今」から抽象される。このことから、普遍は、至る所に常に在ると言われる。し
たがって、人間であれ天であれ石であれ、常に在ったのではないことは証明され
えない。同様に、意志によって働く作用因の側からも、証明されえない。なぜな
ら、神の意志は理性によって探究されえないからである。ただし、神が意志する
ことが無条件的に必然であるような事柄に関しては例外だが、被造物に関
して神が意志することは、そのようなものではない。

\\


Potest autem voluntas divina
 homini manifestari per revelationem, cui fides innititur. Unde mundum
 incoepisse est credibile, non autem demonstrabile vel scibile. Et hoc
 utile est ut consideretur, ne forte aliquis, quod fidei est demonstrare
 praesumens, rationes non necessarias inducat, quae praebeant materiam
 irridendi infidelibus, existimantibus nos propter huiusmodi rationes
 credere quae fidei sunt.
&

ところで、神の意志は、啓示によって人間に明らかにされるが、信仰は啓示に依
 拠する。したがって、世界が始まったということは、信じられうる事柄であり、
 論証されうる事柄や知られうる事柄ではない。また、次のことを考慮するのもよ
 いだろう。すなわち、信仰に属することを論証しようとする人が、必然的でない
 理由を導入し、その理由が、私たちが信仰に属することをそのような理由で信じ
 ているのだと考える不信心者たちに、嘲笑の材料を与えることがないように、と
 いうことを。

\\


{\sc Ad primum ergo dicendum quod}, sicut dicit Augustinus, XI {\it de
 Civ.~Dei}, philosophorum ponentium aeternitatem mundi, duplex fuit
 opinio. Quidam enim posuerunt quod substantia mundi non sit a Deo. Et
 horum est intollerabilis error; et ideo ex necessitate
 refellitur.
%\footnote{r\u{e}-fello , felli, 3, v.~a.~I.~to show to be false; to
% disprove, rebut, confute, refute, repel} 
Quidam autem sic posuerunt mundum aeternum, quod tamen
 mundum a Deo factum dixerunt. {\it Non enim} mundum {\it temporis
 volunt habere, sed suae creationis initium, ut quodam modo vix
 intelligibili semper sit factus. Id autem quomodo intelligant,
 invenerunt}, ut idem dicit in X {\it de Civ. Dei}. {\it Sicut enim,
 inquiunt, si pes ex aeternitate semper fuisset in pulvere,
%\footnote
%{pulvis , \u{e}ris, m. 
%I.~Lit., dust, powder:
%II.~Transf.
%A.~A place of contest, arena, lists:
%2.~In gen., a scene of action, field (cf.~arena):
%B.~Toil, effort, labor (poet.):
%} 
semper
 subesset vestigium, quod a calcante
%\footnote
%{calco , \={a}vi, \={a}tum, 1, v.~a.~1.~calx,
%I.to tread something or upon something, to tread under foot. 
%} 
factum nemo dubitaret; sic et
 mundus semper fuit, semper existente qui fecit}. Et ad hoc
 intelligendum, considerandum est quod causa efficiens quae agit per
 motum, de necessitate praecedit tempore suum effectum, quia effectus
 non est nisi in termino actionis, agens autem omne oportet esse
 principium actionis. Sed si actio sit instantanea, et non successiva,
 non est necessarium faciens esse prius facto duratione; sicut patet in
 illuminatione. Unde dicunt quod non sequitur ex necessitate, si Deus
 est causa activa mundi, quod sit prior mundo duratione, quia creatio,
 qua mundum produxit, non est mutatio successiva, ut supra dictum est.

&

第一に対しては、それゆえ、次のように言われるべきである。
アウグスティヌスが『神の国』11巻で述べているように、世界が永遠であるとし
 ている哲学者たちには、二通りの意見があった。
ある人々は、世界の実体が神からではないとした。この人々には、容認しがたい
 誤りがあるので、必然的に偽であることが示される。
他方、ある人々は、世界が神によって作られたと語りながら、世界が永遠だとし
 た。「彼らは、世界が、時間ではなく自らの創造の始まりをもつことを欲する。
 ちょうど、可知的な力があるいみで常に作られているように」。しかるに、同
 じ人が『神の国』第10巻で言うように、「どのように知性認識するかを、彼ら
 は見出した。つまり、彼らが言うには、ちょうど、もし足が常に砂塵のなかに
 在ったならば、踏みしめる者によって作られた足跡が、常にその下に在っただ
 ろうということを、だれも疑わないだろう。同様に、世界を作る者が常に存在
 しているならば、世界もまた常に在った」。
そして、これを理解するためには、次のことが考察されるべきである。運動を通
 して働く作出因は、必然的に、時間において自らの結果に先行する。なぜなら、
 結果は働きの終端においてでなければ存在しないが、すべて作用者は、働きの
 端緒でなければならないからである。しかし、たとえば照明において明らかで
 あるように、もし働きが瞬間的であって継起的でないならば、作るものが、作
 られるものに、持続において先である必要はない。したがって、彼らは、もし
 神が、世界の能動因であるとしても、それによって世界を生み出した創造は、
 上で述べられたとおり
%\footnote
%{ ``Ad tertium dicendum quod in his quae fiunt sine motu, simul est fieri et
% factum esse, sive talis factio sit terminus motus, sicut illuminatio
% (nam simul aliquid illuminatur et illuminatum est); sive non sit
% terminus motus, sicut simul formatur verbum in corde et formatum
% est. Et in his, quod fit, est, sed cum dicitur fieri, significatur ab
% alio esse, et prius non fuisse. Unde, cum creatio sit sine motu, simul
% aliquid creatur et creatum est.'' (STI q.~45, a.~2, ad 3)
%「第三に対しては、次のように言われるべきである。運動なしに生じるものどもに
% おいては、作られることと作られたこととは同時である。これは、その「作り」
% が運動の終極であろうと、そうでなかろうと同様である。前者の場合、たとえ
% ば、照明がその例であり(あるものが照明されることと、照明されたことと
% は同時である)、後者の場合は、たとえば、言葉が心の中で形成され、同時に、
% 形成されたというように。これらのものにおいては、作られていることが、在
% ることである。しかし、作られていると言われるとき、他のものによって存在
% すること、そして、以前には存在しなかったことが意味される。したがって、
% 創造は運動なしにあるので、何かが創造されていることと創造されたこととは
% 同時である。」
%}
継起的な運動ではないので、持続において世界に先立つ
 必要はないと言う。

\\


{\sc Ad secundum dicendum} quod illi qui ponerent mundum aeternum, dicerent
 mundum factum a Deo ex nihilo, non quod factus sit post nihilum,
 secundum quod nos intelligimus per nomen creationis; sed quia non est
 factus de aliquo. Et sic etiam non recusant
%\footnote
%{r\u{e}-c\={u}so , \={a}vi, \={a}tum, 1 (
%I.gen. plur. of the part. pres. recusant\={u}m, Verg. A. 7, 16), v. a. causa.
%I. In gen., to make an objection against, in statement or reply; to decline, reject, refuse, be reluctant or unwilling to do a thing, etc. (freq. and class.; cf.: abnuo, renuo, denego); constr. with acc., an inf., an object-clause, with de, ne, quin, quominus, or absol. 
%} 
aliqui eorum creationis
 nomen, ut patet ex Avicenna in sua {\it Metaphysica}.

&
第二に対しては、次のように言われるべきである。
世界が永遠であることを措定した人々は、私たちが創造という名で理解するよう
 に、無のあとに作られたからではなく、何かから作られたのではないから、世
 界が神によって無から作られたと言う。この意味で、彼ら(世界が永遠だとす
 る人々)のうちには、創造という名を拒絶しない人すらいる。たとえば、彼の
 『形而上学』におけるアヴィセンナから明らかなとおり。



\\


{\sc Ad tertium dicendum} quod illa est ratio Anaxagorae, quae ponitur
 in III {\it Physic}. Sed non de necessitate concludit, nisi de intellectu qui
 deliberando investigat quid agendum sit, quod est simile motui. Talis
 autem est intellectus humanus, sed non divinus, ut supra patet.


&

第三に対しては、次のように言われるべきである。
かの説は、『自然学』第3巻に示されているアナクサゴラスの説である。
しかし、何が為されるべきかを思量しながら探究する知性、これは運動に似てい
 るが、についてでなければ、必然的に帰結しない。
しかし、そのような知性は人間の知性であり、上で明らかなとおり、神の知性で
 はない。


\\


{\sc Ad quartum dicendum} quod ponentes aeternitatem mundi, ponunt
 aliquam regionem infinities esse mutatam de inhabitabili in
 habitabilem, et e converso. Et similiter ponunt quod artes, propter
 diversas corruptiones et accidentia, infinities fuerunt inventae, et
 iterum corruptae. Unde Aristoteles dicit, in libro {\it Meteor}., quod
 ridiculum est ex huiusmodi particularibus mutationibus opinionem
 accipere de novitate mundi totius.

&
第四に対しては、次のように言われるべきである。
世界の永遠性を主張する人々は、ある地域が、無限回、居住不可能なところから
 可能なところへ、またその逆へ変化したと主張する。
同様に、技術も、さまざまな消滅とアクシデントのために、無限回、見出され、
 そして失われたと主張する。
このことからアリストテレスは、『気象論』で、このような個別的な変化にもと
 づいて、全世界の新しさについての意見を受け取るのは馬鹿げていると述べて
 いる。


\\


{\sc Ad quintum dicendum} quod, etsi mundus semper fuisset, non tamen
 parificaretur Deo in aeternitate, ut dicit Boetius, in fine {\it de
 Consolat}., quia esse divinum est esse totum simul, absque successione;
 non autem sic est de mundo.

&
第五に対しては、次のように言われるべきである。
かりに世界が常に在ったとしても、ボエティウスが『哲学の慰め』末尾で述べる
 ように、永遠性において神と等しくされることはない。なぜなら、神の存在は、
 継起なく、全体が同時にであるのに対し、世界についてはそうでないからであ
 る。


\\


{\sc Ad sextum dicendum} quod transitus semper intelligitur a termino in
 terminum. Quaecumque autem praeterita dies signetur, ab illa usque ad
 istam sunt finiti dies, qui pertransiri poterunt. Obiectio autem
 procedit ac si, positis extremis, sint media infinita.

&

第六に対しては、次のように言われるべきである。
移行は、つねに、端から端へ理解される。
しかし、どの過去の日が指定されても、その日からこの日までは有限な日数であ
 り、それを通過することは可能であった。しかし、反論は、あたかも、両端が
 指定されても中間が無限であるかのように論じている。

\\


Ad septimum dicendum quod in causis efficientibus impossibile est
 procedere in infinitum per se; ut puta si causae quae per se
 requiruntur ad aliquem effectum, multiplicarentur in infinitum; sicut
 si lapis moveretur a baculo, et baculus a manu, et hoc in
 infinitum. Sed per accidens in infinitum procedere in causis agentibus
 non reputatur
%\footnote
%{r\u{e}-p\u{u}to , \={a}vi, \={a}tum, 1, v. a.,
%I.to count over, reckon, calculate, compute (syn.: numero, expendo).
%II. Trop.
%A. To think over, ponder, meditate, reflect upon (freq. and class.; a favorite word of Sall. and Tac.; not used by Caes.;
%B. (Late Lat.) To impute, ascribe: 
%} 
 impossibile; ut puta si omnes causae quae in infinitum
 multiplicantur, non teneant ordinem nisi unius causae, sed earum
 multiplicatio sit per accidens; sicut artifex agit multis martellis per
 accidens, quia unus post unum frangitur. Accidit ergo huic martello,
 quod agat post actionem alterius martelli. Et similiter accidit huic
 homini, inquantum generat, quod sit generatus ab alio, generat enim
 inquantum homo, et non inquantum est filius alterius hominis; omnes
 enim homines generantes habent gradum unum in causis efficientibus,
 scilicet gradum particularis generantis. Unde non est impossibile quod
 homo generetur ab homine in infinitum. Esset autem impossibile, si
 generatio huius hominis dependeret ab hoc homine, et a corpore
 elementari, et a sole, et sic in infinitum.

&
第七に対しては、次のように言われるべきである。
作出因において、自体的に、無限に遡ることはできない。ちょうど、ある結果に
 自体的に必要とされる原因が、無限に多数化される場合がそれに当たる。
たとえば、石が杖によって動かされ、杖が手によって動かされ、そして無限に続
 くように。
しかし、附帯的に、作用因において無限に進むことは、不可能なことは考えられ
 ない。ちょうど、もし、無限に多いすべての原因が、一つの原因の秩序だけを
 保つ場合、しかしそれらの多数化は、附帯的である。たとえば、職人が、次々
 と壊れるために、多くの金槌によって働く場合のように。したがって、この金
 槌には、他の金槌の働きのあとに働くことが附帯する。
同様に、この人間には、産む限りにおいて、他のものから生まれたことが附帯す
 る。なぜなら、人間である限りにおいて産むのであり、他の人間の息子である
 限りにおいて産むのではないから。
じっさい、すべての産む人間は、作出因において、一つの段階、すなわち、個別
 的生成の段階を持つ。したがって、人間が人間から、無限に生み出されること
 は不可能ではない。
しかし、もし、この人間の生成が、この人間に、元素的物体に、太陽に、という
 ように無限に依存したら、これは不可能だっただろう。

\\


Ad octavum dicendum quod hanc rationem ponentes aeternitatem mundi
 multipliciter effugiunt. Quidam enim non reputant impossibile esse
 infinitas animas actu; ut patet in {\it Metaphysica} Algazelis, dicentis hoc
 esse infinitum per accidens. Sed hoc improbatum est superius. Quidam
 vero dicunt animam corrumpi cum corpore. Quidam vero quod ex omnibus
 animabus remanet una tantum. Alii vero, ut Augustinus dicit, posuerunt
 propter hoc circuitum animarum; ut scilicet animae separatae a
 corporibus, post determinata temporum curricula, iterum redirent ad
 corpora. De quibus omnibus in sequentibus est agendum. Considerandum
 tamen quod haec ratio particularis est. Unde posset dicere aliquis quod
 mundus fuit aeternus, vel saltem aliqua creatura, ut Angelus; non autem
 homo. Nos autem intendimus universaliter, an aliqua creatura fuerit ab
 aeterno.

&

第八に対しては、次のように言われるべきである。世界の永遠性を主張する人々
は、この異論を、多くのかたちで避けている。ある人々は、無限の魂が現実に存
在することが不可能ではないと考える。たとえば、アルガゼルの『形而上学』に
おいて明らかなように。彼は、これは附帯性による無限だと言う。しかし、これ
は上で否定された。\footnote{(ST I, q.~7, a.~4, c.)}
%\footnote{(ST I, q.~7, a.~4, c.) ``Respondeo dicendum
%quod circa hoc fuit duplex opinio. Quidam enim, sicut Avicenna et
%Algazel, dixerunt quod impossibile est esse multitudinem actu infinitam
%per se, sed infinitam per accidens multitudinem esse, non est
%impossibile. Dicitur enim multitudo esse infinita per se, quando
%requiritur ad aliquid ut multitudo infinita sit. Et hoc est impossibile
%esse, quia sic oporteret quod aliquid dependeret ex infinitis; unde eius
%generatio nunquam compleretur, cum non sit infinita pertransire. Per
%accidens autem dicitur multitudo infinita, quando non requiritur ad
%aliquid infinitas multitudinis, sed accidit ita esse. Et hoc sic
%manifestari potest in operatione fabri, ad quam quaedam multitudo
%requiritur per se, scilicet quod sit ars in anima, et manus movens, et
%martellus. Et si haec in infinitum multiplicarentur, nunquam opus
%fabrile compleretur, quia dependeret ex infinitis causis. Sed multitudo
%martellorum quae accidit ex hoc quod unum frangitur et accipitur aliud,
%est multitudo per accidens, accidit enim quod multis martellis operetur;
%et nihil differt utrum uno vel duobus vel pluribus operetur, vel
%infinitis, si infinito tempore operaretur. Per hunc igitur modum,
%posuerunt quod possibile est esse actu multitudinem infinitam per
%accidens. Sed hoc est impossibile. Quia omnem multitudinem oportet esse
%in aliqua specie multitudinis. Species autem multitudinis sunt secundum
%species numerorum. Nulla autem species numeri est infinita, quia
%quilibet numerus est multitudo mensurata per unum. Unde impossibile est
%esse multitudinem infinitam actu, sive per se, sive per accidens. Item,
%multitudo in rerum natura existens est creata, et omne creatum sub
%aliqua certa intentione creantis comprehenditur, non enim in vanum agens
%aliquod operatur. Unde necesse est quod sub certo numero omnia creata
%comprehendantur. Impossibile est ergo esse multitudinem infinitam in
%actu, etiam per accidens. Sed esse multitudinem infinitam in potentia,
%possibile est. Quia augmentum multitudinis consequitur divisionem
%magnitudinis, quanto enim aliquid plus dividitur, tanto plura secundum
%numerum resultant. Unde, sicut infinitum invenitur in potentia in
%divisione continui, quia proceditur ad materiam, ut supra ostensum est;
%eadem ratione etiam infinitum invenitur in potentia in additione
%multitudinis.''}  
また、ある人々は、魂が身体と共に滅びると言う。またある
人々は、すべての魂から、ただ一つの魂が残存すると言う。また、アウグスティ
ヌスが言うように、ある人々は、このために、魂の循環を主張した。すなわち、
身体から分離した魂は、一定の時間の運行の後に、再び身体に戻る。これらすべ
てについては、後に論じられるべきである。\footnote {(STI, q.~75, a.~6)人間
の魂は可滅的か。(STI, q.~76, a.~2)知性的根源は身体と共に多数化されず、す
べての人間に一つの知性があるか。(STI, q.~118, a.~6)とレオニナにあるが、
118問は第三項までしかない。} しかし、この異論は個別的であることが考察され
るべきである。したがって、世界、または、ある被造物が、たとえば天使が、永
遠であったが、人間はそうでない、と言うことができただろう。しかし私たちは、
普遍的に、なんらかの被造物が永遠から存在したかどうかを意図している。

\\

\end{longtable}

\newpage
\rhead{a.~3}
\begin{center}
 {\Large {\bf ARTICULUS TERTIUS}}\\
 {\large UTRUM CREATIO RERUM FUERIT IN PRINCIPIO TEMPORIS}\\
 {\footnotesize II {\itshape Sent.}, d.~1, q.~1, a.~6.}\\
 {\Large 第三項\\事物の創造は時間の始まりにあったか}
\end{center}

\begin{longtable}{p{21em}p{21em}}

{\huge A}{\scshape d tertium sic proceditur}. Videtur quod
creatio rerum non fuit in principio temporis. Quod enim non est in
tempore, non est in aliquo temporis. Sed creatio rerum non fuit in
tempore, per creationem enim rerum substantia in esse producta est;
tempus autem non mensurat substantiam rerum, et praecipue
incorporalium. Ergo creatio non fuit in principio temporis.

&
第三に対しては次のように進められる。
事物の創造は、時間の始まりにあったのではないと思われる。なぜなら、
時間のうちにないものは、時間のどこにもない。ところが、事物の創造は時間の
 うちにあったのではない。
なぜなら、創造によって、事物の実体が存在へと生み出されたのだが、時間は事
 物の実体の尺度ではなく、とくに、非物体的事物の実体の尺度ではない。
ゆえに、創造は時間の始まりにあったのではない。

\\


{\scshape 2 Praeterea}, philosophus probat quod omne
quod fit, fiebat, et sic omne fieri habet prius et posterius. In
principio autem temporis, cum sit indivisibile, non est prius et
posterius. Ergo, cum creari sit quoddam fieri, videtur quod res non sint
creatae in principio temporis.


&

さらに
哲学者は、すべて生じているものは、生じたのであり、その意味で、生じるとい
 うことは、前後をもつということを証明している。
ところで、時間の始まりは不可分だから、時間の始まりにおいて前後はない。ゆ
 えに、創造とは一種の生じることだから、事物が時間の始まりにおいて創造さ
 れたのではないと思われる。
\\

{\scshape 3 Praeterea}, ipsum etiam tempus creatum
est. Sed non potest creari in principio temporis, cum tempus sit
divisibile, principium autem temporis indivisibile. Non ergo creatio
rerum fuit in principio temporis.

&
さらに、時間そのものもまた創造された。ところで、時間が、時間の始まりにお
 いて創造されることは不可能である。なぜなら、時間は分割可能だが、時間の始まり
 は不可分だからである。ゆえに、事物の創造が時間の始まりにあったのではな
 い。

\\


{\scshape Sed contra est} quod {\itshape Gen}.~{\scshape i} dicitur,
{\itshape in principio creavit Deus caelum et terram}.

&
しかし反対に、『創世記』1章で「始めに、神は天と地を創造した」と言われて
 いる。

\\


{\scshape Respondeo dicendum} quod illud verbum {\itshape
Genes}.~{\scshape i}, {\itshape in principio creavit Deus caelum et
terram}, tripliciter exponitur, ad excludendum tres errores. Quidam enim
posuerunt mundum semper fuisse, et tempus non habere principium. Et ad
hoc excludendum, exponitur, {\itshape in principio}, scilicet {\itshape
temporis}.


&
答えて言わなければならない。
『創世記』1章「始めに、神は天と地を創造した」というあの言葉は、三つの誤
 りを排除するために、三通りに説明されている。
ある人々は、世界は常にあったのであり、時間は始まりをもたないと考えた。そ
 して、この誤りを排除するために「始めに」はすなわち「時間の始めに」だと
 説明される。

\\



-- Quidam vero posuerunt duo esse creationis principia, unum bonorum,
aliud malorum. Et ad hoc excludendum, exponitur, {\itshape in
principio}, idest {\itshape in filio}. Sicut enim principium effectivum
appropriatur patri, propter potentiam, ita principium exemplare
appropriatur filio, propter sapientiam, ut sicut dicitur, {\itshape
omnia in sapientia fecisti}, ita intelligatur Deum omnia fecisse
{\itshape in principio}, idest in filio; secundum illud apostoli
{\itshape ad Coloss}.~{\scshape i}, {\itshape in ipso}, scilicet filio,
{\itshape condita sunt universa}.

&

他方、ある人々は、創造の根源が二つ、すなわち、善の根源と悪の根源が
 あると考えた。そして、この誤りを排除するために、「始めに」はすなわち
 「子において」であると説明される。というのも、作出的根源が、その能力のた
 めに父に固有化されるように、範型的根源は、その知恵のために子に固有化され
 るからである。それはちょうど、「あなた方は万物を知恵において作った」と言
 われるように、神は万物を「始めに」すなわち子において作ったと理解されるよ
 うにである。かの『コロサイの信徒への手紙』1章の使徒によれば、「彼において」つま
 り子において、「宇宙が作られた」\footnote{「万物は御子において造られた
 からです」(1:16)}のである。

\\


-- Alii vero dixerunt corporalia esse creata a Deo mediantibus creaturis
spiritualibus. Et ad hoc excludendum, exponitur, {\itshape in principio
creavit Deus caelum et terram}, idest {\itshape ante omnia}. Quatuor
enim ponuntur simul creata, scilicet caelum Empyreum, materia corporalis
(quae nomine {\itshape terrae} intelligitur), tempus, et natura angelica.

&
また他の人々は、物体的なものどもが、霊的被造物を媒介にして、神に創造され
 たと言った。そしてこの誤りを排除するために、「始めに、神は天と地を創造
 した」というのはすなわち「すべてに先立って」ということである、と説明さ
 れる。つまり、四つのものが同時に創造されたとされる。その四つとは、浄火
 天、物体的質料(これは、「地」という名前で理解されている)、時間、そし
 て天使の本性である。

\\


{\scshape Ad primum ergo dicendum} quod non dicuntur in
principio temporis res esse creatae, quasi principium temporis sit
creationis mensura sed quia simul cum tempore caelum et terra creata
sunt.


&

第一に対して、それゆえ、次のように言われるべきである。
時間の始めに事物が創造されたと言われるのは、時間の始めが創造の尺度だとい
 う意味ではなく、時間とともに、天と地が創造されたという意味である。

\\


{\scshape Ad secundum dicendum} quod verbum illud
philosophi intelligitur de fieri quod est per motum, vel quod est
terminus motus. Quia cum in quolibet motu sit accipere prius et
posterius, ante quodcumque signum in motu signato, dum scilicet aliquid
est in moveri et fieri, est accipere prius, et etiam aliquid post ipsum,
quia quod est in principio motus, vel in termino, non est in
moveri. Creatio autem neque est motus neque terminus motus, ut supra
dictum est. Unde sic aliquid creatur, quod non prius creabatur.


&

第二に対しては、次のように言われるべきである。
哲学者のその言葉は、運動による「生じる」についてか、または、運動の端であ
 るような「生じる」について理解される。というのも、どんな運動においても
 前後を理解することができるから、あるものが運動や生成の中にあるかぎり、
 ある特定の運動の中のどの特定の時点の前にも「より先」を理解することがで
 きるし、また[どんな時点の]後にも何かを理解することができるからである。
 運動の始まり、あるいは終わりにあるものは、運動の中にはないのだから。
しかし、上で述べられたように、創造は運動でも運動の終端でもない。したがっ
 て、それより先に何も創造されていなかったというかたちで、あるものは創造
 される。

\\


{\scshape Ad tertium dicendum} quod nihil fit nisi
secundum quod est. Nihil autem est temporis nisi nunc. Unde non potest
fieri nisi secundum aliquod nunc, non quia in ipso primo nunc sit
tempus, sed quia ab eo incipit tempus.

&

第三に対しては、次のように言われるべきである。
何ものも、それが在るという点に即してでなければ、生じない。
ところが、何も、「今」でなければ時間に属さない。したがって、何らかの「今」
 であるという点に即してでなければ、何ものも生じえない。しかしそれは、そ
 の第一の今の中に時間がある、というのではなく、その第一の今から、時間が
 始まるということである。

\end{longtable}

\end{document}