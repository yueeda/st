\documentclass[10pt]{jsarticle}
\usepackage{okumacro}
\usepackage{longtable}
\usepackage[polutonikogreek,english,japanese]{babel}
\usepackage{latexsym}
\usepackage{color}
\usepackage{schemata}
\usepackage[T1]{fontenc}
\usepackage{lmodern}

%----- header -------
\usepackage{fancyhdr}
\pagestyle{fancy}
\lhead{{\it Summa Theologiae} I-II, q.63}
%--------------------

\bibliographystyle{jplain}

\title{{\bf PRIMA SECUNDAE}\\{\HUGE Summae Theologiae}\\Sancti Thomae
Aquinatis\\{\sffamily QUEAESTIO SEXAGESIMATERTIA}\\DE CAUSA VIRTUTUM}
\author{Japanese translation\\by Yoshinori {\sc Ueeda}}
\date{Last modified \today}

%%%% コピペ用
%\rhead{a.~}
%\begin{center}
% {\Large {\bf }}\\
% {\large }\\
% {\footnotesize }\\
% {\Large \\}
%\end{center}
%
%\begin{longtable}{p{21em}p{21em}}
%
%&
%
%
%\\
%
%\end{longtable}
%\newpage

\begin{document}

\maketitle
\thispagestyle{empty}

\begin{center}
{\LARGE 『神学大全』第二部の一}\\
{\Large 第六十三問\\徳の原因について}
\end{center}

\begin{longtable}{p{21em}p{21em}}

Deinde considerandum est de causa virtutum. Et circa hoc quaeruntur
quatuor.

\begin{enumerate}
 \item utrum virtus sit in nobis a natura.
 \item utrum aliqua virtus causetur in nobis ex assuetudine operum.
 \item utrum aliquae virtutes morales sint in nobis per infusionem.
 \item utrum virtus quam acquirimus ex assuetudine operum, sit eiusdem
 speciei cum virtute infusa.
\end{enumerate}

&

次に徳の原因について考察されるべきである。これを巡って四つのことが問われる。

\begin{enumerate}
 \item 徳は私たちの中に本性によってあるか。
 \item ある徳は私たちの中に業の習慣化が原因で生じるか。
 \item 道徳的徳は私たちの中に注入が原因となって生じるか。
 \item 業の習慣化から私たちが獲得する徳は注入された徳と種において同じか。
\end{enumerate}
\end{longtable}

\newpage


\rhead{a.~1}
\begin{center}
{\Large {\bf ARTICULUS PRIMUS}}\\
{\large UTRUM VIRTUS INSIT NOBIS A NATURA}\\
{\footnotesize Supra, q.55, a.1; I {\itshape Sent.}, d.17, q.1, a.3; II, d.39, q.2, a.1; III, d.33, q.1, a.2, qu$^{a}$1;\\ {\itshape De Verit.}, q.9, a.1; {\itshape De Virtut.}, q.1, a.8; II {\itshape Ethic.}, lect.1.}\\
{\Large 第一項\\徳は本性によって私たちに内在するか}
\end{center}

\begin{longtable}{p{21em}p{21em}}

{\scshape Ad primum sic proceditur}. Videtur quod virtus sit in nobis
a natura. Dicit enim Damascenus, in III libro, {\itshape naturales
sunt virtutes, et aequaliter insunt omnibus}. Et Antonius dicit, in
sermone ad monachos, {\itshape si naturam voluntas mutaverit,
perversitas est; conditio servetur, et virtus est}. Et
Matth.~{\scshape iv}, super illud, {\itshape circuibat Iesus} etc.,
dicit Glossa, {\itshape docet naturales iustitias, scilicet
castitatem, iustitiam, humilitatem, quas naturaliter habet homo}.

&

 第一項の問題へ、議論は以下のように進められる。徳は私たちの中で本性に
 よってあると思われる。理由は以下の通り。ダマスケヌスはその書物の第三
 巻で「徳は本性的でありすべての人に等しく内在する」と言う。またアント
 ニウスはモナコの人々への説教の中で「もし意志が本性を動かすならば、そ
 れは歪みである。状態が保持されるようにしなさい。そうすれば徳である」
 と言う。また『マタイによる福音書』第4章「イエスは巡回した」云々につい
 て『註解』は「彼は本性的な正義、すなわち、人間が本性的に持っている貞
 潔、正義、謙遜を教える」と言っている。

 
\\


2.~{\scshape Praeterea}, bonum virtutis est secundum rationem esse, ut
ex dictis patet. Sed id quod est secundum rationem, est homini
naturale, cum ratio sit hominis natura. Ergo virtus inest homini a
natura.

&

 さらに、徳の善は、上述のことから明らかなとおり、理性に即してあること
 である。しかるに理性に即してあるものは、人間にとって本性的である。な
 ぜなら、理性は人間の本性だから。ゆえに徳は本性によって人間に内在する。

 
\\



3.~{\scshape Praeterea}, illud dicitur esse nobis naturale, quod nobis
a nativitate inest. Sed virtutes quibusdam a nativitate insunt,
dicitur enim Iob.~{\scshape xxxi}, {\itshape ab infantia
crevit\footnote{ind.3.s.pf. of cresco or cerno} mecum miseratio, et de
utero egressa est mecum}. Ergo virtus inest homini a natura.


&

 さらに、私たちに生まれながら内在するものが、私たちにとって本性的なも
 のと言われる。しかるに徳はある人々に生まれながらに内在している。たと
 えば『ヨブ記』31章「憐れみは小さいときから私と共に育ち、私と共に胎内
 から出た」\footnote{そうだ、私は若い頃から父親のように彼を育て/私の
 母の胎にいたときから彼女を導いたのだ。(31:18)}と言われている。ゆえに
 徳は本性によって人間に内在する。
 
 
\\



{\scshape Sed contra}, id quod inest homini a natura, est omnibus
hominibus commune, et non tollitur per peccatum, quia etiam in
Daemonibus bona naturalia manent, ut Dionysius dicit, in {\scshape iv}
cap.~{\itshape de Div.~Nom}. Sed virtus non inest omnibus hominibus;
et abiicitur per peccatum. Ergo non inest homini a natura.


&

 しかし反対に、人間に本性によって内在するものはすべての人間に共通し、
 罪によってなくなることもない。なぜなら、ディオニュシウスが『神名論』
 第4章で言うように、悪魔においてさえ、本性的な善いものが残っているから
 である。しかるに徳はすべての人間に内在しないし、罪によって捨てられる。
 ゆえに人間に本性によって内在するのではない。

 
\\



 {\scshape Respondeo dicendum} quod circa formas corporales, aliqui
 dixerunt quod sunt totaliter ab intrinseco, sicut ponentes
 latitationem formarum. Aliqui vero, quod totaliter sint ab
 extrinseco, sicut ponentes formas corporales esse ab aliqua causa
 separata. Aliqui vero, quod partim sint ab intrinseco, inquantum
 scilicet praeexistunt in materia in potentia; et partim ab
 extrinseco, inquantum scilicet reducuntur ad actum per agens.


&

 解答する。以下のように言われるべきである。物体的形相を巡って,ある人々
 はそれらが全く内側からであると言った。たとえば諸形相の隠れを考えた人々
 である。他方、ある人々は、それらが全く外側からであると言った。たとえ
 ば物体的形相は何らかの離在した原因によってあると考えた人々である。さ
 らに他方、ある人々は、それが部分的に内側から、すなわち質料の中に可能
 態において先在し、部分的に外側から、すなわち作用者による作用へ還元さ
 れると言った。

 
\\



 Ita etiam circa scientias et virtutes, aliqui quidem posuerunt eas
 totaliter esse ab intrinseco, ita scilicet quod omnes virtutes et
 scientiae naturaliter praeexistunt in anima; sed per disciplinam et
 exercitium impedimenta scientiae et virtutis tolluntur, quae
 adveniunt animae ex corporis gravitate; sicut cum ferrum clarificatur
 per limationem. Et haec fuit opinio Platonicorum.


&

 このように、知や徳を巡っても、ある人々はそれらが全く内側からであると
 考えた。すなわちすべての徳と知は本性的に魂の中に内在していて、学習と
 訓練によって、ちょうど鉄が磨かれることによって輝く場合のように、知と
 徳の障害が取り除かれると考えた。そしてこれがプラトン派の人々の意見だっ
 た。

 
 
\\


 Alii vero dixerunt quod sunt totaliter ab extrinseco, idest ex
 influentia intelligentiae agentis, ut ponit Avicenna.


&

 他の人々は、それらが全く外側から、すなわち能動知性の流入によってある
 と言った。アヴィセンナがそう考えているように。

 
\\


 Alii vero dixerunt quod secundum aptitudinem scientiae et virtutes
 sunt in nobis a natura, non autem secundum perfectionem, ut
 philosophus dicit, in II {\itshape Ethic}. Et hoc verius est.


&

 さらに他の人々は、知と徳の適合性に即しては、本性によって私たちの中に
 あるが、完全性に即してあるのではない、と言った。ちょうど哲学者が『ニ
 コマコス倫理学』第2巻で言うように。そしてこれがより真である。

 
\\


 Ad cuius manifestationem, oportet considerare quod aliquid dicitur
 alicui homini naturale dupliciter, uno modo, ex natura speciei; alio
 modo, ex natura individui. Et quia unumquodque habet speciem secundum
 suam formam, individuatur vero secundum materiam; forma vero hominis
 est anima rationalis, materia vero corpus, id quod convenit homini
 secundum animam rationalem, est ei naturale secundum rationem
 speciei; id vero quod est ei naturale secundum determinatam corporis
 complexionem, est ei naturale secundum naturam individui. Quod enim
 est naturale homini ex parte corporis secundum speciem, quodammodo
 refertur ad animam, inquantum scilicet tale corpus est tali animae
 proportionatum.


&

 これを明らかにするために、以下のことが考察されるべきである。あるもの
 がある人に本性的であるのに二通りの仕方がある。一つには種の本性に基づ
 いてであり、もう一つは個人の本性に基づいてである。そして各々のものは
 自らの形相に即して種を持ち、質料に即して個体化されるが、人間の形相は
 理性的魂であり質料は身体なので、人間に理性的魂に即して適合するものは
 種の性格に即して人間に本性的だが、身体の特定の構成に即して人間に本性
 的であるものは、個人の本性に即して本性的である。じっさい、種に即して
 身体の側から人間にとって本性的なものは、ある意味で、すなわちそのよう
 な身体がそのような魂に比例している限りにおいて、魂に関係づけられる。
 
\\

 Utroque autem modo virtus est homini naturalis secundum quandam
 inchoationem. Secundum quidem naturam speciei, inquantum in ratione
 homini insunt naturaliter quaedam principia naturaliter cognita tam
 scibilium quam agendorum, quae sunt quaedam seminalia intellectualium
 virtutum et moralium; et inquantum in voluntate inest quidam
 naturalis appetitus boni quod est secundum rationem.


&

 さて、どちらの仕方によっても、徳は人間にとってその端緒に関して本性的
 である。実に種の本性に即して言えば、それは以下の点においてそうである。
 すなわち、人間の理性の中に、知られうる事柄についてと同様に為されるべ
 き事柄についても本性的に知られたある種の原理が本性的に内在しており、
 それらは知性的徳と道徳的徳の種子のようなものである。また、意志の中に
 も理性に従う本性的な善への欲求が内在する。
 

 
\\

 Secundum vero naturam individui, inquantum ex corporis dispositione
 aliqui sunt dispositi vel melius vel peius ad quasdam virtutes, prout
 scilicet vires quaedam sensitivae actus sunt quarundam partium
 corporis, ex quarum dispositione adiuvantur vel impediuntur huiusmodi
 vires in suis actibus, et per consequens vires rationales, quibus
 huiusmodi sensitivae vires deserviunt.

&

 他方、個人の本性に即してそうであるのは、以下の点においてである。すな
 わち、身体の態勢に基づいて、ある人々はある徳に、よりよく、あるいはよ
 り悪く態勢付けられる。すなわち、ある種の感覚的な力の作用は身体のある
 部分に属し、それらの態勢によってそのような力がその作用において、また
 結果的に理性的な力が、助けられたり妨げられたりする。このような感覚的
 な力は理性的な力に奉仕するからである。

 
\\


 Et secundum hoc, unus homo habet naturalem aptitudinem ad scientiam,
 alius ad fortitudinem, alius ad temperantiam. Et his modis tam
 virtutes intellectuales quam morales, secundum quandam aptitudinis
 inchoationem, sunt in nobis a natura.


&

 そしてこの限りにおいて、ある人間は学知への本性的な適合性を持つが、別
 の人は勇気への、また別の人は節制への適合性を持つ。このしかたで、知性
 的徳も道徳的徳も、ある種の端緒の適合性に即して、私たちに本性によって
 内在する。

 
\\


 Non autem consummatio earum. Quia natura determinatur ad unum:
 consummatio autem huiusmodi virtutum non est secundum unum modum
 actionis, sed diversimode, secundum diversas materias in quibus
 virtutes operantur, et secundum diversas circumstantias.

&

しかし、それらの達成\footnote{consummatioを「達成」と訳してみる。「成
就」「完成」「仕上げ」などと訳しうる。}はそうでない。なぜなら、本性は
一つのものへ限定される。しかしこのような徳の達成は、作用の一つの仕方に
よるのではなく、徳が働くさまざまな質料(対象)に応じて、またさまざまな
環境に応じてさまざまな仕方による。
 
\\


 Sic ergo patet quod virtutes in nobis sunt a natura secundum
 aptitudinem et inchoationem, non autem secundum perfectionem, praeter
 virtutes theologicas, quae sunt totaliter ab extrinseco.

&

 ゆえに徳は私たちの中に適合性と端緒に即して本性によってあるが、完全性
 に即してではない。ただし、全く外的なものに由来する神学的徳の場合を除
 いて。

 
\\


Et per hoc patet responsio ad obiecta. Nam primae duae rationes
procedunt secundum quod seminalia virtutum insunt nobis a natura,
inquantum rationales sumus. Tertia vero ratio procedit secundum quod
ex naturali dispositione corporis, quam habet ex nativitate, unus
habet aptitudinem ad miserendum, alius ad temperate vivendum, alius ad
aliam virtutem.


&

そしてこれによって、異論に対する解答は明らかである。すなわち、最初の二
つの論は徳の種子的なものが、私たちが理性的なものである限りにおいて私た
ちの中に本性によって内在することに即して論じている。第三異論は、生まれ
たときから持っている身体の本性的な態勢に即して、ある人が憐れむ事へ、別
の人は節度をもって生きることへ、また別の人は別の徳への適合性をもってい
ることから論じている。
 


\end{longtable}
\newpage

\rhead{a.~2}
\begin{center}
{\Large {\bf ARTICULUS SECUNDUS}}\\
{\large UTRUM ALIQUA VIRTUS CAUSETUR IN NOBIS EX ASSUETUDINE OPERUM}\\
{\footnotesize Supra, q.51, a.2; II {\itshape Sent.}, d.44, q.1, a.1, ad 6; III, d.33, q.1, a.2, qu$^{a}$2; {\itshape De Virtut.}, q.1, a.9; II {\itshape Ethic.}, lect.1.}\\
{\Large 第二項\\ある徳は業の習慣化が原因となって私たちの中に生じるか}
\end{center}

\begin{longtable}{p{21em}p{21em}}
{\scshape Ad secundum sic proceditur}. Videtur quod virtutes in nobis
causari non possint ex assuetudine operum. Quia super illud {\itshape
Rom}.~{\scshape xiv}, {\itshape omne quod non est ex fide, peccatum
est, dicit Glossa Augustini, omnis infidelium vita peccatum est; et
nihil est bonum sine summo bono. Ubi deest cognitio veritatis, falsa
est virtus etiam in optimis moribus}. Sed fides non potest acquiri ex
operibus, sed causatur in nobis a Deo; secundum illud {\itshape
Ephes}.~{\scshape ii}, {\itshape gratia estis salvati per fidem}. Ergo
nulla virtus potest in nobis acquiri ex assuetudine operum.

&

 第二項の問題へ、議論は以下のように進められる。徳が私たちの中に、業の
 習慣化が原因となって生じることは不可能だと思われる。理由は以下の通り。
 かの『ローマの信徒への手紙』14章「すべて信仰からでないものは罪である」
 \footnote{「しかし、疑いながら食べる人は、罪に定められます。信仰に基
 づいていないからです。信仰に基づいていないことはすべて、罪なのです。」
 (13:23)}を注釈して、アウグスティヌスの註解は次のように述べている。
 「不信仰な人々の人生はすべてが罪である。何者も最高善なしに善ではない。
 真理の認識がないところに最善の習俗においてすら徳は偽である」。しかる
 に、信仰は業によって獲得されるのではなく神が原因となって私たちのうち
 に生じる。これは『エフェソの信徒への手紙』第2章「恩恵によってあなたた
 ちは信仰を通して救われた」\footnote{「あなたがたは恵みにより、信仰を
 通して救われたのです。それは、あなたがたの力によるのではなく、神の賜
 物です。」(2:8)}による。ゆえにどんな徳も私たちの中に、業の習慣化によっ
 て獲得されることはできない。
 
\\



2.~{\scshape Praeterea}, peccatum, cum contrarietur virtuti, non
compatitur secum virtutem. Sed homo non potest vitare peccatum nisi
per gratiam Dei; secundum illud {\itshape Sap}.~{\scshape viii},
{\itshape didici quod non possum esse aliter continens, nisi Deus
det}. Ergo nec virtutes aliquae possunt in nobis causari ex
assuetudine operum; sed solum dono Dei.

&

さらに、罪は徳に反対するので、徳と両立しない。しかるに人間は神の恩恵な
しに罪を避けることはできない。これは『知恵の書』8章、「神が与えないな
らば、それ以外の仕方で私は節制していることはできない、ということを学ん
だ」\footnote{「しかし、神が与えてくださるのでなければ/知恵を得ること
はできないのであり/誰からの恵みであるかを知るのは/賢明なことであると
私は知っていた。/そこで私は主に向かって祈り/心の底からこのように言っ
た。」(8:21)}による。ゆえに業の習慣化によって私たちの中に何らかの徳が
生じることはありえず、それはただ神の賜物にのみ基づく。
 
\\



3.~{\scshape Praeterea}, actus qui sunt in virtutem, deficiunt a
perfectione virtutis. Sed effectus non potest esse perfectior
causa. Ergo virtus non potest causari ex actibus praecedentibus
virtutem.

&

さらに、徳へ向かう行為は徳の完全性には届かない。しかるに結果は原因より
も完全ではありえない。ゆえに徳は徳に先行する行為が原因となって生じるこ
とはできない。
 
\\



{\scshape Sed contra est} quod Dionysius dicit, {\scshape iv}
cap.~{\itshape de Div.~Nom}., quod bonum est virtuosius quam
malum. Sed ex malis actibus causantur habitus vitiorum. Ergo multo
magis ex bonis actibus possunt causari habitus virtutum.

&

しかし反対に、ディオニュシウスは『神名論』第4章で善は悪よりも有徳だと
言う。しかるに悪い行為が原因となって悪徳の習慣が生じる。ゆえに、まして
や、善い行為が原因となって徳の習慣が生じうる。
 
\\



{\scshape Respondeo dicendum} quod de generatione habituum ex actibus,
supra in generali dictum est. Nunc autem specialiter quantum ad
virtutem, considerandum est quod sicut supra dictum est, virtus
hominis perficit ipsum ad bonum.

&

解答する。以下のように言われるべきである。作用からの習慣の生成について
は一般的に前に\footnote{Q.51, a.2, a.3.}語られた。今は特殊的に徳につい
て、上述のように\footnote{Q.55, a.3, a.4.}人間の徳が人間を善に向けて完
成することが考慮されるべきである。
 
\\


 Cum autem ratio boni consistat in {\itshape modo, specie et ordine},
ut Augustinus dicit in libro {\itshape de Natura Boni}; sive in
{\itshape numero, pondere et mensura}, ut dicitur {\itshape
Sap}.~{\scshape xi}, oportet quod bonum hominis secundum aliquam
regulam consideretur.

&

 さて、アウグスティヌスが『善の本性について』で言うように、善の性格は
 「限度、形象、秩序」において、あるいは『知恵の書』第11章で言われるよ
 うに「数、重さ、尺度」\footnote{「しかしあなたは、物差しと数と秤とに
 よって/すべてを按配された。」(11:20)}において成立するので、人間の善
 は何らかの規則に即して考察されなければならない。

 
\\

 Quae quidem est duplex, ut supra dictum est, scilicet ratio humana,
et lex divina. Et quia lex divina est superior regula, ideo ad plura
se extendit, ita quod quidquid regulatur ratione humana, regulatur
etiam lege divina, sed non convertitur.

&

 上述の如く、規則は二通りにある。すなわち、人間の理性と神の法である。
 神の法は規則の上にあるので複数のものへと及ぶ。それゆえ、人間の理性に
 よって測られるものはなんであれ,神の法によっても測られるが、逆は成り
 立たない。

 
\\

 Virtus igitur hominis ordinata ad bonum quod modificatur secundum
regulam rationis humanae, potest ex actibus humanis causari, inquantum
huiusmodi actus procedunt a ratione, sub cuius potestate et regula
tale bonum consistit.

&

 ゆえに、人間の理性の規則に即して変容された善へと秩序付けられている人
 間の徳は、人間の行為が理性から発し、その権能と規則の元に善が成立する
 限りにおいて、人間の作用が原因となって生じうる。

 
\\


 Virtus vero ordinans hominem ad bonum secundum quod modificatur per
legem divinam, et non per rationem humanam, non potest causari per
actus humanos, quorum principium est ratio, sed causatur solum in
nobis per operationem divinam. Et ideo, huiusmodi virtutem definiens,
Augustinus posuit in definitione virtutis, {\itshape quam Deus in
nobis sine nobis operatur}.


&

 他方で、人間の理性ではなく神の法によって変容されたかぎりでの善へ人間
 を秩序付ける徳は、その根源が理性である人間の作用が原因となって生じる
 ことはできず、ただ神の働きが原因となって私たちのうちに生じる。ゆえに
 このような徳を定義して、アウグスティヌスは徳の定義の中に、「神が私た
 ちの中に私たちなしに働く」と入れたのである。
 
 
\\


Et de huiusmodi etiam virtutibus prima ratio procedit.

&

そしてこのような徳について、第一異論は論じ進めている。
 
\\

 {\scshape Ad secundum dicendum} quod virtus divinitus infusa, maxime
 si in sua perfectione consideretur, non compatitur secum aliquod
 peccatum mortale. Sed virtus humanitus acquisita potest secum compati
 aliquem actum peccati, etiam mortalis, quia usus habitus in nobis est
 nostrae voluntati subiectus, ut supra dictum est; non autem per unum
 actum peccati corrumpitur habitus virtutis acquisitae; habitui enim
 non contrariatur directe actus, sed habitus.

&

 第二異論に対しては以下のように言われるべきである。神から注入された徳
 は、最大限に、もしその完全性において考察されるならば、何らかの大罪と
 両立したりはしない。しかし、人間的な仕方で獲得された徳は、何らかの罪
 の作用と、それが大罪であったとしても、両立しうる。なぜなら、私たちの
 中での習慣の使用は、上述の如く私たちの意志に従属するが、一つの罪の作
 用によって獲得された徳の習慣が消滅することはないからである。というの
 も、習慣に直接的に反対するのは習慣であって作用ではないからである。
 
\\



 Et ideo, licet sine gratia homo non possit peccatum mortale vitare,
 ita quod nunquam peccet mortaliter; non tamen impeditur quin possit
 habitum virtutis acquirere, per quam a malis operibus abstineat ut in
 pluribus, et praecipue ab his quae sunt valde rationi contraria. Sunt
 etiam quaedam peccata mortalia quae homo sine gratia nullo modo
 potest vitare, quae scilicet directe opponuntur virtutibus
 theologicis, quae ex dono gratiae sunt in nobis. Hoc tamen infra
 manifestius fiet.


&

ゆえに、たしかに人間は恩恵なしに、決して大罪を犯さないというかたちで大
罪を避けることができないが、しかし、徳の習慣を獲得し、その徳によって悪
い業から多くの場合に、とくに理性に大いに反対するものから遠ざかることは
可能である。さらに、人間が恩恵なしには避けることがけっしてできないある
種の大罪、つまり神学的徳に真っ向から対立するものは、恩恵の賜物に基づい
て私たちの中にある。しかしこれは、後により明らかにされる。
 
\\



{\scshape Ad tertium dicendum} quod, sicut dictum est, virtutum
acquisitarum praeexistunt in nobis quaedam semina sive principia,
secundum naturam. Quae quidem principia sunt nobiliora virtutibus
eorum virtute acquisitis, sicut intellectus principiorum speculabilium
est nobilior scientia conclusionum; et naturalis rectitudo rationis
est nobilior rectificatione appetitus quae fit per participationem
rationis, quae quidem rectificatio pertinet ad virtutem moralem. Sic
igitur actus humani, inquantum procedunt ex altioribus principiis,
possunt causare virtutes acquisitas humanas.


&

 第三異論に対しては以下のように言われるべきである。すでに述べられたと
 おり、獲得された諸徳の何らかの種や根源は、本性的に私たちの中に先在す
 る。これらの根源は、それらのちからによって獲得される諸徳よりも高貴で
 ある。たとえば観照的な諸原理の直知は結論についての学知よりも高貴であ
 り、理性の本性的な正しさは理性の分有によって生じる欲求を正しくするこ
 と高貴であるように。そしてこの正しくすることは、道徳的徳に属する。ゆ
 えにこのようにして、より上位の根源から出てくる限りにおいて、人間的な
 行為が原因となって獲得された人間の徳が生じうる。
 

\end{longtable}
\newpage


\rhead{a.~3}
\begin{center}
{\Large {\bf ARTICULUS TERTIUS}}\\
{\large UTRUM ALIQUAE VIRTUTES MORALES SINT IN NOBIS PER INFUSIONEM}\\
{\footnotesize Supra, q.51, a.4; III {\itshape Sent.}, d.33, q.1, a.2, qu$^{a}$3; {\itshape De Virtut.}, q.1, a.10.}\\
{\Large 第三項\\何らかの道徳的徳が私たちの中に注入によってあるか}
\end{center}

\begin{longtable}{p{21em}p{21em}}
{\scshape Ad tertium sic proceditur}. Videtur quod praeter virtutes
theologicas, non sint aliae virtutes nobis infusae a Deo. Ea enim quae
possunt fieri a causis secundis, non fiunt immediate a Deo, nisi forte
aliquando miraculose, quia, ut Dionysius dicit, {\itshape lex
divinitatis est ultima per media adducere}. Sed virtutes
intellectuales et morales possunt in nobis causari per nostros actus,
ut dictum est. Non ergo convenienter causantur in nobis per
infusionem.

&

 第三項の問題へ、議論は以下のように進められる。神学的徳以外に、私たち
 に神から注入された他の徳はないと思われる。理由は以下の通り。第二原因
 によって生じうるものは、奇跡以外には、直接的に神によって生じることは
 ない。なぜならディオニュシウスが言うように「神の法は中間のものを通っ
 て究極のものをもたらす」からである。しかるに知性的徳と道徳的徳は、上
 述の如く、私たちの作用が原因となって私たちの中に生じる。ゆえに注入が
 原因となって私たちの中に生じるのは適切でない。
 
\\



2.~{\scshape Praeterea}, in operibus Dei multo minus est aliquid
superfluum quam in operibus naturae. Sed ad ordinandum nos in bonum
supernaturale, sufficiunt virtutes theologicae. Ergo non sunt aliae
virtutes supernaturales, quas oporteat in nobis causari a Deo.

&

 さらに、神の業の中には、自然の業におけるよりも無駄なものははるかに少
 ない。しかるに私たちを本性を越える善へと秩序付けることは神学的徳で十
 分である。ゆえに私たちの中に神が原因となって生じなければならないよう
 な自然を越える他の徳は存在しない。
 
\\



3.~{\scshape Praeterea}, natura non facit per duo, quod potest facere
per unum, et multo minus Deus. Sed Deus inseruit animae nostrae semina
virtutum, ut dicit Glossa {\itshape Heb}.~{\scshape i}. Ergo non
oportet quod alias virtutes in nobis per infusionem causet.

&

 さらに、自然は一つでできることを二つで行うことはない。ましてや神もそ
 うである。しかるに、『へブル人への手紙』への注釈が言うように、神は私
 たちの魂に徳の種を差し入れた。ゆえに他の徳を私たちの中に注入によって
 生み出す必要はない。
 
\\



{\scshape Sed contra est} quod dicitur {\itshape Sap}.~{\scshape
viii}, {\itshape sobrietatem et iustitiam docet, prudentiam et
virtutem}.

&

 しかし反対に、『知恵の書』第8章では「節度と正義、思慮と徳を彼は教える」
 \footnote{「人が義を愛するのならば/知恵の働きこそが徳である。/節制
 と賢明、正義と勇気の徳を教えてくれるのは/知恵であり/人生においてこ
 れらの徳よりも有益なものはない。」(8:7)} と言われている。
 
\\



{\scshape Respondeo dicendum} quod oportet effectus esse suis causis
et principiis proportionatos. Omnes autem virtutes tam intellectuales
quam morales, quae ex nostris actibus acquiruntur, procedunt ex
quibusdam naturalibus principiis in nobis praeexistentibus, ut supra
 dictum est.


&

 解答する。以下のように言われるべきである。結果はその原因と根源に比例
 したものでなければならない。しかるに私たちの作用から獲得されるすべて
 の知性的徳も道徳的徳も、上述の如く、私たちの中に先在する何らかの本性
 的な根源から出てくる。
 
\\


 Loco quorum naturalium principiorum, conferuntur nobis a
Deo virtutes theologicae, quibus ordinamur ad finem supernaturalem,
sicut supra dictum est. Unde oportet quod his etiam virtutibus
theologicis proportionaliter respondeant alii habitus divinitus
causati in nobis, qui sic se habeant ad virtutes theologicas sicut se
habent virtutes morales et intellectuales ad principia naturalia
virtutum.

&

 本性的な根源の場所に、神によって神学的徳が私たちにもたらされ、上述の
 如く、それによって私たちは本性を越えた目的へ秩序付けられる。このこと
 から、この神学的徳にも比例的に対応する神に由来する他の習慣が私たちの
 中に生じる。これらは神学的徳に対して、ちょうど道徳的徳と知性的徳が徳
 の本性的根源に対応するように関係する。
 
\\



{\scshape Ad primum ergo dicendum} quod aliquae quidem virtutes
morales et intellectuales possunt causari in nobis ex nostris actibus,
tamen illae non sunt proportionatae virtutibus theologicis. Et ideo
oportet alias, eis proportionatas, immediate a Deo causari.

&

 第一異論に対しては、それゆえ以下のように言われるべきである。たしかに
 私たちの中に私たちの行為が原因となって生じる道徳的徳や知性的徳がある。
 しかしそれらは神学的徳に比例的に関係しない。ゆえに他の、それに対応す
 るような徳が、直接神が原因となって生じなければならない。
 
\\

{\scshape Ad secundum dicendum} quod virtutes theologicae sufficienter
nos ordinant in finem supernaturalem, secundum quandam inchoationem,
quantum scilicet ad ipsum Deum immediate. Sed oportet quod per alias
virtutes infusas perficiatur anima circa alias res, in ordine tamen ad
Deum.

&

 第二異論に対しては以下のように言われるべきである。神学的徳は、そのあ
 る端緒に即して、すなわち直接的に神へ、十分に私たちを本性を越えた目的
 へと秩序付ける。しかし魂は、他の注入された諸徳によって、他の事物を巡っ
 て、ただし神への秩序において、完成される必要がある。
 
\\



{\scshape Ad tertium dicendum} quod virtus illorum principiorum
naturaliter inditorum, non se extendit ultra proportionem naturae. Et
ideo in ordine ad finem supernaturalem, indiget homo perfici per alia
principia superaddita.

&

 第三異論に対しては以下のように言われるべきである。本性的に植え付けら
 れているかの根源の徳は、本性の比を越えて自らを拡張させることはない。
 ゆえに本性を越えた目的への秩序において、人間は、付加された他の根源に
 よって完成されることを必要とする。


\end{longtable}
\newpage

\rhead{a.~4}
\begin{center}
{\Large {\bf ARTICULUS QUARTUS}}\\
{\large UTRUM VIRTUS QUAM ACQUIRIMUS EX OPERUM ASSUETUDINE,\\SIT EIUSDEM SPECIEI CUM VIRTUTE INFUSA}\\
{\footnotesize III {\itshape Sent.}, d.33, q.1, a.2, qu$^{a}$4; {\itshape De Virtut.}, q.1, a.10, ad 7, 8, 9; q.5, a.4.}\\
{\Large 第四項\\私たちが業の習慣化によって獲得する徳は\\注入された徳と種において同じか}
\end{center}

\begin{longtable}{p{21em}p{21em}}
{\scshape Ad quartum sic proceditur}. Videtur quod virtutes infusae
non sint alterius speciei a virtutibus acquisitis. Virtus enim
acquisita et virtus infusa, secundum praedicta, non videntur differre
nisi secundum ordinem ad ultimum finem. Sed habitus et actus humani
non recipiunt speciem ab ultimo fine, sed a proximo. Non ergo virtutes
morales vel intellectuales infusae differunt specie ab acquisitis.


&

 第四項の問題へ、議論は以下のように進められる。注入された徳が獲得され
 た徳と違う種に属するわけではないと思われる。理由は以下の通り。獲得さ
 れた徳と注入された徳は、上述のことによれば、究極目的への秩序に即して
 のみ異なると思われる。しかし人間的な習慣と作用は究極目的から種を受け
 取るのではなく最近接の目的から受け取る。ゆえに注入された道徳的徳と知
 性的徳は、獲得されたそれらから種において異なるわけではない。
 
\\


2.~{\scshape Praeterea}, habitus per actus cognoscuntur. Sed idem est
actus temperantiae infusae, et acquisitae, scilicet moderari
concupiscentias tactus. Ergo non differunt specie.

&

 さらに、習慣は作用を通して認識される。しかるに注入された節制と獲得さ
 れた節制の作用は同一、すなわち、触覚の欲情を穏やかにすることである。
 ゆえにそれらは種において違わない。
 
\\




3.~{\scshape Praeterea}, virtus acquisita et infusa differunt secundum
illud quod est immediate a Deo factum, et a creatura. Sed idem est
specie homo quem Deus formavit, et quem generat natura; et oculus quem
caeco nato dedit, et quem virtus formativa causat. Ergo videtur quod
est eadem specie virtus acquisita, et infusa.

&

 さらに獲得された徳と注入された徳は、神によって直接的に作られたか、被
 造物によって作られたかの違いである。しかるに神が形成した人間と、自然
 が生み出す人間は種において同一であり、生まれつきの盲目の人に(神が)
 与えた目と、(自然の)形成する力が原因となって生じた目とは種において
 同一である。ゆえに獲得された徳と注入された徳とは種において同一だと思
 われる。
 
\\




{\scshape Sed contra}, quaelibet differentia in definitione posita,
mutata diversificat speciem. Sed in definitione virtutis infusae
ponitur, quam Deus in nobis sine nobis operatur, ut supra dictum
est. Ergo virtus acquisita, cui hoc non convenit, non est eiusdem
speciei cum infusa.

&

 しかし反対に、定義の中に置かれたどんな差異も、もしそれが変化するなら、
 種が変わる。しかるに注入された徳の定義の中には、上述の如く、それを神
 が、私たちの中に私たちなしに働くことが置かれている。ゆえに獲得された
 徳は、これに適合しないのだから、注入された徳と同じ種に属すことはない。
 
\\


{\scshape Respondeo dicendum} quod dupliciter habitus distinguuntur
specie. Uno modo, sicut praedictum est, secundum speciales et formales
rationes obiectorum. Obiectum autem virtutis cuiuslibet est bonum
consideratum in materia propria, sicut temperantiae obiectum est bonum
delectabilium in concupiscentiis tactus.

&

 解答する。以下のように言われるべきである。習慣は二通りの仕方で種にお
 いて区別される。一つには、すでに述べられたとおり、対象の種的で形相的
 な性格に即してである。さて、どんな徳の対象も固有の質料において考察さ
 れた善である。たとえば節制の対象は触覚の欲情における快楽の善である。
 
\\


Cuius quidem obiecti formalis ratio est a ratione, quae instituit
modum in his concupiscentiis, materiale autem est id quod est ex parte
 concupiscentiarum.

 &

 この対象の形相的性格は理性によってあり、理性はこれらの欲情の中に中庸
 を課す。これに対して質料的な性格は欲情的部分の側からある。

 \\

 Manifestum est autem quod alterius rationis est
modus qui imponitur in huiusmodi concupiscentiis secundum regulam
rationis humanae, et secundum regulam divinam.

&

ところで、このような欲情において人間の理性の規則に従って課される中庸と、
神の規則において課される中庸とでは性格が異なる。
 
\\


Puta in sumptione ciborum, ratione humana modus statuitur ut non
noceat valetudini corporis, nec impediat rationis actum, secundum
autem regulam legis divinae, requiritur quod homo {\itshape castiget corpus
suum, et in servitutem redigat}, per abstinentiam cibi et potus, et
aliorum huiusmodi. Unde manifestum est quod temperantia infusa et
acquisita differunt specie, et eadem ratio est de aliis virtutibus.

&

 たとえば、食べ物の摂取において、人間的性格によって立てられる中庸は、
 身体の強さを損なわず、また理性の作用を妨げないということだが、神の法
 の規則によれば、人間は、飲食や他のそのようなものから離れて、「自己の
 身体を矯正し、奉仕させること」\footnote{「競技をする人は皆、すべてに
 節制します。彼らは朽ちる冠を受けるためにそうするのですが、私たちは朽
 ちない冠を受けるために節制するのです。ですから、私は、やみくもに走っ
 たりしないし、空を打つような拳闘もしません。 むしろ、自分の体を打ち叩
 いて従わせます。他の人に宣教しておきながら、自分のほうが失格者となら
 ないためです。」(I {\itshape Ad Cor}, 9:25-27)}が求められる。したがって注入された節
 制と獲得された節制が種において異なることは明らかであり、考え方は他の
 徳についても同じである。
 
\\


Alio modo habitus distinguuntur specie secundum ea ad quae ordinantur,
non enim est eadem specie sanitas hominis et equi, propter diversas
naturas ad quas ordinantur. Et eodem modo dicit philosophus, in III
{\itshape Polit}., quod diversae sunt virtutes civium, secundum quod
bene se habent ad diversas politias. Et per hunc etiam modum differunt
specie virtutes morales infusae, per quas homines bene se habent in
ordine ad hoc quod sint {\itshape cives sanctorum et domestici Dei};
et aliae virtutes acquisitae, secundum quas homo se bene habet in
ordine ad res humanas.


&

 他の仕方で、習慣はそれへと秩序付けられるところのものに即して、種にお
 いて区別される。たとえば人間と馬の健康は、それへと秩序付けられている
 ところの本性の異なりのために、種において同一でない。同じしかたで哲学
 者は『政治学』第3巻で「異なる政治体制へ善く関係する限りにおいて、諸々
 の市民の徳は異なっていると言う。この仕方でも、人間がそれによって「聖
 なる人々と神の家族に属する」\footnote{「ですから、あなたがたは、もは
 やよそ者でも寄留者でもなく、聖なる者たちと同じ民であり、神の家族の一
 員です。」({\itshape Ad Ephes}. (2:19)}ことへの秩序において善くあると
 ころの注入された道徳的徳と、人間が人間的な事柄への秩序において善くあ
 るところの獲得された徳とは、種において異なる。
 
\\

{\scshape Ad primum ergo dicendum} quod virtus infusa et acquisita non
solum differunt secundum ordinem ad ultimum finem; sed etiam secundum
ordinem ad propria obiecta, ut dictum est.


&

 第一異論に対しては、それゆえ、以下のように言われるべきである。注入さ
 れた徳と獲得された徳は、究極目的への秩序において異なるだけでなく、す
 でに述べられたとおり、固有の対象への秩序においても異なる。
 
\\




{\scshape Ad secundum dicendum} quod alia ratione modificat
concupiscentias delectabilium tactus temperantia acquisita, et
temperantia infusa, ut dictum est. Unde non habent eundem actum.

&

 第二異論に対しては以下のように言われるべきである。すでに述べられたと
 おり、獲得された節制と注入された節制は、異なる観点から、触覚の快楽へ
 の欲情を制御する。したがって同一の作用を持つわけではない。
 
\\




{\scshape Ad tertium dicendum} quod oculum caeci nati Deus fecit ad
eundem actum ad quem formantur alii oculi secundum naturam, et ideo
fuit eiusdem speciei. Et eadem ratio esset, si Deus vellet miraculose
causare in homine virtutes quales acquiruntur ex actibus. Sed ita non
est in proposito, ut dictum est.

 &

 第三異論に対しては以下のように言われるべきである。神が生まれつき盲目
 の人の目を作ったのは、自然に即して他の目がそのために形成されるところ
 の作用と同一の作用のためにであり、それゆえ、同じ種に属した。もし神が
 原因となって、奇跡的に、人間の中に行為から獲得されるような徳を生じさ
 せることを意志したならば、同じ理屈となったであろう。しかしすでに述べ
 られたとおり、神の意図はそうではない。
 
\end{longtable}
\end{document}