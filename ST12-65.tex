\documentclass[10pt]{jsarticle}
\usepackage{okumacro}
\usepackage{longtable}
\usepackage[polutonikogreek,english,japanese]{babel}
\usepackage{latexsym}
\usepackage{color}
\usepackage{schemata}
\usepackage[T1]{fontenc}
\usepackage{lmodern}

%----- header -------
\usepackage{fancyhdr}
\pagestyle{fancy}
\lhead{{\it Summa Theologiae} I-II, q.65}
%--------------------

\bibliographystyle{jplain}

\title{{\bf PRIMA SECUNDAE}\\{\HUGE Summae Theologiae}\\Sancti Thomae
Aquinatis\\{\sffamily QUEAESTIO SEXAGESIMAQUINTA}\\DE CONNEXIONE VIRTUTUM}
\author{Japanese translation\\by Yoshinori {\sc Ueeda}}
\date{Last modified \today}

%%%% コピペ用
%\rhead{a.~}
%\begin{center}
% {\Large {\bf }}\\
% {\large }\\
% {\footnotesize }\\
% {\Large \\}
%\end{center}
%
%\begin{longtable}{p{21em}p{21em}}
%
%&
%
%
%\\
%
%\end{longtable}
%\newpage

\begin{document}

\maketitle
\thispagestyle{empty}

\begin{center}
{\LARGE 『神学大全』第二部の一}\\
{\Large 第六十五問\\徳の結びつきについて}
\end{center}

\begin{longtable}{p{21em}p{21em}}
Deinde considerandum est de connexione virtutum. Et circa hoc
quaeruntur quinque. 

\begin{enumerate}
 \item utrum virtutes morales sint ad invicem connexae.
 \item utrum virtutes morales possint esse sine caritate.
 \item utrum caritas possit esse sine eis.
 \item utrum fides et spes possint esse sine caritate.
 \item utrum caritas possit esse sine eis.
\end{enumerate}

&

 次に徳の結びつきについて考察されるべきである。これを巡っては五つのことが問われる。

\begin{enumerate}
 \item 道徳的徳は相互に結びついているか。
 \item 道徳的徳は愛徳なしにありうるか。
 \item 愛徳はそれらなしにありうるか。
 \item 信仰と希望は愛徳なしにありうるか。
 \item 愛徳はそれらなしにありうるか。
\end{enumerate}
\end{longtable}

\newpage



\rhead{a.~1}
\begin{center}
{\Large {\bf ARTICULUS PRIMUS}}\\
{\large UTRUM VIRTUTES MORALES SINT AD INVICEM CONNEXAE}\\
 {\footnotesize III {\itshape Sent.}, d.36, a.1; d.33, q.3, a.2, ad 6; {\itshape De Virtut.}, q.5, a.2; {\itshape Quodl}.~XII, q.15, a.1; VI {\itshape Ethic.}, lect.11.}\\
{\Large 第一項\\道徳的徳は相互に結びついているか}
\end{center}

\begin{longtable}{p{21em}p{21em}}
{\scshape Ad primum sic proceditur}. Videtur quod virtutes morales non
sint ex necessitate connexae. Virtutes enim morales quandoque
causantur ex exercitio actuum, ut probatur in II {\itshape Ethic}. Sed
homo potest exercitari in actibus alicuius virtutis sine hoc quod
exercitetur in actibus alterius virtutis. Ergo una virtus moralis
potest haberi sine altera.

&

 第一項の問題へ、議論は以下のように進められる。道徳的徳は必然的に結び
 ついているのではないと思われる。理由は以下の通り。『ニコマコス倫理学』
 第2巻で証明されているように、道徳的徳は作用の遂行が原因となって生じる
 ことがある。しかるに人間は、ある徳の作用において働きながら、別の徳の
 作用においては働かないことがありうる。ゆえにある道徳的徳が、別の道徳
 的徳なしに所有されうる。

\\




2. {\scshape Praeterea}, magnificentia et magnanimitas sunt quaedam
virtutes morales. Sed aliquis potest habere alias virtutes morales,
sine hoc quod habeat magnificentiam et magnanimitatem, dicit enim
philosophus, in IV {\itshape Ethic}., quod {\itshape inops non potest
esse magnificus}, qui tamen potest habere quasdam alias virtutes; et
quod {\itshape ille qui parvis est dignus, et his se dignificat,
temperatus est, magnanimus autem non est}. Ergo virtutes morales non
sunt connexae.

&

 さらに、豪毅と高邁は道徳的徳に数えられる。しかるに人は、豪毅と高邁を
 もたずに、他の道徳的徳を持つことがありうる。というのも哲学者が『ニコ
 マコス倫理学』第4巻で「困窮している人は豪毅でありえない」が、しかし他
 の何らかの徳を持つことはありうるし、「小さいことに相応しく、自分をそ
 ういうことに相応しいとする人は、節制ある人だが高邁ではない」と言って
 いるからである。ゆえに道徳的徳は結びついていない。

\\

3.~{\scshape Praeterea}, sicut virtutes morales perficiunt partem
appetitivam animae, ita virtutes intellectuales perficiunt partem
intellectivam. Sed virtutes intellectuales non sunt connexae, potest
enim aliquis habere unam scientiam, sine hoc quod habeat aliam. Ergo
etiam neque virtutes morales sunt connexae.

&

 さらに、道徳的徳が魂のの欲求的部分を完成させるように、知性的徳は知性
 的部分を完成させる。しかるに知性的徳は結び付けられていない。なぜなら、
 人はある知をもちながら別の知をもたないことがあるからである。ゆえに道
 徳的徳も結び付けられていない。

\\




4.~{\scshape Praeterea}, si virtutes morales sint connexae, hoc non
est nisi quia connectuntur in prudentia. Sed hoc non sufficit ad
connexionem virtutum moralium. Videtur enim quod aliquis possit esse
prudens circa agibilia quae pertinent ad unam virtutem, sine hoc quod
sit prudens in his quae pertinent ad aliam, sicut etiam aliquis potest
habere artem circa quaedam factibilia, sine hoc quod habeat artem
circa alia. Prudentia autem est recta ratio agibilium. Ergo non est
necessarium virtutes morales esse connexas.


&

 さらに、道徳的徳が結び付けられているならば、思慮において結びつけられ
 ているからに他ならない。しかしこれは、道徳的徳を結び付けるのには十分
 でない。なぜなら、人はある徳に属する作用可能な事柄を巡って、別の徳に
 属する事柄においての思慮なしに、思慮がありうるからである。たとえば、
 人は、制作可能な何かを巡って別の何かを巡っては技術を持たないのに、技
 術を持つことができるように。しかるに思慮は、作用可能な事柄についての
 正しい理である。ゆえに、道徳的徳が結び付けられていることは必然ではな
 い。
 

\\

{\scshape Sed contra est} quod Ambrosius dicit, {\itshape super Lucam}
: {\itshape Connexae sibi sunt, concatenataeque virtutes, ut qui unam
habet, plures habere videatur}. Augustinus etiam dicit, in VI
{\itshape de Trin}., quod {\itshape virtutes quae sunt in animo
humano, nullo modo separantur ab invicem}. Et Gregorius dicit, XXII
{\itshape Moral}., quod {\itshape una virtus sine aliis aut omnino
nulla est, aut imperfecta}. Et Tullius dicit, in II {\itshape de
Tuscul.~Quaest}., {\itshape si unam virtutem confessus es te non
habere, nullam necesse est te habiturum}.

&

 しかし反対に、アンブロシウスは『ルカによる福音書註解』で「諸徳は互い
 に結び付けられ結合されているので、一つの徳をもつ人は複数の徳をもつと
 思われる」と述べている。またアウグスティヌスは『三位一体論』第6巻で
 「人間の魂の中にある諸徳は決して相互に切り離されていない」と言う。ま
 たグレゴリウスは『道徳論』第22巻で「一つの徳は、もし別の徳がないなら
 ば、まったく徳でないか、あるいは不完全かである」と言う。そしてキケロ
 は『トゥスクルム荘対談集』で「もしある徳を持っていないことをあなたが
 白状するならば、一つも持つことがないだろうということが必然である」と
 言っている。

\\




{\scshape Respondeo dicendum} quod virtus moralis potest accipi vel
perfecta vel imperfecta. Imperfecta quidem moralis virtus, ut
temperantia vel fortitudo, nihil aliud est quam aliqua inclinatio in
nobis existens ad opus aliquod de genere bonorum faciendum, sive talis
 inclinatio sit in nobis a natura, sive ex assuetudine.


&

 解答する。以下のように言われるべきである。道徳的徳は完全なものとして、
 あるいは不完全なものとして理解されうる。不完全な道徳的徳は、節制や勇
 気のように、なされるべき善の類に属する何らかの業に向かう、私たちの中
 にある何らかの傾向性に他ならない。そしてこのような傾向性は、本性によっ
 て、または習慣化によって、私たちの中にある。

\\

 Et hoc modo accipiendo virtutes morales, non sunt connexae, videmus
enim aliquem ex naturali complexione, vel ex aliqua consuetudine, esse
promptum ad opera liberalitatis, qui tamen non est promptus ad opera
castitatis.


&

 そしてこのように理解するならば、道徳的徳は結び付けられていない。とい
 うのも、私たちは人が、生まれつきの成り立ちに基づいて、あるいは慣れに
 基づいて、気前の善さに属する業はすぐ行うが、禁欲に属する業はそうでな
 いことを見るからである。

\\


 Perfecta autem virtus moralis est habitus inclinans in bonum opus
 bene agendum. Et sic accipiendo virtutes morales, dicendum est eas
 connexas esse; ut fere ab omnibus ponitur. Cuius ratio duplex
 assignatur, secundum quod diversimode aliqui virtutes cardinales
 distinguunt.

&

 これに対して完全な道徳的徳は善くなされるべき善い業へと傾ける習慣であ
 る。道徳的徳をこのように理解するとき、ほとんどすべての人によって考え
 られているように、それらは結びついていると言わなければならない。この
 理由は、人々がさまざまに枢要徳を区別していることに即して二通りに指定
 されうる。

\\

 Ut enim dictum est, quidam distinguunt eas secundum quasdam generales
 conditiones virtutum, utpote quod discretio pertineat ad prudentiam,
 rectitudo ad iustitiam, moderantia ad temperantiam, firmitas animi ad
 fortitudinem, in quacumque materia ista considerentur. 

 &

 すでに述べられたとおり、ある人々はそれらを徳の何らかの類的な条件に即
 して区別する。たとえば、それらがどんな質料にいおいて考察される場合で
 も、分別は思慮に、正直さ正義に、穏健さは節制に、魂の堅固さは勇気に属
 する、というように。


\\

 Et secundum hoc, manifeste apparet ratio connexionis, non enim
 firmitas habet laudem virtutis, si sit sine moderatione, vel
 rectitudine, aut discretione; et eadem ratio est de aliis. Et hanc
 rationem connexionis assignat Gregorius, XXII {\itshape Moral}.,
 dicens quod {\itshape virtutes, si sint disiunctae, non possunt esse
 perfectae, secundum rationem virtutis, quia nec prudentia vera est
 quae iusta, temperans et fortis non est}; et idem subdit de aliis
 virtutibus. Et similem rationem assignat Augustinus, in VI de Trin.

&

 そしてこの限りでは、結びつきの根拠は明らかである。すなわち、堅固さは、
 もし穏健さや正直さや分別を伴わなければ徳の賞賛を持たないし、その他に
 ついても同様である。そしてこの結びつきの根拠をグレゴリウスは『道徳講
 話』第22巻で次のように述べて指定している。「諸徳は、もし結びついてい
 ないならば徳の性格に即して完全なものではありえない。なぜなら正しく、
 節制され、強くないならば、思慮も真のものではないからである」。そして
 同じ性格をアウグスティヌスは『三位一体論』第6巻で指定している。

\\


 Alii vero distinguunt praedictas virtutes secundum materias. Et
 secundum hoc assignatur ratio connexionis ab Aristotele, in VI
 Ethic. Quia sicut supra dictum est, nulla virtus moralis potest sine
 prudentia haberi, eo quod proprium virtutis moralis est facere
 electionem rectam, cum sit habitus electivus; ad rectam autem
 electionem non solum sufficit inclinatio in debitum finem, quod est
 directe per habitum virtutis moralis; sed etiam quod aliquis directe
 eligat ea quae sunt ad finem, quod fit per prudentiam, quae est
 consiliativa et iudicativa et praeceptiva eorum quae sunt ad finem.

&

 他方、他の人々は上述の徳を質料に即して区別している。そしてこの限りで、
 結びつきの根拠がアリストテレスによって『ニコマコス倫理学』第6巻で指定
 されている。というのも、上述の如く、どんな道徳的徳も思慮なしに所有さ
 れることはありえないからである。その理由は以下の通りである。道徳的徳
 は選択的徳なので、正しい選択をなすことがそれに固有である。しかるに正
 しい選択には、道徳的徳の習慣によって直接的に行われる、しかるべき目的
 への傾向性だけでは十分でなく、人が直接的に、目的のためにあるものを選
 択することもまた必要である。そしてこれは、目的のためにあるものを比量
 し判断し命令する思慮によってなされる。
 

\\

 Similiter etiam prudentia non potest haberi nisi habeantur virtutes
 morales, cum prudentia sit recta ratio agibilium, quae, sicut ex
 principiis, procedit ex finibus agibilium, ad quos aliquis recte se
 habet per virtutes morales. Unde sicut scientia speculativa non
 potest haberi sine intellectu principiorum, ita nec prudentia sine
 virtutibus moralibus. Ex quo manifeste sequitur virtutes morales esse
 connexas.

&

 同様に、道徳的な徳が所持されていないならば思慮もまた所持されえない。
 というのも、思慮は実行可能な事柄にかんする正しい理であり、その理は、
 あたかも原理からであるようにして、実行可能な目的から出てくるが、その
 目的へ人が正しく関係するのは道徳的徳によってだからである。したがって、
 ちょうど観照的学知が第一諸原理なしに所有されえないように、思慮も道徳
 的徳なしには所有されえない。このことから明らかに、道徳的徳が結びつい
 ていることが帰結する。

\\



{\scshape Ad primum ergo dicendum} quod virtutum moralium quaedam
perficiunt hominem secundum communem statum, scilicet quantum ad ea
quae communiter in omni vita hominum occurrunt agenda. Unde oportet
quod homo simul exercitetur circa materias omnium virtutum
 moralium.


&

 第一異論に対しては、それゆえ、以下のように言われるべきである。道徳的
 徳の中のあるものは、人間を共通の状態において、すなわち人々のすべての
 生活において共通に生じる為されるべき事柄にかんして、完成する。したがっ
 て人はみな、全ての道徳的な徳の質料をめぐって、実践することが必要であ
 る。
 

\\



 Et si quidem circa omnes exercitetur bene operando, acquiret habitus
omnium virtutum moralium. Si autem exercitetur bene operando circa
unam materiam, non autem circa aliam, puta bene se habendo circa iras,
non autem circa concupiscentias; acquiret quidem habitum aliquem ad
refrenandum iras, qui tamen non habebit rationem virtutis, propter
defectum prudentiae, quae circa concupiscentias corrumpitur. Sicut
etiam naturales inclinationes non habent perfectam rationem virtutis,
si prudentia desit.

&

 そしてもし全てのことを巡って善く働くことによって実践されるなら、その
 人は全ての道徳的徳の習慣を獲得するだろう。しかしもし、ある質料につい
 ては善く働き、別の質料についてはそうでないならば、たとえば怒りを巡っ
 ては善く行うが、欲情についてはそうでないならば、怒りをなだめることに
 ついてはある習慣を獲得するが、それは徳の性格を持たないだろう。なぜな
 ら、それは欲情を巡っては消滅する思慮を欠くためである。ちょうど生まれ
 ながらの傾向性もまた、思慮を欠くならば徳の完全な性格を持たないのと同
 様である。

\\

Quaedam vero virtutes morales sunt quae perficiunt hominem secundum
aliquem eminentem statum, sicut magnificentia, et magnanimitas. Et
quia exercitium circa materias harum virtutum non occurrit unicuique
communiter, potest aliquis habere alias virtutes morales, sine hoc
quod habitus harum virtutum habeat actu, loquendo de virtutibus
acquisitis.

&

 他方で、ある道徳的徳は、豪毅や高邁のように、人間を何らかの卓越する状
 態に即して完成する。そしてこれらの徳の質料を巡る実践は、だれにでも共
 通に起こるというわけではないので、獲得された徳について語るならば、あ
 る人はこれらの徳の習慣を現実に持つことなしに、他の道徳的徳を持つとい
 うことがありうる。

\\

 Sed tamen, acquisitis aliis virtutibus, habet istas virtutes in
potentia propinqua. Cum enim aliquis per exercitium adeptus est
liberalitatem circa mediocres donationes et sumptus, si superveniat ei
abundantia pecuniarum, modico exercitio acquiret magnificentiae
habitum, sicut geometer modico studio acquirit scientiam alicuius
conclusionis quam nunquam consideravit. Illud autem habere dicimur,
quod in promptu est ut habeamus; secundum illud philosophi, in II
{\itshape Physic}., quod {\itshape parum deest, quasi nihil deesse
videtur.}

&

 しかし、他の徳が獲得されていれば、その人はこれらの徳を、最近接の可能
 態において所有している。というのも、実践を通して少額の贈与と消費を巡っ
 て「気前の善さ」を獲得した人は、もし潤沢な資金が手に入ったら、少しの
 実践で豪毅の習慣を獲得するだろうからである。それはちょうど、幾何学者
 が少しの努力で、今まで一度も考えたことがない何らかの結論についての知
 を獲得するようなものである。ところで、これを私たちが持つと言われるの
 は、それを持つのが迅速だからであり、それは『自然学』第2巻の哲学者「少
 し欠けているものはまったく欠けていないように見える」による。
 

\\


Et per hoc patet responsio ad secundum.

&

 これによって、第二異論への解答は明らかである。

\\


{\scshape Ad tertium dicendum} quod virtutes intellectuales sunt circa
diversas materias ad invicem non ordinatas, sicut patet in diversis
scientiis et artibus. Et ideo non invenitur in eis connexio quae
invenitur in virtutibus moralibus existentibus circa passiones et
 operationes, quae manifeste habent ordinem ad invicem.

&

第三異論に対しては以下のように言われるべきである。知性的徳は、さまざま
な学知や技術において明らかなように、相互に秩序付けられた様々な質料(対
象)を巡ってある。それゆえそれらにおいて、明らかに相互の秩序をもつ情念
や働きを巡ってある道徳的徳において見出されるような結びつきは見出されな
い。
 
\\

 Nam omnes passiones, a quibusdam primis procedentes, scilicet amore
et odio, ad quasdam alias terminantur, scilicet delectationem et
tristitiam. Et similiter omnes operationes quae sunt virtutis moralis
materia, habent ordinem ad invicem, et etiam ad passiones. Et ideo
tota materia moralium virtutum sub una ratione prudentiae cadit.


&

 すなわち、全ての情念は、愛や憎しみのようないくつかの第一のものから出
 てきて、快楽や悲しみのようないくつかの他のものへと終局する。同様に、
 道徳的徳の質料である全ての働きも、相互に、そして情念へも秩序を持って
 いる。ゆえに、道徳的徳の質料全体は、思慮の一つの観点(ratio)のもとに含
 まれる。

\\



 Habent tamen omnia intelligibilia ordinem ad prima principia. Et
secundum hoc, omnes virtutes intellectuales dependent ab intellectu
principiorum; sicut prudentia a virtutibus moralibus, ut dictum est.


&

 これに対して、全ての可知的なものは第一原理への秩序をもつ。そしてその
 かぎりにおいて、全ての知性的徳は諸原理についての直知に依存する。それ
 はちょうど、すでに述べられたとおり\footnote{本項主文、最後のパラグラ
 フを参照。}、思慮が道徳的徳に依存するようにである。

\\

 Principia autem universalia, quorum est intellectus principiorum, non
dependent a conclusionibus, de quibus sunt reliquae intellectuales
virtutes; sicut morales dependent a prudentia, eo quod appetitus movet
quodammodo rationem, et ratio appetitum, ut supra dictum est.

&

 しかし、諸原理についての直知がかかわる普遍的な諸原理は、その他の知性
 的徳がかかわる結論に依存しない。これは道徳的徳が思慮に依存するのとは
 異なる。後者が起こるのは、上述の如く、あるしかたで欲求は理性を、理性
 は欲求を動かすからである。

\\



{\scshape Ad quartum dicendum} quod ea ad quae inclinant virtutes
morales, se habent ad prudentiam sicut principia, non autem factibilia
se habent ad artem sicut principia, sed solum sicut
 materia.

 &

 第四異論に対しては次のように言われるべきである。道徳的徳がそれへと傾
 く対象は、思慮に対して、根源として関係するが、作られうるものは技術に
 対して根源としては関係せず、ただ質料としてのみ関係する。
 
 \\

 Manifestum est autem quod, etsi ratio possit esse recta in una parte
materiae, et non in alia; nullo tamen modo potest dici ratio recta, si
sit defectus cuiuscumque principii.


 &

 ところで、理性が質料(対象)のある部分において正しいが、別の部分にお
 いて正しくないということがありうることは明らかである。しかし、もしそ
 の理性がなんであれ原理を欠いていれば、決して正しいとは言われえない。
 
 \\


 Sicut si quis erraret circa hoc principium, omne totum est maius sua
parte, non posset habere scientiam geometricam, quia oporteret multum
recedere a veritate in sequentibus.

 &

 たとえば、もし人が「全ての全体はその部分よりも大きい」という原理を巡っ
 て謝るならば、幾何学を所有することはできない。なぜなら、そこから帰結
 する事柄において大いに真理から外れるはずだからである。
 
 \\

 Et praeterea, agibilia sunt ordinata ad invicem; non autem
factibilia, ut dictum est. Et ideo defectus prudentiae circa unam
partem agibilium, induceret defectum etiam circa alia agibilia. Quod
in factibilibus non contingit.
 
 &

 さらに、為されうる事柄は相互に秩序付けられている。しかしすでに述べら
 れたとおり、作られうるものはそうでない。ゆえに、為されうる事柄の一つ
 の部分を巡る思慮の欠点は、他の為されうる事柄を巡る欠点を引き起こす。
 作られうるものにおいては、このようなことは起こらない。
 
\end{longtable}
\newpage



\rhead{a.~2}
\begin{center}
{\Large {\bf ARTICULUS SECUNDUS}}\\
{\large UTRUM VIRTUTES MORALES POSSINT ESSE SINE CARITATE}\\
{\footnotesize II$^{a}$II$^{ae}$, q.23, a.7; III {\itshape Sent.}, d.27, q.2, a.4, qu$^{a}$3, ad 2; d.36, a.1; {\itshape De Virtut.}, q.5, a.2.}\\
{\Large 第二項\\道徳的徳は愛徳なしにありうるか}
\end{center}

\begin{longtable}{p{21em}p{21em}}
 {\scshape Ad secundum sic proceditur}. Videtur quod virtutes morales
 possint esse sine caritate. Dicitur enim in libro {\itshape
 Sententiarum} Prosperi, quod {\itshape omnis virtus praeter
 caritatem, potest esse communis bonis et malis. Sed caritas non
 potest esse nisi in bonis}, ut dicitur ibidem. Ergo aliae virtutes
 possunt haberi sine caritate.

 
&

 第二項の問題へ、議論は以下のように進められる。道徳的徳は愛徳なしにあ
 りうると思われる。理由は以下の通り。プロスペルスの『命題論集』という
 書物で「愛徳以外の全ての徳は、善と悪に共通するが、愛徳は善においての
 みありうる」と言われている。ゆえに他の諸徳は愛徳なしに所有されうる。

\\

2.~{\scshape Praeterea}, virtutes morales possunt acquiri ex actibus
humanis, ut dicitur in II {\itshape Ethic}. Sed caritas non habetur
nisi ex infusione; secundum illud {\itshape Rom}.~{\scshape v},
{\itshape caritas Dei diffusa est in cordibus nostris per spiritum
sanctum, qui datus est nobis}. Ergo aliae virtutes possunt haberi sine
caritate.
 
&

 さらに、『ニコマコス倫理学』第2巻で言われるように、道徳的徳は人間的行
 為に基づいて獲得されうる。しかるに愛徳は注入によってのみ所有される。
 これはかの『ローマの信徒への手紙』第5章「神の愛徳が、私たちに与えられ
 た聖霊を通して私たちの心に注がれた」\footnote{「この希望が失望に終わ
 ることはありません。私たちに与えられた聖霊によって、神の愛が私たちの
 心に注がれているからです。」(5:5)}による。ゆえに他の徳は愛徳なしに所
 有されうる。

\\

3.~{\scshape Praeterea}, virtutes morales connectuntur ad invicem,
inquantum dependent a prudentia. Sed caritas non dependet a prudentia;
immo prudentiam excedit, secundum illud {\itshape Ephes}.~{\scshape
iii}, {\itshape supereminentem scientiae caritatem Christi}. Ergo
virtutes morales non connectuntur caritati, sed sine ea esse possunt.

 
&

 さらに、道徳的徳は、思慮に依存する限りにおいて、相互に結びついている。
 しかるに愛徳は思慮に依存しない。むしろかの『エフェソの信徒への手紙』
 第3章「キリストの、知識を越える愛を」\footnote{「人知をはるかに超えた
 キリストの愛を知ることができ、神の満ち溢れるものすべてに向かって満た
 されますように。」(3:19)}によれば、愛徳は思慮を越える。ゆえに道徳的徳
 は愛徳に結び付けられず、むしろそれなしにありうる。

\\


 {\scshape Sed contra est} quod dicitur I {\itshape Ioan}.~{\scshape
 iii}, {\itshape qui non diligit, manet in morte}. Sed per virtutes
 perficitur vita spiritualis, ipsae enim sunt {\itshape quibus recte
 vivitur}, ut Augustinus dicit, in II {\itshape de Lib.~Arbit}. Ergo
 non possunt esse sine dilectione caritatis.
 
&

 しかし反対に、『ヨハネの手紙一』第3章「愛さない人は死に留まる」
 \footnote{「私たちは、自分が死から命へと移ったことを知っています。きょ
 うだいを愛しているからです。愛することのない者は、死の内にとどまって
 います。」(3:14)}と言われる。しかるに徳によって、霊的な生は完成される。
 なぜなら、アウグスティヌスが『自由意志』第2巻で言うように、徳は「それ
 によって正しくいきられるところのもの」だからである。ゆえに愛徳の愛な
 しに徳はありえない。

\\


 {\scshape Respondeo dicendum} quod, sicut supra dictum est, virtutes
 morales prout sunt operativae boni in ordine ad finem qui non excedit
 facultatem naturalem hominis, possunt per opera humana acquiri.  Et
 sic acquisitae sine caritate esse possunt, sicut fuerunt in multis
 gentilibus.
 
&

 解答する。以下のように言われるべきである。上述の如く、人間本性の機能
 を越えない目的への秩序において善に働きかけるものとしての道徳的徳は、
 人間の業によって獲得されうる。そしてそのようにして、それらは愛徳なし
 に獲得されうる。それはちょうど多くの異教徒において起こっているように
 である。


\\


 Secundum autem quod sunt operativae boni in ordine ad ultimum finem
 supernaturalem, sic perfecte et vere habent rationem virtutis; et non
 possunt humanis actibus acquiri, sed infunduntur a Deo. Et huiusmodi
 virtutes morales sine caritate esse non possunt.
 
&

 しかし、本性を越えた究極の目的へと秩序づけられた善に働きかける限りに
 おいて、徳は完全で真の徳の性格をもち、人間的行為によっては獲得されず、
 神から注入される。そしてこのような道徳的徳は愛徳なしにはありえない。

\\

 Dictum est enim supra quod aliae virtutes morales non
 possunt esse sine prudentia; prudentia autem non potest esse sine
 virtutibus moralibus, inquantum virtutes morales faciunt bene se
 habere ad quosdam fines, ex quibus procedit ratio prudentiae.
 
&

その理由は以下の通りである。他の道徳的徳が思慮なしにありえないことは前
に述べられた。さらに思慮は道徳的徳なしにありえない。それは道徳的徳が、
思慮の性格がそこから出てくるある特定の目的へと関係させる限りにおいてで
ある。


\\

 Ad rectam autem rationem prudentiae multo magis requiritur quod homo
 bene se habeat circa ultimum finem, quod fit per caritatem, quam
 circa alios fines, quod fit per virtutes morales, sicut ratio recta
 in speculativis maxime indiget primo principio indemonstrabili, quod
 est contradictoria non simul esse vera.
 
&

 しかるに思慮の正しい性格には、道徳的徳によって生じる他の目的を巡って
 よりも、愛徳によって生じる究極の目的を巡って人が善く関係することが必
 要とされる。それはちょうど観照的な事柄における正しい理性が、論証され
 えない第一原理、すなわち矛盾するものは同時に真でありえない、という原
 理を最大限に必要とするのと同じである。
 
\\

 Unde manifestum fit quod nec
 prudentia infusa potest esse sine caritate; nec aliae virtutes
 morales consequenter, quae sine prudentia esse non possunt.

 
&

 したがって、注入された思慮も愛徳なしにありえないことが明らかとなる。
 また思慮なしにありえない他の道徳的徳も、結果的に愛徳がないとありえな
 い。

\\


 Patet igitur ex dictis quod solae virtutes infusae sunt perfectae, et
 simpliciter dicendae virtutes, quia bene ordinant hominem ad finem
 ultimum simpliciter.
 
&

 ゆえに、すでに述べられたことから、注入された徳だけが完全であり、端的
 に徳と言われることが明らかである。なぜならそれは究極の目的へ端的に人
 を善く秩序付けるからである。

\\

 Aliae vero virtutes, scilicet acquisitae, sunt secundum quid
 virtutes, non autem simpliciter, ordinant enim hominem bene respectu
 finis ultimi in aliquo genere, non autem respectu finis ultimi
 simpliciter.
 
&

 他の徳、すなわち獲得された徳は、ある意味において徳であって端的にでは
 ない。なぜならそれらは人間を、ある類における究極の目的へと善く秩序付
 けるが、端的に究極目的に関してそうするのではないからである。

\\




 Unde {\itshape Rom}.~{\scshape xiv} super illud, {\itshape omne quod
 non est ex fide, peccatum est}, dicit Glossa Augustini, {\itshape ubi
 deest agnitio veritatis, falsa est virtus etiam in bonis moribus}.
 
&

 このことから『ローマの信徒への手紙』第14章「信仰から出ていないものは
 全て罪である」\footnote{「しかし、疑いながら食べる人は、罪に定められ
 ます。信仰に基づいていないからです。信仰に基づいていないことはすべて、
 罪なのです。」(14:23)}を註解してアウグスティヌスの註解は「真理の認識
 がないところでは、善い習俗の中にあっても徳は偽物である」と言う。

\\

 {\scshape Ad primum ergo dicendum} quod virtutes ibi accipiuntur
 secundum imperfectam rationem virtutis. Alioquin, si virtus moralis
 secundum perfectam rationem virtutis accipiatur, bonum facit
 habentem; et per consequens in malis esse non potest.
 
&

 第一異論に対しては、それゆえ、以下のように言われるべきである。ここで
 徳は、徳の不完全な性格に即して理解されている。もしそうでなく、道徳的
 徳が徳の完全な性格に即して理解されるならば、所有する人を善い人にし、
 結果的に、その人は悪においてあることができない。

\\



 {\scshape Ad secundum dicendum} quod ratio illa procedit de
 virtutibus moralibus acquisitis.

 
&

第二異論に対しては以下のように言われるべきである。彼の異論は獲得された
道徳的徳について論じている。
 

\\

 {\scshape Ad tertium dicendum} quod, etsi caritas excedat scientiam
 et prudentiam, tamen prudentia dependet a caritate, ut dictum est. Et
 per consequens, omnes virtutes morales infusae.

 
&

第三異論に対しては以下のように言われるべきである。愛徳が学知と思慮を越
えているとしても、すでに述べられたとおり、思慮は愛徳に依存する。そして
結果的に、注入された全ての道徳的な徳も愛徳に依存する。



\end{longtable}
\newpage

\rhead{a.~3}
\begin{center}
{\Large {\bf ARTICULUS TERTIUS}}\\
{\large UTRUM CARITAS POSSIT ESSE SINE ALIIS VIRTUTIBUS MORALIBUS}\\
{\footnotesize III {\itshape Sent.}, d.36, a.2; {\itshape De Virtut.}, q.5, a.2.}\\
{\Large 第三項\\愛徳は他の道徳的徳なしにありうるか}
\end{center}

\begin{longtable}{p{21em}p{21em}}
 {\scshape Ad tertium sic proceditur}. Videtur quod caritas sine aliis
 virtutibus moralibus haberi possit. Ad id enim ad quod sufficit unum,
 indebitum est quod plura ordinentur. Sed sola caritas sufficit ad
 omnia opera virtutis implenda, ut patet per id quod dicitur I
 {\itshape ad Cor}.~{\scshape xiii}, {\itshape Caritas patiens est,
 benigna est}, et cetera. Ergo videtur quod, habita caritate, aliae
 virtutes superfluerent.
 
&

 第三項の問題へ、議論は以下のように進められる。愛徳は他の道徳的徳がな
 くても所持されうると思われる。理由は以下の通り。一つのもので十分であ
 るものへ、複数のものが秩序付けられる必要はない。しかるに『コリントの
 信徒への手紙一』第13章で「愛徳は耐え、優しく云々」\footnote{「愛は忍
 耐強い。愛は情け深い。妬まない。愛は自慢せず、高ぶらない。礼を失せず、
 自分の利益を求めず、怒らず、悪をたくらまない。不正を喜ばず、真理を共
 に喜ぶ。すべてを忍び、すべてを信じ、すべてを望み、すべてに耐える。」
 (13:4-7) }と言われていることによれば、愛徳だけで他のすべての徳の業が
 満たされるのに十分である。ゆえに、愛徳が所持されると、他の徳は余計だ
 と思われる。

\\

2.~{\scshape Praeterea}, qui habet habitum virtutis, de facili
operatur ea quae sunt virtutis, et ei secundum se placent, unde et
{\itshape signum habitus est delectatio quae fit in opere}, ut dicitur
in II {\itshape Ethic}. Sed multi habent caritatem, absque peccato
mortali existentes, qui tamen difficultatem in operibus virtutum
patiuntur, neque eis secundum se placent, sed solum secundum quod
referuntur ad caritatem. Ergo multi habent caritatem, qui non habent
alias virtutes.
 
&

 さらに、徳の習慣を持つ人は、徳に属することを容易に行い、それ自体に即
 してそれを喜ぶ。このことから『ニコマコス倫理学』第2巻で言われるように
 「習慣のしるしは、業において生じる喜びである」。しかるに多くの人々が、
 大罪なしに存在しているので愛徳を持っているが、彼らは徳の業において困
 難を感じ、それ自体においてその業を気に入っているわけではなく、ただ愛
 徳へ関係するかぎにおいて、気に入っている。ゆえに多くの人は、他の諸徳
 を持たないのに愛徳を持っている。

\\



3.~{\scshape Praeterea}, caritas in omnibus sanctis invenitur. Sed
quidam sunt sancti qui tamen aliquibus virtutibus carent, dicit enim
Beda quod sancti magis humiliantur de virtutibus quas non habent, quam
de virtutibus quas habent, glorientur. Ergo non est necessarium quod
qui habet caritatem, omnes virtutes morales habeat.

 
&

 さらに、愛徳は全ての聖人において見出される。しかるにある聖人たちは何
 らかの徳を欠いている。というのも、ベーダが、聖人たちは持っている徳に
 ついて誇るよりは、持っていない徳についてへりくだる、と言っているから
 である。ゆえに愛徳を持つ人が全ての道徳的徳を持つ必然性はないと思われ
 る。

\\




 {\scshape Sed contra est} quod per caritatem tota lex impletur,
 dicitur enim {\itshape Rom}.~{\scshape xiii}, {\itshape qui diligit
 proximum, legem implevit}. Sed tota lex impleri non potest nisi per
 omnes virtutes morales, quia lex praecipit de omnibus actibus
 virtutum, ut dicitur in V {\itshape Ethic}. Ergo qui habet caritatem,
 habet omnes virtutes morales. Augustinus etiam dicit, in quadam
 epistola, quod caritas includit in se omnes virtutes cardinales.
 
&

 しかし反対に、愛徳によって律法全体が満たされる。というのも『ローマの
 信徒への手紙』第13章で「隣人を愛する人は律法を成就した」
 \footnote{「互いに愛し合うことのほかは、誰に対しても借りがあってはな
 りません。人を愛する者は、律法を全うしているのです。」(13:8)}と言われ
 ているからである。しかるに、律法全体が満たされることは、全ての道徳的
 徳によらなければ不可能である。なぜなら法は全ての徳の行為について命令
 するからである。これは『ニコマコス倫理学』第5巻で言われているようにで
 ある。ゆえに愛徳を持つ人は、全ての道徳的徳を持つ。アウグスティヌスは
 さらに、ある書簡で、愛徳は全ての枢要徳も自らの内に含むと言っている。

\\


 {\scshape Respondeo dicendum} quod cum caritate simul infunduntur
 omnes virtutes morales. Cuius ratio est quia Deus non minus perfecte
 operatur in operibus gratiae, quam in operibus naturae.

 
&

 解答する。以下のように言われるべきである。全ての道徳的徳は愛徳と一緒
 に注入される。その理由は以下の通りである。神は自然の業においてに劣ら
 ず、恩恵の業においても完全に働く。

\\


 Sic autem videmus in operibus naturae, quod non invenitur principium
 aliquorum operum in aliqua re, quin inveniantur in ea quae sunt
 necessaria ad huiusmodi opera perficienda, sicut in animalibus
 inveniuntur organa quibus perfici possunt opera ad quae peragenda
 anima habet potestatem.
 
&

 しかるに、たとえば動物において、それを達成するために魂が権能をもつと
 ころの業が完成されうるような器官が見出されるように、自然の業において、
 ある事物の中に何らかの業の根源が見出されるなら、必ずそこには、そのよ
 うな完成されるべき業に必要とされるものが見出される。
 
\\

 Manifestum est autem quod caritas, inquantum ordinat hominem ad finem
 ultimum, est principium omnium bonorum operum quae in finem ultimum
 ordinari possunt. Unde oportet quod cum caritate simul infundantur
 omnes virtutes morales, quibus homo perficit singula genera bonorum
 operum.
 
&

 ところで、愛徳は、人間を究極目的へと秩序付ける限りにおいて、究極目的
 へ秩序づけられうる全ての善い業の根源であることは明らかである。したがっ
 て、愛徳と一緒に、それによって人間が善い業の一つ一つの類を完成しうる
 全ての道徳的徳が注入されなければならない。

\\

 Et sic patet quod virtutes morales infusae non solum habent
 connexionem propter prudentiam; sed etiam propter caritatem. Et quod
 qui amittit caritatem per peccatum mortale, amittit omnes virtutes
 morales infusas.
 
&

 そしてこのようにして、注入された道徳的徳は思慮のための結びつきのみな
 らず、愛徳のための結びつきも有することが明らかである。そして大罪によっ
 て愛徳を失う人は、注入された全ての道徳的徳を失う。

\\



 {\scshape Ad primum ergo dicendum} quod ad hoc quod actus inferioris
 potentiae sit perfectus, requiritur quod non solum adsit perfectio in
 superiori potentia, sed etiam in inferiori, si enim principale agens
 debito modo se haberet, non sequeretur actio perfecta, si
 instrumentum non esset bene dispositum.

 
&

 第一異論に対しては、それゆえ、以下のように言われるべきである。下位の
 能力の作用が完全であるためには、上位の能力の中だけでなく下位の能力の
 中にも完全性があることが求められる。というのも、もし道具が善く調えら
 れていなかったならば、かりに主要な作用者がしかるべき仕方であったとし
 ても、完全な作用は結果として生まれないからである。

\\

 Unde oportet ad hoc quod homo bene operetur in his quae sunt ad
 finem, quod non solum habeat virtutem qua bene se habeat circa finem,
 sed etiam virtutes quibus bene se habeat circa ea quae sunt ad finem,
 nam virtus quae est circa finem, se habet ut principalis et motiva
 respectu earum quae sunt ad finem. Et ideo cum caritate necesse est
 etiam habere alias virtutes morales.
 
&

 このことから、人間が、目的に対してある事柄において善く働くためには、
 それによって目的を巡って善く関係する徳だけでなく、目的に対してある事
 柄を巡っても善く関係する徳も所持しなければならない。なぜなら、目的を
 巡ってある徳は、目的に対してあるものどもに対して、根源や動かしうるも
 のとして関係するからである。ゆえに、愛徳と共に、他の道徳的徳を持つこ
 とも必要である。

\\

 {\scshape Ad secundum dicendum} quod quandoque contingit quod aliquis
 habens habitum, patitur difficultatem in operando, et per consequens
 non sentit delectationem et complacentiam in actu, propter aliquod
 impedimentum extrinsecus superveniens, sicut ille qui habet habitum
 scientiae, patitur difficultatem in intelligendo, propter
 somnolentiam vel aliquam infirmitatem.

 
&

 第二異論に対しては以下のように言われるべきである。時として、習慣を持っ
 ている人が働きにおいて困難を被り、結果的に作用において快感や喜びを感
 じないということがある。それは何らかの外的な障害がやって来るためにで
 ある。たとえば、学知の習慣を持つ人が、眠気や何らかの病気のために、知
 性認識において困難を被る場合がそれにあたる。

\\

 Et similiter habitus moralium virtutum infusarum patiuntur interdum
 difficultatem in operando, propter aliquas dispositiones contrarias
 ex praecedentibus actibus relictas. Quae quidem difficultas non ita
 accidit in virtutibus moralibus acquisitis, quia per exercitium
 actuum, quo acquiruntur, tolluntur etiam contrariae dispositiones.

&

同様に、注入された道徳的徳の習慣が、時として、先行する行為から残った相
反する何らかの態勢のために、働きにおいて困難を被ることがある。この困難
は、獲得された道徳的徳においてはそのように生じることはない。なぜなら、
それによって徳が獲得される作用の遂行を通して、相反する態勢もまた消える
からである。

\\

 {\scshape Ad tertium dicendum} quod aliqui sancti dicuntur aliquas
 virtutes non habere, inquantum patiuntur difficultatem in actibus
 earum, ratione iam dicta; quamvis habitus omnium virtutum habeant.
 
&

 第三異論に対しては以下のように言われるべきである。ある聖人たちが何ら
 かの徳を持っていないと言われるのは、今述べた理由で、それらの徳の作用
 において困難さを被っている限りにおいてである。ただし彼らは全ての徳の
 習慣を持っている。

\end{longtable}
\newpage


\rhead{a.~4}
\begin{center}
{\Large {\bf ARTICULUS QUARTUS}}\\
{\large UTRUM FIDES ET SPES POSSINT ESSE SINE CARITATE}\\
{\footnotesize II$^{a}$II$^{ae}$, q.23, a.7, ad 1; III {\itshape Sent.}, d.23, q.3, a.1, qu$^{a}$2; d.24, q.2, a.3, qu$^{2}$; I {\itshape Cor.}, cap.13, lect.1.}\\
{\Large 第四項\\信仰と希望は愛徳なしにありうるか}
\end{center}

\begin{longtable}{p{21em}p{21em}}
{\scshape Ad quartum sic proceditur}. Videtur quod fides et spes
nunquam sint sine caritate. Cum enim sint virtutes theologicae,
digniores esse videntur virtutibus moralibus, etiam infusis. Sed
virtutes morales infusae non possunt esse sine caritate. Ergo neque
fides et spes.


&

 第四項の問題へ、議論は以下のように進められる。信仰と希望は決して愛な
 しにないと思われる。理由は以下の通り。神学的徳は道徳的徳よりも偉大だ
 と思われる。それは注入されたものであってもそうである。しかるに注入さ
 れた道徳的徳は愛徳なしにありえない。ゆえに信仰と希望も同様である。

\\




2.~{\scshape Praeterea}, {\itshape nullus credit nisi volens}, sed caritas est in
voluntate ut Augustinus dicit, {\itshape super Ioan}. Sicut perfectio eius, ut
supra dictum est. Ergo fides non potest esse sine caritate.

&

 アウグスティヌスが『ヨハネ福音書註解』で言うように「だれも意志しなけ
 れば信じない」。しかるに愛徳は、上述の如く、意志の完成として意志の中
 にある。ゆえに信仰は愛徳なしにありえない。

\\




3.~{\scshape Praeterea}, Augustinus dicit, in {\itshape Enchirid}.,
quod {\itshape spes sine amore esse non potest}. Amor autem est
caritas, de hoc enim amore ibi loquitur. Ergo spes non potest esse
sine caritate.


&

 さらに、アウグスティヌスは『エンキリディオン』で「希望は愛(amor)なし
 にありえない」と述べている。しかるにこの愛は愛徳である。というのもそ
 こでは、そのような愛について語っているから。ゆえに希望は愛徳なしにあ
 りえない。

\\




{\scshape Sed contra est} quod Matth.~{\scshape i}, dicitur in Glossa
quod {\itshape fides generat spem, spes vero caritatem}. Sed generans
est prius generato, et potest esse sine eo. Ergo fides potest esse
sine spe; et spes sine caritate.

&

 しかし反対に、『マタイによる福音書』第1章の『註解』において、「信仰が
 希望を生み、希望が愛徳を生む」とある。しかるに生むものは生まれるもの
 に先行し、それなしにありうる。ゆえに信仰は希望なしに、希望は愛徳なし
 にありうる。

\\


 {\scshape Respondeo dicendum} quod fides et spes, sicut et virtutes
 morales, dupliciter considerari possunt. Uno modo, secundum
 inchoationem quandam; alio modo, secundum perfectum esse
 virtutis. Cum enim virtus ordinetur ad bonum opus agendum, virtus
 quidem perfecta dicitur ex hoc quod potest in opus perfecte bonum,
 quod quidem est dum non solum bonum est quod fit, sed etiam bene fit.


&

 解答する。以下のように言われるべきである。信仰と希望は、道徳的徳と同
 様に二通りに考察されうる。一つにはある種の端緒においてであり、もう一
 つは徳の完全な存在においてである。その理由は以下の通りである。徳は為
 されるべき善い業へと秩序付けられているので、徳が完全と言われるのは、
 完全な仕方で善い業へ向かうことができることに基づくが、このことは、為
 されることが善であるだけでなく、そのが善く為される場合に生じる。

\\

 Alioquin, si bonum sit quod fit, non autem bene fiat, non erit
 perfecte bonum, unde nec habitus qui est talis operis principium,
 habebit perfecte rationem virtutis. Sicut si aliquis operetur iusta,
 bonum quidem facit, sed non erit opus perfectae virtutis, nisi hoc
 bene faciat, idest secundum electionem rectam, quod est per
 prudentiam, et ideo iustitia sine prudentia non potest esse virtus
 perfecta.


&

 もしそうでないならば、つまり、為されたことは善であるが、それが善いし
 かたで為されていない場合には、それは完全な仕方で善でないだろう。した
 がって、そのような業の根源である習慣もまた、徳の性格を完全な仕方では
 所持しないだろう。たとえば、もしある人が正しいことをするとき、為され
 たことは善だが、それが善い仕方で為されない場合には、つまり思慮によっ
 てあるところの正しい選択に即して為されない場合には、完全な徳の業では
 ないだろう。それゆえ、思慮のない正義は完全な徳でありえない。

\\

 Sic igitur fides et spes sine caritate possunt quidem aliqualiter
 esse, perfectae autem virtutis rationem sine caritate non habent. Cum
 enim fidei opus sit credere Deo; credere autem sit alicui propria
 voluntate assentire, si non debito modo velit, non erit fidei opus
 perfectum.

&

 ゆえにこのようにして、信仰と希望は、ある意味で、愛徳なしにありうるが、
 しかし愛徳がなければ完全な徳の性格を持たない。じっさい、信仰の業は神
 を信じることだが、だれかを信じることは、固有の意志によって同意するこ
 とであるので、もししかるべき仕方で意志しないならば、信仰は完全な業で
 ないことになる。

\\


 Quod autem debito modo velit, hoc est per caritatem, quae perficit
 voluntatem, omnis enim rectus motus voluntatis ex recto amore
 procedit, ut Augustinus dicit, in XIV {\itshape de Civ.~Dei}. Sic
 igitur fides est quidem sine caritate, sed non perfecta virtus, sicut
 temperantia vel fortitudo sine prudentia.

&

 しかるに、しかるべき仕方で意志することは愛徳による。愛徳は意志の根源
 であり、それはアウグスティヌスが『神の国』第14巻で言うように、全ての
 意志の正しい運動は正しい愛から出てくるからである。ゆえにこの意味で、
 信仰は愛徳なしにあるが、それは完全な徳ではない。ちょうど節制や勇気も
 思慮がないならば完全でないように。

\\


 Et similiter dicendum est de spe. Nam actus spei est expectare
 futuram beatitudinem a Deo. Qui quidem actus perfectus est, si fiat
 ex meritis quae quis habet, quod non potest esse sine caritate. Si
 autem hoc expectet ex meritis quae nondum habet, sed proponit in
 futurum acquirere, erit actus imperfectus, et hoc potest esse sine
 caritate.

 &

 そして希望についても同様に言われるべきである。すなわち希望の作用は、
 神から将来の至福を希望することである。実に、その作用が、その人が持っ
 ている功徳に由来するならばその作用は完全であるが、このことは愛徳なし
 にはありえない。しかし、もしまだ持っていないが将来獲得する予定の功徳
 に基づいてこのことを期待するならば、その作用は不完全であり、このよう
 な希望は愛徳なしにありうる。


\\


 Et ideo fides et spes possunt esse sine caritate, sed sine caritate,
 proprie loquendo, virtutes non sunt; nam ad rationem virtutis
 pertinet ut non solum secundum ipsam aliquod bonum operemur, sed
 etiam bene, ut dicitur in II {\itshape Ethic}.

&

 ゆえに、信仰と希望は愛徳なしにありうるが、厳密に言えば、愛徳がなけれ
 ばそれらは徳でない。すなわち、徳の性格には、『ニコマコス倫理学』第2巻
 で言われているように、それに即して何か善いことを私たちが為すだけでな
 く、善い仕方で為すことが属するからである。

\\



{\scshape Ad primum ergo dicendum} quod virtutes morales dependent a
prudentia, prudentia autem infusa nec rationem prudentiae habere
potest absque caritate, utpote deficiente debita habitudine ad primum
principium, quod est ultimus finis. Fides autem et spes, secundum
proprias rationes, nec a prudentia nec a caritate dependent. Et ideo
sine caritate esse possunt; licet non sint virtutes sine caritate, ut
dictum est.


&

 第一異論に対しては、それゆえ、以下のように言われるべきである。道徳的
 徳は思慮に依存する。しかるに注入された思慮は、愛徳がなければ、究極目
 的という第一原理へのしかるべき関係を欠のだから、思慮の性格を持つこと
 ができない。これに対して信仰と希望は、その固有の性格に即せば、思慮に
 も愛徳にも依存していない。ゆえに、すでに述べられたとおり、愛徳を欠け
 ばそれらは徳ではないが、愛徳なしにあることもできる。

\\



{\scshape Ad secundum dicendum} quod ratio illa procedit de fide quae
habet perfectam rationem virtutis.


&

 第二異論に対しては以下のように言われるべきである。この異論は徳の完全
 な性格を持つ信仰について論じている。

\\


{\scshape Ad tertium dicendum} quod Augustinus loquitur ibi de spe,
secundum quod aliquis expectat futuram beatitudinem per merita quae
iam habet, quod non est sine caritate.


&

 第三異論に対しては以下のように言われるべきである。アウグスティヌスは
 そこで、人が将来の至福をすでに負っている功徳によって期待する限りにお
 いて、希望について語っているが、そのことは愛徳なしにはない。

\\


\end{longtable}
\newpage



\rhead{a.~5}
\begin{center}
{\Large {\bf ARTICULUS QUINTUS}}\\
{\large UTRUM CARITAS POSSIT ESSE SINE FIDE ET SPE}\\
{\Large 第五項\\愛徳は信仰と希望なしにありうるか}
\end{center}

\begin{longtable}{p{21em}p{21em}}

{\scshape Ad quintum sic proceditur}. Videtur quod caritas possit esse
sine fide et spe. Caritas enim est amor Dei. Sed Deus potest a nobis
amari naturaliter, etiam non praesupposita fide, vel spe futurae
beatitudinis. Ergo caritas potest esse sine fide et spe.

&

 第五項の問題へ、議論は以下のように進められる。愛徳は信仰と希望なしに
 ありうると思われる。理由は以下の通り。愛徳とは神への愛である。しかる
 に神は私たちによって本性的に愛されうるのであり、これは前提とされる信
 仰や将来の至福への希望なしにそうである。ゆえに愛徳は信仰と希望なしに
 ありうる。
 

\\

2.~{\scshape Praeterea}, caritas est radix omnium virtutum; secundum
illud {\itshape Ephes}.~{\scshape iii}, {\itshape In caritate radicati
et fundati}. Sed radix aliquando est sine ramis. Ergo caritas potest
esse aliquando sine fide et spe et aliis virtutibus.


&

 さらに、『エフェソの信徒への手紙』第3章「愛徳の中に根を下ろし、それに
 基づく人々は」\footnote{「あなたがたの信仰によって、キリストがあなた
 がたの心の内に住んでくださいますように。あなたがたが愛に根ざし、愛に
 基づく者となることによって、」(3:17)}によれば、愛徳は全ての徳の根であ
 る。しかるに根はあっても枝がないこともある。ゆえに愛徳は、信仰や希望
 やその他の徳がなくてもあることがある。

\\


3.~{\scshape Praeterea}, in Christo fuit perfecta caritas. Ipse tamen
non habuit fidem et spem, quia fuit perfectus comprehensor, ut infra
dicetur. Ergo caritas potest esse sine fide et spe.

&

 さらに、キリストの中には完全な愛徳がある。しかしキリストは信仰と愛を
 持っていない。なぜなら、後述の如く、完全に把握する者だからである。ゆ
 えに愛徳は信仰と希望なしにありうる。

\\


{\scshape Sed contra est} quod apostolus dicit, Heb.~{\scshape xi},
{\itshape sine fide impossibile est placere Deo}; quod maxime pertinet
ad caritatem, ut patet; secundum illud {\itshape Proverb}.~{\scshape
viii}, {\itshape ego diligentes me diligo}. Spes etiam est quae
introducit ad caritatem, ut supra dictum est. Ergo caritas non potest
haberi sine fide et spe.


&

 しかし反対に、使徒は『ヘブライ人への手紙』第11章で「信仰なしに神に喜
 ばれることはできない」\footnote{「信仰がなければ、神に喜ばれることは
 できません。神に近づく者は、神が存在しておられること、また、神がご自
 分を求める者に報いてくださる方であることを、信じていなければならない
 からです。」(11:6)}と言い、このことは最大限に愛徳に属する。それは『箴
 言』第8章「私は私を愛する人々を愛する」\footnote{「私を愛する人を私も
 愛し/私を探し求める人を私も見いだす。」(8:17)}において明らかである。
 また希望は、上述の如く、愛に導くものである。ゆえに愛徳は信仰と希望な
 しには所持されえない。


\\



{\scshape Respondeo dicendum} quod caritas non solum significat amorem
Dei, sed etiam amicitiam quandam ad ipsum; quae quidem super amorem
addit mutuam redamationem cum quadam mutua communicatione, ut dicitur
 in VIII {\itshape Ethic}.

&

 解答する。以下のように言われるべきである。愛徳は神への愛を意味するだ
 けでなく、神へのある種の友愛も意味する。これは、『ニコマコス倫理学』
 第8巻で言われるように、愛に、ある種の相互コミュニケーションを伴う相互
 の愛し合いを付加する。

\\

 Et quod hoc ad caritatem pertineat, patet per id quod dicitur I
Ioan.~{\scshape iv}, {\itshape qui manet in caritate, in Deo manet, et
Deus in eo}. Et I {\itshape ad Cor}.~{\scshape i} dicitur, {\itshape
fidelis Deus, per quem vocati estis in societatem filii eius}.

&

 そしてこれが愛徳に属することは、『ヨハネの手紙一』第4章「愛徳にとどま
 る人は、神にとどまり、神はその人の中にいる」\footnote{「私たちは、神
 が私たちに抱いておられる愛を知り、信じています。神は愛です。愛の内に
 とどまる人は、神の内にとどまり、神もその人の内にとどまってくださいま
 す。」(4:16)}によって明らかである。また『コリントの信徒への手紙一』で
 は「信頼の置ける神、その人によって、あなたたちは神の息子との交わりへ
 と呼ばれた」\footnote{「神は真実な方です。この方によって、あなたがた
 は神の子、私たちの主イエス・キリストとの交わりに招き入れられたのです。」
 (1:9)}と言われている。

\\


 Haec autem societas hominis ad Deum, quae est quaedam familiaris
conversatio cum ipso, inchoatur quidem hic in praesenti per gratiam,
perficietur autem in futuro per gloriam, quorum utrumque fide et spe
tenetur.


&

 ところで、この人間の神に対する交わり、これは神とのある種の親しい会話
 だが、は、ここ現生においては恩恵によって始まるが、将来は栄光によって
 完成される。そしてこのどちらも、信仰と希望によって保持される。

\\


 Unde sicut aliquis non posset cum aliquo amicitiam habere, si
discrederet vel desperaret se posse habere aliquam societatem vel
familiarem conversationem cum ipso; ita aliquis non potest habere
amicitiam ad Deum, quae est caritas, nisi fidem habeat, per quam
credat huiusmodi societatem et conversationem hominis cum Deo, et
speret se ad hanc societatem pertinere. Et sic caritas sine fide et
spe nullo modo esse potest.

&

 したがって、人がもし、ある人と交わりや親しい会話を持つことを信じず、
 その希望もないときに、その人と友愛を持つことはできないように、人は神
 に対して、そのような人が神との交わりや会話をもつことを信じる信仰や、
 自分がその交わりに参加する希望をもたなかったならば、友愛を持つことは
 できない。この意味で、愛徳は信仰や希望なしにはどんな仕方によってもあ
 りえない。

\\




{\scshape Ad primum ergo dicendum} quod caritas non est qualiscumque
amor Dei, sed amor Dei quo diligitur ut beatitudinis obiectum, ad quod
ordinamur per fidem et spem.

&

 第一異論に対しては、それゆえ、以下のように言われるべきである。愛徳は、
 神への愛であればどんなものであってもいいというわけではなく、それによっ
 て至福の対象として愛され、信仰と希望によってそれへと秩序付けられるよ
 うな神への愛が愛徳と言われる。

\\




{\scshape Ad secundum dicendum} quod caritas est radix fidei et spei,
inquantum dat eis perfectionem virtutis. Sed fides et spes, secundum
rationem propriam, praesupponuntur ad caritatem, ut supra dictum
est. Et sic caritas sine eis esse non potest.


&

 第二異論に対しては以下のように言われるべきである。愛徳が信仰と希望の
 根であるのは、それらに徳の完全性を与える限りにおいてである。しかるに
 信仰と希望は、その固有の性格に即するならば、上述の如く、愛徳の前提と
 される。この意味で愛徳はそれらなしにありえない。

\\

{\scshape Ad tertium dicendum} quod Christo defuit fides et spes,
propter id quod est imperfectionis in eis. Sed loco fidei, habuit
apertam visionem; et loco spei, plenam comprehensionem. Et sic fuit
perfecta caritas in eo.

&


 第三異論に対しては以下のように言われるべきである。キリストに信仰と希
 望がなかったのは、それらの中にある不完全性に属するもののためである。
 しかし信仰の代わりに彼は明らかな直視を、希望の代わりに十全な把握を持っ
 ていた。こうして彼の中には完全な愛徳があった。

\end{longtable}
\end{document}