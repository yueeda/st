\documentclass[10pt]{jsarticle} % use larger type; default would be 10pt
%\usepackage[utf8]{inputenc} % set input encoding (not needed with XeLaTeX)
%\usepackage[round,comma,authoryear]{natbib}
%\usepackage{nruby}
\usepackage{okumacro}
\usepackage{longtable}
%\usepqckage{tablefootnote}
\usepackage[polutonikogreek,english,japanese]{babel}
%\usepackage{amsmath}
\usepackage{latexsym}
\usepackage{color}

%----- header -------
\usepackage{fancyhdr}
\lhead{{\it Summa Theologiae} I, q.15}
%--------------------

\bibliographystyle{jplain}

\title{{\bf PRIMA PARS}\\{\HUGE Summae Theologiae}\\Sancti Thomae
Aquinatis\\{\sffamily QUEAESTIO DECIMAQUINTA}\\DE IDEIS}
\author{Japanese translation\\by Yoshinori {\sc Ueeda}}
\date{Last modified \today}


%%%% コピペ用
%\rhead{a}
%\begin{center}
% {\Large {\bf }}\\
% {\large }\\
% {\footnotesize }\\
% {\Large \\}
%\end{center}
%
%\begin{longtable}{p{21em}p{21em}}
%
%&
%
%
%\\
%\end{longtable}
%\newpage


\begin{document}
\maketitle
\pagestyle{fancy}

\begin{center}
{\Large 第十五問\\イデアについて}
\end{center}

\begin{longtable}{p{21em}p{21em}}
{\huge P}{\scshape ost} considerationem de scientia Dei, restat
 considerare de ideis. Et circa hoc quaeruntur tria. 

\begin{enumerate}
 \item an sint ideae.
 \item utrum sint plures, vel una tantum.
 \item utrum sint omnium quae cognoscuntur a Deo.
\end{enumerate}

&

神の知を考察した後、イデアについて考察することが残っている。
これについて三つのことが問われる。

\begin{enumerate}
 \item イデアは存在するか。
 \item イデアは複数か、あるいは、ただ一つか。
 \item 神によって認識されるすべてのものにイデアがあるか。
\end{enumerate}

\end{longtable}

\newpage

\rhead{a.1}
\begin{center}
 {\Large {\bf ARTICULUS PRIMUS}}\\
 {\large UTRUM IDEAE SINT}\\
 {\footnotesize Infra, q.44, a.3; I {\itshape Sent.}, d.36, q.2,
 a.1; {\itshape De Verit.}, q.3, a.1; I {\itshape Metaphys.}, lect.15.}\\
 {\Large 第一項\\イデアが存在するか}
\end{center}

\begin{longtable}{p{21em}p{21em}}


{\huge A}{\scshape d primum sic proceditur}. Videtur quod ideae non
sint. Dicit enim Dionysius, {\scshape vii} cap.{\itshape de Div.~Nom}.,
quod Deus non cognoscit res secundum ideam. Sed ideae non ponuntur ad
aliud, nisi ut per eas cognoscantur res. Ergo ideae non sunt.

&

 第一項の問題へ、議論は以下のように進められる。イデアは存在しないと思
 われる。理由は以下の通り。ディオニュシウスは『神名論』第7章で、神は事
 物をイデアに即して認識するのではない、と述べている。しかし、イデアが
 あるとされるのは、それによって事物が認識されるために他ならない。ゆえ
 に、イデアは存在しない。

\\


{\scshape 2 Praeterea}, Deus in seipso cognoscit omnia, ut supra
dictum est. Sed seipsum non cognoscit per ideam. Ergo nec alia.

&

さらに、前に述べられたとおり、神は、自己自身の中に万物を認識する。とこ
ろが、自己自身を認識するのはイデアによってではない。ゆえに、他のものも
また[イデアによって認識することはない]。

\\


{\scshape 3 Praeterea}, idea ponitur ut principium
cognoscendi et operandi. Sed essentia divina est sufficiens principium
cognoscendi et operandi omnia. Non ergo necesse est ponere ideas.

&

 さらに、イデアがあるとされるのは、認識と働きの根源としてである。とこ
 ろが、神の本質は、万物を認識し、万物に働きかけるのに十分な根源である。
 ゆえに、イデアを措定する必要はない。


\\


{\scshape Sed contra est} quod dicit Augustinus, in libro {\itshape
Octoginta trium quaest}., {\itshape tanta vis in ideis constituitur, ut,
nisi his intellectis, sapiens esse nemo possit}.

&

しかし反対に、アウグスティヌスは『八十三問題集』の中で「イデアの中には、
 もしこれらが知性認識されなければ、だれも知者でありえないほどの力があ
 る」と述べている。

\\


{\scshape Respondeo dicendum} quod necesse est ponere in mente divina
ideas. {\itshape Idea} enim Graece, Latine {\itshape forma} dicitur,
unde per ideas intelliguntur formae aliarum rerum, praeter ipsas res
existentes. Forma autem alicuius rei praeter ipsam existens, ad duo
esse potest, vel ut sit exemplar eius cuius dicitur forma; vel ut sit
principium cognitionis ipsius, secundum quod formae cognoscibilium
dicuntur esse in cognoscente. Et quantum ad utrumque est necesse
ponere ideas.


&

解答する。以下のように言われるべきである。神の精神の中にイデアを措定す
ることは必要である。理由は以下の通り。ギリシャ語で「イデア」[と言われ
るもの]は、ラテン語で「形相」と言われる。ここから、「イデア」によって、
「他の事物の、その事物を離れて存在する形相」が理解される。ところで、あ
る事物の、その事物を離れて存在する形相は、二つの点にかんしてありうる。
一つには、「Xの形相」と言われるそのXの範型としてであり、もう一つには、
そのXの認識の根源としてである。これは、認識されうるものの形相が、認識
者の中にあると言われることに即してである。そして、このどちらについても、
イデアを措定することが必要である。

\\


Quod sic patet. In omnibus enim quae non a casu generantur, necesse
est formam esse finem generationis cuiuscumque. Agens autem non ageret
propter formam, nisi inquantum similitudo formae est in ipso. Quod
quidem contingit dupliciter. In quibusdam enim agentibus praeexistit
forma rei fiendae secundum esse naturale, sicut in his quae agunt per
naturam; sicut homo generat hominem, et ignis ignem. In quibusdam vero
secundum esse intelligibile, ut in his quae agunt per intellectum;
sicut similitudo domus praeexistit in mente aedificatoris. Et haec
potest dici idea domus, quia artifex intendit domum assimilare formae
quam mente concepit.

&

このことは、以下のようにして明らかである。偶然に生まれるのでないすべて
のものにおいて、形相がそれぞれの生成の目的であることが必要である。とこ
ろが、作用者は、形相の類似が自分の中にないかぎり、その形相を目的として
働くことがない。こういうことは、二通りに生じる。ある作用者の中には、作
られるべき事物の形相が、自然本性的存在に即して先在する。それはちょうど、
人間が人間を生み、火が火を生む場合のように、本性によって働くものの中に
存在するようなかたちである。他方、ある作用者の中には、可知的な存在に即
して[先在する]。それはちょうど、家の類似が、建築家の精神の中に先在す
る場合のように、知性によって働くものの中にあるようなかたちでである。そ
して、この後者[の形相]が、家のイデアと言われうる。なぜなら、技術者は、
精神によって捉えた形相に、家を似せることを意図するからである。


\\


Quia igitur mundus non est casu factus, sed est factus a Deo per
intellectum agente, ut infra patebit, necesse est quod in mente divina
sit forma, ad similitudinem cuius mundus est factus. Et in hoc
consistit ratio ideae.

&

ゆえに、後に明らかになるように、世界は偶然に生じたのではなく、知性によっ
て働く神によって作られたのだから、神の精神の中には形相があり、その類似
に向けて世界が作られたことが必要である。そしてここに、イデアの性格が成
立する。

\\


{\scshape Ad primum ergo dicendum} quod Deus non intelligit res
secundum ideam extra se existentem. Et sic etiam Aristoteles improbat
opinionem Platonis de ideis, secundum quod ponebat eas per se
existentes, non in intellectu.

&

第一に対しては、それゆえ、次のように言われるべきである。神は、自己の外
に存在するイデアにしたがって事物を知性認識するのではない。この意味で、
アリストテレスも、知性の中ではなくそれ自体で存在するイデアを措定した点
で、イデアについてのプラトンの意見を批判している。

\\


{\scshape Ad secundum dicendum} quod, licet Deus per essentiam suam se
et alia cognoscat, tamen essentia sua est principium operativum
aliorum, non autem sui ipsius, et ideo habet rationem ideae secundum
quod ad alia comparatur, non autem secundum quod comparatur ad ipsum
Deum.

&

第二に対しては次のように言われるべきである。たしかに、神は、自分の本質
によって自己と他のものを認識する。そして神の本質は他のものに対して働き
かける根源だが、しかし、自分自身に働きかける根源ではない。ゆえに、神が
他のものへ関係する点で、イデアの性格をもつが、神自身へ関係する点で、イ
デアの性格をもつことはない。

\\


{\scshape Ad tertium dicendum} quod Deus secundum essentiam suam est
similitudo omnium rerum. Unde idea in Deo nihil est aliud quam Dei
essentia.

&

 第三に対しては次のように言われるべきである。神は、自己の本質に即して、
すべての事物の類似である。したがって、神の中のイデアは、神の本質以外の
ものではない。

\end{longtable}
\newpage


\rhead{a.2}
\begin{center}
 {\Large {\bf ARTICULUS SECUNDUS}}\\
 {\large UTRUM SINT PLURES IDEAE}\\
 {\footnotesize Infra, q.44, a.3; q.47, a.1, ad 2; I {\itshape
 Sent.}, d.36, q.2, a.2; III, d.45, a.2, qu$^a$ 2; I {\itshape
 SCG.}, c.54; \\{\itshape De Pot.}, q.3, a.16, ad 12, 13; {\itshape De
 Verit.}, q.3, a.2; {\itshape Quodl}.IV, q.1.}\\
 {\Large 第二項\\複数のイデアが存在するか}
\end{center}

\begin{longtable}{p{21em}p{21em}}


{\huge A}{\scshape d secundum sic proceditur}. Videtur quod
non sint plures ideae. Idea enim in Deo est eius essentia. Sed essentia
Dei est una tantum. Ergo et idea est una.

&

第二項の問題へ、議論は以下のように進められる。複数のイデアがあるのでは
ないと思われる。理由は以下の通り。神の中のイデアは神の本質である。とこ
ろが、神の本質は一つだけである。ゆえに、イデアは一つである。

\\


{\scshape 2 Praeterea}, sicut idea est principium cognoscendi et
operandi, ita ars et sapientia. Sed in Deo non sunt plures artes et
sapientiae. Ergo nec plures ideae.

&

さらに、イデアが認識と働きの根源であるように、技術と知恵もそう[=認識
と働きの根源]である。ところが、神の中に、複数の技術や知恵はない。ゆえ
に、複数のイデアもない。

\\


3. Si dicatur quod ideae multiplicantur secundum respectus ad diversas
creaturas, contra, pluralitas idearum est ab aeterno. Si ergo ideae
sunt plures, creaturae autem sunt temporales, ergo temporale erit
causa aeterni.

&

[この第二異論に対して]もし、イデアはさまざまな被造物にかんする限りで
 複数化されると言われるなら、それに反対して以下のように言う。[イデア
 が複数だとすると、その]イデアの複数性は、永遠からである。ゆえに、も
 しイデアが複数であるならば、被造物は時間的なものだから、時間的なもの
 が永遠的なものの原因であることになる。[しかしこれはありえない。ゆえ
 に、イデアは複数でない。]

\\


{\scshape 4 Praeterea}, respectus isti aut sunt secundum rem in
creaturis tantum, aut etiam in Deo. Si in creaturis tantum, cum
creaturae non sint ab aeterno, pluralitas idearum non erit ab aeterno,
si multiplicentur solum secundum huiusmodi respectus. Si autem
realiter sunt in Deo, sequitur quod alia pluralitas realis sit in Deo
quam pluralitas personarum, quod est contra Damascenum, dicentem quod
in divinis omnia unum sunt, praeter {\itshape ingenerationem,
generationem et processionem}. Sic igitur non sunt plures ideae.

 &

 さらに、[第三異論で言うような]この関係は、実在的には、被造物の中だ
けにあるか、あるいは、[被造物に加えて]神の中にもあるかのいずれかであ
る。もし、被造物の中だけにあるのなら、被造物は永遠から在るのではないか
ら、もし、このような関係だけに即して多数化されるのであれば、イデアの複
数性は永遠からでない。他方、もし、[このような関係が]神の中に実在的に
在るとすると、神の中に、ペルソナの複数性以外の実在的な複数性が在ること
になるが、これは、神において、「不生、出生、発出」以外にはすべてが一つ
である、と言うダマスケヌスに反する。このような理由で、複数のイデアは存
在しない。

\\

{\scshape Sed contra est} quod dicit Augustinus, in libro {\itshape
Octoginta trium quaest}., {\itshape ideae sunt principales quaedam
formae vel rationes rerum stabiles atque incommutabiles, quia ipsae
formatae non sunt, ac per hoc aeternae ac semper eodem modo se habentes,
quae divina intelligentia continentur. Sed cum ipsae neque oriantur
neque intereant, secundum eas tamen formari dicitur omne quod oriri et
interire potest, et omne quod oritur et interit}.

&

しかし反対に、アウグスティヌスは、『八十三問題集』で次のように述べてい
る。「イデアは、諸事物の、一種の主要な安定して変化しえない形相ないし性
格であり、というのも、それらは形成されたものでないからだが、そのために、
それら[イデア]は永遠で、常に同じあり方にあり、神の知性に含まれている。
しかし、それら[イデア]は、生じたり消えたりせず、むしろ、すべて生じた
り消えたりしうるもの、そして、すべて現に生じたり消えたりしているものは、
それらにしたがって、形成されると言われる」。


\\


{\scshape Respondeo dicendum} quod necesse est ponere plures ideas. Ad
cuius evidentiam, considerandum est quod in quolibet effectu illud
quod est ultimus finis, proprie est intentum a principali agente;
sicut ordo exercitus a duce. Illud autem quod est optimum in rebus
existens, est bonum ordinis universi, ut patet per philosophum in XII
{\itshape Metaphys}.

&

解答する。以下のように言われるべきである。複数のイデアを措定することが
必要である。これを明らかにするためには、以下のことが考察されるべきであ
る。どんな結果においても、最終目的が、主要な作用者によって固有に意図さ
れる。たとえば、軍隊の秩序が、指揮官によって[最終目的として固有に意図
されるように]。ところで、『形而上学』第12巻の哲学者によって明らかなと
おり、諸事物において存在する最善のものは、宇宙の秩序の善である。


\\

Ordo igitur universi est proprie a Deo intentus, et non per
accidens proveniens secundum successionem agentium, prout quidam
dixerunt quod Deus creavit primum creatum tantum, quod creatum creavit
secundum creatum, et sic inde quousque producta est tanta rerum
multitudo, secundum quam opinionem, Deus non haberet nisi ideam primi
creati. 


&

ゆえに、宇宙の秩序は、神によって固有に意図されているのであって、ある人々
が言ったように、作用者の継起によって偶然に生じたのではない。つまりその
人々は、神は第一の被造物だけを創造し、この被造物が第二の被造物を創造し、
というようにして、これほど多い諸事物が生み出されたと言ったのだが、この
意見に従うならば、神は第一の被造物のイデアしか持たなかったであろう。

\\

Sed si ipse ordo universi est per se creatus ab eo, et intentus ab
ipso, necesse est quod habeat ideam ordinis universi. Ratio autem
alicuius totius haberi non potest, nisi habeantur propriae rationes
eorum ex quibus totum constituitur, sicut aedificator speciem domus
concipere non posset, nisi apud ipsum esset propria ratio cuiuslibet
partium eius.

&

しかし、もし宇宙の秩序が、神によって自体的に創造され意図されたのであれ
 ば、神が宇宙の秩序のイデアを持つことが必要である。ところで、どんな全
 体の性格も、全体がそれから構成される個々のものの固有の性格が所有され
 ないかぎり、所有されえない。たとえば、もし、建築家が、家のあらゆる部
 分について、それの固有の性格を知っていないならば、彼は家の形象を捉え
 ることができなかったであろうように。

\\

Sic igitur oportet quod in mente divina sint propriae rationes omnium
rerum. Unde dicit Augustinus, in libro {\itshape Octoginta trium
quaest}., quod {\itshape singula propriis rationibus a Deo creata
sunt}. Unde sequitur quod in mente divina sint plures ideae.

&

それゆえ、神の精神の中には、万物の固有の性格(理念)がある。このことから、アウ
グスティヌスは『八十三問題集』で「個々のものが、固有の性格(理念)によって、神
に創造された」と述べる。したがって、神の精神の中に複数のイデアがあるこ
とが帰結する。

\\

Hoc autem quomodo divinae simplicitati non repugnet, facile est
videre, si quis consideret ideam operati esse in mente operantis sicut
quod intelligitur; non autem sicut species qua intelligitur, quae est
forma faciens intellectum in actu.

&

しかしこのこと[=神の精神の中に複数のイデアがあること]が、どうして神
の単純性に反しないかということは、次のことを考えれば容易に分かる。つま
り、働きの結果生じるもののイデアは、働きを行うものの精神の中に、知性認
識されるものとして存在するのであり、知性を現実態にする形相である[可知
的]形象---これによって何かが知性認識されるのだが---として存在するので
はない。


\\


Forma enim domus in mente aedificatoris est aliquid ab eo intellectum,
ad cuius similitudinem domum in materia format. Non est autem contra
simplicitatem divini intellectus, quod multa intelligat, sed contra
simplicitatem eius esset, si per plures species eius intellectus
formaretur. Unde plures ideae sunt in mente divina ut intellectae ab
ipso.


&

たとえば、建築家の精神の中にある家の形相は、彼によって知性認識された何
かであり、彼はそれの類似へ向けて、家を質料の中に形作る。多くのものを知
性認識することは、神の知性の単純性に反しないが、しかしもし、複数の形象
によって、神の知性が形成されるということになると、それは神の単純性に反
することになっただろう。したがって、複数のイデアが、神の精神の中に、神
によって知性認識されたものとして存在する。


\\


Quod hoc modo potest videri. Ipse enim essentiam suam perfecte
cognoscit, unde cognoscit eam secundum omnem modum quo cognoscibilis
est. Potest autem cognosci non solum secundum quod in se est, sed
secundum quod est participabilis secundum aliquem modum similitudinis
a creaturis.

&

このことは、次のようにして理解される。神は、自分の本質を完全に認識する。
したがって、神は自分の本質を、それが認識可能であるあらゆるあり方で認識
する。さらに、[神の本質は、神によって]それ自体である[=神の本質であ
る]という観点からだけでなく、それが被造物によって、何らかの類似のあり
方に即して分有されうるものであるという観点からもまた、認識されうる。

\\

Unaquaeque autem creatura habet propriam speciem, secundum quod aliquo
modo participat divinae essentiae similitudinem. Sic igitur inquantum
Deus cognoscit suam essentiam ut sic imitabilem a tali creatura,
cognoscit eam ut propriam rationem et ideam huius creaturae. Et
similiter de aliis. Et sic patet quod Deus intelligit plures rationes
proprias plurium rerum; quae sunt plures ideae.

&

ところで、各々の被造物は、何らかのかたちで神の本質の類似を分有している
という点で、固有の形象をもつ。ゆえに、神が、そのような被造物によってそ
のように模倣されうるものとして自分の本質を認識する点で、神は自分の本質
を、その被造物の固有の性格つまりイデアとして認識する。他のものについて
も同様である。このようにして、神が複数の事物の複数の固有の性格を、つま
り複数のイデアを知性認識することが明らかである。

\\

{\scshape Ad primum ergo dicendum} quod idea non nominat divinam
essentiam inquantum est essentia, sed inquantum est similitudo vel
ratio huius vel illius rei. Unde secundum quod sunt plures rationes
intellectae ex una essentia, secundum hoc dicuntur plures ideae.

&

第一異論に対しては、それゆえ、次のように言われるべきである。「イデア」
は、それが本質であるという観点から神の本質を名付ける名ではなく、[神の
本質が]これやあの事物の類似ないし性格(ratio)であるという観点から神の
本質を指す名である。したがって、一つの本質に基づいて複数の性格が理解さ
れるという点で、複数のイデアが語られる。

\\


{\scshape Ad secundum dicendum} quod sapientia et ars significantur ut
quo Deus intelligit, sed idea ut quod Deus intelligit. Deus autem uno
intelligit multa; et non solum secundum quod in seipsis sunt, sed
etiam secundum quod intellecta sunt; quod est intelligere plures
rationes rerum. Sicut artifex, dum intelligit formam domus in materia,
dicitur intelligere domum, dum autem intelligit formam domus ut a se
speculatam, ex eo quod intelligit se intelligere eam, intelligit ideam
vel rationem domus. Deus autem non solum intelligit multas res per
essentiam suam, sed etiam intelligit se intelligere multa per
essentiam suam. Sed hoc est intelligere plures rationes rerum; vel,
plures ideas esse in intellectu eius ut intellectas.

&

第二異論に対しては次のように言われるべきである。「知恵」や「技術」は、
神がそれによって知性認識するところのもの[=能力]として表示されるが、
「イデア」は、神が知性認識するもの[=対象]として表示される。また、神
は、一によって多を知性認識するが、[その多が]それぞれ自身であるという
点でだけでなく、それらが知性認識されたものであるという点でも、認識する。
これが、諸事物の複数のイデアを知性認識することである。たとえば、職人が、
質料の中に家の形相を知性認識するときには、「家を知性認識する」と言われ
るが、自分によって観照されたものとして家の形相を知性認識するときは、自
分がそれを知性認識することを知性認識することから、家のイデア、あるいは
家の性格を知性認識する。ところで神は、自分の本質によって多くの事物を知
性認識するだけでなく、自分が自分の本質によって多くのものを知性認識して
いることをも知性認識する。ところがこれは、諸事物の複数の性格、あるいは、
複数のイデアが、彼の知性の中に、知性認識されたものとして存在することを
知性認識していることである。

\\


{\scshape Ad tertium dicendum} quod huiusmodi respectus, quibus
multiplicantur ideae, non causantur a rebus, sed ab intellectu divino,
comparante essentiam suam ad res.

&

第三異論に対しては、次のように言われるべきである。イデアが多数化される
このような関係は、諸事物が原因となってではなく、自分の本質を事物へ関係
づける神の知性が原因となって生じる。

\\

{\scshape Ad quartum dicendum} quod respectus multiplicantes ideas,
non sunt in rebus creatis, sed in Deo. Non tamen sunt reales
respectus, sicut illi quibus distinguuntur personae, sed respectus
intellecti a Deo.

&

第四異論に対しては、次のように言われるべきである。イデアを多数化させる
関係は、創造された事物の中にではなく神の中にある。しかし、それはペルソ
ナが区別されるような実在的な関係ではなく、神に知性認識された関係である。


\end{longtable}
\newpage


\rhead{a.3}
\begin{center}
 {\Large {\bf ARTICULUS TERTIUS}}\\
 {\large UTRUM OMNIUM QUAE COGNOSCIT DEUS, SINT IDEAE}\\
 {\footnotesize I {\itshape Sent.}, d.36, q.2, a.3; {\itshape De
 Pot.}, q.1, a.5, ad 10, 11; q.3, a.1, ad 13; {\itshape De Verit.},
 q.3, a.3 sqq.; {\itshape De Div.Nom.}, cap.5, lect.3.}\\
 {\Large 第三項\\神が認識するすべてのものにイデアがあるか}
\end{center}

\begin{longtable}{p{21em}p{21em}}

{\huge A}{\scshape d tertium sic proceditur}. Videtur quod non
omnium quae cognoscit Deus, sint ideae in ipso. Mali enim idea non est
in Deo, quia sequeretur malum esse in Deo. Sed mala cognoscuntur a
Deo. Ergo non omnium quae cognoscuntur a Deo, sunt ideae.

&

第三項の問題へ、議論は以下のように進められる。神が認識するすべてのもの
のイデアが、神のうちにあるわけではないと思われる。理由は以下の通り。悪
のイデアは神のうちにない。[もし在ったら]神のうちに悪があることになっ
ただろうから。ところが、悪は神によって認識される。ゆえに、神に認識され
るすべてのものにイデアがあるわけではない。

\\


{\scshape 2 Praeterea}, Deus cognoscit ea quae nec sunt nec erunt nec
fuerunt, ut supra dictum est. Sed horum non sunt ideae, quia dicit
Dionysius, {\scshape v} cap.{\itshape de Div.~Nom}., quod {\itshape
exemplaria sunt divinae voluntates, determinativae et effectivae
rerum}. Ergo non omnium quae a Deo cognoscuntur, sunt ideae in ipso.


&

さらに、上で述べられたとおり、神は、現在、過去、未来に存在しないものを
認識する。ところが、これらにはイデアがない。なぜなら、ディオニュシウス
が『神名論』5章で「範型は、事物を限定し作り出しうる神の意志である」と
述べているから。ゆえに、神によって認識されるすべてのものにイデアがある
わけではない。

\\

{\scshape 3 Praeterea}, Deus cognoscit materiam primam, quae non
potest habere ideam, cum nullam habeat formam. Ergo idem quod prius.

&

さらに、神は第一質料を認識するが、これはイデアをもつことができない。な
ぜなら、どんな形相ももたないから。ゆえに、前と同じ。

\\


{\scshape 4 Praeterea}, constat quod Deus scit non solum species, sed
etiam genera et singularia et accidentia. Sed horum non sunt ideae,
secundum positionem Platonis, qui primus ideas introduxit, ut dicit
Augustinus. Non ergo omnium cognitorum a Deo sunt ideae in ipso.

&

さらに、神は種だけでなく類や個や附帯性も知ることが明らかである。ところ
が、これら[=類、個、附帯性]にはイデアがない。これは、アウグスティヌ
スが言うように、最初にイデアを導入したプラトンの意見による。ゆえに、神
によって認識されたすべてのもののイデアが、神の中にあるわけではない。

\\


{\scshape Sed contra}, ideae sunt rationes in mente divina existentes,
ut per Augustinum patet. Sed omnium quae cognoscit, Deus habet
proprias rationes. Ergo omnium quae cognoscit, habet ideam.


&

 しかし反対に、アウグスティヌスによって明らかなとおり、イデアとは、神
 の精神の中に存在する諸々の理念(rationes)である。ところが、神が認識す
 るすべてのものについて、神は固有の理念をもつ。ゆえに、神が認識するす
 べてのものについて、神はイデアをもつ。

\\

{\scshape Respondeo dicendum} quod, cum ideae a Platone ponerentur
principia cognitionis rerum et generationis ipsarum, ad utrumque se
habet idea, prout in mente divina ponitur. Et secundum quod est
principium factionis rerum, {\itshape exemplar} dici potest, et ad
practicam cognitionem pertinet. Secundum autem quod principium
cognoscitivum est, proprie dicitur {\itshape ratio}; et potest etiam
ad scientiam speculativam pertinere.


&

解答する。以下のように言われるべきである。イデアは、プラトンによって、
諸事物の認識と生成の根源として措定されたので、イデアは、神の精神の中に
あるものとして、そのどちらにも関係する。諸事物の作成の根源としては、
「範型」と言われることができ、実践的な認識に属する。他方、認識の根源で
あるという点では、固有に「理念」(ratio)と言われる。また、それは観照的
な知にも属しうる。

\\

Secundum ergo quod exemplar est, secundum hoc se habet ad omnia quae a
Deo fiunt secundum aliquod tempus. Secundum vero quod principium
cognoscitivum est, se habet ad omnia quae cognoscuntur a Deo, etiam si
nullo tempore fiant; et ad omnia quae a Deo cognoscuntur secundum
propriam rationem, et secundum quod cognoscuntur ab ipso per modum
speculationis.

&

ゆえに、[イデアは]何らかの時間に神によって作られるすべてのものに関係
する点で、範型である。他方、[イデアは]認識の根源である点で、神によっ
て認識されるすべてのものに、たとえそれがどんな時間にも作られないとして
も、関係する。また、固有の理念という観点から、神によって観照的なしかた
で認識されるという点で、神に認識されるすべてのものに関係する。

\\


{\scshape Ad primum ergo dicendum} quod malum cognoscitur a Deo non
per propriam rationem, sed per rationem boni. Et ideo malum non habet
in Deo ideam, neque secundum quod idea est exemplar, neque secundum
quod est ratio.


&

第一異論に対しては、それゆえ、次のように言われるべきである。悪は、神に
よって、固有の理念によってではなく、善の理念を通して認識される。ゆえに、
悪は、イデアが範型であるという点でも、また理念であるという点でも、神の
中にイデアをもたない。

\\


{\scshape Ad secundum dicendum} quod eorum quae neque sunt neque erunt
neque fuerunt, Deus non habet practicam cognitionem, nisi virtute
tantum. Unde respectu eorum non est idea in Deo, secundum quod idea
significat exemplar, sed solum secundum quod significat rationem.

&

第二異論に対しては次のように言われるべきである。現在、過去、未来に存在
しないものどもについて、神は、たんに能力的にという意味以外には、実践的
な認識を持たない。したがって、イデアが範型を意味する場合には、それらに
ついて、神の中にイデアはない。しかし、理念(ratio)を表示する場合にかぎ
り、それらについてのイデアが神の中にある。

\\


{\scshape Ad tertium dicendum} quod Plato, secundum quosdam, posuit
materiam non creatam, et ideo non posuit ideam esse materiae, sed
materiae concausam. Sed quia nos ponimus materiam creatam a Deo, non
tamen sine forma, habet quidem materia ideam in Deo, non tamen aliam
ab idea compositi. Nam materia secundum se neque esse habet, neque
cognoscibilis est.

&

第三異論に対しては、次のように言われるべきである。ある人々によれば、プ
ラトンは、質料が創造されたものと考えなかったので、質料にはイデアがなく、
[イデアは]質料の共同原因であるとした。しかし、私たちは、質料が神によっ
て創造されたと考えるので、質料も神の中にイデアをもつ。ただし、質料は形
相なしに創造されたのではないから、[質料のイデアは]複合体のイデアと別
のイデアではない。質料は、それ自体としては、存在をもたず、認識されうる
ものでもないからである。

\\


{\scshape Ad quartum dicendum} quod genera non possunt habere ideam
aliam ab idea speciei, secundum quod idea significat exemplar, quia
nunquam genus fit nisi in aliqua specie. Similiter etiam est de
accidentibus quae inseparabiliter concomitantur subiectum, quia haec
simul fiunt cum subiecto.

&

第四異論に対しては次のように言われるべきである。類は、何らかの種の中に
でなければけっして生じないので、イデアが範型を意味する場合には、類は種
のイデアと異なるイデアをもつことができない。同様に、基体と分離不可能な
かたちで同伴する附帯性については、それは基体と同時に生じるのだから、
[類の場合と]同様である。

\\

Accidentia autem quae superveniunt subiecto, specialem ideam
habent. Artifex enim per formam domus facit omnia accidentia quae a
principio concomitantur domum, sed ea quae superveniunt domui iam
factae, ut picturae vel aliquid aliud, facit per aliquam aliam formam.

&

他方、[基体から分離可能で]基体に到来する附帯性は、種的なイデアをもつ。
たとえば、技術者は、家の形相によって、初めから家に伴うすべての附帯性を
作るが、すでに作られた家に到来するような附帯性、たとえば、絵とかそうい
うものは、何か別の形相によってそれを作る。


\\

Individua vero, secundum Platonem, non habebant aliam ideam quam ideam
speciei, tum quia singularia individuantur per materiam, quam ponebat
esse increatam, ut quidam dicunt, et concausam ideae; tum quia
intentio naturae consistit in speciebus, nec particularia producit,
nisi ut in eis species salventur. Sed providentia divina non solum se
extendit ad species, sed ad singularia, ut infra dicetur.

&

また、プラトンによれば、個体は種のイデア以外のイデアをもたないが、それ
は、個物は質料によって個体化されるが、誰かが言うように、プラトンは質料
が創造されず、イデアの共同原因だと考えたからであり、また、自然の意図は
種にあり、個別的なものの中に種が保存されるためにでなければ、個別的なも
のを生み出さないからである。しかし、後に述べられるとおり、神の摂理は種
だけでなく個物にも及ぶ。[ゆえに、個物のイデアが存在する。]
\end{longtable}
\end{document}