\documentclass[10pt]{jsarticle} % use larger type; default would be 10pt
%\usepackage[utf8]{inputenc} % set input encoding (not needed with XeLaTeX)
%\usepackage[round,comma,authoryear]{natbib}
%\usepackage{nruby}
\usepackage{okumacro}
\usepackage{longtable}
%\usepqckage{tablefootnote}
\usepackage[polutonikogreek,english,japanese]{babel}
%\usepackage{amsmath}
\usepackage{latexsym}
\usepackage{color}
%\usepackage{tikz}

%----- header -------
\usepackage{fancyhdr}
\pagestyle{fancy}
\lhead{{\it Summa Theologiae} I, q.~33}
%--------------------


\title{{\bf PRIMA PARS}\\{\HUGE Summae Theologiae}\\Sancti Thomae
Aquinatis\\{\sffamily QUEAESTIO TRIGESTIMATERTIA}\\DE PERSONA PATRIS}
\author{Japanese translation\\by Yoshinori {\sc Ueeda}}
\date{Last modified \today}

%%%% コピペ用
%\rhead{a.~}
%\begin{center}
% {\Large {\bf }}\\
% {\large }\\
% {\footnotesize }\\
% {\Large \\}
%\end{center}
%
%\begin{longtable}{p{21em}p{21em}}
%
%&
%
%\\
%\end{longtable}
%\newpage



\begin{document}

\maketitle
\thispagestyle{empty}
\begin{center}
{\Large 第三十三問\\父のペルソナについて}
\end{center}


\begin{longtable}{p{21em}p{21em}}

{\Huge C}{\scshape onsequenter} considerandum est de
 personis in speciali. Et primo de persona patris. Circa quam
 quaeruntur quatuor. 

\begin{enumerate}
 \item utrum patri competat esse principium.
 \item utrum persona patris proprie significetur hoc nomine pater.
 \item utrum per prius dicatur in divinis pater secundum quod sumitur personaliter, quam secundum quod sumitur essentialiter.
 \item utrum sit proprium patri esse ingenitum.
\end{enumerate}


&

続いてペルソナについて特殊的に考察されるべきである。第一に父のペルソナ
 について。これをめぐって四つのことが問われる。


\begin{enumerate}
 \item 根源であることは父に適合するか。
 \item 父のペルソナは「父」というこの名称によって固有に表示されるか。
 \item 本質的によりもむしろペルソナ的に取られる限りにおける父が、神に
       おいてより先に語られるか。
 \item 不生であることは父に固有か。
\end{enumerate}

\end{longtable}



\newpage



\rhead{a.~1}
\begin{center}
{\Large {\bf ARTICULUS PRIMUS}}\\
{\large UTRUM COMPETAT PATRI ESSE PRINCIPIUM}\\
{\footnotesize I {\itshape Sent.}, d.12, a.2, ad 1; d.29, a.1; III,
 d.9, a.1, ad 5; {\itshape De Pot.}, q.10, a.1, ad 8 sqq.; {\itshape
 Contra errores Graec.}, cap.1.}\\
{\Large 第一項\\根源であることは父に適合するか}
\end{center}

\begin{longtable}{p{21em}p{21em}}

{\scshape Ad primum sic proceditur}. Videtur quod pater non possit dici
principium filii vel spiritus sancti. Principium enim et causa idem
sunt, secundum philosophum. Sed non dicimus patrem esse causam
filii. Ergo non debet dici quod sit eius principium.


&

第一項の問題へ議論は以下のように進められる。
父は息子や聖霊の根源と言われえないと思われる。理由は以下の通り。
哲学者によれば根源と原因は同じである。しかるに私たちは父が息子の原因で
 あるとは言わない。ゆえにその根源であるとも言われるべきでない。

\\



2. {\scshape Praeterea}, principium dicitur respectu principiati. Si igitur pater
est principium filii, sequitur filium esse principiatum, et per
consequens esse creatum. Quod videtur esse erroneum.


&

さらに、根源は根源から生まれたものにかんして言われる。ゆえにもし父が息
 子の根源であるならば、息子は根源から生まれたものであることが帰結し、
 結果的に、創造されたものであることになる。これは誤りであると思われる。


\\



3. {\scshape Praeterea}, nomen principii a prioritate sumitur. Sed in divinis non
est prius et posterius, ut Athanasius dicit. Ergo in divinis non
debemus uti nomine principii.


&

さらに、根源という名称は先行性からとられている。しかるにアタナシウスが
 言うように、神においてより先とより後はない。ゆえに私たちは神において根源とい
 う名称を用いるべきではない。

\\



{\scshape Sed contra est} quod dicit Augustinus, in IV {\itshape de Trin}., {\itshape pater est
principium totius deitatis}.


&

しかし反対に、アウグスティヌスは『三位一体論』第4巻で「父は全神性の根
 源である」と述べている。

\\



{\scshape Respondeo dicendum} quod hoc nomen principium nihil aliud significat
quam id a quo aliquid procedit, omne enim a quo aliquid procedit
quocumque modo, dicimus esse principium; et e converso. Cum ergo pater
sit a quo procedit alius, sequitur quod pater est principium.


&

解答する。以下のように言われるべきである。
根源というこの名称は、何かがそこから発出するところのものをまさに表示す
 る。なぜなら、どのようなしかたであれそこから何かが発出するところのも
 のはすべて、私たちはそれを根源と言い、逆もまたそうだからである。ゆえ
 に、父はそこから他のものが発出するところのものなので、父が根源である
 ことが帰結する。

\\



{\scshape Ad primum ergo dicendum} quod Graeci utuntur in divinis indifferenter
nomine {\itshape causae}, sicut et nomine principii, sed Latini doctores non
utuntur nomine causae, sed solum nomine principii. Cuius ratio est,
quia {\itshape principium} communius est quam {\itshape causa}, sicut {\itshape causa} communius quam
{\itshape elementum}, primus enim terminus, vel etiam prima pars rei dicitur
principium, sed non causa. 

&

第一異論に対してはそれゆえ以下のように言われるべきである。
ギリシア人たちは神において、根源という名称と同様に「原因」という名称を無差別
 に用いている。しかしラテンの教師たちは原因という名称を用いずただ根
 源という名称だけを用いている。その理由は、ちょうど「原因」が「元素」
 よりも共通であるように、「根源」が「原因」よりも共通だから
 である。たとえば事物の第一の端が、あるいは事物の第一の部分
 \footnote{少し後(2パラグラフ後)に出てくる点や線のことを指していると思われる。直前の元素のことを指す
 と解釈すると包含関係がうまくいかない。}が根源(原
 理)と言われるが原因とは言われないように。

\\


Quanto autem aliquod nomen est communius,
tanto convenientius assumitur in divinis, ut supra dictum est, quia
nomina, quanto magis specialia sunt, tanto magis determinant modum
convenientem creaturae. Unde hoc nomen {\itshape causa} videtur importare
diversitatem substantiae, et dependentiam alicuius ab altero; quam non
importat nomen principii. 



&

しかるに、前に述べられたとおり、ある名称が共通であるほど神においてはよ
 り適合的に採用される。なぜなら名称は、より特殊的であるほど、被造物に
 適したあり方を限定するからである。したがって原因というこの名称は実体
 の異なりやある者の他のものへの依存を含意するように見えるが、これは根
 源という名称が含意しない。



\\


In omnibus enim causae generibus, semper
invenitur distantia inter causam et id cuius est causa, secundum
aliquam perfectionem aut virtutem. Sed nomine principii utimur etiam
in his quae nullam huiusmodi differentiam habent, sed solum secundum
quendam ordinem, sicut cum dicimus punctum esse principium lineae, vel
etiam cum dicimus primam partem lineae esse principium lineae.


&

じっさい、すべての原因の類において、常に原因とそれの原因であるところの
 もの(結果)とのあいだに、何らかの完全性や力における距離が見出される。
 しかし根源という名称を私たちはなんらそのような違いを持たず、たんに何
 らかの秩序においてあるものにおいても用いる。たとえば私たちが点は線の
 根源であると言う場合や、さらに、線の第一の部分は線の根源であると言う
 場合のように。


\\



{\scshape Ad secundum dicendum} quod apud Graecos invenitur de filio vel spiritu
sancto dici quod principientur. Sed hoc non est in usu doctorum
nostrorum. Quia licet attribuamus patri aliquid auctoritatis ratione
principii, nihil tamen ad subiectionem vel minorationem quocumque modo
pertinens, attribuimus filio vel spiritui sancto, ut vitetur omnis
erroris occasio. Secundum quem modum Hilarius dicit, IX {\itshape de Trin}.,
{\itshape donantis auctoritate pater maior est; sed minor non est filius, cui
unum esse donatur}.

&

第二異論に対しては以下のように言われるべきである。
ギリシア人たちのもとでは、息子や聖霊が「根源される」と言われているのが
 見出されるが、これは私たちの教師たちの用例の中にはない。なぜなら、私
 たちは父にその根源という性格に基づいて何らかの権威を帰するが、しかし
どんなかたちでも息子や聖霊に従属や卑小化を帰属させることはない。このよ
 うな帰属はあらゆる誤謬の機会となるように思われる。この意味でヒラリウ
 スは『三位一体論』第9巻で「与えるものとしての権威によって父はより大き
 いが、息子には同じ存在が与えられているのだから息子がより小さいわけで
 はない」。

\\



{\scshape Ad tertium dicendum} quod, licet hoc nomen principium, quantum ad id a
quo imponitur ad significandum, videatur a prioritate sumptum; non
tamen significat prioritatem, sed originem. Non enim idem est quod
significat nomen, et a quo nomen imponitur, ut supra dictum est.


&

第三異論に対しては以下のように言われるべきである。
この「根源」という名称は、表示するためにそこから付けられたところにかん
 する限りでは先行性から取られているが、表示するのは先行性ではなく起源
 である。じっさい、名称が表示するものと、そこからその名称が取られたと
 ころとは異なる。これは前に述べられた。\footnote{Q.13, a.2, ad 2; a.8.}


\\


\end{longtable}
\newpage




\rhead{a.~2}
\begin{center}
{\Large {\bf ARTICULUS SECUNDUS}}\\
{\large UTRUM HOC NOMEN {\itshape PATER} SIT NOMEN PROPRIE DIVINAE PERSONAE}\\
{\footnotesize Infra qu.11, a.2.}\\
{\Large 第二項\\「父」というこの名称は固有に神のペルソナの名称か}
\end{center}

\begin{longtable}{p{21em}p{21em}}

{\scshape Ad secundum sic proceditur}. Videtur quod hoc nomen pater non sit
proprie nomen divinae personae. Hoc enim nomen pater significat
relationem. Persona autem est substantia individua. Non ergo hoc nomen
pater est proprie nomen significativum personae.

&

第二項の問題へ議論は以下のように進められる。
「父」というこの名称は固有に神のペルソナの名称ではないと思われる。
理由は以下の通り。
この「父」という名称は関係を表示する。しかるにペルソナは個的実体である。
 ゆえにこの「父」という名称は固有にペルソナを表示しうるものではない。

\\



2. {\scshape Praeterea}, generans communius est quam pater, nam omnis pater est
generans, sed non e converso. Sed nomen communius magis proprie
dicitur in divinis, ut dictum est. Ergo magis proprium nomen est
personae divinae generans et genitor, quam pater.

&

さらに、「生むもの」は「父」より共通的である。なぜならすべての父は生む
 ものだが逆はそうでないからである。しかるにすでに述べられたとおり、より共通的な名称はより固有
 に神において語られる。ゆえに「父」よりも「生むもの」「産者」という名
 称の方が神のペルソナにとって固有である。

\\



3. {\scshape Praeterea}, nihil quod secundum metaphoram dicitur, potest esse nomen
proprium alicuius. Sed verbum metaphorice apud nos dicitur genitum vel
proles, et per consequens ille cuius est verbum, metaphorice dicitur
pater. Non ergo principium verbi in divinis potest proprie dici pater.

&

さらに、比喩において語られるものはどんなものも、あるものの固有の名称で
 はありえない。しかるに私たちのもとで、言葉は比喩的に「生まれたもの」や「子孫」と言わ
 れ、その結果、それの言葉であるところのそれが比喩的に「父」と言われる。
 ゆえに神における言葉の根源が固有に「父」と言われることはありえない。

\\



4. {\scshape Praeterea}, omne quod proprie dicitur in divinis, per prius dicitur de
Deo quam de creaturis. Sed generatio per prius videtur dici de
creaturis quam de Deo, verior enim ibi videtur esse generatio, ubi
aliquid procedit ab alio distinctum non secundum relationem tantum,
sed etiam secundum essentiam. Ergo nomen patris, quod a generatione
sumitur, non videtur esse proprium alicuius divinae personae.

&

さらに、神において固有に語られるものはすべて被造物よりも神につ
 いてより先に語られる。しかし生成は神よりも被造物についてより先に語ら
 れると思われる。なぜなら、たんに関係においてだけでなく本質においても
 区別された何かが他から発出するところに、より真に生成があると思われる
 からである。ゆえに「父」という名称は生成から取られるので、何らかの神
 のペルソナに固有のものであるとは思われない。

\\



{\scshape Sed contra est} quod dicitur in Psalmo, {\itshape ipse invocabit me, pater meus
es tu}.

&

しかし反対に、『詩編』で「彼が私に呼びかけるであろう。あなたは私の父で
 ある」\footnote{「彼は私に『あなたはわが父/わが神、わが救いの岩』と
 呼びかけるだろう。」(89:27)}と言われている。

\\




{\scshape Respondeo dicendum} quod nomen proprium cuiuslibet personae significat
id per quod illa persona distinguitur ab omnibus aliis. Sicut enim de
ratione hominis est anima et corpus, ita de intellectu huius hominis
est haec anima et hoc corpus, ut dicitur in VII {\itshape Metaphys}.; his autem
hic homo ab omnibus aliis distinguitur. Id autem per quod distinguitur
persona patris ab omnibus aliis, est paternitas. Unde proprium nomen
personae patris est hoc nomen pater, quod significat paternitatem.

&

解答する。以下のように言われるべきである。
どのペルソナがもつ固有の名称もそれによってそのペルソナ型のすべてから区
 別されるものを表示する。理由は以下の通り。
人間の性格に魂と身体が属するように「この人間」の理解内容には「この魂」
 と「この身体」が属する。これは『形而上学』第7巻で言われているとおりで
 ある。そしてこれらによって人間は他のすべてから区別される。ところで、
 父のペルソナが他のすべてのものから区別されるのは父性によってである。
従って、父のペルソナがもつ固有の名称は「父」というこの名称であり、それ
 は父性を表示する。

\\



{\scshape Ad primum ergo dicendum} quod apud nos relatio non est subsistens
persona, et ideo hoc nomen pater, apud nos, non significat personam,
sed relationem personae. Non autem est ita in divinis, ut quidam falso
opinati sunt, nam relatio quam significat hoc nomen pater, est
subsistens persona. Unde supra dictum est quod hoc nomen persona in
divinis significat relationem ut subsistentem in divina natura.

&

第一異論に対してはそれゆえ以下のように言われるべきである。
私たちのもとでは関係は自存するペルソナでないので「父」というこの名称は、
 私たちのもとでは、ペルソナを表示せずペルソナの関係を表示する。しかし
 神においてはそうではない。ある人々は誤ってそうだと考えたのではあるが。
 なぜなら、「父」というこの名称が表示する関係は自存するペルソナだから
 である。したがってこのペルソナという名称は神において神の本性において
 自存する関係を表示するということが、前に\footnote{Q.29, a.4.}語られた。


\\



{\scshape Ad secundum dicendum} quod, secundum philosophum, in II {\itshape de Anima},
denominatio rei maxime debet fieri a perfectione et fine. Generatio
autem significat ut in fieri, sed paternitas significat complementum
generationis. Et ideo potius est nomen divinae personae pater, quam
generans vel genitor.

&

第二異論に対しては以下のように言われるべきである。
『デアニマ』第2巻における哲学者によれば、事物の論証は、最大限に、完全性と目的から
 生じるべきである。しかし生成は、生成過程にあるものとして表示するが、
 父性は生成の完成を表示する。ゆえに神のペルソナの名称は「生むもの」や
 「産者」ではなく「父」である。


\\



{\scshape Ad tertium dicendum} quod verbum non est aliquid subsistens in natura
humana, unde non proprie potest dici genitum vel filius. Sed verbum
divinum est aliquid subsistens in natura divina, unde proprie, et non
metaphorice, dicitur filius, et eius principium, pater.

&

第三異論に対しては以下のように言われるべきである。人間の本性において言
 葉は自存するものではない。したがって固有に「生まれたもの」や「息子」
 とは言われえない。しかし神の言葉は神の本性において自存する何かである。
 したがって固有に、比喩的にではなく、「息子」と言われ、その根源は「父」
 と言われる。

\\



{\scshape Ad quartum dicendum} quod nomen generationis et paternitatis, sicut et
alia nomina quae proprie dicuntur in divinis, per prius dicuntur de
Deo quam de creaturis, quantum ad rem significatam, licet non quantum
ad modum significandi. Unde et apostolus dicit, {\itshape ad
 Ephes}.~{\scshape iii} : {\itshape flecto
genua mea ad patrem domini nostri Iesu Christi, ex quo omnis
paternitas in caelo et in terra nominatur}. 

&

第四異論に対しては以下のように言われるべきである。神において固有に語ら
 れる他の名称と同様に、生成や父性という名称は、表示された事物にかんす
 るかぎりで、被造物についてより先に神について語られるが、表示のしかた
 にかんしてはそうでない。したがって、使途も『エフェソの信徒への手紙』
 第3章で「私の膝を折って私たちの主イエス・キリストの父に、そこからすべ
 ての父性が天と地において名付けられる」\footnote{「このようなわけで、
 私は、天と地にあって家族と呼ばれているあらゆるものの源である御父の前
 に、膝をかがめて祈ります。」(3:14-15)}と言っている。

\\

Quod sic
apparet. Manifestum est enim quod generatio accipit speciem a termino,
qui est forma generati. Et quanto haec fuerit propinquior formae
generantis, tanto verior et perfectior est generatio; sicut generatio
univoca est perfectior quam non univoca, nam de ratione generantis
est, quod generet sibi simile secundum formam. 


&

このことは以下のようにして明らかである。生成は、生み出されたものの形相
 である端から種を受け取る。そしてこれが生むものの形相に近ければ近いほ
 どその生成は真であり完全である。ちょうど一義的な生成が一義的でない生
 成よりも完全であるように。というのも、生むものの性格には形相において
 自らに似たものを生み出すということが属するからである。

\\

Unde hoc ipsum quod in
generatione divina est eadem numero forma generantis et geniti, in
rebus autem creatis non est eadem numero, sed specie tantum, ostendit
quod generatio, et per consequens paternitas, per prius sit in Deo
quam in creaturis. Unde hoc ipsum quod in divinis est distinctio
geniti a generante secundum relationem tantum, ad veritatem divinae
generationis et paternitatis pertinet.

&

したがって神の生成において数的に同一の形相が生むものと生み出されたもの
 に属するが、創造されたものにおいては数的に同じではなく種においてのみ
 同じであるというこのこと自体が、生成が、そして結果的に父性が、被造物
 より先に神においてあることを示している。したがって神において生み出さ
 れたものと生み出すものの区別が単に関係においてのみであること自体が、
 神の生成と父性の真理に属している。


\end{longtable}
\newpage


\rhead{a.~3}
\begin{center}
{\Large {\bf ARTICULUS TERTIUS}}\\
{\large UTRUM HOC NOMEN {\itshape PATER} DICATUR IN DIVINIS PER
 PRIUS\\SECUNDUM QUOD PERSONALITER SUMITUR}\\
{\Large 第三項 \\神において「父」というこの名称は\\ペルソナ的に取られる限
 りでより先に語られるか}
\end{center}

\begin{longtable}{p{21em}p{21em}}
{\scshape Ad tertium sic proceditur}. Videtur quod hoc nomen pater non dicatur in
divinis per prius secundum quod personaliter sumitur. Commune enim,
secundum intellectum, est prius proprio. Sed hoc nomen pater, secundum
quod personaliter sumitur, est proprium personae patris, secundum vero
quod sumitur essentialiter est commune toti Trinitati, nam toti
Trinitati dicimus pater noster. Ergo per prius dicitur pater
essentialiter sumptum, quam personaliter.

&

第三項の問題へ議論は以下のように進められる。
この「父」という名称はペルソナ的に取られるかぎりで神においてより先に語
 られるのではないと思われる。理由は以下の通り。
理解内容において共通なものは固有なものよりも先である。しかるに「父」と
 いうこの名称はペルソナ的に取られるかぎり父のペルソナに固有であり、本
 質的に取られるかぎり三性全体に共通である。というのも、私たちは三性全
 体に「私たちの父」と言うからである。ゆえに本質的に取られた「父」がペ
 ルソナ的に取られた「父」より先に語られる。


\\



2. {\scshape Praeterea}, in his quae sunt eiusdem rationis, non est praedicatio per
prius et posterius. Sed paternitas et filiatio secundum unam rationem
videntur dici secundum quod persona divina est pater filii, et
secundum quod tota Trinitas est pater noster vel creaturae, cum,
secundum Basilium, {\itshape accipere} sit commune creaturae et filio. Ergo non
per prius dicitur pater in divinis secundum quod sumitur
essentialiter, quam secundum quod sumitur personaliter.

&

さらに、同じ性格に属するものどもにおいて、より先、より後というかたちで
 述語されることはない。しかるにバシリウスによれば「受け取ること」は被
 造物と息子に共通なので、神のペルソナが息子の父であることと、全三性が
 私たちと被造物の父であることにおいて父性と息子性は同じ性格において語
 られると思われる。ゆえに父は神においてペルソナ的に取られる限りにおい
 てよりも本質的に取られる限りにおいて、より先に語られるのではないと思わ
 れる。


\\



3. {\scshape Praeterea}, inter ea quae non dicuntur secundum rationem unam, non
potest esse comparatio. Sed filius comparatur creaturae in ratione
filiationis vel generationis, secundum illud {\itshape
 Coloss}.~{\scshape i} : {\itshape Qui est imago
Dei invisibilis, primogenitus omnis creaturae}. Ergo non per prius
dicitur in divinis paternitas personaliter sumpta, quam essentialiter;
sed secundum rationem eandem.

&


さらに、一つの性格において語られるのではないものどものあいだに関連はあ
 りえない。しかるにかの『コロサイの信徒への手紙』第1章「彼は見えざる
 神の像であり、すべての被造物の第一に生まれたもの」\footnote{「御子は、
 見えない神のかたちであり/すべてのものが造られる前に/最初に生まれた
 方です。」(1:15)}によれば息子は被造物に息子性ないし生成という性格にお
 いて関連付けられている。ゆえに神において父性が本質的により先にペルソ
 ナ的に取られるのではなく、むしろ同じ性格において取られる。


\\



{\scshape Sed contra est} quod aeternum prius est temporali. Ab aeterno autem
Deus est pater filii, ex tempore autem pater est creaturae. Ergo per
prius dicitur paternitas in Deo respectu filii, quam respectu
creaturae.

&

しかし反対に、永遠なものは時間的なものより先である。しかるに神は永遠か
 ら息子の父であり、時間的に被造物の父である。ゆえに神において父性は被造物への関係にお
 いてより先に、息子への関係において語られる。

\\



{\scshape Respondeo dicendum} quod per prius dicitur nomen de illo in quo
salvatur tota ratio nominis perfecte, quam de illo in quo salvatur
secundum aliquid, de hoc enim dicitur quasi per similitudinem ad id in
quo perfecte salvatur, quia omnia imperfecta sumuntur a perfectis. Et
inde est quod hoc nomen leo per prius dicitur de animali in quo tota
ratio leonis salvatur, quod proprie dicitur leo, quam de aliquo homine
in quo invenitur aliquid de ratione leonis, ut puta audacia vel
fortitudo, vel aliquid huiusmodi, de hoc enim per similitudinem
dicitur. 

&

解答する。以下のように言われるべきである。
ある名称は、そこにおいてその名称の性格がある点に即して汲み取られるとこ
 ろのものよりは、完全に汲み取られるところのものについて、より先に語ら
 れる。なぜなら、前者のものについては、完全に汲み取られるところのもの
 へのいわば類似によって語られるからである。というのも、すべて不完全な
 ものは完全なものから取られるのだから。このことから「獅子」というこの
 名称は、勇気や強さやそのような獅
 子の性格に属する何かが見出される人間よりも先に、そこにおいて獅子の全性格が汲み取られるところの動物について
語られ、固有に獅子と言われるが、人間については、類似によって語られる。


\\

Manifestum est autem ex praemissis quod perfecta ratio
paternitatis et filiationis invenitur in Deo patre et Deo filio, quia
patris et filii una est natura et gloria. Sed in creatura filiatio
invenitur respectu Dei, non secundum perfectam rationem, cum non sit
una natura creatoris et creaturae; sed secundum aliqualem
similitudinem. 

&

ところで前述の事柄から、父である神と息子である神の中に父性と息子性の完
 全な性格が見出されることが明らかである。なぜなら父と息子には一つの本
 性と栄光が属しているからである。しかし被造物において息子性が神に対し
 て見出されるのは完全な性格に即してではない。なぜなら、創造者と被造物
 に同じ本性は属さず、何らかの類似に即して見出されるからである。

\\

Quae quanto perfectior fuerit, tanto propinquius
acceditur ad veram filiationis rationem. Dicitur enim Deus alicuius
creaturae pater, propter similitudinem vestigii tantum, utpote
irrationalium creaturarum; secundum illud {\itshape Iob} {\scshape
 xxxviii} : {\itshape Quis est
pluviae pater? Aut quis genuit stillas roris?} Alicuius vero creaturae,
scilicet rationalis, secundum similitudinem imaginis; secundum illud
{\itshape Deut}.~{\scshape xxxii} : {\itshape Nonne ipse est pater tuus, qui possedit et fecit et
creavit te?} 

&

これは完全であればあるだけ、真の息子性の性格へ近づく。じっさい、神がある
 被造物の父であると言われるのは、痕跡の類似性のためだけにである場合、た
 とえば『ヨブ記』第38章「だれが雨の父か。あるいはだれが露のしずくを生んだか」\footnote{「雨に
 父親があるだろうか。/誰が露の滴を生んだのか。」(38:28)}によれば非理
 性的な被造物の父である場合や、または『申命記』第32章「彼はあなたの父ではないのか、あなたを所有し、作り、創造した方が」
 \footnote{「あなたがたはこのようにして主に恩を返すのか/愚かで知恵の
 ない民よ。/この方こそあなたを造られた父ではないか。/この方があなた
 を造り、揺るぎない者とされた。」(32:6)}によれば、像の類似において何ら
 かの被造物、すなわち理性的な被造物の父である場合がある。


\\


Aliquorum vero est pater secundum similitudinem gratiae,
qui etiam dicuntur filii adoptivi, secundum quod ordinantur ad
haereditatem aeternae gloriae per munus gratiae acceptum; secundum
illud {\itshape Rom}.~{\scshape viii} : {\itshape Ipse spiritus reddit testimonium spiritui nostro,
quod sumus filii Dei; si autem filii, et haeredes}. Aliquorum vero
secundum similitudinem gloriae, prout iam gloriae haereditatem
possident; secundum illud {\itshape Rom}.~{\scshape v} : {\itshape Gloriamur in spe gloriae filiorum
Dei}. 

&

他方で、恩恵の類似においてある人々の父であり、彼らは永遠の栄光の遺産へ
 恩恵の贈り物によって秩序づけられている限りにおいて養子とされた息子と
も言われる。これは『ローマの信徒への手紙』第8章「この霊が私たちの霊に
 私たちが神の息子であることを証言する。しかしもし息子であるならば相続
 人でもある」\footnote{「この霊こそが、私たちが神の子どもであることを、
 私たちの霊と一緒に証ししてくださいます。子どもであれば、相続人でもあ
 ります。神の相続人、しかもキリストと共同の相続人です。キリストと共に
 苦しむなら、共に栄光をも受けるからです。」(8:16-17) }による。またすで
 に栄光の遺産を所有しているかぎりで栄光の類似においてある人々の父であ
 る。これは『ローマの信徒への手紙』第5章「神の息子の栄光への希望におい
 て私たちは誇らしく思う」\footnote{「このキリストのお陰で、今の恵みに
 信仰によって導き入れられ、神の栄光にあずかる希望を誇りにしています。」
 (5:2)}による。


\\

Sic igitur patet quod per prius paternitas dicitur in divinis
secundum quod importatur respectus personae ad personam, quam secundum
quod importatur respectus Dei ad creaturam.

&

ゆえに、このようにして、父性は神においてペルソナのペルソナへの関係が含
 意される限りにおいて、神の被造物への関係が含意される限りにおいてより
 も先に語られることが明らかである。

\\



{\scshape Ad primum ergo dicendum} quod communia absolute dicta, secundum ordinem
intellectus nostri, sunt priora quam propria, quia includuntur in
intellectu propriorum, sed non e converso; in intellectu enim personae
patris intelligitur Deus, sed non convertitur. 

&

第一異論に対してはそれゆえ以下のように言われるべきである。
非関係的に語られた共通的なものは、私たちの知性への関係において固有なも
 のよりも先である。なぜなら固有なものの理解内容に共通なものは含まれる
 が、逆はそうでないからである。たとえば息子のペルソナの理解内容には神
 が含まれるが、逆はそうでないように。

\\

Sed communia quae
important respectum ad creaturam, per posterius dicuntur quam propria
quae important respectus personales, quia persona procedens in
divinis, procedit ut principium productionis creaturarum. 

&

しかし被造物への関係を含意する共通的なものは、ペルソナ的な関係を含意す
 る固有なものよりも後である。なぜなら神において発出するペルソナは、被
 造物の発出の根源として発出するからである。

\\

Sicut enim
verbum conceptum in mente artificis, per prius intelligitur procedere
ab artifice quam artificiatum, quod producitur ad similitudinem verbi
concepti in mente; ita per prius procedit filius a patre quam
creatura, de qua nomen filiationis dicitur secundum quod aliquid
participat de similitudine filii; ut patet per illud quod dicitur
{\itshape Rom}.~{\scshape viii} : {\itshape Quos praescivit, et praedestinavit fieri conformes imaginis
filii eius}.

&

たとえば、ちょうど技術者の精神の中に懐念された言葉は、精神の中に懐念された言葉
 の類似へと生み出される技術作品よりも先に技術者から発出すると理解され
 るように、息子は被造物より先に父から発出し、その息子の類似の何かを分
 有する限りにおいて被造物について息子性という名称が語られる。これは
 『ローマの信徒への手紙』第8章「その人たちをあらかじめ知り、彼の息子の
 像の形に一致することをあらかじめ定めた」\footnote{「神は前もって知っ
 ておられた者たちを、御子のかたちに似たものにしようとあらかじめ定めら
 れました。それは、御子が多くのきょうだいの中で長子となられるためです。」
 (8:29)}による。

\\



{\scshape Ad secundum dicendum} quod accipere dicitur esse commune creaturae et
filio, non secundum univocationem, sed secundum similitudinem quandam
remotam, ratione cuius dicitur primogenitus creaturae. Unde in
auctoritate inducta subditur, {\itshape ut sit ipse primogenitus in multis
fratribus}, postquam dixerat {\itshape conformes fieri aliquos imaginis filii
Dei}. Sed filius Dei naturaliter habet quoddam singulare prae aliis,
scilicet habere per naturam id quod accipit; ut idem Basilius
dicit. Et secundum hoc dicitur unigenitus, ut patet {\itshape
 Ioan}.~{\scshape i} : {\itshape Unigenitus, qui est in sinu patris, ipse nobis enarravit}.

&

第二異論に対しては以下のように言われるべきである。
受けることは被造物と息子に共通であると言われるが、一義性に即してではな
 くある種の遠い類似に即してである。このために被造物の長子
と言われる。したがって権威の中からの引用の中で、「神の息子の像の誰かが
 同じ形になる」と述べた後に「彼が多くの兄弟の中で長子であるように」と
 続けて述べている。しかし神の息子は本性的に他のものと比較してなにか個的なも
 の、すなわち受けているものを本性によって持つということをもつ。これはバ
 シリウスが述べているのと同じである。そしてこのことに従って、独り子と
 いうことが語られている。これは『ヨハネによる福音書』第1章「神の懐にい
 る独り子を私たちに示した」\footnote{「いまだかつて、神を見た者はいない。父の懐にいる独り子である神、この方
 が神を示されたのである。」(1:18)}によって明らかである。

\\



Et per hoc patet solutio ad tertium.

&

これによって第三異論への解答は明らかである。

\\



\end{longtable}
\newpage


\rhead{a.~4}
\begin{center}
{\Large {\bf ARTICULUS QUARTUS}}\\
{\large UTRUM ESSE INGENITUM SIT PATRI PROPRIUM}\\
{\footnotesize I {\itshape Sent.}, d.13, a.4; d.28, q.1, a.1;
 {\itshape SCG}, cap.8.}\\
{\Large 第四項\\不生であることは父に固有か}
\end{center}

\begin{longtable}{p{21em}p{21em}}

{\scshape Ad quartum sic proceditur}. Videtur quod esse ingenitum non sit patri
proprium. Omnis enim proprietas ponit aliquid in eo cuius est
proprietas. Sed ingenitus nihil ponit in patre, sed removet
tantum. Ergo non significat proprietatem patris.


&

第四項の問題へ議論は以下のように進められる。
不生であることは父に固有ではないと思われる。理由は以下の通り。
すべて固有性は、それの固有性であるところの中に何かを措定する。
しかるに不生は父において何かを措定するのではなくたんに除去するだけであ
 る。ゆえに父の固有性を表示しない。

\\



2. {\scshape Praeterea}, ingenitum aut dicitur privative, aut negative. Si negative,
tunc quidquid non est genitum, potest dici ingenitum. Sed spiritus
sanctus non est genitus, neque etiam essentia divina. Ergo ingenitum
etiam eis convenit, et sic non est proprium patri. Si autem privative
sumatur, cum omnis privatio significet imperfectionem in privato,
sequitur quod persona patris sit imperfecta. Quod est impossibile.


&

さらに不生は欠如的に言われるか否定的に言われるかのどちらかである。
もし否定的に言われるならば生まれないものはなんでも不生と言われうる。し
 かるに聖霊は生まれず神の本質も生まれない。ゆえに不生はそれらにも適合し、
 父の固有性でないことになる。他方でもし欠如的に言われるならばすべての
 欠如は欠如するものにおいて不完成を意味するので父のペルソナは不完全だ
 ということになる。これはありえない。



\\



3. {\scshape Praeterea}, ingenitus in divinis non significat relationem, quia non
dicitur relative, significat ergo substantiam. Ingenitus igitur et
genitus secundum substantiam differunt. Filius autem, qui est genitus,
non differt a patre secundum substantiam. Pater ergo non debet dici
ingenitus.


&

さらに、神における不生は関係を表示しない。なぜなら関係的に語られていな
 いからである。ゆえにそれは実体を意味する。ゆえに、不生のものと生まれ
 たものとは実体において異なる。しかるに息子は生まれたものだが父と実体
 において異なることはない。ゆえに父は不生と言われるべきではない。

\\



4. {\scshape Praeterea}, proprium est quod uni soli convenit. Sed cum sint plures ab
alio procedentes in divinis, nihil videtur prohibere quin etiam sint
plures ab alio non existentes. Non igitur est proprium patri esse
ingenitum.


&

さらに固有のものはただ一つのものだけに適合する。しかるに神において複
 数のものが他から発出するので、複数のものが他から存在しないということもま
 た妨げないように思われる。ゆえに不生であることは父に固有でない。

\\



5. {\scshape Praeterea}, sicut pater est principium personae genitae, ita et
personae procedentis. Si ergo propter oppositionem quam habet ad
personam genitam, proprium patris ponitur esse quod sit ingenitus;
etiam proprium eius debet poni quod sit improcessibilis.


&

さらに、父は生まれたペルソナの根源であるように、発出するペルソナの根源
 でもある。ゆえにもし生まれたペルソナに対して持つ対立性のために不生で
 あるという父に固有なものが措定されるならば、不発出という固有なものも
 また措定されるべきである。


\\



{\scshape Sed contra} est quod dicit Hilarius, IV {\itshape de
 Trin}. : {\itshape Est unus ab uno},
scilicet ab ingenito genitus, {\itshape proprietate videlicet in unoquoque et
innascibilitatis et originis}.


&

しかし反対に、ヒラリウスは『三位一体論』第4巻で次のように述べている。「一人は一人からある」すな
 わち生まれたものは生まれないものからある。「すなわち各々のものにおいて
 不生性と起源という固有性によって」。


\\



{\scshape Respondeo dicendum} quod, sicut in creaturis invenitur principium
primum et principium secundum, ita in personis divinis, in quibus non
est prius et posterius, invenitur {\itshape principium non de principio}, quod
est pater, et {\itshape principium a principio}, quod est filius. 

&

解答する。以下のように言われるべきである。
被造物において第一根源と第二根源が見出されるように、より先より後がない神のペルソナにおい
 ても根源からでない根源すなわち父と、根源からである根源すなわち息子が
 見出される。

\\

In rebus autem
creatis aliquod principium primum innotescit dupliciter, uno quidem
modo, inquantum est principium primum per hoc quod habet relationem ad
ea quae ab ipso sunt; alio modo, inquantum est primum principium per
hoc quod non est ab alio. 

&

ところで被造物においてある第一根源は二通りのしかたで知られる。一つは、
 それによってあるものどもへそれがもつ関係によってそれが第一根源である
 限りにおいてであり、もう一つは、他によってでないことによって第一根源
 である限りにおいてである。

\\


Sic igitur et pater innotescit quidem
paternitate et communi spiratione, per respectum ad personas ab eo
procedentes, inquantum autem est principium non de principio,
innotescit per hoc, quod non est ab alio, quod pertinet ad
proprietatem innascibilitatis, quam significat hoc nomen {\itshape ingenitus}.


&

ゆえに、このようなかたちで父は父から発出するペルソナとの関係を通して父
 性と共通霊発によって知られるが、他方で、根源からではない根源である限
 りにおいて、他によってではないということを通して知られる。このことは
 不生性という固有性に属するが、これを「不生」という名称が表示する。

\\



{\scshape Ad primum ergo dicendum} quod quidam dicunt quod innascibilitas, quam
significat hoc nomen ingenitus, secundum quod est proprietas patris,
non dicitur tantum negative; sed importat vel utrumque simul, scilicet
quod pater a nullo est, et quod est principium aliorum; vel importat
universalem auctoritatem; vel etiam fontalem plenitudinem. 

&

第一異論に対してはそれゆえ以下のように言われるべきである。
ある人々は「不生」というこの名称が表示する不生性が、父の固有性である限
 りにおいて、たんに否定的に語られるのではなく、父は何ものからでもない
 ということと、他の者どもの根源であるということあるいは普遍的な権威で
 あること、あるいはさらに泉のような充溢であることの両方を同時に含意す
 ると述べている。

\\

Sed hoc non
videtur verum. Quia sic innascibilitas non esset alia proprietas a
paternitate et spiratione, sed includeret eas, sicut includitur
proprium in communi, nam fontalitas et auctoritas nihil aliud
significant in divinis quam principium originis. 


&

しかしこれは真だとは思われない。なぜなら、もしそうなら不生性は父性や霊
 発とは別の固有性になって、固有なものが共通なものに含まれる形でそれら
 を含んだであろうから。つまり泉のようであることや権威は神において起源
 という意味での根源を表示するからである。

\\

Et ideo dicendum est,
secundum Augustinum, V {\itshape de Trin}., quod {\itshape ingenitus} negationem
generationis passivae importat, dicit enim quod {\itshape tantum valet quod
dicitur ingenitus, quantum valet quod dicitur non filius}. Nec propter
hoc sequitur quod {\itshape ingenitus} non debeat poni propria notio patris, quia
prima et simplicia per negationes notificantur; sicut dicimus punctum
esse cuius {\itshape pars non est}.


&

ゆえに、アウグスティヌス『三位一体論』第5巻に従って、不生は受動的な生
 成の否定を意味すると言われるべきである。すなわち不生であると言われる
 のは、息子でないと言われるのとちょうど同じ意味だからである。またこの
 ために、不生が父の固有の識標として置かれるべきでないということも帰結
 しない。なぜなら、第一で単純なものは、否定によって識表されるからであ
 る。たとえば私たちは点を部分がないものと言うように。

\\



{\scshape Ad secundum dicendum} quod ingenitum quandoque sumitur negative
tantum. Et secundum hoc Hieronymus dicit spiritum sanctum esse
ingenitum, idest non genitum. Alio modo potest dici ingenitum aliquo
modo privative, non tamen aliquam imperfectionem
importat. 

&

第二異論に対しては以下のように言われるべきである。
時として不生はたんに否定的にとられる。そしてこの意味でヒエロニムスは
聖霊は不生である、すなわち「生まれたのでない」と言っている。
別の場合に不生は、なんらかの意味で欠如的に語られることが可能だが、しか
 しいかなる不完全性も含意しない。

\\


Multipliciter enim dicitur privatio. Uno modo, quando
aliquid non habet quod natum est haberi ab alio, etiamsi ipsum non sit
natum habere illud, sicut si lapis dicatur res mortua, quia caret
vita, quam quaedam res natae sunt habere. Alio modo dicitur privatio,
quando aliquid non habet quod natum est haberi ab aliquo sui generis;
sicut si talpa dicatur caeca. Tertio modo, quando ipsum non habet quod
natum est habere, et hoc modo privatio imperfectionem importat. 

&

理由は以下の通り。「欠如」は多くの意味で言われる。
一つには、たとえば石が、ある事物は生まれつき持っている生命を欠いているがゆえに「死すべき事物」と言われ
 る場合のように、それを生まれつき持つのではないにしても、他のものから
 生まれつき持たれるものを欠いている場合である。
もう一つにの意味で欠如といわれるのは、モグラが盲目と言われる場合のよう
 に、自らの類に属する何かによって生まれつき持たれるものを持たない場合
 である。第三の意味では、それが生まれつき持つものを持たない場合であり、
 この意味で欠如は不完全性を意味する。


\\


Sic
autem ingenitum non dicitur privative de patre, sed secundo modo,
prout scilicet aliquod suppositum divinae naturae non est genitum,
cuius tamen naturae aliquod suppositum est genitum. Sed secundum hanc
rationem, etiam de spiritu sancto potest dici ingenitum. Unde ad hoc
quod sit proprium soli patri, oportet ulterius in nomine ingeniti
intelligere, quod conveniat alicui personae divinae quae sit
principium alterius personae; ut sic intelligatur importare negationem
in genere principii personaliter dicti in divinis. 

&

しかし、この意味で不生が父について語られるわけではなく、それは第二の意
 味において、すなわち、神の本性を持つ何らかの個体が生まれていないが、
 同じ本性をもつ他の個体は生まれている。しかしこの意味において、聖霊も
 また不生と言われうる。したがって、ただ父にのみ固有であるもののために
 は、さらに、不生という名称において、他のペルソナの根源であるペルソナ
 に適合することを理解しなければならない。その結果、神においてペルソナ
 的に語られる根源の類において否定を意味することが理解される。


\\


Vel, ut
intelligatur in nomine ingeniti, quod omnino non sit ab alio, et non
solum quod non sit ab alio per generationem. Sic enim nec spiritui
sancto convenit esse ingenitum, qui est ab alio per processionem ut
persona subsistens, nec etiam divinae essentiae, de qua potest dici
quod est in filio vel in spiritu sancto ab alio, scilicet a patre.


&

あるいは、不生という名称において、たんに他のものから生成によってあるの
 でないということだけでなく、あらゆる意味において他のものからではない
 ということが理解される。この意味では、聖霊にも不生であることは適合し
 ない。聖霊は自存するペルソナとして発出を通して他のものから存在するか
 らである。また、神の本質も(不生ではない)。なぜなら神の本質は息子や
 聖霊の中にあると言われうるが、それらは他から、つまり父からであるから。


\\



{\scshape Ad tertium dicendum} quod, secundum Damascenum, ingenitum uno modo
significat idem quod increatum, et sic secundum substantiam dicitur;
per hoc enim differt substantia creata ab increata. Alio modo
significat id quod non est genitum. Et sic relative dicitur, eo modo
quo negatio reducitur ad genus affirmationis, sicut non homo ad genus
substantiae, et non album ad genus qualitatis. 

&

第三異論に対しては以下のように言われるべきである。
ダマスケヌスによれば、不生は、ある意味では創造されないものを意味し、そ
 の意味で実体に即して語られる。またこれによって被造の実体は創造されな
 い実体から区別される。別の意味では、生まれたものでないことを意味し、
 その意味では関係的に語られる。これはちょうど「人間でないもの」が実体
 の類に、「白くないもの」が性質の類に還元されるように、否定が肯定の類
 に還元されるようにしてである。



\\


Unde, cum genitum in
divinis relationem importet, ingenitum etiam ad relationem
pertinet. Et sic non sequitur quod pater ingenitus distinguatur a
filio genito secundum substantiam; sed solum secundum relationem,
inquantum scilicet relatio filii negatur de patre.


&


したがって、神において生まれたものは関係を意味するので、不生もまた関係
 に属する。こうして、不生である父が生まれたものである息子から実体にお
 いて区別されるということは帰結しない。むしろ帰結するのは関係において、
 すなわち、息子の関係が父について否定される限りにおいて区別されるとい
 うことである。

\\



{\scshape Ad quartum dicendum} quod, sicut in quolibet genere oportet ponere unum
primum, ita in divina natura oportet ponere unum principium quod non
sit ab alio, quod ingenitum dicitur. Ponere igitur duos innascibiles,
est ponere duos deos, et duas naturas divinas. Unde Hilarius dicit, in
libro {\itshape de Synodis} : {\itshape Cum unus Deus sit, duo innascibiles esse non
possunt}. Et hoc praecipue quia, si essent duo innascibiles, unus eorum
non esset ab alio, et sic non distinguerentur oppositione relativa:
oporteret igitur quod distinguerentur diversitate naturae.


&

第四異論に対しては以下のように言われるべきである。
どんな類においても一つの第一のものを措定しなければならないように、神の
 本性においても他からではない一つの根源を措定しなければならない。これ
 が不生と言われる。ゆえに、二つの不生性を措定することは二つの神を、そ
 して二つの神の本性を措定することである。従ってヒラリウスは『教会会議
 について』で「神は一人なのだから二つの不生性があることは不可能である」
 と述べている。そしてこれはとくに、もし二つの不生のものがいたならば、
 それらのうちの一方が他方からではないことになり、関係的対立によって区
 別されないことになり、それゆえ本性の異なりによって区別されなければな
 らなくなるからである。



\\



{\scshape Ad quintum dicendum} quod proprietas patris prout non est ab alio,
potius significatur per remotionem nativitatis filii, quam per
remotionem processionis spiritus sancti. Tum quia processio spiritus
sancti non habet nomen speciale, ut supra dictum est. Tum quia etiam
ordine naturae praesupponit generationem filii. 

&

第五異論に対しては以下のように言われるべきである。
他からでないという父の固有性は、聖霊の発出よりも息子の出生を除去するこ
 とによってより表示される。その理由は、一つにはすでに述べられたとおり聖霊の発出が特別な名称を
 持っていないからであり、また一つには、本性の秩序においても(聖霊の発
 出は)息子の生成を前提するからである。


\\


Unde, remoto a patre
quod non sit genitus, cum tamen sit principium generationis, sequitur
consequenter quod non sit procedens processione spiritus sancti, quia
spiritus sanctus non est generationis principium, sed a genito
procedens.


&

従って、生まれたことを父から除去することで、他方で父は生成の根
 源なので、結果的に、聖霊の発出によって発出するのでないことが帰結する。
 なぜなら聖霊は出生の根源でなく、生み出されたものから発出するからであ
 る。


\end{longtable}
\newpage


\end{document}