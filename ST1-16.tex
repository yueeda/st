\\f\documentclass[10pt]{jsarticle} % use larger type; default would be 10pt
%\usepackage[utf8]{inputenc} % set input encoding (not needed with XeLaTeX)
%\usepackage[round,comma,authoryear]{natbib}
%\usepackage{nruby}
\usepackage{okumacro}
\usepackage{longtable}
%\usepqckage{tablefootnote}
\usepackage[polutonikogreek,english,japanese]{babel}
%\usepackage{amsmath}
\usepackage{latexsym}
\usepackage{color}

%----- header -------
\usepackage{fancyhdr}
\lhead{{\it Summa Theologiae} I, q.~16}
%--------------------

\bibliographystyle{jplain}

\title{{\bf PRIMA PARS}\\{\HUGE Summae Theologiae}\\Sancti Thomae
Aquinatis\\{\sffamily QUEAESTIO DECIMASEXTA}\\DE VERITATE}
\author{Japanese translation\\by Yoshinori {\sc Ueeda}}
\date{Last modified \today}


%%%% コピペ用
%\rhead{a.~}
%\begin{center}
% {\Large {\bf }}\\
% {\large }\\
% {\footnotesize }\\
% {\Large \\}
%\end{center}
%
%\begin{longtable}{p{21em}p{21em}}
%
%&
%
%
%
%\\
%\end{longtable}
%\newpage



\begin{document}
\maketitle
\pagestyle{fancy}

\begin{center}
{\Large 第十六問\\真理について}
\end{center}

\begin{longtable}{p{21em}p{21em}}
{\huge Q}{\scshape uoniam} autem scientia verorum est, post
 considerationem scientiae Dei, de veritate inquirendum est. Circa quam
 quaeruntur octo. 
\begin{enumerate}
 \item utrum veritas sit in re, vel tantum in intellectu.
 \item utrum sit tantum in intellectu componente et dividente.
 \item de comparatione veri ad ens.
 \item de comparatione veri ad bonum.
 \item utrum Deus sit veritas.
 \item utrum omnia sint vera veritate una, vel pluribus.
 \item de aeternitate veritatis.
 \item de incommutabilitate ipsius.
\end{enumerate}

&

知とは、真についての知だから、神の知についての考察のあと、真理について探
 究されるべきである。これについて、8つのことが問われる。
\begin{enumerate}
 \item 真理は事物の中にあるか、あるいは、ただ知性の中だけにあるか。
 \item ただ複合分割する知性の中にあるか。
 \item 真と有の関係について。
 \item 真と善の関係について。
 \item 神は真理か。
 \item 万物は一つに真理によって真か、あるいは、複数の真理によってか。
 \item 真理の永遠性について。
 \item 真理の不変性について。
\end{enumerate}

\end{longtable}


\newpage

\rhead{a.~1}
\begin{center}
 {\Large {\bf ARTICULUS PRIMUS}}\\
 {\large UTRUM VERITAS SIT TANTUM IN INTELLECTU}\\
 {\footnotesize I {\itshape Sent.}, d.~19, q.~5, a.~1; I {\itshape
 SCG.}, c.~60; {\itshape De Verit.}, q.~1, a.~2; I {\itshape Periherm.},
 lect.~3; VI {\itshape Metaphys.}, lect.~4.}\\
 {\Large 第一項\\真理はただ知性の中だけにあるか}
\end{center}

\begin{longtable}{p{21em}p{21em}}



{\huge A}{\scshape d primum sic proceditur}. Videtur quod veritas non
sit tantum in intellectu, sed magis in rebus. Augustinus enim, in
libro {\itshape Soliloq}., reprobat hanc notificationem veri,
{\itshape verum est id quod videtur}, quia secundum hoc, lapides qui
sunt in abditissimo terrae sinu, non essent veri lapides, quia non
videntur. Reprobat etiam istam, {\itshape verum est quod ita se habet
ut videtur cognitori, si velit et possit cognoscere}, quia secundum
hoc sequeretur quod nihil esset verum, si nullus posset cognoscere. Et
definit sic verum, {\itshape verum est id quod est}. Et sic videtur
quod veritas sit in rebus, et non in intellectu.

&

第一項の問題へ、議論は以下のように進められる。真理は、ただ知性の中だけ
にあるのではなく、むしろ事物においてあると思われる。なぜなら、アウグス
ティヌスは、『ソリロクィア』 という書物で「真とは見られるものである」
という真の規定を批判する。なぜなら、これによれば、大地のもっとも深い内
奥にある石は、見られないために真の石でないことになるから。さらに、「真
とは、認識しようと意志し、認識できるならば、認識者に見られるもの」とい
う規定も批判する。なぜなら、これによれば、もしだれも認識できないならば、
何も真でないことになっただろうから。そして彼は、真を、「真とは在るもの
である」と定義する。したがって、真理は事物において在り、知性において在
るのではないと思われる。

\\


{\scshape 2 Praeterea}, quidquid est verum, veritate verum est. Si
igitur veritas est in intellectu solo, nihil erit verum nisi secundum
quod intelligitur, quod est error antiquorum philosophorum, qui
dicebant omne quod videtur, esse verum. Ad quod sequitur
contradictoria simul esse vera, cum contradictoria simul a diversis
vera esse videantur.


&

さらに、なんであれ真であるものは、真理によって真である。ゆえに、もし真
 理が知性の中だけにあるとしたら、何ものも、知性認識されることによって
 でなければ真でないだろう。しかしこれは、すべてそう見えるものが真であ
 る、と言った昔の哲学者たちの誤りである。[なぜ誤りかと言えば]矛盾す
 るものが、同時に、異なる人々に真であると思われる[ことがある]ので、
 このことからは、矛盾するものが同時に真である[ことがある]ということ
 が帰結するからである。

\\


{\scshape 3 Praeterea}, {\itshape propter quod unumquodque, et illud
magis}, ut patet I {\itshape Poster}. Sed {\itshape ex eo quod res est
vel non est, est opinio vel oratio vera vel falsa}, secundum
philosophum in {\itshape Praedicamentis}. Ergo veritas magis est in
rebus quam in intellectu.


&

さらに、『分析論後書』第1巻で明らかなとおり、$a$のゆえに、$b, c, d,
...$がPであるとき、aは$b, c, d, ...$よりもいっそうPである。ところが、
『カテゴリー論』の哲学者によれば、「事物がある、またはあらぬということ
のゆえに、意見や言明は真、または偽である」。ゆえに、真理は、知性よりも
事物において在る。

\\


{\scshape Sed contra est} quod philosophus dicit, VI {\itshape
Metaphys}., quod {\itshape verum et falsum non sunt in rebus, sed in
intellectu}.


&

 しかし反対に、哲学者は『形而上学』第6巻で「真と偽は、事物の中ではなく知性
 の中にある」と述べている。


\\


{\scshape Respondeo dicendum} quod, sicut bonum nominat id in quod
tendit appetitus, ita verum nominat id in quod tendit intellectus. Hoc
autem distat inter appetitum et intellectum, sive quamcumque
cognitionem, quia cognitio est secundum quod cognitum est in
cognoscente, appetitus autem est secundum quod appetens inclinatur in
ipsam rem appetitam. Et sic terminus appetitus, quod est bonum, est in
re appetibili, sed terminus cognitionis, quod est verum, est in ipso
intellectu.


&

解答する。以下のように言われるべきである。ちょうど「善」が、欲求がそこ
へと向かうものを名付ける名であるように、「真」は、知性がそこへと向かう
ものを名付ける名である。ところで、欲求と知性、あるいは、なんであれ認識
は、次の点で異なる。すなわち、認識は、認識されたものが認識者の中にある
ことで成立するが、欲求は、欲求するものが欲求される事物そのものへと傾く
ことによって成り立つ。この意味で、欲求の行き着く先、つまり善は、欲求さ
れうる事物の中にあるが、認識が行き着く先、つまり真は、知性自身の中にあ
る。

\\


\hspace{1em}Sicut autem bonum est in re, inquantum habet ordinem ad
appetitum; et propter hoc ratio bonitatis derivatur a re appetibili in
appetitum, secundum quod appetitus dicitur bonus, prout est boni, ita,
cum verum sit in intellectu secundum quod conformatur rei intellectae,
necesse est quod ratio veri ab intellectu ad rem intellectam derivetur,
ut res etiam intellecta vera dicatur, secundum quod habet aliquem
ordinem ad intellectum. 


&

また、善は、欲求への秩序をもつという点で事物の中にあるが、このために、
 善性という性格が欲求されうる事物から欲求へ派生し、このことによって、
 善にかかわるものとして欲求が善と言われる。ちょうどそのように、真は、
 知性認識された事物に一致するという点で知性の中にあるから、真という性
 格が知性から知性認識された事物へと派生し、その結果、知性へのなんらか
 の秩序をもつ点で、知性認識された事物もまた真と言われる。


\\

\hspace{1em}Res autem intellecta ad intellectum aliquem potest habere
ordinem vel per se, vel per accidens. Per se quidem habet ordinem ad
intellectum a quo dependet secundum suum esse, per accidens autem ad
intellectum a quo cognoscibilis est. Sicut si dicamus quod domus
comparatur ad intellectum artificis per se, per accidens autem
comparatur ad intellectum a quo non dependet.


&

ところで、知性認識された事物は、自体的に、または附帯的に、何らかの知性
に対して秩序をもつことができる。自体的には、自らの存在という点で依存す
る知性へ秩序をもち、附帯的には、それによって認識されうるような知性へ
[秩序をもつ]。たとえば、私たちが、家は[それを作る]技術者の知性に自
体的に関係するが、それに依存しない知性には附帯的に関係する、と言うよう
に。


\\


Iudicium autem de re non sumitur secundum id quod inest ei per
accidens, sed secundum id quod inest ei per se. Unde unaquaeque res
dicitur vera absolute, secundum ordinem ad intellectum a quo
dependet. Et inde est quod res artificiales dicuntur verae per ordinem
ad intellectum nostrum, dicitur enim domus vera, quae assequitur
similitudinem formae quae est in mente artificis; et dicitur oratio
vera, inquantum est signum intellectus veri.

&

ところで、事物についての判断は、その事物に附帯的に内在するものではなく、
 自体的に内在するものにしたがって行われる。したがって、各々の事物は、
 それが依存する知性への秩序という観点から、無条件に真と言われる。人工
 物が真と言われるのが私たちの知性への秩序によってであるのはこのためで
 ある。たとえば、家が真と言われるのは、それが技術者の精神の中にある形
 相の類似に到達しているものだからであり、言明が真であるのは、それが真
 の知性認識のしるしだという点でだからである。


\\


Et similiter res naturales dicuntur esse verae, secundum quod
assequuntur similitudinem specierum quae sunt in mente divina, dicitur
enim verus lapis, qui assequitur propriam lapidis naturam, secundum
praeconceptionem intellectus divini. Sic ergo veritas principaliter
est in intellectu; secundario vero in rebus, secundum quod comparantur
ad intellectum ut ad principium.

&

自然物も同様に、神の精神の中にある形象の類似に到達しているという点で、
 真であると言われる。たとえば、真の石と言われるのは、神の知性があらか
 じめ抱いた概念に即して、石の固有の本性に到達しているものである。それ
 ゆえ、このように、真理は主要に知性の中にあるが、第二義的には、根源と
 しての知性へ関係するという観点から、事物の中にもある。

\\

\hspace{1em}Et secundum hoc, veritas diversimode notificatur. Nam
Augustinus, in libro {\itshape de Vera Relig}., dicit quod {\itshape
veritas est, qua ostenditur id quod est}. Et Hilarius dicit quod
{\itshape verum est declarativum aut manifestativum esse}. Et hoc
pertinet ad veritatem secundum quod est in intellectu.


&

このため、真理はさまざまに規定される。アウグスティヌスは『真の宗教につ
 いて』という書物で「真理とはそれによって存在するものが明示されるもの
 である」と言う。またヒラリウスは「真とは存在を示し明らかにしうるもの
 である」と言う。これは、知性の中にあるものとしての真理に関係する。

\\


--- Ad veritatem autem rei secundum ordinem ad intellectum, pertinet
definitio Augustini in libro {\itshape de Vera Relig}., {\itshape
veritas est summa similitudo principii, quae sine ulla dissimilitudine
est}. Et quaedam definitio Anselmi, {\itshape veritas est rectitudo sola
mente perceptibilis}; nam rectum est, quod principio concordat. Et
quaedam definitio Avicennae, {\itshape veritas uniuscuiusque rei est
proprietas sui esse quod stabilitum est ei}. --- Quod autem dicitur quod
{\itshape veritas est adaequatio rei et intellectus} potest ad utrumque
pertinere.


&

『真の宗教について』の「真理は根源の最高の類似であり、それはどんな不類
 似ももたずに在る」というアウグスティヌスの定義は、知性への秩序という
 観点から事物の真理に関係する。また、アンセルムスの「真理とは精神だけ
 に捉えられる正しさである」という定義も[これに類する]。なぜなら、根
 源に一致するものが「正しい」のだから。そしてアヴィセンナの「各々の事
 物の真理とは、その事物に固定された自分の存在の固有性である」も[同じ
 である]。他方、「真理とは事物と知性の対等である」と言われるものは、
 どちらにも関係しうる。

\\


{\scshape Ad primum ergo dicendum} quod Augustinus loquitur de
veritate rei; et excludit a ratione huius veritatis, comparationem ad
intellectum nostrum. Nam id quod est per accidens, ab unaquaque
definitione excluditur.


&

第一異論に対しては、それゆえ、次のように言われるべきである。アウグスティ
ヌスは、事物の真理について語っていて、その真理の性格から、私たちの知性
への関係を排除している。というのも、附帯的なことがらは、どんなものの定
義からも排除されるからである。

\\


{\scshape Ad secundum dicendum} quod antiqui philosophi species rerum
naturalium non dicebant procedere ab aliquo intellectu, sed eas
provenire a casu, et quia considerabant quod verum importat
comparationem ad intellectum, cogebantur veritatem rerum constituere
in ordine ad intellectum nostrum. Ex quo inconvenientia sequebantur
quae philosophus prosequitur in IV {\itshape Metaphys}. Quae quidem
inconvenientia non accidunt, si ponamus veritatem rerum consistere in
comparatione ad intellectum divinum.

&

第二異論に対しては次のように言われるべきである。昔の哲学者たちは、自然
的事物の種が何らかの知性から出てくるのではなく、偶然たまたま生じたと言っ
ていて、しかも、「真」は知性への関係を意味すると考えていたので、やむを
えず、私たちの知性との関係で、事物の真理を作り上げるしかなかった。この
ことから、哲学者が『形而上学』4巻で追究している不都合が帰結した。この
不都合は、もし私たちが、事物の真理は神の知性との関係において成立すると
するならば、生じない。

\\


{\scshape Ad tertium dicendum} quod, licet veritas intellectus nostri
a re causetur, non tamen oportet quod in re per prius inveniatur ratio
veritatis, sicut neque in medicina per prius invenitur ratio sanitatis
quam in animali; virtus enim medicinae, non sanitas eius, causat
sanitatem, cum non sit agens univocum. Et similiter esse rei, non
veritas eius, causat veritatem intellectus. Unde philosophus dicit
quod opinio et oratio vera est {\itshape ex eo quod res est, non ex eo
quod res vera est}.


&

第三異論に対しては、次のように言われるべきである。たしかに、私たちの知
性の真理は事物が原因となって生じるが、しかし、真理の性格が、事物の中に、
先に見出されなければならないということはない。たとえば、「健康」という
性格は、動物よりも薬の中に先に見出されるということはない。薬は一義的作
用者ではないから、薬の中にある健康ではなく、薬の中にある力が、健康の原
因だからである。同様に、事物の真理ではなく、事物の存在が原因となって、
知性の真理を生み出す。このことから、哲学者は、意見や言明が真であるのは、
「事物が在るからであって、事物が真であるからではない」と述べている。
\end{longtable}
\newpage


\rhead{a.~2}
\begin{center}
 {\Large {\bf ARTICULUS SECUNDUS}}\\
 {\large UTRUM VERITAS SIT IN INTELLECTU COMPONENTE ET DIVIDENTE}\\
 {\footnotesize I {\itshape Sent.}, d.~19, q.~5, a.~1, ad 7; I {\itshape
 SCG.}, c.~59; {\itshape De Verit.}, q.~1, a.~3, 9; I {\itshape
 Periherm.}, lect.~3; VI {\itshape Metaphys.}, lect.~4; III {\itshape de
 Anima}, lect.~9.}\\
 {\Large 第二項\\真理は複合分割する知性の中にあるか}
\end{center}

\begin{longtable}{p{21em}p{21em}}

{\huge A}{\scshape d secundum sic proceditur}. Videtur quod veritas non
sit solum in intellectu componente et dividente. Dicit enim philosophus,
in III {\itshape de Anima}, quod sicut sensus propriorum sensibilium
semper veri sunt, ita et intellectus {\itshape eius quod quid est}. Sed
compositio et divisio non est neque in sensu, neque in intellectu
cognoscente {\itshape quod quid est}. Ergo veritas non solum est in
compositione et divisione intellectus.

&

第二項の問題へ、議論は以下のように進められる。真理は、ただ複合分割する
知性の中だけにあるのではないと思われるなぜなら、哲学者は『デ・アニマ』
第3巻で、感覚が、固有に可感的なものについては常に真であるように、知性
もまた「何であるか」について[常に真である]と述べている。ところが、複
合分割は、感覚の中にも、「何であるか」を認識する知性の中にもない。ゆえ
に、真理は、複合分割する知性の中だけにあるのではない。

 \\


{\scshape 2 Praeterea}, Isaac dicit, in libro {\itshape de
Definitionibus}, quod veritas est adaequatio rei et intellectus. Sed
sicut intellectus complexorum potest adaequari rebus, ita intellectus
incomplexorum, et etiam sensus sentiens rem ut est. Ergo veritas non
est solum in compositione et divisione intellectus.

&

さらに、イサクは『定義について』という書物で、真理とは事物と知性の対等
であると述べている。ところで、複合されたものについての知性が事物と対当
しうるのと同様、複合されてないものについての知性もまた[事物と対当しう
る]。さらに、事物をあるがままに感覚する感覚もまた[事物と対当しうる]。
ゆえに、真理は、複合分割する知性だけの中にあるのではない。

\\

{\scshape Sed contra est} quod dicit philosophus, in VI {\itshape
Metaphys}., quod circa simplicia et {\itshape quod quid est} non est
veritas, nec in intellectu neque in rebus.

&

しかし反対に、哲学者は『形而上学』6巻で、単純なものと「何であるか」に
 ついては、知性の中にも事物の中にも真理はない、と述べている。

\\

{\scshape Respondeo dicendum} quod verum, sicut dictum est, secundum
sui primam rationem est in intellectu. Cum autem omnis res sit vera
secundum quod habet propriam formam naturae suae, necesse est quod
intellectus, inquantum est cognoscens, sit verus inquantum habet
similitudinem rei cognitae, quae est forma eius inquantum est
cognoscens. Et propter hoc per conformitatem intellectus et rei
veritas definitur. Unde conformitatem istam cognoscere, est cognoscere
veritatem.


& 

解答する。以下のように言われるべきである。真は、すでに述べられたとおり、
その第一の性格という点では知性の中にある。ところで、すべての事物は、自
分の本性の固有の形相をもつという点で真であるから、必然的に、知性は、そ
れが認識者であるという観点から言って、認識された事物の類似をもつという
点で真である。というのも、この類似は、認識者であるかぎりでの知性の形相
なのだから。このため、知性と事物の一致によって、真理が定義される。ゆえ
に、この一致を認識することが、「真理」を認識することである。


\\


Hanc autem nullo modo sensus cognoscit, licet enim visus habeat
similitudinem visibilis, non tamen cognoscit comparationem quae est
inter rem visam et id quod ipse apprehendit de ea. Intellectus autem
conformitatem sui ad rem intelligibilem cognoscere potest, sed tamen
non apprehendit eam secundum quod cognoscit de aliquo {\itshape quod
quid est}; sed quando iudicat rem ita se habere sicut est forma quam
de re apprehendit, tunc primo cognoscit et dicit verum.


&

しかし、感覚は、これをけっして認識しない。たしかに、視覚は、見られうる
ものの類似をもつが、しかし、見られる事物と、視覚がその事物についてとら
えるものとの間にある関係を認識しない。これに対して、知性は、自分自身の、
可知的な事物への関係を認識できるが、その関係を、何かについて「何である
か」を認識することでとらえるのではない。それをとらえるのは、事物につい
て、その事物についてとらえている形相と同じようにあると判断するときであ
り、そのとき、最初に、真を認識し、語る。


\\


Et hoc facit componendo et dividendo, nam in omni propositione aliquam
formam significatam per praedicatum, vel applicat alicui rei
significatae per subiectum, vel removet ab ea. Et ideo bene invenitur
quod sensus est verus de aliqua re, vel intellectus cognoscendo
{\itshape quod quid est}, sed non quod cognoscat aut dicat verum. Et
similiter est de vocibus complexis aut incomplexis.


&

そしてこのことを、複合分割することで、すなわち、すべての命題の中で、述
語によって表示されたある形相を、主語によって表示された何らかの事物に適
用したり、あるいは除去したりすることで、行う。ゆえに、感覚が、ある事物
に関して真であるとか、「何であるか」を認識するに際して知性が真である、
ということは十分に成り立つが、しかし、真[という性格それ自体]を認識し
たり語ったりすることはないのであって、これは、複合的音声と非複合的音声
についても同様である。

\\

Veritas quidem igitur potest esse in sensu, vel in intellectu
cognoscente {\itshape quod quid est}, ut in quadam re vera, non autem
ut cognitum in cognoscente, quod importat nomen {\itshape veri};
perfectio enim intellectus est verum ut cognitum. Et ideo, proprie
loquendo, veritas est in intellectu componente et dividente, non autem
in sensu, neque in intellectu cognoscente {\itshape quod quid est}.

&

ゆえに、真理は、感覚や「何であるか」を認識する知性の中に、一種の真の事
物の中にあるかたちであり、認識されたものが認識するものの中にあるという
かたちであるのではない。しかし「真」という言葉は、この後者のことを意味
する。じっさい、知性の完成は、認識されたものとしての真である。ゆえに、
厳密に言うならば、真理は複合分割する知性の中にあり、感覚の中にも、「何
であるか」を認識する知性の中にもない。

\\

Et per hoc patet solutio ad obiecta.

&

以上のことから、異論に対する解答は明らかである。

\end{longtable}
\newpage

\rhead{a.~3}
\begin{center}
 {\Large {\bf ARTICULUS TERTIUS}}\\
 {\large UTRUM VERUM ET ENS CONVERTUNTUR}\\
 {\footnotesize I {\itshape Sent.}, d.~8, q.~1, a.~3; d.~19, q.~5, a.~1,
 ad 3, 7; {\itshape De Verit.}, q.~1, a.~1; a.~2, ad 1.}\\
 {\Large 第三項\\真と有は置換されるか}
\end{center}

\begin{longtable}{p{21em}p{21em}}

{\huge A}{\scshape d tertium sic proceditur}. Videtur quod verum et
ens non convertantur. Verum enim est proprie in intellectu, ut dictum
est. Ens autem proprie est in rebus. Ergo non convertuntur.

&

第三項の問題へ、議論は以下のように進められる。真と有は置換されないと思
われる。なぜなら、すでに言われたとおり、真は固有に知性の中にある。他方、
有は、固有に事物においてある。ゆえに、置換されない。

\\

{\scshape 2 Praeterea}, id quod se extendit ad ens et non ens, non
 convertitur cum ente. Sed verum se extendit ad ens et non ens, nam
 verum est quod est esse, et quod non est non esse. Ergo verum et ens
 non convertuntur.


&

さらに、有と非有へ広がるものは、有と置換されない。ところが、真は、有と
非有にまで広がる。なぜなら、存在するものが存在すること、存在しないもの
が存在しないことは真だからである。ゆえに、真と有は置換されない。

\\

{\scshape 3 Praeterea}, quae se habent secundum prius et posterius,
 non videntur converti. Sed verum videtur prius esse quam ens, nam ens
 non intelligitur nisi sub ratione veri. Ergo videtur quod non sint
 convertibilia.

&

さらに、より先、より後、という関係にあるものどもは、置換されないと思わ
れる。ところが、真は、有より先であると思われる。なぜなら、有は、真とい
う性格のもとでしか知性認識されないからである。ゆえに、これらは置換可能
でない。

\\


{\scshape Sed contra est} quod dicit philosophus, II {\itshape
 Metaphys}., quod eadem est dispositio rerum in esse et veritate.


&

しかし反対に、哲学者は『形而上学』第2巻で、存在と真理において、事物の
状態は同じだと述べている。

\\


{\scshape Respondeo dicendum} quod, sicut bonum habet rationem
 appetibilis, ita verum habet ordinem ad cognitionem. Unumquodque
 autem inquantum habet de esse, intantum est cognoscibile. Et propter
 hoc dicitur in III {\itshape de Anima}, quod {\itshape anima est
 quodammodo omnia} secundum sensum et intellectum. Et ideo, sicut
 bonum convertitur cum ente, ita et verum. Sed tamen, sicut bonum
 addit rationem appetibilis supra ens, ita et verum comparationem ad
 intellectum.


&

解答する。以下のように言われるべきである。善が「欲求されうるもの」とい
う性格をもつように、真は、認識への秩序をもつ。ところが、各々のものは、
それが存在をもつだけ、それだけ、認識されうるものである。このために、
『デ・アニマ』3巻で、感覚と知性の点で「魂はある意味ですべてのものであ
る」と言われている。ゆえに、善が有と置換されるように、真もまた[有と]
置換される。しかし、善が「欲求されうるもの」という性格を有に加えるよう
に、真も、知性への関係を[有に加える]。

\\

{\scshape Ad primum ergo dicendum} quod verum est in rebus et in
intellectu, ut dictum est. Verum autem quod est in rebus, convertitur
cum ente secundum substantiam. Sed verum quod est in intellectu,
convertitur cum ente, ut manifestativum cum manifestato. Hoc enim est
de ratione veri, ut dictum est. Quamvis posset dici quod etiam ens est
in rebus et in intellectu, sicut et verum; licet verum principaliter
in intellectu, ens vero principaliter in rebus. Et hoc accidit propter
hoc, quod verum et ens differunt ratione.


&

第一異論に対しては、それゆえ、次のように言われるべきである。すでに述べ
られたとおり、真は事物の中にも知性の中にもある。事物の中にある真は、有
と、実体の点で置換されるが、知性の中にある真は、有と、明示しうるものが
明示されうるものと置換されるようなかたちで置換される。すでに述べられた
とおり、このことが、「真」という性格に含まれているからである。かりに、
真と同様、有もまた、事物の中にも知性の中にもある、と言われうるとしても、
真は主要に知性の中にあり、有は主要に事物の中にある。これは、真と有が性
格の点で異なるから生じることである。


\\


{\scshape Ad secundum dicendum} quod non ens non habet in se unde
cognoscatur, sed cognoscitur inquantum intellectus facit illud
cognoscibile. Unde verum fundatur in ente, inquantum non ens est
quoddam ens rationis, apprehensum scilicet a ratione.


&

第二異論に対しては、次のように言われるべきである。非有は、自分の中に、
認識されるための根拠をもたず、知性がそれを認識可能なものにするかぎりで、
認識される。このことから、非有は何らかの理性の有、つまり、理性によって
とらえられたものであるという意味で、真は有に基づく。

\\

{\scshape Ad tertium dicendum} quod, cum dicitur quod ens non potest
apprehendi sine ratione veri, hoc potest dupliciter intelligi. Uno
modo, ita quod non apprehendatur ens, nisi ratio veri assequatur
apprehensionem entis. Et sic locutio habet veritatem. Alio modo posset
sic intelligi, quod ens non posset apprehendi, nisi apprehenderetur
ratio veri. Et hoc falsum est. Sed verum non potest apprehendi, nisi
apprehendatur ratio entis, quia ens cadit in ratione veri. Et est
simile sicut si comparemus intelligibile ad ens. Non enim potest
intelligi ens, quin ens sit intelligibile, sed tamen potest intelligi
ens, ita quod non intelligatur eius intelligibilitas. Et similiter ens
intellectum est verum, non tamen intelligendo ens, intelligitur verum.


&

第三異論に対しては、次のように言われるべきである。「有は真の性格なしに
とらえられえない」と言われるとき、これは二つの意味で理解されうる。一つ
には、「真」という性格が有の把握に伴わないかぎり、有がとらえられること
はない、という意味であり、この言い方は正しい。もう一つには、「真」とい
う性格がとらえられないかぎり、有はとらえられない、という意味で理解され
うる。そして、これは間違いである。逆に、真は「有」という性格がとらえら
れないかぎり、とらえられない。なぜなら、「有」は「真」の性格の中に入っ
ているからである。これは、私たちが、「可知的なもの」と有との関係を考え
るときに生じることに似ている。つまり、有が可知的なものでないかぎり、有
は知性認識されえないが、有の可知性が認識されていなくても、有は知性認識
されうる。同様に、知性認識された有が真であるが、「有」を認識するときに、
「真」が認識されているわけではない。

\end{longtable}
\newpage


\rhead{a.~4}
\begin{center}
 {\Large {\bf ARTICULUS QUARTUS}}\\
 {\large UTRUM BONUM SECUNDUM RATIONEM SIT PRIUS QUAM VERUM}\\
 {\footnotesize {\itshape De Verit.}, q.~21, a.~3; {\itshape Hebr.},
 c.~11, lect.~1.}\\
 {\Large 第四項\\善は、性格の点で、真より先か}
\end{center}

\begin{longtable}{p{21em}p{21em}}

{\huge A}{\scshape d quartum sic proceditur}. Videtur quod bonum
secundum rationem sit prius quam verum. Quod enim est universalius,
secundum rationem prius est, ut patet ex I {\itshape Physic}. Sed
bonum est universalius quam verum, nam verum est quoddam bonum,
scilicet intellectus. Ergo bonum prius est secundum rationem quam
verum.

&

 第四項の問題へ、議論は以下のように進められる。善は、その性格の点で、
真より先であると思われる。なぜなら、『自然学』1巻で明らかなとおり、よ
り普遍的なものは、性格の点で、より先である。ところが、善は真より普遍的
である。なぜなら、真は一種の善、すなわち、知性の善だからである。ゆえに、
善は、性格の点で、真より先である。

\\


{\scshape 2 Praeterea}, bonum est in rebus, verum autem in
compositione et divisione intellectus, ut dictum est. Sed ea quae sunt
in re, sunt priora his quae sunt in intellectu. Ergo prius est
secundum rationem bonum quam verum.

&

さらに、すでに述べられたとおり、善は事物の中に、真は、知性の複合分割の
 中にある。ところが、事物の中にあるものは、知性の中にあるものより先で
 ある。ゆえに、善は、その性格の観点から見て、真より先である。

\\


{\scshape 3 Praeterea}, veritas est quaedam species virtutis, ut patet
in IV {\itshape Ethic}. Sed virtus continetur sub bono, est enim bona
qualitas mentis, ut dicit Augustinus. Ergo bonum est prius quam verum.

&


 さらに、『ニコマコス倫理学』4巻で明らかなとおり、真理は徳の種である。
 ところで、徳は善のもとに含まれる。なぜなら、アウグスティヌスが言うよ
 うに、徳は精神の善い性質だからである。ゆえに、善は真より先である。


\\


{\scshape Sed contra}, quod est in pluribus, est prius secundum
rationem. Sed verum est in quibusdam in quibus non est bonum, scilicet
in mathematicis. Ergo verum est prius quam bonum.

&

しかし反対に、より多くのものの中にあるものは、性格として、より先である。
 ところが、真は、善よりも、より多くのものの中にある。すなわち、[真は、
 善がその中にない]数学的なものの中にもある。ゆえに、真は善より先であ
 る。

\\

{\scshape Respondeo dicendum} quod, licet bonum et verum supposito
convertantur cum ente, tamen ratione differunt. Et secundum hoc verum,
absolute loquendo, prius est quam bonum. Quod ex duobus apparet.
Primo quidem ex hoc, quod verum propinquius se habet ad ens, quod est
prius, quam bonum. Nam verum respicit ipsum esse simpliciter et
immediate, ratio autem boni consequitur esse, secundum quod est aliquo
modo perfectum; sic enim appetibile est.  ---Secundo apparet ex hoc,
quod cognitio naturaliter praecedit appetitum. Unde, cum verum
respiciat cognitionem, bonum autem appetitum, prius erit verum quam
bonum secundum rationem.

&

解答する。以下のように言われるべきである。真と善は、基体(suppositum)と
してはともに有と置換されるが、性格としては異なる。そしてこの点で、無条
件的に言えば、真は善より先である。このことは、二つの点から明らかである。
一つには、真は善よりも、より先のものである有に近いことからである。つま
り、真は、存在そのものに端的に直接的に関係するが、善の性格は、存在が何
らかのかたちで完成され、欲求されうるものであるという点で、存在に伴うか
らである。--- 第二には、認識が自然本性的に欲求に先行することからである。
つまり、真は認識に、善は欲求に関係するので、そのことから、その性格の点
で、真は善より先である。

\\

{\scshape Ad primum ergo dicendum} quod voluntas et intellectus mutuo
se includunt, nam intellectus intelligit voluntatem, et voluntas vult
intellectum intelligere. Sic ergo inter illa quae ordinantur ad
obiectum voluntatis, continentur etiam ea quae sunt intellectus; et e
converso. Unde in ordine appetibilium, bonum se habet ut universale,
et verum ut particulare, in ordine autem intelligibilium est e
converso. Ex hoc ergo quod verum est quoddam bonum, sequitur quod
bonum sit prius in ordine appetibilium, non autem quod sit prius
simpliciter.

&

第一異論に対しては、それゆえ、次のように言わなければならない。意志と知
性は相互に含み合う。つまり、知性は意志を認識し、意志は知性が認識するこ
とを欲する。それゆえ、このように、意志の対象に秩序づけられているものど
もの中に、知性に属するものが含まれ、またその逆でもある。したがって、欲
求されうるものどもの秩序の中では、善が普遍的で、真が個別的となるが、可
知的なものどもの秩序の中では、それが逆になる。ゆえに、真が一種の善であ
ることからは、欲求されうるものどもの秩序の中で、善が先であることが帰結
するのであって、端的に先であることが帰結するわけではない。

\\

{\scshape Ad secundum dicendum} quod secundum hoc est aliquid prius
ratione, quod prius cadit in intellectu. Intellectus autem per prius
apprehendit ipsum ens; et secundario apprehendit se intelligere ens;
et tertio apprehendit se appetere ens. Unde primo est ratio entis,
secundo ratio veri, tertio ratio boni, licet bonum sit in rebus.

&

第二異論に対しては、次のように言われるべきである。何かが性格として先で
あるということは、先に知性の中に入ってくる[=知性がそれを認識する]こ
とから言われる。ところで、知性は、より先に有そのものをとらえ、第二に、
自分が有を認識していることをとらえ、第三に、自分が有を欲求していること
をとらえる。したがって、善は事物の中にあるが、一番目が有の性格、二番目
が真の性格、三番目が善の性格となる。

\\


{\scshape Ad tertium dicendum} quod virtus quae dicitur veritas, non
est veritas communis, sed quaedam veritas secundum quam homo in dictis
et factis ostendit se ut est. Veritas autem vitae dicitur
particulariter, secundum quod homo in vita sua implet illud ad quod
ordinatur per intellectum divinum, sicut etiam dictum est veritatem
esse in ceteris rebus. Veritas autem {\itshape iustitiae} est secundum
quod homo servat id quod debet alteri secundum ordinem legum. Unde ex
his particularibus veritatibus non est procedendum ad veritatem
communem.

&

第三異論に対しては、次のように言われるべきである。「真理」と言われる徳
は、一般的に言われる真理ではなく、人間が発言と行為においてあるがままに
自分を示すという意味での、特殊な真理である。また、「人生の真理」という
ことは、人が、神の知性によって秩序づけられたものを自分の人生の中で満た
すという意味で、ちょうど、ある事物の中に真理があると言われるようなしか
たで、個別的に言われる。またさらに、「正義の真理」は、法の秩序にしたがっ
て、人が他者に対して為すべきことを為すという意味で言われる。したがって、
これら個別的な真理から、一般的な意味での真理へと論を進めるべきではない。
\end{longtable}
\newpage


\rhead{a.~5}
\begin{center}
 {\Large {\bf ARTICULUS QUINTUS}}\\
 {\large UTRUM DEUS SIT VERITAS}\\
 {\footnotesize I$^a$ II$^{ae}$, q.~3, a.~7; I {\itshape Sent.}, d.~19,
 q.~5, a.~1; I {\itshape SCG.}, c.~54 sqq.; III, c.~11.}\\
 {\Large 第五項\\神は真理か}
\end{center}

\begin{longtable}{p{21em}p{21em}}

{\huge A}{\scshape d quintum sic proceditur}. Videtur quod Deus non
 sit veritas. Veritas enim consistit in compositione et divisione
 intellectus. Sed in Deo non est compositio et divisio. Ergo non est
 ibi veritas.


&

第五項の問題へ、議論は以下のように進められる。神は真理でないと思われる。
なぜなら、真理は、知性の複合分割の中に成立する。ところが、神の中に複合
分割はない。ゆえに、そこに真理はない。

\\

{\scshape 2 Praeterea}, veritas, secundum Augustinum, in libro
 {\itshape de Vera Relig}., {\itshape est similitudo principii}. Sed
 Dei non est similitudo ad principium. Ergo in Deo non est veritas.


&

さらに、真理は、『真の宗教について』という書物のアウグスティヌスによれ
 ば、「根源への類似」である。ところが、神に、根源への類似はない。ゆえ
 に、神の中に真理はない。

\\

{\scshape 3 Praeterea}, quidquid dicitur de Deo, dicitur de eo ut de
 prima causa omnium, sicut esse Dei est causa omnis esse, et bonitas
 eius est causa omnis boni. Si ergo in Deo sit veritas, ergo omne
 verum erit ab ipso. Sed aliquem peccare est verum. Ergo hoc erit a
 Deo. Quod patet esse falsum.

&

さらに、神について語られることがらはなんであれ、「神の存在はすべての存
 在の原因である」「神の善性はすべての善の原因である」というように、万
 物の第一原因について語られるようにして、神について語られる。ゆえに、
 もし神の中に真理があれば、すべての真は神によって真であることになるだ
 ろう。ところで、「ある人が罪を犯す」ということは真である。ゆえに、こ
 のことが神によってあることになるだろうが、これは明らかに偽である。

\\


{\scshape Sed contra est} quod dicit dominus, {\itshape
Ioan}.~{\scshape xiv}, {\itshape ego sum via, veritas et vita}.


&

しかし反対に、『ヨハネによる福音書』14章で、主が「私は、生命、真理、道
 である」と言っている。\footnote{「イエスは言われた。「わたしは道であ
 り、真理であり、命である。わたしを通らなければ、だれも父のもとに行く
 ことができない。」(14:6)}


\\

{\scshape Respondeo dicendum} quod, sicut dictum est, veritas
 invenitur in intellectu secundum quod apprehendit rem ut est, et in
 re secundum quod habet esse conformabile intellectui. Hoc autem
 maxime invenitur in Deo. Nam esse suum non solum est conforme suo
 intellectui, sed etiam est ipsum suum intelligere; et suum
 intelligere est mensura et causa omnis alterius esse, et omnis
 alterius intellectus; et ipse est suum esse et intelligere. Unde
 sequitur quod non solum in ipso sit veritas, sed quod ipse sit ipsa
 summa et prima veritas.

&

解答する。以下のように言われるべきである。すでに述べられたように、真理
は、それが事物をあるがままに捉えるという点で、知性の中に見出され、知性
に合致しうる存在を持つという点で、事物の中に見出される。しかしこのこと
は、最大限に神の中に見出される。すなわち、その存在は、その知性に合致し
ているだけでなく、その知性認識の活動それ自体であるし、その知性認識の活
動は、他のすべての存在と、他のすべての知性の、尺度であり原因である。ま
た、神は、自らの存在でありかつ知性認識である。したがって、神の中に真理
が見出されるということだけでなく、神は、最高の、第一の真理それ自体だと
いうことが帰結する。

\\

{\scshape Ad primum ergo dicendum} quod, licet in intellectu divino
 non sit compositio et divisio, tamen secundum suam simplicem
 intelligentiam iudicat de omnibus, et cognoscit omnia complexa. Et
 sic in intellectu eius est veritas.

&

第一異論に対しては、それゆえ、次のように言われるべきである。神の知性の
中には複合分割がないけれども、その単純な知性認識によって、すべてについ
て判断し、すべての複合されたものを認識する。このようなかたちで、神の知
性の中に真理がある。

\\

{\scshape Ad secundum dicendum} quod verum intellectus nostri est
 secundum quod conformatur suo principio, scilicet rebus, a quibus
 cognitionem accipit. Veritas etiam rerum est secundum quod
 conformantur suo principio, scilicet intellectui divino. Sed hoc,
 proprie loquendo, non potest dici in veritate divina, nisi forte
 secundum quod veritas appropriatur filio, qui habet principium. Sed
 si de veritate essentialiter dicta loquamur, non potest intelligi,
 nisi resolvatur affirmativa in negativam, sicut cum dicitur,
 {\itshape pater est a se, quia non est ab alio}. Et similiter dici
 potest {\itshape similitudo principii} veritas divina, inquantum esse
 suum non est suo intellectui dissimile.

&

第二異論に対しては、次のように言われるべきである。私たちの知性の真は、
自らの根源、すなわち事物へ一致させられることによる。私たちの知性は事物
から認識を受け取るからである。また、諸事物の真理は、自らの根源、すなわ
ち神の知性に一致させられることによる。しかしこのことは、厳密に言えば、
神の真理においては語られえない。真理が、根源を持つ子に固有化されるかぎ
りで、というのでないかぎり。そうではなく、私たちが本質的に語られた真理
について述べるならば、「父が自らによって在るのは、他によって在るのでは
ないからである」と言われる場合のように、肯定的な真理が否定的な真理に解
体されることによらない限り、それ[=真理とは根源への類似であるというこ
と]が理解されることはできない。同様に、「根源への類似」ということも、
その存在が、自らの知性に類似していないことがない、という意味で、神の真
理について言われうる。

\\

{\scshape Ad tertium dicendum} quod non ens et privationes non habent
ex seipsis veritatem, sed solum ex apprehensione intellectus. Omnis
autem apprehensio intellectus a Deo est, unde quidquid est veritatis
in hoc quod dico, {\itshape istum fornicari est verum}, totum est a
Deo. Sed si arguatur, {\itshape ergo istum fornicari est a Deo}, est
fallacia accidentis.


&

第三異論に対しては、次のように言われるべきである。非有と欠如は、自らに
基づいて真理をもつことがなく、知性の把握にもとづいてのみ真理をもつ。と
ころで、知性の把握はすべて神によってある。したがって、私が「その人が姦
淫することは真である」と言うことの中で、真理に属することは何であれ、す
べて神による。しかし、「それゆえ、その人が姦淫することは神による」と論
じるならば、それは附帯性の誤謬である。

\end{longtable}
\newpage


\rhead{a.~6}
\begin{center}
 {\Large {\bf ARTICULUS SEXTUS}}\\
 {\large UTRUM SIT UNA SOLA VERITAS, SECUNDUM QUAM OMNIA SUNT VERA}\\
 {\footnotesize I {\itshape Sent.}, d.~19, q.~5, a.~2; III {\itshape
 SCG.}, c.~48; {\itshape De Verit.}, q.~1, a.~4; q.~21, a.~4, ad 5;
 q.~27, a.~1, ad 7; {\itshape Quodl}.~X, q.~4, a.~1.}\\
 {\Large 第六項\\それによってすべてが真であるような、ただ一つの真理が存
 在するか}
\end{center}

\begin{longtable}{p{21em}p{21em}}

{\huge A}{\scshape d sextum sic proceditur}. Videtur quod una sola sit
veritas, secundum quam omnia sunt vera. Quia, secundum Augustinum,
nihil est maius mente humana, nisi Deus. Sed veritas est maior mente
humana, alioquin mens iudicaret de veritate; nunc autem omnia iudicat
secundum veritatem, et non secundum seipsam. Ergo solus Deus est
veritas. Ergo non est alia veritas quam Deus.

&

第六項の問題へ、議論は以下のように進められる。すべてがそれによって真で
あるような、ただ一つの真理が存在すると思われる。理由は以下の通り。アウ
グスティヌスによれば、神以外に、人間精神より大きいものはない。しかし、
真理は人間の精神よりも大きい。さもなければ、精神は真理について判断した
であろうが、じっさいには精神はすべてについて真理によって判断するのであ
り、自分自身によって判断するのではない。ゆえに、ただ神だけが真理である。
ゆえに、神以外の真理は存在しない。

\\

2 {\scshape Praeterea}, Anselmus dicit, in libro {\itshape de
Veritate}, quod sicut tempus se habet ad temporalia, ita veritas ad
res veras. Sed unum est tempus omnium temporalium. Ergo una est
veritas, qua omnia vera sunt.

&

さらに、アンセルムスは『真理論』という書物の中で、時間が時間的なものに
関係するように、真理は真の事物に関係する、と述べている。しかし、すべて
の時間的なものが関係する時間は一つである。ゆえに、万物がそれによって真
であるところの真理は一つである。

\\

{\scshape Sed contra est} quod in Psalmo {\scshape xi} dicitur,
{\itshape diminutae sunt veritates a filiis hominum}.


&

しかし反対に、『詩編』11で「人間の息子たちによって、諸真理が減じられた」
\footnote{「人の子らの中から信仰のある人は消え去りました」(12:2)}と言
われている。

\\

{\scshape Respondeo dicendum} quod quodammodo una est veritas, qua
omnia sunt vera, et quodammodo non. Ad cuius evidentiam, sciendum est
quod, quando aliquid praedicatur univoce de multis, illud in quolibet
eorum secundum propriam rationem invenitur, sicut {\itshape animal} in
qualibet specie animalis.

&

解答する。以下のように言われるべきである。ある意味で、それによって万物
が真である真理は一つだが、ある意味ではそうでない。これを明らかにするた
めには、以下のことが知られるべきである。或るもの(X)が多くのもの(Y)に一
義的に述語づけられるとき、たとえば、「動物」が、動物のどんな種の中にも
見出されるように、その或るもの(X)が、多くのもの(Y)のどの中にも、固有の
性格において見出される。

\\

Sed quando aliquid dicitur analogice de multis, illud invenitur
secundum propriam rationem in uno eorum tantum, a quo alia
denominantur. Sicut {\itshape sanum} dicitur de animali et urina et
medicina, non quod sanitas sit nisi in animali tantum, sed a sanitate
animalis denominatur medicina sana, inquantum est illius sanitatis
effectiva, et urina, inquantum est illius sanitatis significativa. Et
quamvis sanitas non sit in medicina neque in urina, tamen in utroque
est aliquid per quod hoc quidem facit, illud autem significat
sanitatem.

 &

しかし、或るもの(X)が多くのもの(Y)に、アナロギア的に語られるとき、それ
(X)は、それら(Y)のうち、ただ一つのものの中に固有の性格において見出され、
それによって他のものが[Xと]名付けられる。たとえば、「健康」は動物と
尿と薬について語られるが、それは、健康が、ただ動物の中だけにある
\footnote{nisiをnon nisiと読む。}、と語られるのではなく、動物の健康を
作り出しうるという点で、動物の健康によって、薬が健康と名付けられ、また、
動物の健康を表示しうるという点で、尿が健康と名付けられる。そして、薬や
尿の中に健康はないが、しかし、この両者に何かが存在し、前者ではそれによっ
て[動物の]健康を作り、後者ではそれを表示する。

\\

Dictum est autem quod veritas per prius est in intellectu, et per
posterius in rebus, secundum quod ordinantur ad intellectum
divinum. Si ergo loquamur de veritate prout existit in intellectu,
secundum propriam rationem, sic in multis intellectibus creatis sunt
multae veritates; etiam in uno et eodem intellectu, secundum plura
cognita.

&

さて、真理は、先に知性の中に在り、事物の中には、それが神の知性に秩序づ
けられる点で、後に在るということが述べられた。ゆえに、私たちが、[真理
の]固有の性格にしたがって、知性の中に存在するものとしての真理について
語るならば、その場合には、被造の多くの知性の中に多くの真理が存在する。
そして一つの同一の知性の中にも、複数のものが認識されるのに応じて[多く
の真理が存在する]。

\\

Unde dicit Glossa super illud Psalmi {\scshape xi}, {\itshape
diminutae sunt veritates a filiis hominum} etc., quod sicut ab una
facie hominis resultant plures similitudines in speculo, sic ab una
veritate divina resultant plures veritates.

&

このことから、かの『詩編』11「人間の息子たちによって、諸真理が減じられ
 た」の『解説』は、ちょうど、人間の一つの顔から、鏡の中に複数の類似が
 結果として生じるように、神の一つの真理から、複数の真理が結果として生
 じる、と語る。

\\

Si vero loquamur de veritate secundum quod est in rebus, sic omnes
sunt verae una prima veritate, cui unumquodque assimilatur secundum
suam entitatem. Et sic, licet plures sint essentiae vel formae rerum,
tamen una est veritas divini intellectus, secundum quam omnes res
denominantur verae.

&

他方、私たちが諸事物の中にあるかぎりでの真理について語るならば、すべて
 の事物は一つの第一真理によって真であり、各々のものは、自らの存在性に
 したがって、それに類似化されている。この意味で、事物の本質や形相は複
 数あるが、しかし、すべての事物がそれによって真と名付けられるところの
 神の知性の真理は一つである。

\\

{\scshape Ad primum ergo dicendum} quod anima non secundum quamcumque
veritatem iudicat de rebus omnibus; sed secundum veritatem primam,
inquantum resultat in ea sicut in speculo, secundum prima
intelligibilia. Unde sequitur quod veritas prima sit maior anima. Et
tamen etiam veritas creata, quae est in intellectu nostro, est maior
anima, non simpliciter, sed secundum quid, inquantum est perfectio
eius; sicut etiam scientia posset dici maior anima. Sed verum est quod
nihil subsistens est maius mente rationali, nisi Deus.

&

第一異論に対しては、それゆえ、以下のように言われるべきである。魂がすべ
ての事物について判断するのは、どんな真理にしたがってでもというわけでは
なく、第一真理にしたがってである。それは、[第一真理は]第一の可知的な
ものである点で、あたかも鏡の中のように、魂の中に[それら諸真理が]結果
として生じる限りにおいてである。したがって、第一真理は魂より大きいこと
が帰結する。しかし、私たちの知性の中にある被造の真理もまた、魂よりも大
きいが、それは、端的にではなく、ある意味において、[被造の真理が]知性
の完成である限りにおいてである。この意味で、知もまた、魂よりも大きいと
言われうるであろう。しかし、自存するもので理性的な精神よりも大きいのは
神以外ない、ということは、真である。

\\

{\scshape Ad secundum dicendum} quod dictum Anselmi veritatem habet,
secundum quod res dicuntur verae per comparationem ad intellectum
divinum.


&

第二異論に対しては、以下のように言われるべきである。アンセルムスが言う
ことは、事物が神の知性との関係によって真と言われるかぎりでは、真理性を
もっている。

\\


\end{longtable}
\newpage

\rhead{a.~7}
\begin{center}
 {\Large {\bf ARTICULUS SEPTIMUS}}\\
 {\large UTRUM VERITAS CREATA SIT AETERNA}\\
 {\footnotesize Supra, q.~10, a.~3, ad 3; I {\itshape Sent.}, d.~19,
 q.~5, a.~3; II {\itshape SCG.}, c.~35; III, c.~82, 84; {\itshape De
 Verit.}, q.~1, a.~5; {\itshape De Pot.}, q.~3, a.~17, ad 27.}\\
 {\Large 第七項\\被造の真理は永遠か}
\end{center}

\begin{longtable}{p{21em}p{21em}}

{\huge A}{\scshape d septimum sic proceditur}. Videtur quod veritas
creata sit aeterna. Dicit enim Augustinus, in libro {\itshape De
Libero Arbitrio}, quod nihil est magis aeternum quam ratio circuli, et
duo et tria esse quinque. Sed horum veritas est veritas creata. Ergo
veritas creata est aeterna.


&

第七項の問題へ、議論は以下のように進められる。被造の真理は永遠だと思わ
れる。理由は以下の通り。アウグスティヌスは、『自由意志』という書物の中
で、円の概念や、二足す三は五であること以上に永遠なものはない、と述べて
いる。ところが、これらのことの真理は、被造の真理である。ゆえに、被造の
真理は永遠である。

\\



{\scshape 2 Praeterea}, omne quod est semper, est aeternum. Sed
 universalia sunt ubique et semper. Ergo sunt aeterna. Ergo et verum,
 quod est maxime universale.


&

 さらに、すべて、常に在るものは永遠である。ところが、普遍的なものは、
いたるところに、そして、常に在る。ゆえに、それは永遠である。ゆえに、最
大限に普遍的である「真」もまた、永遠である。


\\

{\scshape 3 Praeterea}, id quod est verum in praesenti, semper fuit
verum esse futurum. Sed sicut veritas propositionis de praesenti est
veritas creata, ita veritas propositionis de futuro. Ergo aliqua
veritas creata est aeterna.


&

さらに、現在、真であるものは、「それが未来にあるだろう」ということが常
に真であった。ところが、現在についての命題の真理が被造の真理であるよう
に、未来についての命題の真理もまた被造の真理である。ゆえに、なんらかの
被造の真理が永遠である。

\\

{\scshape 4 Praeterea}, omne quod caret principio et fine, est
aeternum. Sed veritas enuntiabilium caret principio et fine. Quia, si
veritas incoepit cum ante non esset, verum erat veritatem non esse, et
utique aliqua veritate verum erat, et sic veritas erat antequam
inciperet. Et similiter si ponatur veritatem habere finem, sequitur
quod sit postquam desierit, verum enim erit veritatem non esse. Ergo
veritas est aeterna.


&

さらに、すべて、始まりと終わりを欠くものは永遠である。ところが、命題さ
れうるものの真理は、始まりと終わりを欠く。というのも、もし、真理が存在
し始めたとすると、その前に真理は存在しなかったのだから、「真理が存在し
ない」ということが真であった。そして、このことは、なんらかの真理によっ
て真であったから、真理が存在し始める前に真理があったことになる。同様に、
もし真理が終わりをもつとされるならば、その帰結として、真理がなくなった
あと、「真理が存在しない」ことが真であるだろう。ゆえに、真理は永遠であ
る。

\\

{\scshape Sed contra est} quod solus Deus est aeternus, ut supra
 habitum est.

&

しかし反対に、前に論じられたとおり、ただ神のみが永遠である。

\\

{\scshape Respondeo dicendum} quod veritas enuntiabilium non est aliud
quam veritas intellectus. Enuntiabile enim et est in intellectu, et
est in voce. Secundum autem quod est in intellectu, habet per se
veritatem. Sed secundum quod est in voce, dicitur verum enuntiabile,
secundum quod significat aliquam veritatem intellectus; non propter
aliquam veritatem in enuntiabili existentem sicut in subiecto.


&

解答する。以下のように言われるべきである。命題の真理は、知性の真理に他
ならない。じっさい、命題は、知性の中にも音声の中にもある。知性の中にあ
るかぎりで、それは自体的に真理をもつが、音声の中にあるかぎりでは、それ
は、なんらかの知性の真理を表示するという点で、真の命題と言われる。つま
り、命題が真と言われるのは、基体としての命題の中に存在する、なんらかの
真理のためにではない。

\\

Sicut urina dicitur sana, non a sanitate quae in ipsa sit, sed a
sanitate animalis, quam significat. Similiter etiam supra dictum est
quod res denominantur verae a veritate intellectus. Unde si nullus
intellectus esset aeternus, nulla veritas esset aeterna. Sed quia
solus intellectus divinus est aeternus, in ipso solo veritas
aeternitatem habet. Nec propter hoc sequitur quod aliquid aliud sit
aeternum quam Deus, quia veritas intellectus divini est ipse Deus, ut
supra ostensum est.

&

たとえば、尿が健康と言われるのは、尿の中にある健康によってではなく、尿
がそれを表示する動物の健康によってである。同様に、事物もまた、知性の真
理によって真と名付けられることが、先に述べられた。したがって、もし、ど
んな知性も永遠でなかったならば、どんな真理もまた永遠でなかったであろう。
ところが、ただ神の知性だけが永遠だから、その中でのみ、真理は永遠性を有
する。だからといって、神以外の何かが永遠だということにはならない。なぜ
なら、神の知性の真理は、前に示されたとおり、神自身だからである。

\\


{\scshape Ad primum ergo dicendum} quod ratio circuli, et duo et tria
esse quinque, habent aeternitatem in mente divina.

&

第一異論に対しては、それゆえ、以下のように言われるべきである。円の概念
や、二足す三は五であるということは、神の精神の中で、永遠性を有する。


\\

{\scshape Ad secundum dicendum} quod aliquid esse semper et ubique,
potest intelligi dupliciter. Uno modo, quia habet in se unde se
extendat ad omne tempus et ad omnem locum, sicut Deo competit esse
ubique et semper. Alio modo, quia non habet in se quo determinetur ad
aliquem locum vel tempus, sicut materia prima dicitur esse una, non
quia habet unam formam, sicut homo est unus ab unitate unius formae,
sed per remotionem omnium formarum distinguentium. Et per hunc modum,
quodlibet universale dicitur esse ubique et semper, inquantum
universalia abstrahunt ab hic et nunc. Sed ex hoc non sequitur ea esse
aeterna, nisi in intellectu, si quis sit aeternus.

&

第二異論に対しては、以下のように言われるべきである。あるものが常にいた
るところに存在するということは、二通りに理解されうる。一つには、自らの
中に、すべての時間とすべての場所に及ぶ根拠をもつという理由によってであ
り、ちょうどそれは、神に、いたるところに常に存在するということが属する
ようにである。もう一つには、自らの中に、なんらかの場所や時間に限定され
る根拠をもたないという理由によってであり、ちょうど、第一質料が一と言わ
れるのが、人間が、一つの形相の一性によって一と言われるようなしかたで、
一つの形相をもつからではなく、区別するあらゆる形相をもたないことによっ
てであるように。この後者のかたちによって、どんな普遍も、「ここ」や「今」
から抽象されているという点で、いたるところに常に存在すると言われる。し
かし、このことからは、それらの普遍が永遠であることは帰結せず、帰結する
のは、もしなんらかの知性が永遠であるならば、その知性の中でのみ永遠だと
いうことだけである。


\\

{\scshape Ad tertium dicendum} quod illud quod nunc est, ex eo futurum
fuit antequam esset, quia in causa sua erat ut fieret. Unde, sublata
causa, non esset futurum illud fieri. Sola autem causa prima est
aeterna. Unde ex hoc non sequitur quod ea quae sunt, semper fuerit
verum ea esse futura, nisi quatenus in causa sempiterna fuit ut essent
futura. Quae quidem causa solus Deus est.

&

第三異論に対しては、以下のように言われるべきである。今存在するものは、
存在するための自らの原因の中に存在していたから、存在する以前に、「将来
存在するだろう」ということ[が、真で]あった。ところが、第一原因だけが
永遠である。ゆえに、それらが現在存在するということから、それらが将来存
在するだろうということが常に真であったということは、将来存在するための
永続的な原因の中に存在していたという意味でないかぎり、帰結しない。そし
て、そのような原因とは、ただ神だけである。

\\



{\scshape Ad quartum dicendum} quod, quia intellectus noster non est
aeternus, nec veritas enuntiabilium quae a nobis formantur, est
aeterna, sed quandoque incoepit. Et antequam huiusmodi veritas esset,
non erat verum dicere veritatem talem non esse, nisi ab intellectu
divino, in quo solum veritas est aeterna.

&

第四異論に対しては、以下のように言われるべきである。私たちの知性は永遠
でないので、私たちによって形成される命題の真理も永遠でなく、ある時点で
始まった。そして、そのような真理が存在する以前に、そのような真理が存在
しないと語ることは、神の知性によるのでない限り、真でない。ただ神の知性
の中でのみ、真理は永遠だからである。

\\

Sed nunc verum est dicere veritatem tunc non fuisse. Quod quidem non
est verum nisi veritate quae nunc est in intellectu nostro, non autem
per aliquam veritatem ex parte rei. Quia ista est veritas de non ente;
non ens autem non habet ex se ut sit verum, sed solummodo ex
intellectu apprehendente ipsum. Unde intantum est verum dicere
veritatem non fuisse, inquantum apprehendimus non esse ipsius ut
praecedens esse eius.

&

しかし、そのとき真理が存在しなかったと、今、語ることは真である。ただし、
これが真であるのは、今、私たちの知性の中にある真理によってであり、事物
の側にある何らかの真理によるのではない。なぜなら、それは、非有について
の真理であって、非有は自らに基づいて真であるようなものはもたず、ただ、
それを捉える知性によってのみ真だからである。したがって、私たちが、その
ものの存在に先立つものとして、それの非有を捉えるかぎりで、真理が存在し
なかったと語ることは真である。
\end{longtable}
\newpage

\rhead{a.~8}
\begin{center}
 {\Large {\bf ARTICULUS OCTAVUS}}\\
 {\large UTRUM VERITAS SIT IMMUTABILIS}\\
 {\footnotesize I {\itshape Sent.}, d.~19, q.~5, a.~3; {\itshape De
 Verit.}, q.~1, a.~6.}\\
 {\Large 第八項\\真理は不変か}
\end{center}

\begin{longtable}{p{21em}p{21em}}

{\huge A}{\scshape d octavum sic proceditur}. Videtur quod veritas sit
immutabilis. Dicit enim Augustinus, in libro II {\itshape de Libero
Arbitrio}, quod veritas non est aequalis menti, quia esset mutabilis,
sicut et mens.

&

第八項の問題へ、議論は以下のように進められる。真理は不変であると思われ
る。理由は以下の通り。アウグスティヌスは『自由意志』2巻で次のように述
べている。真理は、精神と等しいものではない。なぜなら、[もし等しいもの
であったならば、真理は]精神がそうであるように、可変的であっただろうか
ら。


\\


{\scshape 2 Praeterea}, id quod remanet post omnem mutationem, est
immutabile, sicut prima materia est ingenita et incorruptibilis, quia
remanet post omnem generationem et corruptionem. Sed veritas remanet
post omnem mutationem, quia post omnem mutationem verum est dicere
esse vel non esse. Ergo veritas est immutabilis.

&

さらに、ちょうど、第一質料が、あらゆる生成と消滅のあとにも残るので、不
生不滅であるように、あらゆる変化のあとに残るものは不変である。ところが、
真理はあらゆる変化のあとに残る。なぜなら、あらゆる変化のあと、「ある」
とか「ない」と言うことは真だからである。ゆえに、真理は不変である。

\\


{\scshape 3 Praeterea}, si veritas enuntiationis mutatur, maxime
mutatur ad mutationem rei. Sed sic non mutatur. Veritas enim, secundum
Anselmum, est rectitudo quaedam, inquantum aliquid implet id quod est
de ipso in mente divina. Haec autem propositio, {\itshape Socrates
sedet}, accipit a mente divina ut significet Socratem sedere, quod
significat etiam eo non sedente. Ergo veritas propositionis nullo modo
mutatur.

&

さらに、もし命題の真理が変化するならば、最大限に、事物の変化に即して変
 化する。ところが、そのようには変化しない。理由は以下の通り。アンセル
 ムスによれば、真理とは、神の精神の中でそれにかかわることを満たすなに
 かという点での、一種の正直である。ところが、「ソクラテスは座っている」
 というこの命題は、ソクラテスが座っていることを表示することを、神の精
 神から受け取っていて、ソクラテスが座ってないときでも、このことを表示
 している。ゆえに、命題の真理はけっして変化しない

\\

{\scshape 4 Praeterea}, ubi est eadem causa, et idem effectus. Sed
eadem res est causa veritatis harum trium propositionum {\itshape
Socrates sedet, sedebit, et sedit}. Ergo eadem est harum veritas. Sed
oportet quod alterum horum sit verum. Ergo veritas harum propositionum
immutabiliter manet. Et eadem ratione cuiuslibet alterius
propositionis.

&

さらに、同じ原因があるところには、同じ結果がある。ところが、「ソクラテ
スは座っている」「ソクラテスは座るだろう」「ソクラテスは座った」という
三つの命題の真理の原因は同じ事態である。ゆえに、これらの命題の真理は同
じである。ところが、これらの[命題の]うち、一つは真でなければならない。
ゆえに、これらの命題の真理は、変化せずに留まる。同じ理由で、これらの他
の命題の真理についても同じことが言える。

\\

{\scshape Sed contra est} quod dicitur in {\itshape Psalmo} {\scshape
xi}, diminutae sunt veritates a filiis hominum.

&

 しかし反対に、『詩編』11「諸々の真理が、人間たちの子らから減じられた」
 と言われている。


\\

{\scshape Respondeo dicendum} quod, sicut supra dictum est, veritas
proprie est in solo intellectu, res autem dicuntur verae a veritate
quae est in aliquo intellectu. Unde mutabilitas veritatis consideranda
est circa intellectum. Cuius quidem veritas in hoc consistit, quod
habeat conformitatem ad res intellectas. Quae quidem conformitas
variari potest dupliciter, sicut et quaelibet alia similitudo, ex
mutatione alterius extremi. Unde uno modo variatur veritas ex parte
intellectus, ex eo quod de re eodem modo se habente aliquis aliam
opinionem accipit, alio modo si, opinione eadem manente, res
mutetur. Et utroque modo fit mutatio de vero in falsum.

&

解答する。以下のように言われるべきである。前に述べられたとおり、真理は、
固有の意味では、ただ知性の中だけにあり、事物は、何らかの知性の中にある
真理によって真と言われる。したがって、真理の可変性は、知性をめぐって考
察されるべきである。実に知性の真理は、知性認識された事物への一致を有す
ることにおいて成立する。そして、この一致は、他のどんな類似でもそうであ
るように、二通りの仕方で変わりうる。すなわち、その両端の変化によってで
ある。したがって、一つには、事物は同一に留まっているのに、だれかが別の
考えを抱く場合のように、真理は知性の側から変化する。また、もう一つには、
考えが同一に留まっているのに、事物が変化する場合である。この両方の仕方
で、真から偽への変化が生じる。

\\

Si ergo sit aliquis intellectus in quo non possit esse alternatio
opinionum, vel cuius acceptionem non potest subterfugere res aliqua,
in eo est immutabilis veritas. Talis autem est intellectus divinus, ut
ex superioribus patet. Unde veritas divini intellectus est
immutabilis. Veritas autem intellectus nostri mutabilis est. Non quod
ipsa sit subiectum mutationis, sed inquantum intellectus noster
mutatur de veritate in falsitatem; sic enim formae mutabiles dici
possunt. Veritas autem intellectus divini est secundum quam res
naturales dicuntur verae, quae est omnino immutabilis.

&

ゆえに、もし、思いの変化がありえないような知性が存在するならば、あるい
は、どんな事物もその理解から逃れることができないような知性が存在するな
らば、そこに不変の真理がある。ところが、これまでに述べたことから明らか
なとおり、神の知性はそのようなものである。したがって、神の知性の真理は
不変である。他方、私たちの知性の真理は可変である。ただし、真理が変化の
基体だというのではなく、私たちの知性が、真理から虚偽へと変化する限りに
おいてである。じっさい、このような意味で、形相が可変的だと言われうる。
これに対して、神の知性の真理は、それにしたがって、自然的諸事物が真と言
われるものであって、あらゆる点で不変である。

\\

{\scshape Ad primum ergo dicendum} quod
Augustinus loquitur de veritate divina.

&

第一異論に対しては、それゆえ、以下のように言われるべきである。アウグス
ティヌスは、神の真理について語っている。

\\


{\scshape Ad secundum dicendum} quod verum et ens sunt
convertibilia. Unde, sicut ens non generatur neque corrumpitur per se,
sed per accidens, inquantum hoc vel illud ens corrumpitur vel
generatur, ut dicitur in I {\itshape Physic}.; ita veritas mutatur,
non quod nulla veritas remaneat, sed quia non remanet illa veritas
quae prius erat.

& 

第二異論に対しては、以下のように言われるべきである。真と有は置換されう
る。したがって、『自然学』1巻で述べられるとおり、有は自体的に生成消滅
せず、この有やあの有が、消滅したり生成したりするという点で、附帯的に生
成消滅するように、真理も、どんな真理も残らないというかたちで変化するの
ではなく、以前にあったあの真理が留まらない、というかたちで変化する。


\\


{\scshape Ad tertium dicendum} quod propositio non solum habet
veritatem sicut res aliae veritatem habere dicuntur, inquantum implent
id quod de eis est ordinatum ab intellectu divino; sed dicitur habere
veritatem quodam speciali modo, inquantum significat veritatem
intellectus. Quae quidem consistit in conformitate intellectus et
rei. Qua quidem subtracta, mutatur veritas opinionis, et per
consequens veritas propositionis. Sic igitur haec propositio, Socrates
sedet, eo sedente vera est et veritate rei, inquantum est quaedam vox
significativa; et veritate significationis, inquantum significat
opinionem veram. Socrate vero surgente, remanet prima veritas, sed
mutatur secunda.

&

第三異論に対しては、以下のように言われるべきである。命題は、他の事物が
真理をもつと言われると同じように、それらについて神の知性によって秩序づ
けられているものを満たすかぎりで真理をもつだけでなく、一種独特の意味で、
知性の真理を表示するかぎりで、真理をもつと言われる。この知性の真理は、
知性と事物の一致において成立する。この一致がなくなると、思いの真理が変
化し、その結果、命題の真理が変化する。ゆえに、このようにして、「ソクラ
テスが座っている」というこの命題は、彼が座っているとき、ある種の表示的
音声であるかぎりで、事物の真理によって真であり、また、真である思いを表
示するかぎりで、表示の真理によって真である。しかし、ソクラテスが立ち上
がると、第一の真理は留まるが、第二の真理は変化する。

\\


{\scshape Ad quartum dicendum} quod sessio Socratis, quae est causa
veritatis huius propositionis, Socrates sedet, non eodem modo se habet
dum Socrates sedet, et postquam sederit, et antequam sederet. Unde et
veritas ab hoc causata, diversimode se habet; et diversimode
significatur propositionibus de praesenti, praeterito et futuro. Unde
non sequitur quod, licet altera trium propositionum sit vera, quod
eadem veritas invariabilis maneat.

&

第四異論に対しては、以下のように言われるべきである。ソクラテスが座って
いることは、「ソクラテスが座っている」というこの命題の真理の原因である
が、ソクラテスが座っているあいだと、座っていて立ち上がったあとと、座る
前とでは、同じかたちで関係するわけではない。したがって、これが原因となっ
て生じる真理もまた、異なる関係にあるのであって、現在、過去、未来という、
異なるかたちで表示される。したがって、たしかに三つの命題のどれか一つは
真であるけれども、このことから、同一の不変の真理が留まるということが帰
結するわけではない。
\end{longtable}
\end{document}