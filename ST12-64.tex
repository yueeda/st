\documentclass[10pt]{jsarticle}
\usepackage{okumacro}
\usepackage{longtable}
\usepackage[polutonikogreek,english,japanese]{babel}
\usepackage{latexsym}
\usepackage{color}
\usepackage{schemata}
\usepackage[T1]{fontenc}
\usepackage{lmodern}

%----- header -------
\usepackage{fancyhdr}
\pagestyle{fancy}
\lhead{{\it Summa Theologiae} I-II, q.64}
%--------------------

\bibliographystyle{jplain}

\title{{\bf PRIMA SECUNDAE}\\{\HUGE Summae Theologiae}\\Sancti Thomae
Aquinatis\\{\sffamily QUEAESTIO SEXAGESIMAQUARTA}\\DE MEDIO VIRTUTUM}
\author{Japanese translation\\by Yoshinori {\sc Ueeda}}
\date{Last modified \today}

%%%% コピペ用
%\rhead{a.~}
%\begin{center}
% {\Large {\bf }}\\
% {\large }\\
% {\footnotesize }\\
% {\Large \\}
%\end{center}
%
%\begin{longtable}{p{21em}p{21em}}
%
%&
%
%
%\\
%
%\end{longtable}
%\newpage

\begin{document}

\maketitle
\thispagestyle{empty}

\begin{center}
{\LARGE 『神学大全』第二部の一}\\
{\Large 第六十四問\\徳の中間について}
\end{center}

\begin{longtable}{p{21em}p{21em}}

Deinde considerandum est de proprietatibus virtutum. Et primo quidem,
de medio virtutum; secundo, de connexione virtutum; tertio, de
aequalitate earum; quarto, de ipsarum duratione. Circa primum
quaeruntur quatuor. 

\begin{enumerate}
 \item utrum virtutes morales sint in medio.
 \item utrum medium virtutis moralis sit medium rei, vel rationis.
 \item utrum intellectuales virtutes consistant in medio.
 \item utrum virtutes theologicae.
\end{enumerate}

&

次に徳の特性について考察されるべきである。第一に徳の中間\footnote{「中
 庸」が定訳であろうが、「中庸」は徳を含意するので、本項で中立的な意味
 で用いられるmediumの訳語としては相応しくない。あえて一貫して「中間」
 と訳してみる。}について、第二に徳の結びつきについて、第三に徳の等しさ
 について、第四に徳の持続について。第一のものを巡っては四つのことが問
 われる。

\begin{enumerate}
 \item 道徳的徳は中間においてあるか。
 \item 道徳的徳の中間は事物の中間か、理性の中間か。
 \item 知性的徳は中間においてあるか。
 \item 神学的徳はどうか。
\end{enumerate}
\end{longtable}

\newpage

\rhead{a.~1}
\begin{center}
{\Large {\bf ARTICULUS PRIMUS}}\\
{\large UTRUM VIRTUTES MORALES SINT IN MEDIO}\\
{\footnotesize II$^{a}$II$^{ae}$, q.17, a.5, ad 2; III {\itshape Sent.}, d.33, q.1, a.3, qu$^{a}$1; {\itshape De Virtut.}, q.1, a.13; q.4, a.1, ad 7; II {\itshape Ethic.}, lect.6, 7.}\\
{\Large 第一項\\道徳的徳は中間においてあるか}
\end{center}

\begin{longtable}{p{21em}p{21em}}
{\scshape Ad primum sic proceditur}. Videtur quod virtus moralis non
consistat in medio. Ultimum enim repugnat rationi medii. Sed de
ratione virtutis est ultimum, dicitur enim in I {\itshape de Caelo},
quod {\itshape virtus est ultimum potentiae}. Ergo virtus moralis non
consistit in medio.

&

 第一項の問題へ、議論は以下のように進められる。道徳的徳は中間において
 成立するのではないと思われる。理由は以下の通り。究極は中間の性格に反
 する。しかるに徳の性格には究極が属する。なぜなら『天体論』第1巻では
 「徳は能力の究極である」と言われているからである。ゆえに道徳的徳は中
 間において成立するのではない。

\\



2.~{\scshape Praeterea}, illud quod est maximum, non est medium. Sed
quaedam virtutes morales tendunt in aliquod maximum, sicut
magnanimitas est circa maximos honores, et magnificentia circa maximos
sumptus, ut dicitur in IV {\itshape Ethic}. Ergo non omnis virtus
moralis est in medio.

&

 さらに、最大のものは中間ではない。しかるにある道徳的徳はある最大のも
 のへと向かう。たとえば『ニコマコス倫理学』第4巻で言われているように、
 高邁は最大の名誉、豪毅は最大の消費を巡ってある。ゆえにすべての道徳的
 徳が中間にあるのではない。

\\

3.~{\scshape Praeterea}, si de ratione virtutis moralis sit in medio
esse, oportet quod virtus moralis non perficiatur, sed magis
corrumpatur, per hoc quod tendit ad extremum. Sed quaedam virtutes
morales perficiuntur per hoc quod tendunt ad extremum, sicut
virginitas, quae abstinet ab omni delectabili venereo, et sic tenet
extremum, et est perfectissima castitas. Et dare omnia pauperibus est
perfectissima misericordia vel liberalitas. Ergo videtur quod non sit
de ratione virtutis moralis esse in medio.


&

 さらに、もし道徳的徳の性格に中間にあると言うことが属するならば、道徳
 的徳は極端に向かうことによって完成されず、むしろ消滅するだろう。しか
 るにある道徳的徳は極端に向かうことによって完成される。たとえば、処女
 性は、すべての快適なヴィーナス的なものを控え、そのようにして極端を保
 持し、この上なく完全な貞潔となる。また、貧しい人々にすべてを与えるこ
 とはこの上なく完全な憐れみないし気前の善さである。ゆえに道徳的徳の性
 格に、中間にあることは属していないと思われる。

\\




{\scshape Sed contra est} quod philosophus dicit, in II {\itshape
Ethic}., quod {\itshape virtus moralis est habitus electivus in
medietate existens}.


&

 しかし反対に、哲学者は『ニコマコス倫理学』第2巻で「道徳的徳は中間にあっ
 て選択しうる習慣である」と述べている。
 

\\




 {\scshape Respondeo dicendum} quod, sicut ex supradictis patet,
 virtus de sui ratione ordinat hominem ad bonum. Moralis autem virtus
 proprie est perfectiva appetitivae partis animae circa aliquam
 determinatam materiam. Mensura autem et regula appetitivi motus circa
 appetibilia, est ipsa ratio.

&

 解答する。以下のように言われるべきである。上述のことから明らかなとお
 り、徳は自らの性格から人間を善へ秩序付ける。しかるに道徳的徳は固有に、
 何らかの限定された質料をめぐってある魂の欲求的部分を完成しうる。とこ
 ろが欲求されうるものを巡る欲求的運動の尺度ないし規則は理性それ自体で
 ある。

\\


 Bonum autem cuiuslibet mensurati et regulati consistit in hoc quod
 conformetur suae regulae, sicut bonum in artificiatis est ut
 consequantur regulam artis. Malum autem per consequens in huiusmodi
 est per hoc quod aliquid discordat a sua regula vel mensura. Quod
 quidem contingit vel per hoc quod superexcedit mensuram, vel per hoc
 quod deficit ab ea, sicut manifeste apparet in omnibus regulatis et
 mensuratis.

 &

 どんな尺度や規則の善も、自らの規則に一致することにおいて成立する。た
 とえば技術作品における善は技術の規則を獲得することである。他方悪は、
 結果的に、あるものが自らの規則や尺度に一致しないことによってそのよう
 なもののうちにある。これは、尺度を超えることによって、あるいはそれに
 足りないことによって生じる。これは、規制され測られるすべてのものにお
 いて明らかなとおりである。

\\


 Et ideo patet quod bonum virtutis moralis consistit in adaequatione
 ad mensuram rationis. Manifestum est autem quod inter excessum et
 defectum medium est aequalitas sive conformitas. Unde manifeste
 apparet quod virtus moralis in medio consistit.

&

 ゆえに、道徳的徳の善は理性の尺度への対等において成立することが明らか
 である。さらに、超過と不足の間の中間が対等あるいは一致であることは明
 らかである。ゆえに道徳的徳が中間において成立することは明らかである。

\\


{\scshape Ad primum ergo dicendum} quod virtus moralis bonitatem habet
ex regula rationis, pro materia autem habet passiones vel
operationes. Si ergo comparetur virtus moralis ad rationem, sic,
secundum id quod rationis est, habet rationem extremi unius, quod est
conformitas, excessus vero et defectus habet rationem alterius
extremi, quod est difformitas.

&

 第一異論に対しては,それゆえ、以下のように言われるべきである。道徳的
 徳は理性の規則から善性を持ち、情念や働きを質料としてもつ。ゆえにもし
 道徳的徳が理性との関係で見られるならば、それが理性に属する限りにおい
 て一つの端の性格を持ち、それは一致であり、超過と不足はもう一つの端の
 性格、すなわち不一致という性格をもつ。

\\



 Si vero consideretur virtus moralis secundum suam materiam, sic habet
rationem medii, inquantum passionem reducit ad regulam rationis. Unde
philosophus dicit, in II {\itshape Ethic}., quod {\itshape virtus
secundum substantiam medietas est}, inquantum regula virtutis ponitur
circa propriam materiam, {\itshape secundum optimum autem et bene, est
extremitas}, scilicet secundum conformitatem rationis.


&

 しかしもし道徳的徳が自らの質料に即して考察されるならば、その場合には
 情念を理性の規則へ導く限りにおいて、中間の性格を持つ。このことから哲
 学者は『ニコマコス倫理学』第2巻で次のように言う。徳の規則が固有の質料
 をめぐって指定される限りにおいて「徳は実体において中間である」が、理
 性の一致に即しては「最善であり善くある限りにおいては極端である」。

\\



{\scshape Ad secundum dicendum} quod medium et extrema considerantur
in actionibus et passionibus secundum diversas circumstantias, unde
nihil prohibet in aliqua virtute esse extremum secundum unam
circumstantiam, quod tamen est medium secundum alias circumstantias,
per conformitatem ad rationem.

&

 第二異論に対しては以下のように言われるべきである。中間と極端は作用と
 情念においてさまざまな環境に即して考察される。だから、何らかの徳にお
 いて、ある環境において極端であるものが、別の環境においては理性への一
 致によって中間であるということがあってもかまわない。

\\

 Et sic est in magnificentia et magnanimitate. Nam si consideretur
quantitas absoluta eius in quod tendit magnificus et magnanimus,
dicetur extremum et maximum, sed si consideretur hoc ipsum per
comparationem ad alias circumstantias, sic habet rationem medii; quia
in hoc tendunt huiusmodi virtutes secundum regulam rationis, idest ubi
oportet, et quando oportet, et propter quod oportet.

&

 そして、高邁と豪毅においてはこのようである。すなわち、もし高邁な人や
 豪毅な人が向かうものの無条件的な量が考えられるならば、それは極端であ
 り最大であると言われるが、もし、他の環境との比較によってそれが考えら
 れるならば、その場合には中間の性格を持つ。なぜなら、このような徳は理
 性の規制に従って、すなわち、しかるべきところでしかるべき時に、しかる
 べきもののためにそれへと向かうからである。

\\



 Excessus autem, si in hoc maximum tendatur quando non oportet, vel
ubi non oportet, vel propter quod non oportet; defectus autem est, si
non tendatur in hoc maximum ubi oportet, et quando oportet. Et hoc est
quod philosophus dicit, in IV {\itshape Ethic}., quod {\itshape
magnanimus est quidem magnitudine extremus; eo autem quod ut oportet,
medius}.

&

 これに対して超過は、もしこの最大のものへ向かうとしても、あるべきでな
 いときに、あるべきでない場所で、あるべきでないもののためにそうする。
 また不足は、しかるべき場所でしかるべき時に、それへと向かわない場合で
 ある。哲学者が『ニコマコス倫理学』第4巻で「高邁な人は大きさにおいて極
 端だが、しかるべき事という点では、中間である」と言うのもこのことであ
 る。

\\


 {\scshape Ad tertium dicendum quod} eadem ratio est de virginitate et
 paupertate, quae est de magnanimitate. Abstinet enim virginitas ab
 omnibus venereis, et paupertas ab omnibus divitiis, propter quod
 oportet, et secundum quod oportet; idest secundum mandatum Dei, et
 propter vitam aeternam.

&

 第三異論に対しては以下のように言われるべきである。豪毅についてと同じ
 論が処女性と清貧に当てはまる。すなわち処女性はすべてのヴィーナス的な
 ことから、清貧はすべての富から、しかるべきもののために、しかるべき事
 に即して離れる。すなわち、神の命令に則して永遠の生命のためにそうする。

\\

 Si autem hoc fiat secundum quod non oportet, idest secundum aliquam
 superstitionem illicitam, vel etiam propter inanem gloriam; erit
 superfluum. Si autem non fiat quando oportet, vel secundum quod
 oportet, est vitium per defectum, ut patet in transgredientibus votum
 virginitatis vel paupertatis.

&

 他方、もしこれがあるべきでないものに即してなされるならば、すなわち、
 何らかの不正な迷信や空虚な栄光のためにそうするならば、それは無駄なこ
 とになるだろう。また他方、もししかるべき時に、しかるべき事に即してな
 されないならば、それは不足による悪徳となるだろう。それは処女性や清貧
 の誓いを踏み外す人々において明らかなとおりである。


\end{longtable}


\newpage

\rhead{a.~2}
\begin{center}
{\Large {\bf ARTICULUS SECUNDUS}}\\ {\large UTRUM MEDIUM VIRTUTIS
MORALIS SIT MEDIUM REI, VEL RATIONIS}\\{\footnotesize
II$^{a}$II$^{ae}$, q.58, a.10; III {\itshape Sent.}, d.33, q.1, a.3,
qu$^{a}$2; {\itshape De Virtut.}, q.1, a.13.}\\ {\Large 第二項\\道徳的
徳の中間は事物の中間か、それとも理性の中間か}
\end{center}

\begin{longtable}{p{21em}p{21em}}
{\scshape Ad secundum sic proceditur}. Videtur quod medium virtutis
moralis non sit medium rationis, sed medium rei. Bonum enim virtutis
moralis consistit in hoc quod est in medio. Bonum autem, ut dicitur in
VI {\itshape Metaphys}., est in rebus ipsis. Ergo medium virtutis
moralis est medium rei.


&

 第二項の問題へ、議論は以下のように進められる。道徳的徳の中間は理性で
 はなく事物の中間だと思われる。理由は以下の通り。道徳的徳の善は中間に
 あることにおいて成立する。しかるに『形而上学』第6巻で言われるように、
 善は事物自体においてある。ゆえに道徳的徳の中間は事物の中間である。

 
\\

2.~{\scshape Praeterea}, ratio est vis apprehensiva. Sed virtus
moralis non consistit in medio apprehensionum; sed magis in medio
operationum et passionum. Ergo medium virtutis moralis non est medium
rationis, sed medium rei.

&

 さらに、理性は把握する力である。しかるに道徳的徳は把握の中間において
 ではなく、むしろ働きと情念の中間において成立する。ゆえに道徳的徳の中
 間は理性でなく事物の中間である。

\\

3.~{\scshape Praeterea}, medium quod accipitur secundum proportionem
arithmeticam vel geometricam, est medium rei. Sed tale est medium
iustitiae, ut dicitur in V {\itshape Ethic}. Ergo medium virtutis
moralis non est medium rationis, sed rei.

&

 さらに、算術的、あるいは幾何学的な比に即して理解される中間は事物の中
 間である。しかるに、『ニコマコス倫理学』第5巻で言われるように、正義の
 中間ははそのような中間である。ゆえに道徳的徳の中間は理性ではなく事物
 の中間である。


\\


{\scshape Sed contra est} quod philosophus dicit, in II {\itshape
Ethic}., quod {\itshape virtus moralis in medio consistit quoad nos,
determinata ratione}.


&

 しかし反対に、哲学者は『ニコマコス倫理学』第2巻で、「道徳的徳は、私た
 ちにとって、理性によって限定されたものとして、中間において成立する」
 と述べている。

\\




 {\scshape Respondeo dicendum} quod medium rationis dupliciter potest
 intelligi. Uno modo, secundum quod medium in ipso actu rationis
 existit, quasi ipse actus rationis ad medium reducatur. Et sic, quia
 virtus moralis non perficit actum rationis, sed actum virtutis
 appetitivae; medium virtutis moralis non est medium rationis.


&

 解答する。以下のように言われるべきである。理性の中間は二通りに理解さ
 れうる。一つには、理性の作用自体が中間へと還元されるかのように、理性
 の作用自体の中に中間があることに即してである。この意味では、道徳的徳
 は理性の作用ではなく欲求的ちからの作用を完成するので、道徳的徳の中間
 は理性の中間ではない。

\\

 Alio modo potest dici medium rationis id quod a ratione ponitur in
 aliqua materia. Et sic omne medium virtutis moralis est medium
 rationis, quia, sicut dictum est, virtus moralis dicitur consistere
 in medio, per conformitatem ad rationem rectam.


&

もう一つの仕方では、理性の中間は、ある質料の中に理性によって指定され
 るものとして言われうる。この意味では、すべての道徳的徳の中間は理性の
 中間である。なぜなら、すでに述べられたとおり、道徳的徳は、正しい理性
 への一致によって、中間において成立すると言われるからである。
 
\\



 Sed quandoque contingit quod medium rationis est etiam medium rei, et
 tunc oportet quod virtutis moralis medium sit medium rei; sicut est
 in iustitia.


&

 しかし時として、理性の中間が事物の中間でもあるということが生じる。そ
 してその場合には、道徳的徳の中間は事物の中間でなければならない。ちょ
 うど正義においてそうであるように。

\\

 Quandoque autem medium rationis non est medium rei, sed accipitur per
 comparationem ad nos, et sic est medium in omnibus aliis virtutibus
 moralibus. Cuius ratio est quia iustitia est circa operationes, quae
 consistunt in rebus exterioribus, in quibus rectum institui debet
 simpliciter et secundum se, ut supra dictum est, et ideo medium
 rationis in iustitia est idem cum medio rei, inquantum scilicet
 iustitia dat unicuique quod debet, et non plus nec minus.

&

 また、別の場合には、理性の中間が事物の中間ではなく、私たちへの関係に
 よって理解されることがある。そして、他のすべての道徳的徳の中間はその
 ようなものである。その理由は以下の通りである。正義は働きを巡ってある
 が、働きは外的な事物において成立し、外的な事物において、正しい事柄が
 端的に自体的に打ち立てられなければならない。これは上述の如くである。
 ゆえに正義における理性の中間は事物の中間と同じである。それは、正義が
 各々のものに、より多くも少なくもなく、しかるべきものを与える限りにお
 いてである。

\\

 Aliae vero virtutes morales consistunt circa passiones interiores, in
 quibus non potest rectum constitui eodem modo, propter hoc quod
 homines diversimode se habent ad passiones, et ideo oportet quod
 rectitudo rationis in passionibus instituatur per respectum ad nos,
 qui afficimur secundum passiones.


&

 これに対して他の道徳的徳は内的な情念を巡って成立するが、情念において
 は同じしかたで正しい事柄が成立しえない。それは人間がさまざまな仕方で
 情念に関係し、それゆえ、理性の正しさが情念において打ち立てられるのは
 私たちへの関係を通してだからである。私たちはそれらの情念に即して情動
 を受ける。

\\

Et per hoc patet responsio ad obiecta. Nam, primae duae rationes
procedunt de medio rationis quod scilicet invenitur in ipso actu
rationis. Tertia vero ratio procedit de medio iustitiae.


&

 以上のことを通して、異論への解答は明らかである。すなわち、最初の二つ
 の異論は、理性の作用自体の中に見出される、理性の中間について論じてお
 り、他方、第三異論は正義の中間について論じている。

\\

\end{longtable}
\newpage





\rhead{a.~3}
\begin{center}
{\Large {\bf ARTICULUS TERTIUS}}\\
{\large UTRUM VIRTUS INTELLECTUALES CONSISTANT IN MEDIO}\\
{\footnotesize III {\itshape Sent.}, d.33, q.1, a.3, qu$^{a}$3; {\itshape De Virtut.}, q.1, a.13; q.4, a.1, ad 7.}\\
{\Large 第三項\\知性的徳は中間において成立するか}
\end{center}

\begin{longtable}{p{21em}p{21em}}
{\scshape Ad tertium sic proceditur}. Videtur quod virtutes
intellectuales non consistant in medio. Virtutes enim morales
consistunt in medio, inquantum conformantur regulae rationis. Sed
virtutes intellectuales sunt in ipsa ratione; et sic non videntur
habere superiorem regulam. Ergo virtutes intellectuales non consistunt
in medio.


&

 第三項の問題へ、議論は以下のように進められる。知性的徳は中間において
 成立するのではないと思われる。理由は以下の通り。道徳的徳が中間におい
 て成立するのは、理性の規制に一致する限りにおいてである。しかるに知性
 的徳は理性それ自体の中にあり、その意味で、上位の規制をもたないと思わ
 れる。ゆえに知性的徳は中間において成立するのではないと思われる。

\\


2.~{\scshape Praeterea}, medium virtutis moralis determinatur a
virtute intellectuali, dicitur enim in II {\itshape Ethic}., quod
{\itshape virtus consistit in medietate determinata ratione, prout
sapiens determinabit}. Si igitur virtus intellectualis iterum
consistat in medio, oportet quod determinetur sibi medium per aliquam
aliam virtutem. Et sic procedetur in infinitum in virtutibus.


&

 さらに、道徳的徳の中間は、知性的徳によって限定される。というのも『ニ
 コマコス倫理学』第2巻で「徳は、ちょうど知恵が限定するであろうように、
 理性によって限定された中間性において成立する」と言われているからであ
 る。ゆえに、もし知性的徳がさらに中間において成立するならば、何か他の
 徳によって、自らに中間が限定されなければならない。かくして徳の中を無
 限に進むことになったであろう。

\\


3.~{\scshape Praeterea}, medium proprie est inter contraria; ut patet
per philosophum, in X {\itshape Metaphys}. Sed in intellectu non
videtur esse aliqua contrarietas, cum etiam ipsa contraria, secundum
quod sunt in intellectu, non sint contraria, sed simul intelligantur,
ut album et nigrum, sanum et aegrum. Ergo in intellectualibus
virtutibus non est medium.

&

 さらに、『形而上学』第10巻の哲学者によって明らかなとおり、中間とは、
 厳密には相反するものの間にある。しかるに知性の中には相反性は存在しな
 い。なぜなら、相反性それ自体もまた、知性の中にある限りにおいては、相
 反するものではなく、同時に知性認識されるからである。たとえば白と黒、
 健康と病気のように。ゆえに知性的徳の中に中間はない。

\\




{\scshape Sed contra est} quod ars est virtus intellectualis, ut
dicitur in VI {\itshape Ethic}.; et tamen artis est aliquod medium, ut
dicitur in II {\itshape Ethic}. Ergo etiam virtus intellectualis
consistit in medio.


&

 しかし反対に、『ニコマコス倫理学』第6巻で言われるように、技術は知性的
 徳である。しかし『ニコマコス倫理学』第2巻で言われるように、技術には何
 らかの中間が属する。ゆえに知性的徳も中間において成立する。

\\




{\scshape Respondeo dicendum} quod bonum alicuius rei consistit in
medio, secundum quod conformatur regulae vel mensurae quam contingit
transcendere et ab ea deficere, sicut dictum est. Virtus autem
intellectualis ordinatur ad bonum, sicut et moralis, ut supra dictum
est.

 
&

 解答する。以下のように言われるべきである。ある事物の善は中間において
 成立するが、すでに述べられたとおり、それは越えることも不足することも
 ある規制ないし尺度に一致するかぎりにおいてである。さらに、上述の如く、
 知性的徳は、道徳的徳と同じく、善へと秩序付けられている。

\\


 Unde secundum quod bonum virtutis intellectualis se habet ad
 mensuram, sic se habet ad rationem medii. Bonum autem virtutis
 intellectualis est verum, speculativae quidem virtutis, verum
 absolute, ut in VI Ethic. dicitur; practicae autem virtutis, verum
 secundum conformitatem ad appetitum rectum.


&

 したがって、知性的徳の善が尺度に関係するかぎりにおいて、そのようにそ
 れは中間の性格に関係する。しかるに、知性的徳の善は真である。『ニコマ
 コス倫理学』第6巻で言われるように、観照的な知性的徳の善は無条件的な真
 であり、実践的な知性的徳の善は、正しい欲求への一致に即した真である。

\\


 Verum autem intellectus nostri absolute consideratum, est sicut
 mensuratum a re, res enim est mensura intellectus nostri, ut dicitur
 in X {\itshape Metaphys}.; ex eo enim quod res est vel non est,
 veritas est in opinione et in oratione. Sic igitur bonum virtutis
 intellectualis speculativae consistit in quodam medio, per
 conformitatem ad ipsam rem, secundum quod dicit esse quod est, vel
 non esse quod non est; in quo ratio veri consistit. Excessus autem
 est secundum affirmationem falsam, per quam dicitur esse quod non
 est, defectus autem accipitur secundum negationem falsam, per quam
 dicitur non esse quod est.

&

 ところで、私たちの知性の真は、無条件的に考察される場合には、事物によっ
 て測られたものとしてある。なぜなら、『形而上学』第10巻で言われるよう
 に、事物は私たちの知性の尺度だからである。つまり、事物が在るあるいは
 無いことによって、意見や言明の中に真理があるからである。ゆえにこのよ
 うにして、観照的な知性的徳の善は、在るものを在る、無いものを無いと言
 うことにおいて、そして真の性格はここに成り立つのだが、事物自体へ一致
 することによって、ある種の中間において成立する。これに対して超過は、
 無いものが在ると言われる偽の肯定に即して、また不足は在るものが無いと
 言われる偽の否定に即して理解される。

\\



 Verum autem virtutis intellectualis practicae, comparatum quidem ad
 rem, habet rationem mensurati. Et sic eodem modo accipitur medium per
 conformitatem ad rem, in virtutibus intellectualibus practicis, sicut
 in speculativis. Sed respectu appetitus, habet rationem regulae et
 mensurae. Unde idem medium, quod est virtutis moralis, etiam est
 ipsius prudentiae, scilicet rectitudo rationis, sed prudentiae quidem
 est istud medium ut regulantis et mensurantis; virtutis autem
 moralis, ut mensuratae et regulatae. Similiter excessus et defectus
 accipitur diversimode utrobique.


&

 他方、実践的な知性的徳の真は、事物に対して、測られたものの性格を持つ。
 この意味で、実践的な知性的徳において中間は、観照的なそれと同じしかた
 で、事物への一致によって理解される。しかし欲求に対しては、規制や尺度
 という性格を持つ。したがって、道徳的徳と同じ中間、すなわち理性の正し
 さが、思慮自体に属する。しかし思慮は規制し測るものとして中間をもつが、
 道徳的徳は、測られ規制されるものとして中間をもつ。同様に、超過と不足
 も、両者において違った仕方で理解される。

\\




{\scshape Ad primum ergo dicendum} quod etiam virtus intellectualis
habet suam mensuram, ut dictum est, et per conformitatem ad ipsam,
accipitur in ipsa medium.

&

 第一異論に対しては、それゆえ、以下のように言われるべきである。すでに
 述べられたとおり、知性的徳も自らの尺度を持ち、それへの一致によってそ
 の中に中間を受け取る。

\\

{\scshape Ad secundum dicendum} quod non est necesse in infinitum
procedere in virtutibus, quia mensura et regula intellectualis
virtutis non est aliquod aliud genus virtutis, sed ipsa res.

&

 第二異論に対しては以下のように言われるべきである。徳の中を無限に進む
 必要はない。なぜなら知性的徳の尺度や規制は徳の何らかの他の類ではなく、
 事物それ自体だからである。

\\


{\scshape Ad tertium dicendum} quod ipsae res contrariae non habent
contrarietatem in anima, quia unum est ratio cognoscendi alterum, et
tamen in intellectu est contrarietas affirmationis et negationis, quae
sunt contraria, ut dicitur in fine {\itshape Peri Hermeneias}. Quamvis
enim esse et non esse non sint contraria, sed contradictorie opposita,
si considerentur ipsa significata prout sunt in rebus, quia alterum
est ens, et alterum est pure non ens, tamen si referantur ad actum
animae, utrumque ponit aliquid. Unde esse et non esse sunt
contradictoria, sed opinio qua opinamur quod {\itshape bonum est
bonum}, est contraria opinioni qua opinamur quod {\itshape bonum non
est bonum}. Et inter huiusmodi contraria medium est virtus
intellectualis.


&

 第三異論に対しては以下のように言われるべきである。相反する事物自体は
 魂の中に相反性をもたない。なぜなら、一方が他方を認識する根拠(ratio)だ
 からである。しかし知性の中には肯定と否定の相反性があるのであり、これ
 らが相反することは『命題論』の最終巻で言われている。つまり、「在る」
 と「無い」は相反するものではないが、矛盾するものとして対立するのであ
 り、それは、表示された事柄自体が事物においてあるものとして考察された
 場合にそうなる。なぜなら、一方が有であり、他方が純粋な非有だからであ
 る\footnote{難読。contrariumとcontradictorie oppositumとを区別して、
 前者の要件として、関係する両項がともになんらかの意味で存在することを
 求めている。その意味で、有と非有はcontrariaの関係にはない。この有と非
 有の関係を表すのが後者で、「矛盾的に対立する」と言われている。}。しか
 し、もし魂の作用へ関係づけられるならば、どちらも何かを定立する。した
 がって、「在る」と「無い」は矛盾するが、しかし「善は善である」と私た
 ちが意見するその意見は、「善は善でない」と私たちが意見するその意見に
 相反する。そしてこのような相反するものの間で、知性的な徳は中間である。

\end{longtable}
\newpage

\rhead{a.~4}
\begin{center}
{\Large {\bf ARTICULUS QUARTUS}}\\
{\large UTRUM VIRTUTES THEOLOGICAE CONSISTANT IN MEDIO}\\
{\footnotesize II$^{a}$II$^{ae}$, q.17, a.5, ad 2; III {\itshape Sent.}, d.33, q.1, a.3, qu$^{a}$}4; {\itshape De Virtut.}, q.1, a.13; q.2, a.2, ad 10, 13; q.4, a.1, ad 7; {\itshape Rom.}, cap.12, lect.1.\\
{\Large 第四項\\神学的徳は中間において成立するか}
\end{center}

\begin{longtable}{p{21em}p{21em}}
 {\scshape Ad quartum sic proceditur}. Videtur quod virtus theologica
 consistat in medio. Bonum enim aliarum virtutum consistit in
 medio. Sed virtus theologica excedit in bonitate alias virtutes. Ergo
 virtus theologica multo magis est in medio.
 
&

 第四項の問題へ、議論は以下のように進められる。神学的徳は中間において
 成立すると思われる。理由は以下の通り。他の徳の善は中間において成立す
 る。しかるに神学的徳は他の徳を善性において越える。ゆえにましてや神学
 的徳は中間において成立する。

\\


2.~{\scshape Praeterea}, medium virtutis accipitur, moralis quidem
secundum quod appetitus regulatur per rationem; intellectualis vero
secundum quod intellectus noster mensuratur a re. Sed virtus
theologica et perficit intellectum, et appetitum, ut supra dictum
est. Ergo etiam virtus theologica consistit in medio.
 
&

 さらに、徳の中間は、道徳的な場合は欲求が理性によって規制されることに
 即して、知性的な場合は私たちの知性が事物によって測られることに即して
 理解される。しかるに神学的徳は、すでに述べられたとおり、知性も欲求も
 完成する。ゆえに神学的徳も中間において成立する。

\\

3.~{\scshape Praeterea}, spes quae est virtus theologica, medium est
inter desperationem et praesumptionem. Similiter etiam fides incedit
media inter contrarias haereses, ut Boetius dicit, in libro {\itshape
de Duabus Naturis}, quod enim confitemur in Christo unam personam et
duas naturas, medium est inter haeresim Nestorii, qui dicit duas
personas et duas naturas; et haeresim Eutychis, qui dicit unam
personam et unam naturam. Ergo virtus theologica consistit in medio.
 
&

 さらに、神学的徳である希望は、絶望と慢心の中間である。同様に信仰は、
 ボエティウスが『二つの本性について』という書物で言うように、相反する
 異端の間の中間を行く。この書物は、キリストの中に一つのペルソナと二つ
 の本性があることを告白するが、これは、二つのペルソナと二つの本性があ
 るとしたネストリウスの異端と、一つのペルソナと一つの本性があるとした
 エウテュキスの異端の中間である。ゆえに神学的徳は中間において成立する。

\\

 Sed contra, in omnibus in quibus consistit virtus in medio, contingit
 peccare per excessum, sicut et per defectum. Sed circa Deum, qui est
 obiectum virtutis theologicae, non contingit peccare per excessum,
 dicitur enim {\itshape Eccli}.~{\scshape xliii}, {\itshape
 benedicentes Deum, exaltate illum quantum potestis, maior enim est
 omni laude}. Ergo virtus theologica non consistit in medio.

 
&

 しかし反対に、徳が中間において成り立つようなすべてのものには、欠陥に
 よるようにして、過剰によって罪を犯すことが生じる。しかるに神学的徳の
 対象である神を巡っては過剰による罪は生じない。なぜなら『シラ書』第18
 章「神を褒め称える人々よ、それをできる限り高く上げよ。神はあらゆる賞
 賛よりも大きいのだから」\footnote{「その栄光をたたえて主を崇めよ。/
 全力を尽くしても/主はなおもそれを超えておられる。/力の限り主を崇め
 よ。/倦むことなく崇めよ。/もうこれで十分ということはないのだから。」
 (43:30)}と言われているからである。ゆえに神学的徳は中間において成立す
 るのではない。

\\

 {\scshape Respondeo dicendum} quod, sicut dictum est, medium virtutis
 accipitur per conformitatem ad suam regulam vel mensuram, secundum
 quod contingit ipsam transcendere vel ab ea deficere. 

&

 解答する。以下のように言われるべきである。すでに述べられたとおり、徳
 の中間は、自らの規制や尺度に、それを越えたりそれに足りなかったりする
 ことがある限りにおいて、一致することによって理解される。

\\

 Virtutis autem theologicae duplex potest accipi mensura. Una quidem
 secundum ipsam rationem virtutis.  Et sic mensura et regula virtutis
 theologicae est ipse Deus, fides enim nostra regulatur secundum
 veritatem divinam, caritas autem secundum bonitatem eius, spes autem
 secundum magnitudinem omnipotentiae et pietatis eius. Et ista est
 mensura excellens omnem humanam facultatem, unde nunquam potest homo
 tantum diligere Deum quantum diligi debet, nec tantum credere aut
 sperare in ipsum, quantum debet.

 
&

 しかるに神学的徳は二通りに尺度を受け取りうる。一つは徳の性格それ自体
 にそくしてである。その場合、神学的徳の尺度と規制は神自身である。たと
 えば、私たちの信仰は神の真理に即して規制され、愛徳は神の善性に即して、
 また希望は全能の大きさとその敬神に即して規制される。そしてこれは人間
 のすべての機能を超える尺度なので、人間は神を愛すべきほどに愛すること
 はできないし、しかるべき程度に神を信じたり希望したりすることもできな
 い。

\\


Unde multo minus potest ibi esse excessus. Et sic bonum talis virtutis
non consistit in medio, sed tanto est melius, quanto magis acceditur
 ad summum.


&

 したがって、ましてやそこに超過はありえない。この意味で、そのような徳
 の善は中間において成立せず、最高へと近づけば近づくほど善い。

\\



 Alia vero regula vel mensura virtutis theologicae est ex
parte nostra, quia etsi non possumus ferri in Deum quantum debemus,
debemus tamen ferri in ipsum credendo, sperando et amando, secundum
mensuram nostrae conditionis. Unde per accidens potest in virtute
theologica considerari medium et extrema, ex parte nostra.

&

 他方で、神学的徳のもう一つの規制ないし尺度は、私たちの側からある。と
 いうのも、私たちが神へしかるべき程度にまでもたらされることはありえな
 いにしても、しかし、私たちの状態の尺度に即して、信じること、希望する
 こと、愛することにおいて、神へともたらされる。したがって、附帯的に、
 神学的徳において、私たちの側から中間と極端とが考察されうる。

\\

 {\scshape Ad primum ergo dicendum} quod bonum virtutum
 intellectualium et moralium consistit in medio per conformitatem ad
 regulam vel mensuram quam transcendere contingit. Quod non est in
 virtutibus theologicis, per se loquendo, ut dictum est.
 
&

 第一異論に対しては、それゆえ、以下のように言われるべきである。知性的、
 道徳的徳の善は、それを超えることがありうるような規制や尺度への一致に
 よって、中間において成立する。このことは、すでに述べられたとおり、自
 体的に語るならば神学的徳においてはない。
 

\\


{\scshape Ad secundum dicendum} quod virtutes morales et
intellectuales perficiunt intellectum et appetitum nostrum in ordine
ad mensuram et regulam creatam, virtutes autem theologicae in ordine
ad mensuram et regulam increatam. Unde non est similis ratio.
 
&

 第二異論に対しては以下のように言われるべきである。道徳的徳と知性的徳
 は私たちの知性と欲求を被造の尺度と規制への秩序において完成するが、神
 学的徳は想像されない尺度と規制への秩序において完成する。したがって類
 似した論は成り立たない。

\\

{\scshape Ad tertium dicendum} quod spes est media inter
praesumptionem et desperationem, ex parte nostra, inquantum scilicet
aliquis praesumere dicitur ex eo quod sperat a Deo bonum quod excedit
suam conditionem; vel non sperat quod secundum suam conditionem
sperare posset. Non autem potest esse superabundantia spei ex parte
Dei, cuius bonitas est infinita. Similiter etiam fides est media inter
contrarias haereses, non per comparationem ad obiectum, quod est Deus,
cui non potest aliquis nimis credere, sed inquantum ipsa opinio humana
est media inter contrarias opiniones, ut ex supradictis patet.

 &

第三異論に対しては以下のように言われるべきである。希望は慢心と絶望の中
間だが、それは私たちの側から、すなわち、人が自分の条件を越える善を神か
ら希望するかぎりで、あるいは自分の条件に即して希望することができたのに
それを希望しないかぎりにおいてである。しかし、神の側からは、希望の過剰
はありえない。神の善性は無限だからである。同様に、信仰が相反する異端の
中間であるのは、対象すなわち神への関係によってではない。なぜなら人は神
を信じすぎることはできないからである。そうではなく、上述のことから明ら
かなとおり、人の意見が相反する意見の中間にあるかぎりにおいてである。
 
\end{longtable}
\end{document}
