\documentclass[10pt]{jsarticle} % use larger type; default would be 10pt
%\usepackage[utf8]{inputenc} % set input encoding (not needed with XeLaTeX)
%\usepackage[round,comma,authoryear]{natbib}
%\usepackage{nruby}
\usepackage{okumacro}
\usepackage{longtable}
%\usepqckage{tablefootnote}
\usepackage[polutonikogreek,english,japanese]{babel}
\usepackage[GlyphNames]{teubner}
%\usepackage{amsmath}
\usepackage{latexsym}
\usepackage{color}
%\usepackage{tikz}

%----- header -------
\usepackage{fancyhdr}
\pagestyle{fancy}
\lhead{{\it Summa Theologiae} I, q.~39}
%--------------------


\title{{\bf PRIMA PARS}\\{\Huge Summae Theologiae}\\Sancti Thomae
Aquinatis\\{\sffamily QUEAESTIO TRIGESTIMANONA}\\DE PERSONIS AD
ESSENTIAM RELATIS}
\author{Japanese translation\\by Yoshinori {\sc Ueeda}}
\date{Last modified \today}

%%%% コピペ用
%\rhead{a.~}
%\begin{center}
% {\Large {\bf }}\\
% {\large }\\
% {\footnotesize }\\
% {\Large \\}
%\end{center}
%
%\begin{longtable}{p{21em}p{21em}}
%
%&
%
%\\
%\end{longtable}
%\newpage


\begin{document}

\maketitle


\begin{center}
{\Large 聖トマス・アクィナスの神学大全の第一部\\第三十九問\\本質へ関係
 付けられたペルソナについて}
\end{center}




\newpage
\begin{longtable}{p{21em}p{21em}}

{\Huge P}ost ea quae de personis divinis absolute tractata sunt, considerandum
 restat de personis in comparatione ad essentiam, et ad proprietates, et
 ad actus notionales; et de comparatione ipsarum ad invicem. Quantum
 igitur ad primum horum, octo quaeruntur. 

\begin{enumerate}
 \item utrum essentia in divinis sit idem quod persona.
 \item utrum dicendum sit quod tres personae sunt unius essentiae.
 \item utrum nomina essentialia praedicanda sint de personis in plurali
       vel in singulari.
 \item utrum adiectiva notionalia, aut verba vel participia, praedicari
       possint de nominibus essentialibus concretive acceptis.
 \item utrum praedicari possint de nominibus essentialibus in abstracto
       acceptis.
 \item utrum nomina personarum praedicari possint de nominibus
       essentialibus concretis.
 \item utrum essentialia attributa sint approprianda personis.
 \item quod attributum cuique personae debeat appropriari.
\end{enumerate}


&


神のペルソナに属する事柄が非関係的に論じられたあと、本質との関係におい
て、固有性との関係において\footnote{q.40}、識標的作用との関係において
\footnote{q.41}、ペルソナについて考察すること、そしてペルソナ相互の関
係について\footnote{q.42}考察することが残されている。

さてこれらのうち第一のものについて八つのことが問われる。

\begin{enumerate}
 \item 神において本質はペルソナと同一か。
 \item 三つのペルソナに一つの本質が属すると言われるべきか。
 \item 本質的な名は、ペルソナについて複数で述語付けられるべきか、それ
       とも単数で述語付けられるべきか。
 \item 識標的な修飾語、動詞、部分詞は、具体的に理解された本質的名称につ
       いて述語付けられうるか。
 \item それらは抽象的に理解された本質的名称について述語付けられうるか。
 \item ペルソナの名称は具体的な本質的名称について述語付けられうるか。
 \item 本質的属性がペルソナに固有化されるべきか。
 \item どの属性がどのペルソナに固有化されるべきか。

\end{enumerate}


\end{longtable}



\newpage



\rhead{a.~1}
\begin{center}
 {\Large {\bf ARTICULUS PRIMUS}}\\
 {\large UTRUM IN DIVINIS ESSENTIA SIT IDEM QUOD PERSONA}\\
 {\footnotesize Supra q.3, a.3; I {\itshape Sent.}, d.34, q.1, a.1; III,
 d.6, q.2, a.2, ad 2.}\\
 {\Large 第一項\\神において本質はペルソナと同一か}
\end{center}

\begin{longtable}{p{21em}p{21em}}

{\Huge A}{\scshape d primum sic proceditur}. Videtur quod in divinis essentia non sit idem
quod persona. In quibuscumque enim essentia est idem quod persona seu
suppositum, oportet quod sit tantum unum suppositum unius naturae, ut
patet in omnibus substantiis separatis, eorum enim quae sunt idem re,
unum multiplicari non potest, quin multiplicetur et reliquum. Sed in
divinis est una essentia et tres personae, ut ex supra dictis
patet. Ergo essentia non est idem quod persona.


&

第一項の問題へ議論は以下のように進められる。
神において本質はペルソナと同一でないと思われる。理由は以下の通り。
本質がペルソナないし個体と同一であるようなどんなものにおいても、一つの本
 性をもつただ一つの個体があるのでなければならず、これは全ての離在実体
 において明らかなとおりである。実際、実在において同一のものは、その一つが多数
 化されると残りのものも多数化される\footnote{AとBとCが実在的に同一で
 あるならば、Aが多数化されるとBもCも多数化される。ペルソナと本質が同一
 ならペルソナが三つなら本質も三つのはず。}。しかし神におい
 て、前に述べられたことから明らかなとおり\footnote{q.28, a.3; q.30, a.2.}、一つの本質と三つのペルソナが
 ある。ゆえに本質はペルソナと同一でない。


\\



2 {\scshape Praeterea}, affirmatio et negatio simul et semel non verificantur de
eodem. Sed affirmatio et negatio verificantur de essentia et persona,
nam persona est distincta, essentia vero non est distincta. Ergo persona
et essentia non sunt idem.


&

さらに、同一のものについて肯定と否定とがともに同時に真となること
 はない。しかし本質とペルソナについては肯定と否定とが真となる。
すなわちペルソナは区別されるが本質は区別されない。
ゆえにペルソナと本質は同一でない。


\\



3 {\scshape Praeterea}, nihil subiicitur sibi ipsi. Sed persona subiicitur essentiae,
unde {\itshape suppositum} vel {\itshape hypostasis} nominatur. Ergo persona non est idem quod
essentia.


&

さらに、なにものも自分自身のもとにある(基体となる)ことはない。しかしペルソナは本
 質のもとにある。そのことから「個体」とか「ヒュポスタシス」と名付けら
 れる。ゆえにペルソナは本質と同一でない。

\\



{\scshape Sed contra est} quod Augustinus dicit, VII {\itshape de
 Trin}.: {\itshape Cum dicimus personam
patris, non aliud dicimus quam substantiam patris}.


&

しかし反対に、アウグスティヌスは『三位一体論』第7巻で「私たちが父のペルソ
 ナと言うとき父の実体以外のことを言わない」と述べている。


\\



{\scshape Respondeo dicendum} quod considerantibus divinam simplicitatem, quaestio
ista in manifesto habet veritatem. Ostensum est enim supra quod divina
simplicitas hoc requirit, quod in Deo sit idem essentia et suppositum;
quod in substantiis intellectualibus nihil est aliud quam persona. 


&

解答する。以下のように言われるべきである。
神の単純性を考察する人々にとって、この問いに対応する真理は明らかである。
 すなわち、以前に\footnote{STI, q.3, a.3.}、神の単純性は神において本
 質と個体が同一であることを要求することが示されたが、知性的実体において
 この個体はペルソナに他ならない。


\\

Sed
difficultatem videtur ingerere quod, multiplicatis personis divinis,
essentia retinet unitatem. 
Et quia, ut Boetius dicit, {\itshape relatio
multiplicat personarum Trinitatem}, posuerunt aliqui hoc modo in divinis
differre essentiam et personam, quo et relationes dicebant esse
assistentes, considerantes in relationibus solum quod ad alterum sunt,
et non quod res sunt. 


&

しかし神のペルソナが多数になると、本質が一性を保つのに困難が生じると思
 われる。そしてボエティウスが言うように「関係がペルソナの三性を生む」
 ので、ある人々は、神において本質とペルソナは、関係が補助的な存在である
 というかたちで異なると論じたが、それは、関係が、他のものに対してあるこ
 とだけを考察し、それが事物(res)であることを考察しなかったことによる。



\\

Sed, sicut supra ostensum est, sicut relationes in
rebus creatis accidentaliter insunt, ita in Deo sunt ipsa essentia
divina. 
Ex quo sequitur quod in Deo non sit aliud essentia quam persona
secundum rem; et tamen quod personae realiter ab invicem
distinguantur. 

&

しかし、前に示されたとおり\footnote{STI, q.28, a.2.}、被造物における関係
 は附帯的に内在するが、神における関係は神の本質そのものである。このこ
 とから、神において神の本質とペルソナが事物において別のものでないが、
 ペルソナは相互に区別されることが帰結する。

\\

Persona enim, ut dictum est supra, significat relationem,
prout est subsistens in natura divina. Relatio autem, ad essentiam
comparata, non differt re, sed ratione tantum, comparata autem ad
oppositam relationem, habet, virtute oppositionis, realem
distinctionem. Et sic remanet una essentia, et tres personae.

&

理由は以下の通り。ペルソナは、前に述
 べられたとおり\footnote{q.29, a.4.}、神の本性において自存するものとして
 の関係を表示する。
ところでこの関係と本質は実在的にではなく、たんに概念的に異なるだけで
 あるが、対立する関係とは、対立の力によって実在的な区別を有する。
そしてこのようにして、一つの本質と三つのペルソナという状態が保持され
 る。

\\




{\scshape Ad primum ergo dicendum} quod in creaturis non potest esse distinctio
suppositorum per relationes, sed oportet quod sit per essentialia
principia, quia relationes non sunt subsistentes in creaturis. In
divinis autem relationes sunt subsistentes, et ideo, secundum quod
habent oppositionem ad invicem, possunt distinguere supposita. Neque
tamen distinguitur essentia, quia relationes ipsae non distinguuntur ab
invicem secundum quod sunt realiter idem cum essentia.


&

第一異論に対しては、それゆえ以下のように言われるべきである。
被造物において個体が対立する場合、それは関係によることはなく、本質的な根源によっ
 てでなければならない。なぜなら、関係は被造物において自存するものでな
 いからである。しかし神において関係は自存する。それゆえ相互に対立す
 る限りで個体を区別することができる。しかし本質による区別ではない。
 なぜならこれらの関係は、それが実在的に本質と同一である限りにおいて、
 相互に区別されるのではないからである。


\\



{\scshape Ad secundum dicendum} quod, inquantum essentia et persona in divinis
differunt secundum intelligentiae rationem, sequitur quod aliquid possit
affirmari de uno, quod negatur de altero, et per consequens quod,
supposito uno, non supponatur alterum.


&

第二異論に対しては、以下のように言われるべきである。
神において、本質とペルソナが知的理解において異なるかぎりで、あるものが一
 方について肯定され、他方について否定されるということが生じる。
その結果、一方が一つであると想定されても他方も一つであると想定されるこ
 とはない。


\\



{\scshape Ad tertium dicendum quod} rebus divinis nomina imponimus secundum modum
rerum creatarum, ut supra dictum est. Et quia naturae rerum creatarum
individuantur per materiam, quae subiicitur naturae speciei, inde est
quod individua dicuntur {\itshape subiecta}, vel {\itshape supposita}, vel hypostases. Et
propter hoc etiam divinae personae supposita vel hypostases nominantur,
non quod ibi sit aliqua suppositio vel subiectio secundum rem.


&

第三異論に対しては、以下のように言われるべきである。
前に示されたとおり\footnote{STI, q.13, a.1, ad 2; a.3.}、私たちは被造
 物のしかたに即して神的諸事物に名を付ける。そして被造の諸事物の本性
 は質料によって個体化され、質料は種の本性のもとにあるので、個体が「基体
となるもの」「もとにあるもの」ないし「ヒュポスタシス」と言われる。このた
 め、神のペルソナも個体やヒュポスタシスと名付けられるが、そこに実在
 的になんらかの個体性や基体性があるわけではない。


\end{longtable}
\newpage






\rhead{a.~2}
\begin{center}
 {\Large {\bf ARTICULUS SECUNDUS}}\\
 {\large UTRUM SIT DICENDUM TRES PERSONAS ESSE {\itshape UNIUS ESSENTIAE}}\\
 {\footnotesize I {\itshape Sent.}, d.25, exposit.~text.; d.34, q.1, a.2.}\\
 {\Large 第二項\\三つのペルソナに一つの本質があると言われるべきか}
\end{center}

\begin{longtable}{p{21em}p{21em}}

{\Huge A}{\scshape d secundum sic proceditur}. Videtur quod non sit dicendum tres personas
esse unius essentiae. Dicit enim Hilarius, in libro {\itshape de Synod}., quod
pater et filius et spiritus sanctus {\itshape sunt quidem per substantiam tria,
per consonantiam vero unum}. Sed substantia Dei est eius essentia. Ergo
tres personae non sunt unius essentiae.


&

第二項の問題へ、議論は以下のように進められる。
三つのペルソナに一つの本質があると言われるべきでないと思われる。理由は以
 下の通り。ヒラリウスは『教会会議について』で、父と子と聖霊は「三つの
 実体だが協調によって一である」と述べている。しかし神の実体は神の本
 質である。ゆえに三つのペルソナに一つの本質があるわけではない。

\\



2 {\scshape Praeterea}, non est affirmandum aliquid de divinis, quod auctoritate
Scripturae sacrae non est expressum, ut patet per Dionysium, I cap. {\itshape de
Div.~Nom}. Sed nunquam in Scriptura sacra exprimitur quod pater et filius
et spiritus sanctus sunt unius essentiae. Ergo hoc non est asserendum.


&

さらに、ディオニュシウスの『神名論』第1章で明らかなとおり、聖書の権威に
 よって表明されていないことを神について肯定すべきでない。しかし聖書
 の中で、父と子と聖霊に一つの本質があるとは一度も表明されていない。ゆ
 えにこれは主張されるべきでない。

\\



3 {\scshape Praeterea}, natura divina est idem quod essentia. Sufficeret ergo dicere
quod tres personae sunt unius naturae.


&

さらに、神の本性はその本質と同一である。ゆえに三つのペルソナに一つの
本性があると言ったとしてもそれで足りたであろう。


\\



4 {\scshape Praeterea}, non consuevit dici quod persona sit essentiae, sed magis quod
essentia sit personae. Ergo neque convenienter videtur dici quod tres
personae sunt unius essentiae.


&

さらに、ペルソナに本質があると言われる習いはなく、むしろ本質がペルソ
 ナであると言われる。ゆえに三つのペルソナに一つの本質があると言われ
 るのは適切だとは思われない。

\\



5 {\scshape Praeterea}, Augustinus dicit quod non dicimus tres personas esse {\itshape ex una
essentia}, ne intelligatur in divinis aliud esse essentia et persona. Sed
sicut praepositiones sunt transitivae, ita et obliqui. Ergo, pari
ratione, non est dicendum quod tres personae sunt unius essentiae.

&

さらにアウグスティヌスは、神において本質とペルソナが別であると理解され
 ないように、私たちは三つのペルソナが「一つの本質から(ex)」あるとは言わないと述べている。しかし前置詞が他に向かうように、
 斜格もまた他に向かう。ゆえに同じ理由で、三つのペルソナに「一つの本質(斜格)」があると言われるべきでない。\footnote{ラテン
 語文法に基づいた論なので翻訳は至難。アウグスティヌスは``tres personae sunt ex una
 essentia.''と言うべきでないと書いているが、同じ理由で、``tres
 personae sunt unius personae.''とも言うべきでない。なぜなら、exも
 unius personaeという斜格(主格・呼格以外の格。この場合は属格)も他のものを含意するからである、との意。 }

\\

6 {\scshape Praeterea}, id quod potest esse erroris occasio, non est in divinis
dicendum. Sed cum dicuntur tres personae unius essentiae vel substantiae
datur erroris occasio. Quia, ut Hilarius dicit, in libro {\itshape de Synod}: {\itshape Una
substantia patris et filii praedicata, aut unum qui duas nuncupationes
habeat, subsistentem significat; aut divisam unam substantiam duas
imperfectas fecisse substantias; aut tertiam priorem substantiam, quae a
duobus et usurpata sit et assumpta}. Non est ergo dicendum tres personas
esse unius substantiae.


&

さらに、誤謬のきっかけとなることは神において語られるべきでない。しかし、
 一つの本質ないし実体をもつ三つのペルソナと言われるとき、誤謬のきっかけ
 が与えられる。というのも、ヒラリウスが『教会会議について』という書物の
 中で言うように、「父と子の一つの実体が述語されるとき、二つの名を持つ一
 つの自存するものを表示するのか、あるいは一つの実体が分割されて不完
 全な二つの実体となったのか、あるいはこの二つに使用され採用された先
 行する第三の実体を表示するのか」。ゆえに三つのペルソナに一つの実体が
 あると言われるべきでない。

\\



{\scshape Sed contra est} quod Augustinus dicit, in libro II {\itshape contra Maximinum}, quod
hoc nomen {\itshape homousion}, quod in Concilio Nicaeno adversus Arianos firmatum
est, idem significat quod tres personas esse unius essentiae.


&

しかし反対に、アウグスティヌスは『マクシムス駁論』第2巻で「ホモウーシ
 オン」というこの言葉は、ニケア公会議でアリウス派の人々に反対して定めら
 れたが、三つのペルソナに一つの本質があることと同じことを意味すると言っ
 ている。


\\



{\scshape Respondeo dicendum} quod, sicut supra dictum est, intellectus noster res
divinas nominat, non secundum modum earum, quia sic eas cognoscere non
potest; sed secundum modum in rebus creatis inventum. 


&

解答する。以下のように言われるべきである。
前に述べられたとおり、私たちの知性は、神的な事柄をそれらのあり方に即し
 てではなく、というのもそのようにそれらを認識することができないからだ
 が、むしろ被造物の中に見出されたあり方に即して名付ける。


\\


Et quia in rebus
sensibilibus, a quibus intellectus noster scientiam capit, natura
alicuius speciei per materiam individuatur; et sic natura se habet ut
forma, individuum autem ut suppositum formae, propter hoc etiam in
divinis, quantum ad modum significandi, essentia significatur ut forma
trium personarum. 

&

そして私たちの知性は可感的事物から知を受け取るが、その可感的事物におい
 て、特定の種に属する本性は質料によって個体化される。そして、そのようにして、
 本性は形相として、個体は形相の基体として関係し、このた
 め神においても、表示のしかたにかんする限りで、本質が三つのペルソナ
 の形相として表示される。



\\

Dicimus autem in rebus creatis formam quamcumque esse
eius cuius est forma; sicut sanitatem vel pulchritudinem hominis
alicuius. 
Rem autem habentem formam non dicimus esse formae, nisi cum
adiectione alicuius adiectivi, quod designat illam formam, ut cum
dicimus, {\itshape ista mulier est egregiae formae}, {\itshape iste homo est perfectae
virtutis}. 


&

ところで被造物において、どんな形相も、それがXの形相であるとき、そのXに
 属する。ちょうど健康性や美性が何らかの人間に属するように。
逆に私たちは、形相をもつ事物が形相に属するとは言わない。ただし、「この
女性は傑出した容姿を持つ」「この人は完全な徳を持つ」のように、何らか
 の形容詞が付加される場合を除いて\footnote{【文法】ラテン語の属格の用法。形
 容詞+名詞が属格となり、性質を表す。「性質の属格」``genitive of
 description''などと呼ばれる。}。

\\


Et similiter, quia in divinis, multiplicatis personis, non
multiplicatur essentia, dicimus {\itshape unam essentiam esse trium personarum}; et
{\itshape tres personas unius essentiae}, ut intelligantur isti genitivi construi
in designatione formae.


&

同様に神において、ペルソナが多数化されても本質は多数化されないので、「一
 つの本質が三つのペルソナに属する」と言う。そして、「三つのペルソナが一つ
 の本質に属する」と言う場合には、これらの属格\footnote{``tres personas
 unius essentiae''「一つの本質」が属格に
 なっている。}が形相を指示するのに用いられていると理解されるかぎりにおいてである。

\\




{\scshape Ad primum ergo dicendum} quod {\itshape substantia} sumitur pro {\itshape hypostasi}; et non
pro {\itshape essentia}.


&

第一異論に対しては、それゆえ、「実体」は、「ヒュポスタシス」の意味で理解されてい
 るのであり、「本質」の意味ではない、と言われるべきである。

\\



{\scshape Ad secundum dicendum} quod, licet tres personas esse unius essentiae non
inveniatur in sacra Scriptura per haec verba, invenitur tamen quantum ad
hunc sensum, sicut ibi, {\itshape ego et pater unum sumus}; et, {\itshape ego in patre, et
pater in me est}. Et per multa alia haberi potest idem.


&

第二異論に対しては、以下のように言われるべきである。
三つのペルソナが一つの本質に属するとしても、聖書の中に、これらの言葉によっ
 て見出されるわけではない。しかし、この意味にかんしては見出される。たと
 えば、「私と父は一つである」\footnote{「わたしと父とは一つである。」
 『ヨハネによる福音書』(10:30)}や「私は父において、父は私においてある」
 \footnote{「しかし、行っているのであれば、わたしを信じなくても、その業
 を信じなさい。そうすれば、父がわたしの内におられ、わたしが父の内にいる
 ことを、あなたたちは知り、また悟るだろう。」同(10:38)、「わたしが父の内
 におり、父がわたしの内におられることを、信じないのか。わたしがあなたが
 たに言う言葉は、自分から話しているのではない。わたしの内におられる父が、
 その業を行っておられるのである。」同(14:10)}。
 そして、他の多くの箇所によって同じことが理解されうる。


\\



{\scshape Ad tertium dicendum} quod, quia natura designat principium actus,
essentia vero ab essendo dicitur, possunt dici aliqua unius naturae,
quae conveniunt in aliquo actu, sicut omnia calefacientia, sed unius
essentiae dici non possunt, nisi quorum est unum esse. Et ideo magis
exprimitur unitas divina per hoc quod dicitur quod tres personae sunt
unius essentiae, quam si diceretur quod sunt unius naturae.


&

第三異論に対しては、以下のように言われるべきである。
本性は作用の根源を指示し、本質は「存在する」から言われるので、一つの何ら
 かの作用において一致するものどもが、一つの本性に属するとは言われうるが、
 たとえば全ての熱するものが(一つの本性に属するとは言われうるが)、一つの
 本質に属するとは言われ得ない。それらに一つの存在が属するのでない限り。
 ゆえに、三つのペルソナが一つの本質に属すると言われる方が、一つの本性に
 属すると言われた場合よりも、神の一性がよりよく表現される。


\\



{\scshape Ad quartum dicendum} quod forma, absolute accepta, consuevit significari
ut eius cuius est forma, ut {\itshape virtus Petri}. E converso autem, res habens
formam aliquam non consuevit significari ut eius, nisi cum volumus
determinare sive designare formam. 


&


第四異論に対しては、以下のように言われるべきである。
形相は、無条件的に理解された場合、それがXの形相であれば、Xに属するものと
 して表示されるのが通例である。たとえば、「ペトロの力」というように。逆
 に、ある形相をもつ事物は、その形相に属するものとしては、通常、表示され
 ない。ただし、私たちが、形相を限定したり指示したりすることを欲する場合
 を除いて。

\\

Et tunc requiruntur duo genitivi,
quorum unus significet formam, et alius determinationem formae, ut si
dicatur, {\itshape Petrus est magnae virtutis}, vel etiam requiritur unus genitivus
habens vim duorum genitivorum, ut cum dicitur, {\itshape vir sanguinum est iste},
idest effusor multi sanguinis. 


&

そして、その場合には二つの属格が必要とされるのであり、一つは形相を表示し、
 もう一つは形相の限定を表示する。たとえば、「ペトロには大きな力がある
 (直訳:ペトルスは大きな力に属している)」のように。あるいは、二つの属格の
 意味を持つ一つの属格が必要とされる場合もある。たとえば、「こいつは残酷
 な男だ(直訳:この男は複数の血に属する)」すなわち、多くの血を流す者、のように。


\\

Quia igitur essentia divina significatur
ut forma respectu personae, convenienter essentia personae dicitur. Non
autem e converso, nisi aliquid addatur ad designationem essentiae; ut si
dicatur quod pater est persona {\itshape divinae essentiae}, vel quod tres personae
sunt {\itshape unius essentiae}.


&

ゆえに、神の本質は、ペルソナに対して形相として表示されるので、「ペルソナの
 本質」と言われるのは適切である。しかし、逆は、本質の何らかの限定が加え
 られない限り、適切でない。たとえば、「父は、神の本質に属するペルソナで
 ある」や「三つのペルソナは一つの本質に属する」のように。


\\



{\scshape Ad quintum dicendum} quod haec praepositio ex vel de non designat
habitudinem causae formalis, sed magis habitudinem causae efficientis
vel materialis. Quae quidem causae in omnibus distinguuntur ab his
quorum sunt causae, nihil enim est sua materia, neque aliquid est suum
principium activum. 

&

第五異論に対しては、以下のように言われるべきである。
「〜基づいて(ex)」や「〜から(de)」という前置詞は、形相因の関係ではなく、
 作出因や質料因の関係を指示する。これらの原因は、あらゆるものにおいて、
 それがXの原因であるとき、そのXから区別される。自らの質料であったり、自
 らの能動的根源であったりするものはないからである。

\\



Aliquid tamen est sua forma, ut patet in omnibus
rebus immaterialibus. Et ideo per hoc quod dicimus tres personas unius
essentiae, significando essentiam in habitudine formae, non ostenditur
aliud esse essentia quam persona, quod ostenderetur, si diceremus tres
personas {\itshape ex eadem essentia}.

&

しかし、すべての非質料的事物において明らかなように、あるものは自らの形
 相である。ゆえに三つのペルソナが一つの本質に属する、と私たちが言うことによって、
 本質を形相の関係において表示するのであり、本質がペルソナとは別である
 ということが示されることはない。もし、三つのペルソナが「同一の本質か
 ら」あると私たちが言ったならば、そういうことが示されたであろうが。


\\




{\scshape Ad sextum dicendum} quod, sicut Hilarius dicit, in libro {\itshape de Synod}., {\itshape male
sanctis rebus praeiudicatur, si, quia non sanctae a quibusdam habentur,
esse non debeant}. Sic, si {\itshape male intelligitur homousion, quid ad me bene
intelligentem? Sit ergo una substantia ex naturae genitae proprietate,
non sit autem ex portione, aut ex unione, aut ex communione}.


&

第六異論に対しては、以下のように言われるべきである。
ヒラリウスが『教会の会議について』という書物で述べているように、「ある人々
 によって聖なるものと思われていないので、それが聖なるものでない場合があ
 ると判断したら、それは聖なるものにとってよい判断でない」。同様に、も
 し、「ホモウーシオンがよく理解されていなくても、よく理解する私に対し
 て、それが何だろうか。ゆえに、産まれた本性の固有性に基づいて一つの実体
 なのであり、部分や一性や共有に基づいてでない」。



\\


\end{longtable}
\newpage




\rhead{a.~3}
\begin{center}
 {\Large {\bf ARTICULUS TERTIUS}}\\
 {\large UTRUM NOMINA ESSENTIALIA PREDICENTUR\\SINGULARITER DE TRIBUS PERSONIS}\\
 {\footnotesize I {\itshape Sent.}, d.9, q.1, a.2.}\\
 {\Large 第三項\\本質的名称は三つのペルソナに単数で述語されるか}
\end{center}

\begin{longtable}{p{21em}p{21em}}



{\Huge A}{\scshape d tertium sic proceditur}. Videtur quod nomina essentialia, ut hoc nomen
{\itshape Deus}, non praedicentur singulariter de tribus personis, sed
pluraliter. Sicut enim {\itshape homo} significatur ut {\itshape habens humanitatem}, ita {\itshape Deus}
significatur ut {\itshape habens deitatem}. Sed tres personae sunt tres habentes
deitatem. Ergo tres personae sunt {\itshape tres dii}.


&

第三項の問題へ、議論は以下のように進められる。「神」のような本質的名称は、
 三つのペルソナについて単数ではなく複数で述語されるべきである。理由は以
 下の通り。「人間」が「人間性を持つもの」として表示されるように、「神」
 は「神性を持つもの」として表示される。しかし三つのペルソナは神性を
 もつ三つのものである。ゆえに三つのペルソナは「三つの神々」である。


\\



2 {\scshape Praeterea}, {\itshape Gen}.~{\scshape i}, ubi dicitur: {\itshape In principio creavit Deus caelum et
terram}, Hebraica veritas habet {\itshape Elohim}, quod potest interpretari {\itshape dii},
sive {\itshape iudices}. Et hoc dicitur propter pluralitatem personarum. Ergo tres
personae sunt {\itshape plures dii}, et non {\itshape unus Deus}.


&

さらに、「始めに神は天と地を作った」とある『創世記』第1章は、ヘブライ語
 の原文ではElohimとあり、これは「神々」または「裁判官たち」と訳されうる。
 しかしこれは、ペルソナの複数性のために言われている。ゆえに、三つのペル
 ソナは「複数の神々」であって「一つの神」ではない。


\\



3 {\scshape Praeterea}, hoc nomen res, cum absolute dicatur, videtur ad substantiam
pertinere. Sed hoc nomen pluraliter praedicatur de tribus personis,
dicit enim Augustinus, in libro {\itshape de Doctr. Christ}: {\itshape Res quibus fruendum
est, sunt pater et filius et spiritus sanctus}. Ergo et alia nomina
essentialia pluraliter praedicari possunt de tribus personis.


&

さらに、「事物」というこの名は、無条件的に言われたときには、実体に属する
 と思われる。しかし、この名は、三つのペルソナについて複数で述語付けられ
 る。実際、アウグスティヌスは『キリスト教の教え』という書物の中で、「享
 受されるべき諸事物は、父、子、聖霊である」と述べている。ゆえに、他の本
 質的名称もまた、三つのペルソナについて複数で述語されうる。


\\



4 {\scshape Praeterea}, sicut hoc nomen {\itshape Deus} significat habentem deitatem, ita hoc
nomen {\itshape persona} significat subsistentem in natura aliqua
intellectuali. Sed dicimus {\itshape tres personas}. Ergo, eadem ratione, dicere
possumus {\itshape tres deos}.


&

さらに、「神」というこの名が神性を持つものを表示するように、「ペルソナ」
 は、何らかの知性的本性において自存するものを表示する。
しかし、私たちは「三つのペルソナたち」と言う。ゆえに、同じ理由で、「三つの
 神々」と言うことができる。


\\



{\scshape Sed contra est} quod dicitur {\itshape Deut}.~{\scshape vi}: {\itshape Audi, Israel, dominus Deus tuus,
Deus unus est}.


&

しかし反対に、『申命記』第6章に「聞け、イスラエルよ、あなたの主人である
 神、神は一である」\footnote{「聞け、イスラエルよ。我らの神、主は唯一の
 主である。 」(6:4)}と言われている。



\\



{\scshape Respondeo dicendum} quod nominum essentialium quaedam significant
essentiam substantive, quaedam vero adiective. Ea quidem quae
substantive essentiam significant, praedicantur de tribus personis
singulariter tantum, et non pluraliter, quae vero adiective essentiam
significant, praedicantur de tribus personis in plurali. 


&

解答する。以下のように言われるべきである。本質的名称のあるものは、本質を
名詞として表示し、あるものは形容詞として表示する。名詞として本質を表示す
 るものは、三つのペルソナに、単数でのみ述語付けられ、複数では述語付けら
 れない。他方、形容詞として本質を表示するものは、三つのペルソナについて、
 複数で述語付けられる。

\\


Cuius ratio
est, quia nomina substantiva significant aliquid per modum substantiae,
nomina vero adiectiva significant aliquid per modum accidentis, quod
inhaeret subiecto. Substantia autem, sicut per se habet esse, ita per se
habet unitatem vel multitudinem, unde et singularitas vel pluralitas
nominis substantivi attenditur secundum formam significatam per
nomen. 



&

その理由は、名詞はあるものを実体のあり方で表示するのに対して、形容詞はあ
 るものを附帯性のあり方で表示する。そして附帯性は基体に内在する。しかし、
 実体は、ちょうどそれ自身によって存在を持つように、それ自身によって一性
 と多性をもつ。したがって、名詞の単数と複数も、その名によって表示された
 形相に即して見出される。


\\


Accidentia autem, sicut esse habent in subiecto, ita ex subiecto
suscipiunt unitatem et multitudinem, et ideo in adiectivis attenditur
singularitas et pluralitas secundum supposita. 

&

これに対して、附帯性は基体において存在をもつように、基体から一性と多
 性を受け取る。ゆえに形容詞において、単数と複数は個体に即して見出さ
 れる。


\\

In creaturis autem non
invenitur una forma in pluribus suppositis nisi unitate ordinis, ut
forma multitudinis ordinatae. 



&

ところで、被造物において、一つの形相が複数の個体において見出されるという
 ことはない。ただし、秩序の一性、たとえば、秩序付けられた多くのものに一つの形相が属す
 る場合を別とすれば。



\\


Unde nomina significantia talem formam, si
sint substantiva, praedicantur de pluribus in singulari, non autem si
sint adiectiva. Dicimus enim quod multi homines sunt {\itshape collegium} vel
{\itshape exercitus} aut {\itshape populus}, dicimus tamen quod plures homines sunt
{\itshape collegiati}. 



&

したがって、そのような形相を表示する名称は、それが名詞であれば、複数のも
 のに単数で述語付けられるが、形容詞であれば、そうでない。たとえば、私た
 ちは、多くの人間が、同僚(collegium)、軍隊、人民(単数)であると言うが、しかし、複数の人間
 は、同僚たち(collegiati)であると言う。


\\


In divinis autem essentia divina significatur per modum
formae, ut dictum est. Quae quidem simplex est et maxime una, ut supra
ostensum est. Unde nomina significantia divinam essentiam substantive,
singulariter, et non pluraliter, de tribus personis praedicantur. 



&

しかし、すでに述べられたとおり、神において、神の本質は形相として表示され
 る。前に示されたとおり、それは、実際、単純で、最大限に一である。したがっ
 て、神の本質を名詞として表示する名は、複数ではなく単数で、三つのペルソ
 ナに述語付けられる。



\\



Haec
igitur est ratio quare Socratem et Platonem et Ciceronem dicimus tres
homines; patrem autem et filium et spiritum sanctum non dicimus tres
deos, sed unum Deum, quia in tribus suppositis humanae naturae sunt tres
humanitates; in tribus autem personis est una divina essentia. 

&

ゆえに、このことが、私たちが、ソクラテスとプラトンとキケロは三人の人間だと言う理由で
 ある。しかし、父と子と聖霊は、三つの神々とは言わず、一つの神と言う。な
 ぜなら、三つの個体において、人間の本性は三つの人間性だが、三つのペルソ
 ナにおいて、一つの神の本質があるからである。



\\


Ea vero
quae significant essentiam adiective, praedicantur pluraliter de tribus,
propter pluralitatem suppositorum. Dicimus enim {\itshape tres existentes} vel {\itshape tres
sapientes}, aut {\itshape tres aeternos} et {\itshape increatos} et {\itshape immensos}, si adiective
sumantur. Si vero substantive sumantur, dicimus {\itshape unum increatum}, {\itshape immensum}
et {\itshape aeternum}, ut Athanasius dicit.


&

他方、本質を形容詞として表示する名は、個体の複数性のために、三つについて
 複数で述語付けられる。たとえば、三つの存在する者たち、三つの知恵ある
 者たち、三つの永遠な者たち、創造されざる者たち、計り知れない者たちと言うが、これは
 形容詞として理解された場合である。しかし、名詞として理解される場合には、
 私たちは、一つの創造されざる者、計り知れない者、永遠な者、と言う。これ
 は、アタナシウスが言うようにである。


\\



{\scshape Ad primum ergo dicendum} quod, licet {\itshape Deus} significet {\itshape habentem deitatem},
est tamen alius modus significandi, nam {\itshape Deus} dicitur substantive, sed
{\itshape habens deitatem} dicitur adiective. Unde, licet sint {\itshape tres habentes
deitatem}, non tamen sequitur quod sint {\itshape tres dii}.


&

第一異論に対しては、それゆえ、以下のように言われるべきである。
「神」は「神性を持つもの」を表示するが、表示のしかたは異なる。すなわち、
 「神」は名詞として、「神性を持つもの」は形容詞として語られる。したがっ
 て、「神性を持つもの」は三つでも、「三つの神」が帰結するわけではない。


\\



{\scshape Ad secundum dicendum} quod diversae linguae habent diversum modum
loquendi. Unde, sicut propter pluralitatem suppositorum Graeci dicunt
{\itshape tres hypostases}, ita et in Hebraeo dicitur pluraliter Elohim. Nos autem
non dicimus pluraliter neque {\itshape deos} neque {\itshape substantias}, ne pluralitas ad
substantiam referatur.


&

第二異論に対しては、以下のように言われるべきである。
さまざまな言語にはさまざまな語り方がある。それゆえ、個体の複数性のために、
 ギリシャ人たちが「三つのヒュポスタシス」と言うように、ヘブライ人たちの
 もとでは、Elohimが複数で語られる。しかし私たちは、複数性が実体へ関係し
 ないように、「神々」や「諸実体」のように複数では語らない。


\\



{\scshape Ad tertium dicendum} quod hoc nomen {\itshape res} est de transcendentibus. Unde,
secundum quod pertinet ad relationem, pluraliter praedicatur in divinis,
secundum vero quod pertinet ad substantiam, singulariter
praedicatur. Unde Augustinus dicit ibidem quod {\itshape eadem Trinitas quaedam
summa res est}.


&

第三異論に対しては、以下のように言われるべきである。
「事物」というこの名は超越的なものに属する。それゆえ、関係に属するかぎり
 では、神において複数で述語付けられるが、実体に属する限りでは、単数で述
 語付けられる。
このことから、アウグスティヌスは、同箇所で「同じ三位一体が、一種の最高の
 事物(単数)である」と述べている。


\\



{\scshape Ad quartum dicendum} quod forma significata per hoc nomen {\itshape persona}, non
est essentia vel natura, sed {\itshape personalitas}. Unde, cum sint tres
personalitates, idest tres personales proprietates, in patre et filio et
spiritu sancto, non singulariter, sed pluraliter praedicatur de tribus.


&


第四異論に対しては、以下のように言われるべきである。「ペルソナ」というこ
 の名によっては、本質や本性は表示されず、「ペルソナ性」が表示される。し
 たがって、三つのペルソナ性、つまり三つのペルソナ的固有性が、父と子と聖
 霊の中に存在するので、単数ではなく複数で、三つのものについて述語付けら
 れる。


\end{longtable}
\newpage





\rhead{a.~4}
\begin{center}
 {\Large {\bf ARTICULUS QUARTUS}}\\
 {\large UTRUM NOMEN ESSENTIALIA CONCRETA POSSINT SUPPONERE PRO PERSONA}\\
 {\footnotesize I {\itshape Sent.}, d.4, q.1, a.2; d.5, q.1, a.2.}\\
 {\Large 第四項\\具体的な本質的名称はペルソナを代示しうるか}
\end{center}

\begin{longtable}{p{21em}p{21em}}





{\Huge A}{\scshape d quartum sic proceditur}. Videtur quod nomina essentialia concretiva
non possunt supponere pro persona, ita quod haec sit vera, {\itshape Deus genuit
Deum}. Quia, ut sophistae dicunt, {\itshape terminus singularis idem significat et
supponit}. Sed hoc nomen Deus videtur esse terminus singularis, cum
pluraliter praedicari non possit, ut dictum est. Ergo, cum significet
essentiam, videtur quod supponat pro essentia, et non pro persona.

&

第四項の問題へ議論は以下のように進められる。
具体的な本質的名称はペルソナを代示することができず、従って「神は神
を生んだ」は真でないと思われる。
理由は以下の通り。
ソフィストたちが言うように「単称名は同一のものを表示し、かつ代示する」。
 しかし、「神」というこの名は、すでに述べられたとおり複数形で述語付け
 られないのだから、単称名だと思われる。ゆえにそれは本質を意味するの
 で、ペルソナではなく本質を代示すると思われる。



\\



2 {\scshape Praeterea}, terminus in subiecto positus non restringitur per terminum
positum in praedicato, ratione significationis; sed solum ratione
temporis consignificati. Sed cum dico, {\itshape Deus creat}, hoc nomen Deus
supponit pro essentia. Ergo cum dicitur, {\itshape Deus genuit}, non potest iste
terminus {\itshape Deus}, ratione praedicati notionalis, supponere pro persona.

&

さらに、主語に置かれた語は、意味の点で、述語に置かれた語によって制約
 されず、たんに、ともに示された時間の点でのみ制約される。しかし「神が
 創造する」と私が言うとき、「神」というこの名は本質を代示する。ゆえに、
 「神が生んだ」と言われるとき、「神」というこの語が、識標的な述語の点
 で、ペルソナを代示することはできない。


\\



3 Praeterea, si haec est vera, {\itshape Deus genuit}, quia pater generat; pari
ratione haec erit vera, {\itshape Deus non generat}, quia filius non generat. Ergo
est Deus generans, et Deus non generans. Et ita videtur sequi quod sint
{\itshape duo dii}.

&

さらに、もし、父が生むがゆえに「神が生んだ」が真であるならば、子は生まない
 ので「神は生まない」も真である。ゆえに、生む神と生まない神が存在する。
 そしてこのことから、「二人の神」がいることが帰結する。
 
\\



4 {\scshape Praeterea}, si {\itshape Deus genuit Deum}, aut se Deum, aut alium Deum. Sed non se
Deum, quia, ut Augustinus dicit, in I {\itshape de Trin}., {\itshape nulla res generat
seipsam}. Neque alium Deum, quia non est nisi unus Deus. Ergo haec est
falsa, {\itshape Deus genuit Deum}.

&

さらに、もし「神が神を生んだ」のであれば、自分自身である神を生んだのか、
 自分とは違う神を生んだのかのどちらかである。しかし、自分自身である神で
 はない。なぜなら、アウグスティヌスが『三位一体論』第1巻で述べるように、
 「どんなものも自分自身を生まない」からである。また、自分とは違う神でも
 ない。なぜなら、神は一人しかいないからである。ゆえに、「神が神を生んだ」
 は偽である。


\\



5 {\scshape Praeterea}, si {\itshape Deus genuit Deum}, aut Deum qui est Deus pater, aut Deum
qui non est Deus pater. Si Deum qui est Deus pater, ergo Deus pater est
genitus. Si Deum qui non est Deus pater, ergo Deus est qui non est Deus
pater, quod est falsum. Non ergo potest dici quod {\itshape Deus genuit Deum}.

&

さらに、もし「神が神を生んだ」のであれば、父なる神を生んだのか、父でない神
 を生んだのかのどちらかである。もし、父である神を生んだのであれば、父で
 ある神は生み出されたことになる。もし、父でない神を生んだのであれば、神
 は、父でない神であることになり、これは偽である。ゆえに、「神が神を生ん
 だ」と言われることはできない。


\\



{\scshape Sed contra est} quod in Symbolo dicitur {\itshape Deum de Deo}.

&

しかし反対に、ニケア信教に、「神は神から」と言われている。

\\



{\scshape Respondeo dicendum} quod quidam dixerunt quod hoc nomen \textit{Deus}, et similia,
proprie secundum suam naturam supponunt pro essentia, sed ex adiuncto
notionali trahuntur ad supponendum pro persona. Et haec opinio
processisse videtur ex consideratione divinae simplicitatis, quae
requirit quod in Deo idem sit habens et quod habetur, et sic \textit{habens
deitatem}, quod significat hoc nomen \textit{Deus}, est idem quod deitas. 


&

解答する。以下のように言われるべきである。
ある人々はこの「神」という名称やそれに似た名称は、厳密には自らの本性において本質を代示するが、識標的なものが結び付けられることでペルソナを代示する方へ引き寄せられると言った。
この意見は神の単純性の考察から出てきたように思われる。なぜなら単純性は神において持つものと持たれるものが同一であることを要求するので、「神」というこの名称が表示する「神性を持つもの」が神性と同一であることになるからである。



\\


Sed in
proprietatibus locutionum, non tantum attendenda est res significata;
sed etiam modus significandi. Et ideo, quia hoc nomen \textit{Deus} significat
divinam essentiam ut in habente ipsam, sicut hoc nomen \textit{homo} humanitatem
significat in supposito; alii melius dixerunt quod hoc nomen \textit{Deus} ex
modo significandi habet ut proprie possit supponere pro persona, sicut
et hoc nomen \textit{homo}. 


&

しかし語りの特性の中ではたんに表示された事物だけでなく表示のしかたもまた注意されるべきである。
ゆえにこの「神」という名称は、ちょうど「人間」というこの名称が個体の中に人間性を表示するように、
本質を持つものの中に神の本質を表示する。
別の人々は、よりよいしかたで、この「神」という名称は表示のしかたに基づいて、「人間」というこの名称と同様に、厳密にはペルソナを示すことができると言った。

\\


Quandoque ergo hoc nomen Deus supponit pro essentia,
ut cum dicitur, \textit{Deus creat}, quia hoc praedicatum competit subiecto
ratione formae significatae, quae est deitas. Quandoque vero supponit
personam, vel unam tantum, ut cum dicitur, \textit{Deus generat}; vel duas, ut
cum dicitur \textit{Deus spirat}; vel tres, ut cum dicitur, \textit{regi saeculorum
immortali, invisibili, soli Deo etc}., I \textit{Tim}.\textsc{i}.

&

それゆえ、この「神」という名称は、「神が創造する」と言われるときのように、本質を示すこともある。というのも、この述語は表示された形相を根拠に主語に適合しているが、その形相とは神性だからである。
またペルソナを示すこともあるが、「神が生む」と言われるときのようにただ一つのペルソナの場合や、「神が霊発する」と言われるときのように二つのペルソナの場合、そして「世々の王、不滅で不可視で唯一の神に」\footnote{永遠の王、不滅で目に見えない唯一の神に、誉れと栄光が世々限りなくありますように、アーメン。(1:17)}『テモテへの手紙一』第1章と言われるときのように三つのペルソナの場合がある。

\\


\textsc{Ad primum ergo dicendum} quod hoc nomen \textit{Deus}, licet conveniat cum
terminis singularibus in hoc, quod forma significata non multiplicatur;
convenit tamen cum terminis communibus in hoc, quod forma significata
invenitur in pluribus suppositis. Unde non oportet quod semper supponat
pro essentia quam significat.

&

第一異論に対してはそれゆえ以下のように言われるべきである。
「神」というこの名称は、表示された形相が多数化されないという点で単称名と適合的であるが、表示された形相が複数の個体に見出される点で一般名と適合的である。このことから、常にそれが表示する本質を示さないといけないわけではない。

\\


\textsc{Ad secundum dicendum} quod obiectio illa procedit contra illos qui
dicebant quod hoc nomen Deus non habet naturalem suppositionem pro
persona.

&

第二異論に対しては以下のように言われるべきである。
かの反論は「神」というこの名称が本性的にペルソナを示すことがないと言った人々に反対して論じられている。
\\


\textsc{Ad tertium dicendum quod} aliter se habet hoc nomen Deus ad supponendum
pro persona, et hoc nomen homo. Quia enim forma significata per hoc
nomen homo, idest humanitas, realiter dividitur in diversis suppositis,
per se supponit pro persona; etiamsi nihil addatur quod determinet ipsum
ad personam, quae est suppositum distinctum. 



&

第三異論に対しては以下のように言われるべきである。
ペルソナを代示するためのこの「神」という名称は「人間」という名称と異なる。
なぜなら、この「人間」という名称によって表示された形相すなわち人間性は、異なる個体において実在的に分割されているので、自体的にペルソナを代示する。それ自身をペルソナすなわち区別された個体へ限定するものは何も加えないにしても。

\\

Unitas autem sive
communitas humanae naturae non est secundum rem, sed solum secundum
considerationem, unde iste terminus homo non supponit pro natura
communi, nisi propter exigentiam alicuius additi, ut cum dicitur, homo
est species. 



&

これに対して、人間本性の一性や共通性は事物においてあるのではなく、たん
に考察においてのみある。このことから、「人間」というこの語は、「人間は
種である」と言われる場合のように、何らかの加えられた必要性のためにでな
ければ、共通本性を代示しない。

\\


Sed forma significata per hoc nomen Deus, scilicet essentia
divina, est una et communis secundum rem. Unde per se supponit pro
natura communi, sed ex adiuncto determinatur eius suppositio ad
personam. Unde cum dicitur, Deus generat, ratione actus notionalis
supponit hoc nomen Deus pro persona patris. 

&

しかし「神」というこの名称によって表示された形相、すなわち神の本質は、
実在において一でありかつ共通である。したがって、自体的に共通本性を代示
するが、しかし結び付けられたものに基づいてその代示がペルソナへ限定され
る。したがって、「神が生む」と言われるとき、識標的作用のために「神」と
いうこの名称が父のペルソナを代示する。

\\

Sed cum dicitur, Deus non
generat, nihil additur quod determinet hoc nomen ad personam filii, unde
datur intelligi quod generatio repugnet divinae naturae. Sed si addatur
aliquid pertinens ad personam filii, vera erit locutio; ut si dicatur,
Deus genitus non generat. 


&

しかし「神は生まない」と言われるとき、この名称を息子のペルソナへ限定す
 るものは何も加えられていないので、生むことは神の本性に反すると理解さ
 れることが許される。しかしもし、何か息子のペルソナに属することが付加
 されるならば、「生まれた神は生まない」と言われているかのようにして、
 この言説は真となるであろう。

\\



Unde etiam non sequitur, est Deus generans et
est Deus non generans, nisi ponatur aliquid pertinens ad personas; ut
puta si dicamus, pater est Deus generans, et filius est Deus non
generans. Et ita non sequitur quod sint plures dii, quia pater et filius
sunt unus Deus, ut dictum est.

&

従って、ペルソナに属することが置かれないかぎり、生む神と生まない神が存
 在するということも帰結しない。そして、複数の神がいることもまた帰結し
 ない。なぜなら、すでに述べられたとおり父と息子は一人の神だかである。

\\



{\scshape Ad quartum dicendum} quod haec est falsa, \textit{pater genuit se Deum}, quia ly
\textit{se}, cum sit reciprocum, refert idem suppositum. Neque est contrarium
quod Augustinus dicit, \textit{ad Maximum}, quod \textit{Deus pater genuit alterum
se}. Quia ly \textit{se} vel est casus ablativi; ut sit sensus, \textit{genuit alterum a
se}. Vel facit relationem simplicem, et sic refert identitatem naturae,
sed est impropria vel emphatica locutio, ut sit sensus, \textit{genuit alterum
simillimum sibi}. 


&

第四異論に対しては以下のように言われるべきである。
「父が自分である神を生んだ」は偽であり、それは「自分」が再帰的なので同一の個体に関係するからである。
また、これはアウグスティヌスが『マクシムス宛書簡』で「父なる神がalterum
seを生んだ」と言っていることに反しない。
なぜならこのseは、奪格でその意味が「自分と違う者を生んだ」となるか、あるいは単純な関係を表していて、本性の同一性に言及しているかのいずれかだからである。
しかし、この後者は厳密でないあるいは強調した表現であって、その意味は「自分に似た他の者を生んだ」となる。\footnote{Pater
genuit alterum se.
という文がseの解釈によって異なる意味に読めるということであり、訳出は不可能。}

\\



Similiter et haec est falsa, \textit{genuit alium Deum}. Quia
licet filius sit \textit{alius} a patre, ut supra dictum est, non tamen est
dicendum quod sit \textit{alius Deus}, quia intelligeretur quod hoc adiectivum
\textit{alius} poneret rem suam circa substantivum quod est \textit{Deus}; et sic
significaretur distinctio deitatis. 


&

同様に、「他の神を生んだ」という命題も偽である。なぜなら、前に述べられたとおり息子は父と「別の人」だが、「別の神」と言われるべきではない。なぜなら「別の」という形容詞が「神」という名詞をめぐって自分が所有する事物を措定しているように解釈されただろうから。そのようにして、神性の区別が表示されたであろう。

\\


Quidam tamen concedunt istam, \textit{genuit
alium Deum}, ita quod ly \textit{alius} sit substantivum, et ly \textit{Deus} appositive
construatur cum eo. Sed hic est improprius modus loquendi, et evitandus,
ne detur occasio erroris.

&

しかしある人々は「他の神を生んだ」を認めている。それはこの「他の」が名詞であり「神」はそれと同格に置かれているとする。しかしこれは語り方が適切でなく誤謬のきっかけとなるので避けるべきである。


\\



{\scshape Ad quintum dicendum} quod haec est falsa, \textit{Deus genuit Deum qui est Deus
pater}, quia, cum ly \textit{pater} appositive construatur cum ly \textit{Deus}, restringit
ipsum ad standum pro persona patris; ut sit sensus, \textit{genuit Deum qui est
ipse pater}, et sic pater esset genitus, quod est falsum. Unde negativa
est vera, \textit{genuit Deum qui non est Deus pater}. 



&

第五異論に対しては以下のように言われるべきである。
「神が父である神を生んだ」は偽である。なぜなら、この「父」は「神」と同格的に使われているので、父のペルソナを表示するようにそれ自身を制限するからであり、その意味は「父自身である神を生んだ」となり、かくして父が生まれたも野となってしまい、これは偽である。
従って、この否定の命題、つまり「父でない神を生んだ」が真である。

\\


Si tamen intelligeretur
constructio non esse appositiva, sed aliquid esse interponendum; tunc e
converso affirmativa esset vera, et negativa falsa; ut sit sensus,
\textit{genuit Deum qui est Deus qui est pater}. Sed haec est extorta
expositio. Unde melius est quod simpliciter affirmativa negetur, et
negativa concedatur. 


&

しかし、同格ではなく、間に何かが入る構文だと理解されたとしたら、その場合には逆に、
肯定命題が真であり否定命題が偽であっただろう。つまりその意味は、
「父である神、その神を生んだ」となろう。
しかしこれは破格の表現である。
したがって、端的に肯定命題が否定され否定命題が認められる方がよい。

\\


Praepositivus tamen dixit quod tam negativa quam
affirmativa est falsa. Quia hoc relativum qui in affirmativa potest
referre suppositum, sed in negativa refert et significatum et
suppositum. Unde sensus affirmativae est, quod esse Deum patrem
conveniat personae filii. 


&

しかし、プラエポシティーウスは肯定命題と同様に否定命題も偽だと述べた。
なぜなら、肯定命題の中の関係的名称は個体に関係しうるが、否定命題においては
表示されたものと個体に関係しうるからである。
したがって肯定命題の意味は、父なる神であることが息子のペルソナに適合することになる。

\\



Negativae vero sensus est, quod esse Deum
patrem non tantum removeatur a persona filii, sed etiam a divinitate
eius sed hoc irrationabile videtur, cum, secundum philosophum, de eodem
de quo est affirmatio, possit etiam esse negatio.

&

他方、否定命題の意味は、父なる神であることが息子のペルソナからだけでなく息子の神性からも排除されるが、これは不合理だと思われる。なぜなら、哲学者によれば、それについて肯定がある同じものについて、否定もまたありうるからである。

\end{longtable}
\newpage


\rhead{a.~5}
\begin{center}
 {\Large {\bf ARTICULUS QUINTUS}}\\
 {\large UTRUM NOMINA ESSENTIALIA IN ABSTRACTO SIGNIFICATA\\POSSINT SUPPONERE PRO PERSONA}\\
 {\footnotesize I \textit{Sent.}, d.5, q.1, a.1, 2; \textit{De Un.~Verb.}, a.1, ad 12; \textit{Contra errores Graec.}, cap.~\textsc{iv}; In \textit{Decretal.~II}.}\\
 {\Large 第五項\\抽象的に表示された本質的名称はペルソナを代示しうるか}
\end{center}

\begin{longtable}{p{21em}p{21em}}

\textsc{Ad quintum sic proceditur}. Videtur quod nomina essentialia in
abstracto significata possint supponere pro persona, ita quod haec sit vera,
essentia generat essentiam. Dicit enim Augustinus, VII \textit{de Trin}., \textit{pater et
filius sunt una sapientia, quia una essentia; et singillatim sapientia de
sapientia, sicut essentia de essentia}.


&

第五項の問題へ議論は以下のように進められる。
「本質が本質を生む」のように、抽象的に表示された本質的名称はペルソナを代示しうると思われる。
理由は以下の通り。
アウグスティヌスは『三位一体論』第7巻の中で以下のように述べている。
「父と息子は一つの知恵である。なぜなら彼らは一つの本質であるから。
また一人一人が本質からの本質であるように知恵からの知恵である」。

\\

2 \textsc{Praeterea}, generatis nobis vel corruptis, generantur vel
corrumpuntur ea quae in nobis sunt. Sed filius generatur. Ergo, cum essentia
divina sit in filio, videtur quod essentia divina generetur.


&

さらに、私たちが生成消滅するとき、私たちの中にあるものも生成消滅する。
しかるに息子は生成する。ゆえに、息子の中には神の本質があるので、神の本質は生成すると思われる。

\\

3 \textsc{Praeterea}, idem est Deus et essentia divina, ut ex supra dictis
patet. Sed haec est vera, \textsc{Deus generat Deum}, sicut dictum est. Ergo haec est
vera, \textsc{essentia generat essentiam}.


&

さらに、すでに述べられたことから明らかなとおり、神と神の本質は同一である。
しかるにすでに述べられたとおり「神が神を生む」は真である。
ゆえに「本質が本質を生む」も真である。

\\

4 \textsc{Praeterea}, de quocumque praedicatur aliquid, potest supponere pro
illo. Sed essentia divina est pater. Ergo essentia potest supponere pro persona
patris. Et sic essentia generat.


&

さらに、或るものがどんなものに述語付けられようとも、そのものを代示しうる。
しかるに父は神の本質である。ゆえに本質は父のペルソナを代示しうる。
かくして、本質が生む。

\\

5 \textsc{Praeterea}, essentia est res generans, quia est pater, qui est
generans. Si igitur essentia non sit generans, erit essentia res generans et
non generans, quod est impossibile.


&

さらに、本質は、父であり父は生むものなので、生む事物である。
ゆえにもし本質が生むものでないならば、本質は生みかつ生まない事物であることになる。
これは不可能である。

\\

6 \textsc{Praeterea}, Augustinus dicit, in IV \textit{de Trin}., \textit{pater
est principium totius deitatis}. Sed non est principium nisi generando vel
spirando. Ergo pater generat vel spirat deitatem.


&

さらにアウグスティヌスは『三位一体論』第6巻で「父は神性全体の根源である」と述べている。
しかるに生むことないし霊発することによるのでなければ根源であることはない。
ゆえに父は神性を生む、あるいは霊発する。


\\

\textsc{Sed contra est} quod Augustinus dicit, in I \textit{de Trin}., quod
\textit{nulla res generat seipsam}. Sed si essentia generat essentiam, non
generat nisi seipsam, cum nihil sit in Deo, quod distinguatur a divina
essentia. Ergo essentia non generat essentiam.


&

しかし反対に、アウグスティヌスは『三位一体論』第1巻で「どんな事物も自分自身を生まない」と述べている。
しかるに、もし本質が本質を生むなら、それは自分自身を生むことに他ならない。なぜなら神の中に神の本質から区別されるものは何もないからである。ゆえに本質は本質を生まない。

\\


\textsc{Respondeo dicendum} quod circa hoc erravit Abbas Ioachim, asserens quod, sicut
dicitur, \textit{Deus genuit Deum}, ita potest dici quod \textit{essentia genuit essentiam};
considerans quod, propter divinam simplicitatem, non est aliud Deus quam divina
essentia. 


&

解答する。以下のように言われるべきである。
このことをめぐって、アッバス・ヨーアヒムは「神が神を生んだ」と言われるように「本質が本質を生んだ」とも言われうると述べて誤った。彼は神の単純性のために、神は神の本質と別でないと考えたのである。

\\



Sed in hoc deceptus fuit, quia ad veritatem locutionum, non solum
oportet considerare res significatas, sed etiam modum significandi ut dictum
est. Licet autem, secundum re, sit idem Deus quod deitas, non tamen est idem
modus significandi utrobique. 

&

しかし彼が欺かれたのは以下の点である。すなわち、語りの真理のためには表示される事物だけでなく、すでに述べられたとおり、表示のしかたもまた考察しなければならない。
事物において、神が神性と同じであるのはたしかだが、しかし両方に同じ表示のしかたが属しているわけではない。

\\



Nam hoc nomen \textit{Deus}, quia significat divinam
essentiam ut in habente, ex modo suae significationis naturaliter habet quod
possit supponere pro persona, et sic ea quae sunt propria personarum, possunt
praedicari de hoc nomine \textit{Deus}, ut dicatur quod \textit{Deus est genitus} vel \textit{generans},
sicut dictum est. 

&

すなわちこの「神」という名称は、持つ者として神の本質を表示するので、すの表示のしかたから自然にペルソナを代示することができる。そしてペルソナに固有であることがらは、「神」というこの名称に述語付けられうる。
たとえばすでに述べられたとおり、「神は生み出された者である」あるいは「神は生む者である」のように。

\\



Sed hoc nomen essentia non habet ex modo suae significationis
quod supponat pro persona, quia significat essentiam ut formam abstractam. Et
ideo ea quae sunt propria personarum, quibus ab invicem distinguuntur, non
possunt essentiae attribui, significaretur enim quod esset distinctio in
essentia divina, sicut est distinctio in suppositis.


&

しかしこの「本質」という名称はその表示のしかたからペルソナを代示することはできない。
なぜならそれは本質を抽象的な形相として表示するからである。
そしてそれゆえに、それによって相互に区別されるところのペルソナに固有の事柄は
本質に帰せられえない。もしも、個体における区別のように、神の本質において区別があったならば、
それは表示されたであろうが。

\\


\textsc{Ad primum ergo dicendum} quod, ad exprimendam unitatem essentiae et personae,
sancti doctores aliquando expressius locuti sunt quam proprietas locutionis
patiatur. 


&

第一異論に対してはそれゆえ以下のように言われるべきである。
ペルソナと本質の一性を表現するために、聖なる教師たちはときどき、
語りの適切さが許す以上の激しさで語った。

\\


Unde huiusmodi locutiones non sunt extendendae, sed exponendae, ut
scilicet nomina abstracta exponantur per concreta, vel etiam per nomina
personalia, ut, cum dicitur, \textit{essentia de essentia, vel sapientia de sapientia},
sit sensus, \textit{filius, qui est essentia et sapientia, est de patre, qui est
essentia et sapientia}. 


&

このことから、このような語り方は広められるべきではなく、むしろ抽象的な名称が具体的な名称によって、
あるいはペルソナ的な名称によって説明されていると解釈されるべきである。
たとえば「本質から本質が、知恵から知恵が」と言われるとき、その意味は「本質であり知恵である息子は、
本質であり知恵である父から」という意味だ、というように。

\\


In his tamen nominibus abstractis est quidam ordo
attendendus, quia ea quae pertinent ad actum, magis propinque se habent ad
personas, quia actus sunt suppositorum. Unde minus impropria est ista, \textit{natura
de natura}, vel \textit{sapientia de sapientia}, quam \textit{essentia de essentia}.


&

しかしこのの抽象的な名称の中にはある種の順序が見出される。なぜなら、
作用に属する名称はペルソナにより知覚関係するからである。
というのも、働きは個体に属するからである。
このことから、「本性から本性が」や「知恵から知恵が」は、「本質から本質が」に比べて
不適切さが少ない。

\\


\textsc{Ad secundum dicendum} quod in creaturis generatum non accipit naturam eandem
numero quam generans habet, sed aliam numero, quae incipit in eo esse per
generationem de novo, et desinit esse per corruptionem, et ideo generatur et
corrumpitur per accidens. Sed Deus genitus eandem naturam numero accipit quam
generans habet. Et ideo natura divina in filio non generatur, neque per se
neque per accidens.


&

第二異論に対しては以下のように言われるべきである。
被造物において、生み出されたものは生み出すものが持っている数的に同一の本性を受け取るのではなく、
数的に別の本性を受け取る。この本性は生成によってそのものの中に新たに存在し始め、消滅によって
存在をやめるので、附帯的に生成消滅する。
しかし生み出された神は生むものと数的に同一の本性を持つ。
ゆえに神の本性は息子において自体的にも附帯的にも生まれない。

\\


\textsc{Ad tertium dicendum} quod, licet Deus et divina essentia sint idem secundum rem,
tamen, ratione alterius modi significandi, oportet loqui diversimode de
utroque.


&

第三異論に対しては以下のように言われるべきである。
神と神の本質が事物に即して同一であるとしても、表示のしかたが異なるために、この両者について異なるしかたで語らなければならない。

\\


\textsc{Ad quartum dicendum} quod essentia divina praedicatur de patre per modum
identitatis, propter divinam simplicitatem, nec tamen sequitur quod possit
supponere pro patre, propter diversum modum significandi. Ratio autem
procederet in illis, quorum unum praedicatur de altero sicut universale de
particulari.


&

第四異論に対しては以下のように言われるべきである。
神の本質は、神の単純性のために父について同一性のあり方によって述語される。
しかし表示のしかたが異なるために、このことから、それが父を代示しうることは帰結しない。
もしも一方が他方に、個別について普遍が述語されるように述語されるのであったならば、
この異論が言うことは当たっていたであろう。


\\


\textsc{Ad quintum dicendum} quod haec est differentia inter nomina substantiva et
adiectiva, quia nomina substantiva ferunt suum suppositum, adiectiva vero non,
sed rem significatam ponunt circa substantivum. 



&

第五異論に対しては以下のように言われるべきである。
名詞は自らの個体を担うが、形容詞はそれを担わず、むしろ表示された事物を名詞の周りに措定する。
これが名詞と形容詞の違いである。

\\

Unde sophistae dicunt quod
nomina substantiva supponunt; adiectiva vero non supponunt, sed copulant.
Nomina igitur personalia substantiva possunt de essentia praedicari, propter
identitatem rei, neque sequitur quod proprietas personalis distinctam
determinet essentiam; sed ponitur circa suppositum importatum per nomen
substantivum. 



&

このことからソフィストたちは名詞は代示するが形容詞は代示せず結び付けると言う。
ゆえにペルソナ的名詞は事物の同一性のために本質に述語付けられうるが、このことから
ペルソナ的固有性が区別された本質を限定することは帰結せず、むしろ、名詞によって含意される個体の周りに
それを措定する。
\\

Sed notionalia et personalia adiectiva non possunt praedicari de
essentia, nisi aliquo substantivo adiuncto. Unde non possumus dicere quod
\textit{essentia est generans}. 

&

しかし、識標的でペルソナ的な形容詞は、何らかの名詞が結びつかないかぎり本質について述語付けられえない。
このことから、私たちは「本質が生む者である」と言うことができない。
\\

Possumus tamen dicere quod essentia est res generans,
vel Deus generans, si res et Deus supponant pro persona, non autem si supponant
pro essentia. Unde non est contradictio, si dicatur quod essentia est res
generans, et res non generans, quia primo res tenetur pro persona, secundo pro
essentia.


&

しかし私たちは「本質は生む事物である」や「本質は生む神である」と言うことができる。
これは、「事物」や「神」がペルソナを代示する場合に可能なのであって、
それらが本質を代示する場合には可能でない。
このことから、「本質は生む事物である」と「本質は生む事物でない」と言われる場合にも矛盾ではない。
なぜなら、前者の「事物」はペルソナとして、後者は本質として理解されるからである。

\\


\textsc{Ad sextum dicendum} quod deitas, inquantum est una in pluribus suppositis, habet
quandam convenientiam cum forma nominis collectivi. Unde cum dicitur, pater est
principium totius deitatis, potest sumi pro universitate personarum; inquantum
scilicet, in omnibus personis divinis, ipse est principium. 


&

第六異論に対しては以下のように言われるべきである。
神性は複数の個体において一つである限りにおいて、ある種の集合的名称のかたちに一致する。
このことから、「父は神性全体の根源である」と言われるとき、それはペルソナの普遍性として解釈されうる。
すなわち、神のすべてのペルソナにおいて父が根源である限りにおいて。

\\



Nec oportet quod
sit principium sui ipsius, sicut aliquis de populo dicitur rector totius
populi, non tamen sui ipsius. Vel potest dici quod est principium totius
deitatis, non quia eam generet et spiret, sed quia eam, generando et spirando,
communicat.


&

また自分自身の根源だということは帰結しない。
たとえば、ある人が民衆について、全民衆の支配者だと言われても、自分自身の支配者ではないように。
あるいは、神性全体の根源であるのは、神性を生み霊発するからではなくて、生むことや霊発することによって
神性を伝達するからであると言われうる。
\end{longtable}
\newpage


\rhead{a.~6}
\begin{center}
 {\Large {\bf ARTICULUS SEXTUS}}\\
 {\large UTRUM PERSONAE POSSUNT PRAEDICARI DE NOMINIBUS ESSENTIALIBUS}\\
 {\footnotesize I \textit{Sent.}, d.4, q.2, a.2, ad 4, 5.}\\
 {\Large 第六項\\ペルソナは本質的名称について述語付けられうるか}
\end{center}

\begin{longtable}{p{21em}p{21em}}

\textsc{Ad sextum sic proceditur}. Videtur quod personae non possint praedicari de
nominibus essentialibus concretis, ut dicatur, \textit{Deus est tres personae}, vel \textit{est
Trinitas}. Haec enim est falsa, \textit{homo est omnis homo}, quia pro nullo suppositorum
verificari potest, neque enim Socrates est omnis homo, neque Plato, neque
aliquis alius. Sed similiter ista, \textit{Deus est Trinitas}, pro nullo suppositorum
naturae divinae verificari potest, neque enim pater est Trinitas, neque filius,
neque spiritus sanctus. Ergo haec est falsa, \textit{Deus est Trinitas}.


&

第六項の問題へ議論は以下のように進められる。
ペルソナは「神は三つのペルソナである」や「神は三性である」のように具体的な本質的名称について述語されえないと思われる。
理由は以下の通り。
「人間はすべての人間である」という命題は偽である。なぜなら、ソクラテスはすべての人間でないし、プラトンも、他のだれもすべての人間ではないのだから、どの個体についても真であることが確かめられえないからである。
しかるに「神は三性である」も同様に、神の本性を持つどの個体についても真であることが確かめられえない。
父は三性でないし息子も聖霊も三性でないからである。
ゆえに「神は三性である」というこの命題は偽である。


\\

2. \textsc{Praeterea}, inferiora non praedicantur de suis superioribus nisi accidentali
praedicatione, ut cum dico, \textit{animal est homo}, accidit enim animali esse hominem.
Sed hoc nomen Deus se habet ad tres personas sicut commune ad inferiora, ut
Damascenus dicit. Ergo videtur quod nomina personarum non possint praedicari de
hoc nomine Deus, nisi accidentaliter.


&

さらに、附帯的な述語付けでない限り、下位のものが上位のものについて述語付けられることはない。
たとえば私が「動物は人間である」と言うとき、動物は附帯的に人間である。
しかるに「神」というこの名称は、ダマスケヌスが言うように、三つのペルソナに対して、ちょうど共通のものが下位のものに対するように関係する。
ゆえにペルソナに属する名称は「神」というこの名称について、附帯的にでなければ述語付けられえない。

\\

\textsc{Sed contra} est quod Augustinus dicit, in sermone \textit{de Fide : credimus unum Deum
unam esse divini nominis Trinitatem.}


&

しかし反対に、アウグスティヌスは『信仰について』という説教の中で「私たちは一人の神が神の名を持つ一つの三性であることを信じる」と言っている。

\\

\textsc{Respondeo dicendum} quod, sicut iam dictum est, licet nomina personalia vel
notionalia adiectiva non possint praedicari de essentia; tamen substantiva
possunt, propter realem identitatem essentiae et personae. 



&

解答する。以下のように言われるべきである。
すでに述べられたように、形容詞的なペルソナ的名称や識標的名称は本質について述語付けられえない。
しかし、名詞的なそのような名称は述語付けられうる。それは本質とペルソナの実在的な同一性のためである。

\\


Essentia autem
divina non solum idem est realiter cum una persona, sed cum tribus. Unde et una
persona, et duae, et tres possunt de essentia praedicari; ut si dicamus,
\textit{essentia est pater et filius et spiritus sanctus}. 

&

しかるに神の本質は一つのペルソナだけでなく三つのペルソナと実在的に同一である。
このことから、一つのペルソナも二つのペルソナも三つのペルソナも、本質について述語付けられうる。
たとえば「本質は父であり息子であり聖霊である」と私たちが言うように。

\\


Et quia hoc nomen \textit{Deus} per se
habet quod supponat pro essentia, ut dictum est, ideo, sicut haec est vera,
\textit{essentia est tres personae}, ita haec est vera, \textit{Deus est tres personae}.


&

そして「神」というこの名称は、すでに述べられたとおり、自体的に本質を代示する。ゆえに
「本質は三つのペルソナである」が真であるように「神は三つのペルソナである」も真である。

\\

\textsc{Ad primum ergo dicendum} quod, sicut supra dictum est, hoc nomen \textit{homo} per se
habet supponere pro persona; sed ex adiuncto habet quod stet pro natura
communi. Et ideo haec est falsa, \textit{homo est omnis homo}, quia pro nullo supposito
verificari potest. Sed hoc nomen Deus per se habet quod stet pro essentia.
Unde, licet pro nullo suppositorum divinae naturae haec sit vera, \textit{Deus est
Trinitas}, est tamen vera pro essentia. Quod non attendens, Porretanus eam
negavit.


&

第一異論に対してはそれゆえ以下のように言われるべきである。
前に述べられたとおり、「人間」というこの名称は、自体的にペルソナを代示する。
しかし、別の言葉が結び付けられることによって、共通の本性を代示することができる。
それゆえ、どの個体についても真であることが確かめられないので「人間はすべての人間である」は偽である。
しかし「神」というこの名称は、自体的に本質を代示する。このことから、神の本性を持つどの個体についても真であることが
確かめられないが、しかし「神は三性である」は本質として真である。このことに気がつかずに、ポッレターヌスはこの命題を否定した。

\\

\textsc{Ad secundum dicendum} quod, cum dicitur, \textit{Deus} vel \textit{divina essentia est pater}, est
praedicatio per identitatem, non autem sicut inferioris de superiori, quia in
divinis non est universale et singulare. Unde, sicut est per se ista, \textit{pater est
Deus}, ita et ista, \textit{Deus est pater}; et nullo modo per accidens.


&

第二異論に対しては以下のように言われるべきである。
「神ないし神の本質は父である」と言われるとき、それは同一性による述語付けである。
しかし下位のものが上位のものに述語付けられているのではない。
なぜなら、神においては普遍も個別もないからである。
このことから、「父は神である」が自体的であるように「神は父である」も自体的であり、
いかなる意味においても附帯的ではない。


\end{longtable}
\newpage
\rhead{a.~7}
\begin{center}
 {\Large {\bf ARTICULUS SEPTIMUS}}\\
 {\large UTRUM NOMINA ESSENTIALIA SINT APPROPRIANDA PERSONIS}\\
 {\footnotesize I \textit{Sent.}, d.31, q.1, a.2; \textit{De Verit.}, q.7, a.3.}\\
 {\Large 第七項\\本質的名称はペルソナに固有化されるべきか}
\end{center}

\begin{longtable}{p{21em}p{21em}}

    \textsc{Ad septimum sic proceditur}. Videtur quod nomina essentialia non sint
    approprianda personis. Quod enim potest vergere in errorem fidei, vitandum
    est in divinis, quia, ut Hieronymus dicit, \textit{ex verbis inordinate prolatis
    incurritur haeresis}. Sed ea quae sunt communia tribus personis appropriare
    alicui, potest vergere in errorem fidei, quia potest intelligi quod vel
    illi tantum personae conveniant cui appropriantur; vel quod magis
    conveniant ei quam aliis. Ergo essentialia attributa non sunt approprianda
    personis.


&

第七項の問題へ議論は以下のように進められる。
本質的名称はペルソナに固有化されるべきでないと思われる。理由は以下の通り。

\\


2. \textsc{Praeterea}, essentialia attributa, in abstracto significata, significant per
modum formae. Sed una persona non se habet ad aliam ut forma, cum forma ab eo
cuius est forma, supposito non distinguatur. Ergo essentialia attributa, maxime
in abstracto significata, non debent appropriari personis.


&

\\


3. \textsc{Praeterea}, proprium prius est appropriato, \textit{proprium} enim est de ratione
appropriati. Sed essentialia attributa, secundum modum intelligendi, sunt
priora personis, sicut commune est prius proprio. Ergo essentialia attributa
non debent esse appropriata.


&

\\


\textsc{Sed contra} est quod apostolus dicit, I \textit{Cor}. I, \textit{Christum, Dei virtutem et Dei
sapientiam}.


&

\\


\textsc{Respondeo dicendum} quod, ad manifestationem fidei, conveniens fuit essentialia
attributa personis appropriari. Licet enim Trinitas personarum demonstratione
probari non possit, ut supra dictum est, convenit tamen ut per aliqua magis
manifesta declaretur. 



&

\\


Essentialia vero attributa sunt nobis magis manifesta
secundum rationem, quam propria personarum, quia ex creaturis, ex quibus
cognitionem accipimus, possumus per certitudinem devenire in cognitionem
essentialium attributorum; non autem in cognitionem personalium proprietatum,
ut supra dictum est. 



&

\\

Sicut igitur similitudine vestigii vel imaginis in
creaturis inventa utimur ad manifestationem divinarum personarum, ita et
essentialibus attributis. Et haec manifestatio personarum per essentialia
attributa, \textit{appropriatio} nominatur. 



&

\\



Possunt autem manifestari personae divinae
per essentialia attributa dupliciter. Uno modo, per viam similitudinis, sicut
ea quae pertinent ad intellectum, appropriantur filio, qui procedit per modum
intellectus ut verbum. 




&

\\


Alio modo, per modum dissimilitudinis, sicut potentia
appropriatur patri, ut Augustinus dicit, quia apud nos patres solent esse
propter senectutem infirmi; ne tale aliquid suspicemur in Deo.


&

\\


\textsc{Ad primum ergo dicendum} quod essentialia attributa non sic appropriantur
personis ut eis esse propria asserantur, sed ad manifestandum personas per viam
similitudinis vel dissimilitudinis, ut dictum est. Unde nullus error fidei
sequitur, sed magis manifestatio veritatis.


&

\\


\textsc{Ad secundum dicendum} quod, si sic appropriarentur essentialia attributa
personis, quod essent eis propria, sequeretur quod una persona se haberet ad
aliam in habitudine formae. 



&

\\



Quod excludit Augustinus, in VII \textit{de Trin}.,
ostendens quod pater non est sapiens sapientia quam genuit, quasi solus filius
sit sapientia; ut sic pater et filius simul tantum possint dici \textit{sapiens}, non
autem pater sine filio. 



&

\\



Sed filius dicitur sapientia patris, quia est sapientia
de patre sapientia, uterque enim per se est sapientia, et simul ambo una
sapientia. Unde pater non est sapiens sapientia quam genuit, sed sapientia quae
est sua essentia.


&

\\


\textsc{Ad tertium dicendum} quod, licet essentiale attributum, secundum rationem
propriam, sit prius quam persona, secundum, modum intelligendi; tamen,
inquantum habet rationem appropriati, nihil prohibet proprium personae esse
prius quam appropriatum. Sicut color posterior est corpore, inquantum est
corpus, prius tamen est naturaliter \textit{corpore albo}, inquantum est album.


&



\end{longtable}
\newpage
\end{document}
%\rhead{a.~}
%\begin{center}
% {\Large {\bf }}\\
% {\large }\\
% {\footnotesize }\\
% {\Large \\}
%\end{center}
%
%\begin{longtable}{p{21em}p{21em}}
%
%&
%
%\\
%\end{longtable}
%\newpage

ARTICULUS 8
[30057] Iª q. 39 a. 8 arg. 1
Ad octavum sic proceditur. Videtur quod inconvenienter a sacris doctoribus sint essentialia personis attributa. Dicit enim Hilarius, in II de Trin., aeternitas est in patre, species in imagine, usus in munere. In quibus verbis ponit tria nomina propria personarum, scilicet nomen patris; et nomen imaginis, quod est proprium filio, ut supra dictum est; et nomen muneris, sive doni, quod est proprium spiritus sancti, ut supra habitum est. Ponit etiam tria appropriata, nam aeternitatem appropriat patri, speciem filio, usum spiritui sancto. Et videtur quod irrationabiliter. Nam aeternitas importat durationem essendi, species vero est essendi principium, usus vero ad operationem pertinere videtur. Sed essentia et operatio nulli personae appropriari inveniuntur. Ergo inconvenienter videntur ista appropriata personis.

[30058] Iª q. 39 a. 8 arg. 2
Praeterea, Augustinus in I de Doctr. Christ., sic dicit, in patre est unitas, in filio aequalitas, in spiritu sancto unitatis aequalitatisque concordia. Et videtur quod inconvenienter. Quia una persona non denominatur formaliter per id quod appropriatur alteri, non enim est sapiens pater sapientia genita, ut dictum est. Sed, sicut ibidem subditur, tria haec unum omnia sunt propter patrem, aequalia omnia propter filium, connexa omnia propter spiritum sanctum. Non ergo convenienter appropriantur personis.

[30059] Iª q. 39 a. 8 arg. 3
Item, secundum Augustinum, patri attribuitur potentia, filio sapientia, spiritui sancto bonitas. Et videtur hoc esse inconveniens. Nam virtus ad potentiam pertinet. Virtus autem invenitur appropriari filio, secundum illud I ad Cor. I, Christum, Dei virtutem; et etiam spiritui sancto, secundum illud Luc. VI, virtus de illo exibat, et sanabat omnes. Non ergo potentia patri est approprianda.

[30060] Iª q. 39 a. 8 arg. 4
Item, Augustinus, in libro de Trin., dicit, non confuse accipiendum est quod ait apostolus, ex ipso, et per ipsum, et in ipso - ex ipso dicens propter patrem; per ipsum propter filium; in ipso propter spiritum sanctum. Sed videtur quod inconvenienter. Quia per hoc quod dicit in ipso, videtur importari habitudo causae finalis, quae est prima causarum. Ergo ista habitudo causae deberet appropriari patri, qui est principium non de principio.

[30061] Iª q. 39 a. 8 arg. 5
Item, invenitur veritas appropriari filio, secundum illud Ioan. XIV, ego sum via, veritas et vita. Et similiter liber vitae, secundum illud Psalmi XXXIX, in capite libri scriptum est de me, Glossa, idest apud patrem, qui est caput meum. Et similiter hoc quod dico, qui est, quia super illud Isa. LXV, ecce ego, ad gentes, dicit Glossa, filius loquitur, qui dixit Moysi, ego sum qui sum. Sed videtur quod propria sint filii, et non appropriata. Nam veritas, secundum Augustinum, in libro de vera religione, est summa similitudo principii, absque omni dissimilitudine, et sic videtur quod proprie conveniat filio, qui habet principium. Liber etiam vitae videtur proprium aliquid esse, quia significat ens ab alio, omnis enim liber ab aliquo scribitur. Hoc etiam ipsum qui est videtur esse proprium filio. Quia si, cum Moysi dicitur, ego sum qui sum, loquitur Trinitas, ergo Moyses poterat dicere, ille qui est pater et filius et spiritus sanctus, misit me ad vos. Ergo et ulterius dicere poterat, ille qui est pater et filius et spiritus sanctus, misit me ad vos, demonstrando certam personam. Hoc autem est falsum, quia nulla persona est pater et filius et spiritus sanctus. Non ergo potest esse commune Trinitati, sed est proprium filii.

[30062] Iª q. 39 a. 8 co.
Respondeo dicendum quod intellectus noster, qui ex creaturis in Dei cognitionem manuducitur, oportet quod Deum consideret secundum modum quem ex creaturis assumit. In consideratione autem alicuius creaturae, quatuor per ordinem nobis occurrunt. Nam primo, consideratur res ipsa absolute, inquantum est ens quoddam. Secunda autem consideratio rei est, inquantum est una. Tertia consideratio rei est, secundum quod inest ei virtus ad operandum et ad causandum. Quarta autem consideratio rei est, secundum habitudinem quam habet ad causata. Unde haec etiam quadruplex consideratio circa Deum nobis occurrit. Secundum igitur primam considerationem, qua consideratur absolute Deus secundum esse suum, sic sumitur appropriatio Hilarii, secundum quam aeternitas appropriatur patri, species filio, usus spiritui sancto. Aeternitas enim, inquantum significat esse non principiatum, similitudinem habet cum proprio patris, qui est principium non de principio. Species autem, sive pulchritudo, habet similitudinem cum propriis filii. Nam ad pulchritudinem tria requiruntur. Primo quidem, integritas sive perfectio, quae enim diminuta sunt, hoc ipso turpia sunt. Et debita proportio sive consonantia. Et iterum claritas, unde quae habent colorem nitidum, pulchra esse dicuntur. Quantum igitur ad primum, similitudinem habet cum proprio filii, inquantum est filius habens in se vere et perfecte naturam patris. Unde, ad hoc innuendum, Augustinus in sua expositione dicit, ubi, scilicet in filio, summa et prima vita est, et cetera. Quantum vero ad secundum, convenit cum proprio filii, inquantum est imago expressa patris. Unde videmus quod aliqua imago dicitur esse pulchra, si perfecte repraesentat rem, quamvis turpem. Et hoc tetigit Augustinus cum dicit, ubi est tanta convenientia, et prima aequalitas, et cetera. Quantum vero ad tertium, convenit cum proprio filii, inquantum est verbum, quod quidem lux est, et splendor intellectus, ut Damascenus dicit. Et hoc tangit Augustinus cum dicit, tanquam verbum perfectum cui non desit aliquid, et ars quaedam omnipotentis Dei, et cetera. Usus autem habet similitudinem cum propriis spiritus sancti, largo modo accipiendo usum, secundum quod uti comprehendit sub se etiam frui; prout uti est assumere aliquid in facultatem voluntatis, et frui est cum gaudio uti, ut Augustinus, X de Trin., dicit. Usus ergo quo pater et filius se invicem fruuntur, convenit cum proprio spiritus sancti, inquantum est amor. Et hoc est quod Augustinus dicit, illa dilectio, delectatio, felicitas vel beatitudo, usus ab illo appellatus est. Usus vero quo nos fruimur Deo, similitudinem habet cum proprio spiritus sancti, inquantum est donum. Et hoc ostendit Augustinus cum dicit, est in Trinitate spiritus sanctus, genitoris genitique suavitas, ingenti largitate atque ubertate nos perfundens. Et sic patet quare aeternitas, species et usus personis attribuantur vel approprientur, non autem essentia vel operatio. Quia in ratione horum, propter sui communitatem, non invenitur aliquid similitudinem habens cum propriis personarum. Secunda vero consideratio Dei est, inquantum consideratur ut unus. Et sic Augustinus patri appropriat unitatem, filio aequalitatem, spiritui sancto concordiam sive connexionem. Quae quidem tria unitatem importare manifestum est, sed differenter. Nam unitas dicitur absolute, non praesupponens aliquid aliud. Et ideo appropriatur patri, qui non praesupponit aliquam personam, cum sit principium non de principio. Aequalitas autem importat unitatem in respectu ad alterum, nam aequale est quod habet unam quantitatem cum alio. Et ideo aequalitas appropriatur filio, qui est principium de principio. Connexio autem importat unitatem aliquorum duorum. Unde appropriatur spiritui sancto, inquantum est a duobus. Ex quo etiam intelligi potest quod dicit Augustinus, tria esse unum propter patrem, aequalia propter filium, connexa propter spiritum sanctum. Manifestum est enim quod illi attribuitur unumquodque, in quo primo invenitur, sicut omnia inferiora dicuntur vivere propter animam vegetabilem, in qua primo invenitur ratio vitae in istis inferioribus. Unitas autem statim invenitur in persona patris, etiam, per impossibile, remotis aliis personis. Et ideo aliae personae a patre habent unitatem. Sed remotis aliis personis, non invenitur aequalitas in patre, sed statim, posito filio, invenitur aequalitas. Et ideo dicuntur omnia aequalia propter filium, non quod filius sit principium aequalitatis patri; sed quia, nisi esset patri aequalis filius, pater aequalis non posset dici. Aequalitas enim eius primo consideratur ad filium, hoc enim ipsum quod spiritus sanctus patri aequalis est, a filio habet. Similiter, excluso spiritu sancto, qui est duorum nexus, non posset intelligi unitas connexionis inter patrem et filium. Et ideo dicuntur omnia esse connexa propter spiritum sanctum, quia, posito spiritu sancto, invenitur unde pater et filius possint dici connexi. Secundum vero tertiam considerationem, qua in Deo sufficiens virtus consideratur ad causandum, sumitur tertia appropriatio, scilicet potentiae, sapientiae et bonitatis. Quae quidem appropriatio fit et secundum rationem similitudinis, si consideretur quod in divinis personis est, et secundum rationem dissimilitudinis, si consideretur quod in creaturis est. Potentia enim habet rationem principii. Unde habet similitudinem cum patre caelesti, qui est principium totius divinitatis. Deficit autem interdum patri terreno, propter senectutem. Sapientia vero similitudinem habet cum filio caelesti, inquantum est verbum, quod nihil aliud est quam conceptus sapientiae. Deficit autem interdum filio terreno, propter temporis paucitatem. Bonitas autem, cum sit ratio et obiectum amoris, habet similitudinem cum spiritu divino, qui est amor. Sed repugnantiam habere videtur ad spiritum terrenum, secundum quod importat violentam quandam impulsionem; prout dicitur Isa. XXV, spiritus robustorum quasi turbo impellens parietem. Virtus autem appropriatur filio et spiritui sancto, non secundum quod virtus dicitur ipsa potentia rei, sed secundum quod interdum virtus dicitur id quod a potentia rei procedit, prout dicimus aliquod virtuosum factum esse virtutem alicuius agentis. Secundum vero quartam considerationem, prout consideratur Deus in habitudine ad suos effectus, sumitur illa appropriatio ex quo, per quem, et in quo. Haec enim praepositio ex importat quandoque quidem habitudinem causae materialis, quae locum non habet in divinis, aliquando vero habitudinem causae efficientis. Quae quidem competit Deo ratione suae potentiae activae, unde et appropriatur patri, sicut et potentia. Haec vero praepositio per designat quidem quandoque causam mediam; sicut dicimus quod faber operatur per martellum. Et sic ly per quandoque non est appropriatum, sed proprium filii, secundum illud Ioan. I, omnia per ipsum facta sunt; non quia filius sit instrumentum, sed quia ipse est principium de principio. Quandoque vero designat habitudinem formae per quam agens operatur; sicut dicimus quod artifex operatur per artem. Unde, sicut sapientia et ars appropriantur filio, ita et ly per quem. Haec vero praepositio in denotat proprie habitudinem continentis. Continet autem Deus res dupliciter. Uno modo, secundum suas similitudines; prout scilicet res dicuntur esse in Deo, inquantum sunt in eius scientia. Et sic hoc quod dico in ipso, esset appropriandum filio. Alio vero modo continentur res a Deo, inquantum Deus sua bonitate eas conservat et gubernat, ad finem convenientem adducendo. Et sic ly in quo appropriatur spiritui sancto, sicut et bonitas. Nec oportet quod habitudo causae finalis, quamvis sit prima causarum, approprietur patri, qui est principium non de principio, quia personae divinae, quarum pater est principium, non procedunt ut ad finem, cum quaelibet illarum sit ultimus finis; sed naturali processione, quae magis ad rationem naturalis potentiae pertinere videtur.

[30063] Iª q. 39 a. 8 ad 1
Ad illud vero quod de aliis quaeritur, dicendum quod veritas, cum pertineat ad intellectum, ut supra dictum est, appropriatur filio, non tamen est proprium eius. Quia veritas, ut supra dictum est, considerari potest prout est in intellectu, vel prout est in re. Sicut igitur intellectus et res essentialiter sumpta sunt essentialia et non personalia, ita et veritas. Definitio autem Augustini inducta, datur de veritate secundum quod appropriatur filio. Liber autem vitae in recto quidem importat notitiam, sed in obliquo vitam, est enim, ut supra dictum est, notitia Dei de his qui habituri sunt vitam aeternam. Unde appropriatur filio, licet vita approprietur spiritui sancto, inquantum importat quendam interiorem motum, et sic convenit cum proprio spiritus sancti, inquantum est amor. Esse autem scriptum ab alio, non est de ratione libri inquantum est liber; sed inquantum est quoddam artificiatum. Unde non importat originem, neque est personale, sed appropriatum personae. Ipsum autem qui est appropriatur personae filii, non secundum propriam rationem, sed ratione adiuncti, inquantum scilicet in locutione Dei ad Moysen, praefigurabatur liberatio humani generis, quae facta est per filium. Sed tamen, secundum quod ly qui sumitur relative, posset referre interdum personam filii, et sic sumeretur personaliter, ut puta si dicatur, filius est genitus qui est; sicut et Deus genitus personale est. Sed infinite sumptum est essentiale. Et licet hoc pronomen iste, grammatice loquendo, ad aliquam certam personam videatur pertinere; tamen quaelibet res demonstrabilis, grammatice loquendo, persona dici potest, licet secundum rei naturam non sit persona; dicimus enim iste lapis, et iste asinus. Unde et, grammatice loquendo, essentia divina, secundum quod significatur et supponitur per hoc nomen Deus, potest demonstrari hoc pronomine iste; secundum illud Exod. XV, iste Deus meus, et glorificabo eum.

