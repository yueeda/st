\documentclass[10pt]{jsarticle} % use larger type; default would be 10pt
%\usepackage[utf8]{inputenc} % set input encoding (not needed with XeLaTeX)
%\usepackage[round,comma,authoryear]{natbib}
%\usepackage{nruby}
\usepackage{okumacro}
\usepackage{longtable}
%\usepqckage{tablefootnote}
\usepackage[polutonikogreek,english,japanese]{babel}
%\usepackage{amsmath}
\usepackage{latexsym}
\usepackage{color}

%----- header -------
\usepackage{fancyhdr}
\pagestyle{fancy}
\lhead{{\it Summa Theologiae} I, q.~26}
%--------------------


\title{{\bf PRIMA PARS}\\{\HUGE Summae Theologiae}\\Sancti Thomae
Aquinatis\\{\sffamily QUEAESTIO VIGESIMASEXTA}\\DE DIVINA
BEATITUDINE}
\author{Japanese translation\\by Yoshinori {\sc Ueeda}}
\date{Last modified \today}

%%%% コピペ用
%\rhead{a.~}
%\begin{center}
% {\Large {\bf }}\\
% {\large }\\
% {\footnotesize }\\
% {\Large \\}
%\end{center}
%
%\begin{longtable}{p{21em}p{21em}}
%
%&
%
%\\
%\end{longtable}
%\newpage



\begin{document}

\maketitle

\begin{center}
{\Large 第二十六問\\神の至福について}
\end{center}


\begin{longtable}{p{21em}p{21em}}



{\Huge U}ltimo autem, post considerationem eorum quae ad divinae essentiae
 unitatem pertinent, considerandum est de divina beatitudine. Et circa
 hoc quaeruntur quatuor. 

\begin{enumerate}
 \item utrum beatitudo Deo competat.
 \item secundum quid dicitur Deus esse beatus, utrum secundum actum
 intellectus. 
 \item utrum sit essentialiter beatitudo cuiuslibet
 beati.
 \item utrum in eius beatitudine omnis beatitudo includatur.
\end{enumerate}

&

さて、神の本質に属することが考察された後、最後に、神の至福について考察さ
 れるべきである。これをめぐっては、四つのことが問われる。

\begin{enumerate}
 \item 至福は神に適合するか。
 \item 神は何に即して至福と言われるか。知性の働きに即してか。
 \item 神は、本質的に、至福であるものの至福か。
 \item 神の至福の中に、あらゆる至福が含まれるか。
\end{enumerate}

\end{longtable}

\newpage

\rhead{a.~1}
\begin{center}
 {\Large {\bf ARTICULUS PRIMUS}}\\
 {\large UTRUM BEATITUDO DEI COMPETAT}\\
 {\footnotesize II {\itshape Sent.}, d.1, q.2, a.2, ad 2; {\itshape
 SCG.}, cap.100}\\
 {\Large 第一項\\至福は神に適合するか}
\end{center}

\begin{longtable}{p{21em}p{21em}}


{\Huge A}{\scshape d primum sic proceditur}. Videtur quod beatitudo Deo non
conveniat. Beatitudo enim, secundum Boetium, in III {\itshape de Consol}., est
{\itshape status omnium bonorum aggregatione perfectus}. Sed aggregatio bonorum non
habet locum in Deo, sicut nec compositio. Ergo Deo non convenit
beatitudo.

&

第一項の問題へ、議論は以下のように進められる。
至福は神にふさわしくないと思われる。理由は以下の通り。
『哲学の慰め』第3巻のボエティウスによれば、至福とは「すべての善を集めて
 完全である状態」である。ところが、ちょうど複合が神の中にない
 ように、善を集めることは、神の中にない。ゆえに、神に至福はふさわしくな
 い。


\\



2. {\scshape Praeterea}, beatitudo, sive felicitas, est {\itshape praemium virtutis}, secundum
philosophum, in I {\itshape Ethic}. Sed Deo non convenit praemium, sicut nec
meritum. Ergo nec beatitudo.

&

さらに、『ニコマコス倫理学』第1巻の哲学者によれば、至福、ないし幸福は
「徳の褒美」である。しかし、神には、功績が適合しないように、褒美も適合
しない。ゆえに至福も適合しない。

\\



{\scshape Sed contra est} quod dicit apostolus, I {\itshape ad Tim.} ultimo, {\itshape quem suis
temporibus ostendet Deus beatus et solus potens, rex regum et dominus
dominantium}.

&

しかし反対に、使徒は『テモテの信徒への手紙一』の最終章で言う。「至福なる
 神、唯一の力ある者、諸王の王、諸主の主は、自らの時間においてそれを示す
 だろう」。

\\



{\scshape Respondeo dicendum} quod beatitudo maxime Deo competit. Nihil enim aliud
sub nomine beatitudinis intelligitur, nisi bonum perfectum
intellectualis naturae; cuius est suam sufficientiam cognoscere in bono
quod habet; et cui competit ut ei contingat aliquid vel bene vel male,
et sit suarum operationum domina. Utrumque autem istorum excellentissime
Deo convenit, scilicet perfectum esse, et intelligentem. Unde beatitudo
maxime convenit Deo.

&

解答する。以下のように言われるべきである。至福は最大限に神に適合する。
理由は以下の通り。「至福」という言葉で理解されるのは、「知性的本性の完
全な善」に他ならない。そのような善には、自分が持っている善において、自
らの十分さを認識することが属し、また、何かがそれに、善くあるいは悪く生
じ、かつ自らの働きの主であることが適合する。しかし、この二つ、すなわち、
完全であることと知性認識者であることというこのどちらも、この上なく卓越
したしかたで神に適合する。したがって、至福は最大限に神に属する。


\\



{\scshape Ad primum ergo dicendum} quod aggregatio bonorum est in Deo non per modum
compositionis, sed per modum simplicitatis, quia quae in creaturis
multiplicia sunt, in Deo praeexistunt simpliciter et unite, ut supra
dictum est.

&

第一異論に対しては、それゆえ、以下のように言われるべきである。善を集め
ることは、神において、複合のかたちによってあるのではない。むしろ、単純
性のかたちによってある。なぜなら、前に述べられたとおり、被造物において
多数化されているものは、神において、単純で一なるかたちで、先在するから
である。


\\



{\scshape Ad secundum dicendum} quod esse praemium virtutis accidit beatitudini vel
felicitati, inquantum aliquis beatitudinem acquirit, sicut esse terminum
generationis accidit enti, inquantum exit de potentia in actum. Sicut
igitur Deus habet esse, quamvis non generetur; ita habet beatitudinem,
quamvis non mereatur.

&

第二異論に対しては、以下のように言われるべきである。ちょうど、生成の終
端であることが有に生じるのが、可能態から現実態へ出ていく限りにおいてで
あるように、徳の褒美であるということが至福に生じるのは、何らかの至福を
獲得する限りにおいてである。ゆえに、ちょうど神が、生成したのでないにも
かかわらず存在を持つように、神は、褒められたのではないにもかかわらず、
至福を持つ。



\end{longtable}

\newpage


\rhead{a.~2}
\begin{center}
 {\Large {\bf ARTICULUS SECUNDUS}}\\
 {\large UTRUM DEUS DICATUR BEATUS SECUNDUM INTELLECTUM}\\
 {\footnotesize II {\itshape Sent.}, d.16, a.2; I {\itshape Tim.},
 cap.6, lect.3.}\\
 {\Large 第二項\\神は知性に即して至福と言われるか}
\end{center}

\begin{longtable}{p{21em}p{21em}}


{\Huge A}{\scshape d secundum sic proceditur}. Videtur quod Deus non dicatur beatus
secundum intellectum. Beatitudo enim est summum bonum. Sed bonum dicitur
in Deo secundum essentiam, quia bonum respicit esse, quod est secundum
essentiam, secundum Boetium, in libro {\itshape de Hebdomad}. Ergo et beatitudo
dicitur in Deo secundum essentiam, et non secundum intellectum.


&

第二項の問題へ、議論は以下のように進められる。神は、知性において、至福
と言われるのではないと思われる。理由は以下の通り。至福は最高の善である。
ところが、善は、神において、本質に即して言われる。なぜなら、『デ・ヘブ
ドマディブス』のボエティウスによれば、善は存在に関係し、存在は本質に即
してあるからである。ゆえに、至福も神において本質に即して言われるのであ
り、知性に即してではない。

\\



2. {\scshape Praeterea}, beatitudo habet rationem finis. Finis autem est obiectum
voluntatis, sicut et bonum. Ergo beatitudo dicitur in Deo secundum
voluntatem, et non secundum intellectum.

&

さらに、至福は目的の性格を持つ。しかし目的は、善と同様、意志の対象であ
る。ゆえに、至福は神において意志に即して言われるのであり、知性に即して
ではない。

\\



{\scshape Sed contra est} quod Gregorius dicit, XXXII {\itshape
Moralium}, {\itshape ipse gloriosus est, qui, dum seipso perfruitur,
accedentis laudis indigens non est}. Esse autem gloriosum significat
esse beatum. Cum igitur Deo fruamur secundum intellectum, quia
{\itshape visio est tota merces}, ut dicit Augustinus, videtur quod
beatitudo dicatur in Deo secundum intellectum.

&

しかし反対に、グレゴリウスは『道徳論』32巻で、「自らを享受する間、降り
かかる賞賛を必要としない人は、輝かしい」と述べている。しかし、輝かしい
ということは、至福であることを意味している。ゆえに、私たちは神を知性に
即して享受するのだから、というのも、アウグスティヌスが言うように、「見
ることが褒美のすべて」だからだが、それゆえ、神における至福は、知性に即
して語られると思われる。


\\



{\scshape Respondeo dicendum} quod beatitudo, sicut dictum est, significat bonum
perfectum intellectualis naturae. Et inde est quod, sicut unaquaeque res
appetit suam perfectionem, ita et intellectualis natura naturaliter
appetit esse beata. Id autem quod est perfectissimum in qualibet
intellectuali natura, est intellectualis operatio, secundum quam capit
quodammodo omnia. Unde cuiuslibet intellectualis naturae creatae
beatitudo consistit in intelligendo. In Deo autem non est aliud esse et
intelligere secundum rem, sed tantum secundum intelligentiae
rationem. Attribuenda ergo est Deo beatitudo secundum intellectum, sicut
et aliis beatis, qui per assimilationem ad beatitudinem ipsius, beati
dicuntur.

&

解答する。以下のように言われるべきである。すでに述べられたとおり、至福
は、知性的本性の完全な善である。したがって、各々の事物が自らの完成を欲
求するように、知性的本性は、本性的に、至福であることを欲求する。ところ
で、どんな知性的本性においても、もっとも完全なものは、知性的な働きであ
り、その働きにしたがって、ある意味ですべてのものを捉える。ゆえに、どん
な被造の知性的本性の至福も、知性認識することにおいて成立する。しかし神
においては、事物に即して、存在と知性認識は別のものでなく、ただ知解の概
念においてのみ異なる。ゆえに、至福は、他の至福者たちと同様に、知性に応
じて神に帰せられるべきである。他の至福者たちは、神の至福への類似を通し
て、至福と言われるのだから。


\\



{\scshape Ad primum ergo dicendum} quod ex illa ratione probatur quod Deus sit
beatus secundum suam essentiam, non autem quod beatitudo ei conveniat
secundum rationem essentiae, sed magis secundum rationem intellectus.

&

第一異論に対しては、それゆえ、以下のように言われるべきである。その議論
によって証明されるのは、神が本質によって至福だということであり、至福が
神に、本質という観点に即して適合するということではない。それはむしろ知
性という観点に即して適合する。



\\



{\scshape Ad secundum dicendum} quod beatitudo, cum sit bonum, est
obiectum voluntatis. Obiectum autem praeintelligitur actui
potentiae. Unde, secundum modum intelligendi, prius est beatitudo
divina, quam actus voluntatis in ea requiescentis. Et hoc non potest
esse nisi actus intellectus. Unde in actu intellectus attenditur
beatitudo.

&


第二異論に対しては、以下のように言われるべきである。至福は、善であるか
ら、意志の対象である。ところで、対象は、能力の作用に先行して、知性によっ
て認識される。ゆえに、知性認識のしかたに即しては、神の至福は、それを求
めていく意志の作用に先行する。そしてこれは、知性の作用なしにはありえな
い。したがって、知性の作用において、至福が見出される。



\end{longtable}



\newpage


\rhead{a.~3}
\begin{center}
 {\Large {\bf ARTICULUS TERTIUS}}\\
 {\large UTRUM DEUS SIT BEATITUDO CUIUSLIBET BEATI}\\
 {\footnotesize I-IIae, q.3, a.1; IV {\itshape Sent.}, d.49, q.1, a.2,
 q$^a$ 1.}\\
 {\Large 第三項\\神はあらゆる至福者の至福か}
\end{center}

\begin{longtable}{p{21em}p{21em}}


{\Huge A}{\scshape d tertium sic proceditur}. Videtur quod Deus sit beatitudo cuiuslibet
beati. Deus enim est summum bonum, ut supra ostensum est. Impossibile
est autem esse plura summa bona, ut etiam ex superioribus patet. Cum
igitur de ratione beatitudinis sit, quod sit summum bonum, videtur quod
beatitudo non sit aliud quam Deus.


&

第三項の問題へ、議論は以下のように進められる。
神はあらゆる至福者の至福であると思われる。理由は以下の通り。
前に示されたとおり、神は最高善である。
しかし、前の諸々の箇所から明らかなとおり、複数のものが最高善であることは
 不可能である。
ゆえに、至福の概念には最高善であることが属するので、至福は神に他ならない
 と思われる。



\\



2. {\scshape Praeterea}, beatitudo est finis rationalis naturae ultimus. Sed esse
ultimum finem rationalis naturae, soli Deo convenit. Ergo beatitudo
cuiuslibet beati est solus Deus.


&

さらに、至福は、理性的本性の究極目的である。しかし、理性的本性の究極目的
 であることは、ただ神にのみ適合する。
ゆえに、どんな至福者の至福も、ただ神だけである。


\\



{\scshape Sed contra}, beatitudo unius est maior beatitudine alterius, secundum
illud I {\itshape Cor}.~{\scshape xv}, {\itshape stella differt a stella in claritate}. Sed Deo nihil est
maius. Ergo beatitudo est aliquid aliud quam Deus.


&

しかし反対に、かの『コリントの信徒への手紙一』「星は他の星から明るさにおいて
 異なる」によれば、ある人の至福は他の人の至福よりも大きい。しかし、神よ
 り大きいものはない。ゆえに、至福は、神以外の何かである。


\\



{\scshape Respondeo dicendum} quod beatitudo intellectualis naturae consistit in
actu intellectus. In quo duo possunt considerari, scilicet obiectum
actus, quod est intelligibile; et ipse actus, qui est intelligere. Si
igitur beatitudo consideretur ex parte ipsius obiecti, sic solus Deus
est beatitudo, quia ex hoc solo est aliquis beatus, quod Deum
intelligit; secundum illud Augustini, in V libro {\itshape Confess}.: {\itshape beatus est
qui te novit, etiam si alia ignoret}. Sed ex parte actus intelligentis,
beatitudo est quid creatum in creaturis beatis, in Deo autem est etiam
secundum hoc, aliquid increatum.


&

解答する。以下のように言われるべきである。
知性的本性の至福は知性の作用・現実態において成立する。
そこにおいては、二つのことが考察されうる。
一つは、作用の対象、すなわち、可知的なものと、作用それ自体、すなわち知性
 認識とである。
ゆえに、もし、対象そのものの側から至福が考察されるならば、その場合には、
 神だけが至福である。
なぜなら、アウグスティヌスの『告白』第5巻「あなたを知る者は、かりにそれ
 以外のことを知らなくても、至福である」によれば、神を知性認識することに
 よってのみ、人は至福であるのだから。
しかし、知性認識者の作用の側からは、至福は至福な被造物の中に創造された何
 かである。ただし、神においては、この意味に即しても、創造されない何かで
 ある。


\\



{\scshape Ad primum ergo dicendum} quod beatitudo, quantum ad obiectum, est summum
bonum simpliciter, sed quantum ad actum, in creaturis beatis, est summum
bonum, non simpliciter, sed in genere bonorum participabilium a
creatura.


&

第一異論に対しては、それゆえ、以下のように言われるべきである。
至福は、対象に関して、端的に最高善であるが、作用に関しては、至福な被造物
 の中にあり、端的に最高善ではなく、被造物に分有可能な善という類の中での
 最高善である。

\\



{\scshape Ad secundum dicendum} quod finis est duplex, scilicet cuius et quo, ut
philosophus dicit, scilicet ipsa res, et usus rei, sicut avaro est finis
pecunia, et acquisitio pecuniae. Creaturae igitur rationalis est quidem
Deus finis ultimus ut res; beatitudo autem creata ut usus, vel magis
fruitio, rei.


&

第二異論に対しては、以下のように言われるべきである。哲学者が言うように、
 目的には二通りある。すなわち、「それの」と「それによって」、つまりもの
 それ自体と、ものの使用とである。たとえば、貪欲なものにとって、目的は、
 金銭であり、また、金銭の獲得である。ゆえに、理性的被造物には、ものとし
 て、神が究極目的であるが、ものの使用、いやむしろ享受としては、被造の至
 福が、究極目的である。



\end{longtable}

\newpage




\rhead{a.~4}
\begin{center}
 {\Large {\bf ARTICULUS QUARTUS}}\\
 {\large UTRUM IN DEI BEATITUDINE OMNIS BEATITUDO INCLUDATUR}\\
 {\footnotesize I {\itshape SCG}, cap.102.}\\
 {\Large 第四項\\神の至福にすべての至福が含まれるか}
\end{center}

\begin{longtable}{p{21em}p{21em}}



{\Huge A}{\scshape d quartum sic proceditur}. Videtur quod beatitudo divina non
complectatur omnes beatitudines. Sunt enim quaedam beatitudines
falsae. Sed in Deo nihil potest esse falsum. Ergo divina beatitudo non
complectitur omnem beatitudinem.


&

第四項の問題へ、議論は以下のように進められる。
神の至福はすべての至福を包含しないと思われる。理由は以下の通り。
ある至福は偽である。しかし、神の中に偽はありえない。
ゆえに、神の至福はすべての至福を包含しない。

\\



2. {\scshape Praeterea}, quaedam beatitudo, secundum quosdam, consistit in rebus
corporalibus, sicut in voluptatibus, divitiis, et huiusmodi, quae quidem
Deo convenire non possunt, cum sit incorporeus. Ergo beatitudo eius non
complectitur omnem beatitudinem.


&

さらに、ある人々によれば、ある至福は、快楽や富のような物体的なものにおい
 て成立する。
しかしこれらは神に適合することができない。ゆえに、神の至福はすべての至福
 を包含しない。


\\



{\scshape Sed contra est} quod beatitudo est perfectio quaedam. Divina autem
perfectio complectitur omnem perfectionem, ut supra ostensum est. Ergo
divina beatitudo complectitur omnem beatitudinem.


&

しかし反対に、至福は一種の完全性である。しかし、神の完全性は、前に示され
 たとおり、すべての完全性を包含する。ゆえに、神の至福はすべての至福を包
 含する。

\\


{\scshape Respondeo dicendum} quod quidquid est desiderabile in quacumque
beatitudine, vel vera vel falsa, totum eminentius in divina beatitudine
praeexistit. De contemplativa enim felicitate, habet continuam et
certissimam contemplationem sui et omnium aliorum, de activa vero,
gubernationem totius universi. De terrena vero felicitate, quae
consistit in voluptate, divitiis, potestate, dignitate et fama, secundum
Boetium, in III {\itshape de Consol}., habet gaudium de se et de omnibus aliis, pro
delectatione, pro divitiis, habet omnimodam sufficientiam, quam divitiae
promittunt, pro potestate, omnipotentiam, pro dignitate, omnium regimen,
pro fama vero, admirationem totius creaturae.


&

解答する。以下のように言われるべきである。
どの至福においても、望まれうるものはなんであれ、それが真であろうと偽であ
 ろうと、全体が、より優れたかたちで、神の至福の中に先在する。
じっさい、神は、観想的幸福について、自分自身と他のすべてのものについての、
 行き渡りこの上なく確実な観想をもち、活動的幸福について、全宇宙の統治を
 もつ。ボエティウス『哲学の慰め』第3巻によれば、快楽、富、権力、威厳ない
 し名声において成り立つ地上的幸福については、自分自身と
 他のすべてのものについての喜びを快楽としてもち、また、富としては、富が
約束するあらゆる面での充足を、権力としては全能を、威厳としては万物の支配
 を、名声としては、被造物全体からの賞賛をもつ。


\\



{\scshape Ad primum ergo dicendum} quod beatitudo aliqua secundum hoc est falsa,
secundum quod deficit a ratione verae beatitudinis, et sic non est in
Deo. Sed quidquid habet de similitudine, quantumcumque tenui,
beatitudinis, totum praeexistit in divina beatitudine.


&


第一異論に対しては、それゆえ、以下のように言われるべきである。
ある至福が偽と言われるのは、それが、真の至福の性格を欠いている限りでであ
 る。そして、この意味では、神の中にない。しかし、それがどれだけ所持され
 ようとも、何らかの至福の類似をもつものは何であれ、その全体が神の至福の
 中に先在する。



\\



{\scshape Ad secundum dicendum} quod bona quae sunt in corporalibus corporaliter,
in Deo sunt spiritualiter, secundum modum suum. 

&

第二異論に対しては、以下のように言われるべきである。
物体的なものの中に物体的にある善は、神の中に、神自身のあり方に即して、霊
 的に存在する。

\\

Et haec dicta sufficiant
de his quae pertinent ad divinae essentiae unitatem.


&

以上で語られたことが、神の本質の一性に属する事柄について、十分だとしよう。



\end{longtable}

\end{document}

