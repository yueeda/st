\documentclass[10pt]{jsarticle} % use larger type; default would be 10pt
\usepackage[utf8]{inputenc} % set input encoding (not needed with XeLaTeX)
\usepackage[T1]{fontenc}
%\usepackage[round,comma,authoryear]{natbib}
%\usepackage{nruby}
\usepackage{okumacro}
\usepackage{longtable}
%\usepqckage{tablefootnote}
\usepackage[polutonikogreek,english,japanese]{babel}
%\usepackage{amsmath}
\usepackage{latexsym}
\usepackage{color}
\usepackage{otf}
\usepackage{schemata}
%----- header -------
\usepackage{fancyhdr}
\pagestyle{fancy}
\lhead{{\it Summa Theologiae} I-II, q.54}
%--------------------


\bibliographystyle{jplain}


\title{{\bf Prima Secundae}\\{\HUGE Summae Theologiae}\\Sancti Thomae
Aquinatis\\{\sffamily QUAESTIO QUINQUAGESIMAQUARTA}\\{\bf DE DISTINCTIONE HABITUUM}}
\author{Japanese translation\\by Yoshinori {\sc Ueeda}}
\date{Last modified \today}

%%%% コピペ用
%\rhead{a.~}
%\begin{center}
% {\Large {\bf }}\\
% {\large }\\
% {\footnotesize }\\
% {\Large \\}
%\end{center}
%
%\begin{longtable}{p{21em}p{21em}}
%
%&
%
%\\
%\end{longtable}
%\newpage

\begin{document}

\maketitle
\thispagestyle{empty}
\begin{center}
{\Large 『神学大全』第二部の一\\第五十四問\\習慣の区別について}
\end{center}


\begin{longtable}{p{21em}p{21em}}
Deinde considerandum est de distinctione habituum. Et circa hoc quaeritur quatuor.



\begin{enumerate}
 \item utrum multi habitus possint esse in una potentia.
 \item utrum habitus distinguantur secundum obiecta.
 \item utrum habitus distinguantur secundum bonum et malum.
 \item utrum unus habitus ex multis habitibus constituatur.
\end{enumerate}

&

次に、習慣の区別について考察されるべきである。これを巡って四つのことが問われる。
\begin{enumerate}
 \item 多くの習慣が一つの能力の中にありうるか。
 \item 習慣は対象に即して区別されるか。
 \item 習慣は善と悪に即して区別されるか。
 \item 一つの習慣が多くの習慣から構成されるか。
\end{enumerate}
\end{longtable}
\newpage
\rhead{a.~1}
\begin{center}
{\Large {\bf ARTICULUS PRIMUS}}\\
{\large UTRUM MULTI HABITUS POSSINT ESSE IN UNA POTENTIA}\\
{\footnotesize III {\itshape Sent.}, d.33, q.1, a.1, qu$^a$1; {\itshape De Verit.}, q.15, a.2, ad 11; {\itshape De Virtut.}, q.1, a. 12, ad 4.}\\
{\Large 第一項\\多くの習慣が一つの能力の中にありうるか}
\end{center}


\begin{longtable}{p{21em}p{21em}}

{\scshape Ad primum sic proceditur}. Videtur quod non possint esse
multi habitus in una potentia. Eorum enim quae secundum idem
distinguuntur multiplicato uno, multiplicatur et aliud. Sed secundum
idem potentiae et habitus distinguuntur, scilicet secundum actus et
obiecta. Similiter ergo multiplicantur. Non ergo possunt esse multi
habitus in una potentia.

&

第一項の問題へ、議論は以下のように進められる。多くの習慣が一つの能力の
中にあることはできないと思われる。理由は以下の通り。同一の点で区別され
るものどもは、一方が多数化されると、他方もまた多数化される。しかるに、
能力と習慣は同一のもの、すなわち、働きと対象に即して区別される。それゆ
え、これらは同じように多数化される。ゆえに、多くの習慣が一つの能力の中
にあることはできない。

\\




2. {\scshape Praeterea}, potentia est virtus quaedam simplex. Sed in
uno subiecto simplici non potest esse diversitas accidentium, quia
subiectum est causa accidentis; ab uno autem simplici non videtur
procedere nisi unum. Ergo in una potentia non possunt esse multi
habitus.

&

さらに、能力は、ある種の単純な徳である。しかるに一つの単純な基体の中に、
附帯性の多様性はありえない。なぜなら、基体は附帯性の原因だが、一つの単
純なものからは一つのものしか出てこないように思われるからである。ゆえに
一つの能力の中に多くの習慣があることはできない。

\\




3. {\scshape Praeterea}, sicut corpus formatur per figuram, ita
potentia formatur per habitum. Sed unum corpus non potest simul
formari diversis figuris. Ergo neque una potentia potest simul formari
diversis habitibus. Non ergo plures habitus possunt simul esse in una
potentia.

&

さらに、物体が形態によって形成されるように、能力は習慣によって形成される。
しかるに一つの物体が、同時にさまざまな形態によって形成さることはできない。
ゆえに一つの能力も、同時にさまざまな習慣によって形成されることはできない。
ゆえに、複数の習慣が同時に一つの能力の中にあることはできない。

\\




{\scshape Sed contra est} quod intellectus est una potentia, in qua
tamen sunt diversarum scientiarum habitus.

&

しかし反対に、知性は一つの能力であり、その能力の中に、さまざまな知の習態がある。

\\




{\scshape Respondeo dicendum} quod, sicut supra dictum est, habitus
sunt dispositiones quaedam alicuius in potentia existentis ad aliquid,
sive ad naturam, sive ad operationem vel finem naturae. Et de illis
quidem habitibus qui sunt dispositiones ad naturam, manifestum est
quod possunt plures esse in uno subiecto, eo quod unius subiecti
possunt diversimode partes accipi, secundum quarum dispositionem
habitus dicuntur.


&

解答する。以下のように言われるべきである。前に述べられたとおり、習慣は、
\kenten{何か}に対して可能態にあるものの、一種の態勢である。その
\kenten{何か}とは、一つには本性であり、もう一つには働き、言い換えれば
本性の目的である。そして、本性への態勢である習慣については、一つの基体
の中に複数の習慣があることが明らかである。なぜなら、一つの基体の部分と
いうのはさまざまに理解されうるのであって、それらの部分の態勢が習慣と言
われるからである。

\\


Sicut, si accipiantur humani corporis partes humores, prout
disponuntur secundum naturam humanam, est habitus vel dispositio
sanitatis, si vero accipiantur partes similes ut nervi et ossa et
carnes, earum dispositio in ordine ad naturam, est fortitudo aut
macies, si vero accipiantur membra, ut manus et pes et huiusmodi,
earum dispositio naturae conveniens, est pulchritudo. Et sic sunt
plures habitus vel dispositiones in eodem.


&

たとえば、人間の身体の部分として体液が理解されるならば、それが人間本性
に即して態勢付けられるものとしては、「健康」という習慣ないし態勢である
が、他方で、神経と骨と肉として同じ(身体の)部分が理解されるならば、そ
れらの本性に対する態勢は「剛健さ」や「ひ弱さ」である。また、もし手や足
やそのようなものとして、四肢として理解されるならば、それらの本性に適合
した態勢は「美しさ」である。このように複数の習慣や態勢が同一のものの中
にある。


\\


Si vero loquamur de habitibus qui sunt dispositiones ad opera, qui
proprie pertinent ad potentias; sic etiam contingit unius potentiae
esse habitus plures. Cuius ratio est, quia subiectum habitus est
potentia passiva, ut supra dictum est, potentia enim activa tantum non
est alicuius habitus subiectum, ut ex supradictis patet.



&

他方、もし私たちが働きへの態勢である習慣について語るならば、その習慣は
厳密には能力に属するが、この意味でもまた、一つの能力に複数の習慣が
属するということが起こる。理由は以下の通り。前に述べられたとおり、習慣
の基体は受動的な能力である。というのも、前述のことから明らかなとおり、
能動的な能力だけが何らかの習慣の基体であることはできないからである。


\\


Potentia autem passiva comparatur ad actum determinatum unius speciei,
sicut materia ad formam, eo quod, sicut materia determinatur ad unam
formam per unum agens, ita etiam potentia passiva a ratione unius
obiecti activi determinatur ad unum actum secundum speciem.


&

しかるに受動的な能力は、一つの種に限定された作用に対して、質料が形相に
対するように関係する。なぜなら、質料が一つの形相に限定されるのが一つの
作用者によってであるように、受動的な能力もまた、一つの能動的な対象の性
格によって、種において一つの作用へと限定されるからである。

\\


Unde sicut plura obiecta possunt movere unam potentiam passivam, ita
una potentia passiva potest esse subiectum diversorum actuum vel
perfectionum secundum speciem. Habitus autem sunt quaedam qualitates
aut formae inhaerentes potentiae, quibus inclinatur potentia ad
determinatos actus secundum speciem. Unde ad unam potentiam possunt
plures habitus pertinere, sicut et plures actus specie differentes.

&

したがって、複数の対象が一つの受動的能力を動かすことができるように、一
つの受動的能力が、種においてさまざまな作用ないし完全性の基体でありうる。
ところが、習慣は、能力に内在する何らかの性質ないし形相であり、それによっ
て能力が、種において限定された作用へと傾くところのものである。したがっ
て、ちょうど種において異なる複数の作用と同じように、複数の習慣が、一つ
の能力に属しうる。


\\




{\scshape Ad primum ergo dicendum} quod, sicut in rebus naturalibus
diversitas specierum est secundum formam, diversitas autem generum est
secundum materiam, ut dicitur in V {\itshape Metaphys}. (ea enim sunt
diversa genere, quorum est materia diversa), ita etiam diversitas
obiectorum secundum genus, facit distinctionem potentiarum (unde
philosophus dicit, in VI {\itshape Ethic}., quod ad ea quae sunt
genere altera, sunt etiam animae particulae aliae); diversitas vero
obiectorum secundum speciem, facit diversitatem actuum secundum
speciem, et per consequens habituum.


&

第一異論に対しては、それゆえ、以下のように言われるべきである。『形而上
学』第5巻で言われるとおり、自然的諸事物において、種の多様性が形相にし
たがってあり、類の多様性が質料にしたがってあるように(なぜなら、類にお
いて異なるものは、それらの質料が異なるから)、類における対象の多様性も
また、能力の区別を生む(このことから哲学者は『ニコマコス倫理学』第6巻
で、類において異なるものに対しては、他の個別的な魂があると述べている)。
他方、種における対象の多様性は、種における作用の多様性を生み、結果的に、
習慣の多様性を生む。


\\



Quaecumque autem sunt diversa genere, sunt etiam specie diversa, sed
non convertitur. Et ideo diversarum potentiarum sunt diversi actus
specie, et diversi habitus, non autem oportet quod diversi habitus
sint diversarum potentiarum, sed possunt esse plures unius. Et sicut
sunt genera generum, et species specierum; ita etiam contingit esse
diversas species habituum et potentiarum.

&

ところで、なんであれ類において異なるものは種においても異なるが、その逆
は言えない。ゆえに、異なる能力に属する作用や習慣は種において異なるが、
異なる習慣が異なる能力に属さないといけないわけではなく、むしろ、複数の
習慣が一つの能力に属しうる。ちょうど、類の類や種の種があるように、習慣
や能力ののさまざまな種が存在する、ということがありうる。


\\




{\scshape Ad secundum dicendum} quod potentia, etsi sit quidem simplex secundum
essentiam, est tamen multiplex virtute, secundum quod ad multos actus
specie differentes se extendit. Et ideo nihil prohibet in una potentia
esse multos habitus specie differentes.

&

第二異論に対しては以下のように言われるべきである。能力は、本質において
単純だが、種において異なる多くの作用へ及ぶ限りにおいて、ちから(virtus)
において多様である。ゆえに一つの能力の中に、種において異なる多くの習慣
があることを何も妨げない。

\\




{\scshape Ad tertium dicendum} quod corpus formatur per figuram sicut
per propriam terminationem, habitus autem non est terminatio
potentiae, sed est dispositio ad actum sicut ad ultimum terminum. Et
ideo non possunt esse unius potentiae simul plures actus, nisi forte
secundum quod unus comprehenditur sub alio, sicut nec unius corporis
plures figurae, nisi secundum quod una est in alia, sicut trigonum in
tetragono. Non enim potest intellectus simul multa actu
intelligere. Potest tamen simul habitu multa scire.

&


第三異論に対しては以下のように言われるべきである。物体は、固有の限界と
しての形態によって形成される。他方、習慣は能力の限界ではなく、究極の終
端としての作用への態勢である。ゆえに、ある作用のもとに別の作用が包含さ
れるのでない限り、一つの能力に同時に複数の作用が属すことがありえない。
それはちょうど、四角形の中に三角形が含まれるようにして、一つの形態が他の形態に
含まれるかぎりにおいてでないならば、一つの物体に複数の携帯がある、とい
うことがありえないのと同様である。それは、知性が同時に多くのものを現実
に知性認識できないからである。しかし、習態においては(習慣においては)
多くのものを同時に知ることができる。


\end{longtable}
\newpage


\rhead{a.~}
\begin{center}
{\Large {\bf }}\\
{\large }\\
{\footnotesize }\\
{\Large \\}
\end{center}

\begin{longtable}{p{21em}p{21em}}


ARTICULUS 2
[35763] Iª-IIae q. 54 a. 2 arg. 1
Ad secundum sic proceditur. Videtur quod habitus non distinguantur secundum obiecta. Contraria enim sunt specie differentia. Sed idem habitus scientiae est contrariorum, sicut medicina sani et aegri. Non ergo secundum obiecta specie differentia, habitus distinguuntur.

&

\\


[35764] Iª-IIae q. 54 a. 2 arg. 2
Praeterea, diversae scientiae sunt diversi habitus. Sed idem scibile pertinet ad diversas scientias, sicut terram esse rotundam demonstrat et naturalis et astrologus, ut dicitur in II Physic. Ergo habitus non distinguuntur secundum obiecta.

&

\\


[35765] Iª-IIae q. 54 a. 2 arg. 3
Praeterea, eiusdem actus est idem obiectum. Sed idem actus potest pertinere ad diversos habitus virtutum, si ad diversos fines referatur, sicut dare pecuniam alicui, si sit propter Deum, pertinet ad caritatem; si vero sit propter debitum solvendum, pertinet ad iustitiam. Ergo etiam idem obiectum potest ad diversos habitus pertinere. Non ergo est diversitas habituum secundum diversitatem obiectorum.

&

\\


[35766] Iª-IIae q. 54 a. 2 s. c.
Sed contra, actus differunt specie secundum diversitatem obiectorum, ut supra dictum est. Sed habitus sunt dispositiones quaedam ad actus. Ergo etiam habitus distinguuntur secundum diversa obiecta.

&

\\


[35767] Iª-IIae q. 54 a. 2 co.
Respondeo dicendum quod habitus et est forma quaedam, et est habitus. Potest ergo distinctio habituum secundum speciem attendi aut secundum communem modum quo formae specie distinguuntur; aut secundum proprium modum distinctionis habituum. Distinguuntur siquidem formae ad invicem secundum diversa principia activa, eo quod omne agens facit simile secundum speciem. Habitus autem importat ordinem ad aliquid. Omnia autem quae dicuntur secundum ordinem ad aliquid, distinguuntur secundum distinctionem eorum ad quae dicuntur. Est autem habitus dispositio quaedam ad duo ordinata, scilicet ad naturam, et ad operationem consequentem naturam. Sic igitur secundum tria, habitus specie distinguuntur. Uno quidem modo, secundum principia activa talium dispositionum; alio vero modo, secundum naturam; tertio vero modo, secundum obiecta specie differentia; ut per sequentia explicabitur.

&

\\


[35768] Iª-IIae q. 54 a. 2 ad 1
Ad primum ergo dicendum quod in distinctione potentiarum, vel etiam habituum, non est considerandum ipsum obiectum materialiter; sed ratio obiecti differens specie, vel etiam genere. Quamvis autem contraria specie differant diversitate rerum, tamen eadem ratio est cognoscendi utrumque, quia unum per aliud cognoscitur. Et ideo inquantum conveniunt in una ratione cognoscibilis, pertinent ad unum habitum cognoscitivum.

&

\\


[35769] Iª-IIae q. 54 a. 2 ad 2
Ad secundum dicendum quod terram esse rotundam per aliud medium demonstrat naturalis, et per aliud astrologus, astrologus enim hoc demonstrat per media mathematica, sicut per figuras eclipsium, vel per aliud huiusmodi; naturalis vero hoc demonstrat per medium naturale, sicut per motum gravium ad medium, vel per aliud huiusmodi. Tota autem virtus demonstrationis, quae est syllogismus faciens scire, ut dicitur in I Poster., dependet ex medio. Et ideo diversa media sunt sicut diversa principia activa, secundum quae habitus scientiarum diversificantur.

&

\\


[35770] Iª-IIae q. 54 a. 2 ad 3
Ad tertium dicendum quod, sicut philosophus dicit, in II Physic. et in VII Ethic., ita se habet finis in operabilibus, sicut principium in demonstrativis. Et ideo diversitas finium diversificat virtutes sicut et diversitas activorum principiorum. Sunt etiam ipsi fines obiecta actuum interiorum, qui maxime pertinent ad virtutes, ut ex supradictis patet.

&




\end{longtable}
\newpage



\end{document}

%\rhead{a.~}
%\begin{center}
% {\Large {\bf }}\\
% {\large }\\
% {\footnotesize }\\
% {\Large \\}
%\end{center}
%
%\begin{longtable}{p{21em}p{21em}}
%
%&
%
%\\
%\end{longtable}
%\newpage

ARTICULUS 3
[35771] Iª-IIae q. 54 a. 3 arg. 1
Ad tertium sic proceditur. Videtur quod habitus non distinguantur secundum bonum et malum. Bonum enim et malum sunt contraria. Sed idem habitus est contrariorum, ut supra habitum est. Ergo habitus non distinguuntur secundum bonum et malum.

[35772] Iª-IIae q. 54 a. 3 arg. 2
Praeterea, bonum convertitur cum ente, et sic, cum sit commune omnibus, non potest sumi ut differentia alicuius speciei; ut patet per philosophum in IV Topic. Similiter etiam malum, cum sit privatio et non ens, non potest esse alicuius entis differentia. Non ergo secundum bonum et malum possunt habitus specie distingui.

[35773] Iª-IIae q. 54 a. 3 arg. 3
Praeterea, circa idem obiectum contingit esse diversos habitus malos, sicut circa concupiscentias intemperantiam et insensibilitatem, et similiter etiam plures habitus bonos, scilicet virtutem humanam et virtutem heroicam sive divinam, ut patet per philosophum in VII Ethic. Non ergo distinguuntur habitus secundum bonum et malum.

[35774] Iª-IIae q. 54 a. 3 s. c.
Sed contra est quod habitus bonus contrariatur habitui malo, sicut virtus vitio. Sed contraria sunt diversa secundum speciem. Ergo habitus differunt specie secundum differentiam boni et mali.

[35775] Iª-IIae q. 54 a. 3 co.
Respondeo dicendum quod, sicut dictum est, habitus specie distinguuntur non solum secundum obiecta et principia activa, sed etiam in ordine ad naturam. Quod quidem contingit dupliciter. Uno modo, secundum convenientiam ad naturam, vel etiam secundum disconvenientiam ab ipsa. Et hoc modo distinguuntur specie habitus bonus et malus, nam habitus bonus dicitur qui disponit ad actum convenientem naturae agentis; habitus autem malus dicitur qui disponit ad actum non convenientem naturae. Sicut actus virtutum naturae humanae conveniunt, eo quod sunt secundum rationem, actus vero vitiorum, cum sint contra rationem, a natura humana discordant. Et sic manifestum est quod secundum differentiam boni et mali, habitus specie distinguuntur. Alio modo secundum naturam habitus distinguuntur, ex eo quod habitus unus disponit ad actum convenientem naturae inferiori; alius autem habitus disponit ad actum convenientem naturae superiori. Et sic virtus humana, quae disponit ad actum convenientem naturae humanae, distinguitur a divina virtute vel heroica, quae disponit ad actum convenientem cuidam superiori naturae.

[35776] Iª-IIae q. 54 a. 3 ad 1
Ad primum ergo dicendum quod contrariorum potest esse unus habitus, secundum quod contraria conveniunt in una ratione. Nunquam tamen contingit quod habitus contrarii sint unius speciei, contrarietas enim habituum est secundum contrarias rationes. Et ita secundum bonum et malum habitus distinguuntur, scilicet inquantum unus habitus est bonus et alius malus, non autem ex hoc quod unus est boni et alius mali.

[35777] Iª-IIae q. 54 a. 3 ad 2
Ad secundum dicendum quod bonum commune omni enti non est differentia constituens speciem alicuius habitus, sed quoddam bonum determinatum, quod est secundum convenientiam ad determinatam naturam, scilicet humanam. Similiter etiam malum quod est differentia constitutiva habitus, non est privatio pura, sed est aliquid determinatum repugnans determinatae naturae.

[35778] Iª-IIae q. 54 a. 3 ad 3
Ad tertium dicendum quod plures habitus boni circa idem specie, distinguuntur secundum convenientiam ad diversas naturas, ut dictum est. Plures vero habitus mali distinguuntur circa idem agendum secundum diversas repugnantias ad id quod est secundum naturam, sicut uni virtuti contrariantur diversa vitia circa eandem materiam.


%\rhead{a.~}
%\begin{center}
% {\Large {\bf }}\\
% {\large }\\
% {\footnotesize }\\
% {\Large \\}
%\end{center}
%
%\begin{longtable}{p{21em}p{21em}}
%
%&
%
%\\
%\end{longtable}
%\newpage


ARTICULUS 4
[35779] Iª-IIae q. 54 a. 4 arg. 1
Ad quartum sic proceditur. Videtur quod unus habitus ex pluribus habitibus constituatur. Illud enim cuius generatio non simul perficitur, sed successive, videtur constitui ex pluribus partibus. Sed generatio habitus non est simul, sed successive ex pluribus actibus, ut supra habitum est. Ergo unus habitus constituitur ex pluribus habitibus.

[35780] Iª-IIae q. 54 a. 4 arg. 2
Praeterea, ex partibus constituitur totum. Sed uni habitui assignantur multae partes, sicut Tullius ponit multas partes fortitudinis, temperantiae et aliarum virtutum. Ergo unus habitus constituitur ex pluribus.

[35781] Iª-IIae q. 54 a. 4 arg. 3
Praeterea, de una sola conclusione potest scientia haberi et actu et habitu. Sed multae conclusiones pertinent ad unam scientiam totam, sicut ad geometriam vel arithmeticam. Ergo unus habitus constituitur ex multis.

[35782] Iª-IIae q. 54 a. 4 s. c.
Sed contra, habitus, cum sit qualitas quaedam, est forma simplex. Sed nullum simplex constituitur ex pluribus. Ergo unus habitus non constituitur ex pluribus habitibus.

[35783] Iª-IIae q. 54 a. 4 co.
Respondeo dicendum quod habitus ad operationem ordinatus, de quo nunc principaliter intendimus, est perfectio quaedam potentiae. Omnis autem perfectio proportionatur suo perfectibili. Unde sicut potentia, cum sit una, ad multa se extendit secundum quod conveniunt in aliquo uno, idest in generali quadam ratione obiecti; ita etiam habitus ad multa se extendit secundum quod habent ordinem ad aliquod unum, puta ad unam specialem rationem obiecti, vel unam naturam, vel unum principium, ut ex supradictis patet. Si igitur consideremus habitum secundum ea ad quae se extendit, sic inveniemus in eo quandam multiplicitatem. Sed quia illa multiplicitas est ordinata ad aliquid unum, ad quod principaliter respicit habitus, inde est quod habitus est qualitas simplex, non constituta ex pluribus habitibus, etiam si ad multa se extendat. Non enim unus habitus se extendit ad multa, nisi in ordine ad unum, ex quo habet unitatem.

[35784] Iª-IIae q. 54 a. 4 ad 1
Ad primum ergo dicendum quod successio in generatione habitus non contingit ex hoc quod pars eius generetur post partem, sed ex eo quod subiectum non statim consequitur dispositionem firmam et difficile mobilem; et ex eo quod primo imperfecte incipit esse in subiecto, et paulatim perficitur. Sicut etiam est de aliis qualitatibus.

[35785] Iª-IIae q. 54 a. 4 ad 2
Ad secundum dicendum quod partes quae singulis virtutibus cardinalibus assignantur, non sunt partes integrales, ex quibus constituatur totum sed partes subiectivae sive potentiales, ut infra patebit.

[35786] Iª-IIae q. 54 a. 4 ad 3
Ad tertium dicendum quod ille qui in aliqua scientia acquirit per demonstrationem scientiam conclusionis unius, habet quidem habitum, sed imperfecte. Cum vero acquirit per aliquam demonstrationem scientiam conclusionis alterius, non aggeneratur in eo alius habitus; sed habitus qui prius inerat fit perfectior, utpote ad plura se extendens; eo quod conclusiones et demonstrationes unius scientiae ordinatae sunt, et una derivatur ex alia.

